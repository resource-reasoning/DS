We explain the connection between abstract executions and kv-stores,
bridging by dependency graphs.
We show the connection between views on the kv-stores
and visibility relations on the abstract executions.
We then show how to construct an \( \ET[\top] \)-trace from an abstract execution, and vice versa.
This means that \( \ET[\top] \)-traces and abstract executions are equally expressive,
which is a key result for proving the correctness of our definitions of consistency models.
In this section, we only consider the traces that do not involve \( \prog \) but only committing fingerprint.
In \cref{sec:et-sound-complete-constructor}, we will go further and discuss the trace installed with \( \prog \).

%Note that abstract executions themselves can be seen as traces 
%where transactions are committed in the order of \( \AR \).

We show how to convert an abstract execution to a dependency graph.
The definition of \( \XToD \) is adapted from \citep{framework-concur,SIanalysis}.
By the definition of \( \DToK \) defined in \cref{def:dependency-graph-kv-store},
we therefore can convert an abstract execution to a kv-store.

\begin{definition}[Abstract executions to dependency graphs, {\( \XToD \)},  and kv-stores, {\( \XToK \)}]
\label{def:aexec-dgraph}
\label{def:aexec-kvstore}
Given an abstract execution \(\aexec \in \AbstractExecutions \),
the dependency graph is defined by
\(\XToD(\aexec) \FuncDef (\txidset[\aexec], \WR[\aexec], \WW[\aexec], \RW[\aexec])\) 
where the dependency relations on abstract execution are defined by:
\begin{enumerate}
\item write-read dependency relation on abstract execution is defined by:
    \( \WR[\aexec] \FuncDef \bigcup_{\key \in \Keys } \WR[\aexec](\key)\) where 
    \[
    \WR[\aexec](\key) \FuncDef \Set{(\txid,\txid') |  
                    \txid = \MaxVisTrans(\aexec, \VISInv[\aexec](\txid'),\key)
                    \\ {} \land \Exists{\val \in \Values} 
                    \opW(\key,\val) \in \aexec(\txid)
                    \land \opR(\key,\val) \in \aexec(\txid') } ,
    \]
\item write-write dependency relation on abstract execution is defined by:
    \( \WW[\aexec] \FuncDef \bigcup_{\key \in \Keys } \WW[\aexec](\key)\) where 
    \[
    \WW[\aexec](\key) \FuncDef \Set{(\txid,\txid') | (\txid,\txid') \in \AR[\aexec] 
                    \land \Exists{\val, \val' \in \Values} 
                    \opW(\key,\val) \in \aexec(\txid)
                    \land \opW(\key,\val') \in \aexec(\txid') } , 
    \]
\item read-write anti-dependency relation on abstract execution is defined by:
    \( \RW[\aexec] \FuncDef \bigcup_{\key \in \Keys } \RW[\aexec](\key)\) where 
    \[
    \RW[\aexec](\key) \FuncDef \Set{(\txid,\txid') |  
                    \Tuple{\MaxVisTrans(\aexec,\VISInv[\aexec](\txid),\key),\txid'} \in \AR[\aexec]
                    \\ {} \land \Exists{\val, \val' \in \Values} 
                    \opR(\key,\val) \in \aexec(\txid)
                    \land \opW(\key,\val') \in \aexec(\txid') } .
    \]
\end{enumerate}
Given \( \DToK \) in \cref{def:dependency-graph-kv-store},
the kv-store induced by the abstract execution \( \aexec \) is defined by:
\[ \XToK(\aexec) = \DToK(\XToD(\aexec)). \] 
\end{definition}


\begin{toappendix}
\label{sec:proof-well-form-xtod}
\begin{proposition}[Well-defined \( \XToD \)]
\label{prop:well-defined-xtod}
Given an abstract execution \( \aexec \in \AbstractExecutions \),
the dependency graph \( \XToD(\aexec)\) is well-formed.
\end{proposition}
\begin{proof}
Given the definition of \( \XToD \), if the dependency graph \( \dgraph = \XToD(\aexec)\) 
for some abstract execution \( \aexec \), then \( \WR[\dgraph] = \WR[\aexec] \),
\( \WW[\dgraph] = \WW[\aexec] \) and  \( \RW[\dgraph] = \RW[\aexec]\)  .
Consider all the well-formed conditions for relations \( \WR,\WW,\RW \) defined in \cref{def:dgraph}.
\begin{enumerate}
\Case{write-read dependency \( \WR \)}
    By the definition of \( \WR[\aexec] \), it is trivial for \cref{equ:dgraph-wr-def,equ:dgraph-wr-minimum}.
    Because visibility relation cannot violate session order by \cref{equ:aexec-vis-so},
    it follows \cref{equ:dgraph-wr-so}.
    Since \(\MaxVisTrans \) in the definition of \( \WR[\aexec] \) returns a unique transaction, it means that \cref{equ:dgraph-wr-unique} holds.
\Case{write-write dependency \( \WW \)}
    By the definition of \( \WW[\aexec] \), it is trivial for \cref{equ:dgraph-ww-def,equ:dgraph-ww-minimum}.
    Because the initialisation transaction \( \txidinit \)
    is \( \AR[\aexec]\)-before any other transactions by  \cref{equ:aexec-ar-init}, 
    this implies \cref{equ:dgraph-ww-init}.
    Since \(\AR[\aexec] \) is a total order and \( \SO \subseteq \AR[\aexec] \), 
    by the definition of \( \WW[\aexec]\), the write-write dependency cannot violate the \( \SO \) 
    and thus \cref{equ:dgraph-ww-so,equ:dgraph-ww-irreflexive} hold.
    Last write-write dependency must be transitive,
    because \( \AR[\aexec] \) is transitive and 
    \( \ToEdge{\txid | \WW[\aexec](\key) -> \txid' | \WW[\aexec](\key) -> \txid'' }\) implies
    \( \ToEdge{\txid | \AR[\aexec] -> \txid' | \AR[\aexec] -> \txid'' }\) and therefore
    \( \ToEdge{\txid | \WW[\aexec](\key) -> \txid'' }\);
    this implies \cref{equ:dgraph-ww-transitive}.
\Case{read-write anti-dependency \( \RW \)}
    Consider two transactions \( \txid, \txid'\) such that \( (\txid,\txid') \in \RW[\aexec](\key) \)
    for some key \( \key \).
    By definition of \( \RW[\aexec] \), \( \opR(\key,\val) \in \aexec(\txid )\) for some value \( \val \).
    There exists a transition \( \txid'' \) such that \( \txid'' = \MaxVisTrans(\aexec,\VISInv[\aexec](\txid),\key) \),
    and therefore \( (\txid'',\txid) \in \WR[\aexec](\key)\).
    Again by definition of \( \RW[\aexec] \), it is must the case that \( (\txid'',\txid') \in \AR[\aexec] \)
    and therefore \( (\txid'',\txid') \in \WW[\aexec](\key) \).

    Now consider three transactions \( \txid, \txid', \txid'' \)
    such that \( (\txid'',\txid) \in \WR[\aexec](\key) \) 
    and \( (\txid'',\txid') \in \WW[\aexec](\key) \) for some key \( \key \).
    First, \( (\txid'',\txid) \in \WR[\aexec](\key) \) implies
    \( \txid'' = \MaxVisTrans(\aexec,\VISInv[\aexec](\txid),\key) \) and \( (\txid'',\txid) \in \AR[\aexec] \);
    then because \( (\txid'',\txid') \in \WW[\aexec](\key) \), it means \( (\txid'',\txid') \in \AR[\aexec](\key) \)
    and thus we have \( (\txid,\txid') \in \RW[\aexec] \) by the definition of \( \RW[\aexec]\). \qedhere
\end{enumerate}
\end{proof}
%\begin{proofsketch}
%Given an abstract execution \( \aexec \), by the definition of \( \dgraph = \XToD(\aexec)\),
%we have \( \WR[\dgraph] = \WR[\aexec] \), \( \WW[\dgraph] = \WW[\aexec] \) and  \( \RW[\dgraph] = \RW[\aexec]\).
%Because \( \aexec \) apply last-write-wins policy, 
%\( \WR[\aexec] \) satisfies \cref{equ:dgraph-wr-minimum,equ:dgraph-wr-def,equ:dgraph-wr-so}.
%Since \( \WW[\aexec] \subseteq \AR[\aexec]\), 
%it is easy to prove 
%\cref{equ:dgraph-ww-def,equ:dgraph-ww-minimum,equ:dgraph-ww-irreflexive,equ:dgraph-ww-transitive,equ:dgraph-ww-init,equ:dgraph-ww-so}.
%Last, \( \RW[\aexec] \) can be derived from \( \WR[\aexec]\) and \( \WW[\aexec]\).
%The full proof is given in \cref{sec:proof-well-form-xtod} on \pageref{sec:proof-well-form-xtod}.
%\end{proofsketch}
\end{toappendix}


Each abstract execution \(\aexec\) determines a \emph{unique and well-formed dependency graph} \(\dgraph = \XToD(\aexec)\),
because:
\begin{enumerate*}
\item \( \AR \) is a total order; and
\item \( \aexec \) applies last-write-wins in the sense that a transaction always reads from 
the latest visible transaction given by \( \MaxVisTrans \) defined in \cref{def:aexec}.
\end{enumerate*}
%It is easy to see that \( \XToD(\aexec) \) is a well-formed dependency graph and 
%thus \( \XToK(\aexec) \) is a well-formed kv-store.
The full proof is given in \cref{sec:proof-well-form-xtod} on \pageref{sec:proof-well-form-xtod}.
Therefore, \( \XToK(\aexec) \) is a uniquely defined and well-formed kv-store, by \cref{prop:dtok-well-defined}.
However, abstract executions are \emph{not} bijective to kv-stores
in that \emph{several} abstract executions may be encoded to the same kv-store.
Because: \begin{enumerate*} 
\item kv-stores do not have the total arbitration order of transactions, and
\item kv-stores only track the actually \( \WR \) dependency, 
    in contrast to the \emph{potential write-read dependency} captured by visibility relation.
\end{enumerate*}

Apart from the correspondence between individual kv-stores and abstract executions,
there is a correspondence between views on kv-stores and visibility relations on abstract executions.
%It is a key step for comparing traces in kv-stores and abstract executions.
Recall that snapshots in kv-stores are computed from views,
while in abstract executions, these snapshots are computed from visibility relations.
A kv-store \(\kvs\) is \emph{compatible} with an abstract execution \(\aexec\), 
written \( \kvs \AKcomp \aexec \), if and only if 
any snapshot made on \(\kvs\) can be obtained by an snapshot made on \(\aexec\), and vice versa. 

\begin{definition}[Compatibility between kv-stores and abstract executions]
\label{def:compatible}
Given an abstract execution \( \aexec \), the view induced by a set of visible transactions \( \txidset \) 
on \( \aexec \), written \( \GetView(\aexec,\txidset) \), is defined by:
\(
    \GetView(\aexec,\txidset) \FuncDef \lambda 
    \key \in \Keys \ldotp \Set{0} \cup \Set{\idx | \WtOf({\XToK(\aexec)(\key,\idx)}) \in \txidset} .
\)
A kv-store \(\kvs\) and an abstract execution \(\aexec\) are compatible, written 
\(\kvs \AKcomp \aexec\), if and only if:
\begin{enumerate}
\item for any transaction \( \txid\),
\begin{align}
\txid \in \kvs \iff \txid \in \aexec \label{equ:ak-compatible-transactions}
\end{align}
\item for any subset of transactions \( \txidset \subseteq \aexec \) such that \( \txidinit \in \txidset \),
\begin{align}
\AexecSnapshot(\aexec,\txidset) = \Snapshot(\kvs,\GetView(\aexec,\txidset)) \label{equ:ak-compatible-vis-to-view}
\end{align}
\item given the definition of \( \VisTrans \), defined in \cref{def:vis-transactions-function},
    for any view \( \vi \in \Views(\kvs) \) on kv-store \( \kvs \),
\begin{align}
    \Snapshot(\kvs,\vi) = \AexecSnapshot(\aexec,\VisTrans(\kvs,\vi)) . \label{equ:ak-compatible-view-to-vis}
\end{align}
\end{enumerate}
\end{definition}

Given a set of transactions \( \txidset \), the  \( \GetView \) function extracts a view 
that includes all versions written by any transaction in \( \txidset \).
A kv-store \( \kvs \) is compatible with an abstract execution \(\aexec \),
if and only if: 
\begin{enumerate*}
\item they contain the same set of transactions (\cref{equ:ak-compatible-transactions}); and 
\item the snapshot induced by a set of transactions \( \txidset \) on \( \aexec \), 
is the same as the snapshot induced by the view \( \GetView(\aexec,\txidset) \) on \( \kvs \) (\cref{equ:ak-compatible-vis-to-view})
and \item vice versa (\cref{equ:ak-compatible-view-to-vis}).
\end{enumerate*}

\begin{toappendix}
\label{sec:proof-xtod-compatibility}
\end{toappendix}
\begin{theoremrep}[Compatibility of \( \aexec \) and \( \XToK(\aexec)\)]
\label{thm:xtod-compatibility}
For any abstract execution \(\aexec\), \(\aexec \AKcomp \XToK(\aexec)\).
\end{theoremrep}
\begin{proof}
Let kv-store \( \kvs = \XToK(\aexec) \).
\Case{\( \AexecSnapshot(\aexec,\txidset) = \Snapshot(\kvs,\GetView(\aexec,\txidset)) \) 
            for all \( \txidset \subseteq \aexec \) such that \( \txidinit \in \txidset \)}
    By the \cref{prop:getview-valid}, the view \( \vi = \GetView(\aexec, \txidset)\) 
    is a valid view on \( \kvs \), that is, \( \vi \in \Views(\kvs) \).
    Given that \( \vi \) is a valid view,
    it is sufficient to prove that for all keys \( \key \),
    \[
        \AexecSnapshot(\aexec,\txidset)(\key) = \Snapshot(\kvs,\GetView(\aexec,\txidset))(\key) .
    \]
    Let transaction \( \txid = \MaxVisTrans(\aexec,\txidset,\key)\).
    Therefore operation \( \opW{\key,\val} \in \aexec(\txid)\) for some \(\val \in \Values \).
    By definition of \(\AexecSnapshot\), it follows that \( \AexecSnapshot(\aexec,\txidset)(\key) \).
    By definition of \( \MaxVisTrans \), any other transaction \( \txid' \in \txidset \setminus \Set{\txid} \) 
    that also wrote the key \( \key \) with \( \opW(\key,\stub) \in \aexec(\txid') \),
    must be \( \AR[\aexec] \)-before \( \txid \), that is \( (\txid',\txid) \in \AR[\aexec] \);
    this means that \( (\txid',\txid) \in \WW[\dgraph] \) for \( \dgraph = \XToD(\aexec) \).
    Since transactions \( \txid, \txid' \) both wrote key \( \key \),
    there must exists two indexes \( \idx, \idx' \) such that \( \txid = \WtOf(\kvs(\key,\idx)) \),
    \( \txid' = \WtOf(\kvs(\key,\idx'))\).
    Because \( \kvs = \DToK(\dgraph)\) and \( (\txid',\txid) \in \WW[\dgraph] \),
    then \( \idx > \idx' \);
    therefore by definition of \( \Snapshot \), the value matches as \( \Snapshot(\kvs,\vi)(\key) = \val \).
\Case{\( \Snapshot(\kvs,\vi) = \AexecSnapshot(\aexec,\VisTrans(\kvs,\vi)) \) 
            for all \( \vi \subseteq \Views(\kvs) \)}
    %This case is non-trivial as \( \txidset \) contains \emph{read-only} transactions.
    %By \cref{prop:erase-read-only}, it is safe to erase read only transactions from \( \txidset \),
    %when calculating the view \( \GetView(\aexec, \txidset) \).
    Let transaction sets \( \txidset = \VisTrans(\kvs,\vi) \).
    It is sufficient to prove that for all keys \( \key \),
    \[
        \Snapshot(\kvs,\vi)(\key) = \AexecSnapshot(\aexec,\VisTrans(\kvs,\vi))(\key) .
    \]
    Let \( \val = \Snapshot(\kvs,\vi)(\key) \)
    By definition of \( \VisTrans \), for versions included in \( vi \) for key \( \key \),
    their writer must be included in \( \txidset \), that is
    \[
        \Forall{\idx \in \vi(\key)} \WtOf(\kvs(\key,\idx)) \in \txidset .
    \]
    Let \( \txid \) be the transaction that wrote the newest version for key \( \key \) in \( \vi \)
    as \( \txid = \WtOf(\kvs(\key,\Max[<](\vi(\key)))) \) and 
    \[
        \Forall{\idx \in \vi(\key)} \ToEdge{ \WtOf(\kvs(\key,\idx)) | \Inv(\WW[\kvs]) -> \txid }.
    \]
    By definition of \( \Snapshot \), the snapshot must include \(\txid \)
    \[
        \ValOf(\kvs(\key,\Max[<](\vi(\key)))) = \val ,
    \]
    which means \( \val = \AexecSnapshot(\aexec,\VisTrans(\kvs,\vi))(\key) \) 
    by definition of \( \AexecSnapshot \).
    Note that the initial version is always included in the view \( 0 \in \vi(\key) \), 
    therefore \( \txidinit \in \txidset \) and the function 
    \( \AexecSnapshot(\aexec,\VisTrans(\kvs,\vi)) \)
    must be defined by \cref{prop:well-defined-aexec-snapshot}.
\end{proof}
\begin{proofsketch}
It can be derived from the definition of \( \XToK \).
Given \( \aexec \), let \( \kvs = \XToK(\aexec)\).
\Cref{equ:ak-compatible-transactions} trivial holds.
Note that, given an abstract execution \( \aexec \)
and a set of transactions \( \txidset \), the view \( \GetView(\aexec, \txidset) \) is well-formed on \( \XToD(\aexec) \)
(detail is given in \cref{prop:getview-valid} on page \pageref{sec:proof-well-formed-get-view}).
For \cref{equ:ak-compatible-vis-to-view}, it is sufficient to prove
\[
    \AexecSnapshot(\aexec,\txidset)(\key) = \Snapshot(\kvs,\GetView(\aexec,\txidset))(\key) .
\]
Note that \( \WW[\kvs] = \WW[\aexec] \subseteq \AR[\aexec] \).
%and therefore \( (\txid,\txid') \in \WW[\kvs](\key) \implies (\txid,\txid') \in \AR[\aexec](\key) \).
This means if two transactions \( \txid, \txid' \) both write to key \( \key \),
and if \( \txid, \txid' \in \txidset \), then the \( \AexecSnapshot(\aexec,\txidset)(\key) \) may include the value written by \( \txid' \) but not \( \txid \).
Similarly by definition of \( \GetView \) and \( \Snapshot \),
the snapshot \( \Snapshot(\kvs,\GetView(\aexec,\txidset))(\key) \) may contain the value of version written by \( \txid' \) but not \( \txid \).
Therefore, we have the prove for \cref{equ:ak-compatible-vis-to-view}.
Similarly,  \cref{equ:ak-compatible-view-to-vis} can be derived by  \( \WW[\kvs] = \WW[\aexec] \subseteq \AR[\aexec] \).
The full proof is given in \cref{sec:proof-xtod-compatibility} on page \pageref{sec:proof-xtod-compatibility}.
\end{proofsketch}

\begin{toappendix}
\label{sec:proof-well-formed-get-view}
\begin{proposition}[Well-formed views of \( \GetView \)]
\label{prop:getview-valid}
For any abstract execution \(\aexec\), and \(\txidset \subseteq \aexec\), 
\(\GetView(\aexec, \txidset) \in \Views(\XToK(\aexec))\).
\end{proposition}
\begin{proof}
Let \( \vi = \GetView(\aexec, \txidset)\) and \( \kvs = \XToK(\aexec) \).
By the definition of \(\GetView\),
the initial version must be included in the view \(0 \in \vi(\key)\) for any key \( \key \) (\cref{equ:view-wf-initial}).
For any index \( \idx \) included in the view for key \( \key \) such that \( \idx \in \vi(\key) \), 
it must be the case that \(0 \leq \idx < \Abs{ \kvs(\key) }\) (\cref{equ:view-wf-with-in-range}).
Now consider two versions \( \kvs(\key,\idx) \) and \( \kvs(\key',\idx') \) such that
\( \idx \in \vi(\key)\) and \( \WtOf(\kvs(\key,\idx)) = \WtOf(\kvs(\key',\idx')) \),
it must the case that \( \WtOf(\kvs(\key,\idx)) \in \txidset \) thus \( \idx' \in \vi(\key')\) (\cref{equ:view-wf-atomic}).
\end{proof}
\end{toappendix}

%\begin{proposition}[erasing read-only transactions]
%\label{prop:erase-read-only}
%Let \(\vi \in \Views(\mkvs_{\aexec})\), and let \(\txidset \subseteq \txidset_{\aexec}\) be a 
%set of read-only transactions in \(\aexec\). Then 
%\(\getView[\aexec, \txidset \cup \Tx[\mkvs_{\aexec}, \vi]] = \vi\). 
%\end{proposition}

%\begin{proof}
%Fix a key \(\key\). Suppose that \(i \in \getView[\aexec, \txidset \cup \Tx[\mkvs_{\aexec}, \vi]](\key)\). 
%By definition, \(\mkvs_{\aexec}(\key, j) = (\stub, \txid, \stub)\) for some \(\txid \in \txidset \cup \Tx[\mkvs_{\aexec}, \vi]\). 
%Because \(\txidset\) only contains read-only transactions, by definition of \(\mkvs_{\aexec}\) there exists 
%no index \(j\) such that \(\mkvs_{\aexec}(\key, j) = (\stub, \txid', \stub)\) for some \(\txid' \in \txidset\), 
%hence it must be the case that \(\txid \in \Tx[\mkvs_{\aexec}, \vi]\). By definition of \(\Tx\), 
%this is possible only if there exist a key \(\key'\) and an index \(j\) such that \(\mkvs_{\aexec}(\key', \vi) = (\stub, \txid, \stub)\). 
%Because \(\vi\) is atomic by definition, and because \(\mkvs_{\aexec}(\key, i) = (\stub, \txid, \stub)\), then we have that \(i \in \vi(\key)\). 

%Now suppose that \(i \in \vi(\key)\), and let \(\mkvs_{\aexec}(\key, i) = (\stub, \txid, \stub)\) for some \(\txid\). 
%This implies that \((\otW, \key, \stub) \in_{\aexec} \txid\).
%By definition \(\txid \in \Tx[\mkvs_{\aexec}, \vi]\), hence \(\txid \in \txidset \cup \Tx[\mkvs_{\aexec}, \vi)]\). 
%Because \(\txid \in \txidset \cup \Tx[\mkvs_{\aexec}, \vi]\), then for any key \(\key'\) such that 
%\((\otW, \key', \stub) \in_{\aexec} \txid\), there exists an index \(j \in \getView[\aexec, \txidset \cup \Tx[\mkvs_{\aexec}, \vi]]\) 
%\(\mkvs(\key', j) = (\stub, \txid, \stub)\); because kv-stores only allow a transaction to write at most one version 
%per key, then the index \(j\) is uniquely determined. In particular, we know that \((\otW, \key, \stub) \in_{\aexec} \txid\), 
%and \(\mkvs_{\aexec}(\key, i) = (\stub, \txid, \stub)\), from which it follows that \(i \in \getView[\aexec, \txidset \cup \Tx[\mkvs_{\aexec}, \vi]](\key)\).
%\end{proof}
