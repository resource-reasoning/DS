\emph{Abstract executions} \citep{ev_transactions,framework-concur} 
are an alternative formalism for defining consistency models. 
As with dependency graphs, an abstract execution \(\aexec\)
is a directed graph where nodes representing transactions,
%(with each node labelled with a transaction identifier and a fingerprint), 
and edges representing certain relations between transactions: that is
\begin{enumerate}
\item \emph{visibility relation} which means that,
if \( (\txid,\txid') \in \VIS \), 
then transaction \( \txid' \) \emph{observes} \( \txid \),
or in other words, \( \txid \) is \emph{visible} to \( \txid \); and
\item \emph{arbitration relation}, also known as \emph{arbitration order},
which means that
if \( (\txid,\txid') \in \AR \),
then the update of the \( \txid' \) \emph{overwrite} the update of \( \txid \),
or in other words, \( \txid \) \emph{happens before} \( \txid' \).
\end{enumerate}
An example abstract execution graph is given in \cref{fig:abstract-execution}. 

\begin{figure}
\centering
\begin{tikzpicture}
\OperationsBox[left]{t0}{\txid_0}{
    /{\opW(\key_1,\val_0)}%
    , /{\opW(\key_2,\val'_0)}%
    , /{\cdots}%
};

\OperationsBox((t0.north east) + (1.5,0.5))[above]{t1}{\txid_1}{
    /{\opR(\key_2,\val'_0)}%
    , /{\opW(\key_1,\val_1)}%
};


\OperationsBox((t0.south east) + (1.5,-0.5))[below]{t2}{\txid_2}{
    /{\opR(\key_1,\val_0)}%
    , /{\opW(\key_2,\val_2)}%
};


\path[->]
(t0.north) edge[bend left=30] node[above, yshift=3pt, xshift=-20pt, pos=0.3] {$\VIS,\AR$} (t1.west)
(t0.south) edge[bend right=30] node[below, yshift=-3pt, xshift=-20pt, pos=0.3] {$\VIS,\AR$} (t2.west)
([xshift=8pt]t1.south) edge[bend left=20] node[right] {$\AR$} ([xshift=8pt]t2.north);

\end{tikzpicture}

\hrulefill

\caption{Abstract execution}
\label{fig:abstract-execution}
\end{figure}


\begin{definition}[Abstract executions]
\label{def:aexec}
Given the definition of \( \SO \) (\cref{def:session-order}),
the set of \emph{abstract executions}, \( \AbstractExecutions \ni \aexec \), 
is defined by:
\(
\AbstractExecutions \TypeDef \Set{(\txidop, \SO, \VIS, \AR) | \WfAExec(\txidop, \VIS, \AR) }
\)
where the well-formed condition, \( \WfAExec \), is defined by:
\begin{align*}
\MoveEqLeft \WfAExec(\txidop, \VIS, \AR) \PredDef 
\\ & \begin{multlined}[t]
    \txidop \in \begin{Bracketed} \begin{Bracketed} \TxIDZs \ToPFFunc \Fingerprints 
            \end{Bracketed} \uplus \Mapping{\txidinit -> \Set{\opW(\key,\valinit) 
                    | \key \in \Keys \land \valinit \in \InitVal(\key) }} \end{Bracketed}
            \\ \TotalRelation(\AR,\txidop) \land \WfVIS(\VIS,\txidop,\AR) .
        \end{multlined}
\end{align*}
The well-formed conditions on visibility and arbitration relations, 
\( \TotalRelation \) and \( \WfVIS \), are defined:
\begin{align}
    \TotalRelation(\AR,\txidop) & \PredDef \Forall{\txid,\txid' \in \Dom(\txidop)} \nonumber
    \\ & \txid = \txid' \lor (\txid,\txid') \in \AR \lor (\txid',\txid) \in \AR \label{equ:aexec-ar-total}
    \\ & \land (\txid,\txid) \notin \AR \label{equ:aexec-ar-irreflexive}
    \\ & \land \AR = \Trasi(\AR) \label{equ:aexec-ar-transitive}
    \\ & (\txidinit,\txid) \in \AR \label{equ:aexec-ar-init}
    \\ & \land \SO \subseteq \AR \label{equ:aexec-ar-so}
    \\ \WfVIS(\VIS,\txidop, \AR) & \PredDef \Forall{\txid,\txid' \in \Dom(\txidop) | \key \in \Keys | \val \in \Values} \nonumber
    \\ & \begin{Bracketed} (\txid,\txid') \in \VIS \implies (\txid',\txid) \notin \SO \end{Bracketed} \label{equ:aexec-vis-so}
    \\ & \land \VIS \subseteq \AR \label{equ:aexec-ar-vis}
    \\ & \land (\txidinit, \txid) \in \VIS  \label{equ:aexec-vis-init}
    \\ & \left( \begin{array}{@{}l@{}} \opR(\key,\val) \in \txidop(\txid) 
                    \\ \implies \opW(\key,\val) \in \txidop(\MaxVisTrans((\txidop,\VIS,\AR),\Inv(\VIS)(\txid),\key)) 
         \end{array} \right) \label{equ:aexec-vis-external}
\end{align}
where \( \MaxVisTrans((\txidop,\VIS,\AR),\txidset,\key) \) defined by:
\[
\MaxVisTrans((\txidop,\VIS,\AR),\txidset,\key) \FuncDef 
            \Max[\AR](\Set{ \txid | \Exists{\val \in \Values } 
                \txid \in \txidset \land \opW(\key,\val) \in \txidop(\txid) }) 
\]
with \( \Max[\AR](\txidset') \) returning the maximum transaction in \( \txidset' \)
with respect to \( \AR \).
Given an abstract execution \( \aexec \),
let \( \txidop[\aexec]\), \( \VIS[\aexec] \) and \( \AR[\dgraph]\) be the first, third and fourth projections.
Notation \( \txid \in \dgraph \) and \( \op \in \dgraph(\txid) \) denotes
\( \txid \in \Dom(\txidop) \) and \( \op \in \txidop(\txid) \) respectively.
Let the initial abstract execution 
\( \aexecinit = \Tuple{\Mapping{\txidinit -> \Set{\opW(\key,\valinit) 
                    | \key \in \Keys \land \valinit \in \InitVal(\key) }},\emptyset, \emptyset} \).
\end{definition}

For a well-formed abstract execution, 
the arbitration relation \(\AR\) is a total order over all the transactions 
(\cref{equ:aexec-ar-total,equ:aexec-ar-irreflexive,equ:aexec-ar-transitive})
starting from the initialisation transaction (\cref{equ:aexec-ar-init}),
and agrees with the session order \( \SO \) (\cref{equ:aexec-ar-so}).
The visibility relation \( \VIS \) agrees with the session order (\cref{equ:aexec-vis-so}) 
and arbitrary order (\cref{equ:aexec-ar-vis}) 
Transactions always see the initialisation transaction \( \txidinit \) (\cref{equ:aexec-vis-init}).
Lastly, we only consider abstract executions that apply the \emph{last-write-wins} policy,
that is, a transaction reading \(\key\) always fetches 
the latest visible write (\(\VIS\)-predecessor) on \(\key\) (\cref{equ:aexec-vis-external}).

%Abstract executions are known as declarative models.
Consistent models are specified by \emph{visibility axioms}
that constrain the shape of visibility relation.
Note that we focus on the consistency model satisfying 
\emph{snapshot property} and \emph{last-write-win} policy%
\footnote{A more general approach than \emph{last-write-win} policy 
would be parametrised how each transaction \( \txid \) picks values for keys,
from its visible transactions \( \Refl(\VIS)(\txid)\).}.

\begin{definition}[Visibility axioms]
\label{def:aexec-axioms}
Two abstract executions \(\aexec, \aexec'\) \emph{agree} on a subset of transaction \( \txidset \),
written \( \aexec \aexeceq[\txidset] \aexec' \), if and only if
\[
    \Forall{\txid,\txid' \in \txidset } \aexec(\txid) = \aexec(\txid')
    \land \begin{Bracketed} \ToEdge{\txid | \VIS[\aexec] -> \txid' } 
                \iff \ToEdge{\txid | \VIS[\aexec'] -> \txid' } \end{Bracketed}
    \land \begin{Bracketed} \ToEdge{\txid | \AR[\aexec] -> \txid' } 
                \iff \ToEdge{\txid | \AR[\aexec'] -> \txid' } \end{Bracketed} .
\]
The set of \emph{visibility relation axioms} or just \emph{axioms}, 
\( \VisAxioms \ni \visaxiom \), is defined by:
\( \VisAxioms \TypeDef \AbstractExecutions \ToTFunc \PowerSet(\TxIDs \times \TxIDs)\)
such that whenever two abstract executions \(\aexec, \aexec'\) agree on a subset of 
transactions \(\txidset\), 
then \(\visaxiom(\aexec) \cap (\txidset \times \txidset) \subseteq \visaxiom(\aexec')\).
The subsets of \( \VisAxioms \) are ranged over 
\( \visaxioms, \visaxioms[0], \visaxioms', \cdots \).
\end{definition}

Given an abstract execution \( \aexec \), a visibility relation axiom \( \visaxiom \) 
specifies the \emph{minimum visibility relation} in \( \aexec \):
that is, the visibility relation \( \VIS[\aexec]\)
satisfies \( \visaxiom(\aexec) \subseteq \VIS[\aexec] \).
The well-formed condition requires an axiom be local in the sense that
given two abstract executions \( \aexec, \aexec' \), 
the minimum visibility relation must be 
the same for the any sub-graph on which \( \aexec, \aexec' \) agree.
Note that, for two abstract executions \( \aexec \aexeceq[\txidset] \aexec' \),
the condition, \(\visaxiom(\aexec) \cap (\txidset \times \txidset) 
            \subseteq \visaxiom(\aexec')\),
implies that \( \visaxiom(\aexec) \cap (\txidset \times \txidset) 
        = \visaxiom(\aexec') \cap (\txidset \times \txidset) \).

\begin{definition}[Consistent models on abstract executions]
An abstract execution \( \aexec \) satisfies a set of axioms \( \visaxioms \), 
written \( \AExecSat(\aexec,\visaxioms) \), if and only if
\(
\Forall{\visaxiom \in \visaxioms} \visaxiom(\aexec) \subseteq \VIS[\aexec] .
\)
The set of abstract executions induced by a set of axioms \( \visaxioms \) is defined by:
\[
    \CMA(\visaxioms) \FuncDef \Set{\aexec | \AExecSat(\aexec,\visaxioms) } .
\]
\end{definition}

