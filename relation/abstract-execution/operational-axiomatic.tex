This operational semantics on abstract executions are equally expressive as the axiomatic definitions
in that, given a set of axioms \( \visaxioms\), both formalisms yields the same set of abstract executions.
We first define the notion of reachable abstract executions of a program \( \prog \),
written \( \EvalAExec{\prog} \).

\begin{definition}[Abstract executions induced by programs]
Given a program \( \prog \), the set of abstract executions obtained by 
executing the program under a set of axioms \( \visaxioms \) is defined by:
\[
\EvalAExec{\prog} \FuncDef \Set{ \aexec | \Exists{n} (\aexec,\stub,\stub) = \Last(\AProgTraces(\visaxioms,\prog,n))  }
\]
where \( \AProgTraces \) is defined by:
\begin{align*}
\AProgTraces(\visaxioms,\prog,0) & \FuncDef \Set{\ToAexecProg{\aexecinit | \clenv | \prog} 
                | \Dom(\prog) \subseteq \Dom(\clenv) }
\\ \AProgTraces(\visaxioms,\prog,n+1) & \FuncDef \Set{\ToAexecProg{\aexectrc | \lb 
                    -> \aexec | \clenv | \prog'}  
                    |  \aexectrc \in \AProgTraces(\prog,n)
                    \land \Dom(\prog') \subseteq \Dom(\clenv)} .
\end{align*}
\end{definition}

We now introduce \emph{cuts} of an abstract execution \( \aexec \), 
which are used to construct traces where the final state \( \aexec \).
This means that an individual abstract execution contains all the history information.

\begin{definition}[Cuts of abstract executions]
\label{def:aexec-cut}
The cut of first \( n \) transactions of an abstract execution \( \aexec \), 
written \( \AexecCut(\aexec,n) \),  is defined by:
\[
    \AexecCut(\aexec,n) \FuncDef 
    \begin{multlined}[t]
    \CodeFont{let} \ \txidset = \ARClose(\aexec,\idx) \ 
    \CodeFont{in} \
    \\ \Tuple{\lambda \txid \in \txidset \ldotp \aexec(\txid)
                , \VIS[\aexec] \cap \Set{(\txid,\txid') | \txid,\txid' \in \txidset }
                , \AR[\aexec] \cap \Set{(\txid,\txid') | \txid,\txid' \in \txidset } }
    \end{multlined}
\]
where \( \ARClose(\aexec,\idx) \) is defined by:
\(
    \ARClose(\aexec,\idx) \FuncDef \Set{\txid_0, \cdots, \txid_\idx}
\)
for transactions \( \txid_0, \cdots, \txid_\idx \) such that
\( \ToEdge{\txid_0 | \AR[\aexec] -> \cdots | \AR[\aexec] 
                -> \txid_\idx | \AR[\aexec] -> \cdots | \AR[\aexec] -> \txid_{\Abs{\aexec} - 1} } \).
\end{definition}

Given an abstract execution \( \aexec \),
we call \(  \AexecCut(\aexec,\idx) \) as the \Th{(\idx + 1)} cut of \( \aexec \).
It is easy to see that the \Th{(\idx + 1)} cut is the result of 
committing next transaction (with respect to \( \AR \)) to the \Th{\idx} cut.
Note that the zero-cut is the graph that only contains the initialisation transaction.
The detail is given in \cref{prop:aexec-cut-to-update} on page \pageref{sec:proof-aexec-cut-to-trace}.

\begin{toappendix}
\label{sec:proof-equal-axiom-semantics}
\end{toappendix}
\begin{theoremrep}[Equal expressibility between declarative and operational semantics on abstract executions]
\label{thm:equivalence-aexec-trace-visaxioms}
For any \( \visaxioms \subseteq \VisAxioms \),
the operational semantics capture the same set of abstract executions as 
direct axiomatic definitions on abstract definitions,
that is, \( \bigcup_{\prog \in \Programs} \EvalAExec{\prog} = \CMA(\visaxioms)\).
\end{theoremrep}
\begin{proof}
This prove is similar to the one in \cref{thm:ettrc-et-prog}.
We prove the set closure for both sides respectively.
\begin{enumerate}
\Case{\( \bigcup_{\prog \in \Programs} \EvalAExec{\prog} \subseteq \CMA(\visaxioms) \)}
    By the definition of \( \bigcup_{\prog \in \Programs}\EvalAExec{\prog}\),
    It suffices to prove that for every trace \( \aexectrc \), 
    initial state \( (\aexecinit,\clenv_0) \) with program \( \prog \)
    and final state \( (\aexec,\clenv) \) with program \( \prog' \),
    \begin{Formulae}
    \begin{Formula}
        \Dom(\prog) \subseteq \Dom(\clenv_0)
        \land \aexectrc = \ToAexecProg{\aexecinit | \clenv_0 | \prog | \lb | * -> \aexec | \clenv | \prog' }
        \\ \aexec \in \CMA(\visaxioms) .
        \label{equ:aexec-prog-to-consistency-model}
    \end{Formula}
    \end{Formulae}
    We prove \cref{equ:aexec-prog-to-consistency-model} by induction on the length of \( \aexectrc \).
    \begin{enumerate*}
    \CaseBase{\(\aexectrc = \ToAexecProg{\aexecinit | \clenv_0 | \prog }\)}
        It is trivial that \( \AExecSat(\aexecinit,\visaxioms)\) and thus \( \aexecinit \in \CMA(\visaxioms)\).
    \CaseInd{\(\aexectrc = \ToAexecProg{ \aexectrc' | \lb -> \aexec | \clenv | \prog' }\)}
        Let \( \LastConf(\aexectrc) =  (\aexec',\clenv'), \prog'' \).
        If \( \lb = \lbPri\), then \( \aexec = \aexec' \);
        therefore by \ih, \cref{equ:aexec-prog-to-consistency-model} holds.
        Consider \( \lb = \lbTrans{\txidset,\fp}  \).
        By the \rAAtomicTrans rule, \( \aexec = \UpdateAExec(\aexec',\txidset,\fp,\txid) \)
        for some \( \txid \in \NextAExecTxid(\aexec,\cl) \).
        Note that the new abstract execution \( \aexec \) contains all transactions and edges in \( \aexec' \),
        and thus \( \aexec \aexeceq[\aexec'\Proj{0}] \aexec' \).
        This means that first by the \ih, \( \AExecSat(\aexec',\visaxioms) \),
        and by the definition of well-formed visibility axioms that every axiom must be local,
        we only need to consider the new visibility edges.
        By the last premise of the \rAAtomicTrans rule,
        for any \( \visaxiom \in \visaxioms \), it must be the case that \( \Inv(\visaxiom)(\aexec')(\txid) \subseteq \txidset \),
        and thus \( \aexec \in \CMA(\visaxioms) \).
    \end{enumerate*}
\Case{\( \bigcup_{\prog \in \Programs} \EvalAExec{\prog} \supseteq \CMA(\visaxioms) \)}
    Assume an abstract execution \( \aexec \in \CMA(\visaxioms) \).
    Let \( (\txidop,\VIS,\AR) =  \aexec \) .
    We prove a stronger result that 
    there is an abstract execution trace \( \aexectrc \) corresponding to the cut of abstract execution, 
    that is, for any number \( \idx \) and cut of abstract execution \( \aexec' \)
    \begin{Formulae}
    \begin{Formula}
        \aexec' = \AexecCut(\aexec,\idx) \implies 
                \Exists{\aexectrc | \prog \in \Programs | \clenv \in \ClientEnvs } 
                \\ \LastConf(\aexectrc) = (\aexec', \clenv), \prog 
                \land \prog = \lambda \cl \in \Dom(\prog) \ldotp \pskip
    \label{equ:aexec-to-aexec-trace}
    \end{Formula}
    \end{Formulae}
    We prove by induction on the number of transactions.
    The key here is to construct a program \( \prog \) and this is done in similar way as in \cref{thm:ettrc-et-prog}.
    \begin{enumerate}
    \CaseBase{\(\AexecCut(\aexec,0)\)}
        We know \( \aexecinit = \AexecCut(\aexec,0) \) by the definition of \( \aexecinit \).
        Let the client set \( C  = \Set{\cl | \Exists{n \in \Indexs} \txid[\cl](n) \in \aexec }\).
        We pick a program \( \prog = \lambda \cl \in C \ldotp \pskip \)
        a client environment \( \clenv_0 = \lambda \cl \in C \ldotp \stk  \) for some stack \( \stk \), 
        and therefore \( \aexectrc = \ToAexecProg{ \aexecinit | \clenv_0 | \prog }\)
        which implies \cref{equ:aexec-to-aexec-trace}.
    \CaseBase{\(\AexecCut(\aexec,\idx)\)}
        Let \( \aexec' = \AexecCut(\aexec,\idx-1) \) and \( \aexec'' = \AexecCut(\aexec,\idx) \).
        Suppose that there is a trace \( \aexectrc' \) for \( \AexecCut(\aexec,\idx-1) \),
        such that
        \[
        \begin{multlined}[t]
            \aexectrc' = \ToAexecProg{\aexecinit | \clenv_0 | \prog | \stub | * -> \aexec' | \clenv' | \prog' }
            \\ {} \land \prog' = \lambda \cl \in \Dom(\prog') \ldotp \pskip .
        \end{multlined}
        \]
        Let the transaction \( \txid = \aexec'' \setminus \aexec' \), 
        fingerprint \( \fp = \aexec(\txid) \), transaction set \( \txidset = \VISInv(\txid) \).
        By \cref{prop:aexec-cut-to-update}, it follows that \( \aexec'' = \UpdateAExec(\aexec',\txidset,\fp,\txid) \).
        Let the transactional command \( \trans = \TransFp(\fp) \) (\TransFp is defined in \cref{thm:ettrc-et-prog})
        and the new initial program \( \prog'' = \ExtendProgram(\prog,\cl,\ptrans{\trans}) \).
        It is easy to see there exists a new trace \( \aexectrc \) such that
        \[
            \aexectrc = \ToAexecProg{ \aexecinit | \clenv_0 | \prog'' | \stub | * 
                -> \aexec' | \clenv' | \prog'\UpdateFunc{\cl -> \ptrans{\trans} } | stub -> \aexec'' | \clenv'' | \prog^* } ,
        \]
        for some client environment \( \clenv'' \) and finial program \( \prog^* \) 
        with \( \prog^* = \lambda \cl \in \Dom(\prog^*) \ldotp \pskip \). \qedhere
    \end{enumerate}
\end{enumerate}
\end{proof}
\begin{proofsketch}
It is easy to see that \( \bigcup_{\prog \in \Programs} \EvalAExec{\prog} \subseteq \CMA(\visaxioms)\)
since each step in the traces from the program \( \prog \) is constrained by \( \visaxioms \).
The opposite way,  \( \bigcup_{\prog \in \Programs} \EvalAExec{\prog} \supseteq \CMA(\visaxioms)\),
can be derived from the following result: for all \( n \),
\[ \AexecCut(\aexec,n) = \UpdateAExec(\aexec',\VISInv[\aexec](\txid),\txid,\aexec(\txid)) \]
where \( \txid \notin \AexecCut(\aexec,n) \) and \(  \txid \in \AexecCut(\aexec,n+1) \).
The full detail is given in \cref{sec:proof-equal-axiom-semantics} 
on page \pageref{sec:proof-equal-axiom-semantics}.
\end{proofsketch}

\begin{toappendix}
\label{sec:proof-aexec-cut-to-trace}
\begin{proposition}[Abstract execution cut to update]
\label{prop:aexec-cut-to-update}
Given an abstract execution \( \aexec \) and the cut \( \aexec' = \AexecCut(\aexec,n) \),
assume that the transaction \( \txid \) is the next transaction in the arbitration order,
that is, \( \ToEdge{ \Max[\AR[\aexec]](\aexec') | \AR[\aexec] -> \txid }\),
assume there exists no transaction \( \txid' \) such that 
\( \ToEdge{ \Max[\AR[\aexec]](\aexec') | \AR[\aexec] -> \txid' | \AR[\aexec] -> \txid  }\),
and then
\[ \AexecCut(\aexec,n+1) = \UpdateAExec(\aexec',\VISInv[\aexec](\txid),\txid,\aexec(\txid)) . \]
\end{proposition}
\begin{proof}
    It is trivial by the definitions of \( \AexecCut\) and \( \UpdateAExec \).
\end{proof}
\end{toappendix}
