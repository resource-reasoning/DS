The wildly used definition on abstract executions for causal consistency is that 
\( \VIS \) is transitive and \( \SO \in \VIS \).
Yet it is for the sack of elegant definition,
while there is a equivalent minimum visibility relation (\cref{thm:cc-visaxioms}) defined by 
\( \visaxioms[\CC] \FuncDef \Set{ \lambda \aexec \ldotp (\WR[\aexec] \cup \SO) ; \VIS[\aexec] \subseteq \VIS[\aexec] , 
                                    \lambda \aexec \ldotp \SO \subseteq \VIS[\aexec]} \),
where \( \WR[\aexec] \) is defined in \cref{def:aexec-dgraph}.

\begin{theorem}[Minimum visibility relation for (\texorpdfstring{\CC}{\texttt{CC}})]
\label{thm:cc-visaxioms}
For two abstract executions \( \aexec,\aexec' \),
the following constrain on visibility,
\begin{Formulae}
\begin{Formula}
    (\WR[\aexec] \cup \SO) ; \VIS[\aexec] \subseteq \VIS[\aexec] \land \SO \subseteq \VIS[\aexec]
    \label{equ:kvstore-cc-spec}
\end{Formula}
\end{Formulae}
is equivalent to
\begin{Formulae}
\begin{Formula}
    \VIS[\aexec'] ; \VIS[\aexec'] \subseteq \VIS[\aexec'] \land \SO \subseteq \VIS[\aexec']
    \label{equ:aexec-cc-spec}
\end{Formula}
\end{Formulae}
in that 
\(
    \Forall{\txid \in \TxIDs | \fp } \left( \fp = \aexec(\txid) \iff \fp = \aexec'(\txid) \right)
    \land \AR[\aexec] = \AR[\aexec'] .
\)
\end{theorem}
\begin{proof}
For an abstract execution \( \aexec \) that satisfies \cref{equ:kvstore-cc-spec},
by \cref{lem:aexec-spec-cc}, there exists \( \aexec' \) that satisfies \cref{equ:aexec-cc-spec}.
Assume an abstract execution \( \aexec' \) that satisfies \cref{equ:aexec-cc-spec}.
Since \( \WR[\aexec'] \subseteq \VIS[\aexec']\) by the definition of \( \WR[\aexec']\),
thus \( \aexec' \) satisfies \cref{equ:kvstore-cc-spec}.
\end{proof}

\begin{toappendix}
\begin{lemma}[Minimum visibility relation for (\texorpdfstring{\CC}{\texttt{CC}})]
\label{lem:aexec-spec-cc}
For any abstract execution \( \aexec \), if it satisfies \( \visaxioms[\CC] \),
there exists a new abstract execution \( \aexec' \) such that \( \SO \in \VIS[\aexec]\) and
\begin{Formulae}
\begin{Formula}
    \Forall{\txid \in \TxIDs | \fp } \left( \fp = \aexec(\txid) \iff \fp = \aexec'(\txid) \right)
    \land \AR[\aexec] = \AR[\aexec'] \land \VIS[\aexec'] ; \VIS[\aexec'] \subseteq \VIS[\aexec'] .
    \label{equ:aexec-spec-cc}
\end{Formula}
\end{Formulae}
\end{lemma}
\begin{proof}
We erase some visibility relation for each transaction following 
the arbitration order \( \AR \) until the visibility is transitive.
Intuitively, the final visibility relation is exactly \( \Trasi((\WR[\aexec] \cup \SO)) \).
Assume the \Th{\idx} transaction \( \txid_\idx \)  with respect to the arbitration order.
Let \( \rel[\idx] \) be a new visibility for the transaction \( \txid_\idx \) such that
\( {\rel[\idx]}\Proj{2} = \Set{\txid_\idx}\) for all indexes \( \idx \)
and the union of visibility relations \( \bigcup_{0 \leq j \leq \idx } \rel[\idx] \) is transitive.
We preserve that, for each index \( \idx \), cut of abstract execution \( \aexec' =  \AexecCut(\aexec, \idx) \)
and visibility relation \( \VIS' = \bigcup_{0 \leq j \leq \idx } \rel[j] \),
the following invariant holds:
\begin{Formulae}
& \begin{Formula} 
    \VIS' ; \VIS' \subseteq \VIS'  ,
    \label{equ:cc-vis-idx-transitive} 
\end{Formula}
\\ & \begin{Formula}
    \Forall{ \txid \in \aexec } (\txid,\txid_i) \in \rel[\idx] \implies (\txid, \txid_i) \in (\WR[\aexec'] \cup \SO) .
    \label{equ:cc-vis-idx-minimum}
\end{Formula}
\end{Formulae}
We prove the above by induction on the number \( \idx \).
\begin{enumerate}
\CaseBase{\( \idx = 0 \)}
    By the definition of \( \AexecCut \), we know that \(\aexecinit = \AexecCut(\aexec,0) \)
    and \cref{equ:cc-vis-idx-transitive,equ:cc-vis-idx-minimum} trivially hold.
\CaseInd{\( \idx > 0 \)}
    Suppose that, for the \Th{(\idx-1)} step,
    the abstract execution \( \aexec'' =  \AexecCut(\aexec, \idx - 1) \)
    and the visibility relation \( \VIS'' = \bigcup_{0 \leq j \leq \idx-1 } \rel[j] \) 
    satisfy \cref{equ:cc-vis-idx-transitive,equ:cc-vis-idx-minimum}.
    Let consider \Th{\idx} step, the transaction \( \txid_i \),
    the cut \( \aexec' =  \AexecCut(\aexec, \idx) \)
    and the visibility relation \( \VIS' = \bigcup_{0 \leq j \leq \idx } \rel[j] \).
    Initially we take \( \rel \) as an empty set.
    First, we include \( \Set{(\txid,\txid_i) | (\txid,\txid_i) \in \WR[\aexec]} \) to \( \rel \)
    and, by the definition of \( \WR[\aexec]\), 
    it trivially does not affect any read operation for the transaction \( \txid_i \).
    Then we do the same for \( \SO \) as that 
    we include \( \Set{(\txid,\txid_i) | (\txid,\txid_i) \in \SO} \) to \( \rel \).
    Note that \( \SO \) cannot affect any read operation for the transaction \( \txid_i \) neither,
    otherwise it contradicts to that \( \SO \subseteq \VIS[\aexec] \) and the definition of \( \WR[\aexec] \).

    For relations \( \rel' = \rel ; \bigcup_{0 \leq j \leq \idx-1 } \rel[j] \) and then \( \rel[\idx] = \rel \cup \rel' \),
    it easy to see that \( \rel \in \VIS[\aexec]\) and, then by \ih, \( \rel' \in \VIS[\aexec] \).
    We prove that the \( \rel[\idx] \) does not affect any read operation for the transaction \( \txid_i \)
    by contradiction.
    Assume distinct transactions \( \txid,\txid' \) such that
    \( \ToEdge{\txid'' | \rel \cup \rel' -> \txid' | \rel \cup \rel' -> \txid_i } \),
    and immediately  by the definition of \( \rel \) and \( \rel' \),
    then \( \ToEdge{\txid'' | \rel' -> \txid' | \rel -> \txid_i } \).
    Assume that \( \txid'' \) change the read operation for a key \( \key \) in \( \txid_i \).
    This means that there exists a transaction \( \txid^* \) such that
    \( (\txid^*,\txid_i) \in \WR[\aexec](\key)\) and \( (\txid^*,\txid'') \in \AR[\aexec] \),
    where the latter implies that \( (\txid'',\txid_i) \in \WR[\aexec](\key) \);
    there is a contradiction and thus 
    \( \rel[\idx] \) does not affect any read operation for the transaction \( \txid_i \).

    We now prove that \cref{equ:cc-vis-idx-transitive,equ:cc-vis-idx-minimum} still hold.
    \begin{enumerate}
    \Case{\cref{equ:cc-vis-idx-transitive}}
        Assume a relation \( \rel^* = \bigcup_{0 \leq j \leq \idx-1 } \rel[j] \) 
        and transactions \( \txid, \txid',\txid'' \) such that 
        \[
            \ToEdge{\txid | \rel^* \cup \rel[\idx] -> \txid' | \rel^* \cup \rel[\idx] -> \txid'' } .
        \]
        If \( \ToEdge{\txid | \rel^*  -> \txid' | \rel^*  -> \txid'' } \), 
        then by \ih, \( \ToEdge{\txid | \rel^*  -> \txid'' } \).
        Note that \( \ToEdge{\txid | \rel[\idx]  -> \txid' | \rel^*  -> \txid'' } \) cannot happen,
        because it contradict to that \( \txid' = \txid_i\) and \( (\txid'',\txid_i) \in \AR[\aexec] \).
        Thus consider \( \ToEdge{\txid | \rel^*  -> \txid' | \rel[\idx]  -> \txid'' } \).
        It must be the case that \( \txid'' = \txid_i \) and by the definition of \( \rel[\idx] \),
        we know that \( \ToEdge{\txid | \rel[\idx]  -> \txid'' } \).
    \Case{\cref{equ:cc-vis-idx-minimum}}
        By the construction, \cref{equ:cc-vis-idx-minimum} hold. \qedhere
    \end{enumerate}
\end{enumerate}
\end{proof}
\end{toappendix}

We pick the invariant as \( \aexecinv[\CC] = \aexecinv[\MR] \cup \aexecinv[\RYW]  \).
\SOUNDLET{\CC}{ \txidsetrd \supseteq
\begin{multlined}[t]
\left( \bigcup_{\Set{\txid[\cl](\idx) | \txid[\cl](\idx) \in \aexec}} 
\VISInv[\aexec](\txid[\cl](\idx)) \cup \Refl((\Inv(\SO)))(\txid[\cl](\idx)) \right) 
\setminus \Set{\txid' | \Forall{l | \key | \val } (l,\key,\val) \in \aexec(\txid') \implies l = \opR } .
\end{multlined} }
Assume 
\[ 
\txidsetrd' = 
\begin{multlined}[t]
\left( \bigcup_{\Set{\txid[\cl](\idx) | \txid[\cl](\idx) \in \aexec}} 
\VISInv[\aexec](\txid[\cl](\idx)) \cup \Refl((\Inv(\SO)))(\txid[\cl](\idx)) \right) 
\setminus \Set{\txid' | \Forall{l | \key | \val } (l,\key,\val) \in \aexec(\txid') \implies l = \opR } .
\end{multlined} 
\]
and \( \txidsetrd'' = \txidsetrd \setminus \txidsetrd' \).
By the definition of soundness, we prove the following result
\begin{Formulae}
& \begin{Formula}
\Inv(\SO)(\txid) \subseteq \txidset \cup \txidsetrd'
\label{equ:cc-so-vis}
\end{Formula}
\\ & \begin{Formula}
\Inv((( \WR[\aexec'] \cup \SO ) ; \VIS[\aexec'] )) (\txid) \subseteq \txidset \cup \txidsetrd' \cup \txidsetrd''
\label{equ:cc-vis-transitive}
\end{Formula}
\\ & \begin{Formula}
\aexecinv[\CC](\aexec',\cl) \subseteq \VisTrans(\XToK(\aexec'),\vi')
\label{equ:cc-inv-preserve}
\end{Formula}
\end{Formulae}
\Cref{equ:cc-so-vis} can be proven in the same way as in \cref{sec:sound-complete-mr}
We now prove \cref{equ:cc-vis-transitive}.
Initially we take \( \txidsetrd'' \) to be an empty set.
Note that \(\VISInv[\aexec'](\txid) = \txidset \cup \txidsetrd' \cup \txidsetrd'' \).
By \cref{thm:view-vis-relation,equ:view-close-to-aexec}, there exists \( \txidsetrd'' \) such that
\( \txidset \cup \txidsetrd'' \) is closed under \( \WR[\aexec'] \cup \SO \).
Now consider a transaction \( \txidrd \in \txidsetrd' \) and
assume a transaction \( \txid' \) such that \( \ToEdge{ \txid' | \WR[\aexec'] \cup \SO -> \txidrd } \).
There are two cases depending on \( \txidrd \).
\begin{enumerate}
\Case{\( \ToEdge{\txidrd | \VIS[\aexec'] -> \txid'' | \SO -> \txid} \) for some \( \txid'' \)}
    For this case, we have
    \begin{align*}
    \ToEdge{\txid' | \WR[\aexec'] \cup \SO -> \txidrd | \VIS[\aexec'] -> \txid'' | \SO -> \txid }
    & 
    \implies \ToEdge{\txid' | \WR[\aexec] \cup \SO -> \txidrd | \VIS[\aexec] -> \txid'' | \SO -> \txid }
    \\ & \implies \ToEdge{\txid' | \VIS[\aexec] -> \txid'' | \SO -> \txid } .
    \end{align*}
    By \( \aexecinv[\MR]\), we know that \(\txid' \in \aexecinv[\MR] \cup \txidsetrd' \).
\Case{\( \ToEdge{\txidrd | \SO -> \txid} \)}
    For this case, we have
    \begin{align*}
    \ToEdge{\txid' | \WR[\aexec'] \cup \SO -> \txidrd | \SO -> \txid }
    & 
    \implies \ToEdge{\txid' | \WR[\aexec] \cup \SO -> \txidrd | \SO -> \txid }
    \\ & \implies \ToEdge{\txid' | \VIS[\aexec] -> \txid'' | \SO -> \txid } .
    \end{align*}
    By \( \aexecinv[\MR]\), we know that \(\txid' \in \aexecinv[\MR] \cup \txidsetrd' \).
\end{enumerate}
Last, \cref{equ:cc-inv-preserve}can be proven in the same way as in \cref{sec:sound-complete-mr,sec:sound-complete-ryw}.

\COMPLETELET{\CC}
By \cref{thm:cc-visaxioms},
it is sufficient to prove with respect to the following visibility axioms,
\( \visaxioms[\CC]' \FuncDef \Set{ \lambda \aexec \ldotp  \VIS[\aexec] ; \VIS[\aexec] \subseteq \VIS[\aexec] , 
                                    \lambda \aexec \ldotp \SO \subseteq \VIS[\aexec]} \).
By the definition of \( \et[\CC] \), we prove \( \CanCommit[\CC]\) and \( \ViewShift[\MR \cup \RYW]\) respectively.
Since \( (\WR[\aexec] \cup \SO) ; \VIS[\aexec]  \subseteq \VIS[\aexec] ; \VIS[\aexec] \subseteq \VIS[\aexec] \),
then \( \CanCommit[\CC]\) can be derived from \cref{thm:view-vis-relation,equ:aexec-close-to-view}
and \( \ViewShift[\RYW] \) can be proven in the same way as in \cref{sec:sound-complete-ryw}.
By \( \VIS[\aexec] ; \SO \subseteq \VIS[\aexec] ; \VIS[\aexec] \subseteq \VIS[\aexec]  \),
\( \ViewShift[\MR] \) can be proven in the same way as in \cref{sec:sound-complete-mr}.


