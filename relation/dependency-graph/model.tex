A dependency graph \(\dgraph\) is a directed labelled graph,
where the nodes denote transactions, 
and the edges denote \emph{dependencies} between transactions,
including session order \( \SO \), \emph{write-read} dependency \( \WR \),
\emph{write-write} dependency \( \WW \) and \emph{read-write} anti-dependency \(\RW \).

\begin{definition}[Dependency graph]
\label{def:dgraph}
Given the session order defined in \cref{def:session-order},
the set of \emph{dependency graphs}, \( \DependencyGraphs \ni \dgraph \), is defined by:
\[
    \DependencyGraphs \TypeDef \Set{(\txidop, \SO, \WR, \WW, \RW) | \WfDGraph(\txidop, \WR, \WW, \RW)  }.
\]
where the well-formed condition for a dependency graph, \( \WfDGraph \), is defined by:
\begin{align*}
\WfDGraph(\txidop, \WR, \WW, \RW) & \PredDef \begin{multlined}[t]
        \txidop \in \begin{Bracketed} \begin{Bracketed} \TxIDZs \ToPFFunc \Fingerprints 
            \end{Bracketed} \uplus \Mapping{\txidinit -> \Set{\opW(\key,\valinit) 
                        | \key \in \Keys }} \end{Bracketed} \land {}
        \\ {} \Forall{\key \in \Keys} \left( \begin{array}{@{}l@{}}
            \Pred{wfwr}(\WR,\txidop,\key) 
            \land \Pred{wfww}(\WW,\txidop,\key)
            \\ \qquad {} \land \Pred{wfrw}(\WR,\WW,\RW,\key)
         \end{array} \right)
        \end{multlined}
\end{align*}
and the well-formed conditions, \( \Pred{wfwr} \), \( \Pred{wfww} \) and \( \Pred{wfrw} \),
for the three dependency relations are defined by:
\begin{Formulae}
\Pred{wfwr}(\WR,\txidop,\key) \PredDef 
& \begin{Formula}
\Forall{\txid, \txid', \txid''} 
(\txid,\txid') \in \WR
\implies \opW(\key,\stub) \in \txidop(\txid) \land
\opR(\key,\stub) \in \txidop(\txid')
\label{equ:dgraph-wr-minimum}
\end{Formula}
\\ & \begin{Formula}
\opR(\key,\stub) \in \txidop(\txid') 
\land \Exists{\txid^*} \opW(\key,\stub) \in \txidop(\txid^*) 
\implies (\txid^*,\txid') \in \WR
\label{equ:dgraph-wr-def}
\end{Formula}
\\ & \begin{Formula}
(\txid,\txid') \in \WR \implies (\txid',\txid) \notin \Refl(\SO)
\label{equ:dgraph-wr-so}
\end{Formula}
\\ & \begin{Formula}
(\txid',\txid) \in \WR \land (\txid'',\txid) \in \WR(\txidop,\key)  \implies \txid' = \txid''
\label{equ:dgraph-wr-unique}
\end{Formula}
\\ \Pred{wfww}(\WR,\txidop,\key) \PredDef & \begin{Formula}
\Forall{\txid,\txid'}
\opW(\key,\stub) \in \txidop(\txid) 
\land \opW(\key,\stub) \in \txidop(\txid')  \nonumber
\end{Formula}
\\ & \begin{Formula}
\quad {} \implies \txid = \txid' \lor (\txid,\txid') \in \WW \lor (\txid',\txid) \in \WW
\end{Formula}
\label{equ:dgraph-ww-def}
\\ & \begin{Formula}
(\txid,\txid') \in \WW
\implies \opW(\key,\stub) \in \txidop(\txid) \land
\opW(\key,\stub) \in \txidop(\txid') 
\label{equ:dgraph-ww-minimum}
\end{Formula}
\\ & \begin{Formula}
(\txid,\txid') \in \WW \implies \txid' \neq \txid
\label{equ:dgraph-ww-irreflexive}
\end{Formula}
\\ & \begin{Formula}
\WW = \Trasi(\WW)
\label{equ:dgraph-ww-transitive}
\end{Formula}
\\ & \begin{Formula}
(\txid,\txid') \in \WW \implies \txid' \neq \txidinit
\label{equ:dgraph-ww-init}
\end{Formula}
\\ & \begin{Formula}
(\txid,\txid') \in \WW \implies (\txid',\txid) \notin \SO
\label{equ:dgraph-ww-so}
\end{Formula}
\\ \Pred{wfrw}(\WR,\WW,\RW,\key) \PredDef 
& \begin{Formula}
{
\RW = \left\{ \begin{array}{@{} c | c @{}}
    (\txid, \txid')
    & 
    \begin{array}{@{}l@{}}
        \Exists{\txid''} (\txid'', \txid) \in \WR
        \\ \quad {} \land (\txid'', \txid') \in \WW
        \land \txid \neq \txid' 
    \end{array}
    \end{array} \right\} .
}
\label{equ:dgraph-rw}
\end{Formula}
\end{Formulae}
Given a \emph{dependency graph} \( \dgraph \),
let \( \txidop[\dgraph], \WR[\dgraph], \WW[\dgraph]\) and \( \RW[\dgraph]\) return the first, the third to fifth projection respectively,
and \( \WR[\dgraph](\key), \WW[\dgraph](\key)\) and \( \RW[\dgraph](\key)\) be the relations on the key \( \key \).
Let notations \( \txid \in \dgraph \) and \( \op \in \dgraph(\txid) \) denote
\( \txid \in \Dom(\txidop) \) and \( \op \in \txidop(\txid) \) respectively.
\end{definition}

\Cref{fig:dependency-graph} gives an example dependency graph.
Each node in a dependency graph is a fingerprint labelled with the unique transaction identifier.
The special initialisation transaction \( \txidinit \) initialises all keys,
with a set of \emph{infinite} write operations.
Except the initialisation transaction, other transactions must be finite.
Edges are labelled with dependency relations \(\SO, \WR, \WW \) and \(\RW\).
Session order \( \SO \) is defined in \cref{def:session-order}.
The write-read dependency, \( (\txid, \txid') \in \WR(\key) \), means that 
a transaction \( \txid' \) read from a value of \( \key \) 
written by another transaction \( \txid' \) (\cref{equ:dgraph-wr-minimum}).
Note that any read operation must read from a transaction (\cref{equ:dgraph-wr-def,equ:dgraph-wr-unique}).
The write-read dependency must agree with session order (\cref{equ:dgraph-wr-so}).
Given a key \( \key \), the write-write dependency on \( \key \) is a total order,
that is, a connexive (\cref{equ:dgraph-ww-def,equ:dgraph-ww-minimum}), 
irreflexive (\cref{equ:dgraph-ww-irreflexive}) and transitive (\cref{equ:dgraph-ww-transitive}) relation.
Intuitively, if \( (\txid, \txid') \in \WW(\key) \), 
then \( \txid' \) overwrites a value of \( \key \) previously written by \( \txid \).
This relation \( \WW(\key) \) is a total order,
The initial transaction \( \txidinit \) writes the initial value of \( \key \),
hence \( (\txidinit, \txid) \in \WW(\key) \) for any transaction \( \txid \) that writes \( \key \)
(\cref{equ:dgraph-ww-init}).
Similarly, write-write dependency must agree on session order (\cref{equ:dgraph-ww-so}).
Last, read-write anti-dependency \( \RW \) is derived from \( \WR \) and \( \WW \).
Intuitively, if \( (\txid, \txid')  \in \RW(\key) \), the transaction \( \txid \) read 
a value \( \val \) on the key \( \key \) written by some transaction \( \txid'' \)
that is later overwritten by \( \txid' \).

\input{dependency-graph/fig}

In dependency graphs, consistency models are specified using
axioms that rule out invalid graphs that contain certain cycles.
For example, snapshot isolation require that graphs only contain cycles with 
at least \emph{two adjacent read-write anti-dependency edges}, that is,
relation \( (\SO \cup \WR[\dgraph] \cup \WW[\dgraph]) ; \RWInv[\dgraph] \) is acyclic \cite{adya,SIanalysis}.

In our kv-store semantics, 
we adopt the same names for the dependency relations between transactions.
This is to emphasis the similarity between the dependency relations on kv-stores 
and the dependency relations on dependency graphs.
In fact, there is a bijection between our global kv-stores and dependency graphs.
We first show how to extract a dependency graph from a kv-store.
