\section{Overview}
\label{sec:overview}

We motivate our key ideas, centralised kv-stores, partial client views and execution tests,
using an single counter example.
We show that our interleaving semantics is an ideal mid-point for verifying distributed protocols 
and proving invariant properties of client programs.


\mypar{Example} We use a simple transactional library, \(\CodeFont{Counter}(\key)\), to
 introduce our operational semantics.  Clients of this counter library can manipulate the
value of key \(\key\) via two transactions:
\( 
\ctrinc(\key) \defeq 
\begin{transaction}
\plookup{\pv{x}}{\key}; \ 
\pmutate{\key}{\pv{x}{+}1}
\end{transaction}
\)
and
\(
\ctrread(\key) \defeq
\begin{transaction}
\plookup{\pv{x}}{\key}
\end{transaction}
\).
Command \( \plookup{\pv{x}}{\key} \) reads the value of key \( \key \) to
local variable \( \vx \), and command \( \pmutate{\key}{\pv{x}{+}1} \)
writes the value of \( \pv{x}{+}1 \) to key \( \key \).  The code of each
operation is wrapped in square brackets, denoting a transaction that 
must be executed \emph{atomically}.  

Consider a replicated database where a client only interacts with one replica.
For such a database, the correctness of atomic transactions is subtle, 
depending heavily on the particular consistency model under consideration.  
Consider the client program, defend by:
\[ 
\prog_{\CodeFont{LU}} \defeq \left(\cl_1 : \ctrinc(\key) \;|| \; \cl_2: \ctrinc(\key) \right)
\]
where we assume that the clients \( \cl_1 \) and \( \cl_2 \) work on different replicas and
the \(\key\) initially holds value \(0\) in all replicas.
Intuitively, since transactions are executed atomically, after both
calls to \(\ctrinc(\key)\) have terminated, the counter should hold the value \(2\).
Indeed, this is the only outcome allowed under serialisability (\(\SER\)), 
where transactions appear to execute in a sequential (serial) order, one after another.
The implementation of \(\SER\) in distributed kv-stores comes at a
significant performance cost. Therefore, implementers are content with
{weaker} consistency models \cite{tango,CORFU,ramp,rola,cops,wren,redblue,PSI,NMSI,gdur,clocksi,distrsi,PSI-RA,si}. 
For example, if the replicas provide no synchronisation mechanism for transactions,
then it is possible for both clients to read the same initial value \(0\) for \(\key\) at their
distinct replicas, update them to \(1\), and eventually propagate the absolute value of \( \key \) to other replicas. 
Thus, both sites are unchanged with value  \(1\) for \(\key\).
This weak behaviour is known as the \emph{lost update} anomaly, which
is  allowed under the consistency model called \emph{causal consistency},
but not \emph{parallel snapshot isolation} and \emph{snapshot isolation}.

\begin{figure*}[t]
\centering
\captionsetup[subfigure]{aboveskip=-5pt, belowskip=0pt}
\begin{tabularx}{\textwidth}{@{} c @{} | c @{} |  c @{} | c@{}}
\hline
\phantom{-}& \phantom{-}& \phantom{-}& \phantom{-}\\[-5pt]
\begin{subfigure}{0.18\textwidth}
\centering
\begin{centertikz}
%Location x
\node(locx) {$\key \mapsto$};
\draw pic at ([xshift=\tikzkvspace]locx.east) {vlist={versionx}{%
    /0/\txid_0/\emptyset
}};

\end{centertikz}
\caption{Initial state}
\label{fig:counter_kv_initial}
\end{subfigure}
&
\begin{subfigure}{0.22\textwidth}
\begin{centertikz}

%Location x
\node(locx) {$\key \mapsto$};
\draw pic at ([xshift=\tikzkvspace]locx.east) {vlist={versionx}{%
    /0/\txid_0/\Set{\txid}
    , /1/\txid/\emptyset
}};

\end{centertikz}
\caption{After \(\txid \)}
\label{fig:counter_kv_first_inc}
\end{subfigure}
&
\begin{subfigure}{0.28\textwidth}
\begin{centertikz}

%Location x
\node(locx) {$\key \mapsto$};
\draw pic at ([xshift=\tikzkvspace]locx.east) {vlist={versionx}{%
    fillbg/0/\txid_0/\Set{\txid}
    , /1/\txid/\emptyset
}};

\end{centertikz}
\caption{A possible view of \( \cl_2 \)}
\label{fig:counter_kv_view}
\end{subfigure} 
&
\begin{subfigure}{0.28\textwidth}
\begin{centertikz}
    
\node(locx) {$\key \mapsto$};
\draw pic at ([xshift=\tikzkvspace]locx.east) {vlist={versionx}{%
    /0/\txid_0/\Set{\txid,\txid'}
    , /1/\txid/\emptyset
    , /1/\txid'/\emptyset
}};

\end{centertikz}%
\caption{After \( \txid' \), lost update}
\label{fig:counter_kv_final}
%\label{fig:ua-disallowed}
\end{subfigure}\\
\hline
\end{tabularx}
\caption{Example key-value stores (\subref{fig:counter_kv_initial}, \subref{fig:counter_kv_first_inc}, \subref{fig:counter_kv_final}) and a client view (\subref{fig:counter_kv_view})}
\end{figure*}


\mypar{Centralised Operational Semantics}
A well-known declarative approach for a range of consistency models
is to use execution graphs \cite{adya-icde,adya,framework-concur,ev_transactions},
where nodes are atomic transactions and edges describe the
known dependencies between transactions. The graphs capture the
behaviour of the whole program, with different consistency models
corresponding to different sets of axioms ruling out the invalid graphs. 
However, execution graphs provide little information about how the 
state of a kv-store evolves throughout the execution of a program.
By contrast, we provide an interleaving operational semantics based on an
abstract centralised state. The centralised state comprises a
centralised, multi-versioned kv-store, which is {\em global} in the
sense that it contains all the versions written by clients, and client views of the store,
which are {\em partial} in the sense that clients may see different 
subsets of the versions in the kv-store. Each update is given by either
a primitive command or an atomic transaction. The atomic
transaction steps are subject to an {\em execution test} which
analyses the state to determine if the update is allowed by 
the associated consistency model. 


Let us introduce our global kv-stores and partial client views by
showing that we can reproduce the lost update anomaly given by 
\(\prog_{\CodeFont{LU}}\).
Our kv-stores are functions mapping keys to lists of versions, where
the versions  record all the values written to each key together with the
meta-data of the transactions that access it. 
These kv-stores are the abstraction of the machine states of real-world key-value stores.
In the \(\prog_{\CodeFont{LU}}\) example, the initial kv-store comprises a single key \(\key\), with only one initialisation version \((0, \txid_{0}, \emptyset)\).
This version represents the initialisations in both replicas,
stating that \(\key\) holds value \(0\), 
that the version \emph{writer} is the initialising transaction
\(\txid_0\) (this version was written by \(\txid_0\)), 
and that the version \emph{reader set} is empty (no transaction has read this version). 
\Cref{fig:counter_kv_initial} depicts this initial kv-store, with the version
represented as a box sub-divided in three sections: the value \(0\);
the writer \(t_0\); and the reader set \(\emptyset\). 


First, suppose that \(\cl_1\) invokes \(\ctrinc\) on \cref{fig:counter_kv_initial}. 
It does this by choosing a fresh transaction identifier, \(\txid\), 
and then proceeds with \(\ctrinc(\key)\). It reads the initial version
of \(\key\) with value \(0\) 
and then writes a new value \(1\) for \(\key\). 
The resulting kv-store is depicted in \cref{fig:counter_kv_first_inc},
where  the initial version of \(\key\)  has been  updated to reflect that it
has been read by \(\txid\) and a new version with value 1 installed to the end of the list.

Second, client \(\cl_2\) invokes \(\ctrinc\) on
\cref{fig:counter_kv_first_inc}.  As there are now two versions
available for \(\key\), we must determine the version from which
\(\cl_2\) fetches its value, before running \(\ctrinc(\key)\).  This is
where \emph{client views} come into play.  Intuitively, a view of
client \(\cl_2\) comprises those versions in the kv-store that are
\emph{visible} to \(\cl_2\), \ie those that can be read by
\(\cl_2\).  If more than one version is visible, then the newest
(right-most) version is selected, modelling the \emph{last-write-wins}
resolution policy used by many distributed key-value stores.
In our example, there are two view candidates for \(\cl_2\) when running
\(\ctrinc(\key)\) on \cref{fig:counter_kv_first_inc}: 
\begin{enumerate*}
\item one containing
only the initial version of \(\key\) depicted in \cref{fig:counter_kv_view}; and
\item the other containing both versions of \(\key\) depicted in \cref{fig:counter_kv_view_all}%
\footnote{As we explain in \cref{sec:mkvs-view}, we always require
  the view of a client to include the initial version of each key.}.
\end{enumerate*}
Given the view in \cref{fig:counter_kv_view},
client \(\cl_2\) chooses a fresh
transaction identifier \(t'\), reads the initial value \(0\) and writes a
new version with value \(1\), as depicted in \cref{fig:counter_kv_final}. 
Such a kv-store does not contain a
version with value \(2\), despite two increments on \(\key\), producing
the lost update anomaly. 
Given the view in \cref{fig:counter_kv_view_all},
client \(cl_2\) reads the newest
value \(1\) and writes a new version with value \(2\).

To avoid undesirable behaviour, such as the \emph{lost update} anomaly in \cref{fig:counter_kv_final},
we use an \emph{execution test} which restricts the possible update at the
point of the transaction commit.  One such test is to enforce a client
to commit a transaction writing to \(\key\) if and only if its view
contains all versions available in the global state for \(\key\),
captured by the following predicate:
\[ \textstyle \closed[\kvs,\vi,{\bigcup_{(\mathtt{w},\key,\val) \in \fp}\WWInv[\kvs](\key)}] . \]
It means a client with a given view \( \vi \) is allowed to commit a transaction to \( \kvs \) 
if the view is \emph{closed} with respect to a relation 
\( \rel  = {\bigcup_{(\mathtt{w},\key,\val) \in \fp}\WWInv[\kvs](\key)} \),
in the sense that:
if a view \( \vi \) includes a versions written by a transaction \( \txid \), 
and if there is an edge \( (\txid',\txid) \in \rel \),
then the view \( \vi \)  must also include versions written by \( \txid' \).
The \emph{fingerprint} \( \fp \) of a transaction is 
the read (label \( \mathtt{r} \)) and  write (label \( \mathtt{w} \)) set of the transaction.
The write-write relation on a key, for example \( (\txid_0,\txid) \in \WW[\kvs](\key) \) in \cref{fig:counter_kv_first_inc},
means that a transaction \( \txid \) overwrites a version of \( \key \) written by another transaction \( \txid_0 \).
Note that \( \rel^{-1} \) denotes the reverse of the relation \( \rel \).
This \( \Pred{closed} \) predicate prevents \(\cl_2\) from running \(\ctrinc(\key)\) on
\cref{fig:counter_kv_first_inc} if its view only contains the initial
version of \(\key\).  Instead, the \(\cl_2\) view must contain both
versions of \(\key\), thus enforcing \(\cl_2\) to write a version with
value \(2\) after running \(\ctrinc(\key)\). This particular test
corresponds to \emph{write-conflict freedom}:
at most one concurrent transaction can write to a key at any one time.

The situation becomes more complicated when the library contains multiple counters
where each client can read and increments several counters in one session.
For instance, consider the following program:
\[
    \prog_{\CodeFont{LF}} \defeq 
    \begin{multlined}[t]
    \cl_1 : \ptrans{\plookup{\var}{\key_1} ; \pmutate{\key_1}{\var + 1 }} ; 
                \ptrans{\plookup{\var(y)}{\key_2} ; \pmutate{\key_2}{\var(y) + 1} }
        \\ || \ \cl_2: \ptrans{\plookup{\var}{\key_1} ; \plookup{\var(y)}{\key_2} }
                 || \ \cl_3:  \ptrans{\plookup{\var}{\key_1} ; \plookup{\var(y)}{\key_2} } .
    \end{multlined}
\]
For simplicity, we assume that the initial kv-store contains two keys (\cref{fig:overview-sec-long-fork-init}).
Suppose that \(\cl_1\) executes first the transactions \( \txid \) and \( \txid' \)
that updates \(\key_1\) and \(\key_2\) to values \(1\) respectively.
This results in \(\key_1\) and \(\key_2\) having two versions with values \(0\) and \(1\) each. 
Client \(\cl_2\) executes its transaction \( \txid_2 \), using a view that 
contains both versions of \(\key_1\), but only the initial version of \(\key_2\) and therefore
client \(\cl_2\) reads \(1\) for \(\key_1\) and \(0\) for \(\key_2\): that is, 
\(\cl_2\) observes the update of \(\key_1\) happening before the increment of \(\key_2\). 
Finally, \(\cl_3\) executes its transaction \( \txid_3 \) using a view that contains both versions for \(\key_2\), 
but only the initial version of \(\key_1\) and therefore
client \(\cl_3\) reads \(0\) for \(\key_1\) and \(1\) for \(\key_2\),
that is, \(\cl_3\) observes the increment of \(\key_2\) happening before the increment of \(\key_1\). 
This behaviour is known as the \emph{long fork} anomaly (\cref{fig:overview-sec-long-fork}). 

\begin{figure}

\begin{tabularx}{\textwidth}{@{} c | c @{} }
\hline
\phantom{-}& \phantom{-}\\[-5pt]
\begin{subfigure}{0.39\textwidth}
\centering
\scalebox{.8}{%
\begin{tikzpicture}%
\KVMapping{x}{\key_1}{
    /0/\txidinit/\emptyset
};
\KVMapping[x]{y}{\key_2}{
    /0/\txidinit/\emptyset
};
\end{tikzpicture}%
}
\caption{Initial kv-store}
\label{fig:overview-sec-long-fork-init}
\end{subfigure}
& \begin{subfigure}{0.58\textwidth}
\centering
\scalebox{.8}{%
\begin{tikzpicture}%
\KVMapping{x}{\key_1}{
    /0/\txidinit/\Set{\txid,\txid_3}
    , /1/\txid/\Set{\txid_2}
};
\KVMapping[x]{y}{\key_2}{
    /0/\txidinit/\Set{\txid',\txid_2}
    , /1/\txid'/\Set{\txid_3}
};
\end{tikzpicture}%
}
\caption{Transactions \( \txid_2 \) and \( \txid_3\)
            observe the update to \( \key_1 \) and \( \key_2 \) 
            in different order (\emph{long fork} anomaly)}
\label{fig:overview-sec-long-fork}
\end{subfigure}
\\
\hline
\end{tabularx} 

\begin{subfigure}{\textwidth}
\centering
\begin{tikzpicture}

\node at (0,0) (a) {\(\txidinit\)};
\node at (2,-0.5) (b) {\(\txid\)};
\node at (4.5,0) (c) {\(\txid_3\)};

\coordinate (a1) at ($(a) + (3,0)$);
\coordinate (b1) at ($(b) + (7,0)$);
\coordinate (c1) at ($(c) + (3,0)$);
%\coordinate (d1) at ($(d) + (3,0)$);

\draw[|-,densely dashed] (a) -- ($(a)!0.5!(a1)$);
\draw[-|] ($(a)!0.5!(a1)$) -- (a1);
\draw[|-,densely dashed] (b) -- ($(b)!0.5!(b1)$);
\draw[-|] ($(b)!0.5!(b1)$) -- (b1);
\draw[|-] (c) -- ($(c)!0.5!(c1)$);
\draw[-|,densely dashed] ($(c)!0.5!(c1)$) -- (c1);
%\draw[|-|] (d) -- (d1);

%\path[->, thick] (a1) edge[bend left=50] node[above, text opacity=1] {$\WR$} ($(c)+(0.3,0.05)$);
%\path[->, thick] (a1) edge node[below, text opacity=1] {$\WW$} (b1);
%\path[->, thick] (c) edge node[above, text opacity=1] {$\RW$} (b1);

\end{tikzpicture}

\caption{An example of dependencies between transactions with respect to 
the time line of the starts and commits of these transactions 
(dashed line being able to stretched)}
\label{fig:overview-dependencies-time-line}

\end{subfigure}

\hrulefill


\caption{Long fork anomaly: multiple counters}
\label{fig:mult-counter}
\end{figure}


The long fork anomaly is disallowed under strong models 
such as serialisability (\(\SER\)) and snapshot isolation (\(\SI\)), 
but is allowed under weaker models such as parallel snapshot isolation (\(\PSI\)), causal consistency (\(\CC\)) and update atomic (\( \UA \)).
To capture such consistency models and rule out the long fork anomaly as a possible result 
of \(\prog_{\CodeFont{LF}}\), we must strengthen the execution test associated with the kv-store.
For \(\SER\), we strengthen the execution test by ensuring that a client can execute a transaction 
only if its view contains all versions available in the global state, 
captured by the following predicate \( \closed[\kvs,\vi,{\WWInv[\kvs]}]  \),
which means the view \( \vi \) before commit the transaction must contains all versions in \( \kvs \).
For \(\SI\), the execution test recovers the order in which 
updates of versions have been observed by different clients:
\[
    \PreClosed(\kvs,\vi,\Trasi(\left( \WR[\kvs] \cup \SO \cup \WW[\kvs] \right)) ; \Refl(\RW[\kvs]) ) 
\]
where: 
\begin{enumerate*} 
\item write-read dependency relation,
such as \( (\txid,\txid'') \in \WR[\kvs] \)
in \cref{fig:overview-sec-long-fork}, states that
a transaction, \( \txid \), reads a version written by another transaction, \( \txid'' \);
\item session order, \( \SO \), determines, for each client \( \cl \),
the commit order of transactions of \( \cl \); and
\item read-write anti-dependency relation, 
such as \( (\txid_3,\txid) \in \RW[\kvs] \),
in \cref{fig:overview-sec-long-fork}, states that
a transaction, \( \txid_3 \), reads a version that has been overwritten by another transaction, \( \txid \).
\end{enumerate*} 
Given this strengthen, client \( \cl_3 \) must
observe the second version of \( \key_1 \),
because \( \ToEdge{\txid | \WR[\kvs] -> \txid'' | \RW[\kvs] -> \txid' } \)
Under such strengthened execution tests for \(\SER \) and \( \SI \), 
in the \( \prog_{\CodeFont{LF}} \) example,
\(\cl_2\) cannot observe \(1\) for \(\key_2\) after observing \(0\) for \(\key_1\),
if \(\cl_1\) has already established that the increment on \(\key_2\) happens after 
the one of \(\key_1\). 
%We will give more detail about the formal definitions of many execution tests 
%for well-known consistency models in \cref{sec:model}.

In \cref{sec:cm}, we give many examples of execution tests and their
associated consistency models on kv-stores. 
\ifTechRepEdits%
In the technical report,
\else%
In \cref{app:et_sound_complete},
\fi
we show the equivalence of our operational definitions of consistency models and 
the declarative ones based on abstract executions. 

\mypar{Verifying Implementation Protocols} 
The first application of our operational
semantics is for showing that implementations of distributed
key-value stores satisfy certain consistency models. 
Our abstract states provide a 
faithful abstraction of replicated and partitioned
databases, and execution tests provide a powerful abstraction of the synchronisation mechanisms 
enforced by these databases when committing a transaction. 
This then allows us to use our 
formalism to verify the correctness of distributed database protocols,
via trace refinement.
To demonstrate this, we show that the
COPS protocol \citep{cops} for implementing a replicated database 
satisfies causal consistency  
(%
\ifTechRepEdits%
\cref{sec:verify-impl} and the technical report%
\else%
\cref{sec:verify-impl,sec:cops}%
\fi%
), 
and the Clock-SI protocol \citep{clocksi} for implementing a
partitioned database satisfies snapshot isolation
(%
\ifTechRepEdits%
the technical report%
\else%
\cref{sec:cops}%
\fi%
). 

\mypar{Proving Invariant Properties of Client Programs} 
The second application of our operational semantics is to prove
invariant properties of client programs (\cref{sec:robustness}).
One property is the robustness for client programs.
A program \(\prog\) is robust if any kv-stores obtained 
by executing \(\prog\) under a weak consistency model can also be obtained under serialisability.
To demonstrate this, we prove the robustness of the single
counter library discussed above against \(\PSI\), 
and the robustness of a multi-counter library and the banking library of \citet{bank-example-wsi}
against \(\SI\).
The latter is done through general conditions on invariant
which guarantees robustness against our new proposed model \( \WSI \),
and hence implies robustness of nay stronger models such as  \( \SI \).
Apart from robustness,
we show that a lock paradigm is correct under \( \UA \), 
although it is not robust.
Thanks to our operational semantics, 
our invariant-based approaches only need to work with single program steps 
rather than whole program traces.
