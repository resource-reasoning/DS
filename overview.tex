\section{Overview}
\label{sec:overview}

Suppose that we have an integer valued key-value store. Four different 
clients from the set $\{\cl_i\}_{i=1}^{4}$ interact with the key-value store by requesting 
to execute transactions from the set $\{\trans_{i}\}_{i=1}^{4}$, 
whose code is defined below, respectively:

\sx{
    Maybe present in the following way, and we can say that
    we assume the allocated transaction identifiers \( \txid \) are the ones shown in the code.
    Also clients are like threads etc.
\[
\begin{array}{@{} l @{} }
\begin{array}{@{} c ||  c @{} }
\begin{session}
\txid_1 :
\begin{transaction}
\pmutate{\ke_{1}}{1}
\end{transaction} 
\end{session}
&
\begin{session}
\txid_2 :
\begin{transaction}
\pmutate{\ke_2}{1}
\end{transaction}
\end{session}
\end{array} \\[1em]
\qqqquad \begin{array}{@{} || c || c @{} }
\txid_3 :
\begin{session}
\begin{transaction}
\pderef{\pv{a}}{\ke_1};
\quad \pderef{\pv{b}}{\ke_2};\\
\pifs{\pv{a} = 0 \wedge \pv{b} = 1} \\
\quad \pmutate{\ke_{3}}{1}
\pife
\end{transaction}
\end{session}
&
\txid_4 :
\begin{session}
\begin{transaction}
\pderef{\pv{a}}{\ke_1};
\quad \pderef{\pv{b}}{\ke_2};\\
\pifs{\pv{a} = 1 \wedge \pv{b} = 0}  \\
\quad \pmutate{\ke_{4}}{1}
\pife
\end{transaction}
\end{session}
\end{array}
\end{array}
\]
}
\[
\begin{array}{lrlr}
\trans_{1} = &
\begin{session}
\begin{transaction}
\pmutate{\ke_{1}}{1}
\end{transaction}
\end{session}
&
\trans_{2} = &
\begin{session}
\begin{transaction}
\pmutate{\ke_2}{1}
\end{transaction}
\end{session}
\\[5pt]
\trans_{3} = &
\begin{session}
\begin{transaction}
\pderef{\pv{a}}{\ke_1};\\
\pderef{\pv{b}}{\ke_2};\\
\pifs{\pv{a} = 0 \wedge \pv{b} = 1} 
 \pmutate{\ke_{3}}{1}
\pife
\end{transaction}
\end{session}
&
\trans_{4} = &
\begin{session}
\begin{transaction}
\pderef{\pv{a}}{\ke_1};\\
\pderef{\pv{b}}{\ke_2};\\
\pifs{\pv{a} = 1 \wedge \pv{b} = 0} 
 \pmutate{\ke_{4}}{1}
\pife
\end{transaction}
\end{session}
\end{array}
\]
Here and in the rest of the paper, we use 
brackets to wrap fragments of code into transactions, 
which should appear as being executed atomically by the clients of the key-value store. 
Assume that initially $\ke_{i} = 0$ for $i=1,\cdots,4$.
The semantics of the set of transactions above 
depends from the data consistency model guaranteed by the key-value store. 
For example, if the key-value store is serialisable \cite{papadimitriou}, i.e. 
transactions are executed according to some serial order $\rightarrow$, 
then it is impossible to schedule the execution of the transactions $\{\trans_{i}\}_{i=1}^{4}$ 
as to obtain $\ke_3 = \ke_4 = 1$. 
If $\ke_3$ is updated to value $1$, then it must be the case that 
$\trans_3$ has been executed after  $\trans_2$, but before $\trans_1$: 
$\trans_2 \rightarrow \trans_{3} \rightarrow \trans_1$.
Similarly, the value of $\ke_4$ is updated 
to $1$ only if $\trans_4$ 
is executed after $\trans_1$, 
but before $\trans_2$: $\trans_{1} \rightarrow \trans{4} \rightarrow \trans_{2}$. 
Then, when both $\ke_3, \ke_4$ are updated to value $1$, we must have that 
$\trans_1 \rightarrow \trans_4 \rightarrow \trans_2 \rightarrow \trans_3 
\rightarrow \trans_1$, thus violating the serialisability requirement imposed by 
the key-value store. 
%A similar argument applies even it the key-value stores 
%is assumed to guarantee \emph{Snapshot Isolation} \cite{ansisql}, a weaker consistency model 
%provided by several centralised, geo-replicated and distributed databases \cite{ansisql,dias-tm,gsi,clocksi,distrsi}.

However, executing the set of transactions $\{\trans_{i}\}_{i=1}^{4}$ depicted above 
may lead to both $\ke_3 = 1$ and $\ke_4 = 1$ if a weaker consistency model than 
serialisability is employed by the key-value store. This is true especially in the case 
that the key-value store is replicated at several sites, and little to no coordination 
between replicas is present when clients commit transactions: examples of such geo-replicated 
datastores are those that provide the \emph{Causal Consistency} model \cite{cops}. 
For example, transactions $\trans_1$ and $\trans_4$ may be executed sequentially 
on a replica of the key value store, leading to $\ke_4 = 1$, while transactions 
$\trans_2$ and $\trans_3$ may be executed sequentially on a different replica, leading to $\ke_3 = 1$. 
When each replica communicates the state updates to the other, both of them will 
end up in a state where $\ke_3 = \ke_4 = 1$. 


\begin{figure*}[t]
\begin{center}
%
%\hrulefill
\begin{subfigure}{0.32\textwidth}
\begin{tikzpicture}
\begin{pgfonlayer}{foreground}
%\draw[help lines] grid(5,4);

%Location x
\node(locx) {$\ke_1 \mapsto$};
\matrix(versionx) [version list]
    at ([xshift=\tikzkvspace]locx.east) {
    {a} & $\txid_0$ \\
    {a} & $\emptyset$ \\
};
\tikzvalue{versionx-1-1}{versionx-2-1}{locx-v0}{$0$};

%Location y
\path (locx.south) + (0,\tikzkeyspace) node (locy) {$\ke_2 \mapsto$};
\matrix(versiony) [version list]
   at ([xshift=\tikzkvspace]locy.east) {
 {a} & $\txid_0$\\
  {a} & $\emptyset$ \\
};
\tikzvalue{versiony-1-1}{versiony-2-1}{locy-v0}{$0$};

\path (locy.south) + (0,\tikzkeyspace) node (locz) {$\ke_3 \mapsto$};
\matrix(versionz) [version list]
   at ([xshift=\tikzkvspace]locz.east) {
 {a} & $\txid_0$\\
  {a} & $\emptyset$ \\
};
\tikzvalue{versionz-1-1}{versionz-2-1}{locz-v0}{$0$};

\path (locz.south) + (0,\tikzkeyspace) node (locm) {$\ke_4 \mapsto$};
\matrix(versionm) [version list]
   at ([xshift=\tikzkvspace]locm.east) {
 {a} & $\txid_0$\\
  {a} & $\emptyset$ \\
};
\tikzvalue{versionm-1-1}{versionm-2-1}{locm-v0}{$0$};

\end{pgfonlayer}
\end{tikzpicture}
\caption{Key-value store}
\label{fig:hheap-a}
\label{fig:key-value-store}
\label{fig:initial-kv}
\end{subfigure}
%
%
\begin{subfigure}{0.32\textwidth}
\begin{tikzpicture}
\begin{pgfonlayer}{foreground}
%\draw[help lines] grid(5,4);

%Location x
\node(locx) {$\ke_1 \mapsto$};
\matrix(versionx) [version list, column 2/.style={text width=8mm}, column 4/.style={text width=8mm}]
    at ([xshift=\tikzkvspace]locx.east) {
        {a} & $\txid_0$   & {a} & $\txid_1$ \\
        {a} & $\emptyset$ & {a} & $\Set{\txid_{\cl_4}}$\\
};
\tikzvalue{versionx-1-1}{versionx-2-1}{locx-v0}{$0$};
\tikzvalue{versionx-1-3}{versionx-2-3}{locx-v1}{$1$};

%Location y
\path (locx.south) + (0,\tikzkeyspace) node (locy) {$\ke_2 \mapsto$};
\matrix(versiony) [version list, column 2/.style={text width=8mm}]
   at ([xshift=\tikzkvspace]locy.east) {
 {a} & $\txid_0$\\
  {a} & $\Set{\txid_{\cl_4}}$ \\
};
\tikzvalue{versiony-1-1}{versiony-2-1}{locy-v0}{$0$};

\path (locy.south) + (0,\tikzkeyspace) node (locz) {$\ke_3 \mapsto$};
\matrix(versionz) [version list, column 2/.style={text width=8mm}]
   at ([xshift=\tikzkvspace]locz.east) {
 {a} & $\txid_0$\\
  {a} & $\emptyset$ \\
};
\tikzvalue{versionz-1-1}{versionz-2-1}{locz-v0}{$0$};

\path (locz.south) + (0,\tikzkeyspace) node (locm) {$\ke_4 \mapsto$};
\matrix(versionm) [version list, column 2/.style={text width=8mm}, column 4/.style={text width=8mm}]
   at ([xshift=\tikzkvspace]locm.east) {
  {a} & $\txid_0$ & {a} & $\txid_{\cl_4}$ \\              
  {a} & $\emptyset$  & {a} & $\emptyset$\\  
};
\tikzvalue{versionm-1-1}{versionm-2-1}{locm-v0}{$0$};
\tikzvalue{versionm-1-3}{versionm-2-3}{locm-v0}{$1$};

\end{pgfonlayer}
\end{tikzpicture}
\caption{\( \txid_4 \)}
\label{fig:long-fork-t4}
\end{subfigure}
%
%
%
\begin{subfigure}{0.32\textwidth}
\begin{tikzpicture}
\begin{pgfonlayer}{foreground}
%\draw[help lines] grid(5,4);

%Location x
\node(locx) {$\ke_1 \mapsto$};
\matrix(versionx) [version list, column 2/.style={text width=8mm}, column 4/.style={text width=8mm}]
    at ([xshift=\tikzkvspace]locx.east) {
        {a} & $\txid_0$   & {a} & $\txid_1$ \\
        {a} & $\emptyset$ & {a} & $\Set{\txid_{\cl_4}}$\\
};
\tikzvalue{versionx-1-1}{versionx-2-1}{locx-v0}{$0$};
\tikzvalue{versionx-1-3}{versionx-2-3}{locx-v1}{$1$};

%Location y
\path (locx.south) + (0,\tikzkeyspace) node (locy) {$\ke_2 \mapsto$};
\matrix(versiony) [version list, column 2/.style={text width=8mm}, column 4/.style={text width=8mm}]
   at ([xshift=\tikzkvspace]locy.east) {
       {a} & $\txid_0$ & {a} & $\txid_{\cl_2}$ \\
       {a} & $\Set{\txid_{\cl_4}}$ & {a} & $\emptyset$ \\
};
\tikzvalue{versiony-1-1}{versiony-2-1}{locy-v0}{$0$};
\tikzvalue{versiony-1-3}{versiony-2-3}{locy-v1}{$1$};

\path (locy.south) + (0,\tikzkeyspace) node (locz) {$\ke_3 \mapsto$};
\matrix(versionz) [version list, column 2/.style={text width=8mm}]
   at ([xshift=\tikzkvspace]locz.east) {
 {a} & $\txid_0$\\
  {a} & $\emptyset$ \\
};
\tikzvalue{versionz-1-1}{versionz-2-1}{locz-v0}{$0$};

\path (locz.south) + (0,\tikzkeyspace) node (locm) {$\ke_4 \mapsto$};
\matrix(versionm) [version list, column 2/.style={text width=8mm}, column 4/.style={text width=8mm}]
   at ([xshift=\tikzkvspace]locm.east) {
  {a} & $\txid_0$ & {a} & $\txid_{\cl_4}$ \\              
  {a} & $\emptyset$  & {a} & $\emptyset$\\  
};
\tikzvalue{versionm-1-1}{versionm-2-1}{locm-v0}{$0$};
\tikzvalue{versionm-1-3}{versionm-2-3}{locm-v0}{$1$};

\end{pgfonlayer}
\end{tikzpicture}
\caption{\( \txid_2 \)}
\label{fig:long-fork-t2}
\end{subfigure}
%
%
%
\begin{subfigure}{0.32\textwidth}
\begin{tikzpicture}
\begin{pgfonlayer}{foreground}
%\draw[help lines] grid(5,4);

%Location x
\node(locx) {$\ke_1 \mapsto$};
\matrix(versionx) [version list, column 2/.style={text width=8mm}, column 3/.style={dotted}, column 4/.style={dotted,text width=8mm}]
    at ([xshift=\tikzkvspace]locx.east) {
        {a} & $\txid_0$   & {a} & \color{gray}$\txid_1$ \\
        {a} & $\emptyset$ & {a} & \color{gray}$\Set{\txid_{\cl_4}}$\\
};
\tikzvalue{versionx-1-1}{versionx-2-1}{locx-v0}{$0$};
\node[version node,draw=none,fit=(versionx-1-3) (versionx-2-3),fill=white, inner sep=0pt] (locx-v1) {\color{gray}$1$};

%Location y
\path (locx.south) + (0,\tikzkeyspace) node (locy) {$\ke_2 \mapsto$};
\matrix(versiony) [version list, column 2/.style={text width=8mm}, column 4/.style={text width=8mm}]
   at ([xshift=\tikzkvspace]locy.east) {
       {a} & $\txid_0$ & {a} & $\txid_{\cl_2}$ \\
       {a} & $\Set{\txid_{\cl_4}}$ & {a} & $\emptyset$ \\
};
\tikzvalue{versiony-1-1}{versiony-2-1}{locy-v0}{$0$};
\tikzvalue{versiony-1-3}{versiony-2-3}{locy-v1}{$1$};

\path (locy.south) + (0,\tikzkeyspace) node (locz) {$\ke_3 \mapsto$};
\matrix(versionz) [version list, column 2/.style={text width=8mm}]
   at ([xshift=\tikzkvspace]locz.east) {
 {a} & $\txid_0$\\
  {a} & $\emptyset$ \\
};
\tikzvalue{versionz-1-1}{versionz-2-1}{locz-v0}{$0$};

\path (locz.south) + (0,\tikzkeyspace) node (locm) {$\ke_4 \mapsto$};
\matrix(versionm) [version list, column 2/.style={text width=8mm}, column 3/.style={dotted}, column 4/.style={text width=8mm,dotted}]
   at ([xshift=\tikzkvspace]locm.east) {
  {a} & $\txid_0$ & {a} & \color{gray}$\txid_{\cl_4}$ \\              
  {a} & $\emptyset$  & {a} &  \color{gray}$\emptyset$\\  
};
\tikzvalue{versionm-1-1}{versionm-2-1}{locm-v0}{$0$};
\node[version node,draw=none,fit=(versionm-1-3) (versionm-2-3),fill=white, inner sep=0pt] (locx-v1) {\color{gray}$1$};

\end{pgfonlayer}
\end{tikzpicture}
\caption{view of \( \txid_3 \)}
\label{fig:view-of-cl-prime}
\label{fig:long-fork-t3-view}
\end{subfigure}
%
%
%
%
\begin{subfigure}{0.32\textwidth}
\begin{tikzpicture}
\begin{pgfonlayer}{foreground}
%\draw[help lines] grid(5,4);

%Location x
\node(locx) {$\ke_1 \mapsto$};
\matrix(versionx) [version list, column 2/.style={text width=8mm}, column 4/.style={text width=8mm}]
    at ([xshift=\tikzkvspace]locx.east) {
        {a} & $\txid_0$   & {a} & $\txid_1$ \\
        {a} & $\Set{\txid_{\cl_3}}$ & {a} & $\Set{\txid_{\cl_4}}$\\
};
\tikzvalue{versionx-1-1}{versionx-2-1}{locx-v0}{$0$};
\tikzvalue{versionx-1-3}{versionx-2-3}{locx-v1}{$1$};

%Location y
\path (locx.south) + (0,\tikzkeyspace) node (locy) {$\ke_2 \mapsto$};
\matrix(versiony) [version list, column 2/.style={text width=8mm}, column 4/.style={text width=8mm}]
   at ([xshift=\tikzkvspace]locy.east) {
       {a} & $\txid_0$ & {a} & $\txid_{\cl_2}$ \\
       {a} & $\Set{\txid_{\cl_4}}$ & {a} & $\Set{\txid_{\cl_3}}$ \\
};
\tikzvalue{versiony-1-1}{versiony-2-1}{locy-v0}{$0$};
\tikzvalue{versiony-1-3}{versiony-2-3}{locy-v1}{$1$};

\path (locy.south) + (0,\tikzkeyspace) node (locz) {$\ke_3 \mapsto$};
\matrix(versionz) [version list, column 2/.style={text width=8mm}, column 4/.style={text width=8mm}]
   at ([xshift=\tikzkvspace]locz.east) {
       {a} & $\txid_0$ & {a} & $\txid_{\cl_3}$ \\
       {a} & $\emptyset$ & {a} & $\emptyset$ \\
};
\tikzvalue{versionz-1-1}{versionz-2-1}{locz-v0}{$0$};
\tikzvalue{versionz-1-3}{versionz-2-3}{locz-v1}{$1$};

\path (locz.south) + (0,\tikzkeyspace) node (locm) {$\ke_4 \mapsto$};
\matrix(versionm) [version list, column 2/.style={text width=8mm}, column 4/.style={text width=8mm}]
   at ([xshift=\tikzkvspace]locm.east) {
  {a} & $\txid_0$ & {a} & $\txid_{\cl_4}$ \\              
  {a} & $\emptyset$  & {a} & $\emptyset$\\  
};
\tikzvalue{versionm-1-1}{versionm-2-1}{locm-v0}{$0$};
\tikzvalue{versionm-1-3}{versionm-2-3}{locm-v0}{$1$};

\end{pgfonlayer}
\end{tikzpicture}
\caption{\( \txid_3 \)}
\label{fig:long-fork-t3}
\label{fig:cp-disallowed-1}
\end{subfigure}

\end{center}

\hrulefill

\caption{Multi-version key-value stores and views}
\label{fig:hheap}
\label{fig:key-value-and-view}
\end{figure*}

It has been observed before that reasoning about the behaviour of a replicated data-store at a low level 
may be excessively complicated, as it requires the knowledge of the state of each 
replica, as well as understanding the synchronisation protocol among replicas \cite{framework-concur,adya,seebelieve}. 
\ac{After trying to write this paragraph several time, comparing with the three proposals in the references: 
do not apologise here about why another framework when there are already other three around. This should 
be done in introduction and conclusions. Just give the details about how we do stuff.} 
We propose an approach where we study the semantics to programs in a setting where 
clients interact with a centralised (i.e. non-replicated, non-distributed), multi-version key value store, 
using a mechanism that we refer to as \emph{views}: in \cref{fig:view-of-cl} we show that this approach 
is sufficient to capture the behaviour of programs under several consistency models 
employed by distributed transactional storage systems.

Let us show how, in our model, it is possible to exhibit a computation of 
the set of transactions $\{\trans_{i}\}_{i=1}^{4}$ that leads to updating 
both $\ke_3$ and $\ke_4$ to $1$.
In our model, when a piece of code corresponding to a transaction executes, 
it gets assigned a \emph{transaction identifier} $\txid, \txid',\cdots$. 
As we will see, transaction identifiers 
embed the information about the client that executes them. 
The state of the system consists of a mapping from keys to non-empty lists of versions, each 
of which comprises value and the meta-data of the transaction identifiers 
that read and wrote such a version. Initially, each key is initialised to a singleton carrying 
value $0$, written by a special initialisation transaction $\txid_0$ that is not associated 
with any client. This key-value store is depicted graphically in \cref{fig:key-value-store}. Versions are represented graphically
as boxes divided in three containers: the leftmost container contains the value of the version, 
the right-top container contains the identifier of the transaction that wrote the version, 
and the right-bottom container contains the set of transaction identifiers that read the version. 
Initially, one of the four clients from $\{\cl_{i}\}_{i=1}^{4}$, say 
$\cl_1$, executes the transaction $\trans_1$. To do so, the client 
must first select the portion of state that it observes from the key-value store: 
specifically, it must select a non-empty sub-list of versions of each key that it observes from 
the key-value store. The portion of state selected by the client is referred to as its view, 
and it determines the local state in which the transactional code $\trans_1$ is executed.
Because there is only one version for each key, each carrying value $0$, 
the choice of view for $\cl_1$ is compulsory, 
and it leads to client $\cl_1$ executing the transaction $\trans_1$ in the local state 
where $\ke_1 = \ke_2 = \ke_3 = \ke_4 = 0$. This transaction gets assigned an identifier 
that also carries the information about the client that executes the transaction $\txid_{\cl_{1}}^{1}$ 
(ignore for the moment the superscript $1$). Upon executing $\trans_1$ 
atomically, a new version carrying value $1$ and written by $\txid_{\cl_{1}}^{1}$ 
is created for $\ke_1$. 

Next, client $\cl_4$ is scheduled to execute a transaction. As for $\cl_1$, 
we must select a view for $\cl_4$, which we use to determine the state in which $\trans_4$ 
is executed. The choice for which versions to observe for $\ke_2, \ke_3, \ke_4$ is compulsory, 
and it leads to execute $\trans_4$ in a state where $\ke_2 = \ke_3 = \ke_4 = 0$.
Because we have two versions for $\ke_1$, carrying values $0$ and $1$ respectively, 
there are three different choices for selecting the view that $\cl_4$ has on $\ke_1$: 
only the initial version carrying value $0$ is observed, in which case $\trans_4$ executes in 
a state where $\ke_1 = 0$; or the final version carrying value $1$ is observed, in which 
case $\trans_4$ executes in a state where $\ke_1 = 1$; orboth versions for 
$\ke_1$ are observed: because the version of $\ke_1$, carrying value $1$ is more recent than the one 
carrying value $0$ (i.e. it appears after in the list of versions for $\ke_1$), then $\trans_4$ 
executes in a state in which $\ke_1 = 1$. This approach models the \emph{Last Write Wins} 
policy employed by many transactional key-value stores \cite{cassandra,...........}. If we assume that the view 
of $\trans_4$ comprises both versions for $\ke_1$, then the execution of $\trans_4$ 
results in reading the only version available for $\ke_2$ (carrying value $0$), the last version available 
for $\ke_1$, and in the creation of a new version for $\ke_4$ carrying value $1$. If we associate 
the transaction identifier $\txid_{\cl_{4}}^1$ to the execution of $\trans_4$, the resulting 
key-value store is the one depicted in \cref{fig:long-fork-t4}.

Next, client $\cl_2$ executes transaction $\trans_2$. Independently from the view that 
$\cl_2$ chooses to execute this transaction, the outcome will be a new version carrying 
value $1$ to be installed for value $\ke_1$. The identifier of the transaction installing such 
a version is given by $\txid_{\cl_2}{1}$. The resulting key-value store is given in \cref{fig:long-fork-t2}. 
Finally, client $\cl_3$ executes transaction $\trans_3$. Prior to executing a transaction, client 
$\cl_3$ selects a view of the key-value store where that comprises only the initial version of $\ke_1$, 
and both versions for $\ke_2$. In \cref{fig:long-fork-t3-view} we show the same key-value store from \cref{fig:long-fork-t2},  
where we highlight the versions that are included in the view of $\cl_3$. The choice of view for $\cl_3$ 
mandates that the transaction $\txid_3$ is executed in a state where $\ke_1 = 0, \ke_2 = 1$, which 
means that a new version carrying value $1$ for $\ke_4$ is installed by the execution of $\trans_3$. 
If $\txid_{\cl_3}^{1}$ is the transaction identifier associated with the execution of $\trans_3$, then the resulting 
key value store is given in \cref{fig:long-fork-t3}.

The example above implicitly assumes that the key-value store employs a very weak consistency model, 
where transactions are always allowed to execute. To express stronger consistency models, 
we introduce a notion of \emph{execution tests}, which restrict the set of possible scenarios in which 
a transaction is allowed to execute in a given initial state. For example, the execution test for 
(strict) serialisability mandates that the only view that a client can select, prior to executing a transaction, 
is the one that comprises all versions of all keys: this amounts to requiring that the client observes the totality 
of the state of the key-value store. If we were to replay the example from above using the execution test 
for serialisability, we would notice that the execution of transaction $\trans_3$ starting from a state 
where $\ke_1 = 0, \ke_2 = 1$, would not be possible anymore: because at the moment $\trans_3$ is 
executed the key-value store contains a version carrying value $1$ for $\ke_1$, this version must be 
included in the view used by $\cl_3$ to determine the state in which $\trans_3$ is executed, leading 
to $\trans_3$ being executed in a state where $\ke_1 = \ke_2 = 1$. A consequence of this fact is 
that the execution of $\trans_3$ will not install a new version for $\ke_4$ anymore. Furthermore, 
if the execution test for serialisability is enforced by the key-value store, there exists no scheduling 
of the transactions $\{\trans_i\}_{i=1}^{4}$ that leads to installing a version carrying value 
$1$ for both $\ke_3$ and $\ke_4$.

%\ac{
%
%    \sx{Done}
%    Pictures will need to be drawn: Shale, do you mind doing this? 
%For each figure I describe the updates that need to be made w.r.t to the previous 
%Figure.\\
%First picture: 4 keys $\ke_1, \ke_2, \ke_3, \ke_4$, 
%all initialised to 0 by $\txid_{0}$ and with no read transactions.\\
%Second picture: a version with value $1$, written by $\txid_{\cl^{1}}^1$ 
%and read by no transaction, is inserted for $\ke_1$. Second picture: $\txid_4$.
%reads value $1$ for $\ke_1$, value $0$ for $\ke_2$, and writes value $1$ for $\ke_4$. 
%Third picture: transaction $\txid_{\cl_2}^{1}$ writes value $1$ for $\ke_2$. 
%Fourth picture: same as the third picture, but the version with value $1$ of $\ke_1$ 
%should be dotted (as to represent the view of $\txid_{\cl_3}^{3}$. \\
%Fifth picture: $\txid_{\cl_3}^{1}$ reads value $1$ for $\ke_2$, value $0$ for $\ke_1$, 
%and updates $\ke_4$ to $1$.}


%\begin{figure}[!ht]
\hrule
\[
\intass :
\begin{array}[t]{@{} c @{\quad} c @{} }
\begin{rclarray}[t]
    \CB{L1} & : & \key_1 \fpW 1 \sep \null \fpA \cass{\CB{L1}}{\lrid} \\
\end{rclarray}
&
\begin{rclarray}[t]
    \CB{R1} & : & \key_2 \fpW 1 \sep \null \fpA \cass{\CB{R1}}{\lrid} \\
\end{rclarray}
\\
\begin{rclarray}[t]
    \CB{L2} & : & \key_1 \fpR 1 \sep \key_2 \fpR 0 \sep \key_3 \fpW 1 \sep \null \fpA \cass{\CB{L2}}{\lrid} \\
\end{rclarray}
&
\begin{rclarray}[t]
    \CB{R2} & : & \key_1 \fpR 0 \sep \key_2 \fpR 1  \sep \key_4 \fpW 1 \sep \null \fpA \cass{\CB{R2}}{\lrid} \\
\end{rclarray}  
\end{array}
\]


\hrule\vspace{5pt}
\centering
\scalebox{.9}{%
\(
\begin{session}
    \specline{ \cass{\CB{L1}}{\lrid} \sep \cass{\CB{L2}}{\lrid} \sep \cass{\CB{R1}}{\lrid} \sep \cass{\CB{R2}}{\lrid} \sep \boxass{\key_1 \pt 0 \sep \key_2 \pt 0 \sep \key_3 \pt 0 \sep \key_4 \pt 0}{\lrid}{\intass}  } \\ \\
\begin{array}{@{}l@{}}
\begin{parl}
\begin{session}
    \specline{ 
        \cass{\CB{L1}}{\lrid} \sep \boxass{\key_1 \pt 0 \sep \asstrue }{\lrid}{\intass}
    } \\
    \begin{transaction}
        \specline{ \key_1 \fpI 0 } \\
        \pmutate{\key_1}{1} ; \\
        \specline{ \key_1 \fpW 1 } \\
    \end{transaction} \\
    \specline{ 
        \boxass{\key_1 \pt 1 \sep \cass{\CB{L1}}{\lrid} \sep \asstrue }{\lrid}{\intass} \\
    } \\
\end{session}
&
\begin{session}
    \specline{ 
        \cass{\CB{R1}}{\lrid} \sep \boxass{\key_2 \pt 0 \sep \asstrue }{\lrid}{\intass} 
    } \\
    \begin{transaction}
        \specline{ \key_2 \fpI 0 } \\
        \pmutate{\key_2}{1} ; \\
        \specline{ \key_2 \fpW 1 } \\
    \end{transaction} \\
    \specline{ 
        \boxass{\key_2 \pt 1 \sep \cass{\CB{R1}}{\lrid} \sep \asstrue }{\lrid}{\intass} \\
    } \\
\end{session}
\end{parl} \\ \\
\qqqquad \qqqquad \begin{array}{@{} || c ||@{} c @{}}
\begin{session}
    \specline{ 
        \cass{\CB{L2}}{\lrid} \sep 
        \boxass{ 
            \left( 
                \begin{array}{@{}l@{}}
                    \key_1 \pt 0 \lor \cass{\CB{L1}}{\lrid} \sep {} \\
                    ( \key_1 \pt 0 \lor \key_1 \pt 1 ) \\
                \end{array}
            \right)
            \sep {} \\
            \left( 
                \begin{array}{@{}l@{}}
                    \key_2 \pt 0 \lor \cass{\CB{R1}}{\lrid} \sep{} \\
                    ( \key_2 \pt 0 \lor \key_2 \pt 1 ) \\
                \end{array}
            \right) \sep {} \\
            \key_3 \pt 0 \sep \key_4 \pt \anyval 
        }{\lrid}{\intass}  \\
        {} \lor \cass{\CB{L2}}{\lrid} \sep 
        \boxass{\key_1 \pt 1 \sep  \key_2 \pt 1 \sep  {} \\
            \key_3 \pt 0 \sep \key_4 \pt \anyval \sep {} \\
            \cass{\CB{R1}}{\lrid} \sep
            \cass{\CB{R2}}{\lrid} 
            \sep \cass{\CB{L1}}{\lrid} }{\lrid}{\intass} 
    } \\
    \begin{transaction}
        \specline{ 
            \exsts{n,m \in \Set{0,1}} \\
            \key_1 \fpI n \sep \key_2 \fpI m
            \sep \key_3 \fpI 0
        } \\
        \plookup{\pv{a}}{\key_1};
        \quad \plookup{\pv{b}}{\key_2};\\
        \pifs{ \pv{a} = 1 \land \pv{b} = 0 } 
        \passign{\key_3}{1} ; 
        \pife \\
        \specline{ 
            \exsts{n,m \in \Set{0,1}} \\
            \key_1 \fpR n \sep \key_2 \fpR m  \\
            {} \sep \mathtt{if} \ n = 1 \land m = 0 \\
            \ \mathtt{then} \ \key_3 \fpW 1 
            \ \mathtt{else} \ \key_3 \fpI 0
        } \\
    \end{transaction} \\
    \specline{ 
        \cass{\CB{L2}}{\lrid} \sep 
        \boxass{ 
            \left( 
                \begin{array}{@{}l@{}}
                    \key_1 \pt 0 \lor \cass{\CB{L1}}{\lrid} \sep {} \\
                    ( \key_1 \pt 0 \lor \key_1 \pt 1 ) \\
                \end{array}
            \right)
            \sep {} \\
            \left( 
                \begin{array}{@{}l@{}}
                    \key_2 \pt 0 \lor \cass{\CB{R1}}{\lrid} \sep{} \\
                    ( \key_2 \pt 0 \lor \key_2 \pt 1 ) \\
                \end{array}
            \right) \sep {} \\
            \key_3 \pt 0 \sep \key_4 \pt \anyval 
        }{\lrid}{\intass}  \\
        {} \lor \boxass{
            \key_1 \pt 1 \sep ( \key_2 \pt 0 \lor \key_2 \pt 1 ) \\ 
            {} \sep \key_3 \pt 1 \sep \key_4 \pt \anyval \sep {} \\
            \cass{\CB{R1}}{\lrid} \sep \cass{\CB{L1}}{\lrid} \sep \cass{\CB{L2}}{\lrid} }{\lrid}{\intass} \\
        {} \lor \cass{\CB{L2}}{\lrid} \sep 
        \boxass{\key_1 \pt 1 \sep  \key_2 \pt 1 \sep  {} \\
            \key_3 \pt 0 \sep \key_4 \pt \anyval \sep {} \\
            \cass{\CB{R1}}{\lrid} \sep
            \cass{\CB{R2}}{\lrid} 
            \sep \cass{\CB{L1}}{\lrid} }{\lrid}{\intass} 
    } \\
\end{session}
&
\begin{session}
    \specline{ 
        \cass{\CB{R2}}{\lrid} \sep 
        \boxass{ 
            \left( 
                \begin{array}{@{}l@{}}
                    \key_1 \pt 0 \lor \cass{\CB{L1}}{\lrid} \sep {} \\
                    ( \key_1 \pt 0 \lor \key_1 \pt 1 ) \\
                \end{array}
            \right)
            \sep {} \\
            \left( 
                \begin{array}{@{}l@{}}
                    \key_2 \pt 0 \lor \cass{\CB{R1}}{\lrid} \sep {} \\
                    ( \key_2 \pt 0 \lor \key_2 \pt 1 ) \\
                \end{array}
            \right) \sep {} \\
            \key_3 \pt \anyval \sep \key_4 \pt 0
        }{\lrid}{\intass}  \\
        {} \lor \cass{\CB{R2}}{\lrid} \sep 
        \boxass{
            \key_1 \pt 1 \sep  \key_2 \pt 1 \sep {} \\
            \key_3 \pt \anyval \sep \key_4 \pt 0 \sep {} \\
            \cass{\CB{R1}}{\lrid} \sep \cass{\CB{L1}}{\lrid} \sep \cass{\CB{L2}}{\lrid} }{\lrid}{\intass} 
    } \\
    \begin{transaction}
        \specline{ 
            \exsts{n,m \in \Set{0,1}} \\
            \key_1 \fpI n \sep \key_2 \fpI m
            \sep \key_4 \fpI 0
        } \\
        \plookup{\pv{c}}{\key_1};
        \quad \plookup{\pv{d}}{\key_2};\\
        \pifs{ \pv{c} = 0 \land \pv{d} = 1 } 
        \passign{\pv{f2}}{1} ; 
        \pife \\
        \specline{ 
            \exsts{n,m \in \Set{0,1}} \\
            \key_1 \fpR n \sep \key_2 \fpR m  \\
            {} \sep \mathtt{if} \ n = 0 \land m = 1 \\
            \ \mathtt{then} \ \key_4 \fpW 1 
            \ \mathtt{else} \ \key_4 \fpI 0
        } \\
    \end{transaction} \\
    \specline{ 
        \cass{\CB{R2}}{\lrid} \sep 
        \boxass{ 
            \left( 
                \begin{array}{@{}l@{}}
                    \key_1 \pt 0 \lor \cass{\CB{L1}}{\lrid} \sep {} \\
                    ( \key_1 \pt 0 \lor \key_1 \pt 1 ) \\
                \end{array}
            \right)
            \sep {} \\
            \left( 
                \begin{array}{@{}l@{}}
                    \key_2 \pt 0 \lor \cass{\CB{R1}}{\lrid} \sep {} \\
                    ( \key_2 \pt 0 \lor \key_2 \pt 1 ) \\
                \end{array}
            \right) \sep {} \\
            \key_3 \pt \anyval \sep \key_4 \pt 0
        }{\lrid}{\intass}  \\
        {} \lor \boxass{( \key_1 \pt 0 \lor \key_1 \pt 1 ) \sep \key_2 \pt 1 \\
            {} \sep \key_3 \pt \anyval \sep \key_4 \pt 1 \sep {} \\
        {} \cass{\CB{L1}}{\lrid} \sep \cass{\CB{R1}}{\lrid} \sep \cass{\CB{R2}}{\lrid} }{\lrid}{\intass}  \\
        {} \lor \cass{\CB{R2}}{\lrid} \sep 
        \boxass{
            \key_1 \pt 1 \sep  \key_2 \pt 1 \sep {} \\
            \key_3 \pt \anyval \sep \key_4 \pt 0 \sep {} \\
            \cass{\CB{R1}}{\lrid} \sep \cass{\CB{L1}}{\lrid} \sep \cass{\CB{L2}}{\lrid} }{\lrid}{\intass} 
    } \\
\end{session}
\end{array} \\
\end{array} \\ \\
\specline{ 
    \cass{\CB{L2}}{\lrid} \sep \boxass{
        \key_3 \pt 0 \sep \key_4 \pt 1 \sep 
        \cass{\CB{L1}}{\lrid} \sep \cass{\CB{R1}}{\lrid}
        \sep \cass{\CB{R2}}{\lrid} \sep \asstrue 
    }{\lrid}{\intass} \\
    {} \lor \cass{\CB{R2}}{\lrid} \sep \boxass{
        \key_3 \pt 1 \sep \key_4 \pt 0 \sep 
        \cass{\CB{L1}}{\lrid} \sep \cass{\CB{R1}}{\lrid} 
        \sep \cass{\CB{L2}}{\lrid} \sep \asstrue 
    }{\lrid}{\intass} \\
    {} \lor \cass{\CB{L2}}{\lrid} \sep \cass{\CB{R2}}{\lrid} 
    \sep \boxass{
        \key_3 \pt 0 \sep \key_4 \pt 0 
        \sep \cass{\CB{L1}}{\lrid} \sep \cass{\CB{R1}}{\lrid} 
        \sep \asstrue 
    }{\lrid}{\intass} \\
} \\
\end{session} 
\)%
}

\hrulefill

\caption{Sketch proofs for long fork under parallel snapshot isolation}
\label{fig:long-fork-proof}
\end{figure}


%We build a programming logic based on the multi-version key-value stores and views.
%The syntax is close to concurrent-abstract-predicate (CAP) \cite{cap}, 
%we generalise the meaning of the assertions and the \emph{atomic update rule}.
%\cref{fig:long-fork-proof} is the sketch proof for the long fork.

%The transactional assertions are normal separation assertions with read (\(\otR\)) and write (\(\otW\)) labels.
%There are four forms of transactional assertions for a key:
%\( \ke \fpI 1 \), \( \ke \fpR 1\), \( \ke \fpW 1 \) means that the current state of the key \( \ke \) 
%in the local snapshot is 1 and it has no operation, has been read with value 1 or has been overwritten to value 1;
%\( \ke \fpRW (1,2) \) means the state of the key is 2 and 
%the first read before write fetched 1 and then has been overwritten to 2.
%The rules for reasoning the transaction are extension of standard sequential separation logic rules by 
%adding read label to the assertion when reading from keys and write label when writing to keys.

%The assertions in the form \( \boxass{\bar{\lpre}}{\lrid}{\intass}\) 
%are \emph{shared region assertions}, which describe the local views on the shared key-value stores.
%Each share region has a unique region identifier \( \lrid \) and a interference \( \intass \)
%that specifies what operations are allowed to commit, 
%together with some transformation of \emph{capabilities}, ghost resources for reasoning purpose.
%For example, \( \CB{L1}  :  \ke_1 \fpW 1 \sep \null \fpA \cass{\CB{L1}}{\lrid} \) in \cref{fig:long-fork-proof} 
%means if a client holds \( \CB{L1} \), it is allowed to commit a transaction with fingerprint \(  \ke_1 \fpW 1 \),
%and afterwards by \( \null \fpA \cass{\CB{L1}}{\lrid}\) the capability \( \CB{L1} \) should return to the shared region.
%The interference \( \intass \)  induces a invariant of the region that 
%includes all states reachable from the initial states shown in \cref{fig:initial-kv}.
%This means, even though the assertions describe the local views, but they also provide information about possible key-value store
%if taking the invariant into account.

%The notation of stable assertions is generalised: if assertions are stable under a consistency model,  
%then the assertions are close under the portion of invariant where each step satisfies the consistency model.
%The \emph{update atomic rule} is similar to the one from CAP.
%We generalise the repartitioning \( \repartition[]{\gpre}{\gpost}{\lpre}{\lpost}\):
%for any initial view and key-value stores satisfying the precondition \( \gpre \),
%any final view and key-value stores that are yielded by committing the fingerprints satisfying \( \gpost \) under the consistency model, should satisfy the postcondition \( \lpost \).


%\sx{
%    \begin{itemize}
%        \item long-fork example
%        \item Multi-version Key-value, View.
%        \item how does the semantics work on long fork example
%        \item a sketch proof for long fork 
%    \end{itemize}
%}
%
%
%\begin{wrapfigure}[7]{r}{0.33\textwidth}
%\vspace{-10pt}
%\begin{verbatim}
%interface Transaction {
%    Start(); 
%    Read(Key k);
%    Write(Key k, Value v); 
%    Commit();    }
%\end{verbatim}
%\vspace{-10pt}
%\caption{Example of Transaction API.}
%\label{fig:api}
%\end{wrapfigure}
%We focus on an abstract computational model where multiple client programs can access and update keys in a key-value store using atomic transactions. 
%In general, clients are provided with a simple \textit{API} such as the one depicted in \cref{fig:api} \cite{gdur,physicsnmsi,clockSI}\footnote{It is 
%often the case that key-value stores provide a mechanism to wrap more transactions inside a session, and give 
%provide appropriate APIs to handle sessions. For the sake of simplicity, in this paper we assume that each client executes transactions 
%within a single session.}, while both the implementation details and the system architecture are hidden from clients. 
%Because (distributed) key-value stores only give weak consistency guarantees of the data to their clients, the latter are 
%not ensured to read the most up-to-date version of a key.
%%In an ideal world, when executing a transaction clients would read the most up-to-date version of a key. In a distributed setting 
%%this approach, known as (strict) serialisability, would require a continuous synchronisation between the different components of 
%%the system, which impacts performance and limits scalability. To this end, the database only provides weak consistency model
%%\ac{This sentence should probably be in the introduction.}
%
%Following these intuitions, we model a key-value stores, or \emph{kv-store}, as a centralised unit where multiple versions 
%are stored for each key (\cref{sec:mkvs-view}). Versions consist of a value and the meta-data of the transactions that wrote and 
%read such a version. In practical, distributed systems, the meta-data is usually encoded using either timestamps 
%\cite{physicsnmsi,clockSI} or vector clocks \cite{gdur}. We focus on key-value stores whose transactions 
%enjoy \emph{atomic visibility}, meaning that \textbf{(i)} transactions read their data from an atomic 
%snapshot of the key-value store, and \textbf{(ii)} a transaction can observe either none or all 
%of the updates performed by another transaction. In other words, a transaction only reads (writes) at most 
%one version for each key.
%Because clients may observe potentially out-of-date versions of the system, we introduce the notion of \emph{views}. 
%Intuitively, a view records the version of each key that a client observes at a given time. We use views 
%to determine the snapshot taken by transactions executed by clients.
%
%A consistency model is a contract between the key-value store and its clients. We distinguish 
%between \emph{client-centric} consistency models \cite{terry1994session}, 
%which impose constraints on the observations and updates made by a single client, 
%and \emph{data-centric} consistency models \cite{framework-concur}, which impose constraints 
%on the structure of the key value store.
%To specify weak consistency models, 
%we introduce the notion of \emph{execution tests} (\cref{sec:execution.tests}). An execution test
% specifies when a client is allowed to execute a transaction carrying a given 
%set of read and write operations, or \emph{fingerprint}. Therefore, an execution 
%test constrains how the state of the key-value store may evolve;
%by considering all the possible evolutions of the key-value store under said execution test, 
%we determine a consistency model. For example, an execution test for (strict) serialisability 
%requires that a transaction can be executed by a client only if it observes the most up-to-date 
%version for each key. 
%We give several examples of execution tests that capture both client-centric and 
%data-centric consistency models. 
%
%The idea of specifying consistency models using execution tests has been 
%already proposed in \cite{seebelieve}; however, their notion of execution 
%test is intrinsically more complex than ours: to determine 
%whether a transaction can commit, the total order in which all past transactions 
%have committed must be known. This knowledge is not needed in our setting.
%
%
%\begin{figure}[t]
%\begin{center}
%
%\hrulefill
%
%\begin{onethirdsubfig}
%\begin{tikzpicture}
%\begin{pgfonlayer}{foreground}
%%\draw[help lines] grid(5,4);
%
%%Location x
%\node(locx) {$\ke_1 \mapsto$};
%
%\matrix(versionx) [version list, column 2/.style={text width=8mm}]
%    at ([xshift=\tikzkvspace]locx.east) {
%    {a} & $\txid_0$ & {a} & $\txid_{\cl}^{1}$\\
%    {a} & $\left\{\txid_{\cl'}^{1}\right\}$ & {a} & $\emptyset$ \\
%};
%\tikzvalue{versionx-1-1}{versionx-2-1}{locx-v0}{$v_0$};
%\tikzvalue{versionx-1-3}{versionx-2-3}{locx-v1}{$v_1$};
%
%%Location y
%\path (locx.south) + (0,\tikzkeyspace) node (locy) {$\ke_2 \mapsto$};
%\matrix(versiony) [version list, column 2/.style={text width=8mm}]
%   at ([xshift=\tikzkvspace]locy.east) {
% {a} & $\txid_0$ & {a} & $\txid_{\cl'}^{1}$ \\
%  {a} & $\left\{\txid_{\cl}^1\right\}$ & {a} & $\emptyset$\\
%};
%
%\tikzvalue{versiony-1-1}{versiony-2-1}{locy-v0}{$v'_0$};
%\tikzvalue{versiony-1-3}{versiony-2-3}{locy-v1}{$v'_1$};
%
%% \draw[-, red, very thick, rounded corners] ([xshift=-5pt, yshift=5pt]locx-v1.north east) |- 
%%  ($([xshift=-5pt,yshift=-5pt]locx-v1.south east)!.5!([xshift=-5pt, yshift=5pt]locy-v0.north east)$) -| ([xshift=-5pt, yshift=5pt]locy-v0.south east);
%
%%%blue view - I should  check whether I can use pgfkeys to just declare the list of locations, and then add the view automatically.
%%\draw[-, blue, very thick, rounded corners=10pt]
% %([xshift=-3pt, yshift=20pt]locx-v1.north east) node (tid1start) {} -- 
% %([xshift=-3pt, yshift=-5pt]locx-v1.south east) --
% %([xshift=-3pt, yshift=5pt]locy-v0.north east) -- 
% %([xshift=-3pt, yshift=-5pt]locy-v0.south east);
% 
% %\path (tid1start) node[anchor=south, rectangle, fill=blue!20, draw=blue, font=\small, inner sep=1pt] {$\client$};
%
%%%red view
%%\draw[-, red, very thick, rounded corners = 10pt]
% %([xshift=-16pt, yshift=5pt]locx-v1.north east) node (tid2start) {}-- 
% %([xshift=-16pt, yshift=-5pt]locy-v1.south east) node {};
% 
%%\path (tid2start) node[anchor=south, rectangle, fill=red!20, draw=red, font=\small, inner sep=1pt] {$\client'$};
%
%\end{pgfonlayer}
%\end{tikzpicture}
%\caption{A key-value store $\hh$}
%\label{fig:hheap-a}
%\label{fig:key-value-store}
%\label{fig:ser-disallowed}
%\end{onethirdsubfig}
%%
%\begin{onethirdsubfig}
%\begin{tikzpicture}
%\begin{pgfonlayer}{foreground}
%%\draw[help lines] grid(5,4);
%
%%Location x
%\node(locx) {$\ke_1 \mapsto$};
%
%\matrix(versionx) [version list, column 2/.style={text width=8mm}]
%    at ([xshift=\tikzkvspace]locx.east) {
%    {a} & $\txid_0$ & {a} & $\txid_{\cl}^{1}$\\
%    {a} & $\left\{\txid_{\cl'}^{1}\right\}$ & {a} & $\emptyset$ \\
%};
%\tikzvalue{versionx-1-1}{versionx-2-1}{locx-v0}{$v_0$};
%\tikzvalue{versionx-1-3}{versionx-2-3}{locx-v1}{$v_1$};
%
%%Location y
%\path (locx.south) + (0,\tikzkeyspace) node (locy) {$\ke_2 \mapsto$};
%\matrix(versiony) [version list, column 2/.style={text width=8mm}, column 3/.style={dotted}, column 4/.style={dotted}]
%   at ([xshift=\tikzkvspace]locy.east) {
%       {a} & $\txid_0$ & {a} & \color{gray}$\txid_{\cl'}^{1}$ \\
%  {a} & $\left\{\txid_{\cl}^1\right\}$ & {a} & \color{gray}$\emptyset$\\
%};
%
%\tikzvalue{versiony-1-1}{versiony-2-1}{locy-v0}{$v'_0$};
%\node[version node,draw=none,fit=(versiony-1-3) (versiony-2-3),fill=white, inner sep=0pt] (locy-v1) {\color{gray}$v'_1$};
%
%\end{pgfonlayer}
%\end{tikzpicture}
%\caption{A view for client \( \cl \)}
%\label{fig:view-of-cl}
%\end{onethirdsubfig}
%%
%\begin{onethirdsubfig}
%\begin{tikzpicture}
%\begin{pgfonlayer}{foreground}
%%\draw[help lines] grid(5,4);
%
%%Location x
%\node(locx) {$\ke_1 \mapsto$};
%
%\matrix(versionx) [version list, column 2/.style={text width=8mm}, column 3/.style={dotted}, column 4/.style={dotted}]
%    at ([xshift=\tikzkvspace]locx.east) {
%    {a} & $\txid_0$ & {a} & \color{gray}$\txid_{\cl}^{1}$\\
%    {a} & $\left\{\txid_{\cl'}^{1}\right\}$ & {a} & \color{gray}$\emptyset$ \\
%};
%\tikzvalue{versionx-1-1}{versionx-2-1}{locx-v0}{$v_0$};
%\node[version node,draw=none,fit=(versionx-1-3) (versionx-2-3),fill=white, inner sep=0pt] (locx-v1) {\color{gray}$v_1$};
%
%%Location y
%\path (locx.south) + (0,\tikzkeyspace) node (locy) {$\ke_2 \mapsto$};
%\matrix(versiony) [version list, column 2/.style={text width=8mm}]
%   at ([xshift=\tikzkvspace]locy.east) {
% {a} & $\txid_0$ & {a} & $\txid_{\cl'}^{1}$ \\
%  {a} & $\left\{\txid_{\cl}^1\right\}$ & {a} & $\emptyset$\\
%};
%
%\tikzvalue{versiony-1-1}{versiony-2-1}{locy-v0}{$v'_0$};
%\tikzvalue{versiony-1-3}{versiony-2-3}{locy-v1}{$v'_1$};
%
%\end{pgfonlayer}
%\end{tikzpicture}
%\caption{A view for client \( \cl' \)}
%\label{fig:view-of-cl-prime}
%\end{onethirdsubfig}
%
%\end{center}
%
%\hrulefill
%
%\caption{Multi-version key-value stores and views}
%\label{fig:hheap}
%\label{fig:key-value-and-view}
%\end{figure}
%\ac{
%It will be often convenient to depict key-value stores graphically: an 
%example is given by the kv-store $\hh$ depicted in \cref{fig:hheap-a}
%(ignore for the moment the vertical lines labelled $\client$ and $\client'$).
%It comprises two keys \( \ke_1\) and \( \ke_2 \), 
%each of which is associated with two versions carrying values $\val_0$ and $\val_1$, and $\val'_0$ and $\val'_1$, respectively.
%The versions of a key are listed in order from left to right. 
%We represent each version as a three-cell box, with the left cell storing the value, the top right cell recording the writer, and the bottom right cell recording the readers. 
%For example, the version carrying value $\val_0$ in $\ke_1$ has been written by $\txid_0$, and has been read by $\txid_{\cl'}^1$.
%}
%
%%the author require 
%%the knowledge of the total order in which all past transactions have been 
%%executed, to determine whether a new transaction can commit.
%
%%%\ac{Can't cite documentations of real databases, as they usually have a much more complicated API.}
%%
%%%Transactions in our model execute atomically, though they have different effects on the key-value stores depending on their associated \emph{consistency model}.
%%%A consistency model controls how the key-value store evolves.
%%%A common model is \emph{serialisability}, where transactions appear to execute one after another in a sequential order.
%%%This notion of sequential execution is however not necessary for many weaker models. As such, upon commencing execution, a transaction may not observe the most up-to-date values for keys. 
%%
%%To address this, we 
%%first model 
%%the state of the system using \emph{multi-version key-value stores (MKVSs)} (\cref{sec:mkvs-view}). 
%%An MKVS keeps track of all versions (values) written for keys, as well as the information about the transactions that read and wrote such versions. 
%%To model the potential out-of-date observation, we introduce \emph{views}.
%%A view decides the observable versions of keys for a client.
%%Therefore, in order to execute a transaction, the client first takes \emph{a snapshot} of the system with the view, executes the transaction locally with respect to its snapshot (\cref{sec:trans-semantics}), and afterwards commits the effect of the transaction if the change is allowed by the underlying consistency model (\cref{sec:prog-semantics}).
%
%
%%We first introduce MKVSs and views (\secref{sec:mkvs-view}) and.
%%We starts with the syntax of programs followed by the semantics of transaction.
%%Finally, we will give the semantics for the entire programs.
\begin{figure}[t]

\hrulefill

\begin{subfigure}{0.38\textwidth}
\begin{verbatim}
interface Transaction {
    Start(); 
    Read(Key k);
    Write(Key k, Value v); 
    Commit();    
}
\end{verbatim}
\caption{Example of Transaction API.}
\label{fig:api}
\end{subfigure}

\hrulefill

\end{figure}
%
%We focus on an abstract computational model where multiple client programs can access and update keys in a key-value store using atomic transactions. 
%In general, clients are provided with a simple \textit{API} such as the one depicted in \cref{fig:api} \cite{gdur,physicsnmsi,clockSI}\footnote{It is 
%often the case that key-value stores provide a mechanism to wrap more transactions inside a session, and give 
%provide appropriate APIs to handle sessions. For the sake of simplicity, in this paper we assume that each client executes transactions 
%within a single session.}, while both the implementation details and the system architecture are hidden from clients. 
%Because (distributed) key-value stores only give weak consistency guarantees of the data to their clients, the latter are 
%not ensured to read the most up-to-date version of a key.
%%In an ideal world, when executing a transaction clients would read the most up-to-date version of a key. In a distributed setting 
%%this approach, known as (strict) serialisability, would require a continuous synchronisation between the different components of 
%%the system, which impacts performance and limits scalability. To this end, the database only provides weak consistency model
%%\ac{This sentence should probably be in the introduction.}
%
%Following these intuitions, we model a key-value stores, or \emph{kv-store}, as a centralised unit where multiple versions 
%are stored for each key (\cref{sec:mkvs-view}). Versions consist of a value and the meta-data of the transactions that wrote and 
%read such a version. In practical, distributed systems, the meta-data is usually encoded using either timestamps 
%\cite{physicsnmsi,clockSI} or vector clocks \cite{gdur}. We focus on key-value stores whose transactions 
%enjoy \emph{atomic visibility}, meaning that \textbf{(i)} transactions read their data from an atomic 
%snapshot of the key-value store, and \textbf{(ii)} a transaction can observe either none or all 
%of the updates performed by another transaction. In other words, a transaction only reads (writes) at most 
%one version for each key.
%Because clients may observe potentially out-of-date versions of the system, we introduce the notion of \emph{views}. 
%Intuitively, a view records the version of each key that a client observes at a given time. We use views 
%to determine the snapshot taken by transactions executed by clients.
%
%A consistency model is a contract between the key-value store and its clients. We distinguish 
%between \emph{client-centric} consistency models \cite{terry1994session}, 
%which impose constraints on the observations and updates made by a single client, 
%and \emph{data-centric} consistency models \cite{framework-concur}, which impose constraints 
%on the structure of the key value store.
%To specify weak consistency models, 
%we introduce the notion of \emph{execution tests} (\cref{sec:execution.tests}). An execution test
% specifies when a client is allowed to execute a transaction carrying a given 
%set of read and write operations, or \emph{fingerprint}. Therefore, an execution 
%test constrains how the state of the key-value store may evolve;
%by considering all the possible evolutions of the key-value store under said execution test, 
%we determine a consistency model. For example, an execution test for (strict) serialisability 
%requires that a transaction can be executed by a client only if it observes the most up-to-date 
%version for each key. 
%We give several examples of execution tests that capture both client-centric and 
%data-centric consistency models. 
%
%The idea of specifying consistency models using execution tests has been 
%already proposed in \cite{seebelieve}; however, their notion of execution 
%test is intrinsically more complex than ours: to determine 
%whether a transaction can commit, the total order in which all past transactions 
%have committed must be known. This knowledge is not needed in our setting.
%
%
%\begin{figure}[t]
%\begin{center}
%
%\hrulefill
%
%\begin{onethirdsubfig}
%\begin{tikzpicture}
%\begin{pgfonlayer}{foreground}
%%\draw[help lines] grid(5,4);
%
%%Location x
%\node(locx) {$\vx \mapsto$};
%
%\matrix(versionx) [version list, column 2/.style={text width=8mm}]
%    at ([xshift=\tikzkvspace]locx.east) {
%    {a} & $\txid_0$ & {a} & $\txid_{\cl}^{1}$\\
%    {a} & $\emptyset$ & {a} & $\emptyset$ \\
%};
%\tikzvalue{versionx-1-1}{versionx-2-1}{locx-v0}{$v_0$};
%\tikzvalue{versionx-1-3}{versionx-2-3}{locx-v1}{$v_1$};
%
%%Location y
%\path (locx.south) + (0,\tikzkeyspace) node (locy) {$\vy \mapsto$};
%\matrix(versiony) [version list, column 2/.style={text width=8mm}]
%   at ([xshift=\tikzkvspace]locy.east) {
% {a} & $\txid_0$ & {a} & $\txid_{\cl'}^{1}$ \\
% {a} & $\Set{\txid_{\cl}^{2}}$ & {a} & $\emptyset$\\
%};
%
%\tikzvalue{versiony-1-1}{versiony-2-1}{locy-v0}{$v'_0$};
%\tikzvalue{versiony-1-3}{versiony-2-3}{locy-v1}{$v'_1$};
%
%% \draw[-, red, very thick, rounded corners] ([xshift=-5pt, yshift=5pt]locx-v1.north east) |- 
%%  ($([xshift=-5pt,yshift=-5pt]locx-v1.south east)!.5!([xshift=-5pt, yshift=5pt]locy-v0.north east)$) -| ([xshift=-5pt, yshift=5pt]locy-v0.south east);
%
%%%blue view - I should  check whether I can use pgfkeys to just declare the list of locations, and then add the view automatically.
%%\draw[-, blue, very thick, rounded corners=10pt]
% %([xshift=-3pt, yshift=20pt]locx-v1.north east) node (tid1start) {} -- 
% %([xshift=-3pt, yshift=-5pt]locx-v1.south east) --
% %([xshift=-3pt, yshift=5pt]locy-v0.north east) -- 
% %([xshift=-3pt, yshift=-5pt]locy-v0.south east);
% 
% %\path (tid1start) node[anchor=south, rectangle, fill=blue!20, draw=blue, font=\small, inner sep=1pt] {$\client$};
%
%%%red view
%%\draw[-, red, very thick, rounded corners = 10pt]
% %([xshift=-16pt, yshift=5pt]locx-v1.north east) node (tid2start) {}-- 
% %([xshift=-16pt, yshift=-5pt]locy-v1.south east) node {};
% 
%%\path (tid2start) node[anchor=south, rectangle, fill=red!20, draw=red, font=\small, inner sep=1pt] {$\client'$};
%
%\end{pgfonlayer}
%\end{tikzpicture}
%\caption{A key-value store $\hh$}
%\label{fig:hheap-a}
%\label{fig:key-value-store}
%\label{fig:ser-disallowed}
%\end{onethirdsubfig}
%%
%\begin{onethirdsubfig}
%\begin{tikzpicture}
%\begin{pgfonlayer}{foreground}
%%\draw[help lines] grid(5,4);
%
%%Location x
%\node(locx) {$\vx \mapsto$};
%
%\matrix(versionx) [version list, column 2/.style={text width=8mm}]
%    at ([xshift=\tikzkvspace]locx.east) {
%    {a} & $\txid_0$ & {a} & $\txid_{\cl}^{1}$\\
%    {a} & $\emptyset$ & {a} & $\emptyset$ \\
%};
%\tikzvalue{versionx-1-1}{versionx-2-1}{locx-v0}{$v_0$};
%\tikzvalue{versionx-1-3}{versionx-2-3}{locx-v1}{$v_1$};
%
%%Location y
%\path (locx.south) + (0,\tikzkeyspace) node (locy) {$\vy \mapsto$};
%\matrix(versiony) [version list, column 2/.style={text width=8mm}, column 3/.style={dotted}, column 4/.style={dotted}]
%   at ([xshift=\tikzkvspace]locy.east) {
%       {a} & $\txid_0$ & {a} & \color{gray}$\txid_{\cl'}^{1}$ \\
%   {a} & $\Set{\txid_{\cl}^{2}}$ & {a} & \color{gray}$\emptyset$\\
%};
%
%\tikzvalue{versiony-1-1}{versiony-2-1}{locy-v0}{$v'_0$};
%\node[version node,draw=none,fit=(versiony-1-3) (versiony-2-3),fill=white, inner sep=0pt] (locy-v1) {\color{gray}$v'_1$};
%
%\end{pgfonlayer}
%\end{tikzpicture}
%\caption{A view for client \( \cl \)}
%\label{fig:view-of-cl}
%\end{onethirdsubfig}
%%
%\begin{onethirdsubfig}
%\begin{tikzpicture}
%\begin{pgfonlayer}{foreground}
%%\draw[help lines] grid(5,4);
%
%%Location x
%\node(locx) {$\vx \mapsto$};
%
%\matrix(versionx) [version list, column 2/.style={text width=8mm}, column 3/.style={dotted}, column 4/.style={dotted}]
%    at ([xshift=\tikzkvspace]locx.east) {
%    {a} & $\txid_0$ & {a} & \color{gray}$\txid_{\cl}^{1}$\\
%    {a} & $\emptyset$ & {a} & \color{gray}$\emptyset$ \\
%};
%\tikzvalue{versionx-1-1}{versionx-2-1}{locx-v0}{$v_0$};
%\node[version node,draw=none,fit=(versionx-1-3) (versionx-2-3),fill=white, inner sep=0pt] (locx-v1) {\color{gray}$v_1$};
%
%%Location y
%\path (locx.south) + (0,\tikzkeyspace) node (locy) {$\vy \mapsto$};
%\matrix(versiony) [version list, column 2/.style={text width=8mm}]
%   at ([xshift=\tikzkvspace]locy.east) {
%  {a} & $\txid_0$ & {a} & $\txid_{\cl'}^{1}$ \\
%  {a} & $\left\{\txid_{\cl}^2\right\}$ & {a} & $\emptyset$\\
%};
%
%\tikzvalue{versiony-1-1}{versiony-2-1}{locy-v0}{$v'_0$};
%\tikzvalue{versiony-1-3}{versiony-2-3}{locy-v1}{$v'_1$};
%
%\end{pgfonlayer}
%\end{tikzpicture}
%\caption{A view for client \( \cl' \)}
%\label{fig:view-of-cl-prime}
%\end{onethirdsubfig}
%
%\sx{The key-value store corresponds to the last step of long fork.}
%\end{center}
%
%\hrulefill
%
%\caption{Multi-version key-value stores and views}
%\label{fig:hheap}
%\label{fig:key-value-store-and-view}
%\label{fig:key-value-and-view}
%\end{figure}
%\ac{
%It will be often convenient to depict key-value stores graphically: an 
%example is given by the kv-store $\hh$ depicted in \cref{fig:hheap-a}
%(ignore for the moment the vertical lines labelled $\client$ and $\client'$).
%It comprises two keys \( \ke_1\) and \( \ke_2 \), 
%each of which is associated with two versions carrying values $\val_0$ and $\val_1$, and $\val'_0$ and $\val'_1$, respectively.
%The versions of a key are listed in order from left to right. 
%We represent each version as a three-cell box, with the left cell storing the value, the top right cell recording the writer, and the bottom right cell recording the readers. 
%For example, the version carrying value $\val_0$ in $\ke_1$ has been written by $\txid_0$, and has been read by $\txid_{\cl'}^1$.
%}
%
%%the author require 
%%the knowledge of the total order in which all past transactions have been 
%%executed, to determine whether a new transaction can commit.
%
%%%\ac{Can't cite documentations of real databases, as they usually have a much more complicated API.}
%%
%%%Transactions in our model execute atomically, though they have different effects on the key-value stores depending on their associated \emph{consistency model}.
%%%A consistency model controls how the key-value store evolves.
%%%A common model is \emph{serialisability}, where transactions appear to execute one after another in a sequential order.
%%%This notion of sequential execution is however not necessary for many weaker models. As such, upon commencing execution, a transaction may not observe the most up-to-date values for keys. 
%%
%%To address this, we 
%%first model 
%%the state of the system using \emph{multi-version key-value stores (MKVSs)} (\cref{sec:mkvs-view}). 
%%An MKVS keeps track of all versions (values) written for keys, as well as the information about the transactions that read and wrote such versions. 
%%To model the potential out-of-date observation, we introduce \emph{views}.
%%A view decides the observable versions of keys for a client.
%%Therefore, in order to execute a transaction, the client first takes \emph{a snapshot} of the system with the view, executes the transaction locally with respect to its snapshot (\cref{sec:trans-semantics}), and afterwards commits the effect of the transaction if the change is allowed by the underlying consistency model (\cref{sec:prog-semantics}).
%
%
%%We first introduce MKVSs and views (\secref{sec:mkvs-view}) and.
%%We starts with the syntax of programs followed by the semantics of transaction.
%%Finally, we will give the semantics for the entire programs.f
