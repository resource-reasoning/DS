\subsection{Soundness}

\label{sec:mcapSound}

The soundness of mCAP is established by relating its proof judgements to the operational semantics. The link is created through a \textit{reification} function that transforms worlds, as defined in Section \ref{worlds}, to concrete machine states. As we constructed mCAP as an instantiation of the views framework \cite{views}, its soundness follows from the soundness of views, given that the transactions (atomic commands) are shown to be sound with respect to their operational semantics.

\begin{param}
	(Machine state reification).
	Given the partial commutative monoid for machine states $(\mathbb{M}, \bullet_\mathbb{M}, \mathbf{0}_\mathbb{M})$, assume a \emph{reification function} $\lfloor - \rfloor_\mathbb{M} : \mathbb{M} \rightarrow \mathcal{P}(\mathcal{S})$ which associates machine states to sets of concrete ones.
\end{param}

The views framework guarantees soundness of an instantiated logic once we are able to prove its \textit{Axiom Soundness} property, specifically \textbf{Property L} in \cite{views}. This is formulated and proven in Theorem \ref{thm:soundTrans}. Given the structure of the operational semantics for transactions, the proof relies on two other results, namely the axiom soundness of sequential commands and elementary commands. The latter is provided to the framework as a further parameter since elementary commands are themselves parametric.

\begin{param}
	\label{param:ecmdSound}
	(Elementary command soundness).
	Given the pcm for machine states $(\mathbb{M}, \bullet_\mathbb{M}, \mathbf{0}_\mathbb{M})$ and its associated reification function $\lfloor - \rfloor_\mathbb{M}$, assume that for every elementary command $\hat{\mathds{C}} \in \mathsf{ECmd}$, the corresponding axiom $(M_1, \hat{\mathds{C}}, M_2) \in \textsc{Ax}_{\hat{\mathsf{C}}}$ and any given machine state $m \in \mathbb{M}$ the following \emph{soundness} property holds:
	\[
		\tsem{\hat{\mathds{C}}}(\lfloor M_1 \bullet_\mathbb{M} \{m\} \rfloor_\mathbb{M}) \subseteq \lfloor M_2 \bullet_\mathbb{M} \{m\} \rfloor_\mathbb{M}
	\]
\end{param}
Intuitively, the property establishes that, for any given axiom of elementary commands $(M_1, \hat{\mathds{C}}, M_2)$, the operational semantics of $\hat{\mathds{C}}$ are able to transform elements of the reificated machine states set $M_1$ into ones that belong to the reification of $M_2$. Moreover, the property also enforces frame preservation. In fact, it suggests that if we apply the command to the composition of the elements in $\lfloor M_1 \rfloor_\mathbb{M}$ with an arbitrary machine state $m$, then the result will be a subset the reification of elements in $\lfloor M_2 \rfloor_\mathbb{M}$, composed with the same $m$.

\begin{defn}
	(Reification).
	The \emph{reification of worlds}, $\lfloor - \rfloor_W : \mathsf{World} \rightarrow \mathcal{P}(\mathcal{S})$, is defined as follows.
	\[
		\lfloor (l, g, \mathcal{J}) \rfloor_W \triangleq \lfloor (l \circ g)_\mathsf{M} \rfloor_\mathbb{M}
	\]
\end{defn}
We transform worlds into machine states by dropping the action model, composing the local and the shared state together, and removing all of the region capabilities from it in order to only end up with the machine component of the resulting logical state.

\begin{defn}
	(Judgements).
	The syntactic triple $\Delta \vdash \triple{P}{\mathds{P}}{Q}$ is defined in terms of the semantic triple as stated in Views \cite{views} and it holds if and only if
	\[
		\forall e, \delta \ldotp \delta \in \tsem{\Delta}^\textsc{p} \implies \vDash \triple{\tsem{P}_{e, \delta}}{\mathds{P}}{\tsem{Q}_{e, \delta}}
	\]
\end{defn}

In a comparable way to elementary commands, the soundness of transactions is proven by considering an arbitrary transactions' axiom $(W_1, \mathds{T}, W_2)$, built according to Definition \ref{defn:transAx}, together with a world $w$. It is required to show that the operational semantics of $\mathds{T}$, when applied to elements of the reification of $W_1 \bullet_\mathbb{M} \{w\}$, output a concrete state which is a subset of $\lfloor W_1 \bullet_\mathbb{M} R(\{w\}) \rfloor_\mathbb{M}$. Not only we require the semantics to work correctly in terms of what the axioms impose, but also to preserve frames even under the effects of the rely relation. The property in fact ensures that whatever additional world component, $w$, which is not needed by $\mathds{T}$, i.e. its corresponding axiom does not describe it, will remain unchanged upto its rely relation evolutions.
\begin{thm}
	\label{thm:soundTrans}
	(Transaction soundness).
	For all $\mathds{T} \in \mathsf{Trans}, (W_1, \mathds{T}, W_2) \in \textsc{Ax}_\mathsf{T}$ and $w \in \mathsf{World}$:
	\[
		\tsem{\mathds{T}}_\mathsf{T}(\lfloor W_1 \bullet_\mathbb{M} \{ w \} \rfloor_W) \subseteq \lfloor W_2 \bullet_\mathbb{M} R(\{ w \}) \rfloor_W
	\]
	
	{\parindent0pt
	\begin{proof}
	By induction on the structure of $\mathds{T}$. \\	
	
	\textit{Case}: $\ptdef{\mathds{C}}$
	
	Let's pick an arbitrary $\mathds{C} \in \mathsf{Cmd}, w \in \mathsf{World}$ and $W_1, W_2 \in \mathcal{P}(\mathsf{World})$ such that the following holds:
	\[
		(W_1, \ptdef{\mathds{C}}_\iota, W_2) \in \textsc{Ax}_\mathsf{T}
	\]
	From the definition of $\textsc{Ax}_\mathsf{T}$ we know that there exists $M_1, M_2 \in \mathcal{P}(\mathbb{M})$ such that:
	\begin{gather}\label{tsound:1}
		(M_1, \mathds{C}, M_2) \in \textsc{Ax}_\mathsf{C} \land W_1 \Rrightarrow^{\{M_1\}\{M_2\}} W_2
	\end{gather}
	
	\textit{To show}:
	\[
		\tsem{\ptdef{\mathds{C}}}_\mathsf{T}(\lfloor W_1 \bullet \{w\} \rfloor_W) \subseteq \lfloor W_2 \bullet R(\{w\}) \rfloor_W
	\]
	
	Let's pick an arbitrary $w_1 = (l_1, g_1, \mathcal{J}_1) \in W_1$. We are now left to show that there exists a $w_2 \in \mathsf{World}$ and $w' \in R(w)$ such that:
	\[
		\tsem{\ptdef{\mathds{C}}}_\mathsf{T}(\lfloor w_1 \bullet w \rfloor_W) = \lfloor w_2 \bullet w' \rfloor_W
	\]
	
	From the definition of $\tsem{-}_\mathsf{T}, \lfloor - \rfloor_W$, and the properties of $\bullet$ and $\bullet_\mathbb{M}$ we have:
	\begin{align}
		\tsem{\ptdef{\mathds{C}}}_\mathsf{T}(\lfloor w_1 \bullet w \rfloor_W) =\
			&\tsem{\mathds{C}}_\mathsf{C}(\lfloor w_1 \bullet w \rfloor_W) \\
			\label{tsound:2} =\ &\tsem{\mathds{C}}_\mathsf{C}\left( \lfloor (l_1 \oplus g_1)_\mathsf{M} \bullet_\mathbb{M} (w_\mathsf{L})_\mathsf{M} \rfloor_\mathbb{M} \right)
	\end{align}
	
	From (\ref{tsound:1}) and the definition of $\Rrightarrow$ we know there exists $m_1 \in M_1$ and $m' \in \mathbb{M}$ such that:
	\begin{gather}
	\label{tsound:3} m_1 \bullet_\mathbb{M} m' = (l_1 \oplus g_1)_\mathsf{M} \land \\
	\label{tsound:4} \forall m_2 \in M_2 \ldotp \exists w_2 = (l_2, g_2, \mathcal{J}_2) \in W_2 \ldotp m_2 \bullet_\mathbb{M} m' = (l_2 \oplus g_2)_\mathsf{M} \land (w_1, w_2) \in G
	\end{gather}
	
	From (\ref{tsound:2}) and (\ref{tsound:3}) we get:
	\begin{gather}\label{tsound:5}
	\tsem{\ptdef{\mathds{C}}}_\mathsf{T}(\lfloor w_1 \bullet w \rfloor_W) =
	\tsem{\mathds{C}}_\mathsf{C}(\lfloor m_1 \bullet_\mathbb{M} m' \bullet_\mathbb{M} (w_\mathsf{L})_\mathsf{M} \rfloor_\mathbb{M})
	\end{gather}
	
	From (\ref{tsound:1}) and the soundness of commands shown in Theorem \ref{thm:cSound}, we can rewrite (\ref{tsound:5}) as:
	\[
		\tsem{\ptdef{\mathds{C}}}_\mathsf{T}(\lfloor w_1 \bullet w \rfloor_W)
		\subseteq
		\lfloor M_2 \bullet_\mathbb{M} \{m' \bullet_\mathbb{M} (w_\mathsf{L})_\mathsf{M}\} \rfloor_\mathbb{M}
	\]
	
	Which means that there exists a $m_2 \in M_2$ such that:
	\begin{gather}\label{tsound:6}
		\tsem{\ptdef{\mathds{C}}}_\mathsf{T}(\lfloor w_1 \bullet w \rfloor_W)
		=
		\lfloor m_2 \bullet_\mathbb{M} m' \bullet_\mathbb{M} (w_\mathsf{L})_\mathsf{M} \rfloor_\mathbb{M}
	\end{gather}
	
	From (\ref{tsound:4}) we know that there exists $w_2 \in \mathsf{World}$ such that:
	\begin{gather}
	\label{tsound:8} w_2 = (l_2, g_2, \mathcal{J}_2) \in W_2 \land m_2 \bullet_\mathbb{M} m' = (l_2 \oplus g_2)_\mathsf{M} \\
	\label{tsound:7} \land\ (w_1, w_2) \in G
	\end{gather}
	
	From the definition of $\lfloor - \rfloor_W$ and the properties of $\bullet_\mathbb{M}$ and $\bullet$ we can rewrite (\ref{tsound:6}) as:
	\begin{gather}\label{tsound:9}
		\tsem{\ptdef{\mathds{C}}}_\mathsf{T}(\lfloor w_1 \bullet w \rfloor_W)
		=
		\lfloor (l_2 \circ w_\mathsf{L}, g_2, \mathcal{J}_2) \rfloor_W
	\end{gather}
	
	From (\ref{tsound:7}) and Lemma \ref{lem:R} we know that there exists a $w' \in \mathsf{World}$ such that:
	\begin{gather}\label{tsound:10}
		w' = (w_\mathsf{L}, g_2, \mathcal{J}_2) \land w' \in R(w)
	\end{gather}
	
	From (\ref{tsound:8}), (\ref{tsound:9}) and (\ref{tsound:10}) we know that there exists a $w_2 \in W_2$ and $w' \in R(w)$ such that:
	\[
		\tsem{\ptdef{\mathds{C}}}_\mathsf{T}(\lfloor w_1 \bullet w \rfloor_W) = \lfloor w_2 \bullet R(w) \rfloor_W
	\]
	\end{proof}
	}
\end{thm}

\begin{thm}
	\label{thm:mcapSound}
	(Soundness).
	For all predicate axioms $\Delta \in \mathsf{Ax}$, assertions $P, Q \in \mathsf{Assn}$ and program $\mathds{P} \in \mathsf{Prog}$, if $\Delta \vdash \triple{P}{\mathds{P}}{Q}$ then $\Delta \vDash \triple{P}{\mathds{P}}{Q}$.
\end{thm}