\vspace*{\fill}
\section*{Abstract}

Two-phase locking (\tpl) is a concurrency control protocol, often implemented as part of a database or transactional system, in order to manage concurrent access to its underlying data. The protocol guarantees the serializability of any execution that happens in the system. This is a strong consistency property which requires the outcome of any schedule of operations, coming from concurrent transactions, to be equivalent to a serial one. \\ \\
We present a formal model and an operational semantics that fully express the two-phase locking behaviours as part of a generic software system, supporting transactions which can interact with a global memory storage. This framework enables the proof of the aforementioned serializability property, which is done as part of this work, and unties  reasoning in a \tpl\ setting from its low-level details, by instead allowing users to think in a serial way. \\ \\
The obtained results are then used to formulate a logic for concurrent transactional programs, that are able to manipulate a memory heap. The program logic is proven to be sound with respect to our two-phase locking semantics, by first showing its soundness to semantics that treat transactions as atomic blocks, and then proving that any \tpl\ execution has an equivalent in the atomic world.
\vspace*{\fill}