\subsection{Worlds} \label{worlds}

A \textit{world} is the structure that represents the entirety of the state of existing resources together with a way to describe how these can be modified. We now informally present all of a world's components and properties which are later formalised. A world is in fact a \textit{well-formed} triple $(l, g, \mathcal{J})$, where:
\begin{itemize}
	\item The \textit{logical local state} $l$ represents the resources which are locally owned by a thread and are not externally accessible, i.e. no other thread can see nor update them.
	\item The \textit{shared state} $g$ represents all of the globally shared resources which are divided into regions and accessible to all threads.
	\item The \textit{action model} $\mathcal{J}$ describes how, for every region in the shared state, a thread holding a particular \textit{capability} can update the region. Action models include partial functions, one per region, which associate capabilities to \textit{actions}. An action is a pair $(s, s')$ of logical states where $s$ is the \textit{pre-state}, the state of the region before the action is applied, and $s'$ is the \textit{post-state} which is the state of the region after the action takes place.
\end{itemize}

Worlds can be composed whenever they have the same shared state, action model and a disjoint local logical state. We now proceed to define the latter as tuples whose first component is a machine state, the heap, while the second one is a mapping from regions to capabilities held by the thread for that particular region. Given that mCAP is parametric with respect to the partial commutative monoid representing machine states and capabilities, we allow users of the framework to chose a suitable instantiation to tackle a particular program verification.

\begin{param}
	(Machine states pcm).
	Assume a partial, commutative monoid with multiple units which represents \emph{machine states}, $(\mathbb{M}, \bullet_\mathbb{M}, \mathbf{0}_\mathbb{M})$. Elements of $\mathbb{M}$ are ranged over by $m, m_1, \ldots, m_n$.
\end{param}

\begin{param}
	(Primitive capability pcm).
	Assume a partial, commutative monoid with multiple units which represents \emph{primitive capability resources}, $(\mathbb{K}, \bullet_\mathbb{K}, \mathbf{0}_\mathbb{K})$. Elements of $\mathbb{K}$ are ranged over by $\kappa, \kappa_1, \ldots, \kappa_n$.
\end{param}

Given that the shared state in a mCAP world is organized into regions, we require a way to uniquely characterize each of those. We therefore define distinct region identifiers.
\begin{defn}
	(Region identifiers).
	Assume a set of \emph{region identifiers} $\mathsf{Rid}$, ranged over by $r, r_1, \ldots, r_n$.
\end{defn}

\begin{defn}
	(Capability).
	Given a pcm for primitive capabilities, $(\mathbb{K}, \bullet_\mathbb{K}, \mathbf{0}_\mathbb{K})$, the set of \emph{capabilities}, $\mathsf{RKap}$, is defined as the set of partial finite functions from region identifiers to primitive capabilities.
	\[
		\mathsf{RKap} \triangleq \mathsf{Rid} \overset{\text{fin}}{\rightharpoonup} \mathbb{K}
	\]
	The $\mathsf{RKap}$ set is ranged over by $\rho, \rho_1, \ldots, \rho_n$. Composition on capabilities, $\circ : \mathsf{RKap} \times \mathsf{RKap} \rightharpoonup \mathsf{RKap}$, is defined as follows:
	\[
		(\rho \circ \rho') (r)
			\triangleq
		\begin{cases}
			\rho(r), & \text{if } r \not\in \pred{dom}{\rho'}
			\\
			\rho'(r), & \text{if } r \not\in \pred{dom}{\rho}
			\\
			\rho(r) \bullet_\mathbb{K} \rho'(r), & \text{otherwise}
		\end{cases}
	\]
	The capabilities pcm is defined as $(\mathsf{RKap}, \circ, \mathbf{0}_{\mathsf{RK}})$, where the capability unit set, $\mathbf{0}_{\mathsf{RK}} : \mathsf{Rid} \overset{\text{fin}}{\rightharpoonup} \mathbb{K}$, is the function with an empty domain. 
\end{defn}
One can see region capabilities as a way to record what capabilities are held for regions present in the world's shared state. Such capabilities can be owned by a thread in its local state or be included inside of a region.

A logical state is therefore a pair $(m, \rho)$ where $m \in \mathbb{M}$ is a machine state and $\rho \in \mathsf{RKap}$ is a region capability, describing what resources are owned and which capabilities are held.
\begin{defn}
	(Logical states).
	Given a pcm for machine states $(\mathbb{M}, \bullet_\mathbb{M}, \mathbf{0}_\mathbb{M})$, and one for capabilities $(\mathbb{K}, \bullet_\mathbb{K}, \mathbf{0}_\mathbb{K})$, the set of \emph{logical states}, $\mathsf{LStates}$, ranged over by $l, l_1, \ldots, l_n$ is defined as:
	\[
		\mathsf{LState} \triangleq \mathbb{M} \times \mathsf{RKap}
	\]
	Composition on logical states, $\circ : \mathsf{LState} \times \mathsf{LState} \rightharpoonup \mathsf{LState}$, is defined as:
	\[
		(m, \rho) \circ (m', \rho') \triangleq (m \bullet_\mathbb{M} m', \rho \circ \rho')
	\]
	The logical states unit set is $\mathbf{0}_\mathsf{L} = \{ (m, \rho)\ |\ m \in \mathbf{0}_\mathbb{M} \land \rho \in \mathbf{0}_\mathsf{RK} \}$ and the pcm of logical states is defined as $(\mathsf{LState}, \circ, \mathbf{0}_\mathsf{L})$.
\end{defn}
Given a logical state $l$ we use $l_\mathsf{M}$ and $l_\mathsf{K}$ to refer to its first and second projections respectively.

\begin{defn}
	(Shared states).
	The set of \emph{shared states} $\mathsf{GState}$, ranged over by $g, g_1, \ldots g_n$, is defined as the set of finite partial functions, mapping region identifiers to logical states:
	\[
		\mathsf{GState} \triangleq \mathsf{Rid} \overset{\text{fin}}{\rightharpoonup} \mathsf{LState}
	\]
	The \emph{combination function}, $\llfloor - \rrfloor : \mathsf{GState} \rightarrow \mathsf{LState}$ is defined as:
	\[
		\llfloor g \rrfloor \triangleq \prod^{\circ}_{r \in \pred{dom}{g}} g(r)
	\]
	The \textit{cross-composition} function between logical states and shared states, $\oplus : \mathsf{LState} \times \mathsf{GState} \rightharpoonup \mathsf{LState}$, is defined as:
	\[
		l \oplus g \triangleq l \circ \llfloor g \rrfloor
	\]
\end{defn}

\begin{defn}
	(Action models).
	The set of \emph{actions} $\mathsf{Action}$, is defined as the set of tuples of logical states.
	\[
		\mathsf{Action} \triangleq \mathsf{LState} \times \mathsf{LState}
	\]
	Actions are used as part of of action models. The set of \emph{action models}, $\mathsf{AMod}$, ranged over by $\mathcal{J}, \mathcal{J}_1, \ldots, \mathcal{J}_n$, is defined as follows.
	\[
		\mathsf{AMod} \triangleq \mathsf{Rid} \overset{\text{fin}}{\rightharpoonup} \mathbb{K} \overset{\text{fin}}{\rightharpoonup} \mathcal{P}(\mathsf{Action})
	\]
	An action model with an empty domain is denoted by $\emptyset$.
\end{defn}
We refer to the first component of an action $(l, l')$ as the \emph{pre-state} while the second one as the \emph{post-state}. 

\begin{defn}
	(Well-formedness).
	A given triple $(l, g, \mathcal{J}) : \mathsf{LState} \times \mathsf{GState} \times \mathsf{AMod}$ is \emph{well-formed}, written $\pred{wf}{(l, g, \mathcal{J})}$ when the cross-composition of the local and shared state is defined, the resulting region capability is contained in the action model and the regions in the shared state are the same as the ones described by the action model.
	\[
		\pred{wf}{(l, g, \mathcal{J})} \iff \exists l' \ldotp l \oplus g = l' \land \pred{dom}{l'_\mathsf{K}} \subseteq \pred{dom}{\mathcal{J}} \land \pred{dom}{g} = \pred{dom}{\mathcal{J}}
	\]
\end{defn}

\begin{defn}
	(World).
	The set of all \emph{worlds}, $\mathsf{World}$, ranged over by $w, w_1, \ldots w_n$, is defined as the set of well-formed triples containing a local state, a global one and an action model.
	\[
		\mathsf{World} \triangleq \{ w \in \mathsf{LState} \times \mathsf{GState} \times \mathsf{AMod}\ |\ \pred{wf}{w} \}
	\]
	Composition on worlds, $\bullet : \mathsf{World} \times \mathsf{World} \rightharpoonup \mathsf{World}$, is defined by composing local states and requiring that shared states and action models be identical.
	\[
		(l, g, \mathcal{J}) \bullet (l', g', \mathcal{J}') \triangleq
		\begin{cases}
			(l \circ l', g, \mathcal{J}), & \text{if } g = g' \text{ and } \mathcal{J} = \mathcal{J}' \\ & \text{and } \pred{wf}{(l \circ l', g, \mathcal{J})}
			\\
			\mathsf{undef} & \text{otherwise}
		\end{cases}
	\]
	The worlds pcm is defined as $(\mathsf{World}, \bullet, \mathbf{0}_\mathsf{W})$, where the worlds unit set is defined as any well-defined world whose local state is part of the logical state unit set, $\mathbf{0}_\mathsf{L}$:
	\[
		\mathbf{0}_\mathsf{W} = \{ (l, g, \mathcal{J})\ |\ (l, g, \mathcal{J}) \in \mathsf{World} \land l \in \mathbf{0}_\mathsf{L} \}
	\]
\end{defn}
Given a world $w$, we write $w_\mathsf{L}, w_\mathsf{S}$ and $w_\mathsf{A}$ for its first, second and third projections.