\chapter{The mCAP Program Logic}

\label{sec:mcapModel}

The path to the definition of a program logic for serializable transactions starts in this section, where we lay its foundations by taking elements from existing logics for concurrency, and expand it to a general framework that can be instantiated for one's needs. As a consequence, we provide the first formalisation of the mCAP logic model (where the \textit{m} stands for \textit{modern}) and later of its semantics. This work builds upon the CAP program logic, presented in Section \ref{sec:cap} and first introduced in \cite{cap}, and borrows some of the definitions and proof strategies from CoLoSL \cite{colosl} \cite{azaleaThesis}. The main novelties introduced in mCAP, as compared to regular CAP, are concerned with freeing its usage as a logic from some hard constraints and create the necessary space for transactional reasoning. mCAP is in fact parametric with respect to the region capabilities (or \textit{guards}) used, it allows multiple region updates at once through the repartitioning operation, and does not enforce the entirety of a region capability to be held in order to construct the region itself.

The program logic that we later provide in Section \ref{sec:transLogic} will be a concrete instantiation of mCAP and will enable us to reason about systems adopting real transactional atomicity and later about ones using \tpl\ to handle their concurrency control.

\subsection{Separation Algebra}

The concept of a \textit{separation algebra} was introduced in \cite{sepalgebra} to abstract from the standard separation logic model for heaps equipped with an operator to compose them. We introduce such concept here and list some of its core properties.

\defn (Separation algebra). A separation agebra $(\mathcal{M}, \bullet, I)$ \cite{views} is a partial, commutative monoid with multiple units, where $\mathcal{M}$ is a set equipped with a partial operator $\bullet : \mathcal{M} \times \mathcal{M} \rightharpoonup \mathcal{M}$ and a unit set $I \subseteq \mathcal{M}$ satisfying:
\begin{itemize}
\item Commutativity: $m_1 \bullet m_2 = m_2 \bullet m_1$ when either is defined.
\item Associativity: $m_1 \bullet (m_2 \bullet m_3) = (m_1 \bullet m_2) \bullet m_3$ when either is defined.
\item Existence of unit: for all $m \in \mathcal{M}$ there exists $i \in I$ such that $i \bullet m = m$.
\item Minimality of unit: for all $m \in \mathcal{M}$ and $i \in I$, if $i \bullet m$ is defined then $i \bullet m = m$.
\item Cancellativity: for all $m_1, m_2, y, z \in \mathcal{M}$ if $m_1 \bullet y = z$ and $m_2 \bullet y = z$ then $m_1 = m_2$.
\end{itemize}

\defn (Ordering). Given a separation algebra $(\mathcal{M}, \bullet, \mathbf{0})$, the ordering relation $\leq : \mathcal{M} \times \mathcal{M}$ is defined as:
\[
	\leq \triangleq \{ (m_1, m_2)\ |\ \exists m \ldotp m_1 \bullet m = m_2 \}
\]
We write $m_1 \leq m_2$ for $(m_1, m_2) \in \leq$.

\defn (Compatibility). Given a separation algebra $(\mathcal{M}, \bullet, \mathbf{0})$, the compatibility property $\sharp : \mathcal{M} \times \mathcal{M}$ is defined as:
\[
	\sharp \triangleq \{ (m_1, m_2)\ |\ \exists m \ldotp m_1 \bullet m_2 = m \}
\]

\defn (Orthogonal). Given a separation algebra $(\mathcal{M}, \bullet, \mathbf{0})$ and an element $m \in \mathcal{M}$, its orthogonal $(-)^\bot_\mathcal{M} : \mathcal{M} \rightarrow \mathcal{P}(\mathcal{M})$ is the set of all elements in $\mathcal{M}$ which are compatible with it.
\[
	(m)^\bot_\mathcal{M} \triangleq \{m'\ |\ m\ \sharp\ m' \}
\]

\defn (Cross-split property) A separation agebra $(\mathcal{M}, \bullet, I)$ complies with the cross-split property iff:
\begin{gather*}
	\forall a, b, c, d, z \ldotp a \bullet b = z \land c \bullet d = z \implies \\ \exists ac, ad, bc, bd \ldotp ac \bullet ad = a \land ac \bullet bc = c \land bc \bullet bd = b \land ad \bullet bd = d
\end{gather*}

\subsection{Worlds} \label{worlds}

A \textit{world} is the structure that represents all of the existing resources and their states, together with a way to describe how these can be modified. We now informally present all of a world's components and properties which are later formalised. A world is in fact a \textit{well-formed} triple $(l, g, \mathcal{J})$, where:
\begin{itemize}
	\item The \textit{logical local state} $l$ represents the resources which are locally owned by a thread and are not externally accessible, i.e. no other thread can see nor update them.
	\item The \textit{shared state} $g$ represents all of the globally shared resources which are divided into regions and accessible to all threads.
	\item The \textit{action model} $\mathcal{J}$ describes how, for every region in the shared state, a thread holding a particular \textit{capability} can update the region. Action models include partial functions, one per region, which associate capabilities to \textit{actions}. An action is a pair $(s, s')$ of logical states where $s$ is the \textit{pre-state}, the state of the region before the action is applied, and $s'$ is the \textit{post-state} which is the state of the region after the action takes place.
\end{itemize}

Worlds can be composed whenever they have the same shared state, action model and a disjoint local logical state. We now proceed to define the latter as tuples whose first component is a machine state, the heap, while the second one is a mapping from regions to capabilities held by the thread for that particular region. Given that mCAP is parametric with respect to the partial commutative monoid representing machine states and capabilities, we allow users of the framework to chose a suitable instantiation to tackle a particular program verification.

\begin{param}
	\label{param:machineStates}
	(Machine states pcm).
	Assume a partial, commutative monoid with multiple units which represents \emph{machine states}, $(\mathbb{M}, \bullet_\mathbb{M}, \mathbf{0}_\mathbb{M})$. Elements of $\mathbb{M}$ are ranged over by $m, m_1, \ldots, m_n$.
\end{param}

\begin{param}
	(Primitive capability pcm).
	Assume a partial, commutative monoid with multiple units which represents \emph{primitive capability resources}, $(\mathbb{K}, \bullet_\mathbb{K}, \mathbf{0}_\mathbb{K})$. Elements of $\mathbb{K}$ are ranged over by $\kappa, \kappa_1, \ldots, \kappa_n$.
\end{param}

Given that the shared state in a mCAP world is organized into regions, we require a way to uniquely characterize each of those. We therefore define distinct region identifiers.
\begin{defn}
	(Region identifiers).
	Assume a set of \emph{region identifiers} $\mathsf{Rid}$, ranged over by $r, r_1, \ldots, r_n$.
\end{defn}

\begin{defn}
	(Capability).
	Given a pcm for primitive capabilities, $(\mathbb{K}, \bullet_\mathbb{K}, \mathbf{0}_\mathbb{K})$, the set of \emph{capabilities}, $\mathsf{RKap}$, is defined as the set of partial functions with a finite domain from region identifiers to primitive capabilities.
	\[
		\mathsf{RKap} \triangleq \mathsf{Rid} \overset{\text{fin}}{\rightharpoonup} \mathbb{K}
	\]
	The $\mathsf{RKap}$ set is ranged over by $\rho, \rho_1, \ldots, \rho_n$. Composition on capabilities, $\circ : \mathsf{RKap} \times \mathsf{RKap} \rightharpoonup \mathsf{RKap}$, is defined as follows:
	\[
		(\rho \circ \rho') (r)
			\triangleq
		\begin{cases}
			\rho(r), & \text{if } r \not\in \pred{dom}{\rho'}
			\\
			\rho'(r), & \text{if } r \not\in \pred{dom}{\rho}
			\\
			\rho(r) \bullet_\mathbb{K} \rho'(r), & \text{otherwise}
		\end{cases}
	\]
	The capabilities pcm is defined as $(\mathsf{RKap}, \circ, \mathbf{0}_{\mathsf{RK}})$, where the capability unit set, $\mathbf{0}_{\mathsf{RK}} : \mathsf{Rid} \overset{\text{fin}}{\rightharpoonup} \mathbb{K}$, is the function with an empty domain. 
\end{defn}
One can see region capabilities as a way to record what capabilities are held for regions present in the world's shared state. Such capabilities can be owned by a thread in its local state or be included inside of a region.

A logical state is therefore a pair $(m, \rho)$ where $m \in \mathbb{M}$ is a machine state and $\rho \in \mathsf{RKap}$ is a region capability, describing what resources are owned and which capabilities are held.
\begin{defn}
	(Logical states).
	Given a pcm for machine states $(\mathbb{M}, \bullet_\mathbb{M}, \mathbf{0}_\mathbb{M})$, and one for capabilities $(\mathbb{K}, \bullet_\mathbb{K}, \mathbf{0}_\mathbb{K})$, the set of \emph{logical states}, $\mathsf{LStates}$, ranged over by $l, l_1, \ldots, l_n$ is defined as:
	\[
		\mathsf{LState} \triangleq \mathbb{M} \times \mathsf{RKap}
	\]
	Composition on logical states, $\circ : \mathsf{LState} \times \mathsf{LState} \rightharpoonup \mathsf{LState}$, is defined as:
	\[
		(m, \rho) \circ (m', \rho') \triangleq (m \bullet_\mathbb{M} m', \rho \circ \rho')
	\]
	The logical states unit set is $\mathbf{0}_\mathsf{L} = \{ (m, \rho)\ |\ m \in \mathbf{0}_\mathbb{M} \land \rho \in \mathbf{0}_\mathsf{RK} \}$ and the pcm of logical states is defined as $(\mathsf{LState}, \circ, \mathbf{0}_\mathsf{L})$.
\end{defn}
Given a logical state $l$ we use $l_\mathsf{M}$ and $l_\mathsf{K}$ to refer to its first and second projections respectively.

\begin{defn}
	(Shared states).
	The set of \emph{shared states} $\mathsf{GState}$, ranged over by $g, g_1, \ldots g_n$, is defined as the set of partial functions with a finite domain, mapping region identifiers to logical states:
	\[
		\mathsf{GState} \triangleq \mathsf{Rid} \overset{\text{fin}}{\rightharpoonup} \mathsf{LState}
	\]
	The \emph{combination function}, $\llfloor - \rrfloor : \mathsf{GState} \rightarrow \mathsf{LState}$ is defined as:
	\[
		\llfloor g \rrfloor \triangleq \prod^{\circ}_{r \in \pred{dom}{g}} g(r)
	\]
	The \textit{cross-composition} function between logical states and shared states, $\oplus : \mathsf{LState} \times \mathsf{GState} \rightharpoonup \mathsf{LState}$, is defined as:
	\[
		l \oplus g \triangleq l \circ \llfloor g \rrfloor
	\]
\end{defn}

\begin{defn}
	(Action models).
	The set of \emph{actions} $\mathsf{Action}$, is defined as the set of tuples of logical states.
	\[
		\mathsf{Action} \triangleq \mathsf{LState} \times \mathsf{LState}
	\]
	Actions are used as part of of action models. The set of \emph{action models}, $\mathsf{AMod}$, ranged over by $\mathcal{J}, \mathcal{J}_1, \ldots, \mathcal{J}_n$, is defined as follows.
	\[
		\mathsf{AMod} \triangleq \mathsf{Rid} \overset{\text{fin}}{\rightharpoonup} \mathbb{K} \overset{\text{fin}}{\rightharpoonup} \mathcal{P}(\mathsf{Action})
	\]
	An action model with an empty domain is simply denoted by $\emptyset$.
\end{defn}
We refer to the first component of an action $(l, l')$ as the \emph{pre-state} while the second one as the \emph{post-state}. 

\begin{defn}
	(Well-formedness).
	A given triple $(l, g, \mathcal{J}) : \mathsf{LState} \times \mathsf{GState} \times \mathsf{AMod}$ is \emph{well-formed}, written $\pred{wf}{(l, g, \mathcal{J})}$ when the cross-composition of the local and shared state is defined, the resulting region capability is contained in the action model and the regions in the shared state are the same as the ones described by the action model.
	\[
		\pred{wf}{(l, g, \mathcal{J})} \iff \exists l' \ldotp l \oplus g = l' \land \pred{dom}{l'_\mathsf{K}} \subseteq \pred{dom}{\mathcal{J}} \land \pred{dom}{g} = \pred{dom}{\mathcal{J}}
	\]
\end{defn}

\begin{defn}
	(World).
	The set of all \emph{worlds}, $\mathsf{World}$, ranged over by $w, w_1, \ldots w_n$, is defined as the set of well-formed triples containing a local state, a global one and an action model.
	\[
		\mathsf{World} \triangleq \{ w \in \mathsf{LState} \times \mathsf{GState} \times \mathsf{AMod}\ |\ \pred{wf}{w} \}
	\]
	Composition on worlds, $\bullet : \mathsf{World} \times \mathsf{World} \rightharpoonup \mathsf{World}$, is defined by composing local states and requiring that shared states and action models be identical.
	\[
		(l, g, \mathcal{J}) \bullet (l', g', \mathcal{J}') \triangleq
		\begin{cases}
			(l \circ l', g, \mathcal{J}), & \text{if } g = g' \text{ and } \mathcal{J} = \mathcal{J}' \\ & \text{and } \pred{wf}{(l \circ l', g, \mathcal{J})}
			\\
			\mathsf{undef} & \text{otherwise}
		\end{cases}
	\]
	The worlds pcm is defined as $(\mathsf{World}, \bullet, \mathbf{0}_\mathsf{W})$, where the worlds unit set is defined as any well-defined world whose local state is part of the logical state unit set, $\mathbf{0}_\mathsf{L}$:
	\[
		\mathbf{0}_\mathsf{W} = \{ (l, g, \mathcal{J})\ |\ (l, g, \mathcal{J}) \in \mathsf{World} \land l \in \mathbf{0}_\mathsf{L} \}
	\]
\end{defn}
Given a world $w$, we write $w_\mathsf{L}, w_\mathsf{S}$ and $w_\mathsf{A}$ for its first, second and third projections.

\section{Assertions}

Assertions in mCAP follow the ones in standard CAP with the only difference of being parametric with respect to the machine state assertions and capability assertions that describe elements of the given sets $\mathbb{M}$ and $\mathbb{K}$ respectively. Judgements in mCAP have the shape $\Delta \vdash \triple{P}{\mathds{P}}{Q}$ where $P, Q$ are \textit{assertions}, $\mathds{P}$ is a program and $\Delta$ is an \textit{axiom definition}, all defined in this section.

We assume the presence of an infinite set of logical variables, $x \in \mathsf{LVar}$ and logical environments, $\mathsf{LEnv}$, such that $e \in \mathsf{LEnv} \triangleq \mathsf{LVar} \rightarrow \mathsf{Val}$. Logical environments associate logical variables with their values. Also, since mCAP is an extension of CAP, we provide support for abstract predicates used to express concrete properties. We assume to be given a set of predicate environments, $\mathsf{PEnv}$ that associate each predicate name (coming from the set $\mathsf{PName}$) and its arguments, to a set of worlds that satisfy them. Formally $\delta \in \mathsf{PEnv} \triangleq \mathsf{PName} \times \mathsf{Val}^* \rightarrow \mathcal{P}(\mathsf{World})$.

\begin{param}
	(Machine state assertions).
	Assume a set of \emph{machine state assertions} $\mathsf{MAssn}$, ranged over by $\mathcal{M}, \mathcal{M}_1, \ldots, \mathcal{M}_n$. Given the machine states pcm $(\mathbb{M}, \bullet_\mathbb{M}, \mathbf{0}_\mathbb{M})$, assume an assertion semantics function, that maps machine state assertions to the elements of $\mathbb{M}$ it describes:
	\[
		\tsem{-}_-^\textsc{m} : \mathsf{MAssn} \rightarrow \mathsf{LEnv} \rightarrow \mathcal{P}(\mathbb{M})
	\]
\end{param}

\begin{param}
	(Capability assertions).
	Assume a set of \emph{capability assertions} $\mathsf{KAssn}$ ranged over by $\mathcal{K}, \mathcal{K}_1, \ldots, \mathcal{K}_n$ and an associated semantics function that maps such assertions to elements of the capability separation algebra given as $(\mathbb{K}, \bullet_\mathbb{K}, \mathbf{0}_\mathbb{K})$.
	\[
		\tsem{-}_-^\textsc{k} : \mathsf{KAssn} \rightarrow \mathsf{LEnv} \rightarrow \mathcal{P}(\mathbb{K})
	\]
\end{param}

\begin{defn}
	(Assertion syntax).
	Assertions are elements of the $\mathsf{Assn}$ set defined by the following grammar, where $x \in \mathsf{LVar}, r \in \mathsf{Rid}$ and $\alpha \in \mathsf{PName}$. 
	\begin{align*}
		A &::= \mathtt{false}\ |\ \mathtt{emp}\ |\ \mathcal{M}\ |\ [\mathcal{K}]^r \\
		p, q \in \mathsf{LAssn} &::= A\ |\ \lnot p\ |\ \exists x \ldotp p\ |\ p \lor q\ |\ p \sep q\ |\ p \sepimp q \\
		P, Q \in \mathsf{Assn} &::= p\ |\ P \lor Q\ |\ \exists x \ldotp P\ |\ P \sep Q\ |\ \boxed{P}_I^r\ |\ \alpha(\mathds{E}_1, \ldots, \mathds{E}_n) \\
		I \in \mathsf{IAssn} &::= \emptyset\ |\ \{ \mathcal{K} : \exists \vec{y} \ldotp P \leadsto Q \} \cup I \\
		\Delta \in \mathsf{Ax} &::= \emptyset\ |\ \forall \vec{x} \ldotp P \implies Q\ |\ \forall \vec{x} \ldotp \alpha(\vec{x}) \equiv P\ |\ \Delta_1, \Delta_2
	\end{align*}
\end{defn}

The structure of assertions follows from the separation logic one, with the addition of the given capability assertions $[\mathcal{K}]^r$, shared region (or \textit{boxed}) assertions $\boxed{P}^r_I$ and \textit{interference} assertions $I$. Machine state assertions, $\mathcal{M}$ are interpreted over a world's local logical state, i.e. $\mathcal{M}$ holds true when it is satisfied by $m$, as part of $(m, \rho)$, for a $\rho \in \mathbf{0}_\mathsf{RK}$. In a similar way, a capability assertion $[\mathcal{K}]^r$ is true in a logical state $(m, [r \mapsto \kappa])$ for $m \in\mathbf{0}_\mathbb{M}$, a shared region $r$ and capability $\kappa$ that satisfies $\mathcal{K}$. The assertion $\mathtt{emp}$ is only satisfied by the unit of logical states, formally any $l \in \mathbf{0}_\mathsf{L}$. The separating assertion $P \sep Q$ is true of worlds that can be split in two parts such that the first one satisfies $P$ while the second one $Q$. A predicate assertion holds in all worlds belonging to its semantic interpretation as given by the environment $\delta$. The predicate's arguments are evaluated with respect to the logical environment $e$.

An interference assertion $I$, describes actions that are enabled by a particular capability in the form of pre-condition and post-condition. $I$ is checked against the action model $\mathcal{J}$ to verify that, for the region $r$ it describes, every pre-condition $P$ and post-condition $Q$ are satisfied by $\mathcal{J}$'s actions pre- and post-state. The variables that are existentially bound as part of $I$'s action ($\vec{y}$) are set in the logical environment $e$ to be the same when checking for satisfaction of $P$ and $Q$. This is practically done by building (in Definition \ref{defn:interferenceSem}) a \textit{tiny} action model out of the assertions in $I$ and then checking that it is contained inside of $\mathcal{J}$.

Boxed assertions of the shape $\boxed{P}^r_I$ are true of worlds $(l, g, \mathcal{J})$ where the local state is empty, $l \in \mathbf{0}_\mathsf{L}$, and the logical state associated to $r$ inside of $g$ satisfies $P$. We also need to make sure that the interference assertion $I$, attached to region $r$, is satisfied by $\mathcal{J}(r)$, as described in the previous paragraph.

Finally, a syntactic predicate environment $\Delta$, allows to use assertions in order to build a parametric predicate and store its meaning throughout a proof. In Section \ref{sec:mcapLogic} we will study their semantic interpretation and see how are they used.

\begin{defn}
	\label{defn:interferenceSem}
	(Interference assertion semantics).
	The semantics of \emph{interference assertions}, $\tsem{-}^\textsc{i}_{r, e, \delta, \mathcal{J}} : \mathsf{IAssn} \times \mathsf{Rid} \times \mathsf{LEnv} \times \mathsf{PEnv} \times \mathsf{AMod} \rightarrow \mathsf{Rid} \overset{\text{fin}}{\rightharpoonup} \mathbb{K} \overset{\text{fin}}{\rightharpoonup} \mathcal{P}(\mathsf{Action})$, are defined as:
	\begin{align*}
		\tsem{\emptyset}^\textsc{i}_{r, e, \delta, \mathcal{J}}(r')(\kappa) &\triangleq \emptyset
		\\
		\tsem{\mathcal{K}: \exists \vec{y} \ldotp P \transto Q \cup I}^\textsc{i}_{r, e, \delta, \mathcal{J}}(r')(\kappa) &\triangleq
			\tsem{I}^\textsc{i}_{r, e, \delta, \mathcal{J}}(r')(\kappa)
			\cup 
			\begin{cases}
				\emptyset, & \text{if } r' \neq r
				\\
				(l_p, l_q), & \text{if } \exists g_p, g_q, e', \vec{v} \ldotp
				\kappa \in \tsem{\mathcal{K}}^\textsc{k}_e
				\\
				& \land g_p(r) = l_p \land g_q(r) = l_q
				\\
				& \land \forall \dot{r} \neq r \ldotp g_p(\dot{r}) = g_q(\dot{r})
				\\
				& \land (l_p, g_p, \mathcal{J}), e', \delta \vDash P
				\\
				& \land (l_q, g_q, \mathcal{J}), e', \delta \vDash Q
				\\
				& \land e' = e[\vec{y} \mapsto \vec{v}] 
			\end{cases}
	\end{align*}
\end{defn}
The way we determine semantics for interference assertions requires a precise analysis. We are effectively building a function that, once given an interference assertion $I$, region identifier $r$, logical environment, predicate environment and action model returns a new action model, $\mathcal{J}_I$. The latter only describes all of the actions modelled by $I$ and referring to region $r$. In fact, when we intepret a single action's intereference assertion $\mathcal{K}: \exists \vec{y} \ldotp P \transto Q$, we build a function that, once queried for actions in region $r$ associated to a given capability $\kappa$, will return a tuple of logical states $(l_p, l_q)$. Here, $l_p$ is the local state of the a world built as $(l_p, g_p, \mathcal{J})$ that is used, together with a logical environment $e'$ where all of the $\vec{y}$ in $P$ and $Q$ are bound to values, in order to satisfy assertion $P$. In a similary way, $(l_q, g_q, \mathcal{J})$ is constructed to satisfy $Q$ with the same environment $e'$. The shared state components of these \textit{artificial} worlds are such that they are equivalent for all regions but $r$. On top of this, $g_p(r) = l_p$ and $g_q(r) = l_q$. For interference assertions describing a number of actions, we recursively union the result of a single interpretation with the rest of the actions in $I$.

\begin{defn}
	(Assertion semantics).
	Assertion semantics are given with respect to a world $w \in \mathsf{World}$, a logical environment $e \in \mathsf{LEnv}$ and a predicate environment $\delta \in \mathsf{PEnv}$. They tell us whether a configuration $w, e, \delta$ satisfies a particular assertion.
	\begingroup
	\renewcommand*{\arraystretch}{1.5}
	\[
	\begin{array}{r c l}
		(l, g, \mathcal{J}), e, \delta \vDash p
		&
		\iff
		&
		l, e \vDash_\mathsf{SL} p
	\\
		(l, g, \mathcal{J}), e, delta \vDash \boxed{P}_I^r
		&
		\iff
		&
		l \in \mathbf{0}_\mathsf{L} \text{ and } \exists l' \ldotp (l', g, \mathcal{J}), e, \delta \vDash P
		\\ && \text{and } \exists r' \ldotp r' = e(r) \land g(r') = l' \land \tsem{I}^\textsc{i}_{r', e, \delta, \mathcal{J}} \subseteq \mathcal{J}
	\\
		w, e, \delta \vDash \alpha(\mathds{E}_1, \ldots, \mathds{E}_n)
		&
		\iff
		&
		w \in \delta(\alpha, \tsem{\mathds{E}_1}_e^\textsc{e}, \ldots, \tsem{\mathds{E}_n}_e^\textsc{e})
	\\
		w, e, \delta \vDash \exists x \ldotp P
		&
		\iff
		&
		\exists v \ldotp w, e[x \mapsto v], \delta \vDash P
	\\
		w, e, \delta \vDash P \lor Q
		&
		\iff
		&
		w, e, \delta \vDash P \text{ or } w, e, \delta \vDash Q
	\\
		w, e, \delta \vDash P \sep Q
		&
		\iff
		&
		\exists w_1, w_2 \ldotp w = w_1 \bullet w_2 \text{ and } \\ && w_1, e \vDash P \text{ and } w_2, e \vDash Q
	\\
		l, e \vDash_\mathsf{SL} \mathtt{false}
		&&
		\text{never}
	\\
		l, e \vDash_\mathsf{SL}  \mathtt{emp}
		&
		\iff
		&
		l \in \mathbf{0}_\mathsf{L}
	\\
		l, e \vDash_\mathsf{SL} \mathcal{M}
		&
		\iff
		&
		\exists m, \rho \ldotp l = (m, \rho) \text{ and } m \in \tsem{\mathcal{M}}_e^\textsc{m} \text{ and } \rho \in \mathbf{0}_\mathsf{RK}
	\\
		l, e \vDash_\mathsf{SL} [\mathcal{K}]^r
		&
		\iff
		&
		\exists m, \kappa \ldotp l = (m, [r \mapsto \kappa]) \text{ and } \kappa \in \tsem{\mathcal{K}}_e^\textsc{k} \text{ and } m \in \mathbf{0}_\mathbb{M}
	\\
		l, e \vDash_\mathsf{SL} \lnot p
		&
		\iff
		&
		l, e \not\vDash_\mathsf{SL} p
	\\
		l, e \vDash_\mathsf{SL} p \sepimp q
		&
		\iff
		&
		\forall l' \ldotp l', e \vDash_\mathsf{SL} p \text{ implies } l \circ l', e \vDash_\mathsf{SL} q
	\\
		l, e \vDash_\mathsf{SL} p \sep q
		&
		\iff
		&
		\exists l_1, l_2 \ldotp l = l_1 \circ l_2 \text{ and } \\
		&& l_1, e \vDash_\mathsf{SL} p \text{ and } l_2, e \vDash_\mathsf{SL} q
	\\
		 l, e \vDash_\mathsf{SL} p \lor q
		 &
		 \iff
		 &
		 l, e \vDash_\mathsf{SL} p \text{ or } l, e \vDash_\mathsf{SL} q
	\\
		l, e \vDash_\mathsf{SL} \exists x \ldotp p
		&
		\iff
		&
		\exists v \ldotp l, e[x \mapsto v] \vDash_\mathsf{SL} p
	\end{array}
	\]
	\endgroup
\end{defn}
Now, given a logical environment $e \in \mathsf{LEnv}$ and a predicate environment $\delta \in \mathsf{PEnv}$, we write $\tsem{P}_{e, \delta}$ for the set of all worlds satisfying assertion $P$ under the environments $e$ and $\delta$.
\[
	\tsem{P}_{e, \delta} \triangleq \{ w\ |\ w, e, \delta \vDash P \}
\]

Similarly to CAP, the separating conjunction of shared state assertions on the same region is interpreted as a regular non-separating conjunction, $\land$, formally:
\[
	\boxed{P}^r_I \sep \boxed{Q}^r_I \iff \boxed{P \land Q}^r_I
\]
We also syntactically allow nested regions, but they can always be separated. Their semantic meaning is therefore expressed as follows:
\[
	\boxed{\boxed{P}^r_I \sep Q}^{r'}_{I'} \iff \boxed{P}^r_I \sep \boxed{Q}^{r'}_{I'}
\]