\section{Trace equivalence}

\[
	\alpha(\iota, k) \triangleq \alpha \text{ s.t. }
	\alpha \in 
		\{
			\actread{\iota}{k}{v},
			\actwrite{\iota}{k}{v},
			\actlock{\iota}{k}{\kappa},
			\actunlock{\iota}{k}\
			|\ v \in \mathsf{Val}, \kappa \in \mathsf{Lock}
		\}
\]
\begin{align*}
	\pred{tgen}{[], h, \underline{h}, \Phi, S, \mathds{P}}
		\iff&
	h = \underline{h} \land \mathds{P} = \pskip \land \Phi = \emptyset
		\\
	\pred{tgen}{(\alpha, n) : \tau, h, \underline{h}, \Phi, S, \mathds{P}}
		\iff&
	\exists h', \Phi', S', \mathds{P}' \ldotp (h, \Phi, S, \mathds{P}) \xrightarrow{\alpha} (h', \Phi', S', \mathds{P}') \\ &\land \pred{tgen}{\tau, h', \underline{h}, \Phi', S', \mathds{P}'}
\end{align*}
\[
	\pred{absent}{\iota, k, \tau}
		\iff
	\lnot \exists v, n \ldotp (\actread{\iota}{k}{v}, n) \in \tau \lor (\actwrite{\iota}{k}{v}, n) \in \tau
\]
\[
	\pred{clean}{\tau} \iff \forall \iota, k, \kappa, n \ldotp \left( (\actlock{\iota}{k}{\kappa}, n) \in \tau \lor (\actunlock{\iota}{k}, n) \in \tau \right) \implies \lnot \pred{absent}{\iota, k, \tau}
\]

\lem \label{lem:alman} A lock on an item is not needed for any reductions a part from a read, a write or an unlock action performed by the same transaction on the same item.
\begin{gather*}
	\forall \mathds{P}, \mathds{P}', h, h', \Phi, \Phi', S, S', \alpha, i, k, v, I, \kappa \ldotp \\
	(h, \Phi, S, \mathds{P}) \xrightarrow{\alpha} (h', \Phi', S', \mathds{P}')
		\land
	(\{i\} \uplus I, \kappa) = \Phi(k)
		\land \\
	\alpha \not\in \{ \actread{i}{k}{v}, \actwrite{i}{k}{v}, \actunlock{i}{k} \}
		\implies
	\exists \Phi_m, \Phi_m', I', \kappa', \kappa'' \ldotp \\
	(h, \Phi_m, S, \mathds{P}) \xrightarrow{\alpha} (h', \Phi_m', S, \mathds{P}')
		\land
	\Phi_m = \Phi[k \mapsto (I, \kappa')]
		\land
	\Phi_m' = \Phi'[k \mapsto (I', \kappa'')]
		\land
	\kappa' \leq \textsc{s}
\end{gather*}
\begin{proof}
Let's pick arbitrary $\mathds{P}, \mathds{P}' \in \mathsf{Prog}, h, h' \in \mathsf{Storage}, \Phi, \Phi' \in \mathsf{LMan}, S, S' \in \mathsf{TState}, \alpha \in \mathsf{Act}, i \in \mathsf{Tid}, k \in \mathsf{Key}, v \in \mathsf{Val}, I \in \mathcal{P}(\mathsf{Tid}), \kappa \in \mathsf{Lock}$. We now assume that the following holds:
\begin{gather}
	\label{lem:alman1}
	(h, \Phi, S, \mathds{P}) \xrightarrow{\alpha} (h', \Phi', S', \mathds{P}')
		\land
	(\{i\} \uplus I, \kappa) = \Phi(k)
		\land
	\alpha \not\in \{ \actread{i}{k}{v}, \actwrite{i}{k}{v} \}
\end{gather}
From (\ref{lem:alman1}) we directly obtain that $\kappa \geq \textsc{s}$ given that $i$ is in the owners' set for item $k$. The proof proceeds with a case-by-case analysis on $\alpha$.
\begin{itemize}
	\item If $\alpha = \actprog$, $\alpha = \actid{\iota}$ or $\alpha = \actalloc{\iota}{n}{l}$ for some $\iota \in \mathsf{Tid}, n \in \mathds{N}, l \in \mathsf{Key}$, then the result trivially follows given that in these cases $\alpha$ has no requirementes on $\Phi$ to succesfully reduce.
	
	\item If $\alpha = \actread{j}{k'}{v'}$ for $j \in \mathsf{Tid}, k' \in \mathsf{Key}, v' \in \mathsf{Val}$ then from (\ref{lem:alman1}) we know that $i \neq j$. Next we consider the following two cases:
		\begin{itemize}
			\item If $k = k'$ then from (\ref{lem:alman1}) we obtain that, given the $\alpha$ action has succesfully reduced, $\kappa = \textsc{s}$ and $j \in I$. Therefore we can find $\kappa' = \textsc{s}, \kappa'' = \textsc{s}$ and $I' = I$.
			\item If $k \neq k'$ then $\alpha$ has no requirement on $\Phi(k)$ to succesfully reduce and the result follows.
		\end{itemize}
		
	\item If $\alpha = \actwrite{j}{k'}{v'}$ for $j \in \mathsf{Tid}, k' \in \mathsf{Key}, v' \in \mathsf{Val}$ then from (\ref{lem:alman1}) we know that $i \neq j$. Next we consider the following two cases:
		\begin{itemize}
			\item If $k = k'$ then it is not possible that $\alpha$ succesfully reduced since from (\ref{lem:alman1}) we know that $i$ was in the owners set for key $k$, then it must be the case that $k \neq k'$.
			\item If $k \neq k'$ then $\alpha$ has no requirement on $\Phi(k)$ to succesfully reduce and the result follows.
		\end{itemize}
		
	\item If $\alpha = \actlock{j}{k'}{\kappa_j}$ for some $j \in \mathsf{Tid}, k' \in \mathsf{Key}, \kappa_j \in \mathsf{Lock}$.
		\begin{itemize}
			\item If $k \neq k'$ then $\alpha$ has no requirement on $\Phi(k)$ to succesfully reduce and the result follows.
			\item If $k = k'$ and $i \neq j$ then from (\ref{lem:alman1}) we know that $\alpha$ succesfully reduced, and therefore $\kappa = \textsc{s}$ and $\kappa_j = \textsc{s}$. This also implies that we can find $\kappa' = \textsc{s}, I' = I \cup \{j\}$ and $\kappa'' = \textsc{s}$.
			\item If $k = k'$ and $i = j$ then given that $\Phi(k)$ already had $i$ as part of the owners, it must be the case that $\kappa = \textsc{s}, I = \emptyset$ and $\kappa_j = \textsc{x}$ in order for $\alpha$ to reduce as imposed by (\ref{lem:alman1}). It follows that we can find $\kappa' = \textsc{u}, I' = \{i\}$ and $\kappa'' = \textsc{x}$.
		\end{itemize}
		
	\item If $\alpha = \actunlock{j}{k'}$ for $j \in \mathsf{Tid}, k' \in \mathsf{Key}$ then from (\ref{lem:alman1}) we know that $i \neq j$. Next we consider the following two cases:
		\begin{itemize}
			\item If $k \neq k'$ then $\alpha$ has no requirement on $\Phi(k)$ to succesfully reduce and the result follows.
			\item If $k = k'$ then from (\ref{lem:alman1}) we know that $\alpha$ succesfully reduced meaning that $i$ and $j$ were holding the lock at the same time, making $\kappa = \textsc{s}, \kappa' = \textsc{s}, \kappa'' = \textsc{s}$ and $I' = I \setminus \{ i, j \}$.
		\end{itemize}
\end{itemize}
\end{proof}


\lem \label{lem:lockAbsent} Lock and unlock operations done by a transaction on items which it does not read or write can be removed without affecting the program or the global state.
\begin{gather*}
	\forall \tau, \tau', h, h', \Phi, S, \mathds{P}, n, n', \iota, k, \kappa, x, y \ldotp
		\\
	\pred{tgen}{\tau, h, h', \Phi, S, \mathds{P}} \land  \pred{absent}{\iota, k, \tau} \land x = (\actlock{\iota}{k}{\kappa}, n) \land y = (\actunlock{\iota}{k}, n') \\ \land x \in \tau \land y \in \tau
	\land \tau' = \tau \setminus \{ x, y \}
		\implies
	\pred{tgen}{\tau', h, h', \Phi, S, \mathds{P}}
\end{gather*}
\begin{proof}
Let's pick arbitrary $\tau, \tau' \in [\mathsf{Act} \times \mathds{N}], h, h' \in \mathsf{Storage}, \Phi \in \mathsf{LMan}, S \in \mathsf{TState}, \mathds{P} \in \mathds{Prog}, n, n' \in \mathds{N}, \iota \in \mathsf{Tid}, k \in \mathsf{Key}, \kappa \in \mathsf{Lock}, x, y \in \mathsf{Act} \times \mathds{N}$. We now assume that the following holds:
\begin{gather*}
	\pred{tgen}{\tau, h, h', \Phi, S, \mathds{P}} \land  \pred{absent}{\iota, k, \tau} \land x = (\actlock{\iota}{k}{\kappa}, n) \land y = (\actunlock{\iota}{k}, n') \\ \land x \in \tau \land y \in \tau
	\land \tau' = \tau \setminus \{ x, y \}
\end{gather*}
From Lemma \ref{lem:2phase} we obtain that $\tau \vDash x < y$. From the definition of $\mathsf{tgen}$ and the fact that both $x$ and $y$ are in $\tau$ it follows that $\kappa \geq \textsc{s}$ and:
\begin{gather}
	(h, \Phi, S, \mathds{P}) \rightarrow^* (h_1, \Phi_1, S_1, \mathds{P}_1) \xrightarrow{\actlock{\iota}{k}{\kappa}} (h_1', \Phi_1', S_1', \mathds{P}_1') \\ \rightarrow^* (h_2, \Phi_2, S_2, \mathds{P}_2) \xrightarrow{\actunlock{\iota}{k}} (h_2', \Phi_2', S_2', \mathds{P}_2') \rightarrow^* (h', \emptyset, S', \pskip)
\end{gather}
From the semantic interpretation of $\mathsf{lock}$ and $\mathsf{unlock}$, we know that it is the case that $h_1' = h_1, S_1' = S_1, \mathds{P}_1' = \mathds{P}_1, h_2' = h_2, S_2' = S_2, \mathds{P}_2' = \mathds{P}_2$ and $\Phi_1' = \Phi_1[k \mapsto (\{\iota\} \cup I, \kappa)], \Phi_2' = \Phi_2[k \mapsto (I', \kappa')]$ for $I, I' \in \mathcal{P}(\mathsf{Tid})$ such that $\iota \not\in I'$ and $\kappa' \leq \textsc{s}$. From the assumption that $\pred{absent}{\iota, k, \tau}$ holds, we know there is no read or write action on $k$ done by $\iota$ happening in $(h_1', \Phi_1', S_1', \mathds{P}_1') \rightarrow^* (h_2, \Phi_2, S_2, \mathds{P}_2)$ meaning that actions which need a presence of $\iota$'s lock acquisition on $k$ to succeed (i.e. read and write) are not there. From Lemma \ref{lem:alman} we obtain that all actions that are part of the sequence of reductions $(h_1', \Phi_1', S_1', \mathds{P}_1') \rightarrow^* (h_2, \Phi_2, S_2, \mathds{P}_2)$ will succesfully reduce with the $\Phi_1$ lock manager not containing $\iota$ as an owner for $k$. It follows that, for $(\alpha, n+1) \in \tau$ and $(\alpha', n'+1) \in \tau$:
\begin{gather}
	\label{lem:spur1} (h, \Phi, S, \mathds{P}) \rightarrow^* (h_1, \Phi_1, S_1, \mathds{P}_1) \xrightarrow{\alpha} (h_1'', \Phi_1'', S_1'', \mathds{P}_1'')
		\\
	\label{lem:spur2} \rightarrow^* (h_2'', \Phi_2'', S_2'',\mathds{P}_2'') \xrightarrow{\alpha'} (h_2, \Phi_2, S_2, \mathds{P}_2) \rightarrow^* (h', \emptyset, S', \pskip)
\end{gather}
From the initial assumption we also know that $\tau' = \tau \setminus \{ x, y \}$ meaning that $\tau'$ has all of $\tau$'s actions a part from the ones at position $n$ and $n'$. As a consequence we have that by following $\tau'$ up to (and not including) position $n$ we have $(h, \Phi, S, \mathds{P}) \rightarrow^* (h_1, \Phi_1, S_1, \mathds{P}_1)$. Skipping the operation at position $n$ which is not present in $\tau'$, we proceed with the one in position $n + 1$ all the way to (and not including) the one in position $n'$ to get by (\ref{lem:spur1}) and (\ref{lem:spur2}) that $(h_1, \Phi_1, S_1, \mathds{P}_1) \xrightarrow{\alpha} (h_1'', \Phi_1'', S_1'', \mathds{P}_1'') \rightarrow^* (h_2'', \Phi_2'', S_2'',\mathds{P}_2'')$ holds. Now we apply the action from position $n' + 1$ to the end in $\tau'$ to obtain $(h_2'', \Phi_2'', S_2'',\mathds{P}_2'') \xrightarrow{\alpha'} (h_2, \Phi_2, S_2, \mathds{P}_2) \rightarrow^* (h', \emptyset, S', \pskip)$. From the definition of $\mathsf{tgen}$ we can state that $\pred{tgen}{\tau', h, h', \Phi, S, \mathds{P}}$ holds as needed.
\end{proof}

\lem \label{lem:rr} The order of two consecutive reads can be swapped as long as the transactions performing them are distinct.
\begin{gather*}
	\forall \tau, \tau', h, h', \Phi, S, \mathds{P}, i, j, k, k', v, v', \alpha, \alpha', n \ldotp \\
	i \neq j \land \alpha = \actread{i}{k}{v} \land \alpha' = \actread{j}{k'}{v'} \land (\alpha, n) \in \tau \land (\alpha', n+1) \in \tau \\ \land \pred{tgen}{\tau, h, h', \Phi, S, \mathds{P}} \land \tau' = \tau \setminus \{(\alpha, n), (\alpha', n+1)\} \cup \{ (\alpha, n+1), (\alpha', n) \}
		\\	 
	 \implies \pred{tgen}{\tau', h, h', \Phi, S, \mathds{P}}
\end{gather*}
\begin{proof}
Let's pick arbitrary $\tau, \tau' \in [\mathsf{Act} \times \mathds{N}], h, h' \in \mathsf{Storage}, \Phi \in \mathsf{LMan}, S \in \mathsf{TState}, \mathds{P} \in \mathsf{Prog}, i, j \in \mathsf{Tid}, k, k' \in \mathsf{Key}, v, v' \in \mathsf{Val}, \alpha, \alpha' \in \mathsf{Act}, n \in \mathds{N}$. We assume that the following holds:
\begin{gather*}
	i \neq j \land \alpha = \actread{i}{k}{v} \land \alpha' = \actread{j}{k'}{v'} \land (\alpha, n) \in \tau \land (\alpha', n+1) \in \tau \\ \land \pred{tgen}{\tau, h, h', \Phi, S, \mathds{P}} \land \tau' = \tau \setminus \{(\alpha, n), (\alpha', n+1)\} \cup \{ (\alpha, n+1), (\alpha', n) \}
\end{gather*}
The above means that the two transactions performing the consecutive read actions $\alpha$ and $\alpha'$ are distinct and in $\tau$. Also, $\tau'$ is equivalent to $\tau$ with the $\alpha$ and $\alpha'$ actions swapped. From the definition of $\mathsf{tgen}$ we know that following the actions in $\tau$ we obtain the following:
\begin{gather}
	\label{lem:rr1} (h, \Phi, S, \mathds{P}) \rightarrow^* (h_1, \Phi_1, S_1, \mathds{P}_1) \xrightarrow{\alpha} (h_0, \Phi_0, S_0, \mathds{P}_0) \\
	\label{lem:rr2} \xrightarrow{\alpha'} (h_2, \Phi_2, S_2, \mathds{P}_2) \rightarrow^* (h', \emptyset, S, \pskip)
\end{gather}
It is now required to show that trace $\tau'$ is executing the following:
\[
	(h, \Phi, S, \mathds{P}) \rightarrow^* (h_1, \Phi_1, S_1, \mathds{P}_1) \xrightarrow{\alpha'} (h_0', \Phi_0', S_0', \mathds{P}_0') \xrightarrow{\alpha} (h_2, \Phi_2, S_2, \mathds{P}_2) \rightarrow^* (h', \emptyset, S', \pskip)
\]
Since $i \neq j$ we know that the two action labels $\alpha$ and $\alpha'$ were produced by the two transactions running in parallel executing a single step each meaning we can write $\mathds{P}_1 = \mathds{P}_i \| \mathds{P}_j$ (or equivalently $\mathds{P}_j \| \mathds{P}_i$) for some $\mathds{P}_i, \mathds{P}_j \in \mathsf{Prog}$. It follows that $\mathds{P}_2 = \mathds{P}_i' \| \mathds{P}_j'$ for $(h_1, \Phi_1, S_1, \mathds{P}_i) \xrightarrow{\alpha} (h_0, \Phi_0, S_0, \mathds{P}_i')$ and $(h_0, \Phi_0, S_0, \mathds{P}_j) \xrightarrow{\alpha'} (h_2, \Phi_2, S_2, \mathds{P}_j')$. Given the effect of the $\mathsf{read}$ action, we know that $h_1 = h_0 = h_2, \Phi_1 = \Phi_0 = \Phi_2$. We can immediately find a $h_0' = h_1 = h_2$ and a $\Phi_0' = \Phi_1 = \Phi_2$. $\mathds{P}_0'$ will be the program $\mathds{P}_1$ that has executed a step in the program where transaction $j$ resides, formally $\mathds{P}_0' = \mathds{P}_i \| \mathds{P}_j'$ for $(h_1, \Phi_1, S_1, \mathds{P}_j) \xrightarrow{\alpha'} (h_0', \Phi_0', S_0', \mathds{P}_j')$. We know that this will always succeed since the $\mathsf{read}$ action requirements on $h_0, \Phi_0$ are all satisfied by (\ref{lem:rr2}). From this, $\mathds{P}_0'$ can always reduce to $\mathds{P}_2$ by chosing to run the program in which transaction $i$ is, i.e. $\mathds{P}_i$ as part of $(h_0', \Phi_0', S_0', \mathds{P}_i) \xrightarrow{\alpha} (h_2, \Phi_2, S_2, \mathds{P}_i')$, which is possible thanks to the assumption in (\ref{lem:rr1}). Given that by assumption $i \neq j$, it must be the case that $S(i)$ and $S(j)$ are disjoint therefore the relative ordering on the updates to the local variables does not matter.
\end{proof}

\lem \label{lem:rwlu} The order of two consecutive read, write, lock or unlock operations can be swapped as long as the transactions performing them are distinct and the keys they refer to are different.
\begin{gather*}
	\forall \tau, \tau', h, h', \Phi, S, \mathds{P}, i, j, k, k', x, y, n \ldotp \\
		i \neq j \land x = \alpha(i, k) \land y = \alpha(j, k') \land k \neq k' \land (x, n) \in \tau \land (y, n+1) \in \tau \\ \land \pred{tgen}{\tau, h, h', \Phi, S, \mathds{P}} \land \tau' = \tau \setminus \{(x, n), (y, n+1)\} \cup \{ (x, n+1), (y, n) \}
		\\	 
	 \implies \pred{tgen}{\tau', h, h', \Phi, S, \mathds{P}}
\end{gather*}
\begin{proof}
Let's pick arbitrary $\tau, \tau' \in [\mathsf{Act} \times \mathds{N}], h, h' \in \mathsf{Storage}, \Phi \in \mathsf{LMan}, S \in \mathsf{TState}, \mathds{P} \in \mathsf{Prog}, i, j \in \mathsf{Tid}, k, k' \in \mathsf{Key}, x, y \in \mathsf{Act}, n \in \mathds{N}$. We assume that the following holds:
\begin{gather}
	i \neq j \land x = \alpha(i, k) \land y = \alpha(j, k') \land (x, n) \in \tau \land (y, n+1) \in \tau \\ \land \pred{tgen}{\tau, h, h', \Phi, S, \mathds{P}} \land \tau' = \tau \setminus \{(x, n), (y, n+1)\} \cup \{ (x, n+1), (y, n) \}
\end{gather}
The above means that the two transactions performing the consecutive actions $x$ and $y$ are distinct and in $\tau$. Also, $\tau'$ is equivalent to $\tau$ with the $x$ and $y$ actions swapped. From the definition of $\mathsf{tgen}$ we know that following the actions in $\tau$ we obtain the following:
\begin{gather}
	\label{lem:xy1} (h, \Phi, S, \mathds{P}) \rightarrow^* (h_1, \Phi_1, S_1, \mathds{P}_1) \xrightarrow{x} (h_0, \Phi_0, S_0, \mathds{P}_0) \\
	\label{lem:xy2} \xrightarrow{y} (h_2, \Phi_2, S_2, \mathds{P}_2) \rightarrow^* (h', \emptyset, S, \pskip)
\end{gather}
It is now required to show that trace $\tau'$ is executing the following:
\[
	(h, \Phi, S, \mathds{P}) \rightarrow^* (h_1, \Phi_1, S_1, \mathds{P}_1) \xrightarrow{y} (h_0', \Phi_0', S_0', \mathds{P}_0') \xrightarrow{x} (h_2, \Phi_2, S_2, \mathds{P}_2) \rightarrow^* (h', \emptyset, S', \pskip)
\]
Since $i \neq j$ we know that the two action labels $x$ and $y$ were produced by the two transactions running in parallel executing a single step each meaning we can write $\mathds{P}_1 = \mathds{P}_i \| \mathds{P}_j$ (or equivalently $\mathds{P}_j \| \mathds{P}_i$) for some $\mathds{P}_i, \mathds{P}_j \in \mathsf{Prog}$. It follows that $\mathds{P}_2 = \mathds{P}_i' \| \mathds{P}_j'$ for $(h_1, \Phi_1, S_1, \mathds{P}_i) \xrightarrow{x} (h_0, \Phi_0, S_0, \mathds{P}_i')$ and $(h_0, \Phi_0, S_0, \mathds{P}_j) \xrightarrow{y} (h_2, \Phi_2, S_2, \mathds{P}_j')$. We will proceed with a case-by-case analysis on $x$ and $y$ in order to find suitable $h_0'$ and $\Phi_0'$.
\begin{itemize}
	\item If $x = \actread{i}{k}{v}$ and $y = \actread{j}{k'}{v'}$ for $v, v' \in \mathsf{Val}$ then the result follows directly from Lemma \ref{lem:rr}.
	
	\item If $x = \actwrite{i}{k}{v}$ and $y = \actwrite{j}{k'}{v'}$ for $v, v' \in \mathsf{Val}$ then $h_2 = h_1[k \mapsto v][k' \mapsto v']$ since $k \neq k'$ and $\Phi_2 = \Phi_1$ meaning we can find $h_0' = h_1[k \mapsto v']$ and $\Phi_0' = \Phi_1$.
	
	\item If $x = \actread{i}{k}{v}$ and $y = \actwrite{j}{k'}{v'}$ for $v, v' \in \mathsf{Val}$ then $h_2 = h_1[k' \mapsto v']$ and $\Phi_2 = \Phi_1$ meaning we can find $h_0' = h_1[k' \mapsto v']$ and $\Phi_0' = \Phi_1$.
	
	\item If $x = \actlock{i}{k}{\kappa}$ and $y = \actunlock{j}{k'}$ for $\kappa \in \mathsf{Lock}$ then $h_2 = h_1$ and $\Phi_2 = \Phi_1[k \mapsto (I, \kappa)][k' \mapsto (I' \setminus \{j\}, \kappa')]$ since $k \neq k'$ for $I, I' \in \mathcal{P}(\mathsf{Tid})$ and $\kappa' \in \mathsf{Lock}$, meaning we can find $h_0' = h_1$ and $\Phi_0' = \Phi_1[k' \mapsto (I' \setminus \{j\}, \kappa')]$.
	
	\item If $x = \actlock{i}{k}{\kappa}$ and $y = \actlock{j}{k'}{\kappa'}$ for $\kappa, \kappa' \in \mathsf{Lock}$ then $h_2 = h_1$ and $\Phi_2 = \Phi_1[k \mapsto (I, \kappa)][k' \mapsto (I', \kappa')]$ since $k \neq k'$ for $I, I' \in \mathcal{P}(\mathsf{Tid})$ meaning we can find $h_0' = h_1$ and $\Phi_0' = \Phi_1[k' \mapsto (I', \kappa')]$.
	
	\item If $x = \actunlock{i}{k}$ and $y = \actunlock{j}{k'}$ then $h_2 = h_1$ and $\Phi_2 = \Phi_1[k \mapsto (I \setminus \{i\}, \kappa)][k' \mapsto (I' \setminus \{j\}, \kappa')]$ since $k \neq k'$ for $I, I' \in \mathcal{P}(\mathsf{Tid})$ and $\kappa, \kappa' \in \mathsf{Lock}$, meaning we can find $h_0' = h_1$ and $\Phi_0' = \Phi_1[k' \mapsto (I' \setminus \{j\}, \kappa')]$.
	
	\item If $x = \actlock{i}{k}{\kappa}$ and $y = \actread{j}{k'}{v}$ for $\kappa \in \mathsf{Lock}$ and $v \in \mathsf{Val}$ then $h_2 = h_1$ and $\Phi_2 = \Phi_1[k \mapsto (I, \kappa)]$ for $I \in \mathcal{P}(\mathsf{Tid})$, meaning we can find $h_0' = h_1$ and $\Phi_0' = \Phi_1$.
	
	\item If $x = \actlock{i}{k}{\kappa}$ and $y = \actwrite{j}{k'}{v}$ for $\kappa \in \mathsf{Lock}$ and $v \in \mathsf{Val}$ then $h_2 = h_1[k' \mapsto v]$ and $\Phi_2 = \Phi_1[k \mapsto (I, \kappa)]$ for $I \in \mathcal{P}(\mathsf{Tid})$, meaning we can find $h_0' = h_1[k \mapsto v]$ and $\Phi_0' = \Phi_1$.
	
	\item If $x = \actunlock{i}{k}$ and $y = \actread{j}{k'}{v}$ for $v \in \mathsf{Val}$ then $h_2 = h_1$ and $\Phi_2 = \Phi_1[k \mapsto (I \setminus \{j\}, \kappa)]$ for $\kappa \in \{\textsc{u}, \textsc{s}\}$ and $I \in \mathcal{P}(\mathsf{Tid})$, meaning we can find $h_0' = h_1$ and $\Phi_0' = \Phi_1$.
	
	\item If $x = \actunlock{i}{k}$ and $y = \actwrite{j}{k'}{v}$ for $v \in \mathsf{Val}$ then $h_2 = h_1[k' \mapsto v]$ and $\Phi_2 = \Phi_1[k \mapsto (I \setminus \{j\}, \kappa)]$ for $\kappa \in \{\textsc{u}, \textsc{s}\}$ and $I \in \mathcal{P}(\mathsf{Tid})$, meaning we can find $h_0' = h_1[k \mapsto v]$ and $\Phi_0' = \Phi_1$.
\end{itemize}
The inverted cases that are not included in the list can be trivially found as a consequence of the presented ones, with the appropriate substitions.

From (\ref{lem:xy1}) we obtain that $\mathds{P}_0'$ is the program $\mathds{P}_1$ that has executed a step in the program where transaction $j$ resides, formally $\mathds{P}_0' = \mathds{P}_i \| \mathds{P}_j'$ for $(h_1, \Phi_1, S_1, \mathds{P}_j) \xrightarrow{y} (h_0', \Phi_0', S_0', \mathds{P}_j')$. We know that this will always succeed since the actions act on disjoint parts of the global heap and lock manager, as showed in the case-by-case analysis above, meaning that their requirements are all satisfied by (\ref{lem:xy2}). From this, $\mathds{P}_0'$ can always reduce to $\mathds{P}_2$ by chosing to run the program in which transaction $i$ is, i.e. $\mathds{P}_i$ as part of $(h_0', \Phi_0', S_0', \mathds{P}_i) \xrightarrow{x} (h_2, \Phi_2, S_2, \mathds{P}_i')$, which is possible thanks to the assumption in (\ref{lem:xy1}). Given that by assumption $i \neq j$, it must be the case that $S(i)$ and $S(j)$ are disjoint therefore the relative ordering on the eventual updates to the local variables does not matter.
\end{proof}

\lem \label{lem:aa} The order of two consecutive allocations can be swapped as long as the transactions performing them are distinct.
\begin{gather*}
	\forall \tau, \tau', h, h', \Phi, S, \mathds{P}, i, j, m, m', l, l', \alpha, \alpha', n \ldotp \\
		i \neq j \land \alpha = \actalloc{i}{m}{l} \land \alpha' = \actalloc{j}{m'}{l'} \land (\alpha, n) \in \tau \land (\alpha', n+1) \in \tau \\ \land \pred{tgen}{\tau, h, h', \Phi, S, \mathds{P}} \land \tau' = \tau \setminus \{(\alpha, n), (\alpha', n+1)\} \cup \{ (\alpha, n+1), (\alpha', n) \}
		\\	 
	 \implies \pred{tgen}{\tau', h, h', \Phi, S, \mathds{P}}
\end{gather*}
\begin{proof}
Let's pick arbitrary $\tau, \tau' \in [\mathsf{Act} \times \mathds{N}], h, h' \in \mathsf{Storage}, \Phi \in \mathsf{LMan}, S \in \mathsf{TState}, \mathds{P} \in \mathsf{Prog}, i, j \in \mathsf{Tid}, l, l' \in \mathsf{Key}, \alpha, \alpha' \in \mathsf{Act}, n, m, m' \in \mathds{N}$. We assume that the following holds:
\begin{gather}
	i \neq j \land \alpha = \actalloc{i}{m}{l} \land \alpha' = \actalloc{j}{m'}{l'} \land (\alpha, n) \in \tau \land (\alpha', n+1) \in \tau \\ \land \pred{tgen}{\tau, h, h', \Phi, S, \mathds{P}} \land \tau' = \tau \setminus \{(\alpha, n), (\alpha', n+1)\} \cup \{ (\alpha, n+1), (\alpha', n) \}
\end{gather}
The above means that the two transactions performing the consecutive allocation actions $\alpha$ and $\alpha'$ are distinct and in $\tau$. Also, $\tau'$ is equivalent to $\tau$ with the $\alpha$ and $\alpha'$ actions swapped. From the definition of $\mathsf{tgen}$ we know that following the actions in $\tau$ we obtain the following:
\begin{gather}
	\label{lem:aa1} (h, \Phi, S, \mathds{P}) \rightarrow^* (h_1, \Phi_1, S_1, \mathds{P}_1) \xrightarrow{\alpha} (h_0, \Phi_0, S_0, \mathds{P}_0) \\
	\label{lem:aa2} \xrightarrow{\alpha'} (h_2, \Phi_2, S_2, \mathds{P}_2) \rightarrow^* (h', \emptyset, S, \pskip)
\end{gather}
It is now required to show that trace $\tau'$ is executing the following:
\[
	(h, \Phi, S, \mathds{P}) \rightarrow^* (h_1, \Phi_1, S_1, \mathds{P}_1) \xrightarrow{\alpha'} (h_0', \Phi_0', S_0', \mathds{P}_0') \xrightarrow{\alpha} (h_2, \Phi_2, S_2, \mathds{P}_2) \rightarrow^* (h', \emptyset, S', \pskip)
\]
Since $i \neq j$ we know that the two action labels $\alpha$ and $\alpha'$ were produced by the two transactions running in parallel executing a single step each meaning we can write $\mathds{P}_1 = \mathds{P}_i \| \mathds{P}_j$ (or equivalently $\mathds{P}_j \| \mathds{P}_i$) for some $\mathds{P}_i, \mathds{P}_j \in \mathsf{Prog}$. It follows that $\mathds{P}_2 = \mathds{P}_i' \| \mathds{P}_j'$ for $(h_1, \Phi_1, S_1, \mathds{P}_i) \xrightarrow{\alpha} (h_0, \Phi_0, S_0, \mathds{P}_i')$ and $(h_0, \Phi_0, S_0, \mathds{P}_j) \xrightarrow{\alpha'} (h_2, \Phi_2, S_2, \mathds{P}_j')$. Given the effect of the $\mathsf{alloc}$ action, we know that $\Phi_2 = \Phi_1 = \Phi_0$. We can immediately find a $\Phi_0' = \Phi_1 = \Phi_2$. We also know that $\{l, \ldots, l + n - 1\} \subseteq \pred{dom}{h_0}$ and in order for $\actalloc{j}{n'}{l'}$ to suceed, which it does by (\ref{lem:aa2}), $\{l', \ldots, l' + n' - 1\} \cap \pred{dom}{h_0} \equiv \emptyset$ which means that the two ranges of memory locations are disjoint. As a consequence the order of allocation does not matter in terms of reaching the final heap $h_2$; our $h_0'$ will therefore be $h_1[l' \mapsto 0]\ldots[l' + n' - 1 \mapsto 0]$. $\mathds{P}_0'$ will be the program $\mathds{P}_1$ that has executed a step in the program where transaction $j$ resides, formally $\mathds{P}_0' = \mathds{P}_i \| \mathds{P}_j'$ for $(h_1, \Phi_1, S_1, \mathds{P}_j) \xrightarrow{\alpha'} (h_0', \Phi_0', S_0', \mathds{P}_j')$. We know that this will always succeed since the $\mathsf{alloc}$ action requirements are all satisfied by (\ref{lem:aa2}). From this, $\mathds{P}_0'$ can always reduce to $\mathds{P}_2$ by chosing to run the program in which transaction $i$ is, i.e. $\mathds{P}_i$ as part of $(h_0', \Phi_0', S_0', \mathds{P}_i) \xrightarrow{\alpha} (h_2, \Phi_2, S_2, \mathds{P}_i')$, which is possible thanks to the assumption in (\ref{lem:aa1}). Given that by assumption $i \neq j$, it must be the case that $S(i)$ and $S(j)$ are disjoint therefore the relative ordering on the updates to the local variables does not matter.
\end{proof}

\lem \label{lem:ax} The order of an allocation followed by a read, write, lock or unlock can be swapped as long as the transactions performing them are distinct and the keys accessed are not part of the ones created by the allocation.
\begin{gather*}
	\forall \tau, \tau', h, h', \Phi, S, \mathds{P}, i, j, k, x, y, n, m, l \ldotp \\
		i \neq j \land x = \alpha(j, k) \land y = \actalloc{i}{m}{l} \land (x, n) \in \tau \land (y, n+1) \in \tau \land (k < l \lor k \geq l + n) \\ \land \pred{tgen}{\tau, h, h', \Phi, S, \mathds{P}} \land \tau' = \tau \setminus \{(x, n), (y, n+1)\} \cup \{ (x, n+1), (y, n) \}
		\\	 
	 \implies \pred{tgen}{\tau', h, h', \Phi, S, \mathds{P}}
\end{gather*}
\begin{proof}
Let's pick arbitrary $\tau, \tau' \in [\mathsf{Act} \times \mathds{N}], h, h' \in \mathsf{Storage}, \Phi \in \mathsf{LMan}, S \in \mathsf{TState}, \mathds{P} \in \mathsf{Prog}, i, j \in \mathsf{Tid}, k, l \in \mathsf{Key}, x, y \in \mathsf{Act}, n, m \in \mathds{N}$. We assume that the following holds:
\begin{gather*}
	i \neq j \land x = \alpha(j, k) \land y = \actalloc{i}{m}{l} \land (x, n) \in \tau \land (y, n+1) \in \tau \land (k < l \lor k \geq l + n) \\ \land \pred{tgen}{\tau, h, h', \Phi, S, \mathds{P}} \land \tau' = \tau \setminus \{(x, n), (y, n+1)\} \cup \{ (x, n+1), (y, n) \}
\end{gather*}
The above means that the two transactions performing the consecutive actions $x$ and $y$ are distinct and in $\tau$. Also, $\tau'$ is equivalent to $\tau$ with the $x$ and $y$ actions swapped. From the definition of $\mathsf{tgen}$ we know that following the actions in $\tau$ we obtain the following:
\begin{gather}
	\label{lem:ax1} (h, \Phi, S, \mathds{P}) \rightarrow^* (h_1, \Phi_1, S_1, \mathds{P}_1) \xrightarrow{x} (h_0, \Phi_0, S_0, \mathds{P}_0) \\
	\label{lem:ax2} \xrightarrow{y} (h_2, \Phi_2, S_2, \mathds{P}_2) \rightarrow^* (h', \emptyset, S, \pskip)
\end{gather}
It is now required to show that trace $\tau'$ is executing the following:
\[
	(h, \Phi, S, \mathds{P}) \rightarrow^* (h_1, \Phi_1, S_1, \mathds{P}_1) \xrightarrow{y} (h_0', \Phi_0', S_0', \mathds{P}_0') \xrightarrow{x} (h_2, \Phi_2, S_2, \mathds{P}_2) \rightarrow^* (h', \emptyset, S', \pskip)
\]
Since $i \neq j$ we know that the two action labels $x$ and $y$ were produced by the two transactions running in parallel executing a single step each meaning we can write $\mathds{P}_1 = \mathds{P}_i \| \mathds{P}_j$ (or equivalently $\mathds{P}_j \| \mathds{P}_i$) for some $\mathds{P}_i, \mathds{P}_j \in \mathsf{Prog}$. It follows that $\mathds{P}_2 = \mathds{P}_i' \| \mathds{P}_j'$ for $(h_1, \Phi_1, S_1, \mathds{P}_i) \xrightarrow{x} (h_0, \Phi_0, S_0, \mathds{P}_i')$ and $(h_0, \Phi_0, S_0, \mathds{P}_j) \xrightarrow{y} (h_2, \Phi_2, S_2, \mathds{P}_j')$. Given the effect of the $\mathsf{alloc}$ action, we know that $\Phi_2 = \Phi_1 = \Phi_0$. Given the effect of the $\mathsf{alloc}$ action, we know that $\Phi_2 = \Phi_1 = \Phi_0$ and $h_0 = h_1[l \mapsto 0]\ldots[l + n - 1 \mapsto 0]$. In order to find $h_0'$ and $\Phi_0'$, we now proceed with a case-by-case analysis on the kind of action $x$.
\begin{itemize}
	\item If $x = \actread{j}{k}{v}$ for $v \in \mathsf{Val}$ then $h_2 = h_1$ and $\Phi_2 = \Phi_1$ meaning we can find $h_0' = h_1$ and $\Phi_0' = \Phi_1$.
	
	\item If $x = \actwrite{j}{k}{v}$ for $v \in \mathsf{Val}$ then $h_2 = h_1[k \mapsto v]$ and $\Phi_2 = \Phi_1$ meaning we can find $h_0' = h_1[k \mapsto v]$ and $\Phi_0' = \Phi_1$.
	
	\item If $x = \actlock{j}{k}{\kappa}$ for some $\kappa \in \mathsf{Lock}$ then $h_2 = h_1$ and $\Phi_2 = \Phi_1[k \mapsto (I, \kappa)]$ meaning we can find $h_0' = h_1$ and $\Phi_0' = \Phi_1[k \mapsto (I, \kappa)]$ for $I \in \mathcal{P}(\mathsf{Tid})$.
	
	\item If $x = \actunlock{j}{k}$ then $h_2 = h_1$ and $\Phi_2 = \Phi_1[k \mapsto (I \setminus \{j\}, \kappa)]$ for $I \in \mathcal{P}(\mathsf{Tid})$ and $\kappa \in \{\textsc{u}, \textsc{s}\}$ meaning we can find $h_0' = h_1$ and $\Phi_0' = \Phi_1[k \mapsto (I \setminus \{j\}, \kappa)]$.
\end{itemize}
$\mathds{P}_0'$ will be the program $\mathds{P}_1$ that has executed a step in the program where transaction $j$ resides, formally $\mathds{P}_0' = \mathds{P}_i \| \mathds{P}_j'$ for $(h_1, \Phi_1, S_1, \mathds{P}_j) \xrightarrow{y} (h_0', \Phi_0', S_0', \mathds{P}_j')$. We know that this will always succeed since the $\mathsf{alloc}$ action requirements are all satisfied by (\ref{lem:ax2}). From this, $\mathds{P}_0'$ can always reduce to $\mathds{P}_2$ by chosing to run the program in which transaction $i$ is, i.e. $\mathds{P}_i$ as part of $(h_0', \Phi_0', S_0', \mathds{P}_i) \xrightarrow{x} (h_2, \Phi_2, S_2, \mathds{P}_i')$, which is possible thanks to the assumption in (\ref{lem:ax1}). Given that by assumption $i \neq j$, it must be the case that $S(i)$ and $S(j)$ are disjoint therefore the relative ordering on the updates to the local variables does not matter.
\end{proof}

\lem \label{lem:idx} The order of an $\mathsf{id}$ operation followed by any other action performed by a different transaction can be swapped.
\begin{gather*}
	\forall \tau, \tau', h, h', \Phi, S, \mathds{P}, i, j, x, y, n \ldotp \\
		i \neq j \land x = \actid{i} \land y = \alpha(j) \land (x, n) \in \tau \land (y, n+1) \in \tau \\ \land \pred{tgen}{\tau, h, h', \Phi, S, \mathds{P}} \land \tau' = \tau \setminus \{(x, n), (y, n+1)\} \cup \{ (x, n+1), (y, n) \}
		\\	 
	 \implies \pred{tgen}{\tau', h, h', \Phi, S, \mathds{P}}
\end{gather*}
Let's pick arbitrary $\tau, \tau' \in [\mathsf{Act} \times \mathds{N}], h, h' \in \mathsf{Storage}, \Phi \in \mathsf{LMan}, S \in \mathsf{TState}, \mathds{P} \in \mathsf{Prog}, i, j \in \mathsf{Tid}, x, y \in \mathsf{Act}, n \in \mathds{N}$. We assume that the following holds:
\begin{gather*}
	i \neq j \land x = \actid{i} \land y = \alpha(j) \land (x, n) \in \tau \land (y, n+1) \in \tau \land (k < l \lor k \geq l + n) \\ \land \pred{tgen}{\tau, h, h', \Phi, S, \mathds{P}} \land \tau' = \tau \setminus \{(x, n), (y, n+1)\} \cup \{ (x, n+1), (y, n) \}
\end{gather*}
The above means that the two transactions performing the consecutive actions $x$ and $y$ are performed by distinct transactions and in $\tau$. Also, $\tau'$ is equivalent to $\tau$ with the $x$ and $y$ actions swapped. From the definition of $\mathsf{tgen}$ we know that following the actions in $\tau$ we obtain the following:
\begin{gather}
	\label{lem:idx1} (h, \Phi, S, \mathds{P}) \rightarrow^* (h_1, \Phi_1, S_1, \mathds{P}_1) \xrightarrow{x} (h_0, \Phi_0, S_0, \mathds{P}_0) \\
	\label{lem:idx2} \xrightarrow{y} (h_2, \Phi_2, S_2, \mathds{P}_2) \rightarrow^* (h', \emptyset, S, \pskip)
\end{gather}
It is now required to show that trace $\tau'$ is executing the following:
\[
	(h, \Phi, S, \mathds{P}) \rightarrow^* (h_1, \Phi_1, S_1, \mathds{P}_1) \xrightarrow{y} (h_0', \Phi_0', S_0', \mathds{P}_0') \xrightarrow{x} (h_2, \Phi_2, S_2, \mathds{P}_2) \rightarrow^* (h', \emptyset, S', \pskip)
\]
Since $i \neq j$ we know that the two action labels $x$ and $y$ were produced by the two transactions running in parallel executing a single step each meaning we can write $\mathds{P}_1 = \mathds{P}_i \| \mathds{P}_j$ (or equivalently $\mathds{P}_j \| \mathds{P}_i$) for some $\mathds{P}_i, \mathds{P}_j \in \mathsf{Prog}$. It follows that $\mathds{P}_2 = \mathds{P}_i' \| \mathds{P}_j'$ for $(h_1, \Phi_1, S_1, \mathds{P}_i) \xrightarrow{x} (h_0, \Phi_0, S_0, \mathds{P}_i')$ and $(h_0, \Phi_0, S_0, \mathds{P}_j) \xrightarrow{y} (h_2, \Phi_2, S_2, \mathds{P}_j')$. Given the effect of the $\mathsf{id}$ action, we know that $h_1 = h_0, \Phi_1 = \Phi_0, S_1 = S_0$ meaning that $y$ reduced succesfully as per (\ref{lem:idx2}), with a configuration equivalent to the one after $x$ reduced. It follows that we can find $h_0' = h_1, \Phi_0' = \Phi_1, S_0' = S_1$.

$\mathds{P}_0'$ will be the program $\mathds{P}_1$ that has executed a step in the program where transaction $j$ resides, formally $\mathds{P}_0' = \mathds{P}_i \| \mathds{P}_j'$ for $(h_1, \Phi_1, S_1, \mathds{P}_j) \xrightarrow{y} (h_0', \Phi_0', S_0', \mathds{P}_j')$. We know that this will always succeed since the $\mathsf{alloc}$ action requirements are all satisfied by (\ref{lem:idx2}). From this, $\mathds{P}_0'$ can always reduce to $\mathds{P}_2$ by chosing to run the program in which transaction $i$ is, i.e. $\mathds{P}_i$ as part of $(h_0', \Phi_0', S_0', \mathds{P}_i) \xrightarrow{x} (h_2, \Phi_2, S_2, \mathds{P}_i')$, which is possible thanks to the assumption in (\ref{lem:idx1}).

\subsection{Strict total order}

\defn (Reflexive image). The reflexive image of a set $X$, written $\pred{Id}{X}$ is defined as:
\[
	\pred{Id}{X} = \{ (x, x)\ |\ x \in X \}
\]

\defn (Reflexive closure). The reflexive closure of a given relation $R$ on a set $X$, written $R^\mathsf{id}$, is defined as:
\[
	R^\mathsf{id} = R \cup \pred{Id}{X}
\]

\defn (\tpl\ Transactions order)
\begin{align*}
	\text{let } (N, E) = \pred{SG}{\tau} \text{ in }
	\sqsubset_0 &= E^* \\
	\sqsubset_{n + 1} &=\ \sqsubset_n \cup \left( \sqsubset_n^\mathsf{id} ; \{ (i, j) \} ; \sqsubset_n^\mathsf{id} \right) \\
	&\text{where } i, j \in N \text{ and } i < j \\
	&\text{and } i \not\sqsubset_n j \text{ and } j \not\sqsubset_n i
\end{align*}

\thm \label{thm:totOrder} The $\sqsubset$ relation is a strict total order on $N$ for $(N, E) = \pred{SG}{\tau}, \tau = \pred{trace}{h, \emptyset, \emptyset, \mathds{P}}, \mathds{P} \in \mathsf{Prog}, h \in \mathsf{Storage}$.

\begin{proof}
In order to show the theorem, we are required to prove that for all $a, b, c \in N$:
\begin{itemize}
	\item (Irreflexivity). $a \not\sqsubset a$
	\item (Asymmetry). If $a \sqsubset b$ then $b \not\sqsubset a$
	\item (Transitivity). If $a \sqsubset b$ and $b \sqsubset c$ then $a \sqsubset c$
	\item (Totality). $a \sqsubset b$ or $b \sqsubset a$ or $a = b$
\end{itemize}

Let's pick an arbitrary program $\mathds{P} \in \mathsf{Prog}$, initial storage $h \in \mathsf{Storage}$ and get a trace out of it $\tau = \pred{trace}{h, \emptyset, \emptyset, \mathds{P}}$. We now consider the incrementally built $\sqsubset$ relation on $N$, where $(N, E) = \pred{SG}{\tau}$. \\

(Irreflexivity). The proof follows by induction on the number of $\sqsubset$ relation construction steps, $n$. Let's pick an arbitrary transaction identifier $a \in N$.

{\parindent0pt
\textit{Base case}: $n = 0$

\textit{To show}: $a \not\sqsubset_0 a$

By definition we know that $\sqsubset_0 = E^*$, i.e. the transitive closure on the edges of the serialization graph $\pred{SG}{\tau}$. We directly obtain that $a \not\sqsubset_0 a$ from Theorem \ref{thm:sgAcyclic}. \\

\textit{Inductive case}: $n > 0$

\textit{Inductive hypothesis}: $a \not\sqsubset_n a$

\textit{To show}: $a \not\sqsubset_{n+1} a$

Let's assume that $a \sqsubset_{n+1} a$ and by definition we know it means that, for some $i, j \in N$ such that $i < j, i \not\sqsubset_n j, j \not\sqsubset_n i$ we have:
\[
	(a, a) \in \sqsubset_n \cup \left( \sqsubset_n^\mathsf{id} ; \{ (i, j) \} ; \sqsubset_n ^\mathsf{id} \right)
\]
and by I.H. we can rewrite it as $(a, a) \in \left( \sqsubset_n^\mathsf{id} ; \{ (i, j) \} ; \sqsubset_n ^\mathsf{id} \right)$ given we assumed that $\sqsubset_n$ is irreflexive. It follows that it must be the case that $(a, i)$ and $(j, a)$ are in $\sqsubset_n^\mathsf{id}$ and moreover they must be in $\sqsubset_n$ given that $i < j$ and therefore $i \neq j$. By transitivity of $\sqsubset_n$, there must be a $(j, i) \in \sqsubset_n$. By contradiction we state that $a \not\sqsubset_{n+1} a$. \\
}

(Asymmetry). The proof follows by induction on the number of $\sqsubset$ relation construction steps, $n$. Let's pick arbitrary transaction identifiers $a, b \in N$.

{\parindent0pt
\textit{Base case}: $n = 0$

\textit{To show}: $a \sqsubset_0 b \implies b \not\sqsubset_0 a$

By definition we know that $\sqsubset_0 = E^*$, i.e. the transitive closure on the edges of the serialization graph $\pred{SG}{\tau}$. Let's assume that $a \sqsubset_0 b$ meaning that $a \rightarrow^* b \in E$. We directly obtain that $b \not\sqsubset_0 a$ from Theorem \ref{thm:sgAcyclic}. \\

\textit{Inductive case}: $n > 0$

\textit{Inductive hypothesis}: $a \sqsubset_n b \implies b \not\sqsubset_n a$

\textit{To show}: $a \sqsubset_{n + 1} b \implies b \not\sqsubset_{n + 1} a$

Let's assume that $a \sqsubset_{n + 1} b$ and by definition we know it means that, for some $i, j \in N$ such that $i < j, i \not\sqsubset_n j, j \not\sqsubset_n i$ we have:
\[
	(a, b) \in \sqsubset_n \cup \left( \sqsubset_n^\mathsf{id} ; \{ (i, j) \} ; \sqsubset_n ^\mathsf{id} \right)
\]
\begin{itemize}
	\item If $a \sqsubset_n b$ we know by I.H. that $b \not\sqsubset_n a$. Let's instead assume that $(b, a) \in \left( \sqsubset_n^\mathsf{id} ; \{ (i, j) \} ; \sqsubset_n ^\mathsf{id} \right)$ from which it follows that there is a $(b, i) \in \sqsubset_n^\mathsf{id}$ and $(j, a) \in \sqsubset_n^\mathsf{id}$. By transitivity of $\sqsubset_n$ we obtain that $(j, i) \in \sqsubset_n^\mathsf{id}$ and moreover that $j \sqsubset_n i$ since $i \neq j$ as $i < j$. By contradiction we obtain that $(b, a) \not\in \left( \sqsubset_n^\mathsf{id} ; \{ (i, j) \} ; \sqsubset_n ^\mathsf{id} \right)$. We conclude that $b \not\sqsubset_{n + 1} a$.
	\item If $(a, b) \in \left( \sqsubset_n^\mathsf{id} ; \{ (i, j) \} ; \sqsubset_n ^\mathsf{id} \right)$ which means there is a $(a, i) \in \sqsubset_n^\mathsf{id}$ and $(j, b) \in \sqsubset_n^\mathsf{id}$. Let's now assume that $(b, a) \in \left( \sqsubset_n^\mathsf{id} ; \{ (i, j) \} ; \sqsubset_n ^\mathsf{id} \right)$ meaning that there is a $(b, i) \in \sqsubset_n^\mathsf{id}$ and $(j, a) \in \sqsubset_n^\mathsf{id}$. By transitivity of $\sqsubset_n$ we obtain that $(j, i) \in \sqsubset_n^\mathsf{id}$ and moreover that $j \sqsubset_n i$ since $i \neq j$ as $i < j$. By contradiction we obtain that $(b, a) \not\in \left( \sqsubset_n^\mathsf{id} ; \{ (i, j) \} ; \sqsubset_n ^\mathsf{id} \right)$. We now assume that $b \sqsubset_n a$ which implies that $(b, a) \in \sqsubset_n^\mathsf{id}$. By transitivity of $\sqsubset_n$ we obtain that $(j, i) \in \sqsubset_n^\mathsf{id}$ and moreover that $j \sqsubset_n i$ since $i \neq j$ as $i < j$. By contradiction we obtain that $(b, a) \not\in \sqsubset_n$. We conclude that $b \not\sqsubset_{n + 1} a$.
\end{itemize}
}

(Transitivity). We are required to show that $\forall m \geq 1 \ldotp \sqsubset^m\ \subseteq\ \sqsubset$ The proof follows by induction on the number of self-composition steps, $m$.

{\parindent0pt
\textit{Base case}: $m = 1$

\textit{To show}: $\sqsubset^1\ \subseteq\ \sqsubset$ \\

The result follows directly by definition $\sqsubset^1\ =\ \sqsubset\ \subseteq\ \sqsubset$. \\

\textit{Inductive case}: $m > 1$

\textit{Inductive hypothesis}: $\sqsubset^m\ \subseteq\ \sqsubset$

\textit{To show}: $\sqsubset^{m + 1}\ \subseteq\ \sqsubset$
\begin{align*}
	\sqsubset^{m + 1}
	&=\ \sqsubset^m ; \sqsubset \text{ by associativity} \\
	&\subseteq\ \sqsubset ; \sqsubset \text{ by I.H.} \\
	&=\ \sqsubset^2 \text{by definition} \\
	&\subseteq\ \sqsubset \text{ by Lemma \ref{lem:total2}}
\end{align*}
}

(Totality). Let's pick arbitrary transaction identifiers $a, b \in N$ (\textsc{i}) for a finite $N$ and build the $\sqsubset$ relation on it until convergence, i.e. in a finite number of steps. If $(a, b) \in E^*$ or $(b, a) \in E*$ then we know that either $a \sqsubset b$ or $b \sqsubset a$ holds. On the other hand if there is no edge connecting $a$ to $b$ or $b$ to $a$ in $E^*$ (\textsc{ii}) then:
\begin{itemize}
	\item If $a = b$ then by irreflexivity of $\sqsubset$ we are done as totality is met.
	\item Without loss of generality, we say that $a < b$ (\textsc{iii}). Given that the construction of $\sqsubset$ terminated in some $m > 0$ steps (being $N$ a finite set), by (\textsc{i}), (\textsc{ii}) and (\textsc{iii}) we know that there must exist a construction step $n$ such that $0 < n < m$ where the tuple $(a, b)$ was inserted in the relation given that $\sqsubset_{n-1} \cup \left( \sqsubset_{n-1}^\mathsf{id} ; \{(a,b)\} ; \sqsubset_{n-1}^\mathsf{id} \right) \implies a \sqsubset_n b \implies a \sqsubset b$. 
\end{itemize}
\end{proof}

\lem \label{lem:total2} Given a serialization graph $(N, E) = \pred{SG}{\tau}$ for $\tau = \pred{trace}{h, \emptyset, \emptyset, \mathds{P}}, \mathds{P} \in \mathsf{Prog}, h \in \mathsf{Storage}$, and the $\sqsubset$ relation on the set $N$ we say that $\sqsubset^2\ \subseteq\ \sqsubset$.

{\parindent0pt
\begin{proof}
We proceed by induction on the number of $\sqsubset$ construction steps, $n$. \\

\textit{Base case}: $n = 0$

\textit{To show}: $\sqsubset_0^2\ \subseteq\ \sqsubset_0$

By definition we know that $\sqsubset_0 = E^*$, i.e. the transitive closure on the edges of the serialization graph $\pred{SG}{\tau}$. It follows that by definition of transitive closure, $\sqsubset_0^2\ = E^* ; E^* = E^*$ meaning that $\sqsubset_0^2\ \subseteq\ \sqsubset_0$. \\

\textit{Inductive case}: $n > 0$

\textit{Inductive hypothesis}: $\sqsubset_n^2\ \subseteq\ \sqsubset_n$

\textit{To show}: $\sqsubset_{n + 1}^2\ \subseteq\ \sqsubset_{n + 1}$

We can rewrite the formula to be proven as the following, for some $i, j \in N$ such that $i < j, i \not\sqsubset_n j, j \not\sqsubset_n i$:
\begin{align}
	\left( \sqsubset_n \cup \underbrace{\left( \sqsubset_n^\mathsf{id} ; \{ (i, j) \} ; \sqsubset_n^\mathsf{id} \right)}_{R} \right) ; \left( \sqsubset_n \cup \left( \sqsubset_n^\mathsf{id} ; \{ (i, j) \} ; \sqsubset_n^\mathsf{id} \right) \right) &\subseteq \sqsubset_{n + 1} \\
	\label{thm:total1} \sqsubset_n ; \sqsubset_n \cup \sqsubset_n ; R \cup R ; \sqsubset_n \cup R ; R &\subseteq\ \sqsubset_{n + 1} \text{ by distributivity}
\end{align}
It now suffices to show that each unioned set in the L.H.S. of (\ref{thm:total1}) is a subset of $\sqsubset_{n + 1}$ itself.
\begin{itemize}
	\item \textit{To show}: $\sqsubset_n ; \sqsubset_n\ \subseteq\ \sqsubset_{n + 1}$
		\begin{align}
			\sqsubset_n ; \sqsubset_n\ &=\ \sqsubset_n^2 \\
			\text{by I.H.}&\subseteq\ \sqsubset_n\ \subseteq\ \sqsubset_{n + 1}
		\end{align}
	\item \textit{To show}: $\sqsubset_n ; \left( \sqsubset_n^\mathsf{id} ; \{ (i, j) \} ; \sqsubset_n^\mathsf{id} \right) \subseteq\ \sqsubset_{n + 1}$
		\begin{align}
			S  &=\ \sqsubset_n ; \sqsubset_n^\mathsf{id} \\
				&=\ \sqsubset_n ; \left( \sqsubset_n \cup\ \pred{Id}{N} \right) \text{by definition} \\
				&=\ \sqsubset_n ; \sqsubset_n \cup \sqsubset_n ; \pred{Id}{N} \\
				&=\ \sqsubset_n^2 \cup \sqsubset_n \\
				&\subseteq\ \sqsubset_n \text{by I.H.} \\
				\label{thm:total2} &\subseteq\ \sqsubset_n^\mathsf{id} \text{by definition} \\
			\sqsubset_n ; \left( \sqsubset_n^\mathsf{id} ; \{ (i, j) \} ; \sqsubset_n^\mathsf{id} \right) &= \left( \sqsubset_n ; \sqsubset_n^\mathsf{id} ; \{ (i, j) \} \right) ; \sqsubset_n^\mathsf{id} \text{ by associativity} \\
			&= \left( S ; \{ (i, j) \} \right) ; \sqsubset_n^\mathsf{id} \text{ by associativity} \\
			&\subseteq\ \sqsubset_n^\mathsf{id} ; \{ (i, j) \} ; \sqsubset_n^\mathsf{id} \text{by (\ref{thm:total2})} \\
			&\subseteq\ \sqsubset_{n + 1}
		\end{align}
	\item \textit{To show}: $\left( \sqsubset_n^\mathsf{id} ; \{ (i, j) \} ; \sqsubset_n^\mathsf{id} \right) ; \sqsubset_n \subseteq\ \sqsubset_{n + 1}$
		\begin{align}
			S' &=\ \sqsubset_n^\mathsf{id} ; \sqsubset_n \\
				&= \left( \sqsubset_n \cup\ \pred{Id}{N} \right) ; \sqsubset_n \text{by definition} \\
				&=\ \sqsubset_n ; \sqsubset_n \cup\ \pred{Id}{N} ; \sqsubset_n \\
				&=\ \sqsubset_n^2 \cup \sqsubset_n \\
				&\subseteq\ \sqsubset_n \text{by I.H.} \\
				\label{thm:total3} &\subseteq\ \sqsubset_n^\mathsf{id} \text{by definition} \\
			\left( \sqsubset_n^\mathsf{id} ; \{ (i, j) \} ; \sqsubset_n^\mathsf{id} \right) ; \sqsubset_n &=\ \sqsubset_n^\mathsf{id} ; \left( \{ (i, j) \} ; \sqsubset_n^\mathsf{id} ; \sqsubset_n \right) \text{ by associativity} \\
			&=\ \sqsubset_n^\mathsf{id} ; \left( \{ (i, j) \} ; S' \right) \text{ by associativity} \\
			&\subseteq\ \sqsubset_n^\mathsf{id} ; \{ (i, j) \} ; \sqsubset_n^\mathsf{id} \text{by (\ref{thm:total3})} \\
			&\subseteq\ \sqsubset_{n + 1}
		\end{align}
	\item \textit{To show}: $\left( \sqsubset_n^\mathsf{id} ; \{ (i, j) \} ; \sqsubset_n^\mathsf{id} \right) ; \left( \sqsubset_n^\mathsf{id} ; \{ (i, j) \} ; \sqsubset_n^\mathsf{id} \right) \subseteq\ \sqsubset_{n + 1}$
	
		Let's assume that $\left( \sqsubset_n^\mathsf{id} ; \{ (i, j) \} ; \sqsubset_n^\mathsf{id} \right) ; \left( \sqsubset_n^\mathsf{id} ; \{ (i, j) \} ; \sqsubset_n^\mathsf{id} \right) \neq \emptyset$ meaning that the set at least contains a tuple $(a, b)$, for $a, b \in N$. 
		\begin{align}
			(a, b) \in \left( \sqsubset_n^\mathsf{id} ; \{ (i, j) \} ; \sqsubset_n^\mathsf{id} \right) ; \left( \sqsubset_n^\mathsf{id} ; \{ (i, j) \} ; \sqsubset_n^\mathsf{id} \right)
				&\iff \\
			\exists c \ldotp (a, c) \in \left( \sqsubset_n^\mathsf{id} ; \{ (i, j) \} ; \sqsubset_n^\mathsf{id} \right) \land (c, b) \in \left( \sqsubset_n^\mathsf{id} ; \{ (i, j) \} ; \sqsubset_n^\mathsf{id} \right)
				&\iff \\
			\exists c \ldotp (a, i) \in\ \sqsubset_n^\mathsf{id} \land (i, j) \in  \{(i, j)\} \land (j, c) \in\ \sqsubset_n^\mathsf{id} \\ \land (c, i) \in\ \sqsubset_n^\mathsf{id} \land (i, j) \in  \{(i, j)\} \land (j, b) \in\ \sqsubset_n^\mathsf{id}
				&\implies \\
			\label{thm:total4} \exists c \ldotp (j, c) \in\ \sqsubset_n^\mathsf{id} \land (c, i) \in\ \sqsubset_n^\mathsf{id}
		\end{align}
		We know that $i \neq j$ since $i < j$ so the proof proceeds with a case-by-case analysis on $c$.
		\begin{itemize}
			\item If $c = i$ then we know that $c \neq j$ and by (\ref{thm:total4}) we have that $(j, i) \in\ \sqsubset_n^\mathsf{id} \land (i, i) \in\ \sqsubset_n^\mathsf{id}$ from which it follows that $(j, i) \in\ \sqsubset_n$, a contradiction.
			\item If $c = j$ then we know that $c \neq i$ and by (\ref{thm:total4}) we have that $(j, j) \in\ \sqsubset_n^\mathsf{id} \land (j, i) \in\ \sqsubset_n^\mathsf{id}$ from which it follows that $(j, i) \in\ \sqsubset_n$, a contradiction.
			\item If $c \neq i$ and $c \neq j$ then (\ref{thm:total4}) it follows that $(j, c) \in\ \sqsubset_n \land (c, i) \in\ \sqsubset_n$. By I.H. we know that $\sqsubset_n^2\ \subseteq\ \sqsubset_n$. By definition we know that if $(j, c) \in\ \sqsubset_n$ and $(c, i) \in\ \sqsubset_n$ then $(j, i) \in\ \sqsubset_n^2$. By I.H. this means that $(j, i) \in\ \sqsubset_n$ which is again a contradiction.
		\end{itemize}
		By contradiction we conclude that $\left( \sqsubset_n^\mathsf{id} ; \{ (i, j) \} ; \sqsubset_n^\mathsf{id} \right) ; \left( \sqsubset_n^\mathsf{id} ; \{ (i, j) \} ; \sqsubset_n^\mathsf{id} \right) = \emptyset$ meaning that $\left( \sqsubset_n^\mathsf{id} ; \{ (i, j) \} ; \sqsubset_n^\mathsf{id} \right) ; \left( \sqsubset_n^\mathsf{id} ; \{ (i, j) \} ; \sqsubset_n^\mathsf{id} \right) \subseteq\ \sqsubset_{n + 1}$.
		
		Direct proof.
		\begin{align*}
			\{ (i, j) \} ; \sqsubset_n^\mathsf{id} ; \{ (i, j) \}
				&=
			\{ (i, j) \} ; \left( \sqsubset_n \cup \pred{Id}{N} \right) ; \{ (i, j) \} \\
				&=
			\{ (i, j) \} ; \left( \sqsubset_n ; \{ (i, j) \} \cup \{ (i, j) \} \right) \\
				&=
			\left( \{ (i, j) \} ; \sqsubset_n ; \{ (i, j) \} \right) \cup \left(\{ (i, j) \} ; \{ (i, j) \} \right) \\
			 	&=
			 \left( \{ (i, j) \} ; \sqsubset_n ; \{ (i, j) \} \right) = \emptyset \\
			\left( \sqsubset_n^\mathsf{id} ; \{ (i, j) \} ; \sqsubset_n^\mathsf{id} \right) ; \left( \sqsubset_n^\mathsf{id} ; \{ (i, j) \} ; \sqsubset_n^\mathsf{id} \right)
				&=
			\left( \sqsubset_n^\mathsf{id} ; \{ (i, j) \} \right) ; \left( \sqsubset_n^\mathsf{id} ; \left( \sqsubset_n^\mathsf{id} ; \{ (i, j) \} ; \sqsubset_n^\mathsf{id} \right) \right) \\
				&=
			\left( \sqsubset_n^\mathsf{id} ; \{ (i, j) \} \right) ; \left( \left( \sqsubset_n^\mathsf{id} ;  \sqsubset_n^\mathsf{id} \right) ; \left( \{ (i, j) \} ; \sqsubset_n^\mathsf{id} \right) \right) \\
				&=
			\left( \sqsubset_n^\mathsf{id} ; \{ (i, j) \} \right) ; \left( \sqsubset_n^\mathsf{id} ; \left( \{ (i, j) \} ; \sqsubset_n^\mathsf{id} \right) \right) \\
				&=\
			\sqsubset_n^\mathsf{id} ; \left( \{ (i, j) \} ; \sqsubset_n^\mathsf{id} ; \{ (i, j) \} \right) ; \sqsubset_n^\mathsf{id} \\
				&=\
			\sqsubset_n^\mathsf{id} ; \emptyset ; \sqsubset_n^\mathsf{id} = \emptyset
		\end{align*}
\end{itemize}
\end{proof}
}
