\section{Conclusion}

\subsection{Future Work}

The results obtained as part of this project, together with the approach taken in building an overall framework to reason about concurrency in a \tpl\ setting, open up the possibility of both tayloring its use to a particular application or extending the work to wider scenarios. We explore the latter by looking at how this work can be expanded or improved.
\begin{itemize}
	\item Extend the focus of the framework to non-serializable or weak models of transactional concurrency. Nowadays, specially in the context of distributed systems, full serializability appears to be too expensive in terms of performance as the explicit synchronization of single items is a bottleneck. This is the reason why weaker consistency and isolation properties, usually enforced by optimistic concurrency control protocols, are preferred. It would therefore be extremely interesting to investigate how the work done here can be modified and ported to these different level of consistency.  Parallel work to this is being done by a member of the Program Specification and Verification Group, Shale Xiong. The focus there is on the analysis of the snapshot isolation guarantee, as opposed to serializability. The goal is to take a similar path as compared to the one in this thesis, and define a formal model, operational semantics and a program logic on top of it.
	
	\item Model other related protocols and more than that, expand the framework to a general one for serializability. This would effectively mean to keep the parts of our model that do not directly refer to \tpl, while parameterizing all the rest. Instantiations could then be built by providing the required well formed constructs that are able to express the behaviour of the particular protocol. In terms of operational semantics, we would need to find a set of rules that, while abstracting large parts of the particular scenario, still allow to prove serializability and atomic equivalence of instantiations.
	
	\item If the timescale of the project was larger, the next step would have been to prove the simulation between a concrete implementation of a system and our operational semantics. The mentioned system would be required to adopt a particular flavour of two-phase locking to manage concurrent access to data items. An implementation has already been written in the C language and, given the premature effort to prove its equivalence, is not included in this report.
	
	\item Take inspiration from the work done in TaDA \cite{tada} in terms of abstract atomicity, in order to investigate the formal meaning of combining multiple atomic transactions in a larger and \textit{fictionally-atomic} one. In relation to this, it would also be interesting to further explore the inverse situation, where we divide an existing transaction into multiple ones, without affecting the overall result of executions. The latter technique is referred to in research as \textit{transaction chopping} and understanding how it applies to our model and semantics would give us the possibility of establishing more powerful properties.
\end{itemize}