\thm \label{thm:eSound} (Elementary command soundness). For all $\hat{\mathds{C}} \in \mathsf{ECmd}$, their corresponding axiom $(M_1, \hat{\mathds{C}}, M_2) \in \mathsf{Ax}_{\hat{\mathsf{C}}}$ and any given machine state $m = (h, s) \in \mathsf{Storage} \times \mathsf{Stack}$, the following must hold.
\[
	\tsem{\hat{\mathds{C}}}_{\hat{\mathsf{C}}} \left( \lfloor M_1 \bullet_\mathbb{M} \{m\} \rfloor_\mathbb{M} \right) \subseteq \lfloor M_2 \bullet_\mathbb{M} \{m\} \rfloor_\mathbb{M}
\]
{\parindent0pt
\begin{proof}
By induction on the structure of $\hat{\mathds{C}}$. \\

\textit{Case}: $\passign{\pvar{x}}{\mathds{E}}$

\textit{To show}: $\tsem{\passign{\pvar{x}}{\mathds{E}}}_{\hat{\mathsf{C}}} \left( \lfloor M_1 \bullet_\mathbb{M} \{m\} \rfloor_\mathbb{M} \right) \subseteq \lfloor M_2 \bullet_\mathbb{M} \{m\} \rfloor_\mathbb{M}$

Let's pick an arbitrary $m_1 \in M_1$, now by definition of $\lfloor - \rfloor_\mathbb{M}$, it is now sufficient to show that the following holds for some $m_2 \in M_2$:
\begin{gather}
	\label{thm:CH1} \tsem{\passign{\pvar{x}}{\mathds{E}}}_{\hat{\mathsf{C}}} \left(m_1 \bullet_\mathbb{M} m \right) = \{ m_2 \bullet_\mathbb{M} m \}
\end{gather}

From the definition of the \textsc{Assign} axiom, we know that $m_1 = (h_1, s_1)$ can be any machine state and $m_2 = (h_1, s_1[\pvar{x} \mapsto v])$ where $v = \tsem{\mathds{E}}^\textsc{e}_{s_1}$. Then by definition of $\tsem{\passign{\pvar{x}}{\mathds{E}}}_{\hat{\mathsf{C}}}$, the first projection of the return value is $m_1 \bullet_\mathbb{M} m$. We can conclude that (\ref{thm:CH1}) holds.  \\

\textit{Case}: $\palloc{\pvar{x}}{\mathds{E}}$

\textit{To show}: $\tsem{\palloc{\pvar{x}}{\mathds{E}}}_{\hat{\mathsf{C}}} \left( \lfloor M_1 \bullet_\mathbb{M} \{m\} \rfloor_\mathbb{M} \right) \subseteq \lfloor M_2 \bullet_\mathbb{M} \{m\} \rfloor_\mathbb{M}$

Let's pick an arbitrary $m_1 \in M_1$, now by definition of $\lfloor - \rfloor_\mathbb{M}$, it is now sufficient to show that the following holds for some $m_2 \in M_2$:
\begin{gather}
	\label{thm:CH3} \tsem{\palloc{\pvar{x}}{\mathds{E}}}_{\hat{\mathsf{C}}} \left(m_1 \bullet_\mathbb{M} m, s \right) = \{ m_2 \bullet_\mathbb{M} m \}
\end{gather}

From the definition of the \textsc{Alloc} axiom, we know that $m_1 \in \{ (\emptyset, \emptyset) \}$ meaning that $m_1 = (\emptyset, \emptyset)$ and $m_2 = \left([l \mapsto 0]\ldots[l + n - 1 \mapsto 0], [\pvar{x} \mapsto l] \right)$ for $n = \tsem{\mathds{E}}_s^\textsc{e}$, $n > 0$ and some arbitrary $l \in \mathsf{Key}$. Then by definition of $\tsem{\palloc{\pvar{x}}{\mathds{E}}}_{\hat{\mathsf{C}}}$, the return value is:
\begin{align*}
	m' &= ((h \uplus \emptyset)[l \mapsto 0]\ldots[l + n - 1 \mapsto 0], (s \uplus \emptyset)[\pvar{x} \mapsto l]) \\
	&= (h[l \mapsto 0]\ldots[l + n - 1 \mapsto 0], s[\pvar{x} \mapsto l])
\end{align*}
We can pick $l$ such that $\{ l, \ldots, l + n -1 \} \cap \pred{dom}{h} \equiv \emptyset$ and as a consequence express $m' = m_2 \bullet_\mathbb{M} m$. We can conclude that (\ref{thm:CH3}) holds. \\

\textit{Case}: $\pderef{\pvar{x}}{\mathds{E}}$

\textit{To show}: $\tsem{\pderef{\pvar{x}}{\mathds{E}}}_{\hat{\mathsf{C}}} \left( \lfloor M_1 \bullet_\mathbb{M} \{m\} \rfloor_\mathbb{M} \right) \subseteq \lfloor M_2 \bullet_\mathbb{M} \{m\} \rfloor_\mathbb{M}$

Let's pick an arbitrary $m_1 \in M_1$, it is now sufficient to show that the following holds for some $m_2 \in M_2$:
\begin{gather}
	\label{thm:CH4} \tsem{\pderef{\pvar{x}}{\mathds{E}}}_{\hat{\mathsf{C}}} \left( m_1 \bullet_\mathbb{M} m\right) = \{m_2 \bullet_\mathbb{M} m\}
\end{gather}

From the definition of the \textsc{Read} axiom, we know that for $k = \tsem{\mathds{E}}_s^\textsc{e}$ and some $v \in \mathsf{Val}$:
\begin{gather}
	\label{thm:CH7}
	m_1 = (h_1, s_1) = ([k \mapsto v], [\pvar{x} \mapsto -]) \land m_2 = ([k \mapsto v], [\pvar{x} \mapsto v])
\end{gather}
therefore we know that $k \in \pred{dom}{h_1} \land \pvar{x} \in \pred{dom}{m_1 \downarrow_2}$ and it follows that $k \not\in \pred{dom}{h} \land \pvar{x} \not\in \pred{dom}{s}$ since $m_1 \bullet_\mathbb{M} m$ is defined. Then from the definition of $\tsem{\pderef{\pvar{x}}{\mathds{E}}}_{\hat{\mathsf{C}}}$ and \ref{thm:CH7} we have:
\begin{align*}
	\tsem{\pderef{\pvar{x}}{\mathds{E}}}_{\hat{\mathsf{C}}}
		&=
	\{ (h_1 \uplus h, (s_1 \uplus s)[\pvar{x} \mapsto v]) \} \\
		&=
	\{ ([k \mapsto v] \uplus h, ([\pvar{x} \mapsto -] \uplus s)[\pvar{x} \mapsto v]) \} \\
		&=
	\{ ([k \mapsto v] \uplus h, [\pvar{x} \mapsto v] \uplus s) \} \\
		&=
	\{ ([k \mapsto v], [\pvar{x} \mapsto v]) \uplus_2 (h, s) \} \\
		&=
	\{ m_2 \uplus m \} = \{ m_2 \bullet_\mathbb{M} m \}
\end{align*}

\textit{Case}: $\pmutate{\mathds{E}_1}{\mathds{E}_2}$

\textit{To show}: $\tsem{\pmutate{\mathds{E}_1}{\mathds{E}_2}}_{\hat{\mathsf{C}}} \left( \lfloor M_1 \bullet_\mathbb{M} \{m\} \rfloor_\mathbb{M} \right) \subseteq \lfloor M_2 \bullet_\mathbb{M} \{m\} \rfloor_\mathbb{M}$

Let's pick an arbitrary $m_1 \in M_1$, now by definition of $\lfloor - \rfloor_\mathbb{M}$, it is sufficient to show that the following holds for some $m_2 \in M_2$:
\begin{gather}
	\label{thm:CH5} \tsem{\pmutate{\mathds{E}_1}{\mathds{E}_2}}_{\hat{\mathsf{C}}} \left( m_1 \bullet_\mathbb{M} m \right) = \{ m_2 \bullet_\mathbb{M} m \}
\end{gather}

From the definition of the \textsc{Write} axiom, we know that for $k = \tsem{\mathds{E}_1}_s^\textsc{e}$ and $v = \tsem{\mathds{E}_2}_s^\textsc{e}$,
\begin{gather}
	\label{thm:CH6}
	m_1 = (h_1, s_1) = ([k \mapsto -], s_1) \land m_2 = ([k \mapsto v], s_1)
\end{gather}
This implies that $k \in \pred{dom}{h_1}$ which requires that $k \not\in \pred{dom}{h}$ given that $m_1 \bullet_\mathbb{M} m$ is defined by assumption. Then, from the definition of $\tsem{\pmutate{\mathds{E}_1}{\mathds{E}_2}}_{\hat{\mathsf{C}}}$ and (\ref{thm:CH6}), we have:
\begin{align*}
	\tsem{\pmutate{\mathds{E}_1}{\mathds{E}_2}}_{\hat{\mathsf{C}}} \left( m_1 \bullet_\mathbb{M} m \right)
		&=
	\{((h_1 \uplus h)[k \mapsto v], s_1 \uplus s)\} \\
		&=
	\{(([k \mapsto -] \uplus h)[k \mapsto v], s_1 \uplus s)\} \\
		&=
	\{([k \mapsto v] \uplus h, s_1 \uplus s)\} \\
		&=
	\{ ([k \mapsto v], s_1) \uplus_2 (h, s) \} \\
		&=
	\{ m_2 \uplus_2 m \} = \{ m_2 \bullet_\mathbb{M} m \}
\end{align*}
\end{proof}
}