\section{Atom Proof}

\[
\begin{array}{r l}
	\pred{atom}{\mathds{P}} \triangleq&
	\begin{array}{l}
	\forall h, h', S, S', \Phi \ldotp \\
	(h, \Phi, S, \mathds{P}) \rightarrow^* (h', \emptyset, S', \pskip) \implies 
	(h, \mathds{P}) \tred^* (h', \pskip)
	\end{array}
\end{array}
\]

{\parindent0pt
\begin{proof}
$\forall \mathds{P} \in \mathsf{Prog} \ldotp \pred{atom}{\mathds{P}}$ by induction on the structure of programs $\mathsf{Prog}$. \\

\textit{Base case 1}: $\pskip \in \mathsf{Prog}$

\textit{To show}: $\pred{atom}{\pskip}$

For arbitrary $h, h', S, S', \Phi$ we assume that $(h, \Phi, S, \pskip) \rightarrow^* (h', \emptyset, S', \pskip)$ holds, and given that $\pskip$ has no possible one-step reductions, it must be the case that it is a zero-step reduction. Therefore we have $h = h', \Phi = \emptyset, S = S'$. Starting from $(h, \pskip)$ through the $\tred$ relation, we can always reach $(h, \pskip)$ via a zero-step reduction $(h, \pskip) \tred^0 (h, \pskip)$. We can conclude that $(h, \Phi, S, \pskip) \rightarrow^* (h', \emptyset, S', \pskip) \implies (h, \pskip) \tred^* (h', \pskip)$ where $h = h'$. \\

\textit{Base case 2}: $\mathds{T} \in \mathsf{Prog}$

\textit{To show}: $\pred{atom}{\mathds{T}}$ by induction on the structure of transactions $\mathsf{Trans}$. This leaves us to prove that $\pred{atom}{\mathtt{begin}\ \mathds{C}\ \mathtt{end}_\iota}$ holds, given that $\mathtt{begin}\ \mathds{C}\ \mathtt{end} \in \mathsf{UTrans}$ will always be reduced to the previous case by the \textsc{Start} rule. We proceed by proving the $\mathsf{cAtom}$ property by induction on the structure of commands $\mathsf{Cmd}$.


-\\

\textit{Inductive case 1}: $\mathds{P}_1 + \mathds{P}_2 \in \mathsf{Prog}$

\textit{To show}: $\pred{cAtom}{\mathds{P}_1 + \mathds{P}_2}$

\textit{Inductive hypothesis}: $\pred{atom}{\mathds{P}_1} \land \pred{atom}{\mathds{P}_2}.$

For arbitrary $h, h', S, S', \Phi$ we assume that $(h, \Phi, S, \mathds{P}_1 + \mathds{P}_2) \rightarrow^* (h', \emptyset, S', \pskip)$ holds. Now we are presented with two cases:
\begin{enumerate}
\item We can reduce $(h, \Phi, S, \mathds{P}_1 + \mathds{P}_2) \xrightarrow{\actid} (h, \Phi, S, \mathds{P}_1)$ with one step through the rule \textsc{ChoiceL}, which we can always apply since it has an empty premiss. We can also always reduce $(h, \mathds{P}_1 + \mathds{P}_2) \tred (h, \mathds{P}_1)$ through the rule \textsc{AtChoiceL} given it has an empty premiss. By inductive hypothesis on $\mathds{P}_1$ we obtain that $(h, \Phi, S, \mathds{P}_1) \rightarrow^* (h', \emptyset, S', \pskip) \implies (h, \mathds{P}_1) \tred^* (h', \pskip)$. Therefore we can conclude that $(h, \Phi, S, \mathds{P}_1 + \mathds{P}_2) \rightarrow^* (h', \emptyset, S', \pskip) \implies  (h, \mathds{P}_1 + \mathds{P}_2) \tred^* (h', \pskip)$.
\item We can reduce $(h, \Phi, S, \mathds{P}_1 + \mathds{P}_2) \xrightarrow{\actid} (h, \Phi, S, \mathds{P}_2)$ with one step through the rule \textsc{ChoiceR}, which we can always apply since it has an empty premiss. We can also always reduce $(h, \mathds{P}_1 + \mathds{P}_2) \tred (h, \mathds{P}_2)$ through the rule \textsc{AtChoiceR} given it has an empty premiss. By inductive hypothesis on $\mathds{P}_2$ we obtain that $(h, \Phi, S, \mathds{P}_2) \rightarrow^* (h', \emptyset, S', \pskip) \implies (h, \mathds{P}_2) \tred^* (h', \pskip)$. Therefore we can conclude that $(h, \Phi, S, \mathds{P}_1 + \mathds{P}_2) \rightarrow^* (h', \emptyset, S', \pskip) \implies  (h, \mathds{P}_1 + \mathds{P}_2) \tred^* (h', \pskip)$. \\
\end{enumerate}

\textit{Inductive case 2}: $\mathds{P}_1 ; \mathds{P}_2 \in \mathsf{Prog}$

\textit{To show}: $\pred{atom}{\mathds{P}_1 ; \mathds{P}_2}$

\textit{Inductive hypothesis}: $\pred{atom}{\mathds{P}_1} \land \pred{atom}{\mathds{P}_2}$

For arbitrary $h, h', S, S', \Phi$ we assume that $(h, \Phi, S, \mathds{P}_1 ; \mathds{P}_2) \rightarrow^* (h', \emptyset, S', \pskip)$ holds. Given the overall reduction from $\mathds{P}_1 ; \mathds{P}_2$ to $\pskip$ we must have a chain of reductions of the following shape, for some $h'', \Phi'', S''$ where $\Phi'' = \emptyset$ by Lemma \ref{ref:phiemp}.
\[
\underbrace{(h, \Phi, S, \mathds{P}_1 ; \mathds{P}_2) \rightarrow^* (h'', \Phi'', S'', \pskip; \mathds{P}_2)}_{(\textsc{i})}
\rightarrow (h'', \Phi'', S'', \mathds{P}_2) \rightarrow^* (h', \emptyset, S', \pskip)
\]
\begin{enumerate}
\item \label{seq:1} By (\textsc{i}) and Lemma \ref{ref:2seq} we get that $(h, \Phi, S, \mathds{P}_1) \rightarrow^* (h'', \Phi'', S'', \pskip)$ holds.
\item \label{seq:2} By \ref{seq:1} and the inductive hypothesis on $\mathds{P}_1$ we obtain that $(h, \mathds{P}_1) \tred^* (h'', \pskip)$.
\item By \ref{seq:2} and Lemma \ref{ref:aseq} we get that $(h, \mathds{P}_1 ; \mathds{P}_2) \tred^* (h'', \pskip ; \mathds{P}_2)$
\end{enumerate}

At this point we can apply the \textsc{PSeqSkip} rule to reduce $(h'', \Phi'', S'', \pskip ; \mathds{P}_2) \rightarrow^* (h'', \Phi'', S'', \mathds{P}_2)$ and rule \textsc{AtPSeqSkip} to reduce $(h'', \pskip ; \mathds{P}_2) \tred^* (h'', \mathds{P}_2)$. By inductive hypothesis on $\mathds{P}_2$ we can conclude that $(h, \Phi, S, \mathds{P}_1 ; \mathds{P}_2) \rightarrow^* (h', \emptyset, S', \pskip) \implies (h, \mathds{P}_1 ; \mathds{P}_2) \tred^* (h', \pskip)$. \\

\textit{Inductive case 3}: $\mathds{P}^* \in \mathsf{Prog}$

\textit{To show}: $\pred{atom}{\mathds{P}^*}$

\textit{Inductive hypothesis}: $\pred{atom}{\mathds{P}}$

For arbitrary $h, h', S, S', \Phi$ we assume that $(h, \Phi, S, \mathds{P}^*) \rightarrow^* (h', \emptyset, S', \pskip)$ holds (\textsc{i}). Given the overall reduction from $\mathds{P}^*$ to $\pskip$ we must have a chain of reductions of the following shape, for some $h'', \Phi'', S''$.
\[
(h, \Phi, S, \mathds{P}^*) \rightarrow^* (h'', \Phi'', S'', (\pskip + \mathds{P} ; \mathds{P}^*)) \rightarrow^*  (h', \emptyset, S', (\pskip + \mathds{P} ; \mathds{P}^*)) \xrightarrow{\actid} (h', \emptyset, S', \pskip)
\]
Through the \textsc{Loop} rule (\textsc{ii}), we can always reduce $(h, \Phi, S, \mathds{P}^*) \xrightarrow{\actid} (h, \Phi, S, \pskip + (\mathds{P} ; \mathds{P}^*))$ given that it has an empty premiss. Similarly, we can always reduce any $(h'', \mathds{P}^*) \tred (h'', \pskip + (\mathds{P} ; \mathds{P}^*))$ via the \textsc{AtLoop} rule as it also has an empty premiss. We now consider two possible cases:
\begin{enumerate}
\item \label{loop:1} We reduce $(h, \Phi, S, \pskip + (\mathds{P} ; \mathds{P}^*)) \xrightarrow{\actid} (h, \Phi, S, \pskip)$ through the \textsc{ChoiceL} rule which we can always do, together with \textsc{AtChoiceL} that reduces $(h, \pskip + (\mathds{P} ; \mathds{P}^*)) \tred (h, \pskip)$ where $\Phi = \emptyset$ by Lemma \ref{ref:phiemp}. In this scenario we directly obtain the result.
\item We reduce $(h, \Phi, S, \pskip + (\mathds{P} ; \mathds{P}^*)) \xrightarrow{\actid} (h, \Phi, S, \mathds{P} ; \mathds{P}^*)$ through the \textsc{ChoiceL} rule which we can always do, together with \textsc{AtChoiceL} that reduces $(h, \pskip + (\mathds{P} ; \mathds{P}^*)) \tred (h, \mathds{P} ; \mathds{P}^*)$. By Lemma \ref{ref:2seq}, Lemma \ref{ref:aseq} and the inductive hypothesis on $\mathds{P}$ we get that $(h, \Phi, S, \mathds{P} ; \mathds{P}^*) \rightarrow^* (h'', \Phi'', S'', \pskip ; \mathds{P}^*) \implies (h, \mathds{P} ; \mathds{P}^*) \tred^* (h'', \pskip ; \mathds{P}^*)$. It is now possible to further reduce $(h'', \Phi'', S'', \pskip ; \mathds{P}^*) \xrightarrow{\actid} (h'', \Phi'', S'', \mathds{P}^*)$ via \textsc{PSeqSkip} and $(h'', \pskip ; \mathds{P}^*) \tred (h'', \mathds{P}^*)$ through \textsc{AtPSeqSkip}. We are now in the position to repeat the process from point (\textsc{ii}) until case \ref{loop:1} is encountered at which point we have a reduction to $\pskip$. This will eventually happen given our initial assumption (\textsc{i}). \\
\end{enumerate}

\textit{Inductive case 4}: $\mathds{P}_1 \| \mathds{P}_2 \in \mathsf{Prog}$

\textit{To show}: $\pred{atom}{\mathds{P}_1 \| \mathds{P}_2}$

\textit{Inductive hypothesis}: $\pred{atom}{\mathds{P}_1} \land \pred{atom}{\mathds{P}_2}$

For arbitrary $h, h', S, S', \Phi$ we assume that $(h, \Phi, S, \mathds{P}_1 \| \mathds{P}_2) \rightarrow^* (h', \emptyset, S', \pskip)$ holds.
\end{proof}
}

\subsection{Lemmas}

\lem \label{lem:catom}
\[
\begin{array}{r l}
	\pred{cAtom}{\mathds{C}} \triangleq&
	\begin{array}{l}
	\forall \iota, s, s', p, p', \Phi, h, h', \mathds{C}', \alpha \ldotp \\
	(s, p, \mathds{C}) \xrightarrow{\alpha}_\iota (s', p', \mathds{C}') \land h' \in \left( \llbracket \alpha \rrbracket(h, \Phi) \downarrow_1 \right) \\
	\implies
	(h, s, \mathds{C}) \tred (h', s', \mathds{C}')
	\end{array}
\end{array}
\]

{\parindent0pt
\begin{proof}
$\forall \mathds{C} \in \mathsf{Cmd} \ldotp \pred{cAtom}{\mathds{C}}$ by induction on the structure of commands $\mathsf{Cmd}$. \\

\textit{Base case 1}: $\pskip \in \mathsf{Cmd}$

\textit{To show}: $\pred{cAtom}{\pskip}$

For arbitrary $\iota, s, s', p, p', \Phi, h, h', \mathds{C}', \alpha$ we assume that $(s, p, \pskip) \xrightarrow{\alpha}_\iota (s', p', \mathds{C}') \land h' \in \left( \llbracket \alpha \rrbracket(h, \Phi) \downarrow_1 \right)$ holds. Given that $\pskip$ has no possible one-step reductions, it must be a zero-step reduction such that $s = s', \mathds{C}' = \pskip, \alpha = \actid, h = h'$. Now, we can reduce $(h, s, \pskip) \tred (h', s', \pskip)$ through a zero step reduction, again where $s = s', h = h'$. \\

\textit{Base case 2}: $\passign{\pvar{x}}{\mathds{E}} \in \mathsf{Cmd}$

\textit{To show}: $\pred{cAtom}{\passign{\pvar{x}}{\mathds{E}}}$

For arbitrary $\iota, s, s', p, p', \Phi, h, h', \mathds{C}', \alpha$ we assume that $(s, p, \passign{\pvar{x}}{\mathds{E}}) \xrightarrow{\alpha}_\iota (s', p', \mathds{C}') \land h' \in \left( \llbracket \alpha \rrbracket(h, \Phi) \downarrow_1 \right)$ holds. The only way to reduce $\passign{\pvar{x}}{\mathds{E}}$ is through the \textsc{Assign} rule which makes $\alpha = \actid, \mathds{C}' = \pskip, s' = s[\pvar{x} \mapsto v], p' = p, h' = h$ where $v = \llbracket \mathds{E} \rrbracket_s$. In a similar way we can reduce $(h, s, \passign{\pvar{x}}{\mathds{E}}) \tred (h, s[\pvar{x} \mapsto v], \pskip)$ through the \textsc{AtAssign} rule. \\

\textit{Base case 3}: $\pderef{\pvar{x}}{\mathds{E}} \in \mathsf{Cmd}$

\textit{To show}: $\pred{cAtom}{\pderef{\pvar{x}}{\mathds{E}}}$

For arbitrary $\iota, s, s', p, p', \Phi, h, h', \mathds{C}', \alpha$ we assume that $(s, p, \pderef{\pvar{x}}{\mathds{E}}) \xrightarrow{\alpha} (s', p', \mathds{C}') \land h' \in \left( \llbracket \alpha \rrbracket(h, \Phi) \downarrow_1 \right)$ holds. The only way to reduce $\pderef{\pvar{x}}{\mathds{E}}$ is through the \textsc{Read} rule which makes $\mathds{C}' = \pskip, h' = h, s' = s[\pvar{x} \mapsto v], \alpha = \actread{\iota}{k}{v}$ where $k = \llbracket \mathds{E} \rrbracket_s$ and $v = h(k)$. In a similar way we can reduce $(h, s, \pderef{\pvar{x}}{\mathds{E}}) \tred (h, s[\pvar{x} \mapsto v], \pskip)$ through the \textsc{AtRead} rule. \\

\textit{Base case 4}: $\palloc{\pvar{x}}{\mathds{E}} \in \mathsf{Cmd}$

\textit{To show}: $\pred{cAtom}{\palloc{\pvar{x}}{\mathds{E}}}$

For arbitrary $\iota, s, s', p, p', \Phi, h, h', \mathds{C}', \alpha$ we assume that $(s, p, \palloc{\pvar{x}}{\mathds{E}}) \xrightarrow{\alpha} (s', p', \mathds{C}') \land h' \in \left( \llbracket \alpha \rrbracket(h, \Phi) \downarrow_1 \right)$ holds. The only way to reduce $\palloc{\pvar{x}}{\mathds{E}}$ is through the \textsc{Alloc} rule which makes $\mathds{C}' = \pskip, h' = h[l \mapsto 0]\ldots[l + n - 1 \mapsto 0], s' = s[\pvar{x} \mapsto l], \alpha = \actalloc{\iota}{n}{l}$ where $n = \llbracket \mathds{E} \rrbracket_s$. In a similar way we can reduce $(h, s, \palloc{\pvar{x}}{\mathds{E}}) \tred (h[l \mapsto 0]\ldots[l + n - 1 \mapsto 0], s[\pvar{x} \mapsto l], \pskip)$ through the \textsc{AtAlloc} rule. \\

\textit{Base case 5}: $\pmutate{\mathds{E}_1}{\mathds{E}_2} \in \mathsf{Cmd}$

\textit{To show}: $\pred{cAtom}{\pmutate{\mathds{E}_1}{\mathds{E}_2}}$

For arbitrary $\iota, s, s', p, p', \Phi, h, h', \mathds{C}', \alpha$ we assume that $(s, p, \pmutate{\mathds{E}_1}{\mathds{E}_2}) \xrightarrow{\alpha} (s', p', \mathds{C}') \land h' \in \left( \llbracket \alpha \rrbracket(h, \Phi) \downarrow_1 \right)$ holds. The only way to reduce $\pmutate{\mathds{E}_1}{\mathds{E}_2}$ is through the \textsc{Write} rule which makes $\mathds{C}' = \pskip, h' = h[k \mapsto v], s' = s, \alpha = \actwrite{\iota}{k}{v}$ where $k = \llbracket \mathds{E}_1 \rrbracket_s, v = \llbracket \mathds{E}_2 \rrbracket_s$.  In a similar way we can reduce $(h, s, \pmutate{\mathds{E}_1}{\mathds{E}_2}) \tred (h[k \mapsto v], s, \pskip)$ through the \textsc{AtWrite} rule. \\

\textit{Inductive case 1}: $\mathds{C}_1 ; \mathds{C}_2 \in \mathsf{Cmd}$

\textit{Inductive hypothesis}: $\pred{cAtom}{\mathds{C}_1}$

\textit{To show}: $\pred{cAtom}{\mathds{C}_1 ; \mathds{C}_2}$

For arbitrary $\iota, s, s', p, p', \Phi, h, h', \mathds{C}', \alpha$ we assume that $(s, p, \mathds{C}_1 ; \mathds{C}_2) \xrightarrow{\alpha} (s', p', \mathds{C}') \land h' \in \left( \llbracket \alpha \rrbracket(h, \Phi) \downarrow_1 \right)$ holds. There are two possible ways to reduce $\mathds{C}_1 ; \mathds{C}_2$.
\begin{enumerate}
\item When $\mathds{C}_1 = \pskip$, we can reduce $(s, p, \pskip ; \mathds{C}_2) \xrightarrow{\actid} (s, p, \mathds{C}_2)$ through the \textsc{SeqSkip} rule, therefore $\mathds{C}' = \mathds{C}_2$. We can do the same using the \textsc{AtSeqSkip} rule by reducing $(h, s, \pskip ; \mathds{C}_2) \tred (h, s, \mathds{C}_2)$.
\item When $\mathds{C}_1 \neq \pskip$, we can reduce $(s, p, \mathds{C}_1 ; \mathds{C}_2) \xrightarrow{\alpha} (s', p', \mathds{C}_1' ; \mathds{C}_2)$ through the \textsc{SeqSkip} rule, making $\mathds{C}' = \mathds{C}_1'; \mathds{C}_2$, by running $(s, p, \mathds{C}_1) \xrightarrow{\alpha} (s', p', \mathds{C}_1')$. By induction hypothesis on $\mathds{C}_1$ we obtain that $(h, s, \mathds{C}_1) \tred (h', s', \mathds{C}_1')$ which we can combine with rule \textsc{AtSeq} to obtain $(h, s, \mathds{C}_1 ; \mathds{C}_2) \tred (h', s', \mathds{C}_1' ; \mathds{C}_2)$. \\
\end{enumerate}

\textit{Inductive case 2}: $\pif{\mathds{B}}{\mathds{C}_1}{\mathds{C}_2} \in \mathsf{Cmd}$

\textit{Inductive hypothesis}: $\pred{cAtom}{\mathds{C}_1} \land \pred{cAtom}{\mathds{C}_2}$

\textit{To show}: $\pred{cAtom}{\pif{\mathds{B}}{\mathds{C}_1}{\mathds{C}_2}}$

For arbitrary $\iota, s, s', p, p', \Phi, h, h', \mathds{C}', \alpha$ we assume that $(s, p, \pif{\mathds{B}}{\mathds{C}_1}{\mathds{C}_2}) \xrightarrow{\alpha} (s', p', \mathds{C}') \land h' \in \left( \llbracket \alpha \rrbracket(h, \Phi) \downarrow_1 \right)$ holds. There are two possible ways to reduce $\pif{\mathds{B}}{\mathds{C}_1}{\mathds{C}_2}$ for $b = \llbracket \mathds{B} \rrbracket_{s}^{\textsc{b}}$.
\begin{enumerate}
\item When $b = \top$, we can reduce $(s, p, \pif{\mathds{B}}{\mathds{C}_1}{\mathds{C}_2}) \xrightarrow{\actid} (s, p, \mathds{C}_1)$ through the \textsc{CondT} rule. It is possible to do the same using the \textsc{AtCondT} rule to get $(h, s, \pif{\mathds{B}}{\mathds{C}_1}{\mathds{C}_2}) \tred (h, s, \mathds{C}_1)$.
\item When $b = \bot$, we can reduce $(s, p, \pif{\mathds{B}}{\mathds{C}_1}{\mathds{C}_2}) \xrightarrow{\actid} (s, p, \mathds{C}_2)$ through the \textsc{CondF} rule. It is possible to do the same using the \textsc{AtCondF} rule to get $(h, s, \pif{\mathds{B}}{\mathds{C}_1}{\mathds{C}_2}) \tred (h, s, \mathds{C}_2)$.
\end{enumerate}

\textit{Inductive case 3}: $\ploop{\mathds{B}}{\mathds{C}} \in \mathsf{Cmd}$

\textit{Inductive hypothesis}: $\pred{cAtom}{\mathds{C}}$

\textit{To show}: $\pred{cAtom}{\ploop{\mathds{B}}{\mathds{C}}}$

For arbitrary $\iota, s, s', p, p', \Phi, h, h', \mathds{C}', \alpha$ we assume that $(s, p, \ploop{\mathds{B}}{\mathds{C}}) \xrightarrow{\alpha} (s', p', \mathds{C}') \land h' \in \left( \llbracket \alpha \rrbracket(h, \Phi) \downarrow_1 \right)$ holds. There are two possible ways to reduce $\ploop{\mathds{B}}{\mathds{C}}$ for $b = \llbracket \mathds{B} \rrbracket_{s}^{\textsc{b}}$.
\begin{enumerate}
\item When $b = \top$, we can reduce $(s, p, \ploop{\mathds{B}}{\mathds{C}}) \xrightarrow{\actid} (s, p, \mathds{C} ; \ploop{\mathds{B}}{\mathds{C}})$ through the \textsc{LoopT}. In a similar way we can reduce $(h, s, \ploop{\mathds{B}}{\mathds{C}}) \tred (h, s, \mathds{C} ; \ploop{\mathds{B}}{\mathds{C}})$ via the \textsc{AtLoopT} rule.
\item When $b = \bot$, we can reduce $(s, p, \ploop{\mathds{B}}{\mathds{C}}) \xrightarrow{\actid} (s, p, \pskip)$ through the \textsc{LoopF}. In a similar way we can reduce $(h, s, \ploop{\mathds{B}}{\mathds{C}}) \tred (h, s, \pskip)$ via the \textsc{AtLoopF} rule.
\end{enumerate}

\end{proof}
}

\lem \label{ref:aseq}
\[
\begin{array}{r l}
	\pred{aseq}{\mathds{P}_1} \triangleq
	&
	\begin{array}{l}
	\forall h, h', \mathds{P}_2 \ldotp \\
	(h, \mathds{P}_1) \tred^* (h', \pskip) \Rightarrow 
	(h, \mathds{P}_1; \mathds{P}_2) \tred^* (h', \pskip; \mathds{P}_2)
	\end{array}
\end{array}
\]

{\parindent0pt
\begin{proof}
$\forall \mathds{P}_1 \ldotp \pred{aseq}{\mathds{P}_1}$ by induction on $n$, i.e. the number of reduction steps in $\tred^*$. \\

\textit{Base case}: $n = 0$

\textit{To show}: 
\[
\begin{array}{l}
\forall h, h', \mathds{P}_2 \ldotp \\
(h, \mathds{P}_1) \tred^0 (h', \pskip) \implies 
(h, \mathds{P}_1; \mathds{P}_2) \tred^* (h', \pskip; \mathds{P}_2)
\end{array}
\]
We assume $(h, \mathds{P}_1) \tred^0 (h', \pskip)$ holds and given it is a zero-step reduction, the only possible case is for $\mathds{P}_1 = \pskip$. Therefore $(h, \pskip) \tred^0 (h', \pskip)$ where $h = h'$. Now for $m = 0$ we have $(h, \pskip; \mathds{P}_2) \tred^0 (h', \pskip; \mathds{P}_2)$ given that $\tred^*$ is a reflexive relation, again for $h = h'$. \\

\textit{Inductive case}: For some arbitrary $n > 0$

\textit{Inductive hypothesis}: Assume the property holds for some arbitrary program $\mathds{P}_1'$ and $n$ steps, and such that $(h, \mathds{P}_1) \tred (h'', \mathds{P}_1')$.
\[
\begin{array}{l}
\forall h, h', \mathds{P}_2 \ldotp \\
(h, \mathds{P}_1') \tred^n (h', \pskip) \implies 
(h, \mathds{P}_1'; \mathds{P}_2) \tred^* (h', \pskip; \mathds{P}_2)
\end{array}
\]
\textit{To show}:
\[
\begin{array}{l}
\forall h, h', \mathds{P}_2 \ldotp \\
(h, \mathds{P}_1) \tred^{n+1} (h', \pskip) \implies 
(h, \mathds{P}_1; \mathds{P}_2) \tred^* (h', \pskip; \mathds{P}_2)
\end{array}
\]
We assume $(h, \mathds{P}_1) \tred^{n+1} (h', \pskip)$ holds and we also know that $(h, \mathds{P}_1) \tred (h'', \mathds{P}_1')$. The latter is the premiss of rule \textsc{PSeq} for the conclusion $(h, \mathds{P}_1; \mathds{P}_2) \tred (h'', \mathds{P}_1'; \mathds{P}_2)$ in one step of reduction. Therefore by this fact and the inductive hypothesis we obtain that $\pred{aseq}{\mathds{P}_1}$ holds for $n + 1$ steps.

\end{proof}
}

\lem \label{ref:2seq}
\[
\begin{array}{r l}
	\pred{2seq}{\mathds{P}_1} \triangleq
	&
	\begin{array}{l}
	\forall h, h', S, S', \Phi, \mathds{P}_2 \ldotp \\ 
	(h, \Phi, S, \mathds{P}_1; \mathds{P}_2) \rightarrow^* (h', \emptyset, S', \pskip; \mathds{P}_2)
	\Rightarrow
	(h, \Phi, S, \mathds{P}_1) \rightarrow^* (h', \emptyset, S', \pskip)
	\end{array}
\end{array}
\]

{\parindent0pt
\begin{proof}
$\forall \mathds{P}_1 \ldotp \pred{2seq}{\mathds{P}_1}$ by induction on $n$, i.e. the number of reduction steps in $\rightarrow^*$. \\

\textit{Base case}: $n = 0$

\textit{To show}:
\[
\begin{array}{l}
\forall h, h', S, S', \Phi, \mathds{P}_2 \ldotp \\ 
	(h, \Phi, S, \mathds{P}_1; \mathds{P}_2) \rightarrow^0 (h', \emptyset, S', \pskip; \mathds{P}_2)
	\Rightarrow
	(h, \Phi, S, \mathds{P}_1) \rightarrow^* (h', \emptyset, S', \pskip)
\end{array}
\]
We assume that $(h, \Phi, S, \mathds{P}_1; \mathds{P}_2) \rightarrow^0 (h', \emptyset, S', \pskip; \mathds{P}_2)$ holds. A zero-step reduction means that it must be the case that $\mathds{P}_1 = \pskip$. Therefore $(h, \Phi, S, \pskip; \mathds{P}_2) \rightarrow^0 (h', \emptyset, S', \pskip; \mathds{P}_2)$ for $h = h', \Phi = \emptyset, S = S'$. \\

\textit{Inductive case}: For some arbitrary $n > 0$

\textit{Inductive hypothesis}: Assume the property holds for some arbitrary program $\mathds{P}_1'; \mathds{P}_2$ and $n$ steps, and such that $(h, \Phi, S, \mathds{P}_1; \mathds{P}_2) \tred (h'', \Phi'', S'', \mathds{P}_1'; \mathds{P}_2)$.
\[
\begin{array}{l}
\forall h, h', S, S', \Phi, \mathds{P}_2 \ldotp \\ 
	(h, \Phi, S, \mathds{P}_1'; \mathds{P}_2) \rightarrow^n (h', \emptyset, S', \pskip; \mathds{P}_2)
	\Rightarrow
	(h, \Phi, S, \mathds{P}_1') \rightarrow^* (h', \emptyset, S', \pskip)
\end{array}
\]

\textit{To show}:
\[
\begin{array}{l}
\forall h, h', S, S', \Phi, \mathds{P}_2 \ldotp \\ 
	(h, \Phi, S, \mathds{P}_1; \mathds{P}_2) \rightarrow^{n+1} (h', \emptyset, S', \pskip; \mathds{P}_2)
	\Rightarrow
	(h, \Phi, S, \mathds{P}_1) \rightarrow^* (h', \emptyset, S', \pskip)
\end{array}
\]

We assume that $(h, \Phi, S, \mathds{P}_1; \mathds{P}_2) \rightarrow^{n+1} (h', \emptyset, S', \pskip; \mathds{P}_2)$ holds and we also know that in one step of reduction we get $(h, \Phi, S, \mathds{P}_1; \mathds{P}_2) \tred (h'', \Phi'', S'', \mathds{P}_1'; \mathds{P}_2)$. The latter is the conclusion of rule \textsc{PSeq}, so we know that if it happened then it must be the case that $(h, \Phi, S, \mathds{P}_1) \rightarrow (h'', \Phi'', S'', \mathds{P}_1')$ holds in the premiss. Now, by inductive hypothesis we can conclude that the property holds for $n + 1$ steps.

\end{proof}
}

\lem \label{ref:phiemp}
\[
\forall h, h', S, S', \Phi, \Phi', \mathds{P} \ldotp
(h, \Phi, S, \mathds{P}) \rightarrow^* (h', \Phi', S', \pskip) \implies \Phi' \equiv \emptyset
\]
