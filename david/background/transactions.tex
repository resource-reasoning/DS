\subsection{Transactions}

A database is a collection of data items that hold values. A database system (\textit{DBS}) is a tool comprised of software and hardware components that manages access to the underlying database. It allows its users to perform specific operations on the database items. Those can be summarized as being one of two kinds, reads and writes. In the following, we will use the notation \tread{x} to represent a read operation of the database item named \textit{x} while \twrite{x}{v} to depict a write operation which sets item \textit{x}'s value to \textit{v}. We will use the shorthand notation \twritee{x} to mean that some value is written to item $x$. Both the mentioned kind of operations are executed atomically \cite{ccontrol}. This means that the execution of a set of such operations appears to be sequential in terms of its constituents.

It is often the case that several operations are grouped together in order to make them behave as if they were a single unit of work. In such situations, database transactions are used: an ordered collection of operations. For example, considering a database for bank accounts, we consider a transfer of funds as a single operation from the user's point of view even if in reality the \textit{DBS} performs multiple sequential reads and writes to achieve the goal, as shown in Figure \ref{fig:transfer}. A transaction execution must provide guarantees on the well known \textbf{ACID} properties \cite{dbconcepts}, an acronym for:
\begin{enumerate}[ref=(\arabic*)]
\item \label{acid.a} Atomicity: all of the transaction's operations are executed or none of them; there is no possible partial execution of a transaction.
\item \label{acid.c} Consistency: an isolated execution of a transaction maintains the consistency constraints of the database. In our bank example, the property would be maintained when money is neither created nor destroyed as part of a transfer, but simply \textit{moved}.
\item \label{acid.i} Isolation: a transaction executes as if no other one is running in the \textit{DBS} at the same time. More generally, a given isolation level determines the correct behaviours of transactions that run concurrently. For example, serializability implies that any two transactions running in parallel believe that the other one has either finished before they started or it will start after they finished executing.
\item \label{acid.d} Durability: once a transaction has completed its work, all of its changes are kept in the database even if failures happen in the future.
\end{enumerate}

In order to maintain property \ref{acid.a}, a transaction needs to communicate the completion of its operations or its interruption in case of failure. This is done by introducing two new types of operation, namely commit and abort, which are notated \tcommit and \tabort respectively. In the case of an abort, the \textit{DBS} must roll-back the state of the database to what it was prior to the start of the transaction in order to undo the changes performed up to the failure point.

\begin{center}
\begin{tabular}{c@{\hskip 0.5in}c}
\begin{lstlisting}
BEGIN
	balance := R[a]
	W[a, balance-100]
	balance := R[b]
	W[b, balance+100]
	C
END
\end{lstlisting}
&
\begin{lstlisting}
BEGIN
	balance := R[a]
	W[a, balance-5000]
	A
END
\end{lstlisting}
\end{tabular}
\captionof{figure}{Two examples of transactions on the bank database. Note how the second one aborts due to a failure (potentially insufficient funds).}
\label{fig:transfer}
\end{center}

For the example in Figure \ref{fig:transfer} we would have the following transaction definition.
\begin{gather*}
\theta \equiv \left\lbrace \tmread{\texttt{a}}^1, \tmread{\texttt{b}}^2, \tmwritee{\texttt{a}}^3, \tmwritee{\texttt{b}}^4, \tcommit \right\rbrace
\\
\sqsubset \equiv \{ (\tmread{\texttt{a}}^1, \tmwritee{\texttt{a}}^3), (\tmread{\texttt{b}}^2, \tmwritee{\texttt{b}}^4), (\tmread{\texttt{a}}^1, \tcommit), (\tmread{\texttt{b}}^2, \tcommit), (\tmwritee{\texttt{a}}^3, \tcommit), (\tmwritee{\texttt{b}}^4, \tcommit) \}
\end{gather*}
This gives the \textit{DBS} the option to arbitrarily schedule the operations which are not ordered according to $\sqsubset$. In fact, under the mentioned example, one could invert the order of the two reads (or even run them in parallel) and still always get to the same final result. As we will see in section \ref{histories}, if a history is proven to be serializable, then we can guarantee valuable properties about it.

\subsection{Histories} \label{histories}

Modern database systems employ concurrency in order to run multiple transactions at the same time to achieve a higher performance and better operations throughput. On the downside, running operations in parallel on shared data items can lead to race conditions and a potential loss of two of the aforementioned ACID properties, consistency and isolation.

% In order to investigate the matter further, we will introduce some definitions which will prove useful.
The interleaving of operations which arises from a concurrent run of transactions, is referred to as a history. It records the order in which operations actually happen on the database and given that some of them could effectively execute in parallel, they are not totally ordered. A history $H$ is therefore a tuple $(\Theta, \pord)$ such that for a set of $n$ transactions $T_H = \left\lbrace T_i \right\rbrace_{i \in I}$, where $I = \{ i\ |\ 1 \leq i \leq n\}$, we have the operations set $\Theta \triangleq \bigcup_{i = 1}^{n} \theta_i$ and the partial order $\pord \subseteq \bigcup_{i = 1}^{n} \sqsubset_i$ \cite{ccontrol}. Let's also define two operations as conflicting if they both access the same data item and one of them is a write; formally we describe the property as $\pred{conflict}{op_\alpha, op_\beta} \Leftrightarrow \left( op_\alpha \in \{ \tread{x}, \twritee{x} \} \land op_\beta = \twritee{x} \right) \lor \left( op_\alpha = \twritee{x} \land op_\beta \in \{ \tread{x}, \twritee{x} \} \right)$. We then require that $\forall op_\alpha, op_\beta \ldotp (\pred{conflict}{op_\alpha, op_\beta} \land\left\lbrace op_\alpha, op_\beta \right\rbrace \subseteq \Theta) \Rightarrow (op_\alpha \pord op_\beta \lor op_\beta \pord op_\alpha)$. This means that all conflicting operations appearing in a history must be somehow ordered by $\pord$. When considering a history $H_{total}$ which is a total order over the operations of its transactions, we can refer to it simply by $\twritee{x}_i^1, $ $\twritee{y}_j^1, $ $\tcommit_i, $ $\tread{x}_j^2, $ $\twritee{x}_j^3, $ $\tcommit_j$ for example.

For every history $H = (\Theta, \pord)$ we define its committed transactions as the set $\pred{committed}{H} \triangleq \{ T_i\ |\ \tcommit_i \in \Theta \}$. We can now build its corresponding precedence graph \cite{dbconcepts} $G(H) = (V, E)$ where the set of vertices $V = \pred{committed}{T_H}$ includes all committed transactions in $H$ and the edges $E \triangleq \{ (T_i, T_j)\ |\ i \neq j \land \exists op_\alpha, op_\beta \ldotp op_\alpha \in \theta_i \land op_\beta \in \theta_j \land op_\alpha \pord op_\beta \}$.

\begin{center}
\begin{tikzpicture}[->, semithick]
\node (1) {$T_1$};
\node (2) [right = 0.5cm of 1] {$T_2$};
\node (3) [right = 0.5cm of 2] {$T_3$};

\path
(1) edge node {} (2)
(1) edge [bend left] node {} (3);
\end{tikzpicture}
\\
$H \equiv \tmread{x}_1^1, \tmread{x}_2^1, \tmread{y}_1^2, \tmwritee{x}_2^2, \tcommit_2, \tmwritee{y}_3^1, \tcommit_1, \tcommit_3$
\captionof{figure}{An example history with its corresponding precedence graph.}
\end{center}

Two histories $H_i$ and $H_j$ are equivalent \cite{ccontrol}, written $H_i \equiv H_j$, if and only if $\pred{committed}{H_i} \equiv \pred{committed}{H_j} \land \forall op_\alpha, op_\beta \ldotp (\pred{conflict}{op_\alpha, op_\beta} \land \left\lbrace op_\alpha, op_\beta \right\rbrace \subseteq \Theta_i) \Rightarrow (op_\alpha \pord_i op_\beta \Leftrightarrow op_\alpha \pord_j op_\beta)$. The latter condition states that any two conflicting operations in two histories must be ordered in the same way. This is necessary since the result of a concurrent execution of a set of transactions only depends upon the relative ordering of conflicting operations. We describe a history as serial if all of its constituent transactions are run one after the other, without an interleaving of their operations. Such histories guarantee the highest possible database isolation. It is possible to further state that any history $H$ is serializable if and only if it is equivalent to some serial history $H'$. The serializability theorem \cite{ccontrol} states that any history $H$, such that its precedence graph $G(H)$ contains no cycles, is serializable.

\subsection{Anomalies and Isolation Levels}

The \textsc{Ansi/Iso Sql-92} standard \cite{ansi92} defines a set of different transaction isolation levels based on partial histories they allow as part of the possible interleavings. We will call these sequences of actions phenomena. The original standard describes 3 of such phenomena, while \cite{isolationansi} adds an extra one \ref{ansi:0} for completeness.

\begin{enumerate}[label=(\textbf{P\arabic*})]\addtocounter{enumi}{-1}
\item \label{ansi:0} A \textbf{dirty write} appears when a transaction $T_i$ writes to a data item and, before it commits or aborts, another transaction $T_j$ writes to the same item. If any of $T_i$ and $T_j$ were to abort, it is not clear what value the item should contain.
\\
E.g. $H_{DW} \equiv \ldots \tmwritee{x}_i \ldots \tmwritee{x}_j \ldots$
\item We have the \textbf{dirty read} phenomenon in the case where transaction $T_i$ writes to an item, then, before it commits or aborts, another transaction $T_j$ reads that same data item. In fact if $T_i$ would abort after $T_j$ has already read, the latter would have retrieved a value for the item that never actually existed.
\\
E.g. $H_{DR} \equiv \ldots \twritee{x}_i \ldots \tread{x}_j \ldots$
\item A \textbf{non-repeatable read} happens when a transaction $T_i$ reads a data item which is later written to by another transaction $T_j$ that also commits. At this point, if $T_i$ is to read the same item again, it would either discover a new value or fail due to a removal.
\\
E.g. $H_{NR} \equiv \ldots \tread{x}_i \ldots \twritee{x}_j \ldots \tcommit_j \ldots$
\item The \textbf{phantom} phenomenon appears when transaction $T_i$ queries all data items satisfying a particular condition $\gamma$. Transaction $T_j$ then inserts some new items that satisfy $\gamma$. Now, if $T_i$ were to reproduce the initial query, it would see new items appear.
\\
E.g. $H_{P} \equiv \ldots \tmread{\gamma}_i \ldots \tmwritee{\gamma}_j \ldots \tmread{\gamma}_i \ldots$ here we abuse the transaction notation in the following way $\tmread{\gamma} \triangleq \left\lbrace x\ |\ x \in \pred{items}{db} \land \gamma(x) \right\rbrace$ and $\tmwritee{\gamma} \triangleq \exists x, v \ldotp \gamma(x) \land \tmwrite{x}{v}$.
\end{enumerate}

The listed definitions can be used to formulate the \textsc{Ansi} isolation levels as they appear in Table \ref{table:isolation}. The choice of how to concurrently run several transactions given by a $DBS$ allows clients to achieve a better performance or stronger consistency depending on how strict the isolation level is. In fact, allowing some of the phenomena can lead to failing constraint checks in the database.
\begin{center}
\captionof{table}{\textsc{Ansi} isolation levels and the phenomena they allow \cite{ansi92} \cite{isolationansi}.}
\def\arraystretch{1.4}
\begin{tabular}{|c|c|c|c|c|}
\hline
\textbf{Isolation Level} & \textbf{Dirty Write} & \textbf{Dirty Read} & \textbf{N-R Read} & \textbf{Phantom} \\
\hline
Read Uncommitted & $\times$ & \checkmark & \checkmark & \checkmark \\
\hline
Read Committed & $\times$ & $\times$ & \checkmark & \checkmark \\
\hline
Repeatable Read & $\times$ & $\times$ & $\times$ & \checkmark \\
\hline
Serializable & $\times$ & $\times$ & $\times$ & $\times$ \\
\hline
\end{tabular}
\label{table:isolation}
\end{center}

A concrete example of how the presence of the described phenomena can lead to a lack of correctness is shown in Table \ref{table:inconsistent} where transactions run at the read committed level. Let's assume that $T_i$ is executed by a bank's agent that wants to read what percentage of the overall funds are held by customer $a$ and $T_j$ is directly performed by $a$ as a withdrawal from an ATM machine. The former would first read the balance of $a$ then sum the funds of all customers (in this case $a$, $b$, $c$) and perform a simple division. On the other hand, $T_j$ will first read the cash availability, then subtract $\$100$ and provide the money to the customer. Given the isolation level selected, a non-repeatable read appears as part of history $H$ as $T_j$ removes the money after $T_i$ has read the initial balance for $a$ and before it has computed the overall sum. Therefore the end result $(\texttt{b}\div\texttt{total})$ will be incorrect and larger than the actual one. Moreover, this is a consequence to the fact that $H$ is not serializable. We can show this by first noticing that there are two possible serial histories $H_1 = \tmread{a}_1^1, \tmread{a}_1^2, \tmread{b}_1^3, \tmread{c}_1^4, \tcommit_1, \tmread{a}_2^1, \tmwritee{a}_2^2, \tcommit_2$ and $H_2 = \tmread{a}_2^1, \tmwritee{a}_2^2, \tcommit_2, \tmread{a}_1^1, \tmread{a}_1^2, \tmread{b}_1^3, \tmread{c}_1^4, \tcommit_1$. The conflicting operations are $(\tmread{a}_1^1, \tmwritee{a}_2^2), (\tmread{a}_1^2, \tmwritee{a}_2^2)$ and the order in which the operations appear in $H$ is $\tmread{a}_1^1 \pord_H$ $\tmwritee{a}_2^2 \pord_H \tmread{a}_1^2$ which is neither equivalent to the one in $H_1$, $\tmread{a}_1^1 \pord_{H_1} \tmread{a}_1^2 \pord_{H_1} \tmwritee{a}_2^2$, nor to the one in $H_2$, $\tmwritee{a}_2^2 \pord_{H_2} \tmread{a}_1^1 \pord_{H_2} \tmread{a}_1^2$. Therefore since $H$ is not equivalent to any of $H_1$ and $H_2$, it is clearly not serializable. A plausible solution for this kind of issue is to change the isolation level of the $DBS$ to repeatable read which forbids the presence of the culprit phenomenon.
\begin{center}
\captionof{table}{History $H$ showing a concurrent run of two transactions $T_i$ and $T_j$.}
\def\arraystretch{1.4}
\begin{tabular}{|c|c|c|}
\hline
\textbf{Time $t$} & $T_i$ & $T_j$ \\
\hline
0 & \texttt{b} = $\tmread{a}_i^1$ & \texttt{c} = $\tmread{a}_j^1$ \\
\hline
1 & - & $\tmwrite{a}{\texttt{c}-100}_j^2$ \\
\hline
2 & - & \tcommit \\
\hline
3 & \texttt{total} = $\tmread{a}_i^2$ + $\tmread{b}_i^3$ + $\tmread{c}_i^4$ & - \\
\hline
4 & \tcommit & - \\
\hline
\end{tabular}
\label{table:inconsistent}
\end{center}

\subsection{Locking Protocols}

The most popular mechanism to enforce a particular isolation level is to use some form of locking. We can consider each data item $x \in \pred{items}{db}$ to be associated with a lock that manages the access to its value. When a transaction runs an operation which accesses $x$, it is required to hold the lock relative to the item. The exact synchronization structure used in this case is a read/write lock which works under two modes, a shared one and an exclusive one. The former allows multiple read operations to happen in parallel while the latter makes sure that there is only one write operation executing at any point in time. This way blocking only happens when transactions perform concurrent conflicting operations on the same data item. In terms of notation, a transaction $T_i$ can lock an item using $\tlock{\kappa}{x}_i$ and unlock it with $\tunlock{\kappa}{x}_i$ where $\kappa \in \{ \tshared, \texclusive \}$ is the lock mode (shared or exclusive). If a transaction requests access to item $x$ and gets an immediate grant even if $x$'s lock is held by another transaction, we say that the two locks are compatible \cite{dbconcepts}. We define the property for lock modes as follows $\forall \kappa_i, \kappa_j \ldotp (\exists x \ldotp \tlock{\kappa_i}{x} \land \tlock{\kappa_j}{x}) \Rightarrow (\pred{compatible}{\kappa_i, \kappa_j} \Leftrightarrow (\kappa_i = \tshared \land \kappa_j = \tshared))$.

We can specify the usage of locks in a formal way by describing histories $H$ that arise from locking protocols. All accesses to items must be guarded by the corresponding locks, so $\forall x, i \ldotp (\tmwritee{x}_i \in \Theta_H \Rightarrow \tlock{\texclusive}{x}_i \pord \tmwritee{x}_i \pord \tunlock{\texclusive}{x}_i) \land (\tmread{x}_i \in \Theta_H \Rightarrow \exists \kappa \ldotp \tlock{\kappa}{x}_i \pord \tmread{x}_i \pord \tunlock{\kappa}{x}_i)$. Given that a transaction requesting an already acquired and non-compatible lock must wait for the counterpart to release it we will see that $\forall x, \kappa_a, \kappa_b, i, j \ldotp (i \neq j \land \lnot \pred{compatible}{\kappa_a, \kappa_b} \land \{ \tlock{\kappa_a}{x}_i, \tunlock{\kappa_a}{x}_i, \tlock{\kappa_b}{x}_j, \tunlock{\kappa_b}{x}_j \}\subseteq \Theta_H) \Rightarrow (\tlock{\kappa_a}{x}_i \pord \tunlock{\kappa_b}{x}_j \lor \tlock{\kappa_b}{x}_j \\ \pord \tunlock{\kappa_a}{x}_i)$.

Two-Phase-Locking (\tpl) is a concurrency control protocol which is vastly used in commercial $DBS$ products. In its base version, as the name suggests, each transaction $T_i$'s execution can be divided into two phases \cite{ccontrol}: a growing phase where $T_i$ sequentially acquires locks for all the data items it is going to access, and a shrinking phase that starts on the first lock release and will unlock all the items it holds, one after the other. According to \tpl, during the growing phase no locks are released, and once the shrinking phase has started, no locks are acquired. Formally, all histories that result from the use of \tpl\ will have an additional property in comparison to the ones already mentioned for locking protocols. In fact lock operations will be further ordered as to comply with the two phase separation $\forall x, y, \kappa_a, \kappa_b, i, j \ldotp (\{ \tlock{\kappa_a}{x}_i, \tunlock{\kappa_b}{y}_j \} \subset \Theta_H \Rightarrow \tlock{\kappa_a}{x}_i \pord \tunlock{\kappa_b}{y}_j)$.

A variation of the method is Conservative Two-Phase-Locking (\textsc{c2pl}) that works in a similar way with the key difference that all locks ever needed as part of a transaction are acquired before any read or write operation happens. As a consequence, we can state that all histories $H$ adhering to \textsc{c2pl} will be such that $\forall x, op, \kappa, i \ldotp (op \neq \textsf{L} \land \{ \tlock{\kappa}{x}_i, op \} \subseteq \Theta_H) \Rightarrow \tlock{\kappa}{x}_i \pord op$. This is done primarily in order to prevent situations in which \tpl\ would exhibit deadlocks.

The most used version of \tpl\ inside concrete implementations is Strict Two-Phase-Locking (\stpl) which imposes a different kind of constraint. Specifically, it only allows transactions to release the locks they hold after they have either committed or aborted. Its histories follow the rules listed for standard \tpl\ and moreover they guarantee that $\forall x, op, \kappa, i \ldotp (op \in \{ \tcommit_i, \tabort_i \} \land \{ \tunlock{\kappa}{x}_i, op \} \subset \Theta_H) \Rightarrow op \pord \tunlock{\kappa}{x}_i$. The Strong-Strict Two-Phase-Locking (\textsc{ss2pl}) protocol works by combining the two previous approaches in a way that all locking operations happen before any access to items and all unlocking operations appear after a transaction's commit or abort completion. We summarize the pecularities of the four variants of the protocol in Table \ref{table:2plvariants}.
\begin{center}
\captionof{table}{Two-Phase-Locking variants' constraints. Gradual locking refers to a transaction acquiring locks for cells as it needs to, while the extreme version makes it obtain all needed locks before starting. Similarly, gradual unlocking releases locks whenever they are no longer needed (while still following the two phases rule) and, on the other hand, extreme unlocking only releases locks after a transaction has committed.}
\def\arraystretch{1.4}
\begin{tabular}{|c|c|c|c|c|}
\hline
\textbf{Constraint} & \textsc{2pl} & \textsc{c2pl} & \textsc{s2pl} & \textsc{ss2pl} \\
\hline
Gradual Locking & \checkmark & $\times$ & \checkmark & $\times$ \\
\hline
Extreme Locking & $\times$ & \checkmark & $\times$ & \checkmark \\
\hline
Gradual Unlocking & \checkmark & \checkmark & $\times$ & $\times$ \\
\hline
Extreme Unlocking & $\times$ & $\times$ & \checkmark & \checkmark \\
\hline
\end{tabular}
\label{table:2plvariants}
\end{center}

\subsection{Serializability \& Transactional Models}

\label{sec:serTransMod}

\paragraph{Push/Pull Model}
The model of transactions described in \cite{pushPull} is presented as a general theory of serializability, that abstracts away all of the algorithm and implementation details, in order to find a compact set of operations used in the vast majority of cases. The model does not use a concrete machine state, i.e. a global storage or memory heap, to track the effect of transactions' operations such as writes. Instead, it assigns a local \textit{log} of operations to each transaction and considers a unique shared log, which records the history of all globally visible operations. The semantics allow transactions to \textsc{Push} or \textsc{Unpush}, in order to share their local effects with the global log or to recall them from it. Conversely, they can \textsc{Pull} an operation from the shared log into their local view, together with detangling from one through an \textsc{Unpull}. These two types of actions can be seen as reading or \textit{forgetting} a read from the global storage. The framework comes with a proof of serializability within the semantic model, meaning that once users map their algorithms to push \textit{Push/Pull} semantic rules, they obtain a proof of serializability as a consequence.