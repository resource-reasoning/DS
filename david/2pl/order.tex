\subsubsection{Strict total order}

Given that we want to be able to compare traces produced under the \tpl\ operational semantics, to the ones achieved from the \textsc{Atom} one, we need to establish a strict total order on the transactions that appear in a trace. This enables us to effectively simulate a serial reduction, as we know that, from an abstract point of view, there is a precise order of transaction where one \textit{happens before} another.

The serialization graph structure, which was formalised in Definition \ref{defn:sg}, implicitly gives us a relation on the transactions participating to a given trace through its edges. The latter is in fact a partial order relation on the transaction identifiers which represents the set of ordered conflicts inside of a trace. It is acyclic as shown in Theorem \ref{thm:sgAcyclic} and therefore a great starting point from which to build the total order we need.

\begin{defn}
	(Reflexive image).
	The reflexive image of a set $X$, written $\pred{Id}{X}$, is defined as:
	\[
		\pred{Id}{X} = \{ (x, x)\ |\ x \in X \}
	\]
\end{defn}

\begin{defn}
	(Reflexive closure).
	The reflexive closure of a given relation $R$ on a set $X$, written $R^\mathsf{id}$, is defined as:
	\[
		R^\mathsf{id} = R \cup \pred{Id}{X}
	\]
\end{defn}

\begin{defn}
	(\tpl\ Transactions order)
	\begin{align*}
		\text{let } (N, E) = \pred{SG}{\tau} \text{ in }
		\sqsubset_0 &= E^* \\
		\sqsubset_{n + 1} &=\ \sqsubset_n \cup \left( \sqsubset_n^\mathsf{id} ; \{ (i, j) \} ; \sqsubset_n^\mathsf{id} \right) \\
		&\text{where } i, j \in N \text{ and } i < j \\
		&\text{and } i \not\sqsubset_n j \text{ and } j \not\sqsubset_n i
	\end{align*}
\end{defn}

\begin{thm}
	\label{thm:totOrder}
	(Order of transactions).
	The $\sqsubset$ relation is a strict total order on the set of transactions $N$ in $(N, E) = \pred{SG}{\tau}, \tau = \pred{trace}{h, \emptyset, \emptyset, \mathds{P}}, \mathds{P} \in \mathsf{Prog}, h \in \mathsf{Storage}$.

	\begin{proof}
	In order to show the theorem, we are required to prove that for all $a, b, c \in N$:
	\begin{itemize}
		\item (Irreflexivity). $a \not\sqsubset a$
		\item (Asymmetry). If $a \sqsubset b$ then $b \not\sqsubset a$
		\item (Transitivity). If $a \sqsubset b$ and $b \sqsubset c$ then $a \sqsubset c$
		\item (Totality). $a \sqsubset b$ or $b \sqsubset a$ or $a = b$
	\end{itemize}
	
	Let's pick an arbitrary program $\mathds{P} \in \mathsf{Prog}$, initial storage $h \in \mathsf{Storage}$ and get a trace out of it $\tau = \pred{trace}{h, \emptyset, \emptyset, \mathds{P}}$. We now consider the incrementally built $\sqsubset$ relation on $N$, where $(N, E) = \pred{SG}{\tau}$. \\
	
	(Irreflexivity). The proof follows by induction on the number of $\sqsubset$ relation construction steps, $n$. Let's pick an arbitrary transaction identifier $a \in N$.
	
	{\parindent0pt
	\textit{Base case}: $n = 0$
	
	\textit{To show}: $a \not\sqsubset_0 a$
	
	By definition we know that $\sqsubset_0 = E^*$, i.e. the transitive closure on the edges of the serialization graph $\pred{SG}{\tau}$. We directly obtain that $a \not\sqsubset_0 a$ from Theorem \ref{thm:sgAcyclic}. \\
	
	\textit{Inductive case}: $n > 0$
	
	\textit{Inductive hypothesis}: $a \not\sqsubset_n a$
	
	\textit{To show}: $a \not\sqsubset_{n+1} a$
	
	Let's assume that $a \sqsubset_{n+1} a$ and by definition we know it means that, for some $i, j \in N$ such that $i < j, i \not\sqsubset_n j, j \not\sqsubset_n i$ we have:
	\[
		(a, a) \in \sqsubset_n \cup \left( \sqsubset_n^\mathsf{id} ; \{ (i, j) \} ; \sqsubset_n ^\mathsf{id} \right)
	\]
	and by I.H. we can rewrite it as $(a, a) \in \left( \sqsubset_n^\mathsf{id} ; \{ (i, j) \} ; \sqsubset_n ^\mathsf{id} \right)$ given we assumed that $\sqsubset_n$ is irreflexive. It follows that it must be the case that $(a, i)$ and $(j, a)$ are in $\sqsubset_n^\mathsf{id}$ and moreover they must be in $\sqsubset_n$ given that $i < j$ and therefore $i \neq j$. By transitivity of $\sqsubset_n$, there must be a $(j, i) \in \sqsubset_n$. By contradiction we state that $a \not\sqsubset_{n+1} a$. \\
	}
	
	(Asymmetry). The proof follows by induction on the number of $\sqsubset$ relation construction steps, $n$. Let's pick arbitrary transaction identifiers $a, b \in N$.
	
	{\parindent0pt
	\textit{Base case}: $n = 0$
	
	\textit{To show}: $a \sqsubset_0 b \implies b \not\sqsubset_0 a$
	
	By definition we know that $\sqsubset_0 = E^*$, i.e. the transitive closure on the edges of the serialization graph $\pred{SG}{\tau}$. Let's assume that $a \sqsubset_0 b$ meaning that $a \rightarrow^* b \in E$. We directly obtain that $b \not\sqsubset_0 a$ from Theorem \ref{thm:sgAcyclic}. \\
	
	\textit{Inductive case}: $n > 0$
	
	\textit{Inductive hypothesis}: $a \sqsubset_n b \implies b \not\sqsubset_n a$
	
	\textit{To show}: $a \sqsubset_{n + 1} b \implies b \not\sqsubset_{n + 1} a$
	
	Let's assume that $a \sqsubset_{n + 1} b$ and by definition we know it means that, for some $i, j \in N$ such that $i < j, i \not\sqsubset_n j, j \not\sqsubset_n i$ we have:
	\[
		(a, b) \in \sqsubset_n \cup \left( \sqsubset_n^\mathsf{id} ; \{ (i, j) \} ; \sqsubset_n ^\mathsf{id} \right)
	\]
	\begin{itemize}
		\item If $a \sqsubset_n b$ we know by I.H. that $b \not\sqsubset_n a$. Let's instead assume that $(b, a) \in \left( \sqsubset_n^\mathsf{id} ; \{ (i, j) \} ; \sqsubset_n ^\mathsf{id} \right)$ from which it follows that there is a $(b, i) \in \sqsubset_n^\mathsf{id}$ and $(j, a) \in \sqsubset_n^\mathsf{id}$. By transitivity of $\sqsubset_n$ we obtain that $(j, i) \in \sqsubset_n^\mathsf{id}$ and moreover that $j \sqsubset_n i$ since $i \neq j$ as $i < j$. By contradiction we obtain that $(b, a) \not\in \left( \sqsubset_n^\mathsf{id} ; \{ (i, j) \} ; \sqsubset_n ^\mathsf{id} \right)$. We conclude that $b \not\sqsubset_{n + 1} a$.
		\item If $(a, b) \in \left( \sqsubset_n^\mathsf{id} ; \{ (i, j) \} ; \sqsubset_n ^\mathsf{id} \right)$ which means there is a $(a, i) \in \sqsubset_n^\mathsf{id}$ and $(j, b) \in \sqsubset_n^\mathsf{id}$. Let's now assume that $(b, a) \in \left( \sqsubset_n^\mathsf{id} ; \{ (i, j) \} ; \sqsubset_n ^\mathsf{id} \right)$ meaning that there is a $(b, i) \in \sqsubset_n^\mathsf{id}$ and $(j, a) \in \sqsubset_n^\mathsf{id}$. By transitivity of $\sqsubset_n$ we obtain that $(j, i) \in \sqsubset_n^\mathsf{id}$ and moreover that $j \sqsubset_n i$ since $i \neq j$ as $i < j$. By contradiction we obtain that $(b, a) \not\in \left( \sqsubset_n^\mathsf{id} ; \{ (i, j) \} ; \sqsubset_n ^\mathsf{id} \right)$. We now assume that $b \sqsubset_n a$ which implies that $(b, a) \in \sqsubset_n^\mathsf{id}$. By transitivity of $\sqsubset_n$ we obtain that $(j, i) \in \sqsubset_n^\mathsf{id}$ and moreover that $j \sqsubset_n i$ since $i \neq j$ as $i < j$. By contradiction we obtain that $(b, a) \not\in \sqsubset_n$. We conclude that $b \not\sqsubset_{n + 1} a$.
	\end{itemize}
	}
	
	(Transitivity). We are required to show that $\forall m \geq 1 \ldotp \sqsubset^m\ \subseteq\ \sqsubset$ The proof follows by induction on the number of self-composition steps, $m$.
	
	{\parindent0pt
	\textit{Base case}: $m = 1$
	
	\textit{To show}: $\sqsubset^1\ \subseteq\ \sqsubset$ \\
	
	The result follows directly by definition $\sqsubset^1\ =\ \sqsubset\ \subseteq\ \sqsubset$. \\
	
	\textit{Inductive case}: $m > 1$
	
	\textit{Inductive hypothesis}: $\sqsubset^m\ \subseteq\ \sqsubset$
	
	\textit{To show}: $\sqsubset^{m + 1}\ \subseteq\ \sqsubset$
	\begin{align*}
		\sqsubset^{m + 1}
		&=\ \sqsubset^m ; \sqsubset \text{ by associativity} \\
		&\subseteq\ \sqsubset ; \sqsubset \text{ by I.H.} \\
		&=\ \sqsubset^2 \text{by definition} \\
		&\subseteq\ \sqsubset \text{ by Lemma \ref{lem:total2}}
	\end{align*}
	}
	
	(Totality). Let's pick arbitrary transaction identifiers $a, b \in N$ (\textsc{i}) for a finite $N$ and build the $\sqsubset$ relation on it until convergence, i.e. in a finite number of steps. If $(a, b) \in E^*$ or $(b, a) \in E*$ then we know that either $a \sqsubset b$ or $b \sqsubset a$ holds. On the other hand if there is no edge connecting $a$ to $b$ or $b$ to $a$ in $E^*$ (\textsc{ii}) then:
	\begin{itemize}
		\item If $a = b$ then by irreflexivity of $\sqsubset$ we are done, as totality is met.
		\item Without loss of generality, we say that $a < b$ (\textsc{iii}). Given that the construction of $\sqsubset$ terminated in some $m > 0$ steps (being $N$ a finite set), by (\textsc{i}), (\textsc{ii}) and (\textsc{iii}) we know that there must exist a construction step $n$ such that $0 < n < m$ where the tuple $(a, b)$ was inserted in the relation given that $\sqsubset_{n-1} \cup \left( \sqsubset_{n-1}^\mathsf{id} ; \{(a,b)\} ; \sqsubset_{n-1}^\mathsf{id} \right) \implies a \sqsubset_n b \implies a \sqsubset b$. 
	\end{itemize}
	\end{proof}
\end{thm}
	
\begin{lem}
	\label{lem:total2}
	Given a serialization graph $(N, E) = \pred{SG}{\tau}$ for $\tau = \pred{trace}{h, \emptyset, \emptyset, \mathds{P}}, \mathds{P} \in \mathsf{Prog}, h \in \mathsf{Storage}$, and the $\sqsubset$ relation on the set $N$ we say that $\sqsubset^2\ \subseteq\ \sqsubset$.
	
	{\parindent0pt
	\begin{proof}
	We proceed by induction on the number of $\sqsubset$ construction steps, $n$. \\
	
	\textit{Base case}: $n = 0$
	
	\textit{To show}: $\sqsubset_0^2\ \subseteq\ \sqsubset_0$
	
	By definition we know that $\sqsubset_0 = E^*$, i.e. the transitive closure on the edges of the serialization graph $\pred{SG}{\tau}$. It follows that by definition of transitive closure, $\sqsubset_0^2\ = E^* ; E^* = E^*$ meaning that $\sqsubset_0^2\ \subseteq\ \sqsubset_0$. \\
	
	\textit{Inductive case}: $n > 0$
	
	\textit{Inductive hypothesis}: $\sqsubset_n^2\ \subseteq\ \sqsubset_n$
	
	\textit{To show}: $\sqsubset_{n + 1}^2\ \subseteq\ \sqsubset_{n + 1}$
	
	We can rewrite the formula to be proven as the following, for some $i, j \in N$ such that $i < j, i \not\sqsubset_n j, j \not\sqsubset_n i$:
	\begin{align}
		\left( \sqsubset_n \cup \underbrace{\left( \sqsubset_n^\mathsf{id} ; \{ (i, j) \} ; \sqsubset_n^\mathsf{id} \right)}_{R} \right) ; \left( \sqsubset_n \cup \left( \sqsubset_n^\mathsf{id} ; \{ (i, j) \} ; \sqsubset_n^\mathsf{id} \right) \right) &\subseteq \sqsubset_{n + 1} \\
		\label{thm:total1} \sqsubset_n ; \sqsubset_n \cup \sqsubset_n ; R \cup R ; \sqsubset_n \cup R ; R &\subseteq\ \sqsubset_{n + 1} \text{ by distributivity}
	\end{align}
	It now suffices to show that each unioned set in the L.H.S. of (\ref{thm:total1}) is a subset of $\sqsubset_{n + 1}$ itself.
	\begin{itemize}
		\item \textit{To show}: $\sqsubset_n ; \sqsubset_n\ \subseteq\ \sqsubset_{n + 1}$
			\begin{align}
				\sqsubset_n ; \sqsubset_n\ &=\ \sqsubset_n^2 \\
				\text{by I.H.}&\subseteq\ \sqsubset_n\ \subseteq\ \sqsubset_{n + 1}
			\end{align}
		\item \textit{To show}: $\sqsubset_n ; \left( \sqsubset_n^\mathsf{id} ; \{ (i, j) \} ; \sqsubset_n^\mathsf{id} \right) \subseteq\ \sqsubset_{n + 1}$
			\begin{align}
				S  &=\ \sqsubset_n ; \sqsubset_n^\mathsf{id} \\
					&=\ \sqsubset_n ; \left( \sqsubset_n \cup\ \pred{Id}{N} \right) \text{by definition} \\
					&=\ \sqsubset_n ; \sqsubset_n \cup \sqsubset_n ; \pred{Id}{N} \\
					&=\ \sqsubset_n^2 \cup \sqsubset_n \\
					&\subseteq\ \sqsubset_n \text{by I.H.} \\
					\label{thm:total2} &\subseteq\ \sqsubset_n^\mathsf{id} \text{by definition} \\
				\sqsubset_n ; \left( \sqsubset_n^\mathsf{id} ; \{ (i, j) \} ; \sqsubset_n^\mathsf{id} \right) &= \left( \sqsubset_n ; \sqsubset_n^\mathsf{id} ; \{ (i, j) \} \right) ; \sqsubset_n^\mathsf{id} \text{ by associativity} \\
				&= \left( S ; \{ (i, j) \} \right) ; \sqsubset_n^\mathsf{id} \text{ by associativity} \\
				&\subseteq\ \sqsubset_n^\mathsf{id} ; \{ (i, j) \} ; \sqsubset_n^\mathsf{id} \text{by (\ref{thm:total2})} \\
				&\subseteq\ \sqsubset_{n + 1}
			\end{align}
		\item \textit{To show}: $\left( \sqsubset_n^\mathsf{id} ; \{ (i, j) \} ; \sqsubset_n^\mathsf{id} \right) ; \sqsubset_n \subseteq\ \sqsubset_{n + 1}$
			\begin{align}
				S' &=\ \sqsubset_n^\mathsf{id} ; \sqsubset_n \\
					&= \left( \sqsubset_n \cup\ \pred{Id}{N} \right) ; \sqsubset_n \text{by definition} \\
					&=\ \sqsubset_n ; \sqsubset_n \cup\ \pred{Id}{N} ; \sqsubset_n \\
					&=\ \sqsubset_n^2 \cup \sqsubset_n \\
					&\subseteq\ \sqsubset_n \text{by I.H.} \\
					\label{thm:total3} &\subseteq\ \sqsubset_n^\mathsf{id} \text{by definition} \\
				\left( \sqsubset_n^\mathsf{id} ; \{ (i, j) \} ; \sqsubset_n^\mathsf{id} \right) ; \sqsubset_n &=\ \sqsubset_n^\mathsf{id} ; \left( \{ (i, j) \} ; \sqsubset_n^\mathsf{id} ; \sqsubset_n \right) \text{ by associativity} \\
				&=\ \sqsubset_n^\mathsf{id} ; \left( \{ (i, j) \} ; S' \right) \text{ by associativity} \\
				&\subseteq\ \sqsubset_n^\mathsf{id} ; \{ (i, j) \} ; \sqsubset_n^\mathsf{id} \text{by (\ref{thm:total3})} \\
				&\subseteq\ \sqsubset_{n + 1}
			\end{align}
		\item \textit{To show}: $\left( \sqsubset_n^\mathsf{id} ; \{ (i, j) \} ; \sqsubset_n^\mathsf{id} \right) ; \left( \sqsubset_n^\mathsf{id} ; \{ (i, j) \} ; \sqsubset_n^\mathsf{id} \right) \subseteq\ \sqsubset_{n + 1}$
		
			Let's assume that $\left( \sqsubset_n^\mathsf{id} ; \{ (i, j) \} ; \sqsubset_n^\mathsf{id} \right) ; \left( \sqsubset_n^\mathsf{id} ; \{ (i, j) \} ; \sqsubset_n^\mathsf{id} \right) \neq \emptyset$ meaning that the set at least contains a tuple $(a, b)$, for $a, b \in N$. 
			\begin{align}
				(a, b) \in \left( \sqsubset_n^\mathsf{id} ; \{ (i, j) \} ; \sqsubset_n^\mathsf{id} \right) ; \left( \sqsubset_n^\mathsf{id} ; \{ (i, j) \} ; \sqsubset_n^\mathsf{id} \right)
					&\iff \\
				\exists c \ldotp (a, c) \in \left( \sqsubset_n^\mathsf{id} ; \{ (i, j) \} ; \sqsubset_n^\mathsf{id} \right) \land (c, b) \in \left( \sqsubset_n^\mathsf{id} ; \{ (i, j) \} ; \sqsubset_n^\mathsf{id} \right)
					&\iff \\
				\exists c \ldotp (a, i) \in\ \sqsubset_n^\mathsf{id} \land (i, j) \in  \{(i, j)\} \land (j, c) \in\ \sqsubset_n^\mathsf{id} \\ \land (c, i) \in\ \sqsubset_n^\mathsf{id} \land (i, j) \in  \{(i, j)\} \land (j, b) \in\ \sqsubset_n^\mathsf{id}
					&\implies \\
				\label{thm:total4} \exists c \ldotp (j, c) \in\ \sqsubset_n^\mathsf{id} \land (c, i) \in\ \sqsubset_n^\mathsf{id}
			\end{align}
			We know that $i \neq j$ since $i < j$ so the proof proceeds with a case-by-case analysis on $c$.
			\begin{itemize}
				\item If $c = i$ then we know that $c \neq j$ and by (\ref{thm:total4}) we have that $(j, i) \in\ \sqsubset_n^\mathsf{id} \land (i, i) \in\ \sqsubset_n^\mathsf{id}$ from which it follows that $(j, i) \in\ \sqsubset_n$, a contradiction.
				\item If $c = j$ then we know that $c \neq i$ and by (\ref{thm:total4}) we have that $(j, j) \in\ \sqsubset_n^\mathsf{id} \land (j, i) \in\ \sqsubset_n^\mathsf{id}$ from which it follows that $(j, i) \in\ \sqsubset_n$, a contradiction.
				\item If $c \neq i$ and $c \neq j$ then (\ref{thm:total4}) it follows that $(j, c) \in\ \sqsubset_n \land (c, i) \in\ \sqsubset_n$. By I.H. we know that $\sqsubset_n^2\ \subseteq\ \sqsubset_n$. By definition we know that if $(j, c) \in\ \sqsubset_n$ and $(c, i) \in\ \sqsubset_n$ then $(j, i) \in\ \sqsubset_n^2$. By I.H. this means that $(j, i) \in\ \sqsubset_n$ which is again a contradiction.
			\end{itemize}
			By contradiction we conclude that $\left( \sqsubset_n^\mathsf{id} ; \{ (i, j) \} ; \sqsubset_n^\mathsf{id} \right) ; \left( \sqsubset_n^\mathsf{id} ; \{ (i, j) \} ; \sqsubset_n^\mathsf{id} \right) = \emptyset$ meaning that $\left( \sqsubset_n^\mathsf{id} ; \{ (i, j) \} ; \sqsubset_n^\mathsf{id} \right) ; \left( \sqsubset_n^\mathsf{id} ; \{ (i, j) \} ; \sqsubset_n^\mathsf{id} \right) \subseteq\ \sqsubset_{n + 1}$.
			
			Direct proof.
			\begin{align*}
				\{ (i, j) \} ; \sqsubset_n^\mathsf{id} ; \{ (i, j) \}
					&=
				\{ (i, j) \} ; \left( \sqsubset_n \cup \pred{Id}{N} \right) ; \{ (i, j) \} \\
					&=
				\{ (i, j) \} ; \left( \sqsubset_n ; \{ (i, j) \} \cup \{ (i, j) \} \right) \\
					&=
				\left( \{ (i, j) \} ; \sqsubset_n ; \{ (i, j) \} \right) \cup \left(\{ (i, j) \} ; \{ (i, j) \} \right) \\
				 	&=
				 \left( \{ (i, j) \} ; \sqsubset_n ; \{ (i, j) \} \right) = \emptyset \\
				\left( \sqsubset_n^\mathsf{id} ; \{ (i, j) \} ; \sqsubset_n^\mathsf{id} \right) ; \left( \sqsubset_n^\mathsf{id} ; \{ (i, j) \} ; \sqsubset_n^\mathsf{id} \right)
					&=
				\left( \sqsubset_n^\mathsf{id} ; \{ (i, j) \} \right) ; \left( \sqsubset_n^\mathsf{id} ; \left( \sqsubset_n^\mathsf{id} ; \{ (i, j) \} ; \sqsubset_n^\mathsf{id} \right) \right) \\
					&=
				\left( \sqsubset_n^\mathsf{id} ; \{ (i, j) \} \right) ; \left( \left( \sqsubset_n^\mathsf{id} ;  \sqsubset_n^\mathsf{id} \right) ; \left( \{ (i, j) \} ; \sqsubset_n^\mathsf{id} \right) \right) \\
					&=
				\left( \sqsubset_n^\mathsf{id} ; \{ (i, j) \} \right) ; \left( \sqsubset_n^\mathsf{id} ; \left( \{ (i, j) \} ; \sqsubset_n^\mathsf{id} \right) \right) \\
					&=\
				\sqsubset_n^\mathsf{id} ; \left( \{ (i, j) \} ; \sqsubset_n^\mathsf{id} ; \{ (i, j) \} \right) ; \sqsubset_n^\mathsf{id} \\
					&=\
				\sqsubset_n^\mathsf{id} ; \emptyset ; \sqsubset_n^\mathsf{id} = \emptyset
			\end{align*}
	\end{itemize}
	\end{proof}
	}
\end{lem}