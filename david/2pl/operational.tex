\subsection{Operational Semantics}

The formal behaviour of transactional programs that run under \tpl\ is expressed through operational semantics.

\begin{defn}
	(Action labels).
	Atomic actions performed by transactions are represented using transition \emph{action labels} from the set \textsf{Act}, ranged over by $\alpha, \alpha_1, \ldots, \alpha_n$ and defined with the following grammar:
	\begin{align*}
		\alpha \in \mathsf{Act} ::=
		\ &\actprog\
		|\ \actid{\iota}\
		|\ \actalloc{\iota}{n}{l}\
		|\ \actread{\iota}{k}{v}\
		|\ \actwrite{\iota}{k}{v}\\
		|\ &\actlock{\iota}{k}{\kappa}\
		|\ \actunlock{\iota}{k}
	\end{align*}
	where $k, l \in \mathsf{Key}, n \in \mathds{N}, v \in \mathsf{Val}, \kappa \in \mathsf{Lock}, \iota \in \mathsf{Tid}$.
\end{defn}

System transitions are labelled with $\actprog$ and describe program-level execution. For this reason they are not part of any transaction. Control flow actions that happen at the level of transactions are labelled through $\actid{\iota}$, where $\iota$ is the identifier of the transaction performing the action. A transaction $\iota$ reads a particular value $v$ from cell indexed with $k$ via $\actread{\iota}{k}{v}$ while similarly it can write a value $v$ to $k$ through the $\actwrite{\iota}{k}{v}$.