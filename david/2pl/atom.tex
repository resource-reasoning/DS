\subsubsection{Proof}

\begin{thm}

\label{thm:atom}

\[
	\forall h, h', S, S' \ldotp
	(h, \emptyset, S, \mathds{P}) \rightarrow^* (h', \emptyset, S', \pskip) \implies 
	(h, \mathds{P}) \tred^* (h', \pskip)
\]

{\parindent0pt
\begin{proof}
The proof is done by induction on the structure of programs $\mathsf{Prog}$. \\

\textit{Base case 1}: $\pskip \in \mathsf{Prog}$

\textit{To show}:
\[
	\forall h, h', S, S' \ldotp
	(h, \emptyset, S, \pskip) \rightarrow^* (h', \emptyset, S', \pskip) \implies 
	(h, \pskip) \tred^* (h', \pskip)
\]

For arbitrary $h, h', S, S'$ we assume that $(h, \emptyset, S, \pskip) \rightarrow^* (h', \emptyset, S', \pskip)$ holds, and given that $\pskip$ has no possible one-step reductions, it must be the case that it is a zero-step reduction. Therefore we have $h = h'$ and $S = S'$. Starting from $(h, \pskip)$ through the $\tred$ relation, we can always reach $(h, \pskip)$ via a zero-step reduction $(h, \pskip) \tred^0 (h, \pskip)$. We can conclude that $(h, \emptyset, S, \pskip) \rightarrow^* (h', \emptyset, S', \pskip) \implies (h, \pskip) \tred^* (h', \pskip)$ where $h = h'$. \\

\textit{Base case 2}: $\mathds{T} \in \mathsf{Prog}$

\textit{To show}:
\[
	\forall h, h', S, S' \ldotp
	(h, \emptyset, S, \mathds{T}) \rightarrow^* (h', \emptyset, S', \pskip) \implies 
	(h, \mathds{T}) \tred^* (h', \pskip)
\]

We will proceed with the proof by induction on the structure of transactions $\mathsf{Trans}$. Given that the \textsc{Atom} semantics only support user transactions, all that is required to show is:
\begin{gather*}
	\forall h, h', S, S' \ldotp \\
	(h, \emptyset, S, \ptdef{\mathds{C}}) \rightarrow^* (h', \emptyset, S', \pskip) \implies 
	(h, \ptdef{\mathds{C}}) \tred^* (h', \pskip)
\end{gather*}

For arbitrary $h, h', S, S'$ we assume that $(h, \emptyset, S, \ptdef{\mathds{C}}) \rightarrow^* (h', \emptyset, S', \pskip)$ holds. Given the overall reduction from $\ptdef{\mathds{C}}$ to $\pskip$ it must be the case that the following holds.
\begin{gather*}
	(h, \emptyset, S, \ptdef{\mathds{C}})
	\xrightarrow{\actid{\iota}} (h, \emptyset, S[\iota \mapsto (\emptyset, \pgrow)], \ptdef{\mathds{C}}_\iota) \\
	\rightarrow^* (h', \emptyset, S', \ptdef{\pskip}_\iota)
	\xrightarrow{\actprog} (h', \emptyset, S', \pskip)
\end{gather*}
Which implies that $\mathds{C}$ reduces to $\pskip$ through the repeated use of the \textsc{Exec} rule. From the transitive closure of the $\rightarrow$ relation and Lemma \ref{lem:catom} we obtain the result that $(h, \ptdef{\mathds{C}}) \tred^* (h', \pskip)$. \\

\textit{Inductive case 1}: $\mathds{P}_1 + \mathds{P}_2 \in \mathsf{Prog}$

\textit{To show}:
\[
	\forall h, h', S, S' \ldotp
	(h, \emptyset, S, \mathds{P}_1 + \mathds{P}_2) \rightarrow^* (h', \emptyset, S', \pskip) \implies 
	(h, \mathds{P}_1 + \mathds{P}_2 ) \tred^* (h', \pskip)
\]

\textit{Inductive hypothesis}:
\begin{gather*}
	\forall h, h', S, S' \ldotp
	(h, \emptyset, S, \mathds{P}_1) \rightarrow^* (h', \emptyset, S', \pskip) \implies 
	(h, \mathds{P}_1) \tred^* (h', \pskip)
	\\ \land \\
	\forall h, h', S, S' \ldotp
	(h, \emptyset, S, \mathds{P}_2) \rightarrow^* (h', \emptyset, S', \pskip) \implies 
	(h, \mathds{P}_2) \tred^* (h', \pskip)
\end{gather*}

For arbitrary $h, h', S, S'$ we assume that $(h, \emptyset, S, \mathds{P}_1 + \mathds{P}_2) \rightarrow^* (h', \emptyset, S', \pskip)$ holds. Now we are presented with two cases:
\begin{enumerate}
	\item We can reduce $(h, \emptyset, S, \mathds{P}_1 + \mathds{P}_2) \xrightarrow{\actprog} (h, \emptyset, S, \mathds{P}_1)$ with one step through the \textsc{ChoiceL} rule, which we can always apply since it has an empty premiss. We can also always reduce $(h, \mathds{P}_1 + \mathds{P}_2) \tred (h, \mathds{P}_1)$ through the rule \textsc{AtChoiceL} given it has an empty premiss. By inductive hypothesis on $\mathds{P}_1$ we obtain that $(h, \emptyset, S, \mathds{P}_1) \rightarrow^* (h', \emptyset, S', \pskip) \implies (h, \mathds{P}_1) \tred^* (h', \pskip)$. Therefore we can conclude that $(h, \emptyset, S, \mathds{P}_1 + \mathds{P}_2) \rightarrow^* (h', \emptyset, S', \pskip) \implies  (h, \mathds{P}_1 + \mathds{P}_2) \tred^* (h', \pskip)$.
	
	\item We can reduce $(h, \emptyset, S, \mathds{P}_1 + \mathds{P}_2) \xrightarrow{\actprog} (h, \emptyset, S, \mathds{P}_2)$ with one step through the \textsc{ChoiceR} rule, which we can always apply since it has an empty premiss. We can also always reduce $(h, \mathds{P}_1 + \mathds{P}_2) \tred (h, \mathds{P}_2)$ through the rule \textsc{AtChoiceR} given it has an empty premiss. By inductive hypothesis on $\mathds{P}_2$ we obtain that $(h, \emptyset, S, \mathds{P}_2) \rightarrow^* (h', \emptyset, S', \pskip) \implies (h, \mathds{P}_2) \tred^* (h', \pskip)$. Therefore we can conclude that $(h, \emptyset, S, \mathds{P}_1 + \mathds{P}_2) \rightarrow^* (h', \emptyset, S', \pskip) \implies  (h, \mathds{P}_1 + \mathds{P}_2) \tred^* (h', \pskip)$. \\
\end{enumerate}

\textit{Inductive case 2}: $\mathds{P}_1 ; \mathds{P}_2 \in \mathsf{Prog}$

\textit{To show}:
\[
	\forall h, h', S, S' \ldotp
	(h, \emptyset, S, \mathds{P}_1 ; \mathds{P}_2) \rightarrow^* (h', \emptyset, S', \pskip) \implies 
	(h, \mathds{P}_1 ; \mathds{P}_2 ) \tred^* (h', \pskip)
\]

\textit{Inductive hypothesis}:
\begin{gather*}
	\forall h, h', S, S' \ldotp
	(h, \emptyset, S, \mathds{P}_1) \rightarrow^* (h', \emptyset, S', \pskip) \implies 
	(h, \mathds{P}_1) \tred^* (h', \pskip)
	\\ \land \\
	\forall h, h', S, S' \ldotp
	(h, \emptyset, S, \mathds{P}_2) \rightarrow^* (h', \emptyset, S', \pskip) \implies 
	(h, \mathds{P}_2) \tred^* (h', \pskip)
\end{gather*}

For arbitrary $h, h', S, S'$ we assume that $(h, \emptyset, S, \mathds{P}_1 ; \mathds{P}_2) \rightarrow^* (h', \emptyset, S', \pskip)$ holds. Given the overall reduction from $\mathds{P}_1 ; \mathds{P}_2$ to $\pskip$ we must have a chain of reductions of the following shape, for some $h'', S''$: \iffalse where $\Phi'' = \emptyset$ by Lemma \ref{ref:phiemp}. \fi
\[
	\underbrace{(h, \emptyset, S, \mathds{P}_1 ; \mathds{P}_2) \rightarrow^* (h'', \emptyset, S'', \pskip; \mathds{P}_2)}_{(\textsc{i})}
	\xrightarrow{\actprog} (h'', \emptyset, S'', \mathds{P}_2) \rightarrow^* (h', \emptyset, S', \pskip)
\]
\begin{enumerate}
	\item \label{seq:1} By (\textsc{i}) and Lemma \ref{ref:2seq} we get that $(h, \emptyset, S, \mathds{P}_1) \rightarrow^* (h'', \emptyset, S'', \pskip)$ holds.
	
	\item \label{seq:2} By \ref{seq:1}. and the inductive hypothesis on $\mathds{P}_1$ we obtain that $(h, \mathds{P}_1) \tred^* (h'', \pskip)$.
	
	\item By \ref{seq:2}. and Lemma \ref{ref:aseq} we get that $(h, \mathds{P}_1 ; \mathds{P}_2) \tred^* (h'', \pskip ; \mathds{P}_2)$.
\end{enumerate}

At this point we can apply the \textsc{PSeqSkip} rule to reduce $(h'', \emptyset, S'', \pskip ; \mathds{P}_2) \xrightarrow{\actprog} (h'', \emptyset, S'', \mathds{P}_2)$ and rule \textsc{AtPSeqSkip} to reduce $(h'', \pskip ; \mathds{P}_2) \tred^* (h'', \mathds{P}_2)$. By inductive hypothesis on $\mathds{P}_2$ we can conclude that $(h, \emptyset, S, \mathds{P}_1 ; \mathds{P}_2) \rightarrow^* (h', \emptyset, S', \pskip) \implies (h, \mathds{P}_1 ; \mathds{P}_2) \tred^* (h', \pskip)$. \\

\textit{Inductive case 3}: $\mathds{P}^* \in \mathsf{Prog}$

\textit{To show}:
\[
	\forall h, h', S, S' \ldotp
	(h, \emptyset, S, \mathds{P}^*) \rightarrow^* (h', \emptyset, S', \pskip) \implies 
	(h, \mathds{P}^* ) \tred^* (h', \pskip)
\]

\textit{Inductive hypothesis}:
\[
	\forall h, h', S, S' \ldotp
	(h, \emptyset, S, \mathds{P}) \rightarrow^* (h', \emptyset, S', \pskip) \implies 
	(h, \mathds{P}) \tred^* (h', \pskip)
\]

For arbitrary $h, h', S, S'$ we assume that $(h, \emptyset, S, \mathds{P}^*) \rightarrow^* (h', \emptyset, S', \pskip)$ holds (\textsc{i}). Given the overall reduction from $\mathds{P}^*$ to $\pskip$ we must have a chain of reductions of the following shape, for some $h'', \Phi'', S''$:
\[
(h, \emptyset, S, \mathds{P}^*) \rightarrow^* (h'', \Phi'', S'', \pskip + (\mathds{P} ; \mathds{P}^*)) \rightarrow^*  (h', \emptyset, S', \pskip + (\mathds{P} ; \mathds{P}^*)) \xrightarrow{\actprog} (h', \emptyset, S', \pskip)
\]
Through the \textsc{Loop} rule (\textsc{ii}), we can always reduce $(h, \emptyset, S, \mathds{P}^*) \xrightarrow{\actprog} (h, \emptyset, S, \pskip + (\mathds{P} ; \mathds{P}^*))$ given that it has an empty premiss. Similarly, we can always reduce any $(h'', \mathds{P}^*) \tred (h'', \pskip + (\mathds{P} ; \mathds{P}^*))$ via the \textsc{AtLoop} rule as it also has an empty premiss. We now consider two possible cases:
\begin{enumerate}
	\item \label{loop:1} We reduce $(h, \emptyset, S, \pskip + (\mathds{P} ; \mathds{P}^*)) \xrightarrow{\actprog} (h, \emptyset, S, \pskip)$ through the \textsc{ChoiceL} rule which we can always do, together with \textsc{AtChoiceL} that reduces $(h, \pskip + (\mathds{P} ; \mathds{P}^*)) \tred (h, \pskip)$. In this scenario we directly obtain the result.
	
	\item We reduce $(h, \emptyset, S, \pskip + (\mathds{P} ; \mathds{P}^*)) \xrightarrow{\actprog} (h, \emptyset, S, \mathds{P} ; \mathds{P}^*)$ through the \textsc{ChoiceR} rule which we can always do, together with \textsc{AtChoiceR} that reduces $(h, \pskip + (\mathds{P} ; \mathds{P}^*)) \tred (h, \mathds{P} ; \mathds{P}^*)$. By Lemma \ref{ref:2seq}, Lemma \ref{ref:aseq} and the inductive hypothesis on $\mathds{P}$ we get that $(h, \emptyset, S, \mathds{P} ; \mathds{P}^*) \rightarrow^* (h'', \emptyset, S'', \pskip ; \mathds{P}^*) \implies (h, \mathds{P} ; \mathds{P}^*) \tred^* (h'', \pskip ; \mathds{P}^*)$. It is now possible to further reduce $(h'', \emptyset, S'', \pskip ; \mathds{P}^*) \xrightarrow{\actprog} (h'', \emptyset, S'', \mathds{P}^*)$ via \textsc{PSeqSkip} and $(h'', \pskip ; \mathds{P}^*) \tred (h'', \mathds{P}^*)$ through \textsc{AtPSeqSkip}. We are now in the position to repeat the process from point (\textsc{ii}) until case \ref{loop:1} is encountered at which point we have a reduction to $\pskip$. This will eventually happen given our initial assumption (\textsc{i}). \\
\end{enumerate}

\textit{Inductive case 4}: $\mathds{P}_1 \| \mathds{P}_2 \in \mathsf{Prog}$

\textit{To show}:
\[
	\forall h, h', S, S' \ldotp
	(h, \emptyset, S, \mathds{P}_1 \| \mathds{P}_2) \rightarrow^* (h', \emptyset, S', \pskip) \implies 
	(h, \mathds{P}_1 \| \mathds{P}_2 ) \tred^* (h', \pskip)
\]

We will prove the parallel composition case by mathematical induction on the number of reduction steps $n$.

\textit{Base case}: $n = 0$

\textit{To show}:
\begin{gather*}
	\forall h, h', S, S' \ldotp \\
	(h, \emptyset, S, \mathds{P}_1 \| \mathds{P}_2) \rightarrow^0 (h', \emptyset, S', \pskip)
	\implies
	(h, \mathds{P}_1 \| \mathds{P}_2) \tred^* (h', \pskip)
\end{gather*}

For arbitrary $h, h', S, S'$ we assume that $(h, \emptyset, S, \mathds{P}_1 \| \mathds{P}_2) \rightarrow^0 (h', \emptyset, S', \pskip)$ holds. Since a zero-step reduction happened, it must be the case that $\mathds{P}_1 = \mathds{P}_2 = \pskip$, $h' = h$ and $\pskip \| \pskip$ reduced to $\pskip$ through the \textsc{ParEnd} rule. Now, we immediately obtain that $(h, \pskip \| \pskip) \tred (h, \pskip)$ from rule \textsc{AtParEnd}. \\

\textit{Inductive case 4.1}: $n > 0$

\textit{Inductive hypothesis}:
\begin{gather*}
	\forall m \leq n, h, h', S, S', \mathds{P}_1, \mathds{P}_2 \ldotp \\
	(h, \emptyset, S, \mathds{P}_1 \| \mathds{P}_2) \rightarrow^m (h', \emptyset, S', \pskip)
	\implies
	(h, \mathds{P}_1 \| \mathds{P}_2) \tred^* (h', \pskip)
\end{gather*}

\textit{To show}:
\begin{gather*}
	\forall h, h', \Phi, S, S' \ldotp \\
	(h, \emptyset, S, \mathds{P}_1 \| \mathds{P}_2) \rightarrow^{n+1} (h', \emptyset, S', \pskip)
	\implies
	(h, \mathds{P}_1 \| \mathds{P}_2) \tred^* (h', \pskip)
\end{gather*}

\newpage 
For arbitrary $h, h', S, S'$ we assume that $(h, \emptyset, S, \mathds{P}_1 \| \mathds{P}_2) \rightarrow^{n+1} (h', \emptyset, S', \pskip)$ holds. As a consequence, from the definition of $\mathsf{trace}$ and $\mathsf{tgen}$ we can state there is a trace $\tau \in [\mathsf{Act} \times \mathds{N}]$ of length $n + 1$ such that:
\begin{gather}
	\label{thm:atom1}
	\tau = \pred{trace}{h, \emptyset, S, \mathds{P}_1 \| \mathds{P}_2} \land \pred{tgen}{\tau, h, h', \emptyset, S, \mathds{P}_1 \| \mathds{P}_2}
\end{gather}
By repeatedly applying Lemma \ref{lem:lockAbsent} until convergence, we obtain a trace $\tau_c \in [\mathsf{Act} \times \mathds{N}]$ such that $\tau_c$ does not contain spurious lock and unlock operations, i.e. $\pred{clean}{\tau_c}$ holds, and for which the following is true, from (\ref{thm:atom1}).
\begin{gather}
	\label{thm:atom2} \pred{tgen}{\tau_c, h, h', \emptyset, S, \mathds{P}_1 \| \mathds{P}_2}
\end{gather}

From Theorem \ref{thm:totOrder} we are always able to build the strict total order $\sqsubset$ on the set of transactions $N$ that appear in $\tau_c$, for which $(N, E) = \pred{SG}{\tau}$.

From the definition of strict total order, we know we can find the minimal (or first) element $\iota$ of $\sqsubset$ such that:
\begin{gather}
	\label{thm:atom3} \forall j \in N \ldotp \iota \neq j \implies \iota \sqsubset j
\end{gather}
At this point we repeatedly apply Lemma \ref{lem:rr}, Lemma \ref{lem:rwlu}, Lemma \ref{lem:aa}, Lemma \ref{lem:ax} until convergence in order to swap and move to the left all operations performed by transaction $\iota$ inside $\tau_c$. Once no more swap is possible, i.e. no aforementioned lemmata can be applied, we obtain a trace $\tau_{seq}$ for which, from (\ref{thm:atom2}) and the fact that $\tau_c$ is clean, we state the following:
\begin{gather}
	\label{thm:atom4} \pred{tgen}{\tau_{seq}, h, h', \emptyset, S, \mathds{P}_1 \| \mathds{P}_2} \\
		\land\
	\label{thm:atom7} \pred{clean}{\tau_{seq}}
\end{gather}
We now claim that all of $\iota$'s operations appear in $\tau_{seq}$ before the actions performed by any other transaction $j$ in $N$. Let's instead assume that there is a transaction $j$ which has an action in $\tau_{seq}$ happening before one done by $\iota$, formally:
\begin{gather}
	\label{thm:atom6}
	\begin{array}{c}
		\exists x, y, n, n', j \ldotp
		\iota \neq j
		\land x = (\alpha(j), n)
		\land y = (\alpha(\iota), n')
		\land \tau_{seq} \vDash x < y
	\end{array}
\end{gather}
Let's now look at one particular consecutive sequence of labelled reductions as part of $\tau_{seq}$ that follows from (\ref{thm:atom4}) and the definition of $\mathsf{tgen}$, that arises as a consequence of (\ref{thm:atom6}). For $j \in \mathsf{Tid}$ we have $\alpha = \alpha(j)$ and $\alpha' = \alpha(\iota)$ being two consecutive actions in $\tau_{seq}$ performed by $j$ and $\iota$ respectively: %such that $\tau \not\vDash \alpha(\iota) < \alpha'$.
\begin{gather*}
	(h, \emptyset, S, \mathds{P}_1 \| \mathds{P}_2)
		\rightarrow^*
	(h_a, \Phi_a, S_a, \mathds{P}_a)
		\xrightarrow{\alpha}
	(h_b, \Phi_b, S_b, \mathds{P}_b)
		\xrightarrow{\alpha'}
	(h_c, \Phi_c, S_c, \mathds{P}_c)
		\rightarrow^*
	(h', \emptyset, S', \pskip)
\end{gather*}
We proceed with a case-by-case analysis on $\alpha$ and $\alpha'$ to determine that in all feasible situations we either have a contradiction or we could have swapped the two actions to have $\alpha'$ occurring right before $\alpha$. In this context, we say \textit{feasible} to mean that the consecutive reductions labelled with $\alpha$ and $\alpha'$ can be actually generated by the operational semantics. In the cases that follow we use $k, k' \in \mathsf{Key}, \kappa, \kappa' \in \mathsf{Lock}, I \in \mathcal{P}(\mathsf{Tid}), v, v' \in \mathsf{Val}$. Also, we only focus on $k = k'$ as in the case where $k \neq k'$ all action types can always be swapped.
\begin{enumerate}[label=({\roman*})]
	\item If $\alpha = \actunlock{j}{k}$ and $\alpha' = \actlock{\iota}{k}{\kappa}$ for $\kappa = \textsc{x}$, then from (\ref{thm:atom7}) we know that $\tau_{seq}$ does not contain any spurious locks which implies that $\iota$ is obtaining the exclusive lock on $k$ in order to later read or write to it. Given that $j$ is releasing a lock on $k$, it means that it either read or wrote to it beforehand through action $\alpha_c$ (since again $\pred{clean}{\tau_{seq}}$ holds from (\ref{thm:atom7}))
	\begin{enumerate}
		\item \label{thm:atom8a} If $\iota$ later writes to $k$ through $\alpha'' = \actwrite{\iota}{k}{v}$ then, from the definition of $\mathsf{conflict}$ it follows that the actions $\alpha_c$ and $\alpha''$ must be conflicting. From the definition of $\mathsf{SG(\tau_{seq})}$ and the fact that the strict total order $\sqsubset$ keeps the serialization graph's edges, it must be the case that $j \sqsubset \iota$ which is not possible due to the fact that $\iota$ is the minimal element of the $\sqsubset$ relation from (\ref{thm:atom3}) and we get a contradiction.
		
		\item If $\iota$ later only reads storage cell $k$ and if $j$ wrote to cell $k$ then we are back to the previous case as we have a conflict. Otherwise, if $j$ also only read from $k$ then we can convert $\alpha$ into $\actlock{j}{k}{\textsc{s}}$ and $\alpha'$ into $\actlock{\iota}{k}{\textsc{s}}$ since the exclusive lock is never used. Then we can swap $\alpha'$ in the place of $\alpha$.
	\end{enumerate}

	\item If one of $\alpha$ or $\alpha'$ is an $\mathsf{id}$ action and the other one is any type of action, then we can trivially swap the two. This is because, from the semantic interpretation of $\mathsf{id}$, we know that the transition has no effect on the global state $h, \Phi$.
	
	\item If $\alpha = \actread{j}{k}{v}$ and $\alpha' = \actread{\iota}{k}{v'}$ then we can always swap $\alpha'$ with $\alpha$ to move $\iota$'s operation to the left of $\tau_{seq}$. This follows since, given that the two reductions were successful, $\Phi_a$ must be such that $\Phi_a(k) = (\{\iota, j\} \uplus I, \textsc{s})$.
	
	\item $\alpha = \actalloc{\iota}{n}{l}$ and $\alpha' = \actalloc{\iota}{n'}{l'}$ then we have $\{l, \ldots, l + n - 1\} \cap \{l', \ldots, l' + n - 1\} \equiv \emptyset$  which means that we can always swap $\alpha'$ in the place of $\alpha$.
	
	\item If $\alpha = \actlock{j}{k}{\kappa}$ and $\alpha' = \actlock{\iota}{k}{\kappa'}$ then, when one of $\kappa$ or $\kappa'$ is \textsc{x} then the operational semantics do not allow the reductions. When $\kappa = \kappa' = \textsc{s}$ then we can always swap $\alpha'$ in the place of $\alpha$.
	
	\item If $\alpha = \actunlock{j}{k}$ and $\alpha' = \actunlock{\iota}{k}$ and $(\{j\}, \textsc{x}) = \Phi_a(k)$, then there is no way action $\alpha'$ could have reduced from $(h_b, \Phi_b, S_b, \mathds{P}_b)$ since from the semantic interpretation of $\mathsf{unlock}$ we obtain $\Phi_b = \Phi_a[k \mapsto (\emptyset, \textsc{u})]$.
	
	\item If $\alpha = \actlock{j}{k}{\kappa}$ and $\alpha' = \actunlock{\iota}{k}$ for $\kappa = \textsc{x}$, then there is no possible way for action $\alpha'$ to succesfully reduce from $(h_b, \Phi_b, S_b, \mathds{P}_b)$, since $j$ acquires the lock on $k$ through $\alpha$ before $\iota$ unlocks it making $(\{j\}, \textsc{x}) = \Phi_b(k)$ and $\iota \neq j$ by assumption.
	
	\item If $\alpha = \actlock{j}{k}{\kappa}$ and $\alpha' = \alpha(\iota, k)$ and $\kappa = \textsc{x}$ then in no way $\alpha'$ could have reduced with $\Phi_b = \Phi_a[k \mapsto (\{j\}, \textsc{x})]$ and $\iota \neq j$. In the case where $\kappa = \textsc{s}$ and $\alpha'$ is a $\mathsf{write}$ we have a similar problem, as $\Phi_b = \Phi_a[k \mapsto (\{j\} \uplus I, \textsc{s})]$ for an $I \in \mathcal{P}(\mathsf{Tid})$.
	
	\item If $\alpha = \actunlock{j}{k}$ and $\alpha' = \alpha(\iota, k)$ and $\alpha'$ is a $\mathsf{write}$, then from (\ref{thm:atom7}) we know that there must be a read or write operation $\alpha_c$ performed by transaction $j$ before $\alpha$. It follows that $\alpha_c$ is conflicting with $\alpha'$ and this leads back to case \ref{thm:atom8}.
	
	\item If $\alpha = \alpha(j, k)$ and $\alpha' = \actlock{\iota}{k}{\kappa}$ and ($\kappa = \textsc{x}$ or $\kappa = \textsc{s}$ and $\alpha$ is a $\mathsf{write}$ action), then from (\ref{thm:atom7}) there must be an action $\alpha_c$ done by transaction $\iota$ happening after $\alpha'$ which is conflicting with $\alpha$. This scenario leads back to \ref{thm:atom8}.
	
	\item If $\alpha = \alpha(j, k)$ and $\alpha' = \actunlock{\iota}{k}$ and $\alpha$ is a $\mathsf{write}$ action then it must be the case that $(\emptyset, \textsc{u}) = \Phi_b(k)$ and $\alpha'$ could have never succesfully reduced from $(h_b, \Phi_b, S_b, \mathds{P}_b)$.
\end{enumerate}

We have now established (\ref{thm:atom6}), i.e. that the minimal transaction $\iota$'s operations appear in $\tau_{seq}$ before the ones of any other transaction. Let's now analyse the structure of the reduction described by $\tau_{seq}$, for some fresh $\alpha \in \mathsf{Act}$.
\[
	(h, \emptyset, S, \mathds{P}_1 \| \mathds{P}_2) \xrightarrow{\alpha}
	\underbrace{(h'', \Phi'', S'', \mathds{P}'') \rightarrow^* (h', \emptyset, S', \pskip)}_{n \text{ steps}}
\]
\begin{itemize}
	\item If $\alpha = \actprog$ then by Lemma \ref{lem:sameSys} we obtain that $(h, \mathds{P}_1 \| \mathds{P}_2) \tred (h'', \mathds{P}'')$ and the final result follows by I.H.
	
	\item If $\alpha \neq \actprog$ then from (\ref{thm:atom6}) we know that the action was performed by transaction $\iota$, the minimal one according to $\sqsubset$. Without loss of generality, we can assume that the program $\mathds{P}_1 \| \mathds{P}_2$ is of the following shape:
	\begin{gather}
		\left( \mathds{T}_\iota ; \mathds{P}_1' \right) \| \mathds{P}_2
	\end{gather}
	From the assumption that $\alpha \neq \actprog$ and Lemma \ref{lem:sysSwap} we know that all of labels generated from the reduction $\iota$ will appear before any system transition. This means that under $\tau_{seq}$ we are able to reduce the initial state and program as follows, for some $m < n + 1$:
	\begin{gather}
		\label{thm:atom10}
		(h, \emptyset, S, \left( \mathds{T}_\iota ; \mathds{P}_1' \right) \| \mathds{P}_2) \rightarrow^{m} (h_{fin}, \Phi_{fin}, S_{fin}, \left( \pskip ; \mathds{P}_1' \right) \| \mathds{P}_2)
	\end{gather}
	From (\ref{thm:atom10}), \textit{Base case 2} of this proof and the fact that by rule \textsc{AtTrans} a transaction can always run without conditions on the global storage (i.e. empty premiss), we obtain that:
	\begin{gather}
		\label{thm:atom11}
		(h, \left( \mathds{T}_\iota ; \mathds{P}_1' \right) \| \mathds{P}_2) \tred (h_{fin}, \left( \pskip ; \mathds{P}_1' \right) \| \mathds{P}_2)
	\end{gather}
	From (\ref{thm:atom10}), (\ref{thm:atom11}) and I.H given that we have reduced the starting program for $m$ steps, we know that:
	\begin{gather}
		(h_{fin}, \Phi_{fin}, S_{fin}, \left( \pskip ; \mathds{P}_1' \right) \| \mathds{P}_2) \rightarrow^* (h', \emptyset, S', \pskip) \\
		\implies (h_{fin}, \left( \pskip ; \mathds{P}_1' \right) \| \mathds{P}_2) \tred^* (h', \pskip)
	\end{gather}
	which concludes our proof.
\end{itemize}
\end{proof}
}
\end{thm}