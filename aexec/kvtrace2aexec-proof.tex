\subsection{AUX proof}
\label{sec:kvtrace2aexec}

\begin{lemma}
\label{lem:graph.extend}
Let $\aexec$ be an abstract execution, 
%$\cl$,  be a client such that $\txid \in \nextTxid(\cl, \T_{\aexec})$, 
$\txid \notin \T_{\aexec} \cup \{\txid_0\}$ be a transaction identifier $\T_{\aexec}$, and $\opset \in \Tx$. 
Let $\T \subseteq \T_{\aexec}$ be a set of transaction identifiers.
%\ac{$\nextTxId$ has been defined only for kv-stores, it must be defined for sets of transactions.}
Define $\Gr := \graphof(\aexec), \Gr' := \graphof(\extend(\aexec, \txid, \T, \opset))$. 
We have the following: 
\begin{enumerate}
\item for any $\txid' \in \T_{\Gr'}$, either $\txid' \in \T_{\Gr}$ and $\TtoOp{T}_{\Gr}(\txid') = \TtoOp{T}_{\Gr'}(\txid')$, 
or $\txid' = \txid$ and $\TtoOp{T}_{\Gr'}(\txid) = \opset$.
\item $\txid' \xrightarrow{\RF_{\Gr'}(\ke)} \txid''$ if and only if either 
$\txid' \xrightarrow{\RF_{\Gr}(\ke)_{\Gr}} \txid''$, or $(\otR, \ke, \_) \in \opset$, $\txid'' = \txid$ and 
$\txid' = \max_{\VO_{\Gr}(\ke)}(\T)$, 
\item $\txid' \xrightarrow{\VO_{\Gr'}(\ke)} \txid''$ if and only if 
either $\txid' \xrightarrow{\VO_{\Gr}(\ke)} \txid''$, or $(\otW, \ke, \_) \in \opset$, $\txid'' = \txid$, 
and $(\otW, \ke, \_) \in_{\Gr} \txid'$.
\end{enumerate}
\end{lemma}

\begin{proof}
Fix a key $\ke$. Let $\aexec' = \extend(\aexec, \txid, \T, \opset)$. Recall that $\Gr' = \graphof(\aexec')$.

\begin{enumerate}
\item By definition of $\extend$, and 
because $\txid \notin \T_{\aexec}$, we have that 
$\T_{\aexec'} = \T_{\aexec} \uplus \{\txid\}$. Furthermore, $\TtoOp{T}_{\aexec'}(\txid) = \opset$, 
from which it follows that $\TtoOp{T}_{\Gr'}(\txid) = \opset$.
For all $\txid' \in \T_{\aexec}$, we have that $\TtoOp{T}_{\aexec'}(\txid') = 
\TtoOp{T}_{\aexec}(\txid') = \TtoOp{T}_{\Gr}(\txid')$.
\item
Suppose that $\txid' \xrightarrow{\RF(\ke)_{\Gr}} \txid''$ for some $\txid', \txid'' \in \T_{\Gr}$. 
By definition, $(\otR, \ke, \_) \in_{\aexec} \txid''$,  
and $\txid' = \max_{\AR_{\aexec}}(\VIS_{\aexec}^{-1}(\txid'') \cap \{\txid''' \mid (\otW, \ke, \_) \in_{\aexec} \txid'''\})$. 
Because $\txid'' \in \T_{\Gr} = \T_{\aexec}$, it follows that $\txid'' \neq \txid$. By definition, 
$\VIS^{-1}_{\aexec'}(\txid'') = \VIS^{-1}_{\aexec}(\txid)$: also, whenever 
$\txid_{a}, \txid_{b} \in \VIS^{-1}_{\aexec'}(\txid)$ we have that $\txid_{a}, \txid_{b} \in \T_{\aexec}$, 
and therefore if $\txid_{a} \xrightarrow{\AR_{\aexec'}} \txid_{b}$, then it must be the case 
that $\txid_{a} \xrightarrow{\AR_{\aexec}} \txid_b$; also, $\TtoOp{T}_{\aexec}(\txid_{a}) = \TtoOp{T}_{\aexec'}(\txid_{a})$. 
As a consequence, we have that 
\[\max_{\AR_{\aexec'}}(\VIS^{-1}_{\aexec'}(\txid) \cap \{ \txid''' \mid (\otW, \ke, \_) \in_{\aexec'} \txid'''\}) = 
\max_{\AR_{\aexec}}(\VIS^{-1}_{\aexec}(\txid) \cap \{ \txid''' \mid (\otW, \ke, \_) \in_{\aexec} \txid'''\}) = \txid', \] 
and therefore $\txid' \xrightarrow{\RF_{\Gr'}} \txid$. 

Suppose now that $(\otR,\ke, \_) \in \opset$, and $\txid' = \max_{\VO(\ke)_{\Gr}}(\T)$. 
By Definition, $\txid' = \max_{\AR_{\aexec}}(\T) \cap \{ \txid''' \mid (\otW, \ke, \_) \in_{\aexec} \txid'''\}$, 
Also, $\T = \VIS^{-1}_{\aexec'}(\txid)$, and because $\T \subseteq \T_{\aexec}$, we have 
that for any $\txid_{a}, \txid_{b}$, if $\txid_{a} \xrightarrow{\AR_{\aexec}} \txid_{b}$, 
then $\txid_{a} \xrightarrow{\AR_{\aexec'}} \txid_{b}$; and $\TtoOp{T}_{\aexec'}(\txid_{a}) = 
TtoOp{T}_{\aexec}(\txid_a)$. Therefore, 
\[
\txid' = \max_{\AR_{\aexec'}}(\VIS^{-1}_{\aexec'}(\txid) \cap \{ \txid''' \mid (\otW, \ke, \_) \in_{\aexec'} \txid'''\}, 
\] 
from which it follows that $\txid' \xrightarrow{\RF_{\Gr'}(\ke)}\txid$.

Now, suppose that $\txid' \xrightarrow{\RF_{\Gr'}(\ke)} \txid''$ for some $\txid', \txid'' \in \T_{\Gr'} = 
\T_{\aexec'}$. We have that $ (\otR, \ke, \_) \in_{\aexec'} \txid''$, 
$(\otW, \ke, \_) \in_{\aexec'} \txid'$, and $\txid'' = \max_{\AR_{\aexec'}}(\VIS_{\aexec'}^{-1}(\txid'') 
\cap \{ \txid''' \mid (\otW, \ke, \_) \in_{\aexec'} \txid'''\}$. 
We also have that $\T_{\aexec'} = \T_{\aexec} \uplus \{\txid\}$. We perform a case 
analysis on $\txid''$. 

If $\txid'' = \txid$, then by definition of $\extend$ we have that 
$\VIS^{-1}_{\aexec'}(\txid) = \T$. Note that $\T \subseteq \T_{\aexec}$, so that 
for any $\txid_{a}, \txid_{b} \in \T_{\aexec}$, we have that $\txid_{a} \xrightarrow{\AR_{\aexec'}} \txid_{b}$ 
if and only if $\txid_{a} \xrightarrow{\AR_{\aexec}} \txid_{b}$, 
and $(\otW, \ke, \val) \in_{\aexec'} \txid_{a}$ if and only if $(\otW, \ke, \val) \in_{\aexec} \txid_{a}$. 
Thus, $\txid' = \max_{\AR_{\aexec}}(\T 
\cap \{\txid''' \mid (\otW, \ke, \_) \in_{\aexec} \txid'''\}) = \max_{\VO_{\Gr}(\ke)}(\T)$. 

If $\txid'' \in \T_{\aexec}$, then it is the case that 
$\txid' = \max_{\AR_{\aexec'}}(\VIS^{-1}_{\aexec'}(\txid'') \cap \{ \txid''' \mid (\otW, \ke, \_) \in_{\aexec'} \txid'''\}$. 
Similarly to the case above, we can prove that $\VIS^{-1}_{\aexec'}(\txid'') = \VIS^{-1}_{\aexec}(\txid)$, 
for any $\txid_{a}, \txid_{b} \in \VIS^{-1}_{\aexec}(\txid)$, $(\otW, \ke, \val) \in_{\aexec'} \txid_{a}$ 
implies $(\otW, \ke, \val) \in_{\aexec} \txid_{a}$, and $\txid_{a} \xrightarrow{\AR_{\aexec'}} \txid_{b}$ 
implies $\txid_{a} \xrightarrow{\AR_{\aexec}} \txid_{b}$, from which it follows that 
$\txid' = \max_{\AR_{\aexec}}(\VIS^{-1}_{\aexec}(\txid'') \cap \{ \txid''' \mid (\otW, \ke \_) \in_{\aexec} \txid'''\})$, 
and therefore $\txid' \xrightarrow{\RF_{\Gr}(\ke)} \txid''$.

\item Suppose that $\txid' \xrightarrow{\VO_{\Gr}(\ke)} \txid''$ for some $\txid', \txid'' \in \T_{\aexec}$. 
Then $(\otW,\ke,\_) \in_{\aexec} \txid', (\otW, \ke, \_) \in_{\aexec} \txid''$, and $\txid' \xrightarrow{\AR_{\aexec}} \txid''$. 
By definition of $\extend$, it follows that $\txid' \xrightarrow{\AR_{\aexec'}} \txid''$, and because 
$\txid', \txid'' \in \T_{\aexec}$, hence $\txid', \txid'' \neq \txid$, then 
$(\otW,\ke, \_) \in_{\aexec'} \txid'$, $(\otW, \ke, \_) \in_{\aexec'} \txid''$. By definition, 
we have that $\txid' \xrightarrow{\VO_{\aexec'}(\ke)} \txid''$.

Suppose that $(otW, \ke, \_) \in_{\aexec} \txid'$, $(\otW, \ke, \_) \in \opset$. Because $\txid' \in \T_{\aexec}$, 
we have that $\txid' \neq \txid$, hence $(\otW, \ke, \_) \in_{\aexec' }\txid'$. By definition, 
$\TtoOp{T}_{\aexec'}(\txid) = \opset$, hence $(\otW, \ke, \_) \in_{\aexec'} \txid$. Finally, 
the definition of $\extend$ ensures that $\txid' \xrightarrow{\AR_{\aexec'}} \txid$. Combining 
these three facts together, we obtain that  
$\txid' \xrightarrow{\VO_{\Gr'}(\ke)} \txid$. 

Now, suppose that $\txid' \xrightarrow{\VO_{\Gr'}(\ke)} \txid''$ for some $\txid', \txid'' \in \T_{\aexec}$. 
Then $\txid' \xrightarrow{\AR_{\aexec'}} \txid''$, $(\otW, \ke, \_) \in_{\aexec'} \txid'$, $(\otW, \ke, \_) 
\in_{\aexec'} \txid''$. 
Recall that $\T_{\Gr'} = \T_{\aexec'} = \T_{\aexec} \uplus \{ \txid \}$. We perform a case analysis on $\txid''$. 

If $\txid'' = \txid$, then the definition of $\extend$ ensures that $\txid' \xrightarrow{\AR_{\aexec'}} \txid$ 
implies that $\txid \in \T_{\aexec}$, hence $\txid' \neq \txid$. 
Together with $(\otW, \ke, \_) \in_{\aexec'} 
\txid'$, this leads to $(\otW, \ke, \_) \in_{\aexec} \txid'$. 

If $\txid'' \in \T_{\aexec}$, then $\txid'' \neq \txid$. The definition of $\extend$ ensures that $\txid' \xrightarrow{\AR_{\aexec}} \txid''$. 
This implies that $\txid', \txid'' \in \T_{\aexec}$, hence $\txid', \txid'' \neq \txid$, and $\TtoOp{T}_{\aexec'}(\txid') = \TtoOp{T}_{\aexec}(\txid')$, 
$\TtoOp{T}_{\aexec'}(\txid'') = \TtoOp{T}_{\aexec}(\txid'')$. It follows that $(\otW, \ke, \_) \in_{\aexec} \txid'$, 
$(\otW, \ke, \_) \in_{\aexec} \txid''$, and therefore $\txid' \xrightarrow{\VO_{\Gr}}(\ke) \txid''$.

\end{enumerate}
\end{proof}

\begin{lemma}
\label{lem:graph.update}
Let $\hh$ be a kv-store, and $\vi \in \Views(\hh)$. Let $\txid \notin \hh$, and 
$\opset \subseteq \powerset{\Ops}$, and let $\hh' = \updateKV(\hh, \vi, \txid, \opset)$. 
Let $\Gr = \Gr_{\hh}$, $\Gr' = \Gr_{\hh'}$; then for all $\txid', \txid'' \in \T_{\Gr'}$ and keys $\ke$, 
\begin{itemize}
\item $\TtoOp{T}_{\Gr'} = \TtoOp{T}_{\Gr}\rmto{\txid}{\opset}$, 
\item $\txid' \xrightarrow{\RF_{\Gr'}(\ke)} \txid''$ if and only if either 
$\txid' \xrightarrow{\RF_{\Gr}(\ke)} \txid''$, or $(\otR, \ke, \_) \in \opset$ and 
$\txid' = \max_{\VO_{\Gr}(\ke)}(\{\WTx(\hh(\ke, i)) \mid i \in \vi(\ke)\})$, 
\item $\txid' \xrightarrow{\VO_{\Gr'}(\ke)} \txid''$ if and only if either 
$\txid' \xrightarrow{\VO_{\Gr}(\ke)} \txid''$, or $(\otW, \ke, \_) \in \opset$ 
and $\txid' = \WTx(\hh(\ke, \_))$. 
\end{itemize}

\end{lemma}

\begin{proof}
Fix $\ke \in \Keys$. Because $\txid \notin \hh$, then $\txid \notin \T_{\Gr}$, 
and by definition of $update$ we obtain that $\{\txid' \mid \txid' \in \hh'\} = 
\{\txid' \mid \txid' \in \hh\} \cup \{\txid\}$. It follows that $\T_{\Gr'} = \T_{\Gr} \uplus \{\txid \}$.

\begin{enumerate}
\item Suppose that $(\otR, \ke, \val) \in_{\Gr} \txid'$. By definition, 
there exists an index $i = 0,\cdots, \lvert \hh(\ke) \rvert - 1$ such 
that $\hh(\ke, i) = (\val, \_, \{\txid'\} \cup \_)$. Because $\hh' = \updateKV(\hh, \vi, \txid, \opset)$, 
it is immediate to observe that $\hh'(\ke, i) = (\val, \_, \{\txid'\} \cup \_)$, and therefore 
$(\otR,\ke, \val) \in_{\Gr'} \txid'$. Conversely, note that if $(\otR, \ke, \val) \in_{\Gr'} \txid$, 
then there exists an index $i = 0,\cdots, \lvert \hh'(\ke) \rvert - 1$ such that 
$\hh'(\ke, i) = (\val, \_, \{\txid'\} \cup \_)$. As a simple consequence of \cref{cor:updatekv.singlecell} 
it follows that it must be the case that $i \leq \lvert \hh(\ke) \rvert - 1$, and because 
$\txid' \neq \txid$, we have that $\hh(\ke, i) = (\val, \_, \{\txid'\} \cup \_)$. Therefore 
$(\otR, \ke, \val) \in_{\Gr} \txid'$. 

Similarly, if $(\otW, \ke, \val) \in_{\Gr} \txid'$, 
then there exists an index $i=0,\cdots, \lvert \hh(\ke) \rvert - 1$ such that 
$\hh(\ke, i) = (\val, \txid', \val)$. It follows that $\hh'(\ke, i) = (\val, \txid', \_)$, hence 
$(\otW, \ke, \val) \in_{\Gr'} \txid'$. If $(\otW, \ke, \val) \in \opset$, then we 
have from \cref{cor:updatekv.singlecell} that $\hh'(\ke, \lvert \hh'(\ke) \rvert - 1) = (\val, \txid', \_)$, 
hence $(\otW, \ke, \val) \in_{\Gr'} \txid'$. 
Conversely, if $(\otW, \ke, \val) \in_{\Gr'} \txid'$, then there exists an index 
$i = 0, \cdots, \lvert \hh'(\ke) \rvert - 1$ such that $\hh(\ke, i) = (\val, \txid', \_)$. 
We have two possible cases: either $i < \lvert \hh'(\ke, i) \rvert - 1$, leading to  
$\txid' \neq \txid$ and $\hh(\ke, i) = (\val, \txid', \_)$, or equivalently 
$(\otR,\ke, \val) \in_{\Gr} \txid'$; or $i = \lvert \hh'(\ke, i) \rvert - 1$, 
leading to $\txid' = \txid$, and $\hh(\ke, i) = (\val, \txid, \emptyset)$ 
for some $\val$ such that $(\otW, \ke, \val) \in \opset$. 

Putting together the facts above, we obtain that $\TtoOp{T}_{\Gr'} = 
\TtoOp{T}_{\Gr}\rmto{\txid}{\opset}$, as we wanted to prove.

\item Suppose that $\txid' \xrightarrow{\RF_{\Gr}(\ke)} \txid''$. 
By definition, there exists an index $i = 0,\cdots, \lvert \hh(\ke) \rvert - 1$ 
such that $\hh(\ke, i) = (\_, \txid', \{\txid''\} \cup \_)$. It is immediate 
to observer, from the definition of $\updateKV$, that $\hh'(\ke, i) = (\_, \txid', \{\txid''\} \cup \_)$, 
and therefore $\txid' \xrightarrow{\RF_{\Gr'}(\ke)} \txid''$. 

Next, suppose that $(\otR, \ke, \_) \in \opset$, and $\txid' = \max_{\VO_{\Gr}(\ke)}(\{\WTx(\hh(\ke, i)) \mid i \in \vi(\ke)\}$. 
By Definition, $\hh(\ke, i) = (\_, \txid', \_)$, where $i = \max(\vi(\ke))$. This is because 
$\txid' \rightarrow{\VO_{\Gr}(\ke)} \txid''$ if and only if $\txid' = \WTx(\hh(\ke, j_1)), \txid'' = 
\WTx(\hh(\ke, j_2))$ for some $j_1, j_2$ such that $j_1 < j_2$. 
The definition of $\updateKV$ now ensures that $\hh'(\ke, i) = (\_, \txid', \{\txid \} \cup \_)$, 
from which it follows that $\txid' \xrightarrow{\RF_{\Gr'}(\ke)} \txid$.

Conversely, suppose that $\txid' \xrightarrow{\RF_{\Gr'}(\ke)} \txid''$. 
Recall that $\T_{\Gr'} = \T_{\Gr} \cup \{ \txid \}$, hence either 
$\txid'' \in \T_{\Gr}$ or $\txid'' = \txid$. 

If $\txid'' = \txid$, then it must be the case that there exists an index $i = 0,\cdots, \lvert \hh'(\ke) \rvert - 1$ 
such that $\hh'(\ke, i) = (\_, \txid', \{\txid \} \cup \_)$. Note that if $\hh'(\ke, \lvert \hh'(\ke) \rvert -1)$ is 
defined, then it must be the case that $\hh'(\ke, \lvert \hh'(\ke) \rvert -1) = (\_, \txid, \emptyset)$, 
hence it must be the case that $i < \lvert \hh'(\ke) \rvert - 1$. Because $\txid \notin \hh$, 
then by the definition of $\updateKV$ it must be the case that $(\otR, \ke, \_) \in \opset$, 
$\hh(\ke, i) = (\_, \txid', \_)$ and $i = \max(\vi(\ke))$; this also implies that $\txid' = 
\max_{\VO(\ke)}\{\WTx(\hh(\ke, i)) \mid i \in \vi(\ke)\}$. 

If $\txid'' \in \T_{\Gr}$, then  it must be the case that $\txid'' \neq \txid$. 
In this case, it also must exist an index $i = 0,\cdots, \lvert \hh'(\ke) \rvert - 1$ 
such that $\hh'(\ke, i) = (\_, \txid', \{\txid''\} \cup \_)$. As in the previous 
case, we note that $i < \lvert \hh'(\ke) \rvert - 1$, which together 
with the fact that $\txid'' \neq \txid$ leads to $\hh(\ke, i) = (\_, \txid', \{\txid''\} \cup \_)$. 
It follows that $\txid' \xrightarrow{\RF_{\Gr}(\ke)} \txid''$.

\item Suppose that $\txid' \xrightarrow{\VO_{\Gr}(\ke)} \txid''$. 
By definition, there exist two indexes $i, j$ such that 
$\hh(\ke, i) = (\_, \txid', \_)$, $\hh(\ke, j) = (\_, \txid'', \_)$ 
and $i < j$. The definition of $\updateKV$ ensures that 
$\hh'(\ke, i) = (\_, \txid', \_)$, $\hh'(\ke, j) = (\_, \txid'', \_)$, 
and because $i < j$ we obtain that $\txid' \xrightarrow{\VO_{\Gr'}(\ke)} \txid''$. 

Suppose that $(\otW, \ke, \_) \in \opset$. Then $\hh'(\ke, \lvert \hh(\ke) \rvert) = (\_, \txid, \_)$.
Let $\txid' \in \T_{\Gr}$; by definition there exists an index $i = 0,\cdots, \lvert \hh(\ke) \rvert$ 
such that $\hh(\ke, i) = (\_, \txid', \_)$. It follows that $\hh'(\ke, i) = (\_, \txid', \_)$, and 
because $i < \lvert \hh(\ke) \rvert$, then we have that $\txid' \xrightarrow{\VO_{\Gr'}(\ke)} \txid$. 

Conversely, suppose that $\txid' \xrightarrow{\VO_{\Gr'}(\ke)} \txid''$. Because 
$\T_{\Gr'} = \T_{\Gr} \cup \{ \txid \}$, we have two possibilities. Either $\txid'' = \txid$, 
or $\txid'' \in \T_{\Gr}$. 

If $\txid'' = \txid$, then it must be the case that $(\otW, \ke, \_) \in_{\Gr'} \txid$, 
or equivalently there exists an index $i=0,\cdots, \lvert \hh'(\ke) \rvert -1 $ such that 
$\hh'(\ke, i) = (\_, \txid, \_)$. Because $\txid \notin \hh$, and because for any 
$i = 0, \cdots, \lvert \hh(\ke) \rvert - 1$, $\hh'(\ke, i) = (\_, \txid, \_) \implies 
\hh(\ke, i) = (\_, \txid, \_)$, then it necessarily has to be $i = \hh'(\ke) \rvert - 1$. 
According to the definition of $\updateKV$, this is possible only if $(\otW,\ke, \_) \in \opset$. 
Finally, note that because $\txid' \xrightarrow{\VO_{\Gr'}(\ke)} \txid$, then 
there exists an index $j < \lvert \hh'(\ke, i) \rvert - 1$ such that 
$\hh'(\ke, j) = (\_, \txid' ,\_)$. The fact that $j < \lvert \hh'(\ke, i) \rvert - 1$ 
From \cref{cor:updatekv.singlecell} we obtain that $\hh(\ke, j) = (\_, \txid', \_)$, 
or equivalently $\txid' = \WTx(\hh(\ke, \_))$. 

If $\txid'' \in \T_{\Gr}$, then there exist two indexes $i,j$ such that 
$j < \lvert \hh'(\ke, j) \rvert - 1$, $\hh'(\ke, j) = (\_, \txid'', \_)$, 
$i < j$, and $\hh'(\ke, i) = (\_, \txid', \_)$. It is immediate to observe 
that $\hh(\ke, i) = (\_, \txid', \_)$, $\hh(\ke, j) = (\_, \txid'', \_)$, 
from which $\txid' \xrightarrow{\VO_{\Gr}(\ke)} \txid''$ follows. 

\end{enumerate}
\end{proof}

