\section{Operational Semantics on KV-Stores}
\label{sec:full-semantics}

\begin{definition}[Multi-version Key-value Stores]
\label{def:mkvs-appendix}
Assume a countably infinite set of \emph{keys} $\Keys \ni \key$, 
and a countably infinite set of  \emph{values} $\Val \ni \val$, 
including an \emph{initialisation value} $\val_0 $.
The set of \emph{versions}, $\Versions \ni \ver$, is: $\Versions \defeq \Val \times \TxID \times \pset{\TxID_{0}}$. 
A \emph{kv-store} is a function $\mkvs: \Keys \to \func{List}[\Versions]$, 
where $\func{List}[\Versions] \ni \vilist$ is the set of lists of versions $\Versions$. 
Well-formed key-values store satisfy:
\begin{align}
& \fora{\key, i, j} 
\rsOf(\mkvs(\key, i)) \cap \rsOf(\mkvs(\key, j)) \neq \emptyset \lor
\wtOf(\mkvs(\key, i)) = \wtOf(\mkvs(\key, j))
\implies i = j  \\
& \fora{\key} \mkvs(\key, 0) = (\val_0, \txid_0, \stub) \\
& \fora{ \key, \cl, i,j, n, m} 
\txid_{\cl}^{n} = \wtOf(\mkvs(\key,i)) \land \txid_{\cl}^{m} \in
\Set{\wtOf(\mkvs(\key,j))} \cup \rsOf(\mkvs(\key, i)) \implies n < m
\end{align}
\end{definition}

The full semantics is in \cref{def:program_semantics} and the full definition of consistency models is in  \cref{fig:app-execution-tests}.
% composition of kv


\begin{figure*}[!t]
\[
\begin{rclarray}
\toL & : & ((\Stacks \times \Snapshots \times \Fingerprints) \times \Transactions) \times ((\Stacks \times \Snapshots \times \Fingerprints) \times \Transactions)
\end{rclarray}
\]
\begin{mathpar}
    \infer[\rl{TPrimitive}]{%
        (\stk, \h, \fp) , \transpri \ \toL \  (\stk', \h', \fp \addO \op) , \pskip 
    }{%
        (\stk, \h) \toLTS{\transpri} (\stk', \h')
        \\ \op = \func{op}{\stk, \h, \transpri}
    }
    \\\
    \infer[\rl{TChoice}]{%
        (\stk, \h, \fp) , \trans_{1} \pchoice \trans_{2} \ \toL \  (\stk, \h, \fp) , \trans_{i}
    }{
        i \in \Set{1,2}
    }
    \and
    \infer[\rl{TIter}]{%
        (\stk, \h, \fp),  \trans\prepeat \ \toL \  (\stk, \h, \fp), \pskip \pchoice (\trans \pseq \trans\prepeat)
    }{ } 
    \and
    \infer[\rl{TSeqSkip}]{%
        (\stk, \h, \fp), \pskip \pseq \trans \ \toL \  (\stk, \h, \fp), \trans
    }{ }
    \and
    \infer[\rl{TSeq}]{%
        (\stk, \h, \fp), \trans_{1} \pseq \trans_{2} \ \toL \  (\stk', \h', \fp'), \trans_{1}' \pseq \trans_{2}
    }{%
        (\stk, \h, \fp), \trans_{1} \ \toL \  (\stk', \h', \fp'), \trans_{1}'
    }
\end{mathpar}
\hrulefill
\[
\begin{rclarray}
	\toT{}  & : &
    \begin{array}[t]{@{}c@{}}
    \Clients \; \times \;
	\left( ( \HisHeaps \times \Views \times \Stacks ) \times \Commands \right)  
    \; \times\; \ETs \;\times \sort{Labels} \times \;
	\left( ( \HisHeaps \times \Views \times \Stacks ) \times \Commands \right) 
    \end{array}
\end{rclarray}
\]
\begin{mathpar}
    \infer[\rl{CAtomicTrans}]{%
        \cl \vdash 
        ( \mkvs, \vi, \stk ), \ptrans{\trans} \ 
        \toT{(\cl, \vi'', \fp)}_{\ET} \ 
        (\mkvs',\vi', \stk' ) , \pskip
    }{%
		\begin{array}{@{} c @{}}
			\vi \orderVI  \vi''
			\quad \h = \clpsHH{\mkvs,\vi''}
			\quad \txid \in \nextTxId(\cl, \mkvs) \\
			(\stk, \h, \emptyset), \trans \toL^{*}   (\stk', \stub,  \fp) , \pskip
            \quad \mkvs' = \updateKV(\mkvs, \vi'', \fp, \txid) 
            \quad \ET \vdash (\mkvs, \vi'') \csat \f : (\mkvs',\vi')
			%\qquad \ET \vdash (\mkvs, \vi'') \triangleright \fp : \vi'
		\end{array}
    }
    \and
    \infer[\rl{CPrimitive}]{%
    \cl \vdash ( \mkvs, \vi, \stk ) , \cmdpri \ \toT{(\cl,\iota)}_{\ET} \  ( \mkvs, \vi, \stk' ) , \pskip
    }{
        \stk \toLTS{\cmdpri} \stk'
    }
    \and
    \infer[\rl{CChoice}]{%
        \cl \vdash ( \mkvs, \vi, \stk ) , \cmd_{1} \pchoice \cmd_{2} \ \toT{(\cl,\iota)}_{\ET} \  ( \mkvs, \vi, \stk ) , \cmd_{i}
    }{
        i \in \Set{1,2}
    }
    \quad
    \infer[\rl{CIter}]{%
        \cl \vdash ( \mkvs, \vi, \stk ) , \cmd\prepeat \ \toT{(\cl,\iota)}_{\ET} \  ( \mkvs, \vi, \stk ) , \pskip \pchoice (\cmd \pseq \cmd\prepeat)
    }{ }
    \and
    \infer[\rl{CSeqSkip}]{%
        \cl \vdash ( \mkvs, \vi, \stk ) , \pskip \pseq \cmd \ \toT{(\cl,\iota)}_{\ET} \  ( \mkvs, \vi, \stk ) , \cmd
    }{ }
    \quad
    \infer[\rl{CSeq}]{%
        \cl \vdash ( \mkvs, \vi, \stk ) , \cmd_{1} \pseq \cmd_{2} \ \toT{(\cl,\iota)}_{\ET} \ ( \mkvs, \vi', \stk' ) , {\cmd_{1}}' \pseq \cmd_{2}
    }{% 
        \cl \vdash ( \mkvs, \vi, \stk ) , \cmd_{1} \ \toT{(\cl,\iota)}_{\ET} \  ( \mkvs, \vi', \stk' ) , {\cmd_{1}}' 
    }
\end{mathpar}
\vspace*{5pt}

\hrulefill

\[
	\toG{} : 
    ( \Confs \times \ThdEnv \times \Programs) 
    \;\times\; \ETs \;\times \sort{Label} \times \;
    ( \Confs \times \ThdEnv \times \Programs) 
\]
\begin{mathpar}
    \infer[\rl{PProg}]{%
    ( \mkvs, \viewFun, \thdenv ) , \prog \ \toG{\lambda}_{\ET} \  ( \mkvs', \viewFun\rmto{\cl}{\vi'}, \thdenv\rmto{\cl}{\stk'}) , \prog\rmto{\cl}{\cmd'} 
    }{%
        \cl \vdash ( \mkvs, \viewFun(\cl), \thdenv(\cl) ), \prog(\cl), \ \toT{\lambda}_{\ET} \  ( \mkvs', \vi', \stk' ) , \cmd'  
    }
\end{mathpar}
%
\hrulefill
\caption{Operational Semantics on Key-value Store}
\label{fig:transaction_semantics}
\label{def:thread_semantics}
\label{fig:thread_semantics}
\label{def:thread_pool_semantics}
\label{fig:thread_pool_semantics}
\label{def:program_semantics}
\label{fig:program_semantics}
\label{fig:full-semantics}
\end{figure*}

\begin{figure*}[t]
%\renewcommand{\true}{\ensuremath{\mathsf{true}}}
\small
\centering
\scalebox{.8}{%
\begin{tabular}{ @{} l @{\hspace{2pt}} || @{\hspace{2pt}} c | @{\hspace{2pt}} l @{\hspace{2pt}} | @{\hspace{2pt}}  c @{} }
\hline
	\ET 
	& $\cancommit \mkvs \vi \fp$
	& Closure Relation (where applicable)
    & $\vshift \mkvs \vi {\mkvs'} {\vi'}$ 
	\\
	\hline
%	
	\MR 
	& \true 
	& 
	& $\vi \viewleq \vi'$
	\\ \hline  
%
	\RYW
	& \true
	& 
	& 
	\protect{$
	\begin{array}[t]{@{} l @{}}
		\fora{\txid \in \mkvs' \setminus \mkvs} \fora{\key, i} \\
		\;\;\wtOf(\mkvs'(\key, i) ) \toEDGE{\!\!\SO\rflx\!\!} \txid \implies i \!\in\! \vi'(\key) 
	\end{array}
	$}
	\\ \hline  
    \MW 
    & \( \closed(\mkvs, \vi, \rel_{\MW} ) \)
    & \(\rel_{\MW} \defeq \SO \cap \WW_\mkvs\)
    & \true  
    \\ \hline
%	
    \WFR
    & $\closed(\mkvs, \vi, \rel_{\WFR})$
    & $\rel_{\WFR}  \defeq \WR_{\mkvs} ; (\SO \cup \RW_\mkvs)\rflx $ 
    & \true \\ \hline
	\CC
	& $\closed(\mkvs, \vi, \rel_{\CC})$
	& $\rel_{\CC}   \defeq \SO \cup \WR_{\mkvs}$ 
	& $\vshift[\MR \cap \RYW] \mkvs \vi {\mkvs'} {\vi'}$
	\\ \hline  
%
	\UA 
	& $\closed(\mkvs, \vi, \rel_{\UA})$
	& $\rel_{\UA}  \defeq {\textstyle\bigcup_{(\otW, \key, \stub) \in \fp}} \WW^{-1}_{\mkvs}(\key) $ 
	& \true  
	\\ \hline  
% 
	\PSI
	& $\closed(\mkvs, \vi, \rel_{\PSI})$
	& $\rel_{\PSI} \defeq \rel_{\UA} \cup \rel_{\CC} \cup \WW_\mkvs$ 
	& $\vshift[\MR \cap \RYW] \mkvs \vi {\mkvs'} {\vi'}$
	\\ \hline   
%
	\CP 
	& $\closed(\mkvs, \vi, \rel_{\CP})$
	& $\rel_{\CP} \defeq \SO;\RW\rflx_\mkvs \cup \WR_\mkvs;\RW\rflx_\mkvs  \cup \WW_\mkvs$ 	
	& $\vshift[\MR \cap \RYW] \mkvs \vi {\mkvs'} {\vi'}$
    \\ \hline 
%	
	\SI
	& $\closed(\mkvs,\vi, \rel_{\SI})$
	& $  \rel_{\SI}  \defeq \rel_{\UA} \cup \rel_{\CP} \cup (\WW_\mkvs; \RW_\mkvs)$ 
	& $\vshift[\MR \cap \RYW] \mkvs \vi {\mkvs'} {\vi'}$
	\\ \hline  
%% 
	\SER
	& $\closed(\mkvs,\vi, \rel_{\SER})$
	&$\rel_{\SER} \defeq \WW^{-1}$
	& \true	
	\\ \hline
    \SER*
    & $\vshift[\MR \cap \RYW] \mkvs \vi {\mkvs'} {\vi'}$
    & $ \closed(\mkvs,\vi, \rel_{\SER^*})$
    & $\rel_{\SER^*} \defeq \rel_{\UA} \cup \SO \cup \WW_\mkvs \cup \WR_{\mkvs} \cup \RW_\mkvs$ 
    \\ \hline  
\end{tabular}%
}
%
\vspace{0pt}
\caption{Execution tests of well-known consistency models, where \SER* denotes an alternative equivalent $\SER$ specification and $\SO$ is as given in \cref{subsec:kvstores}.
%\( \WR_\mkvs, \WW_\mkvs, \RW_\mkvs\) are given in \cref{subsec:cm_examples}.
}
\label{fig:app-execution-tests}
\end{figure*}


%\newpage
%\subsection{Well-defined \( \updateKV \)}
\label{sec:updatekv-well-defined}
The \cref{lem:updatekv.explicit} and then \cref{cor:updatekv.singlecell} shows that
swapping the operations of one key yields the same result.

\begin{lemma}[Swapping Operation]
\label{lem:updatekv.explicit}
Let $\mkvs$ be a kv-store, $\vi \in \Views(\mkvs)$, $\txid \in \TxID$ and $\fp \in \pset{\Ops}$. 
Let also $\key \in \Keys$. Then
\begin{enumerate}
    \item\label{item:updatekv.explicit.none} 
        $\fora{\val} (\otR, \key, \val) \notin \fp \land (\otW, \key, \val) \notin \fp \implies \updateKV[\mkvs, \vi, \fp, \txid](\key) = \mkvs(\key)$
\item\label{item:updatekv.explicit.rd} 
    $\fora{\val} (\otR, \key, \stub) \in \fp \land (\otW, \key, \val) \notin \fp 
    \implies 
    \begin{array}[t]{@{}l@{}}
    \updateKV[\mkvs, \vi, \fp, \txid](\key) = \\
    \qquad \text{let} \ (\val', \txid', \txidset') = \mkvs(\key, \max<(\vi(\key))) \\
    \qquad \text{in} \ \mkvs(\key)\rmto{\max<(\vi(\key))}{(\val', \txid', \txidset' \cup \Set{\txid})}
    \end{array}
    $
\item\label{item:updatekv.explicit.wr} 
    $
    \begin{array}[t]{@{}l@{}}
    \fora{ \val, \val'}(\otR, \key, \val) \notin \fp \land (\otW, \key, \val) \in \fp 
    \implies \updateKV[\mkvs, \vi, \fp, \txid](\key) = \mkvs(\key) \lcat (\val, \txid, \emptyset)
    \end{array}
    $
\item\label{item:updatekv.explicit.rdwr}
    $
    \fora{\val} (\otR, \key, \stub) \in \fp \land (\otW, \key, \val) \in \fp 
    \implies 
    \begin{array}[t]{@{}l@{}}
    \updateKV[\mkvs,\vi,\fp,\txid](\key) =  \\
    \qquad \text{let} \ (\val', \txid', \txidset') = \mkvs(\key, \max(\vi(\key)))  \\
    \qquad \text{in} \ \mkvs(\key)\rmto{\max(\vi(\key))}{(\val', \txid', \txidset' \cup \Set{\txid}) } \lcat (\val, \txid, \emptyset)
    \end{array}
    $
\end{enumerate}
\end{lemma}

\begin{proof}
All the four statements are proved by induction on $\fp$, by keeping the variable $\mkvs$ universally quantified in the inductive hypothesis. 
Statement \cref{item:updatekv.explicit.rd} and \cref{item:updatekv.explicit.wr} requires 
proving \cref{item:updatekv.explicit.none} first, while Statement \cref{item:updatekv.explicit.rdwr} requires proving all the other statements. 
Fix then an arbitrary $\key \in \Keys$.
\begin{enumerate}
	\item 
	Suppose that for any $\val$, $(\otR, \key, \val) \notin \fp$ and $(\otW, \key, \val)) \notin \fp$. We prove that $\updateKV[\mkvs, \vi, \fp, \txid](\key) = 
	\mkvs(\key)$.
	\begin{itemize}
        \item \caseB{$\fp = \emptyset$} in this case we have that 
		\[
		\updateKV[\mkvs, \vi, \emptyset, \txid](\key) \stackrel{\cref{eq:updatekv}}{=} \mkvs(\key).
		\]
    \item  
        Suppose that $\fp = \fp' \uplus \Set{(\otR, \key', \val')}$ for some $\key', \val'$. Because we are assuming that 
		$(\otR, \key, \val) \notin \fp$ for any $\val \in \Val$, then it must be the case that 
		\begin{equation}
		\label{eq:updatekv.explicit.none.keneqkepRD}
		\key \neq \key'.
		\end{equation}
		Also, we have that $(\otR,\key, \val) \notin \fp'$ and $(\otW, \key, \val) \notin \fp$ for any $\val \in \Val$. 
		By inductive hypothesis we can assume 
		\begin{equation}
            \fora{ \mkvs' }\updateKV[\mkvs', \vi, \fp', \txid](\key) = \mkvs'(\key)
		\label{eq:updatekv.explicit.none.IHrd}
		\end{equation} 
		Therefore we have 
		\begin{align*}
        \updateKV[\mkvs, \vi, \fp, \txid](\key) 
            & =  
            \updateKV[\mkvs, \vi, \fp' \uplus \Set{(\otR, \key', \val')}, \txid](\key) \\
            & \stackrel{\cref{eq:updatekv}}{=}
            \begin{multlined}[t]
                \text{let} \ (\val', \txid', \txidset') = \mkvs(\key', \max_{<}(\vi(\key))) \\
                \text{in} \ \updateKV[\mkvs\rmto{\key'}{\mkvs(\key')\rmto{\max_{<}(\vi(\key'))}{\left(\val', \txid', \txidset' \cup \Set{\txid}\right)}}, \vi, \fp', \txid](\key) 
            \end{multlined} \\
            & \stackrel{\cref{eq:updatekv.explicit.none.IHrd}}{=}
            \begin{multlined}[t]
		    \text{let} \ (\val', \txid', \txidset') = \mkvs(\key', \max_{<}(\vi(\key'))) \\
            \text{in} \ \mkvs\rmto{\key'}{\mkvs(\key')\rmto{\max_{<}(\vi(\key'))}{\left(\val', \txid', \txidset' \Set{\txid}\right)}}(\key) 
            \end{multlined} \\
            &\stackrel{\cref{eq:updatekv.explicit.none.keneqkepRD}}{=} 
		    \text{let} \ (\val', \txid', \txidset') = \mkvs(\key', \max_{<}(\vi(\key'))) \text{ in } \mkvs(\key) \big) \\
            & = \mkvs(\key)
		\end{align*}

		\item Suppose that $\fp = \fp' \uplus \Set{(\otW, \key', \val')}$ for some $\val' \in \Val$. Then it must be the 
		case that 
		\begin{equation}
		\label{eq:updatekv.explicit.none.keneqkepWR}
		\key \neq \key'
		\end{equation}
		Also, we have that $(\otR,\key, \val) \notin \fp'$ and $(\otW, \key, \val) \notin \fp$ for any $\val \in \Val$. 
		By inductive hypothesis we can assume 
		\begin{equation}
            \fora{ \mkvs'}\updateKV[\mkvs', \vi, \fp', \txid](\key) = \mkvs'(\key)
		\label{eq:updatekv.explicit.none.IHwr}
		\end{equation}
		Therefore we have 
        \begin{align*}
            \updateKV[\mkvs, \key, \fp, \txid](\key)
            & =
            \updateKV[\mkvs, \key, \fp \uplus \Set{(\otW, \key', \val')}, \txid](\key) \\
            & \stackrel{\cref{eq:updatekv}}{=} 
            \updateKV[\mkvs\rmto{\key'}{\mkvs(\key')\lcat (\val', \txid, \emptyset)}, \vi, \fp, \txid ](\key)  \\
            &\stackrel{\cref{eq:updatekv.explicit.none.IHwr}}{=}
            \mkvs\rmto{\key'}{\mkvs(\key') \lcat (\val', \txid, \emptyset)}(\key) \\
            & \stackrel{\cref{eq:updatekv.explicit.none.keneqkepWR}}{=} \mkvs(\key)
		\end{align*}
	\end{itemize}

	\item Suppose $(\otR, \key, \stub) \in \fp$, and $(\otW, \key, \val) \notin \fp$ for all $\val \in \Val$. 
        Let $(\val, \txid', \txidset) = \mkvs(\key, \max_{<}(\vi(\key)))$. We prove that 
    \[
        \updateKV[\mkvs, \vi, \fp, \txid](\key) = \mkvs(\key)\rmto{\vi(\key)}{(\val, \txid', \txidset \cup \Set{\txid})}
    \]
		\begin{itemize}
        \item \caseB{$\fp = \emptyset$} this case is vacuous, as $(\otR, \key, \val) \notin \fp$ for all $\val \in \Val$, 
		against the assumption that $(\otR, \key, \stub) \in \fp$. 

		\item Suppose that $\fp = \fp' \cup \Set{(\otR, \key', \stub)}$ for some $\key'$. 
            We have two possible cases: 
			\begin{enumerate}
			\item $\key = \key'$, in which case we know that $(\otR, \key, \val') \notin \fp'$ for all $\val' \in \Val$ because of 
			the assumptions that we make on the structure of $\fp$. 
            By \cref{lem:updatekv.explicit}\cref{item:updatekv.explicit.none} we have that
			\begin{equation}
            \fora{ \mkvs' } \updateKV[\mkvs', \vi, \fp', \txid](\key) = \mkvs'(\key).
			\label{eq:updatekv.explicit.rd.applynone}
			\end{equation}
			In this case we have that 
            \begin{align*}
                \updateKV[\mkvs, \vi, \fp, \txid](\key) 
                & =
                \updateKV[\mkvs,\vi, \fp' \uplus \Set{(\otR, \key, \val')}, \txid](\key) \\
                & \stackrel{\cref{eq:updatekv}}{=} 
                \updateKV[\mkvs\rmto{\key}{\mkvs(\key)\rmto{\max_{<}(\vi(\key))}{(\val, \txid', \txidset \cup \Set{\txid})}}, \vi, \fp', \txid](\key) \\
                & \stackrel{\cref{eq:updatekv.explicit.rd.applynone}}{=}
                \mkvs\rmto{\key}{\mkvs(\key)\rmto{\max_{<}(\vi(\key))}{(\val, \txid', \txidset \cup \Set{\txid})}}(\key) \\
                & = \mkvs(\key)\rmto{\max(\vi(\key))}{(\val, \txid', \txidset \cup \Set{\txid})}
            \end{align*}
            \item \( \key \neq \key' \).
			In this case we know that because $(\otR, \key, \stub) \in \fp$, then 
            it must be $(\otR, \key, \stub) \in \fp'$. We also know that $\fora{ \val } (\otW, \key, \val) \notin \fp$. 
			By the inductive hypothesis, we have that 
			\begin{equation}
			\label{eq:updatekv.explicit.rd.IHrd}
            \begin{array}{l}
			\fora{ \mkvs' } \updateKV[\mkvs', \vi, \fp', \txid](\key) 
            = \mkvs'(\key)\rmto{\max_{<}(\vi(\key))}{(\val, \txid', \txidset \cup \Set{\txid})}
            \end{array}
			\end{equation}
			In this case we have 
			\begin{align*}
                \updateKV[\mkvs, \vi, \fp, \txid](\key) 
                & =
                \updateKV[\mkvs, \vi, \fp' \uplus \Set{(\otR, \key', \stub)}, \txid](\key) \\
                & \stackrel{\cref{eq:updatekv}}{=}
			    \updateKV[\mkvs\rmto{\key'}{\stub}, \vi, \fp, \txid](\key) \\
                & \stackrel{\cref{eq:updatekv.explicit.rd.IHrd}}{=} 
                \mkvs\rmto{\key'}{\stub}(\key)\rmto{\max_{<}(\vi(\key))}{(\val, \txid', \txidset' \cup \Set{\txid})} \\
                &\stackrel{\key \neq \key'}{=}
                \mkvs(\key)\rmto{\max_{<}(\vi(\key))}{(\val, \txid', \txidset' \cup \Set{\txid})}
			\end{align*}
		\end{enumerate}

		\item $\fp = \fp' \uplus \Set{(\otW, \key', \val')}$ for some $\val' \in \Val$. Because $(\otW, \key, \val) \notin \fp$ 
		for any $\val \in \Val$, it must be the case that 
		\begin{equation}
		\key \neq \key'
		\label{eq:updatekv.explicit.rd.keneqkepWR}
		\end{equation}
		Because $(\otR, \key, \stub) \in \fp$, it must also be the case that $(\otR, \key, \stub) \in \fp'$. By the inductive hypothesis, 
		we have that 
		\begin{equation}
		\label{eq:updatekv.explicit.rd.IHwr}
        \begin{array}{l}
        \fora{ \mkvs' }\updateKV[\mkvs', \vi, \fp', \txid](\key) 
        {} = \mkvs(\key)\rmto{\max_{<}(\vi(\key))}{(\val, \txid', \txidset \cup \Set{\txid})}
        \end{array}
		\end{equation}
		It follows that 
        \begin{align*}
		    \updateKV[\mkvs, \vi, \fp, \txid](\key) 
            & =
            \updateKV[\mkvs, \vi, \fp' \uplus \Set{(\otW, \key', \val')}, \txid](\key) \\
            & \stackrel{\cref{eq:updatekv}}{=}
		    \updateKV[\mkvs\rmto{\key'}{\stub}, \vi, \fp', \txid](\key) \\
            & \stackrel{\cref{eq:updatekv.explicit.rd.IHwr}}{=} 
            \mkvs(\rmto{\key'}{\stub}(\key)\rmto{\max_{<}(\vi(\key))}{(\val, \txid', \txidset \cup \Set{\txid})} \\
            & \stackrel{\cref{eq:updatekv.explicit.rd.keneqkepWR}}{=} 
            \mkvs(\key)\rmto{\max_{<}(\vi(\key))}{(\val, \txid', \txidset \cup \Set{\txid})}
        \end{align*}
	\end{itemize}
	
	\item Suppose that $(\otW, \key, \val) \in \fp$ for some $\val \in \Val$, and 
	$(\otR, \key, \val') \notin \fp$ for any $\val' \in \Val$. We prove that 
	$\updateKV[\mkvs, \vi, \fp, \txid](\key) = \mkvs(\key) \lcat (\val, \txid, \emptyset)$. 
		\begin{itemize}
        \item \caseB{$\fp = \emptyset$} This case is vacuous, as $(\otW, \key, \val) \in \fp$.
		\item Suppose that $\fp = \fp' \uplus \Set{(\otR, \key', \stub)}$ for some 
		$\key'$. Note that, because we are assuming that $\Set{(\otR, \key, \val')} \notin \fp$ 
		for all $\val' \in \Val$, then it must be the case that 
		\begin{equation}
		\key \neq \key'
		\label{eq:updatekv.explicit.wr.kenqkepRD}
		\end{equation}	
		We also have that $\Set{(\otR, \key, \val')} \notin \fp'$ for all $\val' \in \Val$, and 
		$(\otW, \key, \val) \in \fp'$. By the inductive hypothesis we have that 
		\begin{equation}
        \fora{\mkvs'} \updateKV[\mkvs', \vi, \fp', \txid](\key) = \mkvs'(\key) \lcat (\val, \txid, \emptyset)
		\label{eq:updatekv.explicit.wr.IHrd}
		\end{equation}
		Therefore, we have that 
        \begin{align*}
		    \updateKV[\mkvs, \vi, \fp, \txid]]\key) 
            & = 
            \updateKV[\mkvs, \vi, \fp' \uplus \Set{(\otR, \key', \stub)}, \txid](\key) \\
            & \stackrel{\cref{eq:updatekv}}{=}
		    \updateKV[\mkvs\rmto{\key'}{\stub}, \vi, \fp',\txid ](\key) \\
            & \stackrel{\cref{eq:updatekv.explicit.wr.IHrd}}{=} 
            \mkvs\rmto{\key'}{\stub}(\key) \lcat (\val, \txid, \emptyset) \\
            & \stackrel{\cref{eq:updatekv.explicit.wr.kenqkepRD}}{=} 
		    \mkvs(\key) \lcat (\val, \txid, \emptyset)
		\end{align*}
		
		\item Suppose that $\fp = \fp' \uplus \Set{(\otW, \key', \val')}$ 
		for some $\key'$. We distinguish two possible cases:
			\begin{enumerate}
			\item $\key = \key'$. In this case the structure of $\fp$ also imposes that $\val = \val'$, 
			and $(\otW, \key, \val'') \notin \fp'$ for any $\val'' \in \Val$. Furthermore, we have 
			that $(\otR, \key, \val'') \notin \fp'$ for any $\val'' \in \Val$. 
			By \cref{lem:updatekv.explicit}\cref{item:updatekv.explicit.none}, we have that 
			\begin{equation}
            \fora{\mkvs'} \updateKV[\mkvs', \vi, \fp', \txid](\key) = \mkvs(\key)
			\label{eq:updatekv.explicit.wr.applynone}
			\end{equation}
			from which it follows 
            \begin{align*}
			    \updateKV[\mkvs, \vi, \fp, \txid](\key) 
                & =
			    \updateKV[\mkvs, \vi, \fp' \uplus \Set{(\otW, \key', \val')}, \txid](\key) \\
                & = 
                \updateKV[\mkvs, \vi, \fp' \uplus \Set{(\otW, \key, \val)}, \txid](\key) \\
                & \stackrel{\cref{eq:updatekv}}{=} 
                \updateKV[\mkvs\rmto{\key}{\mkvs(\key) \lcat (\val, \txid, \emptyset)}, \vi, \fp', \txid](\key) \\
                & \stackrel{\cref{eq:updatekv.explicit.wr.applynone}}{=} 
                \mkvs\rmto{\key}{\mkvs(\key) \lcat (\val, \txid, \emptyset)}(\key) \\
                & = \mkvs(\key) \lcat (\val, \txid, \emptyset)
			\end{align*}
			
            \item \( \key \neq \key'\).
			In this case we have that, because $(\otW, \key, \val) \in \fp$, then it must 
			be $(\otW, \key, \val) \in \fp'$. Furthermore, we also have that $(\otR, \key, \val'') \notin \fp'$ 
			for any $\val'' \in \Val$. By the inductive hypothesis, we have that 
			\begin{equation}
            \fora{\mkvs'}\updateKV[\mkvs', \vi, \fp', \txid](\key) = \mkvs(\key) \lcat (\val, \txid, \emptyset)
			\label{eq:updatekv.explicit.wr.IHwr}
			\end{equation}
			from which it follows 
            \begin{align*}
			    \updateKV[\mkvs, \vi, \fp, \txid](\key) 
                & =
                \updateKV[\mkvs, \vi, \fp' \uplus \Set{(\otW,\key', \val')}, \txid](\key) \\
                & \stackrel{\cref{eq:updatekv}}{=}
			    \updateKV[\mkvs\rmto{\key'}{\stub}, \vi, \fp, \txid](\key) \\
                & \stackrel{\cref{eq:updatekv.explicit.wr.IHwr}}{=} 
                \mkvs\rmto{\key'}{\stub}(\key) \lcat (\val, \txid, \emptyset) \\
                & \stackrel{\key \neq \key'}{=} 
                \mkvs(\key) \lcat (\val, \txid, \emptyset)
			\end{align*}
			\end{enumerate}
		\end{itemize}
		
		\item Suppose that $(\otW, \key, \val) \in \fp$ for some $\val \in \Val$, and $(\otR, \key, \stub) \in \fp$. 
		Let $\mkvs(\key, \vi) = (\val', \txid', \txidset')$. We prove that 
        \[ 
            \updateKV[\mkvs, \vi, \fp, \txid](\key) = 
            \mkvs(\key)\rmto{\vi(\key)}{(\val', \txid', \txidset' \cup \Set{\txid})} \lcat (\val, \txid, \emptyset)
        \]
		by induction on $\fp$:
			\begin{itemize}
			\item $\fp = \emptyset$; this case is vacuous.
			\item $\fp = \fp' \uplus \Set{(\otR, \key', \stub)}$. We distinguish two cases, according to 
			whether $\key = \key'$ or $\key \neq \key'$. If $\key = \key'$, then we know that 
			$(\otW, \key, \val) \in \fp'$ and $(\otR, \key, \val'') \notin \fp$ for any $\val'' \in \Val$. 
			By Lemma \cref{lem:updatekv.explicit}\cref{item:updatekv.explicit.wr} we have that 
			\begin{equation}
            \fora{\mkvs'}\updateKV[\mkvs,\vi,\fp',\txid](\key) = \mkvs(\key) \lcat (\val, \txid, \emptyset)
			\label{eq:updateKV.explicit.rdwr.applyWR}
			\end{equation}
			from which it follows that 
			\begin{align*}
			    \updateKV[\mkvs, \vi, \fp, \txid](\key)
                & =
                \updateKV[\mkvs, \vi, \fp' \uplus \Set{(\otR, \key', \stub)}, \txid](\key) \\
                & = 
			    \updateKV[\mkvs, \vi, \fp' \uplus \Set{(\otR, \key, \stub)}, \txid](\key) \\
                & \stackrel{\cref{eq:updatekv}}{=}
                \updateKV[\mkvs\rmto{\key}{\mkvs(\key)\rmto{\max_{<}(\vi(\key))}{(\val', \txid', \txidset' \cup \Set{\txid})}}, \vi, \fp', \txid](\key) \\
                & \stackrel{\cref{eq:updateKV.explicit.rdwr.applyWR}}{=} 
                \mkvs\rmto{\key}{\mkvs(\key)\rmto{\max_{<}(\vi(\key))}{(\val', \txid', \txidset' \cup \Set{\txid})}}(\key) \lcat (\val, \txid, \emptyset) \\
                & = 
			    \mkvs(\key)\rmto{\max_{<}(\vi(\key))}{(\val' \txid', \txidset' \cup \Set{\txid}} \lcat (\val, \txid, \emptyset)
            \end{align*}
			If $\key \neq \key'$, then we have that both $(\otR, \key, \stub) \in \fp'$ and 
			$(\otW, \key, \val) \in \fp'$. In this case, by the inductive hypothesis we have that 
			\begin{equation}
			\label{eq:updatekv.explicit.rdwr.IHrd}
            \begin{array}{l}
            \fora{\mkvs'}\updateKV[\mkvs,\vi,\fp',\txid](\key) = 
            \mkvs'(\key)\rmto{\max_{<}(\vi(\key))}{(\val', \txid', \txidset' \cup \Set{\txid})} \lcat (\val, \txid, \emptyset)
            \end{array}
			\end{equation}
			from which it follows that 
            \begin{align*}
			    \updateKV[\mkvs, \vi, \fp,\txid](\key)
                & = 
                \updateKV[\mkvs, \vi, \fp' \uplus \Set{(\otR, \key', \stub)}, \txid](\key) \\
                & \stackrel{\cref{eq:updatekv}}{=}
			    \updateKV[\mkvs\rmto{\key'}{\stub}, \vi, \fp', \txid](\key) \\
                & \stackrel{\cref{eq:updatekv.explicit.rdwr.IHrd}}{=} 
                \mkvs\rmto{\key'}{\stub}(\key)\rmto{\max_{<}(\vi(\key))}{(\val', \txid', \txidset' \cup \Set{\txid})} \lcat (\val, \txid, \emptyset) \\
                & = 
                \mkvs(\key)\rmto{\max_{<}(\vi(\key))}{(\val', \txid', \txidset' \cup \Set{\txid})} \lcat (\val, \txid, \emptyset)
            \end{align*}
			
			\item $\fp = \fp' \uplus \Set{(\otW, \key'', \val'')}$ for some $\key'', \val''$. Again, 
			there are two possible cases to consider. If $\key = \key''$, then $\val = \val''$ because of the structure imposed on $\fp$.
			Furthermore, we have that 
			$(\otR, \key, \stub) \in \fp'$ and $(\otW, \key, \val''') \notin \fp$ for all $\val''' \in \Val$.
			By \cref{lem:updatekv.explicit}\cref{item:updatekv.explicit.rd} we have that 
			\begin{equation}
			\label{eq:updatekv.explicit.rdwr.applyRD}
            \begin{array}{l}
            \fora{\mkvs'}\updateKV[\mkvs', \vi, \fp', \txid](\key) =
            \mkvs'(\key)\rmto{\max_{<}(\vi(\key))}{(\val', \txid', \txidset' \cup \Set{\txid})} 
            \end{array}
			\end{equation}
			We have that 
            \begin{align*}
			    \updateKV[\mkvs,\vi,\txid, \fp](\key)
                & =
                \updateKV[\mkvs, \vi, \fp' \cup \Set{(\otW, \key'', \val'')}, \txid](\key) \\
                & =
			    \updateKV[\mkvs,\vi, \fp' \cup \Set{(\otW, \key, \val)}, \txid](\key) \\
                & \stackrel{\cref{eq:updatekv}}{=}
			    \updateKV[\mkvs\rmto{\key}{\mkvs(\key) \lcat (\val, \txid, \emptyset)}, \vi, \fp', \txid](\key) \\
                & \stackrel{\cref{eq:updatekv.explicit.rdwr.applyRD}}{=}
			    \mkvs\rmto{\key}{\mkvs(\key) \lcat (\val, \txid, \emptyset)}(\key)\rmto{\max_{<}(\vi(\key))}{(\val', \txid', \txidset' \cup \Set{\txid})} \\
                & =
			    \left(\mkvs(\key) \lcat (\val, \txid, \emptyset)\right)\rmto{\max_{<}(\vi(\key))}{(\val', \txid', \txidset' \cup \Set{\txid})} \\
                & = 
			    \mkvs(\key)\rmto{\max_{<}(\vi(\key))}{(\val',\txid', \txidset' \cup \Set{\txid})} \lcat (\val, \txid, \emptyset)
            \end{align*}
			Finally, if $\key \neq \key'$, then we have that $(\otR, \key, \stub) \in \fp'$ and $(\otW, \key, \val) \in \fp'$. 
			By the inductive hypothesis, we obtain 
			\begin{equation}
			\label{eq:updatekv.explicit.rdwr.IHwr}
            \begin{array}{l}
            \fora{\mkvs'}\updateKV[\mkvs', \vi, \fp', \txid](\key) = 
            \mkvs'(\key)\rmto{\max_{<}(\vi(\key))}{(\val', \txid', \txidset' \cup \Set{\txid})} \lcat (\val, \txid, \emptyset)
            \end{array}
			\end{equation}
			It follows that 
            \begin{align*}
			    \updateKV[\mkvs, \vi, \fp, \txid](\key)
                & =
                \updateKV[\mkvs, \vi, \fp' \uplus \Set{(\otW, \key', \stub)}, \txid](\key) \\
                & \stackrel{\cref{eq:updatekv}}{=} 
			    \updateKV[\mkvs\rmto{\key'}{\stub}, \vi, \txid, \fp'](\key) \\
                & \stackrel{\cref{eq:updatekv.explicit.rdwr.IHwr}}{=}
                \mkvs\rmto{\key'}{\stub}(\key)\rmto{\max_{<}(\vi(\key))}{(\val', \txid', \txidset' \cup \Set{\txid})} \lcat (\val, \txid, \emptyset) \\
                & =
                \mkvs\rmto{\max_{<}(\vi(\key))}{(\val', \txid', \txidset' \cup \Set{\txid})} \lcat (\val, \txid, \emptyset)
            \end{align*}
			\end{itemize}
\end{enumerate}
\end{proof}

In the following, given a version $\ver = (\val, \txid', \txidset)$ and a set of 
transaction identifiers $\txidset'$, we let $\ver \oplus \txidset' = (\val, \txid', \txidset \cup \txidset')$. 
Clearly the operator $\oplus$ is commutative over sets of transactions: 
$\fora{ \ver, \txidset, \txidset' } (\ver \oplus \txidset) \oplus \txidset' = (\ver \oplus \txidset') \oplus \txidset = 
\ver \oplus (\txidset \cup \txidset')$.

\begin{corollary}
\label{cor:updatekv.singlecell}
Let $\mkvs$ be a kv-store, $\vi \in \Views(\mkvs)$, $\txid \in \TxID$ and $\fp \in \pset{\Ops}$. 
Let also $\key \in \Keys$. Then 
\begin{enumerate}
\item\label{item:updatekv.singlecell.noview} 
    $ 
    \begin{array}[t]{l}
        \fora{ i } 0 \leq i < \abs{\mkvs(\key) } - 1 \land i \neq \max_{<}(\vi(\key)) 
        \implies \updateKV[\mkvs, \vi, \fp, \txid](\key, i) = \mkvs(\key, i)
    \end{array}
    $
\item\label{item:updatekv.singlecell.rd} $\fora{ \val } (\otR, \key, \stub) \in \fp \implies \updateKV[\mkvs,\vi, \fp,\txid](\key, \vi) = \mkvs(\key, \max_{<}(\vi(\key))) \oplus \Set{\txid}$
\item\label{item:updatekv.singlecell.nord} $\fora{ \val } (\otR,\key, \val) \notin \fp \implies \updateKV[\mkvs,\vi, \fp,\txid](\key,\vi) = \mkvs(\key, \max_{<}(\vi(\key)))$
\item\label{item:updatekv.singlecell.wr} 
    $\begin{array}[t]{@{}l@{}}
        \fora{\val} (\otW, \key, \val) \in \fp \\
        \quad {} \implies
        \lvert \updateKV[\mkvs,\vi,\fp,\txid](\key) \rvert = 
        \lvert \mkvs(\key) \rvert + 1 \wedge
        \updateKV[\mkvs,\vi,\fp,\txid](\key, \lvert \mkvs(\key) \rvert) = (\val, \txid, \emptyset)
    \end{array}$
\item\label{item:updatekv.singlecell.nowr} $\fora{ \val } (\otW, \key, \val) \notin \fp \implies \lvert \updateKV[\mkvs,\vi,\fp,\txid](\key) \rvert = \lvert \mkvs(\key) \rvert$
\end{enumerate}
\end{corollary}

\begin{proof}
A simple consequence of \cref{lem:updatekv.explicit}.
\end{proof}


\begin{definition}[ET-reduction]
An \emph{\(\ET\)-reduction}, \((\mkvs, \vienv) \toET{(\cl, \alpha)} (\mkvs', \vienv')\), is defined by:
%we define the \(\ET\)-trace as a sequence of \(\ET\)-reductions on configurations that a client
\begin{enumerate}
    \item either \(\alpha = \varepsilon\), \(\mkvs' = \mkvs\) and $\vienv' =
      \vienv\rmto{\cl}{\vi}\) for some \(u\) such that \( \vienv(\cl) \sqsubseteq u\); or
\item \(\alpha = \fp\) for some \(\fp\), and \(\ET \vdash (\mkvs, \vi ) \csat \fp : (\mkvs', \vi')\), where \(\mkvs' = \updateKV[\mkvs, \vi, \fp, \txid]\) 
   for some \(\txid \in \nextTxid(\cl, \mkvs)\), \(\vienv' = \vienv\rmto{\cl}{\vi'}\).
\end{enumerate}
A finite sequence of \(\ET\)-reductions starting in an
initial configuration \(\conf_{0}\) is called  an \emph{\(\ET\)-trace}. 
\end{definition}
Each \(\ET\)-trace  starting with an initial configuration
(\cref{def:configuration}) terminates in a configuration \((\mkvs, \stub)\) where \(\mkvs\) is obtained as a result of several clients committing transactions under the 
execution test \(\ET\). The consistency model induced by \(\ET\), 
written \(\CMs(\ET)\), is the set of all such terminal kv-stores.

Note that in the definition of \(\ET\)-traces, the view-shifts and 
transaction commits are decoupled. This is in contrast to the
operational semantics (\cref{sec:model}, \cref{fig:semantics-commands}), 
where view-shifts and transaction commits are combined in a single transition of programs (\rl{CAtomicTrans}). 
The reason for this mismatch is best understood when looking at the
intended applications. 
ET-traces are useful for 
proving that a distributed transactional 
protocol implements a given consistency model: in this case, it is convenient to separate shifting a view from committing a transaction, 
as these two steps often take place separately in distributed
protocols. The operational semantics  is particularly useful for  reasoning about transactional 
programs: in this case, the treatment of the view-shifts and transaction commits as a single transition reduces the number of interleavings in programs.
The \(\ET\)-traces and operational semantics are equally expressive as
the following theorem states. 

\begin{theorem}
	\label{thm:ettraces2sem}
	Let \(\interpr{\prog}_{\ET}\) be the set of kv-stores reachable by executing \(\prog\) under the execution test \(\ET\) and  \( \interpr{\et} \) be the the set of kv-stores reachable by \( \et \)-tarces.
    Then for all \(\ET\), \(\interpr{\et} = \bigcup_{\prog} \interpr{\prog}_{\ET}\).
\end{theorem}
