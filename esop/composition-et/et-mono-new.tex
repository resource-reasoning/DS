\subsection{Extra Monotonic \( \ET \)}
The relation \( \ET_1(\txidset_1,\rel_1) \etleq \ET_2(\txidset_2,\rel_2) \) is defined as the following:
\begin{align*}
    \ET_1(\txidset_1,\rel_1) \etleq \ET_2(\txidset_2,\rel_2) 
    & \defiff \fora{\mkvs,\vi,\cl,\fp} \txidset_2(\mkvs,\vi,\cl,\fp) \subseteq \txidset_1(\mkvs,\vi,\cl,\fp) \land \rel_2(\mkvs,\vi,\cl,\fp) \subseteq \rel_1(\mkvs,\vi,\cl,\fp)
\end{align*}
%For brevity, we write \(\txidset\) and \(\rel\) which are implicitly parametrised by \( \mkvs,\vi,\cl,\fp \).
%\begin{proposition}
%if $\ET_1(\txidset_1, \rel_1) \etleq \ET_2(\txidset_2,\rel_2)$ then $\ET_1 \subseteq \ET_2$.
%\end{proposition}
%\begin{proof}
%Assume \( (\mkvs,\vi,\cl,\fp) \in \ET_2 \).
%By definition
%\begin{centermultline}
    %\vi = \getView[\mkvs,\mu X \ldotp \txidset_2 \cup \rel^{-1}_2(X)]
%\end{centermultline}
%By \( \ET_1(\txidset_1, \rel_1) \etleq \ET_2(\txidset_2,\rel_2) \), 
%it follows \( \mu X \ldotp \txidset_2 \cup \rel^{-1}_2(X) \subseteq \mu X \ldotp \txidset_1 \cup \rel^{-1}_1(X)\),
%which means
%\begin{centermultline}
    %\vi = \getView[\mkvs,\mu X \ldotp \txidset_2 \cup \rel^{-1}_2(X)]
%\end{centermultline}
%Thus, \( (\mkvs,\vi,\cl,\fp) \in \ET_2 \).
%\end{proof}
