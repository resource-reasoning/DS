%\subsection{Normal \( \ET \) Traces}
\label{sec:normal-form-exist}
Given a program $\prog$ and an execution test $\ET$, we say that an \emph{execution fragment} of 
$\prog$ under $\ET$ is a sequence of transitions 
$(\mkvs_{0}, \vienv_{0}, \thdenv_{0} ), \prog \toCMD{\_}_{\ET}^{\ast} ( \mkvs, \vienv, \thdenv ) , \prog' )$, 
where $(\mkvs_{0}, \vienv_{0})$ is an initial configuration and $\thdenv_{0}$ is a partial function mapping 
clients in $\lambda x. 0$. If $\prog'(\cl) = \pskip$ whenever $\prog'(\cl)$ is defined, then 
we refer to the sequence of execution above as an \emph{execution} of $\prog$ under $\ET$, and 
we say that $\mkvs$ is \emph{reachable} by executing $\prog$ under $\ET$.

Throughout this Section, we assume that the execution test $\ET$ is fixed. 
Our aim is that of relating  executions of programs under the execution test $\ET$, and $\ET$-traces. 
In particular, we prove  \cref{thm:ettraces2sem}. To this end, we show the following: 

\begin{proposition}
\label{prop:sem2ettraces}
$\bigcup_{\prog} \interpr{\prog}_{\ET} \subseteq \CMs(\ET)$.
\end{proposition}

\begin{proposition}
\label{prop:ettraces2sem}
$\CMs(\ET) \subseteq \bigcup_{\prog} \interpr{\prog}_{\ET}$
\end{proposition}

Let us focus first on the proof of \cref{prop:sem2ettraces}. First, given a program $\prog$, 
we encode the execution of some transactional code $\ptrans{\trans}$ in a program, 
terminating in a configuration $\conf$, into a sequence of $\ET$-reductions also terminating 
in $\conf$. Then we lift this result to execution fragments generated by the program $\prog$ by using 
a simple inductive argument. Because an execution of $\prog$ under $\ET$ is also an execution fragment, 
each execution of $\prog$ terminating in a configuration $\conf$ induces a $\ET$-trace terminating in 
the same configuration, thus showing that $\conf \in \CMs(\ET)$.

\begin{lemma}
\label{lem:sem2ettrace}
For all programs $\prog$ and transitions of the form 
form $\vdash (\mkvs, \vienv, \thdenv ), \prog \toCMD{(\cl, \vi, \fp)}_{\ET}^{\ast} ( \mkvs', \vienv', \thdenv' ) , \prog' )$,  
we have that  
\[
(\mkvs, \vienv) \toET{(\cl, \varepsilon)} (\mkvs, \vienv\rmto{\cl}{\vi}) \toET{(\cl, \fp)} (\mkvs', \vienv').
\]
\end{lemma}

\ac{The semantics of programs for clients has some degree of redundancy that 
should be eliminated. In particular in the transition 
$\cl \vdash 
        ( \mkvs, \vi, \stk ), \ptrans{\trans} 
        \toCMD{(\cl, \vi'', \fp)}_{\ET}
        (\mkvs',\vi', \stk' ) , \pskip$, the parameter $\cl$ is used twice. 
Choose which of the two occurrences you want to keep, and remove the other. 
In the rest of the proof I will assume that the parameter $\cl$ has been removed from the 
action of the transition, and that Rule PProg has been modified to reintroduce the parameter 
$\cl$ into the action. Note that keeping the parameter $\cl$ in both places in the transition above 
complicates the proof of \cref{lem:sem2ettrace}, as it requires proving the following (trivial) lemma: 
if $\cl \vdash (\mkvs, \vi, \stk, \cmd) \toCMD(\cl', \mu) (\mkvs', \vi', \stk', \cmd')$, then $\cl = \cl'$.}

\begin{proof}
Suppose that $ \vdash (\mkvs, \vienv, \thdenv ), \prog \toCMD{(\cl, \vi, \varepsilon)}_{\ET} 
( \mkvs', \vienv', \thdenv') , \prog' )$. This transition could have been derived only by 
applying Rule $\rl{PProg}$, from which it follows that $\cl \vdash (\mkvs, \vi'', \stk), \cmd \toCMD{(\vi, \fp)} 
(\mkvs', \vi', \stk'), \cmd'$, where $\vi'' = \vienv(\cl)$, $\vienv' = \vienv\rmto{\cl}{\vi'}$, $\stk = \thdenv(\cl)$, $\thdenv' = \thdenv\rmto{\cl}{\stk'}$, $\cmd = \prog(\cl)$ 
and $\prog' = \prog\rmto{\cl}{\cmd'}$. The rest of the proof is performed by Rule induction on the 
derivation of the transition $\cl \vdash (\mkvs, \vi'', \stk), \cmd \toCMD{(\vi, \fp)} 
(\mkvs', \vi', \stk')$. 
\begin{itemize}
\item The only base case corresponds to such a transition being derived via an 
application of  Rule $\rl{CAtomicTrans}$. In this case we have that 
$\vi'' \sqsubseteq \vi$, $(\stk, \sn, \emptyset) \toTRANS^{\ast} (\stk', \_, \fp)$ for $\sn = \snapshot[\mkvs, \vi]$, 
$\mkvs' = \updateKV[\mkvs, \vi, \fp \txid]$ for some $\txid \in \nextTxid[\cl, \mkvs]$ and 
$\ET \vdash (\mkvs,\vi) \csat \fp: (\mkvs', \vi')$. Because $\vi'' \sqsubseteq \vi'$ and $\vienv(\cl) = \vi''$, 
we have that $(\mkvs, \vienv) \toET(\cl, \varepsilon) \mkvs, \vienv\rmto{\cl}{\vi}$. Also, because 
$\mkvs' = \updateKV[\mkvs, \vi, \fp, \txid]$, $\txid \in \nextTxid[\cl, \mkvs]$, and $\ET \vdash (\mkvs, \vi) 
\csat \fp: (\mkvs', \vi')$, we have that 
$(\mkvs, \vienv\rmto{\cl}{\vi}) \toET{(\cl, \fp)} (\mkvs', \vienv\rmto{\cl}{\vi'}) = (\mkvs', \vienv')$, as we wanted 
to prove.
\item For the inductive case, the only rule that can have been applied to infer the transition 
$\cl \vdash (\mkvs, \vi'', \stk), \cmd \toCMD{(\vi, \fp)} (\mkvs', \vi', \stk'), \cmd'$ is Rule $\rl{CSeq}$. 
In this case we have that $\cmd = \cmd_1 ; \cmd_2$, $\cmd' = \cmd_1' ; \cmd_2$ for some $\cmd_1, \cmd_1', \cmd_2$, 
$\cl \vdash (\mkvs, \vi'', \stk), \cmd_1 \toCMD{(\vi, \fp)} (\mkvs', \vi', \stk'), \cmd_1'$. By the inductive hypothesis, 
we obtain that 
$(\mkvs, \vienv) \toET{(\cl, \varepsilon)} (\mkvs, \vienv\rmto{\cl}{\vi}) \toET{(\cl, \fp)} (\mkvs', \vienv')$.
\end{itemize}
\end{proof}

\paragraph{Proof of \cref{prop:sem2ettraces}.} 
Recall that we let $\mkvs_{0}$ be the initial kv-store that maps 
each key in the list of locations containing the single version $(0, \txid_{0}, \emptyset)$, 
and we use $\vienv_{0}$ to denote an arbitrary partial mapping of clients into the only valid view 
of $\mkvs_{0}$.

Fix an arbitrary program $\prog$. We prove a more general result: 
whenever $(\mkvs_{0}, \vienv_{0}, \thdenv_{0}) \prog \toCMD{\stub}^{\ast} (\mkvs', \vienv', \thdenv'), prog'$, 
then there exists an $\ET$-trace $\tau$ terminating in the configuration $(\mkvs', \vienv')$, 
and therefore $\mkvs' \in \CMs(\ET)$.
Then \cref{prop:sem2ettraces} follows immediately by considering the 
special case when $\prog'$ is such that $\forall \cl \in \dom(\prog').\; \prog'(\cl) = \pskip$.

Suppose that $(\mkvs,_{0} \vienv_{0}, \thdenv_{0}) \prog \toCMD{\stub}^{\ast} (\mkvs', \vienv', \thdenv'), \prog'$. 
By definition, then $(\mkvs, \vienv, \thdenv_{0}) \prog \toCMD{\stub}^{n} (\mkvs', \vienv', \thdenv'), \prog'$ 
for some $n \geq 0$. We prove that there exists a $\ET$-trace $\tau$ terminating in $(\mkvs\, \vienv')$ 
by induction on $n$. 
\begin{itemize}
\item Base case: $n = 0$. In this case $(\mkvs', \vienv') = (\mkvs_{0}, \vienv_{0})$, and the latter 
is also an $\ET$-trace with no $\ET$-reductions. This case is trivial. 
\item Inductive step: $n > 0$. In this case we have that $(\mkvs_{0}, \vienv_{0}, \thdenv_{0}), \prog \toCMD{\stub}^{n-1} (\mkvs'', \vienv'', \thdenv''), \prog''
\toCMD{\lambda} (\mkvs', \vienv', \thdenv'), \prog'$ for some $\mkvs'', \vienv'', \thdenv'',\prog''$. By inductive hypothesis, 
there exists an $\ET$-trace $\tau_{n-1}$ terminating in the configutration $(\mkvs'', \vienv'')$. 
We distinguish between two cases, according to action $\lambda$: 
\begin{itemize}
\item $\lambda = (\cl, \iota)$ for some client $\cl$. A simple rule induction on the rules of \cref{fig:full-semantics} 
reveals that in this case we have that $(\mkvs', \vienv') = (\mkvs'', \vienv'')$, and therefore we let $\tau := \tau_{n-1}$; 
\item of $\lambda = (\cl, \vi, \fp)$ for some $\cl, \vi, \fp$. By \cref{lem:sem2ettrace}, there exists a sequence 
of $\ET$-reductions $(\mkvs'', \vienv'') \toET{(\cl, \varepsilon)} (\mkvs'', \vienv''\rmto{\cl}{\vi}) \toET{(\cl, \fp)} (\mkvs', \vienv')$. 
Because the $\ET$-trace $\tau_{n-1}$ terminates in the configuration $(\mkvs'', \vienv'')$, then
\[
\tau := \tau_{n-1} \toET{(\cl, \varepsilon)} (\mkvs'', \vienv''\rmto{\cl}{\vi}) \toET{(\cl, \fp)} (\mkvs', \vienv')
\]
is a well-defined $\ET$-trace that terminates in the configuration $(\mkvs', \vienv')$, which concludes the proof.\hfil\qed
\end{itemize}
\end{itemize}

Let us turn our attention to the proof of \cref{prop:ettraces2sem}. The proof of this result is more involved 
technically than the previous proof, because of two reasons. The first reason is that  
executions of programs under the execution test $\ET$ and $\ET$-traces. Whereas in executions of 
programs under the execution test $\ET$
a client can only shift its view immediately before executing a transaction, in $\ET$-traces a client 
can perform a view shift arbitrarily. It would seem then that $\ET$-traces are more expressive than 
executions of programs. However, this difference in expressive power is only apparent: for any 
$\ET$-trace $\tau$ ending in a configuration $(\mkvs, \_)$, we can always construct another 
$\ET$-trace $\tau'$ that ends in the same configuration and that is in \emph{normal form}: 
all view shifts performed by a client immediately precede a transaction commit performed by 
the same client.

%
%First, we show how to convert an $\ET$-trace $\tau$, ending in a configuration of the form $(\mkvs, \_)$, 
%into a \emph{normal} trace $\tau'$, also ending in $(\mkvs, \_)$ 
%and with the additional property that in view shifts of clients, i.e. actions of the form $(\cl, \varepsilon)$, 
%immediately precede a transaction commit performed by the same client, i.e. an action of the form $(\cl, \fp)$. 
\begin{definition}
Let $\ET$ be an execution test. The $\ET$-trace
\[
\conf_0 \toET{(\cl_{0}, \alpha_0)} \conf_1 \toET{(\cl_1, \alpha_1)} \cdots \toET{(\cl_{(2\cdot n) - 1}, \alpha_{2\ cdot n - 1})} \conf_{2 \cdot n}
\]
is in \emph{normal form} if \textbf{(i)} $\conf_0$ is initial, and 
\textbf{(ii)} $\forall  i.\; 0 \leq i < n$ 
\ac{Throughout the text sometimes we use $\forall i. P$, other times we use $\forall i: P$. The style should be 
made consistent. Also, I know you really like to use define macros for everything, but sometimes you define too 
many of them. I still don't see the point of the \textbf{fora} command.}

there exists a client $\cl_i$ and set of operations $\fp_{i}$ such that 
$\alpha_{2 \cdot i} = \varepsilon$, $\cl_{2 \cdot i + 1} = \cl_{2 \cdot i}$ and $\alpha_{2 \cdot i + 1} = \fp_{i}$ for some $\fp_{i} \subseteq \opset$.
\end{definition}


%Note that the structure of $\ET$-reductions in normal $\ET$-traces matches closely the structure of 
%transitions in executions of  programs under $\ET$. 
%In fact, in the latter a transition of the form $(\mkvs, \vienv, \thdenv ), \prog \toCMD{(\cl, \vi, \fp)}_{\ET} ( \mkvs', \vienv', \thdenv' ) , \prog' )$ 
%corresponds to client $\cl$ shifting its view to $\vi$, and then committing a transaction with 
%fingerprint $\fp$. 
%Converting a $\ET$-trace in normal form is a necessary step towards proving \cref{thm:ettraces2sem}. 
\begin{proposition}[Normal \( \ET \) Traces]
\label{prop:et.normalform}
Let $\ET$ be an execution test, and suppose that $\mkvs \in \CMs(\ET)$. Then there exists a $\ET$-trace  
\[
(\mkvs_0, \vienv_0) \toET{\stub} \cdots \toET{\stub} (\mkvs_n, \vienv_{n})
\]
that is in normal form, and such that $\mkvs_{n} = \mkvs$.
\end{proposition}
%
%The proof \cref{prop:et.normalform}, which we will illustrate shortly, relies on constructing
%repeatedly re-arranging the view-shifts appearing in a $\ET$-trace $\tau$, while preserving 
%the order in which fingerprints are committed in $\tau$, as well as the view that clients use 
%to commit such fingerprints. To achieve this, we identify two particular transformations over 
%$\ET$-traces: view-shift absorptions and view-shift  right-moves. 
%
%\begin{lemma}[View-shift Absorption]
%\label{lem:et.absorb}
%If $\conf \toET{(\cl, \varepsilon)} \conf' \toET{(\cl, \varepsilon)} \conf''$, then 
%$\conf \toET{(\cl, \varepsilon)} \conf''$.
%\end{lemma}
%
%\begin{proof}
%Let $\conf = (\mkvs, \vienv)$, $\conf' = (\mkvs', \vienv')$, $\conf'' = (\mkvs', \vienv'')$. 
%By the reduction it must be the case that $\mkvs = \mkvs'$, and $\vienv' = \vienv\rmto{\cl}{\vi'}$ 
%for some $\vi' : \vi \sqsubseteq \vi'$. It must also be the case that $\mkvs' = \mkvs''$, and $\vienv'' = \vienv'\rmto{\cl}{\vi''}$ 
%for some $\vi'': \vi' \sqsubseteq \vi''$. Therefore we have that $\mkvs'' = \mkvs' = \mkvs$, and 
%$\vienv'' = \vienv'\rmto{\cl}{\vi''} = (\vienv\rmto{\cl}{\vi'})\rmto{\cl}{\vi''} = \vienv\rmto{\cl}{\vi''}$, 
%and $\vi \sqsubseteq \vi''$. 
%It follows that $\conf = (\mkvs, \vienv) \toET{(\cl, \varepsilon)} (\mkvs'', \vienv'') = \conf''$.
%\end{proof}
%
%\begin{lemma}[View-shift Swap]
%\label{lem:viewshift.rightmover}
%Let $\conf \toET{(\cl, \varepsilon)} \conf_1 \toET{(\cl', \alpha)} \conf'$ 
%for some $\conf, \conf_1, \conf''$ and $\cl, \cl'$ such that $\cl' \neq \cl$. 
%Then $\conf \toET{(\cl', \alpha)} \conf_2 \toET{(\cl, \varepsilon)} \conf'$ 
%\end{lemma}
%
%\begin{proof}
%We only consider the case where $\alpha = \fp$ for some fingerprint $\fp$. The case where 
%$\alpha = \varepsilon$ is simpler to prove.
%Let $\conf = (\mkvs, \vienv)$, $\conf_1 = (\mkvs_1, \vienv_1)$, $\conf' = (\mkvs', \vienv')$. 
%Let also $\vi = \vienv(\cl)$.
%By the reduction rule we have that $\mkvs_1 = \mkvs, \vienv_1 = \vienv\rmto{\cl}{\vi_1}$ for 
%some $\vi_1: \vi_1 \sqsubseteq \vi_1$. Let $\vi' = \vienv(\cl')$: then we have that $\vienv_1(\cl') = 
%\vi'$. Because $(\mkvs_1, \vienv_1) \toET{(\cl', \fp)} (\mkvs', \vienv')$, we have that 
%$\ET \vdash (\mkvs_1, \vi') \csat \fp : (\mkvs',\vi'') $, where $\vi'' = \vienv'(\cl')$. Because $\mkvs_1 = \mkvs$, 
%that means that $\ET \vdash (\mkvs, \vi') \csat \fp: (\mkvs', \vi'')$, then it follows that 
%$(\mkvs, \vienv) \toET{(\cl', \fp)} (\mkvs', \vienv\rmto{\cl'}{\vi''}) 
%\toET{(\cl, \varepsilon)} (\mkvs', \vienv\rmto{\cl'}{\vi''}\rmto{\cl}{\vi_1}) = 
%(\mkvs', \vienv\rmto{\cl}{\vi_1}\rmto{\cl'}{\vi''}) = (\mkvs', \vienv_1\rmto{\cl'}{\vi''}) = 
%(\mkvs', \vienv')$, as we wanted to prove.
%\end{proof}

\begin{proof}
Let $\mkvs \in \CMs(\ET)$. By definition, there exists a $\ET$-trace  
\begin{equation}
\label{eq:normalform.sequence}
\tau = (\mkvs_{0}, \vienv_{0}) \toET{(\cl_0, \alpha_0)} \cdots \toET{(\cl_{n-1}, \alpha_{n-1})} (\mkvs_n, \vienv_{n})
\end{equation}
such that $\mkvs_{n} = \mkvs$. 
We show that there exists a $\ET$-trace 
\[
\tau' = (\mkvs'_{0} \vienv'_{0}) \toET{(\cl'_{0}, \alpha'_{0})} \cdots \toET{(\cl'_{m-1}, \alpha'_{m-1})} (\mkvs'_{m}, \vienv'_{m})
\]
that is in normal form, such that $\mkvs'_{m} = \mkvs$ and for any client $\cl \in \dom(\vienv_{n})$, $\vienv'_{m}(\cl) \sqsubseteq 
\vienv_{n}(\cl)$. The proof is done by induction on the length $n$ of $\tau$. 
\begin{itemize}
\item Base case: $n = 0$. In this case $\tau = (\mkvs, \vienv_{0})$, which by definition is in normal form. Then it 
suffices to choose $\tau' = \tau$; also, for any client $\cl \in \dom(\vienv_{0})$, the inequality $\vienv_{0}(\cl) \sqsubseteq 
\vienv_{0}(\cl)$ trivially holds, 
\item Inductive case: let $n > 0$, and consider the fragment of $\tau$ given by 
\[
\tau'' = (\mkvs_{0}, \vienv_{0}) \toET{(\cl_0, \alpha_0)} \cdots \toET{(\cl_{n-2}, \alpha_{n-2})} (\mkvs_{n-1}, \vienv_{n-1}).
\]
By inductive hypothesis, there exists a trace $\tau'''$ of the form 
\[
\tau''' = (\mkvs'_{0} \vienv'_{0}) \toET{(\cl'_{0}, \alpha'_{0})} \cdots \toET{(\cl'_{m-1}, \alpha'_{j-1})} (\mkvs'_{j}, \vienv'_{j})
\]
that is in normal form, and such that $\mkvs'_{j} = \mkvs_{n-1}$, for any $\cl \in \dom(\vienv_{n-1})$, $\vienv'_{j}(\cl) \sqsubseteq 
\vienv_{n-1}(\cl)$. Next, consider the transition $(\mkvs_{n-1}, \vienv_{n-1}) \toET{(\cl_{n-1}, \alpha_{n-1})} (\mkvs_{n}, \alpha_{n})$. 
There are two different cases, according to the structure of the action $\alpha_{n-1}$: 
\begin{itemize}
\item $\alpha_{n-1} = \varepsilon$, or $\alpha_{n-1} = \emptyset$. In this case, $\mkvs_{n} = \mkvs_{n-1}$, and $\vienv_{n} = \vienv_{n-1}\rmto{\cl_{n-1}}{\vi}$, 
for some $\vi : \vienv_{n-1}(\cl_{n-1}) \sqsubseteq \vi$. In particular, we have that $\dom(\vienv_{n-1}) = \dom(\vienv_{n})$, 
and for any $\cl \in \dom(\vienv_{n-1})$, then $\vienv_{n-1}(\cl) \sqsubseteq \vienv_{n}(\cl)$. In this case, we let $\tau' := \tau'''$. 
By the inductive hypothesis, we have that $\mkvs'_{j} = \mkvs_{n-1} = \mkvs$, and for any $\cl \in \dom(vienv_{n})$, 
$\vienv'_{j}(\cl) \sqsubseteq \vienv_{n-1}(\cl) \sqsubseteq \vienv_{n}(\cl)$, 
\item $\alpha_{n-1} = \fp$ for some $\fp \neq \emptyset$. 
Let $\vi_{n-1} = \vienv_{n-1}(\cl_{n-1})$, $\vi_{n} = \vienv_{n}(\cl_{n-1})$. Then \textbf{(a)} there exists $\txid \in \nextTxid(\cl, \mkvs)$ 
such that $\mkvs_{n} = \updateKV(\mkvs_{n-1}, \vi_{n-1}, \fp, \txid)$, $\ET \vdash (\mkvs_{n-1}, \vi_{n-1}) \triangleright \fp: (\mkvs_{n}, \vi_{n})$ 
and $\vienv_{n} = \vienv_{n-1}\rmto{\cl_{n-1}}{\vi_{n}}$. 
We also have that \textbf{(b)} $\vienv_{n-1}(\cl)$ is defined and equal to $\cl$, which means that $\vienv'_{j}(\cl) \sqsubseteq \vienv_{n-1}$. 
Using \textbf{(a)} and \textbf{(b)} above, and knowing that $\mkvs'_{j} = \mkvs_{n-1}$ by induction,   we can extend 
$\tau'''$ with the sequence of $\ET$-reduction 
\[ (\mkvs'_{j}, \vienv'_{j}) = (\mkvs_{n-1}, \vienv'_{j}) 
 \toET{(\cl_{n-1}, \varepsilon)} (\mkvs_{n}, \vienv'_{j}\rmto{\cl_{n-1}}{\vi_{n-1}}) \toET{(\cl_{n-1}, \fp)} (\mkvs_{n}, \vienv'_{j}\rmto{\cl_{n-1}}{\vi_{n}}.
 \]
Let $\tau''$ be the resulting trace, and let $(\mkvs'', \vienv'')$ be the last configuration of $\tau''$. 
Obviously, $\mkvs'' = \mkvs_{n}$. Furthermore, let $\cl \in \dom(\vienv_{n})$. Then either $\cl = \cl_{n-1}$, 
in which case $\vienv_{n}(\cl_{n-1}) = \vi_{n} = \vienv''(\cl_{n-1})$, or 
$\vienv_{n}(\cl) = \vienv_{n-1}(\cl) \sqsubseteq \vienv'_{j}(\cl) = \vienv''(\cl)$. That is, for any $\cl \in \dom(\vienv_{n})$, 
$\vienv''(\cl) \sqsubseteq \vienv_{n}(\cl)$, and there is nothing left to prove.
\end{itemize}

\end{itemize}

\end{proof}

The second problem that we face, when proving \cref{prop:ettraces2sem}, 
is that for any $\ET$-trace $\tau$ that is in normal form and 
that terminates in the configuration $(\mkvs, \_)$, we need to define a program 
$\prog_{\tau}$ which has an execution terminating in the same configuration. 
To this end, we first map fingerprints into transactional code, and then we lift 
this mapping to $\ET$-traces and programs. 

\begin{definition}
\label{def:fp2prog}
\ac{This mapping assumes that fingerprints are finite. This assumption was lifted when we 
decided to admit the initial transaction into abstract executions, which leads to such a transaction 
having an infinite fingerprint. It may be useful to state that $\txid_{0}$ is the only transaction that is 
allowed to have an infinite fingerprint.}

Note that any fingerprint set 
$\fp$ is either the empty set $\fp$; or it has the form $\{(\otR, \key, \val)\} \uplus \fp'$ for 
some $\fp'$ such that $(\otR, \key, \val') \notin \fp'$ for any $\val' \in \Val$; 
or it has the form $\{(\otW, \key, \val)\} \uplus \fp'$ for some $\key, \val$ such that 
$(\otW, \key', \val') \notin \fp'$ for any $\key' \in \Keys, \val' \in \Val$, and 
$(\otR, \key, \val') \notin \fp'$, for any $\val' \in \Val$.  

Given a (finite) fingerprint $\fp$, we define the transactional code $\trans_{\fp}$ by induction 
on $\fp$, via an induction schema based on the characterisation of finite fingerprints given above.
\begin{itemize}
\item if $\fp = \emptyset$, then $\trans_{\fp} = \pskip$, 
\item if $\fp = \{(\otR, \key, \val)\} \uplus \fp'$ for some $\key, \val, \fp'$, and $(\otR,\key,\val') \notin \fp'$ for any 
$\val' \in \Val$, then $\trans_{\fp'} = \plookup{\_}{\key} ; \trans_{\fp'}$, 
\item if $\fp = \{(\otW, \key, \val)\} \uplus \fp'$ for some $\key, \val, \fp'$, $(\otR,\key', \val') \notin \fp'$ for all 
$\val' \in \Val$ and $\key \in \Keys$, $(\otW, \key, \val') \notin \fp'$ for all $\val' \in \Val$, 
then $\trans_{\fp} = \pmutate{\key}{\val} ; \trans_{\fp'}$.
\end{itemize}
\ac{To be pedantic, the definition above is ill-formed, as the same fingerprint can lead to multiple 
definitions of $\fp$. This should be fixed by defining a set of transactional commands for 
fingerprints, and prove the necessary properties for all the transactional commands induced 
by a fingerprint.}

Consider an $\ET$-trace $\tau$ that is in normal form and that terminates in the configuration $(\mkvs, \vienv)$. Let $C = \dom(\vienv)$. 
We define the program $\prog_{\tau}$ by induction on the structure of $\tau$. 
\begin{itemize}
\item if $\tau = (\mkvs_{0}, \vienv_{0})$, i.e. it has no transitions, then $\prog_{\tau} = \lambda \cl \in C. \pskip$, 
\item if $\tau = \tau' \toET{(\cl, \varepsilon)} (\mkvs', \vienv') \toET{(\cl, \fp)} (\mkvs, \vienv)$, then 
$\prog_{\tau} = \prog_{\tau'}\rmto{\cl}{\prog_{\tau'}(\cl)}; \ptrans{\trans_{\fp}}$.
\end{itemize}
\end{definition}

Clearly, the command $\trans_{\fp}$  is defined in a way that its 
execution from a client whose local state agrees with the read operations 
of $\fp$,  leads to the fingerprint $\fp$ itself. More formally, we have the 
following result: 
\begin{lemma}
\label{lem:fp2trans}
Let $\fp$ be an arbitrary, finite fingerprint, and let $\sn$ be such that, whenever 
$(\otR, \key, \val) \in \fp$, then $\sn(\key) = \val$. Then, for any $\stk \in \Vars \rightarrow \Val$,
\[
(\stk, \sn, \emptyset), \trans_{\fp} \trans_{\fp} \toTRANS^{\ast} (\stub, \stub, \fp), \pskip
\]
\end{lemma}

\begin{proof}
The proof of \cref{lem:fp2trans} is by induction on the structure $\fp$, via an 
induction schema that follows the characterisation of fingerprints from \cref{def:fp2prog}.

If $\fp = \emptyset$ then $\trans_{\fp} = \pskip$, and $(\stk, \sn, \emptyset), 
\pskip \toTRANS^{\ast} (\stk, \sn, \emptyset)$. 
Suppose then that $\fp = \{(\otR, \key, \val)\} \uplus \fp'$, and that 
$(\otR, \key, \val') \notin \fp'$ for any $\val' \in \Val$. 
Suppose also that $\sn$ is such that whenever $(\otR, \key', \val') \in \fp$ then $\sn(\key') = \val'$. 
In particular, because $(\otR,\ ke, \val) \in \fp$, then $\sn(\key) = \val$; and for any $(\otR,\ ke', \val') \in \fp'$, 
then $\sn(\key') = \val'$. 
By definition, $\trans_{\fp} = \plookup{\_}{\key} ; \trans_{\fp'}$. By applying Rule \rl{Tprimitive} 
we obtain the derivation 
\[
(\_, \sn, \emptyset), \plookup{\_}{\key} ; \trans_{\fp'} \toTRANS (\_, \sn, \{(\otR, \key, \val)\}) \trans_{\fp'}
\]
Also,because whenever $(\otR,\ ke', \val') \in \fp'$, 
then $\sn(\key') = \val'$,  we can apply the inductive hypothesis and infer that $(\_, \sn, \emptyset), \trans_{\fp'} \toTRANS^{\ast} (\stub, \stub, \fp'), \pskip$. 
\ac{To be pedantic the result below should be proved: it requires first a rule induction on 
transitions of the form $(\stk, \sn, \fp), \trans \toTRANS (\stk', \sn', \fp'), \trans'$, followed by an induction on the 
number of transitions in a judgement of the form $(\_, \sn, \fp), \trans^{\ast} \toTRANS (\_, \_, \fp'), \trans' $.}
An immediate consequence of this fact is that 
$(\_, \sn, \{(\otR,\key, \val)\}) \trans_{\fp'} \toTRANS^{\ast} (\stub \stub, \{(\otR,\ ke, \val)\} \uplus \fp'), \pskip$. 
That is, we have that 
\[
(\_, \sn, \emptyset), \plookup{\_}{\key} ; \trans_{\fp'} \toTRANS (\_, \_, \{(\otR, \key, \val)\}) \trans_{\fp'} \toTRANS^{\ast} (\_, \_, \fp'), \pskip.
\]

Finally, suppose that $\fp = \{(\otW, \key, \val)\} \uplus \fp'$, and 
for no  $\key', \val'$ $(\otR,\key', \val') \in \fp'$, and for no $\val'$ 
$(\otW, \key, \val') \in \fp'$. In this case we have that $\trans_{\fp} = 
\pmutate{\key}{\val} \uplus \fp'$. Let $\sn$ be an arbitrary snapshot. 
By applying Rule \rl{Tprimitive} we obtain that $(\_, \sn, \emptyset) \pmutate{\key}{\val} ; \trans_{\fp'} 
\toTRANS (\_, \sn\rmto{\key}{\val}, \{(\otW, \key, \val)\}), \trans_{\fp'}$. Because $\fp'$ does not contain 
any operation of the form $(\otR, \key', \val')$, the statement \emph{if $(\otR,\key',\val') \in \fp'$ then 
$\sn\rmto{\key}{\val}(\key') = \val'$} is trivially satisfied. By the inductive hypothesis, we have that 
$(\_, \sn\rmto{\key}{\val}, \emptyset), \trans_{\fp'} \toTRANS^{\ast} (\_, \_, \fp'), \pskip$, 
from which it is immediate to infer that 
\[
(\_, \sn\rmto{\key}{\val}, \{(\otW, \key, \val)\}), \trans_{\fp'} \toTRANS^{\ast} (\_, \_, \{(\otW, \key, \val)\} \uplus \fp'), \pskip. 
\]
Finally, we have that 
\[
(\_, \sn, \emptyset), \pmutate{\key}{\val} ; \trans_{\fp'} \toTRANS (\_, \_, \{(\otW, \key, \val)\}), \trans_{\fp'} \toTRANS^{\ast} (\_, \_, \fp), \pskip
\]
as we wanted to prove.
\end{proof}

We now have everything we need to prove \cref{prop:ettraces2sem}.
\paragraph{Proof of \cref{prop:ettraces2sem}.}
Let $\mkvs \in \CMs(\ET)$. By definition there exists a $\ET$-trace $\tau$ that 
terminates in a configuration of the form $(\mkvs, \_)$. By \cref{prop:et.normalform}, we 
can assume that $\tau$ is in normal form, hence the program $\prog_{\tau}$ is well-defined. 
In particular, let 
\[
\tau = (\mkvs_{0}, \vienv_{0}), \toET{(\cl_{0}, \varepsilon_{0})} (\mkvs_{0}', \vienv_{0}') \toET{(\cl_{0}, \fp_{0})} (\mkvs_{1}, \vienv_{1}) 
\toET{(\cl_{2}, \varepsilon_{2})} \cdots \toET{(\cl_{n-1}, \fp_{n-1})} (\mkvs_{n}, \vienv_{n}),
\]
where $\mkvs_{n} = \mkvs$. Then for any $i =0,\cdots, n-1$, we have that $\mkvs_{i} = \mkvs'_{i}$, 
and $\vienv'_{i} = \vienv_{i}\rmto{\cl_{i}}{\vi_{i}}$ for some $\vi_{i}. \vienv_{i}(\cl_{i}) \sqsubseteq \vi_{i}$.
Let $\mathsf{Cl}_{\tau} = \dom(\vienv_{0})$, and $\thdenv_{0} = \lambda \cl \in \mathsf{Cl}_{\tau}. \lambda x \in \Vars. \val_{0}$. 
For $i=0,\cdots, n$, let $\tau_{i}$ be the prefix of $\tau$ terminating in the configuration $(\mkvs_{i}, \vienv_{i})$.
We prove that, for any $i=0,\cdots, n-1$, 
then $(\mkvs_{0}, \vienv_{0}), \thdenv_{0}), \prog_{\tau_{i}} \toCMD{\stub}^{\ast}_{\ET} (\mkvs_{i}, \vienv_{i}, \thdenv_{i}), (\lambda \cl \in \mathsf{Cl}_{\tau}.\pskip)$, 
for some $\thdenv_{i}$. By definition $\mkvs = \mkvs_{n} \in \interpr{\prog_{\tau}}_{\ET}$. 
The proof of this claim is by induction on $n$.

\begin{itemize}
\item The base case where  $n = 0$, as $\tau_{0}$ consists of the initial configuration $(\mkvs_{0}, \_)$. 
By definition $\prog_{\tau_{0}}(\cl) = \lambda \cl \in \mathsf{Cl}_{\tau}.\pskip$, and there is nothing to prove.

\item Next let $i: 0 < i < n$, and suppose that the claim holds for $\tau_{i-1}$: specifically, there exists 
$\thdenv_{i}$ such that $(\mkvs_{0}, \vienv_{0}, \thdenv_{0}), \prog_{\tau_{i-1}} \toCMD{-}^{\ast}_{\ET} 
(\mkvs_{i}, \vienv_{i}, \thdenv_{i}), \lambda \cl \in \dom(\mathsf{Cl}_{\tau}).\pskip$. Note that 
$\prog_{\tau_{i}} = \prog_{\tau_{i-1}}\rmto{\cl_{i-1}}{\prog_{\tau_{i-1}}(\cl_{i-1}) ; \ptrans{\trans_{\fp_{i-1}}}}$, 
from which it follows that 
\[
(\mkvs_{0}, \vienv_{0}, \thdenv_{0}), \prog_{\tau_{i}} \toCMD{-}^{\ast}_{\ET} 
(\mkvs_{i}, \vienv_{i}, \thdenv_{i}), (\lambda \cl \in \dom(\mathsf{Cl}_{\tau}).\pskip)\rmto{\cl_{i-1}}{\ptrans{\trans_{\fp_{i-1}}}}.
\]
Next, we prove that $\cl_{i} \vdash (\mkvs_i, \vienv_{i}(\cl_{i}), \thdenv_{i}(\cl_{i}), \ptrans{\fp_{i}} 
\toCMD{(\cl_{i-1}, \_, \fp_{i-1})} (\mkvs_{i}, \vienv_{i}(\cl_{i-1}), \_), \pskip$, 
from which we can infer 
\[
(\mkvs_{i}, \vienv_{i}, \thdenv_{i}), (\lambda \cl \in \dom(\mathsf{Cl}_{\tau}).\pskip)\rmto{\cl_{i}}{\ptrans{\trans_{\fp_{i}}}} 
\toPROG{(\cl_{i-1}, \_, \fp_{i-1})} (\mkvs_{i+1}, \vienv_{i+1}, \_), \lambda \cl \in \dom(\mathsf{Cl}_{\tau}.\pskip),
\]
which is what we want to prove. 

To prove that $\cl_{i} \vdash (\mkvs_i, \vienv_{i}(\cl_{i}), \thdenv_{i}(\cl_{i}), \ptrans{\fp_{i}} 
\toCMD{(\cl_{i-1}, \_, \fp_{i-1})} (\mkvs_{i}, \vienv_{i}(\cl_{i-1}), \_), \pskip$, 
we consider the sequence of $\ET$-reductions $(\mkvs_{i-1}, \vienv_{i-1}) \toET{(\cl_{i-1}, \varepsilon} (\mkvs_{i-1}', \vienv_{i-1}') 
\toET{(\cl_{i-1}, \fp_{i-1})} (\mkvs_{i}, \fp_{i})$. The $\ET$-reduction ensures the following: 
\begin{itemize}
\item let $\vi_{i-1} = \vienv_{i-1}(\cl_{i-1})$. Then there exists $\vi_{i-1}'$ such that $\vi_{i-1} \sqsubseteq \vi_{i-1}'$ and
$\vienv_{i-1}'('cl_{i-1}) = \vienv_{i-1}\rmto{\cl_{i-1}}(\vi_{i-1}')$, 
\item $\mkvs_{i-1}' = \mkvs_{i-1}$.
\end{itemize}
The $\ET$-reduction $(\mkvs_{i-1}', \vienv_{i-1}') 
\toET{(\cl_{i-1}, \fp_{i-1})} (\mkvs_{i}, \fp_{i})$ ensures the following: 
\begin{itemize}
\item $\mkvs_i = \updateKV(\mkvs_{i-1}, \vi_{i-1}', \fp_{i-1}, \txid)$ for some $\txid \in \nextTxid(\cl_{i-1}, \mkvs_{i-1}$, 
\item $\ET \vdash (\mkvs_{i-1}, \vi_{i-1}') \triangleright \fp_{i-1} (\mkvs_{i}, \vi_{i}$, where $\vi_{i} = \vienv_{i}(\cl_{i-1})$; 
by the definition of execution test, this also ensures that 
\item whenever $(\otR, \key, \val) \in \fp_{i-1}$, then $\snapshot(\mkvs_{i-1}, \vi_{i-1}')(\key) = \val$.
\end{itemize}
The latter statement, together with \cref{lem:fp2trans}, leads to the following: let $\stk_{i-1} = \thdenv_{i-1}(\cl_{i-1}$, 
$\sn_{i-1} = \snapshot(\mkvs_{i-1}, \vi_{i-1}')$; then 
$(\stk_{i-1}, \sn_{i-1}, \emptyset), \trans_{\fp_{i-1}} \toTRANS^{\ast} (\_, \_, \fp_{i-1}) \pskip$. 
By putting all the statements together, we infer the transition 
\[\cl_{i} \vdash (\mkvs_i, \vienv_{i}(\cl_{i}), \thdenv_{i}(\cl_{i}), \ptrans{\fp_{i}} 
\toCMD{(\cl_{i-1}, \_, \fp_{i-1})} (\mkvs_{i}, \vienv_{i}(\cl_{i-1}), \_), \pskip,\]
which is exactly 
what we wanted to prove.
\end{itemize}
%
%\ac{Old Appendix starts here}
%
%\[
%\conf_0 \toET{(\cl_{0}, \alpha_{0})} \conf_{1} \toET{(\cl_{1}, \alpha_1)} \cdots \toET{(\cl_{n-1}, \alpha_{n-1})} \conf_{n},
%\] 
%where for any $i=0,\cdots, n$, $\conf_{i} = (\mkvs_{i}, \vienv_{i})$ for some $\mkvs_{i},\vienv_{i}$, 
%can be converted into a \emph{normal trace}, where a view shift from
%
%\ac{Context is missing. What's the purpose of this appendix? 
%Normal $\ET$-traces is not enough. There should be an introductory 
%paragraph. Try to point to the main text. E.g. \emph{In \cref{sec:?} we 
%stated that $\cdots$. Here we give a formal proof of the result.} 
%Then you give an high level outline of the proof. E.g. \emph{First, we prove 
%that $\ET$-traces can be converted into a normal form. Then we show how to convert 
%$\ET$-traces in normal form into a trace of a program $\prog$ in the operational semantics.}}
%
%\ac{I believe that the aim of this Section is that of proving Theorem 4.4. This proof is not in the appendix 
%as of now, but it follows immediately from the results that we already have. I will put the structure of the 
%proof at the end of this section.}
%
%For technical reasons, it will be convenient to adopt a reduction strategy for inferring kv-stores induced by an 
%execution test: such an execution strategy require that clients only commit transactions with non-empty fingerprints, 
%and a client updates its view only immediately before committing a transaction. 
%The next proposition states that all kv-stores induced by an execution test $\ET$ can be 
%obtained via a sequence of reductions that adhere to the reduction strategy outlined above. 
%Throughout this section, we assume that the execution test $\ET$ is fixed.
%
%\begin{definition}
%Let $\ET$ be an execution test. The $\ET$-trace
%\[
%\conf_0 \toET{\alpha_0} \conf_1 \toET{\alpha_1} \cdots \toET{\alpha_{2n}} \conf_{2n + 1}
%\]
%is in \emph{normal form} if \textbf{(i)} $\conf_0$ is initial, and 
%\textbf{(ii)} $\fora{ i : 0 \leq i \leq n}$ there exists a client $\cl_i$ and set of operations $\fp_{i}$ such that 
%$\alpha_{2 \cdot i} = (\cl_{i}, \varepsilon)$, and $\alpha_{2 \cdot i + 1}$ is defined and equal to $(\cl_{i}, \fp_{i})$ where \( \fp_i \neq \emptyset \).
%\end{definition}
%\ac{In the main text $\alpha$ ranges over $\{\varepsilon\} \cup \{\fp_{i}\}$, but here it ranges over pairs 
%of the form $\{(\cl, \varepsilon)\} \cup \{(\cl, \fp_{i})\}$, while $\mu$ ranges over $\{\varepsilon\} \cup \{\fp_{i}\}$. 
%Fix.}
%\ac{In the explicit sequence of reductions above, the last index on $\alpha$ is $2 \cdot n$, but the definition 
%mandates that the last index on $\alpha$ must have the form $2\ cdot i + 1$ for some $i$. Fix.
%}
%
%For any trace satisfying \( \CMs(\ET) \), 
%there exists a equivalent normal trace ends up with the same state (\cref{prop:et.normalform}).
%
%\ac{The proposition below uses several side results (see items (1) and (2) below, and the notion of 
%transitions in place. The best way to present the result clearly is to \textbf{(a)} state the main result 
%that you want to prove, i.e. \cref{prop:et.normalform}. \textbf{(b)} Give an intuition about how the 
%proof is performed (i.e. you transform the original $\ET$-reduction by merging view-shifts and pushing client commits to the left of view-shifts from 
%different clients, see lemmas \ref{lem:et.absorb} and \ref{lem:viewshift.rightmover}). 
%\textbf{(c)} Once you have introduced these two lemmas, give their proof straight away.
%\textbf{(d)} Having a high-level overview of what you are doing you can introduce  
%in-place and out of place reductions. At this point it should be clear that a reduction in-place is one that does 
%not need to be  moved around anymore, when constructing a normal $\ET$-trace. 
%\textbf{(e)} Give the proof of \cref{prop:et.normalform}} 
%\begin{proposition}[Normal \( \ET \) Traces]
%\label{prop:et.normalform}
%Let $\ET$ be an execution test, and suppose that $\mkvs \in \CMs(\ET)$. Then there exists a $\ET$-trace  
%\[
%(\mkvs_0, \vienv_0) \toET{\stub} \cdots \toET{\stub} (\mkvs_n, \vienv_{n})
%\]
%that is in normal form, and such that $\mkvs_{n} = \mkvs$.
%\end{proposition}
%\begin{proof}
%Let $\mkvs \in \CMs(\ET)$. By definition, there exists a sequence of reductions 
%\begin{equation}
%\label{eq:normalform.sequence}
%(\mkvs_{0}, \vienv_{0}) \toET{(\cl_0, \mu_0)} \cdots \toET{(\cl_{n-1}, \mu_{n-1})} (\mkvs_n, \vienv_{n})
%\end{equation}
%such that $\mkvs_{n} = \mkvs$. Given an index $i = 1,\cdots, n-1$, we say that the action $(\cl_{i}, \mu_{i})$ is \emph{in place} 
%if, $\mu_{i} = \fp_{i}$ for some $\fp_{i}$, $\cl_{i-1} = \cl_{i}$, $\mu_{i-1} = \varepsilon$, and if $(\cl_{j}, \mu_{j}) = (\cl_{i}, \varepsilon)$, 
%for some  $j = 0,\cdots, i-2$, then there exists $j': j < j' < i$ such that $(\cl_{j'}, \mu_{j'}) = (\cl_i, \fp_{j'})$. An action of the 
%form $(\cl_{i}, \mu_{i})$ is \emph{out of place} if it is not in place. 
%
%Given the sequence of reductions in \cref{eq:normalform.sequence}, we show the following: 
%\begin{enumerate}
%\item if the sequence has no action out of place, then there exists a sequence 
%\[
%(\mkvs'_{0}, \vienv'_{0}) \toET{(\cl'_{0}, \mu'_{0})} \cdots \toET{(\cl'_{m-1}, \mu'_{m-1})} (\mkvs'_{m}, \vienv'_{m})
%\]
%that is in normal form, and such that $\mkvs'_{m} = \mkvs_{n}$, and 
%\item if the sequence has $h$ actions out of place, for some $h > 0$, then there exists a sequence 
%\[
%(\mkvs'_{0}, \vienv'_{0}) \toET{(\cl'_{0}, \mu'_{0})} \cdots \toET{(\cl'_{m-1}, \mu'_{m-1})} (\mkvs'_{m}, \vienv'_{m})
%\]
%that has $h-1$ actions out of place, and such that $\mkvs'_{m} = \mkvs_{n}$.
%\end{enumerate}
%Combining the two facts above, we obtain that if $\mkvs \in \CMs(\ET)$, then there exists a sequence of reductions in formal form whose final 
%configuration is $(\mkvs, \_)$, as we wanted to prove.
%
%\begin{enumerate}
%\item 
%Suppose that the sequence of reductions from \cref{eq:normalform.sequence} has no action out of place. 
%Let $i=0,\cdots, n-1$, and consider the greatest index $i=0,\cdots, n-1$ such that  
%$\mu_{i} = \varepsilon$, and either $i = n-1$, or 
%$\fora{ \fp } (\cl_{i+1}, \mu_{i+1}) \neq (\cl_{i}, \fp)$. 
%If such an index does not exist, then the sequence of transitions from \cref{eq:normalform.sequence} is in 
%normal form, and there is nothing to prove. Otherwise, note that for any $j = i+1,\cdots, n-1$, 
%$\fora{ \fp } (\cl_{j}, \mu_{j}) \neq (\cl_{i}, \fp)$. 
%
%Suppose in fact that there existed 
%an index $j = i+1,\cdots, n-1$ such that $(\cl_{j}, \mu_{j}) = (\cl_{i}, \fp_{j})$ for some 
%$\fp_{j}$, and without loss of generality assume that $j$ is the smallest such index. This implies that 
%there exists no index $j': i < j' < j$ such that $(\cl_{j'}, \mu_{j'}) = (\cl_{i}, \fp_{j'})$ for some 
%$\fp_{j'}$. Also, it cannot be $j = i+1$, because we are assuming that $\fora{ \fp } (\cl_{i+1}, \mu_{i+1}) \neq 
%(\cl_{i}, \fp)$.  We have that $j \geq i+2$; we also have that  $(\cl_{j}, \mu_{j}) = (\cl_{i}, \fp_{j})$, 
%$(\cl_{i}, \mu_{i}) = (\cl_{i}, \varepsilon)$, $\fora{ j': i < j < j', \fp } (\cl_{j'}, \mu_{j'}) \neq (\cl_{i}, \fp)$. 
%By definition, the action $(\cl_{j}, \mu_{j})$ is out of place, contradicting the assumption that the sequence of 
%reduction of \cref{eq:normalform.sequence} has no actions out of place.
%
%We have proved that $\fora{ j : i+1 \leq j \leq n-1, \fp } (\cl_{j}, \mu_{j}) \neq (\cl_{i}, \fp)$. 
%Also, because we are assuming that $\mu_{i}$ is the greatest index such that $\mu_{i} = \varepsilon$, 
%and either $i= n-1$ or $\fora{ \fp } (\cl_{i+1},\mu_{i+1}) \neq (\cl_{i}, \fp)$, 
%then $\fora{ j : i+1 \leq j \leq n-1, \mu } (\cl_{j}, \mu_{j}) \neq (\cl_{i}, \mu)$. 
%Consider the transition 
%\[
%(\mkvs_{i}, \vienv_{i}) \toET{(\cl_{i}, \mu_{i})} (\mkvs_{i+1}, \vienv_{i+1}).
%\]
%Let $\vi = \vienv_{i}(\cl)$. Because $\mu_{i} = \varepsilon$, then it must be the case that 
%$\mkvs_{i} = \mkvs_{i+1}$, $\vienv_{i+1} = \vienv_{i}\rmto{\cl}{\vi'}$ for some $\vi' : \vi \sqsubseteq \vi'$. 
%For any $j \geq i$, we have that $\cl_{j} \neq \cl_{i}$. We can replace the transition 
%\[
%(\mkvs_{j}, \vienv_{j}) \toET{(\cl_{j}, \mu_{j})} (\mkvs_{j+1}, \vienv_{j+1})
%\]
%with 
%\[
%(\mkvs_{j}, \vienv_{j}\rmto{\cl_{i}}{\vi}) \toET{(\cl_{j}, \mu_{j})} (\mkvs_{j+1}, \vienv_{j+1}\rmto{\cl_{i}}{\vi}.
%\]
%It follows that the sequence of transitions 
%\[ 
%\begin{array}{@{}l@{}}
%(\mkvs_{0}, \vienv_{0}) \toET{(\cl_{0}, \mu_{0})} \cdots \toET{(\cl_{i-1},\mu_{i-1})} 
%(\mkvs_{i}, \vienv_{i}) = (\mkvs_{i+1}, \vienv_{i+1}\rmto{\cl_{i}}{\vi})  \\
%\quad \toET{(\cl_{i+1}, \mu_{i+1})} \cdots 
%\toET{(\cl_{n-1}, \mu_{n-1})} (\cl_{n}, \vienv_{n}\rmto{\cl_{i}, \vi})
%\end{array}
%\]
%Note that this sequence has one reduction less than the original sequence from \eqref{eq:normalform.sequence} (specifically, 
%the reduction $(\mkvs_{i}, \vienv_{i}) \toET{(\cl_{i}, \mu_{i})} (\mkvs_{i+1}, \vienv_{i+1})$ has 
%been removed). We can repeat this procedure until the resulting sequence of reductions has no index $i=0,\cdots, n-1$ such that  
%$\mu_{i} = \varepsilon$, and either $i = n-1$, or 
%$\fora{ \fp } (\cl_{i+1}, \mu_{i+1}) \neq (\cl_{i}, \fp)$. That is, the resulting sequence of reductions is in normal form, 
%and its final configuration is $(\mkvs_{n}, \_)$.
%
%\item Suppose that the sequence from \cref{eq:normalform.sequence} has $h$ actions out of place, 
%where $h > 0$. Let $i$ be the smallest index such that $(\cl_{i}, \mu_{i})$ is out of place. 
%This means that either $i = 0$, or $(\cl_{i-1}, \mu_{i-1}) \neq (\cl_{i}, \varepsilon)$, 
%or there exists an index $j < i -1 $ such that $(\cl_{j}, \mu_{j}) = (\cl_{i}, \varepsilon)$ 
%and, $\fora{ j': j < j' < i, \fp } (\cl_{j'}, \mu_{j'}) \neq (\cl_{i}, \fp)$. 
%Without loss of generality, we can assume that $i \neq 0$ and $(\cl_{i-1}, \mu_{i-} = (\cl_{i}, \varepsilon)$. 
%This is because we can always transform the sequence of reductions of \cref{eq:normalform.sequence} by 
%introducing a transition of the form $(\mkvs_{i}, \vienv_{i}) \toET{(\cl_{i}, \varepsilon)}
%(\mkvs_{i}, \vienv_{i})$, leading to the sequence of reductions
%\[
%\begin{array}{@{}l@{}}
%(\mkvs_{0}, \vienv_{0}) \toET{(\cl_{0}, \mu_{0})} \cdots \toET{(\cl_{i-1}, \mu_{i-1})}
%(\mkvs_{i}, \vienv_{i}) \\
%\quad \toET{(\cl_{i}, \varepsilon)} (\mkvs_{i}, \vienv_{i}) \toET{(\cl_{i+1}, \mu_{i+1})}
%\cdots \toET{(\cl_{n-1}, \mu_{n-1})} (\mkvs_{n}, \vienv_{n})
%\end{array}
%\]
%
%Therefore, it must be the case that there exists an index $j < i-1$ such that $(\cl_{j}, \mu_{j}) = (\cl_{i}, \varepsilon)$, 
%and $\fora{ j': j< j' < i, \fp } (\cl_{j'}, \mu_{j'}) \neq (\cl_{i}, \fp)$. Let then $j$ be the smallest such index. 
%Let $d = (i-1)-j$ be the number of reductions that separate the configuration $(\mkvs_{j}, \mu_{j})$ from 
%$(\mkvs_{i-1}, \mu_{i-1})$ in \cref{eq:normalform.sequence}. Note that it must be the case that $d > 0$. We show that we can 
%construct a sequence of reductions where the distance between these two configurations is reduced to $0$: 
%a consequence of this fact is such a sequence of reductions would have exactly $h-1$ actions out of place.
%Consider the following fragment in the sequence of reductions from \cref{eq:normalform.sequence}:
%\[
%(\mkvs_{j}, \vienv_{j}) \toET{(\cl_{j}, \mu_{j})} (\mkvs_{j+1}, \vienv_{j+1}) 
%\toET{(\cl_{j+1}, \mu_{j+1})} (\mkvs_{j+2}, \vienv_{j+2})
%\]
%We have two possible cases: 
%\begin{itemize}
%\item $\cl_{j+1} \neq \cl_{j}$. In this case we can apply \cref{lem:viewshift.rightmover} and infer the sequence of 
%reductions 
%\[
%(\mkvs_{j}, \vienv_{j}) \toET{(\cl_{j+1}, \mu_{j+1})} (\mkvs_{j+1}', \vienv_{j+1}') 
%\toET{(\cl_{j}, \mu_{j})} (\mkvs_{j+1}, \vienv_{j+1})
%\]
%which leads to the whole sequence of reductions 
%\[
%\begin{array}{@{}l@{}}
%(\mkvs_{0}, \vienv_{0}) \toET{(\cl_{0}, \mu_{0})} \cdots 
%\toET{(\cl_{j-1}, \mu_{j-1})} (\cl_{j}, \mu_{j}) 
%\toET{(\cl_{j+1}, \mu_{j+1})} (\mkvs'_{j+1}, \vienv'_{j+1})  \\
%\quad \toET{(\cl_{j}, \mu_{j})} (\mkvs_{j+1}, \vienv_{j+1})  
%\toET{(\cl_{j+2}, \mu_{j+2})} \cdots \toET{(\cl_{n-1}, \mu_{n-1})} \mkvs_{n}, \vienv_{n}
%\end{array}
%\]
%\item $\cl_{j+1} = \cl_{j}$. In this case we can apply \cref{lem:et.absorb} and infer the reduction 
%\[
%(\mkvs_{j}, \vienv_{j}) \toET{(\cl_{j}, \varepsilon)} (\mkvs_{j+2}, \vienv_{j+2})
%\]
%which leads to the sequence of reductions 
%\[
%\begin{array}{@{}l@{}}
%(\mkvs_{0}, \vienv_{0}) \toEDGE{(\cl_{0}, \mu_{0})}_{\ET} \cdots 
%\toET{(\cl_{j-1}, \mu_{j-1})}[\ET] (\mkvs_{j}, \vienv_{j}) \toET{(\cl_{j}, \varepsilon)}  \\
%\quad (\mkvs_{j+2}, \vienv_{j+2}) \toET{(\cl_{j+2}, \mu_{j+2})} \cdots 
%\toET{(\cl_{n-1}, \mu_{n-1})} (\mkvs_{n}, \vienv_{n})
%\end{array}
%\]
%\end{itemize}
%In both cases, in the resulting sequence of reductions the number of reductions that separate 
%the configuration $(\mkvs_{j}, \vienv_{j})$ from $(\mkvs_{i-1}, \vienv_{i-1})$ is strictly 
%less than $d$. We can repeating applying the procedure outlined above until there are 
%no reductions that separate the configuration $(\mkvs_{j}, \vienv_{j})$ from 
%$(\mkvs_{i}, \vienv_{i})$.
%\end{enumerate}
%\end{proof}
%\ac{This was more of a proof sketch, rather than a real proof. For the moment it will suffice, though 
%I will need to go back at it when all the other results are sorted.}
%
%
%\begin{lemma}[Absorption]
%\label{lem:et.absorb}
%If $\conf \toET{(\cl, \varepsilon)} \conf' \toET{(\cl, \varepsilon)} \conf''$, then 
%$\conf \toET{(\cl, \varepsilon)} \conf''$.
%\end{lemma}
%
%\begin{proof}
%Let $\conf = (\mkvs, \vienv)$, $\conf' = (\mkvs', \vienv')$, $\conf'' = (\mkvs', \vienv'')$. 
%By the reduction it must be the case that $\mkvs = \mkvs'$, and $\vienv' = \vienv\rmto{\cl}{\vi'}$ 
%for some $\vi' : \vi \sqsubseteq \vi'$. It must also be the case that $\mkvs' = \mkvs''$, and $\vienv'' = \vienv'\rmto{\cl}{\vi''}$ 
%for some $\vi'': \vi' \sqsubseteq \vi''$. Therefore we have that $\mkvs'' = \mkvs' = \mkvs$, and 
%$\vienv'' = \vienv'\rmto{\cl}{\vi''} = (\vienv\rmto{\cl}{\vi'})\rmto{\cl}{\vi''} = \vienv\rmto{\cl}{\vi''}$, 
%and $\vi \sqsubseteq \vi''$. 
%It follows that $\conf = (\mkvs, \vienv) \toET{(\cl, \varepsilon)} (\mkvs'', \vienv'') = \conf''$.
%\end{proof}
%
%\begin{lemma}[Independence of commit]
%\label{lem:viewshift.rightmover}
%Let $\conf \toET{(\cl, \varepsilon)} \conf_1 \toET{(\cl', \mu)} \conf'$ 
%for some $\conf, \conf_1, \conf''$ and $\cl, \cl'$ such that $\cl' \neq \cl$. 
%Then $\conf \toET{(\cl', \mu)} \conf_2 \toET{(\cl, \varepsilon)} \conf'$ 
%\end{lemma}
%
%\begin{proof}
%We only consider the case where $\mu = \fp$ for some fingerprint $\fp$. The case where 
%$\mu = \varepsilon$ is simpler to prove.
%Let $\conf = (\mkvs, \vienv)$, $\conf_1 = (\mkvs_1, \vienv_1)$, $\conf' = (\mkvs', \vienv')$. 
%Let also $\vi = \vienv(\cl)$.
%By the reduction rule we have that $\mkvs_1 = \mkvs, \vienv_1 = \vienv\rmto{\cl}{\vi_1}$ for 
%some $\vi_1: \vi_1 \sqsubseteq \vi_1$. Let $\vi' = \vienv(\cl')$: then we have that $\vienv_1(\cl') = 
%\vi'$. Because $(\mkvs_1, \vienv_1) \toET{(\cl', \fp)} (\mkvs', \vienv')$, we have that 
%$\ET \vdash (\mkvs_1, \vi') \csat \fp : (\mkvs',\vi'') $, where $\vi'' = \vienv'(\cl')$. Because $\mkvs_1 = \mkvs$, 
%that means that $\ET \vdash (\mkvs, \vi') \csat \fp: (\mkvs', \vi'')$, then it follows that 
%$(\mkvs, \vienv) \toET{(\cl', \fp)} (\mkvs', \vienv\rmto{\cl'}{\vi''}) 
%\toET{(\cl, \varepsilon)} (\mkvs', \vienv\rmto{\cl'}{\vi''}\rmto{\cl}{\vi_1}) = 
%(\mkvs', \vienv\rmto{\cl}{\vi_1}\rmto{\cl'}{\vi''}) = (\mkvs', \vienv_1\rmto{\cl'}{\vi''}) = 
%(\mkvs', \vienv')$, as we wanted to prove.
%\end{proof}
