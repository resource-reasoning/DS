\subsection{\( \ET \)s matching views}
\label{sec:et-match-view}

To recall, 
two execution tests $\ET_1$ and $\ET_2$ have matching views if there exists a minimum client view function \( \func{minClientViews}[\ET,\mkvs,\cl] \) for \( \ET_1 \), \( \ET_2 \) and \( \ET_1 \cap \ET_2 \) such that:
\begin{itemize}
    \item for any \( \mkvs, \cl\):
    \begin{centermultline}
        \func{minClientViews}[\ET_1, \mkvs, \cl] \viewcup \func{minClientViews}[\ET_2, \mkvs, \cl] \viewleq \func{minClientViews}[\ET_1 \cap \ET_2, \mkvs, \cl]
    \end{centermultline}
    \item for any \(\mkvs, \mkvs',\fp,\vi_1,\vi_2,\vi_1',\vi_2',\cl\):
\begin{centermultline}
    (\vi_1,\vi'_1) =  \func{minFpViews}[\ET_1,\mkvs,\fp,\mkvs',\cl]
    \land (\vi_2,\vi'_2) =  \func{minFpViews}[\ET_2,\mkvs,\fp,\mkvs',\cl] \\
    \implies 
    \vi'_1 = \func{minClientView}[\ET_1, \mkvs',\cl]
    \land \vi'_2 = \func{minClientView}[\ET_2, \mkvs',\cl]  \\
    \exsts{\vi,\vi'} 
    (\vi,\vi') =  \func{minFpViews}[\ET_1 \cap \ET_2,\mkvs,\fp,\mkvs',\cl]
    \land \vi_1 \sqcup \vi_2 \viewleq \vi
    \land \vi'_1 \sqcup \vi'_2 \viewleq \vi' \\
    {} \land \vi' =  \func{minClientViews}[\ET_1 \cap \ET_2, \mkvs', \cl]
\end{centermultline}
where,
\begin{align*}
    (\vi,\vi') = \func{minFpViews}[\ET, \mkvs,\fp,\mkvs',\cl] & \defiff 
    \begin{multlined}[t]
        \func{minClientViews}[\ET, \mkvs, \cl] \viewleq \vi 
        \land \func{minClientViews}[\ET, \mkvs', \cl] \viewleq \vi' \\
        {} \land \ET \vdash (\mkvs,\vi) \csat \fp : (\mkvs',\vi') \\
        {} \land \fora{\vi'',\vi'''} \ET \vdash (\mkvs,\vi'') \csat \fp : (\mkvs',\vi''')
        \implies \vi \viewleq \vi'' \land \vi' \viewleq \vi'''
    \end{multlined} \\
    \vi \sqcup \vi' & \defeq \lambda \key \ldotp \vi(\key) \cup \vi'(\key)
\end{align*}
    \item for \( \ET_1 \) (\(\ET_2\) and \( \ET_1 \cap \ET_2\) respectively),
    if any kv-store \( \mkvs \) contains less information than \( \mkvs' \) then
    \begin{centermultline}
        \func{minClientView}[\ET_1,\mkvs,\cl] \viewleq \func{minClientView}[\ET_1, \mkvs',\cl]
    \end{centermultline}
\end{itemize}

\Cref{tab:et-minimum-view} presents the minimum client views for execution tests.
\begin{table}[h!]
    \centering
    \caption{Minimum view for clients}
    \label{tab:et-minimum-view}
\begin{tabular}{ @{} l r  @{} }
\hline
Model & \(\func{minClientViews}(\ET, \mkvs, \cl)\)
\\
\hline
\MR & $\getView[\mkvs,\Set{\txid }[\exsts{\txid_{\cl}} \txid \toEDGE{\WR_\mkvs} \txid_{\cl}] \cup \Set{\txid_0} ]$
\\
\MW & \(\vi_0\) \\
\RYW & $\getView[\mkvs,\Set{\txid_{\cl} \in \mkvs }\cup \Set{\txid_0} ]$  \\
\WFR & \(\vi_0\) \\
\(\MR \cap \MW \cap \RYW \cap \WFR\) & 
\(
\getView[\mkvs,\func{lfxTx}[%
        \mkvs,\Set{\txid }[\exsts{\txid_{\cl}} \txid \toEDGE{\WR_\mkvs} \txid_{\cl}] 
        \cup \Set{\txid_{\cl} \in \mkvs }  \cup \Set{\txid_0}, \SO\rflx \cup (\WR_\mkvs ; \SO\rflx)
    ] ]
\) \\
\hline
\end{tabular}%
\end{table}
\begin{align*}
    \func{lfpTx}[\mkvs,\txidset, R] 
    & \defeq 
    \mu X . 
    \begin{bracketarray}
        \txidset \subseteq X  
        \land \fora{ \txid \in X, \txid' \in \mkvs} \txid' \toEDGE{R} \txid \implies \txid' \in X 
    \end{bracketarray} 
\end{align*}


\begin{theorem}
    \( \MR \), \( \MW \), \( \RYW \) and \( \WFR \) have matching views.
\end{theorem}
\begin{proof}
Given \cref{tab:et-minimum-view}, it is easy to see for any \( \mkvs, \cl \):
\begin{centermultline}
    \begin{bracketarray}
    \func{minClientViews}[\ET_{\MR},\mkvs,\cl]
    \viewcup \func{minClientViews}[\ET_{\MW},\mkvs,\cl] \\
    \quad {} \viewcup \func{minClientViews}[\ET_{\RYW},\mkvs,\cl]
    \viewcup \func{minClientViews}[\ET_{\WFR},\mkvs,\cl] 
    \end{bracketarray} \\
    \viewleq \func{minClientViews}[\ET_{\MR} \cap \ET_{\MW} \cap \ET_{\RYW} \cap \ET_{\WFR},\mkvs,\cl]
\end{centermultline}
Also for any \( \ET \in \Set{\ET_{\MR}, \ET_{\MW}, \ET_{\RYW}, \ET_{\WFR}, \ET_{\MR} \cap \ET_{\MW} \cap \ET_{\RYW} \cap \ET_{\WFR}} \),
kv-stores \( \mkvs, \mkvs'\) where \( \mkvs \) contains less information than \( \mkvs' \), given \cref{tab:et-minimum-view} it follows:
\begin{centermultline}
    \func{minClientViews}[\ET,\mkvs,\cl] \viewleq \func{minClientViews}[\ET,\mkvs',\cl] 
\end{centermultline}
Now take arbitrary kv-stores \(\mkvs,\mkvs'\), fingerprint \( \fp \) and client \( \cl \), the \cref{tab:et-minimum-fingerprint}
shows the views given by the \( (\vi,\vi') = \func{minFpviews}[\ET,\mkvs,\fp,\mkvs',\cl] \) functions.
\begin{table}[h!]
    \centering
    \caption{Minimum views for fingerprints}
    \label{tab:et-minimum-fingerprint}
\begin{tabular}{ @{} l r r @{} }
\hline
Model & \(\vi\) & \(\vi'\)
\\
\hline
\MR 
& $\getView[\mkvs,\Set{\txid}[\exsts{\txid_{\cl}} \txid \toEDGE{\WR_{\mkvs}} \txid_{\cl}] \cup \Set{\txid_0} \cup \dagger ]$ 
& $\getView[\mkvs',\Set{\txid}[\exsts{\txid_{\cl}} \txid \toEDGE{\WR_{\mkvs'}} \txid_{\cl}] \cup \Set{\txid_0} ]$ \\
\MW 
& \(\getView[\mkvs,\func{lfpTx}[\mkvs,\dagger,\SO\rflx] ]\)
& \(\vi_0\) \\
\RYW 
& $\getView[\mkvs,\Set{\txid_{\cl} \in \mkvs} \cup \Set{\txid_0} \cup \dagger ]$
& $\getView[\mkvs',\Set{\txid_{\cl} \in \mkvs'} \cup \Set{\txid_0} ]$ \\
\WFR 
& \(\getView[\mkvs,\func{lfpTx}[\mkvs,\dagger,\WR_\mkvs ; \SO\rflx] ]\)
& \(\vi_0\) \\
\hline
\end{tabular}%
\end{table}

The \( \dagger \) is the versions read by the new transaction (\(\txid'\)) from client \( \cl \):
\begin{align*}
    \dagger & \equiv \Set{\txid}[\exsts{\txid'} \txid \in \mkvs' \land \txid' \notin \mkvs \land \txid \toEDGE{\WR_{\mkvs'}} \txid]
\end{align*}
Note that the \( \vi' \) in \cref{tab:et-minimum-fingerprint} already satisfies \( \vi' = \func{minClientViews}[\ET,\mkvs',\cl]\).
Assume the following where \( \vi_i, \vi'_i \) are taken the ones from \cref{tab:et-minimum-fingerprint}:
\begin{centermultline}
    \ET_{\MR} \vdash (\mkvs,\vi_1) \csat \fp : (\mkvs', \vi_1' ) 
    \land \ET_{\MW} \vdash (\mkvs,\vi_2) \csat \fp : (\mkvs', \vi_2' ) \\
    {} \land \ET_{\RYW} \vdash (\mkvs,\vi_3) \csat \fp : (\mkvs', \vi_3' )
    \land \ET_{\WFR} \vdash (\mkvs,\vi_4) \csat \fp : (\mkvs', \vi_4' ) 
\end{centermultline}
Let \( \vi \)
\begin{centermultline}
    \vi = \getView[\mkvs,\func{lfxTx}[%
        \mkvs,\Set{\txid }[\exsts{\txid_{\cl}} \txid \toEDGE{\WR_\mkvs} \txid_{\cl}] 
        \cup \Set{\txid_{\cl} \in \mkvs }  \cup \dagger \cup \Set{\txid_0}, \SO\rflx \cup (\WR_\mkvs ; \SO\rflx)
    ] ]
\end{centermultline}
and \( \vi' \)
\begin{centermultline}
    \vi' = \getView[\mkvs',\func{lfxTx}[%
        \mkvs,\Set{\txid }[\exsts{\txid_{\cl}} \txid \toEDGE{\WR_{\mkvs'}} \txid_{\cl}] 
        \cup \Set{\txid_{\cl} \in \mkvs' } \cup \Set{\txid_0}, \SO\rflx \cup (\WR_{\mkvs'} ; \SO\rflx)
    ] ]
\end{centermultline}
It is easy to see \( \vi_1 \viewcup \vi_2 \viewcup \vi_3 \viewcup \vi_4 \viewleq \vi \) 
and  \( \vi'_1 \viewcup \vi'_2 \viewcup \vi'_3 \viewcup \vi'_4 \viewleq \vi = \func{minClientViews}[\MR \cap \MW \cap \RYW \cap \WFR, \mkvs',\cl] \).
We now prove that \( \ET_\MR \cap \ET_\MW \cap \ET_\RYW \cap \ET_\WFR \vdash (\mkvs,\vi) \csat \fp : (\mkvs',\vi')\), which can implied by proving \( \bigwedge_{1 \leq i \leq 4} \max(\vi)(\key) = \max(\vi_i)(\key) \) for any key \( \key \) being read.
\end{proof}




\begin{lemma}
\( \MR \) and \( \MW \) have matching views.
\end{lemma}
\begin{proof}
Assume kv-stores \( \mkvs, \mkvs' \), a fingerprints \( \fp \) and views \( \vi_1, \vi'_1, \vi_2, \vi'_2 \):
\begin{centermultline}
    \ET_{\MR} \vdash (\mkvs,\vi_1) \csat \fp : (\mkvs', \vi_1' ) \land
    \ET_{\MW} \vdash (\mkvs,\vi_2) \csat \fp : (\mkvs', \vi_2' )
    \land \begin{bracketarray}\fora{l, \key, \val} (l, \key, \val) \in \fp \implies \max(\vi_1(\key)) = \max(\vi_2(\key))\end{bracketarray} 
\end{centermultline}
We pick \(\vi, \vi' \) such that:
\begin{centermultline}
    \fora{\key} 
    \begin{bracketarray}\exsts{l,\val} (l,\key,\val) \in \fp \implies \vi_1(\key) \cup \vi_2(\key) = \vi(\key)\end{bracketarray} \\
    {} \land \begin{bracketarray}
        \nexists l,\val \ldotp \; (l,\key,\val) \in \fp 
        \implies \vi_1(\key) \cup \vi_2(\key) \cup \vi'_1(\key) \cup \vi'_2(\key) = \vi(\key)
\end{bracketarray}
    \land \vi'(\key) = \vi'_1(\key) \cup \vi'_2(\key)
\end{centermultline}
Given \( \vi_1 \viewleq \vi'_1 \) it is easy to see that \( \vi \viewleq \vi' \) and then \( \ET_\MR \vdash (\mkvs,\vi) \csat \fp : (\mkvs',\vi') \).
Given that \( \fora{\key} \exsts{l,\val}  (l,\key,\val) \in \fp \implies \vi_1(\key) \cup \vi_2(\key) = \vi(\key) \), it means \( \allowed[\mkvs,\vi,\fp,\SO\rflx]\),
and thus \( \ET_\MW \vdash (\mkvs,\vi) \csat \fp : (\mkvs',\vi') \).
\end{proof}

\begin{lemma}
    \( \MR \) and \( \RYW \) have matching views.
\end{lemma}
\begin{proof}
Assume kv-stores \( \mkvs, \mkvs' \), a fingerprints \( \fp \) and views \( \vi_1, \vi'_1, \vi_2, \vi'_2 \):
\begin{centermultline}
    \ET_{\MR} \vdash (\mkvs,\vi_1) \csat \fp : (\mkvs', \vi_1' ) \land
    \ET_{\RYW} \vdash (\mkvs,\vi_2) \csat \fp : (\mkvs', \vi_2' )
    \land \begin{bracketarray}\fora{l, \key, \val} (l, \key, \val) \in \fp \implies \max(\vi_1(\key)) = \max(\vi_2(\key))\end{bracketarray} 
\end{centermultline}
We pick \(\vi, \vi' \) such that:
\begin{centermultline}
    \fora{\key} 
    \begin{bracketarray}\exsts{l,\val} (l,\key,\val) \in \fp \implies \vi_1(\key) \cup \vi_2(\key) = \vi(\key)\end{bracketarray} \\
    {} \land \begin{bracketarray}
        \nexists l,\val \ldotp \; (l,\key,\val) \in \fp 
        \implies \vi_1(\key) \cup \vi_2(\key) \cup \vi'_1(\key) \cup \vi'_2(\key) = \vi(\key)
\end{bracketarray}
    \land \vi'(\key) = \vi'_1(\key) \cup \vi'_2(\key)
\end{centermultline}
Given \( \vi_1 \viewleq \vi'_1 \) it is easy to see that \( \vi \viewleq \vi' \) and then \( \ET_\MR \vdash (\mkvs,\vi) \csat \fp : (\mkvs',\vi') \).
Since that \( \vi'(\key) = \vi'_1(\key) \cup \vi'_2(\key) \) for any key \(\key\), 
it follows that \( \vi' \) contains all the version written by the client of the new transaction \( txid \):
\[
    \fora{key,i} \txid \in \mkvs' \land \txid \notin \mkvs \wtOf[\mkvs'(\key,i)] \toEDGE{\SO\rflx} \txid \implies i \in \vi'(\key)
\]
That is, \( \ET_\RYW \vdash (\mkvs,\vi) \csat \fp : (\mkvs',\vi') \).
\end{proof}

