\subsection{Transaction Soundness}


\begin{thm}[Transaction soundness]
\label{thm:transaction-soundness}
The transaction soundness is as follows:
\[
    \begin{array}{@{}l@{}}
        \for{ \lpre, \trans, \lpost } \tripleL{\lpre}{\trans}{\lpost} \implies \ \tripleSemL{\lpre}{\trans}{\lpost} \\
    \end{array}
\]
where,
\[
    \begin{rclarray}
    \tripleSemL{\lpre}{\trans}{\lpost} & \eqdef &
    \begin{array}[t]{@{}l@{}}
        \for{\lenv, \stk_{p}, \stk_{q}, \h_{p}, \h_{q}, \opset_{p}, \opset_{q} } 
        (\h_{p}, \opset_{p}) \in \evalLS[\lenv, \stk_{p}]{\lpre} \\
        \quad {} \land \vdash (\stk_{p}, \h_{p}, \opset_{p} ), \trans \toL^{*}  (\stk_{q}, \h_{q}, \opset_{q} ), \pskip 
        \implies (\h_{q}, \opset_{q}) \in \evalLS[\lenv, \stk_{q}]{\lpost}
    \end{array}
    \end{rclarray}
\]
\end{thm}
\begin{proof}
Induction on the derivations.

\caseB{\rl{TRSkip}}

We have  \(\trans \equiv \pskip\), \( \lpre \equiv \lpost \equiv \assemp \), thus \( \h_{p} = \h_{q} = \unitH \), \( \opset_{p} = \opset_{q} \) and \( \stk_{p} = \stk_{q} \), and then \( (\unitH,\unitO ) \in \evalLS[\lenv, \stk_{q}]{\assemp} \) holds.

\caseB{\rl{TRAss}}

We have \(\trans \equiv ( \pass{\var}{\expr} ) \), \( \lpre \equiv ( \var \doteq \lexpr ) \) and \( \lpost \equiv ( \var \doteq \expr\sub{\var}{\lexpr} ) \) for some \( \expr, \lexpr \) and \( \var \) such that \( \var \notin \func{fv}{\lexpr} \land \var \in \Vars\).
Given the transaction semantics (\figref{fig:thread_semantics}), it has \( \stk_{q} = \stk_{p}\rmto{\var}{\val} \) where \( \val = \evalLE[\lenv, \stk_{p}]{\expr\sub{\var}{\lexpr}} \).
Since \( \var \notin \func{fv}{\lexpr} \), we know \( \evalLE[\lenv, \stk_{p}]{\lexpr} = \evalLE[\lenv, \stk_{q}]{\lexpr} \), and then \( \evalLE[\lenv, \thstk \uplus \txstk_{p}]{\expr\sub{\var}{\lexpr}} = \evalLE[\lenv, \thstk \uplus \txstk_{q}]{\expr\sub{\var}{\lexpr}} \).
This means the assertions related to stack holds even thought the stack changes.
Also because the heap and events set remains unchanged, this is \( \h_{p} = \h_{q} \) and \( \opset_{p} = \opset_{q} \), we have \( (\h_{q}, \opset_{q} ) \in \evalLS[\lenv, \stk_{q}]{\lpost} \).

\sx{Need to change once settle down the fingerprint}
\caseB{\rl{TRDeref}}

We have  \(\trans \equiv ( \pderef{\txvar}{\expr} ) \), \( \lpre \equiv ( \expr \pt \lexpr \sep \fpF ) \) and \( \lpost \equiv ( \txvar \doteq \lexpr \sep \expr \pt \lexpr \sep \fpto{\fp'} ) \) for some \( \txvar, \lexpr, \expr , \fp \) and \( \fp' \) such that \( \txvar \notin \func{fv}{\expr} \cup \func{fv}{\lexpr}\), \( \txvar \in \TxVars \) and \( \fp' = \fp \addO (\etR, \expr, \lexpr)\).
Given the transaction semantics (\figref{fig:thread_semantics}), it has \( \txstk_{q} = \txstk_{p}\remapsto{\txvar}{\val} \) where \( \val = \h_{p}(\evalLE[\lenv, \thstk \uplus \txstk_{p}]{\expr}) \).
Since the heap \( \h_{p}\) satisfies the pre-condition \( \lpre\), so that \( \val =  \evalLE[\lenv, \thstk \uplus \txstk_{p}]{\lexpr} \).
Given that \( \txvar \notin \func{fv}{\expr} \cup \func{fv}{\lexpr} \), it must have \(  \evalLE[\lenv, \thstk \uplus \txstk_{p}]{\lexpr} = \evalLE[\lenv, \thstk \uplus \txstk_{q}]{\lexpr} \) and \( \evalLE[\lenv, \thstk \uplus \txstk_{p}]{\expr} = \evalLE[\lenv, \thstk \uplus \txstk_{q}]{\expr} \).
This means the assertions related to stack states and syntactic values still holds even thought the stack changes.
It also implies that \( \evalF[\lenv, \thstk \uplus \txstk_{p}]{\fp} = \evalF[\lenv, \thstk \uplus \txstk_{q}]{\fp} = \opset_{p} \), therefore \( \evalF[\lenv, \thstk \uplus \txstk_{q}]{\fp'} = \evalF[\lenv, \thstk \uplus \txstk_{q}]{\fp \addO (\etR, \expr, \lexpr)}= \opset_{p} \addO (\etR, \evalLS[\lenv, \thstk \uplus \txstk_{q}]{\expr}, \evalLS[\lenv, \thstk \uplus \txstk_{q}]{\lexpr}) = \opset_{p} \addO (\etR, \evalLS[\lenv, \thstk \uplus \txstk_{p}]{\expr}, \evalLS[\lenv, \thstk \uplus \txstk_{p}]{\lexpr}) = \opset_{q} \).
Given above and \( \h_{p} = \h_{q} \), we have \( ( \h_{q},\opset_{q} ) \in \evalLS[\lenv, \thstk \uplus \txstk_{q}]{\lpost} \).

\sx{Need to change once settle down the fingerprint}
\caseB{ \rl{TRMutate} }

We have \( \trans \equiv (\pmutate{\expr_{1}}{\expr_{2}}) \), \( \lpre \equiv ( \expr_{1} \pt \stub \sep \fpF ) \) and \( \lpost \equiv ( \expr_{1} \pt \expr_{2} \sep \fpto{\fp'}) \), for some \( \expr_{1}, \expr_{2}, \fp \) and \( \fp' \) where \( \fp' = \fp \addO (\etW, \expr_{1}, \expr_{2} ) \).
Given the transaction semantics (\figref{fig:thread_semantics}), it has \( \txstk_{p} = \txstk_{q} \), so \(  \evalLE[\lenv, \thstk \uplus \txstk_{p}]{\expr_{1}} = \evalLE[\lenv, \thstk \uplus \txstk_{q}]{\expr_{1}} \) and \(  \evalLE[\lenv, \thstk \uplus \txstk_{p}]{\expr_{2}} = \evalLE[\lenv, \thstk \uplus \txstk_{q}]{\expr_{2}} \).
This implies \( \h_{q} = \Set{\evalLE[\lenv, \thstk \uplus \txstk_{p}]{\expr_{1}} \mapsto \evalLE[\lenv, \thstk \uplus \txstk_{p}]{\expr_{2}} } = \Set{\evalLE[\lenv, \thstk \uplus \txstk_{q}]{\expr_{1}} \mapsto \evalLE[\lenv, \thstk \uplus \txstk_{q}]{\expr_{2}} } \) , and \( \evalF[\lenv, \thstk \uplus \txstk_{q}]{\fp'} = \evalF[\lenv, \thstk \uplus \txstk_{q}]{\fp \addO (\etW, \expr_{1}, \expr_{2})}= \opset_{p} \addO (\etW, \evalLS[\lenv, \thstk \uplus \txstk_{q}]{\expr_{1}}, \evalLS[\lenv, \thstk \uplus \txstk_{q}]{\expr_{2}}) = \opset_{p} \addO (\etW, \evalLS[\lenv, \thstk \uplus \txstk_{p}]{\expr_{1}}, \evalLS[\lenv, \thstk \uplus \txstk_{p}]{\expr_{2}}) = \opset_{q} \).
Thus, \( ( \h_{q},\opset_{q} ) \in \evalLS[\lenv, \thstk \uplus \txstk_{q}]{\lpost} \). 

\caseI{\rl{TRChoice}}

We have  \(\trans \equiv \trans_{1} + \trans_{2} \), where \( \tripleL{\lpre}{\trans_{1}}{\lpost} \) and \( \tripleL{\lpre}{\trans_{2}}{\lpost} \) hold, for some \( \trans_{1}, \trans_{2}, \lpre, \lpost \).
Given the transaction semantics (\figref{fig:thread_semantics}), it either has \( \vdash ( \stk_{p}, \h_{p}, \opset_{p} ), \trans_{1} \pchoice \trans_{2} \toL ( \stk_{p}, \h_{p}, \opset_{p} ), \trans_{1} \) or  \( \vdash ( \stk_{p}, \h_{p}, \opset_{p} ), \trans_{1} \pchoice \trans_{2} \toL ( \stk_{p}, \h_{p}, \opset_{p} ), \trans_{2} \).
Let us pick \( \trans_{1} \).
Assume \( \trans_{1} \) can be reduced to \( \pskip \), \ie \( \thstk \vdash ( \txstk_{p}, \h_{p}, \opset_{p} ), \trans_{1}  \toL^{*} ( \txstk_{q}, \h_{q}, \opset_{q} ), \pskip \).
By the \ih that \( \tripleSemL{\lpre}{\trans_{1}}{\lpost} \), we know \( (\h_{q}, \opset_{q}) \in \evalLE[\lenv, \stk_{q}]{\lpost} \).
Symmetrically, if we pick \( \trans_{2} \), it yields the same result.

\caseI{\rl{TRSeq}}

We have \( \trans \equiv \trans_{1} \pseq \trans_{2} \) where \( \tripleL{\lpre}{\trans_{1}}{\lframe} \) and \( \tripleL{\lframe}{\trans_{2}}{\lpost} \) hold, for some \( \trans_{1}, \trans_{2}, \lpre, \lpost, \lframe \).
Given the transaction semantics (\figref{fig:thread_semantics}), it has \( \vdash ( \stk_{p}, \h_{p}, \opset_{p} ), \trans_{1} \pseq \trans_{2} \toL^{*} ( \stk_{r}, \h_{r}, \opset_{r} ), \pskip \pseq \trans_{1} \toL ( \stk_{r}, \h_{r}, \opset_{r} ), \trans_{1} \toL^{*} ( \stk_{q}, \h_{q}, \opset_{q} ), \pskip \) for some \( \stk_{r}, \h_{r}, \opset_{r} \).
By the \ih that \( \tripleSemL{\lpre}{\trans_{1}}{\lframe} \), we have \( (\h_{r}, \opset_{r}) \in \evalLE[\lenv, \stk_{r}]{\lframe} \).
The elimination of prefix \( \pskip\) does not change any state, so \( (\h_{r}, \opset_{r}) \in \evalLE[\lenv, \stk_{r}]{\lframe} \) still holds.
Then, by the \ih that \( \tripleSemL{\lframe}{\trans_{2}}{\lpost} \), we prove \( (\h_{q}, \opset_{q}) \in \evalLE[\lenv, \stk_{q}]{\lpost} \).

\caseI{\rl{TRLoop}}

Since the triple is only partial correct, meaning that if the transaction \( \trans \) terminates it will reach a state satisfying the post-condition \( \lpost \), it is sufficient to prove the follows,
\[
    \for{\lpre, \trans, \nat > 0} \tripleL{\lpre}{\trans^{\nat}}{\lpre} \implies \ \tripleSemL{\lpre}{\trans^{\nat}}{\lpre} \\
\]
where,
\[
\begin{rclarray}
    \trans^{1} & \defeq  & \trans \\
    \trans^{\nat} & \defeq  & \trans \pseq \trans^{\nat - 1} \\
\end{rclarray}
\]

We prove that by induction on the number \( \nat \).
For \( \nat = 1 \), it is proven directly by the \ih
For \( \nat > 1 \), we have \( \vdash (\stk_{p}, \h_{p}, \opset_{p}), \trans \pseq \trans^{\nat - 1} \toL^{*} (\stk_{r}, \h_{r}, \opset_{r}), \trans^{\nat - 1} \toL^{*} (\stk_{q}, \h_{q}, \opset_{q}), \pskip \) for some \( \stk_{r}, \h_{r}, \opset_{r} \).
By the \ih that \(\tripleSemL{\lpre}{\trans}{\lpre} \), we know \(  (\h_{r}, \opset_{r}) \in \evalLS[\lenv, \stk_{r}]{\lpre} \).
Then by the \ih that \(\tripleSemL{\lpre}{\trans^{\nat - 1}}{\lpre} \), we have the proof \(  (\h_{q}, \opset_{q}) \in \evalLS[\lenv, \stk_{q}]{\lpre} \).

\sx{Need change once settle down the fingerprint}

\caseI{\rl{TRFrame}}

We need to prove \( \tripleSemL{\lpre \sep \lframe }{\trans}{\lpost \sep \lframe} \) given that \( \tripleSemL{\lpre}{\trans}{\lpost} \).
there exists \( \h_{p} \), \( \h_{r} \), \( \h_{q} \), \( \opset_{p}\), \( \opset_{r}\)  and \( \opset_{q} \) that \( (\h_{p} \composeH \h_{r}, \opset_{p} \uplus \opset_{r}) \in \evalLS[\lenv, \thstk \uplus \txstk_{p}]{\lpre \sep \lframe} \land ( \h_{p}, \opset_{p} ) \in \evalLS[\lenv, \thstk \uplus \txstk_{p}]{\lpre} \land ( \h_{r}, \opset_{r} ) \in \evalLS[\lenv, \thstk \uplus \txstk_{p}]{\lframe}\) and similarly \( ( \h_{q} \composeH \h_{r}, \opset_{q} \uplus \opset_{r} ) \in \evalLS[\lenv, \thstk \uplus \txstk_{q}]{\lpost \sep \lframe} \land ( \h_{q} ,\opset_{q} ) \in \evalLS[\lenv, \thstk \uplus \txstk_{q}]{\lpost} \land ( \h_{r}, \opset_{r} ) \in \evalLS[\lenv, \thstk \uplus \txstk_{q}]{\lframe}\).
By the \ih that \( \tripleSemL{\lpre}{\trans}{\lpost} \), it means \( \thstk \vdash ( \txstk_{p}, \h_{p}, \opset_{p} ), \trans \toL^{*} ( \txstk_{q}, \h_{q}, \opset_{q} ), \pskip \).
Now we need to prove the follows,
\[
    \thstk \vdash ( \txstk_{p}, \h_{p}  \composeH \h_{r}, \opset_{p} \composeO \opset_{r}), \trans \toL^{*} ( \txstk_{q}, \h_{q} \composeH \h_{r}, \opset_{q} \composeO \opset_{r}), \pskip 
\]
First for the heaps part, since both \( (\h_{p} \composeH \h_{r}) \) and  \( (\h_{q} \composeH \h_{r}) \) are defined, this means the domain of the frame \( \h_{r} \) are separate from the ones of \( \h_{p}\) and \( \h_{q} \).
Then by the \ih, the transaction \( \trans \) does not need any resource from \( \h_{r} \) to progress.
Second for the event sets part, since it has \( ( \h_{q} \composeH \h_{r}, \opset_{q} \composeO \opset_{r} ) \in \evalLS[\lenv, \thstk \uplus \txstk_{q}]{\lpost \sep \lframe} \), so \( \opset_{q} = \unitO \lor \opset_{r} = \unitO \).
if \( \opset_{q} = \unitO \), it must be that \( \opset_{p} = \unitO \), it holds by the \ih
If \( \opset_{r} = \unitO \), it also holds by the \ih
Given above we have the prove that \( \tripleSemL{\lpre \sep \lframe }{\trans}{\lpost \sep \lframe} \).


\end{proof}
