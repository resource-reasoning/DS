\subsection{Local/Transaction}

 
\begin{definition}[Logical Expressions]
\label{def:logical-expr}
Assume a countably infinite set of \emph{logical variables} $\V x \in \LVar$.
The set of \emph{logical expressions}, $ \lexpr \in \LExpr$ is defined by the following inductive grammar, where \(\val \in \Val\) (\defref{def:program_values}), \(\txvar \in \TxVars\) and \( \thvar \in \ThdVars \) (\defref{def:stacks}),
\[
\begin{rclarray}
   \lexpr & ::= & \val \mid \txvar \mid \thvar \mid \lvar \mid \lexpr + \lexpr \mid \lexpr * \lexpr \mid \dots 
\end{rclarray}
\]
Assume a set of \emph{logical environments} \(\lenv \in \LEnv: \LVar \parfun \Val\) which associates logical variables with values.
Given a stack $\stk \in \Stacks$ (\defin\ref{def:stacks}) and a logical environment $\lenv \in \LEnv$, the \emph{logical expression evaluation} function, $\evalLE[(., .)]{.}:\LExpr \times \Stacks \times \LEnv\rightharpoonup \Val$, is defined inductively over the structure of logical expressions as follows,
%
\[
    \begin{rclarray}
        \evalLE{\val} & \defeq & \val \\
        \evalLE{\txvar} & \defeq & \stk(\txvar) \\
        \evalLE{\thvar} & \defeq & \stk(\thvar) \\
        \evalLE{\lvar} & \defeq & \lenv(\lvar) \\
        \evalLE{\lexpr_1 + \lexpr_2} & \defeq & \evalLE{\lexpr_1} + \evalLE{\lexpr_2} \\
        \evalLE{\lexpr_1 * \lexpr_2} & \defeq & \evalLE{\lexpr_1} * \evalLE{\lexpr_2} \\
        \dots & \defeq & \dots \\
    \end{rclarray}
\]
\end{definition}

Note the domain of the stack \( \stk \) includes transaction variables and thread variables.

\begin{definition}[Local assertions]
\label{def:local_assertions}
Given the set of logical expressions \( \LExpr \) and logical variables \( \LVar \) (\defin\ref{def:logical-expr}), the set of \emph{local assertions}, $\lpre,  \lpost \in \LAst$, is defined inductively by the following grammar, 
\[
\begin{rclarray}
	\lpre, \lpost  & ::= & \False \mid \True \mid \lpre \land \lpost \mid \lpre \lor \lpost \mid \exsts{\lvar} \lpre \mid \Emp \mid \lexpr \pt \lexpr \fp \mid \lpre \sep \lpost  \\
    \fp & \subseteq & \Setcon{ (\etag,\val) }{\etag \in \ETags \land \val \in \Val}  \\
\end{rclarray}	 
\]
Given a logical environment $\lenv \in \LEnv$, the \emph{local interpretation function}, $\evalLS[(.,.)]{.}: \LAst \times \LEnv \parfun \powerset{\Heaps \times \Events}$, is defined over the structure of local assertions as follows,
\sx{there is variable clash for event tags and logical expression, leave it for now}
\[
\begin{rclarray}
	\evalLS{\assfalse} & \eqdef & (\emptyset, \emptyset)  \\
	\evalLS{\asstrue} & \defeq & \powerset{\Heaps \times \Events}  \\
	\evalLS{\lpre \land \lpost} & \defeq & \evalLS{\lpre} \cap \evalLS{\lpost} \\
	\evalLS{\lpre \lor \lpost} & \defeq & \evalLS{\lpre} \cup \evalLS{\lpost} \\
	\evalLS{\exsts{\lvar} \lpre} & \defeq & \bigcup\limits_{\val \in \textnormal{\Val}}\evalLS[\lenv\remapsto{\lvar}{\val}, \stk]{\lpre}  \\
	\evalLS{\assemp} & \defeq & \Set{(\unitH, \emptyset)}  \\
	\evalLS{\lexpr_1 \pt \lexpr_2 \fp} & \defeq & \Set{ \left(\Set{\evalLE{\lexpr_1} \pt \evalLE{\lexpr_2}}, \Setcon{(\etag, \evalLE{\lexpr_1}, \val)}{(\etag, \val) \in \fp}\right) } \\
	\evalLS{\lpre \sep \lpost} & \defeq & 
    \Setcon{ (\h_1 \composeH \h_2, \evset_{1} \uplus \evset_{2} ) }{ (\h_{1}, \evset_{1}) \in \evalLS{\lpre} \land (\h_{2}, \evset_{2}) \in \evalLS{\lpost} } 
\end{rclarray}
\]
\end{definition}

Observe that program expressions $\Expr$  (\defin\ref{def:language}) are contained in logical expressions $\LExpr$ (\defin\ref{def:local_assertions} above). That is, $\Expr \subset \LExpr$. 

\subsection{Global/Program}

\sx{
    Current idea is, for each box(region) it records a single value for each address corresponding to the ``real'' value, \ie compute through arbitrary relation. 
    And a read function(or invariable of the box), which is a function that make sense with respect to the interference and consistency model.
    By given the current values of addresses the read buffer returns all observable values.
    For example, the read function for snapshot isolation, and serialisibility will be identity, but for PSI and causal consistency probably not the identity.
    The read buffer will need semantics checking.
}
\sx{Bring back the region and capability transfer}
\begin{definition}[Capabilities]
\label{def:capabilities}
Assume a partial commutative monoid for \emph{capabilities} \( (\Caps, \composeC, \unitC) \) with \( \kap, \ca \in \Caps \), where the  \( \composeC: \Caps \times \Caps \parfun \Caps \), is \emph{capability composition function}, and \( \unitC \) is the set of units.
\end{definition}

\sx{event set + capabilities might be a way to represent actions?}
\begin{defn}
Given the set of transaction events \( \Events \) (\defref{def:transaction-event}), the set of \emph{actions} is defined as the follows,
\[
    \begin{rclarray}
        \action \in \Actions & \eqdef & \powerset{\Events \uplus \Caps}
    \end{rclarray}
\]
\end{defn}

\begin{definition}[Actions]
\label{def:action}
Given the set of heaps \( \Heaps \) (\defref{def:heaps}) and transaction events \( \Events \) (\defref{def:transaction-event}), the set of \emph{actions} is defined as the follows,
%
\[
    \begin{rclarray}
	\action \in \Actions & \eqdef &
	\Setcon{
		((\h,\evset), (h',\evset'))
	}{
		(\h, \h') \in \Heaps \land \orth{\h} = \orth{\h'}
	}
    \end{rclarray}
\] 
where, the \emph{orthogonal} \(\orth{(.)} \) is defined as follows, for all domains \( \sort M \), all \( m \in \sort{M} \) and its composition function \( \compose{} \),
\[
    \begin{rclarray}
    \orth{m} & \eqdef & \Setcon{m}{m \compose{} m \isdef} \\
    \end{rclarray}
\]
Given the set of capabilities $\Caps$ (\defin\ref{def:capabilities}), the set of \emph{interference environments} is $\inter \in \Interference \defeq \Caps \parfun \powerset{\Actions}$.
\end{definition}

\begin{defn}[Interference]
\label{def:intf}
The set of \emph{interference assertions}, \( \intass \in \IAst \), are defined by the following grammar:
\[
\begin{rclarray}
	\intass & \eqdef  &
	\emptyset \mid \Set{ \perm{\kap} : \exsts{\vec{\lvar}} \lpre \transfersto \lpost } \cup \intass 
\end{rclarray}
\]
Given a logical environment $\lenv \in \LEnv$ and a stack $\stk \in \Stacks$, the \emph{interference interpretation} function, $\evalI[(., .)]{.}: \IAst \times \LEnv \times \Stacks \parfun \Interference$, is defined as follows, for all $\kap \in \Caps$:
%
\[
\begin{rclarray}
	\evalI{\emptyset}(\kap) & \eqdef & \emptyset \\
	\evalI{\Set{ \perm{\kap} : \exsts{\vec{\lvar}} \lpre \transfersto \lpost } \cup \intass }(\kap) & \eqdef &
	\Setcon{
		((\h_\lpre,\evset_{\lpre}), (\h_\lpost,\evset_{\lpost}))	 
    }{
		((\h_\lpre,\evset_{\lpre}), (\h_\lpost,\evset_{\lpost})) \in \Actions \\
        \quad \land \exsts{\rid, \vec{v}, \lenv'} \land \lenv' = \lenv \remapsto{\vec{\lvar}}{\vec v} \land {} \\
		\quad (\h_\lpre,\evset_{\lpre}) \in \evalLS[\lenv', \stk]{\lpre} \land (\h_\lpost,\evset_{\lpost}) \in \evalLS[\lenv', \stk]{\lpost}
	}
	\cup 
	\evalI{\intass}(\kap)
\end{rclarray}
\] 
\end{defn}


\begin{definition}[Worlds]
\label{def:world}
Given the set of heaps $\Heaps$ (\defref{def:heaps}) and a set of \emph{region identifiers} \( \rid \in \RegionID \), the set of \emph{shared states} is \( \SStates \eqdef \RegionID \to \Heaps \) and the \emph{shared state composition function}, $\composeS: \SStates \times \SStates \parfun \SStates$, is defined as $\composeS \eqdef \composeEq$, where for all domains $\sort M$ and all $m, m' \in \sort M$,
%
\[
\begin{rclarray}
	m \composeEq m' &  \eqdef  &
	\begin{cases}
		m & \text{if } m = m'\\
		\text{undefined} & \text{otherwise}
	\end{cases}
\end{rclarray}
\]
Combining the above with capabilities (\defref{def:capabilities}), the set of \emph{worlds} is defined as follows,
%
\[
\begin{rclarray}
	\world \in \World  & \eqdef & \Caps \times \SStates
\end{rclarray}
\]
% 
The \emph{world composition function}, $\composeW: \World \times \World \parfun \World$, is defined component-wise as: $\composeW \eqdef (\composeC, \composeS)$.
The \emph{world unit set} is $\unitW \eqdef \Setcon{(\ca, \gs)}{(\ca, \gs) \in \World \land \ca \in \unitC}$.
The \emph{partial commutative monoid of worlds} is $(\World, \composeW, \unitW)$.
\end{definition}
 
\begin{definition}[Assertions]
\label{def:assertion}
The set of \emph{assertions}, $\gpre, \gpost \in \Ast$, are defined by the following inductive grammar:
\[
\begin{rclarray}
	\gpre , \gpost & \defeq & \False \mid \True \mid \gpre \land \gpost \mid \gpre \lor \gpost  \mid \exsts{\lvar}\gpre \mid \Emp \mid \lexpr_1 \pointsto \lexpr_2 \mid  \cass{\kap}{\lrid} \mid \gpre \sep \gpost \mid \boxass{\gpre}{\lrid}{\intass}\\
\end{rclarray}
\]
%
where $\lvar, \lrid \in \LVar$, $\lexpr_1, \lexpr_2 \in \LExpr$ (\defin\ref{def:local_assertions}), $\kap \in \Kaps$ (\defin\ref{def:capabilities}) and $\intass \in \IAst$ (\defin\ref{def:intf}).
Given a logical environment $\lenv \in \LEnv$ and a stack $\stk \in \Stacks$, the \emph{assertion interpretation} function, $\evalW[(., .)]{.}: \Ast \times \LEnv \times \Stacks \rightarrow \World$, is defined as follows:
%
\[
\begin{rclarray}
	\evalW{\False} & \defeq & \emptyset \\
	\evalW{\True} & \defeq & \World \\
	\evalW{\emp} & \defeq & \unitW \\
	\evalW{\gpre \land \gpost} & \defeq & \evalW{\gpre} \cap \evalW{\gpost} \\
	\evalW{\gpre \lor \gpost} & \defeq & \evalW{\gpre} \cup \evalW{\gpost} \\
	\evalW{\exsts{\lvar}  \gpre} & \defeq 
	& \bigcup\limits_{\val \in \textnormal{\Val}} \evalW[\lenv\remapsto{\lvar}{\val}, \stk]{\gpre} \\
	\evalW{\lexpr_1 \pt \lexpr_2} & \defeq & 
    \Setcon{
		(\ca, \h) 
    }{
		\exsts{ \h' } \h = \Set{\evalLE{\lexpr_1} \mapsto \evalLE{\lexpr_2} } \composeH \h' \land \ca \in \unitC
	} \\
	\evalW{\perm{\kap}} & \defeq & 
    \Setcon{
		(\kap, \h) 
    }{
        \h \in \Heaps
	} \\
	\evalW{ \gpre \sep \gpost } & \defeq & 
	\Setcon{
	   (\world_1 \composeW \world_2) 
    }{
       \world_1 \in \evalW{\gpre} \land \world_2 \in \evalW{\gpost}
	}   
\end{rclarray}
\]
\end{definition}
