\subsection{Local/Transaction}

\begin{definition}[Logical Expressions]
\label{def:logical-expr}
Assume a countably infinite set of \emph{logical variables} $\V x \in \LVar$.
The set of \emph{logical expressions}, $ \lexpr \in \LExpr$ is defined by the following inductive grammar, where \(\val \in \Val\) (\defref{def:program_values}), \(\txvar \in \TxVars\) and \( \thvar \in \ThdVars \) (\defref{def:stacks}),
\[
\begin{rclarray}
   \lexpr & ::= & \val \mid \txvar \mid \thvar \mid \lvar \mid \lexpr + \lexpr \mid \lexpr \times \lexpr \mid \dots 
\end{rclarray}
\]
Assume a set of \emph{logical environments} \(\lenv \in \LEnv: \LVar \parfun \Val\) which associates logical variables with values.
Given a stack $\stk \in \Stacks$ (\defin\ref{def:stacks}) and a logical environment $\lenv \in \LEnv$, the \emph{logical expression evaluation} function, $\evalLE[(., .)]{.}:\LExpr \times \Stacks \times \LEnv\rightharpoonup \Val$, is defined inductively over the structure of logical expressions as follows,
%
\[
    \begin{rclarray}
        \evalLE{\val} & \defeq & \val \\
        \evalLE[\lenv, \thstk \uplus \txstk]{\thvar} & \defeq & \thstk(\thvar) \\
        \evalLE[\lenv, \thstk \uplus \txstk]{\txvar} & \defeq & \txstk(\txvar) \\
        \evalLE{\lvar} & \defeq & \lenv(\lvar) \\
        \evalLE{\lexpr_1 + \lexpr_2} & \defeq & \evalLE{\lexpr_1} + \evalLE{\lexpr_2} \\
        \evalLE{\lexpr_1 \times \lexpr_2} & \defeq & \evalLE{\lexpr_1} \times \evalLE{\lexpr_2} \\
        \dots & \defeq & \dots \\
    \end{rclarray}
\]
Note that the stack \( \stk \) includes transaction variables and thread variables.
\end{definition}

\emph{Fingerprint assertion} or \emph{fingerprint} is a set of tuples in the form of \( (\otag, \lexpr_{1}, \lexpr_{2}) \) where \( \otag \) is either read tag \( \etR \) or write \( \etW \) and the second and third elements are logical assertions representing the address and value respectively.
This assertion is interpreted to a set of transaction events as expected.

\begin{defn}[Fingerprint Assertions]
\label{def:fingerprint}
The \emph{fingerprint assertion} also \emph{fingerprint}, \( \fp \in \FAst \), is defined as the follows, 
\[
\begin{rclarray}
    \fp & \subseteq & \Setcon{ (\otag,\lexpr_{1},\lexpr_{2}) }{ \otag \in \OTags \land \lexpr_{1}, \lexpr_{2} \in \LExpr } \\
\end{rclarray}
\] 
Given a logical environment $\lenv \in \LEnv$ and a stack $\stk \in \Stacks$, the \emph{fingerprint interpretation} function, $\evalF[(., .)]{.}: \FAst \times \LEnv \times \Stacks \parfun \powerset{\Events}$, is defined as follows,
\[
\begin{rclarray}
    \evalF{\emptyset} & \defeq & \emptyset  \\
    \evalF{\fp \addF (\otag, \lexpr_{1}, \lexpr_{2})} & \defeq & \evalF{\fp} \addO (\otag, \evalLE{\lexpr_{1}}, \evalLE{\lexpr_{2}})
\end{rclarray}
\]
\end{defn}

The local assertions includes normal separation logic assertions and extra fingerprint assertions, which are interpreted as sets of heaps and a set of events respectively.
Notice that the fingerprint assertion cannot be split.

\begin{definition}[Local assertions]
\label{def:local_assertions}
Given the set of logical expressions \( \LExpr \), logical variables \( \LVar \) and fingerprint assertion \( \FAst \), the set of \emph{local assertions}, $\lpre,  \lpost \in \LAst$, is defined inductively by the following grammar, 
\[
\begin{rclarray}
	\lpre, \lpost  & ::= & \False \mid \True \mid \lpre \land \lpost \mid \lpre \lor \lpost \mid \exsts{\lvar} \lpre \mid \lpre \implies \lpost \mid \Emp \mid \lexpr \pt \lexpr \mid \fpF \mid \lpre \sep \lpost  \\
\end{rclarray}	 
\]
Given a logical environment $\lenv \in \LEnv$, the \emph{local interpretation function}, $\evalLS[(.,.)]{.}: \LAst \times \LEnv \times \LAst \parfun \Heaps \times \powerset{ \Events } $, is defined over the structure of local assertions as follows,
\[
\begin{rclarray}
	\evalLS{\assfalse} & \eqdef & \emptyset \\
	\evalLS{\asstrue} & \defeq & \Heaps \times \powerset{ \Events } \\
	\evalLS{\lpre \land \lpost} & \defeq & \evalLS{\lpre} \cap \evalLS{\lpost} \\
	\evalLS{\lpre \lor \lpost} & \defeq & \evalLS{\lpre} \cup \evalLS{\lpost} \\
	\evalLS{\exsts{\lvar} \lpre} & \defeq & \bigcup\limits_{\val \in \textnormal{\Val}}\evalLS[\lenv\remapsto{\lvar}{\val}, \stk]{\lpre}  \\
	\evalLS{\lpre \implies \lpost} & \defeq & \Setcon{\h}{\h \in \evalLS{\lpre} \implies \h \in \evalLS{\lpost}}\\
	\evalLS{\assemp} & \defeq & \Set{ ( \unitH, \unitE) }  \\
	\evalLS{\lexpr_{1} \pt \lexpr_2 } & \defeq & \Set{ (\evalLE{\lexpr_1} \pt \evalLE{\lexpr_2}, \unitE) } \\
	\evalLS{ \fpF } & \defeq & \Set{ (\unitH, \evalF{\fp}) } \\
	\evalLS{\lpre \sep \lpost} & \defeq & 
    \Setcon{
        (\h_1 \composeH \h_2, \evset_{1} \composeE \evset_{2})
    }{ 
        (\h_{1},\evset_{1}) \in \evalLS{\lpre} 
        \land (\h_{2}, \evset_{2} ) \in \evalLS{\lpost} \\
        \land \evset_{1} = \unitE 
        \lor  \evset_{2} = \unitE 
    } 
\end{rclarray}
\]
\end{definition}

Observe that program expressions $\Expr$  (\defin\ref{def:language}) are contained in logical expressions $\LExpr$ (\defin\ref{def:local_assertions} above), \ie $\Expr \subset \LExpr$. 
For readability, we will write angle brackets, \eg \( \fpass{(\etR, \vx, 0)} \) instead of curly brackets \( \fpto{\Set{(\etR, \vx, 0)}} \) for fingerprint assertions.

\subsection{Global/Program}

\begin{definition}[Capabilities]
\label{def:capabilities}
Assume a \emph{partial commutative monoid (PCM)} of \emph{client-specified capabilities} \( (\Kaps, \composeK, \unitK) \) with \( \kap \in \Kaps \), the composition \( \composeK \) the units set \( \unitK \).
Then given a set of \emph{region identifiers} \( \rid \in \RegionID \), the \emph{capability composition function} or \emph{capabilities} \( \ca \in \Caps \defeq \RegionID \parfun \Kaps \), where the composition \( \composeC \) is defined as the follows,
\[
    \begin{rclarray}
        (\ca_{l} \composeC \ca_{r})(\rid) & \defeq  &
        \begin{cases}
            \ca_{l}(\rid) \composeK \ca_{r}(\rid) & \rid \in \dom(\ca_{l}) \cap \dom(\ca_{l}) \\
            \ca_{l}(\rid)  & \rid \in \dom(\ca_{l}) \setminus \dom(\ca_{l}) \\
            \ca_{r}(\rid) & \rid \in \dom(\ca_{r}) \setminus \dom(\ca_{l}) \\
            \text{undefined} & \text{otherwise} \\
        \end{cases}
    \end{rclarray}
\]
, and the units set \( \unitC \defeq \Setcon{\ca}{\for{\rid} \ca(\rid) \in \unitK } \) .
\end{definition}

\begin{defn}[Interference]
\label{def:intf}
Given the fingerprint assertion \( \fp \in \Fingerprint \) (\defref{def:fingerprint}) and local assertion \( \lpre \in \LAst \) (\defref{def:local_assertions}), the grammar of \emph{interference assertions}, \( \intass \in \IAst \), is defined as the follows,
\[
\begin{rclarray}
	\intass & ::=  &
	\emptyset \mid \Set{ \perm{\kap} :  \exsts{\vec{\lvar}} \lpre \mat \fp } \cup \intass 
\end{rclarray}
\]
It will be interpreted to a set of \emph{interference environments}, this is,
\[
\begin{rclarray}
    \inter \in \Interference & \defeq & \Kaps \parfun \powerset{\Heaps} \times \Opsets
\end{rclarray}
\]
Given a logical environment $\lenv \in \LEnv$ and a stack $\stk \in \Stacks$, the \emph{interference interpretation} function, $\evalI[(., .)]{.}: \IAst \times \LEnv \times \Stacks \to \Interference$, is defined as follows,
%
\[
\begin{rclarray}
	\evalI{\emptyset}(\kap) & \eqdef & \text{undefined} \\
	\evalI{\Set{ \perm{\kap} : \exsts{\vec{\lvar}} \lpre \mat \fp } \cup \intass }(\kap') & \eqdef &
    \begin{cases}
    (\evalLS[\lenv',\stk]{\lpre}, \evalF[\lenv',\stk]{\fp}) \cup \evalI{\intass}(\kap')  & \kap = \kap' \\
    \evalI{\intass}(\kap') & \text{ otherwise} \\
    \end{cases} \\
    & & \text{where there exists a vector of values \( \vec{\val}\) such that } \lenv' = \lenv\rmto{\vec{\lvar}}{\vec{\val}} \\
\end{rclarray}
\] 
\end{defn}

\begin{defn}[Labelled transition system]
The labelled transition system is a tuple, \( (\aexecset,\actionset,\toLTS{}, \aexecset_{0}, \como) \), consisting of a set of abstract executions \( \aexecset \subseteq \Aexecs \), a set of actions \( \actionset \subseteq \Actions \), a relation \( \toLTS{} : \Aexecs \times \Actions \times \Aexecs \), a set of initial abstract executions \( \aexecset_{0}\) and the consistency model associated with the transition system \( \como \).
Assume all the initial abstract executions satisfies the consistency model.
The relation \( \toLTS{}\) is defined as the follows,
\[
\begin{rclarray}
    \aexec \toLTS{\evset} \aexec' & \defeq &
        \begin{array}[t]{@{}l}
        \exsts{\vis, \po, \ar, \txid } \\
        \quad {} \land \aexec' = (\aexec\prjT \uplus \Set{ \txid \mapsto \evset }, \aexec\prjP \uplus \po, \aexec\prjV \uplus \vis, \aexec\prjA \uplus \ar) \\
        \quad {} \land \vis, \po \subseteq \ar = \Setcon{(\txid', \txid)}{\txid' \in \dom(\aexec\prjT)} 
        \land \aexec' \in \evalCOM{\como}
    \end{array}
\end{rclarray}
\]
\end{defn}
 
\begin{defn}[Invariant a region]
\label{def:invariant-region}
\label{def:world2aexec}
\label{def:state2aexec}
Assume two global functions, \( \funcn{init} : \RegionID \to \powerset{\Aexecs} \) that returns initial abstract executions for regions, and \( \funcn{como} : \RegionID \to \ConsisModels \) that returns the consistency models associated with regions.
Also assume all the initial states for a region satisfy the consistency model, \ie
\[
\for{\rid, \aexec_{0}} \aexec_{0} \in \func{init}{\rid} \implies \aexec_{0} \in \evalCOM{\func{como}{\rid}}
\]
The invariant of a region, namely \( \func{transinv}{\rid, \intf} \), is the labelled transition system where the initial state is \( \func{init}{\rid}\) and all the actions are included in the interference.
\[
\begin{rclarray}
    \func{inv}{\rid, \intf} & \defeq & (\aexecset,\actionset \cup \Set{\unitE},\toLTS{}, \func{init}{\rid}, \func{como}{\rid}) \\
    & & \text{where } \for{\evset} \evset \in \actionset \implies \exsts{\kap} \evset \in \dom(\intf(\kap))
\end{rclarray}
\]
For brevity, \( \aexec \in \func{inv}{\rid, \intf} \) is short-hand for \( \aexec \in \aexecset \), and similarly \( \aexec \toLTS{\evset} \aexec' \in \func{inv}{\rid, \intf} \).
\end{defn}

The empty (unit) event \( \unitE \) in the invariant is a place-holder for other regions.
They might append a concrete event, while the composition of abstract executions composes events point-wise if the two executions have the same structure.

\begin{defn}[Well-form of a region]
\label{def:well-form-region}
The well-form condition of the interference, namely \( \pred{wfintf}{\rid, \intf} \) predicate, assertions for any concrete events \( \evset \), the state before the events must be included in the interference.
\[
\begin{rclarray}
    \pred{wfintf}{\rid, \intf} & \defeq & \for{\aexec, \aexec', \evset} \aexec \toLTS{\evset} \aexec' \in \func{inv}{\rid, \intf} \land ( \evset \neq \unitE \implies \pred{wfabs}{\rid, \intf, \aexec, \evset, \aexec'} ) \\ 
    \pred{wfabs}{\rid, \intf, \aexec, \evset, \aexec'} & \defeq & 
    \begin{array}[t]{@{}l}
        \exsts{ \kap }
        \evset \in ( \dom(\intf( \kap )) ) \land \obsstate{\aexec, \aexec\prjT,\func{como}{\rid}\projection{2}} \subseteq \intf(\kap)(\evset)
    \end{array} \\
\end{rclarray}
\]
Given a region (identifier) \(\rid\), its current state \( \h \), and its interference \( \intf \), the function \(\funcn{r2e} \) returns all the possible abstract executions that are included in the invariant of a region and satisfy the state \( \h \),
\[
\begin{rclarray}
    \func{r2e}{\rid, \h, \intf} & \eqdef & \Setcon{\aexec}{\aexec \in \func{inv}{\rid, \intf} \land \h \in \obsstate{\aexec,\aexec\prjT,\func{como}{\rid}\projection{2}}} \\
\end{rclarray}
\]
A set of heaps \( \hset \) approximates the observation of a region \( \rid \) under state \( \h \), namely \( \pred{approx}{\rid, \h, \hset, \intf} \), when an abstract execution satisfies the state \( \h \) and it can reach a new state by appending a new transaction \( \txid \), the observable state of the new transaction must included in the approximation \( \hset \),
\[
\begin{rclarray}
    \pred{approx}{\rid, \aexec, \hset, \intf} & \eqdef & 
    \begin{array}[t]{@{}l}
    \for{ \aexec' } \aexec \toLTS{\stub} \aexec' \in \func{inv}{\rid, \intf} \\
    \quad {} \land \exsts{ \txid } \txid = \max_{\ar}\Set{\aexec'\prjT}
    \land \obsstate{\aexec,\aexec\prjV^{-1}(\txid),\func{como}{\rid}\projection{2}} \subseteq \hset
    \end{array}
\end{rclarray}
\]
\end{defn}

\begin{definition}[Worlds]
\label{def:world}
Given the set of heaps $\Heaps$ (\defref{def:heaps}) and a set of \emph{region identifiers} \( \rid \in \RegionID \), the set of \emph{shared states} is \( \SStates \eqdef \RegionID \to \Heaps \times \powerset{\Heaps} \times \Interference \).
Each region has its current state, a set of possible initial states for transitions and the interference.
The \emph{shared state composition function}, $\composeS: \SStates \times \SStates \parfun \SStates$, is defined as $\composeS \eqdef \composeEq$, where for all domains $\sort M$ and all $m, m' \in \sort M$,
%
\[
\begin{rclarray}
	m \composeEq m' &  \eqdef  &
	\begin{cases}
		m & \text{if } m = m'\\
		\text{undefined} & \text{otherwise}
	\end{cases}
\end{rclarray}
\]
A \emph{world} \( \w \in \World \) is a pair of a shared state \( \gs \) and capabilities \( \ca \) (\defref{def:capabilities}), where regions are associated with the same consistency model and the collapse of the pair exists, \ie regions are well-form and compatible.
\[
\begin{rclarray}
	\world \in \World  & \eqdef & 
    \Setcon{
        (\ca, \gs) 
    }{ 
        \ca \in \Caps 
        \land \gs \in \SStates
        \land \clpsW{\gs} \neq \emptyset
        \land \dom(\ca) \subseteq \dom(\gs) \\
        \quad {} \land \for{\rid, \rid'}
        \func{como}{\rid} = \func{como}{\rid'} \\
        \quad {} \land \for{\h, \h' }
        \h \in \Set{\gs(\rid)\projection{1}} \cup \gs(\rid)\projection{2}
        \land \h' \in \Set{\gs(\rid')\projection{1}} \cup \gs(\rid')\projection{2} 
        \land ( \h \composeH \h' )\isdef
    }
\end{rclarray}
\]
The function, \( \clpsW{.} : \SStates \parfun \powerset{\Aexecs} \), collapses shared states to sets of abstract executions as the follows,
\[
\begin{rclarray}
    \clpsW{\emptyset} & \defeq & \unitAEX \\
    \clpsW{\Set{\rid \mapsto (\h, \hset, \intf)} \uplus \gs } & \defeq & 
        \Setcon{ \aexec \composeAEX \aexec' }{ \aexec \in \func{r2e}{\rid, \h, \intf} \land \pred{approx}{\rid, \aexec, \hset, \intf} \land \pred{wfintf}{\rid, \intf} \land \aexec' \in \clpsW{\gs} }\\
\end{rclarray}
\] 
% 
The \emph{world composition function}, $\composeW: \World \times \World \parfun \World$, is defined component-wise as: $\composeW \eqdef (\composeC, \composeS)$.
The \emph{world unit set} is $\unitW \eqdef \Setcon{(\ca, \gs)}{(\ca, \gs) \in \World \land \ca \in \unitC}$.
The \emph{partial commutative monoid of worlds} is $(\World, \composeW, \unitW)$.
\end{definition}

\sx{point-wise composition}
\begin{defn}[Invariant of worlds]
Because regions in a well-defined world must disjointed with each other and have the same consistency model, it is easy to lift the invariant of a region to a shared state,
\[
\begin{rclarray}
    \func{inv}{\emptyset} & \defeq & (\aexecset, ... ) \\
    \func{inv}{\Set{\rid \mapsto (\stub, \stub, \intf)} \uplus \gs} & \defeq & \Setcon{\aexec \composeAEX \aexec' }{\aexec \in \func{inv}{\rid, \intf} \land \aexec' \in \func{inv}{\gs}} \\
    \func{transinv}{\emptyset} & = & \Setcon{ ( \aexec , \unitE, \aexec' ) }{\aexec, \aexec' \in \unitAEX } \\
    \func{transinv}{\Set{\rid \mapsto (\stub, \stub, \intf)} \uplus \gs} & = & 
    \Setcon{
        ( \aexec \composeAEX \aexec_{f}, \evset \composeE \evset_{f}, \aexec' \composeAEX \aexec_{f}' ) 
    }{
        (\aexec, \evset, \aexec') \in \func{transinv}{\rid, \intf} \\
        \quad \land (\aexec_{f}, \evset_{f}, \aexec_{f}') \in \func{transinv}{\gs}
    }
\end{rclarray}
\]
\end{defn}

\begin{definition}[Assertions]
\label{def:assertion}
Assume standard separation logic assertion \( \bar{\lpre}, \bar{\lpost }\) (the local assertion \( \LAst \) without fingerprint) and the interpretation function, The set of \emph{assertions}, $\gpre, \gpost \in \Ast$, are defined by the following inductive grammar:
\[
\begin{rclarray}
	\gpre , \gpost & \defeq & \False \mid \True \mid \gpre \land \gpost \mid \gpre \lor \gpost \mid \exsts{\lvar}\gpre \mid \gpre \implies \gpost \mid \assemp \mid \cass{\kap}{\lrid} \mid \gpre \sep \gpost \mid \sptboxass{\bar{\lpre}}{\bar{\lpost}}{\lrid}{\intass}\\
\end{rclarray}
\]
%
where $\lvar, \lrid \in \LVar$, $\lexpr_1, \lexpr_2 \in \LExpr$ (\defin\ref{def:local_assertions}), $\kap \in \Kaps$ (\defin\ref{def:capabilities}) and $\intass \in \IAst$ (\defin\ref{def:intf}).
Given a logical environment $\lenv \in \LEnv$ and a stack $\stk \in \Stacks$, the \emph{assertion interpretation} function, $\evalW[(., .)]{.}: \Ast \times \LEnv \times \Stacks \to \powerset{\World}$, is defined as follows:
%
\[
\begin{rclarray}
	\evalW{\False} & \defeq & \emptyset \\
	\evalW{\True} & \defeq & \World \\
	\evalW{\emp} & \defeq & \unitW \\
	\evalW{\gpre \land \gpost} & \defeq & 
    \Setcon{
        (\ca, \gs)
    }{
        \exsts{\gs_{p}, \gs_{q}} 
        (\ca, \gs_{p}) \in \evalW{\gpre} 
        \land (\ca, \gs_{q}) \in \evalW{\gpost} \\
        \quad {} \land \for{\rid} 
        \exsts{\h, \hset_{p}, \hset_{q}, \intf} 
        \gs(\rid) = (\h, \hset_{p} \cap \hset_{q}, \intf) \\
        \qquad {} \land \gs_{p}(\rid) = (\h, \hset_{p}, \intf)
        \land \gs_{q}(\rid) = (\h, \hset_{q}, \intf)
    } \\
	\evalW{\gpre \lor \gpost} & \defeq & 
    \Setcon{
        (\ca, \gs)
    }{
        \exsts{\gs_{p}, \gs_{q}} 
        (\ca, \gs_{p}) \in \evalW{\gpre} 
        \land (\ca, \gs_{q}) \in \evalW{\gpost} \\
        \quad {} \land \for{\rid} 
        \exsts{\h, \hset_{p}, \hset_{q}, \intf} 
        \gs(\rid) = (\h, \hset_{p} \cup \hset_{q}, \intf) \\
        \qquad {} \land \gs_{p}(\rid) = (\h, \hset_{p}, \intf)
        \land \gs_{q}(\rid) = (\h, \hset_{q}, \intf)
    } \\
	\evalW{\exsts{\lvar}  \gpre} & \defeq & \bigcup\limits_{\val \in \textnormal{\Val}} \evalW[\lenv\remapsto{\lvar}{\val}, \stk]{\gpre} \\
	\evalW{\gpre \implies \gpost} & \defeq & \Setcon{\w}{\w \in \evalW{\gpre} \implies \w \in \evalW{\gpost}} \\
	\evalW{\cass{\kap}{\lrid}} & \defeq & \Setcon{ (\Set{\lrid \mapsto \kap}, \gs) }{\gs \in \SStates} \\
	\evalW{ \gpre \sep \gpost } & \defeq & 
	\Setcon{
	   (\world_1 \composeW \world_2) 
    }{
       \world_1 \in \evalW{\gpre} \land \world_2 \in \evalW{\gpost}
	} \\
	\evalW{ \sptboxass{\bar{\lpre}}{\bar{\lpost}}{\lrid}{\intass} } & \defeq & 
    \Setcon{
        (\ca,\Set{\lrid \mapsto (\h, \hset, \intf)} \uplus \gs)
    }{
        \ca \in \unitC 
        \land \h \in \evalLS{\bar{\lpre}}
        \land \hset = \evalLS{\bar{\lpost}}
        \land \intf  = \evalI{\intass}
    } \\
\end{rclarray}
\]
\end{definition}

We will write \( \boxass{\bar{\lpre}}{\lrid}{\intass} \) as a short-hand for \( \sptboxass{\bar{\lpre}}{\bar{\lpre}}{\lrid}{\intass} \) and \(\expr \pt N\) for \( \exsts{\nat \in N} \expr \pt \nat\) where \( N \subseteq \Val\).

