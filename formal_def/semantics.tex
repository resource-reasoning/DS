\sx{
    For Philippa and others who read Andrea's report, this is a re-edit version of the semantics, mainly for the reason of properly definition blocks, references, etc.
    Steal many words from that report and add more explanation.
    Few notations difference,
    \begin{itemize}
        \item address \( \addr \) .
            At some point we use \( l \)  for heap/history heap locations, yet there is a variable clash so using \( \addr \).
        \item Transaction identifier \( \txid \).
        \item \( \hh \)  history heap, and we assume index starts from 1 instead of 0.
            We use superscripts \( \hhV(\addr)(3) \), \( \hhW(\addr)(3) \) and \( \hhR(\addr)(3) \) for the value, write, and reads of the version corresponding to the third version of the address \( \addr \) in the history \( \hh \).
        \item \( \stk \) per-thread stack, and no transaction stack.
        \item \( \vi \) single view, as \( \val \) is for value and \( V \) looks like a set.
        \item \( (\otR, \addr, \val ) \) denotes a operation read address \( \addr \) with value is \( \val \) and similarly the write \( (\otW, \addr, \val ) \)
            Also we use, for example \( \opset \addO (\otR, \stub, \stub) \) instead of \( \oplus \), as the latter looks like a commutative operator??
        \item We call \( \funcn{op}\) instead of \( \funcn{Fprint} \) for the function extract the operation/fingerprint from the primitive commands.
            Because we use \emph{fingerprint} to refer the \emph{fingerprint assertion} later in the logic.
        \item We call \( \func{localHp}{\hh, \vi} \) instead of \( \func{snapshot}{\hh, \vi} \) as the latter is a bit misleading.
        \item \( \func{updHisHp}{\hh, \vi, \txid, \opset} \) instead of \( \funcn{HHupdate}_{\txid}(\hh, \vi, \opset)\).
        \( \func{updView}{\hh, \vi, \opset}\) instead of  \( \func{ViewUpdate}{\hh, \vi, \opset} \), but we also assume the \( \hh \) here is the new one not the old one.
        \item Consistency model \( \como \) as it is a set of quadruples and \( C \) is used for command, so we pick a capital \( \como \).
    \end{itemize}
    }
\section{Semantics\label{sec:semantics}}

\sx{The small intro is stolen from Andrea :)}
We focus on an abstract computational model for database where multi-threaded programs can access and update addresses in a heap through atomic transactions. 
Transactions in our model execute atomically, though they have different effect on database depending on the consistency model that does not necessarily correspond to \emph{serialisability}. 
This means that, at the moment of executing, a transaction may not observe the most up-to-date value of an address. 
To overcome this issue, we model the state of the database using \emph{history heaps}. 
A history heap keeps track of all the versions written for any address, as well as the information about the transactions that read and wrote such versions. 
To model the potential out-of-date observation, we use \emph{views}.
A view decides the observable versions of addresses for a thread.
When executing, a transaction extracts a local state from the history heap and the view, and afterwards the transaction commits a set of operations that might change the history heap and the view, if the change is allowed by the consistency model.

We starts with the syntax of programs followed by the semantics of transaction.
Then we will formally define history heaps and views, and how we specify consistency models.
Finally, we will give the semantics for the entire programs.


\begin{defn}[Program values]
\label{def:program_values}
Assume a countably infinite set of \emph{addresses}, $\addr \in \Addr$. The set of \emph{program values} is $\val \in \Val \eqdef \Nat \cup \Addr$, where $\Nat$ denotes the set of natural numbers.
\end{defn}
 
 \sx{Design choice, total stack vs partial stack, make sure it is consistent.}

\begin{defn}[Stacks]
\label{def:stacks}
Given the program values (\defref{def:program_values}) and a set of \emph{variables} \( \Vars \defeq \Set{\var, \dots}\), a \emph{stack} is a partial function from transaction variables to values \( \txstk, \thstk, \stk \in \Stacks \defeq \Vars \parfun \Val \).
\end{defn}

%{ \color{gray}
%\begin{defn}[Stacks]
%Given the program values (\defref{def:program_values}) and a set of \emph{transaction variables} \( \TxVars \defeq \Set{\txvar, \dots}\), a \emph{transaction stack} is a partial function from transaction variables to values \( \txstk \in \TxStacks \defeq \TxVars \parfun \Val \).
%Similarly, assuming a set of \emph{thread variables} \( \ThdVars \defeq \Set{\thvar, \dots}\), a \emph{thread stack} is defined as \( \thstk \in \ThdStacks \defeq \ThdVars \parfun \Val \).
%Then, the set of \emph{stacks} is defined as the union of transaction stacks and thread stacks \( \stk \in \Stacks \eqdef \TxStacks \uplus \ThdStacks \).
%\end{defn}
%}

\begin{definition}[Heaps]
\label{def:heaps}
Given the sets of program values $\Val$  and addresses \( \Addr\)  (\defin\ref{def:program_values}), the set of \emph{heaps} is: $\h \in \Heaps \eqdef \Addr \parfinfun \Val$.
The \emph{heap composition function}, $\composeH: \Heaps \times \Heaps \parfun \Heaps$, is defined as $\composeH \eqdef \uplus$, where $\uplus$ denotes the standard disjoint function union. The \emph{ heap unit element} is $\unitH \eqdef \emptyset$, denoting a function with an empty domain.
The \emph{partial commutative monoid of  heaps} is $(\Heaps, \composeH, \{\unitH\})$.
\end{definition}

For simplify, a program \( \prog \) contains fixed numbers of top level threads and there is no dynamic fork and join.
Each thread has a unique thread identifier \( \thid \in \ThreadID \) and associated commands.
The commands for each thread, ranged over by $\cmd$, are defined by an inductive grammar comprising the standard constructs of $\pskip$, sequential composition ($\cmd; \cmd$), non-deterministic choice ($\cmd+\cmd$) and loops ($\cmd^*$).
To simulate conditional branching and loop, we have primitive commands assume (\( \passume{\expr}\)) and assignment (\( \passign{\thvar}{\expr} \)), where \( \expr \) denotes arithmetic expressions.
Additionally, the programming language contains the \emph{transaction} construct $\ptrans{\trans}$ denoting the \emph{atomic} execution of the transaction $\trans$. 
The atomicity guarantees the execution are dictated by the underlying consistency model.
\emph{Transactions}, ranged over by $\trans$, are defined by a similar inductive grammar comprising $\pskip$, non-deterministic choice, loops and sequential composition, as well as primitive constructs, ranged over by \( \transpri \), for assignment (\( \passign{\var}{\expr}\)), assume (\( \passume{\expr}\)), lookup (\( \pderef{\expr}{\expr}\)), mutation (\( \pmutate{\expr}{\expr}\)) and return(\( \preturn{\expr}\)). 
Transactions do \emph{not} contain the \emph{parallel} composition construct ($\ppar$) as they are to be executed atomically.
A valid transactions codes, for simplicity, should only have the return at the end.
For better presentation, sometime we omit the default return zero \( \preturn{0} \).
%Transactions can only assign to their own variables, namely transaction variables (\defref{def:program_values}), but it can read from both the thread and transaction stacks.

%\sx{Should mention the return must at the very end.}

\begin{defn}[Programming language]
\label{def:language}
A \emph{program}, $\prog \in \Programs$, is a partial function from thread identifiers to commands.
Given the set of variables \( \var \in \Vars \) (\defref{def:stacks}), the commands \( \cmd \in \Commands \) are defined by the following grammar,
\[
    \begin{rclarray}
    \cmd & ::= &
        \pskip \mid 
        \passign{\thvar}{\expr} \mid
        \passume{\expr} \mid
        \ptrans{\trans} \mid 
        \cmd \pseq \cmd \mid 
        \cmd \pchoice \cmd \mid 
        \cmd \prepeat 
    \end{rclarray}
\]
The $\trans \in \Transactions$ in the grammar above denotes a \emph{transaction} defined by the following grammar.
For simplicity, the transaction codes \( \ptrans{\trans} \) is well-form that there is only a single return at the end, \ie \( \ptrans{\trans} \iff \exsts{\trans'} \trans \equiv \trans' \pseq \preturn{\stub} \land \pred{noRet}{\trans'} \).
\[
    \begin{rclarray}
        \transpri & ::= &
        \pass{\txvar}{\expr} \mid
        \pderef{\txvar}{\expr} \mid
        \pmutate{\expr}{\expr} \mid
        \passume{\expr} \mid
        \preturn{\expr} \\
        \trans & ::= &
        \pskip \mid
        \transpri \mid 
        \trans \pseq \trans \mid
        \trans \pchoice \trans \mid
        \trans\prepeat
    \end{rclarray}
\]
The $\expr \in \Expressions$ denotes an \emph{arithmetic expression} defined by the grammar below with $\val \in \Val$ (\defin\ref{def:program_values}),
\[
    \begin{rclarray}
        \expr & ::= &
        \val \mid
        \var \mid
        \expr + \expr \mid
        \expr \times \expr \mid
        \dots 
    \end{rclarray}
\]
Given a stack $\stk \in \Stacks$ (\defin\ref{def:stacks}), the \emph{arithmetic expression evaluation} function, $\evalE[(.)]{.}:\Expressions \times \Stacks \parfun \Val$, is defined inductively over the structure of expressions as follows: 
%
\[
    \begin{rclarray}
        \evalE{\val} & \defeq & \val \\
        \evalE{\var} & \defeq & \stk(\var) \\
        \evalE{\expr_{1} + \expr_{2}} & \defeq & \evalE{\expr_{1}} + \evalE{\expr_{2}} \\
        \evalE{\expr_{1} \times \expr_{2}} & \defeq & \evalE{\expr_{1}} \times \evalE{\expr_{2}} \\
        \dots & \eqdef & \dots \\
    \end{rclarray}
\]
\end{defn}

\subsection{Local/Transaction Semantics}

The operational semantics for transactions \(\trans\) with respect to a configuration of the form \((\stk, \h, \opset)\) comprising a stack, a (local) heap and a set of operations that might affect other transactions.
The operations are the first read and the last write of each address.
This means a valid set of operations contains at most one read and one write for each address.

\sx{Maybe design choice: we constraint the composition for operations. If we allowed arbitrary composition of two operations, we will need some side conditions for the frame rule within the transactions.}
\begin{defn}[Operation and valid operations]
\label{def:transaction-event}
\label{def:transactions}
Assume a set of \emph{operations tags}, a \emph{transaction operation} \( \op \in \Ops \) is a tuple of an operation tag, an address and a value.
The tuple represents either read or write of the address.
The \emph{operation tags} \( \etR \) and \( \etW \) correspond to read and write respectively.
\[
\begin{rclarray}
\OTags & \defeq & \Set{\otR, \otW} \\
\op \in \Ops & \defeq  & \OTags \times \Addr \times \Val
\end{rclarray}
\]
\emph{a well-formed set of operations}, \( \opset \in \Opsets \), is a subset of \( \Ops \) in which any two elements contain either different tags or different address.
\[
    \begin{rclarray}
        \Opsets & \defeq & \Setcon{\opset}{\opset \subseteq \Events \land \wfO{\opset} } \\
        \wfO{\opset} & \defeq & \for{\op, \op' \in \opset} \op\projection{1} \neq  \op'\projection{1} \lor \op\projection{2} \neq  \op'\projection{2}
    \end{rclarray}
\]
The unit element is \( \unitE \defeq \emptyset\) and the composition of two set of operations is only defined when the two sets contains disjointed addresses,
\[ 
\begin{rclarray}
    \opset \composeO \opset' & \defeq & 
    \begin{cases}
        \opset \uplus \opset' & \text{if } \opset\projection{2} \cap \opset'\projection{2} = \emptyset \\
        \text{undefined} & \text{otherwise}
    \end{cases}
\end{rclarray}
\]
A partial binary operation \( \addO \) adds a new operation to valid operations \( \opset \) that ensures the set contains the first read and last write.
For technical reason, if the right hand side is a special token \( \emptyop \), which represents no operation, the operations remains the same.
\[
\begin{rclarray}
    \opset \addO (\etR, \addr, \val) & \defeq & 
    \begin{cases}
        \opset \uplus \Set{(\etR, \addr, \val)} & (\stub, \addr, \stub) \notin \opset \\
        \opset &  \text{otherwise} \\
    \end{cases} \\
    \opset \addO (\etW, \addr, \val) & \defeq & \left( \opset \setminus \Set{(\etW, \addr, \stub)} \right) \uplus \Set{(\etW, \addr, \val)} \\
    \opset \addO \emptyop & \defeq & \opset \\
\end{rclarray}
\]
\end{defn}

\begin{lem}
The \( \addO \) operator preserves the well-form property.
\end{lem}

We define a labelled transition system for primitive transaction commands where the labels are commands.
The \( \funcn{op} \) extracts the read or write operation from primitive transition commands, otherwise returns \( \emptyop \).

\begin{defn}[Transaction operational semantics]
Given the set of stacks \( \Stacks \) (\defref{def:stacks}), heaps \( \Heaps \) (\defin\ref{def:heaps}) and transactions \( \Transactions \) (\defin\ref{def:language}), the \emph{operational semantics of transactions}, 
\[
\begin{rclarray}
\toL & : & \ThdStacks \times \\
& & \quad ((\TxStacks \times \Heaps \times \Opsets) \times \Transactions) \times ((\TxStacks \times \Heaps \times \Opsets) \times \Transactions)
\end{rclarray}
\]
is given in \fig\ref{fig:transaction_semantics}.
Note that arithmetic expression evaluation \( \evalE{\expr} \) is defined in \defref{def:language} and the \( \addO \) operators is defined in \defref{def:transactions}.
\end{defn}

\begin{figure}[!t]
\hrule\vspace{5pt}
\[
\begin{array}{@{} r c l r  c l @{}}
    (\stk, \h) & \toLTS{\passign{\var}{\expr}} & (\stk\rmto{\var}{\evalE{\expr}}, \h) & \func{op}{\stk, \h, \passign{\var}{\expr}} & \defeq & \emptyop \\
    (\stk, \h) & \toLTS{\pderef{\var}{\expr}} & (\stk\rmto{\var}{\h(\evalE{\expr})}, \h) & \func{op}{\stk, \h, \pderef{\var}{\expr}} & \defeq & (\etR, \evalE{\expr}, \h(\evalE{\expr})) \\
    (\stk, \h) & \toLTS{\pmutate{\expr_{1}}{\expr_{2}}} & (\stk, \h\rmto{\evalE{\expr_{1}}}{\evalE{\expr_{2}}}) & \func{op}{\stk, \h, \pmutate{\expr_{1}}{\expr_{2}}} & \defeq & (\etW, \evalE{\expr_{1}}, \evalE{\expr_{2}}) \\
    (\stk, \h) & \toLTS{\passume{\expr}} & (\stk, \h) \text{ where } \evalE{\expr} = 0 & \func{op}{\stk, \h, \passume{\expr}} & \defeq & \emptyop \\
    (\stk, \h) & \toLTS{\preturn{\expr}} & (\stk\rmto{\ret}{\evalE{\expr}}, \h) & \func{op}{\stk, \h, \preturn{\expr}} & \defeq & \emptyop \\
\end{array}
\]
\hrule\vspace{5pt}
\[	
    \infer[\rl{TPrimitive}]{%
        \vdash (\stk, \h, \opset) , \transpri \ \toL \  (\stk', \h', \opset \addO \op) , \pskip
    }{%
        (\stk, \h) \toLTS{\transpri} (\stk', \h')
        && \op = \func{op}{\stk, \h, \transpri}
    }
\]

\[
    \infer[\rl{TChoice}]{%
        \stk \vdash (\txstk, \h, \opset) , \trans_{1} \pchoice \trans_{2} \ \toL \  (\stk, \h, \opset) , \trans'
    }{
        \trans' \in \Set{\trans_{1}, \trans_{2}}
    }
\]

\[
    \infer[\rl{TLoop}]{%
        \stk \vdash (\txstk, \h, \opset),  \trans\prepeat \ \toL \  (\stk, \h, \opset), \pskip \pchoice (\trans \pseq \trans\prepeat)
    }{}
\]


\[
    \infer[\rl{TSeqSkip}]{%
        \stk \vdash (\txstk, \h, \opset), \pskip \pseq \trans \ \toL \  (\stk, \h, \opset), \trans
    }{%
    }
\]

\[
    \infer[\rl{TSeq}]{%
        \stk \vdash (\txstk, \h, \opset), \trans_{1} \pseq \trans_{2} \ \toL \  (\stk', \h', \opset'), \trans_{1}' \pseq \trans_{2}
    }{%
        \stk \vdash (\txstk, \h, \opset), \trans_{1} \ \toL \  (\stk', \h', \opset'), \trans_{1}'
    }
\]

\hrule\vspace{5pt}
\caption{The transaction operational semantics}
\label{fig:transaction_semantics}
\end{figure}


%\subsection{Program Semantics}

To model the global machine states, instead of heap-based states, we use \emph{abstract executions} (\defref{def:abs-exec}).
A \emph{abstract execution} is a graph where each node represents a committed transaction with a unique transaction identifier, and its associated events that have global effect, \ie the first reads and last writes.
There are three types of edges in the graph, a \emph{program order} that is a total order for transactions from the same thread, a \emph{visibility relation} that decides the observable history (a set of transactions) for each transaction, and an \emph{arbitration order} that decides the actual global state that is not necessary to be the same as the observable state for each transaction \cite{eventually-consistent-transactions,Burckhardt:2014:RDT:2535838.2535848,cerone_et_al:LIPIcs:2015:5375}.

\begin{defn}[Runtime abstract executions and abstract executions]
\label{def:run-abs-exec}
\label{def:abs-exec}
Assuming a set of \emph{transactions identifiers} \( \TxID \defeq \Set{\txid, \dots}\), the \emph{transactions} \( \tx \in \Tx \) is defined as a finite partial function from transactions identifiers \( \TxID \) to valid sets of events \( \Evsets \),
\[
\begin{rclarray}
\tx \in \Tx & \defeq & \TxID \parfinfun \Evsets
\end{rclarray}
\]
An \emph{runtime abstract execution} is a tuple \( \aexecrun = (\tx, \setthid, \porun, \vis, \ar) \in \Aexecrun \) that satisfies the following conditions,
\begin{itemize}
\item
The \emph{transactions} \( \tx \) is a partial function from transaction identifiers to their corresponding events (\defref{def:transactions}).
\item 
Assume a countably infinite set of thread identifiers \( \thid \in \setthid \subseteq \ThreadID \). 
The \emph{runtime threads} \( \setthid \) is a set of transaction identifiers.
\item 
The \emph{arbitration order} $\ar \subseteq \dom(\tx) \times \dom(\tx)$ is a strict, total order%
\footnote{Recall that a relation $R \subseteq A \times A$ is a strict partial order if it is irreflexive and transitive.
It is a strict total order if for any $a_1, a_2 \in A$, either $a_1 = a_2$, $(a_1, a_2) \in R$ or $(a_2, a_1) \in R$.}.
\item 
The \emph{runtime program order} $\porun \subseteq \dom(\tx) \times ( \dom(\tx) \uplus \setthid )$ is the union of several disjoint, strict total orders \( \porun_{i} \).
That is, there exists a partition $\Set{ \dom(\tx)_{i} }_{i \in I}$ of $\dom(\tx)$ such that $\porun = \biguplus_{i \in I} \porun_{i}$, where $\porun_{i}$ is a strict, total order.
It also requires \( \po \subseteq \ar\).
\item 
The \emph{visibility relation} $\vis \subseteq \dom(\tx) \times \dom(\tx)$ is a relation such that \( \vis \subseteq \ar \).
\end{itemize} 
The set of \emph{abstract executions} $\aexec  = (\tx, \po, \vis, \ar) \in \Aexecs$, where the \emph{program order} \( \po \subseteq \dom(\tx) \times \dom(\tx)\), is defined by erasing the runtime threads from the runtime abstract executions \( \Aexecrun \),
\[
\begin{rclarray}
    \Aexecs & \defeq & \Setcon{\eraseAEX{\aexecrun}}{\aexecrun \in \Aexecrun} \\
\end{rclarray} 
\]
where the erasing function \( \eraseAEX{.}: \Aexecrun \to \Aexecs \) converts a runtime abstract execution to an abstract execution by erasing the second element, \ie the set of thread identifiers, and also any thread identifiers from the runtime program order,
\[
    \begin{rclarray}
        \eraseAEX{(\tx, \setthid, \porun, \vis, \ar)} & \defeq & (\tx, \hat{\po} \setminus \Setcon{(\txid, \thid)}{ \txid \in \dom(\tx) \land \thid \in \setthid}, \vis, \ar)
    \end{rclarray}
\]
\begin{figure}
\centering
\begin{tikzpicture}
\node[draw] (n1) { \( \txid_{1}:\emptyset\) };
\node[draw,below=1cm of n1] (n2) { \( \txid_{2}:\emptyset\) };
\node[draw,below=1cm of n2] (n3) { \( \txid_{3}:\emptyset\) };
\coordinate (n1n3) at ($(n2) + (1.5,0)$);
\draw[->] (n1) -- (n2) node[pos=0.5]{\ar,\vis};
\draw[->] (n2) -- (n3) node[pos=0.5]{\ar};
\draw[->] (n1) to[out=-20, in=90] (n1n3) to[out=-90,in=20]  (n3);
\node at (n1n3) {\(\ar, \vis\)};
\node[below] at (current bounding box.south) {\(\aexec\)};
\end{tikzpicture}
%
\begin{tikzpicture}
\node[draw] (n1) { \( \txid_{1}:\Set{(\etW, 1, 0), (\etW, 2, 0)}\) };
\node[draw,below=1cm of n1] (n2) { \( \txid_{2}:\Set{(\etR, 1, 0), (\etW, 2, 1)}\) };
\node[draw,below=1cm of n2] (n3) { \( \txid_{3}:\Set{(\etW, 1, 1), (\etR, 2, 0)}\) };
\coordinate (n1n3) at ($(n2) + (2.5,0)$);
\draw[->] (n1) -- (n2) node[pos=0.5]{\ar,\vis};
\draw[->] (n2) -- (n3) node[pos=0.5]{\ar};
\draw[->] (n1) to[out=-20, in=90] (n1n3) to[out=-90,in=20]  (n3);
\node at (n1n3) {\(\ar, \vis\)};
\node[below] at (current bounding box.south) {\(\aexec'\)};
\end{tikzpicture}
\caption{The abstract execution \( \aexec \) is a element of the unit set \( \unitAEX\), and here \( \aexec \composeAEX \aexec' = \aexec' \) as the two abstract executions have the same structure the compositions of each nodes exist.}
\end{figure}
For brevity, the \( \aexecrun\prjT \), \( \aexecrun\prjI \), \( \aexecrun\prjP \), \( \aexecrun\prjV \) and \( \aexecrun\prjA \) denote the corresponding elements in the tuple and similarly for \( \aexec\projection{(.)} \).
The composition of two runtime abstract executions, \( \composeAEXRUN : \Aexecrun \times \Aexecrun \parfun \Aexecrun \), is defined as the follows,
\[
\begin{rclarray}
    \aexecrun_{1} \composeAEXRUN \aexecrun_{2} & \defeq & 
    \begin{cases}
        \left( \lambda \txid \ldotp \aexecrun_{1}\prjT(\txid) \composeE \aexecrun_{2}\prjT(\txid), \aexecrun_{1}\prjI, \aexecrun_{1}\prjP, \aexecrun_{1}\prjV, \aexecrun_{1}\prjA \right) & \dagger \\
        \text{undefined} & \text{ otherwise}
    \end{cases} \\
    \dagger & \equiv &  
    \begin{array}[t]{@{}l@{}}
        \dom(\aexecrun_{1}\prjT) = \dom(\aexecrun_{2}\prjT)
        \land \aexecrun_{1}\prjI = \aexecrun_{2}\prjI \\
        \quad {} \land \aexecrun_{1}\prjP = \aexecrun_{2}\prjP
        \land \aexecrun_{1}\prjV = \aexecrun_{2}\prjV
        \land \aexecrun_{1}\prjA = \aexecrun_{2}\prjA \\
    \end{array} \\
\end{rclarray}
\]
The set of the units is \( \unitAEX \defeq \Setcon{\aexec}{\for{\txid} \aexec\prjT(\txid) = \unitE} \).
Then, the order between two runtime abstract executions \( \aexecrun_{1} \) and \( \aexecrun_{2} \) is defined as point-wise set inclusions,
\[
\begin{rclarray}
\aexecrun_{1} \ordAEXRUN \aexecrun_{2} & \iffdef & 
    \begin{array}[t]{@{}l@{}}
        ( \for{\txid} \aexecrun_{1}\prjT(\txid) \implies \aexecrun_{2}\prjT(\txid) )  
        \land \aexecrun_{1}\prjI \subseteq  \aexecrun_{2}\prjI  \\
        {} \quad \land \aexecrun_{1}\prjP \subseteq  \aexecrun_{2}\prjP 
        \land \aexecrun_{1}\prjV \subseteq  \aexecrun_{2}\prjV 
        \land \aexecrun_{1}\prjA \subseteq  \aexecrun_{2}\prjA 
    \end{array}
\end{rclarray}
\]
Last, by erasing the runtime, the order between two abstract executions is defined as the follows,
\[
\begin{rclarray}
\aexec_{1} \ordAEX \aexec_{2} & \iffdef & 
    \exsts{\aexecrun_{1}, \aexecrun_{2}} \aexec_{1} = \eraseAEX{\aexecrun_{1}}
    \land \aexec_{2} = \eraseAEX{\aexecrun_{2}}
    \land \aexecrun_{1} \ordAEXRUN \aexecrun_{2}
\end{rclarray}
\]
\end{defn}

We parametrise the consistency models in our semantics.
A \emph{consistency model} contains two parts, a \emph{resolution policy} and a \emph{consistency guarantee} (\defref{def:consistency-models}) \cite{cerone_et_al:lipics:2017:7794}.
Given a abstract execution \( \aexec \), a set of observable transactions \( \txidset \) and an address \( \addr \), the \emph{resolution policy} \( \respo(\aexec, \txidset, \addr) \) decides the observable values for the address \( \addr \) through some computation on the observable transactions \( \txidset \).
A common resolution policy is \emph{last-write-win} that if a transaction observes several writes for the same address, it always reads the last write (by the arbitration order).
The consistency guarantee gives the minimum constraint for the visibility relation.


\begin{defn}[Consistency Models]
\label{def:consistency-models}
A \emph{Consistency model} is a tuple, \( \como = (\respo, \conguar) \in \Como\), including a \emph{resolution policy} and a \emph{consistency guarantee}.
A \emph{resolution policy} \( \respo \) is a function such that for a given address \( \addr \) and a set of observable transactions from a abstract execution, it returns a set of possible observable values.
\[
\begin{rclarray}
    \ResPos & \defeq & 
    \Setcon{%
        \respo
     }{%
        \respo \in \Aexecs \times \powerset{\TxID} \times \Addr \to \powerset{\Val}\\
        \quad {} \land \for{\aexec, \txidset, \addr } \respo(\aexec, \txidset, \addr)\isdef \implies \txidset \subseteq \dom(\aexec\prjT)
    }
\end{rclarray}
\]
A \emph{consistency guarantee} \( \conguar \) is a function such that for an abstract execution, it returns a relation which corresponds to the minimum visibility relation.
\[ 
\begin{rclarray}
\ConGuar & \defeq & 
\Setcon{%
        \conguar
    }{%
        \conguar \in \Aexecs \to \powerset{\TransID \times \TransID}
        \land \for{\aexec} \\
        \quad \conguar(\aexec) \subseteq \aexec\prjA
        \land \for{\txid, \txid'} (\txid, \txid') \in \conguar(\aexec) 
        \land \txid,\txid'  \in \dom(\aexec\prjT)
        
    }
\end{rclarray}
\]
Note that a well-formed consistency guarantee must not violate the arbitrary order.
The order \( \ordCOM \)  between two consistency model \( \como_{1}, \como_{2} \) is defined as the follows,
\[
\begin{rclarray}
    (\respo_{1}, \conguar_{1}) \ordCOM (\respo_{2}, \conguar_{2}) & \iffdef & 
    \begin{array}[t]{@{}l@{}}
    \for{\aexec, \txidset, \addr} \\
    \quad \respo_{2}(\aexec, \txidset, \addr) \subseteq \respo_{1}(\aexecrun, \txidset, \addr) \land \conguar_{1}(\aexec) \subseteq  \conguar_{2}(\aexec)
    \end{array} \\
\end{rclarray}
\]
The bottom element is \( \btmCOM \defeq (\lambda (\aexec, \txidset, \addr ) \ldotp \Val, \lambda (\aexec) \ldotp \emptyset) \), which means one is able to observe any arbitrary value for each address and there is no constraint for visibility relation.
\end{defn}

\begin{example}[Last write win]
\[
\begin{rclarray}
        \respo_{LWW}(\aexec, \txidset, \addr) & \defeq & 
        \Setcon{\val}{%
            \exsts{\ar = \aexec\prjA}
            \Set{(\etW, \addr, \val)} = \max_{\ar}\Setcon{\txid}{\txid \in \txidset \land (\etW, \addr, \stub) \in \tx(\txid) } \\
            \quad {} \lor \emptyset = \max_{\ar}\Setcon{\txid}{\txid \in \txidset \land (\etW, \addr, \stub) \in \tx(\txid) } \land v = 0
        }\\
\end{rclarray}
\]
\end{example}

\begin{example}[Write-write conflict]
\[
\begin{rclarray}
        \conguar_{WW}(\aexec) & \defeq & \Setcon{(\txid, \txid')}{ \exsts{ \addr } (\etW, \addr, \stub) \in \aexec\prjT(\txid) \land (\etW, \addr, \stub) \in \aexec\prjT(\txid') \land (\txid, \txid') \in \aexec\prjA } \\
\end{rclarray}
\]
\end{example}

Given two sets of relations \( A \) and \( B \), the notation \( A ; B \) denotes that \( A;B \defeq \Setcon{(a,b)}{(a,c) \in A \land (c,b) \in B} \).

\begin{example}[Serialisibility(SER)]
\[
    \begin{rclarray}                                   
        \respo_{SER} & \defeq & \respo_{LWW} \\
        \conguar_{SER}(\aexec) & \defeq & \aexec\prjA \\
    \end{rclarray}                                                      
\]
\end{example}
%\ac{Why $\como_{SER}$ and not $\respo_{SER}?$ Also, the resolution policy is the last write wins, which you need to define only once.}

\begin{example}[Snapshot isolation(SI)]
\[
    \begin{rclarray}                                   
        \respo_{SI} & \defeq & \respo_{LWW} \\
        \conguar_{SI}(\aexec) & \defeq & ( \aexec\prjA ; \aexec\prjV )  \cup \conguar_{WW}(\aexec) \\
    \end{rclarray}
\]
\end{example}
%\ac{Why not just say that $\conguar_{SI}(\aexec) = \aexec\prjV ; \aexec\prjA$? Also, you are missing 
%write conflict detection ($\conguar_{\mathsf{WWconf}} = \bigcup_{\addr \in \Addr} [\mathsf{Write}_\addr] ; \aexec\prjA ; [\mathsf{Write}_\addr]$.}

\begin{example}[Parallel snapshot isolation(PSI)]
\[
    \begin{rclarray}                                   
        \respo_{PSI} & \defeq & \respo_{LWW} \\
        \conguar_{PSI}(\aexec) & \defeq & ( \aexec\prjV ; \aexec\prjV ) \cup \conguar_{WW}(\aexec) \\
    \end{rclarray}
\]
\end{example}

\begin{defn}[Thread transition labels]
\label{def:label}
Given the set of thread identifiers \(\ThreadID\) (\defref{def:abs-exec}), the set of \emph{thread transition labels}, $\lb \in \Translabel$, is defined by the following grammar, where $\prog$ denotes a program (\defref{def:language}), the $\txid$ demotes a transaction identifier and $\thstk$ denotes a thread stack (\defref{def:stacks}),
\[
    \begin{rclarray}
	\iota \in \Translabel & ::= & \lbID \mid \lbC{\txid} \mid \lbF{\thid,\prog} \mid \lbJ{\thid,\thstk}
    \end{rclarray}
\]
\end{defn}

\begin{defn}[Thread semantics]
\label{def:thread_semantics}
Given the thread identifiers $\thid \in \ThreadID$, the set of \emph{intermediate programs}, $\iprog \in \IntermediatePrograms$, is defined by the following grammar:
\[
    \iprog ::= \prog \mid \iprog \pseq \pwait{\thid}
\]
Given the set of consistency model \( \ConsisModels \) (\defref{def:consistency-models}), the set of thread stacks \( \ThdStacks \) (\defref{def:stacks}) and runtime abstract executions \( \Aexecrun \) (\defref{def:run-abs-exec}), the \emph{per-thread operational semantics} of programs,
\[
\begin{rclarray}
	\toT{} & : &
    \begin{array}[t]{@{}l@{}}
    \Como \times \ThreadID 
    \times \\
	\quad \left( ( \ThdStacks \times \Aexecrun ) \times \IntermediatePrograms \right) 
	\times  \Translabel \times
	\left( ( \ThdStacks \times \Aexecrun ) \times \IntermediatePrograms \right) 
    \end{array}
\end{rclarray}
\]
is defined in \figref{fig:thread_semantics}.
\end{defn}

The \rl{PCommit} rule ``substitutes'' the dummy node for the thread \( \thid \) in the runtime abstract execution \( \aexecrun \) with a newly allocated transaction \( \txid \) with its associated events \( \evset \).
To obtain the events \( \evset \), it prophesies a set of observable transactions \( \txidset \), which will be added into the visibility relation later.
Given the observable transactions, the \( \obsstateName \) function computes possible initial heaps, by applying the resolution policy \( \respo \) for each address.
Note that the resolution policy might return more than one value for an address, so there are more than many possible initial heaps.
Given the transaction code \( \trans \), first pick an initial heap \( \h \) and then given the transaction semantics (\figref{fig:transaction_semantics}), we can get the events set \( \evset \).
To \emph{extend the runtime abstract execution}, it replaces the dummy node \( \thid \) with the new transaction \( \txid \), links all transaction from the observable transactions \( \txidset \) to the new transaction, and puts the new transaction at the end of arbitration order.
Then, it extends the program order by adding back the dummy node \( \thid \) after the transaction \( \txid \) for preserving program order for the future transactions from the same thread.

The \rl{Par} rule forks a new thread and inserts the appropriate joining point by appending the auxiliary \( \pwait{\thid} \) command, where the parameter \(\thid\) denotes the identifier of the newly forked thread.
The \rl{Wait} rule dually awaits the termination of the child thread \(\thid\) indicated by the auxiliary \( \pwait{\thid} \) command, and subsequently updates the thread stack.
Note that these two rules are labelled with the \(\lbF{\thid, \prog}\) and \(\lbJ{\thid, \thstk}\) which are used by the semantics of the thread pool described shortly (\figref{fig:thread_pool_semantics}).

\begin{figure}
%
\hrule
%
\[
    \infer[\rl{PCommit}]{%
        (\respo, \conguar), \thid \vdash ( \thstk, \aexecrun ), \ptrans{\trans} \ \toT{\lbC{\txid}} \ ( \thstk\rmto{\ret}{\txstk(\ret)}, \aexecrun' ) , \pskip
    }{%
        \begin{array}{c}
            \txid \in \func{fresh}{\aexecrun}
            %\quad \txidset \in \addpotvis{\aexecrun, \thid}
            \quad \txidset \subseteq \dom(\aexecrun\prjT)
            \quad \h \in \obsstate{\aexecrun, \txidset, \respo} \\
            %\quad \fph = \lambda \addr \ldotp  (\respo(\eraseAEX{\aexecrun}, \txidset , \addr), \emptyset) \\
            \thstk \vdash (\emptyset, \h, \emptyset) , \trans \ \toL^{*} \  (\txstk, \h', \evset) , \pskip 
            %\quad \evset = \getevent{\fph, \fph'}
            \quad \aexecrun' = \newaexec{\aexecrun, \thid, \evset, \txid, A, \conguar }
        \end{array}
    }
\]

\[
    \infer[\rl{PAssign}]{%
        \como, \thid \vdash ( \thstk, \aexecrun ) , \passign{\thvar}{\expr} \ \toT{\lbID} \  ( \thstk\rmto{\thvar}{\val}, \aexecrun  ) , \pskip
    }{
        \val = \evalE[\thstk]{\expr}
        && \thvar \in \ThdVars
    }
\]

\[
    \infer[\rl{PAssume}]{%
        \como, \thid \vdash ( \thstk, \aexecrun ) , \passume{\expr} \ \toT{\lbID} \  ( \thstk, \aexecrun ) , \pskip
    }{%
        \evalE[\thstk]{\expr} = 0
    }
\]

\[
    \infer[\rl{PChoice}]{%
        \como, \thid \vdash ( \thstk, \aexecrun ) , \prog_{1} \pchoice \prog_{2} \ \toT{\lbID} \  ( \thstk, \aexecrun ) , \prog'
    }{
        \prog' \in \Set{\prog_{1}, \prog_{2}}
    }
\]

\[
    \infer[\rl{PLoop}]{%
        \como, \thid \vdash ( \thstk, \aexecrun ) , \prog\prepeat \ \toT{\lbID} \  ( \thstk, \aexecrun ) , \pskip \pchoice (\prog \pseq \prog\prepeat)
    }{}
\]

\[
    \infer[\rl{PSeqSkip}]{%
        \como, \thid \vdash ( \thstk, \aexecrun ) , \pskip \pseq \iprog \ \toT{\lbID} \  ( \thstk, \aexecrun ) , \iprog
    }{}
\]

\[
    \infer[\rl{PSeq}]{%
        \como, \thid \vdash ( \thstk, \aexecrun ) , \iprog_{1} \pseq \iprog_{2} \ \toT{\lb} \ ( \thstk', \aexecrun' ) , {\iprog_{1}}' \pseq \iprog_{2}
    }{%
        \como, \thid \vdash ( \thstk, \aexecrun ) , \iprog_{1} \ \toT{\lb} \  ( \thstk', \aexecrun' ) , {\iprog_{1}}' 
    }
\]

\[
    \infer[\rl{PPar}]{%
        \como, \thid \vdash ( \thstk, \aexecrun ) , \prog_{1} \ppar \prog_{2} \ \toT{\lbF{\thid', \prog_{2}}} \  ( \thstk, \aexecrun' ) , \prog_{1} \pseq \pwait{\thid'}
    }{
        \aexecrun' = \func{extend\_thread}{\aexecrun, \thid, \thid'}
    }
\]

\[
    \infer[\rl{PWait}]{%
        \como, \thid \vdash ( \thstk, \aexecrun ) , \pwait{\thid'} \ \toT{\lbJ{\thid', \thstk'}} \  (  \thstk_{1} \uplus \thstk_{2} \uplus \thstk_{f}, \aexecrun' ) , \pskip 
    }{
        \thstk = \thstk_{1} \uplus \thstk_{f}
        && \thstk' = \thstk_{2} \uplus \thstk_{f}
        && \aexecrun = \func{erase\_thread}{\aexecrun', \thid'}
    }
\]

\sx{The stack could be polluted by the child thread, but they must agree?}
 
where,
\[
\begin{rclarray}                                 
    \consis{\aexecrun, \thid, \txid, \conguar} & \defeq & 
    \begin{array}[t]{@{}l@{}}
        \thid \in \aexecrun\prjI \land  \for{ \txid' } ( (\txid', \txid) \in \conguar(\eraseAEX{\aexecrun}) \implies (\txid', \txid) \in \aexecrun\prjV ) \\
    \end{array} \\
    \newaexecName & : & 
    \left(\begin{array}{l}
        \Aexecrun \times \ThreadID \times \TxID \times \\
        \quad  \powerset{\Events} \times \powerset{\TxID} \times \ConGuar \end{array} \right)
        \parfun \Aexecrun \\
    \newaexec{\aexecrun, \thid, \txid, \evset, \txidset, \conguar } & \defeq & 
    \begin{cases}
        \aexecrun' & \consis{\aexecrun', \thid, \txid, \conguar} \\
        \text{undefined} & \text{otherwise} \\
    \end{cases} \\
    \aexecrun' & \equiv & 
        \left(
        \begin{array}{@{}l@{}}
            \aexecrun\prjT \uplus \Set{\txid \mapsto \evset},
            \aexecrun\prjI, 
            \aexecrun\prjP \uplus \Setcon{(\txid', \txid)}{(\txid', \thid) \in \aexecrun\prjP} \uplus \Set{(\txid, \thid)}, \\
            \quad \aexecrun\prjV \uplus \Setcon{(\txid', \txid)}{\txid' \in \txidset}, 
            \aexecrun\prjA \uplus \Setcon{(\txid', \txid)}{\txid' \in \dom(\aexecrun\prjT)}
        \end{array}
        \right) \\
%	
%
    \obsstateName & : & \Aexecrun \times \powerset{\TxID} \times \ResPos \parfun \powerset{\FPHeaps} \\
    \obsstate{\aexecrun, \txidset, \respo} & \defeq & 
    \Setcon{%
        \h
    }{%            
        \for{\addr, \val}  \val \in \respo(\eraseAEX{\aexecrun}, \txidset, \addr) \iff \h(\addr) = \val \\
    } \\
%
%              
	\func{fresh}{\aexecrun}  & \defeq & \Setcon{ \txid }{ \neg \txid \in \dom(\aexecrun\prjT) } \\
%
%
    %\geteventName & : & \FPHeaps \times \FPHeaps \to \powerset{\Events} \\
    %\getevent{\fph, \fph'} & \defeq & 
    %\begin{array}[t]{@{}l@{}}
        %\Setcon{ (\etW, \addr, \val ) }{ \exsts{ \fp } \fph(\addr) = ( \val, \fp ) \land \fpW \in \fp} \\
        %\quad {} \uplus \Setcon{ (\etR, \addr, \val ) }{ \exsts{ \fp } \fph(\addr) = (\val, \stub) \land \fph(\addr) = ( \stub, \fp ) \land \fpR \in \fp} \\ 
    %\end{array} \\
%
%
    \func{extend\_thread}{\aexecrun, \thid, \thid'} & \defeq & (\aexecrun\prjT, \aexecrun\prjI \uplus \Set{\thid'}, \aexecrun\prjP \uplus \Setcon{(\txid, \thid')}{ (\txid, \thid) \in \aexecrun\prjP}, \aexecrun\prjV, \aexecrun\prjA) \\
    \func{erase\_thread}{\aexecrun, \thid} & \defeq & (\aexecrun\prjT, \aexecrun\prjI \setminus \Set{\thid}, \aexecrun\prjP \setminus \Setcon{(\txid, \thid)}{ \txid \in \dom(\aexecrun\prjT)}, \aexecrun\prjV, \aexecrun\prjA)                                                                                                                                                                                                                   
    \end{rclarray}
\]
\hrule
\caption{Per-thread operational semantics}
\label{fig:thread_semantics}
\end{figure}

In order to model concurrency, we use thread pools.
A \emph{thread pool} is a finite partial function from thread identifiers to triples of the form \((\thstk, \iprog)\). That is, each thread is associated with a thread stack \(\thstk\) and an intermediate program \(\iprog\) to be executed. 

\begin{defn}[Thread pools]
\label{def:thread_pools}
Given the set of thread stacks $\ThdStacks$ (\defref{def:stacks}) and intermediate programs $\IntermediatePrograms$ (\defref{def:thread_semantics}), a \emph{thread pool} is a a finite partial function from thread identifiers to triples of thread stacks and intermediate programs, \(\thpl \in \TPool \eqdef \ThreadID \parfinfun \ThdStacks \times \IntermediatePrograms\).
\end{defn}
 
\begin{defn}[Thread pool semantics] 
\label{def:thread_pool_semantics}
Given the set of consistent models \( \ConsisModels \) (\defref{def:consistency-models}), runtime abstract executions \(\Aexecrun\) (\defref{def:abs-exec}), transition labels \( \Translabel \) (\defref{def:label}) and thread pools  \( \TPool \) (\defref{def:thread_pools}), the \emph{thread pool semantics}, 
\[
	\toG{} : \Como \times (\Aexecrun \times \TPool) \times \Translabel \times (\Aexecrun \times \TPool) 
\]
is defined in \figref{fig:thread_pool_semantics}.
\end{defn}
 
The thread pool operational semantics is given in \figref{fig:thread_pool_semantics}, where an arbitrary thread in the pool \(\thpl\) is picked to run for one step.
If the next execution step is a thread fork, then a new thread \(\thid'\) is allocated in the pool to be executed with its thread stack copied from its parent (forking) thread.
Conversely, when the next execution step is the joining of thread \(\thid'\), then \(\thid'\) is removed from the thread pool and the stack from the child thread merges into the parent thread.

\begin{figure}
\hrule\vspace{5pt}
%
\[
    \infer[\rl{PSingle}]{%
        \como \vdash ( \aexecrun, \thpl \uplus \Set{ \thid \mapsto (\thstk, \iprog) } ) \ \toG{\lb} \  ( \aexecrun, \thpl \uplus \Set{ \thid \mapsto (\thstk', {\iprog}') } ) 
    }{%
        \como, \thid \vdash ( \thstk, \aexecrun ) , \iprog \ \toT{\lb} \  ( \thstk', \aexecrun ' ) , {\iprog}' 
        \quad \lb \in \Set{ \lbID, \lbC{\stub} }
    }
\]

\[
    \infer[\rl{PFork}]{%
        \como \vdash ( \aexecrun, \thpl \uplus \Set{ \thid \mapsto (\thstk, \iprog) } ) \ \toG{\lbF{\thid', \prog}} \  ( \aexecrun', \thpl \uplus \Set{ \thid \mapsto (\thstk, {\iprog}'), \thid' \mapsto (\thstk, \prog) } )
    }{%
        \como, \thid \vdash ( \thstk, \aexecrun ) , \iprog \ \toT{\lbF{\thid', \prog}} \  ( \thstk, \aexecrun' ) , {\iprog}' 
    }
\]

\[
    \infer[\rl{PJoin}]{%
        \como \vdash ( \aexecrun, \thpl \uplus \Set{ \thid \mapsto (\thstk, \iprog), \thid' \mapsto (\thstk', \pskip) } )  \ \toG{\lbJ{\thid',\thstk''}} \ ( \aexecrun', \thpl \uplus \Set{ \thid \mapsto (\thstk'', {\iprog}')} )
    }{%
        \como, \thid \vdash ( \thstk, \aexecrun ) , \iprog \ \toT{\lbJ{\thid',\thstk'}} \  ( \thstk'', \aexecrun' ) , {\iprog}' 
    }
\]
%
\hrule\vspace{5pt}
\caption{Thread pool semantics}
\label{fig:thread_pool_semantics}
\end{figure}

\subsection{Soundness and Completeness}

Given a consistency model, we can define the set of abstract executions that satisfy a consistency model (\defref{def:valid-aexec}) \cite{cerone:snapshot,cerone_et_al:lipics:2017:7794,cerone_et_al:LIPIcs:2015:5375}.
The \lemref{lem:consistency-monotonicity} is for sanity check.

\begin{defn}[Valid abstract executions]
\label{def:valid-aexec}
Given a consistency model \( (\respo, \conguar) = \como \in \Como \) (\defref{def:consistency-models}), the set of valid abstract executions under the model, denoted by \( \evalCOM{\como} \),  is defined as the follows,
\[
    \begin{rclarray}
        \evalCOM{(\respo, \conguar)} & \defeq & 
        \Setcon{%
            \aexec = (\tx, \po, \vis, \ar)
        }{%
            \conguar(\aexec) \subseteq \vis 
            \land \for{\txid, \addr, \val}  \\
            \qquad (\etR, \addr, \val) \in \tx(\txid) 
            \implies \val \in \respo(\aexec, \aexec\prjV(\txid), \addr)
        }
    \end{rclarray}
\]
where \( \aexec\prjV(\txid) \) returns all the predecessors of \( \txid \) with respect to the visibility relation.
This is, for any relation \( R \),
\[
\begin{rclarray}
    R(x) & \defeq & \Setcon{ x' }{ (x', x) \in R}
\end{rclarray}
\]
\end{defn}    
 
\begin{lem}[Consistency models monotonicity]
\label{lem:consistency-include}
\label{lem:consistency-monotonicity}
The abstract executions allowed by a stronger consistency model is allowed by a weaker consistency model, this is,
\[
    \como_{1} \ordCOM \como_{2} \implies \evalCOM{\como_{2}} \subseteq \evalCOM{\como_{1}}
\]
\end{lem}
\begin{proof}
For any abstract execution \( \aexec \) that satisfies the stronger consistency model \( \como_{2} = (\respo_{2}, \conguar_{2}) \), first, it has \( \conguar_{2}(\aexec) \subseteq \aexec\prjV \) by the \defref{def:valid-aexec}.
Given the hypothesis \( \como_{1} \ordAEX \como_{2} \) and the order definition (\defref{def:consistency-models}), we know \( \conguar_{1}(\aexec)  \subseteq \conguar_{2}(\aexec)  \) so that \( \conguar_{1}(\aexec) \subseteq \aexec\prjV \).
Second, assume a transaction \( \txid \) and any of its read event with an address \( \addr \) and a \( \val \) such that \( (\etR, \addr, \val) \in \tx(\txid) \) therefore \( \val \in \respo_{2}(\aexec, \aexec\prjV(\txid), \addr) \).
Similarly by the hypothesis \( \como_{1} \ordAEX \como_{2} \), we know \( \respo_{2}(\aexec, \aexec\prjV(\txid), \addr) \subseteq \respo_{1}(\aexec, \aexec\prjV(\txid), \addr)\), thus \( \val \in \respo_{1}(\aexec, \aexec\prjV(\txid), \addr)\).
Therefore we have \( \aexec \in \evalCOM{\como_{1}} \).
\end{proof}

\begin{thm}[Soundness of the semantics]
\label{thm:soundness-semantics}
For any runtime abstract abstract execution \( \aexecrun \) that satisfies a consistency model \( \como \), if the semantics under consistency model \( \como \) take one step  to  a new abstract execution \( \aexecrun' \), the new execution should satisfy the consistency model.
This is,
 \[
 \begin{array}{@{}l@{}}
    \for{\como, \thid, \aexecrun, \aexecrun', \thstk, \thstk', \iprog, {\iprog}', \lb} \\
    \qquad \eraseAEX{\aexecrun} \in \evalCOM{\como}
    \land \como, \thid \vdash (\aexecrun, \thstk), \iprog \toT{\lb} (\aexecrun', \thstk'), {\iprog}' 
    \implies \eraseAEX{\aexecrun'} \in \evalCOM{\como}
 \end{array}
 \]
\end{thm}
\begin{proof}
We prove it by induction on the derivations.

\caseB{\rl{PCommit}}

By the \rl{PCommit} rule, it has \( \iprog = \ptrans{\trans} \), \( \iprog' = \pskip \) and \( \lb = \lbC{\txid} \), for some transaction code \( \trans \) and identifier \( \txid \).
Let variables \( \aexec = \eraseAEX{\aexecrun} \) and \( \aexec' = \eraseAEX{\aexecrun'} \) in the following discussion.
We need to prove the follows,
\begin{align}
    & \for{ \txid', \txid'', \addr, \val}  \nonumber \\
    & \quad (\etR, \addr, \val) \in \aexec'\prjT(\txid') \implies \val \in \respo(\aexec',\aexec'\prjV(\txid'), \addr) \label{equ:res_policy}\\
    & \quad (\txid', \txid'') \in \conguar(\aexec') \implies (\txid', \txid'') \in \aexec'\prjV \label{equ:con_guarantee}
\end{align}
First for the \equref{equ:res_policy}, it only needs to check the new transaction \( \txid \) as others are proved directly from the hypothesis.
Given an initial heap \( \h \), a set of events \( \evset \) associated with the new transaction \( \txid \), a set of transactions \( \txidset \) observed by the new transaction \( \txid \), assume these variables satisfy the follows,
\[
\begin{array}{@{}l@{}}
    \exsts{\thstk, \txstk, \h'} \nonumber \\
    \quad \thstk \vdash (\emptyset, \h, \emptyset), \trans \toL^{*} (\txstk, \h', \evset), \pskip 
    \land \h \in \obsstate{\aexecrun, \txidset, \respo} \nonumber \\
\end{array}
\]
therefore the following hold,
\[
    (\etR, \addr, \val) \in \evset \implies \h(\addr) = \val
\]
Because by the transaction semantics (\figref{fig:transaction_semantics}), a transaction only records the first read event for each address.
It can be proved by induction on the derivations for transaction operational semantics, where the only rule that involves read event is \rl{TRead}.
If the new read event is included in the events set after flushing read (\defref{def:transaction-event}), \ie \( (\etR, \addr, \val) \in ( \evset \flushR (\etR, \addr, \val) ) \), this means there is no other read and write to the same address before, so that the value \( \val  \) associate with the address \( \addr \) is the initial value, \ie \( \h(\addr)\), and after that no other read event can over-write.
By the \( \obsstateName \) function (\figref{fig:thread_semantics}), where \( \h(\addr) = \val \iff \respo(\aexec, \txidset, \addr) \), the following hold,
\[
    (\etR, \addr, \val) \in \evset \implies \respo(\aexec, \txidset, \addr)
\]
Since the \( \newaexecName \) function only extends abstract execution, which means \( \aexec \ordAEX \aexec' \), so that,
\[
    (\etR, \addr, \val) \in \evset \implies \respo(\aexec', \txidset, \addr)
\]
By the \( \obsstateName \) function, we have \( \evset = \aexec'\prjT(\txid)\) and \( \txidset = \aexec'\prjV(\txid)\) therefore we prove \equref{equ:res_policy}.

Second for \equref{equ:con_guarantee}, the visibility relation of the new abstract execution \( \aexec'\prjV \) contains the minimum relation required by the consistency guarantee.
Similarly, it is sufficient to consider those visibility edges related to the new transaction \( \txid \).
Note that for any transaction \( \txid'' \in \dom(\aexec'\prjT) \), it has \( (\tsid, \tsid'') \notin \conguar(\aexec') \).
Given \( \newaexecName \) function, so that \( (\tsid'', \tsid) \in \aexec'\prjA \).
Then, given the consistency guarantee (\defref{def:consistency-models}), it cannot violate arbitration order, this is, \( \conguar(\aexec') \subseteq \aexec'\prjA\).
Thus, it is safe to assume a transaction \( \tsid' \in \dom(\aexec'\prjT) \) such that \( (\tsid', \tsid) \in \conguar(\aexec') \).
By the \( \predn{consis}\) predicate, we have \( (\tsid', \tsid) \in \conguar(\aexec') \implies (\tsid', \tsid) \in \aexec'\prjV\), so that we prove \equref{equ:con_guarantee}.

\caseB{\rl{PAssign}, \rl{PAssume}, \rl{PChoice}, \rl{PLoop}, \rl{PSeqSkipS}}

For these base cases, the runtime abstract execution remains the same, \ie \( \aexecrun = \aexecrun' \), so they trivially hold because of the hypothesis.

\caseB{\rl{PPar}, \rl{PWait}}

For these two base cases, since the \( \funcn{extend\_thread} \)  and \( \funcn{erase\_thread} \) functions only change relations related to the corresponding threads, therefore \ie \( \eraseAEX{\aexecrun} = \eraseAEX{\aexecrun'} \), so they hold because of the hypothesis.

\caseI{\rl{PSeq}}

It is proved directly by applying the \ih
\end{proof}

For sanity check and also proving the completeness of this semantics, we first prove the semantics is monotonic (\lemref{lem:semantics-monotonicity}), which means that for any reduction that can happen in the stronger consistency model, it can also happen in the weaker one.

\begin{lem}[Semantics monotonicity]
\label{lem:semantics-monotonicity}
Given an initial runtime abstract execution \( \aexecrun \), if it can transfer to an abstract execution \( \aexecrun' \) by reducing one step of the semantics under stronger consistency model \( \como_{2}\), it is also possible by reducing one step of the semantics under a weaker consistency model \( \como_{1} \).
\[
\begin{array}{@{}l@{}}
    \for{\como_{1}, \como_{2}, \thid, \aexecrun, \aexecrun', \iprog, \iprog', \thstk, \thstk', \lb}  \\
    \quad \como_{2}, \thid \vdash ( \aexecrun, \thstk ), \iprog \toT{\lb} ( \aexecrun', \thstk' ), \iprog'
    \land \como_{1} \ordCOM \como_{2} \\
    \quad \implies \como_{1}, \thid \vdash ( \aexecrun, \thstk ), \iprog \toT{\lb} ( \aexecrun', \thstk' ), \iprog'
\end{array}
\]
\end{lem}
\begin{proof}
We prove it by induction on the derivations.
The only interesting case is the \rl{PCommit} rule.

\caseB{\rl{PCommit}}

Let variables \( (\respo_{1}, \conguar_{1}) = \como_{1} \) and  \( (\respo_{2}, \conguar_{2}) = \como_{2} \) respectively.
Given an initial runtime abstract execution \( \aexecrun \), a set of observable transactions \( \txidset \), a new transaction identifier \( \txid \) and a thread identifier \( \thid \), by the \rl{PCommit} rule (\figref{fig:thread_pool_semantics}), it is sufficient to prove, first, all the observable states under the stronger consistency model can also be observed under the weaker one,
\begin{equation}
    \label{equ:obs-state-included}
    \obsstate{\aexecrun, \txidset, \respo_{2}} \subseteq  \obsstate{\aexecrun, \txidset, \respo_{1}} 
\end{equation}
and second if the new runtime abstract execution \( \aexecrun' \) exists under the stronger concurrency model, it should also exist under weaker one,
\begin{equation}
    \label{equ:consis-both-exist}
    \consis{\aexecrun', \txid, \thid, \conguar_{2}} \implies \consis{\aexecrun', \txid, \thid, \conguar_{1}}
\end{equation}
To prove \equref{equ:obs-state-included}, assume an observable heap \( \h \) under stronger consistency model \( \como_{2}\), which means \( \h \in \obsstate{\aexecrun, \txidset, \respo_{2}} \).
Then, assume an address \( \addr\) and the corresponding value \( \val \) such that \(  \h(\addr) = \val \).
By the \( \obsstateName \) function  (\figref{fig:thread_semantics}) and resolution policy (\defref{def:consistency-models}), it is known that \( \val \in \respo_{2}(\eraseAEX{\aexecrun}, \txidset, \addr)  \).
Because of \( \respo_{2}(\eraseAEX{\aexecrun}, \txidset, \addr) \subseteq \respo_{1}(\eraseAEX{\aexecrun}, \txidset, \addr) \) (\defref{def:consistency-models}), we have \(  \val \in \respo_{1}(\eraseAEX{\aexecrun}, \txidset, \addr) \), so \( \h \in \obsstate{\aexecrun, \txidset, \respo_{1}} \).
For \equref{equ:consis-both-exist}, assume the \( \consis{\aexecrun, \txid, \thid, \conguar_{2}} \) predicate holds and assume an edge \( (\txid', \txid) \in \conguar_{2}(\eraseAEX{\aexecrun'}) \) for some \( \txid' \).
Thus, the edge \( (\txid', \txid) \) will be included in the visibility relation of the new runtime abstract execution, \ie \( (\txid', \txid) \in \aexecrun'\prjV \).
Since \( \conguar_{1}(\eraseAEX{\aexecrun}) \subseteq \conguar_{2}(\eraseAEX{\aexecrun})\), so \( (\txid', \txid) \in \conguar_{1}(\eraseAEX{\aexecrun'}) \implies (\txid', \txid) \in \aexecrun'\prjV \) holds.
This means for any edges that satisfy the consistency guarantee for stronger model, they also satisfy the weaker consistency guarantee, thus the \equref{equ:consis-both-exist} holds.
Combining \equref{equ:obs-state-included} and \equref{equ:consis-both-exist}, we prove \rl{PCommit}.

\caseB{\rl{PAssign}, \rl{PAssume}, \rl{PChoice}, \rl{PLoop}, \rl{PSeqSkipS}, \rl{PPar}, \rl{PWait}}

These base cases do not depend on the consistency model, so they trivial hold because of the hypothesis.

\caseI{\rl{PSeq}}

It is proved directly by applying the \ih
\end{proof}

\begin{lem}[Preservation of the consistency model]
\label{lem:preserve-of-consistency}
\[
 \begin{array}{@{}l@{}}
    \for{\como_{1}, \como_{2}, \thid, \aexecrun, \aexecrun', \thstk, \thstk', \iprog, {\iprog}', \lb} \\
    \qquad \como_{1}, \thid \vdash (\aexecrun, \thstk), \iprog \toT{\lb} (\aexecrun', \thstk'), {\iprog}'
    \land \eraseAEX{\aexecrun'} \in \evalCOM{\como_{2}}
    \implies \eraseAEX{\aexecrun} \in \evalCOM{\como_{2}}
 \end{array}
\]
\end{lem}
\begin{proof}
We prove it by induction on the derivations.
The only interesting case is the \rl{Commit}.

\caseB{\rl{Commit}}

By the rule it has \( \iprog = \ptrans{\trans} \), \( \iprog' = \pskip \) and \( \lb = \lbC{\txid} \).
We prove this case by deriving contradiction.
Assume \( \eraseAEX{\aexecrun} \notin \evalCOM{\como_{2}} \), which means that there exists an edge \( (\txid', \txid'') \) such that it is in the consistency guarantee, \( (\txid', \txid'') \in \conguar_{2}(\eraseAEX{\aexecrun}) \) but not is not included in the visibility relation.
\begin{equation}
    \label{equ:not-in-vis}
    (\txid', \txid'') \notin \aexecrun\prjV 
\end{equation}
Another possibility is that there is a read event from a transaction, \( (\etW, \addr, \val) \in \aexecrun\prjT(\txid') \) where the value is not observable under the stronger consistency model, \ie
\begin{equation}
    \label{equ:not-observable}
    \val \notin \respo_{2}(\eraseAEX{\aexecrun}, \aexecrun\prjV(\txid'), \addr) 
\end{equation}
Because the rule only extend the runtime abstract execution \( \aexecrun \ordAEXRUN \aexecrun'\), this means the edge \( (\txid', \txid'') \) is not in the runtime abstract execution after reduction \( \aexecrun'\), or the transaction \( \txid' \) reads a unobservable value, 
\[
    (\txid', \txid'') \notin \aexecrun'\prjV \lor \val \notin \respo_{2}(\eraseAEX{\aexecrun'}, \aexecrun'\prjV(\txid'), \addr) 
\]
Both cases lead to \( \aexecrun' \notin \evalCOM{\como_{2}} \) so there is contradiction to the hypothesis.
Therefore we have the proof for this base case.

\caseB{\rl{PAssign}, \rl{PAssume}, \rl{PChoice}, \rl{PLoop}, \rl{PSeqSkipS}}

For these base cases, the runtime abstract execution remains the same, \ie \( \aexecrun = \aexecrun' \), so they trivially hold because of the hypothesis.

\caseB{\rl{PPar}, \rl{PWait}}

For these two base cases, since the \( \funcn{extend\_thread} \)  and \( \funcn{erase\_thread} \) functions change edges only related to the corresponding threads, therefore \ie \( \eraseAEX{\aexecrun} = \eraseAEX{\aexecrun'} \), so it holds because of the hypothesis.

\caseI{\rl{PSeq}}

It is proved directly by applying the \ih
\end{proof}

The completeness means that if an abstract execution \( \aexecrun \) satisfies a consistency model, it always is possible to produce such execution through the semantics under the corresponding consistency model for some initial configurations.
To define the completeness, we introduce \emph{anarchic semantics}, which is the semantics under the bottom element for consistency model \( \btmCOM \) as for the fact that there is no constraint for the visibility relations and for the observable value for each address.
\sx{citation for anarchic semantics?}
The \thmref{thm:semantics-completeness} says, given some initial configurations, after one step under the anarchic semantics, if one is ``lucky'' that it ends up with a runtime abstract execution \( \aexecrun'\) that satisfies a consistency model \( \como \), it is possible to get the same result using the semantics specifically for the consistency model \( \como \).

\begin{thm}[Completeness of the semantics]
\label{thm:semantics-completeness}
For any initial configuration \( ( (\aexecrun, \thstk), \iprog ) \), after one step under the \emph{anarchic semantics}, \ie the semantics under the bottom element \( \btmCOM \), it ends up with \( \aexecrun' \), and if the new runtime abstract execution \( \aexecrun' \) satisfies a consistency model \( \como \), then there is a corresponding step using semantics under consistency model \( \como \).
 \[
 \begin{array}{@{}l@{}}
    \for{\como, \thid, \aexecrun, \aexecrun', \thstk, \thstk', \iprog, {\iprog}', \lb} \\
    \qquad \btmCOM, \thid \vdash (\aexecrun, \thstk), \iprog \toT{\lb} (\aexecrun', \thstk'), {\iprog}' \land \eraseAEX{\aexecrun'} \in \evalCOM{\como} \\
    \qqquad \implies \como, \thid \vdash (\aexecrun, \thstk), \iprog \toT{\lb} (\aexecrun', \thstk'), {\iprog}' 
 \end{array}
 \]
\end{thm}
\begin{proof}
For any runtime abstract execution after one step such that \( \eraseAEX{\aexecrun'} \in \evalCOM{\como} \), by the \lemref{lem:preserve-of-consistency}, it is known that the initial configuration also satisfies the consistency model, this is, \( \eraseAEX{\aexecrun} \in \evalCOM{\como} \).
Since \( \btmCOM \) is the bottom element such that \( \btmCOM \ordCOM \como \), because the semantics are monotonic with respect to the order of consistency model (\lemref{lem:semantics-monotonicity}), we have the proof.
\end{proof}

\subsection{Program Semantics}
\begin{defn}[History Heaps]
\label{def:his_heap}
Assuming a set of \emph{transactions identifiers} \( \TxID \defeq \Set{\txid, \dots}\), the \emph{transactions} \( \tx \in \Tx \) is defined as a finite partial function from transactions identifiers \( \TxID \) to valid sets of events \( \Evsets \),
\[
\begin{rclarray}
\tx \in \Tx & \defeq & \TxID \parfinfun \Evsets
\end{rclarray}
\]
A \emph{history heap}, \( \hh \in \HisHeaps \), is a `heap' such that each address maps to a list of values.
Each value in the history heap is associated with a transaction (identifier) who writes the value and a set of transactions (identifiers) who read the it.
\[
\begin{rclarray}
    \HisHeaps & \defeq & \Addr \parfinfun (\Val \times \TxID \times \powerset{\TxID})^{*}
\end{rclarray}
\]
Let \( \hh(\addr)(i)\) denotes the \emph{i-th} (starting from 1) tuple of the list associated with address \( \addr \), and \( \hhV(\addr)(i) \), \( \hhW(\addr)(i) \) and \( \hhR(\addr)(i) \) denote the first, second and third element of the tuple.
\end{defn}

\begin{defn}[Cuts of history heaps]
\label{def:cuts}
\emph{A cut of of a history heap} is a partial function from addresses to indexes,
\[
\begin{rclarray}
    \cu \in \Cuts & \defeq & \Addr \parfinfun \Nat
\end{rclarray}
\]                                                                      
A cut \( \cu \) can collapse a history heap \( \hh \) to a (normal) heap \( h \) by the function \(\funcn{clps} : \HisHeaps \times \Cuts \parfun \Heaps\),
\[
\begin{rclarray}
    \func{clps}{\hh, \cu} & \defeq & 
    \begin{cases}
        \lambda \addr \ldotp \hhV(\addr)(\cu(\addr)) & \text{if } \dom(\hh) = \dom(\cu) \land \for{\addr'} \cu(\addr') \leq \left|\hh(\addr')\right| \\
        \text{undefined} & \text{otherwise}
    \end{cases}
\end{rclarray}
\]
The order between two cuts that have the same domain is the order of all the indexes, 
\[
\begin{rclarray}
    \cu \orderCU \cu' & \defiff & \dom(\cu) = \dom(\cu') \land \for{\addr} \cu(\addr) \leq \cu'(\addr) \\
\end{rclarray}
\]
Assuming a countably infinite set of thread identifiers \( \setthid \subseteq \ThreadID \defeq \Set{\thid, \dots} \), \emph{a cuts' runtime} is a function from thread identifiers to cuts, \ie \( \thcu \in \ThCuts \defeq \ThreadID \parfun \Cuts\).
\end{defn}

\begin{defn}[Consistency Models]
\label{def:consistency-models}
A \emph{Consistency model} \( \como \) is a set of tuples in the form of \( ((\hh,\thcu),(\hh',\thcu')) \) that the transition from the state \( (\hh,\thcu)\) to the state \( (\hh',\thcu') \) satisfies certain constraints.
\end{defn}

\begin{defn}[Thread transition labels]
\label{def:label}
Given the set of thread identifiers \(\ThreadID\) (\defref{def:abs-exec}), the set of \emph{thread transition labels}, $\lb \in \Translabel$, is defined by the following grammar, where $\prog$ denotes a program (\defref{def:language}), the $\txid$ demotes a transaction identifier and $\thstk$ denotes a thread stack (\defref{def:stacks}),
\[
    \begin{rclarray}
	\iota \in \Translabel & ::= & \lbID \mid \lbC{\txid} \mid \lbF{\thid,\prog} \mid \lbJ{\thid,\thstk}
    \end{rclarray}
\]
\end{defn}

\begin{defn}[Thread semantics]
\label{def:thread_semantics}
Given the thread identifiers \(\thid \in \ThreadID\), the set of \emph{intermediate programs}, \(\iprog \in \IntermediatePrograms\), is defined by the following grammar,
\[
    \iprog ::= \prog \mid \iprog \pseq \pwait{\thid}
\]
Given the set of consistency model \( \ConsisModels \) (\defref{def:consistency-models}), thread stacks \( \ThdStacks \) (\defref{def:stacks}), history heaps \( \HisHeaps \) (\defref{def:his_heap}) and cuts' runtime (\defref{def:cuts}), the \emph{per-thread operational semantics} of programs,
\[
\begin{rclarray}
	\toT{} & : &
    \begin{array}[t]{@{}l@{}}
    \Como \times \ThreadID 
    \times \\
	\quad \left( ( \ThdStacks \times \HisHeaps \times \ThCuts ) \times \IntermediatePrograms \right) 
	\times  \Translabel \times
	\left( ( \ThdStacks \times \HisHeaps \times \ThCuts ) \times \IntermediatePrograms \right) 
    \end{array}
\end{rclarray}
\]
is defined in \figref{fig:thread_semantics}.
\end{defn}

\begin{figure}[!t]
%
\hrule
%
\[
    \infer[\rl{PCommit}]{%
        \como, \thid \vdash ( \thstk, \hh, \thcu ), \ptrans{\trans} \ \toT{\lbC{\txid}} \ ( \thstk', \hh', \thcu' ) , \pskip
    }{%
        \begin{array}{c}
            \txid \in \func{fresh}{\hh}  
            \quad \h = \func{clps}{\hh,\thcu(\thid)}, \emptyset) \\
            \thstk \vdash (\txstk_{0}, \h, \unitO), \trans \ \toL^{*} \  (\txstk, \h', \opset) , \pskip \\
            \thstk' = \thstk\rmto{\ret}{\txstk(\ret)} 
            \quad \hh' = \func{commit}{\hh, \thcu(\thid), \txid, \opset}  \\
            \cu = \func{update}{\hh', \thcu(\thid), \opset} 
            \quad \thcu\rmto{\thid}{\cu} \orderCU \thcu
            \quad ((\hh,\thcu),(\hh',\thcu')) \in \como
        \end{array}
    }
\]

\[
    \infer[\rl{PAssign}]{%
        \como, \thid \vdash ( \thstk, \hh, \thcu ) , \passign{\thvar}{\expr} \ \toT{\lbID} \  ( \thstk\rmto{\thvar}{\val}, \hh, \thcu ) , \pskip
    }{
        \val = \evalE[\thstk]{\expr}
        && \thvar \in \ThdVars
    }
\]

\[
    \infer[\rl{PAssume}]{%
        \como, \thid \vdash ( \thstk, \hh, \thcu ) , \passume{\expr} \ \toT{\lbID} \  ( \thstk, \hh, \thcu ) , \pskip
    }{%
        \evalE[\thstk]{\expr} = 0
    }
\]

\[
    \infer[\rl{PCutShift}]{
        \como, \thid \vdash ( \thstk, \hh, \thcu ) , \prog \ \toT{\lbID} \  ( \thstk, \hh, \thcu' ) , \prog
    }{
        \thcu \leq \thcu'
    }
\]

\[
    \infer[\rl{PChoice}]{%
        \como, \thid \vdash ( \thstk, \hh, \thcu ) , \prog_{1} \pchoice \prog_{2} \ \toT{\lbID} \  ( \thstk, \hh, \thcu ) , \prog'
    }{
        \prog' \in \Set{\prog_{1}, \prog_{2}}
    }
\]

\[
    \infer[\rl{PLoop}]{%
        \como, \thid \vdash ( \thstk, \hh, \thcu ) , \prog\prepeat \ \toT{\lbID} \  ( \thstk, \hh, \thcu ) , \pskip \pchoice (\prog \pseq \prog\prepeat)
    }{}
\]

\[
    \infer[\rl{PSeqSkip}]{%
        \como, \thid \vdash ( \thstk, \hh, \thcu ) , \pskip \pseq \iprog \ \toT{\lbID} \  ( \thstk, \hh, \thcu ) , \iprog
    }{}
\]

\[
    \infer[\rl{PSeq}]{%
        \como, \thid \vdash ( \thstk, \hh, \thcu ) , \iprog_{1} \pseq \iprog_{2} \ \toT{\lb} \ ( \thstk', \hh, \thcu' ) , {\iprog_{1}}' \pseq \iprog_{2}
    }{%
        \como, \thid \vdash ( \thstk, \hh, \thcu ) , \iprog_{1} \ \toT{\lb} \  ( \thstk', \hh, \thcu' ) , {\iprog_{1}}' 
    }
\]

\[
    \infer[\rl{PPar}]{%
        \como, \thid \vdash ( \thstk, \hh, \thcu ) , \prog_{1} \ppar \prog_{2} \ \toT{\lbF{\thid', \prog_{2}}} \  \left( \thstk, \hh, \thcu \uplus \Set{\thid' \mapsto \thcu(\thid)} \right) , \prog_{1} \pseq \pwait{\thid'}
    }{}
\]

\[
    \infer[\rl{PWait}]{%
        \como, \thid \vdash \left( \thstk_{1} \uplus \thstk_{f}, \hh, \thcu \uplus \Set{\thid' \mapsto \thcu(\thid)} \right) , \pwait{\thid'} \ \toT{\lbJ{\thid', \thstk_{2} \uplus \thstk_{f}}} \  ( \thstk_{1} \uplus \thstk_{2} \uplus \thstk_{f}, \hh, \thcu ) , \pskip 
    }{}
\]

\sx{The stack could be polluted by the child thread, but they must agree?}
 
where,
\[
\begin{rclarray}                                 
    \func{commit}{\hh, \cu, \txid, \opset} & \defeq & 
    \begin{cases}
        \hh & \text{if } \opset = \unitO \\
        \func{commit}{\hh', \cu, \txid, \opset'} & \text{if } \opset = \opset' \addO (\otR, \addr, \val) \\
        \func{commit}{\hh'', \cu, \txid, \opset''} & \text{if } \opset = \opset'' \addO (\otW, \addr, \val) \\
    \end{cases} \\
    \hh' & \equiv & \hh\rmto{\addr}{\hh(\addr)\rmto{\cu(\addr)}{\left(\hhV(\addr)(\cu(\addr)),\hhW(\addr)(\cu(\addr)),\hhR(\addr)(\cu(\addr)) \uplus \Set{\txid} \right)}} \\
    \hh'' & \equiv & \hh\rmto{\addr}{\hh(\addr) +\!\!+ [(\val, \txid, \emptyset)] } \\
%
%
    \func{update}{\hh, \cu, \opset} & \defeq &
    \begin{cases}
        \cu & \text{if } \opset = \unitO \\
        \func{update}{\hh, \cu, \opset'} & \text{if } \opset = \opset' \addO (\otR, \addr, \val) \\
        \func{update}{\hh, \cu\rmto{\addr}{\left| \hh(\addr) \right|}, \opset''} & \text{if } \opset = \opset'' \addO (\otW, \addr, \val) \\
    \end{cases} \\
%
%              
	\func{fresh}{\aexecrun}  & \defeq & \Setcon{ \txid }{ \txid \in \TxID \land \for{\addr, i} \txid \neq \hhW(\addr)(i) \land \txid \notin \hhR(\addr)(i) } \\
\end{rclarray}
\]
\hrule
\caption{Per-thread operational semantics}
\label{fig:thread_semantics}
\end{figure}

\sx{The full version should support dynamic thread, but we might only need fixed thread so that the thread pool semantics will be pick one thread and run one step}

\begin{defn}[Thread pools]
\label{def:thread_pools}
Given the set of thread stacks $\ThdStacks$ (\defref{def:stacks}) and intermediate programs $\IntermediatePrograms$ (\defref{def:thread_semantics}), a \emph{thread pool} is a a finite partial function from thread identifiers to triples of thread stacks and intermediate programs, \(\thpl \in \TPool \eqdef \ThreadID \parfinfun \ThdStacks \times \IntermediatePrograms\).
\end{defn}
 
\begin{defn}[Thread pool semantics] 
\label{def:thread_pool_semantics}
Given the set of consistent models \( \ConsisModels \) (\defref{def:consistency-models}), history heaps \(\HisHeaps\) (\defref{def:his_heap}), cuts' runtime \(\ThCuts\) (\defref{def:cuts}), transition labels \( \Translabel \) (\defref{def:label}) and thread pools  \( \TPool \) (\defref{def:thread_pools}), the \emph{thread pool semantics}, 
\[
	\toG{} : \Como \times (\HisHeaps \times \ThCuts \times \TPool) \times \Translabel \times (\HisHeaps \times \ThCuts \times \TPool) 
\]
is defined in \figref{fig:thread_pool_semantics}.
\end{defn}
 
The thread pool operational semantics is given in \figref{fig:thread_pool_semantics}, where an arbitrary thread in the pool \(\thpl\) is picked to run for one step.
If the next execution step is a thread fork, then a new thread \(\thid'\) is allocated in the pool to be executed with its thread stack copied from its parent (forking) thread.
Conversely, when the next execution step is the joining of thread \(\thid'\), then \(\thid'\) is removed from the thread pool and the stack from the child thread merges into the parent thread.

\begin{figure}
\hrule\vspace{5pt}
%
\[
    \infer[\rl{PSingle}]{%
        \como \vdash ( \hh, \thcu, \thpl \uplus \Set{ \thid \mapsto (\thstk, \iprog) } ) \ \toG{\lb} \  ( \hh', \thcu', \thpl \uplus \Set{ \thid \mapsto (\thstk', {\iprog}') } ) 
    }{%
        \como, \thid \vdash ( \thstk, \hh, \thcu ) , \iprog \ \toT{\lb} \  ( \thstk', \hh', \thcu' ) , {\iprog}' 
        \quad \lb \in \Set{ \lbID, \lbC{\stub} }
    }
\]

\[
    \infer[\rl{PFork}]{%
        \como \vdash ( \hh, \thcu, \thpl \uplus \Set{ \thid \mapsto (\thstk, \iprog) } ) \ \toG{\lbF{\thid', \prog}} \  ( \hh', \thcu', \thpl \uplus \Set{ \thid \mapsto (\thstk, {\iprog}'), \thid' \mapsto (\thstk, \prog) } )
    }{%
        \como, \thid \vdash ( \thstk, \hh, \thcu ) , \iprog \ \toT{\lbF{\thid', \prog}} \  ( \thstk, \hh', \thcu' ) , {\iprog}' 
    }
\]

\[
    \infer[\rl{PJoin}]{%
        \como \vdash ( \hh, \thcu, \thpl \uplus \Set{ \thid \mapsto (\thstk, \iprog), \thid' \mapsto (\thstk', \pskip) } )  \ \toG{\lbJ{\thid',\thstk''}} \ ( \hh', \thcu', \thpl \uplus \Set{ \thid \mapsto (\thstk'', {\iprog}')} )
    }{%
        \como, \thid \vdash ( \thstk, \hh, \thcu ) , \iprog \ \toT{\lbJ{\thid',\thstk'}} \  ( \thstk'', \hh', \thcu' ) , {\iprog}' 
    }
\]
%
\hrule\vspace{5pt}
\caption{Thread pool semantics}
\label{fig:thread_pool_semantics}
\end{figure}

