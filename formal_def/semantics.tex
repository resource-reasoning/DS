
\section{Semantics}
\label{sec:semantics}

\sx{
    For Philippa and others who read Andrea's report, this is a re-edit version of the semantics, mainly for the reason of properly definition blocks, references, etc.
    Steal many words from that report and add more explanation.
    Few notations difference,
    \begin{itemize}
        \item address \( \addr \) .
            At some point we use \( l \)  for heap/history heap locations, yet there is a variable clash so using \( \addr \).
        \item Transaction identifier \( \txid \).
        \item \( \hh \)  history heap, and we assume index starts from 1 instead of 0.
            We use superscripts \( \hhV(\addr)(3) \), \( \hhW(\addr)(3) \) and \( \hhR(\addr)(3) \) for the value, write, and reads of the version corresponding to the third version of the address \( \addr \) in the history \( \hh \).
        \item \( \stk \) per-thread stack, and no transaction stack.
        \item \( \vi \) single view, as \( \val \) is for value and \( V \) looks like a set.
        \item \( (\otR, \addr, \val ) \) denotes a operation read address \( \addr \) with value is \( \val \) and similarly the write \( (\otW, \addr, \val ) \)
            Also we use, for example \( \opset \addO (\otR, \stub, \stub) \) instead of \( \oplus \), as the latter looks like a commutative operator??
        \item We call \( \funcn{op}\) instead of \( \funcn{Fprint} \) for the function extract the operation/fingerprint from the primitive commands.
            Because we use \emph{fingerprint} to refer the \emph{fingerprint assertion} later in the logic.
        \item We call \( \func{localHp}{\hh, \vi} \) instead of \( \func{snapshot}{\hh, \vi} \) as the latter is a bit misleading.
        \item \( \func{updHisHp}{\hh, \vi, \txid, \opset} \) instead of \( \funcn{HHupdate}_{\txid}(\hh, \vi, \opset)\).
        \( \func{updView}{\hh, \vi, \opset}\) instead of  \( \func{ViewUpdate}{\hh, \vi, \opset} \), but we also assume the \( \hh \) here is the new one not the old one.
        \item Consistency model \( \como \) as it is a set of quadruples and \( C \) is used for command, so we pick a capital \( \como \).
    \end{itemize}
    \ac{
        \( \hh(\addr, 3).\texttt{val} \) instead of \( \hhV(\addr)(3) \) 

        \( \vartriangleleft \) to \( \addO \) so that not clash with \( \csat \)
        
        \emph{Heaps} in general are not sure, maybe \emph{key-value stores}. Just a terminology change.

        Commit tests/Execution tests \( \como \) than consistency models.
        }
    }

\sx{The small intro is stolen from Andrea :)}
We focus on an abstract computational model for database where multi-threaded programs can access and update addresses in a heap through atomic transactions. 
Transactions in our model execute atomically, though they have different effect on database depending on the consistency model that does not necessarily correspond to \emph{serialisability}. 
This means that, at the moment of executing, a transaction may not observe the most up-to-date value of an address. 
To overcome this issue, we model the state of the database using \emph{history heaps}. 
A history heap keeps track of all the versions written for any address, as well as the information about the transactions that read and wrote such versions. 
To model the potential out-of-date observation, we use \emph{views}.
A view decides the observable versions of addresses for a thread.
When executing, a transaction extracts a local state from the history heap and the view, and afterwards the transaction commits a set of operations that might change the history heap and the view, if the change is allowed by the consistency model.

We starts with the syntax of programs followed by the semantics of transaction.
Then we will formally define history heaps and views, and how we specify consistency models.
Finally, we will give the semantics for the entire programs.


\subsection{Programming language}

For simplify, \emph{a program} \( \prog \) contains fixed numbers of top level threads and there is no dynamic fork and join.
Each thread has a unique thread identifier \( \thid \in \ThreadID \) and associated \emph{commands}.
The \emph{commands}, ranged over by $\cmd$, are defined by an inductive grammar comprising the standard constructs of $\pskip$, sequential composition ($\cmd; \cmd$), non-deterministic choice ($\cmd+\cmd$) and loops ($\cmd^*$).
To simulate conditional branching and loops, we have primitive commands assume (\( \passume{\expr}\)) and assignment (\( \passign{\var}{\expr} \)), where \( \var \) denotes stack variable and \( \expr \) denotes arithmetic expressions which have no side effect.
Additionally, the programming language contains the \emph{transaction} construct $\ptrans{\trans}$ denoting the \emph{atomic} execution of the transaction $\trans$. 
The atomicity guarantees the execution are dictated by the underlying consistency model.
\emph{Transactions}, ranged over by $\trans$, are defined by a similar inductive grammar comprising $\pskip$, non-deterministic choice, loops and sequential composition, as well as primitive constructs, ranged over by \( \transpri \), including assignment (\( \passign{\var}{\expr}\)), assume (\( \passume{\expr}\)), lookup (\( \pderef{\expr}{\expr}\)), mutation (\( \pmutate{\expr}{\expr}\)) and return(\( \preturn{\expr}\)). 
Transactions do \emph{not} contain the \emph{parallel} composition construct ($\ppar$) as they are to be executed atomically.
We assume a valid transactions codes must have the only return at the end.
For better presentation, sometime we omit the default return zero \( \preturn{0} \).
%Transactions can only assign to their own variables, namely transaction variables (\defref{def:program_values}), but it can read from both the thread and transaction stacks.

\begin{defn}[Program values]
\label{def:program_values}
Assume a countably infinite set of \emph{addresses}, $\addr \in \Addr$. The set of \emph{program values} is $\val \in \Val \eqdef \Nat \cup \Addr$, where $ \nat \in \Nat$ denotes the set of natural numbers.
\end{defn}

\begin{defn}[Programming language]
\label{def:language}
A \emph{program}, $\prog \in \Programs$, is a partial function from thread identifiers to commands.
Assuming the set of \emph{variables} \( \var \in \Vars \), the commands \( \cmd \in \Commands \) are defined by the following grammar,
\[
    \begin{rclarray}
    \cmd & ::= &
        \pskip \mid 
        \passign{\thvar}{\expr} \mid
        \passume{\expr} \mid
        \ptrans{\trans} \mid 
        \cmd \pseq \cmd \mid 
        \cmd \pchoice \cmd \mid 
        \cmd \prepeat 
    \end{rclarray}
\]
The $\trans \in \Transactions$ in the grammar above denotes a \emph{transaction} defined by the following grammar.
The transaction codes \( \ptrans{\trans} \) satisfy a well-form condition that there is exactly a return at the end, \ie \( \ptrans{\trans} \iff \exsts{\trans', \expr} \trans \equiv ( \trans' \pseq \preturn{\expr} )  \land \pred{noRet}{\trans'} \).
\[
    \begin{rclarray}
        \transpri & ::= &
        \pass{\txvar}{\expr} \mid
        \pderef{\txvar}{\expr} \mid
        \pmutate{\expr}{\expr} \mid
        \passume{\expr} \mid
        \preturn{\expr} \\
        \trans & ::= &
        \pskip \mid
        \transpri \mid 
        \trans \pseq \trans \mid
        \trans \pchoice \trans \mid
        \trans\prepeat
    \end{rclarray}
\]
The $\expr \in \Expressions$ denotes an \emph{arithmetic expression} defined by the grammar below with $\val \in \Val$ (\defin\ref{def:program_values}),
\[
    \begin{rclarray}
        \expr & ::= &
        \val \mid
        \var \mid
        \expr + \expr \mid
        \expr \times \expr \mid
        \dots 
    \end{rclarray}
\]
\end{defn}

\subsection{Local/Transaction Semantics}

Each thread has it own stack, where local variables are stored.
A thread can access the stack inside or outside transactions.
When a transaction start, it determine a local heap from the current state of database and local view, which we will explain the process later.
Yet this local heap does not affect outside world, instead we use \emph{operation} to connect a transaction to outside world.
A operation could be read or write to a address with a value.
Intuitively, when a transaction is about to commit, it will have \emph{a set of operations} containing the first read and last write for each address.
Because a transaction is executed atomically, all the intermediate steps are not observable from the outside world.

we introduce \emph{a well-formed set of operations} \( \opset \in \Opsets\) that is a subset of operations in which there are at most one read and one write for each address.
The composition, then, is defined as set disjointed union as long as the result is well-formed.
To help us write down the semantics, we assume a binary operator \( \opset \addO \op \) that specifies the effects of adding a new operation \( \op \) to the set \( \opset \).
If the new operation is a read, for example \(\otR, \addr, \val\) that \( \addr \) is the address and \( \val\) is the value, and there is no other operation related to the same address, this new read operation will be included in the set.
Note that if there is already a write in the set, this mean the following reads are all local.
Meanwhile, if the new operation is a write, it will overwrite all preview write operations to the same address.

\sx{Single stack for simplicity. Design choice, total stack vs partial stack, make sure it is consistent.}

\begin{defn}[Stacks]
\label{def:stacks}
A \emph{stack} is a partial function from variables \( \Vars \) (\defref{def:language}) to values program values \( \Val \) (\defref{def:program_values}), this is \( \stk \in \Stacks \defeq \Vars \parfun \Val \).
\end{defn}

%{ \color{gray}
%\begin{defn}[Stacks]
%Given the program values (\defref{def:program_values}) and a set of \emph{transaction variables} \( \TxVars \defeq \Set{\txvar, \dots}\), a \emph{transaction stack} is a partial function from transaction variables to values \( \txstk \in \TxStacks \defeq \TxVars \parfun \Val \).
%Similarly, assuming a set of \emph{thread variables} \( \ThdVars \defeq \Set{\thvar, \dots}\), a \emph{thread stack} is defined as \( \thstk \in \ThdStacks \defeq \ThdVars \parfun \Val \).
%Then, the set of \emph{stacks} is defined as the union of transaction stacks and thread stacks \( \stk \in \Stacks \eqdef \TxStacks \uplus \ThdStacks \).
%\end{defn}
%}

\begin{definition}[Heaps]
\label{def:heaps}
Given the sets of program values $\Val$  and addresses \( \Addr\)  (\defin\ref{def:program_values}), the set of \emph{heaps} is: $\h \in \Heaps \eqdef \Addr \parfinfun \Val$.
The \emph{heap composition function}, $\composeH: \Heaps \times \Heaps \parfun \Heaps$, is defined as $\composeH \eqdef \uplus$, where $\uplus$ denotes the standard disjoint function union. The \emph{ heap unit element} is $\unitH \eqdef \emptyset$, denoting a function with an empty domain.
The \emph{partial commutative monoid of  heaps} is $(\Heaps, \composeH, \{\unitH\})$.
\end{definition}

\begin{defn}[Evaluation of expression]
Given a stack $\stk \in \Stacks$ (\defin\ref{def:stacks}), the \emph{arithmetic expression evaluation} function, $\evalE[(.)]{.}:\Expressions \times \Stacks \parfun \Val$, is defined inductively over the structure of expressions as follows: 
%
\[
    \begin{rclarray}
        \evalE{\val} & \defeq & \val \\
        \evalE{\var} & \defeq & \stk(\var) \\
        \evalE{\expr_{1} + \expr_{2}} & \defeq & \evalE{\expr_{1}} + \evalE{\expr_{2}} \\
        \evalE{\expr_{1} \times \expr_{2}} & \defeq & \evalE{\expr_{1}} \times \evalE{\expr_{2}} \\
        \dots & \eqdef & \dots \\
    \end{rclarray}
\]
\end{defn}


\begin{defn}[Operation and valid operations]
\label{def:transaction-event}
\label{def:transactions}
Assume a set of \emph{operations tags}, a \emph{transaction operation} \( \op \in \Ops \) is a tuple of an operation tag, an address and a value.
The tuple represents either read or write of the address.
The \emph{operation tags} \( \etR \) and \( \etW \) correspond to read and write respectively.
\[
\begin{rclarray}
\OTags & \defeq & \Set{\otR, \otW} \\
\op \in \Ops & \defeq  & \OTags \times \Addr \times \Val
\end{rclarray}
\]
\emph{a well-formed set of operations}, \( \opset \in \Opsets \), is a subset of \( \Ops \) in which any two elements contain either different tags or different address.
\[
    \begin{rclarray}
        \Opsets & \defeq & \Setcon{\opset}{\opset \subseteq \Events \land \wfO{\opset} } \\
        \wfO{\opset} & \defeq & \for{\op, \op' \in \opset} \op\projection{1} \neq  \op'\projection{1} \lor \op\projection{2} \neq  \op'\projection{2}
    \end{rclarray}
\]
The unit element is \( \unitE \defeq \emptyset\) and the composition of two set of operations is only defined when the two sets contains disjointed addresses,
\[ 
\begin{rclarray}
    \opset \composeO \opset' & \defeq & 
    \begin{cases}
        \opset \uplus \opset' & \text{if } \opset\projection{2} \cap \opset'\projection{2} = \emptyset \\
        \text{undefined} & \text{otherwise}
    \end{cases}
\end{rclarray}
\]
A partial binary operation \( \addO \) adds a new operation to valid operations \( \opset \) that ensures the set contains the first read and last write.
For technical reason, if the right hand side is a special token \( \emptyop \), which represents no operation, the operations remains the same.
\[
\begin{rclarray}
    \opset \addO (\etR, \addr, \val) & \defeq & 
    \begin{cases}
        \opset \uplus \Set{(\etR, \addr, \val)} & (\stub, \addr, \stub) \notin \opset \\
        \opset &  \text{otherwise} \\
    \end{cases} \\
    \opset \addO (\etW, \addr, \val) & \defeq & \left( \opset \setminus \Set{(\etW, \addr, \stub)} \right) \uplus \Set{(\etW, \addr, \val)} \\
    \opset \addO \emptyop & \defeq & \opset \\
\end{rclarray}
\]
\end{defn}

\begin{lem}
The \( \addO \) operator preserves the well-form property.
\end{lem}

\azalea{
\label{comm:operations}
This can all be very simplified as follows. Each location is associated with a value and a set of tags in $\pset{\Set{\otR, \otW}}$, where $\emptyset$ means no finger print and the rest have the obvious meaning. 
This way you won't need the well-formedness condition or the lemma. 
%
\begin{defn}[Operations]
An \emph{operation map} \( \opset \in \Opsets \eqdef \Addr \fm \Val \times \pset{\Set{\otR, \otW}} \) is function associating each address with its tags.
The \emph{operation tags} \( \etR \) and \( \etW \) correspond to read and write respectively.

The unit element is the function with empty domain, i.e.\ $\emptyset$. 

The composition of two operation maps is defined as the standard disjoint function union: 
\[ 
\begin{rclarray}
    (\opset \composeO \opset')(a) & \defeq & 
    \begin{cases}
        \opset(a)  & \text{if } a \in \dom(\opset) \land a \not\in \dom(\opset') \\
        \opset'(a)  & \text{if } a \in \dom(\opset') \land a \not\in \dom(\opset) \\
        \text{undefined} & \text{otherwise}
    \end{cases}
\end{rclarray}
\]
%
A partial binary operation \( \addO \) updates an operation map $\opset$ with a tuple $(t, \addr, \val)$ where $t \in \Set{\etR, \etW}$ as follows:
\[
\begin{rclarray}
    \opset \addO (\etR, \addr, \val) & \defeq & 
    \begin{cases}
        \opset[\addr \mapsto (\val, \{\etR\})] & \opset(\addr) = (\val, \emptyset) \\
        \opset &  \text{otherwise} \\
    \end{cases} \\
%
%
	\opset \addO (\etW, \addr, \val) & \defeq & 
    \begin{cases}
        \opset[\addr \mapsto (\val, S \cup \{\etW\})] & \opset(\addr) = (-, S) \\
        \opset &  \text{otherwise} \\
    \end{cases} \\
%      
    \opset \addO \emptyop & \defeq & \opset \\
\end{rclarray}
\]
%
For technical reason, if the right hand side is a special token \( \emptyop \), which represents no operation, the operations remains the same.
\end{defn}
}


%We define a labelled transition system for primitive transaction commands where the labels are commands.
The operational semantics for transactions \(\trans\) is defined with respect to a configuration of the form \((\stk, \h, \opset)\) comprising a stack, a (local) heap and a set of operations.
The operations are those that might affect other transactions, meaning the first read and the last write for each address.
We first define a transition between pairs of stacks and heaps for the primitive commands \( \trans_{p}\).
We also define its operation by \( \funcn{op} \) function, which denotes the contribution of the primitive command that might observed by the external environment, \ie transactions from other threads.
The \( \funcn{op} \) extracts the read or write operation from loop-up and mutation respectively, otherwise returns \( \emptyop \).
The semantics for non-deterministic choices, sequential compositions, and loops have the expected behaviours.


\begin{defn}[Transaction operational semantics]
Given the set of stacks \( \Stacks \) (\defref{def:stacks}), heaps \( \Heaps \) (\defin\ref{def:heaps}) and transactions \( \Transactions \) (\defin\ref{def:language}), the \emph{operational semantics of transactions}, 
\[
\begin{rclarray}
\toL & : & \ThdStacks \times \\
& & \quad ((\TxStacks \times \Heaps \times \Opsets) \times \Transactions) \times ((\TxStacks \times \Heaps \times \Opsets) \times \Transactions)
\end{rclarray}
\]
is given in \fig\ref{fig:transaction_semantics}.
Note that arithmetic expression evaluation \( \evalE{\expr} \) is defined in \defref{def:language} and the \( \addO \) operators is defined in \defref{def:transactions}.
\end{defn}

\begin{figure}[!t]
\hrule\vspace{5pt}
\[
\begin{array}{@{} r c l r  c l @{}}
    (\stk, \h) & \toLTS{\passign{\var}{\expr}} & (\stk\rmto{\var}{\evalE{\expr}}, \h) & \func{op}{\stk, \h, \passign{\var}{\expr}} & \defeq & \emptyop \\
    (\stk, \h) & \toLTS{\pderef{\var}{\expr}} & (\stk\rmto{\var}{\h(\evalE{\expr})}, \h) & \func{op}{\stk, \h, \pderef{\var}{\expr}} & \defeq & (\etR, \evalE{\expr}, \h(\evalE{\expr})) \\
    (\stk, \h) & \toLTS{\pmutate{\expr_{1}}{\expr_{2}}} & (\stk, \h\rmto{\evalE{\expr_{1}}}{\evalE{\expr_{2}}}) & \func{op}{\stk, \h, \pmutate{\expr_{1}}{\expr_{2}}} & \defeq & (\etW, \evalE{\expr_{1}}, \evalE{\expr_{2}}) \\
    (\stk, \h) & \toLTS{\passume{\expr}} & (\stk, \h) \text{ where } \evalE{\expr} = 0 & \func{op}{\stk, \h, \passume{\expr}} & \defeq & \emptyop \\
    (\stk, \h) & \toLTS{\preturn{\expr}} & (\stk\rmto{\ret}{\evalE{\expr}}, \h) & \func{op}{\stk, \h, \preturn{\expr}} & \defeq & \emptyop \\
\end{array}
\]
\hrule\vspace{5pt}
\[	
    \infer[\rl{TPrimitive}]{%
        \vdash (\stk, \h, \opset) , \transpri \ \toL \  (\stk', \h', \opset \addO \op) , \pskip
    }{%
        (\stk, \h) \toLTS{\transpri} (\stk', \h')
        && \op = \func{op}{\stk, \h, \transpri}
    }
\]

\[
    \infer[\rl{TChoice}]{%
        \stk \vdash (\txstk, \h, \opset) , \trans_{1} \pchoice \trans_{2} \ \toL \  (\stk, \h, \opset) , \trans'
    }{
        \trans' \in \Set{\trans_{1}, \trans_{2}}
    }
\]

\[
    \infer[\rl{TLoop}]{%
        \stk \vdash (\txstk, \h, \opset),  \trans\prepeat \ \toL \  (\stk, \h, \opset), \pskip \pchoice (\trans \pseq \trans\prepeat)
    }{}
\]


\[
    \infer[\rl{TSeqSkip}]{%
        \stk \vdash (\txstk, \h, \opset), \pskip \pseq \trans \ \toL \  (\stk, \h, \opset), \trans
    }{%
    }
\]

\[
    \infer[\rl{TSeq}]{%
        \stk \vdash (\txstk, \h, \opset), \trans_{1} \pseq \trans_{2} \ \toL \  (\stk', \h', \opset'), \trans_{1}' \pseq \trans_{2}
    }{%
        \stk \vdash (\txstk, \h, \opset), \trans_{1} \ \toL \  (\stk', \h', \opset'), \trans_{1}'
    }
\]

\hrule\vspace{5pt}
\caption{The transaction operational semantics}
\label{fig:transaction_semantics}
\end{figure}


%\input{./formal_def/semantics/graph.tex}
\subsection{Program Semantics}

We model the state of database as a history heap, where each address has a list \emph{versions} from the initial one to the latest one.
Each version contains a value and a transaction identifier who writes it and a set of transactions who read it.

\begin{defn}[History Heaps]
\label{def:his_heap}
Assuming a set of \emph{transactions identifiers} \( \TxID \defeq \Set{\txid, \dots}\), a \emph{history heap}, \( \hh \in \HisHeaps \), is a partial function from addresses to lists of \emph{versions}.
Each version a tuple contains a value, a transaction (identifier) and a set of transactions (identifiers).
\[
\begin{rclarray}
    \Versions & \defeq &  \Val \times \TxID \times \powerset{\TxID} \\
    \HisHeaps & \defeq & \Addr \parfinfun \Versions^{*}
\end{rclarray}
\]
The composition of two history heaps is \( \hh \composeHH \hh' \defeq \hh \uplus \hh' \) and the unit element is \( \unitHH \defeq \emptyset \). 
\end{defn}
 
Let \( \hh(\addr)(i)\) denotes the \emph{i-th} version associated with address \( \addr \), and \( \hhV(\addr)(i) \), \( \hhW(\addr)(i) \) and \( \hhR(\addr)(i) \) denote the first (value), second (write) and third element (read) of the version.
We assume the version number starts from 1.

For weak consistency models, or weak isolation levels which are more common term used in database community, a session is not necessary to work on the up-to-date version of a database in exchange for better performance. 
Even in a single machine database, a session running under weak consistency model can make less synchronisation with the hard drivers and other running threads, which means the session could observe out-of-date state.
We use \emph{views} to model sessions of a database.
A \emph{view} is a cut in a history heap that corresponds the version that a session work with.

\begin{defn}[Views]
\label{def:cuts}
\label{def:views}
\emph{A view of of a history heap}, short as \emph{view}, is a partial function from addresses to indexes,
\[
\begin{rclarray}
    \vi \in \viset \subseteq \Views & \defeq & \Addr \parfinfun \Nat
\end{rclarray}
\]                                                                     
The composition is \( \vi \composeVI \vi' \defeq \vi \uplus \vi'\) and the unit is \( \unitVI \defeq \emptyset\).
The order between two views if they have the same domain is the order of all the indexes, 
\[
\begin{rclarray}
    \cu \orderCU \cu' & \defiff & \dom(\cu) = \dom(\cu') \land \for{\addr} \cu(\addr) \leq \cu'(\addr) \\
\end{rclarray}
\]
Assuming a set of thread identifiers \( \setthid \subseteq \ThreadID \), \emph{a view environment} is a function from thread identifiers to views, this is, \( \vienv \in \ViEnv \defeq \ThreadID \parfun \Views \).
\end{defn}

Consistency models are specified as relations over pairs of history heaps and cut environments, and it describes what transitions are allowed.

\begin{defn}[Consistency Models]
\label{def:consistency-models}
A \emph{Consistency model} \( \como \) is a \emph{reflexive and transitive relation} over pairs of history heaps and cut environments, written \( ((\hh,\viset),(\hh',\viset')) \in \como \) or \( (\hh, \viset) \toCO{\como} (\hh', \viset' )\), that the transition from the state \( (\hh,\viset)\) to the state \( (\hh',\viset') \) satisfies certain constraints.
Each tuple in the set should satisfy a well-formed condition that the cut environment should contain no more address than the history heap,
\[
    \bigwedge\limits_{\vi \in \viset} \dom(\vi) \subseteq \dom(\hh)
\]
\sx{Might be useful constraint for soundness of logic, if never used, just delete this.}
The specification of a certain consistency model should be local to the owned history heaps, so that the combination of two disjointed parts also satisfies the specification.
\[
    \begin{array}{@{}l}
    \for{\hh,\hh',\hh'',\hh''', \viset, \viset', \viset'', \viset'''} \\
        \quad ((\hh, \viset),(\hh',\viset')) \in \como 
        \land ((\hh'', \viset''),(\hh''',\viset''')) \in\como 
        \land ( \hh \composeHH \hh'' )\isdef
        \land ( \hh' \composeHH \hh''' )\isdef \\
        \qquad \implies  ((\hh \composeHH \hh'', \viset \uplus \viset''),(\hh' \composeHH \hh''',\viset' \uplus \viset''')) \in \como
    \end{array}
\]
\sx{The following says the consistency model transfer from a valid state to another valid state. }
A consistency model also requires to transfer from a \emph{valid} state to another \emph{valid} state where the validity is independent from the transition.
\[
    \for{\como} \exsts{\funcn{valid}} (\hh, \viset) \in \como \implies \func{valid}{\hh, \viset}
\]
where \( (\hh , \viset) \in \como\) means the state \((\hh, \viset)\) appears in the relation.
\end{defn}

\sx{Need Andrea for more clarify about defining consistency model}

There are two ways to specify consistency models, one by converting the history heaps to dependant graphs and checking if the cycles satisfy certain property, while another by checking if the view environment satisfy certain property.
We introduce \( \funcn{graph} : \HisHeaps \to \sort{DGraph} \) function for converting history heaps to dependant graphs.
\[
\begin{rclarray}
    \DGraph & \defeq & \Setcon{(\txidset, \ww, \wrr, \rw)}{ \txidset \subseteq \TxID \land \ww, \wrr, \rw \subseteq (\txidset \times \txidset)} \\
    \func{graph}{\hh} & \defeq & \left(
    \begin{array}{@{}l}
        \Setcon{\txid}{\exsts{\addr, i} \txid = \hhW(\addr)(i) \lor \txid \in \hhR(\addr)(i) }, \\
        \Setcon{(\txid, \txid')}{\exsts{\addr, i, i'} \txid = \hhW(\addr)(i) \lor \txid' = \hhW(\addr)(i') \land i < i' }, \\
        \Setcon{(\txid, \txid')}{\exsts{\addr, i} \txid = \hhW(\addr)(i) \lor \txid' \in \hhR(\addr)(i) }, \\
        \Setcon{(\txid, \txid')}{\exsts{\addr, i, i'} \txid \in \hhR(\addr)(i) \lor \txid' = \hhW(\addr)(i') \land i < i' }, \\
    \end{array}
    \right)
\end{rclarray}
\]

\begin{example}[Serialisibility]
\end{example}

\begin{example}[Snapshot]
Any cycle in the corresponding dependant graph of the history heap after the transition must have adjacent anti-dependent edges (\( \rw \) edges).
\[
    \begin{rclarray}
    \SI & \defeq & \Setcon{((\hh, \thcu),(\hh', \thcu'))}{\exsts{\ww, \rw, \wrr} (\stub, \ww, \wrr, \rw)  = \func{graph}{\hh'} \land (\wrr \cup \ww) ; \rw? \text{ is acyclic}}
    \end{rclarray}
\]
\end{example}

%{ \color{gray}
%\begin{defn}[Thread transition labels]
%\label{def:label}
%Given the set of thread identifiers \(\thid \in \ThreadID\), the set of \emph{thread transition labels}, $\lb \in \Translabel$, is defined by the following grammar, where $\prog$ denotes a program (\defref{def:language}), the $\txid$ demotes a transaction identifier and $\thstk$ denotes a thread stack (\defref{def:stacks}),
%\[
    %\begin{rclarray}
	%\iota \in \Translabel & ::= & \lbID \mid \lbC{\txid} \mid \lbF{\thid,\prog} \mid \lbJ{\thid,\thstk}
    %\end{rclarray}
%\]
%\end{defn}
%}

%{ \color{gray}
%A spacial version of this function only takes a history heap, and it returns a normal heap that is the latest state of the database.
%}

%{ \color{gray}
%The function is overloaded with one parameter version that collapses history heaps by picking the last values,
%\[
%\begin{rclarray}
    %\clpsHH{\hh} & \defeq & \lambda \addr \ldotp \hhV(\addr)(\left|\hh(\addr)\right|) 
%\end{rclarray}
%\]
%}
%\end{defn}

\begin{defn}[Thread semantics]
\label{def:thread_semantics}
%{ \color{gray}
%Given the thread identifiers \(\thid \in \ThreadID\), the set of \emph{intermediate programs}, \(\iprog \in \IntermediatePrograms\), is defined by the following grammar,
%\[
    %\iprog ::= \prog \mid \iprog \pseq \pwait{\thid}
%\]
%}
Given the set of consistency model \( \ConsisModels \) (\defref{def:consistency-models}), stacks \( \Stacks \) (\defref{def:stacks}), history heaps \( \HisHeaps \) (\defref{def:his_heap}) and views \( \Views \) (\defref{def:cuts}), the \emph{per-thread operational semantics} of programs,
\[
\begin{rclarray}
	\toT{} & : &
    \begin{array}[t]{@{}l@{}}
    \Como 
    \times {} \\
	\quad \left( ( \Stacks \times \HisHeaps \times \Views ) \times \Commands \right) 
	\times \Translabel \times
	\left( ( \Stacks \times \HisHeaps \times \Views ) \times \Commands \right) 
    \end{array}
\end{rclarray}
\]
is defined in \figref{fig:thread_semantics}.
The labels are for technical reason because some steps need to satisfy certain constraints related other threads. 
The labels includes commit \( \lbC{\txid} \), view shift \( \lbV \) and identify \( \lbID \).
\end{defn}

\begin{figure}[!t]
%
\hrule
%
\[
    \infer[\rl{PCommit}]{%
        \como \vdash ( \stk, \hh, \vi ), \ptrans{\trans} \ \toT{\lbC{\txid}} \ ( \stk', \hh', \vi' ) , \pskip
    }{%
        \begin{array}{c}
            \txid \in \func{fresh}{\hh}  
            \quad \h = \clpsHH{\hh,\vi}
            \quad \vdash (\stk, \h, \unitO), \trans \ \toL^{*} \  (\stk', \h', \opset) , \pskip \\
            \hh' = \func{updHisHp}{\hh, \vi, \txid, \opset}  
            \quad \func{updView}{\hh', \vi, \opset} \orderVI \vi 
            \quad \pred{wfView}{\hh', \vi'}  \\
            \hh,\Set{\vi} \toCO{\como} \hh',\Set{\vi}
        \end{array}
    }
\]

\[
    \infer[\rl{PAssign}]{%
        \como \vdash ( \stk, \hh, \vi ) , \passign{\var}{\expr} \ \toT{\lbID} \  ( \stk\rmto{\var}{\val}, \hh, \vi ) , \pskip
    }{
        \val = \evalE{\expr}
    }
\]

\[
    \infer[\rl{PAssume}]{%
        \como \vdash ( \stk, \hh, \vi ) , \passume{\expr} \ \toT{\lbID} \  ( \stk, \hh, \vi ) , \pskip
    }{%
        \evalE[\thstk]{\expr} = 0
    }
\]

\[
    \infer[\rl{PViewLShift}]{
        \como \vdash ( \stk, \hh, \vi ) , \cmd \ \toT{\lbV} \  ( \stk, \hh, \cu' ) , \cmd
    }{
        \vi \orderVI \vi'
    }
\]

\[
    \infer[\rl{PChoice}]{%
        \como \vdash ( \stk, \hh, \vi ) , \cmd_{1} \pchoice \cmd_{2} \ \toT{\lbID} \  ( \stk, \hh, \vi ) , \cmd'
    }{
        \cmd' \in \Set{\cmd_{1}, \cmd_{2}}
    }
\]

\[
    \infer[\rl{PLoop}]{%
        \como \vdash ( \stk, \hh, \vi ) , \cmd\prepeat \ \toT{\lbID} \  ( \stk, \hh, \vi ) , \pskip \pchoice (\cmd \pseq \cmd\prepeat)
    }{}
\]

\[
    \infer[\rl{PSeqSkip}]{%
        \como \vdash ( \stk, \hh, \vi ) , \pskip \pseq \cmd \ \toT{\lbID} \  ( \stk, \hh, \vi ) , \cmd
    }{}
\]

\[
    \infer[\rl{PSeq}]{%
        \como \vdash ( \stk, \hh, \vi ) , \cmd_{1} \pseq \cmd_{2} \ \toT{\lb} \ ( \stk', \hh, \vi' ) , {\cmd_{1}}' \pseq \cmd_{2}
    }{% 
        \como \vdash ( \stk, \hh, \vi ) , \cmd_{1} \ \toT{\lb} \  ( \stk', \hh, \vi' ) , {\cmd_{1}}' 
    }
\]

%{ \color{gray}
%\[
    %\infer[\rl{PPar}]{%
        %\como, \thid \vdash ( \thstk, \hh, \thcu ) , \prog_{1} \ppar \prog_{2} \ \toT{\lbF{\thid', \prog_{2}}} \  \left( \thstk, \hh, \thcu \uplus \Set{\thid' \mapsto \thcu(\thid)} \right) , \prog_{1} \pseq \pwait{\thid'}
    %}{}
%\]

%\[
    %\infer[\rl{PWait}]{%
        %\como, \thid \vdash \left( \thstk_{1} \uplus \thstk_{f}, \hh, \thcu \uplus \Set{\thid' \mapsto \thcu(\thid)} \right) , \pwait{\thid'} \ \toT{\lbJ{\thid', \thstk_{2} \uplus \thstk_{f}}} \  ( \thstk_{1} \uplus \thstk_{2} \uplus \thstk_{f}, \hh, \thcu ) , \pskip 
    %}{}
%\]
%}

\hrule
 
\[
\begin{rclarray}                                 
    \pred{wfView}{\hh,\vi} & \defeq & \for{\addr} \vi(\addr) \leq \left|\hh(\addr) \right| \\
    \clpsHH{\hh, \cu} & \defeq & 
    \begin{cases}
        \lambda \addr \ldotp \hhV(\addr)(\cu(\addr)) & \text{if } \dom(\hh) = \dom(\cu) \land \for{\addr'} \cu(\addr') \leq \left|\hh(\addr')\right| \\
        \text{undefined} & \text{otherwise}
    \end{cases} \\
    \func{updHisHp}{\hh, \vi, \txid, \opset} & \defeq & 
    \begin{cases}
        \hh & \text{if } \opset = \unitO \\
        \func{updHisHp}{\hh', \vi, \txid, \opset'} & \text{if } \opset = \opset' \addO (\otR, \addr, \val) \\
        \func{updHisHp}{\hh'', \vi, \txid, \opset'} & \text{if } \opset = \opset' \addO (\otW, \addr, \val) \\
    \end{cases} \\
    \hh' & \equiv & \hh\rmto{\addr}{\hh(\addr)\rmto{\vi(\addr)}{\left(\hhV(\addr)(\vi(\addr)),\hhW(\addr)(\vi(\addr)),\hhR(\addr)(\vi(\addr)) \uplus \Set{\txid} \right)}} \\
    \hh'' & \equiv & \hh\rmto{\addr}{\hh(\addr) \lcat \List{(\val, \txid, \emptyset)} } \\
%
%
    \func{updView}{\hh, \vi, \opset} & \defeq &
    \begin{cases}
        \vi & \text{if } \opset = \unitO \\
        \func{updView}{\hh, \vi, \opset'} & \text{if } \opset = \opset' \addO (\otR, \addr, \val) \\
        \func{updView}{\hh, \vi\rmto{\addr}{\left| \hh(\addr) \right|}, \opset'} & \text{if } \opset = \opset' \addO (\otW, \addr, \val) \\
    \end{cases} \\
%
%              
	\func{fresh}{\hh}  & \defeq & \Setcon{ \txid }{ \txid \in \TxID \land \for{\addr, i} \txid \neq \hhW(\addr)(i) \land \txid \notin \hhR(\addr)(i) } \\
\end{rclarray}
\]
\hrule
\caption{Per-thread operational semantics}
\label{fig:thread_semantics}
\end{figure}

A new transaction is committed through the \rl{PCommit} rule.
The transaction code \( \trans \) is executed locally given the local state that is decided by the current state of database and the local view, \ie the history heap \( \hh \) and view \( \vi \) respectively.
The \( \funcn{clps} \) function collapses a history heap and a view to a normal heap by picking the versions corresponding to the view.
After local execution, the transaction picks a fresh identifier \( \txid \) and commits the set of operations \( \opset \).
The operations are the first read and last write of each address, which are the operations might affect the database, because of the atomicity of transactions.
The \funcn{updHisHp} function updates the history heap.
If there is a read operation, it includes the new identifier to the version relates to the address and the local view \( \vi \).
If there is a write operation, it extends a new version to the end and puts the new identifier as the writer.
Since we assume strong session, we set the lower bound for the new local view by \funcn{updView} function.
This function shifts the view to the latest version if the version is written by the current transaction.
This guarantees if there is no currency, the following transaction will read the write.
Since it is a lower bound, the new local view \( \vi' \) can advance the view as long as it is not out-of-bound.
Finally, the transition from \( (\hh, \vi) \) to \( (\hh', \vi') \) should be allowed by the consistency model \( \como \).

To model synchronisation between sessions, the \rl{PViewLShift} rule allows to non-deterministically advance a view.
Except \rl{PCommit} and \rl{PViewLShift}, the rest rules have the expected behaviours.

\begin{lem}[Confluence of \funcn{updHisHp} and \funcn{updView}]
Given a valid set of operations \( \opset \), all the computation paths for \( \funcn{updHisHp} \) function (\( \funcn{updView} \) function) yield the same result.
\end{lem}

%\sx{The full version should support dynamic thread, but we might only need fixed thread so that the thread pool semantics will be pick one thread and run one step}

%{ \color{gray}
%\begin{defn}[Thread pools]
%\label{def:thread_pools}
%Given the set of thread stacks $\ThdStacks$ (\defref{def:stacks}) and intermediate programs $\IntermediatePrograms$ (\defref{def:thread_semantics}), a \emph{thread pool} is a a finite partial function from thread identifiers to triples of thread stacks and intermediate programs, \(\thpl \in \TPool \eqdef \ThreadID \parfinfun \ThdStacks \times \IntermediatePrograms\).
%\end{defn}
%}
 
\begin{defn}[Programs semantics] 
\label{def:thread_pool_semantics}
\label{def:program_semantics}
Given the set of consistent models \( \ConsisModels \) (\defref{def:consistency-models}), history heaps \(\HisHeaps\) (\defref{def:his_heap}) and view environment \(\ViEnv \) (\defref{def:cuts}), the \emph{thread pool semantics}, 
\[
	\toG{} : \Como 
    \times ( \StkEnv \times \HisHeaps \times \ViEnv \times \Programs) 
    \times ( \StkEnv \times \HisHeaps \times \ViEnv \times \Programs) 
\]
is defined in \figref{fig:thread_pool_semantics}.
The stack environment \( \StkEnv \) is a partial function from thread identifiers to stacks.
\end{defn}
 
The program has standard interleaving semantics by picking a thread and then running one step.
If the step is commit \( \lbC{\txid} \), the overall transition of history heap and view environment should satisfy consistency model.
In anther words, the view that has been changed should be compatible with those has not been changed.
It is necessary to check the compatibility because some consistency models are specified as constraints among all views.
We write \( \hh, \vienv \toCO{\como} \hh', \vienv' \) as short-hand for \( \hh, \func{range}{\vienv} \toCO{\como} \hh', \func{range}{\vienv'} \).
If the step is view shift \( \lbV \), it should shift to a valid state\footnote{It is not necessary to always shift to a valid state, yet it is a design choice for abstract operational semantics.}.
 
%The thread pool operational semantics is given in \figref{fig:thread_pool_semantics}, where an arbitrary thread in the pool \(\thpl\) is picked to run for one step.
%If the next execution step is a thread fork, then a new thread \(\thid'\) is allocated in the pool to be executed with its thread stack copied from its parent (forking) thread.
%Conversely, when the next execution step is the joining of thread \(\thid'\), then \(\thid'\) is removed from the thread pool and the stack from the child thread merges into the parent thread.

\begin{figure}
\hrule\vspace{5pt}
%
\[
    \infer[\rl{PSingle}]{%
        \como \vdash ( \stkenv, \hh, \vienv, \prog ) \ \toG \  ( \stkenv\rmto{\thid}{\stk'}, \hh', \vienv\rmto{\thid}{\vi'}, \prog\rmto{\thid}{\cmd'} ) 
    }{%
    \begin{array}{c}
        \stk = \stkenv(\thid)
        \quad \vi = \vienv(\thid)
        \quad \cmd = \prog(\thid) \\
        \como \vdash ( \stk, \hh, \vi ), \cmd, \ \toT{\lb} \  ( \stk', \hh', \vi' ) , \cmd'  \\
        \begin{B}
        \lb = \lbID \lor (\lb = \lbC{\stub} \land \hh, \vienv \toCO{\como} \hh', \vienv') \lor \begin{B}\lb = \lbV \land \pred{valid}{\hh',\vienv'}\end{B}
        \end{B}
    \end{array}
    }
\]

%{ \color{gray}
%\[
    %\infer[\rl{PFork}]{%
        %\como \vdash ( \hh, \thcu, \thpl \uplus \Set{ \thid \mapsto (\thstk, \iprog) } ) \ \toG{\lbF{\thid', \prog}} \  ( \hh', \thcu', \thpl \uplus \Set{ \thid \mapsto (\thstk, {\iprog}'), \thid' \mapsto (\thstk, \prog) } )
    %}{%
        %\como, \thid \vdash ( \thstk, \hh, \thcu ) , \iprog \ \toT{\lbF{\thid', \prog}} \  ( \thstk, \hh', \thcu' ) , {\iprog}' 
    %}
%\]

%\[
    %\infer[\rl{PJoin}]{%
        %\como \vdash ( \hh, \thcu, \thpl \uplus \Set{ \thid \mapsto (\thstk, \iprog), \thid' \mapsto (\thstk', \pskip) } )  \ \toG{\lbJ{\thid',\thstk''}} \ ( \hh', \thcu', \thpl \uplus \Set{ \thid \mapsto (\thstk'', {\iprog}')} )
    %}{%
        %\como, \thid \vdash ( \thstk, \hh, \thcu ) , \iprog \ \toT{\lbJ{\thid',\thstk'}} \  ( \thstk'', \hh', \thcu' ) , {\iprog}' 
    %}
%\]
%}
%
\hrule\vspace{5pt}
\caption{Thread pool semantics}
\label{fig:thread_pool_semantics}
\end{figure}

\subsection{Example of Consistency models}
\ac{This Section is going to become heavy in pictures, which should be organised into figures.}
In this Section we present different consistency models specifications. 
For each of them, we give: 
\begin{itemize}
\item the intuition of the commit tests for different consistency models, and the formal definitions with respect to the \(\Como\) (\defref{def:consistency-models}).
%describing the consistency guarantees that schedules of the database should have in plain English, 
%\item a formal consistency model specification, in the style described in \S \ref{sec:semantics.programs},
\item examples of litmus tests that, when executed, give rise to the anomalies that are forbidden from the consistency model, 
\item an explanation of why the consistency model forbids the litmus tests to exhibit the anomaly that should be forbidden. 
\end{itemize}
Later, we will show how to compare our consistency models specifications with those already existing in the 
literature.
\ac{There is still a long-way to go before proving correspondence with dependency graph specifications, 
but this should be mentioned here.}


\subsection{Read Atomic} 
Read atomic (RA) \cite{ramp} is the weakest consistency model among those that enjoy \emph{atomic visibility} \cite{framework-concur}. 
It requires of a transaction to read an atomic snapshot of the database and never observe the partial effects of other transactions.
This is also known as the \emph{all-or-nothing} property: a transaction observes either none or all the updates performed by other transactions. 

\sx{example RAMP, not sure the meaning}
One litmus test that should \textbf{not} be failed in RAMP consists of the program $\prog_1$ from \S \ref{sec:semantics.example}, which we already observed to produce a violation of atomic visibility if no constraints on the consistency model are placed.
\ac{not be failed. Double negation. Bad English.}

\sx{Rephrase}
Intuitively, in such a program, it violate the atomic visibility because we allowed to execute the transaction \( \trans_1^2\) in the thread-local configuration of $\mathcal{C}'$ relative to $\thid_2$, which is obtained by removing all the information about $\thid_1$ (view and stack) in Figure \ref{fig:opsem.example}(c).
\[
\trans_1^2 = \begin{array}{c} 
            \begin{transaction}
            		\pderef{\pvar{a}}{\vx};\\
            		\pderef{\pvar{b}}{\vy};\\
            		\pifs{\pvar{a}=1 \wedge \pvar{b}=0}\\
            			\quad \passign{\retvar}{\Large \frownie{}} \\
                    \pife
             \end{transaction}
     \end{array}
\]

\ac{
To avoid transactions to only observe the partial effects of other transactions, we 
must ensure that transactional code cannot be executed by a thread whose 
views is up-to-date with respect to some transaction $\tsid$ for some location $[\loc_x]$, 
but not for some other location $[\loc_y]$. This leads to the following definition.
}
To avoid a transaction to observe the partial effects of other transactions, we need to ensure that transactional code cannot be executed by a thread whose views is partially up-to-date with respect to some transactions. This leads to the following definition.
\begin{defn}
\label{def:readatomic}
%Let $\hh$ be a history heap,$V$ be a view, $[\loc_x]$ be 
%a location and $\nu$ be a version. We say that $V$ $[\loc_x]$-\emph{sees} version 
%$\nu$ if there exists an index $i \leq V([\loc_x])$ such that $V(i) = \nu$. 
%We say that $V$ $[\loc_x]$-\emph{sees} transaction $\tsid$ if 
%$V$ $[\loc_x]$-sees a version $\nu = (\_, \tsid, \_)$. 
Given a view $\vi \in \Views$, a history heap $\hh \in \HisHeaps$, and a transaction identifier $\txid \in \TxID$, the view \emph{sees} the transaction in the history heap, written $\pred{visible}{\txid, \vi, \hh}$, if the view sees all the writes from the transaction,
%We say that $V$ \emph{sees} transaction $\tsid$ in $\hh$, written 
%$\mathsf{Visible}(\tsid, V, \hh)$, iff 
\sx{\( \exsts{i} \) might be enough}
\[
\begin{rclarray}
\pred{visible}{\txid, \vi, \hh} & \eqdef & \fora{\addr, i} \hh(\addr)(i) = (\stub, \txid, \stub) \implies i \leq \vi(\addr).
\end{rclarray}
\]
\ac{In English: the view is up-to-date with respect to all the updates 
performed by transaction $\tsid$.}

Then given a history heap \( \hh \), the view $V$ is \emph{consistent} with respect to \emph{atomic visibility}, written $\pred{atomic}{\vi, \hh}$, if the view $V$ is up-to-date with some of the updates performed by $\txid$, then it should be up-to-date with all the updates performed by $\txid$,
\[
\begin{rclarray}
\pred{atomic}{\vi ,\hh} & \eqdef & \fora{\txid } \exsts{\addr, i} i \leq \vi(\addr) \land \hh(\addr)(i) = (\stub, \txid, \stub) \implies \pred{vusible}{\txid, \vi, \hh}
\end{rclarray}
\]
\ac{In English: if the view $V$ is up-to-date with some of the updates performed 
by $\tsid$, then it must be up-to-date with all the updates performed by $\tsid$. 
This is the all-or-nothing property.}
%for all location 
%$[\loc_x]$, if there exists an index $i = 0,\cdots, \lvert \hh([\loc_x]) \rvert - 1$, 
%such that $\hh([\loc_x])(i) = (\_, \tsid, \_)$, then $i \leq V([\loc_x])$.

The consistency model specification $\mathsf{RA}$ is defined as the smallest set such that  
\sx{what is the meaning of smallest?}
\[
\pred{atomic}{\hh, \vi} \implies (\hh, \vi) \csat[\mathsf{RA}] \stub: \stub
\]
\ac{In English: Before executing a transaction, either you observe all or none the 
updates of all other transactions. We may strengthen the consistency model and 
require that the same property must be satisfied at the end as well, though 
this is not strictly necessary. In this case the check becomes: 
\[
\mathsf{atomic}(\hh, V) \wedge \mathsf{atomic}(\hh, V') \wedge \mathsf{UpdateView}(\hh, V, \mathcal{O}) 
\sqsubseteq V' \implies (\hh, V) \triangleright_{\mathsf{RA}} \mathcal{O}: V'.
\]
}
%written $\mathsf{up-to-date}(\hh, V, \tsid, [\loc_x])$, 
%if either 
%
%\begin{itemize}
%\item for all indexes $i = 0,\cdots, \lvert \hh([\loc_x]) - 1 \rvert$, 
%$\WS(\hh([\loc_x])(i)) \neq \tsid)$, or 
%\item if $\WS(\hh([\loc_x])(i)) = \tsid$ for some $i = 0,\cdots, \lvert \hh([\loc_n]) -1 \rvert$, 
%then $i \leq V([\loc_n])$.
%\end{itemize}
\end{defn}

\sx{Not sure how to link the explanation from Andrea's document, sort out later }
Suppose that we execute the program $\prog_1$ under the consistency model specification $\mathsf{RA}$.
We can proceed as in Section \ref{sec:semantics.example} to infer the transition 
$\langle \mathcal{C_0}, \prog_1 \rangle \xrightarrow{\mathsf{RA}} \langle \mathcal{C}_1, \prog_1' \rangle$, 
where we recall that $\mathcal{C}_0$, $\mathcal{C}_1$ are depicted in Figure \ref{fig:opsem.exampe}(a), 
\ref{fig:opsem.example}(b), respectively. 

It is immediate to observe that the only way in which the execution of transaction $\ptrans{\trans}$ from $\thid_2$ in $\prog_1'$ can return value ${\Large \frownie}$ is the following: 
\begin{itemize}
\item first, push the view $V$ of thread $\txid_2$ in the configuration 
$\mathcal{C}_1$ of Figure \ref{fig:opsem.example}(b) to observe the update of location $[\vx]$, but not the update of 
$[\vy]$. This view is the one labelled with $\txid_2$ in Figure \ref{fig:opsem.example}(c), and we refer 
to it as $V'$;
\item then, execute the transaction $\ptrans{\trans}$ in $\thid_2$. 
\end{itemize}

\subsection{Causal Consistency}
%\begin{figure}
%\begin{tabular}{|c|c|}
%\hline
%\begin{tikzpicture}[font=\large]

%\begin{pgfonlayer}{foreground}
%%Uncomment line below for help lines
%%\draw[help lines] grid(5,4);

%%Location x
%\node(locx) at (1,3) {$[\loc_x] \mapsto$};

%\matrix(locxcells) [version list, text width=7mm, anchor=west]
   %at ([xshift=10pt]locx.east) {
 %{a} & $T_0$ \\
  %{a} & $\emptyset$ \\
%};
%\node[version node, fit=(locxcells-1-1) (locxcells-2-1), fill=white, inner sep= 0cm, font=\Large] (locx-v0) {$0$};

%%Location y
%\path (locx.south) + (0,-1.5) node (locy) {$[\loc_y] \mapsto$};
%\matrix(locycells) [version list, text width=7mm, anchor=west]
   %at ([xshift=10pt]locy.east) {
 %{a} & $T_0$ \\
  %{a} & $\emptyset$ \\
%};
%\node[version node, fit=(locycells-1-1) (locycells-2-1), fill=white, inner sep= 0cm, font=\Large] (locy-v0) {$0$};

%% \draw[-, red, very thick, rounded corners] ([xshift=-5pt, yshift=5pt]locx-v1.north east) |- 
%%  ($([xshift=-5pt,yshift=-5pt]locx-v1.south east)!.5!([xshift=-5pt, yshift=5pt]locy-v0.north east)$) -| ([xshift=-5pt, yshift=5pt]locy-v0.south east);

%%blue view - I should  check whether I can use pgfkeys to just declare the list of locations, and then add the view automatically.
%\draw[-, blue, very thick, rounded corners=10pt]
 %([xshift=-2pt, yshift=20pt]locx-v0.north east) node (tid1start) {} -- 
%% ([xshift=-2pt, yshift=-5pt]locx-v0.south east) --
%% ([xshift=-2pt, yshift=5pt]locy-v0.north east) -- 
 %([xshift=-2pt, yshift=-5pt]locy-v0.south east);
 
 %\path (tid1start) node[anchor=south, rectangle, fill=blue!20, draw=blue, font=\small, inner sep=1pt] {$\tid_3$};

%%red view
%\draw[-, red, very thick, rounded corners = 10pt]
 %([xshift=-5pt, yshift=5pt]locx-v0.north east) -- 
%% ([xshift=-8pt, yshift=-5pt]locx-v0.south east) --
%% ([xshift=-8pt, yshift=5pt]locy-v0.north east) -- 
 %([xshift=-5pt, yshift=-10pt]locy-v0.south east) node (tid2start) {};
 
%\path (tid2start) node[anchor=north, rectangle, fill=red!20, draw=red, font=\small, inner sep=1pt] {$\tid_2$};
 
 %%green view
%\draw[-, DarkGreen, very thick, rounded corners = 10pt]
 %([xshift=-16pt, yshift=8pt]locx-v0.north east) node (tid3start) {}-- 
%% ([xshift=-15pt, yshift=-5pt]locx-v0.south east) --
%% ([xshift=-15pt, yshift=5pt]locy-v0.north east) -- 
 %([xshift=-16pt, yshift=-5pt]locy-v0.south east);
 
 %\path (tid3start) node[anchor=south, rectangle, fill=DarkGreen!20, draw=DarkGreen, font=\small, inner sep=1pt] {$\tid_1$};

%\end{pgfonlayer}
%\end{tikzpicture}
%&
%\begin{tikzpicture}[font=\large]

%\begin{pgfonlayer}{foreground}
%%Uncomment line below for help lines
%%\draw[help lines] grid(5,4);

%%Location x
%\node(locx) at (1,3) {$[\loc_x] \mapsto$};

%\matrix(locxcells) [version list, text width=7mm, anchor=west]
   %at ([xshift=10pt]locx.east) {
 %{a} & $\tsid_0$ & {a} & $\tsid_1$\\
  %{a} & $\emptyset$ & {a} & $\emptyset$ \\
%};
%\node[version node, fit=(locxcells-1-1) (locxcells-2-1), fill=white, inner sep= 0cm, font=\Large] (locx-v0) {$0$};
%\node[version node, fit=(locxcells-1-3) (locxcells-2-3), fill=white, inner sep=0cm, font=\Large] (locx-v-1) {$1$};
%%Location y
%\path (locx.south) + (0,-1.5) node (locy) {$[\loc_y] \mapsto$};
%\matrix(locycells) [version list, text width=7mm, anchor=west]
   %at ([xshift=10pt]locy.east) {
 %{a} & $\tsid_0$ \\
   %{a} & $\emptyset$ \\
%};
%\node[version node, fit=(locycells-1-1) (locycells-2-1), fill=white, inner sep= 0cm, font=\Large] (locy-v0) {$0$};
%% \draw[-, red, very thick, rounded corners] ([xshift=-5pt, yshift=5pt]locx-v1.north east) |- 
%%  ($([xshift=-5pt,yshift=-5pt]locx-v1.south east)!.5!([xshift=-5pt, yshift=5pt]locy-v0.north east)$) -| ([xshift=-5pt, yshift=5pt]locy-v0.south east);

%%blue view - I should  check whether I can use pgfkeys to just declare the list of locations, and then add the view automatically.
%\draw[-, blue, very thick, rounded corners=10pt]
 %([xshift=-2pt, yshift=20pt]locx-v0.north east) node (tid1start) {} -- 
%% ([xshift=-2pt, yshift=-5pt]locx-v0.south east) --
%% ([xshift=-2pt, yshift=5pt]locy-v0.north east) -- 
 %([xshift=-2pt, yshift=-5pt]locy-v0.south east);
 
 %\path (tid1start) node[anchor=south, rectangle, fill=blue!20, draw=blue, font=\small, inner sep=1pt] {$\tid_3$};

%%red view
%\draw[-, red, very thick, rounded corners = 10pt]
 %([xshift=-5pt, yshift=5pt]locx-v0.north east) -- 
%% ([xshift=-8pt, yshift=-5pt]locx-v0.south east) --
%% ([xshift=-8pt, yshift=5pt]locy-v0.north east) -- 
 %([xshift=-5pt, yshift=-10pt]locy-v0.south east) node (tid2start) {};
 
%\path (tid2start) node[anchor=north, rectangle, fill=red!20, draw=red, font=\small, inner sep=1pt] {$\tid_2$};
 
 %%green view
%\draw[-, DarkGreen, very thick, rounded corners = 10pt]
 %([xshift=-16pt, yshift=8pt]locx-v1.north east) node (tid3start) {}-- 
 %([xshift=-16pt, yshift=-5pt]locx-v1.south east) --
 %([xshift=-16pt, yshift=5pt]locy-v0.north east) -- 
 %([xshift=-16pt, yshift=-5pt]locy-v0.south east);
 
 %\path (tid3start) node[anchor=south, rectangle, fill=DarkGreen!20, draw=DarkGreen, font=\small, inner sep=1pt] {$\tid_1$};

%\end{pgfonlayer}
%\end{tikzpicture}
%\\
%{\small (a)} & {\small (b)}\\
%\hline
%\begin{tikzpicture}[font=\large]

%\begin{pgfonlayer}{foreground}
%%Uncomment line below for help lines
%%\draw[help lines] grid(5,4);

%%Location x
%\node(locx) at (1,3) {$[\loc_x] \mapsto$};

%\matrix(locxcells) [version list, text width=7mm, anchor=west]
   %at ([xshift=10pt]locx.east) {
 %{a} & $\tsid_0$ & {a} & $\tsid_1$\\
  %{a} & $\emptyset$ & {a} & $\emptyset$ \\
%};
%\node[version node, fit=(locxcells-1-1) (locxcells-2-1), fill=white, inner sep= 0cm, font=\Large] (locx-v0) {$0$};
%\node[version node, fit=(locxcells-1-3) (locxcells-2-3), fill=white, inner sep=0cm, font=\Large] (locx-v-1) {$1$};
%%Location y
%\path (locx.south) + (0,-1.5) node (locy) {$[\loc_y] \mapsto$};
%\matrix(locycells) [version list, text width=7mm, anchor=west]
   %at ([xshift=10pt]locy.east) {
 %{a} & $\tsid_0$ \\
   %{a} & $\emptyset$ \\
%};
%\node[version node, fit=(locycells-1-1) (locycells-2-1), fill=white, inner sep= 0cm, font=\Large] (locy-v0) {$0$};
%% \draw[-, red, very thick, rounded corners] ([xshift=-5pt, yshift=5pt]locx-v1.north east) |- 
%%  ($([xshift=-5pt,yshift=-5pt]locx-v1.south east)!.5!([xshift=-5pt, yshift=5pt]locy-v0.north east)$) -| ([xshift=-5pt, yshift=5pt]locy-v0.south east);

%%blue view - I should  check whether I can use pgfkeys to just declare the list of locations, and then add the view automatically.
%\draw[-, blue, very thick, rounded corners=10pt]
 %([xshift=-2pt, yshift=20pt]locx-v0.north east) node (tid1start) {} -- 
%% ([xshift=-2pt, yshift=-5pt]locx-v0.south east) --
%% ([xshift=-2pt, yshift=5pt]locy-v0.north east) -- 
 %([xshift=-2pt, yshift=-5pt]locy-v0.south east);
 
 %\path (tid1start) node[anchor=south, rectangle, fill=blue!20, draw=blue, font=\small, inner sep=1pt] {$\tid_3$};

%%red view
%\draw[-, red, very thick, rounded corners = 10pt]
 %([xshift=-5pt, yshift=5pt]locx-v1.north east) -- 
 %([xshift=-5pt, yshift=-5pt]locx-v1.south east) --
 %([xshift=-5pt, yshift=3pt]locy-v0.north east) -- 
 %([xshift=-5pt, yshift=-10pt]locy-v0.south east) node (tid2start) {};
 
%\path (tid2start) node[anchor=north, rectangle, fill=red!20, draw=red, font=\small, inner sep=1pt] {$\tid_2$};
 
 %%green view
%\draw[-, DarkGreen, very thick, rounded corners = 10pt]
 %([xshift=-16pt, yshift=8pt]locx-v1.north east) node (tid3start) {}-- 
 %([xshift=-16pt, yshift=-5pt]locx-v1.south east) --
 %([xshift=-16pt, yshift=5pt]locy-v0.north east) -- 
 %([xshift=-16pt, yshift=-5pt]locy-v0.south east);
 
 %\path (tid3start) node[anchor=south, rectangle, fill=DarkGreen!20, draw=DarkGreen, font=\small, inner sep=1pt] {$\tid_1$};

%\end{pgfonlayer}
%\end{tikzpicture}
%&
%\begin{tikzpicture}[font=\large]

%\begin{pgfonlayer}{foreground}
%%Uncomment line below for help lines
%%\draw[help lines] grid(5,4);

%%Location x
%\node(locx) at (1,3) {$[\loc_x] \mapsto$};

%\matrix(locxcells) [version list, text width=7mm, anchor=west]
   %at ([xshift=10pt]locx.east) {
 %{a} & $\tsid_0$ & {a} & $\tsid_1$\\
  %{a} & $\emptyset$ & {a} & $\{\tsid_2\}$ \\
%};
%\node[version node, fit=(locxcells-1-1) (locxcells-2-1), fill=white, inner sep= 0cm, font=\Large] (locx-v0) {$0$};
%\node[version node, fit=(locxcells-1-3) (locxcells-2-3), fill=white, inner sep=0cm, font=\Large] (locx-v-1) {$1$};
%%Location y
%\path (locx.south) + (0,-1.5) node (locy) {$[\loc_y] \mapsto$};
%\matrix(locycells) [version list, text width=7mm, anchor=west]
   %at ([xshift=10pt]locy.east) {
 %{a} & $\tsid_0$ & {a} & $\tsid_2$ \\
   %{a} & $\emptyset$ & {a} & $\emptyset$\\
%};
%\node[version node, fit=(locycells-1-1) (locycells-2-1), fill=white, inner sep= 0cm, font=\Large] (locy-v0) {$0$};
%\node[version node, fit=(locycells-1-3) (locycells-2-3), fill=white, inner sep=0cm, font=\Large] (locy-v-1) {$1$};
%% \draw[-, red, very thick, rounded corners] ([xshift=-5pt, yshift=5pt]locx-v1.north east) |- 
%%  ($([xshift=-5pt,yshift=-5pt]locx-v1.south east)!.5!([xshift=-5pt, yshift=5pt]locy-v0.north east)$) -| ([xshift=-5pt, yshift=5pt]locy-v0.south east);

%%blue view - I should  check whether I can use pgfkeys to just declare the list of locations, and then add the view automatically.
%\draw[-, blue, very thick, rounded corners=10pt]
 %([xshift=-2pt, yshift=20pt]locx-v0.north east) node (tid1start) {} -- 
%% ([xshift=-2pt, yshift=-5pt]locx-v0.south east) --
%% ([xshift=-2pt, yshift=5pt]locy-v0.north east) -- 
 %([xshift=-2pt, yshift=-5pt]locy-v0.south east);
 
 %\path (tid1start) node[anchor=south, rectangle, fill=blue!20, draw=blue, font=\small, inner sep=1pt] {$\tid_3$};

%%red view
%\draw[-, red, very thick, rounded corners = 10pt]
 %([xshift=-5pt, yshift=5pt]locx-v1.north east) -- 
%% ([xshift=-5pt, yshift=-5pt]locx-v1.south east) --
%% ([xshift=-5pt, yshift=3pt]locy-v0.north east) -- 
 %([xshift=-5pt, yshift=-10pt]locy-v1.south east) node (tid2start) {};
 
%\path (tid2start) node[anchor=north, rectangle, fill=red!20, draw=red, font=\small, inner sep=1pt] {$\tid_2$};
 
 %%green view
%\draw[-, DarkGreen, very thick, rounded corners = 10pt]
 %([xshift=-16pt, yshift=8pt]locx-v1.north east) node (tid3start) {}-- 
 %([xshift=-16pt, yshift=-5pt]locx-v1.south east) --
 %([xshift=-16pt, yshift=5pt]locy-v0.north east) -- 
 %([xshift=-16pt, yshift=-5pt]locy-v0.south east);
 
 %\path (tid3start) node[anchor=south, rectangle, fill=DarkGreen!20, draw=DarkGreen, font=\small, inner sep=1pt] {$\tid_1$};

%\end{pgfonlayer}
%\end{tikzpicture}\\
%{\small (c)} & {\small (d)} \\
%\hline
%\begin{tikzpicture}[font=\large]

%\begin{pgfonlayer}{foreground}
%%Uncomment line below for help lines
%%\draw[help lines] grid(5,4);

%%Location x
%\node(locx) at (1,3) {$[\loc_x] \mapsto$};

%\matrix(locxcells) [version list, text width=7mm, anchor=west]
   %at ([xshift=10pt]locx.east) {
 %{a} & $\tsid_0$ & {a} & $\tsid_1$\\
  %{a} & $\emptyset$ & {a} & $\{\tsid_2\}$ \\
%};
%\node[version node, fit=(locxcells-1-1) (locxcells-2-1), fill=white, inner sep= 0cm, font=\Large] (locx-v0) {$0$};
%\node[version node, fit=(locxcells-1-3) (locxcells-2-3), fill=white, inner sep=0cm, font=\Large] (locx-v-1) {$1$};
%%Location y
%\path (locx.south) + (0,-1.5) node (locy) {$[\loc_y] \mapsto$};
%\matrix(locycells) [version list, text width=7mm, anchor=west]
   %at ([xshift=10pt]locy.east) {
 %{a} & $\tsid_0$ & {a} & $\tsid_2$ \\
   %{a} & $\emptyset$ & {a} & $\emptyset$\\
%};
%\node[version node, fit=(locycells-1-1) (locycells-2-1), fill=white, inner sep= 0cm, font=\Large] (locy-v0) {$0$};
%\node[version node, fit=(locycells-1-3) (locycells-2-3), fill=white, inner sep=0cm, font=\Large] (locy-v-1) {$1$};
%% \draw[-, red, very thick, rounded corners] ([xshift=-5pt, yshift=5pt]locx-v1.north east) |- 
%%  ($([xshift=-5pt,yshift=-5pt]locx-v1.south east)!.5!([xshift=-5pt, yshift=5pt]locy-v0.north east)$) -| ([xshift=-5pt, yshift=5pt]locy-v0.south east);

%%blue view - I should  check whether I can use pgfkeys to just declare the list of locations, and then add the view automatically.
%\draw[-, blue, very thick, rounded corners=10pt]
 %([xshift=-2pt, yshift=20pt]locx-v0.north east) node (tid1start) {} -- 
 %([xshift=-2pt, yshift=-5pt]locx-v0.south east) --
 %([xshift=-2pt, yshift=5pt]locy-v1.north east) -- 
 %([xshift=-2pt, yshift=-5pt]locy-v1.south east);
 
 %\path (tid1start) node[anchor=south, rectangle, fill=blue!20, draw=blue, font=\small, inner sep=1pt] {$\tid_3$};

%%red view
%\draw[-, red, very thick, rounded corners = 10pt]
 %([xshift=-5pt, yshift=5pt]locx-v1.north east) -- 
%% ([xshift=-5pt, yshift=-5pt]locx-v1.south east) --
%% ([xshift=-5pt, yshift=3pt]locy-v0.north east) -- 
 %([xshift=-5pt, yshift=-10pt]locy-v1.south east) node (tid2start) {};
 
%\path (tid2start) node[anchor=north, rectangle, fill=red!20, draw=red, font=\small, inner sep=1pt] {$\tid_2$};
 
 %%green view
%\draw[-, DarkGreen, very thick, rounded corners = 10pt]
 %([xshift=-16pt, yshift=8pt]locx-v1.north east) node (tid3start) {}-- 
 %([xshift=-16pt, yshift=-5pt]locx-v1.south east) --
 %([xshift=-16pt, yshift=5pt]locy-v0.north east) -- 
 %([xshift=-16pt, yshift=-5pt]locy-v0.south east);
 
 %\path (tid3start) node[anchor=south, rectangle, fill=DarkGreen!20, draw=DarkGreen, font=\small, inner sep=1pt] {$\tid_1$};

%\end{pgfonlayer}
%\end{tikzpicture}
%&
%\begin{tikzpicture}[font=\large]

%\begin{pgfonlayer}{foreground}
%%Uncomment line below for help lines
%%\draw[help lines] grid(5,4);

%%Location x
%\node(locx) at (1,3) {$[\loc_x] \mapsto$};

%\matrix(locxcells) [version list, text width=7mm, anchor=west]
   %at ([xshift=10pt]locx.east) {
 %{a} & $\tsid_0$ & {a} & $\tsid_1$\\
  %{a} & $\{\tsid_3\}$ & {a} & $\{\tsid_2\}$ \\
%};
%\node[version node, fit=(locxcells-1-1) (locxcells-2-1), fill=white, inner sep= 0cm, font=\Large] (locx-v0) {$0$};
%\node[version node, fit=(locxcells-1-3) (locxcells-2-3), fill=white, inner sep=0cm, font=\Large] (locx-v1) {$1$};
%%Location y
%\path (locx.south) + (0,-1.5) node (locy) {$[\loc_y] \mapsto$};
%\matrix(locycells) [version list, text width=7mm, anchor=west]
   %at ([xshift=10pt]locy.east) {
 %{a} & $\tsid_0$ & {a} & $\tsid_2$ \\
   %{a} & $\emptyset$ & {a} & $\{\tsid_3\}$\\
%};
%\node[version node, fit=(locycells-1-1) (locycells-2-1), fill=white, inner sep= 0cm, font=\Large] (locy-v0) {$0$};
%\node[version node, fit=(locycells-1-3) (locycells-2-3), fill=white, inner sep=0cm, font=\Large] (locy-v1) {$1$};
%% \draw[-, red, very thick, rounded corners] ([xshift=-5pt, yshift=5pt]locx-v1.north east) |- 
%%  ($([xshift=-5pt,yshift=-5pt]locx-v1.south east)!.5!([xshift=-5pt, yshift=5pt]locy-v0.north east)$) -| ([xshift=-5pt, yshift=5pt]locy-v0.south east);

%%blue view - I should  check whether I can use pgfkeys to just declare the list of locations, and then add the view automatically.
%\draw[-, blue, very thick, rounded corners=10pt]
 %([xshift=-2pt, yshift=20pt]locx-v0.north east) node (tid1start) {} -- 
 %([xshift=-2pt, yshift=-5pt]locx-v0.south east) --
 %([xshift=-2pt, yshift=5pt]locy-v1.north east) -- 
 %([xshift=-2pt, yshift=-5pt]locy-v1.south east);
 
 %\path (tid1start) node[anchor=south, rectangle, fill=blue!20, draw=blue, font=\small, inner sep=1pt] {$\tid_3$};

%%red view
%\draw[-, red, very thick, rounded corners = 10pt]
 %([xshift=-5pt, yshift=5pt]locx-v1.north east) -- 
%% ([xshift=-5pt, yshift=-5pt]locx-v1.south east) --
%% ([xshift=-5pt, yshift=3pt]locy-v0.north east) -- 
 %([xshift=-5pt, yshift=-10pt]locy-v1.south east) node (tid2start) {};
 
%\path (tid2start) node[anchor=north, rectangle, fill=red!20, draw=red, font=\small, inner sep=1pt] {$\tid_2$};
 
 %%green view
%\draw[-, DarkGreen, very thick, rounded corners = 10pt]
 %([xshift=-16pt, yshift=8pt]locx-v1.north east) node (tid3start) {}-- 
 %([xshift=-16pt, yshift=-5pt]locx-v1.south east) --
 %([xshift=-16pt, yshift=5pt]locy-v0.north east) -- 
 %([xshift=-16pt, yshift=-5pt]locy-v0.south east);
 
 %\path (tid3start) node[anchor=south, rectangle, fill=DarkGreen!20, draw=DarkGreen, font=\small, inner sep=1pt] {$\tid_1$};

%\end{pgfonlayer}
%\end{tikzpicture}
%\\
%{\small (e)} & {\small (f)} \\
%\hline
%\end{tabular}
%\caption{History heaps obtained in a execution of $\prog_2$.}
%\label{fig:cc.exec}
%\end{figure}


\sx{Should we give intuition about causal dependencies here ?}
The next consistency model that we are interested is \emph{transactional causal consistency} \cite{cops}. 
Intuitively, it ensures that versions read by transactions are closed with respect to \emph{causal dependencies}. 
Consider for example the following program: 
\[
    \prog_2 \equiv \begin{session}
        \begin{array}{@{}c || c || c@{}}
            \txid_{1} : 
            \begin{transaction}
                \pmutate{\vx}{1};\\
            \end{transaction} &
            \txid_{2} : 
            \begin{transaction} 
                \pderef{\pvar{a}}{\vx};\\
                \pmutate{\vy}{\pvar{a}};\\
            \end{transaction} &
            \txid_{3} :
             \begin{transaction}
               	   \pderef{\pvar{a}}{\vx};\\
               	   \pderef{\pvar{b}}{\vy};\\
               	   \pifs{\pvar{a}=0 \wedge \pvar{b}=1}\\
               			\quad \passign{\retvar}{\Large \frownie{}}
               		\pife
             \end{transaction}
        \end{array}
    \end{session}
 \]
For the sake of simplicity, we label the code of the three transactions above as $\txid_{1}, \txid_2, \txid_3$ from left to right.
It is easy to see that, if no constraints or even under read atomic, the third transaction $\txid_{3}$ can return ${\Large \frownie{}}$. 
%The same is true even if the consistency model specification $\mathsf{RA}$ is assumed. 
Informally, the return of value ${\Large \frownie{}}$ by $\txid_3$ can be obtained from the execution outlined below. 
\sx{Did not edit, change later}
\begin{itemize}
\item The initial configuration of this execution is depicted in Figure \ref{fig:cc.exec}(a).
\item $\ptrans{\trans_1}$ executes with the initial view, which points to the 
initial (and only) version for each location; after this transaction is 
executed, a new version $\langle 1, T_1, \emptyset \rangle$ is appended 
at the end of $\hh(\loc_{x})$. The resulting history heap is depicted in Figure \ref{fig:cc.exec}(b).
\item next, $\thid_2$ updates its view as to see the version of $\loc_x$ installed by $\thid_1$, after 
which it proceeds to execute $\ptrans{\trans_2}$. This results in a new version with value $1$ 
to be installed for $\loc_y$. The configurations before, and after the execution of $\ptrans{\trans_2}$, 
are depicted in figures \ref{fig:cc.exec}(c) and \ref{fig:cc.exec}(d), respectively.
\item Finally, thread $\thid_3$ updates its view to observe the update of location $[\loc_y]$, but not the update of 
location $[\loc_y]$, before executing transaction $\ptrans{\trans_3}$. The execution of $\ptrans{\trans_3}$ will 
return the value ${\Large \frownie{}}$. The history heaps immediately before and after 
the execution of $\thid_3$, are depicted in figures \ref{fig:cc.exec}(e) and \ref{fig:cc.exec}(f), respectively. 
\end{itemize}

\sx{Change above}

In the last step, the thread to the right commits the transaction $\txid_3$ in a state where its initial view observes the second version of the address $\vy$, which is created by \( \txid_{2} \).
However, because the transaction \( \txid_{2} \) read the second version of \( \vx \) and create the second version of \( \vy \), this means the latter depends on the former.
Yet the transaction \( \txid_{3} \) does not read from the second version of \( \vx \), which is disallowed by \emph{transactional causal consistency}.
Summarising, under transactional causal consistency if a transaction sees updates for an address \( \addr \), it should also observes those addresses that \( \addr \) depends on.

\ac{
but not the update to address $\vx$.
However, the update of $[\vy]$ committed by $\txid_{2}$, consisted in copying the value of the update 
of $[\loc_y]$: that is, the update of $[\loc_y]$ \textbf{depends} from the update of $[\loc_x]$. 
Summarising, the execution of transaction $\ptrans{\trans_3}$ resulted in a violation of 
causality: the update of $[\loc_y]$ is observed, but not the update of $[\loc_x]$ on which 
it depends.
}

\sx{in RA up-to-date view, and here consistent view, not sure are good words, let re-think later}
To formally specify transactional causal consistency, we inductively define the set of views that are consistent with respect to a history heap $\hh$. 
\sx{What do you mean by the word??}
The definition below models the fact that, if we start from a causally consistent view, and we wish to update the view for some location $\txid_2$, 

\begin{defn}
Given two versions $\ver_{1} = (\val_1, \txid_1, \txidset_1)$, $\ver_2 = (\val_2, \txid_2, \txidset_2)$, $\ver_{1}$ \emph{directly depends on} $\ver_2$, written $\pred{ddep}{\ver_{1}, \ver_{2}}$, if $\txid_1 \in \txidset_2$. 
\ac{Note to self: the notion of directly depends here has little to do with dependencies 
between transactions. $\nu_1 \xrightarrow{\mathsf{ddep}} \nu_2$ means that 
some transaction $\tsid$ touches both versions. However, it reads $\nu_2$ and 
writes $\nu_1$.}
Given $\hh \in \HisHeaps$, the set of views that are \emph{causally consistent} with respect to $\hh$, $\func{CCViews}{\hh}$, is defined as the smallest set such that: 
\begin{enumerate} 
\item the initial view \( \vi_0\)  is in the set, \ie $\vi_0 \in \func{CCViews}{\hh}$ where \( \fora{\addr \in \dom(\hh)} \vi_{0}(\addr) = 1 \),
\item assume any view $\vi \in \pred{CCViews}{\hh}$ and a new view \( \vi' \) by observing one more version for an address $\vi' = \vi\rmto{\addr}{\vi(\addr) + 1}$, where \( \vi'(\addr) \leq \left| \hh(\addr) \right| \).
If some versions directly depend on the version corresponding to \( \vi'(\addr)\) and those versions are aware by \( \vi'\), the new view is included in \( \func{CCViews}{\hh}\),
%for some $i: V([\loc_x]) < i \leq \lvert \hh \rvert -1$
%for some $[\loc_x]$ such that $V([\loc_x]) < \lvert \hh([\loc_x]) \rvert - 1$.
%Suppose that 
\[
\begin{array}{@{}l}
\fora{\vi,\vi'} \exsts{\addr}
\vi \in \func{CCViews}{\hh} 
\land \addr \in \dom(\vi)
\land \vi' = \vi\rmto{\addr}{\vi(\addr) + 1}
\land \vi'(\addr) \leq \left| \hh(\addr) \right|  \\
\quad {} \land 
\begin{B}
\fora{\addr', i}  
1 \leq i \leq \left| \hh(\addr') \right|
\land \pred{ddep}{\hh(\addr')(i), \hh(\addr)(\vi'(\addr))}
\implies i \leq \vi(\addr')
\end{B} \\
\qquad {} \implies \vi' \in \func{CCViews}{\hh}
\end{array}
\]
%for any location $[\loc_y]$ and 
%index $j = 0, \cdots, \lvert \hh([\loc_y]) \rvert -1$ such that $\hh([\loc_{x})(V'([\loc_x]))$ 
%directly depends on $\hh([\loc_y])(j)$, then $j \leq V([\loc_y])$. Then 
%$V' \in \mathsf{CCViews}(\hh)$.
% and suppose that $\hh([\loc_{x}])(V'([\loc_x])) = 
%(\_, \tsid, \_ )$ for some $\tsid$. If for all locations $\loc_{y}$ and 
%indexes $j$ such that $\hh([\loc_y])(j) = (\_, \_, \_ \cup \{\tsid\})$, 
%then $j \leq V'([\loc_y])$, then $V' \in \mathsf{CCViews}(\hh)$.
\end{enumerate}
\end{defn}

