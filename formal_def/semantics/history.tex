\subsection{Program Semantics}
\begin{defn}[History Heaps]
\label{def:his_heap}
Assuming a set of \emph{transactions identifiers} \( \TxID \defeq \Set{\txid, \dots}\), a \emph{history heap}, \( \hh \in \HisHeaps \), is a partial function from addresses to lists of values.
Each element in the list is a value associated with a transaction (identifier) who writes the value and a set of transactions (identifiers) who read it.
\[
\begin{rclarray}
    \HisHeaps & \defeq & \Addr \parfinfun (\Val \times \TxID \times \powerset{\TxID})^{*}
\end{rclarray}
\]
Let \( \hh(\addr)(i)\) denotes the \emph{i-th} (starting from 1) tuple of the list associated with address \( \addr \), and \( \hhV(\addr)(i) \), \( \hhW(\addr)(i) \) and \( \hhR(\addr)(i) \) denote the first, second and third element of the tuple.
The composition of two history heaps is \( \hh \composeHH \hh' \defeq \hh \uplus \hh' \) and the unit element is \( \unitHH \defeq \emptyset \). 
\end{defn}



\begin{defn}[Cuts of history heaps]
\label{def:cuts}
\emph{A cut of of a history heap} is a partial function from addresses to indexes,
\[
\begin{rclarray}
    \cu \in \Cuts & \defeq & \Addr \parfinfun \Nat
\end{rclarray}
\]                                                                     
The composition is \( \cu \composeCU \cu' \defeq \cu \uplus \cu'\) and the unit is \( \unitCU \defeq \emptyset\).
The order between two cuts that have the same domain is the order of all the indexes, 
\[
\begin{rclarray}
    \cu \orderCU \cu' & \defiff & \dom(\cu) = \dom(\cu') \land \for{\addr} \cu(\addr) \leq \cu'(\addr) \\
\end{rclarray}
\]
Assuming a countably infinite set of thread identifiers \( \setthid \subseteq \ThreadID \defeq \Set{\thid, \dots} \), \emph{a cuts' runtime} is a function from thread identifiers to cuts, \ie \( \thcu \in \ThCuts \defeq \ThreadID \parfun \Cuts\).
\end{defn}

\begin{defn}[Collapse of history heaps]
A cut \( \cu \) can collapse a history heap \( \hh \) to a (normal) heap \( h \) through the function \(\funcn{clps} : \HisHeaps \times \Cuts \parfun \Heaps\),
\[
\begin{rclarray}
    \clpsHH{\hh, \cu} & \defeq & 
    \begin{cases}
        \lambda \addr \ldotp \hhV(\addr)(\cu(\addr)) & \text{if } \dom(\hh) = \dom(\cu) \land \for{\addr'} \cu(\addr') \leq \left|\hh(\addr')\right| \\
        \text{undefined} & \text{otherwise}
    \end{cases}
\end{rclarray}
\]
The function is overloaded with one parameter version that collapses history heaps by picking the last values,
\[
\begin{rclarray}
    \clpsHH{\hh} & \defeq & \lambda \addr \ldotp \hhV(\addr)(\left|\hh(\addr)\right|) 
\end{rclarray}
\]
\end{defn}

\begin{defn}[Consistency Models]
\label{def:consistency-models}
A \emph{Consistency model} \( \como \) is a \emph{reflexive and transitive relation} over pairs of history heaps and cut environments, written \( ((\hh,\thcu),(\hh',\thcu')) \in \como  \), that the transition from the state \( (\hh,\thcu)\) to the state \( (\hh',\thcu') \) satisfies certain constraints.
Each tuple in the set should satisfy a well-formed condition that the cut environment should contain no more address than the history heap,
\[
    \bigcup\limits_{\thid \in \dom(\thcu)} \dom(\thcu(\thid)) \subseteq \dom(\hh)
\]
The constraints should be local to the owned history heaps, so that the combination of two disjointed parts also satisfies the constraints,
\[
    \begin{array}{@{}l}
    \for{\hh,\hh',\hh'',\hh''', \thcu, \thcu', \thcu'', \thcu'''} \\
        \quad ((\hh, \thcu),(\hh',\thcu')) \in \como 
        \land ((\hh'', \thcu''),(\hh''',\thcu''')) \in\como 
        \land ( \hh \composeHH \hh'' )\isdef
        \land ( \hh' \composeHH \hh''' )\isdef \\
        \qquad \implies  ((\hh \composeHH \hh'', \thcu \uplus \thcu''),(\hh' \composeHH \hh''',\thcu' \uplus \thcu''')) \in \como
    \end{array}
\]
\end{defn}

\sx{Need Andrea for more clarify about defining consistency model, especially what kind of style we should use here, for example we can constraint history heap or/and the cuts. }

There are two ways to specify the consistency models, one by converting the history heap to dependant graphs while another by checking if the cuts satisfy certain property.
We introduce \( \funcn{graph} : \HisHeaps \to \sort{DGraph} \) function for converting history heaps to dependant graphs.
\[
\begin{rclarray}
    \DGraph & \defeq & \Setcon{(\txidset, \ww, \wrr, \rw)}{ \txidset \subseteq \TxID \land \ww, \wrr, \rw \subseteq (\txidset \times \txidset)} \\
    \func{graph}{\hh} & \defeq & \left(
    \begin{array}{@{}l}
        \Setcon{\txid}{\exsts{\addr, i} \txid = \hhW(\addr)(i) \lor \txid \in \hhR(\addr)(i) }, \\
        \Setcon{(\txid, \txid')}{\exsts{\addr, i, i'} \txid = \hhW(\addr)(i) \lor \txid' = \hhW(\addr)(i') \land i < i' }, \\
        \Setcon{(\txid, \txid')}{\exsts{\addr, i} \txid = \hhW(\addr)(i) \lor \txid' \in \hhR(\addr)(i) }, \\
        \Setcon{(\txid, \txid')}{\exsts{\addr, i, i'} \txid \in \hhR(\addr)(i) \lor \txid' = \hhW(\addr)(i') \land i < i' }, \\
    \end{array}
    \right)
\end{rclarray}
\]

\begin{example}[Serialisibility]
\end{example}

\begin{example}[Snapshot]
Any cycle in the corresponding dependant graph of the history heap after the transition must have adjacent anti-dependent edges (\( \rw \) edges).
\[
    \begin{rclarray}
    \SI & \defeq & \Setcon{((\hh, \thcu),(\hh', \thcu'))}{\exsts{\ww, \rw, \wrr} (\stub, \ww, \wrr, \rw)  = \func{graph}{\hh'} \land (\wrr \cup \ww) ; \rw? \text{ is acyclic}}
    \end{rclarray}
\]
\end{example}

\begin{defn}[Thread transition labels]
\label{def:label}
Given the set of thread identifiers \(\ThreadID\) (\defref{def:abs-exec}), the set of \emph{thread transition labels}, $\lb \in \Translabel$, is defined by the following grammar, where $\prog$ denotes a program (\defref{def:language}), the $\txid$ demotes a transaction identifier and $\thstk$ denotes a thread stack (\defref{def:stacks}),
\[
    \begin{rclarray}
	\iota \in \Translabel & ::= & \lbID \mid \lbC{\txid} \mid \lbF{\thid,\prog} \mid \lbJ{\thid,\thstk}
    \end{rclarray}
\]
\end{defn}

\begin{defn}[Thread semantics]
\label{def:thread_semantics}
Given the thread identifiers \(\thid \in \ThreadID\), the set of \emph{intermediate programs}, \(\iprog \in \IntermediatePrograms\), is defined by the following grammar,
\[
    \iprog ::= \prog \mid \iprog \pseq \pwait{\thid}
\]
Given the set of consistency model \( \ConsisModels \) (\defref{def:consistency-models}), thread stacks \( \ThdStacks \) (\defref{def:stacks}), history heaps \( \HisHeaps \) (\defref{def:his_heap}) and cuts' runtime (\defref{def:cuts}), the \emph{per-thread operational semantics} of programs,
\[
\begin{rclarray}
	\toT{} & : &
    \begin{array}[t]{@{}l@{}}
    \Como \times \ThreadID 
    \times \\
	\quad \left( ( \ThdStacks \times \HisHeaps \times \ThCuts ) \times \IntermediatePrograms \right) 
	\times  \Translabel \times
	\left( ( \ThdStacks \times \HisHeaps \times \ThCuts ) \times \IntermediatePrograms \right) 
    \end{array}
\end{rclarray}
\]
is defined in \figref{fig:thread_semantics}.
\end{defn}

\begin{figure}[!t]
%
\hrule
%
\[
    \infer[\rl{PCommit}]{%
        \como, \thid \vdash ( \thstk, \hh, \thcu ), \ptrans{\trans} \ \toT{\lbC{\txid}} \ ( \thstk', \hh', \thcu' ) , \pskip
    }{%
        \begin{array}{c}
            \txid \in \func{fresh}{\hh}  
            \quad \h = \clpsHH{\hh,\thcu(\thid)}
            \quad \txstk_{0} = \emptyset \\
            \thstk \vdash (\txstk_{0}, \h, \unitO), \trans \ \toL^{*} \  (\txstk, \h', \opset) , \pskip \\
            \thstk' = \thstk\rmto{\ret}{\txstk(\ret)} 
            \quad \hh' = \func{commit}{\hh, \thcu(\thid), \txid, \opset}  \\
            \cu = \func{update}{\hh', \thcu(\thid), \opset} 
            \quad \thcu = \thcu\rmto{\thid}{\cu} 
            \quad ((\hh,\thcu),(\hh',\thcu')) \in \como
        \end{array}
    }
\]

\[
    \infer[\rl{PAssign}]{%
        \como, \thid \vdash ( \thstk, \hh, \thcu ) , \passign{\thvar}{\expr} \ \toT{\lbID} \  ( \thstk\rmto{\thvar}{\val}, \hh, \thcu ) , \pskip
    }{
        \val = \evalE[\thstk]{\expr}
        && \thvar \in \ThdVars
    }
\]

\[
    \infer[\rl{PAssume}]{%
        \como, \thid \vdash ( \thstk, \hh, \thcu ) , \passume{\expr} \ \toT{\lbID} \  ( \thstk, \hh, \thcu ) , \pskip
    }{%
        \evalE[\thstk]{\expr} = 0
    }
\]

\[
    \infer[\rl{PCutShift}]{
        \como, \thid \vdash ( \thstk, \hh, \thcu ) , \prog \ \toT{\lbID} \  ( \thstk, \hh, \thcu' ) , \prog
    }{
        \thcu \leq \thcu'
    }
\]

\[
    \infer[\rl{PChoice}]{%
        \como, \thid \vdash ( \thstk, \hh, \thcu ) , \prog_{1} \pchoice \prog_{2} \ \toT{\lbID} \  ( \thstk, \hh, \thcu ) , \prog'
    }{
        \prog' \in \Set{\prog_{1}, \prog_{2}}
    }
\]

\[
    \infer[\rl{PLoop}]{%
        \como, \thid \vdash ( \thstk, \hh, \thcu ) , \prog\prepeat \ \toT{\lbID} \  ( \thstk, \hh, \thcu ) , \pskip \pchoice (\prog \pseq \prog\prepeat)
    }{}
\]

\[
    \infer[\rl{PSeqSkip}]{%
        \como, \thid \vdash ( \thstk, \hh, \thcu ) , \pskip \pseq \iprog \ \toT{\lbID} \  ( \thstk, \hh, \thcu ) , \iprog
    }{}
\]

\[
    \infer[\rl{PSeq}]{%
        \como, \thid \vdash ( \thstk, \hh, \thcu ) , \iprog_{1} \pseq \iprog_{2} \ \toT{\lb} \ ( \thstk', \hh, \thcu' ) , {\iprog_{1}}' \pseq \iprog_{2}
    }{%
        \como, \thid \vdash ( \thstk, \hh, \thcu ) , \iprog_{1} \ \toT{\lb} \  ( \thstk', \hh, \thcu' ) , {\iprog_{1}}' 
    }
\]

\[
    \infer[\rl{PPar}]{%
        \como, \thid \vdash ( \thstk, \hh, \thcu ) , \prog_{1} \ppar \prog_{2} \ \toT{\lbF{\thid', \prog_{2}}} \  \left( \thstk, \hh, \thcu \uplus \Set{\thid' \mapsto \thcu(\thid)} \right) , \prog_{1} \pseq \pwait{\thid'}
    }{}
\]

\[
    \infer[\rl{PWait}]{%
        \como, \thid \vdash \left( \thstk_{1} \uplus \thstk_{f}, \hh, \thcu \uplus \Set{\thid' \mapsto \thcu(\thid)} \right) , \pwait{\thid'} \ \toT{\lbJ{\thid', \thstk_{2} \uplus \thstk_{f}}} \  ( \thstk_{1} \uplus \thstk_{2} \uplus \thstk_{f}, \hh, \thcu ) , \pskip 
    }{}
\]

\sx{The stack could be polluted by the child thread, but they must agree?}
 
where,
\[
\begin{rclarray}                                 
    \func{commit}{\hh, \cu, \txid, \opset} & \defeq & 
    \begin{cases}
        \hh & \text{if } \opset = \unitO \\
        \func{commit}{\hh', \cu, \txid, \opset'} & \text{if } \opset = \opset' \addO (\otR, \addr, \val) \\
        \func{commit}{\hh'', \cu, \txid, \opset''} & \text{if } \opset = \opset'' \addO (\otW, \addr, \val) \\
    \end{cases} \\
    \hh' & \equiv & \hh\rmto{\addr}{\hh(\addr)\rmto{\cu(\addr)}{\left(\hhV(\addr)(\cu(\addr)),\hhW(\addr)(\cu(\addr)),\hhR(\addr)(\cu(\addr)) \uplus \Set{\txid} \right)}} \\
    \hh'' & \equiv & \hh\rmto{\addr}{\hh(\addr) +\!\!+ [(\val, \txid, \emptyset)] } \\
%
%
    \func{update}{\hh, \cu, \opset} & \defeq &
    \begin{cases}
        \cu & \text{if } \opset = \unitO \\
        \func{update}{\hh, \cu, \opset'} & \text{if } \opset = \opset' \addO (\otR, \addr, \val) \\
        \func{update}{\hh, \cu\rmto{\addr}{\left| \hh(\addr) \right|}, \opset''} & \text{if } \opset = \opset'' \addO (\otW, \addr, \val) \\
    \end{cases} \\
%
%              
	\func{fresh}{\aexecrun}  & \defeq & \Setcon{ \txid }{ \txid \in \TxID \land \for{\addr, i} \txid \neq \hhW(\addr)(i) \land \txid \notin \hhR(\addr)(i) } \\
\end{rclarray}
\]
\hrule
\caption{Per-thread operational semantics}
\label{fig:thread_semantics}
\end{figure}

\sx{The full version should support dynamic thread, but we might only need fixed thread so that the thread pool semantics will be pick one thread and run one step}

\begin{defn}[Thread pools]
\label{def:thread_pools}
Given the set of thread stacks $\ThdStacks$ (\defref{def:stacks}) and intermediate programs $\IntermediatePrograms$ (\defref{def:thread_semantics}), a \emph{thread pool} is a a finite partial function from thread identifiers to triples of thread stacks and intermediate programs, \(\thpl \in \TPool \eqdef \ThreadID \parfinfun \ThdStacks \times \IntermediatePrograms\).
\end{defn}
 
\begin{defn}[Thread pool semantics] 
\label{def:thread_pool_semantics}
Given the set of consistent models \( \ConsisModels \) (\defref{def:consistency-models}), history heaps \(\HisHeaps\) (\defref{def:his_heap}), cuts' runtime \(\ThCuts\) (\defref{def:cuts}), transition labels \( \Translabel \) (\defref{def:label}) and thread pools  \( \TPool \) (\defref{def:thread_pools}), the \emph{thread pool semantics}, 
\[
	\toG{} : \Como \times (\HisHeaps \times \ThCuts \times \TPool) \times \Translabel \times (\HisHeaps \times \ThCuts \times \TPool) 
\]
is defined in \figref{fig:thread_pool_semantics}.
\end{defn}
 
The thread pool operational semantics is given in \figref{fig:thread_pool_semantics}, where an arbitrary thread in the pool \(\thpl\) is picked to run for one step.
If the next execution step is a thread fork, then a new thread \(\thid'\) is allocated in the pool to be executed with its thread stack copied from its parent (forking) thread.
Conversely, when the next execution step is the joining of thread \(\thid'\), then \(\thid'\) is removed from the thread pool and the stack from the child thread merges into the parent thread.

\begin{figure}
\hrule\vspace{5pt}
%
\[
    \infer[\rl{PSingle}]{%
        \como \vdash ( \hh, \thcu, \thpl \uplus \Set{ \thid \mapsto (\thstk, \iprog) } ) \ \toG{\lb} \  ( \hh', \thcu', \thpl \uplus \Set{ \thid \mapsto (\thstk', {\iprog}') } ) 
    }{%
        \como, \thid \vdash ( \thstk, \hh, \thcu ) , \iprog \ \toT{\lb} \  ( \thstk', \hh', \thcu' ) , {\iprog}' 
        \quad \lb \in \Set{ \lbID, \lbC{\stub} }
    }
\]

\[
    \infer[\rl{PFork}]{%
        \como \vdash ( \hh, \thcu, \thpl \uplus \Set{ \thid \mapsto (\thstk, \iprog) } ) \ \toG{\lbF{\thid', \prog}} \  ( \hh', \thcu', \thpl \uplus \Set{ \thid \mapsto (\thstk, {\iprog}'), \thid' \mapsto (\thstk, \prog) } )
    }{%
        \como, \thid \vdash ( \thstk, \hh, \thcu ) , \iprog \ \toT{\lbF{\thid', \prog}} \  ( \thstk, \hh', \thcu' ) , {\iprog}' 
    }
\]

\[
    \infer[\rl{PJoin}]{%
        \como \vdash ( \hh, \thcu, \thpl \uplus \Set{ \thid \mapsto (\thstk, \iprog), \thid' \mapsto (\thstk', \pskip) } )  \ \toG{\lbJ{\thid',\thstk''}} \ ( \hh', \thcu', \thpl \uplus \Set{ \thid \mapsto (\thstk'', {\iprog}')} )
    }{%
        \como, \thid \vdash ( \thstk, \hh, \thcu ) , \iprog \ \toT{\lbJ{\thid',\thstk'}} \  ( \thstk'', \hh', \thcu' ) , {\iprog}' 
    }
\]
%
\hrule\vspace{5pt}
\caption{Thread pool semantics}
\label{fig:thread_pool_semantics}
\end{figure}
