\subsection{Example of Consistency models}
\ac{This Section is going to become heavy in pictures, which should be organised into figures.}
In this Section we present different consistency models specifications. 
For each of them, we give: 
\begin{itemize}
\item the intuition of the commit tests for different consistency models, and the formal definitions with respect to the \(\Como\) (\defref{def:consistency-models}).
%describing the consistency guarantees that schedules of the database should have in plain English, 
%\item a formal consistency model specification, in the style described in \S \ref{sec:semantics.programs},
\item examples of litmus tests that, when executed, give rise to the anomalies that are forbidden from the consistency model, 
\item an explanation of why the consistency model forbids the litmus tests to exhibit the anomaly that should be forbidden. 
\end{itemize}
Later, we will show how to compare our consistency models specifications with those already existing in the 
literature.
\ac{There is still a long-way to go before proving correspondence with dependency graph specifications, 
but this should be mentioned here.}


\subsection{Read Atomic} 
Read atomic (RA) \cite{ramp} is the weakest consistency model among those that enjoy \emph{atomic visibility} \cite{framework-concur}. 
It requires of a transaction to read an atomic snapshot of the database and never observe the partial effects of other transactions.
This is also known as the \emph{all-or-nothing} property: a transaction observes either none or all the updates performed by other transactions. 

\sx{example RAMP, not sure the meaning}
One litmus test that should \textbf{not} be failed in RAMP consists of the program $\prog_1$ from \S \ref{sec:semantics.example}, which we already observed to produce a violation of atomic visibility if no constraints on the consistency model are placed.
\ac{not be failed. Double negation. Bad English.}

\sx{Rephrase}
Intuitively, in such a program, it violate the atomic visibility because we allowed to execute the transaction \( \trans_1^2\) in the thread-local configuration of $\mathcal{C}'$ relative to $\thid_2$, which is obtained by removing all the information about $\thid_1$ (view and stack) in Figure \ref{fig:opsem.example}(c).
\[
\trans_1^2 = \begin{array}{c} 
            \begin{transaction}
            		\pderef{\pvar{a}}{\vx};\\
            		\pderef{\pvar{b}}{\vy};\\
            		\pifs{\pvar{a}=1 \wedge \pvar{b}=0}\\
            			\quad \passign{\retvar}{\Large \frownie{}} \\
                    \pife
             \end{transaction}
     \end{array}
\]

\ac{
To avoid transactions to only observe the partial effects of other transactions, we 
must ensure that transactional code cannot be executed by a thread whose 
views is up-to-date with respect to some transaction $\tsid$ for some location $[\loc_x]$, 
but not for some other location $[\loc_y]$. This leads to the following definition.
}
To avoid a transaction to observe the partial effects of other transactions, we need to ensure that transactional code cannot be executed by a thread whose views is partially up-to-date with respect to some transactions. This leads to the following definition.
\begin{defn}[Read atomic]
\label{def:readatomic}
%Let $\hh$ be a history heap,$V$ be a view, $[\loc_x]$ be 
%a location and $\nu$ be a version. We say that $V$ $[\loc_x]$-\emph{sees} version 
%$\nu$ if there exists an index $i \leq V([\loc_x])$ such that $V(i) = \nu$. 
%We say that $V$ $[\loc_x]$-\emph{sees} transaction $\tsid$ if 
%$V$ $[\loc_x]$-sees a version $\nu = (\_, \tsid, \_)$. 
Given a view $\vi \in \Views$, a history heap $\hh \in \HisHeaps$, and a transaction identifier $\txid \in \TxID$, the view \emph{sees} the transaction in the history heap, written $\pred{visible}{\txid, \vi, \hh}$, if the view sees all the writes from the transaction,
%We say that $V$ \emph{sees} transaction $\tsid$ in $\hh$, written 
%$\mathsf{Visible}(\tsid, V, \hh)$, iff 
\sx{\( \exsts{i} \) might be enough}
\[
\begin{rclarray}
\pred{visible}{\txid, \vi, \hh} & \eqdef & \fora{\addr, i} \hh(\addr)(i) = (\stub, \txid, \stub) \implies i \leq \vi(\addr).
\end{rclarray}
\]
\ac{In English: the view is up-to-date with respect to all the updates 
performed by transaction $\tsid$.}

Then given a history heap \( \hh \), the view $V$ is \emph{consistent} with respect to \emph{atomic visibility}, written $\pred{atomic}{\vi, \hh}$, if the view $V$ is up-to-date with some of the updates performed by $\txid$, then it should be up-to-date with all the updates performed by $\txid$,
\[
\begin{rclarray}
\pred{atomic}{\vi ,\hh} & \eqdef & \fora{\txid } \exsts{\addr, i} i \leq \vi(\addr) \land \hh(\addr)(i) = (\stub, \txid, \stub) \implies \pred{vusible}{\txid, \vi, \hh}
\end{rclarray}
\]
\ac{In English: if the view $V$ is up-to-date with some of the updates performed 
by $\tsid$, then it must be up-to-date with all the updates performed by $\tsid$. 
This is the all-or-nothing property.}
%for all location 
%$[\loc_x]$, if there exists an index $i = 0,\cdots, \lvert \hh([\loc_x]) \rvert - 1$, 
%such that $\hh([\loc_x])(i) = (\_, \tsid, \_)$, then $i \leq V([\loc_x])$.

The consistency model specification $\mathsf{RA}$ is defined as the smallest set such that  
\sx{what is the meaning of smallest?}
\[
\pred{atomic}{\hh, \vi} \implies (\hh, \vi) \csat[\mathsf{RA}] \stub: \stub
\]
\ac{In English: Before executing a transaction, either you observe all or none the 
updates of all other transactions. We may strengthen the consistency model and 
require that the same property must be satisfied at the end as well, though 
this is not strictly necessary. In this case the check becomes: 
\[
\mathsf{atomic}(\hh, V) \wedge \mathsf{atomic}(\hh, V') \wedge \mathsf{UpdateView}(\hh, V, \mathcal{O}) 
\sqsubseteq V' \implies (\hh, V) \triangleright_{\mathsf{RA}} \mathcal{O}: V'.
\]
}
%written $\mathsf{up-to-date}(\hh, V, \tsid, [\loc_x])$, 
%if either 
%
%\begin{itemize}
%\item for all indexes $i = 0,\cdots, \lvert \hh([\loc_x]) - 1 \rvert$, 
%$\WS(\hh([\loc_x])(i)) \neq \tsid)$, or 
%\item if $\WS(\hh([\loc_x])(i)) = \tsid$ for some $i = 0,\cdots, \lvert \hh([\loc_n]) -1 \rvert$, 
%then $i \leq V([\loc_n])$.
%\end{itemize}
\end{defn}

\sx{Not sure how to link the explanation from Andrea's document, sort out later }
Suppose that we execute the program $\prog_1$ under the consistency model specification $\mathsf{RA}$.
We can proceed as in Section \ref{sec:semantics.example} to infer the transition 
$\langle \mathcal{C_0}, \prog_1 \rangle \xrightarrow{\mathsf{RA}} \langle \mathcal{C}_1, \prog_1' \rangle$, 
where we recall that $\mathcal{C}_0$, $\mathcal{C}_1$ are depicted in Figure \ref{fig:opsem.exampe}(a), 
\ref{fig:opsem.example}(b), respectively. 

It is immediate to observe that the only way in which the execution of transaction $\ptrans{\trans}$ from $\thid_2$ in $\prog_1'$ can return value ${\Large \frownie}$ is the following: 
\begin{itemize}
\item first, push the view $V$ of thread $\txid_2$ in the configuration 
$\mathcal{C}_1$ of Figure \ref{fig:opsem.example}(b) to observe the update of location $[\vx]$, but not the update of 
$[\vy]$. This view is the one labelled with $\txid_2$ in Figure \ref{fig:opsem.example}(c), and we refer 
to it as $V'$;
\item then, execute the transaction $\ptrans{\trans}$ in $\thid_2$. 
\end{itemize}

\sx{transactional causal consistency ?}
\subsection{Causal Consistency}
%\begin{figure}
%\begin{tabular}{|c|c|}
%\hline
%\begin{tikzpicture}[font=\large]

%\begin{pgfonlayer}{foreground}
%%Uncomment line below for help lines
%%\draw[help lines] grid(5,4);

%%Location x
%\node(locx) at (1,3) {$[\loc_x] \mapsto$};

%\matrix(locxcells) [version list, text width=7mm, anchor=west]
   %at ([xshift=10pt]locx.east) {
 %{a} & $T_0$ \\
  %{a} & $\emptyset$ \\
%};
%\node[version node, fit=(locxcells-1-1) (locxcells-2-1), fill=white, inner sep= 0cm, font=\Large] (locx-v0) {$0$};

%%Location y
%\path (locx.south) + (0,-1.5) node (locy) {$[\loc_y] \mapsto$};
%\matrix(locycells) [version list, text width=7mm, anchor=west]
   %at ([xshift=10pt]locy.east) {
 %{a} & $T_0$ \\
  %{a} & $\emptyset$ \\
%};
%\node[version node, fit=(locycells-1-1) (locycells-2-1), fill=white, inner sep= 0cm, font=\Large] (locy-v0) {$0$};

%% \draw[-, red, very thick, rounded corners] ([xshift=-5pt, yshift=5pt]locx-v1.north east) |- 
%%  ($([xshift=-5pt,yshift=-5pt]locx-v1.south east)!.5!([xshift=-5pt, yshift=5pt]locy-v0.north east)$) -| ([xshift=-5pt, yshift=5pt]locy-v0.south east);

%%blue view - I should  check whether I can use pgfkeys to just declare the list of locations, and then add the view automatically.
%\draw[-, blue, very thick, rounded corners=10pt]
 %([xshift=-2pt, yshift=20pt]locx-v0.north east) node (tid1start) {} -- 
%% ([xshift=-2pt, yshift=-5pt]locx-v0.south east) --
%% ([xshift=-2pt, yshift=5pt]locy-v0.north east) -- 
 %([xshift=-2pt, yshift=-5pt]locy-v0.south east);
 
 %\path (tid1start) node[anchor=south, rectangle, fill=blue!20, draw=blue, font=\small, inner sep=1pt] {$\thid_3$};

%%red view
%\draw[-, red, very thick, rounded corners = 10pt]
 %([xshift=-5pt, yshift=5pt]locx-v0.north east) -- 
%% ([xshift=-8pt, yshift=-5pt]locx-v0.south east) --
%% ([xshift=-8pt, yshift=5pt]locy-v0.north east) -- 
 %([xshift=-5pt, yshift=-10pt]locy-v0.south east) node (tid2start) {};
 
%\path (tid2start) node[anchor=north, rectangle, fill=red!20, draw=red, font=\small, inner sep=1pt] {$\thid_2$};
 
 %%green view
%\draw[-, DarkGreen, very thick, rounded corners = 10pt]
 %([xshift=-16pt, yshift=8pt]locx-v0.north east) node (tid3start) {}-- 
%% ([xshift=-15pt, yshift=-5pt]locx-v0.south east) --
%% ([xshift=-15pt, yshift=5pt]locy-v0.north east) -- 
 %([xshift=-16pt, yshift=-5pt]locy-v0.south east);
 
 %\path (tid3start) node[anchor=south, rectangle, fill=DarkGreen!20, draw=DarkGreen, font=\small, inner sep=1pt] {$\thid_1$};

%\end{pgfonlayer}
%\end{tikzpicture}
%&
%\begin{tikzpicture}[font=\large]

%\begin{pgfonlayer}{foreground}
%%Uncomment line below for help lines
%%\draw[help lines] grid(5,4);

%%Location x
%\node(locx) at (1,3) {$[\loc_x] \mapsto$};

%\matrix(locxcells) [version list, text width=7mm, anchor=west]
   %at ([xshift=10pt]locx.east) {
 %{a} & $\tsid_0$ & {a} & $\tsid_1$\\
  %{a} & $\emptyset$ & {a} & $\emptyset$ \\
%};
%\node[version node, fit=(locxcells-1-1) (locxcells-2-1), fill=white, inner sep= 0cm, font=\Large] (locx-v0) {$0$};
%\node[version node, fit=(locxcells-1-3) (locxcells-2-3), fill=white, inner sep=0cm, font=\Large] (locx-v-1) {$1$};
%%Location y
%\path (locx.south) + (0,-1.5) node (locy) {$[\loc_y] \mapsto$};
%\matrix(locycells) [version list, text width=7mm, anchor=west]
   %at ([xshift=10pt]locy.east) {
 %{a} & $\tsid_0$ \\
   %{a} & $\emptyset$ \\
%};
%\node[version node, fit=(locycells-1-1) (locycells-2-1), fill=white, inner sep= 0cm, font=\Large] (locy-v0) {$0$};
%% \draw[-, red, very thick, rounded corners] ([xshift=-5pt, yshift=5pt]locx-v1.north east) |- 
%%  ($([xshift=-5pt,yshift=-5pt]locx-v1.south east)!.5!([xshift=-5pt, yshift=5pt]locy-v0.north east)$) -| ([xshift=-5pt, yshift=5pt]locy-v0.south east);

%%blue view - I should  check whether I can use pgfkeys to just declare the list of locations, and then add the view automatically.
%\draw[-, blue, very thick, rounded corners=10pt]
 %([xshift=-2pt, yshift=20pt]locx-v0.north east) node (tid1start) {} -- 
%% ([xshift=-2pt, yshift=-5pt]locx-v0.south east) --
%% ([xshift=-2pt, yshift=5pt]locy-v0.north east) -- 
 %([xshift=-2pt, yshift=-5pt]locy-v0.south east);
 
 %\path (tid1start) node[anchor=south, rectangle, fill=blue!20, draw=blue, font=\small, inner sep=1pt] {$\thid_3$};

%%red view
%\draw[-, red, very thick, rounded corners = 10pt]
 %([xshift=-5pt, yshift=5pt]locx-v0.north east) -- 
%% ([xshift=-8pt, yshift=-5pt]locx-v0.south east) --
%% ([xshift=-8pt, yshift=5pt]locy-v0.north east) -- 
 %([xshift=-5pt, yshift=-10pt]locy-v0.south east) node (tid2start) {};
 
%\path (tid2start) node[anchor=north, rectangle, fill=red!20, draw=red, font=\small, inner sep=1pt] {$\thid_2$};
 
 %%green view
%\draw[-, DarkGreen, very thick, rounded corners = 10pt]
 %([xshift=-16pt, yshift=8pt]locx-v1.north east) node (tid3start) {}-- 
 %([xshift=-16pt, yshift=-5pt]locx-v1.south east) --
 %([xshift=-16pt, yshift=5pt]locy-v0.north east) -- 
 %([xshift=-16pt, yshift=-5pt]locy-v0.south east);
 
 %\path (tid3start) node[anchor=south, rectangle, fill=DarkGreen!20, draw=DarkGreen, font=\small, inner sep=1pt] {$\thid_1$};

%\end{pgfonlayer}
%\end{tikzpicture}
%\\
%{\small (a)} & {\small (b)}\\
%\hline
%\begin{tikzpicture}[font=\large]

%\begin{pgfonlayer}{foreground}
%%Uncomment line below for help lines
%%\draw[help lines] grid(5,4);

%%Location x
%\node(locx) at (1,3) {$[\loc_x] \mapsto$};

%\matrix(locxcells) [version list, text width=7mm, anchor=west]
   %at ([xshift=10pt]locx.east) {
 %{a} & $\tsid_0$ & {a} & $\tsid_1$\\
  %{a} & $\emptyset$ & {a} & $\emptyset$ \\
%};
%\node[version node, fit=(locxcells-1-1) (locxcells-2-1), fill=white, inner sep= 0cm, font=\Large] (locx-v0) {$0$};
%\node[version node, fit=(locxcells-1-3) (locxcells-2-3), fill=white, inner sep=0cm, font=\Large] (locx-v-1) {$1$};
%%Location y
%\path (locx.south) + (0,-1.5) node (locy) {$[\loc_y] \mapsto$};
%\matrix(locycells) [version list, text width=7mm, anchor=west]
   %at ([xshift=10pt]locy.east) {
 %{a} & $\tsid_0$ \\
   %{a} & $\emptyset$ \\
%};
%\node[version node, fit=(locycells-1-1) (locycells-2-1), fill=white, inner sep= 0cm, font=\Large] (locy-v0) {$0$};
%% \draw[-, red, very thick, rounded corners] ([xshift=-5pt, yshift=5pt]locx-v1.north east) |- 
%%  ($([xshift=-5pt,yshift=-5pt]locx-v1.south east)!.5!([xshift=-5pt, yshift=5pt]locy-v0.north east)$) -| ([xshift=-5pt, yshift=5pt]locy-v0.south east);

%%blue view - I should  check whether I can use pgfkeys to just declare the list of locations, and then add the view automatically.
%\draw[-, blue, very thick, rounded corners=10pt]
 %([xshift=-2pt, yshift=20pt]locx-v0.north east) node (tid1start) {} -- 
%% ([xshift=-2pt, yshift=-5pt]locx-v0.south east) --
%% ([xshift=-2pt, yshift=5pt]locy-v0.north east) -- 
 %([xshift=-2pt, yshift=-5pt]locy-v0.south east);
 
 %\path (tid1start) node[anchor=south, rectangle, fill=blue!20, draw=blue, font=\small, inner sep=1pt] {$\thid_3$};

%%red view
%\draw[-, red, very thick, rounded corners = 10pt]
 %([xshift=-5pt, yshift=5pt]locx-v1.north east) -- 
 %([xshift=-5pt, yshift=-5pt]locx-v1.south east) --
 %([xshift=-5pt, yshift=3pt]locy-v0.north east) -- 
 %([xshift=-5pt, yshift=-10pt]locy-v0.south east) node (tid2start) {};
 
%\path (tid2start) node[anchor=north, rectangle, fill=red!20, draw=red, font=\small, inner sep=1pt] {$\thid_2$};
 
 %%green view
%\draw[-, DarkGreen, very thick, rounded corners = 10pt]
 %([xshift=-16pt, yshift=8pt]locx-v1.north east) node (tid3start) {}-- 
 %([xshift=-16pt, yshift=-5pt]locx-v1.south east) --
 %([xshift=-16pt, yshift=5pt]locy-v0.north east) -- 
 %([xshift=-16pt, yshift=-5pt]locy-v0.south east);
 
 %\path (tid3start) node[anchor=south, rectangle, fill=DarkGreen!20, draw=DarkGreen, font=\small, inner sep=1pt] {$\thid_1$};

%\end{pgfonlayer}
%\end{tikzpicture}
%&
%\begin{tikzpicture}[font=\large]

%\begin{pgfonlayer}{foreground}
%%Uncomment line below for help lines
%%\draw[help lines] grid(5,4);

%%Location x
%\node(locx) at (1,3) {$[\loc_x] \mapsto$};

%\matrix(locxcells) [version list, text width=7mm, anchor=west]
   %at ([xshift=10pt]locx.east) {
 %{a} & $\tsid_0$ & {a} & $\tsid_1$\\
  %{a} & $\emptyset$ & {a} & $\{\tsid_2\}$ \\
%};
%\node[version node, fit=(locxcells-1-1) (locxcells-2-1), fill=white, inner sep= 0cm, font=\Large] (locx-v0) {$0$};
%\node[version node, fit=(locxcells-1-3) (locxcells-2-3), fill=white, inner sep=0cm, font=\Large] (locx-v-1) {$1$};
%%Location y
%\path (locx.south) + (0,-1.5) node (locy) {$[\loc_y] \mapsto$};
%\matrix(locycells) [version list, text width=7mm, anchor=west]
   %at ([xshift=10pt]locy.east) {
 %{a} & $\tsid_0$ & {a} & $\tsid_2$ \\
   %{a} & $\emptyset$ & {a} & $\emptyset$\\
%};
%\node[version node, fit=(locycells-1-1) (locycells-2-1), fill=white, inner sep= 0cm, font=\Large] (locy-v0) {$0$};
%\node[version node, fit=(locycells-1-3) (locycells-2-3), fill=white, inner sep=0cm, font=\Large] (locy-v-1) {$1$};
%% \draw[-, red, very thick, rounded corners] ([xshift=-5pt, yshift=5pt]locx-v1.north east) |- 
%%  ($([xshift=-5pt,yshift=-5pt]locx-v1.south east)!.5!([xshift=-5pt, yshift=5pt]locy-v0.north east)$) -| ([xshift=-5pt, yshift=5pt]locy-v0.south east);

%%blue view - I should  check whether I can use pgfkeys to just declare the list of locations, and then add the view automatically.
%\draw[-, blue, very thick, rounded corners=10pt]
 %([xshift=-2pt, yshift=20pt]locx-v0.north east) node (tid1start) {} -- 
%% ([xshift=-2pt, yshift=-5pt]locx-v0.south east) --
%% ([xshift=-2pt, yshift=5pt]locy-v0.north east) -- 
 %([xshift=-2pt, yshift=-5pt]locy-v0.south east);
 
 %\path (tid1start) node[anchor=south, rectangle, fill=blue!20, draw=blue, font=\small, inner sep=1pt] {$\thid_3$};

%%red view
%\draw[-, red, very thick, rounded corners = 10pt]
 %([xshift=-5pt, yshift=5pt]locx-v1.north east) -- 
%% ([xshift=-5pt, yshift=-5pt]locx-v1.south east) --
%% ([xshift=-5pt, yshift=3pt]locy-v0.north east) -- 
 %([xshift=-5pt, yshift=-10pt]locy-v1.south east) node (tid2start) {};
 
%\path (tid2start) node[anchor=north, rectangle, fill=red!20, draw=red, font=\small, inner sep=1pt] {$\thid_2$};
 
 %%green view
%\draw[-, DarkGreen, very thick, rounded corners = 10pt]
 %([xshift=-16pt, yshift=8pt]locx-v1.north east) node (tid3start) {}-- 
 %([xshift=-16pt, yshift=-5pt]locx-v1.south east) --
 %([xshift=-16pt, yshift=5pt]locy-v0.north east) -- 
 %([xshift=-16pt, yshift=-5pt]locy-v0.south east);
 
 %\path (tid3start) node[anchor=south, rectangle, fill=DarkGreen!20, draw=DarkGreen, font=\small, inner sep=1pt] {$\thid_1$};

%\end{pgfonlayer}
%\end{tikzpicture}\\
%{\small (c)} & {\small (d)} \\
%\hline
%\begin{tikzpicture}[font=\large]

%\begin{pgfonlayer}{foreground}
%%Uncomment line below for help lines
%%\draw[help lines] grid(5,4);

%%Location x
%\node(locx) at (1,3) {$[\loc_x] \mapsto$};

%\matrix(locxcells) [version list, text width=7mm, anchor=west]
   %at ([xshift=10pt]locx.east) {
 %{a} & $\tsid_0$ & {a} & $\tsid_1$\\
  %{a} & $\emptyset$ & {a} & $\{\tsid_2\}$ \\
%};
%\node[version node, fit=(locxcells-1-1) (locxcells-2-1), fill=white, inner sep= 0cm, font=\Large] (locx-v0) {$0$};
%\node[version node, fit=(locxcells-1-3) (locxcells-2-3), fill=white, inner sep=0cm, font=\Large] (locx-v-1) {$1$};
%%Location y
%\path (locx.south) + (0,-1.5) node (locy) {$[\loc_y] \mapsto$};
%\matrix(locycells) [version list, text width=7mm, anchor=west]
   %at ([xshift=10pt]locy.east) {
 %{a} & $\tsid_0$ & {a} & $\tsid_2$ \\
   %{a} & $\emptyset$ & {a} & $\emptyset$\\
%};
%\node[version node, fit=(locycells-1-1) (locycells-2-1), fill=white, inner sep= 0cm, font=\Large] (locy-v0) {$0$};
%\node[version node, fit=(locycells-1-3) (locycells-2-3), fill=white, inner sep=0cm, font=\Large] (locy-v-1) {$1$};
%% \draw[-, red, very thick, rounded corners] ([xshift=-5pt, yshift=5pt]locx-v1.north east) |- 
%%  ($([xshift=-5pt,yshift=-5pt]locx-v1.south east)!.5!([xshift=-5pt, yshift=5pt]locy-v0.north east)$) -| ([xshift=-5pt, yshift=5pt]locy-v0.south east);

%%blue view - I should  check whether I can use pgfkeys to just declare the list of locations, and then add the view automatically.
%\draw[-, blue, very thick, rounded corners=10pt]
 %([xshift=-2pt, yshift=20pt]locx-v0.north east) node (tid1start) {} -- 
 %([xshift=-2pt, yshift=-5pt]locx-v0.south east) --
 %([xshift=-2pt, yshift=5pt]locy-v1.north east) -- 
 %([xshift=-2pt, yshift=-5pt]locy-v1.south east);
 
 %\path (tid1start) node[anchor=south, rectangle, fill=blue!20, draw=blue, font=\small, inner sep=1pt] {$\thid_3$};

%%red view
%\draw[-, red, very thick, rounded corners = 10pt]
 %([xshift=-5pt, yshift=5pt]locx-v1.north east) -- 
%% ([xshift=-5pt, yshift=-5pt]locx-v1.south east) --
%% ([xshift=-5pt, yshift=3pt]locy-v0.north east) -- 
 %([xshift=-5pt, yshift=-10pt]locy-v1.south east) node (tid2start) {};
 
%\path (tid2start) node[anchor=north, rectangle, fill=red!20, draw=red, font=\small, inner sep=1pt] {$\thid_2$};
 
 %%green view
%\draw[-, DarkGreen, very thick, rounded corners = 10pt]
 %([xshift=-16pt, yshift=8pt]locx-v1.north east) node (tid3start) {}-- 
 %([xshift=-16pt, yshift=-5pt]locx-v1.south east) --
 %([xshift=-16pt, yshift=5pt]locy-v0.north east) -- 
 %([xshift=-16pt, yshift=-5pt]locy-v0.south east);
 
 %\path (tid3start) node[anchor=south, rectangle, fill=DarkGreen!20, draw=DarkGreen, font=\small, inner sep=1pt] {$\thid_1$};

%\end{pgfonlayer}
%\end{tikzpicture}
%&
%\begin{tikzpicture}[font=\large]

%\begin{pgfonlayer}{foreground}
%%Uncomment line below for help lines
%%\draw[help lines] grid(5,4);

%%Location x
%\node(locx) at (1,3) {$[\loc_x] \mapsto$};

%\matrix(locxcells) [version list, text width=7mm, anchor=west]
   %at ([xshift=10pt]locx.east) {
 %{a} & $\tsid_0$ & {a} & $\tsid_1$\\
  %{a} & $\{\tsid_3\}$ & {a} & $\{\tsid_2\}$ \\
%};
%\node[version node, fit=(locxcells-1-1) (locxcells-2-1), fill=white, inner sep= 0cm, font=\Large] (locx-v0) {$0$};
%\node[version node, fit=(locxcells-1-3) (locxcells-2-3), fill=white, inner sep=0cm, font=\Large] (locx-v1) {$1$};
%%Location y
%\path (locx.south) + (0,-1.5) node (locy) {$[\loc_y] \mapsto$};
%\matrix(locycells) [version list, text width=7mm, anchor=west]
   %at ([xshift=10pt]locy.east) {
 %{a} & $\tsid_0$ & {a} & $\tsid_2$ \\
   %{a} & $\emptyset$ & {a} & $\{\tsid_3\}$\\
%};
%\node[version node, fit=(locycells-1-1) (locycells-2-1), fill=white, inner sep= 0cm, font=\Large] (locy-v0) {$0$};
%\node[version node, fit=(locycells-1-3) (locycells-2-3), fill=white, inner sep=0cm, font=\Large] (locy-v1) {$1$};
%% \draw[-, red, very thick, rounded corners] ([xshift=-5pt, yshift=5pt]locx-v1.north east) |- 
%%  ($([xshift=-5pt,yshift=-5pt]locx-v1.south east)!.5!([xshift=-5pt, yshift=5pt]locy-v0.north east)$) -| ([xshift=-5pt, yshift=5pt]locy-v0.south east);

%%blue view - I should  check whether I can use pgfkeys to just declare the list of locations, and then add the view automatically.
%\draw[-, blue, very thick, rounded corners=10pt]
 %([xshift=-2pt, yshift=20pt]locx-v0.north east) node (tid1start) {} -- 
 %([xshift=-2pt, yshift=-5pt]locx-v0.south east) --
 %([xshift=-2pt, yshift=5pt]locy-v1.north east) -- 
 %([xshift=-2pt, yshift=-5pt]locy-v1.south east);
 
 %\path (tid1start) node[anchor=south, rectangle, fill=blue!20, draw=blue, font=\small, inner sep=1pt] {$\thid_3$};

%%red view
%\draw[-, red, very thick, rounded corners = 10pt]
 %([xshift=-5pt, yshift=5pt]locx-v1.north east) -- 
%% ([xshift=-5pt, yshift=-5pt]locx-v1.south east) --
%% ([xshift=-5pt, yshift=3pt]locy-v0.north east) -- 
 %([xshift=-5pt, yshift=-10pt]locy-v1.south east) node (tid2start) {};
 
%\path (tid2start) node[anchor=north, rectangle, fill=red!20, draw=red, font=\small, inner sep=1pt] {$\thid_2$};
 
 %%green view
%\draw[-, DarkGreen, very thick, rounded corners = 10pt]
 %([xshift=-16pt, yshift=8pt]locx-v1.north east) node (tid3start) {}-- 
 %([xshift=-16pt, yshift=-5pt]locx-v1.south east) --
 %([xshift=-16pt, yshift=5pt]locy-v0.north east) -- 
 %([xshift=-16pt, yshift=-5pt]locy-v0.south east);
 
 %\path (tid3start) node[anchor=south, rectangle, fill=DarkGreen!20, draw=DarkGreen, font=\small, inner sep=1pt] {$\thid_1$};

%\end{pgfonlayer}
%\end{tikzpicture}
%\\
%{\small (e)} & {\small (f)} \\
%\hline
%\end{tabular}
%\caption{History heaps obtained in a execution of $\prog_2$.}
%\label{fig:cc.exec}
%\end{figure}


\sx{Should we give intuition about causal dependencies here ?}
The next consistency model that we are interested is \emph{transactional causal consistency} \cite{cops}. 
Intuitively, it ensures that versions read by transactions are closed with respect to \emph{causal dependencies}. 
Consider for example the following program: 
\[
    \prog_2 \equiv \begin{session}
        \begin{array}{@{}c || c || c@{}}
            \txid_{1} : 
            \begin{transaction}
                \pmutate{\vx}{1};\\
            \end{transaction} &
            \txid_{2} : 
            \begin{transaction} 
                \pderef{\pvar{a}}{\vx};\\
                \pmutate{\vy}{\pvar{a}};\\
            \end{transaction} &
            \txid_{3} :
             \begin{transaction}
               	   \pderef{\pvar{a}}{\vx};\\
               	   \pderef{\pvar{b}}{\vy};\\
               	   \pifs{\pvar{a}=0 \wedge \pvar{b}=1}\\
               			\quad \passign{\retvar}{\Large \frownie{}}
               		\pife
             \end{transaction}
        \end{array}
    \end{session}
 \]
For the sake of simplicity, we label the code of the three transactions above as $\txid_{1}, \txid_2, \txid_3$ from left to right.
It is easy to see that, if no constraints or even under read atomic, the third transaction $\txid_{3}$ can return ${\Large \frownie{}}$. 
%The same is true even if the consistency model specification $\mathsf{RA}$ is assumed. 
Informally, the return of value ${\Large \frownie{}}$ by $\txid_3$ can be obtained from the execution outlined below. 
\sx{Did not edit, change later}
\begin{itemize}
\item The initial configuration of this execution is depicted in Figure \ref{fig:cc.exec}(a).
\item $\ptrans{\trans_1}$ executes with the initial view, which points to the 
initial (and only) version for each location; after this transaction is 
executed, a new version $\langle 1, T_1, \emptyset \rangle$ is appended 
at the end of $\hh(\loc_{x})$. The resulting history heap is depicted in Figure \ref{fig:cc.exec}(b).
\item next, $\thid_2$ updates its view as to see the version of $\loc_x$ installed by $\thid_1$, after 
which it proceeds to execute $\ptrans{\trans_2}$. This results in a new version with value $1$ 
to be installed for $\loc_y$. The configurations before, and after the execution of $\ptrans{\trans_2}$, 
are depicted in figures \ref{fig:cc.exec}(c) and \ref{fig:cc.exec}(d), respectively.
\item Finally, thread $\thid_3$ updates its view to observe the update of location $[\loc_y]$, but not the update of 
location $[\loc_y]$, before executing transaction $\ptrans{\trans_3}$. The execution of $\ptrans{\trans_3}$ will 
return the value ${\Large \frownie{}}$. The history heaps immediately before and after 
the execution of $\thid_3$, are depicted in figures \ref{fig:cc.exec}(e) and \ref{fig:cc.exec}(f), respectively. 
\end{itemize}

\sx{Change above}

In the last step, the thread to the right commits the transaction $\txid_3$ in a state where its initial view observes the second version of the address $\vy$, which is created by \( \txid_{2} \).
However, because the transaction \( \txid_{2} \) read the second version of \( \vx \) and create the second version of \( \vy \), this means the latter depends on the former.
Yet the transaction \( \txid_{3} \) does not read from the second version of \( \vx \), which is disallowed by \emph{transactional causal consistency}.
Summarising, under transactional causal consistency if a transaction sees updates for an address \( \addr \), it should also observes those addresses that \( \addr \) depends on.

\ac{
but not the update to address $\vx$.
However, the update of $[\vy]$ committed by $\txid_{2}$, consisted in copying the value of the update 
of $[\loc_y]$: that is, the update of $[\loc_y]$ \textbf{depends} from the update of $[\loc_x]$. 
Summarising, the execution of transaction $\ptrans{\trans_3}$ resulted in a violation of 
causality: the update of $[\loc_y]$ is observed, but not the update of $[\loc_x]$ on which 
it depends.
}

\sx{in RA up-to-date view, and here consistent view, not sure are good words, let re-think later}
To formally specify transactional causal consistency, we inductively define the set of views that are consistent with respect to a history heap $\hh$. 
\sx{What do you mean by the word??}
The definition below models the fact that, if we start from a causally consistent view, and we wish to update the view for some location $\txid_2$, 

\begin{defn}[Transactional causal dependency]
\label{def:causal}
Given two versions $\ver_{1} = (\val_1, \txid_1, \txidset_1)$, $\ver_2 = (\val_2, \txid_2, \txidset_2)$, $\ver_{1}$ \emph{directly depends on} $\ver_2$, written $\pred{ddep}{\ver_{1}, \ver_{2}}$, if $\txid_1 \in \txidset_2$. 
\ac{Note to self: the notion of directly depends here has little to do with dependencies 
between transactions. $\nu_1 \xrightarrow{\mathsf{ddep}} \nu_2$ means that 
some transaction $\tsid$ touches both versions. However, it reads $\nu_2$ and 
writes $\nu_1$.}
Given $\hh \in \HisHeaps$, the set of views that are \emph{causally consistent} with the history heap $\hh$, $\func{CCViews}{\hh}$, is defined as the smallest set such that: 
\begin{itemize} 
\item the initial view \( \vi_0\)  is in the set, \ie $\vi_0 \in \func{CCViews}{\hh}$ where \( \fora{\addr \in \dom(\hh)} \vi_{0}(\addr) = 1 \),
\item assume any view $\vi \in \pred{CCViews}{\hh}$ and a new view \( \vi' \) by observing one more version for an address $\vi' = \vi\rmto{\addr}{\vi(\addr) + 1}$, where \( \vi'(\addr) \leq \left| \hh(\addr) \right| \).
If some versions directly depend on the version corresponding to \( \vi'(\addr)\) and those versions are aware by \( \vi'\), the new view is included in \( \func{CCViews}{\hh}\),
%for some $i: V([\loc_x]) < i \leq \lvert \hh \rvert -1$
%for some $[\loc_x]$ such that $V([\loc_x]) < \lvert \hh([\loc_x]) \rvert - 1$.
%Suppose that 
\[
\begin{array}{@{}l}
\fora{\vi,\vi'} \exsts{\addr}
\vi \in \func{CCViews}{\hh} 
\land \addr \in \dom(\vi)
\land \vi' = \vi\rmto{\addr}{\vi(\addr) + 1}
\land \vi'(\addr) \leq \left| \hh(\addr) \right|  \\
\quad {} \land 
\begin{B}
\fora{\addr', i}  
1 \leq i \leq \left| \hh(\addr') \right|
\land \pred{ddep}{\hh(\addr')(i), \hh(\addr)(\vi'(\addr))}
\implies i \leq \vi(\addr')
\end{B} \\
\qquad {} \implies \vi' \in \func{CCViews}{\hh}
\end{array}
\]

%for any location $[\loc_y]$ and 
%index $j = 0, \cdots, \lvert \hh([\loc_y]) \rvert -1$ such that $\hh([\loc_{x})(V'([\loc_x]))$ 
%directly depends on $\hh([\loc_y])(j)$, then $j \leq V([\loc_y])$. Then 
%$V' \in \mathsf{CCViews}(\hh)$.
% and suppose that $\hh([\loc_{x}])(V'([\loc_x])) = 
%(\_, \tsid, \_ )$ for some $\tsid$. If for all locations $\loc_{y}$ and 
%indexes $j$ such that $\hh([\loc_y])(j) = (\_, \_, \_ \cup \{\tsid\})$, 
%then $j \leq V'([\loc_y])$, then $V' \in \mathsf{CCViews}(\hh)$.
\end{itemize}
Causal consistency is stronger than read atomic (\defref{def:readatomic}) by further ensuring that a transaction can only observe a causally consistent state of the database, i.e. the view when the transaction starts is causally consistent with the history heap,
\[
\begin{rclarray}
    (\hh, \vi) \csat[\mathsf{CC}] \opset : \vi' & \defeq & (\hh, \vi) \csat[\mathsf{CC}] \opset : \vi' \land \vi \in \func{CCViews}{\hh}
\end{rclarray}
\]
\end{defn}

%\begin{figure}
%\begin{tabular}{|c|c|} 
%\hline
%\begin{tikzpicture}[font=\large]

%\begin{pgfonlayer}{foreground}
%%Uncomment line below for help lines
%%\draw[help lines] grid(5,4);

%%Location x
%\node(locx) at (1,3) {$[\loc_x] \mapsto$};

%\matrix(locxcells) [version list, text width=7mm, anchor=west]
   %at ([xshift=10pt]locx.east) {
 %{a} & $\tsid_0$ & {a} & $\tsid_1$\\
  %{a} & $\emptyset$ & {a} & $\{\tsid_2\}$ \\
%};
%\node[version node, fit=(locxcells-1-1) (locxcells-2-1), fill=white, inner sep= 0cm, font=\Large] (locx-v0) {$0$};
%\node[version node, fit=(locxcells-1-3) (locxcells-2-3), fill=white, inner sep=0cm, font=\Large] (locx-v-1) {$1$};
%%Location y
%\path (locx.south) + (0,-1.5) node (locy) {$[\loc_y] \mapsto$};
%\matrix(locycells) [version list, text width=7mm, anchor=west]
   %at ([xshift=10pt]locy.east) {
 %{a} & $\tsid_0$ & {a} & $\tsid_2$ \\
   %{a} & $\emptyset$ & {a} & $\{\tsid_3\}$\\
%};
%\node[version node, fit=(locycells-1-1) (locycells-2-1), fill=white, inner sep= 0cm, font=\Large] (locy-v0) {$0$};
%\node[version node, fit=(locycells-1-3) (locycells-2-3), fill=white, inner sep=0cm, font=\Large] (locy-v1) {$1$};
%% \draw[-, red, very thick, rounded corners] ([xshift=-5pt, yshift=5pt]locx-v1.north east) |- 
%%  ($([xshift=-5pt,yshift=-5pt]locx-v1.south east)!.5!([xshift=-5pt, yshift=5pt]locy-v0.north east)$) -| ([xshift=-5pt, yshift=5pt]locy-v0.south east);

%\path (locy.south) + (0,-1.5) node (locy) {$[\loc_z] \mapsto$};
%\matrix(loczcells) [version list, text width=7mm, anchor=west]
   %at ([xshift=10pt]locy.east) {
 %{a} & $\tsid_0$ & {a} & $\tsid_3$ \\
   %{a} & $\emptyset$ & {a} & $\emptyset$\\
%};
%\node[version node, fit=(loczcells-1-1) (loczcells-2-1), fill=white, inner sep= 0cm, font=\Large] (locz-v0) {$0$};
%\node[version node, fit=(loczcells-1-3) (loczcells-2-3), fill=white, inner sep=0cm, font=\Large] (locz-v1) {$1$};

%%blue view - I should  check whether I can use pgfkeys to just declare the list of locations, and then add the view automatically.
%\draw[-, blue, very thick, rounded corners=10pt]
 %([xshift=-5pt, yshift=5pt]locx-v0.north east) node (tid1start) {} -- 
%% ([xshift=-5pt, yshift=-5pt]locx-v0.south east) --
%% ([xshift=-5pt, yshift=5pt]locy-v1.north east) -- 
 %([xshift=-5pt, yshift=-5pt]locz-v0.south east);
 
%% \path (tid1start) node[anchor=south, rectangle, fill=blue!20, draw=blue, font=\small, inner sep=1pt] {$\thid_3$};
%\end{pgfonlayer}
%\end{tikzpicture}
%&
%\begin{tikzpicture}[font=\large]

%\begin{pgfonlayer}{foreground}
%%Uncomment line below for help lines
%%\draw[help lines] grid(5,4);

%%Location x
%\node(locx) at (1,3) {$[\loc_x] \mapsto$};

%\matrix(locxcells) [version list, text width=7mm, anchor=west]
   %at ([xshift=10pt]locx.east) {
 %{a} & $\tsid_0$ & {a} & $\tsid_1$\\
  %{a} & $\emptyset$ & {a} & $\{\tsid_2\}$ \\
%};
%\node[version node, fit=(locxcells-1-1) (locxcells-2-1), fill=white, inner sep= 0cm, font=\Large] (locx-v0) {$0$};
%\node[version node, fit=(locxcells-1-3) (locxcells-2-3), fill=white, inner sep=0cm, font=\Large] (locx-v1) {$1$};

%%Location y
%\path (locx.south) + (0,-1.5) node (locy) {$[\loc_y] \mapsto$};
%\matrix(locycells) [version list, text width=7mm, anchor=west]
   %at ([xshift=10pt]locy.east) {
 %{a} & $\tsid_0$ & {a} & $\tsid_2$ \\
   %{a} & $\emptyset$ & {a} & $\{\tsid_3\}$\\
%};
%\node[version node, fit=(locycells-1-1) (locycells-2-1), fill=white, inner sep= 0cm, font=\Large] (locy-v0) {$0$};
%\node[version node, fit=(locycells-1-3) (locycells-2-3), fill=white, inner sep=0cm, font=\Large] (locy-v1) {$1$};



%\path (locy.south) + (0,-1.5) node (locz) {$[\loc_z] \mapsto$};
%\matrix(loczcells) [version list, text width=7mm, anchor=west]
   %at ([xshift=10pt]locz.east) {
 %{a} & $\tsid_0$ & {a} & $\tsid_3$ \\
   %{a} & $\emptyset$ & {a} & $\emptyset$\\
%};

%\node[version node, fit=(loczcells-1-1) (loczcells-2-1), fill=white, inner sep= 0cm, font=\Large] (locz-v0) {$0$};
%\node[version node, fit=(loczcells-1-3) (loczcells-2-3), fill=white, inner sep=0cm, font=\Large] (locz-v1) {$1$};

%%blue view - I should  check whether I can use pgfkeys to just declare the list of locations, and then add the view automatically.
%\draw[-, blue, very thick, rounded corners=10pt]
 %([xshift=-5pt, yshift=5pt]locx-v1.north east) node (tid1start) {} -- 
 %([xshift=-5pt, yshift=-5pt]locx-v1.south east) --
 %([xshift=-5pt, yshift=5pt]locy-v0.north east) -- 
%% ([xshift=-5pt, yshift=-5pt]locy-v0.south east) --
%% ([xshift=-5pt, yshift=5pt]locz-v0.north east) -- 
 %([xshift=-5pt, yshift=-5pt]locz-v0.south east);
 
%% \path (tid1start) node[anchor=south, rectangle, fill=blue!20, draw=blue, font=\small, inner sep=1pt] {$\thid_3$};
%\end{pgfonlayer}
%\end{tikzpicture}
%\\
%{\small(a)} & {\small(b)}\\
%\hline
%\begin{tikzpicture}[font=\large]

%\begin{pgfonlayer}{foreground}
%%Uncomment line below for help lines
%%\draw[help lines] grid(5,4);

%%Location x
%\node(locx) at (1,3) {$[\loc_x] \mapsto$};

%\matrix(locxcells) [version list, text width=7mm, anchor=west]
   %at ([xshift=10pt]locx.east) {
 %{a} & $\tsid_0$ & {a} & $\tsid_1$\\
  %{a} & $\emptyset$ & {a} & $\{\tsid_2\}$ \\
%};
%\node[version node, fit=(locxcells-1-1) (locxcells-2-1), fill=white, inner sep= 0cm, font=\Large] (locx-v0) {$0$};
%\node[version node, fit=(locxcells-1-3) (locxcells-2-3), fill=white, inner sep=0cm, font=\Large] (locx-v1) {$1$};

%%Location y
%\path (locx.south) + (0,-1.5) node (locy) {$[\loc_y] \mapsto$};
%\matrix(locycells) [version list, text width=7mm, anchor=west]
   %at ([xshift=10pt]locy.east) {
 %{a} & $\tsid_0$ & {a} & $\tsid_2$ \\
   %{a} & $\emptyset$ & {a} & $\{\tsid_3\}$\\
%};
%\node[version node, fit=(locycells-1-1) (locycells-2-1), fill=white, inner sep= 0cm, font=\Large] (locy-v0) {$0$};
%\node[version node, fit=(locycells-1-3) (locycells-2-3), fill=white, inner sep=0cm, font=\Large] (locy-v1) {$1$};



%\path (locy.south) + (0,-1.5) node (locz) {$[\loc_z] \mapsto$};
%\matrix(loczcells) [version list, text width=7mm, anchor=west]
   %at ([xshift=10pt]locz.east) {
 %{a} & $\tsid_0$ & {a} & $\tsid_3$ \\
   %{a} & $\emptyset$ & {a} & $\emptyset$\\
%};

%\node[version node, fit=(loczcells-1-1) (loczcells-2-1), fill=white, inner sep= 0cm, font=\Large] (locz-v0) {$0$};
%\node[version node, fit=(loczcells-1-3) (loczcells-2-3), fill=white, inner sep=0cm, font=\Large] (locz-v1) {$1$};

%%blue view - I should  check whether I can use pgfkeys to just declare the list of locations, and then add the view automatically.
%\draw[-, blue, very thick, rounded corners=10pt]
 %([xshift=-5pt, yshift=5pt]locx-v1.north east) node (tid1start) {} -- 
%% ([xshift=-5pt, yshift=-5pt]locx-v1.south east) --
%% ([xshift=-5pt, yshift=5pt]locy-v0.north east) -- 
 %([xshift=-5pt, yshift=-5pt]locy-v1.south east) --
 %([xshift=-5pt, yshift=5pt]locz-v0.north east) -- 
 %([xshift=-5pt, yshift=-5pt]locz-v0.south east);
 
%% \path (tid1start) node[anchor=south, rectangle, fill=blue!20, draw=blue, font=\small, inner sep=1pt] {$\thid_3$};
%\end{pgfonlayer}
%\end{tikzpicture}
%&
%\begin{tikzpicture}[font=\large]

%\begin{pgfonlayer}{foreground}
%%Uncomment line below for help lines
%%\draw[help lines] grid(5,4);

%%Location x
%\node(locx) at (1,3) {$[\loc_x] \mapsto$};

%\matrix(locxcells) [version list, text width=7mm, anchor=west]
   %at ([xshift=10pt]locx.east) {
 %{a} & $\tsid_0$ & {a} & $\tsid_1$\\
  %{a} & $\emptyset$ & {a} & $\{\tsid_2\}$ \\
%};
%\node[version node, fit=(locxcells-1-1) (locxcells-2-1), fill=white, inner sep= 0cm, font=\Large] (locx-v0) {$0$};
%\node[version node, fit=(locxcells-1-3) (locxcells-2-3), fill=white, inner sep=0cm, font=\Large] (locx-v1) {$1$};

%%Location y
%\path (locx.south) + (0,-1.5) node (locy) {$[\loc_y] \mapsto$};
%\matrix(locycells) [version list, text width=7mm, anchor=west]
   %at ([xshift=10pt]locy.east) {
 %{a} & $\tsid_0$ & {a} & $\tsid_2$ \\
   %{a} & $\emptyset$ & {a} & $\{\tsid_3\}$\\
%};
%\node[version node, fit=(locycells-1-1) (locycells-2-1), fill=white, inner sep= 0cm, font=\Large] (locy-v0) {$0$};
%\node[version node, fit=(locycells-1-3) (locycells-2-3), fill=white, inner sep=0cm, font=\Large] (locy-v1) {$1$};



%\path (locy.south) + (0,-1.5) node (locz) {$[\loc_z] \mapsto$};
%\matrix(loczcells) [version list, text width=7mm, anchor=west]
   %at ([xshift=10pt]locz.east) {
 %{a} & $\tsid_0$ & {a} & $\tsid_3$ \\
   %{a} & $\emptyset$ & {a} & $\emptyset$\\
%};

%\node[version node, fit=(loczcells-1-1) (loczcells-2-1), fill=white, inner sep= 0cm, font=\Large] (locz-v0) {$0$};
%\node[version node, fit=(loczcells-1-3) (loczcells-2-3), fill=white, inner sep=0cm, font=\Large] (locz-v1) {$1$};

%%blue view - I should  check whether I can use pgfkeys to just declare the list of locations, and then add the view automatically.
%\draw[-, blue, very thick, rounded corners=10pt]
 %([xshift=-5pt, yshift=5pt]locx-v1.north east) node (tid1start) {} -- 
%% ([xshift=-5pt, yshift=-5pt]locx-v1.south east) --
%% ([xshift=-5pt, yshift=5pt]locy-v0.north east) -- 
%% ([xshift=-5pt, yshift=-5pt]locy-v1.south east) --
%% ([xshift=-5pt, yshift=5pt]locz-v0.north east) -- 
 %([xshift=-5pt, yshift=-5pt]locz-v1.south east);
 
%% \path (tid1start) node[anchor=south, rectangle, fill=blue!20, draw=blue, font=\small, inner sep=1pt] {$\thid_3$};
%\end{pgfonlayer}
%\end{tikzpicture}\\
%{\small(c)} & {\small(d)}\\
%\hline
%\end{tabular}
%\caption{Building a causally consistent view.}
%\label{fig:cc.view}
%\end{figure}

\sx{Ignore the explanation of example for now}
\ac{Note to self: I got this example and the definition wrong several times before getting them
right. Which means that inductive definition of causally dependent views is not really that 
intuitive after all...}
Consider the history heap $\hh$, depicted in Figure \ref{fig:cc.view}(a) 
- ignore for the moment the view in the Figure.
%\begin{center}
%\begin{tikzpicture}[font=\large]
%
%\begin{pgfonlayer}{foreground}
%%Uncomment line below for help lines
%%\draw[help lines] grid(5,4);
%
%%Location x
%\node(locx) at (1,3) {$[\loc_x] \mapsto$};
%
%\matrix(locxcells) [version list, text width=7mm, anchor=west]
%   at ([xshift=10pt]locx.east) {
% {a} & $\tsid_0$ & {a} & $\tsid_1$\\
%  {a} & $\emptyset$ & {a} & $\{\tsid_2\}$ \\
%};
%\node[version node, fit=(locxcells-1-1) (locxcells-2-1), fill=white, inner sep= 0cm, font=\Large] (locx-v0) {$0$};
%\node[version node, fit=(locxcells-1-3) (locxcells-2-3), fill=white, inner sep=0cm, font=\Large] (locx-v-1) {$1$};
%%Location y
%\path (locx.south) + (0,-1.5) node (locy) {$[\loc_y] \mapsto$};
%\matrix(locycells) [version list, text width=7mm, anchor=west]
%   at ([xshift=10pt]locy.east) {
% {a} & $\tsid_0$ & {a} & $\tsid_2$ \\
%   {a} & $\emptyset$ & {a} & $\{\tsid_3\}$\\
%};
%\node[version node, fit=(locycells-1-1) (locycells-2-1), fill=white, inner sep= 0cm, font=\Large] (locy-v0) {$0$};
%\node[version node, fit=(locycells-1-3) (locycells-2-3), fill=white, inner sep=0cm, font=\Large] (locy-v1) {$1$};
%% \draw[-, red, very thick, rounded corners] ([xshift=-5pt, yshift=5pt]locx-v1.north east) |- 
%%  ($([xshift=-5pt,yshift=-5pt]locx-v1.south east)!.5!([xshift=-5pt, yshift=5pt]locy-v0.north east)$) -| ([xshift=-5pt, yshift=5pt]locy-v0.south east);
%
%\path (locy.south) + (0,-1.5) node (locy) {$[\loc_z] \mapsto$};
%\matrix(loczcells) [version list, text width=7mm, anchor=west]
%   at ([xshift=10pt]locy.east) {
% {a} & $\tsid_0$ & {a} & $\tsid_3$ \\
%   {a} & $\emptyset$ & {a} & $\emptyset$\\
%};
%\node[version node, fit=(loczcells-1-1) (loczcells-2-1), fill=white, inner sep= 0cm, font=\Large] (locz-v0) {$0$};
%\node[version node, fit=(loczcells-1-3) (loczcells-2-3), fill=white, inner sep=0cm, font=\Large] (locz-v1) {$1$};
%
%%blue view - I should  check whether I can use pgfkeys to just declare the list of locations, and then add the view automatically.
%%\draw[-, blue, very thick, rounded corners=10pt]
%% ([xshift=-5pt, yshift=5pt]locx-v0.north east) node (tid1start) {} -- 
%% ([xshift=-5pt, yshift=-5pt]locx-v0.south east) --
%% ([xshift=-5pt, yshift=5pt]locy-v1.north east) -- 
%% ([xshift=-5pt, yshift=-5pt]locy-v1.south east);
% 
%% \path (tid1start) node[anchor=south, rectangle, fill=blue!20, draw=blue, font=\small, inner sep=1pt] {$\thid_3$};
%\end{pgfonlayer}
%\end{tikzpicture}
%\end{center}
We want to find a view $V$ that is up-to-date with the update of location $[\loc_z]$; 
that is, $V[\loc_z] = 1$, and $\hh([\loc_z])(V[\loc_z]) = (1, \txid_3, \emptyset)$.
Furthermore, we want the view $V$ to be causally consistent. 

We construct such a view incrementally, as outlined below. 
Note that for the history heapp $\hh$, we have that 
$\hh([\loc_z])(1) \xrightarrow{\mathsf{ddep}} \hh([\loc_x])(1)$, 
and $\hh([\loc_y])(1) \xrightarrow{\mathsf{ddep}} \hh([\loc_x])(1)$.


We start from the initial view $V_0$, $V_0([\loc_{\_}]) = 0$. This view is 
depicted in Figure \ref{fig:cc.view}(a), and it is 
causally consistent by definition. However, it does not include the update to location $[\loc_z]$: in
 fact, $V_0([\loc_z]) = 0$. As a first try, one could immediately consider a view $V'$ 
where the value for location $[\loc_z]$ is updated to $1$, 
that is $V' = V_0[ [\loc_z] \mapsto 1]$. However, in this case 
the view $V'$ is not causally consistent. In fact, 
we have that $\hh([\loc_z])(V'([\loc_z])) = (1, \txid_3, \emptyset) \xrightarrow{\mathsf{ddep}} 
(1, \tsid_2, \{\tsid_3\}) = \hh([\loc_y])(1)$, but $V'([\loc_y]) = 0 < 1$. 
That is, the version of $[\loc_z]$ observed by $V'$ directly depends on a version 
of $\loc_y$ that is not observed by $V'$.
%
%the version observed 
%by $V'$ for location $[\loc_z]$ is $\hh([\loc_z])(V'([\loc_z]) = 
%\{1, \tsid_3, \emptyset\}$, meaning that the version has been written 
%by transaction $\tsid_3$. On the other hand, we also have 
%that $V'([\loc_y]) = 0$, while $\hh([\loc_{y}])(1) = (1, \tsid_2, 
%\{\tsid_3\}$. That is, the version observed by $V'$ for $[\loc_z]$ 
%directly depends from the version $\hh([\loc_y])(1)$, but the latter 
%is not observed by $V'$; in fact, $V'([\loc_y]) = 1$. 

As a second attempt, one could then try to update the the view of location 
$[\loc_y]$ to value $1$, resulting in the value $V'' = V_0[ [\loc_y] \mapsto 1]$, 
but similarly we would find that the version 
$\hh([\loc_y])(1) = (\_, \txid_2, \_)$ directly depends from the version 
$\hh([\loc_x])(1) = (\_, \_, \{\txid_2\})$, and $V''([\loc_x]) = 0 < 1$. 

Finally, we can update the view $V_0$ to include the update of $[\loc_x]$: 
this results in the view $V_1 = V_0[ [\loc_x] \mapsto 1$. Because $\hh([\loc_x])(1) 
= (\_, \tsid_1, \_)$, and there is no version $\nu$ in $\hh$ on which 
$\hh([\loc_x])(1)$ directly depends - i.e. such that  
$\nu = (\_, \_, \_ \cup \{\txid_1\})$ - we can conclude that $V_1$ is causally 
consistent. The view $V_1$ is depicted in Figure \ref{fig:cc.view}(b).
%We can now proceed to update the view of location $[\loc_y]$ in $T_1$. 
%Because 

%\begin{center}
%\begin{tikzpicture}[font=\large]
%
%\begin{pgfonlayer}{foreground}
%%Uncomment line below for help lines
%%\draw[help lines] grid(5,4);
%
%%Location x
%\node(locx) at (1,3) {$[\loc_x] \mapsto$};
%
%\matrix(locxcells) [version list, text width=7mm, anchor=west]
%   at ([xshift=10pt]locx.east) {
% {a} & $\tsid_0$ & {a} & $\tsid_1$\\
%  {a} & $\emptyset$ & {a} & $\{\tsid_2\}$ \\
%};
%\node[version node, fit=(locxcells-1-1) (locxcells-2-1), fill=white, inner sep= 0cm, font=\Large] (locx-v0) {$0$};
%\node[version node, fit=(locxcells-1-3) (locxcells-2-3), fill=white, inner sep=0cm, font=\Large] (locx-v1) {$1$};
%
%%Location y
%\path (locx.south) + (0,-1.5) node (locy) {$[\loc_y] \mapsto$};
%\matrix(locycells) [version list, text width=7mm, anchor=west]
%   at ([xshift=10pt]locy.east) {
% {a} & $\tsid_0$ & {a} & $\tsid_2$ \\
%   {a} & $\emptyset$ & {a} & $\{\tsid_3\}$\\
%};
%\node[version node, fit=(locycells-1-1) (locycells-2-1), fill=white, inner sep= 0cm, font=\Large] (locy-v0) {$0$};
%\node[version node, fit=(locycells-1-3) (locycells-2-3), fill=white, inner sep=0cm, font=\Large] (locy-v1) {$1$};
%
%
%
%\path (locy.south) + (0,-1.5) node (locz) {$[\loc_z] \mapsto$};
%\matrix(loczcells) [version list, text width=7mm, anchor=west]
%   at ([xshift=10pt]locz.east) {
% {a} & $\tsid_0$ & {a} & $\tsid_3$ \\
%   {a} & $\emptyset$ & {a} & $\emptyset$\\
%};
%
%\node[version node, fit=(loczcells-1-1) (loczcells-2-1), fill=white, inner sep= 0cm, font=\Large] (locz-v0) {$0$};
%\node[version node, fit=(loczcells-1-3) (loczcells-2-3), fill=white, inner sep=0cm, font=\Large] (locz-v1) {$1$};
%
%%blue view - I should  check whether I can use pgfkeys to just declare the list of locations, and then add the view automatically.
%\draw[-, blue, very thick, rounded corners=10pt]
% ([xshift=-5pt, yshift=5pt]locx-v1.north east) node (tid1start) {} -- 
% ([xshift=-5pt, yshift=-5pt]locx-v1.south east) --
% ([xshift=-5pt, yshift=5pt]locy-v0.north east) -- 
%% ([xshift=-5pt, yshift=-5pt]locy-v0.south east) --
%% ([xshift=-5pt, yshift=5pt]locz-v0.north east) -- 
% ([xshift=-5pt, yshift=-5pt]locz-v0.south east);
% 
%% \path (tid1start) node[anchor=south, rectangle, fill=blue!20, draw=blue, font=\small, inner sep=1pt] {$\thid_3$};
%\end{pgfonlayer}
%\end{tikzpicture}
%\end{center}
We can now update the view $V_1$ to include the version $\hh([\loc_y])(1)$, 
resulting in the view $V_2 := V_1[ [\loc_y] \mapsto 1]$, depicted in Figure \ref{fig:cc.view}(c). 
Because the view $V_1$ is 
causally consistent, and because all the versions on which $\hh([\loc_y])(V_2([\loc_y]))$ directly depends
are observed by $V_1$, then $V_2$ is also causally consistent. Similarly, we can define 
the view $V_3 := V_2[[\loc_z] \mapsto 1]$, depicted in Figure \ref{fig:cc.view}(d), 
and prove that it casually consistent. 

%\begin{center}
%\begin{tikzpicture}[font=\large]
%
%\begin{pgfonlayer}{foreground}
%%Uncomment line below for help lines
%%\draw[help lines] grid(5,4);
%
%%Location x
%\node(locx) at (1,3) {$[\loc_x] \mapsto$};
%
%\matrix(locxcells) [version list, text width=7mm, anchor=west]
%   at ([xshift=10pt]locx.east) {
% {a} & $\tsid_0$ & {a} & $\tsid_1$\\
%  {a} & $\emptyset$ & {a} & $\{\tsid_2\}$ \\
%};
%\node[version node, fit=(locxcells-1-1) (locxcells-2-1), fill=white, inner sep= 0cm, font=\Large] (locx-v0) {$0$};
%\node[version node, fit=(locxcells-1-3) (locxcells-2-3), fill=white, inner sep=0cm, font=\Large] (locx-v1) {$1$};
%
%%Location y
%\path (locx.south) + (0,-1.5) node (locy) {$[\loc_y] \mapsto$};
%\matrix(locycells) [version list, text width=7mm, anchor=west]
%   at ([xshift=10pt]locy.east) {
% {a} & $\tsid_0$ & {a} & $\tsid_2$ \\
%   {a} & $\emptyset$ & {a} & $\{\tsid_3\}$\\
%};
%\node[version node, fit=(locycells-1-1) (locycells-2-1), fill=white, inner sep= 0cm, font=\Large] (locy-v0) {$0$};
%\node[version node, fit=(locycells-1-3) (locycells-2-3), fill=white, inner sep=0cm, font=\Large] (locy-v1) {$1$};
%
%
%
%\path (locy.south) + (0,-1.5) node (locz) {$[\loc_z] \mapsto$};
%\matrix(loczcells) [version list, text width=7mm, anchor=west]
%   at ([xshift=10pt]locz.east) {
% {a} & $\tsid_0$ & {a} & $\tsid_3$ \\
%   {a} & $\emptyset$ & {a} & $\emptyset$\\
%};
%
%\node[version node, fit=(loczcells-1-1) (loczcells-2-1), fill=white, inner sep= 0cm, font=\Large] (locz-v0) {$0$};
%\node[version node, fit=(loczcells-1-3) (loczcells-2-3), fill=white, inner sep=0cm, font=\Large] (locz-v1) {$1$};
%
%%blue view - I should  check whether I can use pgfkeys to just declare the list of locations, and then add the view automatically.
%\draw[-, blue, very thick, rounded corners=10pt]
% ([xshift=-5pt, yshift=5pt]locx-v1.north east) node (tid1start) {} -- 
%% ([xshift=-5pt, yshift=-5pt]locx-v1.south east) --
%% ([xshift=-5pt, yshift=5pt]locy-v0.north east) -- 
% ([xshift=-5pt, yshift=-5pt]locy-v1.south east) --
% ([xshift=-5pt, yshift=5pt]locz-v0.north east) -- 
% ([xshift=-5pt, yshift=-5pt]locz-v0.south east);
% 
%% \path (tid1start) node[anchor=south, rectangle, fill=blue!20, draw=blue, font=\small, inner sep=1pt] {$\thid_3$};
%\end{pgfonlayer}
%\end{tikzpicture}
%\hfill
%\begin{tikzpicture}[font=\large]
%
%\begin{pgfonlayer}{foreground}
%%Uncomment line below for help lines
%%\draw[help lines] grid(5,4);
%
%%Location x
%\node(locx) at (1,3) {$[\loc_x] \mapsto$};
%
%\matrix(locxcells) [version list, text width=7mm, anchor=west]
%   at ([xshift=10pt]locx.east) {
% {a} & $\tsid_0$ & {a} & $\tsid_1$\\
%  {a} & $\emptyset$ & {a} & $\{\tsid_2\}$ \\
%};
%\node[version node, fit=(locxcells-1-1) (locxcells-2-1), fill=white, inner sep= 0cm, font=\Large] (locx-v0) {$0$};
%\node[version node, fit=(locxcells-1-3) (locxcells-2-3), fill=white, inner sep=0cm, font=\Large] (locx-v1) {$1$};
%
%%Location y
%\path (locx.south) + (0,-1.5) node (locy) {$[\loc_y] \mapsto$};
%\matrix(locycells) [version list, text width=7mm, anchor=west]
%   at ([xshift=10pt]locy.east) {
% {a} & $\tsid_0$ & {a} & $\tsid_2$ \\
%   {a} & $\emptyset$ & {a} & $\{\tsid_3\}$\\
%};
%\node[version node, fit=(locycells-1-1) (locycells-2-1), fill=white, inner sep= 0cm, font=\Large] (locy-v0) {$0$};
%\node[version node, fit=(locycells-1-3) (locycells-2-3), fill=white, inner sep=0cm, font=\Large] (locy-v1) {$1$};
%
%
%
%\path (locy.south) + (0,-1.5) node (locz) {$[\loc_z] \mapsto$};
%\matrix(loczcells) [version list, text width=7mm, anchor=west]
%   at ([xshift=10pt]locz.east) {
% {a} & $\tsid_0$ & {a} & $\tsid_3$ \\
%   {a} & $\emptyset$ & {a} & $\emptyset$\\
%};
%
%\node[version node, fit=(loczcells-1-1) (loczcells-2-1), fill=white, inner sep= 0cm, font=\Large] (locz-v0) {$0$};
%\node[version node, fit=(loczcells-1-3) (loczcells-2-3), fill=white, inner sep=0cm, font=\Large] (locz-v1) {$1$};
%
%%blue view - I should  check whether I can use pgfkeys to just declare the list of locations, and then add the view automatically.
%\draw[-, blue, very thick, rounded corners=10pt]
% ([xshift=-5pt, yshift=5pt]locx-v1.north east) node (tid1start) {} -- 
%% ([xshift=-5pt, yshift=-5pt]locx-v1.south east) --
%% ([xshift=-5pt, yshift=5pt]locy-v0.north east) -- 
%% ([xshift=-5pt, yshift=-5pt]locy-v1.south east) --
%% ([xshift=-5pt, yshift=5pt]locz-v0.north east) -- 
% ([xshift=-5pt, yshift=-5pt]locz-v1.south east);
% 
%% \path (tid1start) node[anchor=south, rectangle, fill=blue!20, draw=blue, font=\small, inner sep=1pt] {$\thid_3$};
%\end{pgfonlayer}
%\end{tikzpicture}
%\end{center}

\ac{Note to self: here there is something subtle going on. We also need to ensure that dependencies 
caused by the information flow of the program are tracked down. For example, we could have a 
transaction returning the value of a location $[\loc_x]$, and then another transaction copy such a value 
into another location $[\loc_y]$. There is a notion of dependency between $[\loc_x], [\loc_y]$ that 
is not captured by the notion of \emph{directly depends}. On the other hand, the fact that 
stacks are thread local, and we have per-thread view monotonicity, should ensure that also program 
dependencies are preserved. A definitive proof that $\mathsf{CC}$ is equivalent to caual consistency 
specified either in terms of abstract executions or dependency graphs, would settle the argument.}

Note that, the view $V$ of $\thid_3$ in Figure \ref{fig:cc.view}(e) is not causally consistent. This is 
because $V([\loc_y]) = 1$, and $V([\loc_x]) = 0$. However, $\hh([\loc_y])(1)$ directly depends 
in $\hh([\loc_x])(1)$, which is not included in the view. 
In general, the only case in which an execution of $\prog_2$ causes transaction $\ptrans{\trans_3}$ 
to return value ${\Large \frownie{}}$ is when such a transaction is executed using a snapshot determined 
by a non-causally consistent view. There exists no execution of $\prog_2$ under $\mathsf{CC}$ that 
causes $\ptrans{\trans_3}$ to return value ${\Large \frownie{}}$.
%if we execute the program $\prog_2$, illustrated previously in this section, under $\mathsf{CCViews}$,
%it is not possible any more to have the transaction $\ptrans{\trans_3}$ return value ${\Large \frownie}$. 
%This is because, in order for $\ptrans{\trans_3}$ to return such a value, it must be executed in a 
%configuration such as the one of Figure \ref{fig:cc.view}(e) (only the view of $\thid_3$ is relevant here), 
%%state where the view thread of $\thid_3$ observes the update to location $[\loc_y]$, but not the 
%%update of $[\loc_x]$ from which the latter causally depends. 

\ac{In theory, I can do better, and require that I see only the causal dependencies 
of what I read. But at the end of the day, who cares?}

%\begin{definition}
%Let $\hh, V$ and $[\loc_x]$ be a history heap, a view, and a location, respectively. 
%Given two transactions $\tsid_1, \tsid_2$, we say that $\tsid_2$ write-read depends 
%on $\tsid_1$ via $[\loc_x]$, written $\tsid_1 \xrightarrow{\RF([\loc_x])_{\hh}} \tsid_2$, 
%if there exists an index $i = 0,\cdots, \llvert \hh, \rvert -1$ such that $\hh([\loc_n])(x) = 
%(\_, \tsid_1, \mathcal{T})$, and $\tsid_2 \in \mathcal{T}$.
%\end{definition}
%
%\begin{definition}
%Let $\hh$ be a history heap, and $V$ be a view.
%Let $[\loc_x]$ be a location, and let 
%$(\_, \tsid, \_) := \hh([\loc_{n}])(V([\loc_{x}]))$. 
%We say that $V$ respects causality for $[\loc_{x}]$ if, 
%whenever $\tsid' \xrightarrow{\RF([\loc_{x}])}_{\tsid}'$, 
%$\hh([\loc_x])(i)$ as follows: 
%\begin{enumerate}
%\end{definition}



\subsection{Update Atomic}
%\begin{figure}
%\begin{tabular}{|c|c|}
%\hline
%\begin{tikzpicture}[font=\large]

%\begin{pgfonlayer}{foreground}
%%Uncomment line below for help lines
%%\draw[help lines] grid(5,4);

%%Location x
%\node(locx) at (1,3) {$[\loc_x] \mapsto$};

%\matrix(locxcells) [version list, text width=7mm, anchor=west]
   %at ([xshift=10pt]locx.east) {
 %{a} & $T_0$ \\
  %{a} & $\emptyset$ \\
%};
%\node[version node, fit=(locxcells-1-1) (locxcells-2-1), fill=white, inner sep= 0cm, font=\Large] (locx-v0) {$0$};

%%Location y
%\path (locx.south) + (0,-1.5) node (locf1) {$[\loc_{f_1}] \mapsto$};
%\matrix(locf1cells) [version list, text width=7mm, anchor=west]
   %at ([xshift=10pt]locf1.east) {
 %{a} & $T_0$ \\
  %{a} & $\emptyset$ \\
%};
%\node[version node, fit=(locf1cells-1-1) (locf1cells-2-1), fill=white, inner sep= 0cm, font=\Large] (locf1-v0) {$0$};

%%Location y
%\path (locf1.south) + (0,-1.5) node (locf2) {$[\loc_{f_2}] \mapsto$};
%\matrix(locf2cells) [version list, text width=7mm, anchor=west]
   %at ([xshift=10pt]locf2.east) {
 %{a} & $T_0$ \\
  %{a} & $\emptyset$ \\
%};
%\node[version node, fit=(locf2cells-1-1) (locf2cells-2-1), fill=white, inner sep= 0cm, font=\Large] (locf2-v0) {$0$};

%% \draw[-, red, very thick, rounded corners] ([xshift=-5pt, yshift=5pt]locx-v1.north east) |- 
%%  ($([xshift=-5pt,yshift=-5pt]locx-v1.south east)!.5!([xshift=-5pt, yshift=5pt]locy-v0.north east)$) -| ([xshift=-5pt, yshift=5pt]locy-v0.south east);

%%blue view - I should  check whether I can use pgfkeys to just declare the list of locations, and then add the view automatically.
%\draw[-, blue, very thick, rounded corners=10pt]
 %([xshift=-2pt, yshift=20pt]locx-v0.north east) node (tid1start) {} -- 
%% ([xshift=-2pt, yshift=-5pt]locx-v0.south east) --
%% ([xshift=-2pt, yshift=5pt]locf1-v0.north east) -- 
%% ([xshift=-2pt, yshift=-5pt]locf1-v0.south east) --
%% ([xshift=-2pt, yshift=5pt]locf2-v0.north east) -- 
 %([xshift=-2pt, yshift=-5pt]locf2-v0.south east);
 
 %\path (tid1start) node[anchor=south, rectangle, fill=blue!20, draw=blue, font=\small, inner sep=1pt] {$\thid_3$};

%%red view
%\draw[-, red, very thick, rounded corners = 10pt]
 %([xshift=-5pt, yshift=5pt]locx-v0.north east) -- 
%% ([xshift=-5pt, yshift=-5pt]locx-v0.south east) --
%% ([xshift=-5pt, yshift=5pt]locf1-v0.north east) -- 
%% ([xshift=-5pt, yshift=-5pt]locf1-v0.south east) --
%% ([xshift=-5pt, yshift=5pt]locf2-v0.north east) -- 
 %([xshift=-5pt, yshift=-10pt]locf2-v0.south east) node (tid2start) {};
 
%\path (tid2start) node[anchor=north, rectangle, fill=red!20, draw=red, font=\small, inner sep=1pt] {$\thid_2$};
 
 %%green view
%\draw[-, DarkGreen, very thick, rounded corners = 10pt]
 %([xshift=-16pt, yshift=8pt]locx-v0.north east) node (tid3start) {}-- 
%% ([xshift=-16pt, yshift=-5pt]locx-v0.south east) --
%% ([xshift=-16pt, yshift=5pt]locf1-v0.north east) -- 
%% ([xshift=-16pt, yshift=-5pt]locf1-v0.south east) --
%% ([xshift=-16pt, yshift=5pt]locf2-v0.north east) -- 
 %([xshift=-16pt, yshift=-5pt]locf2-v0.south east);
 
 %\path (tid3start) node[anchor=south, rectangle, fill=DarkGreen!20, draw=DarkGreen, font=\small, inner sep=1pt] {$\thid_1$};

%\end{pgfonlayer}
%\end{tikzpicture}
%%
%&
%%
%\begin{tikzpicture}[font=\large]

%\begin{pgfonlayer}{foreground}
%%Uncomment line below for help lines
%%\draw[help lines] grid(5,4);

%%Location x
%\node(locx) at (1,3) {$[\loc_x] \mapsto$};

%\matrix(locxcells) [version list, text width=7mm, anchor=west]
   %at ([xshift=10pt]locx.east) {
 %{a} & $\tsid_0$ & {a} & $\tsid_1$\\
  %{a} & $\{\tsid_1\}$ & {a} & $\emptyset$\\
%};
%\node[version node, fit=(locxcells-1-1) (locxcells-2-1), fill=white, inner sep= 0cm, font=\Large] (locx-v0) {$0$};
%\node[version node, fit=(locxcells-1-3) (locxcells-2-3), fill=white, inner sep= 0cm, font=\Large] (locx-v1) {$1$};

%%Location y
%\path (locx.south) + (0,-1.5) node (locf1) {$[\loc_{f_1}] \mapsto$};
%\matrix(locf1cells) [version list, text width=7mm, anchor=west]
   %at ([xshift=10pt]locf1.east) {
 %{a} & $\tsid_0$ & {a} & $\tsid_1$\\
  %{a} & $\{\tsid_1\}$ & {a} & $\emptyset$\\
%};
%\node[version node, fit=(locf1cells-1-1) (locf1cells-2-1), fill=white, inner sep= 0cm, font=\Large] (locf1-v0) {$0$};
%\node[version node, fit=(locf1cells-1-3) (locf1cells-2-3), fill=white, inner sep= 0cm, font=\Large] (locf1-v1) {$1$};

%%Location y
%\path (locf1.south) + (0,-1.5) node (locf2) {$[\loc_{f_2}] \mapsto$};
%\matrix(locf2cells) [version list, text width=7mm, anchor=west]
   %at ([xshift=10pt]locf2.east) {
 %{a} & $T_0$ \\
  %{a} & $\emptyset$ \\
%};
%\node[version node, fit=(locf2cells-1-1) (locf2cells-2-1), fill=white, inner sep= 0cm, font=\Large] (locf2-v0) {$0$};

%% \draw[-, red, very thick, rounded corners] ([xshift=-5pt, yshift=5pt]locx-v1.north east) |- 
%%  ($([xshift=-5pt,yshift=-5pt]locx-v1.south east)!.5!([xshift=-5pt, yshift=5pt]locy-v0.north east)$) -| ([xshift=-5pt, yshift=5pt]locy-v0.south east);

%%blue view - I should  check whether I can use pgfkeys to just declare the list of locations, and then add the view automatically.
%\draw[-, blue, very thick, rounded corners=10pt]
 %([xshift=-2pt, yshift=20pt]locx-v0.north east) node (tid1start) {} -- 
%% ([xshift=-2pt, yshift=-5pt]locx-v0.south east) --
%% ([xshift=-2pt, yshift=5pt]locf1-v0.north east) -- 
%% ([xshift=-2pt, yshift=-5pt]locf1-v0.south east) --
%% ([xshift=-2pt, yshift=5pt]locf2-v0.north east) -- 
 %([xshift=-2pt, yshift=-5pt]locf2-v0.south east);
 
 %\path (tid1start) node[anchor=south, rectangle, fill=blue!20, draw=blue, font=\small, inner sep=1pt] {$\thid_3$};

%%red view
%\draw[-, red, very thick, rounded corners = 10pt]
 %([xshift=-5pt, yshift=5pt]locx-v0.north east) -- 
%% ([xshift=-5pt, yshift=-5pt]locx-v0.south east) --
%% ([xshift=-5pt, yshift=5pt]locf1-v0.north east) -- 
%% ([xshift=-5pt, yshift=-5pt]locf1-v0.south east) --
%% ([xshift=-5pt, yshift=5pt]locf2-v0.north east) -- 
 %([xshift=-5pt, yshift=-10pt]locf2-v0.south east) node (tid2start) {};
 
%\path (tid2start) node[anchor=north, rectangle, fill=red!20, draw=red, font=\small, inner sep=1pt] {$\thid_2$};
 
 %%green view
%\draw[-, DarkGreen, very thick, rounded corners = 10pt]
 %([xshift=-16pt, yshift=8pt]locx-v1.north east) node (tid3start) {}-- 
%% ([xshift=-16pt, yshift=-5pt]locx-v0.south east) --
%% ([xshift=-16pt, yshift=5pt]locf1-v0.north east) -- 
 %([xshift=-16pt, yshift=-5pt]locf1-v1.south east) --
 %([xshift=-16pt, yshift=5pt]locf2-v0.north east) -- 
 %([xshift=-16pt, yshift=-5pt]locf2-v0.south east);
 
 %\path (tid3start) node[anchor=south, rectangle, fill=DarkGreen!20, draw=DarkGreen, font=\small, inner sep=1pt] {$\thid_1$};

%\end{pgfonlayer}
%\end{tikzpicture}
%\\
%{\small (a)} & {\small (b)}\\
%\hline
%\begin{tikzpicture}[font=\large]

%\begin{pgfonlayer}{foreground}
%%Uncomment line below for help lines
%%\draw[help lines] grid(5,4);

%%Location x
%\node(locx) at (1,3) {$[\loc_x] \mapsto$};

%\matrix(locxcells) [version list, column 2/.style = {text width=14mm}, text width=7mm, anchor=west]
   %at ([xshift=10pt]locx.east) {
 %{a} & $\tsid_0$ & {a} & $\tsid_1$ & {a} & $\tsid_2$\\
  %{a} & $\{\tsid_1, \tsid_2\}$ & {a} & $\emptyset$ & {a} & $\emptyset$\\
%};
%\node[version node, fit=(locxcells-1-1) (locxcells-2-1), fill=white, inner sep= 0cm, font=\Large] (locx-v0) {$0$};
%\node[version node, fit=(locxcells-1-3) (locxcells-2-3), fill=white, inner sep= 0cm, font=\Large] (locx-v1) {$1$};
%\node[version node, fit=(locxcells-1-5) (locxcells-2-5), fill=white, inner sep= 0cm, font=\Large] (locx-v2) {$1$};

%%Location y
%\path (locx.south) + (0,-1.5) node (locf1) {$[\loc_{f_1}] \mapsto$};
%\matrix(locf1cells) [version list, text width=7mm, anchor=west]
   %at ([xshift=10pt]locf1.east) {
 %{a} & $\tsid_0$ & {a} & $\tsid_1$\\
  %{a} & $\{\tsid_1\}$ & {a} & $\emptyset$\\
%};
%\node[version node, fit=(locf1cells-1-1) (locf1cells-2-1), fill=white, inner sep= 0cm, font=\Large] (locf1-v0) {$0$};
%\node[version node, fit=(locf1cells-1-3) (locf1cells-2-3), fill=white, inner sep= 0cm, font=\Large] (locf1-v1) {$1$};

%%Location y
%\path (locf1.south) + (0,-1.5) node (locf2) {$[\loc_{f_2}] \mapsto$};
%\matrix(locf2cells) [version list, text width=7mm, anchor=west]
   %at ([xshift=10pt]locf2.east) {
 %{a} & $\tsid_0$ & {a} & $\tsid_2$\\
  %{a} & $\{\tsid_2\}$ & {a} & $\emptyset$ \\
%};
%\node[version node, fit=(locf2cells-1-1) (locf2cells-2-1), fill=white, inner sep= 0cm, font=\Large] (locf2-v0) {$0$};
%\node[version node, fit=(locf2cells-1-3) (locf2cells-2-3), fill=white, inner sep= 0cm, font=\Large] (locf2-v1) {$1$};

%% \draw[-, red, very thick, rounded corners] ([xshift=-5pt, yshift=5pt]locx-v1.north east) |- 
%%  ($([xshift=-5pt,yshift=-5pt]locx-v1.south east)!.5!([xshift=-5pt, yshift=5pt]locy-v0.north east)$) -| ([xshift=-5pt, yshift=5pt]locy-v0.south east);

%%blue view - I should  check whether I can use pgfkeys to just declare the list of locations, and then add the view automatically.
%\draw[-, blue, very thick, rounded corners=10pt]
 %([xshift=-2pt, yshift=20pt]locx-v0.north east) node (tid1start) {} -- 
%% ([xshift=-2pt, yshift=-5pt]locx-v0.south east) --
%% ([xshift=-2pt, yshift=5pt]locf1-v0.north east) -- 
%% ([xshift=-2pt, yshift=-5pt]locf1-v0.south east) --
%% ([xshift=-2pt, yshift=5pt]locf2-v0.north east) -- 
 %([xshift=-2pt, yshift=-5pt]locf2-v0.south east);
 
 %\path (tid1start) node[anchor=south, rectangle, fill=blue!20, draw=blue, font=\small, inner sep=1pt] {$\thid_3$};

%%red view
%\draw[-, red, very thick, rounded corners = 10pt]
 %([xshift=-5pt, yshift=5pt]locx-v2.north east) -- 
 %([xshift=-5pt, yshift=-5pt]locx-v2.south east) --
 %([xshift=-5pt, yshift=5pt]locf1-v0.north east) -- 
 %([xshift=-5pt, yshift=-5pt]locf1-v0.south east) --
 %([xshift=-5pt, yshift=5pt]locf2-v1.north east) -- 
 %([xshift=-5pt, yshift=-10pt]locf2-v1.south east) node (tid2start) {};
 
%\path (tid2start) node[anchor=north, rectangle, fill=red!20, draw=red, font=\small, inner sep=1pt] {$\thid_2$};
 
 %%green view
%\draw[-, DarkGreen, very thick, rounded corners = 10pt]
 %([xshift=-16pt, yshift=8pt]locx-v1.north east) node (tid3start) {}-- 
 %([xshift=-16pt, yshift=-5pt]locx-v1.south east) --
 %([xshift=-16pt, yshift=5pt]locf1-v1.north east) -- 
 %([xshift=-16pt, yshift=-5pt]locf1-v1.south east) --
 %([xshift=-16pt, yshift=5pt]locf2-v0.north east) -- 
 %([xshift=-16pt, yshift=-5pt]locf2-v0.south east);
 
 %\path (tid3start) node[anchor=south, rectangle, fill=DarkGreen!20, draw=DarkGreen, font=\small, inner sep=1pt] {$\thid_1$};

%\end{pgfonlayer}
%\end{tikzpicture}
%%
%&
%%
%\begin{tikzpicture}[font=\large]

%\begin{pgfonlayer}{foreground}
%%Uncomment line below for help lines
%%\draw[help lines] grid(5,4);

%%Location x
%\node(locx) at (1,3) {$[\loc_x] \mapsto$};

%\matrix(locxcells) [version list, column 2/.style = {text width=14mm}, text width=7mm, anchor=west]
   %at ([xshift=10pt]locx.east) {
 %{a} & $\tsid_0$ & {a} & $\tsid_1$ & {a} & $\tsid_2$\\
  %{a} & $\{\tsid_1, \tsid_2\}$ & {a} & $\emptyset$ & {a} & $\emptyset$\\
%};
%\node[version node, fit=(locxcells-1-1) (locxcells-2-1), fill=white, inner sep= 0cm, font=\Large] (locx-v0) {$0$};
%\node[version node, fit=(locxcells-1-3) (locxcells-2-3), fill=white, inner sep= 0cm, font=\Large] (locx-v1) {$1$};
%\node[version node, fit=(locxcells-1-5) (locxcells-2-5), fill=white, inner sep= 0cm, font=\Large] (locx-v2) {$1$};

%%Location y
%\path (locx.south) + (0,-1.5) node (locf1) {$[\loc_{f_1}] \mapsto$};
%\matrix(locf1cells) [version list, text width=7mm, anchor=west]
   %at ([xshift=10pt]locf1.east) {
 %{a} & $\tsid_0$ & {a} & $\tsid_1$\\
  %{a} & $\{\tsid_1\}$ & {a} & $\emptyset$\\
%};
%\node[version node, fit=(locf1cells-1-1) (locf1cells-2-1), fill=white, inner sep= 0cm, font=\Large] (locf1-v0) {$0$};
%\node[version node, fit=(locf1cells-1-3) (locf1cells-2-3), fill=white, inner sep= 0cm, font=\Large] (locf1-v1) {$1$};

%%Location y
%\path (locf1.south) + (0,-1.5) node (locf2) {$[\loc_{f_2}] \mapsto$};
%\matrix(locf2cells) [version list, text width=7mm, anchor=west]
   %at ([xshift=10pt]locf2.east) {
 %{a} & $\tsid_0$ & {a} & $\tsid_2$\\
  %{a} & $\{\tsid_2\}$ & {a} & $\emptyset$ \\
%};
%\node[version node, fit=(locf2cells-1-1) (locf2cells-2-1), fill=white, inner sep= 0cm, font=\Large] (locf2-v0) {$0$};
%\node[version node, fit=(locf2cells-1-3) (locf2cells-2-3), fill=white, inner sep= 0cm, font=\Large] (locf2-v1) {$1$};

%% \draw[-, red, very thick, rounded corners] ([xshift=-5pt, yshift=5pt]locx-v1.north east) |- 
%%  ($([xshift=-5pt,yshift=-5pt]locx-v1.south east)!.5!([xshift=-5pt, yshift=5pt]locy-v0.north east)$) -| ([xshift=-5pt, yshift=5pt]locy-v0.south east);

%%blue view - I should  check whether I can use pgfkeys to just declare the list of locations, and then add the view automatically.
%\draw[-, blue, very thick, rounded corners=10pt]
 %([xshift=-2pt, yshift=20pt]locx-v2.north east) node (tid1start) {} -- 
 %([xshift=-2pt, yshift=-7pt]locx-v2.south east) --
 %([xshift=-2pt, yshift=3pt]locf1-v1.north east) -- 
 %([xshift=-2pt, yshift=-5pt]locf1-v1.south east) --
 %([xshift=-2pt, yshift=5pt]locf2-v1.north east) -- 
 %([xshift=-2pt, yshift=-5pt]locf2-v1.south east);
 
 %\path (tid1start) node[anchor=south, rectangle, fill=blue!20, draw=blue, font=\small, inner sep=1pt] {$\thid_3$};

%%red view
%\draw[-, red, very thick, rounded corners = 10pt]
 %([xshift=-5pt, yshift=5pt]locx-v2.north east) -- 
 %([xshift=-5pt, yshift=-5pt]locx-v2.south east) --
 %([xshift=-5pt, yshift=5pt]locf1-v0.north east) -- 
 %([xshift=-5pt, yshift=-5pt]locf1-v0.south east) --
 %([xshift=-5pt, yshift=5pt]locf2-v1.north east) -- 
 %([xshift=-5pt, yshift=-10pt]locf2-v1.south east) node (tid2start) {};
 
%\path (tid2start) node[anchor=north, rectangle, fill=red!20, draw=red, font=\small, inner sep=1pt] {$\thid_2$};
 
 %%green view
%\draw[-, DarkGreen, very thick, rounded corners = 10pt]
 %([xshift=-16pt, yshift=8pt]locx-v1.north east) node (tid3start) {}-- 
 %([xshift=-16pt, yshift=-5pt]locx-v1.south east) --
 %([xshift=-16pt, yshift=5pt]locf1-v1.north east) -- 
 %([xshift=-16pt, yshift=-5pt]locf1-v1.south east) --
 %([xshift=-16pt, yshift=5pt]locf2-v0.north east) -- 
 %([xshift=-16pt, yshift=-5pt]locf2-v0.south east);
 
 %\path (tid3start) node[anchor=south, rectangle, fill=DarkGreen!20, draw=DarkGreen, font=\small, inner sep=1pt] {$\thid_1$};

%\end{pgfonlayer}
%\end{tikzpicture}\\
%{\small (c)} & {\small (d)} \\
%\hline
%\begin{tikzpicture}[font=\large]

%\begin{pgfonlayer}{foreground}
%%Uncomment line below for help lines
%%\draw[help lines] grid(5,4);

%%Location x
%\node(locx) at (1,3) {$[\loc_x] \mapsto$};

%\matrix(locxcells) [version list, text width=7mm, anchor=west]
   %at ([xshift=10pt]locx.east) {
 %{a} & $\tsid_0$ & {a} & $\tsid_1$\\
  %{a} & $\{\tsid_1\}$ & {a} & $\emptyset$\\
%};
%\node[version node, fit=(locxcells-1-1) (locxcells-2-1), fill=white, inner sep= 0cm, font=\Large] (locx-v0) {$0$};
%\node[version node, fit=(locxcells-1-3) (locxcells-2-3), fill=white, inner sep= 0cm, font=\Large] (locx-v1) {$1$};

%%Location y
%\path (locx.south) + (0,-1.5) node (locf1) {$[\loc_{f_1}] \mapsto$};
%\matrix(locf1cells) [version list, text width=7mm, anchor=west]
   %at ([xshift=10pt]locf1.east) {
 %{a} & $\tsid_0$ & {a} & $\tsid_1$\\
  %{a} & $\{\tsid_1\}$ & {a} & $\emptyset$\\
%};
%\node[version node, fit=(locf1cells-1-1) (locf1cells-2-1), fill=white, inner sep= 0cm, font=\Large] (locf1-v0) {$0$};
%\node[version node, fit=(locf1cells-1-3) (locf1cells-2-3), fill=white, inner sep= 0cm, font=\Large] (locf1-v1) {$1$};

%%Location y
%\path (locf1.south) + (0,-1.5) node (locf2) {$[\loc_{f_2}] \mapsto$};
%\matrix(locf2cells) [version list, text width=7mm, anchor=west]
   %at ([xshift=10pt]locf2.east) {
 %{a} & $T_0$ \\
  %{a} & $\emptyset$ \\
%};
%\node[version node, fit=(locf2cells-1-1) (locf2cells-2-1), fill=white, inner sep= 0cm, font=\Large] (locf2-v0) {$0$};

%% \draw[-, red, very thick, rounded corners] ([xshift=-5pt, yshift=5pt]locx-v1.north east) |- 
%%  ($([xshift=-5pt,yshift=-5pt]locx-v1.south east)!.5!([xshift=-5pt, yshift=5pt]locy-v0.north east)$) -| ([xshift=-5pt, yshift=5pt]locy-v0.south east);

%%blue view - I should  check whether I can use pgfkeys to just declare the list of locations, and then add the view automatically.
%\draw[-, blue, very thick, rounded corners=10pt]
 %([xshift=-2pt, yshift=20pt]locx-v0.north east) node (tid1start) {} -- 
%% ([xshift=-2pt, yshift=-5pt]locx-v0.south east) --
%% ([xshift=-2pt, yshift=5pt]locf1-v0.north east) -- 
%% ([xshift=-2pt, yshift=-5pt]locf1-v0.south east) --
%% ([xshift=-2pt, yshift=5pt]locf2-v0.north east) -- 
 %([xshift=-2pt, yshift=-5pt]locf2-v0.south east);
 
 %\path (tid1start) node[anchor=south, rectangle, fill=blue!20, draw=blue, font=\small, inner sep=1pt] {$\thid_3$};

%%red view
%\draw[-, red, very thick, rounded corners = 10pt]
 %([xshift=-5pt, yshift=5pt]locx-v1.north east) -- 
%% ([xshift=-5pt, yshift=-5pt]locx-v0.south east) --
%% ([xshift=-5pt, yshift=5pt]locf1-v0.north east) -- 
 %([xshift=-5pt, yshift=-5pt]locf1-v1.south east) --
 %([xshift=-5pt, yshift=5pt]locf2-v0.north east) -- 
 %([xshift=-5pt, yshift=-10pt]locf2-v0.south east) node (tid2start) {};
 
%\path (tid2start) node[anchor=north, rectangle, fill=red!20, draw=red, font=\small, inner sep=1pt] {$\thid_2$};
 
 %%green view
%\draw[-, DarkGreen, very thick, rounded corners = 10pt]
 %([xshift=-16pt, yshift=8pt]locx-v1.north east) node (tid3start) {}-- 
%% ([xshift=-16pt, yshift=-5pt]locx-v0.south east) --
%% ([xshift=-16pt, yshift=5pt]locf1-v0.north east) -- 
 %([xshift=-16pt, yshift=-5pt]locf1-v1.south east) --
 %([xshift=-16pt, yshift=5pt]locf2-v0.north east) -- 
 %([xshift=-16pt, yshift=-5pt]locf2-v0.south east);
 
 %\path (tid3start) node[anchor=south, rectangle, fill=DarkGreen!20, draw=DarkGreen, font=\small, inner sep=1pt] {$\thid_1$};

%\end{pgfonlayer}
%\end{tikzpicture}
%&
%\begin{tikzpicture}[font=\large]

%\begin{pgfonlayer}{foreground}
%%Uncomment line below for help lines
%%\draw[help lines] grid(5,4);

%%Location x
%\node(locx) at (1,3) {$[\loc_x] \mapsto$};

%\matrix(locxcells) [version list, text width=7mm, anchor=west]
   %at ([xshift=10pt]locx.east) {
 %{a} & $\tsid_0$ & {a} & $\tsid_1$ & {a} & $\tsid_2$\\
  %{a} & $\{\tsid_1\}$ & {a} & $\{\tsid_2\}$ & {a} & $\emptyset$\\
%};
%\node[version node, fit=(locxcells-1-1) (locxcells-2-1), fill=white, inner sep= 0cm, font=\Large] (locx-v0) {$0$};
%\node[version node, fit=(locxcells-1-3) (locxcells-2-3), fill=white, inner sep= 0cm, font=\Large] (locx-v1) {$1$};
%\node[version node, fit=(locxcells-1-5) (locxcells-2-5), fill=white, inner sep= 0cm, font=\Large] (locx-v2) {$2$};

%%Location y
%\path (locx.south) + (0,-1.5) node (locf1) {$[\loc_{f_1}] \mapsto$};
%\matrix(locf1cells) [version list, text width=7mm, anchor=west]
   %at ([xshift=10pt]locf1.east) {
 %{a} & $\tsid_0$ & {a} & $\tsid_1$\\
  %{a} & $\{\tsid_1\}$ & {a} & $\emptyset$\\
%};
%\node[version node, fit=(locf1cells-1-1) (locf1cells-2-1), fill=white, inner sep= 0cm, font=\Large] (locf1-v0) {$0$};
%\node[version node, fit=(locf1cells-1-3) (locf1cells-2-3), fill=white, inner sep= 0cm, font=\Large] (locf1-v1) {$1$};

%%Location y
%\path (locf1.south) + (0,-1.5) node (locf2) {$[\loc_{f_2}] \mapsto$};
%\matrix(locf2cells) [version list, text width=7mm, anchor=west]
   %at ([xshift=10pt]locf2.east) {
 %{a} & $T_0$ & {a} & $\tsid_2$\\
  %{a} & $\emptyset$ & {a} & $\emptyset$ \\
%};
%\node[version node, fit=(locf2cells-1-1) (locf2cells-2-1), fill=white, inner sep= 0cm, font=\Large] (locf2-v0) {$0$};
%\node[version node, fit=(locf2cells-1-3) (locf2cells-2-3), fill=white, inner sep= 0cm, font=\Large] (locf2-v1) {$1$};

%% \draw[-, red, very thick, rounded corners] ([xshift=-5pt, yshift=5pt]locx-v1.north east) |- 
%%  ($([xshift=-5pt,yshift=-5pt]locx-v1.south east)!.5!([xshift=-5pt, yshift=5pt]locy-v0.north east)$) -| ([xshift=-5pt, yshift=5pt]locy-v0.south east);

%%blue view - I should  check whether I can use pgfkeys to just declare the list of locations, and then add the view automatically.
%\draw[-, blue, very thick, rounded corners=10pt]
 %([xshift=-2pt, yshift=20pt]locx-v0.north east) node (tid1start) {} -- 
%% ([xshift=-2pt, yshift=-5pt]locx-v0.south east) --
%% ([xshift=-2pt, yshift=5pt]locf1-v0.north east) -- 
%% ([xshift=-2pt, yshift=-5pt]locf1-v0.south east) --
%% ([xshift=-2pt, yshift=5pt]locf2-v0.north east) -- 
 %([xshift=-2pt, yshift=-5pt]locf2-v0.south east);
 
 %\path (tid1start) node[anchor=south, rectangle, fill=blue!20, draw=blue, font=\small, inner sep=1pt] {$\thid_3$};

%%red view
%\draw[-, red, very thick, rounded corners = 10pt]
 %([xshift=-5pt, yshift=5pt]locx-v2.north east) -- 
 %([xshift=-5pt, yshift=-5pt]locx-v2.south east) --
 %([xshift=-5pt, yshift=5pt]locf1-v1.north east) -- 
%% ([xshift=-5pt, yshift=-5pt]locf1-v0.south east) --
%% ([xshift=-5pt, yshift=5pt]locf2-v0.north east) -- 
 %([xshift=-5pt, yshift=-10pt]locf2-v1.south east) node (tid2start) {};
 
%\path (tid2start) node[anchor=north, rectangle, fill=red!20, draw=red, font=\small, inner sep=1pt] {$\thid_2$};
 
 %%green view
%\draw[-, DarkGreen, very thick, rounded corners = 10pt]
 %([xshift=-16pt, yshift=8pt]locx-v1.north east) node (tid3start) {}-- 
%% ([xshift=-16pt, yshift=-5pt]locx-v0.south east) --
%% ([xshift=-16pt, yshift=5pt]locf1-v0.north east) -- 
 %([xshift=-16pt, yshift=-5pt]locf1-v1.south east) --
 %([xshift=-16pt, yshift=5pt]locf2-v0.north east) -- 
 %([xshift=-16pt, yshift=-5pt]locf2-v0.south east);
 
 %\path (tid3start) node[anchor=south, rectangle, fill=DarkGreen!20, draw=DarkGreen, font=\small, inner sep=1pt] {$\thid_1$};

%\end{pgfonlayer}
%\end{tikzpicture}\\
%{\small(e)} & {\small(f)}\\
%\hline
%\end{tabular}
%\caption{History heaps obtained in a execution of $\prog_3$.}
%\label{fig:cu.exec}
%\end{figure}
\ac{This Consistency Model shows why the notion of consistent views must 
depend on the set of operations that need to be executed.}

The next consistency model that we consider is \emph{Update Atomic}. 
Although we did not find any implementation of this model, it has been proposed in \cite{framework-concur} as a strengthening to Read Atomic to avoid write-write conflicts.
This model states that: \textbf{(i)} transactions satisfy atomic visibility (\defref{def:readatomic}); and \textbf{(ii)} transactions writing to one same addressees cannot be executed concurrently.
\sx{ This appears too earlier:
Update Atomic is also needed to specify more sophisticated consistency models, 
such as \emph{Parallel Snapshot Isolation} and \emph{Snapshot Isolation}.}
\ac{Check: Nobi said he was interested in implementing Update Atomic 
at some point, maybe he ended up doing something.}

Programs under Update Atomic do not exhibit the \emph{lost update} anomaly: two or more transactions update the same address, for example , both increment its value by $1$, but only one of them will be observed by future transactions, for example, only one of the increments takes effect.
To illustrate the lost update anomaly, consider the following program where two transactions concurrently increment $\vx$ and the third transaction read the value. 
Note that the \( \pvar{f1} \) and \( \pvar{f2} \) are two flags indicating the corresponding transactions has been committed.
\ac{Intuitive behaviour of the litmus test: two transactions concurrently increment $[\loc_x]$. 
 A third transaction observes that the first two transactions have been executed. 
 However, it only observes one of the two increments taking place.
 }
\[
    \prog_3 \equiv \begin{session}
        \begin{array}{@{}c || c || c@{}}
        \txid_1 : 
        \begin{transaction} 
            \pmutate{\pvar{f1}}{1};\\
            \pderef{\pvar{a}}{\vx};\\
            \pmutate{\vx}{a + 1};\\
        \end{transaction} & 
        \txid_2 : 
        \begin{transaction}
            \pmutate{\pvar{f2}}{1};\\
            \pderef{\pvar{a}}{\vx};\\
            \pmutate{\vx}{a + 1};\\
        \end{transaction} &
        \txid_3 : 
        \begin{transaction}
            \pderef{\pvar{a}}{\vx};\\
            \pderef{\pvar{b}}{\pvar{f1}};\\
            \pderef{\pvar{c}}{\pvar{f2}};\\
            \pifs{\pvar{a}=1 \wedge \pvar{b}=1 \wedge \pvar{c} = 1}\\ 
                \quad \passign{\retvar}{\Large \frownie{}}
            \pife
        \end{transaction}
        \end{array}
    \end{session}
 \]

\sx{Did not check the example explanation}
We consider an execution in which the transactions contained in the code of threads 
$\thid_1, \thid_2$ both execute w.r.t. the snapshot determined by the initial view. 
Initially, the configuration of the program coincides with the one given in 
Figure \ref{fig:cu.exec}(a). After executing the transaction of $\thid_1$, the resulting 
configuration is the one depicted in Figure \ref{fig:cu.exec}(b). Next, after $\thid_2$ executed 
its own transaction, we obtain the configuration in \ref{fig:cu.exec}(c). Note that 
in this configuration, both transactions $\txid_1, \txid_2$ read the initial version 
for location $[\loc_x]$. Finally, thread $\thid_3$ updates its view to include the most recent 
version for all the locations $[\loc_x], [\loc_{f_1}], [\loc_{f_2}]$. When executing the 
transaction in its code, all the locations will be found to have value $1$, and 
the return variable will be set to ${\Large \frownie{}}$.

The program $\prog_3$ might exhibit the lost-update anomaly because when the second transaction $\txid_2$ starts, its view did not include the most up-to-date version for address $\vx$ provided that \( \txid_{2}\) will update the address \( \vx \).
\sx{
However, the same transaction installed a new version for $[\loc_x]$. That is, 
}
As consequence, the database \emph{lost the update} of a version of \( \vx \) installed by the transaction $\txid_1$, in a sense that no transaction will observe such the version.
To forbid this anomaly, the \emph{update atomic} requires that if a transaction writes to an address, the transactions must start with a view including the most recent version for the address.
\sx{
then it must have been executed 
in a snapshot obtained from a view including the most recent version of 
$[\loc_x]$.
}

\begin{defn}
\label{def:update-atomic}
\emph{Update atomic} is stronger than then read atomic (\defref{def:readatomic}) by further requiring for all addresses written, it should starts with including the most recent version for those addresses,
\[
\begin{rclarray}
 (\hh, \vi) \csat[\mathsf{UA}] \opset: \vi' & \defeq &
 \begin{array}[t]{@{}l}
 (\hh, \vi) \csat[\mathsf{RA}] \opset: \vi' \land \fora{\addr} \\ 
 \quad (\otW, \addr, \stub) \in \opset \implies \vi(\addr)  = \left| \hh(\addr) \right|
 \end{array} \\
\end{rclarray}
\]
\sx{
$(\hh, V) \triangleright_{\mathsf{UA}} \mathcal{O}: V'$ iff 
$(\hh, V) \triangleright_{\mathsf{RA}} \mathcal{O}: V'$, and 
for all locations $[\loc_x]$ such that $(\otW\; [\loc_x] : \_ \in \mathcal{O})$ 
then $V([\loc_{x}]) = \lvert \hh([\loc_x]) \rvert - 1$. 
}
\end{defn}

Note that in $prog_3$, under $\mathsf{UA}$, 
we cannot execute the transaction of $\thid_2$ starting from 
the configuration depicted in Figure \ref{fig:cu.exec}(b). This 
is because the view of $\thid_2$, in such a configuration, does 
not include the most recent version for location $[\loc_x]$. 
Instead, before executing its transaction $\thid_2$ must update 
its view to include the most recent version of $[\loc_x]$. The 
resulting configuration is depicted in Figure \ref{fig:cu.exec}(e). 
After executing its transaction using the snapshot obtained 
from the view of $\thid_2$ in this configuration, the resulting 
configuration is the one obtained in Figure \ref{fig:cu.exec}(f). 
There are now three different possible configurations in which 
$\thid_3$ can execute its transaction: 
\begin{itemize}
\item the initial one, in which case the value $0$ will be observed 
for the three locations $[\loc_x], [\loc_{f1}]$ and $[\loc_{f_2}]$. 
In this case the transaction will not return value ${\Large \frownie{}}$, 
\item one in which the view of $\thid_3$ for $[\loc_x]$ points to the 
version $(1, \tsid_1, \{\tsid_2\})$. Because of atomic visibility, 
it must also be the case that the same view for location $[\loc_{f_2}]$ 
does not include the update performed by $\tsid_2$ (otherwise, 
the view for $[\loc_x]$ should point to its most recent version, which 
was also installed by $\tsid_2$. Thus in this case the transaction executes 
w.r.t a snapshot where $\loc_{f_2} = 0$, and the value ${\Large \frownie{}}$ 
will not be returned, 
\item one in which the view of $\thid_3$ for $[\loc_x]$ points to 
its most recent version $(2, \tsid_2, \emptyset)$; also in this case, 
the value ${\Large \frownie{}}$ will not be returned from the transaction.
\end{itemize}


\subsection{Consistent Prefix} 
%\begin{figure}
%\begin{center}
%\begin{tabular}{|@{}c|c@{}|}
%\hline
%\begin{tikzpicture}[font=\large]

%\begin{pgfonlayer}{foreground}
%%Uncomment line below for help lines
%%\draw[help lines] grid(5,4);

%%Location x
%\node(locx) at (1,3) {$[\loc_x] \mapsto$};

%\matrix(locxcells) [version list, text width=7mm, anchor=west]
   %at ([xshift=10pt]locx.east) {
 %{a} & $T_0$ \\
  %{a} & $\emptyset$ \\
%};
%\node[version node, fit=(locxcells-1-1) (locxcells-2-1), fill=white, inner sep= 0cm, font=\Large] (locx-v0) {$0$};
%%\node[version node, fit=(locxcells-1-3) (locxcells-2-3), fill=white, inner sep=0cm, font=\Large] (locx-v1) {$1$};

%%Location y
%\path (locx.south) + (0,-1.5) node (locy) {$[\loc_y] \mapsto$};
%\matrix(locycells) [version list, text width=7mm, anchor=west]
   %at ([xshift=10pt]locy.east) {
 %{a} & $T_0$ \\
  %{a} & $\emptyset$ \\
%};
%\node[version node, fit=(locycells-1-1) (locycells-2-1), fill=white, inner sep= 0cm, font=\Large] (locy-v0) {$0$};
%%\node[version node, fit=(locycells-1-3) (locycells-2-3), fill=white, inner sep=0cm, font=\Large] (locy-v1) {$1$};

%% \draw[-, red, very thick, rounded corners] ([xshift=-5pt, yshift=5pt]locx-v1.north east) |- 
%%  ($([xshift=-5pt,yshift=-5pt]locx-v1.south east)!.5!([xshift=-5pt, yshift=5pt]locy-v0.north east)$) -| ([xshift=-5pt, yshift=5pt]locy-v0.south east);

%%blue view - I should  check whether I can use pgfkeys to just declare the list of locations, and then add the view automatically.
%\draw[-, blue, very thick, rounded corners=10pt]
 %([xshift=-3pt, yshift=20pt]locx-v0.north east) node (tid1start) {} -- 
%% ([xshift=-2pt, yshift=-5pt]locx-v0.south east) --
%% ([xshift=-2pt, yshift=5pt]locy-v0.north east) -- 
 %([xshift=-3pt, yshift=-5pt]locy-v0.south east);
 
 %\path (tid1start) node[anchor=south, rectangle, fill=blue!20, draw=blue, font=\small, inner sep=1pt] {$\thid_1$};

%%red view
%\draw[-, red, very thick, rounded corners = 10pt]
 %([xshift=-16pt, yshift=5pt]locx-v0.north east) node (tid2start) {}-- 
%% ([xshift=-8pt, yshift=-5pt]locx-v0.south east) --
%% ([xshift=-8pt, yshift=5pt]locy-v0.north east) -- 
 %([xshift=-16pt, yshift=-5pt]locy-v0.south east) node {};
 
%\path (tid2start) node[anchor=south, rectangle, fill=red!20, draw=red, font=\small, inner sep=1pt] {$\thid_2$};

%%%Stack for threads tid_1 and tid_2
%%
%%\draw[-, dashed] let 
%%   \p1 = ([xshift=0pt]locy.west),
%%   \p2 = ([yshift=-5pt]locycells.south),
%%   \p3 = ([xshift=10pt]locycells.east) in
%%   (\x1, \y2) -- (\x3, \y2);
%%   
%%\matrix(stacks) [
%%   matrix of nodes,
%%   anchor=north, 
%%   text=blue, 
%%   font=\normalsize, 
%%   row 1/.style = {text = blue}, 
%%   row 2/.style = {text = red}, 
%%   text width= 13mm ] 
%%   at ([xshift=-10pt,yshift=-8pt]locycells.south) {
%%   $\thid_1:$ & $\retvar = 0$\\
%%   $\thid_2:$ & $\retvar = 0$\\
%%   };
%\end{pgfonlayer}
%\end{tikzpicture} 
%%
%&
%%
%\begin{tikzpicture}[font=\large]

%\begin{pgfonlayer}{foreground}
%%Uncomment line below for help lines
%%\draw[help lines] grid(5,4);

%%Location x
%\node(locx) at (1,3) {$[\loc_x] \mapsto$};

%\matrix(locxcells) [version list, text width=7mm, anchor=west]
   %at ([xshift=10pt]locx.east) {
 %{a} & $T_0$ & {a} & $\tsid_1$\\
  %{a} & $\emptyset$ & {a} & $\emptyset$ \\
%};
%\node[version node, fit=(locxcells-1-1) (locxcells-2-1), fill=white, inner sep= 0cm, font=\Large] (locx-v0) {$0$};
%\node[version node, fit=(locxcells-1-3) (locxcells-2-3), fill=white, inner sep=0cm, font=\Large] (locx-v1) {$1$};

%%Location y
%\path (locx.south) + (0,-1.5) node (locy) {$[\loc_y] \mapsto$};
%\matrix(locycells) [version list, text width=7mm, anchor=west]
   %at ([xshift=10pt]locy.east) {
 %{a} & $T_0$ \\
  %{a} & $\emptyset$ \\
%};
%\node[version node, fit=(locycells-1-1) (locycells-2-1), fill=white, inner sep= 0cm, font=\Large] (locy-v0) {$0$};
%%\node[version node, fit=(locycells-1-3) (locycells-2-3), fill=white, inner sep=0cm, font=\Large] (locy-v1) {$1$};

%% \draw[-, red, very thick, rounded corners] ([xshift=-5pt, yshift=5pt]locx-v1.north east) |- 
%%  ($([xshift=-5pt,yshift=-5pt]locx-v1.south east)!.5!([xshift=-5pt, yshift=5pt]locy-v0.north east)$) -| ([xshift=-5pt, yshift=5pt]locy-v0.south east);

%%blue view - I should  check whether I can use pgfkeys to just declare the list of locations, and then add the view automatically.
%\draw[-, blue, very thick, rounded corners=10pt]
 %([xshift=-3pt, yshift=20pt]locx-v1.north east) node (tid1start) {} -- 
 %([xshift=-3pt, yshift=-5pt]locx-v1.south east) --
 %([xshift=-3pt, yshift=5pt]locy-v0.north east) -- 
 %([xshift=-3pt, yshift=-5pt]locy-v0.south east);
 
 %\path (tid1start) node[anchor=south, rectangle, fill=blue!20, draw=blue, font=\small, inner sep=1pt] {$\thid_1$};

%%red view
%\draw[-, red, very thick, rounded corners = 10pt]
 %([xshift=-16pt, yshift=5pt]locx-v0.north east) node (tid2start) {}-- 
%% ([xshift=-8pt, yshift=-5pt]locx-v0.south east) --
%% ([xshift=-8pt, yshift=5pt]locy-v0.north east) -- 
 %([xshift=-16pt, yshift=-5pt]locy-v0.south east) node {};
 
%\path (tid2start) node[anchor=south, rectangle, fill=red!20, draw=red, font=\small, inner sep=1pt] {$\thid_2$};

%%%Stack for threads tid_1 and tid_2
%%
%%\draw[-, dashed] let 
%%   \p1 = ([xshift=0pt]locy.west),
%%   \p2 = ([yshift=-5pt]locycells.south),
%%   \p3 = ([xshift=10pt]locycells.east) in
%%   (\x1, \y2) -- (\x3, \y2);
%%   
%%\matrix(stacks) [
%%   matrix of nodes,
%%   anchor=north, 
%%   text=blue, 
%%   font=\normalsize, 
%%   row 1/.style = {text = blue}, 
%%   row 2/.style = {text = red}, 
%%   text width= 13mm ] 
%%   at ([xshift=-10pt,yshift=-8pt]locycells.south) {
%%   $\thid_1:$ & $\retvar = 0$\\
%%   $\thid_2:$ & $\retvar = 0$\\
%%   };
%\end{pgfonlayer}
%\end{tikzpicture}
%\\
%{\small(a)} & {\small(b)}\\
%\hline

%\begin{pgfonlayer}{foreground}
%%Uncomment line below for help lines
%%\draw[help lines] grid(5,4);

%%Location x
%\node(locx) at (1,3) {$[\loc_x] \mapsto$};

%\matrix(locxcells) [version list, text width=7mm, anchor=west]
   %at ([xshift=10pt]locx.east) {
 %{a} & $T_0$ & {a} & $\tsid_1$\\
  %{a} & $\emptyset$ & {a} & $\emptyset$ \\
%};
%\node[version node, fit=(locxcells-1-1) (locxcells-2-1), fill=white, inner sep= 0cm, font=\Large] (locx-v0) {$0$};
%\node[version node, fit=(locxcells-1-3) (locxcells-2-3), fill=white, inner sep=0cm, font=\Large] (locx-v1) {$1$};

%%Location y
%\path (locx.south) + (0,-1.5) node (locy) {$[\loc_y] \mapsto$};
%\matrix(locycells) [version list, text width=7mm, anchor=west]
   %at ([xshift=10pt]locy.east) {
 %{a} & $T_0$ & {a} & $\tsid_2$ \\
  %{a} & $\emptyset$ & {a} & $\emptyset$\\
%};
%\node[version node, fit=(locycells-1-1) (locycells-2-1), fill=white, inner sep= 0cm, font=\Large] (locy-v0) {$0$};
%\node[version node, fit=(locycells-1-3) (locycells-2-3), fill=white, inner sep=0cm, font=\Large] (locy-v1) {$1$};

%% \draw[-, red, very thick, rounded corners] ([xshift=-5pt, yshift=5pt]locx-v1.north east) |- 
%%  ($([xshift=-5pt,yshift=-5pt]locx-v1.south east)!.5!([xshift=-5pt, yshift=5pt]locy-v0.north east)$) -| ([xshift=-5pt, yshift=5pt]locy-v0.south east);

%%blue view - I should  check whether I can use pgfkeys to just declare the list of locations, and then add the view automatically.
%\draw[-, blue, very thick, rounded corners=10pt]
 %([xshift=-3pt, yshift=20pt]locx-v1.north east) node (tid1start) {} -- 
 %([xshift=-3pt, yshift=-5pt]locx-v1.south east) --
 %([xshift=-3pt, yshift=5pt]locy-v0.north east) -- 
 %([xshift=-3pt, yshift=-5pt]locy-v0.south east);
 
 %\path (tid1start) node[anchor=south, rectangle, fill=blue!20, draw=blue, font=\small, inner sep=1pt] {$\thid_1$};

%%red view
%\draw[-, red, very thick, rounded corners = 10pt]
 %([xshift=-16pt, yshift=5pt]locx-v0.north east) node (tid2start) {}-- 
 %([xshift=-16pt, yshift=-5pt]locx-v0.south east) --
 %([xshift=-16pt, yshift=5pt]locy-v1.north east) -- 
 %([xshift=-16pt, yshift=-5pt]locy-v1.south east) node {};
 
%\path (tid2start) node[anchor=south, rectangle, fill=red!20, draw=red, font=\small, inner sep=1pt] {$\thid_2$};

%%%Stack for threads tid_1 and tid_2
%%
%%\draw[-, dashed] let 
%%   \p1 = ([xshift=0pt]locy.west),
%%   \p2 = ([yshift=-5pt]locycells.south),
%%   \p3 = ([xshift=10pt]locycells.east) in
%%   (\x1, \y2) -- (\x3, \y2);
%%   
%%\matrix(stacks) [
%%   matrix of nodes,
%%   anchor=north, 
%%   text=blue, 
%%   font=\normalsize, 
%%   row 1/.style = {text = blue}, 
%%   row 2/.style = {text = red}, 
%%   text width= 13mm ] 
%%   at ([xshift=-10pt,yshift=-8pt]locycells.south) {
%%   $\thid_1:$ & $\retvar = 0$\\
%%   $\thid_2:$ & $\retvar = 0$\\
%%   };
%\end{pgfonlayer}
%\end{tikzpicture}
%%
%&
%%
%\begin{tikzpicture}[font=\large]

%\begin{pgfonlayer}{foreground}
%%Uncomment line below for help lines
%%\draw[help lines] grid(5,4);

%%Location x
%\node(locx) at (1,3) {$[\loc_x] \mapsto$};

%\matrix(locxcells) [version list, text width=7mm, anchor=west]
   %at ([xshift=10pt]locx.east) {
 %{a} & $T_0$ & {a} & $\tsid_1$\\
  %{a} & $\{\tsid_4\}$ & {a} & $\emptyset$ \\
%};
%\node[version node, fit=(locxcells-1-1) (locxcells-2-1), fill=white, inner sep= 0cm, font=\Large] (locx-v0) {$0$};
%\node[version node, fit=(locxcells-1-3) (locxcells-2-3), fill=white, inner sep=0cm, font=\Large] (locx-v1) {$1$};

%%Location y
%\path (locx.south) + (0,-1.5) node (locy) {$[\loc_y] \mapsto$};
%\matrix(locycells) [version list, text width=7mm, anchor=west]
   %at ([xshift=10pt]locy.east) {
 %{a} & $T_0$ & {a} & $\tsid_2$ \\
  %{a} & $\{\tsid_3\}$ & {a} & $\emptyset$\\
%};
%\node[version node, fit=(locycells-1-1) (locycells-2-1), fill=white, inner sep= 0cm, font=\Large] (locy-v0) {$0$};
%\node[version node, fit=(locycells-1-3) (locycells-2-3), fill=white, inner sep=0cm, font=\Large] (locy-v1) {$1$};

%% \draw[-, red, very thick, rounded corners] ([xshift=-5pt, yshift=5pt]locx-v1.north east) |- 
%%  ($([xshift=-5pt,yshift=-5pt]locx-v1.south east)!.5!([xshift=-5pt, yshift=5pt]locy-v0.north east)$) -| ([xshift=-5pt, yshift=5pt]locy-v0.south east);

%%blue view - I should  check whether I can use pgfkeys to just declare the list of locations, and then add the view automatically.
%\draw[-, blue, very thick, rounded corners=10pt]
 %([xshift=-3pt, yshift=20pt]locx-v1.north east) node (tid1start) {} -- 
%% ([xshift=-3pt, yshift=-5pt]locx-v1.south east) --
%% ([xshift=-3pt, yshift=5pt]locy-v0.north east) -- 
 %([xshift=-3pt, yshift=-5pt]locy-v1.south east);
 
 %\path (tid1start) node[anchor=south, rectangle, fill=blue!20, draw=blue, font=\small, inner sep=1pt] {$\thid_1$};

%%red view
%\draw[-, red, very thick, rounded corners = 10pt]
 %([xshift=-16pt, yshift=5pt]locx-v1.north east) node (tid2start) {}-- 
%% ([xshift=-16pt, yshift=-5pt]locx-v0.south east) --
%% ([xshift=-16pt, yshift=5pt]locy-v1.north east) -- 
 %([xshift=-16pt, yshift=-5pt]locy-v1.south east) node {};
 
%\path (tid2start) node[anchor=south, rectangle, fill=red!20, draw=red, font=\small, inner sep=1pt] {$\thid_2$};

%%%Stack for threads tid_1 and tid_2
%%
%%\draw[-, dashed] let 
%%   \p1 = ([xshift=0pt]locy.west),
%%   \p2 = ([yshift=-5pt]locycells.south),
%%   \p3 = ([xshift=10pt]locycells.east) in
%%   (\x1, \y2) -- (\x3, \y2);
%%   
%%\matrix(stacks) [
%%   matrix of nodes,
%%   anchor=north, 
%%   text=blue, 
%%   font=\normalsize, 
%%   row 1/.style = {text = blue}, 
%%   row 2/.style = {text = red}, 
%%   text width= 13mm ] 
%%   at ([xshift=-10pt,yshift=-8pt]locycells.south) {
%%   $\thid_1:$ & $\retvar = 0$\\
%%   $\thid_2:$ & $\retvar = 0$\\
%%   };
%\end{pgfonlayer}
%\end{tikzpicture}
%\\
%{\small(c)} & {\small(d)}\\
%\hline
%\begin{tikzpicture}[font=\large]

%\begin{pgfonlayer}{foreground}
%%Uncomment line below for help lines
%%\draw[help lines] grid(5,4);

%%Location x
%\node(locx) at (1,3) {$[\loc_x] \mapsto$};

%\matrix(locxcells) [version list, text width=7mm, anchor=west]
   %at ([xshift=10pt]locx.east) {
 %{a} & $T_0$ & {a} & $\tsid_1$\\
  %{a} & $\emptyset$ & {a} & $\emptyset$ \\
%};
%\node[version node, fit=(locxcells-1-1) (locxcells-2-1), fill=white, inner sep= 0cm, font=\Large] (locx-v0) {$0$};
%\node[version node, fit=(locxcells-1-3) (locxcells-2-3), fill=white, inner sep=0cm, font=\Large] (locx-v1) {$1$};

%%Location y
%\path (locx.south) + (0,-1.5) node (locy) {$[\loc_y] \mapsto$};
%\matrix(locycells) [version list, text width=7mm, anchor=west]
   %at ([xshift=10pt]locy.east) {
 %{a} & $T_0$ & {a} & $\tsid_2$ \\
  %{a} & $\emptyset$ & {a} & $\emptyset$\\
%};
%\node[version node, fit=(locycells-1-1) (locycells-2-1), fill=white, inner sep= 0cm, font=\Large] (locy-v0) {$0$};
%\node[version node, fit=(locycells-1-3) (locycells-2-3), fill=white, inner sep=0cm, font=\Large] (locy-v1) {$1$};

%% \draw[-, red, very thick, rounded corners] ([xshift=-5pt, yshift=5pt]locx-v1.north east) |- 
%%  ($([xshift=-5pt,yshift=-5pt]locx-v1.south east)!.5!([xshift=-5pt, yshift=5pt]locy-v0.north east)$) -| ([xshift=-5pt, yshift=5pt]locy-v0.south east);

%%blue view - I should  check whether I can use pgfkeys to just declare the list of locations, and then add the view automatically.
%\draw[-, blue, very thick, rounded corners=10pt]
 %([xshift=-3pt, yshift=20pt]locx-v1.north east) node (tid1start) {} -- 
 %([xshift=-3pt, yshift=-5pt]locx-v1.south east) --
 %([xshift=-3pt, yshift=5pt]locy-v0.north east) -- 
 %([xshift=-3pt, yshift=-5pt]locy-v0.south east);
 
 %\path (tid1start) node[anchor=south, rectangle, fill=blue!20, draw=blue, font=\small, inner sep=1pt] {$\thid_1$};

%%red view
%\draw[-, red, very thick, rounded corners = 10pt]
 %([xshift=-16pt, yshift=5pt]locx-v1.north east) node (tid2start) {}-- 
%% ([xshift=-16pt, yshift=-5pt]locx-v0.south east) --
%% ([xshift=-16pt, yshift=5pt]locy-v1.north east) -- 
 %([xshift=-16pt, yshift=-5pt]locy-v1.south east) node {};
 
%\path (tid2start) node[anchor=south, rectangle, fill=red!20, draw=red, font=\small, inner sep=1pt] {$\thid_2$};

%%%Stack for threads tid_1 and tid_2
%%
%%\draw[-, dashed] let 
%%   \p1 = ([xshift=0pt]locy.west),
%%   \p2 = ([yshift=-5pt]locycells.south),
%%   \p3 = ([xshift=10pt]locycells.east) in
%%   (\x1, \y2) -- (\x3, \y2);
%%   
%%\matrix(stacks) [
%%   matrix of nodes,
%%   anchor=north, 
%%   text=blue, 
%%   font=\normalsize, 
%%   row 1/.style = {text = blue}, 
%%   row 2/.style = {text = red}, 
%%   text width= 13mm ] 
%%   at ([xshift=-10pt,yshift=-8pt]locycells.south) {
%%   $\thid_1:$ & $\retvar = 0$\\
%%   $\thid_2:$ & $\retvar = 0$\\
%%   };
%\end{pgfonlayer}
%\end{tikzpicture}
%&
%\begin{tikzpicture}[font=\large]

%\begin{pgfonlayer}{foreground}
%%Uncomment line below for help lines
%%\draw[help lines] grid(5,4);

%%Location x
%\node(locx) at (1,3) {$[\loc_x] \mapsto$};

%\matrix(locxcells) [version list, text width=7mm, anchor=west]
   %at ([xshift=10pt]locx.east) {
 %{a} & $T_0$ & {a} & $\tsid_1$\\
  %{a} & $\emptyset$ & {a} & $\{\tsid_4\}$ \\
%};
%\node[version node, fit=(locxcells-1-1) (locxcells-2-1), fill=white, inner sep= 0cm, font=\Large] (locx-v0) {$0$};
%\node[version node, fit=(locxcells-1-3) (locxcells-2-3), fill=white, inner sep=0cm, font=\Large] (locx-v1) {$1$};

%%Location y
%\path (locx.south) + (0,-1.5) node (locy) {$[\loc_y] \mapsto$};
%\matrix(locycells) [version list, text width=7mm, anchor=west]
   %at ([xshift=10pt]locy.east) {
 %{a} & $T_0$ & {a} & $\tsid_2$ \\
  %{a} & $\{\tsid_3\}$ & {a} & $\emptyset$\\
%};
%\node[version node, fit=(locycells-1-1) (locycells-2-1), fill=white, inner sep= 0cm, font=\Large] (locy-v0) {$0$};
%\node[version node, fit=(locycells-1-3) (locycells-2-3), fill=white, inner sep=0cm, font=\Large] (locy-v1) {$1$};

%% \draw[-, red, very thick, rounded corners] ([xshift=-5pt, yshift=5pt]locx-v1.north east) |- 
%%  ($([xshift=-5pt,yshift=-5pt]locx-v1.south east)!.5!([xshift=-5pt, yshift=5pt]locy-v0.north east)$) -| ([xshift=-5pt, yshift=5pt]locy-v0.south east);

%%blue view - I should  check whether I can use pgfkeys to just declare the list of locations, and then add the view automatically.
%\draw[-, blue, very thick, rounded corners=10pt]
 %([xshift=-3pt, yshift=20pt]locx-v1.north east) node (tid1start) {} -- 
%% ([xshift=-3pt, yshift=-5pt]locx-v1.south east) --
%% ([xshift=-3pt, yshift=5pt]locy-v0.north east) -- 
 %([xshift=-3pt, yshift=-5pt]locy-v1.south east);
 
 %\path (tid1start) node[anchor=south, rectangle, fill=blue!20, draw=blue, font=\small, inner sep=1pt] {$\thid_1$};

%%red view
%\draw[-, red, very thick, rounded corners = 10pt]
 %([xshift=-16pt, yshift=5pt]locx-v1.north east) node (tid2start) {}-- 
%% ([xshift=-16pt, yshift=-5pt]locx-v0.south east) --
%% ([xshift=-16pt, yshift=5pt]locy-v1.north east) -- 
 %([xshift=-16pt, yshift=-5pt]locy-v1.south east) node {};
 
%\path (tid2start) node[anchor=south, rectangle, fill=red!20, draw=red, font=\small, inner sep=1pt] {$\thid_2$};

%%%Stack for threads tid_1 and tid_2
%%
%%\draw[-, dashed] let 
%%   \p1 = ([xshift=0pt]locy.west),
%%   \p2 = ([yshift=-5pt]locycells.south),
%%   \p3 = ([xshift=10pt]locycells.east) in
%%   (\x1, \y2) -- (\x3, \y2);
%%   
%%\matrix(stacks) [
%%   matrix of nodes,
%%   anchor=north, 
%%   text=blue, 
%%   font=\normalsize, 
%%   row 1/.style = {text = blue}, 
%%   row 2/.style = {text = red}, 
%%   text width= 13mm ] 
%%   at ([xshift=-10pt,yshift=-8pt]locycells.south) {
%%   $\thid_1:$ & $\retvar = 0$\\
%%   $\thid_2:$ & $\retvar = 0$\\
%%   };
%\end{pgfonlayer}
%\end{tikzpicture}
%\\
%{\small(e)} & {\small(f)}\\
%\hline
%\end{tabular}
%\end{center}
%\caption{Configurations obtained throughout an execution of 
%$\prog_4$.}
%\label{fig:cp.exec}
%\end{figure}

The next consistency model that we illustrate is \emph{consistent prefix}. 
It ensures that once a thread observes the effects of some transaction $\txid$, it also observes all the transactions that were committed before $\txid$. 
\sx{why different locations}
Another way to express this property is that two different transactions never observe the updates to different addresses in a different order.

Consider the program $\prog_4$ below,
 \[
    \prog_4 \equiv  
    \left( 
    \begin{session} 
        \begin{array}{@{}c || c @{}}
            \begin{session} 
            \txid_1 : 
            \begin{transaction}
                \pmutate{\vx}{1};\\
            \end{transaction}; \\
            
            \txid_3 :
            \begin{transaction}
              	\pderef{\pvar{a}}{\vy};\\
              	\pifs{\pvar{a}=0}\\
                    \quad \passign{\retvar}{\Large \frownie{}} 
                \pife 
            \end{transaction}
            \end{session}
            &
            \begin{session}
            \txid_2 :
            \begin{transaction}
                \pmutate{\vy}{1};\\
            \end{transaction} ; \\
            
            \txid_4 :
            \begin{transaction}
              	\pderef{\pvar{a}}{\vx};\\
              	\pifs{\pvar{a}=0}\\
              		\quad \passign{\retvar}{\Large \frownie{}} 
                \pife
            \end{transaction}
            \end{session}
        \end{array}
    \end{session}
    \right)
 \]

We argue that, under $\mathsf{RA}$, it is possible to obtain an execution 
of program $\prog_4$ where both $\ptrans{\trans_2}$ and $\ptrans{\trans_4}$ 
return value ${\Large \frownie{}}$.
Such an execution can be summarised as follows: 

\begin{itemize}
\item initially, $\thid_1$ executes transaction $\ptrans{\trans_1}$, 
leading to the configuration of Figure \ref{fig:cp.exec}(b), and the program 
$\ptrans{\trans_2} \ppar \ptrans{\trans_3} ; \ptrans{\trans_4}$ to be 
executed, 
\item then $\thid_2$ executes transaction $\ptrans{\trans_3}$, leading 
to the configuration of Figure \ref{fig:cp.exec}(c); the remaining 
program to be executed is $\ptrans{\trans_3} \ppar \ptrans{\trans_4}$, 
\item without changing its view, $\thid_1$ executes transaction $\ptrans{\trans_2}$. 
The code $\trans_2$ is executed using the heap $[ [\loc_x] \mapsto 1, [\loc_y] \mapsto 0]$ 
as a snapshot; this means that, by executing $\trans_2$, the variable $\retvar$ of the thread-local 
stack of $\thid_1$ is set to ${\Large \frownie{}}$. Next, the thread $\thid_2$ executes $\ptrans{\trans_4}$ 
without altering its view. Similarly to the execution of $\ptrans{\trans_2}$ in $\thid_1$, this causes the 
$\retvar$ variable of the thread-local stack of $\thid_2$ to be set to ${\Large \frownie{}}$. At this point, 
the final configuration of the program is the one given in Figure \ref{fig:cp.exec}(d).
\end{itemize}

\sx{What is different update ?}
To avoid different threads to observe different updates in different orders, we impose a constraint known as \emph{consistent prefix}.
\sx{ why centralised? 
in a centralised database, where 
transactions have a start and a commit point, 
}
Assuming a transaction has a start and commit point, \emph{consistent prefix} requires that if a transaction $\txid_1$ observes the effects of another transaction $\txid_2$, then it must observe the effects of any other transaction that committed before $\txid_2$.
In the history heaps framework, transactions are executed in a single step in the semantics,
\sx{
this constraint can be formalised as follows: 
\emph{If a transaction $\tsid_1$ observes the effects of another transaction $\tsid_2$, then it must 
observe the effects of any transaction that committed before $\tsid_2$.}
}
In the history heaps framework, transactions are executed in a single step and the step corresponds to the commit order.
Thus, upon the read atomic (\defref{def:readatomic}), \emph{consistent prefix} requires once a thread commits a transaction, it pushes the view to be up-to-date with the state of the database, so the following transactions from the thread will observe the effects of all transactions committed before \( \txid \) included.
\sx{It is easy to see the definition match the intuition when inside a thread, but not easy to see cross-thread. There is something I feel subtle but dont know what it is and I feel it is actually already included in the definition.}

\sx{ dont understand:
In the history heaps framework, transactions are executed in a single step; 
however, one may think of the order in which transactions execute in our 
operational semantics to be consistent with the order in which 
transactions commit (this correspondence will be made precise later in 
the document, when we will relate executions in our operational semantics 
to abstract executions used in the declarative style for specifications of 
consistency models). By requiring that, after a thread $\thid$ executes 
a transaction $\tsid$, it pushes its own view to be up-to-date with the state of 
the system, we model the fact that any future transaction executed 
by $\thid$ will observe the effects of anything that committed before 
$\tsid$ (included).
}

\begin{defn}[consistent prefix]
\label{def:consistent-prefix}
The \emph{consistent prefix} is stronger than read atomic (\defref{def:readatomic}) by further requiring the view after pushes to the latest for all addresses, 
\[
\begin{rclarray}
    (\hh, \vi) \csat[\mathsf{CP}] \opset: \vi' & \defeq &  (\hh, \vi) \csat[\mathsf{RA}] \opset: \vi' \land \fora{\addr \in \dom(\hh)} \vi'(\addr) = \left| \hh(\addr) \right| \\
\end{rclarray}
\]
\sx{
We say that $(hh, V) \triangleright_{\mathsf{CP}} \mathcal{O}: V'$ 
if and only if $(\hh, V) \triangleright_{\mathsf{RA}} \mathcal{O}: V'$, 
and for any location $[\loc_x]$, $V'([\loc_x]) = \lvert \mathsf{HHeapUpdate}(\hh, V, \mathcal{O}) \rvert -1$. 
}
\end{defn}


\sx{did not read} 
Consider again th program $\prog_4$, ths time to be executed 
under $\mathsf{CP}$. We argue that in this case it is not possible to have both 
threads $\thid_1$ and $\thid_2$ to set the value of $\retvar$ to ${\Large \frownie{}}$. 
The initial configuration in which the program $\prog_4$ is executed is 
the one depicted in Figure \ref{fig:cp.exec}(a). Initially, either thread $\thid_1$ executes 
the code $\ptrans{\trans_1}$, or thread $\thid_2$ executes transaction $\ptrans{\trans_3}$; without 
loss of generality, we consider the former option (the case in which $\thid_2$ executes 
first is symmetric). Upon executing the code $\ptrans{\trans_1}$, we obtain the 
configuration of Figure \ref{fig:cp.exec}(b), with the program $\ptrans{\trans_2} \ppar (\ptrans{\trans_3} ; \ptrans{\trans_4})$ 
to be executed next.
At this point, note that under $\mathsf{CP}$ it is not possible for $\thid_2$ to execute $\ptrans{\trans_3}$ and obtain 
the configuration of Figure \ref{fig:cp.exec}(c) as a result. This is because, in $\mathsf{CP}$, we require the view of $\thid_2$ 
\textbf{after} executing $\ptrans{\trans_3}$ to be up-to-date, i.e. to point to the last location of each version. That is, 
assuming that $\thid_2$ executes $\ptrans{\trans_3}$ next, we obtain the configuration of Figure \ref{fig:cp.exec}(e). 
From this point on, whenever $\thid_2$ will execute transaction $\ptrans{\trans_4}$, it will read value $1$ for 
location $[\loc_x]$, hence it won't be able to set the value of $\retvar$ to ${\Large \frownie{}}$. One possible 
final configuration for the program is given in Figure \ref{fig:cp.exec}(f).

%\begin{definition}
%$(\hh, V, \mathcal{V}) \triangleright_{\mathsf{CP}} \mathcal{O}$ iff 
%$(\hh, V, \mathcal{V}) \triangleright_{\mathsf{RA}} \mathcal{O}$, 
%and for any $V' \in \mathcal{V}$, either $V' \sqsubseteq V$ or 
%$V \sqsubseteq V'$.
%\end{definition}
%\ac{I found that this is a very easy way to encode consistent 
%prefix. In English, a thread can execute a transaction if its view does 
%not cross with the views of any other thread.}

%\ac{Very Important - note to self: It seems that the condition of 
%requiring that views do not cross before executing a transaction 
%does not suffice to model snapshot isolation. In fact, it seems that 
%consistent prefix (when transaction are limited to either one read 
%or one write) coincides with TSO.
%Update, it seems that the condition that I need for Snapshot Isolation 
%(besides write confict detection) is that, after you execute a transaction, 
%you bring your view up-to-date. (So here I have to concede that I was wrong, 
%and the state of the view after you execute a transaction is actually important).\\
%
%}

\subsection{Parallel Snapshot Isolation and Snapshot Isolation}
\sx{Need some citations here, what is geo-replicated, distributed database vs distributed system}
\emph{Snapshot Isolation} (SI) is a consistency model that has been widely employed in both centralised and distributed databases. 
Because snapshot isolation does not scale well to geo-replicated and distributed systems, a weaker model called \emph{Parallel Snapshot Isolation} (PSI) has been recently proposed. 

\sx{
Both SI and PSI can be specified in the history heap framework by combining the consistency models that we have already introduced. 
In short, SI combines atomic visibility and the snapshot monotonicity property from consistent prefix property (if a transaction $\tsid_1$ 
observes the effects of another transaction $\tsid_2$, then it also observes 
the effects of any transaction that committed before), and the write-conflict 
detection from Update Atomic (two committing transactions do not write 
concurrently to the same location). In contrast, PSI only requires atomic 
visibility, causal consistency and write-conflict detection. Formally, we have 
the following:
}
Both SI and PSI can be specified in the history heap framework by combining the consistency models that we have already introduced. 
In short, SI combines consistent prefix property and update atomic, while PSI only requires causal consistency and update atomic.
\begin{defn}
The \emph{parallel snapshot isolation} combines causal consistency (\defref{def:causal})  and update atomic (\defref{def:update-atomic}): $\mathsf{PSI} = \mathsf{CC} \cap \mathsf{UA}$.
The {snapshot isolation} combines consistent prefix (\defref{def:consistent-prefix}) and update atomic (\defref{def:update-atomic}): $\mathsf{SI} = \mathsf{CP} \cap \mathsf{UA}$.
\end{defn}

\begin{lem}
SI is stronger than PSI.
\end{lem}

\subsection{Serialisability}
\sx{
Serialisability is the last and strongest consistency 
model that we consider. Informally, under serialisability 
transactions appear to be executed in a sequential order. 
Consider for example the program 
}
\emph{Serialisability} is  the strongest consistency model that we consider. 
Informally, under serialisability transactions appear in a sequential order. 
Consider the following program know as \emph{write skew},
\[
\begin{session}
\begin{array}{@{}c || c@{}}
\begin{transaction}
\pderef{\pvar{a}}{\vy};\\
\pifs{\pvar{a} = 0};\\
    \quad \pmutate{\vx}{1};\\
    \quad \passign{\retvar}{\Large \frownie{}};
\pife
\end{transaction}
&
\begin{transaction}
\pderef{\pvar{a}}{\vx};\\
\pifs{\pvar{a} = 0};\\
\quad \pmutate{\vy}{1};\\
\quad \passign{\retvar}{\Large \frownie{}};
\pife
\end{transaction}
\end{array}
\end{session}
\]


%\begin{figure}
%\begin{center}
%\begin{tabular}{|@{}c|c@{}|}
%\hline
%\begin{tikzpicture}[font=\large]

%\begin{pgfonlayer}{foreground}
%%Uncomment line below for help lines
%%\draw[help lines] grid(5,4);

%%Location x
%\node(locx) at (1,3) {$[\loc_x] \mapsto$};

%\matrix(locxcells) [version list, text width=7mm, anchor=west]
   %at ([xshift=10pt]locx.east) {
 %{a} & $T_0$ \\
  %{a} & $\emptyset$ \\
%};
%\node[version node, fit=(locxcells-1-1) (locxcells-2-1), fill=white, inner sep= 0cm, font=\Large] (locx-v0) {$0$};
%%\node[version node, fit=(locxcells-1-3) (locxcells-2-3), fill=white, inner sep=0cm, font=\Large] (locx-v1) {$1$};

%%Location y
%\path (locx.south) + (0,-1.5) node (locy) {$[\loc_y] \mapsto$};
%\matrix(locycells) [version list, text width=7mm, anchor=west]
   %at ([xshift=10pt]locy.east) {
 %{a} & $T_0$ \\
  %{a} & $\emptyset$ \\
%};
%\node[version node, fit=(locycells-1-1) (locycells-2-1), fill=white, inner sep= 0cm, font=\Large] (locy-v0) {$0$};
%%\node[version node, fit=(locycells-1-3) (locycells-2-3), fill=white, inner sep=0cm, font=\Large] (locy-v1) {$1$};

%% \draw[-, red, very thick, rounded corners] ([xshift=-5pt, yshift=5pt]locx-v1.north east) |- 
%%  ($([xshift=-5pt,yshift=-5pt]locx-v1.south east)!.5!([xshift=-5pt, yshift=5pt]locy-v0.north east)$) -| ([xshift=-5pt, yshift=5pt]locy-v0.south east);

%%blue view - I should  check whether I can use pgfkeys to just declare the list of locations, and then add the view automatically.
%\draw[-, blue, very thick, rounded corners=10pt]
 %([xshift=-3pt, yshift=20pt]locx-v0.north east) node (tid1start) {} -- 
%% ([xshift=-2pt, yshift=-5pt]locx-v0.south east) --
%% ([xshift=-2pt, yshift=5pt]locy-v0.north east) -- 
 %([xshift=-3pt, yshift=-5pt]locy-v0.south east);
 
 %\path (tid1start) node[anchor=south, rectangle, fill=blue!20, draw=blue, font=\small, inner sep=1pt] {$\thid_1$};

%%red view
%\draw[-, red, very thick, rounded corners = 10pt]
 %([xshift=-16pt, yshift=5pt]locx-v0.north east) node (tid2start) {}-- 
%% ([xshift=-8pt, yshift=-5pt]locx-v0.south east) --
%% ([xshift=-8pt, yshift=5pt]locy-v0.north east) -- 
 %([xshift=-16pt, yshift=-5pt]locy-v0.south east) node {};
 
%\path (tid2start) node[anchor=south, rectangle, fill=red!20, draw=red, font=\small, inner sep=1pt] {$\thid_2$};

%%%Stack for threads tid_1 and tid_2
%%
%%\draw[-, dashed] let 
%%   \p1 = ([xshift=0pt]locy.west),
%%   \p2 = ([yshift=-5pt]locycells.south),
%%   \p3 = ([xshift=10pt]locycells.east) in
%%   (\x1, \y2) -- (\x3, \y2);
%%   
%%\matrix(stacks) [
%%   matrix of nodes,
%%   anchor=north, 
%%   text=blue, 
%%   font=\normalsize, 
%%   row 1/.style = {text = blue}, 
%%   row 2/.style = {text = red}, 
%%   text width= 13mm ] 
%%   at ([xshift=-10pt,yshift=-8pt]locycells.south) {
%%   $\thid_1:$ & $\retvar = 0$\\
%%   $\thid_2:$ & $\retvar = 0$\\
%%   };
%\end{pgfonlayer}
%\end{tikzpicture} 
%%
%&
%%
%\begin{tikzpicture}[font=\large]

%\begin{pgfonlayer}{foreground}
%%Uncomment line below for help lines
%%\draw[help lines] grid(5,4);

%%Location x
%\node(locx) at (1,3) {$[\loc_x] \mapsto$};

%\matrix(locxcells) [version list, text width=7mm, anchor=west]
   %at ([xshift=10pt]locx.east) {
 %{a} & $\tsid_0$ & {a} & $\tsid_1$\\
  %{a} & $\emptyset$ & {a} & $\emptyset$ \\
%};
%\node[version node, fit=(locxcells-1-1) (locxcells-2-1), fill=white, inner sep= 0cm, font=\Large] (locx-v0) {$0$};
%\node[version node, fit=(locxcells-1-3) (locxcells-2-3), fill=white, inner sep=0cm, font=\Large] (locx-v1) {$1$};

%%Location y
%\path (locx.south) + (0,-1.5) node (locy) {$[\loc_y] \mapsto$};
%\matrix(locycells) [version list, text width=7mm, anchor=west]
   %at ([xshift=10pt]locy.east) {
 %{a} & $T_0$ \\
  %{a} & $\{\tsid_1\}$ \\
%};
%\node[version node, fit=(locycells-1-1) (locycells-2-1), fill=white, inner sep= 0cm, font=\Large] (locy-v0) {$0$};
%%\node[version node, fit=(locycells-1-3) (locycells-2-3), fill=white, inner sep=0cm, font=\Large] (locy-v1) {$1$};

%% \draw[-, red, very thick, rounded corners] ([xshift=-5pt, yshift=5pt]locx-v1.north east) |- 
%%  ($([xshift=-5pt,yshift=-5pt]locx-v1.south east)!.5!([xshift=-5pt, yshift=5pt]locy-v0.north east)$) -| ([xshift=-5pt, yshift=5pt]locy-v0.south east);

%%blue view - I should  check whether I can use pgfkeys to just declare the list of locations, and then add the view automatically.
%\draw[-, blue, very thick, rounded corners=10pt]
 %([xshift=-3pt, yshift=20pt]locx-v1.north east) node (tid1start) {} -- 
 %([xshift=-3pt, yshift=-5pt]locx-v1.south east) --
 %([xshift=-3pt, yshift=5pt]locy-v0.north east) -- 
 %([xshift=-3pt, yshift=-5pt]locy-v0.south east);
 
 %\path (tid1start) node[anchor=south, rectangle, fill=blue!20, draw=blue, font=\small, inner sep=1pt] {$\thid_1$};

%%red view
%\draw[-, red, very thick, rounded corners = 10pt]
 %([xshift=-16pt, yshift=5pt]locx-v0.north east) node (tid2start) {}-- 
%% ([xshift=-8pt, yshift=-5pt]locx-v0.south east) --
%% ([xshift=-8pt, yshift=5pt]locy-v0.north east) -- 
 %([xshift=-16pt, yshift=-5pt]locy-v0.south east) node {};
 
%\path (tid2start) node[anchor=south, rectangle, fill=red!20, draw=red, font=\small, inner sep=1pt] {$\thid_2$};

%%%Stack for threads tid_1 and tid_2
%%
%%\draw[-, dashed] let 
%%   \p1 = ([xshift=0pt]locy.west),
%%   \p2 = ([yshift=-5pt]locycells.south),
%%   \p3 = ([xshift=10pt]locycells.east) in
%%   (\x1, \y2) -- (\x3, \y2);
%%   
%%\matrix(stacks) [
%%   matrix of nodes,
%%   anchor=north, 
%%   text=blue, 
%%   font=\normalsize, 
%%   row 1/.style = {text = blue}, 
%%   row 2/.style = {text = red}, 
%%   text width= 13mm ] 
%%   at ([xshift=-10pt,yshift=-8pt]locycells.south) {
%%   $\thid_1:$ & $\retvar = 0$\\
%%   $\thid_2:$ & $\retvar = 0$\\
%%   };
%\end{pgfonlayer}
%\end{tikzpicture}
%\\
%{\small(a)} & {\small(b)}\\
%\hline
%\begin{tikzpicture}[font=\large]

%\begin{pgfonlayer}{foreground}
%%Uncomment line below for help lines
%%\draw[help lines] grid(5,4);

%%Location x
%\node(locx) at (1,3) {$[\loc_x] \mapsto$};

%\matrix(locxcells) [version list, text width=7mm, anchor=west]
   %at ([xshift=10pt]locx.east) {
 %{a} & $T_0$ & {a} & $\tsid_1$\\
  %{a} & $\{\tsid_2\}$ & {a} & $\emptyset$ \\
%};
%\node[version node, fit=(locxcells-1-1) (locxcells-2-1), fill=white, inner sep= 0cm, font=\Large] (locx-v0) {$0$};
%\node[version node, fit=(locxcells-1-3) (locxcells-2-3), fill=white, inner sep=0cm, font=\Large] (locx-v1) {$1$};

%%Location y
%\path (locx.south) + (0,-1.5) node (locy) {$[\loc_y] \mapsto$};
%\matrix(locycells) [version list, text width=7mm, anchor=west]
   %at ([xshift=10pt]locy.east) {
 %{a} & $T_0$ & {a} & $\tsid_2$ \\
  %{a} & $\{\tsid_1\}$ & {a} & $\emptyset$\\
%};
%\node[version node, fit=(locycells-1-1) (locycells-2-1), fill=white, inner sep= 0cm, font=\Large] (locy-v0) {$0$};
%\node[version node, fit=(locycells-1-3) (locycells-2-3), fill=white, inner sep=0cm, font=\Large] (locy-v1) {$1$};

%% \draw[-, red, very thick, rounded corners] ([xshift=-5pt, yshift=5pt]locx-v1.north east) |- 
%%  ($([xshift=-5pt,yshift=-5pt]locx-v1.south east)!.5!([xshift=-5pt, yshift=5pt]locy-v0.north east)$) -| ([xshift=-5pt, yshift=5pt]locy-v0.south east);

%%blue view - I should  check whether I can use pgfkeys to just declare the list of locations, and then add the view automatically.
%\draw[-, blue, very thick, rounded corners=10pt]
 %([xshift=-3pt, yshift=20pt]locx-v1.north east) node (tid1start) {} -- 
 %([xshift=-3pt, yshift=-5pt]locx-v1.south east) --
 %([xshift=-3pt, yshift=5pt]locy-v0.north east) -- 
 %([xshift=-3pt, yshift=-5pt]locy-v0.south east);
 
 %\path (tid1start) node[anchor=south, rectangle, fill=blue!20, draw=blue, font=\small, inner sep=1pt] {$\thid_1$};

%%red view
%\draw[-, red, very thick, rounded corners = 10pt]
 %([xshift=-16pt, yshift=5pt]locx-v1.north east) node (tid2start) {}-- 
%% ([xshift=-16pt, yshift=-5pt]locx-v0.south east) --
%% ([xshift=-16pt, yshift=5pt]locy-v1.north east) -- 
 %([xshift=-16pt, yshift=-5pt]locy-v1.south east) node {};
 
%\path (tid2start) node[anchor=south, rectangle, fill=red!20, draw=red, font=\small, inner sep=1pt] {$\thid_2$};

%%%Stack for threads tid_1 and tid_2
%%
%%\draw[-, dashed] let 
%%   \p1 = ([xshift=0pt]locy.west),
%%   \p2 = ([yshift=-5pt]locycells.south),
%%   \p3 = ([xshift=10pt]locycells.east) in
%%   (\x1, \y2) -- (\x3, \y2);
%%   
%%\matrix(stacks) [
%%   matrix of nodes,
%%   anchor=north, 
%%   text=blue, 
%%   font=\normalsize, 
%%   row 1/.style = {text = blue}, 
%%   row 2/.style = {text = red}, 
%%   text width= 13mm ] 
%%   at ([xshift=-10pt,yshift=-8pt]locycells.south) {
%%   $\thid_1:$ & $\retvar = 0$\\
%%   $\thid_2:$ & $\retvar = 0$\\
%%   };
%\end{pgfonlayer}
%\end{tikzpicture}
%%
%&
%%
%\begin{tikzpicture}[font=\large]

%\begin{pgfonlayer}{foreground}
%%Uncomment line below for help lines
%%\draw[help lines] grid(5,4);

%%Location x
%\node(locx) at (1,3) {$[\loc_x] \mapsto$};

%\matrix(locxcells) [version list, text width=7mm, anchor=west]
   %at ([xshift=10pt]locx.east) {
 %{a} & $\tsid_0$ & {a} & $\tsid_1$\\
  %{a} & $\emptyset$ & {a} & $\emptyset$ \\
%};
%\node[version node, fit=(locxcells-1-1) (locxcells-2-1), fill=white, inner sep= 0cm, font=\Large] (locx-v0) {$0$};
%\node[version node, fit=(locxcells-1-3) (locxcells-2-3), fill=white, inner sep=0cm, font=\Large] (locx-v1) {$1$};

%%Location y
%\path (locx.south) + (0,-1.5) node (locy) {$[\loc_y] \mapsto$};
%\matrix(locycells) [version list, text width=7mm, anchor=west]
   %at ([xshift=10pt]locy.east) {
 %{a} & $T_0$ \\
  %{a} & $\{\tsid_1\}$ \\
%};
%\node[version node, fit=(locycells-1-1) (locycells-2-1), fill=white, inner sep= 0cm, font=\Large] (locy-v0) {$0$};
%%\node[version node, fit=(locycells-1-3) (locycells-2-3), fill=white, inner sep=0cm, font=\Large] (locy-v1) {$1$};

%% \draw[-, red, very thick, rounded corners] ([xshift=-5pt, yshift=5pt]locx-v1.north east) |- 
%%  ($([xshift=-5pt,yshift=-5pt]locx-v1.south east)!.5!([xshift=-5pt, yshift=5pt]locy-v0.north east)$) -| ([xshift=-5pt, yshift=5pt]locy-v0.south east);

%%blue view - I should  check whether I can use pgfkeys to just declare the list of locations, and then add the view automatically.
%\draw[-, blue, very thick, rounded corners=10pt]
 %([xshift=-3pt, yshift=20pt]locx-v1.north east) node (tid1start) {} -- 
 %([xshift=-3pt, yshift=-5pt]locx-v1.south east) --
 %([xshift=-3pt, yshift=5pt]locy-v0.north east) -- 
 %([xshift=-3pt, yshift=-5pt]locy-v0.south east);
 
 %\path (tid1start) node[anchor=south, rectangle, fill=blue!20, draw=blue, font=\small, inner sep=1pt] {$\thid_1$};

%%red view
%\draw[-, red, very thick, rounded corners = 10pt]
 %([xshift=-16pt, yshift=5pt]locx-v1.north east) node (tid2start) {}-- 
 %([xshift=-16pt, yshift=-3pt]locx-v1.south east) --
 %([xshift=-16pt, yshift=7pt]locy-v0.north east) -- 
 %([xshift=-16pt, yshift=-5pt]locy-v0.south east) node {};
 
%\path (tid2start) node[anchor=south, rectangle, fill=red!20, draw=red, font=\small, inner sep=1pt] {$\thid_2$};

%%%Stack for threads tid_1 and tid_2
%%
%%\draw[-, dashed] let 
%%   \p1 = ([xshift=0pt]locy.west),
%%   \p2 = ([yshift=-5pt]locycells.south),
%%   \p3 = ([xshift=10pt]locycells.east) in
%%   (\x1, \y2) -- (\x3, \y2);
%%   
%%\matrix(stacks) [
%%   matrix of nodes,
%%   anchor=north, 
%%   text=blue, 
%%   font=\normalsize, 
%%   row 1/.style = {text = blue}, 
%%   row 2/.style = {text = red}, 
%%   text width= 13mm ] 
%%   at ([xshift=-10pt,yshift=-8pt]locycells.south) {
%%   $\thid_1:$ & $\retvar = 0$\\
%%   $\thid_2:$ & $\retvar = 0$\\
%%   };
%\end{pgfonlayer}
%\end{tikzpicture}
%\\
%{\small(c)} & {\small(d)}\\
%\hline
%\multicolumn{2}{|c|}{
%\begin{tikzpicture}[font=\large]

%\begin{pgfonlayer}{foreground}
%%Uncomment line below for help lines
%%\draw[help lines] grid(5,4);

%%Location x
%\node(locx) at (1,3) {$[\loc_x] \mapsto$};

%\matrix(locxcells) [version list, text width=7mm, anchor=west]
   %at ([xshift=10pt]locx.east) {
 %{a} \pgfmatrixnextcell $\tsid_0$ \pgfmatrixnextcell {a} \pgfmatrixnextcell $\tsid_1$\\
  %{a} \pgfmatrixnextcell $\emptyset$ \pgfmatrixnextcell {a} \pgfmatrixnextcell $\{\tsid_2\}$ \\
%};
%\node[version node, fit=(locxcells-1-1) (locxcells-2-1), fill=white, inner sep= 0cm, font=\Large] (locx-v0) {$0$};
%\node[version node, fit=(locxcells-1-3) (locxcells-2-3), fill=white, inner sep=0cm, font=\Large] (locx-v1) {$1$};

%%Location y
%\path (locx.south) + (0,-1.5) node (locy) {$[\loc_y] \mapsto$};
%\matrix(locycells) [version list, text width=7mm, anchor=west]
   %at ([xshift=10pt]locy.east) {
 %{a} \pgfmatrixnextcell $\tsid_0$ \\
  %{a} \pgfmatrixnextcell $\{\tsid_1\}$ \\
%};
%\node[version node, fit=(locycells-1-1) (locycells-2-1), fill=white, inner sep= 0cm, font=\Large] (locy-v0) {$0$};
%%\node[version node, fit=(locycells-1-2) (locycells-1-3), fill=white, inner sep= 0cm, font= \Large] (locy-v0-ws) {$\tsid_0$};
%%\node[version node, fit=(locycells-2-2) (locycells-2-3), fill=white, inner sep=0cm, font=\Large] (locy-v1-rs) {$\{\tsid_1, \tsid_2\}$};

%% \draw[-, red, very thick, rounded corners] ([xshift=-5pt, yshift=5pt]locx-v1.north east) |- 
%%  ($([xshift=-5pt,yshift=-5pt]locx-v1.south east)!.5!([xshift=-5pt, yshift=5pt]locy-v0.north east)$) -| ([xshift=-5pt, yshift=5pt]locy-v0.south east);

%%blue view - I should  check whether I can use pgfkeys to just declare the list of locations, and then add the view automatically.
%\draw[-, blue, very thick, rounded corners=10pt]
 %([xshift=-3pt, yshift=20pt]locx-v1.north east) node (tid1start) {} -- 
 %([xshift=-3pt, yshift=-5pt]locx-v1.south east) --
 %([xshift=-3pt, yshift=5pt]locy-v0.north east) -- 
 %([xshift=-3pt, yshift=-5pt]locy-v0.south east);
 
 %\path (tid1start) node[anchor=south, rectangle, fill=blue!20, draw=blue, font=\small, inner sep=1pt] {$\thid_1$};

%%red view
%\draw[-, red, very thick, rounded corners = 10pt]
 %([xshift=-16pt, yshift=5pt]locx-v1.north east) node (tid2start) {}-- 
 %([xshift=-16pt, yshift=-3pt]locx-v1.south east) --
 %([xshift=-16pt, yshift=7pt]locy-v0.north east) -- 
 %([xshift=-16pt, yshift=-5pt]locy-v0.south east) node {};
 
%\path (tid2start) node[anchor=south, rectangle, fill=red!20, draw=red, font=\small, inner sep=1pt] {$\thid_2$};

%%%Stack for threads tid_1 and tid_2
%%
%%\draw[-, dashed] let 
%%   \p1 = ([xshift=0pt]locy.west),
%%   \p2 = ([yshift=-5pt]locycells.south),
%%   \p3 = ([xshift=10pt]locycells.east) in
%%   (\x1, \y2) -- (\x3, \y2);
%%   
%%\matrix(stacks) [
%%   matrix of nodes,
%%   anchor=north, 
%%   text=blue, 
%%   font=\normalsize, 
%%   row 1/.style = {text = blue}, 
%%   row 2/.style = {text = red}, 
%%   text width= 13mm ] 
%%   at ([xshift=-10pt,yshift=-8pt]locycells.south) {
%%   $\thid_1:$ & $\retvar = 0$\\
%%   $\thid_2:$ & $\retvar = 0$\\
%%   };
%\end{pgfonlayer}
%\end{tikzpicture}
%}\\
%\multicolumn{2}{|c|}{{\small(e)}}\\
%\hline
%\end{tabular}
%\end{center}
%\caption{Graphical representation of configurations obtained by 
%executing $\prog_5$.}
%\label{fig:ser.exec}
%\end{figure}

\sx{did not read }
If we execute the program $\prog_5$ under any of the consistency model presented so far, it is possible to infer 
an execution in which both threads $\thid_1$ and $\thid_2$ set the 
thread-local variable $\retvar$ to value ${\Large \frownie{}}$.
For example, under $\mathsf{SI}$, this can happen as follows: 
\begin{itemize}
\item first, thread $\thid_1$ executes its transactions in the 
configuration of Figure \ref{fig:ser.exec}(a). The snapshot 
under which the transaction is executed is given by $[[\loc_x] \mapsto 0, [\loc_y] \mapsto 0]$, 
hence the execution of the transaction results in the thread-local variable $\retvar$ of $\thid_1$ 
to be set to ${\Large \frownie{}}$, and in the configuration of Figure \ref{fig:ser.exec}(b).
\item Next, thread $\thid_2$ executes its transaction in the initial 
view $[ [\loc_x] \mapsto 0, [\loc_y] \mapsto 0]$. The snapshot under which 
the transaction is executed is again $[[\loc_x \mapsto 0, [\loc_y] \mapsto 0]$. After 
thetransaction has been executed, the thread-local variable $\retvar$ of $\thid_2$ is 
set to ${\Large \frownie{}}$, and the final configuration is the one of Figure \ref{fig:ser.exec}(c). 
Note that, because we are assuming SI as our consistency model, in this configuration 
we require that the view of thread $\thid_2$ points to the last version of each location.
\end{itemize}


To avoid the \emph{write skew} anomaly, it suffices to ensure that, prior to executing a transaction, the view of a thread is always up-to-date for all addresses. 

\begin{defn}[Serialisability]
The \emph{serialisability} requires  a transaction starts with a view that is up-to-date for all addresses,
\[
\begin{rclarray}
(\hh, \vi) \csat[\mathsf{SER}] \opset : \vi' & \defeq & \fora{\addr \in \dom(\hh)} \vi(\addr) = \left| \hh(\addr) \right|
\end{rclarray}
\]
\sx{
$(\hh, V) \triangleright_{\mathsf{SER}} \mathcal{O} : V'$ iff, for any location $[\loc_x]$, 
$V([\loc_x]) = \lvert \hh([\loc_x]) \rvert -1$. 
}
\end{defn}

Consider again the program $\prog_5$, this time to be executed under $\mathsf{SER}$. 
Similar as for $\mathsf{SI}$, after thread $\thid_1$ has executed its transaction, 
we end up with the variable $\retvar$ of such a thread to be set to value ${\Large \frownie{}}$, 
and with the configuration of Figure \ref{fig:ser.exec}(b). However, at this point we can 
not execute the transaction of thread $\thid_2$ without updating his view beforehand. 
This is because such a view does not point to the most up-to-date version for location 
$[\loc_x]$. Instead, before executing its transaction $\thid_2$, updates its view to 
point to the most up-to-date version of each location, Figure \ref{fig:ser.exec}(d). 
Thus, thread $\thid_2$ will execute its transaction using the snapshot $[[\loc_x] \mapsto 1, [\loc_y] \mapsto 0]$. 
A consequence of this fact is that the thread-local variable $\retvar$ of $\thid_2$ will not be 
set to ${\Large \frownie{}}$, and no new version for location $[\loc_y]$ will be created 
by the execution of the transaction. The final configuration is given in Figure \ref{fig:ser.exec}(e).

\ac{Contents: Read Atomic, Causal Consistency, Update Atomic, Consistent Prefix, Parallel Snapshot Isolation, Snapshot Isolation, 
Serializability. Well-formedness constraint to be placed on consistency models: progress must always be possible - i.e. it is 
always possible to execute a transaction if the view of all threads are up-to-date.}  
