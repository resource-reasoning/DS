\section{Soundness}


\begin{thm}[Transaction soundness]
\label{thm:transaction-soundness}
Assume the standard lift for transaction interpretation and for fingerprint heaps composite \( \composeFP \), the transaction soundness is as follows:
\[
    \begin{array}{@{}l@{}}
        \for{ \lpre, \trans, \lpost } \tripleL{\lpre}{\trans}{\lpost} \implies \ \tripleSemL{\lpre}{\trans}{\lpost} \\
    \end{array}
\]
where,
\[
    \begin{rclarray}
    \tripleSemL{\lpre}{\trans}{\lpost} & \eqdef &
    \begin{array}[t]{@{}l@{}}
        \for{\lenv, \stk_{p}, \stk_{q}, \fph_{p}, \fph_{q} } 
        \fph_{p} \in \evalLS[\lenv, \stk_{p}]{\lpre}
        \land (\stk_{p}, \fph_{p} ), \trans \toL^{*}  (\stk_{q}, \fph_{q} ), \pskip 
        \implies \fph_{q} \in \evalLS[\lenv, \stk_{q}]{\lpost}
    \end{array}
    \end{rclarray}
\]
\end{thm}
\begin{proof}
Induction on the rules for transactions.

\caseB{\rl{TRSkip}}

We have  \(\trans \equiv \pskip\), \( \lpre \equiv \lpost \equiv \assemp \).
Given the semantics in \fig \ref{fig:thread_semantics}, we have \( \fph_{p} = \fph_{q} = \unitFP \) and \( \stk_{p} = \stk_{q} \), so \( \fph_{q} \in \evalLS[\lenv, \stk_{q}]{\lpost} \).

\caseB{\rl{TRAss}}

We have \(\trans \equiv ( \pass{\var}{\expr} ) \), \( \lpre \equiv ( \var \doteq \lexpr ) \) and \( \lpost \equiv ( \var \doteq \expr\sub{\var}{\lexpr} ) \) for some \( \var, \lexpr, \expr \).
Given the semantics in \fig \ref{fig:thread_semantics}, there exists \( \stk \) such that \( \stk = \stk_{p} \setminus \Set{\var \mapsto \stub} = \stk_{q} \setminus \Set{\var \mapsto \stub} \).
Given the premiss of \rl{TRAss}  in \fig \ref{fig:rule-trans} that \( \var \notin \func{fv}{\lexpr} \), it has \( \evalLE{\lexpr} = \evalLE[\lenv, \stk_{p}]{\lexpr} = \evalLE[\lenv, \stk_{q}]{\lexpr} \), and then must exist \( \val \) so that \( \val = \evalLE{\expr\sub{\var}{\lexpr}} = \evalLE[\lenv, \stk_{q}]{\expr\sub{\var}{\lexpr}} \) and \( \stk_{q} = \stk_{p}\remapsto{\var}{\val} \).
Also because \( \fph_{p} = \fph_{q} = \unitFP \), we have \( \fph_{q} \in \evalLS[\lenv, \stk_{q}]{\lpost} \).

\caseB{\rl{TRDeref}}

We have  \(\trans \equiv ( \pderef{\var}{\expr} ) \), \( \lpre \equiv ( \expr \fpt{\fp} \lexpr ) \) and \( \lpost \equiv ( \var \doteq \lexpr \sep \expr \fpt{\addFPR{\fp}} \lexpr ) \) for some \( \var, \fp, \lexpr, \expr \).
Given the semantics in \fig \ref{fig:thread_semantics}, there exists \( \stk \) such that \( \stk = \stk_{p} \setminus \Set{\var \mapsto \stub} = \stk_{q} \setminus \Set{\var \mapsto \stub} \).
Given the premiss of \rl{TRDeref} in \fig \ref{fig:rule-trans} that \( \var \notin \func{fv}{\lexpr} \), it must exist \( \val \) and \( \addr \) such that \( \val = \evalLE{\lexpr} = \evalLE[\lenv, \stk_{p}]{\lexpr} = \evalLE[\lenv, \stk_{q}]{\lexpr} \), \( \addr = \evalLE{\expr} = \evalLE[\lenv, \stk_{p}]{\expr} = \evalLE[\lenv, \stk_{q}]{\expr} \) and \(  \stk_{q} = \stk_{p}\remapsto{\var}{\val} \).
Also, because \( \lpre \equiv ( \expr \fpt{\fp} \lexpr ) \), we have \( \fph_{p} = \Set{\addr \mapsto (\val, \fp) }\).
Given above and \( \fph_{q} = \fph_{p}\remapsto{\addr}{ ( \val, \addFPR{\fp} ) } \), we have \( \fph_{q} \in \evalLS[\lenv, \stk_{q}]{\lpost} \).

\caseB{ \rl{TRMutate} }

We have \( \trans \equiv (\pmutate{\expr_{1}}{\expr_{2}}) \), \( \lpre \equiv ( \expr_{1} \fpt{\fp} \stub ) \) and \( \lpost \equiv ( \expr_{1} \fpt{\addFPW{\fp}} \expr_{2} ) \), for some \( \expr_{1}, \expr_{2}, \fp \).
Therefore, \( \fph_{p} \in \Setcon{ \addr \mapsto (\val_{p} , \fp) }{ \val_{p} \in \Val } \), where \( \addr = \evalLE[\lenv, \stk_{p}]{\expr_{1}} \).
Given the semantics in \fig \ref{fig:thread_semantics}, we have \( \stk_{p} = \stk_{q} \) and \( \fph_{q} \in \Set{\addr \mapsto ( \evalLE[\lenv, \stk_{q}]{\expr_{2}},  \addFPW{\fp} ) } \), so \( \fph_{q} \in \evalLS[\lenv, \stk_{q}]{\lpost} \).

\caseI{\rl{TRChoice}}

We have  \(\trans \equiv \trans_{1} + \trans_{2} \), where \( \tripleL{\lpre}{\trans_{1}}{\lpost} \) and \( \tripleL{\lpre}{\trans_{2}}{\lpost} \) hold, for some \( \lpre, \lpost, \trans_{1}, \trans_{2} \).
Given the semantics in \fig \ref{fig:thread_semantics}, for any \( \lenv, \stk_{p}, \fph_{p} \) that \( \fph_{p} \in \evalLS[\lenv, \stk_{p}]{\lpre} \), it has either \( ( \stk_{p}, \fph_{p} ), \trans_{1} \pchoice \trans_{2} \toL ( \stk_{p}, \fph_{p} ), \trans_{1} \) or  \( ( \stk_{p}, \fph_{p} ), \trans_{1} \pchoice \trans_{2} \toL ( \stk_{p}, \fph_{p} ), \trans_{2} \).
Since it is symmetric, assume picking \( \trans_{1} \).
Therefore we have \( ( \stk_{p}, \fph_{p} ), \trans_{1} \pchoice \trans_{2} \toL ( \stk_{p}, \fph_{p} ), \trans_{1} \toL^{*} ( \stk_{q}, \fph_{q} ), \pskip \) for some \( \stk_{q} \) and \( \fph_{q} \).
By the \ih and the premiss of \rl{TRChoice} we have \( \tripleSemL{\lpre}{\trans_{1}}{\lpost} \), then we have \( \fph_{q} \in \evalLE[\lenv, \stk_{q}]{\lpost} \).
Symmetrically, if it picks \( \trans_{2} \), it yields the same result.

\caseI{\rl{TRSeq}}

We have \( \trans \equiv \trans_{1} \pseq \trans_{2} \) where \( \tripleL{\lpre}{\trans_{1}}{\lframe} \) and \( \tripleL{\lframe}{\trans_{2}}{\lpost} \) hold, for some \( \lpre, \lframe, \lpost, \trans_{1}, \trans_{2} \).
Given the semantics in \fig \ref{fig:thread_semantics}, for any \( \lenv, \stk_{p}, \fph_{p} \) that \( \fph_{p} \in \evalLS[\lenv, \stk_{p}]{\lpre} \), it has \( ( \stk_{p}, \fph_{p} ), \trans_{1} \pseq \trans_{2} \toL^{*} ( \stk_{r}, \fph_{r} ), \pskip \pseq \trans_{1} \toL ( \stk_{r}, \fph_{r} ), \trans_{1} \toL^{*} ( \stk_{q}, \fph_{q} ), \pskip \) for some \( \stk_{r}, \stk_{q}, \fph_{r}, \fph_{q} \).
By the \ih, we have \( \fph_{r} \in \evalLS[\lenv, \stk_{r}]{\lframe} \), then by the \ih, we have \( \fph_{q} \in \evalLS[\lenv, \stk_{q}]{\lpost} \).

\caseI{\rl{TRLoop}}

Since the triple is only partial correct, meaning that if the transaction \( \trans \) terminates it will reach a state satisfying the post-condition \( \lpost \), it is sufficient to prove the follows,
\[
    \for{\lpre, \trans, \nat > 0} \tripleL{\lpre}{\trans^{\nat}}{\lpre} \implies \ \tripleSemL{\lpre}{\trans^{\nat}}{\lpre} \\
\]
where,
\[
\begin{rclarray}
    \trans^{1} & \defeq  & \trans \\
    \trans^{\nat} & \defeq  & \trans \pseq \trans^{\nat - 1} \\
\end{rclarray}
\]
given the \ih that \(\tripleL{\lpre}{\trans}{\lpre} \implies \ \tripleSemL{\lpre}{\trans}{\lpre} \) holds.

We prove that by induction on the number \( \nat \).
For \( \nat = 1 \), it is proven by the \ih.
For \( \nat > 1 \), assume \( \stk, \fph, \lenv \) that satisfy \( \fph \in \evalLS{\lpre} \). 
Given the semantics in \fig \ref{fig:thread_semantics}, we have \( (\stk, \fph), \trans \pseq \trans^{\nat - 1} \toL^{*} (\stk' \fph'), \trans^{\nat - 1} \) for some \( \stk', \fph' \).
By the \ih that \(\tripleSemL{\lpre}{\trans}{\lpre} \), we have \( \fph' \in \evalLS[\lenv, \stk']{\lpre} \).
Then for any \( \stk'', \fph'' \) that \( (\stk', \fph'), \trans^{\nat - 1} \toL^{*} (\stk'' \fph''), \pskip \), by \ih that \( \tripleSemL{\lpre}{\trans}{\lpre} \), we have \( \fph'' \in \evalLS[\lenv, \stk'']{\lpre} \).
\caseI{\rl{TRFrame}}

We have \( \tripleL{\lpre \sep \lframe }{\trans}{\lpost \sep \lframe} \) and \( \tripleL{\lpre}{\trans}{\lpost} \) for some \( \lpre, \lpost, \lframe, \trans\).
For the precondition, we know \( \fph_{p} \composeFP \fph_{r} \in \evalLS[\lenv, \stk_{p}]{\lpre \sep \lframe} \) where \(  \fph_{p} \in \evalLS[\lenv, \stk_{p}]{\lpre} \) and \( \fph_{r} \in \evalLS[\lenv, \stk_{p}]{\lframe} \) for some \( \fph_{p}, \fph_{r}, \stk_{p}, \lenv \).
By the \( \pred{tsound}{\lpre, \trans, \lpost } \) from \ih, we have \( ( \stk_{p}, \fph_{p} ), \trans \toL^{*} (\stk_{q}, \fph_{q}), \pskip \) and \( \fph_{q} \in \evalLS[\lenv, \stk_{q}]{\lpost}\), for some \( \fph_{q}, \stk_{q} \).
Therefore, it is also true that \( ( \stk_{p}, \fph_{p} \composeLS \fph_{r} ), \trans \toL^{*} (\stk_{q}, \fph_{q} \composeLS \fph_{r}), \pskip \), so that \( \fph_{q} \composeFP \fph_{r} \in \evalLS[\lenv, \stk_{q}]{\lpost \sep \lframe} \).

\end{proof}

\begin{defn}[World to time-stamp heaps]
\label{def:w2tsh}
Given the set of world \( \world \in \World \) and the set of \( \tsh \in \TSHeaps \), the function \( \funcn{w2tsh} : \World \to \powerset{\TSHeaps} \) is defined as follows:
\[
    \begin{rclarray}
        \func{w2tsh}{\world} & \defeq & 
        \Setcon{%
            (\tsh,\ts) 
        }{%
            \exsts{ \h }
            \flattenW{\world} = (\h, \stub) 
            \land \snapshot{\tsh}{\ts}(\addr) = (\val, \stub) 
            \land \for{ \addr } \h(\addr) = \val 
         }
    \end{rclarray}
\]
where the \( \snapshotName \) function (\fig \ref{fig:thread_semantics}) returns a fingerprint heap corresponding the state at time \( \ts \), and here we match first projection, i.e. the values, with the flattened world.
\end{defn}

\begin{defn}[Semantic triple]
\label{def:semantic-triple}
    The semantic triple \( \tripleSemG{\gpre}{\prog}{\gpost}\) is defined as the follows,
    \[
        \begin{rclarray}
            \tripleSemG{\gpre}{\prog}{\gpost} & \defeq &
            \begin{array}[t]{@{}l@{}}
                \for{\world_{p}, \tsh_{p}, \tsh_{q}, \lenv, \stk_{p}, \stk_{q}, \ts_{p}, \ts_{q} }  \\
                \quad \world_{p} \in \evalW[\lenv, \stk_{p}]{\gpre}
                \land (\tsh_{p}, \ts_{p}) \in \func{w2tsh}{\world_{p}}
                \land (\stk_{p}, \tsh_{p}, \ts_{p}), \prog \toT{}^{*} (\stk_{q}, \tsh_{q}, \ts_{q}), \pskip \\
                \quad \implies \exsts{ \world_{q} \in \evalW[\lenv, \stk_{q}]{\gpost} }
                (\tsh_{q}, \ts_{q}) \in \func{w2tsh}{\world_{q}}  \\
                \qqquad {} \land \for{\world_{q}'} (\world_{q}, \world_{q}') \in \myrely \implies  \world_{q}' \in  \evalW[\lenv, \stk_{q}]{\gpost}
            \end{array}
        \end{rclarray}
    \]
\end{defn}


\begin{thm}[Soundness]
The soundness is defined as follows:
\[
    \begin{array}{@{}l@{}}
        \for{\gpre, \gpost, \prog } \tripleG{\gpre}{\prog}{\gpost} \implies \tripleSemG{\gpre}{\prog}{\gpost}
    \end{array}
\]
\end{thm}
\begin{proof}
Induction on the rules for program \( \prog \).

\caseB{\rl{PRCommit}}
We have \( \tripleG{\gpre}{\ptrans{\trans}}{\gpost} \) given that \( \tripleL{\lpre}{\trans}{\lpost} \) and \( \repartition{\gpre}{\gpost}{\lpre}{\lpost} \) for any \( \trans, \gpre, \gpost, \lpre, \lpost, \inter \). 
For any \( \stk_{p}, \lenv \), let variables \( \world_{p}, \fpw_{p}, \fph_{p}, \fph_{f} \) to satisfy the follows,
\begin{equation}
    \label{equ:def-wwhh}
    \world_{p} \in \evalW[\lenv, \stk_{p}]{\gpre} 
    \land \eraseFW{\fpw_{p}} = \world_{p}
    \land \flattenFW{\fpw_{p}} = (\fph_{p} \composeFP \fph_{f}, \stub)
    \land \fph_{p} \in \evalLS[\lenv, \stk_{p}]{\lpre}
\end{equation}
Note that we pick the names which are consistent with repartitioning in \defin \ref{def:repartitioning}.
By the soundness of \( \tripleL{\lpre}{\trans}{\lpost} \) (\thmref{thm:transaction-soundness}) and \equref{equ:def-wwhh}, for all \( \fph_{q}, \stk_{q} \), we have the follows,
\begin{equation}
    \label{equ:transaction-soundness}
    (\stk_{p}, \fph_{p} ), \trans \toL^{*}  (\stk_{q}, \fph_{q} ), \pskip 
    \implies \fph_{q} \in \evalLS[\lenv, \stk_{q}]{\lpost}
\end{equation}
Also, given the semantic triple (\defref{def:semantic-triple}) and the operational semantics (\figref{fig:thread_semantics}), let variables \( \tsh_{p}, \tsh_{q}, \ts_{p}, \ts_{q}, \tsid \) satisfy the follows,
\begin{equation}
    \label{equ:commit-current-trans}
    (\stk_{p}, \tsh_{p}, \ts_{p}), \ptrans{\trans} \toT{\lbC{\tsid}} (\stk_{q}, \tsh_{q}, \ts_{q}), \pskip 
\end{equation}
Now we consider addresses being written and read separably.
First, for any address \( \addr_{w} \) being tagged as write or read/write in \( \fph_{q} \), assume the value is \( \val_{w} \).
\begin{equation}
    \label{equ:local-write}
    \exsts{\fp} 
    \fph_{q}(\addr_{w}) = (\val_{w}, \fp)
    \land \fpW \in \fp
\end{equation}
By the \( \commitName \) function in operational semantics (\defref{fig:thread_semantics}), \equref{equ:commit-current-trans}, \equref{equ:transaction-soundness} and \equref{equ:local-write}, we have,
\begin{equation}
    \label{equ:global-write}
    \tsh_{q}(\addr_{w})(\ts_{q}) = (\val_{w}, \etW, \stub)
\end{equation}
By the repartitioning (\defin \ref{def:repartitioning}), \equref{equ:def-wwhh} and \equref{equ:transaction-soundness}, for any \( \fpw_{q} \) we have,
\begin{equation}
    \label{equ:repartition}
    \fpw_{q} \in \mergeR{\fpw_{p}}{\fpw_{i}}{\inter} \land \eraseFW{\fpw_{q}} \in \evalW[\lenv, \stk_{q}]{\gpost}
\end{equation}
Given the \( \mergeName[R] \) function (\defin \ref{def:repartitioning}) that uses several levels of merges until \( \mergeName[\val] \) function (\defin \ref{def:merge-finger-heap}), \equref{equ:transaction-soundness}, \equref{equ:local-write} and \equref{equ:repartition}, we have,
\begin{equation}
    \label{equ:write-remain-the-same}
    \exsts{\fph} \flattenFW{\fpw_{q}} = (\fph, \stub) \land \fph(\addr_{w}) = \fph_{q}(\addr_{w}) = (\val_{w}, \etW)
\end{equation}
which matches with Eq. \eqref{equ:global-write}.
Intuitively, because the address being written cannot be merged with others.

Second, we consider addresses \( \addr_{r} \) that are only being read with the value \( \val_{r} \),
\begin{equation}
    \label{equ:local-read}
    \fph_{q}(\addr_{r}) = (\val_{r}, \Set{\fpR})
\end{equation}
Therefore we have:
\begin{equation}
    \label{equ:tsh-read-value}
    \tsh_{q}(\addr_{r})(\ts_{p}) = (\val_{r}, \etR )
    \land \tsh_{q}(\addr_{r})(\ts_{q}) = (\val_{r}, \etE )
\end{equation}
However, note that:
\[
\neg\left(
    \begin{array}{@{}l@{}}
        \for{\addr_{r}} \exsts{\val_{r}} 
        \tsh_{q}(\addr_{r})(\ts_{p}) = (\val_{r}, \etR, \tsid )
        \land \tsh_{q}(\addr_{r})(\ts_{q}) = (\val_{r}, \etE , \tsid) \\
        \quad \implies \snapshot{\tsh_{q}}{\ts_{q}}(\addr_{r}) = (\val_{r}, \stub)
    \end{array}
\right)
\]
Because there might be other transactions that commits between times \( \ts_{p} \) and \( \ts_{q} \) and writes to some addresses \( \addr_{r} \), which is allowed by the \( \consistentName \) predicate shown in \fig \ref{fig:thread_semantics}.
Let \( \tsid_{1} \) to \( \tsid_{\nat} \) be the transactions that commits between times \( \ts_{p} \) and \( \ts_{q} \) and writes to some addresses \( \addr_{r} \).
This is:
\[
    \begin{array}{@{}l@{}}
        \exsts{\nat} 
        \bigwedge\limits_{1 \leq i \leq \nat} \exsts{\ts, \tsid} \ts_{p} < \ts \leq \ts_{q} 
        \land \tsh_{q}(\addr_{r})(\ts) = (\stub, \etW, \tsid) \\
    \end{array}
\]
Given the operational semantics in \fig \ref{fig:thread_semantics}, if two transactions write to the same addresses, it must be strictly one after another.
We also know the first transaction \( \tsid_{1} \) must take the same snap shot as the transaction \( \tsid \) (Eq. \eqref{equ:commit-current-trans}), therefore they must read the same value.
Note that the variable \( \tsid \) is from Eq. \eqref{equ:commit-current-trans}.
\begin{equation}
\label{equ:other-writes}
    \begin{array}{@{}l@{}}
        \for{\addr_{r}} \exsts{\val_{r}} 
        \tsh_{q}(\addr_{r})(\ts_{p}) = (\val_{r}, \etR, \tsid )
        \land \tsh_{q}(\addr_{r})(\ts_{q}) = (\val_{r}, \etE , \tsid) \\
        \quad \implies \exsts{\nat, \ts_{1}, \ts_{1}', \dots \ts_{\nat}, \ts_{\nat}', \tsid_{1}, \dots, \tsid_{\nat}, \val_{0}, \val_{1}, \dots, \val_{\nat} }  \\
        \qquad \val_{0}  = \val_{r} 
        \land \ts_{p} < \ts_{1}' < \ts_{1} < \dots < \ts_{\nat}' < \ts_{\nat} < \ts_{q} \land {} \\
        \qquad \bigwedge\limits_{1 \leq i \leq \nat} 
        \exsts{ \etag } 
        \land ( \etag = \etR \lor \fp = \etS ) 
        \land \tsh_{q}(\addr_{r})(\ts_{i}') = (\val_{i-1}, \etag, \tsid_{i}) 
        \land \tsh_{q}(\addr_{r})(\ts_{i}) = (\val_{i}, \etW, \tsid_{i}) \\
        \quad \implies \snapshot{\tsh_{q}}{\ts_{q}}(\addr_{r}) = (\val_{n}, \stub)
    \end{array}
\end{equation}
\sx{Below is pending part.}
where for all transactions \( \tsid_{i} \) it must be allowed by \( \relyU \).
To address that, we firstly introduce new variables \( \ls, \fpgs_{0} \), where \( \fpw_{p} \) is introduced before.
\[
    (\ls, \fpgs_{0}) = \fpw_{p}
\]
Given the \( \tsh_{q} \) from Eq. \eqref{equ:tsh-read-value} and \( \addr_{r}, \val_{r}, \nat, \ts_{1}, \ts_{1}', \dots, \ts_{\nat}, \ts_{\nat}', \tsid_{1}, \dots, \tsid_{\nat}, \val_{0}, \dots, \val_{\nat}  \) from Eq. \eqref{equ:other-writes}, we have:
\begin{equation}                              
\label{equ:allowed-by-rely}
\begin{array}{@{}l@{}}
    \exsts{\fpgs_{1}, \dots, \fpgs_{\nat}} \\
    \quad \bigwedge\limits_{1 \leq i \leq \nat } 
    \begin{formulea}
    ( \tsh_{q}, \ts_{i} ) \in \func{w2tsh}{\eraseFW{(\ls, \fpgs_{i})}} \land {} \\
    \quad \exsts{\fpgs_{f}, \fpgs_{f}'} ( (\ls, \fpgs_{i} \composeFPGS \fpgs_{f}) , (\ls, \fpgs_{i+1} \composeFPGS \fpgs_{f}') ) \in \relyU
    \end{formulea}
\end{array}
\end{equation}
\sx{
    Above is pending part.
    Here it is not correct at all, need a induction  to prove every step of \( \tsid_i \) is in \( \myrely \), given the mixing of merge and transitive closure.
}
By Eq. \eqref{equ:repartition}, Eq. \eqref{equ:other-writes} and Eq. ...... , we have the follows:
\begin{equation}
\label{equ:read-match}
\begin{array}{@{}l@{}}
  \for{\addr_{r}} \exsts{\val_{r}} 
  \tsh_{q}(\addr_{r})(\ts_{p}) = (\val_{r}, \etR, \tsid )
  \land \tsh_{q}(\addr_{r})(\ts_{q}) = (\val_{r}, \etE , \tsid) \land {} \\
  \quad \exsts{ \ts , \val', \tsid' } \ts' = 
    \max\Setcon{%
        \ts%
    }{%
        \ts_{p} < \ts \leq \ts_{q} \land \tsh_{q}(\addr_{r})(\ts) = (\stub, \etW, \stub)
    } \land \tsh_{q}(\addr_{r})(\ts') = (\val', \etW, \tsid') \\
    \quad \implies \snapshot{\tsh_{q}}{\ts_{q}}(\addr_{r}) = (\val', \stub) \\
    \quad \implies \exsts{\fpw_{q} \in \mergeR{\fpw_{p}}{\fpw_{i}}{\myrely} } (\tsh_{q}, \ts_{q} ) \in \func{w2tsh}{\eraseFW{\fpw_{q}}}
\end{array}
\end{equation}

Now given the write matching Eq. \eqref{equ:write-match} and read matching Eq. \eqref{equ:read-match}, we have the proof for base case \( \ptrans{\trans} \).


\end{proof}










\sx{No use for below}

\begin{defn}[Transaction interpretation]
\label{def:transactions-interpretation}
Given the set of transaction \( \trans \in \Transactions \) (\defin \ref{def:language}) and the operational semantics (\fig \ref{fig:transaction_semantics}), the \emph{transaction interpretation} function \( \intpSQ{.} : \Transactions \to \FPHeaps \to \powerset{\FPHeaps} \) is defined as follows:
\[
    \begin{rclarray}
        \intpSQ{\trans}(\fph) & \defeq & 
            \Setcon{%
                \fph'
            }{%
                \exsts{ \stk, \stk' } (\stk, \fph ), \trans \toL^{*}  (\stk', \fph' ), \pskip
            }\\
    \end{rclarray}
\]
\end{defn}
\sx{probably use the above to prove%
\[
    \tripleL{\lpre}{\trans}{\lpost}
\]%
Simply for less words, maybe?
}
Note that the stack is local and has no side effect to the fingerprint heap, therefore from now we will fix the stack and treat the stack as the same as logical environment.

\begin{defn}[Reification function]
\label{def:reification}
Given the set of assertions \( \gpre \in \Ast \) and time-stamp heap \( \TSHeaps \), the \emph{reification} function \( \reif{.} : \Ast \to \powerset{\TSHeaps} \) is defined as follows:
\[
\begin{rclarray}
    \reif{\gpre} & \defeq & 
    \Setcon{%
        \tsh
    }{%
        \exsts{\lenv, \stk, \h} \world \in \evalW{\gpre} 
        \land (\h, \stub ) = \flattenW{\world}
        \land \tsh \in\func{h2tsh}{\h} 
    }
\end{rclarray}
\]
where the interpretation of assertion \( \evalW{.} \) is defined in \defin \ref{def:assertion}, the world flattening \( \flattenW{.} \) in \defin \ref{def:world} and \( \funcn{h2tsh} \) function in \defin \ref{def:h2tsh}.
\end{defn}

\begin{defn}[Atomic interpretation]
\label{def:atomic-intp}
%Given the set of transactions \( \trans \in \Transactions \) (\defin \ref{def:language}), the set of programs \( \Programs \) and the operational semantics (\fig \ref{fig:thread_semantics}), the \emph{atomic interpretation} function \( \intfATOM{.} : \Atom \to \TSHeaps \to \powerset{\TSHeaps} \) is defined as follows, where \( \Atom \defeq \Setcon{\ptrans{\trans}}{\trans \in \Transactions \land \ptrans{\trans} \in \Programs} \).
\[
    \begin{rclarray}
        \intpSQ{\ptrans{\trans}}(\tsh) & \defeq & 
            \Setcon{%
                \tsh'
            }{%
                \exsts{ \stk, \stk', \ts, \ts' } (\stk, \tsh, \ts ), \ptrans{\trans} \toG{\stub}  (\stk', \tsh', \ts' ), \pskip
            }\\
    \end{rclarray}
\]
\end{defn}

\begin{thm}[Axiom soundness]
Given the set of transactions \( \trans \in \Transactions \) (\defin \ref{def:language}) and the rely relation \( \myrely \) (\defin \ref{def:rely-guarantee}), and assume stardard lift for reification function (\defin \ref{def:reification}) and for atomic interpretation (\defin \ref{def:atomic-intp}), the axiom soundness is defined as follows:
\[
    \begin{array}{@{}l@{}}
        \for{ \trans, \gpre, \gpost } 
        \tripleG{\gpre}{\ptrans{\trans}}{\gpost} \\
        \quad {} \land \for{\world } \intpSQ{\ptrans{\trans}}\left( \reif{\gpre \sep \Set{\world}} \right) \subseteq \left( \reif{\gpost \sep \myrely(\Set{\world})} \right) 
     \end{array}
\]
where \( \Set{\world} \) denotes assertion satisfying \( \world \) under any logical environment and stack, i.e.\ \( \for{\lenv, \stk} \evalW{\Set{\world}} = \Set{\world} \).
\end{thm}
\begin{proof}
Given the reification function, it is sufficient to prove:
\[
    \begin{array}{@{}l@{}}
        \for{ \trans, \gpre, \gpost } 
        \tripleG{\gpre}{\ptrans{\trans}}{\gpost} \\
        \quad {} \land \for{ \stk ,\lenv, \world_{f}, \world_{p} } 
        \exsts{ \world_{q}, \world_{f}' } 
        \world_{p} \in \evalW{\gpre}
        \land \world_{q} \in \evalW{\gpost}
        \land \world_{f}' \in \myrely(\world_{f})  \\
        \quad {} \land \for{ \h_{p}, \tsh_{p} } 
        \exsts{ \h_{q}, \tsh_{q} }
        (\h_{p}, \stub) = \flattenW{\world_{p} \composeW \world_{f}} 
        \land \tsh_{p} \in \func{h2tsh}{\h_{p}}
        \land (\h_{q}, \stub) = \flattenW{\world_{q} \composeW \world_{f}'} 
        \land \tsh_{q} \in \func{h2tsh}{\h_{q}} \\
        \quad \implies \tsh_{q} \in \intpSQ{\ptrans{\trans}} \left( \tsh_{p} \right)
     \end{array}
\]
First we introduce some new variables \( \fpw_{p}, \fph_{p}, \fpw_{q}, \fph_{q} \) that satisfy the follows and whose names are consistent with those in \defin \ref{def:repartitioning}.
\[
\begin{array}{@{}l@{}}
    \exsts{\fph} \\
    \eraseFW{\fpw_{p}} = \world_{p} 
    \land \flattenFW{\fpw_{p}} = (\fph_{p} \composeFP \fph, \unitC)
    \land \for{\addr} \fph_{p}(\addr) = (\stub, \emptyset) \land {} \\
    \eraseFW{\fpw_{q}} = \world_{q} 
    \land \flattenFW{\fpw_{q}} = (\fph_{q} \composeFP \fph, \unitC)
\end{array}
\]
By the definition of repartition (\defin \ref{def:repartitioning})  and transaction soundness (Theorem \ref{thm:transaction-soundness}):
\[
\begin{array}{@{}l@{}}
    \for{\fph} 
    \fph_{q} \composeFP \fph \in \intpSQ{\trans}( \fph_{p} \composeFP \fph) 
\end{array}
\]
\end{proof}
