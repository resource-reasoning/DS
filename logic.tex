\section{Logic}

We present a logic that is parametrised by any consistency model \emph{at least satisfying atomic  read and monotonic read}.
We motivate it by the \emph{write skew} example under snapshot isolation (\cref{fig:write-skew-si-proof}).
This example distinguishes serialisibility from snapshot isolation (SI).
Under serialisibility that transactions appear one after another, only one key, \( \vx \) or \( \vy \), will be 1 at the end.
While under SI, both keys \( \vx \) and \( \vy \) might be 1.
Because both transactions may take snapshots where \( \vx \) and \( \vy \) are 0, and both can commit because the two transactions write different keys.

\begin{figure}[!t]
\hrule
\[
\intass :
\begin{array}[t]{@{} c @{\quad} c @{\quad} c @{\quad} c @{} }
\begin{rclarray}[t]
    \CB{L} & : & \vx \fpW 1 \sep \vy \fpR 0 \sep \null \fpA \cass{\CB{L}}{\lrid} \\
\end{rclarray}
&
\begin{rclarray}[t]
    \CB{R} & : & \vx \fpR 0 \sep \vy \fpW 1 \sep \null \fpA \cass{\CB{R}}{\lrid} \\
\end{rclarray}
&
\begin{rclarray}[t]
    \CB{0} & : & \exsts{\V n} \vx \fpR \V{n} \\
\end{rclarray} 
&
\begin{rclarray}[t]
    \CB{0} & : & \exsts{\V n} \vy \fpR \V{n} \\
\end{rclarray} \\
\end{array}
\]
\[
\CB{L} \composeK \CB{L} \ \text{is undefined} \quad  \CB{R} \composeK \CB{R} \ \text{is undefined} \quad \CB{U} \ \text{is the unit}
\]
\hrule\vspace{5pt}
\[
\begin{session}
{\color{blue}P : } \specline{ \cass{\CB{L}}{\lrid} \sep \cass{\CB{R}}{\lrid} \sep \boxass{\vx \pt 0 \sep \vy \pt 0 }{\lrid}{\intass}  } \\
\begin{parl}
\begin{session}
    {\color{blue}P1 : } \specline{\cass{\CB{L}}{\lrid} \sep 
            \boxass{\vx \pt 0 \sep \vy \pt 0 }{\lrid}{\intass} \\
            {} \lor \boxass{\vx \pt 0 \sep ( \vy \pt \underline{0}  \lor \vy \pt 1 ) \sep \cass{\CB{R}}{\lrid} }{\lrid}{\intass} 
    } \\
    \txid_1 : \begin{transaction}
        {\color{blue}p1 : } \specline{\vx \fpI 0 \sep ( \vy \fpI 0 \lor \vy \fpI 1 )} \\
        \pderef{\pvar{b}}{\vy} ; 
        \quad \pifs{\pvar{b} = 0} 
        \pmutate{\vx}{1} ;
        \pife \\
        {\color{blue}q1 : } \specline{\vx \fpW 1 \sep  \vy \fpR 0 \lor \vx \fpI 0 \sep \vy \fpR 1 )} \\
    \end{transaction} \\
    {\color{blue}Q1 : } \specline{ 
            \boxass{ \vx \pt 1 \sep \vy \pt 0 \sep \cass{\CB{L}}{\lrid} }{\lrid}{\intass} \\
            {} \lor \boxass{\vx \pt 1 \sep ( \vy \pt \underline{0}  \lor \vy \pt 1 ) \sep \cass{\CB{R}}{\lrid} \sep \cass{\CB{L}}{\lrid} }{\lrid}{\intass}  \\
            {} \lor \cass{\CB{L}}{\lrid} \sep \boxass{\vx \pt 0 \sep \vy \pt 1 \sep \cass{\CB{R}}{\lrid} }{\lrid}{\intass}  \\
    } \\
\end{session}
&
\begin{session}
    {\color{blue}P2 : } \specline{\cass{\CB{R}}{\lrid} \sep 
            \boxass{ ( \vx \pt 0 \sep \vy \pt 0 }{\lrid}{\intass} \\
            {} \lor \boxass{ ( \vx \pt \underline{0} \lor \vx \pt 1 ) \sep \vy \pt 0 \sep \cass{\CB{L}}{\lrid} }{\lrid}{\intass} 
    } \\
    \txid_2 : \begin{transaction}
        {\color{blue}p2 : } \specline{ ( \vx \fpI 0 \lor \vx \fpI 1 ) \sep \vy \fpI 0 )} \\
        \pderef{\pvar{a}}{\vx} ; 
        \quad \pifs{\pvar{a} = 0} 
        \pmutate{\vy}{1} ; 
        \pife \\
        {\color{blue}q2 : } \specline{ \vx \fpR 0 \sep \vy \fpW 1 \lor \vx \fpR 1 \sep \vy \fpI 0 )} \\
    \end{transaction} \\
    {\color{blue}Q2 : } \specline{ 
            \boxass{ \vx \pt 0 \sep \vy \pt 1 \sep \cass{\CB{R}}{\lrid} }{\lrid}{\intass} \\
            {} \lor \boxass{ ( \vx \pt \underline{0} \lor \vx \pt 1 ) \sep \vy \pt 1 \sep \cass{\CB{L}}{\lrid} \sep \cass{\CB{R}}{\lrid} }{\lrid}{\intass}  \\
            {} \lor \cass{\CB{R}}{\lrid} \sep \boxass{\vx \pt 1 \sep \vy \pt 0 \sep \cass{\CB{L}}{\lrid} }{\lrid}{\intass}  \\
    } \\
\end{session}
\end{parl} \\
{\color{blue}Q : } \specline{ 
        \cass{\CB{L}}{\lrid} \sep \boxass{ \vx \pt 0 \sep \vy \pt 1 \sep \cass{\CB{R}}{\lrid} }{\lrid}{\intass} 
        \lor \cass{\CB{R}}{\lrid} \sep \boxass{\vx \pt 1 \sep \vy \pt 0 \sep \cass{\CB{L}}{\lrid} }{\lrid}{\intass}   \\
        {} \lor \boxass{ ( \vx \pt \underline{0} \lor \vx \pt 1 ) \sep \vy \pt 1 \sep \cass{\CB{L}}{\lrid} \sep \cass{\CB{R}}{\lrid} }{\lrid}{\intass}  \\
} \\
\end{session}
\]
\hrule
\caption{Interference, capabilities (the top), and sketch proof (the bottom) for write skew under snapshot isolation}
\label{fig:write-skew-si-proof}
\end{figure}

In the sketch proof (\cref{fig:write-skew-si-proof}), \emph{A shared region assertion} also know as \emph{a boxed assertion} in the form of \( \boxass{\bar{\lpre}}{\lrid}{\intass}\) describes that there is a region with \emph{a unique region identifier} \( \lrid \) and \emph{interference  assertion} \( \intass \), and the assertions inside \( \bar{\lpre} \) describe views on key-value stores.
Note that unlike serialisibility, the view might be out-of-date for SI.
In the assertion, a value with a underline means \emph{a potentially out-of-date value}, otherwise it is the \emph{most recent value}.
For instance, \( \boxass{\vx \pt 0 \sep ( \vy \pt \underline{0} \lor \vy \pt \underline{1} ) }{\lrid}{\intass}\) from \( P1 \) asserts views where \( \vx \) points to the most recent value of 0, but \( \vy \) points to a potentially out-of-date value of either 0 or 1.
Each region are shareable and indivisible, \ie \( \boxass{\bar{\lpre}}{\lrid}{\intass} \sep \boxass{\bar{\lpost}}{\lrid}{\intass} \iff \boxass{\bar{\lpre} \land \bar{\lpost}}{\lrid}{\intass}\).

The \emph{interference assertion} \( \intass \) specifies how the regions can evolve.
It is a set of actions, each of which has the form \( \kap : \fp \) where \( \kap \) is the \emph{client-specified capability} and \( \fp \) is the \emph{fingerprint}.
A client is allowed to execute a transaction with the fingerprint \( \fp \), when the client holds the capability \( \kap \).
The \( \CB{L}\) allows a client to read \( \vy \) when it is 0 and write 1 to \( \vx \), and similarly \( \CB{R} \) allows a client to read \( \vx \) when it is 0 and write 1 to \( \vy \).
The \( \fp \) also specifies capabilities transformation, for example in the example, both \( \CB{L} \) and \( \CB{R} \) need to be move into the region once they have been used.
The \emph{client-specified capability} forms \emph{a partial commutative monoid (PCM)} where \( \composeK \) denotes the composition function.
In the write skew example, both \( \CB{L} \) and \( \CB{R} \) are unique because the compositions with themselves are undefined, and \( \CB{U} \) is the unit.
The unit \( \CB{U} \) allows clients to always read \( \vx \) or \( \vy \) no matter the values.

Given the interference, an assertion \( \gpre \) is \emph{stable}, iff it holds against the interference from the environment.
Since the assertions describe views on key-value stores.
Stabilisation has two meanings: the environment might change the key-value stores but not the views, since the views are local; consequentially, the views can be advanced.
For example, \( P1 \) describes either the environment does nothing or the environment has performed the action associated with \( \cass{\CB{R}}{\lrid}\), 
after which the environment cannot do any further actions since the capabilities \( \cass{\CB{R}}{\lrid}\) is in the shared region.
Let discuss the assertions \( \vy \pt \underline{0}  \lor \vy \pt 1 \) from \( P1 \).
The underline value \( \underline{0} \) means the environment changed the state of the key-value store which make the view out-of-date.
Because the view can be advanced by stabilisation and the capability \( \cass{\CB{R}}{\lrid} \) is in the region, we get a up-to-date \( \vy \pt 1 \).

Because of atomicity, A transaction works on it own snapshot and only commits the fingerprint by the end.
We use \emph{transactional assertions} to describe the state of local snapshots for transactions and also the fingerprints.
The transactional assertions for a single key have the following forms: \( \vx \fpI 0 \), \( \vx \fpR 0\), \( \vx \fpW 0\), and \( \vx \fpRW (0,1) \), where \( \otR \) and \( \otW \) are read and write labels.
The first three asserts the key \( \vx \) in the local snapshot has value 0 and it has not been touched (no label), has been read (\(\otR\)) and has been written (\(\otW\)) respectively.
The last one asserts that the key \( \vx \) currently has value 1, \emph{the first read preceding any write} fetches 0 and the \emph{the last write} updates the key to \( 1 \).
We extend the standard sequential separation logic rules in a ways that the first read before any write to a key adds a read label to the assertion; and a write to a key adds a write label to assertion and updates the value.
Note that when writing to an key that has been read, the assertion after will have two values where the first corresponds to old read value and the second corresponds the written value.

When a transaction starts, it takes a snapshot provided by the view.
For example,  because of \( P1 \) where the view for the \( \vx \) has value 0 and \( \vy \) has value either 0 or 1, the precondition \( p1 \) for transaction \( \txid_1 \) describe exactly the same states, and initial all keys have no fingerprint.
To verify the transaction, it means collecting the fingerprint.
By the end, the postcondition \( q1 \) says the transaction \( \txid_1 \) either reads \( \vy \) with 0 and writes 1 to \( \vx \), or only reads \( \vy \) with 1 and does nothing to \( \vx \).

To committing the fingerprint under SI, we first need to check the view from \( P1 \).
SI requires that if a view observe some transactions, it should observe everything before.
It holds for \( P1 \).
Now given the view on the key-value store \( \gpre \), and committing the fingerprint of the transaction \( \lpost \) under SI means: 
(i) for any key that has been over-written, the view before for the key should be up-to-date and the view after should points to the new version;
(ii) for any key that has been read, the view before points to a version containing the same value and the view after remains the same;
{iii) there are some local capabilities that allows the updates.
This will give us the following assertion before stabilisation:
\[
\boxass{ \vx \pt 1 \sep \vy \pt 0 \sep \cass{\CB{L}}{\lrid} }{\lrid}{\intass} 
\lor \boxass{\vx \pt 1 \sep \vy \pt \underline{0}  \sep \cass{\CB{R}}{\lrid} \sep \cass{\CB{L}}{\lrid} }{\lrid}{\intass}
\lor \cass{\CB{L}}{\lrid} \sep \boxass{\vx \pt 0 \sep \vy \pt 1 \sep \cass{\CB{R}}{\lrid} }{\lrid}{\intass} 
\]
We need to stylise the assertions.
The environment can still change the state that satisfies the first disjunction, because environment still has the ability  to change the value of \( \vy \), which yield the state satisfying the second disjunction.
The second conduction can be stabilised by advancing the view.
Now we have a stable assertions as \( Q1 \) in \cref{fig:write-skew-si-proof}.

Following similar reasoning for right-hand-side client, we get \( Q2 \).
The final \( Q \) is  \( Q = Q1 \sep Q2 \) provided that \( \boxass{\bar{\lpre}}{\lrid}{\intass} \sep \boxass{\bar{\lpost}}{\lrid}{\intass} \iff \boxass{\bar{\lpre} \land \bar{\lpost}}{\lrid}{\intass}\).
It means the final views and key-value stores are those come from both \( Q1 \) and \( Q2 \).



%For serialisibility, the stable PL and PR rules out those views that cannot progress via any possible transactions.
%This mean the view must at the end of the history heap.
%The post conditions QL and QR is stronger because the consistency model check in the repartition 

%\[
%\begin{session}
%\specline{\boxass{\vx \pt 0 \sep \vy \pt 0}{\lrid}{\intass} \sep \cass{\CB{L}}{\lrid} \sep \cass{\CB{R}}{\lrid} } \\
%\begin{parl}
%\begin{session}
    %stable-PL : \specline{\cass{\CB{L}}{\lrid} \sep 
            %\boxass{\vx \pt \Set{0} \sep  \begin{B} \vy \pt \Set{0} \lor \vy \pt \Set{1} \end{B} }{\lrid}{\intass} 
    %} \\
    %%PL2 : \specline{\cass{\CB{L}}{\lrid} \sep 
            %%\boxass{\vx \pt 0 \sep \vy \pt 0 }{\lrid}{\intass} \lor {} \\
            %%\boxass{\vx \pt 0 \sep ( \vy \pt 0 \lor \vy \pt 1 ) \sep \cass{\CB{R}}{\lrid}}{\lrid}{\intass} 
    %%} \\
    %\begin{transaction}
        %\specline{\vx \fpI 0 \sep ( \vy \fpI 0 \lor \vy \fpI 1 )} \\
        %\pderef{\pvar{b}}{\vy} ; \\
        %\pifs{\pvar{b} = 0} 
        %\pmutate{\vx}{1} ;
        %\pife \\
        %\specline{\vx \fpW 1 \sep  \vy \fpR 0 \lor \vx \fpI 0 \sep \vy \fpR 1 )} \\
    %\end{transaction} \\
    %QL : \specline{\cass{\CB{L}}{\lrid} \sep 
            %\boxass{\vx \pt \Set{1} \sep \vy \pt \Set{0} }{\lrid}{\intass} \\
            %{} \lor \boxass{\vx \pt \Set{0} \sep \vy \pt \Set{1} }{\lrid}{\intass} \\
    %} \\
%\end{session}
%&
%\begin{session}
    %PR : \specline{\cass{\CB{R}}{\lrid} \sep 
            %\boxass{ \begin{B} \vx \pt \Set{0} \lor \vx \pt \Set{1} \end{B} {} \sep \vy \pt \Set{0} }{\lrid}{\intass} 
    %} \\
    %\begin{transaction}
        %\pderef{\pvar{a}}{\vx} ; 
        %\quad \pifs{\pvar{a} = 0} 
        %\pmutate{\vy}{1} ; 
        %\pife 
    %\end{transaction} \\
    %QR : \specline{\cass{\CB{R}}{\lrid} \sep 
            %\boxass{\vx \pt \Set{0} \sep \vy \pt \Set{1} }{\lrid}{\intass} \\
            %{} \lor \boxass{\vx \pt \Set{1} \sep \vy \pt \Set{0} }{\lrid}{\intass} \\
    %} \\
%\end{session}
%\end{parl} \\
%QL \sep QR : \specline{\cass{\CB{L}}{\lrid} \sep \cass{\CB{R}}{\lrid} \sep 
            %\boxass{\vx \pt \Set{0} \sep \vy \pt \Set{1} }{\lrid}{\intass} 
            %\lor \boxass{\vx \pt \Set{1} \sep \vy \pt \Set{0} }{\lrid}{\intass} \\
%}
%\end{session}
%\]



\subsection{Reasoning inside transactions}

Recall that a transaction takes a snapshot of the kv-store and commits the fingerprint by the end.
Because of the atomicity, only the \emph{first reads preceding any write} and the \emph{last writes} of keys are contained in the fingerprint.
All the intermediate reads and writes are not observable to other transactions and have no effect on the key-value store.
To capture the state of the local snapshot as well as the fingerprint, 
the \emph{transactional assertions} (\cref{def:local_assertions}) extend normal sequential separate logic assertions with read and/or write labels, 
\eg \( \vx \fpR 0 \), \( \vy \fpW 1 \) and \( \pv{z} \fpRW (0,1) \).


\begin{definition}[Transactional assertions]
\label{def:fingerprint}
\label{def:local_assertions}
\label{def:logical-expr}
Assume a countably infinite set of \emph{logical variables} $\lvar \in \LVar$.
The set of \emph{logical expressions} $\lexpr \in \LExpr$ is defined by the inductive grammar:
\(
\begin{rclarray}
   \lexpr & ::= & \val \mid \lvar \mid \var \mid \lexpr + \lexpr \mid  \dots 
\end{rclarray}
\)
where \(\val \in \Val\)  and \(\var \in \Vars\).
The \emph{logical expression evaluation} function, $\evalLE[(., .)]{.}:\LExpr \times \Stacks \times \LEnv\rightharpoonup \Val$, is defined inductively over the structure of logical expressions,
where the \emph{logical environments} \(\lenv \in \LEnv: \LVar \parfun \Val\) associates logical variables with values:
%
\[
\begin{array}{@{}c@{}}
    \begin{rclarray}
        \evalLE{\val} & \defeq & \val \\
    \end{rclarray}
    \quad
    \begin{rclarray}
        \evalLE{\lvar} & \defeq & \lenv(\lvar) \\
    \end{rclarray}
    \quad
    \begin{rclarray}
        \evalLE{\var} & \defeq & \txstk(\var) \\
    \end{rclarray} 
    \quad
    \begin{rclarray}
        \evalLE{\lexpr_1 + \lexpr_2} & \defeq & \evalLE{\lexpr_1} + \evalLE{\lexpr_2} \\
    \end{rclarray}
    \dots
\end{array}
\]
The set of \emph{transactional assertions}, $\lpre,  \lpost, \fp \in \LAst$, is defined by the following grammars:
\[
\begin{rclarray}
	\lpre, \lpost, \fp & ::= & \False \mid \True \mid \lpre \land \lpost \mid \lpre \lor \lpost \mid \exsts{\lvar} \lpre \mid \lpre \implies \lpost \\
    & & \mid \Emp \mid \lexpr \fpI \lexpr \mid \lexpr \fpR \lexpr \mid \lexpr \fpW \lexpr \mid \lexpr \fpRW (\lexpr, \lexpr)  \mid \lpre \sep \lpost  \\
\end{rclarray}	 
\]
The \emph{transactional assertion interpretation function}, $\evalLS[(.,.)]{.}: \LAst \times \LEnv \times \LAst \parfun \powerset{\Heaps \times \Opsets} $, is defined over the structure of local assertions, where the composition for snapshots \( \composeH \defeq \uplus \) is standard disjointed union on two functions and the composition for fingerprints \( \opset \composeO \opset' \defeq \opset \uplus \opset'\) when they contain different keys, \ie \( \opset\projection{2} \cap \opset'\projection{2} = \emptyset\):
\[
\begin{array}{@{}c @{\qquad} c @{}}
\begin{rclarray}
	\evalLS{\assfalse} & \eqdef & \emptyset \\
	\evalLS{\asstrue} & \defeq & \Heaps \times \Opsets \\
	\evalLS{\lpre \land \lpost} & \defeq & \evalLS{\lpre} \cap \evalLS{\lpost} \\
	\evalLS{\lpre \lor \lpost} & \defeq & \evalLS{\lpre} \cup \evalLS{\lpost} \\
\end{rclarray}
&
\begin{rclarray}
	\evalLS{\exsts{\lvar} \lpre} & \defeq & \bigcup\limits_{\val \in \textnormal{\Val}}\evalLS[\lenv\remapsto{\lvar}{\val}, \stk]{\lpre}  \\
	\evalLS{\lpre \implies \lpost} & \defeq & \Setcon{(\h, \opset)}{(\h , \opset) \in \evalLS{\lpre} \implies (\h , \opset) \in \evalLS{\lpost}}\\
	\evalLS{\assemp} & \defeq & \Set{ ( \unitH, \unitE) }  \\
	\evalLS{ \lexpr_1 \fpI \lexpr_2 } & \defeq & \Set{\left(\Set{\evalLE{\lexpr_1} \mapsto \evalLE{\lexpr_2} }, \unitO\right)} \\
\end{rclarray}
\end{array}
\]
\[
\begin{rclarray}
	\evalLS{ \lexpr_1 \fpR \lexpr_2 } & \defeq & \Set{\left(\Set{\evalLE{\lexpr_1} \mapsto \evalLE{\lexpr_2} }, \Set{(\otR, \evalLE{\lexpr_1},\evalLE{\lexpr_2})}\right)} \\
	\evalLS{ \lexpr_1 \fpW \lexpr_2 } & \defeq & \Set{\left(\Set{\evalLE{\lexpr_1} \mapsto \evalLE{\lexpr_2} }, \Set{(\otW, \evalLE{\lexpr_1},\evalLE{\lexpr_2})}\right)} \\
	\evalLS{ \lexpr_1 \fpRW (\lexpr_2, \lexpr_3) } & \defeq & \Set{\left(\Set{\evalLE{\lexpr_1} \mapsto \evalLE{\lexpr_3} }, \Set{(\otR, \evalLE{\lexpr_1},\evalLE{\lexpr_2}), (\otW, \evalLE{\lexpr_1},\evalLE{\lexpr_3})}\right)} \\
	\evalLS{\lpre \sep \lpost} & \defeq & 
    \Setcon{
        (\h_1 \composeH \h_2, \opset_{1} \composeE \opset_{2})
    }{ 
        (\h_{1},\opset_{1}) \in \evalLS{\lpre} 
        \land (\h_{2}, \opset_{2} ) \in \evalLS{\lpost} 
    } 
\end{rclarray}%
\]
\end{definition}

The \emph{transactional assertions} (\cref{def:local_assertions}) have \( \assfalse \), \(\asstrue \), conjunction \( \land \), disjunction \( \lor \), existential quantification \( \exists \), implication \( \implies  \), empty \( \assemp \), fingerprint assertions \( \stub \stackrel{\stub}{\hookrightarrow} \stub \) and separation conjunction \( \sep \).
They describes the state of local snapshot used by a transaction and more importantly the fingerprint of the transaction.
They are interpreted to pairs of snapshots and fingerprints.

A \emph{fingerprint assertion} describes the possible global effect from a transaction.
It includes the default \(\lexpr_{1} \fpI \lexpr_{2} \), the \emph{first read preceding any write} \( \lexpr_{1} \fpR \lexpr_{2} \), \emph{last write} \( \lexpr_{1} \fpW \lexpr_{2} \) for the key \( \lexpr_{1} \) and the combination of them \( \lexpr_{1} \fpRW \lexpr_{2} \).
The \( \lexpr_{1} \fpI \lexpr_{2} \) means the only key \( \lexpr_{1} \) in the local snapshot has value \( \lexpr_{2} \),
and the key has no associated fingerprint.
The \( \lexpr_{1} \fpR \lexpr_{2} \) means the key has been read before any other write carrying value \( \lexpr_{2} \) and the current value for the key is also \( \lexpr_{2} \).
The \( \lexpr_{1} \fpW \lexpr_{2} \) means the key has been written at least once, and the last written value is \( \lexpr_{2} \).
Because read does not change the state of snapshot, and the write fingerprint corresponds to the last write for the key,
so the state of the snapshot matches the fingerprint for cases \( \lexpr_{1} \fpR \lexpr_{2} \) and  \( \lexpr_{1} \fpW \lexpr_{2} \).
Last, The combined fingerprint \( \lexpr_{1} \fpRW (\lexpr_{2}, \lexpr_{3}) \) means the key has been read and then written at least once, the first read fetched value \( \lexpr_{2} \), and the last written value and the current state of the local snapshot for the key are both \( \lexpr_{3} \).

Other transactional assertions have standard interpretations.
Note that the separation conjunction \( \sep \) asserts two local snapshots and fingerprints when the keys are disjointed.
Observe that program expressions $\Expr$ (\cref{fig:semantics}) are contained in logical expressions $\LExpr$ (\cref{def:local_assertions}), \ie $\Expr \subset \LExpr$. 

The proof rules for transactions (\cref{fig:rule-trans}) are standard except \rl{TRLookup} and \rl{TRMutate}.
The \rl{TRLookup} rule adds read label only if there is no write label.
Because once a key has been written, the following reads are local to the transaction.
However, the local read always needs to match the current state of snapshot.
Especially for the case when the precondition is \( \lexpr \fpRW (\lexpr'', \lexpr') \),
the current state of the key is the last written value \( \lexpr' \).
The\rl{TRMutate} rule changes the state of the key and more importantly adds write label.
For the case when the precondition is \( \lexpr \fpRW (\lexpr'', \lexpr') \), 
the rule changes the value \( \lexpr' \) to the new written value \( \lexpr''' \)
but keeps the old read value \( \lexpr'' \) the same.

\begin{figure}[!t]
\sx{Font for E}
\hrule
\begin{mathpar}
    \inferrule[\rl{TRLookup}]{%
        \var \notin \func{fv}{\lexpr}  
        \\ \lpre \toFP{\otR(\expr, \lexpr)} \lpost
    }{%
        \tripleL{ \lpre }{ \plookup{\var}{\expr} }{\var \dot= \lexpr \sep \lpost\sub{\var}{\lexpr} }
    }
    \and
    \inferrule[\rl{TRMutate}]{
        \lpre \toFP{\otW(\expr_{1},\expr_{2})} \lpost
    }{%
        \tripleL{ \lpre }{ \pmutate{\expr_1}{\expr_2} }{ \lpost } 
    }
    \and
    \inferrule[\rl{TRAss}]{
        \var \notin \func{fv}{\lexpr}
    }{%
        \tripleL{\var \dot= \lexpr }{ \pass{\var}{\expr} }{\var \dot= \expr\sub{\var}{\lexpr} }
    }
    \and
    \inferrule[\rl{TRAssume}]{ }{%
        \tripleL{ \expr \dot\neq 0 }{ \passume{\expr} }{ \expr \dot\neq 0 } 
    }
    \and
    \inferrule[\rl{TRChoice}]{%
        \tripleL{ \lpre }{ \trans_{1} }{ \lpost } 
        \\ \tripleL{ \lpre }{ \trans_{2} }{ \lpost } 
    }{%
        \tripleL{ \lpre }{ \trans_{1} \pchoice \trans_{2} }{ \lpost }
    }
    \and
    \inferrule[\rl{TRSeq}]{%
        \tripleL{ \lpre }{ \trans_{1} }{ \lframe }
        \\ \tripleL{ \lframe }{ \trans_{2} }{ \lpost }
    }{%
        \tripleL{ \lpre }{ \trans_{1} \pseq \trans_{2} }{ \lpost }
    }
    \and
    \inferrule[\rl{TRIter}]{%
        \tripleL{ \lpre }{ \trans }{ \lpre } 
    }{%
        \tripleL{ \lpre }{ \trans\prepeat }{ \lpre }
    }
    \and
    \inferrule[\rl{TRFrame}]{%
        \tripleL{ \lpre }{ \trans }{ \lpost } \and \func{fv}{\lframe} \cup \func{modify}{\trans} = \emptyset
    }{% 
        \tripleL{ \lpre \sep \lframe }{ \trans }{ \lpost \sep \lframe }
    }
\end{mathpar}


\hrule
\[
\begin{array}{@{} c @{\qquad} c @{}}
\begin{rclarray}
    \lexpr \fpI \lexpr' & \toFP{\otR(\lexpr,\lexpr')} & \lexpr \fpR \lexpr' \\
    \lexpr \fpR \lexpr' & \toFP{\otR(\lexpr,\lexpr')} & \lexpr \fpR \lexpr' \\
    \lexpr \fpW \lexpr' & \toFP{\otR(\lexpr,\lexpr')} & \lexpr \fpW \lexpr' \\
    \lexpr \fpRW (\lexpr'', \lexpr') & \toFP{\otR(\lexpr,\lexpr')} & \lexpr \fpRW (\lexpr'', \lexpr') \\
\end{rclarray}
&
\begin{rclarray}
    \lexpr \fpI \lexpr' & \toFP{\otW(\lexpr,\lexpr'')} & \lexpr \fpW \lexpr'' \\
    \lexpr \fpR \lexpr' & \toFP{\otW(\lexpr,\lexpr'')} & \lexpr \fpRW (\lexpr',\lexpr'') \\
    \lexpr \fpW \lexpr' & \toFP{\otW(\lexpr,\lexpr'')} & \lexpr \fpW \lexpr'' \\
    \lexpr \fpRW (\lexpr'', \lexpr') & \toFP{\otW(\lexpr,\lexpr''')} & \lexpr \fpRW (\lexpr'', \lexpr''') \\
\end{rclarray}
\end{array}
\]
\hrule
\caption{The rules for transactions}
\label{fig:rule-trans}
 \end{figure}



\subsection{Reasoning programs}

\emph{Capabilities} (\cref{def:capabilities}) are used to specify the allowed operations on concurrent modules.
Each module is associated with \emph{client-specified capabilities} that forms \emph{a partial commutative monoid (PCM)}.
To recall, \emph{a PCM} is a partially ordered set that is closed under a commutative binary operation \( \compose \) and has a set of identify elements \( \unitelem \).
The client-specified capabilities are lifted to \emph{capability composition function} with their associated region identifiers.
For brevity, we often use \emph{capabilities} for \emph{capability composition function}.
The composition function for \emph{capabilities} \( \ca_l \composeC \ca_r \) is defined as point-wise compositing each region and the units \( \unitC \) are functions where regions map to units of client-specified capabilities.


\begin{definition}[Capabilities]
\label{def:capabilities}
Assume a \emph{partial commutative monoid (PCM)} of \emph{client-specified capabilities} \( (\Kaps, \composeK, \unitK) \) with \( \kap \in \Kaps \), the composition \( \composeK \) the units set \( \unitK \).
Then given a set of \emph{region identifiers} \( \rid \in \RegionID \), 
the \emph{capability composition function} or \emph{capabilities} \( \ca \in \Caps \defeq \RegionID \parfun \Kaps \), where the composition \( \composeC \) is defined as the follows:
\[
    \begin{rclarray}
        (\ca_{l} \composeC \ca_{r})(\rid) & \defeq  &
        \begin{cases}
            \ca_{l}(\rid) \composeK \ca_{r}(\rid) & \rid \in \dom(\ca_{l}) \cap \dom(\ca_{l}) \\
            \ca_{l}(\rid)  & \rid \in \dom(\ca_{l}) \setminus \dom(\ca_{l}) \\
            \ca_{r}(\rid) & \rid \in \dom(\ca_{r}) \setminus \dom(\ca_{l}) \\
            \text{undefined} & \text{otherwise} \\
        \end{cases}
    \end{rclarray}
\]
and the units set \( \unitC \defeq \Setcon{\ca}{\fora{\rid} \ca(\rid) \in \unitK } \) .
A capability assertion is in the form of \( \cass{\kap(\vec{\lvar})}{\lrid} \in \CAst \), where \( \kap(\vec{\lvar}) \) is a token parametrised by logical variables and \( \lrid \) is the region identifiers.
The capability assertion is interpreted to a capability in the model by interpreting all the logical expressions,
\[
\begin{rclarray}
    \evalC{\cass{\kap(\vec{\lvar})}{\lrid}} & \defeq & \Set{\lrid \mapsto \kap(\evalLE{\vec{\lvar}})} \\
\end{rclarray}
\]
\end{definition}

The \emph{capability assertions} are in the form of \( \cass{\kap(\vec{\lvar})}{\lrid} \) where \( \kap(\vec{\lvar}) \) is a syntactic capability and \( \lrid \) is a region identifier
They are interpreted to some capabilities by interpreting the syntactic capabilities \( \kap(\vec{\lvar}) \).:
They are resources that grant abilities to access the module, which we will explain later, or act as ghost resources to provide extra information about the module.

Each region is associated with a \emph{interference} to specify how the region can evolve (\cref{def:invariant-region}).
A action in the interference is in the form of \( \exsts{\vec{\lvar}} \perm{\kap} : \bar{\fp} \) and it says if a client holds the capability \( \perm{\kap} \), it is allowed to commit a transaction that has the \emph{fingerprint with capabilities transformation} \( \bar{\fp} \).
The existential is for binding variables between the capability and the fingerprint assertions.
The \emph{fingerprint with capabilities transformation} are fingerprint assertion (\cref{def:fingerprint}) extended with special one for transferring of capabilities, \ie adding to the shared state \( \fpA \cass{\kap}{\lrid} \), deleting from the shared state \( \fpD \cass{\kap}{\lrid} \) and updating the capabilities \( \cass{\kap(\vec{\lvar})}{\lrid} \fpU \cass{\kap(\vec{\lvar})}{\lrid} \). 

\begin{definition}[Interference]
\label{def:intf}
The \emph{fingerprint with capabilities transformation} is defined by the follows:
\[
\begin{rclarray}    
    \bar{\fp}, \bar{\fp}' & ::= & 
    \lexpr \fpI \lexpr 
    \mid \lexpr \fpR \lexpr 
    \mid \lexpr \fpW \lexpr 
    \mid \lexpr \fpRW (\lexpr, \lexpr) \\
    & & \mid \null \fpA \cass{\kap(\vec{\lvar})}{\lrid}  
    \mid \null \fpD \cass{\kap(\vec{\lvar})}{\lrid} 
    \mid \cass{\kap(\vec{\lvar})}{\lrid} \fpU \cass{\kap(\vec{\lvar})}{\lrid} 
    \mid \bar{\fp} \sep \bar{\fp}'
\end{rclarray}
\] 
Given a logical environment $\lenv \in \LEnv$, a stack $\stk \in \Stacks$ and the fingerprint interpretation function (\cref{def:fingerprint}), the \emph{fingerprint with capabilities transformation} is interpreted through function, $\evalF[(., .)]{.}: \FAst \times \LEnv \times \Stacks \parfun \Heaps \times \Opsets \times \Caps \times \Caps$:
\[
\begin{rclarray}
    \evalF{ \bar{\fp} } & \defeq &
        \Setcon{(\h, \opset, \ca, \ca')}{
            (\h,\opset) \in \evalLS{\bar{\fp}} \land \ca, \ca' \in \unitC
        } \quad \text{where} \ \bar{\fp} \in \LAst \\
    \evalF{\null \fpA \cass{\kap(\vec{\lvar})}{\lrid} } & \defeq & 
        \Setcon{(\unitH, \unitO, \ca, \ca')}{
            \ca = \evalC{\cass{\kap(\vec{\lvar})}{\lrid}} \land \ca' \in \unitC
        } \\
    \evalF{\null \fpD \cass{\kap(\vec{\lvar})}{\lrid} } & \defeq &
        \Setcon{(\unitH, \unitO, \ca, \ca')}{
            \ca \in \unitC \land \ca'  = \evalC{\cass{\kap(\vec{\lvar})}{\lrid}} 
        } \\
    \evalF{\cass{\kap(\vec{\lvar})}{\lrid} \fpU \cass{\kap'(\vec{\lvar}')}{\lrid} } & \defeq &
        \Setcon{(\unitH, \unitO, \ca, \ca')}{
            \ca = \evalC{\cass{\kap(\vec{\lvar})}{\lrid}} \land \ca'  = \evalC{\cass{\kap'(\vec{\lvar}')}{\lrid}} 
        } \\
    \evalF{\fp_{1} \sep \fp_{2}} & \defeq & \Setcon{ ( \h_{1} \composeH \h_{2}, \opset_{1} \composeO \opset_{2}, \ca_{1} \composeC \ca_{2}, \ca'_{1} \composeC \ca'_{2} ) }{(\h_{1}, \opset_{1}, \ca_{1}, \ca'_{1}) \in \evalF{\fp_{1}}  \\ {} \land (\h_{2}, \opset_{2}, \ca_{2}, \ca'_{2}) \in \evalF{\fp_{2}}}\\

\end{rclarray}
\]
The grammar of \emph{interference assertions}, \( \intass \in \IAst \), is defined as the follows:
\[
\begin{rclarray}
	\intass & ::=  & \emptyset \mid \Set{ \exsts{\vec{\lvar}} \perm{\kap} : \fp } \cup \intass 
\end{rclarray}
\]
The interference assertions are interpreted to \emph{interference environments} \( \intf \):
\[
\begin{rclarray}
    \inter \in \Interference & \defeq & \Kaps \to ( \HisHeaps \times \Views \times \Caps ) \times  ( \HisHeaps \times \Views \times \Caps )
\end{rclarray}
\]
The \emph{interference interpretation} function, $\evalI[(., .)]{.}: \IAst \times \LEnv \times \Stacks \to \Interference$, is defined as follows:
\sx{
    Notations are confused, there is different between syntactic \( \kap \) which can be parametrised by logical variables, and client-specified capabilities \( \kap' \).
    Fix the typesetting later.
    } 
\[
\begin{array}{@{}l}
	\evalI{\Set{ \exsts{\vec{\lvar}} \perm{\kap} : \fp } \cup \intass }(\kap') \eqdef \\
    	\quad \left\{ 
            \begin{array}{@{}l @{\qquad} l}
            \multicolumn{2}{@{}l@{}}{
                    \Setcon{
                        \begin{B}
                            (\hh, \vi, \ca_r \composeC \ca_f ), \\ 
                            (\hh',\vi', \ca_f \composeC \ca_a)
                        \end{B}
                    }{ 
                        \exsts{\txid, \opset, \cl} \\
                            \quad ( \stub, \opset, \ca_{a}, \ca_{r} ) \in \evalF[\lenv',\stk]{\fp}   \\
                        \quad {} \land \txid \in \func{nextTxid}{\hh, \cl}  \\
                        \quad {} \land \hh' = \updM{\hh, \vi, \txid, \opset}  \\
                        \quad {} \land \pred{readFrom}{\hh, \vi, \opset} 
                        \land \vi' \geq \vi \\
                    } 
                    \cup \evalI{\intass}(\kap')%
            } \\
            & \text{if there exist a logical environment} \ \lenv' \ \text{by replacing} \ \vec{\lvar} \ \text{with some} \ \vec{\val} \ \text{\ie} \\ 
            & \lenv' = \lenv\rmto{\vec{\lvar}}{\vec{\val}}, \ \text{and under the new logical environment} \ \kap' = \evalI[\lenv', \stk]{\kap} \\
            \evalI{\intass}(\kap') 
            & \text{otherwise} \\
    	    \end{array}
        \right.  \\
\end{array}
\]
The \( \predn{readFrom} \) asserts the fingerprint makes sense with respect to the view:
\[
\begin{rclarray}
    \pred{readFrom}{\hh, \vi, \opset} & \defeq & \fora{\ke, \val} (\otR, \ke, \val) \in \opset \implies \valueOf(\hh(\ke,\vi(\ke))) = \val
\end{rclarray}
\]
\end{definition}


%The interference \( \exsts{\vec{\lvar}} \perm{\kap} : \bar{\lpre} \mat \fp \) says if a thread holds the capability \( \perm{\kap}\) and \emph{the current state of database} satisfies the assertions \( \bar{\lpre} \), the thread is allowed to commit a transaction that has the fingerprint \( \fp \).
%The current state of database refers to the state that all the committed transactions are visible.

%\begin{definition}[Interference]
%\label{def:intf}
%Assume standard separation logic assertion \( \bar{\lpre}\) (the local assertion \( \LAst \) without fingerprint).
%Given the fingerprint assertion \( \fp \in \Fingerprint \) (\defref{def:fingerprint}), the grammar of \emph{interference assertions}, \( \intass \in \IAst \), is defined as the follows,
%\[
%\begin{rclarray}
	%\intass & ::=  &
	%\emptyset \mid \Set{ \perm{\kap} :  \exsts{\vec{\lvar}} \bar{\lpre} \mat \fp } \cup \intass 
%\end{rclarray}
%\]
%The interference assertions are interpreted to a set of \emph{interference environments} that is a function from client-specified capabilities to pairs of history heaps and operations,
%\[
%\begin{rclarray}
    %\inter \in \Interference & \defeq & \Kaps \to \powerset{\HisHeaps \times \Opsets}
%\end{rclarray}
%\]
%Given a logical environment $\lenv \in \LEnv$ and a stack $\stk \in \Stacks$, the \emph{interference interpretation} function, $\evalI[(., .)]{.}: \IAst \times \LEnv \times \Stacks \to \Interference$, is defined as follows,
%%
%\[
%\begin{rclarray}
	%\evalI{\emptyset}(\kap) & \eqdef & \emptyset \\
	%\evalI{\Set{ \perm{\kap} : \exsts{\vec{\lvar}} \bar{\lpre} \mat \fp } \cup \intass }(\kap') & \eqdef &
    %\begin{cases}
    %\Setcon{(\hh, \evalF[\lenv',\stk]{\fp})}{\exsts{\h} \h \in \evalLS[\lenv',\stk]{\bar{\lpre}} \land {} \\ \h = \clpsHH{\hh} } \cup \evalI{\intass}(\kap')  & \kap = \kap' \\
    %\evalI{\intass}(\kap') & \text{ otherwise} \\
    %\end{cases} \\
    %& & \text{where there exists a vector of values \( \vec{\val}\) such that } \lenv' = \lenv\rmto{\vec{\lvar}}{\vec{\val}} \\
%\end{rclarray}
%\] 
%\end{definition}

%We will write \( \intfH(\kap) \)  and \( \intfO(\kap) \) for the first and second projections of all the elements.


%\begin{definition}[Labelled transition system]
%\label{def:labelled-transition-system}
%The labelled transition system is a tuple \( ( \hhset \times \cuset, \opsetset,\toLTS{}, \hhset_{0} \times \cuset_{0}, \como) \) consisting of pairs of history heaps and cuts \( \hhset \times \cuset \), a set of sets of operations \( \opsetset \subseteq \Opsets \), a relation \( \toLTS{} : \HisHeaps \times \Opsets \times \HisHeaps \), a set of initial history heaps and cuts \( \hhset_{0} \times \cuset_{0} \) and the consistency model associated with the transition system \( \como \).
%Assume all the initial abstract executions satisfies the consistency model.
%The relation \( \toLTS{}\) is defined as the follows,
%\[
%\begin{rclarray}
    %(\hh, \cu) \toLTS{\opset} (\hh',\cu') & \defeq &
    %\begin{array}[t]{@{}l}
        %\exsts{\thcu, \thcu', \txid, \thid}
        %\txid \in \fresh{\hh} 
        %\land \hh' = \updM{\hh, \cu, \txid, \opset} 
        %\land \cu' = \updV{\hh', \cu, \opset} \\
        %\quad {} \land ((\hh,\thcu),(\hh',\thcu')) \in \como
        %\land \h = \clpsHH{\hh,\cu} 
        %\land \thcu(\thid) = \cu 
        %\land \thcu'(\thid) = \cu' \\
        %\quad {} \land \fora{\addr,\val} (\otR, \addr, \val)  \in \opset \implies \h(\addr) = \val
    %\end{array}
%\end{rclarray}
%\]
%\end{definition}

%We lift the interference to a invariant.
%The invariant is a labelled transition system that describes how a region evolves providing all the allowed operations.
%Note that the labels are capabilities (with region identifiers) instead of client-specified capabilities, which is only for technical reason.

\begin{definition}[Invariant of a region]
\label{def:invariant-region}
Assume a global function, \( \funcn{init} : \RegionID \to \powerset{ \HisHeaps } \) returning initial key-value stores for regions.
Given the initial states for a region \( \func{init}{\rid}\), the invariant of a region, written \( \func{inv}{\rid, \intf} \), is a set of key-value stores that is closed under the interference \( \intf \):
\[
\begin{array}[t]{@{}l}
    \func{init}{\rid} \in \func{inv}{\rid, \intf} \land 
    \fora{\mkvs, \mkvs'} 
    \exsts{\w, \w', \kap} \\
    \quad \mkvs \in \func{inv}{\rid, \intf} \land  ( \mkvs, \stub ) = \eraseW{\w} \land ( \mkvs', \stub ) = \eraseW{\w'} \land (\w, \w') \in \intf(\kap) \implies \mkvs' \in \func{inv}{\rid, \intf} 
\end{array}
\]
\end{definition}

%For brevity, \( (\hh,\cu) \in \func{inv}{\rid, \intf} \) denotes \( (\hh,\cu) \in \func{inv}{\rid,\intf}\projection{1} \), and similarly \( (\hh,\cu) \toLTS{\opset} (\hh',\cu') \in \func{inv}{\rid, \intf} \).

%\sx{This well form condition allows one to write weaker interference, \eg interference satisfies both SI and SER but to prove the correctness of SER. It is fine since logic only need to be sound?}
%\begin{definition}[Well-form of a region]
%\label{def:well-form-region}
%The well-form condition of the interference, namely \( \pred{wfintf}{\rid, \intf} \) predicate, assertions for any concrete events \( \opset \), the state before the events must be included in the interference.
%\[
%\begin{rclarray}
    %\pred{wfintf}{\rid, \intf} & \defeq & 
    %\begin{array}[t]{@{}l}
        %\fora{\hh, \hh', \opset} 
        %(\hh, \stub) \toLTS{\opset} (\hh',\stub) \in \func{inv}{\rid, \intf} 
        %\implies \exsts{ \kap}
        %(\hh, \opset ) \in \intf( \kap )
    %\end{array} \\
%\end{rclarray}
%\]
%\end{definition}

\begin{definition}[Worlds]
\label{def:world}
Given the set of history heaps $\HisHeaps$ (\cref{def:his_heap}), views \( \Views \) (\cref{def:views}), capabilities \( \Caps\) (\cref{def:capabilities}) and region identifiers \( \RegionID \), the set of \emph{shared states} is \( \SStates \eqdef \RegionID \parfun \HisHeaps \times \Views \times \Caps \times \Interference \).
Each region has its current state and the interference.
The \emph{shared state composition function}, $\composeS: \SStates \times \SStates \parfun \SStates$, is defined as $\composeS \eqdef \composeEq$, where for all domains $\sort M$ and all $m, m' \in \sort M$,
%
\[
\begin{rclarray}
	m \composeEq m' &  \eqdef  &
	\begin{cases}
		m & \text{if } m = m'\\
		\text{undefined} & \text{otherwise}
	\end{cases}
\end{rclarray}
\]
A \emph{world} \( \w \in \World \) is a pair of capabilities \( \ca \) (\cref{def:capabilities}) and a shared state \( \gs \) in which regions are well-formed, \ie (a) they are associated with disjointed part of history heaps; (b) the domain of the view in a region is the same as the domain of the history heap; and (c) the views should not be out of the range of history heaps.
Separately, capabilities from regions and local capabilities are compatible.
These constraints are derived by the clap \(\eraseS{(\ca, \gs)} \neq \emptyset \).
Finally, there is no garbage capability, a capability where the associated region identifier never appear in the shared state.
\[
\begin{rclarray}
	\world \in \World  & \eqdef & 
    \Setcon{
        (\ca, \gs) 
    }{ 
        \ca \in \Caps \land \gs \in \SStates
        \land \exsts{ \ca' } 
        (\stub, \stub, \ca') \in \func{collapse}{\gs}
        \land \dom(\ca \composeC \ca') \subseteq \dom(\gs)  \\
        \quad {} \land \fora{\rid}
        \exsts{\hh, \vi, \intf} 
        \gs(\rid) = (\hh, \vi, \stub, \intf) 
        \land \dom(\hh) = \dom(\vi) 
        \land \mkvs \in \func{inv}{\rid, \intf} \\
        \quad {} \land \fora{ \addr \in \dom(\vi) }
        0 \leq \cu( \addr ) \le \left| \hh(\addr) \right|
    }
\end{rclarray}
\]
where the \(\funcn{collapse} \) function collapses a shared state by erasing the region identifiers:
\[
\begin{rclarray}
    \func{collapse}{\emptyset} & \defeq & \Set{(\unitHH, \unitVI, \unitC )} \\
    \func{collapse}{\Set{\rid \mapsto (\hh, \vi, \ca, \intf)} \uplus \gs } & \defeq & 
        \Setcon{ 
            (\hh \composeHH \hh', \vi \composeCU \vi', \ca \composeC \ca') 
        }{ 
            \land (\hh', \vi', \ca') \in \func{collapse}{\gs} }\\
\end{rclarray}
\] 
% 
The \emph{world composition function}, $\composeW: \World \times \World \parfun \World$, is defined component-wise as: $\composeW \eqdef (\composeC, \composeS)$.
The \emph{world unit set} is $\unitW \eqdef \Setcon{(\ca, \gs)}{(\ca, \gs) \in \World \land \ca \in \unitC}$.
The \emph{partial commutative monoid of worlds} is $(\World, \composeW, \unitW)$.
Because of the well-formedness condition, the function \( \eraseW{.} : \World \to \HisHeaps \times \Views \) collapses a world to \emph{a unique pair of a key-value store and a view}:
\[
\begin{rclarray}
    \eraseW{\w} & \defeq & (\hh, \vi) \text{  where } (\hh, \vi, \stub) \in \func{collapse}{\w\projection{2}}\\
\end{rclarray}
\] 
\end{definition}

\begin{definition}[Assertions]
\label{def:assertion}
Given the set of logical expression \( \lexpr \subseteq \LExpr\), the set of \emph{assertions}, $\gpre, \gpost \in \Ast$, are defined by the following inductive grammar:
\[
\begin{rclarray}
    \bar{\lpre}, \bar{\lpost} & ::= & \False \mid \True \mid \bar{\lpre} \land \bar{\lpost} \mid \bar{\lpre} \lor \bar{\lpost} \mid \exsts{\lvar} \bar{\lpre} \mid \bar{\lpre} \implies \bar{\lpost} \mid \assemp \mid \cass{\kap}{\lrid} \mid \lexpr \pt \lexpr \mid \bar{\lpre} \sep \bar{\lpost} \\
	\gpre , \gpost & ::= & \False \mid \True \mid \gpre \land \gpost \mid \gpre \lor \gpost \mid \exsts{\lvar}\gpre \mid \gpre \implies \gpost \mid \assemp \mid \cass{\kap}{\lrid} \mid \gpre \sep \gpost \mid \boxass{\bar{\lpre}}{\lrid}{\intass}\\
\end{rclarray}
\]
%
where $\lvar, \lrid \in \LVar$, $\lexpr_1, \lexpr_2 \in \LExpr$ (\cref{def:local_assertions}), $\kap \in \Kaps$ (\cref{def:capabilities}) and $\intass \in \IAst$ (\cref{def:intf}).
Given a logical environment $\lenv \in \LEnv$ and a stack $\stk \in \Stacks$, the \emph{assertion interpretation} function, $\evalW[(., .)]{.}: \Ast \times \LEnv \times \Stacks \to \powerset{\World}$, is defined as follows,
%
\[
\begin{rclarray}
	\evalW{\False} & \defeq & \emptyset \\
	\evalW{\True} & \defeq & \World \\
	\evalW{\emp} & \defeq & \unitW \\
	\evalW{\gpre \land \gpost} & \defeq & \evalW{\gpre} \cap \evalW{\gpost} \\
	\evalW{\gpre \lor \gpost} & \defeq & \evalW{\gpre} \cup \evalW{\gpost} \\ 
	\evalW{\exsts{\lvar}  \gpre} & \defeq & \bigcup\limits_{\val \in \textnormal{\Val}} \evalW[\lenv\remapsto{\lvar}{\val}, \stk]{\gpre} \\
	\evalW{\gpre \implies \gpost} & \defeq & \Setcon{\w}{\w \in \evalW{\gpre} \implies \w \in \evalW{\gpost}} \\
	\evalW{\cass{\kap}{\lrid}} & \defeq & \Setcon{ (\Set{\lrid \mapsto \evalI{\kap}}, \gs) }{\gs \in \SStates} \\
	\evalW{ \gpre \sep \gpost } & \defeq & 
	\Setcon{
	   (\world_1 \composeW \world_2) 
    }{
       \world_1 \in \evalW{\gpre} \land \world_2 \in \evalW{\gpost}
	} \\
	\evalW{ \boxass{\bar{\lpre}}{\lrid}{\intass} } & \defeq & 
    \Setcon{
        (\ca, \gs)
    }{         
        \exsts{\hh, \vi, \ca', \intf}
        \ca \in \unitC  \\
        \quad {} \land \intf = \evalI{\intass} 
        \land \gs(\lrid) = (\hh, \vi, \ca', \intf) 
        \land (\hh, \vi, \ca') \in \func{intp}{\bar{\lpre}, \lenv, \stk} 
    } \\
    \\
    \func{intp}{\assfalse,\lenv,\stk} & \defeq & \emptyset \\
    \func{intp}{\asstrue,\lenv,\stk} & \defeq & \HisHeaps \times \Views \times \Caps \\
    \func{intp}{\assemp,\lenv,\stk} & \defeq & \Setcon{ (\unitHH, \unitVI, \ca) }{\ca \in \unitC } \\
    \func{intp}{\bar{\lpre} \land \bar{\lpost},\lenv,\stk} & \defeq & \func{intp}{\bar{\lpre},\lenv,\stk} \cap \func{intp}{\bar{\lpost},\lenv,\stk} \\ 
    \func{intp}{\bar{\lpre} \lor \bar{\lpost},\lenv,\stk} & \defeq & \func{intp}{\bar{\lpre},\lenv,\stk} \cup \func{intp}{\bar{\lpost},\lenv,\stk} \\ 
    \func{intp}{\exsts{\lvar} \bar{\lpre},\lenv,\stk} & \defeq & \bigcup\limits_{\val \in \Val} \func{intp}{\bar{\lpre}, \lenv\rmto{\lvar}{\val}, \stk} \\
    \func{intp}{\bar{\lpre} \implies \bar{\lpost},\lenv,\stk} & \defeq & \Setcon{ (\hh, \vi, \ca) }{ (\hh, \vi, \ca) \in \func{intp}{\bar{\lpre},\lenv,\stk} \implies (\hh, \vi, \ca) \in \func{intp}{\bar{\lpost},\lenv,\stk} }\\
    \func{intp}{\cass{\kap}{\lrid},\lenv,\stk} & \defeq & \Set{ (\unitHH, \unitVI, \Set{\lrid \mapsto \evalC{\kap}}) }\\
    \func{intp}{\lexpr_{1} \pt \luexpr_2,\lenv,\stk} & \defeq & \Setcon{ (\hh, \vi, \ca) }{ \Set{ \evalLE{\lexpr_{1}} \mapsto \evalLE{\luexpr_{1}} } = \clpsHH{\hh, \vi} \land \ca \in \unitC } \\
    \func{intp}{\lexpr_{1} \pt \lexpr_2,\lenv,\stk} & \defeq & 
    \Setcon{ (\hh, \vi, \ca) }{%
        \exsts{\ke = \evalLE{\lexpr_{1}}} \Set{ \ke \mapsto \evalLE{\lexpr_{2}} } = \clpsHH{\hh, \vi} \\ 
        \quad {} \land \vi(\ke) = \lvert \mkvs(\ke) \rvert - 1 \land \ca \in \unitC
    } \\
    \func{intp}{\bar{\lpre} \sep \bar{\lpost},\lenv,\stk} & \defeq & 
    \Setcon{ (\hh \composeHH \hh', \vi \composeVI \vi', \ca \composeC \ca') }{ (\hh, \vi, \ca) \in \func{intp}{\bar{\lpre},\lenv,\stk} \\ {} \land (\hh', \vi', \ca') \in \func{intp}{\bar{\lpost},\lenv,\stk} } \\
\end{rclarray}
\]
\end{definition}



\subsubsection{Rely and Guarantee}

The \emph{rely} and \emph{guarantee} describes the world transformation for the environment and for the current client (\cref{def:rely-guarantee}).
To recall, a world includes local capabilities and a shared state, and a shared state is a client's view for the key-value store.
The \emph{rely} \( \Rely \) is a set of pairs on worlds, describing how the environment can change the state of the key-value store.
Given the local capabilities \( \ca_l \), the environment might own any capabilities \( \ca_e\) that is compatible, \ie \( (\ca_l \composeC \ca_e)\isdef \).
Therefore, the environment can perform actions associated with the their capabilities \( \ca_e \) with their own view \( \vi_e \) to update the key-value store and shared capabilities.
For technical reasons, even though the environment cannot change the view of the current client \( \vi_r\), but it is allowed to arbitrarily shift to the later versions due to the fact that for certain execution tests, the old view might be valid under the new key-value store.

The \emph{guarantee} \( \Guar \) describes the allowed actions for the current client.
The current client can perform actions associated with the local capabilities \( \ca_l \) to update the shared state and the local capabilities.
Yet it should ensure no resource created or deleted by requiring the \emph{orthogonal} of local capabilities and shared capabilities together remains unchanged.
The orthogonal of local capabilities \( \ca \) is a set of capabilities that are compatible with the local capabilities \( \ca \).
This constraint disallow any creation and deletion for capabilities, but it allows to update capabilities.
Note that in the rely and guarantee, it is allowed to update several regions, but each region can be updated at most once.
\begin{definition}[Rely and guarantee]
\label{def:rely-guarantee}
Given the set of worlds $\World$ (\cref{def:world}), the \emph{update rely} relation, $\relyU \subseteq \World \times \World$, is defined as follows:
\sx{In case I get confused again, it is world to world so the shared and local capabilities should always make sense.}
\[	
    \begin{rclarray}
	\relyU & \eqdef &
	\Setcon{
		((\ca_l,\gs), (\ca_l, \gs'))	
	}{
        \exsts{\ca_e}
        (\ca_e \composeC \ca_l) \isdef
        \land \fora{\rid}
        \gs(\rid) = \gs'(\rid) \lor 
        \exsts{\kap, \hh, \hh', \vi_\rid, \vi_{\rid}', \vi_{e}, \vi_{e}', \ca_\rid, \ca_{\rid}', \intf}   \\
        \quad \gs(\rid) = (\hh, \vi_\rid, \ca_\rid, \intf)
        \land \gs'(\rid) = (\hh', \vi_{\rid}', \ca_{\rid}',\intf) \\
        \quad {} \land \kap \sqsubseteq \ca_{e}(\rid) 
        \land ( (\hh, \vi_e, \ca_e), (\hh', \vi_{e}', \ca_{e}') )  \in \intf(\kap)
        \land \vi_{\rid}' \geq \vi_\rid
	} \\
    \end{rclarray}
\]
The \emph{view shift rely} relation $\relyV \subseteq \World \times \World$, is defined as follows:
\[
    \begin{rclarray}
	\relyV & \eqdef &
	\Setcon{
		((\ca_l,\gs), (\ca_l, \gs'))	
	}{
        \exsts{\ca_e}
        (\ca_e \composeC \ca_l) \isdef
        \land \fora{\rid}
        \gs(\rid) = \gs'(\rid) \lor 
        \exsts{\hh, \vi, \vi'\ca, \ca, \intf}   \\
        \quad \gs(\rid) = (\hh, \vi, \ca, \intf)
        \land \gs'(\rid) = (\hh, \vi', \ca, \intf) 
	} \\
    \end{rclarray}
\]
The \emph{rely} relation is transitive closure of updates and view shift: \( \Rely = (\relyU \cup \relyV)^{*} \).
The \emph{guarantee} relation, $\Guar: \World \times \World$, is defined as follows:
\[	
    \begin{rclarray}
	\Guar & \eqdef &
	\Setcon{
		((\ca_l,\gs), (\ca_{l}', \gs'))	
	}{
        \fora{\rid}
        \gs(\rid) = \gs'(\rid) \lor {}
        \exsts{\kap, \hh, \hh', \vi_\rid, \vi_{\rid}', \ca_\rid, \ca_{\rid}', \intf}   \\
        \quad \gs(\rid) = (\hh, \vi_\rid, \ca_\rid,\intf)
        \land \gs'(\rid) = (\hh', \vi_{\rid}', \ca_{\rid}',\intf) 
        \land \kap \sqsubseteq \ca_{l}(\rid)  \\
        \quad {} \land ( (\hh, \vi_\rid, \ca_\rid), (\hh', \vi_{\rid}', \ca_{\rid}') )  \in \intf(\kap)
        \land (\ca_{l} \composeC \ca_\rid)^{\perp} = (\ca_{l}' \composeC \ca_{\rid}')^{\perp}
	} \\
    \end{rclarray}
\]
where for any element \( m \) from its domain \( \sort{M} \), the  \emph{orthogonal} is defined as:
\[
\begin{rclarray}
m^{\perp} & \defeq & \Setcon{m'}{(m \compose{} m')\isdef \land m' \in \sort{M}} 
\end{rclarray}
\]
\end{definition}

The stabilisation says assertions remain true against the environment.
Formally a set of worlds \( \setworld \) is stable under certain execution tests \( \et \) if the set is closed under rely relation under the side conditions: (a) the key-value store transfer is allowed by the execution tests \( (\mkvs, \mkvs') \in \et \);
and (b) the new view under the new key-value store is able to progress.
The first condition says there is at least one view from the environment that can trigger the transformation on the key-value store, and it is allowed by the execution tests.
The second condition is more subtle, as it allows to also update the view to a new view \( \vi' \) so that for all the possible transactions with the fingerprints \( \opset \), if they can execute under the old view, it should be able to execute under the new view.

\begin{definition}[Stable]
\label{def:stable}
A set of worlds $\setworld \subseteq \World$ is \emph{stable}, written $\stable{\setworld, \como}$, if and only if it is closed under the rely relation: 
\[
    \begin{rclarray}
        \stable{\setworld, \como} & \eqdef & 
        \begin{array}[t]{@{}l}
            \fora{\w, \w'} 
            \w \in \setworld 
            \land (\w, \w') \in \Rely  
            \land \exsts{\hh, \hh', \vi, \opset} \\
            \quad (\hh, \vi) = \eraseW{\w}
            \land (\hh', \vi') = \eraseW{\w'} 
            \land (\hh, \hh') \in \como 
            \land \pred{progress}{\w, \et} \\
            \qqquad \implies \w' \in \setworld
        \end{array} \\
    \end{rclarray}
\]
A update history heap update is allowed by consistency model, \ie \( (\hh, \hh') \in \como \), iff there exist some view \( \vi \) and operation set \( \opset \) allowed by the consistency model and the history is updated to \( \hh' \) via them:
\[
    \begin{rclarray}
        (\hh, \hh') \in \como & \eqdef & 
        \begin{array}[t]{@{}l}
            \hh = \hh' \lor 
            \exsts{ \vi, \vi', \opset, \txid, \cl}  \\
            \quad (\hh, \vi) \csat \opset : \vi' 
            \land \txid \in \fresh{\hh, \cl} 
            \land \hh'  = \updM{\hh,\vi, \txid, \opset}
        \end{array}
    \end{rclarray}
\]
A world is able to progress under an execution test \( \et \), iff it is able to execute the empty fingerprint:
\[
    \begin{rclarray}
        \pred{progress}{\w, \et} & \defeq & \exsts{\vi} \eraseW{\w} \csat \unitO :  \vi
    \end{rclarray}
\]
\end{definition}


\subsection{Rules for Global}

The \rl{PRCommit} rule lifts the local effect of transaction \( \trans \) to global level by first converting global state to (local) observable state and then propagating the local fingerprint to the global state.
%The \( \predn{down} \) predicate asserts that the local predicate \( \lpre \) is a over-approximation of the valid observation that is given by the interference.
%The \( \predn{up} \) predicate says the post-condition \( \gpost \) is the result by lifting the local fingerprints \( \fp \) to pre-condition \( \gpre \).


\begin{figure}[t!]
\hrule\vspace{5pt}

\begin{mathpar}
    \inferrule[\rl{PRCommit}]{%
        \tripleL{\lpre}{\trans}{\lpost} 
        \\ \repartition{\gpre}{\gpost}{\lpre}{\lpost}
        \\\\ \stable{\gpre, \como} 
        \\ \stable{\gpost, \como} 
    }{%
        \tripleG{\gpre}{ \ptrans{\trans} }{\gpost}
    }
    \and
    \inferrule[\rl{PRPar}]{%
        \tripleG{ \gpre_{1} }{ \cmd_{1} }{ \gpost_{1} }
        \\ \tripleG{ \gpre_{2} }{ \cmd_{2} }{ \gpost_{2} } 
        \\\\ \stable{\gpre_{1}, \como} 
        \\ \stable{\gpre_{2}, \como} 
    }{%
        \tripleG{ \gpre_{1} \sep \gpre_{2} }{ \cmd_{1} \ppar \cmd_{2} }{ \gpost_{1} \sep \gpost_{2} }
    }
    \and
    \inferrule[\rl{PRAss}]{%
        \thvar \notin \func{fv}{\lexpr} 
    }{%
        \tripleG{\thvar \dot= \lexpr }{ \pass{\thvar}{\expr} }{\thvar \dot= \expr\sub{\thvar}{\lexpr} }
    }
    \and
    \inferrule[\rl{PRAssume}]{ }{%
        \tripleG{ \expr \dot\neq 0 }{ \passume{\expr} }{ \expr \dot\neq 0 } 
    }
    \and
    \inferrule[\rl{PRChoice}]{%
        \tripleG{ \gpre }{ \cmd_{1} }{ \gpost } 
        \\ \tripleG{ \gpre }{ \cmd_{2} }{ \gpost } 
    }{%
        \tripleG{ \gpre }{ \cmd_{1} \pchoice \cmd_{2} }{ \gpost }
    }
    \and
    \inferrule[\rl{PRSeq}]{%
        \tripleG{ \gpre }{ \cmd_{1} }{ \gframe }
        \\ \tripleG{ \gframe }{ \cmd_{2} }{ \gpost }
    }{%
        \tripleG{ \gpre }{ \cmd_{1} \pseq \cmd_{2} }{ \gpost }
    }
    \and
    \inferrule[\rl{PRIter}]{%
        \tripleG{ \gpre }{ \cmd }{ \gpre } 
    }{%
        \tripleG{ \gpre }{ \cmd\prepeat }{ \gpre }
    }
    \and
    \inferrule[\rl{PRFrame}]{%
        \tripleG{ \gpre }{ \cmd }{ \gpost } 
        \\ \stable{\gframe, \como}
        \\ \func{fv}{\gframe} \cap \func{modify}{\cmd} = \emptyset 
    }{%
        \tripleG{ \gpre \sep \gframe }{ \cmd }{ \gpost \sep \gframe }
    }
\end{mathpar}


\hrule\vspace{5pt}
\[
\begin{rclarray}
    \repartition{\gpre}{\gpost}{\lpre}{\lpost} & \defeq & 
    \begin{array}[t]{@{}l@{}}
        \fora{ \w, \lenv, \stk } 
        \w \in \evalW{\gpre} 
        \implies 
        (\getSN{\eraseW{\w}}, \unitO) \in \evalLS{\lpre}  \\
        \quad {} \land \fora{\stk', \txid, \opset, \w', \cl} 
        \txid \in \fresh{\eraseW{\w}\projection{1}, \cl} 
        \land (\stub, \opset) \in \evalF[\lenv, \stk']{\lpost} \\
        \qquad {} \land \eraseW{\w'}\projection{1} = \updM{\eraseW{\w}, \txid, \opset}  \\
        %\land \eraseW{\w}\projection{2} \leq \eraseW{\w'}\projection{2} \\
        \qquad {} \land (\w, \w') \in \Guar  
        \land \eraseW{\w} \csat \opset : \eraseW{\w'}\projection{2}
        \implies \w' \in \evalW[\lenv, \stk']{\gpost}
    \end{array} 
\end{rclarray}                          
\]

\hrule\vspace{5pt}
\caption{The rules for programs}
\label{fig:rule-prog}
\end{figure}

%\azalea{
    %\sx{How to deal with the stack here? As the stack for P and Q are different, just for all quantify two stacks??}
%The quantification seems wrong. Especially, the $\extopset$ needs to be for all quantified, $\h$ needs to be there exist quantified.
%\[
     %\repartition{\gpre}{\gpost}{\lpre}{\lpost} \defeq 
     %\begin{array}[t]{@{}l@{}}
		 %\fora{\w, \hh, \vi, \lenv, \stk, \txid } 
            %\w \in \evalW{\gpre} 
            %\land (\hh, \cu) \in \eraseW{\w}
            %\Rightarrow\\
            %\quad \exsts{\h}
            %\begin{array}[t]{@{} l @{}}
			%\h = \getSN{\hh, \cu}  
                %\land (\h, \unitO) \in \evalLS{\lpre} \\
                %\land\, \fora{\extopset \in \evalF{\lpost}} 
                    %\exsts{\w', \hh', \vi'} \\
                        %\quad \hh' = \updM{\hh, \vi, \txid, \extopset} 
                       %\land \cu' = \updV{\hh, \vi, \extopset} 
                       %\land (\hh',\vi') \in \eraseW{\w'} \\
                        %\quad \land (\w, \w') \in \Guar  
                        %\land (\w, \w') \in \como
                        %\land \w' \in \evalW{\gpost}			
		%\end{array}	
%%		\right)			        	
    %\end{array} 
%\]
%}

%The \( \HHupdate \) and \( \Vupdate \) in the repartition can be replaced by syntactic rules.

%\begin{figure}
%\hrule\vspace{5pt}

%\[
   %\infer[\rl{FInit}]{%
       %\tripleF{ \lexpr \pt \lexpr }{ \lexpr \fpI \lexpr }{ \lexpr \pt \lexpr }
   %}{}
%\]

%\[
   %\infer[\rl{FRead}]{%
       %\tripleF{ \lexpr \pt \lexpr }{ \lexpr \fpR \lexpr }{ \lexpr \pt \lexpr }
   %}{}
%\]

%\[
   %\infer[\rl{FWrite}]{%
       %\tripleF{ \lexpr \pt \lexpr  }{  \lexpr \fpW \lexpr' }{ \lexpr \pt \lexpr' }
   %}{}
%\]

%\[
   %\infer[\rl{FReWrt}]{%
       %\tripleF{ \lexpr \pt \lexpr  }{  \lexpr \fpRW (\lexpr,\lexpr') }{ \lexpr \pt \lexpr' }
   %}{}
%\]

%\[
   %\infer[\rl{FFrame}]{%
       %\tripleF{ \bar{\lpre}_{1} \sep \bar{\lpre}_{2}  }{  \bar{\fp}_{1} \sep \bar{\fp}_{2} }{ \bar{\lpost}_{1} \sep \bar{\lpost}_{2} }
   %}{
       %\tripleF{ \bar{\lpre}_{1} }{ \bar{\fp}_{1} }{ \bar{\lpost}_{1} }
       %&& \tripleF{ \bar{\lpre}_{2}  }{ \bar{\fp}_{2} }{ \bar{\lpost}_{2} }
    %}
%\]


%\hrule\vspace{5pt}
%\caption{Syntactic rule for \( \HHupdate \) and \( \Vupdate \) functions}
%\label{fig:rule-prog}
%\end{figure}


\section{Soundness}


\begin{thm}[Transaction soundness]
\label{thm:transaction-soundness}
Assume the standard lift for transaction interpretation and for fingerprint heaps composite \( \composeFP \), the transaction soundness is as follows:
\[
    \begin{array}{@{}l@{}}
        \for{ \lpre, \trans, \lpost } \tripleL{\lpre}{\trans}{\lpost} \implies \ \tripleSemL{\lpre}{\trans}{\lpost} \\
    \end{array}
\]
where,
\[
    \begin{rclarray}
    \tripleSemL{\lpre}{\trans}{\lpost} & \eqdef &
    \begin{array}[t]{@{}l@{}}
        \for{\lenv, \stk_{p}, \stk_{q}, \fph_{p}, \fph_{q} } 
        \fph_{p} \in \evalLS[\lenv, \stk_{p}]{\lpre}
        \land (\stk_{p}, \fph_{p} ), \trans \toL^{*}  (\stk_{q}, \fph_{q} ), \pskip 
        \implies \fph_{q} \in \evalLS[\lenv, \stk_{q}]{\lpost}
    \end{array}
    \end{rclarray}
\]
\end{thm}
\begin{proof}
Induction on the rules for transactions.

\caseB{\rl{TRSkip}}

We have  \(\trans \equiv \pskip\), \( \lpre \equiv \lpost \equiv \assemp \).
Given the semantics in \fig \ref{fig:thread_semantics}, we have \( \fph_{p} = \fph_{q} = \unitFP \) and \( \stk_{p} = \stk_{q} \), so \( \fph_{q} \in \evalLS[\lenv, \stk_{q}]{\lpost} \).

\caseB{\rl{TRAss}}

We have \(\trans \equiv ( \pass{\var}{\expr} ) \), \( \lpre \equiv ( \var \doteq \lexpr ) \) and \( \lpost \equiv ( \var \doteq \expr\sub{\var}{\lexpr} ) \) for some \( \var, \lexpr, \expr \).
Given the semantics in \fig \ref{fig:thread_semantics}, there exists \( \stk \) such that \( \stk = \stk_{p} \setminus \Set{\var \mapsto \stub} = \stk_{q} \setminus \Set{\var \mapsto \stub} \).
Given the premiss of \rl{TRAss}  in \fig \ref{fig:rule-trans} that \( \var \notin \func{fv}{\lexpr} \), it has \( \evalLE{\lexpr} = \evalLE[\lenv, \stk_{p}]{\lexpr} = \evalLE[\lenv, \stk_{q}]{\lexpr} \), and then must exist \( \val \) so that \( \val = \evalLE{\expr\sub{\var}{\lexpr}} = \evalLE[\lenv, \stk_{q}]{\expr\sub{\var}{\lexpr}} \) and \( \stk_{q} = \stk_{p}\remapsto{\var}{\val} \).
Also because \( \fph_{p} = \fph_{q} = \unitFP \), we have \( \fph_{q} \in \evalLS[\lenv, \stk_{q}]{\lpost} \).

\caseB{\rl{TRDeref}}

We have  \(\trans \equiv ( \pderef{\var}{\expr} ) \), \( \lpre \equiv ( \expr \fpt{\fp} \lexpr ) \) and \( \lpost \equiv ( \var \doteq \lexpr \sep \expr \fpt{\addFPR{\fp}} \lexpr ) \) for some \( \var, \fp, \lexpr, \expr \).
Given the semantics in \fig \ref{fig:thread_semantics}, there exists \( \stk \) such that \( \stk = \stk_{p} \setminus \Set{\var \mapsto \stub} = \stk_{q} \setminus \Set{\var \mapsto \stub} \).
Given the premiss of \rl{TRDeref} in \fig \ref{fig:rule-trans} that \( \var \notin \func{fv}{\lexpr} \), it must exist \( \val \) and \( \addr \) such that \( \val = \evalLE{\lexpr} = \evalLE[\lenv, \stk_{p}]{\lexpr} = \evalLE[\lenv, \stk_{q}]{\lexpr} \), \( \addr = \evalLE{\expr} = \evalLE[\lenv, \stk_{p}]{\expr} = \evalLE[\lenv, \stk_{q}]{\expr} \) and \(  \stk_{q} = \stk_{p}\remapsto{\var}{\val} \).
Also, because \( \lpre \equiv ( \expr \fpt{\fp} \lexpr ) \), we have \( \fph_{p} = \Set{\addr \mapsto (\val, \fp) }\).
Given above and \( \fph_{q} = \fph_{p}\remapsto{\addr}{ ( \val, \addFPR{\fp} ) } \), we have \( \fph_{q} \in \evalLS[\lenv, \stk_{q}]{\lpost} \).

\caseB{ \rl{TRMutate} }

We have \( \trans \equiv (\pmutate{\expr_{1}}{\expr_{2}}) \), \( \lpre \equiv ( \expr_{1} \fpt{\fp} \stub ) \) and \( \lpost \equiv ( \expr_{1} \fpt{\addFPW{\fp}} \expr_{2} ) \), for some \( \expr_{1}, \expr_{2}, \fp \).
Therefore, \( \fph_{p} \in \Setcon{ \addr \mapsto (\val_{p} , \fp) }{ \val_{p} \in \Val } \), where \( \addr = \evalLE[\lenv, \stk_{p}]{\expr_{1}} \).
Given the semantics in \fig \ref{fig:thread_semantics}, we have \( \stk_{p} = \stk_{q} \) and \( \fph_{q} \in \Set{\addr \mapsto ( \evalLE[\lenv, \stk_{q}]{\expr_{2}},  \addFPW{\fp} ) } \), so \( \fph_{q} \in \evalLS[\lenv, \stk_{q}]{\lpost} \).

\caseI{\rl{TRChoice}}

We have  \(\trans \equiv \trans_{1} + \trans_{2} \), where \( \tripleL{\lpre}{\trans_{1}}{\lpost} \) and \( \tripleL{\lpre}{\trans_{2}}{\lpost} \) hold, for some \( \lpre, \lpost, \trans_{1}, \trans_{2} \).
Given the semantics in \fig \ref{fig:thread_semantics}, for any \( \lenv, \stk_{p}, \fph_{p} \) that \( \fph_{p} \in \evalLS[\lenv, \stk_{p}]{\lpre} \), it has either \( ( \stk_{p}, \fph_{p} ), \trans_{1} \pchoice \trans_{2} \toL ( \stk_{p}, \fph_{p} ), \trans_{1} \) or  \( ( \stk_{p}, \fph_{p} ), \trans_{1} \pchoice \trans_{2} \toL ( \stk_{p}, \fph_{p} ), \trans_{2} \).
Since it is symmetric, assume picking \( \trans_{1} \).
Therefore we have \( ( \stk_{p}, \fph_{p} ), \trans_{1} \pchoice \trans_{2} \toL ( \stk_{p}, \fph_{p} ), \trans_{1} \toL^{*} ( \stk_{q}, \fph_{q} ), \pskip \) for some \( \stk_{q} \) and \( \fph_{q} \).
By the \ih and the premiss of \rl{TRChoice} we have \( \tripleSemL{\lpre}{\trans_{1}}{\lpost} \), then we have \( \fph_{q} \in \evalLE[\lenv, \stk_{q}]{\lpost} \).
Symmetrically, if it picks \( \trans_{2} \), it yields the same result.

\caseI{\rl{TRSeq}}

We have \( \trans \equiv \trans_{1} \pseq \trans_{2} \) where \( \tripleL{\lpre}{\trans_{1}}{\lframe} \) and \( \tripleL{\lframe}{\trans_{2}}{\lpost} \) hold, for some \( \lpre, \lframe, \lpost, \trans_{1}, \trans_{2} \).
Given the semantics in \fig \ref{fig:thread_semantics}, for any \( \lenv, \stk_{p}, \fph_{p} \) that \( \fph_{p} \in \evalLS[\lenv, \stk_{p}]{\lpre} \), it has \( ( \stk_{p}, \fph_{p} ), \trans_{1} \pseq \trans_{2} \toL^{*} ( \stk_{r}, \fph_{r} ), \pskip \pseq \trans_{1} \toL ( \stk_{r}, \fph_{r} ), \trans_{1} \toL^{*} ( \stk_{q}, \fph_{q} ), \pskip \) for some \( \stk_{r}, \stk_{q}, \fph_{r}, \fph_{q} \).
By the \ih, we have \( \fph_{r} \in \evalLS[\lenv, \stk_{r}]{\lframe} \), then by the \ih, we have \( \fph_{q} \in \evalLS[\lenv, \stk_{q}]{\lpost} \).

\caseI{\rl{TRLoop}}

Since the triple is only partial correct, meaning that if the transaction \( \trans \) terminates it will reach a state satisfying the post-condition \( \lpost \), it is sufficient to prove the follows,
\[
    \for{\lpre, \trans, \nat > 0} \tripleL{\lpre}{\trans^{\nat}}{\lpre} \implies \ \tripleSemL{\lpre}{\trans^{\nat}}{\lpre} \\
\]
where,
\[
\begin{rclarray}
    \trans^{1} & \defeq  & \trans \\
    \trans^{\nat} & \defeq  & \trans \pseq \trans^{\nat - 1} \\
\end{rclarray}
\]
given the \ih that \(\tripleL{\lpre}{\trans}{\lpre} \implies \ \tripleSemL{\lpre}{\trans}{\lpre} \) holds.

We prove that by induction on the number \( \nat \).
For \( \nat = 1 \), it is proven by the \ih.
For \( \nat > 1 \), assume \( \stk, \fph, \lenv \) that satisfy \( \fph \in \evalLS{\lpre} \). 
Given the semantics in \fig \ref{fig:thread_semantics}, we have \( (\stk, \fph), \trans \pseq \trans^{\nat - 1} \toL^{*} (\stk' \fph'), \trans^{\nat - 1} \) for some \( \stk', \fph' \).
By the \ih that \(\tripleSemL{\lpre}{\trans}{\lpre} \), we have \( \fph' \in \evalLS[\lenv, \stk']{\lpre} \).
Then for any \( \stk'', \fph'' \) that \( (\stk', \fph'), \trans^{\nat - 1} \toL^{*} (\stk'' \fph''), \pskip \), by \ih that \( \tripleSemL{\lpre}{\trans}{\lpre} \), we have \( \fph'' \in \evalLS[\lenv, \stk'']{\lpre} \).
\caseI{\rl{TRFrame}}

We have \( \tripleL{\lpre \sep \lframe }{\trans}{\lpost \sep \lframe} \) and \( \tripleL{\lpre}{\trans}{\lpost} \) for some \( \lpre, \lpost, \lframe, \trans\).
For the precondition, we know \( \fph_{p} \composeFP \fph_{r} \in \evalLS[\lenv, \stk_{p}]{\lpre \sep \lframe} \) where \(  \fph_{p} \in \evalLS[\lenv, \stk_{p}]{\lpre} \) and \( \fph_{r} \in \evalLS[\lenv, \stk_{p}]{\lframe} \) for some \( \fph_{p}, \fph_{r}, \stk_{p}, \lenv \).
By the \( \pred{tsound}{\lpre, \trans, \lpost } \) from \ih, we have \( ( \stk_{p}, \fph_{p} ), \trans \toL^{*} (\stk_{q}, \fph_{q}), \pskip \) and \( \fph_{q} \in \evalLS[\lenv, \stk_{q}]{\lpost}\), for some \( \fph_{q}, \stk_{q} \).
Therefore, it is also true that \( ( \stk_{p}, \fph_{p} \composeLS \fph_{r} ), \trans \toL^{*} (\stk_{q}, \fph_{q} \composeLS \fph_{r}), \pskip \), so that \( \fph_{q} \composeFP \fph_{r} \in \evalLS[\lenv, \stk_{q}]{\lpost \sep \lframe} \).

\end{proof}

\begin{defn}[Conversion to time-stamp heaps]
\label{def:x2tsh}
Given the set of world \( \world \in \World \), fingerprint world \( \fpw \in \FPWorlds \), \( \ls \in \LStates\) the set of \( \tsh \in \TSHeaps \), the overloaded function \( \funcn{x2tsh} : \Set{\World, \FPWorlds, \LStates} \to \powerset{\TSHeaps} \) is defined as follows,
\[
    \begin{rclarray}
        \func{x2tsh}{\world} & \defeq & 
        \Setcon{%
            (\tsh,\ts) 
        }{%
            \exsts{ \h }
            \flattenW{\world} = (\h, \stub) 
            \land \snapshot{\tsh}{\ts}(\addr) = (\val, \stub) 
            \land \for{ \addr } \h(\addr) = \val 
        } \\
        \func{x2tsh}{\fpw} & \defeq & 
        \Setcon{%
            (\tsh,\ts) 
        }{%
            \exsts{ \h }
            \flattenW{\eraseFW{\fpw}} = (\h, \stub) 
            \land \snapshot{\tsh}{\ts}(\addr) = (\val, \stub) 
            \land \for{ \addr } \h(\addr) = \val 
        } \\
        \func{x2tsh}{\ls} & \defeq & 
        \Setcon{%
            (\tsh,\ts) 
        }{%
            \exsts{ \h }
            \ls = (\h, \stub) 
            \land \snapshot{\tsh}{\ts}(\addr) = (\val, \stub) 
            \land \for{ \addr } \h(\addr) = \val 
         }
    \end{rclarray}
\]
where the \( \snapshotName \) function (\fig \ref{fig:thread_semantics}) returns a fingerprint heap corresponding the state at time \( \ts \), and here we match first projection, i.e. the values, with the flattened world.
\end{defn}

\begin{defn}[Semantic triple]
\label{def:semantic-triple}
    The semantic triple \( \tripleSemG{\gpre}{\prog}{\gpost}\) is defined as the follows,
    \[
        \begin{rclarray}
            \tripleSemG{\gpre}{\prog}{\gpost} & \defeq &
            \begin{array}[t]{@{}l@{}}
                \for{\fpw_{p}, \tsh_{p}, \tsh_{q}, \lenv, \stk_{p}, \stk_{q}, \ts_{p}, \ts_{q} }  
                \stable{\gpre}{\intf} \\
                \quad {} \land \eraseFW{\fpw_{p}} \in \evalW[\lenv, \stk_{p}]{\gpre}
                \land (\tsh_{p}, \ts_{p}) \in \func{x2tsh}{\fpw_{p}}
                \land (\stk_{p}, \tsh_{p}, \ts_{p}), \prog \toT{}^{*} (\stk_{q}, \tsh_{q}, \ts_{q}), \pskip \\
                \quad \implies \exsts{ \fpw_{q} } \eraseFW{\fpw_{q}} \in \evalW[\lenv, \stk_{q}]{\gpost}
                \land (\tsh_{q}, \ts_{q}) \in \func{x2tsh}{\fpw_{q}}
                \land \stable{\gpost}{\intf}
            \end{array}
        \end{rclarray}
    \]
\end{defn}


\begin{thm}[Soundness]
The soundness is defined as follows:
\[
    \begin{array}{@{}l@{}}
        \for{\gpre, \gpost, \prog, \intf } \tripleG{\gpre}{\prog}{\gpost} \implies \tripleSemG{\gpre}{\prog}{\gpost}
    \end{array}
\]
\end{thm}
\begin{proof}
Induction on the rules for program \( \prog \).

\caseB{\rl{PRCommit}}
We have \( \tripleG{\gpre}{\ptrans{\trans}}{\gpost} \) given that \( \tripleL{\lpre}{\trans}{\lpost} \), \( \repartition{\gpre}{\gpost}{\lpre}{\lpost} \), \( \stable{\gpre}{\intf} \) and \( \stable{\gpost}{\intf} \) for any \( \trans, \gpre, \gpost, \lpre, \lpost, \inter \). 
For any \( \stk_{p}, \lenv \), let variables \( \world_{p}, \fpw_{p}, \fph_{p}, \fph_{f} \) to satisfy the follows,
\begin{equation}
    \label{equ:def-wwhh}
    \eraseFW{\fpw_{p}} \in \evalW[\lenv, \stk_{p}]{\gpre} 
    \land \flattenFW{\fpw_{p}} = (\fph_{p} \composeFP \fph_{f}, \stub)
    \land \fph_{p} \in \evalLS[\lenv, \stk_{p}]{\lpre}
\end{equation}
Note that we pick the names which are consistent with repartitioning in \defin \ref{def:repartitioning}.
By the soundness of \( \tripleL{\lpre}{\trans}{\lpost} \) (\thmref{thm:transaction-soundness}) and \equref{equ:def-wwhh}, for all \( \fph_{q}, \stk_{q} \), we have the follows,
\begin{equation}
    \label{equ:transaction-soundness}
    (\stk_{p}, \fph_{p} ), \trans \toL^{*}  (\stk_{q}, \fph_{q} ), \pskip 
    \implies \fph_{q} \in \evalLS[\lenv, \stk_{q}]{\lpost}
\end{equation}
Also, given the semantic triple (\defref{def:semantic-triple}) and the operational semantics (\figref{fig:thread_semantics}), let variables \( \tsh_{p}, \tsh_{q}, \ts_{p}, \ts_{q}, \tsid \) satisfy the follows,
\begin{equation}
    \label{equ:commit-current-trans}
    (\stk_{p}, \tsh_{p}, \ts_{p}), \ptrans{\trans} \toT{\lbC{\tsid}} (\stk_{q}, \tsh_{q}, \ts_{q}), \pskip 
\end{equation}
Now we consider addresses being written and read separably.
First, for any address \( \addr_{w} \) being tagged as write or read/write in \( \fph_{q} \), assume the value is \( \val_{w} \).
\begin{equation}
    \label{equ:local-write}
    \exsts{\fp} 
    \fph_{q}(\addr_{w}) = (\val_{w}, \fp)
    \land \fpW \in \fp
\end{equation}
By the \( \commitName \) function in operational semantics (\defref{fig:thread_semantics}), \equref{equ:commit-current-trans}, \equref{equ:transaction-soundness} and \equref{equ:local-write}, we have,
\begin{equation}
    \label{equ:global-write}
    \tsh_{q}(\addr_{w})(\ts_{q}) = (\val_{w}, \etW, \tsid)
\end{equation}
By the repartitioning (\defin \ref{def:repartitioning}), \equref{equ:def-wwhh} and \equref{equ:transaction-soundness}, for any \( \fpw_{q} \) we have,
\begin{equation*}
    \fpw_{q} \in \mergeR{\fpw_{p}}{\fpw_{i}}{\inter} \land \eraseFW{\fpw_{q}} \in \evalW[\lenv, \stk_{q}]{\gpost}
\end{equation*}
Given the \( \mergeName[R] \) function (\defin \ref{def:repartitioning}) that uses several levels of merges until \( \mergeName[\val] \) function (\defin \ref{def:merge-finger-heap}), \equref{equ:transaction-soundness} and \equref{equ:local-write}, we have,
\begin{equation}
    \label{equ:write-remain-the-same}
    \begin{array}{@{}l@{}}
      \for{ \fpw_{q} \in \mergeR{\fpw_{p}}{\fpw_{i}}{\inter} }  
      \eraseFW{\fpw_{q}} \in \evalW[\lenv, \stk_{q}]{\gpost} \\
      \quad {} \land \exsts{\fph} \flattenFW{\fpw_{q}} = (\fph, \stub) \land \fph(\addr_{w}) = \fph_{q}(\addr_{w}) = (\val_{w}, \etW)
    \end{array}
\end{equation}
which matches with Eq. \eqref{equ:global-write}.
Intuitively, because the address being written cannot be merged with others.

Second, we consider addresses \( \addr_{r} \) that are only being read with the value \( \val_{r} \),
\begin{equation}
    \label{equ:local-read}
    \fph_{q}(\addr_{r}) = (\val_{r}, \Set{\fpR})
\end{equation}
By the \( \commitName \) function in operational semantics (\defref{fig:thread_semantics}), \equref{equ:commit-current-trans}, \equref{equ:transaction-soundness} and \equref{equ:local-read}, we have,
\begin{equation}
    \label{equ:global-read}
    \tsh_{q}(\addr_{r})(\ts_{p}) = (\val_{r}, \etR , \tsid)
    \land \tsh_{q}(\addr_{r})(\ts_{q}) = (\val_{r}, \etE , \tsid)
\end{equation}
However, note that,
\[
\neg\left(
    \begin{array}{@{}l@{}}
        \for{\addr_{r}} \exsts{\val_{r}} 
        \tsh_{q}(\addr_{r})(\ts_{p}) = (\val_{r}, \etR, \tsid )
        \land \tsh_{q}(\addr_{r})(\ts_{q}) = (\val_{r}, \etE , \tsid) \\
        \quad \implies \snapshot{\tsh_{q}}{\ts_{q}}(\addr_{r}) = (\val_{r}, \stub)
    \end{array}
\right)
\]
Because there might be other transactions that commits between times \( \ts_{p} \) and \( \ts_{q} \) and writes to some addresses \( \addr_{r} \), which is allowed by the \( \consistentName \) predicate from the operational semantics (\figref{fig:thread_semantics}).
Let \( \tsid_{1} \) to \( \tsid_{\nat} \) be the transactions that commits between times \( \ts_{p} \) and \( \ts_{q} \) and writes to some addresses \( \addr_{r} \) as the follows.
We also assume those transactions are allowed by the \( \relyU \).
\begin{equation}
    \label{equ:concurrent-trans}
    \bigwedge\limits_{1 \leq i \leq \nat} 
    \begin{formulea}
    \exsts{\fpw, \fpw', \ts, \ts', \etag \in \Set{\etS, \etR} } 
    \ts < \ts' 
    \land \ts_{p} < \ts' < \ts_{q} \\
    {} \land \tsh_{q}(\addr_{r})(\ts) = (\stub, \etag, \tsid_{i}) 
    \land \tsh_{q}(\addr_{r})(\ts') = (\stub, \etW, \tsid_{i}) \\
    {} \land \exsts{\ca} \fpw_{p} = (\ca, \stub) 
    \land \fpw = (\ca, \stub) 
    \land \fpw' = (\ca, \stub) \\
    {} \land (\fpw, \fpw') \in \relyU
    \land (\tsh_{q}, \ts) \in \func{x2tsh}{\fpw}
    \land (\tsh_{q}, \ts') \in \func{x2tsh}{\fpw'}
    \end{formulea}
\end{equation}
By the \( \consistentName \) predicate from the operational semantics (\fig \ref{fig:thread_semantics}), we know the first transaction \( \tsid_{1} \) must read the same value as current transaction \( \tsid \).
\begin{equation}
\label{equ:read-the-same-value}   
\exsts{ \ts, \etag } 
\tsh_{q}(\addr_{r})(\ts) = (\val_{r}, \etag, \tsid_{1}) 
\land \etag \in \Set{\etS, \etR}
\end{equation}
Also, if two transactions write to the same addresses, it must be strictly one after another.
\begin{equation}
\label{equ:write-one-after-another}
    \begin{array}{@{}l@{}}
        \bigwedge\limits_{1 \leq i \leq \nat} 
        \exsts{ \ts, \ts', \etag \in \Set{\etS, \etR}, \val } 
        \ts < \ts'
        \land \tsh_{q}(\addr_{r})(\ts_{i}') = (\val, \etW, \tsid_{i - 1}) 
        \land \tsh_{q}(\addr_{r})(\ts_{i}) = (\val, \etag, \tsid_{i})
    \end{array}
\end{equation}
Given the definition of rely (\defref{def:rely-guarantee}), by \equref{equ:global-read}, \equref{equ:concurrent-trans}, \equref{equ:read-the-same-value}, \equref{equ:write-one-after-another}, and then induction on the number \( \nat \), we have,
\begin{equation}
    \label{equ:allowed-by-rely}
    \begin{array}{@{}l@{}}
        \exsts{\fpw, \ts, \etag \in \Set{\etS, \etR}, \ca }
        \tsh_{q}(\addr_{r})(\ts) = (\val_{r}, \etag, \tsid_{i}) 
        \land \fpw_{p} = (\ca, \stub) 
        \land \fpw = (\ca, \stub)  \\
        {} \land \bigwedge\limits_{1 \leq i \leq \nat} 
        \begin{formulea}
        \exsts{ \fpw', \ts' } 
        \ts < \ts' 
        \land \ts_{p} < \ts' < \ts_{q} 
        \land \tsh_{q}(\addr_{r})(\ts') = (\stub, \etW, \tsid_{i}) 
        \land \fpw' = (\ca, \stub) \\
        {} \land (\fpw, \fpw') \in \rely_{i}
        \land (\tsh_{q}, \ts) \in \func{x2tsh}{\fpw}
        \land (\tsh_{q}, \ts') \in \func{x2tsh}{\fpw'}
        \end{formulea}
    \end{array}
\end{equation}
where the \( \rely_{i} \) are defined in \defref{def:rely-guarantee}.
Let looks the first transaction \( \tsid_{1} \) and the last transaction \( \tsid_{\nat} \).
Assume the start state of \( \tsid_{1} \) is \( \fpw_{1} \), the end state of \( \tsid_{\nat} \) is \( \fpw_{\nat} \) and the final value being written to address \( \addr_{r} \) is \( \val_{\nat} \).
By re-writing the \equref{equ:allowed-by-rely}, we have the follows,
\begin{equation}
    \label{equ:first-and-last-concurrent-trans}
    \begin{array}{@{}l@{}}
        \exsts{ \ts_{1}, \ts_{\nat}, \etag \in \Set{\etS, \etR}, \ca }
        \ts_{1} < \ts_{\nat}
        \land \ts_{p} < \ts_{\nat} < \ts_{q}  \\
        \quad {} \land \fpw_{p} = (\ca, \stub) 
        \land \fpw_{1} = (\ca, \stub)  
        \land \fpw_{n} = (\ca, \stub) 
        \land (\fpw_{1}, \fpw_{\nat}) \in \rely_{\nat} \subseteq \myrely \\
        \quad {} \land \tsh_{q}(\addr_{r})(\ts_{1}) = (\val_{r}, \etag, \tsid_{1}) 
        \land \tsh_{q}(\addr_{r})(\ts_{\nat}) = (\val_{\nat}, \etW, \tsid_{\nat})  \\
        \quad {} \land (\tsh_{q}, \ts_{1}) \in \func{x2tsh}{\fpw_{1}}
        \land (\tsh_{q}, \ts_{\nat}) \in \func{x2tsh}{\fpw_{\nat}}
    \end{array}
\end{equation}
Given the  \equref{equ:first-and-last-concurrent-trans}, and the \( \mergeName[R] \) function that is used in repartitioning (\defref{def:repartitioning}), we have,
\begin{equation}
\label{equ:read-can-be-merged}
    \begin{array}{@{}l@{}}
      \exsts{ \fpw_{q} \in \mergeR{\fpw_{p}}{\fpw_{i}}{\inter}, \fph }  
      \eraseFW{\fpw_{q}} \in \evalW[\lenv, \stk_{q}]{\gpost} \\
      \quad {} \land \flattenFW{\fpw_{q}} = (\fph, \stub) \land \fph(\addr_{r}) = \fph_{q}(\addr_{r}) = (\val_{\nat}, \etW)
    \end{array}
\end{equation}

Now combining \equref{equ:write-remain-the-same} and \equref{equ:read-can-be-merged}, we have, 
\begin{equation}
    \eraseFW{\fpw_{q}} \in \evalW[\lenv, \stk_{q}]{\gpost}
    \land (\tsh_{q}, \ts_{q}) \in \func{x2tsh}{\fpw_{q}}  \\
\end{equation}
Then since \( \stable{\gpost}{\intf} \) is proven by the premiss, we have the prove for the \rl{PRCommit}.


\end{proof}










\sx{No use for below}

\begin{defn}[Transaction interpretation]
\label{def:transactions-interpretation}
Given the set of transaction \( \trans \in \Transactions \) (\defin \ref{def:language}) and the operational semantics (\fig \ref{fig:transaction_semantics}), the \emph{transaction interpretation} function \( \intpSQ{.} : \Transactions \to \FPHeaps \to \powerset{\FPHeaps} \) is defined as follows:
\[
    \begin{rclarray}
        \intpSQ{\trans}(\fph) & \defeq & 
            \Setcon{%
                \fph'
            }{%
                \exsts{ \stk, \stk' } (\stk, \fph ), \trans \toL^{*}  (\stk', \fph' ), \pskip
            }\\
    \end{rclarray}
\]
\end{defn}
\sx{probably use the above to prove%
\[
    \tripleL{\lpre}{\trans}{\lpost}
\]%
Simply for less words, maybe?
}
Note that the stack is local and has no side effect to the fingerprint heap, therefore from now we will fix the stack and treat the stack as the same as logical environment.

\begin{defn}[Reification function]
\label{def:reification}
Given the set of assertions \( \gpre \in \Ast \) and time-stamp heap \( \TSHeaps \), the \emph{reification} function \( \reif{.} : \Ast \to \powerset{\TSHeaps} \) is defined as follows:
\[
\begin{rclarray}
    \reif{\gpre} & \defeq & 
    \Setcon{%
        \tsh
    }{%
        \exsts{\lenv, \stk, \h} \world \in \evalW{\gpre} 
        \land (\h, \stub ) = \flattenW{\world}
        \land \tsh \in\func{h2tsh}{\h} 
    }
\end{rclarray}
\]
where the interpretation of assertion \( \evalW{.} \) is defined in \defin \ref{def:assertion}, the world flattening \( \flattenW{.} \) in \defin \ref{def:world} and \( \funcn{h2tsh} \) function in \defin \ref{def:h2tsh}.
\end{defn}

\begin{defn}[Atomic interpretation]
\label{def:atomic-intp}
%Given the set of transactions \( \trans \in \Transactions \) (\defin \ref{def:language}), the set of programs \( \Programs \) and the operational semantics (\fig \ref{fig:thread_semantics}), the \emph{atomic interpretation} function \( \intfATOM{.} : \Atom \to \TSHeaps \to \powerset{\TSHeaps} \) is defined as follows, where \( \Atom \defeq \Setcon{\ptrans{\trans}}{\trans \in \Transactions \land \ptrans{\trans} \in \Programs} \).
\[
    \begin{rclarray}
        \intpSQ{\ptrans{\trans}}(\tsh) & \defeq & 
            \Setcon{%
                \tsh'
            }{%
                \exsts{ \stk, \stk', \ts, \ts' } (\stk, \tsh, \ts ), \ptrans{\trans} \toG{\stub}  (\stk', \tsh', \ts' ), \pskip
            }\\
    \end{rclarray}
\]
\end{defn}

\begin{thm}[Axiom soundness]
Given the set of transactions \( \trans \in \Transactions \) (\defin \ref{def:language}) and the rely relation \( \myrely \) (\defin \ref{def:rely-guarantee}), and assume stardard lift for reification function (\defin \ref{def:reification}) and for atomic interpretation (\defin \ref{def:atomic-intp}), the axiom soundness is defined as follows:
\[
    \begin{array}{@{}l@{}}
        \for{ \trans, \gpre, \gpost } 
        \tripleG{\gpre}{\ptrans{\trans}}{\gpost} \\
        \quad {} \land \for{\world } \intpSQ{\ptrans{\trans}}\left( \reif{\gpre \sep \Set{\world}} \right) \subseteq \left( \reif{\gpost \sep \myrely(\Set{\world})} \right) 
     \end{array}
\]
where \( \Set{\world} \) denotes assertion satisfying \( \world \) under any logical environment and stack, i.e.\ \( \for{\lenv, \stk} \evalW{\Set{\world}} = \Set{\world} \).
\end{thm}
\begin{proof}
Given the reification function, it is sufficient to prove:
\[
    \begin{array}{@{}l@{}}
        \for{ \trans, \gpre, \gpost } 
        \tripleG{\gpre}{\ptrans{\trans}}{\gpost} \\
        \quad {} \land \for{ \stk ,\lenv, \world_{f}, \world_{p} } 
        \exsts{ \world_{q}, \world_{f}' } 
        \world_{p} \in \evalW{\gpre}
        \land \world_{q} \in \evalW{\gpost}
        \land \world_{f}' \in \myrely(\world_{f})  \\
        \quad {} \land \for{ \h_{p}, \tsh_{p} } 
        \exsts{ \h_{q}, \tsh_{q} }
        (\h_{p}, \stub) = \flattenW{\world_{p} \composeW \world_{f}} 
        \land \tsh_{p} \in \func{h2tsh}{\h_{p}}
        \land (\h_{q}, \stub) = \flattenW{\world_{q} \composeW \world_{f}'} 
        \land \tsh_{q} \in \func{h2tsh}{\h_{q}} \\
        \quad \implies \tsh_{q} \in \intpSQ{\ptrans{\trans}} \left( \tsh_{p} \right)
     \end{array}
\]
First we introduce some new variables \( \fpw_{p}, \fph_{p}, \fpw_{q}, \fph_{q} \) that satisfy the follows and whose names are consistent with those in \defin \ref{def:repartitioning}.
\[
\begin{array}{@{}l@{}}
    \exsts{\fph} \\
    \eraseFW{\fpw_{p}} = \world_{p} 
    \land \flattenFW{\fpw_{p}} = (\fph_{p} \composeFP \fph, \unitC)
    \land \for{\addr} \fph_{p}(\addr) = (\stub, \emptyset) \land {} \\
    \eraseFW{\fpw_{q}} = \world_{q} 
    \land \flattenFW{\fpw_{q}} = (\fph_{q} \composeFP \fph, \unitC)
\end{array}
\]
By the definition of repartition (\defin \ref{def:repartitioning})  and transaction soundness (Theorem \ref{thm:transaction-soundness}):
\[
\begin{array}{@{}l@{}}
    \for{\fph} 
    \fph_{q} \composeFP \fph \in \intpSQ{\trans}( \fph_{p} \composeFP \fph) 
\end{array}
\]
\end{proof}

%\subsection{Transaction Soundness}

The \cref{thm:transaction-soundness} is the soundness of for transactions, where the most interesting two rules \rl{TRLookup} and \rl{TRMutate} can be derived from the \cref{lem:fingerprint-op}.
Note that the transactional semantics is defined on snapshots as total functions.
Thus in the \cref{thm:transaction-soundness}, it is safe to extend \( h \) with \( \sn'' \)  to make the snapshot a total function, as there is no allocation and deallocation.

\begin{theorem}[Transaction soundness]
\label{thm:transaction-soundness}
The transaction soundness is as follows:
\[
    \begin{array}{@{}l@{}}
        \fora{ \lpre, \trans, \lpost } \tripleL{\lpre}{\trans}{\lpost}
        \implies 
        \fora{\lenv, \stk, \stk', \sn_\lpre, \sn_\lpost, \sn \fp, \fp' }  \\
        \quad (\sn_\lpre, \fp) \in \evalLS[\lenv, \stk]{\lpre}
        \land \dom(\sn) = \Keys \setminus \dom(\sn) \\
        \quad {} \land \vdash (\stk, \sn_\lpre \composeH \sn, \fp ), \trans \toL^{*}  (\stk', \sn_\lpost \composeH \sn, \fp' ), \pskip  \\
        \qqquad \implies (\sn', \fp') \in \evalLS[\lenv, \stk']{\lpost}
    \end{array}
\]
\end{theorem}
\begin{proof}
Induction on the derivations.

\begin{itemize}

\item \caseB{\rl{TRSkip}}
We have \(\trans \equiv \pskip\), \( \lpre \equiv \lpost \equiv \assemp \), thus \( \sn_{p} = \sn_{q} = \unitH \), \( \fp = \fp' \) and \( \stk = \stk' \), and then \( (\unitH,\unitO ) \in \evalLS[\lenv, \stk']{\assemp} \) holds.

\item \caseB{\rl{TRAss}}
We have \(\trans \equiv ( \pass{\var}{\expr} ) \), \( \lpre \equiv ( \var \doteq \lexpr ) \) and \( \lpost \equiv ( \var \doteq \expr\sub{\var}{\lexpr} ) \) 
for some \( \expr, \lexpr \) and \( \var \) such that \( \var \notin \func{fv}{\lexpr}\).
Given the transaction semantics (\cref{fig:thread_semantics}), it has \( \stk' = \stk\rmto{\var}{\val} \) where \( \val = \evalLE[\lenv, \stk]{\expr\sub{\var}{\lexpr}} \).
Since \( \var \notin \func{fv}{\lexpr} \), we know \( \evalLE[\lenv, \stk]{\lexpr} = \evalLE[\lenv, \stk']{\lexpr} \), and then \( \evalLE[\lenv, \stk]{\expr\sub{\var}{\lexpr}} = \evalLE[\lenv, \stk']{\expr\sub{\var}{\lexpr}} \).
This means the assertions related to stack hold even thought the stack changes.
Also because the snapshot and fingerprint remain unchanged, \ie \( \sn = \sn' \) and \( \fp = \fp' \), so we prove \( (\sn', \fp' ) \in \evalLS[\lenv, \stk']{\lpost} \).

\item \caseB{\rl{TRLookup}}
We have \(\trans \equiv ( \plookup{\var}{\expr} ) \) and four cases for pre- and post-conditions defined by the relation \( \toFP{\otR(\expr, \lexpr)}\).
In all the four cases, the stack is updated to \( \stk' = \stk\rmto{\var}{\val} \), yet since \( \var \notin \func{fv}{\lexpr}\), the logical value \( \lexpr \) and new logical address \( \expr\sub{\var}{\lexpr}\) are evaluated to the same value \( \val \) and address \( \key \) as before.
By \cref{lem:appendix-fingerprint-op}, we know that the final state of snapshot and fingerprint satisfy the postcondition.

\item \caseB{ \rl{TRMutate} }
We have  \( \trans \equiv (\pmutate{\expr_{1}}{\expr_{2}}) \) and four cases for pre- and post-conditions defined by the relation \( \toFP{\otW(\expr, \lexpr)}\). 
In all the four cases, the stack remains untouched and  by \cref{lem:appendix-fingerprint-op} the snapshot and fingerprint match the postcondition.

\item \caseI{\rl{TRChoice}}
We have  \(\trans \equiv \trans_{1} + \trans_{2} \), where \( \tripleL{\lpre}{\trans_{1}}{\lpost} \) and \( \tripleL{\lpre}{\trans_{2}}{\lpost} \) hold, for some \( \trans_{1}, \trans_{2}, \lpre, \lpost \).
Given the transaction semantics (\cref{fig:thread_semantics}), it either has 
\[ 
( \stk, \sn_\lpre \composeH \sn, \fp ), \trans_{1} \pchoice \trans_{2} \toL ( \stk, \sn_\lpre \composeH \sn, \fp ), \trans_{1} 
\]
or  
\[ 
( \stk, \sn_\lpre \composeH \sn, \fp ), \trans_{1} \pchoice \trans_{2} \toL ( \stk, \sn_\lpre \composeH \sn, \fp ), \trans_{2} 
\]
Let us pick \( \trans_{1} \) and  assume it can be reduced to \( \pskip \) from the initial state, \ie \( ( \stk, \sn_\lpre \composeH \sn, \fp ), \trans_{1}  \toL^{*} ( \stk', \sn', \fp' ), \pskip \).
By the premiss of the rule \( \tripleL{\lpre}{\trans_{1}}{\lpost} \) and the \ih, 
there exists a \( \sn_\lpost \) such that \( \sn' = \sn_\lpost \composeH \sn \) and  \( (\sn_\lpost, \fp') \in \evalLE[\lenv, \stk']{\lpost} \).
Symmetrically, if we pick \( \trans_{2} \), it gives the same result.

\item \caseI{\rl{TRSeq}}
We have \( \trans \equiv \trans_{1} \pseq \trans_{2} \) where \( \tripleL{\lpre}{\trans_{1}}{\lframe} \) and \( \tripleL{\lframe}{\trans_{2}}{\lpost} \) hold, for some \( \trans_{1}, \trans_{2}, \lpre, \lpost, \lframe \).
Given the transaction semantics (\cref{fig:thread_semantics}), 
it has 
\[
    \begin{array}{l}
    \vdash ( \stk, \sn_\lpre \compose \sn, \fp ), \trans_{1} \pseq \trans_{2} \toL^{*} ( \stk'', \sn'', \fp'' ), \pskip \pseq \trans_{1} \\
    \qqquad \toL ( \stk'', \sn'', \fp'' ), \trans_{1} \toL^{*} ( \stk', \sn', \fp' ), \pskip 
\end{array}
\] 
for a residue \( \sn \) and  some states \( (\stk', \sn', \fp'), (\stk'', \sn'', \fp'') \).
By the \ih that \( \tripleL{\lpre}{\trans_{1}}{\lframe} \) is sound,
there exists a snapshot \( \sn_\lframe \) such that \( \sn'' = \sn_\lframe \composeH \sn \) and \( (\sn_\lframe, \fp'') \in \evalLE[\lenv, \stk'']{\lframe} \).
The elimination of prefix \( \pskip \) does not change the state, so \( (\sn'', \fp'') \in \evalLE[\lenv, \stk'']{\lframe} \).
Then similarly, 
By the \ih that \( \tripleL{\lframe}{\trans_{2}}{\lpost} \) is sound,
there exists a snapshot \( \sn_\lpost \) such that \( \sn'' = \sn_\lpost \composeH \sn \) and \( (\sn_\lpost, \fp') \in \evalLE[\lenv, \stk']{\lpost} \).

\item \caseI{\rl{TRLoop}}
Since the triple is only partial correct, 
meaning that only when the transaction \( \trans \) terminates, it will reach a state satisfying the post-condition \( \lpost \).
It is sufficient to prove the following is sound:
\[
    \fora{\lpre, \trans, \nat > 0} \tripleL{\lpre}{\trans^{\nat}}{\lpre}
\]
where,
\[
    \trans^{0} \defeq  \pskip \qquad
    \trans^{\nat} \defeq  \trans \pseq \trans^{\nat - 1} 
\]

We prove that by induction on the number \( \nat \).
\begin{itemize}
    \item \caseB{\( \nat = 0 \)} It has been proven before \( \triple{\lpre}{\pskip}{\lpre} \).
    \item \caseI{\( \nat > 0 \)} We have 
    \[ 
        \begin{array}{l}
        \vdash (\stk, \sn_\lpre \composeH \sn, \fp), \trans \pseq \trans^{\nat - 1} \toL^{*} \\
        \qqquad (\stk'', \sn'', \fp''), \trans^{\nat - 1} \toL^{*} (\stk', \sn', \fp'), \pskip  
        \end{array}
    \]
    for a residue \( \sn \) and some states \( ( \stk', \sn', \fp' ), ( \stk'', \sn'', \fp'' ) \).
    This is similar to the \rl{PSeq} case.
\end{itemize}

\item \caseI{\rl{TRFrame}}
We need to prove \( \tripleL{\lpre \sep \lframe }{\trans}{\lpost \sep \lframe} \) is sound, 
given the soundness of  \( \tripleL{\lpre}{\trans}{\lpost} \).
Assume snapshots \( \sn_\lpre, \sn_\lpost, \sn_\lframe \), fingerprints \( \fp, \fp', \fp'' \) and stacks\( \stk, \stk' \) 
such that \( ( \sn_\lpre, \fp ) \in \evalLS[\lenv, \stk]{\lpre} \), \( ( \sn_\lframe, \fp' ) \in \evalLS[\lenv, \stk']{\lpost} \) and \( ( \sn_\lpost, \fp'' ) \in \evalLS[\lenv, \stk]{\lframe}\).
Since \( \lpre \sep \lframe \), the component-wise composition is defined, \ie \( (\sn_\lpre \composeH \sn_\lframe, \fp \composeO \fp'') \in \evalLS[\lenv, \stk]{\lpre \sep \lframe} \).
The keys from the snapshots and fingerprints are disjointed.
By the hypothesis that \( \tripleL{\lpre}{\trans}{\lpost} \) is sound, 
we know \( ( \stk, \sn_\lpre \composeH \sn, \fp ), \trans \toL^{*} ( \stk', \sn_\lpost \composeH \sn, \fp' ), \pskip \) for some residues \( \sn \).
The snapshot after the execution contains the same resources as before, that is \( \dom(\sn_\lpre) = \dom(\sn_\lpost) \) and \( \dom(\fp') \subseteq \dom(\sn_\lpost) \).
We know the composition of \( \sn_\lpost \composeH \sn_\lframe \) and \( \fp' \composeO \fp''\) exists, 
so 
\[ 
    ( \stk, \sn_\lpre \composeH \sn_\lframe \composeH \sn', \fp \composeO \fp''), \trans \toL^{*} ( \stk', \sn_\lpost \composeH \sn_\lframe \composeH \sn', \fp' \composeO \fp'' ), \pskip 
\]
for some residues \( \sn' \).
Note that \( \dom(\sn_\lframe \composeH \sn') = \dom(\sn) \).
Finally, because the transaction \( \trans \) does not modify any variable from the frame \( \lframe \), 
the update of stack does not change the evaluation of the frame, 
\( \evalLS[\lenv, \stk]{\lframe} = \evalLS[\lenv, \stk']{\lframe} \) which then gives us the result \( (\sn_\lpost \composeH \sn_\lframe, \fp' \composeO \fp'') \in \evalLS[\lenv, \stk']{\lpost \sep \lframe} \).
\end{itemize}
\end{proof}

\begin{lemma}
\label{lem:fingerprint-op}
\label{lem:appendix-fingerprint-op}
For \( \mathtt{O} \in \Set{\otR, \otW} \), the relation \( \toFP{\mathtt{O}(\key,\val')}\) is sound with respect to the operator \( \addO \):
\[
\begin{array}{@{}l}
    \fora{\lpre, \lpost, \lenv, \stk, \stk', \fp, \fp', \mathtt{O}, \key, \val} \\
    \quad \mathtt{O} \in \Set{\otR, \otW} 
    \land \lpre \toFP{\mathtt{O}(\key,\val)} \lpost
    \land (\stub, \fp) \in \evalLS{\lpre}
    \land (\stub, \fp') \in \evalLS{\lpost} \\
    \qqquad \implies \fp' = \fp \addO (\mathtt{O}, \key, \val)
\end{array}
\]
\end{lemma}
\begin{proof}
\begin{itemize}
    \item In this case, a read operation is added \( \fp' = \Set{(\otR, \key, \val)} \) and it is included in the interpretation of the post condition \( \lpost \equiv \key \fpR \val \).
If \( \lpre \equiv \key \fpR \val \) and \( \lpost \equiv \key \fpR \val \), 
since there is already a read operation in the fingerprint, adding a new read operation does not change the fingerprint, \ie \( \fp' = \Set{(\otR, \key, \val)} \addO (\otR, \key, \val ) = \Set{(\otR, \key, \val)} \).
This is exactly the post-condition.
It is sound for the rest two cases that  when \( \lpre \equiv \key \fpW \val \) and \( \lpost \equiv \key \fpW \val \), and when \( \lpre \equiv \key \fpRW (\lexpr,\val) \) and \( \lpost \equiv \key \fpRW (\lexpr,\val) \), as the fingerprint remains the same in the rest two cases.

\item When \( \mathtt{O} = \otW\), we also have four cases for pre- and post-conditions.
If \( \lpre \equiv \key \fpI \val' \) for some value \( \val' \), a new write operation is added to the initially empty fingerprint, that is, \( \fp' = \Set{(\otW, \key, \val)}\).
This is exactly the post-condition \( \lpost \equiv \key \fpW \val \).
If \( \lpre \equiv \key \fpW \val' \) for some \( \val' \), the fingerprint before execution is \( \fp = \Set{(\otW, \key, \val')}\).
Since the fingerprint only has the last write for the key because of the property of the \( \addO \) operator,
the fingerprint after is \( \fp' = \fp \addO (\otW, \key, \val) = \Set{(\otW, \key, \val)}\) which satisfies the postcondition \( \lpost \equiv \expr_{1} \fpW \expr_{2} \).
The remaining two cases follow the same argument as the fingerprint only have the last write.
\end{itemize}
\end{proof}

%\subsection{Program Soundness}
\begin{thm}[Program soundness]
The program soundness is the follows,
\[
    \for{\gpre, \prog, \gpost}
    \tripleG{\gpre}{\prog}{\gpost} 
    \implies 
    \tripleSemG{\gpre}{\prog}{\gpost} 
\]
\end{thm}
\begin{proof}
Induction on the derivations.
\caseB{\rl{PRCommit}}
We have \( \prog \equiv \ptrans{\trans} \).
Because a transaction \( \ptrans{\trans} \) is reduced by one step in the semantics, it is sufficient to prove the follows,
\[
\begin{array}{l}
    \begin{B}
        \stable{\gpre} 
        \land \gpre \snap \bar{\lpre}
        \land \tripleL{\bar{\lpre} \sep \fpEMP}{\trans}{\lpost \sep \fpF}
        \land \rpt{\gpre}{\gpost}{\fp} 
        \land \stable{\gpost}
    \end{B} \\
    \implies 
    \for{\w, \w', \w'', \w''', \hh', \hh'', \cu', \cu'', \thcu', \thcu'', \thid, \lenv, \thstk, \thstk''} \\
    \quad \begin{B}
        \w \in \evalW[\lenv, \thstk]{\gpre} 
        \land (\w, \w') \in \Rely^{*} 
        \land (\hh', \cu') \in \clpsW{\w'}
        \land \thcu'(\thid) = \cu' \\
        {} \land \thid, \func{como}{\w'} \vdash (\thstk, \hh', \thcu'), \ptrans{\trans} 
        \toT{\lbC{\txid}} (\thstk'', \hh'', \thcu''), \pskip  \\
        {} \land \thcu''(\thid) = \cu''
        \land (\hh'', \cu'') \in \clpsW{\w''} 
        \land (\w'', \w''') \in \Rely^{*} 
    \end{B} \\
    \quad \implies  \w''' \in \evalW[\lenv, \thstk'']{\gpost} 
\end{array}
\]
\textbf{Stable pre-condition.} 
Given that \( \stable{\gpre} \), for any world \( \w \) satisfies the pre-condition \( \gpre \) if the world can transfer to another world \( \w' \) through any steps of rely, the new world \( \w' \) also satisfies the pre-condition.
This is, 
\begin{equation}
    \label{equ:stable-pre-condition}
    \for{\w, \w',\lenv, \thstk} 
    \stable{\gpre} 
    \land \w \in \evalW[\lenv, \thstk]{\gpre} 
    \land (\w, \w') \in \Rely^{*}
    \implies \w' \in \evalW[\lenv, \thstk]{\gpre}
\end{equation}
\textbf{Commit.}
Given the \equref{equ:stable-pre-condition}, we want to prove the follows,
\begin{equation}
\label{equ:commit-new-transaction}
    \begin{array}{@{}l}
    \begin{B}
        \gpre \snap \bar{\lpre}
        \land \tripleL{\bar{\lpre} \sep \fpEMP}{\trans}{\lpost \sep \fpF}
        \land \rpt{\gpre}{\gpost}{\fp} 
    \end{B} \\
    \implies 
    \for{\w, \w', \hh, \hh', \cu, \cu', \thcu, \thcu', \thid, \lenv, \thstk, \thstk'} \\
    \quad \begin{B}
        \w \in \evalW[\lenv, \thstk]{\gpre}
        \land (\hh, \cu) \in \clpsW{\w}
        \land \thcu(\thid) = \cu \\
        {} \land \thid, \func{como}{\w} \vdash (\thstk, \hh, \thcu), \ptrans{\trans} 
        \toT{\lbC{\txid}} (\thstk', \hh', \thcu'), \pskip  \\
        {} \land \thcu'(\thid) = \cu'
        \land (\hh', \cu') \in \clpsW{\w'} 
    \end{B} \\
    \quad \implies  \w' \in \evalW[\lenv, \thstk']{\gpost} 
    \end{array}
\end{equation}
For any \( \w, \hh, \cu, \lenv, \thstk \) such that \( \w \in \evalW[\lenv,\thstk]{\gpre} \) and \( (\hh, \cu) \in \clpsW{\w} \), by the predicate \( \gpre \snap \bar{\lpre} \) we know \( \clpsHH{\hh, \cu} \in \evalLS[\lenv,\thstk]{\bar{\lpre}} \), so that,
\begin{equation}
\label{equ:local-pre-condition}
(\clpsHH{\hh, \cu}, \unitO) \in \evalLS[\lenv,\thstk]{\bar{\lpre} \sep \fpEMP}
\end{equation}
Because of the soundness of transaction (\thmref{thm:transaction-soundness}), given a thread stack \( \thstk \) and a logical environment \( \lenv \), if a initial configuration \( (\txstk, \h, \unitO), \trans \) satisfies the pre-condition, \ie \( (\h, \unitO) \in \evalLS[\lenv,\thstk \uplus \txstk]{\bar{\lpre} \sep \fpEMP} \), and a final configuration \( (\txstk', \h', \opset), \pskip \) is semantics reachable from the initial configuration, this final configuration will satisfy the post-condition \( \lpost \sep \fpF \).
This is, for any \( \thstk, \txstk, \txstk', \h, \h', \opset \), they satisfy the follows,
\begin{equation}
\label{equ:local-transaction-sound}
\begin{array}{@{}l}
    (\h, \unitO) \in \evalLS[\lenv,\thstk \uplus \txstk]{\bar{\lpre} \sep \fpEMP}
    \land \thstk \vdash (\txstk, \h, \unitO), \trans \toL (\txstk', \h', \opset), \pskip
    \implies (\h, \opset) \in \evalLS[\lenv,\thstk \uplus \txstk']{\bar{\lpost} \sep \fpF}
\end{array}
\end{equation}
The repartition \( \rpt{\gpre}{\gpost}{\fp} \) requires that any machine state \( \hh' \) by committing the operations \( \opset = \evalF{\fp}\) to an initial state \( \hh \) that satisfies the precondition \( \gpre \), should satisfy the \( \gpost \).
Formally, for any \( \w, \w', \hh, \hh', \cu, \cu', \lenv, \thstk, \txid \), we have,
\begin{equation}
\label{equ:repartition}
\begin{array}{@{}l}
    \begin{B}
        \w \in \evalW{\gpre}
        \land (\hh, \cu) \in \eraseW{\w}
        \land \txid \in \func{fresh}{\hh} \\
        {} \land \hh' = \func{commit}{\hh, \cu, \txid, \evalF{\fp}} 
        \land (\hh',\cu') \in \eraseW{\w'}
        \land (\w, \w') \in \Guar 
    \end{B}
    \implies \w' \in \evalW{\gpost}
\end{array}
\end{equation}
The guarantee says regions that has been updated must not violate their invariants,
\begin{equation}
\begin{array}{@{}l}
\end{array}
\end{equation}
\sx{HERE}

\begin{equation}
\label{equ:reachable}
\begin{array}{@{}l}
    \pred{reachable}{\eraseAEX{\aexecrun'}, \eraseAEX{\aexecrun''}, \txid, \evset, \como} \\
    \quad {} \land \h \in \obsstate{\aexecrun', \txidset, \respo} \land \thstk \vdash (\txstk, \h, \emptyset), \trans \toL^{*} (\txstk', \h', \evset ), \pskip
\end{array}
\end{equation}
\begin{equation}
\label{equ:local-transfer}
    (\h, \unitE) \in \evalLS{\lpre} 
    \land (\h', \unitE) \in \evalLS{\lpost} 
    \land (\unitH, \evset ) \in \evalLS{\fp}
    \land \tripleL{\lpre \sep \emptyset}{\trans}{\lpost \sep \fp}
\end{equation}
Since the new abstract execution \( \eraseAEX{\aexecrun''} \) is reachable from \( \eraseAEX{\aexecrun'} \) provided the set of events \( \evset \) (\equref{equ:reachable}) that is an interpretation of the fingerprint assertion \( \fp \) (\equref{equ:local-transfer}), it means there exists \(  \w'' \) satisfying the following because of the reapportion  \( \rpt{ \gpre }{ \gpost }{ \fp }\),
\[
    \w'' \in \evalW{\gpost} 
    \land \eraseAEX{ \aexecrun'' } \in \clpsW{\w''}
\]
\textbf{Stable post-condition} Similar to the stable pre-condition, we know
\[
    \w''' \in \evalW{\gpost} 
    \land \eraseAEX{ \aexecrun''' } \in \clpsW{\w'''}
\]
\sx{ Below is temporarily  }
\begin{equation}
    \aexecrun'' = \newaexec{\aexecrun', \thid, \evset, \txid, A, \conguar } 
\end{equation}
where \( \evset \) is a set of events corresponding to the code inside transaction, \( \txidset \) is  a set of transactions that are visible by the new transaction \( \txid \) and \( \conguar \) relates to the consistency model deciding whether the events \( \evset \) can happen and can be committed with respect to the visible transactions \( \txidset \).
Since \( \aexecrun'' \) is defined and by the \rl{PCommit} rule (\figref{fig:thread_semantics}), we know the following for some \( \h, \h', \thstk \),
\begin{equation}
    \h \in \obsstate{\aexecrun', \txidset, \respo}
    \land \thstk \vdash (\emptyset, \h, \emptyset) , \trans \ \toL^{*} \  (\txstk, \h', \evset) , \pskip 
\end{equation}
By the transaction soundness theorem (\thmref{thm:transaction-soundness}), there exist \( \lpre, \fp, \lpost \) such that
\[ 
\tripleL{\lpre \sep \emptyset}{\trans}{\lpost \sep \fp} \land (\h, \unitE) \in \evalLS{\lpre \sep \emptyset} \land (\h', \evset) \in \evalLS{\lpost \sep \fp}
\]
By the interpretation of assertions, it has,
\begin{equation}
\label{equ:interpretation-local-post}        
(\h', \unitE) \in \evalLS{\lpost} \land (\unitH, \evset) \in \evalLS{\fp}
\end{equation}
Combining \equref{equ:stable-pre-condition-and-the-graph}, \equref{equ:commit-new-transaction} and \equref{equ:interpretation-local-post}, we need to prove the follows,
\[
    \begin{array}[t]{@{}l@{}}
    \w' \in \evalW{\gpre} 
    \land \eraseAEX{ \aexecrun' } \in \clpsW{\w'}
    \land (\unitH, \evset) \in \evalLS{\fp} \\
    {} \land \aexecrun'' = \newaexec{\aexecrun', \thid, \evset, \txid, A, \conguar } 
    \land \rpt{\gpre}{\gpost}{\fp} \\
    \quad {} \implies 
    \exsts{\w''}
    \w'' \in \evalW{\gpost} 
    \land \eraseAEX{ \aexecrun'' } \in \clpsW{\w''}
    \end{array}
\]
\end{proof}

\begin{lem}
\label{lem:reachable}
Given a runtime abstract execution \( \aexecrun \), if it can reach anther runtime abstract execution \( \aexecrun'\) through the \rl{PCommit}, the former can reach the later statically, \ie
\[
\begin{array}{@{}l}
    \for{\aexecrun, \aexecrun', \thstk, \thstk', \como, \txid, \thid}
    \como, \thid \vdash (\thstk, \aexecrun), \ptrans{\trans}
    \toT{\lbC{\txid}} (\thstk', \aexecrun'), \pskip \\
    \quad {} \implies \pred{reachable}{\eraseAEX{\aexecrun}, \eraseAEX{\aexecrun'}, \txid, \evset, \como}
\end{array}
\]  
where \( \evset \) satisfies the conditions the of the \rl{PRCommit},
\[
    \exsts{\txidset, \h, \h', \txstk, \txstk'} \h \in \obsstate{\aexecrun, \txidset, \respo} \land \thstk \vdash (\txstk, \h, \emptyset), \trans \toL^{*} (\txstk', \h', \evset ), \pskip
\]
\end{lem}
\begin{proof}
It is implied by the soundness of the semantics (\thmref{thm:soundness-semantics}).
\end{proof}


%\subsection{Local/Transaction}

\begin{definition}[Logical Expressions]
\label{def:logical-expr}
Assume a countably infinite set of \emph{logical variables} $\V x \in \LVar$.
The set of \emph{logical expressions}, $ \lexpr \in \LExpr$ is defined by the following inductive grammar, where \(\val \in \Val\) (\defref{def:program_values}), \(\txvar \in \TxVars\) and \( \thvar \in \ThdVars \) (\defref{def:stacks}),
\[
\begin{rclarray}
   \lexpr & ::= & \val \mid \txvar \mid \thvar \mid \lvar \mid \lexpr + \lexpr \mid \lexpr \times \lexpr \mid \dots 
\end{rclarray}
\]
Assume a set of \emph{logical environments} \(\lenv \in \LEnv: \LVar \parfun \Val\) which associates logical variables with values.
Given a stack $\stk \in \Stacks$ (\defin\ref{def:stacks}) and a logical environment $\lenv \in \LEnv$, the \emph{logical expression evaluation} function, $\evalLE[(., .)]{.}:\LExpr \times \Stacks \times \LEnv\rightharpoonup \Val$, is defined inductively over the structure of logical expressions as follows,
%
\[
    \begin{rclarray}
        \evalLE{\val} & \defeq & \val \\
        \evalLE[\lenv, \thstk \uplus \txstk]{\thvar} & \defeq & \thstk(\thvar) \\
        \evalLE[\lenv, \thstk \uplus \txstk]{\txvar} & \defeq & \txstk(\txvar) \\
        \evalLE{\lvar} & \defeq & \lenv(\lvar) \\
        \evalLE{\lexpr_1 + \lexpr_2} & \defeq & \evalLE{\lexpr_1} + \evalLE{\lexpr_2} \\
        \evalLE{\lexpr_1 \times \lexpr_2} & \defeq & \evalLE{\lexpr_1} \times \evalLE{\lexpr_2} \\
        \dots & \defeq & \dots \\
    \end{rclarray}
\]
Note that the stack \( \stk \) includes transaction variables and thread variables.
\end{definition}

\emph{Fingerprint assertion} or \emph{fingerprint} is a set of tuples in the form of \( (\otag, \lexpr_{1}, \lexpr_{2}) \) where \( \otag \) is either read tag \( \etR \) or write \( \etW \) and the second and third elements are logical assertions representing the address and value respectively.
This assertion is interpreted to a set of transaction events as expected.

\begin{defn}[Fingerprint Assertions]
\label{def:fingerprint}
The \emph{fingerprint assertion} also \emph{fingerprint}, \( \fp \in \FAst \), is defined as the follows, 
\[
\begin{rclarray}
    \fp & \subseteq & \Setcon{ (\otag,\lexpr_{1},\lexpr_{2}) }{ \otag \in \OTags \land \lexpr_{1}, \lexpr_{2} \in \LExpr } \\
\end{rclarray}
\] 
Given a logical environment $\lenv \in \LEnv$ and a stack $\stk \in \Stacks$, the \emph{fingerprint interpretation} function, $\evalF[(., .)]{.}: \FAst \times \LEnv \times \Stacks \parfun \Opsets$, is defined as follows,
\[
\begin{rclarray}
    \evalF{\emptyset} & \defeq & \unitO  \\
    \evalF{\fp \addF (\otag, \lexpr_{1}, \lexpr_{2})} & \defeq & \evalF{\fp} \addO (\otag, \evalLE{\lexpr_{1}}, \evalLE{\lexpr_{2}})
\end{rclarray}
\]
\end{defn}

The local assertions includes normal separation logic assertions and extra fingerprint assertions, which are interpreted as sets of heaps and a set of events respectively.
Notice that the fingerprint assertion cannot be split.

\begin{definition}[Local assertions]
\label{def:local_assertions}
Given the set of logical expressions \( \LExpr \), logical variables \( \LVar \) and fingerprint assertion \( \FAst \), the set of \emph{local assertions}, $\lpre,  \lpost \in \LAst$, is defined inductively by the following grammar, 
\[
\begin{rclarray}
	\lpre, \lpost  & ::= & \False \mid \True \mid \lpre \land \lpost \mid \lpre \lor \lpost \mid \exsts{\lvar} \lpre \mid \lpre \implies \lpost \mid \Emp \mid \lexpr \pt \lexpr \mid \fpF \mid \lpre \sep \lpost  \\
\end{rclarray}	 
\]
Given a logical environment $\lenv \in \LEnv$, the \emph{local interpretation function}, $\evalLS[(.,.)]{.}: \LAst \times \LEnv \times \LAst \parfun \Heaps \times \powerset{ \Events } $, is defined over the structure of local assertions as follows,
\[
\begin{rclarray}
	\evalLS{\assfalse} & \eqdef & \emptyset \\
	\evalLS{\asstrue} & \defeq & \Heaps \times \powerset{ \Events } \\
	\evalLS{\lpre \land \lpost} & \defeq & \evalLS{\lpre} \cap \evalLS{\lpost} \\
	\evalLS{\lpre \lor \lpost} & \defeq & \evalLS{\lpre} \cup \evalLS{\lpost} \\
	\evalLS{\exsts{\lvar} \lpre} & \defeq & \bigcup\limits_{\val \in \textnormal{\Val}}\evalLS[\lenv\remapsto{\lvar}{\val}, \stk]{\lpre}  \\
	\evalLS{\lpre \implies \lpost} & \defeq & \Setcon{\h}{\h \in \evalLS{\lpre} \implies \h \in \evalLS{\lpost}}\\
	\evalLS{\assemp} & \defeq & \Set{ ( \unitH, \unitE) }  \\
	\evalLS{\lexpr_{1} \pt \lexpr_2 } & \defeq & \Set{ (\evalLE{\lexpr_1} \pt \evalLE{\lexpr_2}, \unitE) } \\
	\evalLS{ \fpF } & \defeq & \Set{ (\unitH, \evalF{\fp}) } \\
	\evalLS{\lpre \sep \lpost} & \defeq & 
    \Setcon{
        (\h_1 \composeH \h_2, \evset_{1} \composeE \evset_{2})
    }{ 
        (\h_{1},\evset_{1}) \in \evalLS{\lpre} 
        \land (\h_{2}, \evset_{2} ) \in \evalLS{\lpost} 
    } 
\end{rclarray}
\]
\end{definition}

Observe that program expressions $\Expr$  (\defin\ref{def:language}) are contained in logical expressions $\LExpr$ (\defin\ref{def:local_assertions} above), \ie $\Expr \subset \LExpr$. 
For readability, we will write angle brackets, \eg \( \fpass{(\etR, \vx, 0)} \) instead of curly brackets \( \fpto{\Set{(\etR, \vx, 0)}} \) for fingerprint assertions.

\subsection{Global/Program}

\begin{definition}[Capabilities]
\label{def:capabilities}
Assume a \emph{partial commutative monoid (PCM)} of \emph{client-specified capabilities} \( (\Kaps, \composeK, \unitK) \) with \( \kap \in \Kaps \), the composition \( \composeK \) the units set \( \unitK \).
Then given a set of \emph{region identifiers} \( \rid \in \RegionID \), the \emph{capability composition function} or \emph{capabilities} \( \ca \in \Caps \defeq \RegionID \parfun \Kaps \), where the composition \( \composeC \) is defined as the follows,
\[
    \begin{rclarray}
        (\ca_{l} \composeC \ca_{r})(\rid) & \defeq  &
        \begin{cases}
            \ca_{l}(\rid) \composeK \ca_{r}(\rid) & \rid \in \dom(\ca_{l}) \cap \dom(\ca_{l}) \\
            \ca_{l}(\rid)  & \rid \in \dom(\ca_{l}) \setminus \dom(\ca_{l}) \\
            \ca_{r}(\rid) & \rid \in \dom(\ca_{r}) \setminus \dom(\ca_{l}) \\
            \text{undefined} & \text{otherwise} \\
        \end{cases}
    \end{rclarray}
\]
, and the units set \( \unitC \defeq \Setcon{\ca}{\for{\rid} \ca(\rid) \in \unitK } \) .
\end{definition}

\begin{defn}[Interference]
\label{def:intf}
Given the fingerprint assertion \( \fp \in \Fingerprint \) (\defref{def:fingerprint}) and local assertion \( \lpre \in \LAst \) (\defref{def:local_assertions}), the grammar of \emph{interference assertions}, \( \intass \in \IAst \), is defined as the follows,
\[
\begin{rclarray}
	\intass & ::=  &
	\emptyset \mid \Set{ \perm{\kap} :  \exsts{\vec{\lvar}} \lpre \mat \fp } \cup \intass 
\end{rclarray}
\]
It will be interpreted to a set of \emph{interference environments}, this is,
\[
\begin{rclarray}
    \inter \in \Interference & \defeq & \Kaps \parfun \powerset{\Heaps} \times \Opsets
\end{rclarray}
\]
Given a logical environment $\lenv \in \LEnv$ and a stack $\stk \in \Stacks$, the \emph{interference interpretation} function, $\evalI[(., .)]{.}: \IAst \times \LEnv \times \Stacks \to \Interference$, is defined as follows,
%
\[
\begin{rclarray}
	\evalI{\emptyset}(\kap) & \eqdef & \text{undefined} \\
	\evalI{\Set{ \perm{\kap} : \exsts{\vec{\lvar}} \lpre \mat \fp } \cup \intass }(\kap') & \eqdef &
    \begin{cases}
    (\evalLS[\lenv',\stk]{\lpre}, \evalF[\lenv',\stk]{\fp}) \cup \evalI{\intass}(\kap')  & \kap = \kap' \\
    \evalI{\intass}(\kap') & \text{ otherwise} \\
    \end{cases} \\
    & & \text{where there exists a vector of values \( \vec{\val}\) such that } \lenv' = \lenv\rmto{\vec{\lvar}}{\vec{\val}} \\
\end{rclarray}
\] 
\end{defn}

\begin{defn}[Labelled transition system]
The labelled transition system is a tuple, \( (\aexecset,\actionset,\toLTS{}, \aexecset_{0}, \como) \), consisting of a set of abstract executions \( \aexecset \subseteq \Aexecs \), a set of actions \( \actionset \subseteq \Actions \), a relation \( \toLTS{} : \Aexecs \times \Actions \times \Aexecs \), a set of initial abstract executions \( \aexecset_{0}\) and the consistency model associated with the transition system \( \como \).
Assume all the initial abstract executions satisfies the consistency model.
The relation \( \toLTS{}\) is defined as the follows,
\[
\begin{rclarray}
    \aexec \toLTS{\evset} \aexec' & \defeq &
        \begin{array}[t]{@{}l}
        \exsts{\vis, \po, \ar, \txid } \\
        \quad {} \land \aexec' = (\aexec\prjT \uplus \Set{ \txid \mapsto \evset }, \aexec\prjP \uplus \po, \aexec\prjV \uplus \vis, \aexec\prjA \uplus \ar) \\
        \quad {} \land \vis, \po \subseteq \ar = \Setcon{(\txid', \txid)}{\txid' \in \dom(\aexec\prjT)} 
        \land \aexec' \in \evalCOM{\como}
    \end{array}
\end{rclarray}
\]
\end{defn}
 
\begin{defn}[Invariant a region]
\label{def:invariant-region}
\label{def:world2aexec}
\label{def:state2aexec}
Assume two global functions, \( \funcn{init} : \RegionID \to \powerset{\Aexecs} \) that returns initial abstract executions for regions, and \( \funcn{como} : \RegionID \to \ConsisModels \) that returns the consistency models associated with regions.
Also assume all the initial states for a region satisfy the consistency model, \ie
\[
\for{\rid, \aexec_{0}} \aexec_{0} \in \func{init}{\rid} \implies \aexec_{0} \in \evalCOM{\func{como}{\rid}}
\]
The invariant of a region, namely \( \func{transinv}{\rid, \intf} \), is the labelled transition system where the initial state is \( \func{init}{\rid}\) and all the actions are included in the interference.
\[
\begin{rclarray}
    \func{inv}{\rid, \intf} & \defeq & (\aexecset,\actionset \cup \Set{\unitE},\toLTS{}, \func{init}{\rid}, \func{como}{\rid}) \\
    & & \text{where } \for{\evset} \evset \in \actionset \implies \exsts{\kap} \evset \in \dom(\intf(\kap))
\end{rclarray}
\]
For brevity, \( \aexec \in \func{inv}{\rid, \intf} \) is short-hand for \( \aexec \in \aexecset \), and similarly \( \aexec \toLTS{\evset} \aexec' \in \func{inv}{\rid, \intf} \).
\end{defn}

The empty (unit) event \( \unitE \) in the invariant is a place-holder for other regions.
They might append a concrete event, while the composition of abstract executions composes events point-wise if the two executions have the same structure.

\begin{defn}[Well-form of a region]
\label{def:well-form-region}
The well-form condition of the interference, namely \( \pred{wfintf}{\rid, \intf} \) predicate, assertions for any concrete events \( \evset \), the state before the events must be included in the interference.
\[
\begin{rclarray}
    \pred{wfintf}{\rid, \intf} & \defeq & \for{\aexec, \aexec', \evset} \aexec \toLTS{\evset} \aexec' \in \func{inv}{\rid, \intf} \land ( \evset \neq \unitE \implies \pred{wfabs}{\rid, \intf, \aexec, \evset, \aexec'} ) \\ 
    \pred{wfabs}{\rid, \intf, \aexec, \evset, \aexec'} & \defeq & 
    \begin{array}[t]{@{}l}
        \exsts{ \kap }
        \evset \in ( \dom(\intf( \kap )) ) \land \obsstate{\aexec, \aexec\prjT,\func{como}{\rid}\projection{2}} \subseteq \intf(\kap)(\evset)
    \end{array} \\
\end{rclarray}
\]
Given a region (identifier) \(\rid\), its current state \( \h \), and its interference \( \intf \), the function \(\funcn{r2e} \) returns all the possible abstract executions that are included in the invariant of a region and satisfy the state \( \h \),
\[
\begin{rclarray}
    \func{r2e}{\rid, \h, \intf} & \eqdef & \Setcon{\aexec}{\aexec \in \func{inv}{\rid, \intf} \land \h \in \obsstate{\aexec,\aexec\prjT,\func{como}{\rid}\projection{2}}} \\
\end{rclarray}
\]
A set of heaps \( \hset \) approximates the observation of a region \( \rid \) under state \( \h \), namely \( \pred{approx}{\rid, \h, \hset, \intf} \), when an abstract execution satisfies the state \( \h \) and it can reach a new state by appending a new transaction \( \txid \), the observable state of the new transaction must included in the approximation \( \hset \),
\[
\begin{rclarray}
    \pred{approx}{\rid, \aexec, \hset, \intf} & \eqdef & 
    \begin{array}[t]{@{}l}
    \for{ \aexec' } \aexec \toLTS{\stub} \aexec' \in \func{inv}{\rid, \intf} \\
    \quad {} \land \exsts{ \txid } \txid = \max_{\ar}\Set{\aexec'\prjT}
    \land \obsstate{\aexec,\aexec\prjV^{-1}(\txid),\func{como}{\rid}\projection{2}} \subseteq \hset
    \end{array}
\end{rclarray}
\]
\end{defn}

\begin{definition}[Worlds]
\label{def:world}
Given the set of heaps $\Heaps$ (\defref{def:heaps}) and a set of \emph{region identifiers} \( \rid \in \RegionID \), the set of \emph{shared states} is \( \SStates \eqdef \RegionID \to \Heaps \times \powerset{\Heaps} \times \Interference \).
Each region has its current state, a set of possible initial states for transitions and the interference.
The \emph{shared state composition function}, $\composeS: \SStates \times \SStates \parfun \SStates$, is defined as $\composeS \eqdef \composeEq$, where for all domains $\sort M$ and all $m, m' \in \sort M$,
%
\[
\begin{rclarray}
	m \composeEq m' &  \eqdef  &
	\begin{cases}
		m & \text{if } m = m'\\
		\text{undefined} & \text{otherwise}
	\end{cases}
\end{rclarray}
\]
A \emph{world} \( \w \in \World \) is a pair of a shared state \( \gs \) and capabilities \( \ca \) (\defref{def:capabilities}), where regions are associated with the same consistency model and the collapse of the pair exists, \ie regions are well-form and compatible.
\[
\begin{rclarray}
	\world \in \World  & \eqdef & 
    \Setcon{
        (\ca, \gs) 
    }{ 
        \ca \in \Caps 
        \land \gs \in \SStates
        \land \clpsW{\gs} \neq \emptyset
        \land \dom(\ca) \subseteq \dom(\gs) \\
        \quad {} \land \for{\rid, \rid'}
        \func{como}{\rid} = \func{como}{\rid'} \\
        \quad {} \land \for{\h, \h' }
        \h \in \Set{\gs(\rid)\projection{1}} \cup \gs(\rid)\projection{2}
        \land \h' \in \Set{\gs(\rid')\projection{1}} \cup \gs(\rid')\projection{2} 
        \land ( \h \composeH \h' )\isdef
    }
\end{rclarray}
\]
The function, \( \clpsW{.} : \SStates \parfun \powerset{\Aexecs} \), collapses shared states to sets of abstract executions as the follows,
\[
\begin{rclarray}
    \clpsW{\emptyset} & \defeq & \unitAEX \\
    \clpsW{\Set{\rid \mapsto (\h, \hset, \intf)} \uplus \gs } & \defeq & 
        \Setcon{ \aexec \composeAEX \aexec' }{ \aexec \in \func{r2e}{\rid, \h, \intf} \land \pred{approx}{\rid, \aexec, \hset, \intf} \land \pred{wfintf}{\rid, \intf} \land \aexec' \in \clpsW{\gs} }\\
\end{rclarray}
\] 
% 
The \emph{world composition function}, $\composeW: \World \times \World \parfun \World$, is defined component-wise as: $\composeW \eqdef (\composeC, \composeS)$.
The \emph{world unit set} is $\unitW \eqdef \Setcon{(\ca, \gs)}{(\ca, \gs) \in \World \land \ca \in \unitC}$.
The \emph{partial commutative monoid of worlds} is $(\World, \composeW, \unitW)$.
\end{definition}

\sx{point-wise composition}
\begin{defn}[Invariant of worlds]
Because regions in a well-defined world must disjointed with each other and have the same consistency model, it is easy to lift the invariant of a region to a shared state,
\[
\begin{rclarray}
    \func{inv}{\emptyset} & \defeq & (\aexecset, ... ) \\
    \func{inv}{\Set{\rid \mapsto (\stub, \stub, \intf)} \uplus \gs} & \defeq & \Setcon{\aexec \composeAEX \aexec' }{\aexec \in \func{inv}{\rid, \intf} \land \aexec' \in \func{inv}{\gs}} \\
    \func{transinv}{\emptyset} & = & \Setcon{ ( \aexec , \unitE, \aexec' ) }{\aexec, \aexec' \in \unitAEX } \\
    \func{transinv}{\Set{\rid \mapsto (\stub, \stub, \intf)} \uplus \gs} & = & 
    \Setcon{
        ( \aexec \composeAEX \aexec_{f}, \evset \composeE \evset_{f}, \aexec' \composeAEX \aexec_{f}' ) 
    }{
        (\aexec, \evset, \aexec') \in \func{transinv}{\rid, \intf} \\
        \quad \land (\aexec_{f}, \evset_{f}, \aexec_{f}') \in \func{transinv}{\gs}
    }
\end{rclarray}
\]
\end{defn}

\begin{definition}[Assertions]
\label{def:assertion}
Assume standard separation logic assertion \( \bar{\lpre}, \bar{\lpost }\) (the local assertion \( \LAst \) without fingerprint) and the interpretation function, The set of \emph{assertions}, $\gpre, \gpost \in \Ast$, are defined by the following inductive grammar:
\[
\begin{rclarray}
	\gpre , \gpost & \defeq & \False \mid \True \mid \gpre \land \gpost \mid \gpre \lor \gpost \mid \exsts{\lvar}\gpre \mid \gpre \implies \gpost \mid \assemp \mid \cass{\kap}{\lrid} \mid \gpre \sep \gpost \mid \sptboxass{\bar{\lpre}}{\bar{\lpost}}{\lrid}{\intass}\\
\end{rclarray}
\]
%
where $\lvar, \lrid \in \LVar$, $\lexpr_1, \lexpr_2 \in \LExpr$ (\defin\ref{def:local_assertions}), $\kap \in \Kaps$ (\defin\ref{def:capabilities}) and $\intass \in \IAst$ (\defin\ref{def:intf}).
Given a logical environment $\lenv \in \LEnv$ and a stack $\stk \in \Stacks$, the \emph{assertion interpretation} function, $\evalW[(., .)]{.}: \Ast \times \LEnv \times \Stacks \to \powerset{\World}$, is defined as follows:
%
\[
\begin{rclarray}
	\evalW{\False} & \defeq & \emptyset \\
	\evalW{\True} & \defeq & \World \\
	\evalW{\emp} & \defeq & \unitW \\
	\evalW{\gpre \land \gpost} & \defeq & 
    \Setcon{
        (\ca, \gs)
    }{
        \exsts{\gs_{p}, \gs_{q}} 
        (\ca, \gs_{p}) \in \evalW{\gpre} 
        \land (\ca, \gs_{q}) \in \evalW{\gpost} \\
        \quad {} \land \for{\rid} 
        \exsts{\h, \hset_{p}, \hset_{q}, \intf} 
        \gs(\rid) = (\h, \hset_{p} \cap \hset_{q}, \intf) \\
        \qquad {} \land \gs_{p}(\rid) = (\h, \hset_{p}, \intf)
        \land \gs_{q}(\rid) = (\h, \hset_{q}, \intf)
    } \\
	\evalW{\gpre \lor \gpost} & \defeq & 
    \Setcon{
        (\ca, \gs)
    }{
        \exsts{\gs_{p}, \gs_{q}} 
        (\ca, \gs_{p}) \in \evalW{\gpre} 
        \land (\ca, \gs_{q}) \in \evalW{\gpost} \\
        \quad {} \land \for{\rid} 
        \exsts{\h, \hset_{p}, \hset_{q}, \intf} 
        \gs(\rid) = (\h, \hset_{p} \cup \hset_{q}, \intf) \\
        \qquad {} \land \gs_{p}(\rid) = (\h, \hset_{p}, \intf)
        \land \gs_{q}(\rid) = (\h, \hset_{q}, \intf)
    } \\
	\evalW{\exsts{\lvar}  \gpre} & \defeq & \bigcup\limits_{\val \in \textnormal{\Val}} \evalW[\lenv\remapsto{\lvar}{\val}, \stk]{\gpre} \\
	\evalW{\gpre \implies \gpost} & \defeq & \Setcon{\w}{\w \in \evalW{\gpre} \implies \w \in \evalW{\gpost}} \\
	\evalW{\cass{\kap}{\lrid}} & \defeq & \Setcon{ (\Set{\lrid \mapsto \kap}, \gs) }{\gs \in \SStates} \\
	\evalW{ \gpre \sep \gpost } & \defeq & 
	\Setcon{
	   (\world_1 \composeW \world_2) 
    }{
       \world_1 \in \evalW{\gpre} \land \world_2 \in \evalW{\gpost}
	} \\
	\evalW{ \sptboxass{\bar{\lpre}}{\bar{\lpost}}{\lrid}{\intass} } & \defeq & 
    \Setcon{
        (\ca,\Set{\lrid \mapsto (\h, \hset, \intf)} \uplus \gs)
    }{
        \ca \in \unitC 
        \land \h \in \evalLS{\bar{\lpre}}
        \land \hset = \evalLS{\bar{\lpost}}
        \land \intf  = \evalI{\intass}
    } \\
\end{rclarray}
\]
\end{definition}

We will write \( \boxass{\bar{\lpre}}{\lrid}{\intass} \) as a short-hand for \( \sptboxass{\bar{\lpre}}{\bar{\lpre}}{\lrid}{\intass} \) and \(\expr \pt N\) for \( \exsts{\nat \in N} \expr \pt \nat\) where \( N \subseteq \Val\).


%\subsection{Rules for Local}

The proof rules are standard except \rl{TRDeref} and \rl{TRMutate}.
The \rl{TRDeref} rule add read fingerprint in finger-tracking set, only if there is no write finger-print.
This is because once a location has been re-written, the rest read are considered as local operations, while the finger-print only records those operations might have effect on global state.

\begin{figure}[t]
\hrule\vspace{5pt}
\sx{Need to be careful about the implication, esp. for fingerprint assertion.}
\[
    \infer[\rl{TRSkip}]{%
        \tripleL{\assemp }{ \pskip }{\assemp }
    }{}
\]

\[
    \infer[\rl{TRAss}]{%
        \tripleL{\txvar \dot= \lexpr }{ \passign{\txvar}{\expr} }{\txvar \dot= \expr\sub{\txvar}{\lexpr} }
    }{%
    \txvar \notin \func{fv}[\lexpr]
        && \txvar \in \TxVars  
    }
\]

\[
    \infer[\rl{TRDeref}]{%
        \tripleL{\expr \pt \lexpr \sep \fpF }{ \pderef{\txvar}{\expr} }{\txvar \dot= \lexpr \sep \expr \pt \lexpr \sep \fpto{\assfp'} }
    }{%
    \txvar \notin \func{fv}[\expr]
        && \txvar \notin \func{fv}[\lexpr]
        && \txvar \in \TxVars  
        && \assfp' = \assfp \addF (\otR, \expr, \lexpr)
    }
\]

\[
    \infer[\rl{TRMutate}]{%
        \tripleL{\expr_1 \pt \stub \sep \fpF }{ \pmutate{\expr_1}{\expr_2} }{ \expr_1 \pt \expr_2 \sep \fpto{\assfp'} } 
    }{
        \assfp' = \assfp \addF (\otW, \expr_{1}, \expr_{2})
    }
\]

\[
    \infer[\rl{TRAssume}]{%
        \tripleL{ \expr \doteq 0 }{ \passume{\expr} }{ \expr \doteq 0 } 
    }{}
\]

\[
    \infer[\rl{TRChoice}]{%
        \tripleL{ \lpre }{ \trans_{1} \pchoice \trans_{2} }{ \lpost }
    }{%
        \tripleL{ \lpre }{ \trans_{1} }{ \lpost } && 
        \tripleL{ \lpre }{ \trans_{2} }{ \lpost } 
    }
\]

\[
    \infer[\rl{TRSeq}]{%
        \tripleL{ \lpre }{ \trans_{1} \pseq \trans_{2} }{ \lpost }
    }{%
        \tripleL{ \lpre }{ \trans_{1} }{ \lframe }  && 
        \tripleL{ \lframe }{ \trans_{2} }{ \lpost }
    }
\]

\sx{check the loop invariant with fingerprint}

\[
    \infer[\rl{TRLoop}]{%
        \tripleL{ \lpre }{ \trans\prepeat }{ \lpre }
    }{%
        \tripleL{ \lpre }{ \trans }{ \lpre } 
    }
\]
 
\[
   \infer[\rl{TRFrame}]{%
       \tripleL{ \lpre \sep \lframe }{ \trans }{ \lpost \sep \lframe }
   }{%
       \tripleL{ \lpre }{ \trans }{ \lpost } 
   }
\]
\hrule\vspace{5pt}
\caption{The rules for transactions}
\label{fig:rule-trans}
 \end{figure}

\subsection{Rely and Guarantee}

\begin{definition}[Rely and guarantee]
\label{def:rely-guarantee}
The \( \func{allowed} \) function asserts that a event is allowed by the owned capabilities,
\[
\begin{rclarray}
    \func{allowed}[\fp, \ca, \gs] & \defeq & \fp = \unitE \\
    \func{allowed}[\fp \composeE \fp', \ca, \gs \uplus \Set{\rid \mapsto (\stub, \stub, \intf)}] & \defeq & 
    \exsts{\kap } \kap \sqsubseteq \ca(\rid)
    \land \fp \in \dom(\intf(\kap)) 
    \land \func{allowed}[\fp', \ca, \gs]
\end{rclarray}
\]
Given the set of worlds $\World$ (\defref{def:world}), the \emph{update rely} relation, $\relyU \subseteq \World \times \World$, is defined as follows,
\[	
    \begin{rclarray}
	\relyU & \defeq &
	\myset{
		((\ca, \gs), (\ca, \gs'))	
	}{
        \exsts{\fp,\aexec, \aexec'}  
        \func{allowed}[\fp, \ca, \gs]  \\
        \quad {} \land \aexec \in \clpsW{\gs} 
        \land \aexec' \in \clpsW{\gs'}
        \land (\aexec, \fp, \aexec') \in \func{transinv}[\gs]
	} \\
    \end{rclarray}
\]
The invariant of a shared state is a lift of the invariants of interferences of regions.
The \emph{rely} relation, $\RelyI \defeq \World \times \World$, is defined as follows:
\[
    \begin{rclarray}
         \RelyI &\defeq & \closure{\left(\relyU\right)} \\
    \end{rclarray}
\]
A set of fingerprint worlds $\setworld \subseteq \World$ is \emph{stable}, written $\stable{\setworld}$, if and only if it is closed under the rely relation: 
\[
    \begin{rclarray}
        \stable{\setworld} & \defeq & \fora{\w_{p}, \w_{q}}  \w_{p} \in \setworld \land (\w_{p}, \w_{q}) \in \RelyI \implies \w_{q} \in \setworld
    \end{rclarray}
\]
The \emph{update guarantee} relation, $\guarU: \World \times \World$, is defined as follows:
\[	
    \begin{rclarray}
	\guarU & \defeq &
	\myset{
		((\ca, \gs), (\ca, \gs'))	
	}{
        \exsts{\fp, \ca', \aexec, \aexec'}  
        (\ca' \composeC \ca)\isdef
        \land \func{allowed}[\fp, \ca, \gs]  \\
        \quad {} \land \aexec \in \clpsW{\gs} 
        \land \aexec' \in \clpsW{\gs'}
        \land (\aexec, \fp, \aexec') \in \func{transinv}[\gs]
	} \\
    \end{rclarray}
\]
The \emph{guarantee} relation, $\GuarI \subseteq \World \times \World$, is defined as follows:
\sx{take away of the closure}
\[
	\GuarI \defeq \guarU
\]
\end{definition}

\subsection{Rules for Global}

The \rl{PRCommit} rule lifts the local effect of transaction \( \trans \) to global level by first converting global state to (local) observable state and then propagating the local fingerprint to the global state.
The \( \pred{down} \) predicate asserts that the local predicate \( \lpre \) is a over-approximation of the valid observation that is given by the interference.
The \( \pred{up} \) predicate says the post-condition \( \gpost \) is the result by lifting the local fingerprints \( \assfp \) to pre-condition \( \gpre \).



\begin{figure}[t!]
\hrule\vspace{5pt}

\sx{not right, we want to say a transaction \( \trans \) can be abstracted to a fingerprint \( \assfp \)}

\[
    \infer[\rl{PRCommit}]{%
        \tripleG{\gpre}{ \ptrans{\trans} }{\gpost}
    }{%
        \begin{array}{c}
        \gpre \snap \lpre
        \quad \tripleL{\lpre \sep \fpass{}}{\trans}{\lpost \sep \fpF}
        \quad \rpt{\gpre}{\gpost}{\assfp} \\
        \stable{\gpre} 
        \quad \stable{\gpost} 
        \end{array}
    }
\]

\[
    \infer[\rl{PRAss}]{%
        \tripleG{\var \dot= \lexpr }{ \passign{\var}{\expr} }{\var \dot= \expr\sub{\var}{\lexpr} }
    }{%
        \var \notin \func{fv}[\lexpr]
        && \var \in \ThdVars  
    }
\]

\[
    \infer[\rl{PRAssume}]{%
        \tripleG{ \expr \doteq 0 }{ \passume{\expr} }{ \expr \doteq 0 } 
    }{}
\]

\[
    \infer[\rl{PRChoice}]{%
        \tripleG{ \gpre }{ \prog_{1} \pchoice \prog_{2} }{ \gpost }
    }{%
        \tripleG{ \gpre }{ \prog_{1} }{ \gpost } && 
        \tripleG{ \gpre }{ \prog_{2} }{ \gpost } 
    }
\]

\[
    \infer[\rl{TRSeq}]{%
        \tripleG{ \gpre }{ \prog_{1} \pseq \prog_{2} }{ \gpost }
    }{%
        \tripleG{ \gpre }{ \prog_{1} }{ \gframe }  && 
        \tripleG{ \gframe }{ \prog_{2} }{ \gpost }
    }
\]

\[
    \infer[\rl{TRLoop}]{%
        \tripleG{ \gpre }{ \prog\prepeat }{ \gpre }
    }{%
        \tripleG{ \gpre }{ \prog }{ \gpre } 
    }
\]
 
\[
   \infer[\rl{TRFrame}]{%
       \tripleG{ \gpre \sep \gframe }{ \prog }{ \gpost \sep \gframe }
   }{%
       \tripleG{ \gpre }{ \prog }{ \gpost } 
   }
\]
 
\[
   \infer[\rl{TRPar}]{%
       \tripleG{ \gpre_{1} \sep \gpre_{2} }{ \prog_{1} \ppar \prog_{2} }{ \gpost_{1} \sep \gpost_{2} }
   }{%
       \tripleG{ \gpre_{1} }{ \prog_{1} }{ \gpost_{1} }
       && \tripleG{ \gpre_{2} }{ \prog_{2} }{ \gpost_{2} }
   }
\]

\[
\begin{rclarray}
    \gpre \snap \lpre & \defeq & \fora{ \w, \sn } \w \in \evalW{\gpre} \land \pred{takeinv}{\w, \sn} \implies \sn \in \evalLS{\lpre}\\
    \pred{takeinv}{\gs, \sn} & \defeq & \gs = \emptyset \land \sn = \unitH \\
    \pred{takeinv}{\Set{\rid \mapsto (\stub, \snset, \intf)} \uplus \gs, \sn \composeH \sn'} & \defeq & \sn \in \snset \land \pred{takeinv}{\gs,\sn'}\\
    \rpt{\gpre}{\gpost}{\assfp} & \defeq & 
    \begin{array}[t]{@{}l@{}}
        \fora{\w, \w', \aexec, \aexec'} \\
        \quad \w \in \evalW{\gpre}
        \land \aexec \in \clpsW{\w}
        \land \func{reachable}{\aexec, \aexec', \stub, \evalF{\assfp}, \ET}  \\
        \quad {} \land \aexec' \in \clpsW{\w'}
        \land (\w, \w') \in \Guar 
        \implies \w' \in \evalW{\gpost}
    \end{array} \\
\end{rclarray}                          
\]

\sx{
    For LLW the propagate can be done by syntactically update those addresses that has been written.
}

\hrule\vspace{5pt}
\caption{The rules for programs}
\label{fig:rule-prog}
\end{figure}

Many consistency model use last write win resolution policy, such as snapshot isolation, therefore the repartition \( \rpt{\gpre}{\gpost}{\assfp} \) can be simplified by checking the guarantee and then syntactically propagating the write fingerprints.
Also in practice, many implementation of consistency model assume strong session constraint.

\begin{figure}
\hrule\vspace{5pt}

\[
   \infer[\rl{FRead}]{%
       \tripleF{ \lexpr \pt \lexpr' \mid \lexpr \pt \lexpr'' }{ \Set{(\otR, \lexpr, \lexpr'')} }{ \lexpr \pt \lexpr' \mid \lexpr \pt \lexpr''}
   }{}
\]

\[
   \infer[\rl{FWriteNS}]{%
       \tripleF{ \lexpr \pt \lexpr' \mid \lexpr \pt \lexpr'' }{ \Set{(\otW, \lexpr, \lexpr''')} }{ \lexpr \pt \lexpr''' \mid ( \lexpr \pt \lexpr'' \lor \lexpr \pt \lexpr''') }
   }{}
\]

\[
   \infer[\rl{FWriteS}]{%
       \tripleF{ \lexpr \pt \lexpr' \mid \lexpr \pt \lexpr'' }{ \Set{(\otW, \lexpr, \lexpr''')} }{ \lexpr \pt \lexpr''' \mid \lexpr \pt \lexpr'''}
   }{}
\]

\[
   \infer[\rl{FFrame}]{%
       \tripleF{ \lpre \sep \lframe  \mid \lpre' \sep \lframe' }{ \assfp }{ \lpost \sep \lframe \mid \lpost' \sep \lframe' }
   }{%
       \tripleF{ \lpre \mid \lpre' }{ \assfp }{ \lpost \mid \lpost' }
   }
\]

\[
   \infer[\rl{FContinue}]{%
       \tripleF{ \lpre \sep \lframe  \mid \lpre' \sep \lframe' }{ \assfp  \uplus \assfp' }{ \lpost \sep \lframe \mid \lpost' \sep \lframe' }
   }{%
       \tripleF{ \lpre \sep \lframe  \mid \lpre' \sep \lframe' }{ \assfp }{ \lpost \sep \lframe \mid \lpost' \sep \lframe' }
   }
\]

\[
\begin{rclarray}
    \rpt{\gpre}{\gpost}{\assfp} & \defeq & 
    \begin{array}[t]{@{}l}
    \fora{\w, \w', \sn, \sn', \lpre, \lpre', \lpost, \lpost'}
    \w \in \evalW{\gpre} \\
    \quad {} \land \pred{unbox}{\w\projection{2}, \sn}
    \land \sn \in \evalLS{\lpre} 
    \land \gpre \snap \lpre' \\
    \quad {} \land {} \tripleF{ \lpre \mid \lpre' }{ \assfp }{ \lpost | \lpost'} \\
    \quad {} \land \pred{unbox}{\w'\projection{2},\sn'}
    \land \sn' \in \evalLS{\lpost} 
    \land \gpost \snap \lpost' \\
    \quad \implies \w' \in \evalW{\gpost}
    \end{array} \\
    \pred{unbox}{\gs, \sn} & \defeq & \gs = \emptyset \land \sn = \unitH \\
    \pred{unbox}{\Set{\rid \mapsto (\sn, \stub, \stub)} \uplus \gs, \sn \composeH \sn' } & \defeq & \pred{unbox}{\gs, \sn'} \\
\end{rclarray}
\]

\hrule\vspace{5pt}
\caption{Simplified repartition for last write win}
\label{fig:rule-prog}
\end{figure}


