\section{Logic}

\sx{small intro for why a logic}

We present a logic that is parametrised by any consistency model \emph{at least satisfying atomic  read and monotonic read}.
We motivate it by the \emph{write skew} example under snapshot isolation (\cref{fig:write-skew-si-proof}).
This example distinguishes serialisibility from snapshot isolation (SI).
Under serialisibility that transactions appear one after another, only one key, \( \vx \) or \( \vy \), will be 1 at the end.
While under SI, both keys \( \vx \) and \( \vy \) might be 1.
Because both transactions may take snapshots where \( \vx \) and \( \vy \) are 0, and both can commit because the two transactions write different keys.

In the sketch proof (\cref{fig:write-skew-si-proof}), \emph{A shared region assertion} also know as \emph{a boxed assertion} in the form of \( \boxass{\gpre}{\lrid}{\intass}\) describes that there is a region with \emph{a unique region identifier} \( \lrid \) and \emph{interference} \( \intass \), and the assertions inside \( \gpre \) satisfies views on key-value stores.
Note that there might be many views on the same key-value store, some of which might be out-of-date.
In the assertion, a value with a underline means \emph{a potentially out-of-date value}, otherwise it is the \emph{most recent value}.
For instance, \( \boxass{\vx \pt 0 \sep ( \vy \pt \underline{0} \lor \vy \pt \underline{1} ) }{\lrid}{\intass}\) from \( P1 \) asserts views where \( \vx \) points to the most recent value of 0, but \( \vy \) points to a potentially out-of-date value of either 0 or 1.
Each region are shareable and indivisible, \ie \( \boxass{\gpre}{\lrid}{\intass} \sep \boxass{\gpost}{\lrid}{\intass} \iff \boxass{\gpre \land \gpost}{\lrid}{\intass}\).

Regions are associated with \emph{interference} to specify how the regions can evolve.
The \emph{interference} \( \intass \) is a set of actions, each of which has the form \( \kap : \fp \) where \( \kap \) is the \emph{client-specified capability} and \( \fp \) is the \emph{fingerprint}.
A client is allowed to execute a transaction with the fingerprint \( \fp \), when the client holds the capability \( \kap \).
The \emph{client-specified capability} forms \emph{a partial commutative monoid (PCM)} where \( \composeK \) denotes the composition function.
In the write skew example, both \( \CB{L} \) and \( \CB{R} \) are unique because the compositions with themselves are undefined, and \( \CB{0} \) is the unit.
The \( \CB{L}\) allows a client to read \( \vy \) when it is 0 and write 1 to \( \vx \), and similarly \( \CB{R} \) allows a client to read \( \vx \) when it is 0 and write 1 to \( \vy \).
The unit \( \CB{0} \) allows clients to always read \( \vx \) or \( \vy \) no matter the values.

When a transaction starts, it take a snapshot give provided by the views.
For example the \( \txid_1 \) in \cref{fig:write-skew-si-proof} has precondition \( p1 \).
The transactional assertions for a single key are combinations of the state of key and the fingerprint.
They are one of the forms: \( \vx \fpI 0 \), \( \vx \fpR 0\), \( \vx \fpW 0\), and \( \vx \fpRW (0,1) \), where \( \otR \) and \( \otW \) are read and write labels.
The first three asserts the key \( \vx \) in the local snapshot has value 0 and it has not been touched, has been read and has been written respectively.
The last one asserts that the key \( \vx \) has value 1, \emph{the first read preceding any write} fetches 0 and the \emph{the last write} updates the key to \( 1 \).

\begin{figure}[!t]
\hrule
\[
\intass :
\begin{array}[t]{@{} c @{\quad} c @{\quad} c @{\quad} c @{} }
\begin{rclarray}[t]
    \CB{L} & : & \vx \fpW 1 \sep \vy \fpR 0  \\
\end{rclarray}
&
\begin{rclarray}[t]
    \CB{R} & : & \vx \fpR 0 \sep \vy \fpW 1  \\
\end{rclarray}
&
\begin{rclarray}[t]
    \CB{0} & : & \exsts{\V n} \vx \fpR \V{n} \\
\end{rclarray} 
&
\begin{rclarray}[t]
    \CB{0} & : & \exsts{\V n} \vy \fpR \V{n} \\
\end{rclarray} \\
\end{array}
\]
\[
\CB{L} \composeK \CB{L} \ \text{is undefined} \quad  \CB{R} \composeK \CB{R} \ \text{is undefined} \quad \CB{0} \ \text{is the unit}
\]
\hrule\vspace{5pt}
\[
\begin{session}
{\color{blue}P : } \specline{ \cass{\CB{L}}{\lrid} \sep \cass{\CB{R}}{\lrid} \sep \boxass{\vx \pt 0 \sep \vy \pt 0 }{\lrid}{\intass}  } \\
\begin{parl}
\begin{session}
    {\color{blue}P1 : } \specline{\cass{\CB{L}}{\lrid} \sep 
            \boxass{\vx \pt 0 \sep ( \vy \pt \underline{0}  \lor \vy \pt \underline{1} ) }{\lrid}{\intass} 
    } \\
    \txid_1 : \begin{transaction}
        {\color{blue}p1 : } \specline{\vx \fpI 0 \sep ( \vy \fpI 0 \lor \vy \fpI 1 )} \\
        \pderef{\pvar{b}}{\vy} ; 
        \quad \pifs{\pvar{b} = 0} 
        \pmutate{\vx}{1} ;
        \pife \\
        {\color{blue}q1 : } \specline{\vx \fpW 1 \sep  \vy \fpR 0 \lor \vx \fpI 0 \sep \vy \fpR 1 )} \\
    \end{transaction} \\
    {\color{blue}Q1 : } \specline{\cass{\CB{L}}{\lrid} \sep 
            \boxass{ \vx \pt 1 \sep \vy \pt \underline{0} \lor {} \\ \vx \pt 1 \sep \vy \pt \underline{1} \lor \vx \pt 0 \sep \vy \pt \underline{1} }{\lrid}{\intass} \\
    } \\
\end{session}
&
\begin{session}
    {\color{blue}P2 : } \specline{\cass{\CB{R}}{\lrid} \sep 
            \boxass{ ( \vx \pt \underline{0} \lor \vx \pt \underline{1} ) \sep \vy \pt 0 }{\lrid}{\intass} 
    } \\
    \txid_2 : \begin{transaction}
        {\color{blue}p2 : } \specline{ ( \vx \fpI 0 \lor \vx \fpI 1 ) \sep \vy \fpI 0 )} \\
        \pderef{\pvar{a}}{\vx} ; 
        \quad \pifs{\pvar{a} = 0} 
        \pmutate{\vy}{1} ; 
        \pife \\
        {\color{blue}q2 : } \specline{ \vx \fpR 0 \sep \vy \fpW 1 \lor \vx \fpR 1 \sep \vy \fpI 0 )} \\
    \end{transaction} \\
    {\color{blue}Q2 : } \specline{\cass{\CB{R}}{\lrid} \sep 
            \boxass{\vx \pt \underline{0} \sep \vy \pt 1 \lor {} \\ \vx \pt \underline{1} \sep \vy \pt 1 \lor \vx \pt \underline{1} \sep \vy \pt 0 }{\lrid}{\intass} \\
    } \\
\end{session}
\end{parl} \\
{\color{blue}Q : } \specline{%
    \cass{\CB{L}}{\lrid} \sep \cass{\CB{R}}{\lrid} \sep 
    \boxass{\vx \pt 0 \sep \vy \pt 1 \lor \vx \pt 1 \sep \vy \pt 1 \lor \vx \pt 1 \sep \vy \pt 0 }{\lrid}{\intass} \\ } \\
\end{session}
\]
\hrule
\caption{Interference, capabilities (the top), and sketch proof (the bottom) for write skew under snapshot isolation}
\label{fig:write-skew-si-proof}
\end{figure}

%For serialisibility, the stable PL and PR rules out those views that cannot progress via any possible transactions.
%This mean the view must at the end of the history heap.
%The post conditions QL and QR is stronger because the consistency model check in the repartition 

%\[
%\begin{session}
%\specline{\boxass{\vx \pt 0 \sep \vy \pt 0}{\lrid}{\intass} \sep \cass{\CB{L}}{\lrid} \sep \cass{\CB{R}}{\lrid} } \\
%\begin{parl}
%\begin{session}
    %stable-PL : \specline{\cass{\CB{L}}{\lrid} \sep 
            %\boxass{\vx \pt \Set{0} \sep  \begin{B} \vy \pt \Set{0} \lor \vy \pt \Set{1} \end{B} }{\lrid}{\intass} 
    %} \\
    %%PL2 : \specline{\cass{\CB{L}}{\lrid} \sep 
            %%\boxass{\vx \pt 0 \sep \vy \pt 0 }{\lrid}{\intass} \lor {} \\
            %%\boxass{\vx \pt 0 \sep ( \vy \pt 0 \lor \vy \pt 1 ) \sep \cass{\CB{R}}{\lrid}}{\lrid}{\intass} 
    %%} \\
    %\begin{transaction}
        %\specline{\vx \fpI 0 \sep ( \vy \fpI 0 \lor \vy \fpI 1 )} \\
        %\pderef{\pvar{b}}{\vy} ; \\
        %\pifs{\pvar{b} = 0} 
        %\pmutate{\vx}{1} ;
        %\pife \\
        %\specline{\vx \fpW 1 \sep  \vy \fpR 0 \lor \vx \fpI 0 \sep \vy \fpR 1 )} \\
    %\end{transaction} \\
    %QL : \specline{\cass{\CB{L}}{\lrid} \sep 
            %\boxass{\vx \pt \Set{1} \sep \vy \pt \Set{0} }{\lrid}{\intass} \\
            %{} \lor \boxass{\vx \pt \Set{0} \sep \vy \pt \Set{1} }{\lrid}{\intass} \\
    %} \\
%\end{session}
%&
%\begin{session}
    %PR : \specline{\cass{\CB{R}}{\lrid} \sep 
            %\boxass{ \begin{B} \vx \pt \Set{0} \lor \vx \pt \Set{1} \end{B} {} \sep \vy \pt \Set{0} }{\lrid}{\intass} 
    %} \\
    %\begin{transaction}
        %\pderef{\pvar{a}}{\vx} ; 
        %\quad \pifs{\pvar{a} = 0} 
        %\pmutate{\vy}{1} ; 
        %\pife 
    %\end{transaction} \\
    %QR : \specline{\cass{\CB{R}}{\lrid} \sep 
            %\boxass{\vx \pt \Set{0} \sep \vy \pt \Set{1} }{\lrid}{\intass} \\
            %{} \lor \boxass{\vx \pt \Set{1} \sep \vy \pt \Set{0} }{\lrid}{\intass} \\
    %} \\
%\end{session}
%\end{parl} \\
%QL \sep QR : \specline{\cass{\CB{L}}{\lrid} \sep \cass{\CB{R}}{\lrid} \sep 
            %\boxass{\vx \pt \Set{0} \sep \vy \pt \Set{1} }{\lrid}{\intass} 
            %\lor \boxass{\vx \pt \Set{1} \sep \vy \pt \Set{0} }{\lrid}{\intass} \\
%}
%\end{session}
%\]



\section{Assertion and Rules\label{sec:assertion}}

\subsection{Local/Transaction}

\azalea{
I would define local assertions as you have PLUS the capability assertion $\captk{\capid(\vec{\lvar})}_{\rid}$. Local assertions are what you gain when you open boxes and they may very well contain capabilities. If you add this here now, then you don't need to redefine $\lpre\uparrow$ later for interference assertions. I would also define LOCAL STATES (pairs of fingerprint heaps and capabilities) over which local assertions are interpreted. \\

Note that if you do this, you need to slightly change your repartitioning definition to ensure that the capability components of $\lpre$ and $\lpost$ are empty.}
\sx{Change repartition}

\begin{definition}[capabilities]\label{def:capabilities}
Assume a partial commutative monoid for \emph{primitive capabilities} $(\Kaps, \composeK, \unitK)$ with $\kap \in \Kaps$.
Assume a countably infinite set of region identifiers $\rid \in \Regionid$. The set of \emph{capabilities} is $\ca \in \Caps \eqdef \Regionid \rightharpoonup \Kaps$.
The \emph{capability composition function}, $\composeC: \Caps \times \Caps \rightharpoonup \Caps$, is defined as follows:
%
\[
	(\ca_1 \composeC \ca_2)(\rid) \defeq 
	\begin{cases}
		\ca_1(\rid) \composeK \ca_2(\rid) & \text{if } \rid \in \dom(\ca_1) \text{ and } \rid \in \dom(\ca_2) \\
		\ca_1(\rid) & \text{if } \rid \in \dom(\ca_1) \text{ and } \rid \not\in \dom(\ca_2) \\
		\ca_2(\rid) & \text{if } \rid \not\in \dom(\ca_1) \text{ and } \rid \in \dom(\ca_2) \\
		\text{undefined} & \text{otherwise}
	\end{cases}
\]
%
The \emph{capability unit set}, $\unitC$, denotes a function with an empty domain.
The \emph{capability partial commutative monoid} is $(\Caps, \composeC, \unitC)$. 
\end{definition}
%
%
\begin{definition}[Local state]\label{def:local_state}
Given the set of fingerprint heaps $\FPHeaps$ (\defin\ref{def:fingerprint_heaps}) and the set of capabilities $\Caps$ (\defin\ref{def:capabilities}), the set of \emph{local states} is $\ls \in \LStates \eqdef \FPHeaps \times \Caps$.
\end{definition}
%
Given a local state $\ls$, we write $\lsFPH{\ls}$ and $\lsCap{\ls}$ for the first and second projections of $\ls$, respectively.
%
%
\begin{definition}[Local assertions]\label{def:local_assertions}
%Given the set of program expressions $\Expressions$ (\defin\ref{def:language}), 
Assume a countably infinite set of \emph{logical variables} $\V x \in \LVars$.
The set of \emph{logical expressions}, $ \lexpr \in \LExpressions$ is defined by the following inductive grammar, where $\val \in \Val$, $\vx \in \Var$ (\defin\ref{def:prgram_values}) and $\V x \in \LVars$:
%
\[
   \lexpr ::= \val \mid \var \mid \V x \mid \lexpr + \lexpr \mid \lexpr * \lexpr \mid \dots 
\]
%
Given the set of values $\Val$ (\defin\ref{def:prgram_values}), assume a set of \emph{logical environments} $\lenv \in \LEnv: \LVars \rightharpoonup \Val$.
Given a stack $\stk \in \Stack$ (\defin\ref{def:stacks}) and a logical environment $\lenv: \LEnv$, the \emph{logical expression evaluation} function, $\leval{.}{(., .)}:\LExpressions \times \Stack \times \LEnv\rightharpoonup \Val$, is defined inductively over the structure of logical expressions as follows: 
%
\[
    \begin{rclarray}
        \leval{\val}{\lenv, \stk} & \defeq & \val \\
        \leval{\var}{\lenv, \stk} & \defeq & \stk(\val) \\
        \leval{\V x}{\lenv, \stk} & \defeq & \lenv(\V x) \\
        \leval{\lexpr_1 + \lexpr_2}{\lenv, \stk} & \defeq & \leval{\lexpr_1}{\lenv, \stk} + \leval{\lexpr_2}{\lenv, \stk}   \\
        \leval{\lexpr_1 * \lexpr_2}{\lenv, \stk} & \defeq & \leval{\lexpr_1}{\lenv, \stk} * \leval{\lexpr_2}{\lenv, \stk}  
    \end{rclarray}
\]
%
The set of \emph{local assertions}, $\lpre,  \lpost \in \Localassertion$, is defined inductively by the following grammar, where $\fp \in \Fingerprint$ denotes a fingerprint (\defin\ref{def:fingerprint_heaps}) and $\V x, \V r \in \LVars$: 
%
\[
\begin{array}{r @{\hspace{2pt}} l}
	\lpre, \lpost ::= & \False \mid \True \mid \lpre \land \lpost \mid \lpre \lor \lpost \mid \exsts{\V x} \lpre \\
	& \mid \emp \mid \lexpr \pt_{\fp} \lexpr \mid \cass{\kap}{\V r} \mid \lpre * \lpost 
\end{array}	 
\]
%
Given a logical environment $\lenv \in \LEnv$, the \emph{local interpretation function}, $: \Localassertion \times \LEnv \rightarrow \powerset{\LStates}$, is defined over the structure of local assertions as follows:
%
\[
\begin{array}{r @{\hspace{2pt}} l}
	\aeval{\assfalse}{\lenv, \stk} \eqdef & \emptyset  \\
%	
	\aeval{\asstrue}{\lenv, \stk} \defeq & \LStates  \\
%
	\aeval{\lpre \land \lpost}{\lenv, \stk} \defeq & \aeval{\lpre}{\lenv, \stk} \cap \aeval{\lpost}{\lenv, \stk} \\
%
	\aeval{\lpre \lor \lpost}{\lenv, \stk} \defeq & \aeval{\lpre}{\lenv, \stk} \cup \aeval{\lpost}{\lenv, \stk} \\
%
	\aeval{\exsts{\V x} \lpre}{\lenv, \stk} \defeq & \bigcup\limits_{\val \in \Val}\aeval{\lpre}{\lenv\remapsto{\V x}{\val}, \stk}  \\
%
	\aeval{\assemp}{\lenv, \stk} \defeq & \Set{(\unitFP, \unitC)}  \\
%
	\aeval{\lexpr_1 \pt_{\fp} \lexpr_2}{\lenv, \stk} \defeq 
	& \Set{
		(\fph, \unitC) \mid
        \begin{array}{@{}l@{}}
			\exsts{\loc, \val} 
			\leval{\lexpr_1}{\lenv, \stk} = \loc 
			\land \leval{\lexpr_2}{\lenv, \stk} = \val  
			\land \dom(\fph) {=} \Set{\loc }
			\land \fph(\loc) {=} (\val, \fp)
		\end{array}
	} \\
%
	\aeval{\cass{\kap}{\V r}}{\lenv, \stk} \eqdef
	& \Set{
		(\unitFP, \ca) \mid
        \begin{array}{@{}l@{}}
			\exsts{\rid} 
			\lenv(\V r) {=} \rid 
			\land \dom(\ca) {=} \Set{\rid }
			\land \ca(\rid) {=} \kap
		\end{array}
	} \\
%		
	\aeval{\lpre \sep \lpost}{\lenv, \stk} \defeq 
	& \Set{ \ls_1 \uplus \ls_2 \middle| \ls_1 \in \aeval{\lpre}{\lenv, \stk} \land \ls_2 \in \aeval{\lpost}{\lenv, \stk} } 
\end{array}
\]
%
\end{definition}
%

\subsection{Fingerprint (temp) -- Shale}
We define an ordering $\sqsubseteq$ on fingerprints as follows:

\begin{center}
    \begin{tikzpicture}
        \node (emp) {\(\emptyset\)};
        \node[above right=0.3cm and 0.3cm of emp] (r) {\(\Set{\rfp}\)};
        \node[above left=0.3cm and 0.3cm of emp] (w) {\(\Set{\wfp}\)};
        \node[above=0.3cm of r] (rw) {\(\Set{\rfp, \wfp}\)};
        \draw[->] (emp) -- (r);
        \draw[->] (emp) -- (w);F
        \draw[->] (r) -- (rw);
    \end{tikzpicture}
\end{center}

\azalea{Your diagram excludes the reflexive and transitive closure. So instead I would write:
%
\[
	\sqsubseteq \ \defeq\ \subseteq \setminus \{(\Set{\wfp}, \Set{\rfp, \wfp})\}
\]
where $\rightarrow$ is as depicted in the diagram and $*$ denotes the reflexive transitive closure of the $\rightarrow$ relation.
}

\sx{
    I can actually define \( \sqsubseteq \) just as \( \subseteq \), then I need a lemma to say: even there is no read,  but it is fine to also tag it with read fingerprint.
    This part is minor, I will change later.
}

Given the order, we define a left merge operator on fingerprints.
It reflects the fact that once a transaction writes to a location, the following reads from the same location are local.

\[
    \begin{rclarray}
        \fp_{1} \bmrg \fp_{2} & \defeq & 
        \begin{funcarray}
            \fp_{1} \cup \fp_{2}  & \fp_{1} \sqsubseteq \fp_{1} \cup \fp_{2} \\
            \fp_{1}  & o.w. \\
        \end{funcarray}
    \end{rclarray}
\]

\subsection{Interference -- Shale}

\begin{defn}[Interferences]
\label{def:interferences}
The set of \emph{Interferences} is \( \Interference \defeq \Regionid \parfun \Kaps \parfun \powerset{\FPHeaps \times \FPHeaps } \),  which are all the possible transitions for a given region identifier \( \rid \in \Regionid \) and primitive capability \( \kap \in \Kaps \).
\end{defn}

\begin{defn}[Interference assertions]
\label{def:interference-assertion}
The \emph{interference assertions} are sets of transitions labelled by \emph{primitive capabilities}, where \( \lpre, \lpost \in \Localassertion \) and \( \kap \in \Kaps \) for same \( \Kaps \):

\[
    \begin{rclarray}
        \intass \in \Interferenceassertion & \defeq & 
              \emptyset \mid \Set{ \captk{\kap} : \exists \vec{\V x} \ldotp \lpre \transfersto \lpost } \uplus \intass \\
    \end{rclarray}
\]
\end{defn}

Given the set of interferences (\defin \ref{def:interferences}), the \emph{interference assertion evaluation} function, \( \eval{.}_{(.,.)}: \Interferenceassertion \parfun \Interference \) is defined inductively over the structure of interference assertions:

\[
    \begin{rclarray}
        \eval{\emptyset}_{ \lenv, \stk}(\dontcare)(\dontcare) & \defeq & \emptyset \\
        \eval{\Set{\cass{\kap}{\V \rid} : \exists \vec{\V x} \ldotp \lpre \transfersto \lpost} \uplus \intass}_{\lenv, \stk}(\rid)(\kap') & \defeq & 
        \begin{funcarray}
            \Set{(\ls,\ls') \middle| 
            \begin{array}{@{}l@{}}
                \exists \vec{\val}, \lenv' = \lenv[\vec{\V x} \mapsto \vec{\val}'] \ldotp \\
                \ls \in \eval{\lpre}_{\lenv', \stk} \land  \ls' \in \eval{\lpost}_{\lenv', \stk} 
            \end{array}} & {\V \rid} = \rid \land \capid = \capid' \\ 
            \eval{\intass}_{\lenv, \stk}(\rid, \capid') & \text{otherwise}
        \end{funcarray} \\
    \end{rclarray}
\]

Given the evaluation, the well-formedness of interference assertions is defined as follows:
\[
    \begin{rclarray} 
        \pred{wfIntf}{\intf} & \defeq & \forall \rid, \kap, \ls, \ls' \ldotp (\ls, \ls' ) \in \eval{\intf}(\rid)(\kap) \land \pred{wf}{\ls, \ls'}\\
        \pred{wf}{\ls,\ls'} & \defeq & \forall \loc \ldotp \lsFPH{\ls} = (\dontcare, \emptyset) \land \dom(\lsFPH{\ls}) = \dom(\lsFPH{\ls'})
    \end{rclarray}
\]



\sx{
How to interpret interference parametrised \( \kap \)?
For example,
\[
    \cass{T(n)}{} \lvar \pt_{\emptyset} n \transfersto \lvar \pt_{\Set{\wfp}} n+1
\]
}


\subsection{Rules for Local}
The proof rules are standard except \rl{TRDeref} and \rl{TRMutate}.
The \rl{TRDeref} rule add read finger-print in finger-tracking set, only if there is no write finger-print.
This is because once a location has been re-written, the rest read are considered as local operations, while the finger-print only records those operations might have effect on global state.

\[
    \infer[\rl{TRDeref}]{%
        \judgement{}{\var = \dontcare \land \lexpr_1 \pt_{\fp} \lexpr_2 }{ \pderef{\var}{\lexpr_1} }{\var = \lexpr_2 \land \lexpr_1 \pt_{\addRFP{\fp}} \lexpr_2 }
    }{%
        \var \notin \func{fv}{\lexpr_1} \quad
        \var \notin \func{fv}{\lexpr_2} \quad 
    }
\]

\[
    \infer[\rl{TRMutate}]{%
        \judgement{}{\lexpr_1 \pt_{\fp} \dontcare }{ \pmutate{\lexpr_1}{\lexpr_2} }{ \lexpr_1 \pt_{\addWFP{\fp}} \lexpr_2} 
    }{}
\]

\subsection{Global/Program}
\begin{definition}[Actions]
Given the set of local states $\LStates$ (\defin\ref{def:local_state}), the set of \emph{actions}, $\action \in \Actions$, is defined as follows:
%
\[
	\Actions \eqdef 
	\myset{
		((\fph, \ca), (\fph', \ca'))
	}{
		((\fph, \ca), (\fph', \ca')) \in \LStates \times \LStates \\
		\land\ \dom(\fph) = \dom(\fph') \\
		\land\ \for{\loc} \fph(\loc) = (\val, \fp) \Rightarrow \\
			\quad 	\big(\fph'(\loc) = (-, \fp') \land \fp' = \addWFP{\fp} \big)
			\lor
			\big(\fph'(\loc) = (\val, \fp') \land \fp' = \addRFP{\fp} \big)
	}
\] 
%
Given the set of primitive capabilities $\Kaps$ (\defin\ref{def:capabilities}), the set of \emph{interference environments} is $\inter \in \Interference \eqdef \Kaps \parfun \powerset{\Actions}$.
\end{definition}

\begin{definition}[Logical states]\label{def:logical_states}
Given the partial commutative monoid of time timestamp heaps $(\TSHeaps, \composeTS, \unitTS)$ in \defin\ref{def:timestamp_heaps}), and the partial commutative monoid of capabilities ($\Caps$, \composeC, \unitC) in \defin\ref{def:capabilities}, the set of \emph{logical states} is: $\lgs \in \LGStates \eqdef \TSHeaps \times \Caps$. 
The \emph{logical state composition function}, $\composeLGS: \LGStates \times \LGStates \parfun \LGStates$, is defined component-wise as: $\composeLGS \eqdef (\composeTS, \composeC)$.
The \emph{logical state unit element} is $\unitLGS \eqdef (\unitTS, \unitC)$.
The \emph{partial commutative monoid of logical states} is $(\LGStates, \composeLGS, \{\unitLGS\})$.
%
\end{definition}
\begin{definition}[Worlds]
Given the set of region identifiers $\Regionid$ (\defin\ref{def:capabilities}) and the partial commutative monoid of logical states $(\LGStates, \composeLGS, \{\unitLGS\})$ in \defin\ref{def:logical_states}, the set of \emph{shared states} is $\SStates \eqdef \Regionid \parfinfun \LGStates$.
The \emph{shared state composition function}, $\composeS: \SStates \times \SStates \parfun \SStates$, is defined as: $\composeS \eqdef \composeEq$, where for all domains $\sort M$ and all $m, m' \in \sort M$: 
%
\[
	m \composeEq m' \eqdef 
	\begin{cases}
		m & \text{if } m = m'\\
		\text{undefined} & \text{otherwise}
	\end{cases}
\]
%
Given the set of timestamps $\Timestamp$ (\defin\ref{def:timestamp_heaps}), the set of \emph{worlds} is $\world \in \World \eqdef (\LGStates \times \SStates \times \Timestamp)$. 
The \emph{world composition function}, $\composeW: \World \times \World \parfun \World$, is defined component-wise as: $\World \eqdef (\composeLGS, \composeS, \composeEq)$.
The \emph{world unit set} is $\unitW \eqdef \myset{(\unitLGS, \gs, \ts)}{\gs \in \SStates \land \ts \in \Timestamp}$.
The \emph{partial commutative monoid of worlds} is $(\World, \composeW, \unitW)$.
\end{definition}
%
%
\begin{definition}[Assertions]
The set of \emph{assertions}, $\gpre, \gpost \in \Globalassertion$, are defined by the following inductive grammar:
\[
\begin{array}{r @{\hspace{2pt}} l}
	\Globalassertion \ni \gpre , \gpost \in \Globalassertion \defeq & 
	\False \mid \True \mid \gpre \land \gpost \mid \gpre \lor \gpost  \mid \exsts{\V x}\gpre \\
    & \mid \emp \mid \lexpr_1 \pointsto \lexpr_2 \mid \cass{\kap}{\V r} \mid \boxass{\gpre}{\V r}{\intass} \mid \gpre \sep \gpost \\
\end{array}
\]
%
where $\V x, \V r \in \LVars$, $\lexpr_1, \lexpr_2 \in \LExpressions$ (\defin\ref{def:local_assertions}), $\kap \in \Kaps$ (\defin\ref{def:capabilities}) and $\intass \in \IAssertions$ denotes an \emph{interference assertion} defined by the following grammar:
%
\[
	\IAssertions \ni \intass \eqdef 
	\emptyset \mid \Set{ \captk{\kap} : \exsts{\vec{\V x}} \lpre \transfersto \lpost } \cup \intass 
\]
%
Given a logical environment $\lenv \in \LEnv$ and a stack $\stk \in \Stack$, the \emph{interference interpretation} function, $\ieval[(., .)]{.}: \IAssertions \times \LEnv \times \Stack \rightarrow \Interference$, is defined as follows, for all $\kap \in \Kaps$:
%
\[
\begin{array}{r @{\hspace{2pt}} l}
	\ieval{\emptyset}(\kap) \eqdef & \emptyset \\
	\ieval{\Set{ \captk{\kap} : \exsts{\vec{\V x}} \lpre \transfersto \lpost } \cup \intass }(\kap) \eqdef &
	\myset{
		(\ls_\lpre, \ls_\lpost)	
	}{
		(\ls_\lpre, \ls_\lpost)	\in \Actions 
		\land \exsts{\rid, \vec v, \lenv'} 
			\lenv(\V r) = \rid 
			\land \lenv' = \lenv \remapsto{\vec{\V x}}{\vec v} \\
			\land\ \ls_\lpre \in \aeval{\lpre}{\lenv', \stk}
			\land \ls_\lpost \in \aeval{\lpost}{\lenv', \stk}
	}
	\cup 
	\ieval{I}(\kap)
\end{array}
\] 
%
Given a logical environment $\lenv \in \LEnv$ and a stack $\stk \in \Stack$, the \emph{assertion interpretation} function, $\aeval[(., .)]{.}: \Globalassertion \times \LEnv \times \Stack \rightarrow \World$, is defined as follows:
%
\newcommand{\auxeval}[2][\lenv, \stk, \ts]{\denote{#2}^{\mathsf{A}}_{#1}}
\[
\begin{rclarray}
	\aeval{\False}{\lenv, \stk} & \defeq & \emptyset \\
%
	\aeval{\True}{\lenv, \stk} & \defeq & \World \\
%
	\aeval{\emp}{\lenv, \stk} & \defeq & \unitW \\
%
	\aeval{\gpre \land \gpost}{\lenv, \stk} & \defeq & \aeval{\gpre}{\lenv, \stk} \cap \aeval{\gpost}{\lenv, \stk} \\
%	
	\aeval{\gpre \lor \gpost}{\lenv, \stk} & \defeq & \aeval{\gpre}{\lenv, \stk} \cup \aeval{\gpost}{\lenv, \stk} \\
%  
	\aeval{\exists \lvar \ldotp \gpre}{\lenv, \stk} & \defeq 
	& \bigcup\limits_{\val \in \Val} \aeval{\gpre}{\lenv\remapsto{\lvar}{\val}, \stk} \\
%
	\aeval{\lexpr_1 \pt \lexpr_2}{\lenv, \stk} & \defeq 
	& \myset{
		((\tshp, \unitC), \gs, \ts)
	}{
		\exsts{\loc, \val, \ts'} 
		\leval{\lexpr_1}{\lenv, \stk} {=} \loc 
		\land \leval{\lexpr_2}{\lenv, \stk} {=} \val 
		\land \ts' \leq \ts 
		\land \dom(\tshp) {=} \Set{\loc} \\
		\land\ \tshp(\loc)(\ts') = (\val,\dontcare, \dontcare) 
		\land \for{\ts'' \in ( \ts', \ts)} \tshp(\loc)(\ts'') \undef \\
		\land\ \gs \in \SStates
	} \\
%	
	\aeval{\cass{\kap}{\V r}}{\lenv, \stk} & \defeq 
	& \myset{
		((\unitTS, \ca), \gs, \ts)
	}{
		\exsts{\rid} 
		\lenv(\V r) = \rid 
		\land \dom(\ca) = \Set{\rid}
		\land \ca(\rid) = \kap
	} \\
%	
	\aeval{\boxass{\gpre}{\rid}{\intass}}{\lenv, \stk} & \defeq 
	& \myset{
		(\unitLGS, \gs, \ts) 
	}{	
		\exsts{\rid, \lgs} 
		\lenv(\V r) = \rid 
		\land \gs(\rid) = (\lgs, \ieval{I})		
		\land (\lgs, \gs) \in \auxeval{P} 
	}\\
%
	\aeval{ \gpre \sep \gpost }{\lenv, \stk} & \defeq & 
	\myset{
		(\world_1 \composeW \world_2)
	}{
		\world_1 \in \aeval{\gpre}{\lenv, \stk, \ts} 
		\land \world_2 \in \aeval{\gpost}{\lenv, \stk, \ts}
	}   
\end{rclarray}
\]
%
with the \emph{auxiliary interpretation function}, $\auxeval[(., ., .)]{(.)}: \Globalassertion \times \LEnv \times \Stack \times \Timestamp \rightarrow \powerset{\LGStates \times \SStates}$, defined as follows:
%
\[
\begin{rclarray}
	\auxeval{\False} & \defeq & \emptyset \\
%
	\auxeval{\True} & \defeq & \LGStates \times \SStates \\
%
	\auxeval{\emp} & \defeq & \myset{(\unitLGS, \gs}{\gs \in \SStates} \\
%
	\auxeval{\gpre \land \gpost} & \defeq & \auxeval{\gpre} \cap \auxeval{\gpost} \\
%	
	\auxeval{\gpre \lor \gpost} & \defeq & \auxeval{\gpre} \cup \auxeval{\gpost} \\
%  
	\auxeval{\exists \lvar \ldotp \gpre} & \defeq 
	& \bigcup\limits_{\val \in \Val} \auxeval[\lenv\remapsto{\lvar}{\val}, \stk, \ts]{\gpre} \\
%
	\auxeval{\lexpr_1 \pt \lexpr_2} & \defeq 
	& \myset{
		((\tshp, \unitC), \gs)
	}{
		\exsts{\loc, \val, \ts'} 
		\leval{\lexpr_1}{\lenv, \stk} {=} \loc 
		\land \leval{\lexpr_2}{\lenv, \stk} {=} \val 
		\land \ts' \leq \ts 
		\land \dom(\tshp) {=} \Set{\loc} \\
		\land\ \tshp(\loc)(\ts') = (\val,\dontcare, \dontcare) 
		\land \for{\ts'' \in ( \ts', \ts)} \tshp(\loc)(\ts'') \undef \\
		\land\ \gs \in \SStates
	} \\
%	
	\auxeval{\cass{\kap}{\V r}} & \defeq 
	& \myset{
		((\unitTS, \ca), \gs)
	}{
		\exsts{\rid} 
		\lenv(\V r) = \rid 
		\land \dom(\ca) = \Set{\rid}
		\land \ca(\rid) = \kap
	} \\
%	
	\auxeval{\boxass{\gpre}{\rid}{\intass}} & \defeq 
	& \myset{
		(\lgs, \gs) 
	}{	
		(\lgs, \gs, \ts) \in \aeval{\boxass{\gpre}{\rid}{\intass}}{\lenv, \stk}
	}\\
%
	\auxeval{ \gpre \sep \gpost } & \defeq & 
	\myset{
		((\lgs_1 \composeLGS \lgs_2), (\gs_1 \composeS \gs_2)
	}{
		(\lgs_1, \gs_1) \in \auxeval{\gpre}
		\land (\lgs_2, \gs_2) \in \auxeval{\gpost}
	}   
\end{rclarray}
\]
%
\end{definition}

\subsection{Old text from Shale}
%\azalea{I would take my capabilities as elements of a PCM instead and make the logic parametric in it. Then, given a \emph{primitive capability} PCM $\textsc{PrimCap} \defeq (\Gamma, \compos_{\Gamma}, \unitelem{\Gamma})$ with $\gamma \in \Gamma$, you can introduce the capability assertion $[\gamma]^{\rid}.$ This way, you don't need to worry about parameterising your capabilities with $\vec{x}$ as the user can instantiate it just so.


\sx{ The nested box is interpreted as flatten box assertion.
Yes, \( \rid \) and \( \intass \) should change position.
}
%
The box assertion \( \boxass{\gpre}{\rid}{\intass} \) asserts part of heap that can be shared, where \( \rid \) is region identifier and \( \intass \) is the interference, i.e.\ all the possible transitions.
The interference is a set of transitions that are labelled by capabilities.

\[
    \begin{rclarray}
            \gpre , \gpost \in \Globalassertion & \defeq & 
            \assfalse \mid \asstrue \mid \gpre \land \gpost \mid \gpre \lor \gpost  \mid \exists \lvar \ldotp \gpre \mid \\
            & & \assemp \mid \lexpr \pointsto \lexpr \mid \cass{\kap}{\V r} \mid \boxass{\gpre}{\V r}{\intass} \mid \gpre \sep \gpost \\
    \end{rclarray}
\]

The global assertions are interpreted as a tuple of a world and a time-stamp.
A world is a pair of shared state and logical state.
The logical state is a pair of time-stamp heap and capability, where capability is a partial finite function from region identifier to a token.
The shared state is a partial finite function from region identifier to logical state and its interference.

\azalea{Again, for the sake of consistency with CAP and CoLoSL, I'd swap the order of logical and shared states in Worlds and have $\World \defeq \Logicalstate \times \Sharestate$ instead.\\

Also I would define worlds as TRIPLES with logical state (capturing privately-owned resources), shared state AND time stamp.  Worlds are what global assertions are interpreted over so add the time stamp here and not later. If you do this here, later when  you give the semantics of global assertions it is just on worlds and not worlds $\times$ time stamps.\\

If you choose to define capabilities as I described above, make sure you propagate the notation changes here too. }

\azalea{You are missing some well-formedness conditions on worlds. This is very important. For instance, you need the local (logical) state to be compatible with the global state. For instance you don't want to have the same resource in both places. Look at CAP or CoLoSL.}

\sx{ I am not happy with the rest formalisation yet. 
So purposely ignore the well-formedness for now.
}

\[
    \begin{rclarray}
        \lstate \in \Logicalstate & \defeq & \Timestampheap \times \powerset{\Capability} \\
        \sstate \in \Sharestate & \defeq & \Regionid \parfinfun \Logicalstate \times \Interference \\
        \world \in \World & \defeq & \Logicalstate \times \Sharestate
    \end{rclarray}
\]


The composition of logical states.

\[
    \begin{rclarray}
        \lstate_{1} \compos_{\lstate} \lstate_{2} & \defeq & (\lstate_{1}\projection{1} \ \uplus \ \lstate_{2}\projection{1}, \lstate_{1}\projection{2} \ \compos_{\cstate} \ \lstate_{2}\projection{2}) \\
        \unitelem{\lstate} & \defeq & ( \emptyset, \emptyset ) \\
    \end{rclarray}
\]

The composition of capabilities and worlds.

\[
    \begin{rclarray}
        \world_{1} \compos_{\world} \world_{2} & \defeq & 
        \begin{funcarray}
            (\sstate, \lstate_{1} \compos_{\lstate} \lstate_{2}) & \world_{1} = (\sstate, \lstate_{1}) \land \world_{2} = (\sstate, \lstate_{2}) \\
            \texttt{undefined} & o.w. \\
        \end{funcarray} \\
        \unitelem{\world} & \defeq & ( \emptyset, \unitelem{\lstate} ) \\
    \end{rclarray}
\]

\azalea{The world unit must be a SET with arbitrary state as its shared state component and NOT the empty function: $\unitelem{\world} \defeq \{(\sstate, \unitelem{\lstate}) \mid \sstate \in \Sharestate\}$.}

The global assertions are interpreted as a set of worlds and their corresponding times.

\[
    \begin{rclarray}
        \eval{\assfalse}_{\lenv, \stk} & \defeq & \emptyset \\
        \eval{\asstrue}_{\lenv, \stk} & \defeq & \powerset{\World \times \Timestamp } \\
        \eval{\assemp}_{\lenv, \stk} & \defeq & \Set{(\emptyset, \ts)} \\
        \eval{\expr_{1} \pt \expr_{2}}_{\lenv, \stk} & \defeq & \Set{ ( ( \emptyset, ( \tshp, \emptyset ) ), \ts ) \middle| 
            \begin{array}{@{}l@{}}
                \exists \loc = \eval{\expr_{1}}_{\lenv, \stk}, \val = \eval{\expr_{2}}_{\lenv, \stk}, \ts' \ldotp \ts' \leq \ts \land 
                \dom(\tshp) = \Set{ \loc } \land \tshp(\loc)(\ts') =  \val \\
                \tshp(\loc)(\ts') = (\val,\dontcare, \dontcare) \land \forall \ts'' \in ( \ts', \ts) \ldotp \tshp(\loc)(\ts'')\undef \\
            \end{array}
        } \\
        \eval{\boxass{\gpre}{\intass}{\rid}}_{\lenv, \stk} & \defeq & \Set{ ( ( \sstate, \lstate ), \ts ) \middle| \exists \sstate', \lstate' \ldotp ( ( \sstate', \lstate \compos_{\lstate} \lstate' ), \ts ) \in \eval{\gpre}_{\lenv, \stk, \ts} \land \sstate = \sstate' \uplus \Set{\rid \mapsto ( \lstate', \eval{\intass}_{\lenv,\stk} ) } } \\
        \eval{ \capass }_{\lenv, \stk} & \defeq & \Set{ ( ( \sstate, ( \tshp, \cstate ) ), \ts ) \middle| \sstate = \emptyset \land \tshp = \emptyset \land \cstate = \eval{ \capass }_{\lenv, \stk} } \\
        \eval{ \gpre \sep \gpost }_{\lenv, \stk} & \defeq & \Set{ ( \world_{1} \compos_{\world} \world_{2}, \ts ) \middle| ( \world_{1}, \ts ) \in \eval{\gpre}_{\lenv, \stk, \ts} \land ( \world_{2}, \ts ) \in \eval{\gpost}_{\lenv, \stk, \ts} } \\
        \eval{\gpre \land \gpost}_{\lenv, \stk} & \defeq & \eval{\gpre}_{\lenv, \stk} \cap \eval{\gpost}_{\lenv, \stk} \\
        \eval{\gpre \lor \gpost}_{\lenv, \stk} & \defeq & \eval{\gpre}_{\lenv, \stk} \cup \eval{\gpost}_{\lenv, \stk} \\
        \eval{\exists \lvar \ldotp \gpre}_{\lenv, \stk} & \defeq & \eval{\gpre}_{\lenv\remapsto{\lvar}{\val}, \stk} \\
    \end{rclarray}
\]

\azalea{
	If you redefine worlds as described above, make sure you propagate the changes here. For instance, you will instead have: $\eval{\asstrue}_{\lenv, \stk} \defeq \World$.\\
	
	There are a few issues:
\[
\begin{array}{r c l}
	\eval{\assemp}_{\lenv, \stk} 
	& \defeq 
	&\Set{(\world, \ts) \mid \world \in \unitelem{\world}} \\
%
	\eval{\exists \lvar \ldotp \gpre}_{\lenv, \stk} & \defeq & \bigcup\limits_{\val \in \Val}\eval{\gpre}_{\lenv\remapsto{\lvar}{\val}, \stk} 
%	
\end{array}        
\]	
%
The boxed assertion must have an EMPTY local component. Also you must allow the region identifier to be a logical variable (this is useful when you existentially quantify them upon creation) and look it up in the logical environment.\\

%I am not sure about the interpretation of $\expr_{1} \pt \expr_{2}$. 
%Given that my time stamp is $t$, you are asking for the value of $\expr_{1}$ to be looked up for the latest time stamp $t'$  such that for all other time stamps between $t$ and $t'$ the value of $\expr_{1}$ is undefined. Now 
%Suppose that I have the world $(\sstate, \lstate)$ with $\lstate {=} (\tshp, \emptyset)$ where $\tshp(x)(t_1) {=} (1, \wfp, \alpha)$; $\tshp(x)(t_2) {=} (1, \rfp, \beta)$; $\tshp(y)(t_3) {=} (2, \wfp, beta)$ such that $t_1 < t_2 < t_3$ and $\tshp$ is undefined for all other locations and time stamps. That is, $\tshp$ is a time stamp heap where transaction $\alpha$ has written $1$ location $x$ at time $t_1$ and then transaction $\beta$ had read 1 from $x$ at time $t_2$ and written $2$ to location $y$ at time $t_3$.
%Intuitively, the $\tshp$ heap must satisfy the $x \mapsto 1 * y \mapsto 2$ assertion. According to the interpretation of $*$, I must find $\tshp_1, \tshp_2$ and $t$ such that $(\sstate, (\tshp_1, \emptyset), t)$ satisfies $x \mapsto 1$ and $(\sstate, (\tshp_2, \emptyset), t)$ satisfies $y \mapsto 2$. The choice of $\tshp_1, \tshp_2$  is simple. I just pick $\tshp_1(x)(t_1) {=} 1$ and undefined for all other locations. Similarly, I pick $\tshp_2(y)(t_2) {=} 2$ and undefined for all other locations. Now, what should I pick for $t$?
%If I pick $t {=} t_1$ then $(\sstate, (\tshp_2, \emptyset), t)$ does NOT satisfy $y \mapsto 2$. On the other hand, if I pick $t {=} t_2$
% where $t_1 < t_2$
}
We define \emph{merge} with respect to two heaps associated with their read sets and write sets.
The operation \( \wmrg \) eagerly merge the left-hand side to the right-hand side.
This means that, first, the domain of the result is the same as the domain of left-hand side.
Second, the result takes left-hand side but propagate all the write effects from right-hand side.

\azalea{I don't quite understand this definition. First, what do you mean by the domain of $\fphp$? $\fphp$ is a triple and not a function.

Second, What is the intuition behind $\wmrg$? When is it used? I think it is used to calculate the postcondition of actions?
}
\[
    \begin{rclarray}
        \dom(\fphp) & \defeq & \dom(\fphp\projection{1}) \\
        \fphp \wmrg \fphp' & \defeq & 
        \left( \begin{array}{@{}l@{}}
        \lambda \loc \ldotp  
        \begin{funcarray}
            \hp'(\loc) & \loc \in \ws' \cap \dom(\hp) \\
            \hp(\loc) & \loc \notin \ws' \land \loc \in \dom(\hp) \\
            \texttt{undefined} & o.w. \\
        \end{funcarray},\ 
        \rs \cup ( \dom(\hp) \cap \rs' ),\ 
        \ws \cup ( \dom(\hp) \cap \ws')
        \end{array} \right)\\
        & & \texttt{where} \ \fphp = (\hp, \rs, \ws) \land \fphp' = (\hp', \rs', \ws') \\
    \end{rclarray}
\]

The first element, also pre-condition \( \lpreext \), only contains empty label, \( \dontcare \pt_{\emptyset} \dontcare \), because the semantics requires the pre-condition interpreted as a heap with empty write set and empty read set.
The second element, also called post-condition \( \lpostext \), should contain the same resources as pre-condition.
If a location has only been read, the value should remain the same pre-condition.

%\azalea{What happens when I write bad actions where I pretend that I have the value of $x$ but I have actually changed its value:
%%
%\[
%	x \pt_{\emptyset} 1 \transfersto x \pt_{\Set{\rfp}} 2
%\]
%%
%These bad actions must be undefined and have empty interpretation. However, I believe at the moment their badness is ignored and they are treated as ID actions.
%}

\sx{For resource moving in can be tagged with write but don't know how to tag resource move out. This is not very right.}


The \rl{PRCommit} rule looks the same but the repartition is much more complicated by  merging all possible transitions from the environment that are allowed to run concurrently and commit in the same time.
Note that after the merging, one still need to check stabilisation.

\[
    \infer[\rl{PRCommit}]{%
        \judgement{}{\gpre}{ \ptrans{\cmd} }{\gpost}
    }{%
        \begin{array}{l}
            \judgement{}{\lpre}{\cmd}{\lpost} \quad 
            \vdash \gpre \Rrightarrow^{\{\lpre\}\{\lpost\}} \gpost
        \end{array}
    }
\]

Given the \emph{merge}for \( \fphp \), we can define \emph{merge} on transitions.
The predicate \( \predn{noFingerPrint} \) asserts the read set and write set are empty sets, and \( \predn{noWriteConflict} \) asserts the write sets are disjointed.
The predicate \( \predn{agree} \) has similar meaning as overlapped separation found in CoLoSL, which means that the state of the overlapped parts must agree.
To merge two transactions, if the pre-conditions agree and both of them have no fingerprint, and if the post-conditions agree and they write to different locations, the merged result is a set of transitions including the left-hand side and another transition by merging the post-condition of right-hand side to left-hand side.
Otherwise, the merge only returns a singleton set of left-hand side.

\[
    \begin{rclarray}
        \pred{noFingerPrint}{\fphp} & \defeq & \exists \rs, \ws, \ldotp \fphp = (\dontcare, \rs, \ws) \land \rs = \ws = \emptyset \\
        \pred{noWriteConflict}{\fphp_{l}, \fphp_{r}} & \defeq & \exists \ws_{l} = \fphp_{l}\projection{3}, \ws_{r} = \fphp_{r}\projection{3} \ldotp \ws_{l} \cap \ws_{r} = \emptyset \\
        \pred{agreeState}{\fphp_{l}, \fphp_{r}} & \defeq & \exists \fphp'_{l}, \fphp_{m}, \fphp'_{r}, \fphp \ldotp \fphp_{l} = \fphp'_{l} \compos_{\fphp} \fphp_{m} \land \fphp_{r} = \fphp'_{r} \compos_{\fphp} \fphp_{m} \land \fphp = \fphp'_{l} \compos_{\fphp} \fphp_{m} \compos_{\fphp} \fphp'_{r} \\
        \pred{agreeCapability}{\cstate_{l}, \cstate_{r}} & \defeq & \exists \cstate'_{l}, \cstate_{m}, \cstate'_{r}, \fphp \ldotp \cstate_{l} = \cstate'_{l} \compos_{\cstate} \cstate_{m} \land \cstate_{r} = \cstate'_{r} \compos_{\cstate} \cstate_{m} \land \cstate = \cstate'_{l} \compos_{\cstate} \cstate_{m} \compos_{\cstate} \cstate'_{r} \\
        \pred{agree}{\fphpcap_{l},\fphpcap_{r}} & \defeq & \pred{agreeState}{\fphpcap_{l}\projection{1}, \fphpcap_{r}\projection{1}} \land \pred{agreeCapability}{\fphpcap_{l}\projection{2}, \fphpcap_{r}\projection{2}} \\
        ( \fphpcap_{\lpre}, \fphpcap_{\lpost} ) \bmrg ( \fphpcap'_{\lpre}, \fphpcap'_{\lpost} ) & \defeq & \Set{( \fphpcap_{\lpre}, \fphpcap_{\lpost} )} \cup \Set{(\fphpcap_{\lpre}, ( \fphp_{\lpost} \wmrg \fphp'_{\lpost}, \cstate_{q} \compos_{\cstate} \cstate'_{q} ) ) \middle| 
        \begin{array}{@{}l@{}}
            \pred{noWriteConflict}{\fphp'_{\lpre}, \fphp'_{\lpost} } \land {} \\
            \pred{agree}{\fphpcap_{\lpre}, \fphpcap_{\lpost}} \land \pred{agree}{\fphpcap'_{\lpre}, \fphpcap'_{\lpost}} 
        \end{array}
    } \\
    \end{rclarray}
\]

The repartition is redefined by adding a merging process.
The notation \( \clps{\world} \) collapses a world to a time-stamp heap and capabilities by compositing the private logical state \( \lstate' \) and all the shared logical states \( \sstate(\rid_{i}) \).
To simplify we reuse the same notation for \( \clps{( \world, \ts) } \), it further collapses the time-stamp heap to a plain heap by taking a snapshot at the time \( \ts \).
The lift of a plain heap is a set of all possible worlds with their times that collapse to the plain heap with any possible capabilities, and similarly the lift of a plain heap with its read and write sets is the lift of the plain heap but lose the information about read and write sets.
Given that \( \fphpcap \) talks about interference, either pre- or post-condition, of a certain region, the lift of \( \fphpcap \) is all possible worlds with their times where the lift of the plain heap, i.e.\ the first projection of \( \fphpcap \) is subset of the time-stamp heap corresponding to the region \( \rid \) and the capabilities of \( \fphpcap \), the second projection, is the subset of the capabilities of region \( \rid \).

The repartition means, for all possible world and time, (\world, \ts), that satisfies global assertion \( \gpre \), there exist a heap with its read set and write set, \( \fphp \), that satisfies local assertion \( \lpre \), \( \fphp' \) for assertion \( \lpost \) and a transition \( (\fphpcap, \fphpcap' ) \) that is allowed by the guarantees.
The \( \fphp \) should agree with \( \fphpcap \), which means \( \fphp = \fphpcap\projection{1} \) and similarly \( \fphp' = \fphpcap'\projection{1} \).
Also, the \( (\world, \ts) \) corresponding to \( \gpre \) should agree with the pre-condition of the transition, i.e.\ \( (\world, \ts) \in \lift{\fphpcap} \).
Then for all possible merging between transition \( (\fphpcap, \fphpcap' ) \) and relies, there must exist a corresponding world and its time satisfies the global assertion \( \gpost \).
It means that the world and its time \( (\world', \ts') \) is in the lift of post-condition of the merging result \( \fphpcap'' \).
At last the worlds \( \world \) and \( \world' \) should be balanced, meaning no creation or destruction of resources.

%Given the \( \fphp \), if there exists a \( \fphp' \) that associated with \( \lpost \) and the transition \( (\fphp, \fphp') \)  is allowed by the guarantees \( \grte(\world) \), then for all the possible post-state \( \fphp'' \) by merging the transition \( (\fphp, \fphp') \) with relies, there must exist a world and its time \( (\world', \ts') \) that satisfies global post-condition \( \gpost \) and their collation equals to the first projection of \( \fphp'' \) which is the plain heap.

\[
    \begin{rclarray}
        \clps{\world} & \defeq & (\tshp, \cstate) \ \texttt{where} \ \exists \sstate, \lstate, \lstate', \rid_{0}, \dots, \rid_{n} \ldotp \\
                      & & \qquad \qquad \world = (\sstate, \lstate') \land \Set{\rid_{0}, \dots, \rid_{n}} = \dom(\sstate) \land \lstate = \lstate' \compos_{\lstate} \sstate(\rid_{1}) \compos_{\lstate} \dots \compos_{\lstate} \sstate(\rid_{n})\\
                             & & \qquad \qquad \tshp = \lstate\projection{1} \land \cstate = \lstate\projection{2} \\
        \clps{(\world, \ts)} & \defeq & ( \hp,\cstate ) \ \texttt{where} \ \hp = \func{startstate}{\clps{\world}\projection{1}, \ts} \land \cstate = \clps{\world}\projection{2} \\
        \clps{\fphpcap} & \defeq & ( \hp,\cstate ) \ \texttt{where} \ \hp = \fphp\projection{1}\projection{1} \land \cstate = \fphpcap\projection{2} \\
        %\lift{\hp} & \defeq & \Set{(\world, \ts) \middle| ( \hp , \cstate ) = \clps{(\world, \ts)}}\\
        %\lift{\fphp} & \defeq & \lift{\fphp\projection{1}} \\
        %\lift{\fphpcap} & \defeq &  \Set{(\world, \ts) \middle| \exists \rid, \tshp, \cstate \ldotp \world\projection{1}(\rid) = ((\tshp, \cstate), \dontcare) \land \lift{\fphpcap\projection{1}} \subseteq \tshp \land \fphpcap\projection{2} \subseteq \cstate}\\
        \pred{balance}{\world_{1}, \world_{2}} & \defeq & \exists \tshp_{1}, \cstate_{1}, \tshp_{2}, \cstate_{2} \ldotp (\tshp_{1}, \cstate_{1}) = \clps{\world_{1} } \land  (\tshp_{2}, \cstate_{2}) = \clps{\world_{2} } \land \dom(\tshp_{1}) = \dom(\tshp_{2} ) \land \cstate_{1} = \cstate_{2} \\
        \repartition{\gpre}{\gpost}{\lpre}{\lpost} & \iff & \forall \lenv, \stk, \world, \ts  \ldotp (\world, \ts) \in \eval{\gpre}_{\lenv, \stk} \ldotp \\
                                                   & & \implies \exists \lenv', \stk', \fphp \in \eval{\lpre}_{\lenv, \stk}, \hpfp' \in \eval{\lpost}_{\lenv', \stk'}, \fphpcap, \fphpcap', \fphpcap_{r} \ldotp \\
                                                   & & \qquad (\fphpcap, \fphpcap' ) \in \grte(\world) \land \fphp = \fphpcap\projection{1} \land \fphp' = \fphpcap'\projection{1} \land \clps{(\world, \ts)} = \clps{\fphpcap \compos_{\fphpcap} \fphpcap_{r}}\\ 
                                                   & & \implies \forall \fphpcap'' \ \ldotp ( \dontcare, \fphpcap'') \in ( (\fphpcap, \fphpcap') \bmrg \rely(\world) ) \\
                                                   & & \exists \world',\ts' \ldotp  \ts < ts' \land (\world',\ts') \in \eval{\gpost}_{\lenv', \stk'} \land \clps{(\world', \ts' )} = \clps{\fphpcap''\compos_{\fphpcap} \fphpcap_{r}} \land \pred{balance}{\world, \world'}
    \end{rclarray}
\]

The relies and guarantees.

\[
    \begin{rclarray}
        \grte & \defeq & \lambda \world \ldotp \Set{(\fphpcap, \fphpcap') \middle| \forall \capb \in \world\projection{2}\projection{2} \land (\fphpcap, \fphpcap') \in \world\projection{1}(\capb\projection{1})\projection{2}(\capb)}\\
        \rely & \defeq & \lambda \world \ldotp \Set{(\fphpcap, \fphpcap') \middle| \forall \capb \ldotp \exists \capb' \in \world\projection{2}\projection{2} \ldotp \capb\projection{1} = \capb'\projection{1} \land ((\capb\projection{2}, \capb\projection{3}) \compos_{\setcap} (\capb'\projection{2}, \capb'\projection{3}))\isdef \land (\fphpcap, \fphpcap') \in \world\projection{1}(\capb\projection{1})\projection{2}(\capb')}
    \end{rclarray}
\]

\section{temp}

To merge two actions \( \lpre \transfersto \lpost \) and \( \lpre' \transfersto \lpost' \), it requires the pre-conditions agrees, which means that if they describe some common heaps, the state of the common part should be consistent.
Then if the post-conditions write to different locations, the \emph{merge} (to the left) operator \( \wmrg \) returns a new action where the effect of right hands side propagate to the left hand side.
\sx{Not sure how to do it properly considering there is \(\vec{x}\) binder and extension \(\vec{y}\) }.

\[
    \begin{rclarray}
        \func{writeSet}{\lvar \pointsto_{\fp \cup \Set{\wfp}} \dontcare } & \defeq & \Set{\lvar} \\
        \func{writeSet}{\lvar \pointsto_{\fp \setminus \Set{\wfp}} \dontcare } & \defeq & \emptyset \\
        \func{writeSet}{ \lpre \sep \lpost } & \defeq & \func{writeSet}{\lpre} \uplus \func{writeSet}{\lpost} \\
        \lpre \bmrg \lpost & \defeq & 
        \begin{funcarray}
            ( \lpre' \bmrg \lpost' ) \sep \lvar \pointsto_{\fp \cup \Set{\wfp}} \dontcare  &  ( \lpre = \lpre' \sep \lvar \pointsto_{\dontcare} \dontcare ) \land  ( \lpost = \lpost' \sep \lvar \pointsto_{\fp \cup \Set{\wfp}} \dontcare ) \\
            \lpre & o.w. \\
        \end{funcarray} \\
        \pred{agree}{\lpre , \lpost} & \defeq & \exists \lpre', \lpost', \lframe, m \ldotp ( \lpre = \lpre \sep \lframe ) \land ( \lpost = \lpost' \sep \lframe)  \land ( m = \lpre' \sep \lpost' \sep \lframe ) \\
        ( \lpre \transfersto \lpost ) \wmrg ( \lpre' \transfersto \lpost' ) & \defeq & \Set{ \lpre \transfersto \lpost } \cup \Set{\lpre \transfersto (\lpost \bmrg \lpost') \middle| \pred{agree}{\lpre , \lpre'} \land \func{writeSet}{\lpost} \cap \func{writeSet}{\lpost'} = \emptyset}
    \end{rclarray}
\]


\subsection{Rules for Local}

The proof rules are standard except \rl{TRDeref} and \rl{TRMutate}.
The \rl{TRDeref} rule add read fingerprint in finger-tracking set, only if there is no write finger-print.
This is because once a location has been re-written, the rest read are considered as local operations, while the finger-print only records those operations might have effect on global state.

\begin{figure}[!t]
\hrule\vspace{5pt}
\[
    \infer[\rl{TRSkip}]{%
        \tripleL{\assemp }{ \pskip }{\assemp }
    }{}
\]

\[
    \infer[\rl{TRAss}]{%
        \tripleL{\var \dot= \lexpr }{ \pass{\var}{\expr} }{\var \dot= \expr\sub{\var}{\lexpr} }
    }{}
\]

\[
    \infer[\rl{TRDeref}]{%
        \tripleL{\expr \pt \lexpr \sep \fp }{ \pderef{\var}{\expr} }{\var \dot= \lexpr \sep \expr \pt \lexpr \sep \fp' }
    }{%
        \var \notin \func{fv}{\expr}
        && \var \notin \func{fv}{\lexpr}  
        && \fp \toFP{\otR(\lexpr)} \fp'
    }
\]

\[
    \infer[\rl{TRMutate}]{%
        \tripleL{\expr_1 \pt \stub \sep \fp }{ \pmutate{\expr_1}{\expr_2} }{ \expr_{1} \pt \expr_{2} \sep \fp' } 
    }{
        \fp \toFP{\otW(\expr_{2})} \fp'
    }
\]

\[
    \infer[\rl{TRAssume}]{%
        \tripleL{ \expr \doteq 0 }{ \passume{\expr} }{ \expr \doteq 0 } 
    }{}
\]

\[
    \infer[\rl{TRChoice}]{%
        \tripleL{ \lpre }{ \trans_{1} \pchoice \trans_{2} }{ \lpost }
    }{%
        \tripleL{ \lpre }{ \trans_{1} }{ \lpost } && 
        \tripleL{ \lpre }{ \trans_{2} }{ \lpost } 
    }
\]

\[
    \infer[\rl{TRSeq}]{%
        \tripleL{ \lpre }{ \trans_{1} \pseq \trans_{2} }{ \lpost }
    }{%
        \tripleL{ \lpre }{ \trans_{1} }{ \lframe }  && 
        \tripleL{ \lframe }{ \trans_{2} }{ \lpost }
    }
\]


\[
    \infer[\rl{TRLoop}]{%
        \tripleL{ \lpre }{ \trans\prepeat }{ \lpre }
    }{%
        \tripleL{ \lpre }{ \trans }{ \lpre } 
    }
\]
 
\sx{Side condition for fingerprint for sure, but need to see how to do it in a better way.
The potential problem here is the mutation rule, as if the precondition is read and write, one can frame off the write, produce a new write, and frame back the write. 
One way is when frame back, need to update the write fingerprint from the post condition.
Or we say fingerprint for a single address cannot be split and then we add 2 more primitive fingerprint transitions.
}

\[
   \infer[\rl{TRFrame}]{%
       \tripleL{ \lpre \sep \lframe }{ \trans }{ \lpost \sep \lframe }
   }{%
       \tripleL{ \lpre }{ \trans }{ \lpost } 
   }
\]
\hrule\vspace{5pt}
\[
\begin{rclarray}
    \expr \fpI & \toFP{\otR(\lexpr)} & \expr \fpR \lexpr \\
    \expr \fpR \expr' & \toFP{\otR(\lexpr)} & \expr \fpR \lexpr' \\
    \expr \fpW \expr' & \toFP{\otR(\lexpr)} & \expr \fpW \lexpr' \\
    \expr \fpI & \toFP{\otW(\lexpr)} & \expr \fpW \lexpr \\
    \expr \fpR \expr' & \toFP{\otW(\lexpr)} & \expr \fpR \lexpr' \sep \expr \fpW \lexpr \\
    \expr \fpW \expr' & \toFP{\otW(\lexpr)} & \expr \fpW \lexpr \\
\end{rclarray}
\]
\hrule\vspace{5pt}
\caption{The rules for transactions}
\label{fig:rule-trans}
 \end{figure}

\subsection{Rely and Guarantee}

\sx{To allow a transaction update multiple regions but a region multiple times. Make sure the definition is confluent.}

\begin{defn}[Rely and guarantee]
\label{def:rely-guarantee}
The \( \predn{updWorlds} \) predicate asserts that the world transfers from \( \w \) to \( \w' \) which is allowed by the capabilities \(\ca\).
\[
\begin{rclarray}
    \pred{updWorlds}{\ca, \w, \w'} & \defeq & \w = \w' \\
    \pred{updWorlds}{\ca, (\ca',\gs \uplus \gs') ,(\ca'',\gs'' \uplus \gs''')} & \defeq & 
    \begin{array}[t]{@{}l}
    \exsts{\kap, \ca'''}
    \kap \sqsubseteq \ca(\rid) \\
    \quad {} \land \pred{updWorlds}{\ca, (\ca''',\gs),(\ca'',\gs'')}
    \land \pred{updW}{\kap, (\ca',\gs),(\ca'',\gs'')}
    \end{array} \\
    \pred{updW}{\kap, (\ca, \gs), (\ca' ,\gs')} & \defeq &  
    \begin{array}[t]{@{}l}
        \exsts{\rid, \hh, \hh', \vi, \vi', \ca'', \ca''', \intf } \\
        \quad \gs = \Set{\rid \mapsto (\hh, \vi, \ca'', \intf)} 
        \land \gs' = \Set{\rid \mapsto (\hh', \vi', \ca''', \intf)}  \\
        \quad {} \land (\hh, \vi, \ca'') \toLTS{\kap} (\hh', \vi', \ca''') \in \func{inv}{\rid, \intf} 
        \land \ca \composeC \ca'' = \ca' \composeC \ca'''
    \end{array}
\end{rclarray}
\]
Given the set of worlds $\World$ (\defref{def:world}), the \emph{update rely} relation, $\relyU \subseteq \World \times \World$, is defined as follows,
\[	
    \begin{rclarray}
	\Rely & \eqdef &
	\Setcon{
		((\ca,\gs), \w')	
	}{
        \exsts{\ca'}  
        (\ca' \composeC \ca)\isdef
        \land \pred{updWorlds}{ \ca', (\ca,\gs), \w'}
	} \\
    \end{rclarray}
\]
%The invariant of a shared state is a lift of the invariants of interferences of regions.
%The \emph{rely} relation, $\RelyI \eqdef \World \times \World$, is defined as follows:
%\[
    %\begin{rclarray}
         %\RelyI &\eqdef & \closure{\left(\relyU\right)} \\
    %\end{rclarray}
%\]
The \emph{update guarantee} relation, $\guarU: \World \times \World$, is defined as follows:
\[	
    \begin{rclarray}
	\Guar & \eqdef &
	\Setcon{
		((\ca,\gs), \w')	
	}{
        \pred{updWorlds}{\ca, (\ca,\gs), \w'}
	} \\
    \end{rclarray}
\]
%The \emph{guarantee} relation, $\GuarI \subseteq \World \times \World$, is defined as follows:
%\sx{take away of the closure}
%\[
	%\GuarI \eqdef \guarU
%\]
\end{defn}

\begin{defn}[Stable]
A set of worlds $\setworld \subseteq \World$ is \emph{stable}, written $\stable{\setworld}$, if and only if it is closed under the rely relation: 
\[
    \begin{rclarray}
        \stable{\setworld} & \eqdef & \for{\w, \w'}  \w \in \setworld \land (\w, \w') \in \Rely \implies \w' \in \setworld
    \end{rclarray}
\]
\end{defn}

\subsection{Rules for Global}

The \rl{PRCommit} rule lifts the local effect of transaction \( \trans \) to global level by first converting global state to (local) observable state and then propagating the local fingerprint to the global state.
The \( \predn{down} \) predicate asserts that the local predicate \( \lpre \) is a over-approximation of the valid observation that is given by the interference.
The \( \predn{up} \) predicate says the post-condition \( \gpost \) is the result by lifting the local fingerprints \( \fp \) to pre-condition \( \gpre \).



\begin{figure}[t!]
\hrule\vspace{5pt}

%\[
    %\infer[\rl{PRCommit}]{%
        %\tripleG{\gpre}{ \ptrans{\trans} }{\gpost}
    %}{%
        %\begin{array}{c}
        %\gpre \snap \lpre
        %\quad \tripleL{\lpre \sep \fpE}{\trans}{\lpost \sep \fp} \\
        %\pred{noFingerprint}{\lpre} 
        %\quad \pred{noFingerprint}{\lpost} \\
        %\rpt{\gpre}{\gpost}{\fp}
        %\quad \stable{\gpre} 
        %\quad \stable{\gpost} 
        %\end{array}
    %}
%\]

\[
    \infer[\rl{PRCommit}]{%
        \tripleG{\gpre}{ \ptrans{\trans} }{\gpost}
    }{%
        \begin{array}{c}
        \gpre \snap \lpre
        \quad \tripleL{\lpre \sep \fpE}{\trans}{\lpost \sep \fp} \\
        \pred{noFingerprint}{\lpre} 
        \quad \pred{noFingerprint}{\lpost} \\
        \como \rpt{\gpre}{\gpost}{\fp}
        \quad \stable{\gpre} 
        \quad \stable{\gpost} 
        \end{array}
    }
\]


\[
    \infer[\rl{PRAss}]{%
        \tripleG{\thvar \dot= \lexpr }{ \pass{\thvar}{\expr} }{\thvar \dot= \expr\sub{\thvar}{\lexpr} }
    }{%
        \thvar \notin \func{fv}{\lexpr} 
        && \thvar \in \ThdVars  
    }
\]

\[
    \infer[\rl{PRAssume}]{%
        \tripleG{ \expr \doteq 0 }{ \passume{\expr} }{ \expr \doteq 0 } 
    }{}
\]

\[
    \infer[\rl{PRChoice}]{%
        \tripleG{ \gpre }{ \prog_{1} \pchoice \prog_{2} }{ \gpost }
    }{%
        \tripleG{ \gpre }{ \prog_{1} }{ \gpost } && 
        \tripleG{ \gpre }{ \prog_{2} }{ \gpost } 
    }
\]

\[
    \infer[\rl{TRSeq}]{%
        \tripleG{ \gpre }{ \prog_{1} \pseq \prog_{2} }{ \gpost }
    }{%
        \tripleG{ \gpre }{ \prog_{1} }{ \gframe }  && 
        \tripleG{ \gframe }{ \prog_{2} }{ \gpost }
    }
\]

\[
    \infer[\rl{TRLoop}]{%
        \tripleG{ \gpre }{ \prog\prepeat }{ \gpre }
    }{%
        \tripleG{ \gpre }{ \prog }{ \gpre } 
    }
\]
 
\[
   \infer[\rl{TRFrame}]{%
       \tripleG{ \gpre \sep \gframe }{ \prog }{ \gpost \sep \gframe }
   }{%
       \tripleG{ \gpre }{ \prog }{ \gpost } 
   }
\]
 
\[
   \infer[\rl{TRPar}]{%
       \tripleG{ \gpre_{1} \sep \gpre_{2} }{ \prog_{1} \ppar \prog_{2} }{ \gpost_{1} \sep \gpost_{2} }
   }{%
       \tripleG{ \gpre_{1} }{ \prog_{1} }{ \gpost_{1} }
       && \tripleG{ \gpre_{2} }{ \prog_{2} }{ \gpost_{2} }
   }
\]

\sx{type mismatch for interpretation of fingerprint in  repartition.}
\[
\begin{rclarray}
    \gpre \snap \lpre & \defeq & 
    \begin{array}[t]{@{}l}
        \for{ \w, \h, \hh, \cu, \lenv, \stk }
        \w \in \evalW{\gpre} 
        \land (\hh, \cu, \stub) \in \eraseW{\w}
        \land \h = \func{clps}{\hh, \cu} 
        \implies (\h, \unitO) \in \evalLS{\lpre}\\
    \end{array} \\
    \como \rpt{\gpre}{\gpost}{\fp} & \defeq & 
    \begin{array}[t]{@{}l@{}}
        \for{\w, \hh, \vi, \ca, \lenv, \stk, \txid} 
        \exsts{\w', \hh', \vi', \ca',\extopset} \\
        \begin{B}
            \w \in \evalW{\gpre}
            \land (\hh, \vi, \ca) \in \eraseW{\w}  \\
            \quad {} \land \txid \in \func{fresh}{\hh} 
            \land \extopset \in \evalF{\fp} \\
            \quad {} \land \hh' = \func{commit}{\hh, \vi, \txid, \func{getOps}{\extopset}} \\
            \quad {} \land \cu' = \func{update}{\hh, \vi, \func{getOps}{\extopset}} \\
            \quad {} \land (\w, \w') \in \Guar  
            \land (\hh, \Set{\vi}) \toCO{\como} (\hh', \Set{\vi'})
        \end{B}
        \implies (\hh',\vi', \ca) \in \eraseW{\w'} \land \w' \in \evalW{\gpost}
    \end{array} \\
\end{rclarray}                          
\]

\hrule\vspace{5pt}
\caption{The rules for programs}
\label{fig:rule-prog}
\end{figure}

Many consistency model use last write win resolution policy, such as snapshot isolation, therefore the repartition \( \rpt{\gpre}{\gpost}{\fp} \) can be simplified by checking the guarantee and then syntactically propagating the write fingerprints.
Also in practice, many implementation of consistency model assume strong session constraint.

\begin{figure}
\hrule\vspace{5pt}

\[
   \infer[\rl{FRead}]{%
       \tripleF{ \lexpr \pt \lexpr' \mid \lexpr \pt \lexpr'' }{ \Set{(\etR, \lexpr, \lexpr'')} }{ \lexpr \pt \lexpr' \mid \lexpr \pt \lexpr''}
   }{}
\]

\[
   \infer[\rl{FWrite}]{%
       \tripleF{ \lexpr \pt \lexpr' \mid \lexpr \pt \lexpr'' }{ \Set{(\etW, \lexpr, \lexpr''')} }{ \lexpr \pt \lexpr''' \mid \lexpr \pt \lexpr'''}
   }{}
\]

\[
   \infer[\rl{FFrame}]{%
       \tripleF{ \lpre \sep \lframe  \mid \lpre' \sep \lframe' }{ \fp }{ \lpost \sep \lframe \mid \lpost' \sep \lframe' }
   }{%
       \tripleF{ \lpre \mid \lpre' }{ \fp }{ \lpost \mid \lpost' }
   }
\]

\[
   \infer[\rl{FContinue}]{%
       \tripleF{ \lpre \sep \lframe  \mid \lpre' \sep \lframe' }{ \fp  \uplus \fp' }{ \lpost \sep \lframe \mid \lpost' \sep \lframe' }
   }{%
       \tripleF{ \lpre \sep \lframe  \mid \lpre' \sep \lframe' }{ \fp }{ \lpost \sep \lframe \mid \lpost' \sep \lframe' }
   }
\]

\[
\begin{rclarray}
    \rpt{\gpre}{\gpost}{\fp} & \defeq & 
    \begin{array}[t]{@{}l}
    \for{\w, \w', \lpre, \lpre', \lpost, \lpost', \lenv, \stk} \\
    \quad 
    \begin{B}
        \w \in \evalW{\gpre}
        \land \pred{unboxleft}{\w, \lpre}
        \land \gpre \snap \lpre' 
        \land {} \tripleF{ \lpre \mid \lpre' }{ \fp }{ \lpost \mid \lpost'} \\
        {} \land (\w, \w') \in \Guar
        \land \pred{unboxleft}{\w',\lpost}
        \land \gpost \snap \lpost'
    \end{B}
    \implies \w' \in \evalW{\gpost}
    \end{array} \\
    \pred{unboxleft}{\w, \lpre} & \defeq & \for{\hh} (\hh, \stub) \in \eraseW{\w} \implies \func{clps}{\hh} \in \evalLS{\lpre}
\end{rclarray}
\]

\hrule\vspace{5pt}
\caption{Syntactic repartition for SI, SER and Causal under strong sessions}
\label{fig:rule-prog}
\end{figure}



\section{Soundness of Logic}

\subsection{Transaction Soundness}


\begin{thm}[Transaction soundness]
\label{thm:transaction-soundness}
The transaction soundness is as follows:
\[
    \begin{array}{@{}l@{}}
        \for{ \lpre, \trans, \lpost } \tripleL{\lpre}{\trans}{\lpost} \implies \ \tripleSemL{\lpre}{\trans}{\lpost} \\
    \end{array}
\]
where,
\[
    \begin{rclarray}
    \tripleSemL{\lpre}{\trans}{\lpost} & \eqdef &
    \begin{array}[t]{@{}l@{}}
        \for{\lenv, \stk, \stk', \h, \h', \opset, \opset' } 
        (\h, \opset) \in \evalLS[\lenv, \stk]{\lpre} \\
        \quad {} \land \vdash (\stk, \h, \opset ), \trans \toL^{*}  (\stk', \h', \opset' ), \pskip 
        \implies (\h', \opset') \in \evalLS[\lenv, \stk']{\lpost}
    \end{array}
    \end{rclarray}
\]
\end{thm}
\begin{proof}
Induction on the derivations.

\caseB{\rl{TRSkip}}

We have  \(\trans \equiv \pskip\), \( \lpre \equiv \lpost \equiv \assemp \), thus \( \h_{p} = \h_{q} = \unitH \), \( \opset = \opset' \) and \( \stk = \stk' \), and then \( (\unitH,\unitO ) \in \evalLS[\lenv, \stk']{\assemp} \) holds.

\caseB{\rl{TRAss}}

We have \(\trans \equiv ( \pass{\var}{\expr} ) \), \( \lpre \equiv ( \var \doteq \lexpr ) \) and \( \lpost \equiv ( \var \doteq \expr\sub{\var}{\lexpr} ) \) for some \( \expr, \lexpr \) and \( \var \) such that \( \var \notin \func{fv}{\lexpr} \land \var \in \Vars\).
Given the transaction semantics (\figref{fig:thread_semantics}), it has \( \stk' = \stk\rmto{\var}{\val} \) where \( \val = \evalLE[\lenv, \stk]{\expr\sub{\var}{\lexpr}} \).
Since \( \var \notin \func{fv}{\lexpr} \), we know \( \evalLE[\lenv, \stk]{\lexpr} = \evalLE[\lenv, \stk']{\lexpr} \), and then \( \evalLE[\lenv, \stk]{\expr\sub{\var}{\lexpr}} = \evalLE[\lenv, \stk']{\expr\sub{\var}{\lexpr}} \).
This means the assertions related to stack holds even thought the stack changes.
Also because the heap and events set remains unchanged, this is \( \h = \h' \) and \( \opset = \opset' \), we have \( (\h', \opset' ) \in \evalLS[\lenv, \stk']{\lpost} \).

\caseB{\rl{TRLookup}}

We have  \(\trans \equiv ( \plookup{\var}{\expr} ) \) and four cases for pre- and post-conditions defined by the relation \( \toFP{\otR(\expr, \lexpr)}\).
In all the four cases, the heap remains the same \( \h = \h' = \Set{\addr \mapsto \val}\) and the stack get updated to \( \stk' = \stk\rmto{\var}{\val} \), yet since \( \var \notin \func{fv}{\lexpr}\), the logic value \( \lexpr \) and new logic address \( \expr\sub{\var}{\lexpr}\) are evaluated to the same value \( \val \) and \( \addr \).
While different cases have different operations.
Also note that the evaluation of the pre- and post-conditions in all four cases are singleton sets.
Now we do case analysis on four case and focus on the operations before and after.

If \( \lpre \equiv \expr \fpI \lexpr \) and \( \lpost \equiv \expr\sub{\var}{\lexpr} \fpR \lexpr \sep \var \dot= \lexpr \), the only interpretation for pre-condition \( \lpre \) is \( (\Set{\addr \mapsto \val}, \emptyset) \) where \( \addr = \evalLE{\expr} \) and \( \val = \evalLE{\lexpr}\).
In this case, a read operation is added \( \opset' = \Set{(\otR, \addr, \val)} \) and it is reflected in the post condition  \( \expr\sub{\var}{\lexpr} \fpR \lexpr  \).

If \( \lpre \equiv \expr \fpR \lexpr \) and \( \lpost \equiv \expr\sub{\var}{\lexpr} \fpR \lexpr \sep \var \dot= \lexpr \), since there is already a read operation, the new operations, adding a new read operation to the same address does not change anything, \ie \( \opset' = \Set{(\otR, \addr, \val)} \addO (\otR, \addr, \val ) = \Set{(\otR, \addr, \val)} \).
This is exactly the post-condition.
For the similar reason that the operations remains the same, it is sound when \( \lpre \equiv \expr \fpW \lexpr \) and \( \lpost \equiv \expr\sub{\var}{\lexpr} \fpW \lexpr \sep \var \dot= \lexpr \) and when \( \lpre \equiv \expr \fpRW (\lexpr,\lexpr') \) and \( \lpost \equiv \expr\sub{\var}{\lexpr} \fpRW (\lexpr,\lexpr') \sep \var \dot= \lexpr' \).

\caseB{ \rl{TRMutate} }

We have  \( \trans \equiv (\pmutate{\expr_{1}}{\expr_{2}}) \) and four cases for pre- and post-conditions defined by the relation \( \toFP{\otW(\expr, \lexpr)}\). 
In all the four cases, the stack remains untouched and the heap is updated to \( \h' = \Set{\addr \mapsto \val' }\) where the logical address \( \evalLE[\lenv,\stk']{\expr_{1}}  = \addr \) and the new values \( \evalLE[\lenv,\stk']{\expr_{2}} = \val'\).
Now we do case analysis on four case and focus on the operations before and after.

If \( \lpre \equiv \expr_{1} \fpI \lexpr \) and \( \lpost \equiv \expr_{1} \fpW \expr_{2} \), a new write operation is added to the initial empty operation set, this is, \( \opset' = \Set{(\otW, \addr, \val)}\) where \( \addr = \evalLE[\lenv,\stk']{\expr_{1}}\) and \( \val = \evalLE[\lenv,\stk']{\expr_{1}}\).
This is exactly the post-condition \( \expr_{1} \fpW \expr_{2} \).
If \( \lpre \equiv \expr_{1} \fpW \lexpr \) and \( \lpost \equiv \expr_{1} \fpW \expr_{2} \), the operations set before execution is \( \opset = \Set{(\otW, \addr, \val)}\) where \( \addr = \evalLE{\expr_{1}}\) and \( \val = \evalLE{\lexpr}\).
Since the set only have the last write, so the set after is \( \opset' = \opset \addO (\otW, \addr, \val') = \Set{(\otW, \addr, \val')}\), where \( \val' = \evalLE[\lenv,\stk']{\expr_{2}}\).
Note that since the stack remains untouched, so we have \( \addr = \evalLE{\expr_{1}} = \evalLE[\lenv,\stk']{\expr_{1}} \).
Thus, we have the proof for this case.
For the remaining two cases, they follows the same argument as the operations set only have the last write.

\caseI{\rl{TRChoice}}

We have  \(\trans \equiv \trans_{1} + \trans_{2} \), where \( \tripleL{\lpre}{\trans_{1}}{\lpost} \) and \( \tripleL{\lpre}{\trans_{2}}{\lpost} \) hold, for some \( \trans_{1}, \trans_{2}, \lpre, \lpost \).
Given the transaction semantics (\figref{fig:thread_semantics}), it either has \( ( \stk, \h, \opset ), \trans_{1} \pchoice \trans_{2} \toL ( \stk, \h, \opset ), \trans_{1} \) or  \( ( \stk, \h, \opset ), \trans_{1} \pchoice \trans_{2} \toL ( \stk, \h, \opset ), \trans_{2} \).
Let us pick \( \trans_{1} \) and  assume it can be reduced to \( \pskip \) from the initial state, \ie \( ( \txstk, \h, \opset ), \trans_{1}  \toL^{*} ( \txstk', \h', \opset' ), \pskip \).
By the premiss of the rule \( \tripleL{\lpre}{\trans_{1}}{\lpost} \) and the \ih, it implies \( \tripleSemL{\lpre}{\trans_{1}}{\lpost} \), so we prove \( (\h', \opset') \in \evalLE[\lenv, \stk']{\lpost} \).
Symmetrically, if we pick \( \trans_{2} \), it gives the same result.

\caseI{\rl{TRSeq}}

We have \( \trans \equiv \trans_{1} \pseq \trans_{2} \) where \( \tripleL{\lpre}{\trans_{1}}{\lframe} \) and \( \tripleL{\lframe}{\trans_{2}}{\lpost} \) hold, for some \( \trans_{1}, \trans_{2}, \lpre, \lpost, \lframe \).
Given the transaction semantics (\figref{fig:thread_semantics}), it has \( \vdash ( \stk, \h, \opset ), \trans_{1} \pseq \trans_{2} \toL^{*} ( \stk'', \h'', \opset'' ), \pskip \pseq \trans_{1} \toL ( \stk'', \h'', \opset'' ), \trans_{1} \toL^{*} ( \stk', \h', \opset' ), \pskip \) for some intermediate state \( (\stk'', \h'', \opset'') \).
By the premiss of the rule and the \ih, we have \( \tripleSemL{\lpre}{\trans_{1}}{\lframe} \) and so \( (\h'', \opset'') \in \evalLE[\lenv, \stk'']{\lframe} \).
The elimination of prefix \( \pskip\) does not change any state, so \( (\h'', \opset'') \in \evalLE[\lenv, \stk'']{\lframe} \) still holds.
Then, by the premiss and the \ih, we know \( \tripleSemL{\lframe}{\trans_{2}}{\lpost} \) and therefore the proof that \( (\h', \opset') \in \evalLE[\lenv, \stk']{\lpost} \).

\caseI{\rl{TRLoop}}

Since the triple is only partial correct, meaning that if the transaction \( \trans \) terminates it will reach a state satisfying the post-condition \( \lpost \), it is sufficient to prove the follows,
\[
    \for{\lpre, \trans, \nat > 0} \tripleL{\lpre}{\trans^{\nat}}{\lpre} \implies \ \tripleSemL{\lpre}{\trans^{\nat}}{\lpre} 
\]
where,
\[
\begin{rclarray}
    \trans^{1} & \defeq  & \trans \\
    \trans^{\nat} & \defeq  & \trans \pseq \trans^{\nat - 1} \\
\end{rclarray}
\]

We prove that by induction on the number \( \nat \).
For \( \nat = 1 \), it is proven directly by the \ih
For \( \nat > 1 \), we have \( \vdash (\stk, \h, \opset), \trans \pseq \trans^{\nat - 1} \toL^{*} (\stk'', \h'', \opset''), \trans^{\nat - 1} \toL^{*} (\stk', \h', \opset'), \pskip \) for some intermediate state \( ( \stk'', \h'', \opset'' ) \).
By the premiss and the \ih, we have \(\tripleSemL{\lpre}{\trans}{\lpre} \) and thus \(  (\h'', \opset'') \in \evalLS[\lenv, \stk'']{\lpre} \).
Then by the \ih that \(\tripleSemL{\lpre}{\trans^{\nat - 1}}{\lpre} \), we prove \(  (\h', \opset') \in \evalLS[\lenv, \stk']{\lpre} \).

\caseI{\rl{TRFrame}}

We need to prove \( \tripleSemL{\lpre \sep \lframe }{\trans}{\lpost \sep \lframe} \) given that \( \tripleSemL{\lpre}{\trans}{\lpost} \).
Assume variables \( \h, \h', \h'', \opset, \opset', \opset'', \stk, \stk' \) such that \( ( \h, \opset ) \in \evalLS[\lenv, \stk]{\lpre} \), \( ( \h', \opset' ) \in \evalLS[\lenv, \stk']{\lpost} \) and \( ( \h'', \opset'' ) \in \evalLS[\lenv, \stk]{\lframe}\).
Since \( \lpre \sep \lframe \), the point-wise composition is defined, \ie \( (\h \composeH \h'', \opset \composeO \opset'') \in \evalLS[\lenv, \stk]{\lpre \sep \lframe} \).
The domain of the heaps and operations sets, therefore, are disjointed.
The domain of a operations set is all the addresses \( \dom(\opset) \defeq \opset\projection{2}\).
We also know the domain of the operation set is a subset of the domain of the heap, \(\dom(\opset) \subseteq \dom(\h) \) which can be proven by induction on the structures of local assertions \( \LAst \).
By the hypothesis \( \tripleSemL{\lpre}{\trans}{\lpost} \), we know \( ( \stk, \h, \opset ), \trans \toL^{*} ( \stk', \h', \opset' ), \pskip \).
The heap after the execution should contain the same resources as before, this is \( \dom(\h) = \dom(\h') \).
Since \( \dom(\opset') \subseteq \dom(\h') \), we know the compositions \( \h' \composeH \h''\) and \( \opset' \composeO \opset''\) exist.
This means the frame does not affect the semantic steps, \ie \( ( \stk, \h \composeH \h'', \opset \composeO \opset''), \trans \toL^{*} ( \stk', \h' \composeH \h'', \opset' \composeO \opset'' ), \pskip \).
Finally, because there is no free variables overlap between \( \lframe \) and \( \lpre, \lpost \), the update of stack does not change the evaluation of the frame, this is, \( \evalLS[\lenv, \stk]{\lframe} = \evalLS[\lenv, \stk']{\lframe} \) which then gives us the result \( (\h' \composeH \h'', \opset' \composeO \opset'') \in \evalLS[\lenv, \stk']{\lpost \sep \lframe} \).


%and similarly \( ( \h_{q} \composeH \h_{r}, \opset_{q} \uplus \opset_{r} ) \in \evalLS[\lenv, \thstk \uplus \txstk_{q}]{\lpost \sep \lframe} \land ( \h_{q} ,\opset_{q} ) \in \evalLS[\lenv, \thstk \uplus \txstk_{q}]{\lpost} \land ( \h_{r}, \opset_{r} ) \in \evalLS[\lenv, \thstk \uplus \txstk_{q}]{\lframe}\).
%By the \ih that \( \tripleSemL{\lpre}{\trans}{\lpost} \), it means .
%Now we need to prove the follows,
%\[
    %\thstk \vdash ( \txstk_{p}, \h_{p}  \composeH \h_{r}, \opset_{p} \composeO \opset_{r}), \trans \toL^{*} ( \txstk_{q}, \h_{q} \composeH \h_{r}, \opset_{q} \composeO \opset_{r}), \pskip 
%\]
%First for the heaps part, since both \( (\h_{p} \composeH \h_{r}) \) and  \( (\h_{q} \composeH \h_{r}) \) are defined, this means the domain of the frame \( \h_{r} \) are separate from the ones of \( \h_{p}\) and \( \h_{q} \).
%Then by the \ih, the transaction \( \trans \) does not need any resource from \( \h_{r} \) to progress.
%Second for the event sets part, since it has \( ( \h_{q} \composeH \h_{r}, \opset_{q} \composeO \opset_{r} ) \in \evalLS[\lenv, \thstk \uplus \txstk_{q}]{\lpost \sep \lframe} \), so \( \opset_{q} = \unitO \lor \opset_{r} = \unitO \).
%if \( \opset_{q} = \unitO \), it must be that \( \opset_{p} = \unitO \), it holds by the \ih
%If \( \opset_{r} = \unitO \), it also holds by the \ih
%Given above we have the prove that \( \tripleSemL{\lpre \sep \lframe }{\trans}{\lpost \sep \lframe} \).


\end{proof}

\subsection{Program Soundness}
\begin{lem}[Locality of cut]
\label{lem:locality-cut}
A thread can only update its own cut when commit a new transaction,
\[
\begin{array}{@{}l}
    \for{\thstk, \thstk', \hh, \hh', \thcu, \thcu', \cu, \cu', \thid, \como} \exsts{\thcu''} \\
    \quad \thid, \como \vdash (\thstk, \hh, \thcu), \ptrans{\trans} \toT{\lbC{\stub}} (\thstk', \hh', \thcu'), \pskip \\
    \qquad \implies \thcu  = \thcu'' \uplus \setminus \Set{\thid \mapsto \cu } \land \thcu' = \thcu'' \uplus \setminus \Set{ \thid \mapsto \cu' }
\end{array}
\]
This means the cut environment can be arbitrary as long as the local update remains the same and the entire environment satisfy the consistency model.
\[
\begin{array}{@{}l}
    \for{\thstk, \thstk', \hh, \hh', \thcu, \thcu', \cu, \cu', \thid, \thid', \como} \\
    \quad \thid, \como \vdash (\thstk, \hh, \thcu \uplus \Set{\thid \mapsto \cu}), \ptrans{\trans} \toT{\lbC{\stub}} (\thstk', \hh', \thcu \uplus \Set{\thid \mapsto \cu'}), \pskip \\
    \quad {} \land ((\hh,\thcu' \uplus \Set{\thid' \mapsto \cu}),(\hh',\thcu' \uplus \Set{\thid' \mapsto \cu'})) \in \como \\
    \qquad \implies \thid', \como \vdash (\thstk, \hh, \thcu' \uplus \Set{\thid' \mapsto \cu}), \ptrans{\trans} \toT{\lbC{\stub}} (\thstk', \hh', \thcu' \uplus \Set{\thid' \mapsto \cu'}), \pskip \\
\end{array}
\]
\end{lem}
\begin{proof}
The first one is trivial as \( \thcu' = \thcu\rmto{\thid}{\cu'}\) in the \rl{PCommit} rule, and no other side condition has side effect on \( \thcu' \).
For the second part, by the hypothesis we have the following which are exactly the side conditions of the \rl{PCommit},
\[
\begin{array}{@{}l}
    \exsts{\h, \h', \txstk_{0}, \txstk, \txid, \opset } \\
    \quad \txid \in \func{fresh}{\hh}  
    \land \h = \clpsHH{\hh,\cu}
    \land \txstk_{0} = \emptyset 
    \land \thstk \vdash (\txstk_{0}, \h, \unitO), \trans \ \toL^{*} \  (\txstk, \h', \opset) , \pskip \\
    \quad {} \land \thstk' = \thstk\rmto{\ret}{\txstk(\ret)} 
    \land \hh' = \func{commit}{\hh, \cu, \txid, \opset}  \\
    \quad {} \land \cu' = \func{update}{\hh', \cu, \opset} 
    \land ((\hh,\thcu \uplus \Set{\thid \mapsto \cu}),(\hh',\thcu \uplus \Set{\thid \mapsto \cu})) \in \como
\end{array}
\]
If we replace \( \thcu \) by \( \thcu' \) and the thread identifier such that \( ((\hh, \thcu' \uplus \Set{\thid' \mapsto \cu}),(\hh',\thcu' \uplus \Set{\thid' \mapsto \cu'})) \in \como \) holds, it is easy to see other side conditions still hold, therefore we have the proof.
\end{proof}
\sx{need change for how we collapse a world}
\begin{lem}
\label{lem:rely-guar-como}
Any transition in the rely or guarantee should satisfies the consistency model,
\[
\begin{array}{@{}l}
    \for{\w, \w', \hh, \hh', \cu, \cu'}
    \exsts{\thcu, \thcu', \thid} \\
    \quad (\w, \w') \in \Rely \cup \Guar
    \land (\hh, \cu) \in \clpsW{\w}
    \land (\hh', \cu') \in \clpsW{\w'} \\
    \qquad \implies 
    \thcu(\thid) = \cu
    \land \thcu'(\thid) = \cu'
    \land ((\hh, \thcu),(\hh', \thcu')) \in \func{como}{\w}
\end{array}
\]
\end{lem}
\begin{proof}
The rely and guarantee are defined using the \( \predn{to}\) predicate by plugging in different capabilities.
It is sufficient to prove that any transition allowed by the \(\predn{to}\) predicate is also allowed by the consistency model,
\[
\begin{array}{@{}l}
    \for{\opset, \ca, \gs, \gs', \hh, \hh', \cu, \cu'}
    \exsts{\thcu, \thcu', \thid} \\
    \quad \pred{to}{\opset, \ca, \gs, \gs'} 
    \land (\hh, \cu) \in \clpsS{\gs}
    \land (\hh', \cu') \in \clpsS{\gs'} \\
    \qquad \implies 
    \thcu(\thid) = \cu
    \land \thcu'(\thid) = \cu'
    \land ((\hh, \thcu),(\hh', \thcu')) \in \como
\end{array}
\]
where the \( \como \) is the consistency model associated to regions.
Since it is a recursive predicate, we are going to prove the property by induction.
\caseB{\(\opset = \unitO\)}
It is trivial because \(\gs = \gs'\), therefore \( \hh = \hh' \) and \( \thcu = \thcu' \) and the consistency model is reflexive.
\caseI{\(\opset \neq \unitO\)}
There exist a region \( \rid \) that has been updated from \( (\hh, \cu) \) to \( (\hh', \cu') \) which is allowed by the invariant, and the update for other regions satisfies the \predn{to} predicate, which is allowed by the consistency model by \ih,
\[
    \begin{array}[t]{@{}l}
    \exsts{\rid, \hh, \hh', \hh'', \hh''', \cu, \cu', \thcu'', \thcu''', \opset', \intf, \gs'',\gs''' }  \\
    \quad \opset' \subseteq \opset
    \land \gs = \gs'' \uplus \Set{\rid \mapsto (\hh, \cu, \intf)} 
    \land \gs' = \gs''' \uplus \Set{\rid \mapsto (\hh', \cu', \intf)}  \\
    \quad {} \land (\hh,\cu) \toLTS{\opset'} (\hh',\cu') \in \func{inv}{\rid, \intf }
    \land ((\hh'', \thcu''),(\hh''',\thcu''')) \in \como
    \end{array}
\]
Given the definition of labelled transition system for a region (\defref{def:labelled-transition-system}), the transition \( (\hh, \cu) \toLTS{\opset} (\hh', \cu') \) should satisfy the consistency model,
\[
\begin{array}{@{}l}
    \exsts{ \thcu, \thcu', \thid } 
    (\hh, \cu) \toLTS{\opset'} (\hh', \cu') \in \func{inv}{\rid, \intf} \\
    \quad \implies
    \begin{B}
    \txid \in \func{fresh}{\hh} 
    \land \hh' = \func{commit}{\hh, \cu, \txid, \opset} 
    \land \cu' = \func{update}{\hh', \cu, \opset} \\
    {} \land ((\hh,\thcu),(\hh',\thcu')) \in \como
    \land \thcu(\thid) = \cu 
    \land \thcu'(\thid) = \cu' 
    \end{B}
\end{array}
\]
Because the well-formed condition for a world (\defref{def:world}), \ie regions must be disjointed, we know the compositions \( \hh \composeHH \hh'' \) and \( \hh' \composeHH \hh''' \) exist.
Also because of the locality of cut (\lemref{lem:locality-cut}), we can pick a minimum cut environment \( \thcu = \Set{\thid \mapsto \cu} \), \( \thcu' = \Set{\thid \mapsto \cu'}\) and a fresh thread identifier, in a way that it satisfies the consistency when combined with the others.
This is,
\[
\begin{array}{@{}l}
    \exsts{\thid} \\
    \begin{B}
    \thid \notin \dom(\thcu'') 
    \land (\hh \composeHH \hh'')\isdef 
    \land (\hh' \composeHH \hh''')\isdef  \\
    \land ((\hh', \Set{\thid \mapsto \cu}),(\hh'',\Set{\thid \mapsto \cu'})) \in \como 
    \land ((\hh'', \thcu''),(\hh''',\thcu''')) \in \como 
    \end{B} \\
    \quad \implies ((\hh' \composeHH \hh'', \Set{\thid \mapsto \cu} \uplus \thcu''), (\hh'' \composeHH \hh''' ,\Set{\thid \mapsto \cu'} \uplus \thcu''')) \in \como
\end{array}
\]
\end{proof}
\begin{thm}[Program soundness]
The program soundness is the follows,
\[
    \for{\gpre, \prog, \gpost}
    \como \tripleG{\gpre}{\prog}{\gpost} 
    \implies 
    \como \tripleSemG{\gpre}{\prog}{\gpost} 
\]
\end{thm}
\begin{proof}
Induction on the derivations.
\caseB{\rl{PRCommit}}
We have \( \prog \equiv \ptrans{\trans} \).
Because a transaction \( \ptrans{\trans} \) is reduced by one step in the semantics, it is sufficient to prove for any state \(\w\) that satisfies pre-condition, if a machine state \((\hh',\vi', \ca') \in \clpsW{\w'}\),  after arbitrary steps of rely, \ie \( (\w, \w') \in \Rely^{*} \) (\equref{equ:stable-pre-condition}), can transfers to a new state \((\hh'',\vi'', \ca'')\) (\equref{equ:commit-transaction}) then followed by arbitrary steps of rely \((\w'',\w''') \in \Rely^{*} \), the final state \( \w''' \) should satisfy the post-condition \(\gpost\) (\equref{equ:stable-post-condition}).
\sx{typesetting is a bit strange}
\begin{gather}
    \for{\w, \w',\lenv, \stk} 
    \stable{\gpre, \como} 
    \land \w \in \evalW{\gpre} 
    \land (\w, \w') \in \Rely
    \land (\w, \w') \in \como
    \implies \w' \in \evalW{\gpre} \tag{Stable Pre} \label{equ:stable-pre-condition} \\
    \begin{array}{@{}l}
    \begin{B}
        \tripleL{\lpre}{\trans}{\lpost}
        \land \repartition{\gpre}{\gpost}{\lpre}{\lpost}
    \end{B} \\
    \implies 
    \for{\w, \w', \hh, \hh', \vi, \vi', \vienv, \vienv', \ca, \ca', \thid, \lenv, \stk, \stk'} \\
    \quad \begin{B}
        \w \in \evalW{\gpre}
        \land (\hh, \vi, \ca) \in \clpsW{\w}
        \land \vienv(\thid) = \vi \\
        {} \land \thid, \como \vdash (\stk, \hh, \vienv), \ptrans{\trans} 
        \toT{\lbC{\txid}} (\stk', \hh', \vienv'), \pskip  \\
        {} \land \vienv'(\thid) = \vi'
        \land (\hh', \vi', \ca') \in \clpsW{\w'} 
    \end{B} 
    \implies  \w' \in \evalW[\lenv, \stk']{\gpost} 
    \end{array} \label{equ:commit-transaction} \tag{Commit} \\
    \for{\w, \w',\lenv, \stk}  
    \stable{\gpost, \como} 
    \land \w \in \evalW{\gpost} 
    \land (\w, \w') \in \Rely
    \land (\w, \w') \in \como
    \implies \w' \in \evalW{\gpost} \tag{Stable Post} \label{equ:stable-post-condition} 
\end{gather}
\sx{make sure the stack is correct}
\textbf{Stable pre-condition.} 
The \( \stable{\gpre, \como} \) predicate asserts any world \( \w \) that satisfies the pre-condition \( \gpre \), if the world can transfer to another world \( \w' \) through rely \( \Rely \), and if the transfer also satisfies the consistency model \( \como \), the new world \( \w' \) satisfies the pre-condition, which implies \equref{equ:stable-pre-condition}. 
\\
\textbf{Commit.}
For any \( \w, \hh, \vi, \lenv, \stk \) such that \( \w \in \evalW{\gpre} \) and \( (\hh, \vi, \ca) \in \clpsW{\w} \), by the predicate \( \pred{unbox}{\gpre, \lpre} \) (wrapped inside the repartitioning) we know \( (\clpsHH{\hh, \vi}, \unitO) \in \evalLS{\lpre} \), this is,
\begin{equation}
\label{equ:local-pre-condition}
\for{\w, \hh, \vi, \lenv, \stk} \w \in \evalW{P} \land (\hh, \vi, \stub) \in \clpsW{\w} \implies (\clpsHH{\hh, \vi}, \unitO) \in \evalLS{\lpre}
\end{equation}
Because of the soundness of transaction (\thmref{thm:transaction-soundness}), given a stack \( \stk \) and a logical environment \( \lenv \), if a initial configuration \( (\stk, \h, \unitO), \trans \) satisfies the pre-condition, \ie \( (\h, \unitO) \in \evalLS[\lenv,\stk]{\lpre} \), and if it can transfer to a final configuration \( (\stk', \h', \opset), \pskip \), the final configuration should satisfy the post-condition \( \lpost \).
This is, for any \( \thstk, \txstk, \txstk', \h, \h', \opset \), they satisfy the follows,
\begin{equation}
\label{equ:local-transaction-sound}
\begin{array}{@{}l}
    \for{\stk, \stk', \hh, \vi, \h, \h', \opset} 
    \h = \func{clps}{\hh,\vi} \\
    \quad (\h, \unitO) \in \evalLS[\lenv,\stk]{\lpre}
    \land \vdash (\stk, \h, \unitO), \trans \toL (\stk', \h', \opset), \pskip
    \implies (\h', \opset) \in \evalLS[\lenv,\stk']{\lpost}
\end{array}
\end{equation}
The repartition \( \repartition{\gpre}{\gpost}{\lpre}{\lpost} \) also asserts that any world \( \w \) satisfying the pre-condition \( \gpre \), if the corresponding machine of the world, \ie \( (\hh, \vi) \), can transfer to a new state \( (\hh',\vi') \), by committing the operations \( \opset \), then if a world \( \w' \) can collapses to the new machine state \( (\hh',\vi') \) and the transition \( (\w, \w') \) is allowed by both the guarantee and the consistency model, the new world \( \w' \) should satisfy the post-condition.
\begin{equation}
\label{equ:repartition}
\begin{array}{@{}l}
    \for{\w, \w', \hh, \hh', \vi, \vi', \stk, \stk', \opset, \lenv, \txid} \\
    \begin{B}
        \w \in \evalW{\gpre}
        \land (\hh, \vi, \stub) \in \eraseW{\w}
        \land \txid \in \func{fresh}{\hh} 
        \land \opset \in \evalLS[\lenv, \stk']{\lpost} \\
        {} \land \hh' = \func{updHisHp}{\hh, \vi, \txid, \opset}  
        \land \vi' = \func{updView}{\hh, \vi, \opset} \\
        {} \land (\hh',\vi', \stub) \in \eraseW{\w'}
        \land (\w, \w') \in \Guar 
        \land (\w, \w') \in \como 
    \end{B}
    \implies \w' \in \evalW[\lenv, \stk']{\gpost}
\end{array}
\end{equation}
First by \equref{equ:local-transaction-sound}, we know that for any world that satisfies \( \gpre \), there exist sets of operations corresponding to the transaction code \( \trans \) which then update the history heap and local view.
If we combine  \equref{equ:local-transaction-sound} and \equref{equ:repartition}, we have the follows,
\begin{equation}
\label{equ:combined-transaction-sound}
\begin{array}{@{}l}
    \for{\w, \w', \hh, \hh', \vi, \vi', \stk, \stk', \h, \h', \opset, \lenv, \txid} \\
    \begin{B}
        \w \in \evalW{\gpre}
        \land (\hh, \vi, \stub) \in \eraseW{\w}
        \land \txid \in \func{fresh}{\hh} 
        \land \h = \clpsHH{\hh, \cu}  \\
        {} \land \vdash (\stk, \h, \unitO), \trans \toL (\stk', \h', \opset), \pskip  \\
        {} \land \hh' = \func{updHisHp}{\hh, \vi, \txid, \opset}  
        \land \vi' = \func{updView}{\hh, \vi, \opset} \\
        {} \land (\hh',\vi', \stub) \in \eraseW{\w'}
        \land (\w, \w') \in \Guar 
        \land (\w, \w') \in \como 
    \end{B}
    \implies \w' \in \evalW[\lenv, \stk']{\gpost}
\end{array}
\end{equation}
\sx{
    Assume by the way \( (\w, \w') \in \como \) is constructed, it implies side condition for the consistency model in the semantics.
    Since how to specify consistency model does not settle down yet.
    Leave the soundness prove like this for now.
}
%Then, by the \lemref{lem:rely-guar-como} that the guarantee \( \Guar \) ensures the transition from \( \w \) to \( \w' \) satisfies the consistent model and picking the empty transaction stack as the initial \( \txstk = \txstk_{0} = \emptyset \) transaction stack for the transaction code \( \trans \), we have the follows,
%\[
%\begin{array}{@{}l}
    %\for{\w, \w', \hh, \hh', \cu, \cu', \lenv, \stk, \txid, \h, \h', \thstk} 
    %\exsts{\thcu, \thcu', \thid } \\
    %\begin{B}
        %\exsts{\h, \h', \txstk_{0}, \txstk', \txid, \opset } \\
        %\quad \w \in \evalW{\gpre} 
        %\land (\hh, \cu) \in \eraseW{\w}
        %\land \txid \in \func{fresh}{\hh} 
        %\land \h = \clpsHH{\hh, \cu}
        %\land \txstk_{0} = \emptyset \\
        %\quad {} \land \thstk \vdash (\txstk, \h, \unitO), \trans \toL (\txstk', \h', \opset), \pskip 
        %\land \hh' = \func{commit}{\hh, \cu, \txid, \opset} \\
        %\quad {} \land \cu' = \func{update}{\hh, \cu, \opset} 
        %\land (\hh',\cu') \in \eraseW{\w'}  \\
        %\quad {} \land ((\hh,\thcu),(\hh',\thcu')) \in \func{como}{\w}
        %\land \thcu(\thid) = \cu 
        %\land \thcu'(\thid) = \cu' \\
    %\end{B}
    %\implies \w' \in \evalW{\gpost}
%\end{array}
%\]
%In the equation above, we have all the side conditions of the \rl{PCommit} rule except the return value, \ie \( \thstk' = \thstk\rmto{\ret}{\txstk(\ret)} \).
%Since the return value does not affect how the post condition \( \gpost \) is interpreted by the repartition, we can fold all the side conditions to the follows,
%\[
%\begin{array}{@{}l}
    %\for{\w, \w', \hh, \hh', \cu, \cu', \lenv, \stk, \txid, \h, \h', \thstk, \thstk', \txid} 
    %\exsts{\thcu, \thcu', \thid } \\
    %\begin{B}
        %\quad \w \in \evalW{\gpre} 
        %\land (\hh, \cu) \in \eraseW{\w}
        %\land \thid, \func{como}{\w} \vdash (\thstk, \hh, \thcu), \ptrans{\trans} 
        %\toT{\lbC{\txid}} (\thstk', \hh', \thcu'), \pskip  \\
        %\quad {} \land (\hh',\cu') \in \eraseW{\w'}  
        %{} \land ((\hh,\thcu),(\hh',\thcu')) \in \func{como}{\w}
        %\land \thcu(\thid) = \cu 
        %\land \thcu'(\thid) = \cu' \\
    %\end{B}
    %\implies \w' \in \evalW{\gpost}
%\end{array}
%\]
%Now we can apply the \lemref{lem:locality-cut} which allows us to convert the existential quantification for \( \thcu, \thcu', \thid\) to global quantification, thus we have the proof for committing a new transaction, \ie \equref{equ:commit-transaction}. 
\textbf{Stable post-condition.} 
It can be proven for the similar reason as the proof for stable pre-condition.
\end{proof}


%\subsection{Local/Transaction}

\begin{definition}[Logical Expressions]
\label{def:logical-expr}
Assume a countably infinite set of \emph{logical variables} $\V x \in \LVar$.
The set of \emph{logical expressions}, $ \lexpr \in \LExpr$ is defined by the following inductive grammar, where \(\val \in \Val\) (\defref{def:program_values}), \(\txvar \in \TxVars\) and \( \thvar \in \ThdVars \) (\defref{def:stacks}),
\[
\begin{rclarray}
   \lexpr & ::= & \val \mid \txvar \mid \thvar \mid \lvar \mid \lexpr + \lexpr \mid \lexpr \times \lexpr \mid \dots 
\end{rclarray}
\]
Assume a set of \emph{logical environments} \(\lenv \in \LEnv: \LVar \parfun \Val\) which associates logical variables with values.
Given a stack $\stk \in \Stacks$ (\defin\ref{def:stacks}) and a logical environment $\lenv \in \LEnv$, the \emph{logical expression evaluation} function, $\evalLE[(., .)]{.}:\LExpr \times \Stacks \times \LEnv\rightharpoonup \Val$, is defined inductively over the structure of logical expressions as follows,
%
\[
    \begin{rclarray}
        \evalLE{\val} & \defeq & \val \\
        \evalLE[\lenv, \thstk \uplus \txstk]{\thvar} & \defeq & \thstk(\thvar) \\
        \evalLE[\lenv, \thstk \uplus \txstk]{\txvar} & \defeq & \txstk(\txvar) \\
        \evalLE{\lvar} & \defeq & \lenv(\lvar) \\
        \evalLE{\lexpr_1 + \lexpr_2} & \defeq & \evalLE{\lexpr_1} + \evalLE{\lexpr_2} \\
        \evalLE{\lexpr_1 \times \lexpr_2} & \defeq & \evalLE{\lexpr_1} \times \evalLE{\lexpr_2} \\
        \dots & \defeq & \dots \\
    \end{rclarray}
\]
Note that the stack \( \stk \) includes transaction variables and thread variables.
\end{definition}

\emph{Fingerprint assertion} or \emph{fingerprint} is a set of tuples in the form of \( (\otag, \lexpr_{1}, \lexpr_{2}) \) where \( \otag \) is either read tag \( \etR \) or write \( \etW \) and the second and third elements are logical assertions representing the address and value respectively.
This assertion is interpreted to a set of transaction events as expected.

\begin{defn}[Fingerprint Assertions]
\label{def:fingerprint}
The \emph{fingerprint assertion} also \emph{fingerprint}, \( \fp \in \FAst \), is defined as the follows, 
\[
\begin{rclarray}
    \fp & \subseteq & \Setcon{ (\otag,\lexpr_{1},\lexpr_{2}) }{ \otag \in \OTags \land \lexpr_{1}, \lexpr_{2} \in \LExpr } \\
\end{rclarray}
\] 
Given a logical environment $\lenv \in \LEnv$ and a stack $\stk \in \Stacks$, the \emph{fingerprint interpretation} function, $\evalF[(., .)]{.}: \FAst \times \LEnv \times \Stacks \parfun \Opsets$, is defined as follows,
\[
\begin{rclarray}
    \evalF{\emptyset} & \defeq & \unitO  \\
    \evalF{\fp \addF (\otag, \lexpr_{1}, \lexpr_{2})} & \defeq & \evalF{\fp} \addO (\otag, \evalLE{\lexpr_{1}}, \evalLE{\lexpr_{2}})
\end{rclarray}
\]
\end{defn}

The local assertions includes normal separation logic assertions and extra fingerprint assertions, which are interpreted as sets of heaps and a set of events respectively.
Notice that the fingerprint assertion cannot be split.

\begin{definition}[Local assertions]
\label{def:local_assertions}
Given the set of logical expressions \( \LExpr \), logical variables \( \LVar \) and fingerprint assertion \( \FAst \), the set of \emph{local assertions}, $\lpre,  \lpost \in \LAst$, is defined inductively by the following grammar, 
\[
\begin{rclarray}
	\lpre, \lpost  & ::= & \False \mid \True \mid \lpre \land \lpost \mid \lpre \lor \lpost \mid \exsts{\lvar} \lpre \mid \lpre \implies \lpost \mid \Emp \mid \lexpr \pt \lexpr \mid \fpF \mid \lpre \sep \lpost  \\
\end{rclarray}	 
\]
Given a logical environment $\lenv \in \LEnv$, the \emph{local interpretation function}, $\evalLS[(.,.)]{.}: \LAst \times \LEnv \times \LAst \parfun \Heaps \times \powerset{ \Events } $, is defined over the structure of local assertions as follows,
\[
\begin{rclarray}
	\evalLS{\assfalse} & \eqdef & \emptyset \\
	\evalLS{\asstrue} & \defeq & \Heaps \times \powerset{ \Events } \\
	\evalLS{\lpre \land \lpost} & \defeq & \evalLS{\lpre} \cap \evalLS{\lpost} \\
	\evalLS{\lpre \lor \lpost} & \defeq & \evalLS{\lpre} \cup \evalLS{\lpost} \\
	\evalLS{\exsts{\lvar} \lpre} & \defeq & \bigcup\limits_{\val \in \textnormal{\Val}}\evalLS[\lenv\remapsto{\lvar}{\val}, \stk]{\lpre}  \\
	\evalLS{\lpre \implies \lpost} & \defeq & \Setcon{\h}{\h \in \evalLS{\lpre} \implies \h \in \evalLS{\lpost}}\\
	\evalLS{\assemp} & \defeq & \Set{ ( \unitH, \unitE) }  \\
	\evalLS{\lexpr_{1} \pt \lexpr_2 } & \defeq & \Set{ (\evalLE{\lexpr_1} \pt \evalLE{\lexpr_2}, \unitE) } \\
	\evalLS{ \fpF } & \defeq & \Set{ (\unitH, \evalF{\fp}) } \\
	\evalLS{\lpre \sep \lpost} & \defeq & 
    \Setcon{
        (\h_1 \composeH \h_2, \evset_{1} \composeE \evset_{2})
    }{ 
        (\h_{1},\evset_{1}) \in \evalLS{\lpre} 
        \land (\h_{2}, \evset_{2} ) \in \evalLS{\lpost} 
    } 
\end{rclarray}
\]
\end{definition}

Observe that program expressions $\Expr$  (\defin\ref{def:language}) are contained in logical expressions $\LExpr$ (\defin\ref{def:local_assertions} above), \ie $\Expr \subset \LExpr$. 
For readability, we will write angle brackets, \eg \( \fpass{(\etR, \vx, 0)} \) instead of curly brackets \( \fpto{\Set{(\etR, \vx, 0)}} \) for fingerprint assertions.

\subsection{Global/Program}

\begin{definition}[Capabilities]
\label{def:capabilities}
Assume a \emph{partial commutative monoid (PCM)} of \emph{client-specified capabilities} \( (\Kaps, \composeK, \unitK) \) with \( \kap \in \Kaps \), the composition \( \composeK \) the units set \( \unitK \).
Then given a set of \emph{region identifiers} \( \rid \in \RegionID \), the \emph{capability composition function} or \emph{capabilities} \( \ca \in \Caps \defeq \RegionID \parfun \Kaps \), where the composition \( \composeC \) is defined as the follows,
\[
    \begin{rclarray}
        (\ca_{l} \composeC \ca_{r})(\rid) & \defeq  &
        \begin{cases}
            \ca_{l}(\rid) \composeK \ca_{r}(\rid) & \rid \in \dom(\ca_{l}) \cap \dom(\ca_{l}) \\
            \ca_{l}(\rid)  & \rid \in \dom(\ca_{l}) \setminus \dom(\ca_{l}) \\
            \ca_{r}(\rid) & \rid \in \dom(\ca_{r}) \setminus \dom(\ca_{l}) \\
            \text{undefined} & \text{otherwise} \\
        \end{cases}
    \end{rclarray}
\]
, and the units set \( \unitC \defeq \Setcon{\ca}{\for{\rid} \ca(\rid) \in \unitK } \) .
\end{definition}

\begin{defn}[Interference]
\label{def:intf}
Given the fingerprint assertion \( \fp \in \Fingerprint \) (\defref{def:fingerprint}) and local assertion \( \lpre \in \LAst \) (\defref{def:local_assertions}), the grammar of \emph{interference assertions}, \( \intass \in \IAst \), is defined as the follows,
\[
\begin{rclarray}
	\intass & ::=  &
	\emptyset \mid \Set{ \perm{\kap} :  \exsts{\vec{\lvar}} \lpre \mat \fp } \cup \intass 
\end{rclarray}
\]
It will be interpreted to a set of \emph{interference environments}, this is,
\[
\begin{rclarray}
    \inter \in \Interference & \defeq & \Kaps \parfun \powerset{\Heaps} \times \Opsets
\end{rclarray}
\]
Given a logical environment $\lenv \in \LEnv$ and a stack $\stk \in \Stacks$, the \emph{interference interpretation} function, $\evalI[(., .)]{.}: \IAst \times \LEnv \times \Stacks \to \Interference$, is defined as follows,
%
\[
\begin{rclarray}
	\evalI{\emptyset}(\kap) & \eqdef & \text{undefined} \\
	\evalI{\Set{ \perm{\kap} : \exsts{\vec{\lvar}} \lpre \mat \fp } \cup \intass }(\kap') & \eqdef &
    \begin{cases}
    (\evalLS[\lenv',\stk]{\lpre}, \evalF[\lenv',\stk]{\fp}) \cup \evalI{\intass}(\kap')  & \kap = \kap' \\
    \evalI{\intass}(\kap') & \text{ otherwise} \\
    \end{cases} \\
    & & \text{where there exists a vector of values \( \vec{\val}\) such that } \lenv' = \lenv\rmto{\vec{\lvar}}{\vec{\val}} \\
\end{rclarray}
\] 
\end{defn}

\begin{defn}[Labelled transition system]
The labelled transition system is a tuple, \( (\aexecset,\actionset,\toLTS{}, \aexecset_{0}, \como) \), consisting of a set of abstract executions \( \aexecset \subseteq \Aexecs \), a set of actions \( \actionset \subseteq \Actions \), a relation \( \toLTS{} : \Aexecs \times \Actions \times \Aexecs \), a set of initial abstract executions \( \aexecset_{0}\) and the consistency model associated with the transition system \( \como \).
Assume all the initial abstract executions satisfies the consistency model.
The relation \( \toLTS{}\) is defined as the follows,
\[
\begin{rclarray}
    \aexec \toLTS{\evset} \aexec' & \defeq &
        \begin{array}[t]{@{}l}
        \exsts{\vis, \po, \ar, \txid } \\
        \quad {} \land \aexec' = (\aexec\prjT \uplus \Set{ \txid \mapsto \evset }, \aexec\prjP \uplus \po, \aexec\prjV \uplus \vis, \aexec\prjA \uplus \ar) \\
        \quad {} \land \vis, \po \subseteq \ar = \Setcon{(\txid', \txid)}{\txid' \in \dom(\aexec\prjT)} 
        \land \aexec' \in \evalCOM{\como}
    \end{array}
\end{rclarray}
\]
\end{defn}
 
\begin{defn}[Invariant a region]
\label{def:invariant-region}
\label{def:world2aexec}
\label{def:state2aexec}
Assume two global functions, \( \funcn{init} : \RegionID \to \powerset{\Aexecs} \) that returns initial abstract executions for regions, and \( \funcn{como} : \RegionID \to \ConsisModels \) that returns the consistency models associated with regions.
Also assume all the initial states for a region satisfy the consistency model, \ie
\[
\for{\rid, \aexec_{0}} \aexec_{0} \in \func{init}{\rid} \implies \aexec_{0} \in \evalCOM{\func{como}{\rid}}
\]
The invariant of a region, namely \( \func{transinv}{\rid, \intf} \), is the labelled transition system where the initial state is \( \func{init}{\rid}\) and all the actions are included in the interference.
\[
\begin{rclarray}
    \func{inv}{\rid, \intf} & \defeq & (\aexecset,\actionset \cup \Set{\unitE},\toLTS{}, \func{init}{\rid}, \func{como}{\rid}) \\
    & & \text{where } \for{\evset} \evset \in \actionset \implies \exsts{\kap} \evset \in \dom(\intf(\kap))
\end{rclarray}
\]
For brevity, \( \aexec \in \func{inv}{\rid, \intf} \) is short-hand for \( \aexec \in \aexecset \), and similarly \( \aexec \toLTS{\evset} \aexec' \in \func{inv}{\rid, \intf} \).
\end{defn}

The empty (unit) event \( \unitE \) in the invariant is a place-holder for other regions.
They might append a concrete event, while the composition of abstract executions composes events point-wise if the two executions have the same structure.

\begin{defn}[Well-form of a region]
\label{def:well-form-region}
The well-form condition of the interference, namely \( \pred{wfintf}{\rid, \intf} \) predicate, assertions for any concrete events \( \evset \), the state before the events must be included in the interference.
\[
\begin{rclarray}
    \pred{wfintf}{\rid, \intf} & \defeq & \for{\aexec, \aexec', \evset} \aexec \toLTS{\evset} \aexec' \in \func{inv}{\rid, \intf} \land ( \evset \neq \unitE \implies \pred{wfabs}{\rid, \intf, \aexec, \evset, \aexec'} ) \\ 
    \pred{wfabs}{\rid, \intf, \aexec, \evset, \aexec'} & \defeq & 
    \begin{array}[t]{@{}l}
        \exsts{ \kap }
        \evset \in ( \dom(\intf( \kap )) ) \land \obsstate{\aexec, \aexec\prjT,\func{como}{\rid}\projection{2}} \subseteq \intf(\kap)(\evset)
    \end{array} \\
\end{rclarray}
\]
Given a region (identifier) \(\rid\), its current state \( \h \), and its interference \( \intf \), the function \(\funcn{r2e} \) returns all the possible abstract executions that are included in the invariant of a region and satisfy the state \( \h \),
\[
\begin{rclarray}
    \func{r2e}{\rid, \h, \intf} & \eqdef & \Setcon{\aexec}{\aexec \in \func{inv}{\rid, \intf} \land \h \in \obsstate{\aexec,\aexec\prjT,\func{como}{\rid}\projection{2}}} \\
\end{rclarray}
\]
A set of heaps \( \hset \) approximates the observation of a region \( \rid \) under state \( \h \), namely \( \pred{approx}{\rid, \h, \hset, \intf} \), when an abstract execution satisfies the state \( \h \) and it can reach a new state by appending a new transaction \( \txid \), the observable state of the new transaction must included in the approximation \( \hset \),
\[
\begin{rclarray}
    \pred{approx}{\rid, \aexec, \hset, \intf} & \eqdef & 
    \begin{array}[t]{@{}l}
    \for{ \aexec' } \aexec \toLTS{\stub} \aexec' \in \func{inv}{\rid, \intf} \\
    \quad {} \land \exsts{ \txid } \txid = \max_{\ar}\Set{\aexec'\prjT}
    \land \obsstate{\aexec,\aexec\prjV^{-1}(\txid),\func{como}{\rid}\projection{2}} \subseteq \hset
    \end{array}
\end{rclarray}
\]
\end{defn}

\begin{definition}[Worlds]
\label{def:world}
Given the set of heaps $\Heaps$ (\defref{def:heaps}) and a set of \emph{region identifiers} \( \rid \in \RegionID \), the set of \emph{shared states} is \( \SStates \eqdef \RegionID \to \Heaps \times \powerset{\Heaps} \times \Interference \).
Each region has its current state, a set of possible initial states for transitions and the interference.
The \emph{shared state composition function}, $\composeS: \SStates \times \SStates \parfun \SStates$, is defined as $\composeS \eqdef \composeEq$, where for all domains $\sort M$ and all $m, m' \in \sort M$,
%
\[
\begin{rclarray}
	m \composeEq m' &  \eqdef  &
	\begin{cases}
		m & \text{if } m = m'\\
		\text{undefined} & \text{otherwise}
	\end{cases}
\end{rclarray}
\]
A \emph{world} \( \w \in \World \) is a pair of a shared state \( \gs \) and capabilities \( \ca \) (\defref{def:capabilities}), where regions are associated with the same consistency model and the collapse of the pair exists, \ie regions are well-form and compatible.
\[
\begin{rclarray}
	\world \in \World  & \eqdef & 
    \Setcon{
        (\ca, \gs) 
    }{ 
        \ca \in \Caps 
        \land \gs \in \SStates
        \land \clpsW{\gs} \neq \emptyset
        \land \dom(\ca) \subseteq \dom(\gs) \\
        \quad {} \land \for{\rid, \rid'}
        \func{como}{\rid} = \func{como}{\rid'} \\
        \quad {} \land \for{\h, \h' }
        \h \in \Set{\gs(\rid)\projection{1}} \cup \gs(\rid)\projection{2}
        \land \h' \in \Set{\gs(\rid')\projection{1}} \cup \gs(\rid')\projection{2} 
        \land ( \h \composeH \h' )\isdef
    }
\end{rclarray}
\]
The function, \( \clpsW{.} : \SStates \parfun \powerset{\Aexecs} \), collapses shared states to sets of abstract executions as the follows,
\[
\begin{rclarray}
    \clpsW{\emptyset} & \defeq & \unitAEX \\
    \clpsW{\Set{\rid \mapsto (\h, \hset, \intf)} \uplus \gs } & \defeq & 
        \Setcon{ \aexec \composeAEX \aexec' }{ \aexec \in \func{r2e}{\rid, \h, \intf} \land \pred{approx}{\rid, \aexec, \hset, \intf} \land \pred{wfintf}{\rid, \intf} \land \aexec' \in \clpsW{\gs} }\\
\end{rclarray}
\] 
% 
The \emph{world composition function}, $\composeW: \World \times \World \parfun \World$, is defined component-wise as: $\composeW \eqdef (\composeC, \composeS)$.
The \emph{world unit set} is $\unitW \eqdef \Setcon{(\ca, \gs)}{(\ca, \gs) \in \World \land \ca \in \unitC}$.
The \emph{partial commutative monoid of worlds} is $(\World, \composeW, \unitW)$.
\end{definition}

\sx{point-wise composition}
\begin{defn}[Invariant of worlds]
Because regions in a well-defined world must disjointed with each other and have the same consistency model, it is easy to lift the invariant of a region to a shared state,
\[
\begin{rclarray}
    \func{inv}{\emptyset} & \defeq & (\aexecset, ... ) \\
    \func{inv}{\Set{\rid \mapsto (\stub, \stub, \intf)} \uplus \gs} & \defeq & \Setcon{\aexec \composeAEX \aexec' }{\aexec \in \func{inv}{\rid, \intf} \land \aexec' \in \func{inv}{\gs}} \\
    \func{transinv}{\emptyset} & = & \Setcon{ ( \aexec , \unitE, \aexec' ) }{\aexec, \aexec' \in \unitAEX } \\
    \func{transinv}{\Set{\rid \mapsto (\stub, \stub, \intf)} \uplus \gs} & = & 
    \Setcon{
        ( \aexec \composeAEX \aexec_{f}, \evset \composeE \evset_{f}, \aexec' \composeAEX \aexec_{f}' ) 
    }{
        (\aexec, \evset, \aexec') \in \func{transinv}{\rid, \intf} \\
        \quad \land (\aexec_{f}, \evset_{f}, \aexec_{f}') \in \func{transinv}{\gs}
    }
\end{rclarray}
\]
\end{defn}

\begin{definition}[Assertions]
\label{def:assertion}
Assume standard separation logic assertion \( \bar{\lpre}, \bar{\lpost }\) (the local assertion \( \LAst \) without fingerprint) and the interpretation function, The set of \emph{assertions}, $\gpre, \gpost \in \Ast$, are defined by the following inductive grammar:
\[
\begin{rclarray}
	\gpre , \gpost & \defeq & \False \mid \True \mid \gpre \land \gpost \mid \gpre \lor \gpost \mid \exsts{\lvar}\gpre \mid \gpre \implies \gpost \mid \assemp \mid \cass{\kap}{\lrid} \mid \gpre \sep \gpost \mid \sptboxass{\bar{\lpre}}{\bar{\lpost}}{\lrid}{\intass}\\
\end{rclarray}
\]
%
where $\lvar, \lrid \in \LVar$, $\lexpr_1, \lexpr_2 \in \LExpr$ (\defin\ref{def:local_assertions}), $\kap \in \Kaps$ (\defin\ref{def:capabilities}) and $\intass \in \IAst$ (\defin\ref{def:intf}).
Given a logical environment $\lenv \in \LEnv$ and a stack $\stk \in \Stacks$, the \emph{assertion interpretation} function, $\evalW[(., .)]{.}: \Ast \times \LEnv \times \Stacks \to \powerset{\World}$, is defined as follows:
%
\[
\begin{rclarray}
	\evalW{\False} & \defeq & \emptyset \\
	\evalW{\True} & \defeq & \World \\
	\evalW{\emp} & \defeq & \unitW \\
	\evalW{\gpre \land \gpost} & \defeq & 
    \Setcon{
        (\ca, \gs)
    }{
        \exsts{\gs_{p}, \gs_{q}} 
        (\ca, \gs_{p}) \in \evalW{\gpre} 
        \land (\ca, \gs_{q}) \in \evalW{\gpost} \\
        \quad {} \land \for{\rid} 
        \exsts{\h, \hset_{p}, \hset_{q}, \intf} 
        \gs(\rid) = (\h, \hset_{p} \cap \hset_{q}, \intf) \\
        \qquad {} \land \gs_{p}(\rid) = (\h, \hset_{p}, \intf)
        \land \gs_{q}(\rid) = (\h, \hset_{q}, \intf)
    } \\
	\evalW{\gpre \lor \gpost} & \defeq & 
    \Setcon{
        (\ca, \gs)
    }{
        \exsts{\gs_{p}, \gs_{q}} 
        (\ca, \gs_{p}) \in \evalW{\gpre} 
        \land (\ca, \gs_{q}) \in \evalW{\gpost} \\
        \quad {} \land \for{\rid} 
        \exsts{\h, \hset_{p}, \hset_{q}, \intf} 
        \gs(\rid) = (\h, \hset_{p} \cup \hset_{q}, \intf) \\
        \qquad {} \land \gs_{p}(\rid) = (\h, \hset_{p}, \intf)
        \land \gs_{q}(\rid) = (\h, \hset_{q}, \intf)
    } \\
	\evalW{\exsts{\lvar}  \gpre} & \defeq & \bigcup\limits_{\val \in \textnormal{\Val}} \evalW[\lenv\remapsto{\lvar}{\val}, \stk]{\gpre} \\
	\evalW{\gpre \implies \gpost} & \defeq & \Setcon{\w}{\w \in \evalW{\gpre} \implies \w \in \evalW{\gpost}} \\
	\evalW{\cass{\kap}{\lrid}} & \defeq & \Setcon{ (\Set{\lrid \mapsto \kap}, \gs) }{\gs \in \SStates} \\
	\evalW{ \gpre \sep \gpost } & \defeq & 
	\Setcon{
	   (\world_1 \composeW \world_2) 
    }{
       \world_1 \in \evalW{\gpre} \land \world_2 \in \evalW{\gpost}
	} \\
	\evalW{ \sptboxass{\bar{\lpre}}{\bar{\lpost}}{\lrid}{\intass} } & \defeq & 
    \Setcon{
        (\ca,\Set{\lrid \mapsto (\h, \hset, \intf)} \uplus \gs)
    }{
        \ca \in \unitC 
        \land \h \in \evalLS{\bar{\lpre}}
        \land \hset = \evalLS{\bar{\lpost}}
        \land \intf  = \evalI{\intass}
    } \\
\end{rclarray}
\]
\end{definition}

We will write \( \boxass{\bar{\lpre}}{\lrid}{\intass} \) as a short-hand for \( \sptboxass{\bar{\lpre}}{\bar{\lpre}}{\lrid}{\intass} \) and \(\expr \pt N\) for \( \exsts{\nat \in N} \expr \pt \nat\) where \( N \subseteq \Val\).


%\subsection{Rules for Local}

The proof rules are standard except \rl{TRDeref} and \rl{TRMutate}.
The \rl{TRDeref} rule add read fingerprint in finger-tracking set, only if there is no write finger-print.
This is because once a location has been re-written, the rest read are considered as local operations, while the finger-print only records those operations might have effect on global state.

\begin{figure}[t]
\hrule\vspace{5pt}
\sx{Need to be careful about the implication, esp. for fingerprint assertion.}
\[
    \infer[\rl{TRSkip}]{%
        \tripleL{\assemp }{ \pskip }{\assemp }
    }{}
\]

\[
    \infer[\rl{TRAss}]{%
        \tripleL{\txvar \dot= \lexpr }{ \pass{\txvar}{\expr} }{\txvar \dot= \expr\sub{\txvar}{\lexpr} }
    }{%
    \txvar \notin \func{fv}[\lexpr]
        && \txvar \in \TxVars  
    }
\]

\[
    \infer[\rl{TRDeref}]{%
        \tripleL{\expr \pt \lexpr \sep \fpF }{ \pderef{\txvar}{\expr} }{\txvar \dot= \lexpr \sep \expr \pt \lexpr \sep \fpto{\fp'} }
    }{%
    \txvar \notin \func{fv}[\expr]
        && \txvar \notin \func{fv}[\lexpr]
        && \txvar \in \TxVars  
        && \fp' = \fp \addF (\etR, \expr, \lexpr)
    }
\]

\[
    \infer[\rl{TRMutate}]{%
        \tripleL{\expr_1 \pt \stub \sep \fpF }{ \pmutate{\expr_1}{\expr_2} }{ \expr_1 \pt \expr_2 \sep \fpto{\fp'} } 
    }{
        \fp' = \fp \addF (\etW, \expr_{1}, \expr_{2})
    }
\]

\[
    \infer[\rl{TRAssume}]{%
        \tripleL{ \expr \doteq 0 }{ \passume{\expr} }{ \expr \doteq 0 } 
    }{}
\]

\[
    \infer[\rl{TRChoice}]{%
        \tripleL{ \lpre }{ \trans_{1} \pchoice \trans_{2} }{ \lpost }
    }{%
        \tripleL{ \lpre }{ \trans_{1} }{ \lpost } && 
        \tripleL{ \lpre }{ \trans_{2} }{ \lpost } 
    }
\]

\[
    \infer[\rl{TRSeq}]{%
        \tripleL{ \lpre }{ \trans_{1} \pseq \trans_{2} }{ \lpost }
    }{%
        \tripleL{ \lpre }{ \trans_{1} }{ \lframe }  && 
        \tripleL{ \lframe }{ \trans_{2} }{ \lpost }
    }
\]

\sx{check the loop invariant with fingerprint}

\[
    \infer[\rl{TRLoop}]{%
        \tripleL{ \lpre }{ \trans\prepeat }{ \lpre }
    }{%
        \tripleL{ \lpre }{ \trans }{ \lpre } 
    }
\]
 
\[
   \infer[\rl{TRFrame}]{%
       \tripleL{ \lpre \sep \lframe }{ \trans }{ \lpost \sep \lframe }
   }{%
       \tripleL{ \lpre }{ \trans }{ \lpost } 
   }
\]
\hrule\vspace{5pt}
\caption{The rules for transactions}
\label{fig:rule-trans}
 \end{figure}

\subsection{Rely and Guarantee}

\begin{definition}[Rely and guarantee]
\label{def:rely-guarantee}
The \( \func{allowed} \) function asserts that a event is allowed by the owned capabilities,
\[
\begin{rclarray}
    \func{allowed}[\evset, \ca, \gs] & \defeq & \evset = \unitE \\
    \func{allowed}[\evset \composeE \evset', \ca, \gs \uplus \Set{\rid \mapsto (\stub, \stub, \intf)}] & \defeq & 
    \exsts{\kap } \kap \sqsubseteq \ca(\rid)
    \land \evset \in \dom(\intf(\kap)) 
    \land \func{allowed}[\evset', \ca, \gs]
\end{rclarray}
\]
Given the set of worlds $\World$ (\defref{def:world}), the \emph{update rely} relation, $\relyU \subseteq \World \times \World$, is defined as follows,
\[	
    \begin{rclarray}
	\relyU & \eqdef &
	\myset{
		((\ca, \gs), (\ca, \gs'))	
	}{
        \exsts{\evset,\aexec, \aexec'}  
        \func{allowed}[\evset, \ca, \gs]  \\
        \quad {} \land \aexec \in \clpsW{\gs} 
        \land \aexec' \in \clpsW{\gs'}
        \land (\aexec, \evset, \aexec') \in \func{transinv}[\gs]
	} \\
    \end{rclarray}
\]
The invariant of a shared state is a lift of the invariants of interferences of regions.
The \emph{rely} relation, $\RelyI \eqdef \World \times \World$, is defined as follows:
\[
    \begin{rclarray}
         \RelyI &\eqdef & \closure{\left(\relyU\right)} \\
    \end{rclarray}
\]
A set of fingerprint worlds $\setworld \subseteq \World$ is \emph{stable}, written $\stable{\setworld}$, if and only if it is closed under the rely relation: 
\[
    \begin{rclarray}
        \stable{\setworld} & \eqdef & \for{\w_{p}, \w_{q}}  \w_{p} \in \setworld \land (\w_{p}, \w_{q}) \in \RelyI \implies \w_{q} \in \setworld
    \end{rclarray}
\]
The \emph{update guarantee} relation, $\guarU: \World \times \World$, is defined as follows:
\[	
    \begin{rclarray}
	\guarU & \eqdef &
	\myset{
		((\ca, \gs), (\ca, \gs'))	
	}{
        \exsts{\evset, \ca', \aexec, \aexec'}  
        (\ca' \composeC \ca)\isdef
        \land \func{allowed}[\evset, \ca, \gs]  \\
        \quad {} \land \aexec \in \clpsW{\gs} 
        \land \aexec' \in \clpsW{\gs'}
        \land (\aexec, \evset, \aexec') \in \func{transinv}[\gs]
	} \\
    \end{rclarray}
\]
The \emph{guarantee} relation, $\GuarI \subseteq \World \times \World$, is defined as follows:
\sx{take away of the closure}
\[
	\GuarI \eqdef \guarU
\]
\end{definition}

\subsection{Rules for Global}

The \rl{PRCommit} rule lifts the local effect of transaction \( \trans \) to global level by first converting global state to (local) observable state and then propagating the local fingerprint to the global state.
The \( \predn{down} \) predicate asserts that the local predicate \( \lpre \) is a over-approximation of the valid observation that is given by the interference.
The \( \predn{up} \) predicate says the post-condition \( \gpost \) is the result by lifting the local fingerprints \( \fp \) to pre-condition \( \gpre \).



\begin{figure}[t!]
\hrule\vspace{5pt}

\sx{not right, we want to say a transaction \( \trans \) can be abstracted to a fingerprint \( \fp \)}

\[
    \infer[\rl{PRCommit}]{%
        \tripleG{\gpre}{ \ptrans{\trans} }{\gpost}
    }{%
        \begin{array}{c}
        \gpre \snap \lpre
        \quad \tripleL{\lpre \sep \fpEMP}{\trans}{\lpost \sep \fpF}
        \quad \rpt{\gpre}{\gpost}{\fp} \\
        \stable{\gpre} 
        \quad \stable{\gpost} 
        \end{array}
    }
\]

\[
    \infer[\rl{PRAss}]{%
        \tripleG{\var \dot= \lexpr }{ \pass{\var}{\expr} }{\var \dot= \expr\sub{\var}{\lexpr} }
    }{%
        \var \notin \func{fv}[\lexpr]
        && \var \in \ThdVars  
    }
\]

\[
    \infer[\rl{PRAssume}]{%
        \tripleG{ \expr \doteq 0 }{ \passume{\expr} }{ \expr \doteq 0 } 
    }{}
\]

\[
    \infer[\rl{PRChoice}]{%
        \tripleG{ \gpre }{ \prog_{1} \pchoice \prog_{2} }{ \gpost }
    }{%
        \tripleG{ \gpre }{ \prog_{1} }{ \gpost } && 
        \tripleG{ \gpre }{ \prog_{2} }{ \gpost } 
    }
\]

\[
    \infer[\rl{TRSeq}]{%
        \tripleG{ \gpre }{ \prog_{1} \pseq \prog_{2} }{ \gpost }
    }{%
        \tripleG{ \gpre }{ \prog_{1} }{ \gframe }  && 
        \tripleG{ \gframe }{ \prog_{2} }{ \gpost }
    }
\]

\[
    \infer[\rl{TRLoop}]{%
        \tripleG{ \gpre }{ \prog\prepeat }{ \gpre }
    }{%
        \tripleG{ \gpre }{ \prog }{ \gpre } 
    }
\]
 
\[
   \infer[\rl{TRFrame}]{%
       \tripleG{ \gpre \sep \gframe }{ \prog }{ \gpost \sep \gframe }
   }{%
       \tripleG{ \gpre }{ \prog }{ \gpost } 
   }
\]
 
\[
   \infer[\rl{TRPar}]{%
       \tripleG{ \gpre_{1} \sep \gpre_{2} }{ \prog_{1} \ppar \prog_{2} }{ \gpost_{1} \sep \gpost_{2} }
   }{%
       \tripleG{ \gpre_{1} }{ \prog_{1} }{ \gpost_{1} }
       && \tripleG{ \gpre_{2} }{ \prog_{2} }{ \gpost_{2} }
   }
\]

\[
\begin{rclarray}
    \gpre \snap \lpre & \defeq & \for{ \w, \h } \w \in \evalW{\gpre} \land \pred{takeinv}{\w, \h} \implies \h \in \evalLS{\lpre}\\
    \pred{takeinv}{\gs, \h} & \defeq & \gs = \emptyset \land \h = \unitH \\
    \pred{takeinv}{\Set{\rid \mapsto (\stub, \hset, \intf)} \uplus \gs, \h \composeH \h'} & \defeq & \h \in \hset \land \pred{takeinv}{\gs,\h'}\\
    \rpt{\gpre}{\gpost}{\fp} & \defeq & 
    \begin{array}[t]{@{}l@{}}
        \for{\w, \w', \aexec, \aexec'} \\
        \quad \w \in \evalW{\gpre}
        \land \aexec \in \clpsW{\w}
        \land \func{reachable}{\aexec, \aexec', \stub, \evalF{\fp}, \como}  \\
        \quad {} \land \aexec' \in \clpsW{\w'}
        \land (\w, \w') \in \Guar 
        \implies \w' \in \evalW{\gpost}
    \end{array} \\
\end{rclarray}                          
\]

\sx{
    For LLW the propagate can be done by syntactically update those addresses that has been written.
}

\hrule\vspace{5pt}
\caption{The rules for programs}
\label{fig:rule-prog}
\end{figure}

Many consistency model use last write win resolution policy, such as snapshot isolation, therefore the repartition \( \rpt{\gpre}{\gpost}{\fp} \) can be simplified by checking the guarantee and then syntactically propagating the write fingerprints.
Also in practice, many implementation of consistency model assume strong session constraint.

\begin{figure}
\hrule\vspace{5pt}

\[
   \infer[\rl{FRead}]{%
       \tripleF{ \lexpr \pt \lexpr' \mid \lexpr \pt \lexpr'' }{ \Set{(\etR, \lexpr, \lexpr'')} }{ \lexpr \pt \lexpr' \mid \lexpr \pt \lexpr''}
   }{}
\]

\[
   \infer[\rl{FWriteNS}]{%
       \tripleF{ \lexpr \pt \lexpr' \mid \lexpr \pt \lexpr'' }{ \Set{(\etW, \lexpr, \lexpr''')} }{ \lexpr \pt \lexpr''' \mid ( \lexpr \pt \lexpr'' \lor \lexpr \pt \lexpr''') }
   }{}
\]

\[
   \infer[\rl{FWriteS}]{%
       \tripleF{ \lexpr \pt \lexpr' \mid \lexpr \pt \lexpr'' }{ \Set{(\etW, \lexpr, \lexpr''')} }{ \lexpr \pt \lexpr''' \mid \lexpr \pt \lexpr'''}
   }{}
\]

\[
   \infer[\rl{FFrame}]{%
       \tripleF{ \lpre \sep \lframe  \mid \lpre' \sep \lframe' }{ \fp }{ \lpost \sep \lframe \mid \lpost' \sep \lframe' }
   }{%
       \tripleF{ \lpre \mid \lpre' }{ \fp }{ \lpost \mid \lpost' }
   }
\]

\[
   \infer[\rl{FContinue}]{%
       \tripleF{ \lpre \sep \lframe  \mid \lpre' \sep \lframe' }{ \fp  \uplus \fp' }{ \lpost \sep \lframe \mid \lpost' \sep \lframe' }
   }{%
       \tripleF{ \lpre \sep \lframe  \mid \lpre' \sep \lframe' }{ \fp }{ \lpost \sep \lframe \mid \lpost' \sep \lframe' }
   }
\]

\[
\begin{rclarray}
    \rpt{\gpre}{\gpost}{\fp} & \defeq & 
    \begin{array}[t]{@{}l}
    \for{\w, \w', \h, \h', \lpre, \lpre', \lpost, \lpost'}
    \w \in \evalW{\gpre} \\
    \quad {} \land \pred{unbox}{\w\projection{2}, \h}
    \land \h \in \evalLS{\lpre} 
    \land \gpre \snap \lpre' \\
    \quad {} \land {} \tripleF{ \lpre \mid \lpre' }{ \fp }{ \lpost | \lpost'} \\
    \quad {} \land \pred{unbox}{\w'\projection{2},\h'}
    \land \h' \in \evalLS{\lpost} 
    \land \gpost \snap \lpost' \\
    \quad \implies \w' \in \evalW{\gpost}
    \end{array} \\
    \pred{unbox}{\gs, \h} & \defeq & \gs = \emptyset \land \h = \unitH \\
    \pred{unbox}{\Set{\rid \mapsto (\h, \stub, \stub)} \uplus \gs, \h \composeH \h' } & \defeq & \pred{unbox}{\gs, \h'} \\
\end{rclarray}
\]

\hrule\vspace{5pt}
\caption{Simplified repartition for last write win}
\label{fig:rule-prog}
\end{figure}


