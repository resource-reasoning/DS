\section{The Soundness and Completeness of Execution Tests}
\label{app:et_sound_complete}
\label{app:et-sound-complete}
\label{sec:kv-sound-complete-proof}
We now show using \cref{def:et_sound,def:et_complete} to prove the soundness and completeness of execution tests with respect to axiomatic definitions.
It is sufficient to match these two definition, 
then by \cref{cor:et-soundness,cor:et-completeness} we have \( \CMs(\ET) = \Set{\mkvs_\aexec}[\aexec \in \CMa(\RP_{\LWW},\Ax)] \).
\label{sec:spec-proof}

\begin{theorem}[View closure to visibility closure]
    \label{thm:view-vis-relation}
    Assume \( \mkvs \) and \( \aexec \) such that \( \mkvs = \mkvs_\aexec \), 
    and \( \rel_\mkvs \) and \( \rel_\aexec \) such that \( \rel_\mkvs = \rel_\aexec \),
    Assume \(\txid, \fp \).
    If there is a view \( \vi = \getView[\mkvs,\lfpTx[\mkvs,\vi,\rel]] \),
    then there a new abstract execution \( \aexec' = \extend[\aexec, \txid, \fp, \Tx[\mkvs, \vi] \cup \txidset_\rd] \),
    that satisfies \( \rel_\aexec^{-1}(\VIS_{\aexec'}^{-1}(\txid)) \subseteq \VIS_{\aexec'}^{-1}(\txid) \),
    for some read only transactions \( \txidset_\rd \).
    Similarly,
    If there a new abstract execution \( \aexec' = \extend[\aexec, \txid, \fp, \Tx[\mkvs, \vi] \cup \txidset_\rd] \),
    that satisfies \( \rel_\aexec^{-1}(\VIS_{\aexec'}^{-1}(\txid)) \subseteq \VIS_{\aexec'}^{-1}(\txid) \),
    for some read only transactions \( \txidset_\rd \),
    then the view \( \vi = \getView[\mkvs,\lfpTx[\mkvs,\vi,\rel]] \).
\end{theorem}
\begin{proof}
    Assume \( \mkvs \) and \( \aexec \) such that \( \mkvs = \mkvs_\aexec \), 
    and \( \rel_\mkvs \) and \( \rel_\aexec \) such that \( \rel_\mkvs = \rel_\aexec \),
    Assume \(\txid, \fp \).
    Assume that a view satisfies \( \vi = \getView[\mkvs,\lfpTx[\mkvs,\vi,\rel_\mkvs]] \),
    where \( \lfpTx[\mkvs,\vi, \rel] \defeq \mu X . \Tx[\mkvs,\vi] \cup \rel^{-1}(X) \).
    We pick
    \begin{centermultline}
        \txidset_\rd = \lfpTx[\mkvs,\vi,\rel] \setminus \Tx[\mkvs,\vi]
    \end{centermultline}
    By the definition of \( \extend \),  the visible transactions 
    \( \VIS_{\aexec'}^{-1}(\txid) = \Tx[\mkvs, \vi] \cup \txidset_\rd \).
    Let consider transactions \( \txid', \txid'' \) such that \( \txid' \toEDGE{R_{\aexec}} \txid'' \toEDGE{\VIS_{\aexec'}} \txid \).
    \begin{itemize}
        \item Assume \( \txid'' \in \Tx[\mkvs, \vi] \).
        Then \( \txid'' \in \Tx[\mkvs, \getView[\mkvs,\lfpTx[\mkvs,\vi,\rel_\mkvs]]]\).
        By the definition of \( \Tx \) and \( \getView \), we have \(  \txid'' \in \lfpTx[\mkvs,\vi,\rel] \).
        This means \( \txid'' \in \mu X . \Tx[\mkvs,\vi] \cup \rel_{\mkvs}^{-1}(X) \).
        Therefore \( \txid' \in \mu X . \Tx[\mkvs,\vi] \cup \rel_{\mkvs}^{-1}(X) \).
        If \( \txid' \) has write then \( \txid \in \Tx[\mkvs,\vi] \), otherwise \( \txid' \in \txidset_\rd \).
        
        \item Assume \( \txid'' \in \txidset_\rd \).
        Then \( \txid'' \in \lfpTx[\mkvs,\vi,\rel] = \mu X . \Tx[\mkvs,\vi] \cup \rel_{\mkvs}^{-1}(X) \).
        Therefore \( \txid' \in \mu X . \Tx[\mkvs,\vi] \cup \rel_{\mkvs}^{-1}(X) \).
        If \( \txid' \) has write then \( \txid \in \Tx[\mkvs,\vi] \), otherwise \( \txid' \in \txidset_\rd \).
    \end{itemize}

    Assume there a new abstract execution \( \aexec' = \extend[\aexec, \txid, \fp, \Tx[\mkvs, \vi] \cup \txidset_\rd] \),
    that satisfies \( \rel_\aexec^{-1}(\VIS_{\aexec'}^{-1}(\txid)) \subseteq \VIS_{\aexec'}^{-1}(\txid) \),
    for some read only transactions \( \txidset_\rd \).
    That is,
    \begin{centermultline}
        \rel_\aexec^{-1}(\Tx[\mkvs, \vi] \cup \txidset_\rd) \subseteq \Tx[\mkvs, \vi] \cup \txidset_\rd
    \end{centermultline}
    This means \( \Tx[\mkvs, \vi] \cup \txidset_\rd = \mu X. \Tx[\mkvs, \vi] \cup \txidset_\rd \cup \rel_\mkvs^{-1}(X)\).
    We now prove 
    \begin{centermultline}
        \getView[\mkvs, \Tx[\mkvs, \vi]] = \getView[\mkvs, \mu X. \Tx[\mkvs, \vi]  \cup \rel_\mkvs^{-1}(X)]
    \end{centermultline}
    It is easy to see for any \( \key,i \), if \( i \in \getView[\mkvs, \Tx[\mkvs, \vi]](\key) \) 
    then \( i \in \getView[\mkvs, \mu X. \Tx[\mkvs, \vi]  \cup \rel_\mkvs^{-1}(X)](\key) \).
    Let assume \( i \in \getView[\mkvs, \mu X. \Tx[\mkvs, \vi]  \cup \rel_\mkvs^{-1}(X)](\key) \).
    Let \( \txid'  = \wtOf[\mkvs(\key,i)] \).
    Let assume \( \txid' \notin \Tx(\mkvs,\vi) \), then \(\txid' \notin \Tx(\mkvs,\vi) \cup \txidset_\rd \) 
    since \( \txidset_\rd \) are read only transactions.
    It means \( \txid' \notin \mu X. \Tx[\mkvs, \vi] \cup \txidset_\rd \cup \rel_\mkvs^{-1}(X) \).
    Now \( i \notin \getView[\mkvs, \mu X. \Tx[\mkvs, \vi]  \cup \rel_\mkvs^{-1}(X)](\key) \), which contradict with our assumption.
    Thus it must be the case \( \txid' \in \Tx[\mkvs,\vi]\).
    By the definition of \( \getView \) and \( \Tx \), we have
    \begin{centermultline}
        \vi = \getView[\mkvs, \Tx[\mkvs, \vi]] = \getView[\mkvs, \mu X. \Tx[\mkvs, \vi]  \cup \rel_\mkvs^{-1}(X)]
    \end{centermultline}
\end{proof}


\emph{Monotonic read} (\( \MR \)) \citep{session-guarantee,repldatatypes} states that after committing a transaction, 
a client cannot lose information in that 
it can only see increasingly more versions from a kv-store.
This prevents, for example, the kv-store in \cref{fig:mr-disallowed},
since client \(\cl\) first reads the latest version of \(\key\) in \(\txid_{\cl}^{1}\), 
and then reads the older, initial version of \(\key\) in \(\txid_{\cl}^{2}\).  
As such, the \(\ViewShift[\MR]\) predicate in \cref{fig:execution-tests} 
ensures that clients can only extend their views,
that is, \( \vi \vileq \vi' \) for views \( \vi, \vi'\) before and after committing.
When this is the case, clients can then \emph{always} commit their transactions,
and thus \(\CanCommit[\MR]\) is simply defined as \(\true\). 

\subsection{Monotonic Write \( \MW \)}
\label{sec:sound-complete-mw}

The execution test $\ET_\MW$ is sound with respect to the axiomatic specification 
$(\RP_{\LWW}, \Set{\lambda \aexec. \PO_{\aexec} ; \VIS_{\aexec} })$.
We pick the invariant as empty set given the fact of no constraint on the view after update:
\[ 
    I( \aexec, \cl ) = \emptyset 
\]
Assume a key-value store $\hh$, an initial and a final view $\vi, \vi'$  a fingerprint $\opset$ 
such that $\ET_{\MW} \vdash (\hh, \vi) \csat \opset: (\hh',\vi')$. 
Also choose an arbitrary $\cl$, a transaction identifier $\txid \in \nextTxId(\hh, \cl)$, 
and an abstract execution $\aexec$ such that $\hh_{\aexec} = \hh$ and 
\( I(\aexec, \cl) =  \emptyset \subseteq \Tx(\hh, \vi) \).
Let \( \aexec' = \extend(\aexec, \txid, \Tx(\mkvs, \vi) \cup \T_\rd, \f ) \).
Note that since the invariant  is empty set, it remains to prove that there exists a set of read-only transactions \( \T_\rd \) such that:
\[
    \begin{array}{@{}l@{}}
        \fora{ \txid' }  (\txid' ,\txid)  \in \PO_{\aexec'} ; \VIS_{\aexec'}
        \implies \txid' \in \Tx(\mkvs, \vi) \cup \T_\rd
    \end{array}
\]
Initially we take \( \T_\rd = \emptyset \), 
and by closing the \( \Tx(\mkvs, \vi) \) with respect to the relation \( \PO_{\aexec'} ; \VIS_{\aexec'} \),
we will add more read-only transactions into the set \( \T_\rd\).
Suppose \( (\txid' ,\txid)  \in \PO_{\aexec'} ; \VIS_{\aexec'} \), 
that is, \( \txid' \toEdge{\SO_{\aexec'}} \txid'' \toEdge{\VIS_{\aexec'}} \txid \).
We perform a case analysis on if \( \txid'' \) has write:
\begin{itemize}
\item If the transaction \( \txid'' \) writes to a key.
For the new abstract execution \( \aexec' \), the visible transactions for \( \txid \) must come from \( \Tx(\mkvs, \vi) \cup \T_\rd \).
It means \( \txid'' \in \Tx(\mkvs, \vi) \cup \T_\rd  \).
Then given that \( \txid'' \) is not a read-only transaction, we have \( \txid'' \in \Tx(\mkvs, \vi) \).
Now there are two cases:
\begin{itemize}
    \item if \( \txid' \) is a read-only transaction, we include \( \txid' \in \T_{\rd} \).
    \item if \( \txid' \) has at least one write, it is easy to see \( \txid' \in \Tx(\mkvs, \vi) \) since \( j \in \vi(\ke) \wedge \WTx(\hh(\ke', i)) \xrightarrow{\PO?} \WTx(\hh(\ke, j)) \implies i \in \vi(\ke') \).
\end{itemize}
\item If the transaction \( \txid'' \in \T_\rd \) is a read-only transaction, 
since \( \T_\rd \) is initial empty, there must exist a later transaction \( \txid''' \) from the same client that writes to a key,
and such transaction \( \txid''' \) is included in \( \Tx(\mkvs, \vi) \):
\[
    \txid' \toEdge{\SO_{\aexec'}} \txid'' 
    \toEdge{\SO_{\aexec'}} \txid''' \toEdge{\VIS_{\aexec'}} \txid 
    \land \txid''' \in \Tx(\mkvs,\vi)
\]
Since \( \SO \) is transitive, 
therefore \( \txid' \toEdge{\SO_{\aexec'}} \txid''' \toEdge{\VIS_{\aexec'}} \txid \),
which we have already proven \( \txid' \in \Tx(\mkvs, \vi) \) or we will include \( \txid' \) in \( \T_\rd \).
Since there are finite transactions from a client in a trace, there must exist a \( \T_\rd \) in the end.
\end{itemize}


The execution test $\ET_{\MW}$ is complete with respect to 
the axiomatic specification $(\RP_{\LWW}, \Set{\lambda \aexec.(\PO_{\aexec} ; \VIS_{\aexec})})$. 
Let $\aexec$ be an abstract execution that satisfies the specification
$\CMa(\RP_{\LWW}, \Set{\lambda \aexec.(\PO_{\aexec} ; \VIS_{\aexec})})$, 
and consider a transaction $\txid \in \T_{\aexec}$. 
Assume i-\emph{th} transaction \( \txid_i \) in the arbitrary order,
and let view \( \vi_{i} = \getView(\aexec, \VIS^{-1}_{\aexec}(\txid_{i}) ) \).
We also pick any final view such that \( \vi'_{i} \subseteq \getView(\aexec, (\AR^{-1}_{\aexec})?(\txid_{i}) ) \).
It suffices to prove \( \ET_\MW \vdash (\hh_{\cut(\aexec, i-1)}, \vi_i ) \csat  \TtoOp{T}_{\aexec}(\txid_{i}) : (\hh_{\cut(\aexec, i-1)}, \vi'_{i}) \).
It means to prove the follows:
\begin{equation}
\label{equ:mw-complete}
\begin{array}{@{}l@{}}
    \fora{j,m,\ke, \ke' } j \in \vi(\ke)  
    \wedge \WTx(\hh_{\cut(\aexec, i-1)}(\ke', m)) \xrightarrow{\PO?} \WTx(\hh_{\cut(\aexec, i-1)}(\ke, j))  
    \implies m \in \vi(\ke')
\end{array}
\end{equation}
Assume \( j \) and \( \ke' \) such that \( j \in \vi(\ke')\), which means \( \WTx(\hh_{\cut(\aexec, i-1)}(\ke', j)) \in \VIS^{-1}_{\aexec}(\txid_{i}) \).
Now let consider transaction \( \txid \) that commits before \( \txid \) from the same session, \ie \( \txid \toEdge{\SO} \WTx(\hh_{\cut(\aexec, i-1)}(\ke, j)) \).
By the constraint \( \lambda \aexec.(\PO_{\aexec} ; \VIS_{\aexec}) \), the transaction \( \txid \in \VIS^{-1}_{\aexec}(\txid_{i}) \).
It means that in the kv-store \(  \hh_{\cut(\aexec, i-1)} \) every version written by \( \txid =  \WTx(\hh_{\cut(\aexec, i-1)}(\ke', m)) \) should be included in the view \( m \in \vi_i(\ke') \).
Thus we have the proof of \cref{equ:mw-complete}.

The execution test \(\et[\RYW]\) is sound with respect to the axiomatic definition
\(\visaxioms[\RYW] = \Set{\lambda \aexec \ldotp \SO_{\aexec} }\) \cite{repldatatypes}.
We pick the following invariant:
\[
    \aexecinv[\RYW](\aexec, \cl) \FuncDef
    \begin{multlined}[t]
    \left( \bigcup_{\Set{\txid[\cl](n) \in \aexec }} \Refl((\Inv(\SO)))(\txid[\cl](n)) \right) 
    \setminus \Set{\txidrd | \txidrd in \aexec \land \Forall{l | \key | \val} (l,\key,\val) \in \aexec(\txid) \implies l = \opR } .
    \end{multlined}
\]

\SOUNDLET{\RYW}{
    \txidsetrd = 
    \left( \bigcup_{\Set{\txid[\cl](n) \in \aexec }} \Refl((\Inv(\SO)))(\txid[\cl](n)) \right) 
    \cap \Set{\txidrd | \txidrd in \aexec \land \Forall{l | \key | \val} (l,\key,\val) \in \aexec(\txid) \implies l = \opR } .
}
\begin{enumerate}
\Case{\(\Forall{\visaxiom \in \visaxioms }
            \Inv(\visaxiom(\aexec'))(\txid) \subseteq \txidset \cup \txidsetrd \)}
    Suppose transactions \( \txid, \txid' \) such that \( \txid,\txid' \in \aexec \) and \( (\txid',\txid) \in \SO \).
    If \( \txid' \) is a read-only transaction, \( \txid' \in \txidsetrd \).
    Otherwise, \( \txid' \) has write, by the definition of \( \aexecinv[\RYW] \), 
    it follows that \( \txid' \in \aexecinv[\RYW](\aexec,\cl) \)
    and therefore \( \txid' \in \txidset \).
\Case{\(\aexecinv(\aexec',\cl) \subseteq \VisTrans(\XToK(\aexec'),\vi') \)}
    Because \( \ToET[\RYW]{\kvs | \vi | \fp | \kvs' | \vi' }\),
    it must be the case that
    \[
        \Forall{\key \in \Keys | \idx \in \Indexs } (\WtOf(\kvs'(\key,\idx)),\txid) \in \Refl(\SO) \implies \idx \in \vi'(\key) 
    \]
    and therefore
    \[
        \Forall{\txid'} \Exists{\key \in \Keys | \val \in \Values} \opW(\key,\val) \in \aexec'(\txid')
        \land (\txid',\txid) \in \SO \implies \txid' \in \VisTrans(\XToK(\aexec'),\vi') .
    \]
    Note that \( \bigcup_{\Set{\txid[\cl](n) \in \aexec }} \Refl((\Inv(\SO)))(\txid[\cl](n)) = \Refl((\Inv(\SO)))(\txid) \).
    Last, we have
    \begin{align*}
        \aexecinv(\aexec',\cl) & = 
            \begin{multlined}[t]
            \left( \bigcup_{\Set{\txid[\cl](n) \in \aexec }} \Inv(\SO)(\txid[\cl](n)) \right) 
            \setminus \Set{\txidrd | \txidrd \in \aexec 
                    \land \Forall{l | \key | \val} (l,\key,\val) \in \aexec(\txid) \implies l = \opR } 
            \end{multlined}
            \\ & = \begin{multlined}[t]
            \left( \Refl((\Inv(\SO)))(\txid) \right) 
            \setminus \Set{\txidrd | \txidrd \in \aexec 
                    \land \Forall{l | \key | \val} (l,\key,\val) \in \aexec(\txid) \implies l = \opR } 
            \end{multlined}
            \\ & \subseteq \VisTrans(\XToK(\aexec'),\vi') 
    \end{align*}
\end{enumerate}

\COMPLETELET{\RYW}
We construct the final view \( \vi'\) depending on whether \( \txid[\cl](n) \) is the last transaction for the client \( \cl \).
\begin{enumerate}
\Case{\( (\txid[\cl](n), \txid') \in \SO \) for \( \txid' \in \aexec \)}
    Let the transaction 
    \( \txid = \Min[\SO](\Set{ \txid' | (\txid[\cl](n), \txid') \in \SO \land \txid' \in \aexec' }) \).
    For this case, let view 
    \( \vi' = \GetView(\aexec, \Refl((\ARInv[\aexec]))(\txid[\cl](\idx)) \cap \VISInv[\aexec](\txid)) \).
    By \( \visaxioms[\RYW] \), it follows that, for any transaction \( \txid' \),
    if \( ( \txid',\txid[\cl](idx) ) \in \Refl(\SO) \), then
    \( \txid' \in \VISInv[\aexec](\txid)) \).
    Since \( \SO \in \AR \), we know that 
    \( \txid' \in \Refl((\ARInv[\aexec]))(\txid[\cl](\idx)) \cap \VISInv[\aexec](\txid)) \).
    Therefore, for any version \( \kvs'(\key,j)\) such that 
    \( ( \WtOf(\kvs'(\key,j)), \txid) \in \Refl(\SO) \),
    then \( j \in \vi'(\key)\).
\Case{\( \neg \left((\txid[\cl](n), \txid') \in \SO \right) \)}
    For this case, let 
    \( \vi' = \GetView(\aexec, \Refl((\ARInv[\aexec]))(\txid[\cl](\idx))) \) be the final view.
    It is easy to see that \( \vi' \) satisfies \( \RYW \). 
\end{enumerate}

The execution test \(\et[\WFR]\) is sound with respect to the axiomatic definition 
\(\visaxioms[\WFR] \Set{\lambda \aexec \ldotp \WR[\aexec] ; \Refl((\SO \cap \RW[\aexec] )) ; \VIS[\aexec] })\) 
\citep{surech-session-guarantee}.
By picking the invariant as \( I( \aexec, \cl ) = \emptyset \), the soundness and completeness
can be derived from \cref{thm:view-vis-relation} in a similar way as the proofs for \( \MW \).

The wildly used definition on abstract executions for causal consistency is that 
\( \VIS \) is transitive and \( \SO \in \VIS \).
Yet it is for the sack of elegant definition,
while there is a equivalent minimum visibility relation (\cref{thm:cc-visaxioms}) defined by 
\( \visaxioms[\CC] \FuncDef \Set{ \lambda \aexec \ldotp (\WR[\aexec] \cup \SO) ; \VIS[\aexec] \subseteq \VIS[\aexec] , 
                                    \lambda \aexec \ldotp \SO \subseteq \VIS[\aexec]} \),
where \( \WR[\aexec] \) is defined in \cref{def:aexec-dgraph}.

\begin{theorem}[Minimum visibility relation for (\texorpdfstring{\CC}{\texttt{CC}})]
\label{thm:cc-visaxioms}
For two abstract executions \( \aexec,\aexec' \),
the following constrain on visibility,
\begin{Formulae}
\begin{Formula}
    (\WR[\aexec] \cup \SO) ; \VIS[\aexec] \subseteq \VIS[\aexec] \land \SO \subseteq \VIS[\aexec]
    \label{equ:kvstore-cc-spec}
\end{Formula}
\end{Formulae}
is equivalent to
\begin{Formulae}
\begin{Formula}
    \VIS[\aexec'] ; \VIS[\aexec'] \subseteq \VIS[\aexec'] \land \SO \subseteq \VIS[\aexec']
    \label{equ:aexec-cc-spec}
\end{Formula}
\end{Formulae}
in that 
\(
    \Forall{\txid \in \TxIDs | \fp } \left( \fp = \aexec(\txid) \iff \fp = \aexec'(\txid) \right)
    \land \AR[\aexec] = \AR[\aexec'] .
\)
\end{theorem}
\begin{proof}
For an abstract execution \( \aexec \) that satisfies \cref{equ:kvstore-cc-spec},
by \cref{lem:aexec-spec-cc}, there exists \( \aexec' \) that satisfies \cref{equ:aexec-cc-spec}.
Assume an abstract execution \( \aexec' \) that satisfies \cref{equ:aexec-cc-spec}.
Since \( \WR[\aexec'] \subseteq \VIS[\aexec']\) by the definition of \( \WR[\aexec']\),
thus \( \aexec' \) satisfies \cref{equ:kvstore-cc-spec}.
\end{proof}

\begin{toappendix}
\begin{lemma}[Minimum visibility relation for (\texorpdfstring{\CC}{\texttt{CC}})]
\label{lem:aexec-spec-cc}
For any abstract execution \( \aexec \), if it satisfies \( \visaxioms[\CC] \),
there exists a new abstract execution \( \aexec' \) such that \( \SO \in \VIS[\aexec]\) and
\begin{Formulae}
\begin{Formula}
    \Forall{\txid \in \TxIDs | \fp } \left( \fp = \aexec(\txid) \iff \fp = \aexec'(\txid) \right)
    \land \AR[\aexec] = \AR[\aexec'] \land \VIS[\aexec'] ; \VIS[\aexec'] \subseteq \VIS[\aexec'] .
    \label{equ:aexec-spec-cc}
\end{Formula}
\end{Formulae}
\end{lemma}
\begin{proof}
We erase some visibility relation for each transaction following 
the arbitration order \( \AR \) until the visibility is transitive.
Intuitively, the final visibility relation is exactly \( \Trasi((\WR[\aexec] \cup \SO)) \).
Assume the \Th{\idx} transaction \( \txid_\idx \)  with respect to the arbitration order.
Let \( \rel[\idx] \) be a new visibility for the transaction \( \txid_\idx \) such that
\( {\rel[\idx]}\Proj{2} = \Set{\txid_\idx}\) for all indexes \( \idx \)
and the union of visibility relations \( \bigcup_{0 \leq j \leq \idx } \rel[\idx] \) is transitive.
We preserve that, for each index \( \idx \), cut of abstract execution \( \aexec' =  \AexecCut(\aexec, \idx) \)
and visibility relation \( \VIS' = \bigcup_{0 \leq j \leq \idx } \rel[j] \),
the following invariant holds:
\begin{Formulae}
& \begin{Formula} 
    \VIS' ; \VIS' \subseteq \VIS'  ,
    \label{equ:cc-vis-idx-transitive} 
\end{Formula}
\\ & \begin{Formula}
    \Forall{ \txid \in \aexec } (\txid,\txid_i) \in \rel[\idx] \implies (\txid, \txid_i) \in (\WR[\aexec'] \cup \SO) .
    \label{equ:cc-vis-idx-minimum}
\end{Formula}
\end{Formulae}
We prove the above by induction on the number \( \idx \).
\begin{enumerate}
\CaseBase{\( \idx = 0 \)}
    By the definition of \( \AexecCut \), we know that \(\aexecinit = \AexecCut(\aexec,0) \)
    and \cref{equ:cc-vis-idx-transitive,equ:cc-vis-idx-minimum} trivially hold.
\CaseInd{\( \idx > 0 \)}
    Suppose that, for the \Th{(\idx-1)} step,
    the abstract execution \( \aexec'' =  \AexecCut(\aexec, \idx - 1) \)
    and the visibility relation \( \VIS'' = \bigcup_{0 \leq j \leq \idx-1 } \rel[j] \) 
    satisfy \cref{equ:cc-vis-idx-transitive,equ:cc-vis-idx-minimum}.
    Let consider \Th{\idx} step, the transaction \( \txid_i \),
    the cut \( \aexec' =  \AexecCut(\aexec, \idx) \)
    and the visibility relation \( \VIS' = \bigcup_{0 \leq j \leq \idx } \rel[j] \).
    Initially we take \( \rel \) as an empty set.
    First, we include \( \Set{(\txid,\txid_i) | (\txid,\txid_i) \in \WR[\aexec]} \) to \( \rel \)
    and, by the definition of \( \WR[\aexec]\), 
    it trivially does not affect any read operation for the transaction \( \txid_i \).
    Then we do the same for \( \SO \) as that 
    we include \( \Set{(\txid,\txid_i) | (\txid,\txid_i) \in \SO} \) to \( \rel \).
    Note that \( \SO \) cannot affect any read operation for the transaction \( \txid_i \) neither,
    otherwise it contradicts to that \( \SO \subseteq \VIS[\aexec] \) and the definition of \( \WR[\aexec] \).

    For relations \( \rel' = \rel ; \bigcup_{0 \leq j \leq \idx-1 } \rel[j] \) and then \( \rel[\idx] = \rel \cup \rel' \),
    it easy to see that \( \rel \in \VIS[\aexec]\) and, then by \ih, \( \rel' \in \VIS[\aexec] \).
    We prove that the \( \rel[\idx] \) does not affect any read operation for the transaction \( \txid_i \)
    by contradiction.
    Assume distinct transactions \( \txid,\txid' \) such that
    \( \ToEdge{\txid'' | \rel \cup \rel' -> \txid' | \rel \cup \rel' -> \txid_i } \),
    and immediately  by the definition of \( \rel \) and \( \rel' \),
    then \( \ToEdge{\txid'' | \rel' -> \txid' | \rel -> \txid_i } \).
    Assume that \( \txid'' \) change the read operation for a key \( \key \) in \( \txid_i \).
    This means that there exists a transaction \( \txid^* \) such that
    \( (\txid^*,\txid_i) \in \WR[\aexec](\key)\) and \( (\txid^*,\txid'') \in \AR[\aexec] \),
    where the latter implies that \( (\txid'',\txid_i) \in \WR[\aexec](\key) \);
    there is a contradiction and thus 
    \( \rel[\idx] \) does not affect any read operation for the transaction \( \txid_i \).

    We now prove that \cref{equ:cc-vis-idx-transitive,equ:cc-vis-idx-minimum} still hold.
    \begin{enumerate}
    \Case{\cref{equ:cc-vis-idx-transitive}}
        Assume a relation \( \rel^* = \bigcup_{0 \leq j \leq \idx-1 } \rel[j] \) 
        and transactions \( \txid, \txid',\txid'' \) such that 
        \[
            \ToEdge{\txid | \rel^* \cup \rel[\idx] -> \txid' | \rel^* \cup \rel[\idx] -> \txid'' } .
        \]
        If \( \ToEdge{\txid | \rel^*  -> \txid' | \rel^*  -> \txid'' } \), 
        then by \ih, \( \ToEdge{\txid | \rel^*  -> \txid'' } \).
        Note that \( \ToEdge{\txid | \rel[\idx]  -> \txid' | \rel^*  -> \txid'' } \) cannot happen,
        because it contradict to that \( \txid' = \txid_i\) and \( (\txid'',\txid_i) \in \AR[\aexec] \).
        Thus consider \( \ToEdge{\txid | \rel^*  -> \txid' | \rel[\idx]  -> \txid'' } \).
        It must be the case that \( \txid'' = \txid_i \) and by the definition of \( \rel[\idx] \),
        we know that \( \ToEdge{\txid | \rel[\idx]  -> \txid'' } \).
    \Case{\cref{equ:cc-vis-idx-minimum}}
        By the construction, \cref{equ:cc-vis-idx-minimum} hold. \qedhere
    \end{enumerate}
\end{enumerate}
\end{proof}
\end{toappendix}

We pick the invariant as \( \aexecinv[\CC] = \aexecinv[\MR] \cup \aexecinv[\RYW]  \).
\SOUNDLET{\CC}{ \txidsetrd \supseteq
\begin{multlined}[t]
\left( \bigcup_{\Set{\txid[\cl](\idx) | \txid[\cl](\idx) \in \aexec}} 
\VISInv[\aexec](\txid[\cl](\idx)) \cup \Refl((\Inv(\SO)))(\txid[\cl](\idx)) \right) 
\setminus \Set{\txid' | \Forall{l | \key | \val } (l,\key,\val) \in \aexec(\txid') \implies l = \opR } .
\end{multlined} }
Assume 
\[ 
\txidsetrd' = 
\begin{multlined}[t]
\left( \bigcup_{\Set{\txid[\cl](\idx) | \txid[\cl](\idx) \in \aexec}} 
\VISInv[\aexec](\txid[\cl](\idx)) \cup \Refl((\Inv(\SO)))(\txid[\cl](\idx)) \right) 
\setminus \Set{\txid' | \Forall{l | \key | \val } (l,\key,\val) \in \aexec(\txid') \implies l = \opR } .
\end{multlined} 
\]
and \( \txidsetrd'' = \txidsetrd \setminus \txidsetrd' \).
By the definition of soundness, we prove the following result
\begin{Formulae}
& \begin{Formula}
\Inv(\SO)(\txid) \subseteq \txidset \cup \txidsetrd'
\label{equ:cc-so-vis}
\end{Formula}
\\ & \begin{Formula}
\Inv((( \WR[\aexec'] \cup \SO ) ; \VIS[\aexec'] )) (\txid) \subseteq \txidset \cup \txidsetrd' \cup \txidsetrd''
\label{equ:cc-vis-transitive}
\end{Formula}
\\ & \begin{Formula}
\aexecinv[\CC](\aexec',\cl) \subseteq \VisTrans(\XToK(\aexec'),\vi')
\label{equ:cc-inv-preserve}
\end{Formula}
\end{Formulae}
\Cref{equ:cc-so-vis} can be proven in the same way as in \cref{sec:sound-complete-mr}
We now prove \cref{equ:cc-vis-transitive}.
Initially we take \( \txidsetrd'' \) to be an empty set.
Note that \(\VISInv[\aexec'](\txid) = \txidset \cup \txidsetrd' \cup \txidsetrd'' \).
By \cref{thm:view-vis-relation,equ:view-close-to-aexec}, there exists \( \txidsetrd'' \) such that
\( \txidset \cup \txidsetrd'' \) is closed under \( \WR[\aexec'] \cup \SO \).
Now consider a transaction \( \txidrd \in \txidsetrd' \) and
assume a transaction \( \txid' \) such that \( \ToEdge{ \txid' | \WR[\aexec'] \cup \SO -> \txidrd } \).
There are two cases depending on \( \txidrd \).
\begin{enumerate}
\Case{\( \ToEdge{\txidrd | \VIS[\aexec'] -> \txid'' | \SO -> \txid} \) for some \( \txid'' \)}
    For this case, we have
    \begin{align*}
    \ToEdge{\txid' | \WR[\aexec'] \cup \SO -> \txidrd | \VIS[\aexec'] -> \txid'' | \SO -> \txid }
    & 
    \implies \ToEdge{\txid' | \WR[\aexec] \cup \SO -> \txidrd | \VIS[\aexec] -> \txid'' | \SO -> \txid }
    \\ & \implies \ToEdge{\txid' | \VIS[\aexec] -> \txid'' | \SO -> \txid } .
    \end{align*}
    By \( \aexecinv[\MR]\), we know that \(\txid' \in \aexecinv[\MR] \cup \txidsetrd' \).
\Case{\( \ToEdge{\txidrd | \SO -> \txid} \)}
    For this case, we have
    \begin{align*}
    \ToEdge{\txid' | \WR[\aexec'] \cup \SO -> \txidrd | \SO -> \txid }
    & 
    \implies \ToEdge{\txid' | \WR[\aexec] \cup \SO -> \txidrd | \SO -> \txid }
    \\ & \implies \ToEdge{\txid' | \VIS[\aexec] -> \txid'' | \SO -> \txid } .
    \end{align*}
    By \( \aexecinv[\MR]\), we know that \(\txid' \in \aexecinv[\MR] \cup \txidsetrd' \).
\end{enumerate}
Last, \cref{equ:cc-inv-preserve}can be proven in the same way as in \cref{sec:sound-complete-mr,sec:sound-complete-ryw}.

\COMPLETELET{\CC}
By \cref{thm:cc-visaxioms},
it is sufficient to prove with respect to the following visibility axioms,
\( \visaxioms[\CC]' \FuncDef \Set{ \lambda \aexec \ldotp  \VIS[\aexec] ; \VIS[\aexec] \subseteq \VIS[\aexec] , 
                                    \lambda \aexec \ldotp \SO \subseteq \VIS[\aexec]} \).
By the definition of \( \et[\CC] \), we prove \( \CanCommit[\CC]\) and \( \ViewShift[\MR \cup \RYW]\) respectively.
Since \( (\WR[\aexec] \cup \SO) ; \VIS[\aexec]  \subseteq \VIS[\aexec] ; \VIS[\aexec] \subseteq \VIS[\aexec] \),
then \( \CanCommit[\CC]\) can be derived from \cref{thm:view-vis-relation,equ:aexec-close-to-view}
and \( \ViewShift[\RYW] \) can be proven in the same way as in \cref{sec:sound-complete-ryw}.
By \( \VIS[\aexec] ; \SO \subseteq \VIS[\aexec] ; \VIS[\aexec] \subseteq \VIS[\aexec]  \),
\( \ViewShift[\MR] \) can be proven in the same way as in \cref{sec:sound-complete-mr}.



\subsection{Update Atomic}
\begin{figure}
\hrule
\begin{tabular}{@{} c c@{}}

\begin{halfsubfig}
\begin{centertikz}

\begin{pgfonlayer}{foreground}
%Uncomment line below for help lines
%\draw[help lines] grid(5,4);

%Location x
\node(locx)  {$\ke_\vx \mapsto$};

\matrix(versionx) [version list]
    at ([xshift=\tikzkvspace]locx.east) {
    {a} & $\txid_0$ \\
    {a} & $\emptyset$ \\
};

\tikzvalue{versionx-1-1}{versionx-2-1}{locx-v0}{0};

%Location y
\path (locx.south) + (0,\tikzkeyspace) node (locf1) {$\ke_{\pv{f1}} \mapsto$};
\matrix(versionf1) [version list]
    at ([xshift=\tikzkvspace]locf1.east) {
    {a} & $\txid_0$ \\
    {a} & $\emptyset$ \\
};
\tikzvalue{versionf1-1-1}{versionf1-2-1}{locf1-v0}{0};

%Location y
\path (locf1.south) + (0,\tikzkeyspace) node (locf2) {$\ke_{\pv{f2}} \mapsto$};
\matrix(versionf2) [version list]
    at ([xshift=\tikzkvspace]locf2.east) {
    {a} & $\txid_0$ \\
    {a} & $\emptyset$ \\
};
\tikzvalue{versionf2-1-1}{versionf2-2-1}{locf2-v0}{0};

% \draw[-, red, very thick, rounded corners] ([xshift=-5pt, yshift=5pt]locx-v1.north east) |- 
%  ($([xshift=-5pt,yshift=-5pt]locx-v1.south east)!.5!([xshift=-5pt, yshift=5pt]locy-v0.north east)$) -| ([xshift=-5pt, yshift=5pt]locy-v0.south east);

%blue view - I should  check whether I can use pgfkeys to just declare the list of locations, and then add the view automatically.
\draw[-, blue, very thick, rounded corners=10pt]
 ([xshift=-2pt, yshift=20pt]locx-v0.north east) node (tid1start) {} -- 
 ([xshift=-2pt, yshift=-5pt]locf2-v0.south east);
 
 \path (tid1start) node[anchor=south, rectangle, fill=blue!20, draw=blue, font=\small, inner sep=1pt] {$\thid_3$};

%red view
\draw[-, red, very thick, rounded corners = 10pt]
 ([xshift=-5pt, yshift=5pt]locx-v0.north east) -- 
 ([xshift=-5pt, yshift=-10pt]locf2-v0.south east) node (tid2start) {};
 
\path (tid2start) node[anchor=north, rectangle, fill=red!20, draw=red, font=\small, inner sep=1pt] {$\thid_2$};
 
 %green view
\draw[-, DarkGreen, very thick, rounded corners = 10pt]
 ([xshift=-16pt, yshift=8pt]locx-v0.north east) node (tid3start) {}-- 
 ([xshift=-16pt, yshift=-5pt]locf2-v0.south east);
 
 \path (tid3start) node[anchor=south, rectangle, fill=DarkGreen!20, draw=DarkGreen, font=\small, inner sep=1pt] {$\thid_1$};

\end{pgfonlayer}
\end{centertikz}%
\caption{Initial configuration}
\label{fig:ua-init}
\end{halfsubfig}
%
&
%
\begin{halfsubfig}
\begin{centertikz}
\begin{pgfonlayer}{foreground}
%Uncomment line below for help lines
%\draw[help lines] grid(5,4);

\node(locx)  {$\ke_\vx \mapsto$};

\matrix(versionx) [version list]
    at ([xshift=\tikzkvspace]locx.east) {
    {a} & $\txid_0$ & {a} & \(\txid_1\)\\
    {a} & $\Set{\txid_1}$ & {a} & \(\emptyset\)\\
};

\tikzvalue{versionx-1-1}{versionx-2-1}{locx-v0}{0};
\tikzvalue{versionx-1-3}{versionx-2-3}{locx-v1}{1};


\path (locx.south) + (0,\tikzkeyspace) node (locf1) {$\ke_{\pv{f1}} \mapsto$};
\matrix(versionf1) [version list]
    at ([xshift=\tikzkvspace]locf1.east) {
    {a} & $\txid_0$ & {a} & $\txid_1$\\
    {a} & $\Set{\txid_1}$ & {a} & $\emptyset$\\
};
\tikzvalue{versionf1-1-1}{versionf1-2-1}{locf1-v0}{0};
\tikzvalue{versionf1-1-3}{versionf1-2-3}{locf1-v1}{1};

%Location y
\path (locf1.south) + (0,\tikzkeyspace) node (locf2) {$\ke_{\pv{f2}} \mapsto$};
\matrix(versionf2) [version list]
    at ([xshift=\tikzkvspace]locf2.east) {
    {a} & $\txid_0$ \\
    {a} & $\emptyset$ \\
};
\tikzvalue{versionf2-1-1}{versionf2-2-1}{locf2-v0}{0};


% \draw[-, red, very thick, rounded corners] ([xshift=-5pt, yshift=5pt]locx-v1.north east) |- 
%  ($([xshift=-5pt,yshift=-5pt]locx-v1.south east)!.5!([xshift=-5pt, yshift=5pt]locy-v0.north east)$) -| ([xshift=-5pt, yshift=5pt]locy-v0.south east);

%blue view - I should  check whether I can use pgfkeys to just declare the list of locations, and then add the view automatically.
\draw[-, blue, very thick, rounded corners=10pt]
 ([xshift=-2pt, yshift=20pt]locx-v0.north east) node (tid1start) {} -- 
 ([xshift=-2pt, yshift=-5pt]locf2-v0.south east);
 
 \path (tid1start) node[anchor=south, rectangle, fill=blue!20, draw=blue, font=\small, inner sep=1pt] {$\thid_3$};

%red view
\draw[-, red, very thick, rounded corners = 10pt]
 ([xshift=-5pt, yshift=5pt]locx-v0.north east) -- 
 ([xshift=-5pt, yshift=-10pt]locf2-v0.south east) node (tid2start) {};
 
\path (tid2start) node[anchor=north, rectangle, fill=red!20, draw=red, font=\small, inner sep=1pt] {$\thid_2$};
 
 %green view
\draw[-, DarkGreen, very thick, rounded corners = 10pt]
 ([xshift=-16pt, yshift=8pt]locx-v1.north east) node (tid3start) {}-- 
 ([xshift=-16pt, yshift=-5pt]locf1-v1.south east) --
 ([xshift=-16pt, yshift=5pt]locf2-v0.north east) -- 
 ([xshift=-16pt, yshift=-5pt]locf2-v0.south east);
 
 \path (tid3start) node[anchor=south, rectangle, fill=DarkGreen!20, draw=DarkGreen, font=\small, inner sep=1pt] {$\thid_1$};

\end{pgfonlayer}
\end{centertikz}
\caption{After \(\txid_1\)}
\label{fig:ua-after-tx1}
\end{halfsubfig}

\\
\begin{subfigure}{0.45\textwidth}
\begin{centertikz}%
\begin{pgfonlayer}{foreground}
%Uncomment line below for help lines
%\draw[help lines] grid(5,4);


\node(locx)  {$\ke_\vx \mapsto$};

\matrix(versionx) [version list, column 2/.style = {text width=14mm}]
    at ([xshift=\tikzkvspace]locx.east) {
    {a} & $\txid_0$ & {a} & $\txid_1$ & {a} & $\txid_2$\\
    {a} & $\Set{\txid_1, \txid_2}$ & {a} & $\emptyset$ & {a} & $\emptyset$\\
};

\tikzvalue{versionx-1-1}{versionx-2-1}{locx-v0}{0};
\tikzvalue{versionx-1-3}{versionx-2-3}{locx-v1}{1};
\tikzvalue{versionx-1-5}{versionx-2-5}{locx-v2}{1};


\path (locx.south) + (0,\tikzkeyspace) node (locf1) {$\ke_{\pv{f1}} \mapsto$};
\matrix(versionf1) [version list]
    at ([xshift=\tikzkvspace]locf1.east) {
    {a} & $\txid_0$ & {a} & $\txid_1$\\
    {a} & $\Set{\txid_1}$ & {a} & $\emptyset$\\
};
\tikzvalue{versionf1-1-1}{versionf1-2-1}{locf1-v0}{0};
\tikzvalue{versionf1-1-3}{versionf1-2-3}{locf1-v1}{1};

%Location y
\path (locf1.south) + (0,\tikzkeyspace) node (locf2) {$\ke_{\pv{f2}} \mapsto$};
\matrix(versionf2) [version list]
    at ([xshift=\tikzkvspace]locf2.east) {
    {a} & $\txid_0$ & {a} & \(\txid_2\) \\
    {a} & $\emptyset$ & {a} & \(\emptyset\) \\
};
\tikzvalue{versionf2-1-1}{versionf2-2-1}{locf2-v0}{0};
\tikzvalue{versionf2-1-3}{versionf2-2-3}{locf2-v1}{1};


% \draw[-, red, very thick, rounded corners] ([xshift=-5pt, yshift=5pt]locx-v1.north east) |- 
%  ($([xshift=-5pt,yshift=-5pt]locx-v1.south east)!.5!([xshift=-5pt, yshift=5pt]locy-v0.north east)$) -| ([xshift=-5pt, yshift=5pt]locy-v0.south east);

%blue view - I should  check whether I can use pgfkeys to just declare the list of locations, and then add the view automatically.
\draw[-, blue, very thick, rounded corners=10pt]
([xshift=-2pt, yshift=20pt]locx-v0.north east) node (tid1start) {} -- 
([xshift=-2pt, yshift=-5pt]locf2-v0.south east);
 
\path (tid1start) node[anchor=south, rectangle, fill=blue!20, draw=blue, font=\small, inner sep=1pt] {$\thid_3$};

%red view
\draw[-, red, very thick, rounded corners = 10pt]
([xshift=-5pt, yshift=5pt]locx-v2.north east) -- 
([xshift=-5pt, yshift=-5pt]locx-v2.south east) --
([xshift=-5pt, yshift=5pt]locf1-v0.north east) -- 
([xshift=-5pt, yshift=-5pt]locf1-v0.south east) --
([xshift=-5pt, yshift=5pt]locf2-v1.north east) -- 
([xshift=-5pt, yshift=-10pt]locf2-v1.south east) node (tid2start) {};

\path (tid2start) node[anchor=north, rectangle, fill=red!20, draw=red, font=\small, inner sep=1pt] {$\thid_2$};
 
 %green view
\draw[-, DarkGreen, very thick, rounded corners = 10pt]
([xshift=-16pt, yshift=8pt]locx-v1.north east) node (tid3start) {}-- 
([xshift=-16pt, yshift=-5pt]locx-v1.south east) --
([xshift=-16pt, yshift=5pt]locf1-v1.north east) -- 
([xshift=-16pt, yshift=-5pt]locf1-v1.south east) --
([xshift=-16pt, yshift=5pt]locf2-v0.north east) -- 
([xshift=-16pt, yshift=-5pt]locf2-v0.south east);

\path (tid3start) node[anchor=south, rectangle, fill=DarkGreen!20, draw=DarkGreen, font=\small, inner sep=1pt] {$\thid_1$};

\end{pgfonlayer}%
\end{centertikz}%
\caption{After \(\txid_2\)}
\label{fig:ua-after-tx2}
\end{subfigure}
%
&
%
\begin{subfigure}{0.45\textwidth}
\begin{centertikz}
\begin{pgfonlayer}{foreground}
%Uncomment line below for help lines
%\draw[help lines] grid(5,4);

\node(locx)  {$\ke_\vx \mapsto$};

\matrix(versionx) [version list, column 2/.style = {text width=14mm}]
    at ([xshift=\tikzkvspace]locx.east) {
    {a} & $\txid_0$ & {a} & $\txid_1$ & {a} & $\txid_2$\\
    {a} & $\Set{\txid_1, \txid_2}$ & {a} & $\emptyset$ & {a} & $\emptyset$\\
};

\tikzvalue{versionx-1-1}{versionx-2-1}{locx-v0}{0};
\tikzvalue{versionx-1-3}{versionx-2-3}{locx-v1}{1};
\tikzvalue{versionx-1-5}{versionx-2-5}{locx-v2}{1};


\path (locx.south) + (0,\tikzkeyspace) node (locf1) {$\ke_{\pv{f1}} \mapsto$};
\matrix(versionf1) [version list]
    at ([xshift=\tikzkvspace]locf1.east) {
    {a} & $\txid_0$ & {a} & $\txid_1$\\
    {a} & $\Set{\txid_1}$ & {a} & $\emptyset$\\
};
\tikzvalue{versionf1-1-1}{versionf1-2-1}{locf1-v0}{0};
\tikzvalue{versionf1-1-3}{versionf1-2-3}{locf1-v1}{1};

%Location y
\path (locf1.south) + (0,\tikzkeyspace) node (locf2) {$\ke_{\pv{f2}} \mapsto$};
\matrix(versionf2) [version list]
    at ([xshift=\tikzkvspace]locf2.east) {
    {a} & $\txid_0$ & {a} & \(\txid_2\) \\
    {a} & $\emptyset$ & {a} & \(\emptyset\) \\
};
\tikzvalue{versionf2-1-1}{versionf2-2-1}{locf2-v0}{0};
\tikzvalue{versionf2-1-3}{versionf2-2-3}{locf2-v1}{1};


% \draw[-, red, very thick, rounded corners] ([xshift=-5pt, yshift=5pt]locx-v1.north east) |- 
%  ($([xshift=-5pt,yshift=-5pt]locx-v1.south east)!.5!([xshift=-5pt, yshift=5pt]locy-v0.north east)$) -| ([xshift=-5pt, yshift=5pt]locy-v0.south east);

%blue view - I should  check whether I can use pgfkeys to just declare the list of locations, and then add the view automatically.
\draw[-, blue, very thick, rounded corners=10pt]
([xshift=-2pt, yshift=20pt]locx-v2.north east) node (tid1start) {} -- 
([xshift=-2pt, yshift=-7pt]locx-v2.south east) --
([xshift=-2pt, yshift=3pt]locf1-v1.north east) -- 
([xshift=-2pt, yshift=-5pt]locf1-v1.south east) --
([xshift=-2pt, yshift=5pt]locf2-v1.north east) -- 
([xshift=-2pt, yshift=-5pt]locf2-v1.south east);

\path (tid1start) node[anchor=south, rectangle, fill=blue!20, draw=blue, font=\small, inner sep=1pt] {$\thid_3$};

%red view
\draw[-, red, very thick, rounded corners = 10pt]
([xshift=-5pt, yshift=5pt]locx-v2.north east) -- 
([xshift=-5pt, yshift=-5pt]locx-v2.south east) --
([xshift=-5pt, yshift=5pt]locf1-v0.north east) -- 
([xshift=-5pt, yshift=-5pt]locf1-v0.south east) --
([xshift=-5pt, yshift=5pt]locf2-v1.north east) -- 
([xshift=-5pt, yshift=-10pt]locf2-v1.south east) node (tid2start) {};

\path (tid2start) node[anchor=north, rectangle, fill=red!20, draw=red, font=\small, inner sep=1pt] {$\thid_2$};
 
 %green view
\draw[-, DarkGreen, very thick, rounded corners = 10pt]
([xshift=-16pt, yshift=8pt]locx-v1.north east) node (tid3start) {}-- 
([xshift=-16pt, yshift=-5pt]locx-v1.south east) --
([xshift=-16pt, yshift=5pt]locf1-v1.north east) -- 
([xshift=-16pt, yshift=-5pt]locf1-v1.south east) --
([xshift=-16pt, yshift=5pt]locf2-v0.north east) -- 
([xshift=-16pt, yshift=-5pt]locf2-v0.south east);

\path (tid3start) node[anchor=south, rectangle, fill=DarkGreen!20, draw=DarkGreen, font=\small, inner sep=1pt] {$\thid_1$};

\end{pgfonlayer}
\end{centertikz}%
\caption{\(\txid_3\) updates the view}
\label{fig:ua-before-tx2}
\end{subfigure} \\
\end{tabular}
\hrule
\caption{An invalid executions under update atomic for $\prog_3$}
\label{fig:cu.exec}
\label{fig:cu-exec}
\end{figure}




\begin{figure}
\hrule
\begin{tabular}{@{} c c@{}}

\begin{subfigure}{0.45\textwidth}
\begin{centertikz}

\begin{pgfonlayer}{foreground}
%Uncomment line below for help lines
%\draw[help lines] grid(5,4);

\node(locx)  {$\ke_\vx \mapsto$};

\matrix(versionx) [version list]
    at ([xshift=\tikzkvspace]locx.east) {
    {a} & $\txid_0$ & {a} & \(\txid_1\)\\
    {a} & $\Set{\txid_1}$ & {a} & \(\emptyset\)\\
};

\tikzvalue{versionx-1-1}{versionx-2-1}{locx-v0}{0};
\tikzvalue{versionx-1-3}{versionx-2-3}{locx-v1}{1};


\path (locx.south) + (0,\tikzkeyspace) node (locf1) {$\ke_{\pv{f1}} \mapsto$};
\matrix(versionf1) [version list]
    at ([xshift=\tikzkvspace]locf1.east) {
    {a} & $\txid_0$ & {a} & $\txid_1$\\
    {a} & $\Set{\txid_1}$ & {a} & $\emptyset$\\
};
\tikzvalue{versionf1-1-1}{versionf1-2-1}{locf1-v0}{0};
\tikzvalue{versionf1-1-3}{versionf1-2-3}{locf1-v1}{1};

%Location y
\path (locf1.south) + (0,\tikzkeyspace) node (locf2) {$\ke_{\pv{f2}} \mapsto$};
\matrix(versionf2) [version list]
    at ([xshift=\tikzkvspace]locf2.east) {
    {a} & $\txid_0$ \\
    {a} & $\emptyset$ \\
};
\tikzvalue{versionf2-1-1}{versionf2-2-1}{locf2-v0}{0};


% \draw[-, red, very thick, rounded corners] ([xshift=-5pt, yshift=5pt]locx-v1.north east) |- 
%  ($([xshift=-5pt,yshift=-5pt]locx-v1.south east)!.5!([xshift=-5pt, yshift=5pt]locy-v0.north east)$) -| ([xshift=-5pt, yshift=5pt]locy-v0.south east);

%blue view - I should  check whether I can use pgfkeys to just declare the list of locations, and then add the view automatically.
\draw[-, blue, very thick, rounded corners=10pt]
([xshift=-2pt, yshift=20pt]locx-v0.north east) node (tid1start) {} -- 
([xshift=-2pt, yshift=-5pt]locf2-v0.south east);

\path (tid1start) node[anchor=south, rectangle, fill=blue!20, draw=blue, font=\small, inner sep=1pt] {$\thid_3$};

%red view
\draw[-, red, very thick, rounded corners = 10pt]
([xshift=-5pt, yshift=5pt]locx-v1.north east) -- 
([xshift=-5pt, yshift=-5pt]locf1-v1.south east) --
([xshift=-5pt, yshift=5pt]locf2-v0.north east) -- 
([xshift=-5pt, yshift=-10pt]locf2-v0.south east) node (tid2start) {};

\path (tid2start) node[anchor=north, rectangle, fill=red!20, draw=red, font=\small, inner sep=1pt] {$\thid_2$};
 
 %green view
\draw[-, DarkGreen, very thick, rounded corners = 10pt]
([xshift=-16pt, yshift=8pt]locx-v1.north east) node (tid3start) {}-- 
([xshift=-16pt, yshift=-5pt]locf1-v1.south east) --
([xshift=-16pt, yshift=5pt]locf2-v0.north east) -- 
([xshift=-16pt, yshift=-5pt]locf2-v0.south east);

\path (tid3start) node[anchor=south, rectangle, fill=DarkGreen!20, draw=DarkGreen, font=\small, inner sep=1pt] {$\thid_1$};

\end{pgfonlayer}
\end{centertikz}%
\caption{\(\thid_2\) updates the view}
\label{fig:ua-thid-2-update-view}
\end{subfigure} 
&
\begin{subfigure}{0.45\textwidth}
\begin{centertikz}

\begin{pgfonlayer}{foreground}
%Uncomment line below for help lines
%\draw[help lines] grid(5,4);

\node(locx)  {$\ke_\vx \mapsto$};

\matrix(versionx) [version list]
    at ([xshift=\tikzkvspace]locx.east) {
    {a} & $\txid_0$ & {a} & $\txid_1$ & {a} & $\txid_2$\\
    {a} & $\Set{\txid_1}$ & {a} & $\Set{\txid_2}$ & {a} & $\emptyset$\\
};

\tikzvalue{versionx-1-1}{versionx-2-1}{locx-v0}{0};
\tikzvalue{versionx-1-3}{versionx-2-3}{locx-v1}{1};
\tikzvalue{versionx-1-5}{versionx-2-5}{locx-v2}{2};


\path (locx.south) + (0,\tikzkeyspace) node (locf1) {$\ke_{\pv{f1}} \mapsto$};
\matrix(versionf1) [version list]
    at ([xshift=\tikzkvspace]locf1.east) {
    {a} & $\txid_0$ & {a} & $\txid_1$\\
    {a} & $\Set{\txid_1}$ & {a} & $\emptyset$\\
};
\tikzvalue{versionf1-1-1}{versionf1-2-1}{locf1-v0}{0};
\tikzvalue{versionf1-1-3}{versionf1-2-3}{locf1-v1}{1};

%Location y
\path (locf1.south) + (0,\tikzkeyspace) node (locf2) {$\ke_{\pv{f2}} \mapsto$};
\matrix(versionf2) [version list]
    at ([xshift=\tikzkvspace]locf2.east) {
    {a} & $\txid_0$ & {a} & \(\txid_2\) \\
    {a} & $\emptyset$ & {a} & \(\emptyset\) \\
};
\tikzvalue{versionf2-1-1}{versionf2-2-1}{locf2-v0}{0};
\tikzvalue{versionf2-1-3}{versionf2-2-3}{locf2-v1}{1};

% \draw[-, red, very thick, rounded corners] ([xshift=-5pt, yshift=5pt]locx-v1.north east) |- 
%  ($([xshift=-5pt,yshift=-5pt]locx-v1.south east)!.5!([xshift=-5pt, yshift=5pt]locy-v0.north east)$) -| ([xshift=-5pt, yshift=5pt]locy-v0.south east);

%blue view - I should  check whether I can use pgfkeys to just declare the list of locations, and then add the view automatically.
\draw[-, blue, very thick, rounded corners=10pt]
([xshift=-2pt, yshift=20pt]locx-v0.north east) node (tid1start) {} -- 
([xshift=-2pt, yshift=-5pt]locf2-v0.south east);

\path (tid1start) node[anchor=south, rectangle, fill=blue!20, draw=blue, font=\small, inner sep=1pt] {$\thid_3$};

%red view
\draw[-, red, very thick, rounded corners = 10pt]
([xshift=-5pt, yshift=5pt]locx-v2.north east) -- 
([xshift=-5pt, yshift=-5pt]locx-v2.south east) --
([xshift=-5pt, yshift=5pt]locf1-v1.north east) -- 
([xshift=-5pt, yshift=-10pt]locf2-v1.south east) node (tid2start) {};

\path (tid2start) node[anchor=north, rectangle, fill=red!20, draw=red, font=\small, inner sep=1pt] {$\thid_2$};
 
 %green view
\draw[-, DarkGreen, very thick, rounded corners = 10pt]
([xshift=-16pt, yshift=8pt]locx-v1.north east) node (tid3start) {}-- 
([xshift=-16pt, yshift=-5pt]locf1-v1.south east) --
([xshift=-16pt, yshift=5pt]locf2-v0.north east) -- 
([xshift=-16pt, yshift=-5pt]locf2-v0.south east);

\path (tid3start) node[anchor=south, rectangle, fill=DarkGreen!20, draw=DarkGreen, font=\small, inner sep=1pt] {$\thid_1$};

\end{pgfonlayer}
\end{centertikz}%
\caption{After \(\txid_2\)}
\label{fig:ua-correct-after-tx2}
\end{subfigure} 
\\
\end{tabular}
\hrule
\caption{A execution of $\prog_3$ without lost-update}
\label{fig:ua-conf-2}
\end{figure}


\ac{This Consistency Model shows why the notion of consistent views must 
depend on the set of operations that need to be executed.}

The next consistency model that we consider is \emph{update atomic}. 
Although we did not find any implementation of this model, it has been proposed in \cite{framework-concur} as a strengthening to Read Atomic to avoid write-write conflicts.
This model states that: \textbf{(i)} transactions satisfy atomic visibility (\cref{def:readatomic}); and \textbf{(ii)} transactions writing to one same keys cannot be executed concurrently.
\sx{ This appears too earlier:
Update Atomic is also needed to specify more sophisticated consistency models, 
such as \emph{Parallel Snapshot Isolation} and \emph{Snapshot Isolation}.}
\ac{Check: Nobi said he was interested in implementing Update Atomic 
at some point, maybe he ended up doing something.}

Programs under update atomic do not exhibit the \emph{lost update} anomaly: two or more transactions update the same address, for example , both increment its value by $1$, but only one of them will be observed by future transactions, for example, only one of the increments takes effect.
To illustrate the \emph{lost-update anomaly}, consider the following program \( \prog_3 \) where two transactions concurrently increment $\vx$ and the third transaction read the value. 
Note that the \( \pvar{f1} \) and \( \pvar{f2} \) are two flags indicating the corresponding transactions has been committed.
\ac{Intuitive behaviour of the litmus test: two transactions concurrently increment $[\loc_x]$. 
 A third transaction observes that the first two transactions have been executed. 
 However, it only observes one of the two increments taking place.
 }
\[
    \prog_3 \equiv \begin{session}
        \begin{array}{@{}c || c || c@{}}
        \txid_1 : 
        \begin{transaction} 
            \pmutate{\pvar{f1}}{1};\\
            \pderef{\pvar{a}}{\vx};\\
            \pmutate{\vx}{a + 1};\\
        \end{transaction} & 
        \txid_2 : 
        \begin{transaction}
            \pmutate{\pvar{f2}}{1};\\
            \pderef{\pvar{a}}{\vx};\\
            \pmutate{\vx}{a + 1};\\
        \end{transaction} &
        \txid_3 : 
        \begin{transaction}
            \pderef{\pvar{a}}{\vx};\\
            \pderef{\pvar{b}}{\pvar{f1}};\\
            \pderef{\pvar{c}}{\pvar{f2}};\\
            \pifs{\pvar{a}=1 \wedge \pvar{b}=1 \wedge \pvar{c} = 1}\\ 
                \quad \passign{\retvar}{\sadface}
            \pife
        \end{transaction}
        \end{array}
    \end{session}
 \]

We consider an execution in which the transactions contained in the code of threads $\thid_1, \thid_2$ both execute on the same snapshot determined by the initial view. 
The initial configuration of the program coincides with the one given in \cref{fig:ua-init}.
After executing the transaction $\txid_1$, the resulting configuration is the one depicted \ref{fig:ua-after-tx1} and then \( \txid_2 \) shown in in \ref{fig:ua-after-tx2}, where both transactions read the initial version for key $\ke_\vx$. 
The third transaction $\txid_3$ choose to update its view to include the most recent version for all the keys (\ref{fig:ua-before-tx3}), then when executing its code, all the keys will have value $1$, and the return variable will be set to ${\sadface}$.

The program $\prog_3$ might exhibit the lost-update anomaly when the second transaction $\txid_2$ starts, its view did not include the most up-to-date version for key $\ke_\vx$ provided that \( \txid_{2}\) will update the key \( \ke_\vx \).
As consequence, the database \emph{lost the update} of a version of \( \ke_\vx \) installed by the transaction $\txid_1$, in a sense that no transaction will observe such the version.
To forbid this anomaly, the \emph{update atomic} requires that if a transaction writes to a key, the transaction should start with a view including the most recent version for the key.

\begin{definition}
\label{def:update-atomic}
\emph{Update atomic} is stronger than then read atomic (\cref{def:readatomic}) by further requiring for all keys written, it should starts with a view including the most recent version for those key:
\[
\begin{rclarray}
(\hh, \vi) \csat[\mathsf{UA}] \opset: \vi' & \defeq &
\begin{array}[t]{@{}l}
(\hh, \vi) \csat[\mathsf{RA}] \opset: \vi' \land \fora{\addr} 
(\otW, \addr, \stub) \in \opset \implies \vi(\addr)  = \left| \hh(\addr) \right| - 1
\end{array} \\
\end{rclarray}
\]
\end{definition}

\begin{proposition}
The execution test $\comoUA$ does not hinder progress. 
For any $\hh, \vi, \opset$, there exist $\vi' : \vi \leq \vi'$ and $\vi'': \Vupdate(\hh, \vi', \opset) \leq \vi''$ such that $(\hh, \vi') \csatUA \opset, \vi''$.
\end{proposition}

The thread $\thid_2$ from the program \( \prog_3\) , under $\mathsf{UA}$, cannot execute the transaction $\txid_2$ starting from the configuration depicted in \cref{fig:ua-after-tx1}.
Because the view of $\thid_2$ does not include the most recent version for key $\ke_\vx$. 
Instead, before executing, $\thid_2$ must update its view to include the most recent version of $\ke_\vx$ (\cref{fig:ua-thid-2-update-view}).
Then the \( \txid_2\) will install a new version for \( \ke_\vx \) with value 2 instead of 1 as shown in \cref{fig:ua-correct-after-tx2}.
There are now three different possible views in which $\thid_3$ can execute its transaction.
First, executing on the initial view, in which the transaction will observe 0 for the three locations, and the transaction will not return value $\sadface$.
Second, executing on the one in which the view of $\thid_3$ for $\ke_\vx$ points to the version $(1, \tsid_1, \Set{\tsid_2})$. 
Because of atomic visibility, it must also includes the most recent version for key $\ke_\pv{f2}$ since it is installed by \( \txid_2 \).
In this case, it will not return \(\sadface \).
Last, executing on the one in which the view of $\thid_3$ for $\ke_\vx$ points to its most recent version $(2, \txid_2, \emptyset)$.
In this case, it will not return \(\sadface \).

\subsection{Consistency Prefix \( \CP \) }
\label{sec:sound-complete-cp}

Given abstract execution \( \aexec \), we define read-write read-write relation:
\[
    \RW(\aexec,\ke) \defeq \Setcon{(\txid, \txid')}{\txid \toEdge{\AR_\aexec} \txid' \land (\otR,\ke, \stub ) \in \txid \land (\otW,\ke, \stub ) \in \txid'  } 
\]
It is easy to see \( \RW(\aexec,\ke) \)  can be derived from \( \WW(\aexec,\ke) \) and \( \WR(\aexec, \ke ) \) as the following:
\[
    \RW(\aexec,\ke) = \Setcon{(\txid, \txid')}{ \exsts{\txid'' } (\txid'', \txid) \in \WR(\aexec, \ke) \land (\txid'', \txid') \in \WW(\aexec, \ke) }
\]
Then, the notation \( \RW_\aexec \defeq \bigcup\limits_{\ke \in \Keys} \RW(\aexec, \ke) \).
Note that for a key-value store \( \mkvs \) such that \( \mkvs = \mkvs_\aexec \),
by the definition of  \(  \mkvs = \mkvs_\aexec \), 
the following holds:
\[
    \RW_\aexec = \Setcon{(\txid, \txid')}{\exsts{\ke, i,j } \txid \in \RTx(\mkvs(\ke, i)) \land \txid' = \WTx(\mkvs(\ke, j)) \land i < j}
\]
The \( \RW_\aexec \) also coincides with \( \RW_\Gr \) and \( \RW_\mkvs \).


An abstract execution \( \aexec \) satisfies consistency prefix (\(\CP\)), 
if it satisfies \( \AR_\aexec ; \VIS_\aexec \subseteq \VIS_\aexec \) and \( \SO_\aexec \subseteq \VIS_\aexec \).
Given the specification, there is a corresponding specification on dependency graph by solve the following inequalities:
\[
    \begin{array}{@{}l@{}}
        \WR \subseteq \VIS \\
        \WW \subseteq \AR \\
        \VIS \subseteq \AR \\
        \VIS ; \RW \subseteq \AR \\
        \AR ; \AR \subseteq \AR  \\
        \SO \subseteq \VIS \\
        \AR ; \VIS \subseteq \VIS
    \end{array}
\]
By solving the inequalities the visibility and arbitration relations are:
\[
    \begin{rclarray}
        \AR & \defeq & \left( (\SO \cup \WR ) ; \RW? \cup \WW \cup R \right)^+ \\
        \VIS & \defeq & \left( (\SO \cup \WR ) ; \RW? \cup \WW \cup R \right)^* ; (\SO \cup \WR )
    \end{rclarray}
\]
for some relation \( R \subseteq \AR \).
When \( R = \emptyset \), it is the smallest solution therefore the minimum visibility required.

\sx{A bit verbal}
\begin{lemma}
    \label{lem:cp-eauiv-spec}
    For any abstract execution \( \aexec \),
    if it satisfies 
    \[
        \left( (\SO \cup \WR ) ; \RW? \cup \WW \right)^* ; \VIS_\aexec \subseteq \VIS_\aexec 
        \qquad \SO_\aexec \subseteq \VIS_\aexec
    \]
    then there exists a new \( \aexec' \) such that \( \T_\aexec = \T_{\aexec'} \), 
    under last-write-win \( \TtoOp{T}_{\aexec}(\txid) = \TtoOp{T}_{\aexec'}(\txid) \) for all transactions \( \txid \),
    and the relations satisfy the following:
    \[ 
        \AR_{\aexec'} ; \VIS_{\aexec'} \subseteq \VIS_{\aexec'}  \qquad \SO_{\aexec'} \subseteq \VIS_{\aexec'}
    \]
    and vice versa.
\end{lemma}
\begin{proof}
    Assume abstract execution \( \aexec' \) that
    satisfies \( \AR_{\aexec'} ; \VIS_{\aexec'} \subseteq \VIS_{\aexec'} \)
    and  \( \SO_{\aexec'} \subseteq \VIS_{\aexec'} \).
    We already show that:
\[
    \begin{rclarray}
        \AR_{\aexec'} & = & \left( (\SO_\aexec \cup \WR_\aexec ) ; \RW_\aexec? \cup \WW_\aexec \cup R \right)^+ \\
        \VIS_{\aexec'} & = & \left( (\SO_\aexec \cup \WR_\aexec ) ; \RW_\aexec? \cup \WW_\aexec \cup R \right)^* ; (\SO_\aexec \cup \WR_\aexec )
    \end{rclarray}
\]
for some relation \( R \subseteq \AR_{\aexec'} \).
If we take \( R  = \emptyset \), we have the proof for:
\[
        \SO \subseteq \VIS_\aexec \qquad 
        \left( (\SO_\aexec \cup \WR_\aexec ) ; \RW_\aexec? \cup \WW_\aexec \right)^* ; \VIS_\aexec \subseteq \VIS_\aexec
\]
For another way, we pick the \( R \) that extends
\( \left( (\SO_\aexec \cup \WR_\aexec ) ; \RW_\aexec? \cup \WW_\aexec \cup R \right)^+ \) 
to a total order.
\end{proof}

By \cref{lem:cp-eauiv-spec} to prove soundness and completeness of \( \ET_\CP \), it is sufficient to use the specification:
\[
    (\RP_{\LWW}, \Set{\lambda \aexec. \left( (\SO \cup \WR ) ; \RW? \cup \WW \right)^* ; \VIS_\aexec, \lambda \aexec \ldotp \SO_\aexec }) 
\]

For the soundness, we pick the invariant as the following:
\[  
\begin{rclarray}
    I_1(\aexec, \cl) & = & \left( \bigcup\limits_{\{\txid_{\cl}^{i} \in \T_{\aexec} \mid i \in \Nat\}} \VIS_{\aexec}^{-1}(\txid^i_\cl) \right) \setminus \T_\rd \\
    I_2(\aexec, \cl) & = & \left( \bigcup\limits_{\{\txid_{\cl}^{i} \in \T_{\aexec} \mid i \in \Nat\}} (\SO_{\aexec}^{-1})?(\txid^i_\cl) \right) \setminus \T_\rd
\end{rclarray}
\]
where \( \T_\rd \) is all the read-only transactions included in both 
\( \left( \bigcup\limits_{\{\txid_{\cl}^{i} \in \T_{\aexec} \mid i \in \Nat\}} \VIS_{\aexec}^{-1}(\txid^i_\cl) \right)\) 
and \( \left( \bigcup\limits_{\{\txid_{\cl}^{i} \in \T_{\aexec} \mid i \in \Nat\}} (\SO_{\aexec}^{-1})?(\txid^i_\cl) \right) \).
Assume a key-value store $\hh$, an initial and a final view $\vi, \vi'$  a fingerprint $\opset$ 
such that $\ET_{\CP} \vdash (\hh, \vi) \triangleright \opset: \vi'$. 
Also choose an arbitrary $\cl$, a transaction identifier $\txid_\cl^n \in \nextTxId(\hh, \cl)$, 
and an abstract execution $\aexec$ such that $\hh_{\aexec} = \hh$ and 
\( I_1(\aexec, \cl) \cup I_2(\aexec, \cl) \subseteq \Tx(\hh, \vi) \).
Let a new abstract execution \( \aexec' = \extend(\aexec, \txid_\cl^n, \f, \Tx(\mkvs, \vi) \cup \T_\rd) \).
We are about to prove that there exists an extra set of read-only transaction \( \T'_\rd \) such that:
\begin{gather}
    \fora{\txid} (\txid, \txid_\cl^n) \in \SO_{\aexec'} \implies \txid \in \Tx(\mkvs, \vi) \cup \T_\rd \cup \T'_\rd \label{equ:cp-sound-update-so}\\
    \begin{array}{l}
    \fora{\txid} (\txid, \txid_\cl^n) \in \left( (\SO_{\aexec'} \cup \WR_{\aexec'} ) ; \RW_{\aexec'}? \cup \WW_{\aexec'} \right)^* ; \VIS_{\aexec'} \\
    \qqquad \implies \txid \in \Tx(\mkvs, \vi) \cup \T_\rd \cup \T'_\rd 
    \end{array}
    \label{equ:cp-sound-update-arvis}\\
    I_1(\aexec',\cl) \cup I_2(\aexec',\cl) \subseteq \Tx(\mkvs_{\aexec'}, \vi') \label{equ:cp-sound-inv} 
\end{gather}
\begin{itemize}
\item the invariant \( I_2 \) implies the \cref{equ:cp-sound-update-so} where the proof is the same as \( \RYW \) in \cref{sec:sound-complete-ryw}.

\item For \cref{equ:cp-sound-update-arvis}, it is sufficient to prove one step inclusion, \ie
\[
    \begin{array}{l}
    \fora{\txid} (\txid, \txid_\cl^n) \in \left( (\SO_{\aexec'} \cup \WR_{\aexec'} ) ; \RW_{\aexec'}? \cup \WW_{\aexec'} \right) ; \VIS_{\aexec'} \\
    \qqquad \implies \txid \in \Tx(\mkvs, \vi) \cup \T_\rd \cup \T'_\rd 
\end{array}
\]
To prove above, let \( \T'_\rd \) initially be empty set.
We will add more read-only transactions until it satisfies \cref{equ:cp-sound-update-arvis}.
Assume a transaction \( \txid \) such that 
\( (\txid, \txid_\cl^n) \in \left( (\SO_{\aexec'} \cup \WR_{\aexec'} ) ; \RW_{\aexec'}? \cup \WW_{\aexec'} \right) ; \VIS_{\aexec'}\).
There exists a transaction \( \txid' \) such that \( \txid \toEdge{(\SO_{\aexec'} \cup \WR_{\aexec'} ) ; \RW_{\aexec'}? \cup \WW_{\aexec'}} \txid' \toEdge{\VIS_{\aexec'}}  \txid_\cl^n \).
It follows \( \txid'  \in \Tx(\mkvs, \vi) \cup \T_\rd \cup \T'_\rd  \).
Note that \( \txid \) and \( \txid' \) must exist in the abstract execution \( \aexec \) before update.
There are two cases: \( \txid' \) writes to at least a key; or \( \txid' \) is a read-only transaction.
\begin{itemize}
    \item
    If \( \txid' \) writes to at least a key, then \( \txid' \in \Tx(\mkvs, \vi)\).
    %\[
        %\begin{rclarray}
            %\func{RW^{-1}}{\mkvs, \ke, i} & \defeq & \Setcon{\txid}{\exsts{ j \leq i } \txid \in \RTx(\mkvs(\ke,j))} \\
            %\ddagger & \equiv &
            %\begin{array}[t]{@{}l@{}}
                %\fora{\ke, \ke', i, j, m, \txid, \txid', \txid''} \\
                %\left( \begin{array}{@{}l@{}}
                %i \in \vi(\ke) 
                %\land \txid \in \Set{\WTx(\mkvs(\ke,i))} \cup \func{RW^{-1}}{\mkvs, \ke, i} \land {} \\
                %\quad \left(
                    %\begin{array}{@{}l @{}}
                        %\left( \begin{array}{@{}l@{}}
                                %\txid' \in \func{SO^{-1}}{\txid} \land {} \\
                                %\txid' \in \Set{\WTx(\mkvs(\ke',j))} \cup  \RTx(\mkvs(\ke',j))
                        %\end{array} \right)  \lor {} \\
                        %\left( \begin{array}{@{}l@{}}
                                %\txid \in \RTx(\mkvs(\ke',j)) \land \txid' = \WTx(\mkvs(\ke',j))
                        %\end{array} \right)
                        %\end{array} \right) 
                    %\end{array}
                    %\right)  \\
                    %{} \lor \left( \begin{array}{@{}l@{}}
                            %i \in \vi(\ke) \land \ke = \ke' \land j < i
                    %\end{array} \right) \\
                    %\qquad \implies j \in \vi(\ke') 
            %\end{array} \\
        %\end{rclarray}
    %\]
    %We link the conditions in \( \ddagger \) to relation:
    %\begin{itemize}
        %\item \( \RW_\aexec\). Assume a key \( \ke \),  an index \( i \) and the writer \( \txid  = \WTx(\mkvs(\ke,i))\),
    %then \( \txid' \in \RW^{-1}(\mkvs, \ke, i)\) if and only if \( \txid' \toEdge{\RW_\aexec} \txid\).
        %\item \( \SO_\aexec\). The transaction identifiers encode the \( \SO_\aexec \).
        %That is, \( \txid' \in \SO^{-1}(\txid)\) if and only if \(\txid' \toEdge{\SO_\aexec} \txid \).
        %\item  \( \WR_\aexec \). It is easy to see \( \txid \in \RTx(\mkvs(\ke',j)) \land \txid' = \WTx(\mkvs(\ke',j)) \) if and only if \( \txid' \toEdge{\WR_\aexec} \txid \).
        %\item \( \WW_\aexec \). The write-write relation describes the order of write operations for a key which corresponds the version orders in key-value store.
        %That is, \( \txid' = \WTx(\mkvs(\ke,j)) \land \txid = \WTx(\mkvs(\ke,i)) \land j < i\) if and only if
        %\( \txid' \toEdge{\WW_\aexec} \txid\).
    %\end{itemize}
    %Let assume \( \txid' \) writes to i-\emph{th} version a key \( \ke \).
    %Given above and 
    %\[ \txid \toEdge{(\SO_{\aexec'} \cup \WR_{\aexec'} ) ; \RW_{\aexec'}? \cup \WW_{\aexec'}} \txid' \toEdge{\VIS_{\aexec'}}  \txid_\cl^n \] we can substitute and rewrite the \( \ddagger \) as the following:
    %\begin{gather}
        %\begin{array}{@{}l@{}}
            %\fora{\txid'',\ke',j}
            %\WTx(\mkvs(\ke,i)) = \txid' \land {} \\
            %\left( \begin{array}{@{}l@{}}
            %\txid'' \toEdge{\RW_{\aexec'}?} \txid' \land {} \\
            %\quad \left(
                %\begin{array}{@{}l @{}}
                    %\left( \begin{array}{@{}l@{}}
                            %\txid \toEdge{\SO_{\aexec'}} \txid'' \land 
                            %\txid \in \Set{\WTx(\mkvs(\ke',j))} \cup  \RTx(\mkvs(\ke',j))
                    %\end{array} \right)  \\
                    %{} \lor 
                    %\left( \begin{array}{@{}l@{}}
                            %\txid \toEdge{\WR_{\aexec'}} \txid'' \land \txid = \WTx(\mkvs(\ke',j))
                    %\end{array} \right)
                    %\end{array} \right) 
                %\end{array}
                %\right)  \\
                %{} \lor \left( \begin{array}{@{}l@{}}
                        %\txid \toEdge{\WW_{\aexec'}} \txid'' \land \txid = \WTx(\mkvs(\ke',j))
                %\end{array} \right) \\
                %\qquad \implies j \in \vi(\ke') 
        %\end{array} 
        %\label{equ:cp-dagger}
    %\end{gather}
    Now we perform case analysis if \( \txid \) is a read-only transaction.
    \begin{itemize}
        \item if \( \txid \) has write, we prove \( \txid \in \Tx(\mkvs, \vi)\).
        Recall the \( \ddagger \) is defined as the following:
        \begin{equation}
        \label{equ:cp-dagger}
        \ddagger  \equiv 
            \fora{\ke, \ke', i, j}
                i \in \vi(\ke)  \wedge \WTx(\hh(\ke', j)) \toEdge{(((\PO \cup \RF_{\hh}) ; \AD_{\hh}?) \cup \VO_{\hh})^{+}} \WTx(\hh(\ke, i))
            \implies j \in \vi(\ke')  
        \end{equation}
        Since \( \WR_\mkvs \), \( \WW_\mkvs \) and \( \RW_\mkvs \) coincide with
        \( \WR_\aexec \), \( \WW_\aexec \) and \( \RW_\aexec \) respectively.
        Also because \( \txid \) write to at least one key,
        it is easy to see there exists some version \( \ke'',m\) such that 
        \( \txid = \WTx(\mkvs(\ke'',m))\) and \( m \in \vi(\ke'')\).
        By definition of \( \Tx \), it follows \( \txid \in \Tx(\mkvs, \vi) \).
        %Therefore by the definition of \( \Tx \), then \( \txid \in \VIS^{-1}(\txid_\cl^n)\).
        \item if \( \txid \) is a read-only transaction, we add it into \( \T'_\rd \).
    \end{itemize}
    \item 
    if \( \txid' \) is a read-only transaction, then either \( \txid' \in \T_\rd \) or \( \txid' \in \T'_\rd \).
    More specifically we have three cases: \textbf{(i)} \( \txid' \in \bigcup\limits_{\{\txid_{\cl}^{i} \in \T_{\aexec} \mid i \in \Nat\}} \VIS_{\aexec}^{-1}(\txid^i_\cl) \), \textbf{(ii)} \( \txid' \in \bigcup\limits_{\{\txid_{\cl}^{i} \in \T_{\aexec} \mid i \in \Nat\}} (\SO_{\aexec}^{-1})?(\txid^i_\cl) \) or \textbf{(iii)} \( \txid' \in \T'_\rd\).
    \begin{itemize}
        \item
        Assume \( \txid' \in \bigcup\limits_{\{\txid_{\cl}^{i} \in \T_{\aexec} \mid i \in \Nat\}} \VIS_{\aexec}^{-1}(\txid^i_\cl) \).
        It means \( \txid' \) is visible for some previous transaction \( \txid_\cl^m \) (\( m < n \)) from the same client \( cl \), 
        \ie 
        \[ 
            \txid \toEdge{(\SO_{\aexec'} \cup \WR_{\aexec'} ) ; \RW_{\aexec'}? \cup \WW_{\aexec'}} \txid' \toEdge{\VIS_{\aexec'}}  \txid_\cl^m 
        \]
        Note that all the edges before \( \txid_\cl^m \) must exist in \( \aexec \).
        Since \( \aexec \) satisfies the \( \left( (\SO \cup \WR ) ; \RW? \cup \WW \right)^* ; \VIS_\aexec \subseteq \VIS_\aexec \),
        we have \( \txid \toEdge{\VIS_{\aexec'}} \txid_\cl^m \) and then \( \txid \in \bigcup\limits_{\{\txid_{\cl}^{i} \in \T_{\aexec} \mid i \in \Nat\}} \VIS_{\aexec}^{-1}(\txid^i_\cl)\).
        By the invariant \( I_1 \), it means \( \txid \in \Tx(\mkvs, \vi) \cup \T_\rd \).
    \item \( \txid' \in \bigcup\limits_{\{\txid_{\cl}^{i} \in \T_{\aexec} \mid i \in \Nat\}} \SO_{\aexec}^{-1}(\txid^i_\cl) \).
    Since \( \txid' \) is a read-only transaction, 
    the edges can be simplified to \( \txid \toEdge{(\SO_{\aexec'} \cup \WR_{\aexec'} )} \txid' \toEdge{\SO_{\aexec'}}  \txid_\cl^n \).
    Given that \( \SO \) is transitive, then  either \( \txid \toEdge{\SO_{\aexec'}} \txid_\cl^n \) or \( \txid \toEdge{\WR_{\aexec'} } \txid' \toEdge{\SO_{\aexec'}}  \txid_\cl^n \).
    \begin{itemize}
        \item \( \txid \toEdge{\SO_{\aexec'}} \txid_\cl^n \).
            It follows \( \txid \in \bigcup\limits_{\{\txid_{\cl}^{i} \in \T_{\aexec} \mid i \in \Nat\}} \SO_{\aexec}^{-1}(\txid^i_\cl) = \Tx(\mkvs, \vi) \cup \T_\rd \).
        \item \( \txid \toEdge{\WR_{\aexec'} } \txid' \toEdge{\SO_{\aexec'}}  \txid_\cl^n \).
            The \( \WR \) edge must exists in \( \aexec \).
            Because \( \WR_\aexec \subseteq \VIS_\aexec \) then  \( \txid \toEdge{\VIS_{\aexec} } \txid' \toEdge{\SO_{\aexec'}}  \txid_\cl^n  \).
            It means 
            \[ 
                \txid \in \bigcup\limits_{\{\txid_{\cl}^{i} \in \T_{\aexec} \mid i \in \Nat\}} \VIS_{\aexec}^{-1}(\txid^i_\cl) = \Tx(\mkvs, \vi) \cup \T_\rd 
            \]
    \end{itemize}
    \item 
    Last, \( \txid' \in \T'_\rd \).
    Since \( \T'_\rd \) initially is empty set, there exists another write transaction \( \txid'' \) such that:
    \[
        \txid \toEdge{(\SO_{\aexec'} \cup \WR_{\aexec'} ) ; \RW_{\aexec'}? \cup \WW_{\aexec'}} \txid' \toEdge{(\SO_{\aexec'} \cup \WR_{\aexec'} ) ; \RW_{\aexec'}? \cup \WW_{\aexec'}} \txid'' \toEdge{\VIS_{\aexec'}}  \txid_\cl^n
    \]
    %Given that \( \txid' \) is a read-only and \( \txid'' \) has write, the edges can be simplified:
    %\[
        %\txid \toEdge{(\SO_{\aexec'} \cup \WR_{\aexec'} )} \txid' \toEdge{\SO_{\aexec'} ; \RW_{\aexec'}?} \txid'' \toEdge{\VIS_{\aexec'}}  \txid_\cl^n
    %\]
    %Because transitivity of  \( \SO \), we have the following two cases:
    %\[
        %\begin{array}{@{}l@{}}
            %\txid \toEdge{ \WR_{\aexec'} } \txid' \toEdge{\SO_{\aexec'} ; \RW_{\aexec'}?} \txid'' \toEdge{\VIS_{\aexec'}}  \txid_\cl^n \\
            %\txid \toEdge{\SO_{\aexec'} ; \RW_{\aexec'}?} \txid'' \toEdge{\VIS_{\aexec'}}  \txid_\cl^n 
        %\end{array}
    %\]
    %\( \txid \toEdge{ \WR_{\aexec'} } \txid' \toEdge{\SO_{\aexec'} ; \RW_{\aexec'}?} \txid'' \toEdge{\VIS_{\aexec'}}  \txid_\cl^n \).
        If \( \txid \) has write, by \cref{equ:cp-dagger} then \( \txid \in \Tx(\mkvs,\vi) \).
        Otherwise if \( \txid \) is a read only transaction, we add it into \( \T'_\rd \).
            %\( \txid \toEdge{\SO_{\aexec'} ; \RW_{\aexec'}?} \txid'' \toEdge{\VIS_{\aexec'}}  \txid_\cl^n \).
            %Similarly by \cref{equ:cp-dagger}, either \( \txid \in \Tx(\mkvs,\vi) \)  or we add it into \( \T'_\rd \).
    \end{itemize}
\end{itemize}

\item Since \( \CP \) satisfies \( \RYW \) and \( \MRd \), thus invariants \( I_1 \) and  \( I_2 \) are preserved after update.

\end{itemize}

    
For completeness, we prove the three parts of the execution test separately.
\begin{itemize}
\item Since \( \SO_\aexec \subseteq \VIS_\aexec  \), the prove for \( \ET_\RYW \) is the as in \cref{sec:sound-complete-mr}.
\item For any \( \VIS_\aexec \)  satisfies the constraint for \( \CP \), by \cref{lem:cp-eauiv-spec} it satisfies that 
\[
    \VIS \defeq \left( (\SO \cup \WR ) ; \RW? \cup \WW \cup R \right)^* ; (\SO \cup \WR )
\]
for some relation \( R \).
It means \( \VIS_\aexec ; \SO_\aexec \subseteq \VIS_\aexec \).
Therefore it is complete with respect to \( \ET_\MRd \).

\item Let consider the \( \ddagger \).
Assume i-\emph{th} transaction \( \txid_i \) in the arbitrary order,
and let view \( \vi_{i} = \getView(\aexec, \VIS^{-1}_{\aexec}(\txid_{i}) ) \).
We also pick any final view such that \( \vi'_{i} \subseteq \getView(\aexec, (\AR^{-1}_{\aexec})?(\txid_{i}) ) \).
Note that there is nothing to prove for \( \vi'_i \) since the \( \ddagger \) does not constrain the \( \vi'_i \).
Recall the \( \ddagger \):
\[
\ddagger  \equiv 
        \fora{\ke, \ke', m, j}
             m \in \vi(\ke)  \wedge \WTx(\hh(\ke', j)) \toEdge{(((\PO \cup \RF_{\hh}) ; \AD_{\hh}?) \cup \VO_{\hh})^{+}} \WTx(\hh(\ke, m))
         \implies j \in \vi(\ke')  
\]
Assume \( j \in \vi_i(\ke) \) for some key \(\ke \) and index \( i \).
It means the writer of the version is visible by the transaction \( \txid_i\),
\ie \( \WTx(\mkvs(\ke,i)) \in \VIS^{-1}_{\aexec}(\txid_{i}) \).
Let the \( \mkvs = \mkvs_{\cut(\aexec, i-1)} \).
We need to prove the following:
\begin{gather}
    \label{equ:cp-complete-arvis}
    \begin{array}{@{}l@{}}
        \fora{\ke, \ke', m, j, \txid, \txid'} 
        m \in \vi(\ke) 
        \land \WTx(\mkvs(\ke,m)) \in \VIS_\aexec^{-1}(\txid_i) \\
        \quad {} \land \WTx(\hh(\ke', j)) \toEdge{(((\PO \cup \RF_{\hh}) ; \AD_{\hh}?) \cup \VO_{\hh})^{+}} \WTx(\hh(\ke, m)) \\
            \qquad \implies \WTx(\mkvs(\ke',j)) \in \VIS_\aexec^{-1}(\txid_i)
    \end{array}
\end{gather}
%Note that \( \txid \in \Set{\WTx(\mkvs(\ke,i))} \cup \func{RW^{-1}}{\mkvs, \ke, i} \) 
%means \( \txid \toEdge{\RW_{\aexec}?} \WTx(\mkvs(\ke,i)) \),
%the formulae \(\left( \begin{array}{@{}l@{}} \txid \in \RTx(\mkvs(\ke',j)) \land \txid' = \WTx(\mkvs(\ke',j)) \end{array} \right) \) 
%means \( \txid \toEdge{\WR_\aexec} \txid' \),
%and \( \left( \begin{array}{@{}l@{}} \txid = \WTx(\mkvs(\ke',m)) \land \txid' = \WTx(\mkvs(\ke',j)) \land m > j \end{array} \right) \) 
%means \( \txid \toEdge{\WW_\aexec} \txid' \).
%Given all the correspondence, the \cref{equ:cp-complete-arvis} holds if the following holds:
%\[
    %\begin{rclarray}
        %\begin{array}[t]{@{}l@{}}
            %\fora{\ke, \ke', i, j, \txid, \txid'} \\
            %\left( \begin{array}{@{}l@{}}
            %i \in \vi(\ke) 
            %\land \WTx(\mkvs(\ke,i)) \in \VIS_\aexec^{-1}(\txid_i) \\
            %{} \land \txid' = \WTx(\mkvs(\ke',j))
            %\land \txid \toEdge{\RW_{\aexec}?} \WTx(\mkvs(\ke,i)) \land {} \\
            %\left(
                %\begin{array}{@{}l @{}}
                    %\txid' \toEdge{\WR_\aexec ; \SO_\aexec}\txid \lor
                    %\txid' \toEdge{\SO_\aexec}\txid \lor
                    %\txid' \toEdge{\WR_\aexec}\txid
                    %\end{array} \right) 
                %\end{array}
                %\right)  \\
                %{} \lor \txid' \toEdge{\WW_\aexec} \WTx(\mkvs(\ke,i)) \\
                %\qquad \implies \txid' \in \VIS_\aexec^{-1}(\txid_i)
        %\end{array} \\
    %\end{rclarray}
%\]
%Then the above holds, if the following holds:
%\begin{gather}
    %\label{equ:cp-complete-arvis-2}
    %\begin{rclarray}
        %\begin{array}[t]{@{}l@{}}
            %\fora{\ke, i, \txid} \\
            %\left( \begin{array}{@{}l@{}}
            %i \in \vi(\ke) 
            %\land \WTx(\mkvs(\ke,i)) \in \VIS_\aexec^{-1}(\txid_i) \\
            %{} \land \txid \toEdge{( (\WR_\aexec; \SO_\aexec) \cup \SO_\aexec \cup \WR_\aexec) ; \RW_{\aexec}? \cup \WW_\aexec} \WTx(\mkvs(\ke,i)) 
                %\end{array}
                %\right)  \\
                %\qquad \implies \txid \in \VIS_\aexec^{-1}(\txid_i)
        %\end{array} \\
    %\end{rclarray}
%\end{gather}
Since \( \WR_\mkvs \), \( \WW_\mkvs \) and \( \RW_\mkvs \) coincide with
\( \WR_\aexec \), \( \WW_\aexec \) and \( \RW_\aexec \) respectively,
and \( \left( (\SO \cup \WR ) ; \RW? \cup \WW \right)^* ; \VIS_\aexec \subseteq \VIS_\aexec \),
It implies \cref{equ:cp-complete-arvis}.
\end{itemize}

\subsection{Parallel Snapshot Isolation \(\PSI\)}
\label{sec:sound-complete-psi}

The axiomatic definition for \( \PSI \) is 
\[ 
    (\RP_{\LWW}, \Set{\lambda \aexec. \VIS_{\aexec} ; \VIS_{\aexec}, \lambda \aexec \ldotp \SO_\aexec, \lambda \aexec. \WW_\aexec })
\]
Given the definition, there is a corresponding definition on dependency graph by solve the following inequalities:
\[
    \begin{array}{@{}l@{}}
        \WR \subseteq \VIS \\
        \WW \subseteq \VIS \\
        \SO \subseteq \VIS \\
        \VIS ; \VIS \subseteq \VIS 
    \end{array}
\]
We have \( \VIS = (\WR \cup \WW \cup \SO \cup R)^{+} \) for some \( R \subseteq \AR \).
Thus, there exist a minimum visibility such that 
\[ 
    (\RP_{\LWW}, \Set{\lambda \aexec. (\WR_{\aexec} \cup \WW_{\aexec} \cup \SO ) ; \VIS_{\aexec}, \lambda \aexec \ldotp \SO_\aexec, \lambda \aexec. \WW_\aexec })
\]

To prove soundness, we pick an invariant for the \( \ET_\PSI \) as the union of those for \( \MR\) and \( \RYW \) shown in the following:
\begin{align*}
    I_1(\aexec, \cl) & =  \left( \bigcup_{\Set{\txid_{\cl}^{n} \in \txidset_{\aexec} }[ n \in \Nat ]} \VIS_{\aexec}^{-1}(\txid^n_\cl) \right) \setminus \txidset_\rd \\
    I_2(\aexec, \cl) & =  \left( \bigcup_{\Set{\txid_{\cl}^{n} \in \txidset_{\aexec} }[ n \in \Nat ]} (\SO_{\aexec}^{-1})\rflx(\txid^n_\cl) \right) \setminus \txidset_\rd
\end{align*}
where \( \txidset_\rd \) is all the read-only transactions included in both 
\( \left( \bigcup_{\Set{\txid_{\cl}^{n} \in \txidset_{\aexec} }[ n \in \Nat ]} \VIS_{\aexec}^{-1}(\txid^n_\cl) \right)\) 
and \( \left( \bigcup_{\Set{\txid_{\cl}^{n} \in \txidset_{\aexec} }[ n \in \Nat ]} (\SO_{\aexec}^{-1})\rflx(\txid^n_\cl) \right) \).
Assume a kv-store $\mkvs$, an initial and a final view $\vi, \vi'$  a fingerprint $\fp$ 
such that $\ET_{\PSI} \vdash (\mkvs, \vi) \csat \fp: (\mkvs',\vi')$. 
Also choose an arbitrary $\cl$, a transaction identifier $\txid_\cl^n \in \nextTxid(\mkvs, \cl)$, 
and an abstract execution $\aexec$ such that $\mkvs_{\aexec} = \mkvs$ and 
\( I_1(\aexec, \cl) \cup I_2(\aexec, \cl) \subseteq \Tx[\mkvs, \vi] \).
We are about to prove there exists an extra set of read-only transactions \( \txidset'_\rd \) such that
the new abstract execution \( \aexec' = \extend[\aexec, \txid_\cl^n, \fp, \Tx[\mkvs, \vi] \cup \txidset_\rd \cup \txidset'_\rd] \) and:
\begin{gather}
    \fora{\txid} (\txid, \txid_\cl^n) \in \SO_{\aexec'} \implies \txid \in \Tx[\mkvs, \vi] \cup \txidset_\rd \cup \txidset'_\rd \label{equ:psi-sound-update-so}\\
    \fora{\txid} (\txid, \txid_\cl^n) \in \WW_{\aexec'} \implies \txid \in \Tx[\mkvs, \vi] \cup \txidset_\rd \cup \txidset'_\rd \label{equ:psi-sound-update-ua}\\
    \fora{\txid} (\txid, \txid_\cl^n) \in ( \SO_{\aexec'} \cup \WR_{\aexec'} \cup \WW_{\aexec'} )^{+} ; \VIS_{\aexec'} \implies \txid \in \Tx[\mkvs, \vi] \cup \txidset_\rd \cup \txidset'_\rd \label{equ:psi-sound-update-closure}\\
    I_1(\aexec',\cl) \cup I_2(\aexec',\cl) \subseteq \Tx[\mkvs_{\aexec'}, \vi'] \label{equ:psi-sound-inv} 
\end{gather}
\begin{itemize}
\item The invariant \( I_2 \) implies \cref{equ:psi-sound-update-so} as the same as \( \RYW \) in \cref{sec:sound-complete-ryw}.
\item Since \( \PSI \) also satisfies \( \UA \), the \cref{equ:si-sound-update-ww} can be proven as the same as \( \UA \) in \cref{sec:sound-complete-ua}.
\item \cref{equ:psi-sound-update-closure}.
    Note that \( (\txid, \txid_\cl^n) \in ( \SO_{\aexec'} \cup \WR_{\aexec'} \cup \WW_{\aexec'}); \VIS_{\aexec'} \implies (\txid, \txid_\cl^n) \in ( \SO_{\aexec} \cup \WR_{\aexec}  \cup \WW_{\aexec} ) ; \VIS_{\aexec'}\).
    Also, recall that \( \SO_\aexec = \SO_\mkvs \), \( \WR_\aexec = \WR_\mkvs \) and  \( \WW_\aexec = \WW_\mkvs \).
    Let \( \txidset'_\rd = \lfpTx[\mkvs,\vi,\SO_{\mkvs} \cup \WR_{\mkvs} \cup \WW_{\mkvs}] \). 
    This means that \( \aexec' = \extend[\aexec, \txid_\cl^n, \fp, \lfpTx[\mkvs, \vi, \SO_{\mkvs} \cup \WR_{\mkvs}] \cup \txidset_\rd ] \).
    Let assume \( \txid \toEDGE{\SO_{\mkvs} \cup \WR_{\mkvs} \cup \WW_{\mkvs}} \txid' \) and \( \txid' \in \lfpTx[\mkvs, \vi, \SO_{\mkvs} \cup \WR_{\mkvs}] \cup \txidset_\rd \).
    We have two possible cases:
    \begin{itemize}
        \item If \( \txid' \in \lfpTx[\mkvs, \vi, \SO_{\mkvs} \cup \WR_{\mkvs} \cup \WW_{\mkvs}] \), by  \cref{thm:view-vis-relation}, we know \( \txid \in \lfpTx[\mkvs, \vi, \SO_{\mkvs} \cup \WR_{\mkvs} \cup \WW_{\mkvs}] \).
        \item If \( \txid' \in \txidset_\rd \), there are two cases:
        \begin{itemize}
            \item \( \txid' \in  \left( \bigcup_{\Set{\txid_{\cl}^{n} \in \txidset_{\aexec} }[ n \in \Nat ]} \VIS_{\aexec}^{-1}(\txid^n_\cl) \right) \).
                Since \( \txid' \) is a read-only transaction, it means \( \txid \toEDGE{\SO_{\mkvs} \cup \WR_{\mkvs} } \txid' \).
                By the property of \( \aexec \) (before update) that \( \SO \cup \WR_\aexec \in \VIS_\aexec \), it is known that \( \txid \in \left( \bigcup_{\Set{\txid_{\cl}^{n} \in \txidset_{\aexec} }[ n \in \Nat ]} \VIS_{\aexec}^{-1}(\txid^n_\cl) \right) \), that is, \( \txid \in \Tx[\mkvs,\vi] \cup \txidset_\rd\).

            \item \( \txid' \in  \left( \bigcup_{\Set{\txid_{\cl}^{n} \in \txidset_{\aexec} }[ n \in \Nat ]} \SO_{\aexec}^{-1}(\txid^n_\cl) \right) \).
                Given that \( \txid' \) is a read only transaction, we know \( \txid \in (\SO \cup \WR_\aexec)^{-1} \left( \bigcup_{\Set{\txid_{\cl}^{n} \in \txidset_{\aexec} }[ n \in \Nat ]} \SO_{\aexec}^{-1}(\txid^n_\cl) \right) \).
                By the property of \( \aexec \) (before update) that \( \SO \cup \WR_\aexec \in \VIS_\aexec \),
                it follows:
                \begin{align*}
                    \txid & \in VIS_\aexec^{-1} \left( \bigcup_{\Set{\txid_{\cl}^{n} \in \txidset_{\aexec} }[ n \in \Nat ]} \SO_{\aexec}^{-1}(\txid^n_\cl) \right) \\
                          & = \left( \bigcup_{\Set{\txid_{\cl}^{n} \in \txidset_{\aexec} }[ n \in \Nat ]} \VIS_{\aexec}^{-1}(\txid^n_\cl) \right)  \\
                          & = \Tx[\mkvs,\vi] \cup \txidset_\rd
                \end{align*}
                
        \end{itemize}
    \end{itemize}
\item Finally the new abstract execution preserves the invariant \( I_1 \) and \( I_2 \) 
because  \( \CC \) satisfies \( \MW \) and \( \RYW \).
\end{itemize}

Given that \( \VIS_\aexec = (\WR_\aexec \cup \WW_\aexec \cup \SO_\aexec \cup R)^{+} \),
we know \( \VIS_\aexec ; \SO_\aexec \subseteq \VIS_\aexec \).
First the completeness follows \( \MR \) in \cref{sec:sound-complete-mr}, \( \RYW \) in \cref{sec:sound-complete-ryw} and  \( \UA \) in \cref{sec:sound-complete-ua}.
Similarly, by \cref{lem:aexec-spec-cc},

An abstract execution \( \aexec \) satisfies snapshot isolation (\(\SI\)), 
if it satisfies
\( \{\lambda \aexec. \AR[\aexec] ; \VIS[\aexec], \lambda \aexec \ldotp \SO, \allowbreak \lambda \aexec. \WW[\aexec] \}) \),
which is intersection of \( \CP \) and \( \UA \) on abstract executions \citep{SIanalysis}.
\citet{SIanalysis} also proposed the minimum visibility relation that gives rise of the following equivalent definition
\[
    \visaxioms[\SI] \FuncDef
    \Set{\lambda \aexec. \left( (\WR[\aexec]  \cup \SO \cup \WW[\aexec] ) ; \Refl(\RW[\aexec]) \right) ; \VIS[\aexec]
            ,\lambda \aexec \ldotp \SO, \lambda \aexec \ldotp \WW[\aexec] }  .
\]

The execution test \( \et[\CP] \) is sound with respect to the axiomatic definition \( \visaxioms[\SI] \)
We pick the invariant \( \aexecinv[\CP] = \aexecinv[\RYW]\).
\SOUNDLET{\CP}{ \txidsetrd \supseteq
\begin{multlined}[t]
\left( \bigcup_{\Set{\txid[\cl](\idx) | \txid[\cl](\idx) \in \aexec}} 
\VISInv[\aexec](\txid[\cl](\idx)) \cup \Refl((\Inv(\SO)))(\txid[\cl](\idx)) \right) 
\setminus \Set{\txid' | \Forall{l | \key | \val } (l,\key,\val) \in \aexec(\txid') \implies l = \opR } .
\end{multlined} }
Assume 
\[ 
\txidsetrd' = 
\begin{multlined}[t]
\left( \bigcup_{\Set{\txid[\cl](\idx) | \txid[\cl](\idx) \in \aexec}} 
\VISInv[\aexec](\txid[\cl](\idx)) \cup \Refl((\Inv(\SO)))(\txid[\cl](\idx)) \right) 
\setminus \Set{\txid' | \Forall{l | \key | \val } (l,\key,\val) \in \aexec(\txid') \implies l = \opR } .
\end{multlined} 
\]
and \( \txidsetrd'' = \txidsetrd \setminus \txidsetrd' \).
By the definition of soundness, we prove the following result:
\begin{Formulae}
& \begin{Formula}
    \Inv(\SO)(\txid) \subseteq \txidset \cup \txidsetrd' 
    \label{equ:si-sound-update-so}
\end{Formula}
\\ & \begin{Formula}
    \Inv(\WW)(\txid) \subseteq \txidset 
    \label{equ:si-sound-update-ua}
\end{Formula}
\\ & \begin{Formula}
    \Inv(\left( (\WR[\aexec] \cup \SO \cup \WW[\aexec] ) ; \Refl(\RW[\aexec]) \right)) (\txid) 
            \subseteq \txidset \cup \txidsetrd' \cup \txidsetrd''
     \label{equ:si-sound-update-closure}
\end{Formula}
\\ & \begin{Formula}
    \aexecinv[\PSI](\aexec',\cl) \subseteq \VisTrans(\XToK(\aexec'),\vi')
    \label{equ:si-inv-preserve}
\end{Formula}
\end{Formulae}
\Cref{equ:si-sound-update-so,equ:si-sound-update-ua} 
can be proven in the same way as in \cref{sec:sound-complete-mr,sec:sound-complete-ua} respectively.
We now prove \cref{equ:si-sound-update-closure}.
Initially we take \( \txidsetrd'' \) to be an empty set.
Note that \(\VISInv[\aexec'](\txid) = \txidset \cup \txidsetrd' \cup \txidsetrd'' \).
By \cref{thm:view-vis-relation,equ:view-close-to-aexec}, there exists \( \txidsetrd'' \) such that
\( \txidset \cup \txidsetrd'' \) is closed under \( \left( (\WR[\aexec] \cup \SO \cup \WW[\aexec] ) ; \Refl(\RW[\aexec]) \right)\).
Now consider a transaction \( \txidrd \in \txidsetrd' \) and
assume a transaction \( \txid' \) such that \( \ToEdge{ \txid' | (\WR[\aexec] \cup \SO \cup \WW[\aexec] ) ; \Refl(\RW[\aexec]) -> \txidrd } \).
Since \( \txidrd \) is a read-only transaction, thus
\( \ToEdge{ \txid' |  (\SO \cup \WR[\aexec] )  -> \txidrd } \)
and the rest proof is exactly the same as as in \cref{sec:sound-complete-cc}.
Last, \cref{equ:si-inv-preserve} can be proven in the same way as in \cref{sec:sound-complete-mr,sec:sound-complete-ryw}.

The execution test $\et[\SI]$ is complete with respect to the axiomatic definition \( \visaxioms[\SI] \).
By \citet{SIanalysis}, it suffices to prove completeness with respect to the following definition,
\[
\Set{\lambda \aexec \ldotp \AR[\aexec] ; \VIS[\aexec], \lambda \aexec \ldotp \SO , \lambda \aexec \ldotp \WW[\aexec] } .
\]
\COMPLETELET{\SI}
By the definition of \( \et[\SI]\), we prove \( \CanCommit[\SI] \), \( \ViewShift[\MR]\) and \( \ViewShift[\RYW]\) respectively.
Recall that \( \CanCommit[\SI] = \PreClosed(\kvs,\vi,\rel[\UA] \cup \left( (\WR[\aexec] \cup \SO \cup \WW[\aexec] ) ; \Refl(\RW[\aexec]) \right)) \).
It is easy to see that 
\[
\begin{multlined}
\PreClosed(\kvs,\vi,\rel[\UA] \cup \WR[\kvs] \cup \SO \cup \WW[\kvs]) \iff {}
    \\ \PreClosed(\kvs,\vi,\rel[\UA]) 
    \land \PreClosed(\kvs,\vi,\left( (\WR[\aexec] \cup \SO \cup \WW[\aexec] ) ; \Refl(\RW[\aexec]) \right)) . 
\end{multlined}
\]
The predicate \( \PreClosed(\kvs,\vi,\rel[\UA]) \) can be proven in the same way as in \cref{sec:sound-complete-ua}
Because
\begin{align*}
\left( (\WR[\aexec] \cup \SO \cup \WW[\aexec]) ; \Refl(\RW[\aexec])  \right) ; \VIS[\aexec] 
        & \subseteq \left( \VIS[\aexec] ; \Refl(\RW[\aexec]) \right) ;  \VIS[\aexec]
        & \subseteq  \AR[\aexec] ; \VIS[\aexec] \subseteq \VIS[\aexec] .
\end{align*}
Then \( \PreClosed(\kvs,\vi,\left( (\WR[\aexec] \cup \SO \cup \WW[\aexec] ) ; \Refl(\RW[\aexec]) \right)) \)
can be derived from \cref{thm:view-vis-relation,equ:aexec-close-to-view}.
The predicate \( \ViewShift[\RYW] \) can be proven in the same way as in \cref{sec:sound-complete-ryw}.
Since \( \VIS[\aexec] ; \SO \subseteq \AR[\aexec] ; \VIS[\aexec] \subseteq \VIS[\aexec] \)
\( \ViewShift[\MR] \) can be proven in the same way as in \cref{sec:sound-complete-mr}.

\paragraph{Strict serialisability (\(\SER\))}  
This model is the strongest consistency model
in any framework that abstracts from aborted transactions, 
requiring that transactions execute in a total sequential order.
The \(\CanCommit[\SER]\) thus allows a client to commit a transaction only 
when the client view on the kv-store is complete
in that the view is closed with respect to \(\WWInv[\kvs]\). 
This requirement prevents the kv-store in  \cref{fig:ser-disallowed}.
Without loss of generality, suppose that \(\txid\) commits before \(\txid'\),
then the client committing \(\txid'\) must see the version of \(\key_1\) written by \(\txid\), 
and thus cannot read the outdated value \(\val_0\) for \(\key_1\). 
This example, known as \emph{write skew anomaly}, 
is allowed by all other execution tests in \cref{fig:execution-tests}.

\begin{figure}
\centering
\begin{tikzpicture}%
\KVMapping{x}{\key_1}{
    /\val_0/\txid_0/\Set{\boldsymbol{\txid'}}
    , /\val_1/\txid/\emptyset
};
\KVMapping[x]{y}{\key_2}{
    /\val_0/\txid_0/\Set{\txid}
    , /\val_2/\boldsymbol{\txid'}/\emptyset
};
\end{tikzpicture}%

\hrulefill

\caption{Write skew anomaly, disallowed by \(\SER\)}
\label{fig:ser-disallowed}
\end{figure}%

