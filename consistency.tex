\section{Consistency Guarantees}
\label{sec:cm}
In this section we address the topic of what it means for a kv-store 
to be in a consistent state. Different consistency models of transactions have 
been proposed in the literature \cite{ev_principles,rola,cops,redblue,PSI,clocksi}, which capture different trade-offs 
between transaction processing performance and application correctness. 

For example, the consistency model known as \emph{serialisability} allows only kv-stores 
obtainable from a serial execution of transactions, where transactions read the most up-to-date 
versions of a key. While this consistency model gives strong consistency guarantees, by ruling out scenarios 
such as the one depicted in \cref{fig:ua-disallowed}, it has obvious performance drawbacks due to client 
having to wait for the most up-to-date version of a key. At the other side of the spectrum, databases 
implementing \emph{eventual consistency} impose few conditions on the structure of a kv-store 
and achieve very fast transaction processing, at the expenses 
of allowing a wide range of non-serialisable kv-stores, or \emph{anomalies}.

Formally, we define a \emph{consistency model} to be a set of kv-stores. We introduce the notion of 
\emph{execution test}, which specifies whether a client is allowed to commit a transaction in a given 
kv-store. Each execution test induces a consistency model as the set of kv-stores that 
can be obtained when clients non-deterministically commit transactions that respect the constraints 
imposed by the execution test. We identify several execution tests that induce well-known consistency 
models in the literature; later in \cref{sec:other_formalisms} we demonstrate that our definitions are equivalent 
to declarative definitions over abstract executions \cite{framework-concur} and dependency graphs \cite{adya}.

%Consistency guarantees of distributed databases describe
%what it means for distributed data to be consistent. 
%They have been formally described axiomatically via dependency graphs~\cite{adya-icde,adya}
%and abstract execution graphs~\cite{ev_transactions,framework-concur}. 
%We formalise the consistency guarantees of our centralised kv-stores by defining a 
%\emph{consistency model}. 
%A consistency model is a set of kv-stores capturing the possible outcomes 
%obtained when multiple clients commit several transactions each, 
%provided that the effects of such transactions comply with the consistency guarantees of the underlying consistency model. 
%To this end, we define consistency models induced by an \emph{execution test}.
%An execution test is a relation which determines whether a client may commit a transaction into a kv-store.  
%We formulate several well-known consistency models over our centralised kv-stores 
%by defining their corresponding execution tests. 
%Later in \cref{sec:other_formalisms} we demonstrate that our definitions over centralised kv-stores are equivalent 
%to their existing definitions over distributed databases.




%An execution test is a set $\ET$ of tuples of the form $(\mkvs, \vi, \fp, \mkvs', \vi')$,
%denoting that a client with view $\vi$ on kv-store $\mkvs$  may commit an atomic transaction 
%with fingerprint $\fp$, and obtain an updated kv-store \( \mkvs' \) and an updated view $\vi'$. 
%We often write
%$\ET \vdash (\mkvs, \vi) \csat \fp: ( \mkvs', \vi')$ for
%$(\mkvs, \vi, \fp, \mkvs', \vi') \in \ET$.

\begin{definition}
\label{def:execution.test}
An \emph{execution test} is a set of tuples $\ET \subseteq \MKVSs \times \Views \times \Fingerprints \times \MKVSs \times \Views$ 
such that for all $(\mkvs, \vi, \fp, \mkvs', \vi') \in \ET$ and all $\key, \val$:
%
{%
\begin{align}
\small
    & 
	(\otR, \key, \val) \in \fp \implies
	\mkvs(\key, \max{}_{<}(\vi(\key))) = \val  
	\tag{Ext} \label{eq:read-external} \\
    & 
    \vi(\key) \neq \vi'(\key) 
    \implies
    \exsts{l} (l, \key, -) \in \fp
    \tag{ValidViewUpd} \label{eq:valid-view-update}
\end{align}%
}%
\end{definition}
%
\noindent 
Intuitively, $(\mkvs, \vi, \fp, \mkvs', \vi') \in \ET$ means that under the execution test $\ET$ 
a client with view $\vi$ over a kv-store $\mkvs$ is allowed to commit a transaction with 
fingerprint $\fp$. The kv-store and view $\mkvs', \vi'$ reflect how $\mkvs, \vi$ change due 
to the transaction being executed, respectively. We adopt the more suggestive notation 
$\ET \vdash (\mkvs, \vi) \csat \fp: (\mkvs', \vi')$ in place of $(\mkvs, \vi, \fp, \mkvs', \vi') \in \ET$.
The first condition of \cref{def:execution.test} enforces the last-write-wins policy~\cite{vogels:2009:ec:1435417.1435432}: 
a transaction always reads the most recent writes from the initial view \(\vi\).  
The second condition states that a transaction is only allowed to update the view for those keys 
that have been recorded in the fingerprint.  The largest execution test is denoted as $\ET_{\top}$: 
in \cref{sec:other_formalisms} we prove that the set of kv-stores induced by $\ET_{\top}$ 
corresponds to the \emph{Read Atomic} consistency model \cite{ramp}. Later, in this Section, 
we define several execution tests that induce well-known consistency models.
%Note that at this initial stage \eqref{eq:read-external} and \eqref{eq:valid-view-update} are the only required conditions and execution tests are otherwise unrestricted. 
%Further restrictions on execution tests are determined by the underlying consistency model, thus prescribing the consistency guarantees of the model.

We now explain how an execution test induces a consistency model. First, 
we define $\ET$-reductions, which capture how a client $\cl$ may interact with $\ET$-configurations. 
An $\ET$-reduction is a triple of the form $(\mkvs, \vienv) \toET{(\cl, \alpha)} (\mkvs', \vienv')$, 
where given an execution test  $\ET$, either 
%we define the $\ET$-trace as a sequence of $\ET$-reductions on configurations that a client
\begin{enumerate}
    \item $\alpha = \varepsilon$, $\mkvs' = \mkvs$ and $\vienv' = \vienv\rmto{\cl}{\vi}$ for some $u: \vienv(\cl) \sqsubseteq u$ - 
	client $\cl$ advances its view to a more up-to-date one, the label $\varepsilon$ denotes that there was no interaction between the client and the kv-store; or 
\item $\alpha = \fp$ for some $\fp$, $\ET \vdash (\mkvs, \vi ) \csat \fp : (\mkvs', \vi')$ for some $\vi'$, $\mkvs' \in \updateKV[\mkvs, \vi, \fp, \cl]$, 
    $\vienv' = \vienv\rmto{\cl}{\vi'}$; client $\cl$ 
	commits a transaction with fingerprint $\fp$.
\end{enumerate}
\ac{Not changing due to time constraints, but I believe the best thing to do is to change the type of execution tests to have a client identifier 
rather than $\mkvs'$, since this is determined by $\mkvs$ and $\cl$. In fact, I believe that the best we can do is to change execution tests to 
have the form $\ET \vdash (\mkvs, \vienv) {\csat}_{\cl} \fp : \vi'$; then one can lift this to a $\ET$-reduction in a style similar to monadic lifting.}

We refer to a finite sequence of $\ET$-reductions starting in an initial configuration $\conf_{0}$ as a $\ET$-trace. 
Each $\ET$-trace terminates in a configuration $(\mkvs, \stub)$: the 
kv-store $\mkvs$ is obtained as a result of several clients committing transactions under the 
execution test $\ET$, and therefore it contributes to the consistency model induced by $\ET$, 
denoted as $\CMs(\ET)$.
%
%\begin{definition}[$\ET$-trace]
%\label{def:reduction}
%An \emph{action} $\alpha \in \Act$ is either of the form $(\cl, \varepsilon)$ or $(\cl, \fp)$, 
%where $\cl$ is a client and $\fp$ is a fingerprint. 
%Given an execution test $\ET$, the $\ET$-\emph{reduction relation},
%$\toET{} \subseteq \Confs \times \Act \times \Confs$, 
%is the smallest relation such that for all $\cl, \mkvs, \mkvs', \vienv, \fp, \vi', \vi''$ and $\vi = \vienv(\cl)$:
%%
%{
%\[
%\small
%\begin{array}{@{}l@{}}
%    \vi \sqsubseteq \vi'' 
%    \land \mkvs' \in \updateKV[\mkvs, \vi'', \fp, \cl]
%    \land (\mkvs, \vi'', \fp, \mkvs', \vi') \in \ET \\
%    \quad \implies
%    (\mkvs, \vienv) \toET{\hspace{-5pt}(\cl, \varepsilon)\hspace{-5pt}} 
%    (\mkvs, \vienv\rmto{\cl}{\vi''}) \toET{\hspace{-5pt}(\cl,\fp)\hspace{-5pt}} (\mkvs', \vienv\rmto{\cl}{\vi'})
%\end{array}
%\]
%}
%%
%Given an execution test $\ET$, an \emph{$\ET$-trace} is a sequence of $\ET$-reductions of the form $\conf_{0} \toET{\alpha_{0}} \cdots \toET{\alpha_{n-1}} \conf_{n}$. $\ET$-traces are ranged over by $\tau, \tau', \cdots$; 
%given a $\ET$-trace $\tau$, $\lvert \tau \rvert$ denotes the number of $\ET$-reductions in $\tau$, and 
%for $i=1,\cdots,n$, $\tau(i)$ denotes the $i$-th reduction of $\tau$.
%\end{definition}
%
%A \emph{consistency model} induced by $\ET$ is a set of kv-stores
%resulting from $\ET$-traces starting in an 
%initial configuration. 

\begin{definition}[Consistency Model]
\label{def:cm}
The \emph{consistency model} induced by an execution test $\ET$ is defined as 
\(
\CMs(\ET) \defeq 
\Set{\mkvs}[ 
\exsts{\conf_0 \in \Confs_0}
\conf_0 \toET{\stub}^{*} (\mkvs, \stub)
]
\).
%Given an execution test $\ET$,  
%the set of \emph{configurations induced by $\ET$},  $\confOf[\ET]$, is   given by: 
%\(
%\confOf[\ET] \defeq 
%\Set{\conf}[ 
%\exsts{\conf_0 \in \Confs_0}
%\conf_0 \toET{\stub}^{*} \conf
%]
%\).
%The \emph{consistency model} induced by $\ET$ is defined as:
%\( 
%\CMs(\ET) \defeq \Set{ \mkvs }[ (\mkvs, \stub) \in \confOf[\ET] ]
%\)
\end{definition}
%
%\noindent In~\cref{sec:mono-et}, we prove that consistency models are 
%\emph{monotonic}: 
%if  $\ET_1 \subseteq \ET_2$ then $\CMs(\ET_1) \subseteq \CMs(\ET_2)$.

Note that in $\ET$-traces the two operations performed by client, view-shift and 
transaction commit, are decoupled. This is in contrast with what happens in the operational model (\cref{sec:model}), 
where view-shifts and transaction commits are coupled together in a single transition of programs. 
The reason for this mismatch is best understood when looking at intended applications of 
$\ET$-traces and the operational model. The former are particularly useful when proving that a distributed 
protocol implements a given consistency model: in this case it is convenient to separate the operation of shifting a view from that of shifting a transactions, 
as these two operations often take place separately in a distributed transactional protocol. The latter is particularly useful when reasoning about transactional 
programs: treating view-shifts and transaction commits as a single transition reduces the number of interleavings in programs.

$\ET$-traces and the operational are equally expressive: 
any kv-store $\mkvs \in \CMs(\ET)$ is obtained as a result of 
executing some program $\prog$ under the execution test $\ET$, and vice versa.
\ac{One may wonder whether this difference in approach leads to a difference in expressive power between $\ET$-traces 
and the operational model: that is, if there exists a kv-store $\mkvs \in \CMs(\ET)$ that can never be obtained as a 
result of executing an arbitrary program $\prog$ under the execution test $\ET$. The next result shows that 
this is not the case.}
%We will prove shortly that this is 
%not the case: let us say that a $\ET$-trace  $\conf_{0} \toET{(\cl_{0}, \alpha_{0})} \cdots \toET{(\cl_{n}, \alpha_{n})} \conf_{n+1}$ is in \emph{normal form} if 
%whenever, for any $i=0, \cdots, n-1$, if $\alpha_{i} = \varepsilon$ then either $\cl_{i+1} = \cl_{i}$, and $\alpha_{i+1} = \fp$ 
%for some $\fp$. For any program $\prog$, we let $\interpr{\prog}_{\ET}$ be the set of 
%kv-stores reachable by executing $\prog$ under the execution test $\ET$.
%
%\begin{theorem}
%For any execution test $\ET$, and $\mkvs \in \CMs(\ET)$, there exists a 
%$\ET$-trace $\conf_{0} \toET{\stub}^{\ast} \conf$ in normal form such that 
%$\conf = (\mkvs, \stub)$.
%\end{theorem}

\begin{theorem}
	Let $\interpr{\prog}_{\ET}$ be the set of kv-stores reachable by executing $\prog$ under the execution test $\ET$. 
	Then for all $\ET$, $\CMs(\ET) = \bigcup_{\prog} \interpr{\prog}$.
	%$\CMs(\ET) = \Set{\mkvs}[\exsts{\mkvs_0, \vienv_0, \thdenv_0, \prog} (\mkvs_0, \vienv_0, \thdenv_0), \prog \toPROG{\stub}^{*} (\mkvs, \stub, \stub), \stub]$.
\end{theorem}

%\mypar{Compositionality}
%We examine the \emph{compositionality} of the consistency models induced by execution tests:  
%\ie given two execution tests $\ET_1, \ET_2$, does 
%$\CMs(\ET_1 \cap \ET_2) = \CMs(\ET_1) \cap \CMs(\ET_2)$ hold? 
%The monotonicity of execution tests guarantees that 
%%It is straightforward to show that the left-to-right direction holds:
% for all $\ET_1, \ET_2$, \( \CMs(\ET_1 \cap \ET_2) \subseteq \CMs(\ET_1) \cap \CMs(\ET_2) \). 
%However, the other direction \( \CMs(\ET_1) \cap \CMs(\ET_2) \subseteq \CMs(\ET_1 \cap \ET_2) \) does not hold for arbitrary consistency models.
%Consider the following:
%%
%\[
%\small
%\begin{array}{@{}l @{\hspace{2pt}} | @{\hspace{2pt}} l@{}}
%    \hline
%    \ET_1 & \ET_2 \\
%%    
%    \hline
%    (\mkvs_{0}, \vi_{0}) \csat\! \Set{(\otW, \key, 1)} {:} ( \mkvs_{\key}, \vi_{0})
%    &
%    (\mkvs_{0}, \vi_{0}) \csat\! \Set{(\otW, \key', 1)} {:} ( \mkvs_{\key'}, \vi_{0}) 
%    \vspace*{-7pt}\\\\
%% %   
%    (\mkvs_{\key}, \vi_{0}) \csat  \!\Set{(\otW, \key', 1)} {:} (\mkvs',\vi_{0}) 
%    &
%    (\mkvs_{\key'}, \vi_{0}) \csat \!\Set{(\otW, \key, 1)} {:} (\mkvs',\vi_{1}) 
%    \\
%\hline
%\end{array}
%\]%
%%
%with%
%%
%{
%\(
%\small
%\begin{array}[t]{l@{} l}
%    \mkvs_{\key} & = \mkvs_{0}[\key \mapsto (0, \txid_{0}, \emptyset) \lcat (1, \_, \emptyset)] \\
%    \mkvs_{\key'} & = \mkvs_{0}[\key' \mapsto (0, \txid_{0}, \emptyset) \lcat (1, \_, \emptyset)] \\
%    \mkvs' & = \mkvs[\key \mapsto (0, \txid_{0}, \emptyset) \lcat (1, \_, \emptyset) 
%                ,\key' \mapsto (0, \txid_{0}, \emptyset) \lcat (1, \_, \emptyset)] \\
%\end{array}
%\)%
%}%
%%
%
%\noindent As both $\ET_1$ and $\ET_2$ allow a version with value $1$ to be written for 
%$\key, \key'$,  we have $\mkvs' \in \CMs(\ET_1) \cap \CMs(\ET_2)$. 
%However, $\ET_1$ and $\ET_2$ enforce a different order in which the writes on $\key, \key'$ must happen; 
%thus $\mkvs' \notin \CMs(\ET_1 \cap \ET_2)$. 
%
%In this example, compositionality fails because execution tests 
%enforced a particular order in which the updates must be committed, even though such updates 
%are non-conflicting: the kv-store obtained after committing such updates is independent of the commit order. 
%This observation is captured in the following definition: 
%
\begin{figure*}[!t]
\small
\begin{tabularx}{\textwidth}{ @{} X r r ||  X  r r @{} }
\hline
Model & Initial \( \txidset \) & Relation \( R \) &
Model & Initial \( \txidset \) & Relation \( R \)
\\
\hline
\MR & \(\clRead[\mkvs,\cl]\) & \( \emptyset \)
&
\UA &   \( \uaWrite[\mkvs,\fp] \) & \( \bigcup_{(\otW,\key,\stub) \in \fp} \WW_{\mkvs}(\key) \)
\\
\MW & \( \emptyset \) & \( \SO\rflx \)
&
\PSI & \( \dagger \cup \uaWrite[\mkvs,\fp] \) & \( \SO \cup \WR_{\mkvs} \cup \WW_{\mkvs}\)
\\
\RYW & \( \clWrite[\mkvs,\cl] \) & \( \emptyset \)
&
\CP & \( \dagger  \)  &\(  \ddagger  \)
\\
\WFR & \( \emptyset \) & \( \WR_{\mkvs} ; \SO\rflx \)
&
$\SI$ & \( \dagger \cup \uaWrite[\mkvs,\fp] \) & \( \ddagger \cup (\WW_\mkvs; \RW_\mkvs) \)
\\
\CC & \( \dagger \)  & \( \SO \cup \WR_{\mkvs} \)
&
\SER & \( \Set{\txid}[\txid \in \mkvs] \) & \( \emptyset \) \\
\hline
\end{tabularx}%
%
\begin{align*}
    \dagger & \equiv 
    \clRead[\mkvs,\cl] \cup \clWrite[\mkvs,\cl] 
    \quad \ddagger 
    \equiv 
    (\SO ; \RW_{\mkvs}\rflx) \cup (\WR_{\mkvs} ; \RW_{\mkvs}\rflx) \cup \WW_\mkvs 
\end{align*}
%
\begin{align*}
    \WR_{\mkvs} &\defeq \bigcup_{\key \in \Keys} \WR_\mkvs(\key)
    & \WR_\mkvs(\key) & \defeq
    \Set{ (\txid, \txid') }[ \exsts{ i }\txid = \wtOf(\mkvs(\key, i)) \land \txid' \in \rsOf(\mkvs(\key, i))]\\
    \WW_{\mkvs} &\defeq  \bigcup_{\key \in \Keys} \WW_\mkvs(\key) 
    & \WW_\mkvs(\key)  & \defeq 
    \Set{ (\txid, \txid') }[ \exsts{ i, j } \txid = \wtOf(\mkvs(\key, i)) \land \txid' = \wtOf(\mkvs(\key, j)) \land i < j]\\
    \RW_{\mkvs} &\defeq   \bigcup_{\key \in \Keys} \RW_\mkvs(\key)  
    & \RW_\mkvs(\key) & \defeq \Set{ (\txid, \txid') }[%
        \exsts{ i, j } \txid \in \rsOf(\mkvs(\key, i)) \land \txid' = \wtOf[\mkvs(\key, j)]%
        \land i < j \land \txid \neq \txid'%
    ]
\end{align*}
%
\begin{align*}
    \lfpTx[\mkvs,\txidset, R] & \defeq \mu X . \txidset \cup R^{-1}(X) \\
    \extRead[\mkvs,\vi,\fp] & \defeq \Set{\wtOf[\mkvs(\key,\max(\vi(\key)))]}[%
        \fora{\key,\val} (\otR,\key,\val) \in \fp 
        \implies 
        \val = \valueOf[\mkvs(\key,\max(\vi(\key)))]%
    ] \\
    \clRead[\mkvs,\cl] & \defeq \Set{\wtOf[\mkvs(\key,i)]}[ \exsts{n} \txid_{\cl}^{n} \in \rsOf[\mkvs(\key,i)] ] 
    \quad \clWrite[\mkvs,\cl] \defeq \Set{\wtOf[\mkvs(\key,i)]}[ \exsts{n} \txid_{\cl}^{n} \in \wtOf[\mkvs(\key,i)] ] \\
    \Tx[\mkvs, \vi] & \defeq 
    \Set{ \wtOf[\mkvs(\key, i)] }[ \key \in \Keys \land i \in \vi(\key) ]
    \quad 
    \getView[\mkvs, \txidset] \defeq 
    \lambda \key. \Set{0} \cup \Set{ i }[\wtOf(\mkvs(\key, i)) \in \txidset] \\
    \uaWrite[\mkvs,\fp] & \defeq \Set{\wtOf[\mkvs(\key,\abs{\mkvs(\key)} - 1)]}[\exsts{\val} (\otW,\key,\val) \in \fp] 
\end{align*}
%
\hrulefill
%
\caption{Execution tests of client-centric (left) and data-centric (right) consistency models, 
with $\SO$ as defined in \cref{subsec:kvstores}. 
All free variables are universally quantified.
}
\label{fig:execution.tests}
\label{fig:execution_tests}
\label{fig:execution-tests}
\end{figure*}


%
%
%\begin{definition}
%\label{def:et-comm}
%Two fingerprints $\fp_1$ and $\fp_2$ are \emph{conflicting} 
%iff there exists $\key$ such that 
%$(\otW, \key, -) \in \fp_1 \land (\otW, \key, -) \in \fp_2$. 
%An execution test $\ET$ is \emph{commutative}, written $\com{\ET}$, if 
%for all \( \mkvs, \mkvs', \vienv, \vienv',\vienv''\), distinct clients \( \cl_1, \cl_2 \), non-conflicting fingerprints \( \fp_1, \fp_2  \) and \( \vi_1, \vi_2 \in \Views(\mkvs) \):% 
%%
%{%
%\[
%\small
%\begin{array}{@{}r @{\hspace{10pt}} l @{}}
%	\text{if} &  
%	(\mkvs, \vienv) \toET{(\cl_1, \fp_1)} 
%	\stub \toET{(\cl_2, \fp_2)} (\mkvs', \vienv') \\
%	\text{then} & (\mkvs, \vienv) \toET{(\cl_2, \fp_2)}
%    \stub \toET{(\cl_1, \fp_1)} (\mkvs', \vienv'') \\
%    & {} \land \vienv''(\cl_2) \viewleq \vienv'(\cl_2)
%\end{array}
%\]%
%}%
%\end{definition}
%
%To guarantee the compositionality of two execution tests $\ET_1, \ET_2$, we 
%require at least one of those to be commutative, say $\ET_1$. The main idea 
%is the following: fix a $\ET_1$-trace $\tau_1$ and a $\ET_2$-trace $\tau_2$, both terminating in a configuration 
%of the form $(\mkvs, \stub)$; then we construct a $(\ET_1 \cap \ET_2)$-trace terminating 
%in a configuration of the same form by re-ordering he sequence of 
%reductions of $\tau_1$ as to match exactly the sequence of 
%reductions of $\tau_2$. 
%In \cref{sec:et-comp}, we show that if $\ET_1$ is commutative, 
%we can indeed re-order the sequence of reductions in the 
%$\ET_1$-trace leading to a trace $\tau_1'$ such that $(\lvert \tau'_1 \rvert = \lvert \tau_2 \rvert)$\footnote{In fact, 
%to ensure that $\lvert \tau_1' \rvert = \lvert \tau_2 \rvert$ we require to further manipulate 
%$\tau_1$ prior to re-ordering its sequence of $\ET_1$-reductions.}, 
%and for any 
%$i=1,\cdots, \lvert \tau_1'\rvert$, the pre and post kv-store of $\tau_1'(i)$,
% coincide with the action, pre and post kv-store 
%of $\tau_2(i)$.
%
%
%Commutativity alone does not ensure that, for $i=1,\cdots,\lvert \tau_2 \rvert$ the 
%pre-views and post-views of $\tau_1'(i)$ 
%match the pre-views and post-views of the $\tau_2(i)$, which is necessary to show 
%that $\tau_1'$ and $\tau_2$ can be recast as a $(\ET_1 \cap \ET_2)$-trace. 
%In \cref{sec:et-comp} we present three other basic requirements to be 
%satisfied by $\ET_1$ and $\ET_2$, that guarantee that $\tau'_1(i)$ and 
%$\tau_2(i)$ agree on the pre-views and post-views for $i=1,\cdots, \lvert \tau_2 \rvert$. 
%The first two of these requirements,  \emph{no blind writes} and \emph{minimum footprint}, 
%ensure that the pre-views of the reductions $\tau_1'(i)$ and $\tau_2(i)$ match, 
%while the third requirement, which we call \emph{monotonic post-views}, 
%guarantees that the post-views of the reductions $\tau_1'(i)$ and $\tau_2(i)$ 
%match. 
% 
%\begin{theorem}[Compositionality]  
%\label{thm:compositional}   
%For all $\ET_1, \ET_2$ with no blind writes, minimum footprints and monotonic post-views: 
%if $\com{\ET_1}$, 
%then $\CMs(\ET_1 \cap \ET_2) {=} \CMs(\ET_1) \cap \CMs(\ET_2)$;
%if $\com{\ET_1} \land \com{\ET_2}$, then $\com{\ET_1 \cap \ET_2}$.
%\end{theorem}
%
%Most of the execution tests associated with well-known consistency models (introduced shortly)
%can be tweaked to satisfy no-blind writes, minimum footprints and monotonic post-views 
%without altering their semantics. However, some of these execution tests
%are inherently non-commutative.


\subsection{Execution Tests for Well-known Consistency Models}\label{subsec:cm_examples}

Execution tests can be used to capture well-known consistency models. 
Here we give an account of several execution tests, from weakest 
to strongest.
The execution tests that we present are defined by 
putting constraints on the view of clients either before or after to committing a transaction: 
such constraints impose properties of the form \emph{if the client 
observes version $\ver$ in the kv-store, then it must also observe 
the set of versions $\func{f}[\ver]$, for some suitable notion of $\func{f}$}.
In practice, this is achieved by requiring views to be prefix-closed with respect 
to some relation that is expressed in terms of dependencies between transactions.
We present the formal notions of transaction dependencies, inspired from 
\cite{adya}, and the definition of closed view, before turning to the formal definitions
of execution tests. 

Let $\mkvs$ be a kv-store; for an arbitrary key $\key$ and 
two indexes $i,j: 0 \leq i < j < \abs{ \mkvs(\key) }$, let 
$(\stub, \txid_{i}, \T_{i}) = \mkvs(\key, i)$, $(\stub, \txid_{j}, \stub) = \mkvs(\key,j)$. 
For any $\txid \in \T_{i}$, we say that there is 
\begin{enumerate*} 
\item a \emph{Write-Read} dependency over 
$\key$ from $\txid_{i}$ to $\txid$, written $\txid_{i} \xrightarrow{\WR_{\mkvs}(\key)} \txid$, 
\item a \emph{Write-Write} dependency over $\key$ from $\txid_{i}$ to $\txid_{j}$, 
written $\txid_{i} \xrightarrow{\WW_{\mkvs}(\key)} \txid_{j}$, and 
\item a \emph{Read-Write} anti-dependency from $\txid$ to $\txid_{j}$, provided that 
$\txid \neq \txid_{j}$, written $\txid \xrightarrow{\RW_{\mkvs}(\key)} \txid_{j}$.
\end{enumerate*}

Note that, although we define dependencies between transactions in a style 
similar to Adya's dependency graphs \cite{adya}, the way in which they are used 
in kv-stores is fundamentally different in its intent. Definitions of consistency 
models given in terms of dependency graphs are declarative: as 
we will see in \cref{sec:other_formalisms} they are equivalent to impose 
constraints on the structure of a kv-store. On the other hand, the execution 
tests use dependencies between transactions to restrict 
the view of clients before and after committing a transaction. In other words, 
whereas Adya's dependency graphs can be used to define a consistency model 
in terms of what is the permissible structure of kv-stores, we use execution tests
to define how such kv-stores can be constructed, in a way that is agnostic 
of the structure of their structure.

\ac{Not sure whether this comment should be here. Maybe it should go in Section 5.
\sx{I think it is a good idea to put here for those readers knowing dependency graphs.}
}

Next, we observe that a view $\vi$ over a kv-store $\mkvs$ naturally induces a 
set of \emph{visible transactions} $\Tx[\mkvs, \vi] \defeq \Set{\wtOf[\mkvs(\key, i] }[ i \in \vi(\key)]$ 
that contains all the transactions whose updates are known to a client with view $\vi$. 
Conversely, given a set of visible transaction $\txidset$ and a kv-store $\mkvs$, it is 
possible to recover the view of the client
who observes updates made by transactions from \( \txidset \):
%having $\T$ has a set of visible transactions: 
$\getView[\mkvs, \T] \defeq \lambda \key. \Set{0} \cup \Set{i }[ \wtOf[\mkvs(\key, i)] \in \T]$. 
The reflection between views and sets of transactions allows to express concisely 
prefix-closure of a view with respect to a relation $\rel$ between 
transactions: given $R \subseteq \TxID \times \TxID$ the predicate
$\closed(\mkvs, \vi, R)$ is defined by $\closed(\mkvs, \vi, R) \defeq \vi = \getView[\mkvs, (R^{-1})^{\ast}(\Tx(\mkvs, \vi))]$.

We are now ready to present the execution tests  associated with different consistency models. 
We start by looking at session guarantees, then we move to stronger consistency models. 
All the formal definitions are given in \cref{fig:execution.tests}. 
%In that table we 
%commit an abuse of notation and let $\closed(\mkvs, \vi, R)$ be the execution test 
%$\{(\mkvs, \vi, \fp, \mkvs', \vi') \mid \closed(\mkvs, \vi, R)\}$.

%Following \cite{distrprinciples}, we distinguish between
%client- and data-centric consistency models.
%Many of the execution tests we present
%We now give examples of execution tests in~\cref{fig:execution.tests},
%where the associated consistency models for kv-stores correspond to
%widely adopted consistency guaranteees for distributed databases.
%Following \cite{distrprinciples}, we distinguish between
%client- and data-centric consistency models: 
%the former constrain the client views; 
%the latter impose conditions on the structure of the kv-store.  
%In \cref{fig:anomalies} we give illustrative
%examples of kv-stores allowed/disallowed by our
%consistency models.

%\sx{Explain functions used in the table}

\mypar{Monotonic Reads ($\MR$)}
This consistency model states that a client
% cannot lose information from the view and hence 
can only see increasingly more up-to-date versions from a kv-store. 
This prevents, for example, the kv-store of \cref{fig:mr-disallowed},
since client $\cl$ first reads the latest version of $\key$ in $\txid_{\cl}^{1}$, 
and then reads the older, initial version of $\key$ in $\txid_{\cl}^{2}$.  
The execution test $\ET_{\MR}$ ensures that clients  can only extend their views. 
Note that the execution test $\ET_{\MR}$ can also be expressed as a closure 
property of the view obtained after committing a transaction: 
let $\rel_{\MR}(\mkvs, \vi) \defeq \Set{(\wtOf[\mkvs(\key, i), \txid] }[ \key \in \Keys \wedge i \in \vi(\key) \wedge \txid \in \mkvs]$; 
%given \(\mkvs, \vi\) before committing a transaction and $\mkvs', \vi'$ after,
the predicate $\closed[\mkvs', \vi', R_{\MR}(\mkvs, \vi)]$ states that all the versions included in $\vi$ on $\mkvs$ 
must also be included in $\vi'$ on $\mkvs'$ (\ie $\vi \sqsubseteq \vi'$), hence 
$\ET_{\MR} \defeq \Set{(\mkvs, \vi, \opset, \mkvs', \vi') }[ \closed[\mkvs', \vi', \rel_{\MR}(\mkvs, \vi)]]$.

\mypar{Monotonic Writes ($\MW$)}
In this consistency model, whenever a transaction sees a version for some key \( \key \) written by a client $\cl$,
then it sees all previous versions for the key \( \key \) written by $\cl$. 
This prevents, for example, the kv-store of \cref{fig:mw-disallowed}, since 
transaction $\txid$ reads the third version of $\key_2$, 
with value $\val_3$ written by client $\cl$, 
but it does not read, hence does not see, the second version of $\key_1$
with value $\val_1$ and previously written by the same client via transaction \( \txid^1_{\cl} \)
who also have written the second version of \( \key_2 \).
We can enforce monotonic writes by requiring the view of a client, prior to committing 
a transaction, to be closed with respect to the session relation $\SO$.
\ac{Why the reflexive closure? \sx{New view should include the new transaction.}}
%The execution test $\ET_{\MW}$  ensures that, prior to executing a transaction,
%the set of versions included in the view of the client are write 
%prefix-closed with respect to the relation $\SO\rflx$.

\mypar{Read Your Writes (\RYW)}
In this consistency model a client must always see the versions previously written by the client itself. 
The execution test $\ET_{\RYW}$ enforces the read-your-writes session guarantee by 
mandating that, after a client executes a transactions, its view contains all the writes of the client 
itself; this ensures these versions will be included in the view of the client when committing future 
transactions.
The execution test $\ET_{\RYW}$ prevents the kv-store in \cref{fig:ryw-disallowed}, 
as the initial version of $\key$ holds value $0$ 
and client $\cl$ tries to increment the value of $\key$ by $1$ twice.  
For its first transaction, it reads the initial value $0$ and then writes a new version with value $1$. 
For its second transaction, since the client need not see its own writes, 
it might read the initial value $0$ again and write a new version with value $1$.
The execution test $\RYW$ ensures that, after committing a transaction, 
the client view includes all the versions it wrote.  
Note that the execution test $\ET_{\RYW}$ can also be expressed 
as a closure property of the view after the transaction commit: given 
two kv-stores $\mkvs, \mkvs'$, let  
$\mkvs' \setminus \mkvs$ be the set of transactions appearing 
in $\mkvs'$ but not in $\mkvs$ --- note that 
%in tuples of the form 
%$(\mkvs, \vi, \fp, \mkvs', \vi')$ taken from execution tests, $\mkvs' \setminus \mkvs$ 
in our operational model this set is a singleton set $\Set{\txid}$ where $\txid$ is the newly committed transaction;
%that committed when updating the kv-store from $\mkvs$ to $\mkvs'$; 
for a given set of transactions $\txidset$, let $\rel_{\RYW}(\txidset) \defeq \Set{(\txid', \txid) }[ \exsts{ \txid \in \T } (\txid', \txid) \in \SO\rflx ]$. 
Then the predicate $\closed[\mkvs', \vi', \rel_{\RYW}(\mkvs' \setminus \mkvs)]$ states that a 
view $\vi'$ includes all the writes performed by the clients that have committed new transactions in $\mkvs'$, 
and therefore $\ET_{\RYW} \defeq \Set{(\mkvs, \vi, \fp, \mkvs', \vi') }[ \closed[\mkvs', \vi', R_{\RYW}(\mkvs' \setminus \mkvs)]]$.
The defining for \( \ET_{\RYW} \) given in \cref{fig:execution-tests} is equivalent.

\begin{figure*}[t]
\captionsetup[subfigure]{aboveskip=-5pt, belowskip=5pt}
\begin{tabular}{@{} c | c | c @{}}
\hline
\phantom{-}& \phantom{-}& \phantom{-}\\
\begin{subfigure}{0.2\textwidth}
\begin{centertikz}

%Location x
\node(locx) {$\ke_1 \mapsto$};
\draw pic at ([xshift=\tikzkvspace]locx.east) {vlist={versionx}{%
    /$\val_0$/$\txid_0$/$\Set{\txid_\cl^2}$
    , /$\val_1$/$\txid_1$/$\Set{\txid_\cl^1}$
}};

\end{centertikz}\vspace{5pt}%
\caption{Disallowed by \(\MRd\)}
\label{fig:mr-disallowed}
\end{subfigure}
%\quad
&
\begin{subfigure}{0.36\textwidth}
\begin{centertikz}

%Location x

\node(locx) {$\ke_1 \mapsto$};
\draw pic at ([xshift=\tikzkvspace]locx.east) {vlist={versionx}{%
    /$\val_0$/$\txid_0$/$\Set{\txid'}$
    , /$\val_1$/$\txid_\cl^1$/$\emptyset$
}};

%Location y
\path (versionx.east) + (1,0) node (locy) {$\ke_2 \mapsto$};
\draw pic at ([xshift=\tikzkvspace]locy.east) {vlist={versiony}{%
    /$\val_0$/$\txid_0$/$\emptyset$
    , /$\val_2$/$\txid_\cl^2$/$\Set{\txid'}$
}};

\end{centertikz}\vspace{5pt}
\caption{Disallowed by \(\MW\)}
\label{fig:mw-disallowed}
\end{subfigure}
%\quad
&
\begin{subfigure}{0.39\textwidth}
\begin{centertikz}

%Location x
\node(locx) {$\ke_1 \mapsto$};
\draw pic at ([xshift=\tikzkvspace]locx.east) {vlist={versionx}{%
    /$\val_0$/$\txid_0$/$\Set{\txid}$
    , /$\val_1$/$\txid'$/$\Set{\txid_\cl^1}$
}};

%Location y
\path (versionx.east) + (1,0) node (locy) {$\ke_2 \mapsto$};
\draw pic at ([xshift=\tikzkvspace]locy.east) {vlist={versiony}{%
    /$\val_0$/$\txid_0$/$\emptyset$
    , /$\val_2$/$\txid_\cl^2$/$\Set{\txid}$
}};

\end{centertikz}

\vspace{5pt}
\caption{Disallowed by \(\WFR\)}
\label{fig:wfr-disallowed}
\end{subfigure}\\
\hline
\end{tabular}
%
%
%
%
\begin{tabular}{@{} c | c | c @{}}
\phantom{-}& \phantom{-}& \phantom{-}\\
\begin{subfigure}{0.25\textwidth}
\begin{centertikz}%

%Location x
\node(locx) {$\ke_1 \mapsto$};
\draw pic at ([xshift=\tikzkvspace]locx.east) {vlist={versionx}{%
    /$0$/$\txid_0$/$\Set{\txid_\cl^1,\txid_\cl^2}$
    , /$1$/$\txid_\cl^1$/$\emptyset$
    , /$1$/$\txid_\cl^2$/$\emptyset$
}};

\end{centertikz}%
\vspace{5pt}
\caption{Disallowed by \(\RYW\)}
\label{fig:ryw-disallowed}
\end{subfigure}
& 

\begin{subfigure}{0.40\textwidth}
\begin{centertikz}%

%Location x
\node(locx) {$\ke_1 \mapsto$};
\draw pic at ([xshift=\tikzkvspace]locx.east) {vlist={versionx}{%
    /$\val_0$/$\txid_0$/$\Set{\txid_2}$
    , /$\val_1$/$\txid_1$/$\emptyset$
}};

%Location y
\path (versionx.east) + (1,0) node (locy) {$\ke_2 \mapsto$};
\draw pic at ([xshift=\tikzkvspace]locy.east) {vlist={versiony}{%
    /$\val_0$/$\txid_0$/$\Set{\txid_1}$
    , /$\val_2$/$\txid_2$/$\emptyset$
}};

\end{centertikz}%
\vspace{5pt}
\caption{Write skew, disallowed by \(\SER\)}
\label{fig:ser-disallowed}
\end{subfigure}%
&
\begin{subfigure}{0.30\textwidth}
\begin{centertikz}

\node(locx) {$\ke_1 \mapsto$};
\draw pic at ([xshift=\tikzkvspace]locx.east) {vlist={versionx}{%
    /$0$/$\txid_0$/$\Set{\txid,\txid'}$
    , /$1$/$\txid$/$\emptyset$
    , /$1$/$\txid'$/$\emptyset$
}};

\end{centertikz}
\vspace{5pt}
\caption{Lost update, disallowed by \(\UA\)}
\end{subfigure}
\\
\hline
\end{tabular}
%
%
%
%
\phantom{x}\vspace{7pt}
\begin{tabular}{@{} c | c @{}}
\phantom{-}& \phantom{-} \\
\begin{subfigure}{0.42\textwidth}
\begin{centertikz}%
%Location x
\node(locx) {$\ke_1 \mapsto$};
\draw pic at ([xshift=\tikzkvspace]locx.east) {vlist={versionx}{%
    /$\val_0$/$\txid_0$/$\Set{\txid_{\cl_2}^1}$
    , /$\val_1$/$\txid$/$\Set{\txid_{\cl_1}^1}$
}};

%Location y
\path (versionx.east) + (1,0) node (locy) {$\ke_2 \mapsto$};
\draw pic at ([xshift=\tikzkvspace]locy.east) {vlist={versiony}{%
    /$\val_0$/$\txid_0$/$\Set{\txid_{\cl_1}^2}$
    , /$\val_1$/$\txid$/$\Set{\txid_{\cl_2}^2}$
}};

\end{centertikz}%
\vspace{5pt}
\caption{Long fork, disallowed by \(\CP\)}
\label{fig:cp-disallowed-2}
\label{fig:cp-disallowed}
\end{subfigure}
&
\begin{subfigure}{0.542\textwidth}%
\begin{centertikz}%
%Location x
\node(locx) {$\ke_1 \mapsto$};
\draw pic at ([xshift=\tikzkvspace]locx.east) {vlist={versionx}{%
    /$\val_0$/$\txid_0$/$\Set{\txid_4}$
    , /$\val_1$/$\txid_1$/$\emptyset$
    , /$\val_2$/$\txid_2$/$\emptyset$
}};

%Location y
\path (versionx.east) + (1,0) node (locy) {$\ke_2 \mapsto$};
\draw pic at ([xshift=\tikzkvspace]locy.east) {vlist={versiony}{%
    /$\val_0$/$\txid_0$/$\Set{\txid_2}$
    , /$\val_3$/$\txid_3$/$\Set{\txid_4}$
    , /$\val_4$/$\txid_4$/$\emptyset$
}};

%%Location z
%\path (versiony.east) + (1,0) node (locz) {$\ke_3 \mapsto$};
%\matrix(versionz) [version list,column 2/.style={text width=7mm}]
   %at ([xshift=\tikzkvspace]locz.east) {
 %{a} & $\txid_0$ & {a} & $\txid_3$ & {a} & $\txid_4$ \\
  %{a} & $\emptyset$ & {a} & $\emptyset$ & {a} & $\emptyset$\\
%};

%\tikzvalue{versionz-1-1}{versionz-2-1}{locz-v0}{$\val_0$};
%\tikzvalue{versionz-1-3}{versionz-2-3}{locz-v1}{$\val_1$};
%\tikzvalue{versionz-1-5}{versionz-2-5}{locz-v2}{$\val_2$};

%%Location w
%\path (versionz.east) + (1,0) node (locw) {$\ke_4 \mapsto$};
%\matrix(versionw) [version list,column 2/.style={text width=7mm}]
    %at ([xshift=\tikzkvspace]locw.east) {
    %{a} & $\txid_0$ & {a} & $\txid_1$ \\
    %{a} & $\{\txid_4\}$ & {a} & $\emptyset$ \\
%};
%\tikzvalue{versionw-1-1}{versionw-2-1}{locw-v0}{$\val_0$};
%\tikzvalue{versionw-1-3}{versionw-2-3}{locw-v1}{$\val_1$};
\end{centertikz}
\vspace{5pt}
\caption{Allowed by \( \UA \) and \( \CP \) but disallowed by \(\SI\)}%
\label{fig:si-disallowed}%
\end{subfigure} \\
\hline
\end{tabular}
%\begin{tabular}{@{}c @{} c @{}}
%\begin{minipage}{0.4\textwidth}
%\begin{subfigure}{\textwidth}
%\begin{centertikz}
%%Location x
%\node(locx) {$\ke_1 \mapsto$};
%
%\matrix(versionx) [version list,,column 2/.style={text width=8mm},column 4/.style={text width=7mm}]
%    at ([xshift=\tikzkvspace]locx.east) {
%    {a} & $\txid_0$ & {a} & $\txid_{\cl}^{1}$\\
%    {a} & $\left\{\txid_{\cl'}^{2}\right\}$ & {a} & $\emptyset$ \\
%};
%\tikzvalue{versionx-1-1}{versionx-2-1}{locx-v0}{$\val_0$};
%\tikzvalue{versionx-1-3}{versionx-2-3}{locx-v1}{$\val_1$};
%
%%Location y
%\path (locx.south) + (0,\tikzkeyspace) node (locy) {$\ke_2 \mapsto$};
%\matrix(versiony) [version list,column 2/.style={text width=8mm},column 4/.style={text width=7mm}]
%   at ([xshift=\tikzkvspace]locy.east) {
% {a} & $\txid_0$ & {a} & $\txid_{\cl'}^1$ \\
%  {a} & $\left\{\txid_{cl}^{2}\right\}$ & {a} & $\emptyset$\\
%};
%
%\tikzvalue{versiony-1-1}{versiony-2-1}{locy-v0}{$\val_0$};
%\tikzvalue{versiony-1-3}{versiony-2-3}{locy-v1}{$\val_2$};
%
%\end{centertikz}
%\caption{Disallowed by \(\CP\)}
%\label{fig:cp-disallowed-2}
%\end{subfigure}
%
%\begin{subfigure}{\textwidth}
%\begin{centertikz}
%%Location x
%\node(locx) {$\ke_1 \mapsto$};
%
%\matrix(versionx) [version list,column 2/.style={text width=7mm},column 4/.style={text width=7mm}]
%    at ([xshift=\tikzkvspace]locx.east) {
%    {a} & $\txid_0$ & {a} & $\txid_1$\\
%    {a} & $\left\{\txid_2\right\}$ & {a} & $\emptyset$ \\
%};
%\tikzvalue{versionx-1-1}{versionx-2-1}{locx-v0}{$\val_0$};
%\tikzvalue{versionx-1-3}{versionx-2-3}{locx-v1}{$\val_1$};
%
%%Location y
%\path (locx.south) + (0,\tikzkeyspace) node (locy) {$\ke_2 \mapsto$};
%\matrix(versiony) [version list,column 2/.style={text width=7mm},column 4/.style={text width=7mm}]
%   at ([xshift=\tikzkvspace]locy.east) {
% {a} & $\txid_0$ & {a} & $\txid_2$ \\
%  {a} & $\left\{\txid_1\right\}$ & {a} & $\emptyset$\\
%};
%
%\tikzvalue{versiony-1-1}{versiony-2-1}{locy-v0}{$\val_0$};
%\tikzvalue{versiony-1-3}{versiony-2-3}{locy-v1}{$\val_2$};
%\end{centertikz}
%\caption{Disallowed by \(\SER\)}
%\label{fig:ser-disallowed}
%\end{subfigure}
%
%\end{minipage}
%
%&
%\begin{subfigure}{0.55\textwidth}%
%\begin{centertikz}%
%%Location x
%\node(locx) {$\ke_1 \mapsto$};
%
%\matrix(versionx) [version list,column 2/.style={text width=7mm}]
%    at ([xshift=\tikzkvspace]locx.east) {
%    {a} & $\txid_0$ & {a} & $\txid_1$ & {a} & $\txid_2$\\
%    {a} & $\emptyset$ & {a} & $\emptyset$ & {a} & $\emptyset$\\
%};
%\tikzvalue{versionx-1-1}{versionx-2-1}{locx-v0}{$\val_0$};
%\tikzvalue{versionx-1-3}{versionx-2-3}{locx-v1}{$\val_1$};
%\tikzvalue{versionx-1-5}{versionx-2-5}{locx-v1}{$\val_2$};
%%Location y
%\path (locx.south) + (0,\tikzkeyspace) node (locy) {$\ke_2 \mapsto$};
%\matrix(versiony) [version list,column 2/.style={text width=7mm}]
%   at ([xshift=\tikzkvspace]locy.east) {
% {a} & $\txid_0$ & {a} & $\txid_3$ \\
%  {a} & $\left\{\txid_2\right\}$ & {a} & $\emptyset$\\
%};
%
%\tikzvalue{versiony-1-1}{versiony-2-1}{locy-v0}{$\val_0$};
%\tikzvalue{versiony-1-3}{versiony-2-3}{locy-v1}{$\val_1$};
%
%%Location z
%\path (locy.south) + (0,\tikzkeyspace) node (locz) {$\ke_3 \mapsto$};
%\matrix(versionz) [version list,column 2/.style={text width=7mm}]
%   at ([xshift=\tikzkvspace]locz.east) {
% {a} & $\txid_0$ & {a} & $\txid_3$ & {a} & $\txid_4$ \\
%  {a} & $\emptyset$ & {a} & $\emptyset$ & {a} & $\emptyset$\\
%};
%
%\tikzvalue{versionz-1-1}{versionz-2-1}{locz-v0}{$\val_0$};
%\tikzvalue{versionz-1-3}{versionz-2-3}{locz-v1}{$\val_1$};
%\tikzvalue{versionz-1-5}{versionz-2-5}{locz-v2}{$\val_2$};
%
%%Location w
%\path (locz.south) + (0,\tikzkeyspace) node (locw) {$\ke_4 \mapsto$};
%\matrix(versionw) [version list,column 2/.style={text width=7mm}]
%    at ([xshift=\tikzkvspace]locw.east) {
%    {a} & $\txid_0$ & {a} & $\txid_1$ \\
%    {a} & $\{\txid_4\}$ & {a} & $\emptyset$ \\
%};
%\tikzvalue{versionw-1-1}{versionw-2-1}{locw-v0}{$\val_0$};
%\tikzvalue{versionw-1-3}{versionw-2-3}{locw-v1}{$\val_1$};
%\end{centertikz}%
%\caption{Disallowed by \(\SI\)}%
%\label{fig:si-disallowed}%
%\end{subfigure} \\
%\end{tabular}
\hrulefill
\vspace*{-5pt}
\caption{Behaviours disallowed under different consistency models}
\label{fig:anomalies}
\vspace*{-10pt}
\end{figure*}


\mypar{Write Follows Reads (\WFR)}
It states that, if a transaction sees a version written by a
client $\cl$, then it must also see the versions previously read by $\cl$ (in $\SO\rflx$ relation).
This prevents the kv-store of \cref{fig:wfr-disallowed},
since transaction $\txid$ reads a version written by $\cl$ but
not a version previously read by $\cl$.
The execution test $\ET_{\WFR}$ ensures
that a view includes all versions previously read by a client 
if the view already includes a write from that client. 

\sx{A cite here \cite{surech-session-guarantee} who mentions composition of 4 session guarantees. 
    \cite{principle-eventual-consistency} mentions causal consistency is combination of session guarantees,
    yet they only define 3 types of session guarantees.
}
\sx{ Sessions to CC cite \cite{principle-eventual-consistency} }
\mypar{Causal Consistency (\CC)}
Causal consistency has been defined in the literature~\cite{session2causal} 
as the conjunction (composition) of the four \emph{session guarantees} \(\MR\), \(\MW\), \(\RYW\) and \(\WFR\). 
One would expect the execution test for causal consistency to be exactly 
the intersection of the execution tests for the individual session guarantees: 
$\ET_{\MR} \cap \ET_{\MW} \cap \ET_{\RYW} \cap \ET_{\WFR}$. Yet it is not the case.
Because $\ET_{\MW}$ and $\ET_{\WFR}$ have been 
defined in terms of closure properties over the view of a client before committing a transaction, with respect 
to two relations $\rel_{\MW} = \SO$ and $\rel_{\WFR}$ respectively, while to build an execution test that composes the two 
consistency guarantees it requires that the view of a client prior to executing a transaction is 
closed simultaneously with respect to $\rel_{\MW}$ and $\rel_{\WFR}$, that is, it is closed with respect to 
$\rel_{\MW} \cup \rel_{\WFR}$. Let 
$\ET_{\MR+\WFR} = \Set{(\mkvs, \vi, \fp, \mkvs', \vi')}[ \closed[\mkvs, \vi, \rel_{\MW} \cup \rel_{\WFR}]]$. 
\cref{fig:wr-wfr-allowed-but-cc} shows an example for $\CMs[\ET_{\MW} \cap \ET_{\RYW}] \neq \CMs(\ET_{\MW+\WFR})$,
where the last transaction \( \txid \), by a client differs from \( \cl \) and \( \cl' \), 
is allowed to committed under \( \MW \) and \( \WFR \) but not \( \MW + \WFR \), given that
\( \txid_{\cl}^1 \toEDGE{\SO} \txid_{\cl}^2 \toEDGE{\WR_\mkvs} \txid_{\cl'}^1 \toEDGE{\SO} \txid_{\cl'}^2 \).
In theory, the same argument would apply to the session guarantees $\MR$ and $\RYW$
%, which 
%can be expressed in terms of closure properties of the views obtained after transactions commit, leading 
%to an execution test of the form $\ET_{\MR+\RYW}$. 
however, in this particular case we obtain that $\ET_{\MR+\RYW} = \ET_{\MR} \cap \ET_{\RYW}$. 
The execution test for causal consistency is therefore 
given by $\ET_{\MR+\RYW} \cap \ET_{\MR} \cap \ET_{\RYW}$, which is equivalent to the one in
\cref{fig:execution.tests}: it states that the view before committing a transaction must be closed with respect to 
causal dependencies of transactions; and the view after committing guarantees \( \MR \) and \( \RYW \).
%$\CMs_{\CC} \defeq \CMs_{\MR} \cap \CMs_{\MW} \cap \CMs_{\RYW} \cap \CMs_{\WFR}$. 
%Analogously, we have defined $\ET_\CC$ as the conjunction of the execution tests corresponding to the four session guarantees.
%As we discuss before, we can do this thanks to the \emph{compositionality} of our execution tests:
%the composition of several consistency models is equivalent to the consistency model induced by the conjunction of the corresponding execution tests. 
%That is, $\CMs(\ET_\CC) = \CMs(\ET_{\MR}) \cap \CMs(\ET_{\MW}) \cap
%\CMs(\ET_{\RYW}) \cap \CMs(\ET_{\WFR}) = \CMs(\ET_{\MR} \cap
%\ET_{\MW} \cap \ET_{\RYW} \cap \ET_{\WFR})$.
%As we discuss in \cref{sec:applications} and prove in \ref{sec:cops}, the COPS
%implementation~\cite{cops} satisfies $\ET_{\CC}$. 

\mypar{Update Atomic ($\UA$)}
This model has been proposed in~\cite{framework-concur} 
and implemented in \cite{rola}.
$\UA$ disallows concurrent transactions writing to the same key,
a property known as \emph{write-write conflict freedom}, that is, 
when two transactions want to write to the same key, one must see another.
This prevents the kv-store of \cref{fig:ua-disallowed},
as $\txid, \txid'$ concurrently increment the initial version of $\key$ by $1$.
%Note that $\UA$ generalises $\RYW$: unlike $\RYW$, $\UA$ does not require $\txid, \txid'$ to be executed by the same client.
$\ET_\UA$ ensures write-conflict freedom by allowing a client to write to key $\key$
only if its view includes all versions of $\key$. Because a view always includes the 
initial version of a key, this is equivalent to require that the view of a client wanting 
to commit a transaction is \emph{suffix-closed} with respect to the relation $\WW(\key)$, 
or equivalently is prefix-closed w.r.t $\WW^{-1}(\key)$ for any key $\key$ written in the 
fingerprint $\fp$.

\mypar{Parallel Snapshot Isolation (\PSI)} 
The guarantees of $\PSI$ have been defined as the conjunction of the guarantees provided by $\CC$ and $\UA$~\cite{framework-concur}. 
As for the case of causal consistency, we cannot define $\ET_{\PSI}$ as $\ET_{\CC} \cap \ET_{\UA}$, because 
this execution test only mandates that views before committing transactions are individually closed with respect to 
the relations $R_{\CC}$ and $R_{\UA}(\fp)$. Furthermore, $\ET_{\UA}$ imposes that a transaction writing 
to a key $\key$ must be able to observe all previous versions of such a key, \ie any version of a key $\key$ 
causally depends from the transactions writing previous versions of such a key (an example is given in \cref{fig:cc-ua-allowed-but-psi}).
To capture this constraint, 
we must include write-write dependencies in the relation with respect to which the view of a client before committing 
a transaction must be closed.
The definition in \cref{fig:execution.tests} ensures that 
views of committing transactions are closed simultaneously with respect to both relations.
%Analogously, we have defined $\ET_\PSI = \ET_\CC \cap \ET_\UA$. 
%This definition exploits the \emph{compositionality} of our execution tests (\cref{thm:compositional}).
%as discussed in \cref{sec:other_formalisms}.

\mypar{Consistent Prefix ($\CP$)}
\label{para:cp}
When the total order in which transactions commit is known, then 
$\CP$ can be described as a strengthening of causal consistency: 
if a client sees a transaction $\txid$,
then it must also see any transaction that commits before $\txid$. 
Kv-stores only provide {\em partial} information about the order in which transactions commit, 
given by the dependency relations between transactions; however, 
this information is sufficient to formalise \emph{Consistent Prefix} \cite{laws}.

The way in which the information about the order in which transactions 
committed can be approximated from the relations $\WR_{\mkvs}, \WW_{\mkvs}, 
RW_{\mkvs}$ and $\SO$, and is best understood in terms of an idealised implementation of 
$\CP$ on a centralised system,
%fine-grained model 
%where transactions are equipped with start and commit points, and operations 
%between transactions in different sessions may interleave with each other. 
where the snapshot of a transaction is determined at its start point and its effects are made visible 
to future transactions at the moment it commits. 
With respect to this implementation, a write-read dependency \(\WR\) from transaction $\txid$ to transaction $\txid'$ 
means that 
%$\txid$ reads the version of some key $\key$ 
%written by $\txid'$, therefore  it {must} be
%the case that 
$\txid$ {must} commit before $\txid'$ starts, and therefore before $\txid'$ commits,
and similarly for $\SO$.
% 
%Consider the transaction relations $\WR_{\mkvs}$, $\WW_{\mkvs}$ and
%$\RW_{\mkvs}$ defined in \ref{fig:execution.tests}, adapted from well-known
%transaction relations associated with dependency graphs~\cite{adya-icde,adya}.
%In \cite{laws} an alternative definition of $\CP$ is given: if a client sees a transaction $\txid$, 
%then it must also see the subset of transactions that committed before $\txid$, that can 
%be computed from $\SO, \WR_{\mkvs}, \WW_{\mkvs}, \RW_{\mkvs}$. This property is captured 
%by $\dagger$ in \cref{fig:execution.tests}.
A read-write anti-dependency \(\RW\) from $\txid''$ to $\txid'$ 
%The pair $(\txid,\txid') \in \WR_\mkvs$ means that $\txid$ reads the version of some key $\key$ 
%written by $\txid'$, therefore  it {must} be
%the case that $\txid$ commits before $\txid'$ starts, and therefore before $\txid'$ commits,
%and similarly for $\SO$.
%The pair $(\txid, \txid') \in \RW_{\mkvs}$ 
means that $\txid''$ reads one version for some key that 
is later overwritten by $\txid'$; then $\txid''$ is prevented from seeing the write of $\txid'$, 
and therefore it {must} be the case that $\txid''$ starts before 
$\txid'$ commits. 
As a consequence, if $\txid$ commits before $\txid''$ starts (which is the case if the two are 
related by a write-read dependency \(\WR\) or by a session order \(\SO\)), and there is an anti-dependency 
from $\txid''$ to $\txid'$, then in {must} also be the case that $\txid$ commits before 
the commit of $\txid'$.
Finally, a write-write dependency from $\txid$ to $\txid'$ means that $\txid$ {must} commit before $\txid'$. 
Therefore the relation $\rel_{\CP} \defeq (\WR_{\mkvs}; \RW_{\mkvs}? \cup \SO;  \RW_{\mkvs} \cup \WW)$ 
approximates the order in which transactions have been executed. It has been shown in \cite{laws} 
that the set $(\rel_{\CP}^{-1})^{+}_{\CP}(\txid)$ contains all the transactions that {must} be observed by $\txid$ under 
$\CP$. We define $\ET_{\CP}$ by requiring that the view of a client prior to committing a transaction is 
closed with respect to $R_{\CP}$, and by enforcing the session guarantees $\MR$ and $\RYW$,
%$(\txid,\txid'') \in \WR_{\mkvs}$ (resp. \SO) and $(\txid'',\txid') \in \RW_\mkvs$, then it {must} also be the case that $\txid$ commits before the commit  of $\txid'$, 
%Last, if $(\txid,\txid') \in \WW_\mkvs$ means that $\txid'$ overwrites a version written by $\txid$ for some 
%key, then it {must} be
%the case that $\txid$ commits before the commit of $\txid'$.
%The relation $((\SO ; \RW_{\mkvs}\rflx) \cup (\WR_{\mkvs} ; \RW_{\mkvs\rflx}) \cup \WW_\mkvs )^{+} \ni (\txid, \txid')$
%captures that, {\em all} the transactions $\txid$ that {must} have already committed to the kv-store before commit of \( \txid' \).
%The execution test $\ET_{\CP}$ is the intersection of $\dagger$ with $\ET_\MR \cap \ET_\RYW$,
where the latter enforces a client sees its own commits.
Consistent prefix disallowed the \emph{long fork anomaly} shown in \cref{fig:cp-disallowed}:
if \( \txid_{\cl_2}^2\) is the last transaction, it reads and thus sees \( \txid' \).
Given the kv-store we have:
\(
\txid \toEDGE{\WR_\mkvs} \txid^1_{\cl_1} \toEDGE{\SO} \txid^2_{\cl_1} \toEDGE{\RW_\mkvs} \txid'
\),
which means, by closure of \( \rel_{\CP} \), transactions \( \txid_{\cl_2}^2 \) must see \( \txid \).
However, in \cref{fig:cp-disallowed} \( \txid_{\cl_2}^2 \) reads a older version of \( \key_1 \) than the one written by \( \txid \).
Symmetrically,
if \( \txid_{\cl_1}^2\) is the last transaction, it sees \( \txid \) and so must see \( \txid' \).
\sx{A reviewer suggests to show the visibility edges for long fork}


\mypar{Snapshot Isolation (\SI)}
When the total order in which transactions commit is known then 
$\SI$ can be defined compositionally from $\CP$ and $\UA$. 
As for the case of $\CC$ and $\PSI$, we cannot define the 
execution test for $\SI$ by simply composing the execution
tests of its individual components, this is, $\ET_{\CP} \cap \ET_{\UA}$.
Rather we need to ensure that the view of a client prior to committing a transaction 
with fingerprint $\fp$
is closed with respect to the relation $\rel_\CP \cup \rel_{\UA}(\fp) \cup \WW;\RW\rflx$; 
the addend $\WW;\RW\rflx$ comes from the fact that, when the centralised implementation 
of $\CP$ (discussed above) is strengthened with write-conflict freedom (\ie concurrent transactions 
write to different key), then a write-write dependency between two transactions $\txid$ and $\txid'$ 
does not only mandate that $\txid$ commits before $\txid'$ commits but also before $\txid'$ starts. 
As a consequence, if $\txid \toEDGE{\WW} \stub \toEDGE{\RW} \txid'$, then $\txid$ must commit 
before $\txid'$ commit.
%When we can rely only on 
%a partial order of transaction execution, inferred from 
%kv-stores and views, then this compositional result does not 
%hold\footnote{This issue arises also if dependency graphs are used in 
%place of kv-stores. See \cref{sec:si-not-intersect-cp-ua}}. For example, the kv-store of \cref{fig:si-disallowed} is
%included in both $\CMs(\ET_{\CP})$ and $\CMs(\ET_{\UA})$, but is
%disallowed by the execution test $\ET_\SI$, introduced presently
%In our definition, \( \ET_\SI \) by replacing property dagger in $\ET_{\CP}$ with 
%\( \ddagger \) (\cref{fig:execution.tests}) and by intersecting the result with $\ET_{\UA}$.
%Similarly as for $\CP$, the $\ddagger$ property captures the 
%fact that if transaction $\txid$ sees the writes of another transaction $\txid'$, then 
%it must see the subset of transactions committing before $\txid'$ that can be computed 
%from $\SO, \WR_{\mkvs}, \WW_{\mkvs}, \RW_{\mkvs}$. However, because snapshot isolation enforces 
%write-conflict freedom, the computation of this subset 
%differs from the one for $\CP$. 
%Under $\UA$ consequently \(\SI\), the pair $(\txid, \txid'') \in \WW_\mkvs$ means not only the case that $\txid$ commits 
%before $\txid''$, but $\txid$ commits before $\txid''$ start. 
%Because $(\txid'',\txid') \in \RW_{\mkvs}$ 
%implies that $\txid''$ starts before $\txid'$ commits, then it must be the case that 
%when $(\txid,\txid') \in \WW_{\mkvs} ; \RW_{\mkvs}$ then $\txid$ commits before $\txid'$ does. 
%In \cref{fig:si-disallowed} we show an anomaly that is allowed by $CP$ and $UA$ 
%(and therefore by the intersection of these two consistency models), but is disallowed by $\SI$. In this kv-store \( \txid_4 \) reads 
%the last version of \( \key_2 \) written by \( \txid_3 \), so that this kv-store is allowed by  \( \UA \). 
%We also have that 
%From the \( \txid_3 \) backwards we have edges:
%\(
%\txid_1 \toEDGE{\WW_\mkvs} \txid_2 \toEDGE{\RW_\mkvs} \txid_3
%\).
%Snapshot isolation \( \SI \) requires that if a transaction sees \( \txid_3 \) it must see \( \txid_1 \) by the \( \ddagger \) (it is not the case in \( \dagger \)).
%However in \cref{fig:si-disallowed} transaction \( \txid_4 \) only see the initial version of \( ke_1 \).
%As we discuss in \cref{sec:applications} and prove in the \ref{sec:clock-si}, 
%the Clock-SI protocol~\cite{clocksi} satisfies $\ET_{\SI}$. 

\mypar{(Strict) serialisability (\SER)}
Serialisability is the strongest consistency model, requiring that there exists a sequential schedule of transactions. 
The execution test $\ET_{\SER}$ thus allows clients to execute transactions only when 
their view of the kv-store store is complete; this can be enforced by requiring that the view 
of a client is closed with respect to the relation $\WW^{-1}$.
This requirement prevents the kv-store in  \cref{fig:ser-disallowed}: under serialisability either $\txid_1$ or $\txid_2$ commits first.
In the former case when $\txid_1$ commits first, then its write to $\key_2$ must be included in the view of $\txid_2$, and thus $\txid_2$ should not read the outdated write to $\key_2$ by $\txid_0$. 
The latter case is analogously prohibited. 
This example is allowed by all the other execution tests in~\cref{fig:execution_tests}.

\subsection{Execution Tests for Discovering Consistency Models}
\label{sec:new_cm}
Kv-stores and execution tests constitute a useful tool for investigating new 
consistency models that are potentially interesting. 
One example of such execution tests is given by 
$\ET_{\CP} \cap \ET_{\UA}$. By definition, this execution 
test is stronger than both $\CP$ and $\UA$: therefore 
it forbids any anomaly forbidden by these consistency models, 
such as the long fork anomaly (\cref{fig:cp-disallowed}) and the lost update (\cref{fig:ua-disallowed}). 
It is also weaker than snapshot isolation, meaning that it allows any of the anomalies 
of SI, such as the write skew (\cref{fig:ser-disallowed}). On the other hand, the inclusion 
$\ET_{\CP} \cap \ET_{\UA} \subseteq \ET_{\SI}$ is strict. The kv-store from  
\cref{fig:si-disallowed} is allowed by $\ET_{\CP} \cap \ET_{\UA}$: 
this kv-store can be obtained by executing  
transaction $\txid_{1}$ using the initial view $[\key_1 \mapsto \{0\}, \key_2 \mapsto \{0\}]$, 
transaction $\txid_{2}$ using the view $[\key_1 \mapsto \{0,1\}, \key_2 \mapsto \{0\}]$ 
transaction $\txid_{3}$ using the view $[\key_1 \mapsto \{0,1\}, \key_2 \mapsto \{0,1\}]$, 
and transaction $\txid_{4}$ using the view $[\key_1 \mapsto \{0\}, \key_2 \mapsto \{0,1\}]$. 
However, transaction $\txid_{4}$ cannot be executed using this same view when $\ET_{\SI}$ is 
assumed. Because $\txid_{4}$ reads the version of $\key_2$ written by $\txid_3$, 
any valid view $\vi$ used to commit $\txid_{4}$ under $\ET_{\SI}$ must include the version written by 
$\txid_{3}$. Since $\txid_{2} \xrightarrow{\RW_{\mkvs}} \txid_{3}$ (because $\txid_{2}$ reads a staler 
version $\key_2$ than the one written by $\txid_{3}$) and $\txid_{1} \xrightarrow{\WW} \txid_{2}$ 
(because $\txid_{2}$ installs a newer version of $\key_{1}$ than the one written by $\txid_{1})$, 
then the view $\vi$ should contain the version of $\key_{1}$ written by $\txid_{1}$, contradicting 
the fact that $\txid_{4}$ reads the initial version of $\key_1$.

We refer to the consistency model induced by $\ET_{\CP} \cap \ET_{\SI}$ as \emph{Weak Snapshot Isolation} ($\WSI$). 
Because it is weaker than Snapshot Isolation, potential implementations of $\WSI$ could perform better than SI ones. 
Also, because of the similarity between these two consistency models, we believe that many applications that 
are correct under SI would still be correctly under $\WSI$. We give an example of such an application in \cref{sec:program-analysis}.


