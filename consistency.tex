\section{Consistency Guarantees}
\label{sec:cm}
Consistency guarantees of distributed databases describe
what it means for distributed data to be consistent. 
They have been formally described axiomatically via dependency graphs~\cite{adya-icde,adya}
and abstract execution graphs~\cite{ev_transactions,framework-concur}. 
We formalise the consistency guarantees of our centralised kv-stores by defining a 
\emph{consistency model}. 
A consistency model is a set of kv-stores capturing the possible outcomes 
obtained when multiple clients commit several transactions each, 
provided that the effects of such transactions comply with the consistency guarantees of the underlying consistency model. 
%That is, the set of such transactions are restricted to those whose effects comply with the consistency guarantees of the underlying consistency model. 
To this end, we define consistency models induced by an \emph{execution test}.
An execution test is a relation which determines whether a client may commit a transaction into a kv-store.  
We formulate several well-known consistency models over our centralised kv-stores 
by defining their corresponding execution tests. 
Later in \cref{sec:other_formalisms} we demonstrate that our definitions over centralised kv-stores are equivalent 
to their existing definitions over distributed databases.




An execution test is a set $\ET$ of tuples of the form $(\mkvs, \vi, \opset, \vi')$,
denoting that a client with view $\vi$ on kv-store $\hh$  may commit an atomic transaction 
with fingerprint $\opset$  and obtain an updated view $\vi'$. 
We often write
$\ET \vdash (\hh, \vi) \triangleright \opset: ( \mkvs', \vi')$ for
$(\mkvs, \vi, \opset, \mkvs', \vi') \in \ET$.


\begin{definition}
\label{def:execution.test}
An \emph{execution test} is a set of tuples $\ET \subseteq \HisHeaps \times \Views \times \powerset{\Ops} \times \HisHeaps \times \Views$ 
such that for all $(\hh, \vi, \opset, \mkvs', \vi') \in \ET$ and all $\otR, \ke, \val$:
\begin{align}
    & 
	(\otR, \ke, \val) \in \opset \Rightarrow
	\hh(\ke, \max{}_{<}(\vi(\ke))) = \val  
	\tag{Ext} \label{eq:read-external} \\
    & 
    \vi(\ke) \neq \vi'(\ke) 
    \Rightarrow
    \exsts{l} (l, \ke, -) \in \opset
    \tag{ValidViewUpd} \label{eq:valid-view-update}
\end{align}
\end{definition}
%
\noindent 
The first condition enforces the last-write-wins policy~\cite{vogels:2009:ec:1435417.1435432}: 
a transaction always reads the most recent writes from the initial view.  
The second condition states that a transaction is only allowed to update the view for those keys 
that have been recorded in the fingerprint.  

Note that at this initial stage \eqref{eq:read-external} and \eqref{eq:valid-view-update} are the only required conditions and execution tests are otherwise unrestricted. 
Further restrictions on execution tests are determined by the underlying consistency model, thus prescribing the consistency guarantees of the model.

Given an execution test  $\ET$, we define the $\ET$-trace as a sequence of $\ET$-reductions on configurations that either 
\begin{enumerate*}
	\item advances the client view to a more up-to-date view; or 
	\item commits a fingerprint of a transaction. 
\end{enumerate*}

\begin{definition}[$\ET$-trace]
\label{def:reduction}
An \emph{action} $\alpha \in \Act$ is either of the form $(\cl, \varepsilon)$ or $(\cl, \opset)$, 
where $\cl$ is a client and $\opset$ is a fingerprint. 
Given an execution test $\ET$, the $\ET$-\emph{reduction relation},
$\xrightarrowtriangle{}_{\ET} \subseteq \Confs \times \Act \times \Confs$, 
is the smallest relation such that for all $\cl, \hh, \hh', \viewFun, \opset, \vi'$ and $\vi = \viewFun(\cl)$:
\begin{enumerate}
	\item
    $\vi \sqsubseteq \vi'
    \Rightarrow
    (\hh, \viewFun) \xrightarrowtriangle{\hspace{-5pt}(\cl, \varepsilon)\hspace{-5pt}}_{\ET} 
    (\hh, \viewFun\rmto{\cl}{\vi'})$; 
	\item 
    $\hh' \in \updateKV(\hh, \vi , \opset, \cl)
     \wedge (\mkvs, \vi, \opset, \mkvs', \vi') \in \ET
	$  \\
	\phantom{a} \hfill $\implies (\hh, \viewFun) \xrightarrowtriangle{\hspace{-5pt}(\cl, \opset)\hspace{-5pt}}_{\ET} (\hh', \viewFun\rmto{\cl}{\vi'})$
\end{enumerate}
Given an execution test $\ET$, an \emph{$\ET$-trace} is a sequence of $\ET$-reductions of the form $\conf_{0} \xrightarrowtriangle{\alpha_{0}}_{\ET} \cdots 
\xrightarrow{\alpha_{n-1}} \conf_{n}$. $\ET$-traces are ranged over by $\tau, \tau', \cdots$.
\end{definition}

A \emph{consistency model} induced by $\ET$ is a set of kv-stores
resulting from $\ET$-traces starting in an 
initial configuration. 

\begin{definition}[Consistency Model]
\label{def:cm}
Given an execution test $\ET$ and an initial configuration $\conf_0$ (\cref{def:configuration}),
the set of \emph{configurations induced by $\ET$},  $\Confs(\ET)$, is   given by: 
\[
\Confs(\ET)\defeq 
\Setcon{ \conf}{ 
	\exsts{\conf_0 \in \Confs_0}
	\conf_0 \xrightarrowtriangle{}_{\ET}^{*} \conf
}
\]
The \emph{consistency model} induced by $\ET$ is defined as:
\( 
\CMs(\ET) \defeq \Setcon{ \hh }{ (\hh, \stub) \in \Confs(\ET) }
\)
\end{definition}


\noindent In~\cref{sec:mono-et}, we prove that consistency models are 
\emph{monotonic}: 
if  $\ET_1 \subseteq \ET_2$ then $\CMs(\ET_1) \subseteq \CMs(\ET_2)$.

\mypar{Compositionality}
We examine the \emph{compositionality} of the consistency models induced by execution tests:  
\ie given two execution tests $\ET_1, \ET_2$, does 
$\CMs(\ET_1 \cap \ET_2) = \CMs(\ET_1) \cap \CMs(\ET_2)$ hold? 
The monotonicity of execution tests guarantees that 
%It is straightforward to show that the left-to-right direction holds:
 for all $\ET_1, \ET_2$, \( \CMs(\ET_1 \cap \ET_2) \subseteq \CMs(\ET_1) \cap \CMs(\ET_2) \) 
However, the other direction \( \CMs(\ET_1) \cap \CMs(\ET_2) \subseteq \CMs(\ET_1 \cap \ET_2) \) does not hold for arbitrary consistency models.
Consider the following:
\[
\small
\begin{array}{@{}l @{\hspace{2pt}} | @{\hspace{2pt}} l@{}}
    \hline
    \ET_1 & \ET_2 \\
%    
    \hline
    (\hh_{0}, \vi_{0}) \csat\! \{(\otW, \ke, 1)\} {:} ( \hh_{\ke}, \vi_{0})
    &
    (\hh_{0}, \vi_{0}) \csat\! \{(\otW, \ke', 1)\} {:} ( \hh_{\ke'}, \vi_{0}) 
    \vspace*{-7pt}\\\\
% %   
    (\hh_{\ke}, \vi_{0}) \csat  \!\{(\otW, \ke', 1)\} {:} (\hh',\vi_{0}) 
    &
    (\hh_{\ke'}, \vi_{0}) \csat \!\{(\otW, \ke, 1)\} {:} (\hh',\vi_{1}) 
    \\
\hline
\end{array}
\]
with 
%\begin{align*}
\[
    \begin{array}{l@{} l}
    \hh_{\ke} & = \hh_{0}[\ke \mapsto (0, \txid_{0}, \emptyset) \lcat (1, \_, \emptyset)] \\
    \hh_{\ke'} & = \hh_{0}[\ke' \mapsto (0, \txid_{0}, \emptyset) \lcat (1, \_, \emptyset)] \\
    \hh' & = \hh[\ke \mapsto (0, \txid_{0}, \emptyset) \lcat (1, \_, \emptyset) 
                ,\ke' \mapsto (0, \txid_{0}, \emptyset) \lcat (1, \_, \emptyset)] \\
\end{array}
\]
%\end{align*}
As both $\ET_1$ and $\ET_2$ allow a version with value $1$ to be written for 
$\ke, \ke'$,  we have $\hh' \in \CMs(\ET_1) \cap \CMs(\ET_2)$. 
However, $\ET_1$ and $\ET_2$ enforce a different order in which the updates on $\ke, \ke'$ must happen; 
thus $\hh' \notin \CMs(\ET_1 \cap \ET_2)$. 

In this example, compositionality fails because execution tests 
enforced a particular order in which the updates must be committed, even though such updates 
are non-conflicting: \ie the kv-store obtained after committing such updates is independent of the commit order. This observation is captures in the following definition: 
\begin{definition}
Two fingerprints $\opset_1$ and $\opset_2$ are \emph{conflicting} 
iff there exists $\ke$ such that 
$(\otW, \ke, -) \in \opset_1 \land (\otW, \ke, -) \in \opset_2$. 

An execution test $\ET$ is \emph{commutative}, written $\com{\ET}$, if 
for all views $\vi_1, \vi_2 \in \Views(\hh_0)$, 
clients $\cl_1, \cl_2$,
fingerprints $\opset_1, \opset_2$, 
kv-stores $\hh'$,
and view functions $\viewFun, \viewFun'$:
\[
\begin{array}{@{}r @{\hspace{10pt}} l @{}}
	\text{if} &  
	(\hh_0, \viewFun) \xrightarrowtriangle{(\cl_1, \opset_1)}_{\ET} 
	\_ \xrightarrowtriangle{(\cl_2, \opset_2)}_{\ET} (\hh', \viewFun') \\
	& \land\ \cl_1 \ne \cl_2 \land \opset_1, \opset_2  \text{ are non-conflicting}\\
%
%	
	\text{then} & (\hh_0, \viewFun) \xrightarrowtriangle{(\cl_2, \opset_2)}_{\ET} 
\_ \xrightarrowtriangle{(\cl_1, \opset_1)}_{\ET} (\hh', \viewFun')
\end{array}
\]
\end{definition}

To guarantee the compositionality of two execution tests $\ET_1, \ET_2$, we 
require at least one of those to be commutative, say $\ET_1$. The main idea 
is the following: fix a $\ET_1$-trace $\tau_1$ and a $\ET_2$-trace $\tau_2$, both terminating in a configuration 
of the form $(\hh, \stub)$; then we construct a $(\ET_1 \cap \ET_2)$-trace terminating 
in a configuration of the same form by re-ordering he sequence of 
reductions of $\tau_1$ as to match exactly the sequence of 
reductions of $\tau_2$. 
In \cref{sec:et-comp}, we show that if $\ET_1$ is commutative, 
we can indeed re-order the sequence of reductions in the 
$\ET_1$-trace leading to a trace $\tau_1'$ such that $\lvert \tau'_1 \rvert = \lvert \tau_2 \rvert$, 
where $\vert \tau'_1 \rvert$ is the 
number of $\ET_1$-reductions of$\tau_1'$ (and similarly for $\tau_2$)\footnote{In fact, 
to ensure that $\lvert \tau_1' \rvert = \lvert \tau_2 \rvert$ we require to further manipulate 
$\tau_1$ prior to re-ordering its sequence of $\ET_1$-reductions.}; and for any 
$i=1,\cdots, \lvert \tau_1'\rvert$, the pre and post kv-store of its
$it$-th reduction in the re-ordered $\ET_1$, coincide with the action, pre and post kv-store 
of the $\ET_2$-trace.
%$\ET_1$ is a commutative execution test, and consider two traces, 
%produced under the execution tests $\ET_1$ and $\ET_2$, respectively, 
%that terminate in a configuration of the form $(\hh, \stub)$. Let $n$ be 
%the number of reductions appearing in the $\ET_2$-trace.
% Intuitively, the kv-store $\hh$ determines the set of fingerprints that appear in 
%such two traces, which therefore must be the same; furthermore, 
%the commutativity of $\ET_1$ guarantees that we can re-order the 
%sequence of reductions of the considered $\ET$-trace so that, for any 
%$i=1,n$, the action of its $i$-th reduction matches the action of the 
%$i$-th reduction in the $\ET_2$-trace. Similarly, we ensure that 
%the pre and post kv-store of the $i$-th reduction of the re-ordered $\ET_1$ 
%trace coincide with the pre and post kv-store of the $i$-th reduction of the 
%$\ET_2$ trace.

Commutativity alone does not ensure that, for $i=1,\cdots,n$ the 
pre-views and post-views of the $i$-th $\ET_1$-reduction of $\tau_1'$ 
coincide with the pre-views and post-views of the $i$-th 
$\ET_2$-reduction in the $\tau_2$-trace, which is a necessary to show 
that $\tau_1'$ and $\tau_2$ can be recast as a $(\ET_1 \cap \ET_2)$-trace. 
In \cref{sec:et-comp} we present three other basic requirements to be 
satisfied by $\ET_1$ and $\ET_2$, that guarantee that $\tau'_1$ and 
$\tau_2$ agree on the pre-views and post-views of each of their reductions. 
The first two of these requirements, which we call 
\emph{no blind writes} and \emph{minimum footprint}, ensure that 
the pre-views of the reductions in $\tau_1'$ and $\tau_2$ match, 
while the third requirement, which we call \emph{monotonic post-views}, 
guarantees that the post-views of the reductions in $\tau_1'$ and $\tau_2$ 
match. 
%
%is not sufficient to ensure the compositionality 
%of $\ET_1$ and $\ET_2$; this is because, although we ensure that 
%the re-ordered $\ET_1$-trace and the $\ET_2$ trace agree on the sequence of 
%actions they perform, as well as on the sequence of key-value stores in which 
%such actions are performed, they may still disagree on the 
%is commutative, then the actions of any $\ET_1$-trace terminating in 
%a configuration of the form $(\hh, \_)$ can be permuted so that 
%the fingerprint 
%
%Commutativity is a necessary, but not sufficient condition for recovering
%the desired compositionality result.
%To recover compositionality, we further require that an execution test have \emph{no blind writes}, 
%\emph{minimal footprint} and \emph{monotonic post-view} (\cref{def:et_properties}).
%We refer the reader to \cref{sec:counter-examples-composition} for counter examples demonstrating why these conditions are necessary.
%
%
%\begin{definition}[$\ET$ properties]
%\label{def:et_properties}
%An execution test $\ET$ has \emph{no blind writes} if
%for all $\hh, \hh', \vi, \vi', \opset, \ke$:
%\[
%\begin{array}{@{} r @{\hspace{10pt}} l @{}}
%\text{if} & \ET \vdash (\hh, \vi) \triangleright \opset: (\hh',\vi' ) 
%\text{and} (\otW, \ke, \_) \in \opset \\
%\text{then} & (\otR, \ke, \_) \in \opset
%\end{array} 
%\]
%An execution test $\ET$ has a \emph{minimum footprint} if for all kv-stores $\hh, \hh'$,
%views $\vi, \vi',\vi''$, and fingerprints $\f$: 
%%
%\[
%\begin{array}{@{} r @{\hspace{10pt}} l @{}}
%    \text{if} & \ET \vdash (\hh, \vi) \triangleright \opset : (\hh', \vi'')  \\
%    \text{and} & \fora{(\stub, \ke, \stub) \in \f} \vi(\ke) {=} \vi'(\ke) \\
%    \text{then} & \ET \vdash (\hh, \vi') \triangleright \opset : (\hh', \vi'')
%\end{array} 
%\]
%%
%An execution test $\ET$ has \emph{monotonic post-views} if 
%for all kv-stores $\hh,\hh'$, 
%views $\vi, \vi',\vi''$ and fingerprints $\f$:
%\[
%\begin{array}{@{} r @{\hspace{10pt}} l @{}}
%    \text{if} & \ET \vdash (\hh, \vi) \triangleright \opset : (\hh',\vi' )
%    \text{ and } \vi' \sqsubseteq \vi''  \\
%    \text{then} & \ET \vdash (\hh, \vi) \triangleright \opset : (\hh',\vi'')
%\end{array} 
%\]
%\end{definition}
%The first two requirements ensure that, in an $\ET$-trace 
%terminates in a configuration $(\hh, \stub)$, 
%the set of client views used to commit a transaction can uniquely determined by $\hh$. 
%If $\ET_1, \ET_2$ have monotonic post-views, then there exists at least one common post-view 
%that can be obtained after a client commits $\f$ to $(\hh, \vi)$ using $\ET_1$ and $\ET_2$. 

\begin{theorem}[Compositionality]     
For all $\ET_1, \ET_2$ with no blind writes, minimum footprints and monotonic post-views: 
if $\com{\ET_1}$, 
then $\CMs(\ET_1 \cap \ET_2) {=} \CMs(\ET_1) \cap \CMs(\ET_2)$;
if $\com{\ET_1} \land \com{\ET_2}$, then $\com{\ET_1 \cap \ET_2}$.
\end{theorem}

Most of the execution tests that we introduce shortly,  \cref{fig:execution_tests} can be minimally adapted to satisfy these 
properties, without excluding any anomalous (weak) behaviour. However, the definitions 
of $\ET_{\SI}$ and $\ET_{\CP}$ are inherently non-commutative (counter example in \cref{sec:comm-counter-cp-si}), which also motivates 
why $\ET_{\UA} \cap \ET_{\CP}$ is not an appropriate execution test for Snapshot Isolation.
\ac{This paragraph does not make sense anymore, as we did not introduce the examples of execution tests yet. Move somewhere else.}


\subsection{Examples}\label{subsec:cm_examples}

\begin{figure*}
\begin{center}
    \scalebox{.9}{%
    \begin{tabular}{ @{} l  r ||  l  r @{} }
\hline
Model & Execution Test: \((\hh, \vi) \csat \opset : (\mkvs',\vi')\) &
Model & Execution Test: \((\hh, \vi) \csat \opset : (\mkvs',\vi')\)
\\
\hline
\MRd & $\vi \viewleq \vi'$ &
\UA & $(\otW, \ke,  \stub) \in \opset \land 0 \leq i < \lvert \hh(\ke)
      \rvert \implies i \in \vi(\ke) $
\\
\MW & 
$j \in \vi(\ke) \wedge \WTx(\hh(\ke', i)) \xrightarrow{\PO^?} \WTx(\hh(\ke, j)) 
\implies i \in \vi(\ke')$ &
\PSI & $\ET_{\PSI} = \ET_{\CC} \cap \ET_{\UA}$
\\
\RYW & $ \txid \in \mkvs' \wedge \txid \notin \mkvs \land \WTx(
\mkvs'(\ke, i) ) \xrightarrow{\PO^?} \txid \implies i \in \vi'(\ke) $ &
\CP & \( \Setcon{(\mkvs, \vi, \f, \vi')}{\dagger} \cap \ET_\MRd \cap \ET_\RYW \) 
\\
\WFR & $j \in \vi(\ke) \wedge \txid \in \RTx(\hh(\ke', i)) \wedge \txid {\xrightarrow{\PO^?}}
\WTx(\mkvs(\ke, j))  \implies i \in \vi(\ke')$ &
$\SI$ & $\Setcon{(\mkvs, \vi, \f, \vi')}{\ddagger} \cap \ET_\MRd \cap \ET_\RYW  \cap \ET_\UA $
\\
\CC & $\ET_{\CC} = \ET_{\MRd} \cap \ET_{\MW} \cap \ET_{\RYW} \cap \ET_{\WFR}$ &
\SER & $ 0 \leq i < \lvert \hh(\ke) \rvert \implies i \in \vi(\ke) $\\
\hline
\end{tabular}%
}
\end{center}
%
with
\vspace*{-5pt}
\[  
    \begin{rclarray}
        \dagger 
        & \eqdef &  
        \fora{\ke, \ke', i, j}
         i \in \vi(\ke)  \wedge 
         \WTx(\hh(\ke', j)) \toEdge{(((\PO \cup \RF_{\hh}) ; \AD_{\hh}^?) \cup \VO_{\hh})^{+}} \WTx(\hh(\ke, i))
        \implies j \in \vi(\ke')  \\
        \ddagger 
        & \eqdef &
        \fora{\ke, \ke', i, j}
        i \in \vi(\ke)
        \wedge \WTx(\hh(\ke', j)) \toEdge{((\PO \cup \RF_{\hh} \cup \VO_{\hh}) ; \AD_{\hh}^?)^{+}} \WTx(\hh(\ke, i))
        \implies j \in \vi(\ke')    
    \end{rclarray}
\]
\vspace*{-5pt}
%
%   \RF_{\hh} \VO_{\hh}  \AD_{\hh}
\[
    \begin{rclarray}
       \RF_{\hh} &\defeq& \{ (\txid, \txid') \mid \exists \ke, i.\; \txid = \WTx(\hh(\ke, i)) \wedge \txid' \in \RTx(\hh(\ke, i))\}\\
     \VO_{\hh} &\defeq& \{ (\txid, \txid') \mid \exists \ke, i, j.\; \txid = \WTx(\hh(\ke, i)) \wedge \txid' = \WTx(\hh(\ke, j)) \wedge i < j\}\\
        \AD_{\hh}&\defeq& \{ (\txid, \txid') \mid \exists \ke, i,
        j.\; \txid \in \RTx(\hh(\ke, i)) \wedge \txid' = \WTx(\hh(\ke,
        j)) \wedge i < j \land \txid \neq \txid'\}
\end{rclarray}
\]
\hrule\vspace{5pt}
\captionsetup{width=\linewidth}
\caption{Execution tests of client-centric (left) and data-centric (right) consistency models, 
with $\PO$ as defined in \cref{subsec:kvstores}. 
All free variables are universally quantified.
}
\label{fig:execution.tests}
\label{fig:execution_tests}
\end{figure*}

\begin{figure*}[t]
\captionsetup[subfigure]{aboveskip=-5pt, belowskip=5pt}
\begin{tabular}{@{} c | c | c @{}}
\hline
\phantom{-}& \phantom{-}& \phantom{-}\\
\begin{subfigure}{0.2\textwidth}
\begin{centertikz}

%Location x
\node(locx) {$\ke_1 \mapsto$};
\draw pic at ([xshift=\tikzkvspace]locx.east) {vlist={versionx}{%
    /$\val_0$/$\txid_0$/$\Set{\txid_\cl^2}$
    , /$\val_1$/$\txid_1$/$\Set{\txid_\cl^1}$
}};

\end{centertikz}\vspace{5pt}%
\caption{Disallowed by \(\MRd\)}
\label{fig:mr-disallowed}
\end{subfigure}
%\quad
&
\begin{subfigure}{0.36\textwidth}
\begin{centertikz}

%Location x

\node(locx) {$\ke_1 \mapsto$};
\draw pic at ([xshift=\tikzkvspace]locx.east) {vlist={versionx}{%
    /$\val_0$/$\txid_0$/$\Set{\txid'}$
    , /$\val_1$/$\txid_\cl^1$/$\emptyset$
}};

%Location y
\path (versionx.east) + (1,0) node (locy) {$\ke_2 \mapsto$};
\draw pic at ([xshift=\tikzkvspace]locy.east) {vlist={versiony}{%
    /$\val_0$/$\txid_0$/$\emptyset$
    , /$\val_2$/$\txid_\cl^2$/$\Set{\txid'}$
}};

\end{centertikz}\vspace{5pt}
\caption{Disallowed by \(\MW\)}
\label{fig:mw-disallowed}
\end{subfigure}
%\quad
&
\begin{subfigure}{0.39\textwidth}
\begin{centertikz}

%Location x
\node(locx) {$\ke_1 \mapsto$};
\draw pic at ([xshift=\tikzkvspace]locx.east) {vlist={versionx}{%
    /$\val_0$/$\txid_0$/$\Set{\txid}$
    , /$\val_1$/$\txid'$/$\Set{\txid_\cl^1}$
}};

%Location y
\path (versionx.east) + (1,0) node (locy) {$\ke_2 \mapsto$};
\draw pic at ([xshift=\tikzkvspace]locy.east) {vlist={versiony}{%
    /$\val_0$/$\txid_0$/$\emptyset$
    , /$\val_2$/$\txid_\cl^2$/$\Set{\txid}$
}};

\end{centertikz}

\vspace{5pt}
\caption{Disallowed by \(\WFR\)}
\label{fig:wfr-disallowed}
\end{subfigure}\\
\hline
\end{tabular}
%
%
%
%
\begin{tabular}{@{} c | c | c @{}}
\phantom{-}& \phantom{-}& \phantom{-}\\
\begin{subfigure}{0.25\textwidth}
\begin{centertikz}%

%Location x
\node(locx) {$\ke_1 \mapsto$};
\draw pic at ([xshift=\tikzkvspace]locx.east) {vlist={versionx}{%
    /$0$/$\txid_0$/$\Set{\txid_\cl^1,\txid_\cl^2}$
    , /$1$/$\txid_\cl^1$/$\emptyset$
    , /$1$/$\txid_\cl^2$/$\emptyset$
}};

\end{centertikz}%
\vspace{5pt}
\caption{Disallowed by \(\RYW\)}
\label{fig:ryw-disallowed}
\end{subfigure}
& 

\begin{subfigure}{0.40\textwidth}
\begin{centertikz}%

%Location x
\node(locx) {$\ke_1 \mapsto$};
\draw pic at ([xshift=\tikzkvspace]locx.east) {vlist={versionx}{%
    /$\val_0$/$\txid_0$/$\Set{\txid_2}$
    , /$\val_1$/$\txid_1$/$\emptyset$
}};

%Location y
\path (versionx.east) + (1,0) node (locy) {$\ke_2 \mapsto$};
\draw pic at ([xshift=\tikzkvspace]locy.east) {vlist={versiony}{%
    /$\val_0$/$\txid_0$/$\Set{\txid_1}$
    , /$\val_2$/$\txid_2$/$\emptyset$
}};

\end{centertikz}%
\vspace{5pt}
\caption{Write skew, disallowed by \(\SER\)}
\label{fig:ser-disallowed}
\end{subfigure}%
&
\begin{subfigure}{0.30\textwidth}
\begin{centertikz}

\node(locx) {$\ke_1 \mapsto$};
\draw pic at ([xshift=\tikzkvspace]locx.east) {vlist={versionx}{%
    /$0$/$\txid_0$/$\Set{\txid,\txid'}$
    , /$1$/$\txid$/$\emptyset$
    , /$1$/$\txid'$/$\emptyset$
}};

\end{centertikz}
\vspace{5pt}
\caption{Lost update, disallowed by \(\UA\)}
\end{subfigure}
\\
\hline
\end{tabular}
%
%
%
%
\phantom{x}\vspace{7pt}
\begin{tabular}{@{} c | c @{}}
\phantom{-}& \phantom{-} \\
\begin{subfigure}{0.42\textwidth}
\begin{centertikz}%
%Location x
\node(locx) {$\ke_1 \mapsto$};
\draw pic at ([xshift=\tikzkvspace]locx.east) {vlist={versionx}{%
    /$\val_0$/$\txid_0$/$\Set{\txid_{\cl_2}^1}$
    , /$\val_1$/$\txid$/$\Set{\txid_{\cl_1}^1}$
}};

%Location y
\path (versionx.east) + (1,0) node (locy) {$\ke_2 \mapsto$};
\draw pic at ([xshift=\tikzkvspace]locy.east) {vlist={versiony}{%
    /$\val_0$/$\txid_0$/$\Set{\txid_{\cl_1}^2}$
    , /$\val_1$/$\txid$/$\Set{\txid_{\cl_2}^2}$
}};

\end{centertikz}%
\vspace{5pt}
\caption{Long fork, disallowed by \(\CP\)}
\label{fig:cp-disallowed-2}
\label{fig:cp-disallowed}
\end{subfigure}
&
\begin{subfigure}{0.542\textwidth}%
\begin{centertikz}%
%Location x
\node(locx) {$\ke_1 \mapsto$};
\draw pic at ([xshift=\tikzkvspace]locx.east) {vlist={versionx}{%
    /$\val_0$/$\txid_0$/$\Set{\txid_4}$
    , /$\val_1$/$\txid_1$/$\emptyset$
    , /$\val_2$/$\txid_2$/$\emptyset$
}};

%Location y
\path (versionx.east) + (1,0) node (locy) {$\ke_2 \mapsto$};
\draw pic at ([xshift=\tikzkvspace]locy.east) {vlist={versiony}{%
    /$\val_0$/$\txid_0$/$\Set{\txid_2}$
    , /$\val_3$/$\txid_3$/$\Set{\txid_4}$
    , /$\val_4$/$\txid_4$/$\emptyset$
}};

%%Location z
%\path (versiony.east) + (1,0) node (locz) {$\ke_3 \mapsto$};
%\matrix(versionz) [version list,column 2/.style={text width=7mm}]
   %at ([xshift=\tikzkvspace]locz.east) {
 %{a} & $\txid_0$ & {a} & $\txid_3$ & {a} & $\txid_4$ \\
  %{a} & $\emptyset$ & {a} & $\emptyset$ & {a} & $\emptyset$\\
%};

%\tikzvalue{versionz-1-1}{versionz-2-1}{locz-v0}{$\val_0$};
%\tikzvalue{versionz-1-3}{versionz-2-3}{locz-v1}{$\val_1$};
%\tikzvalue{versionz-1-5}{versionz-2-5}{locz-v2}{$\val_2$};

%%Location w
%\path (versionz.east) + (1,0) node (locw) {$\ke_4 \mapsto$};
%\matrix(versionw) [version list,column 2/.style={text width=7mm}]
    %at ([xshift=\tikzkvspace]locw.east) {
    %{a} & $\txid_0$ & {a} & $\txid_1$ \\
    %{a} & $\{\txid_4\}$ & {a} & $\emptyset$ \\
%};
%\tikzvalue{versionw-1-1}{versionw-2-1}{locw-v0}{$\val_0$};
%\tikzvalue{versionw-1-3}{versionw-2-3}{locw-v1}{$\val_1$};
\end{centertikz}
\vspace{5pt}
\caption{Allowed by \( \UA \) and \( \CP \) but disallowed by \(\SI\)}%
\label{fig:si-disallowed}%
\end{subfigure} \\
\hline
\end{tabular}
%\begin{tabular}{@{}c @{} c @{}}
%\begin{minipage}{0.4\textwidth}
%\begin{subfigure}{\textwidth}
%\begin{centertikz}
%%Location x
%\node(locx) {$\ke_1 \mapsto$};
%
%\matrix(versionx) [version list,,column 2/.style={text width=8mm},column 4/.style={text width=7mm}]
%    at ([xshift=\tikzkvspace]locx.east) {
%    {a} & $\txid_0$ & {a} & $\txid_{\cl}^{1}$\\
%    {a} & $\left\{\txid_{\cl'}^{2}\right\}$ & {a} & $\emptyset$ \\
%};
%\tikzvalue{versionx-1-1}{versionx-2-1}{locx-v0}{$\val_0$};
%\tikzvalue{versionx-1-3}{versionx-2-3}{locx-v1}{$\val_1$};
%
%%Location y
%\path (locx.south) + (0,\tikzkeyspace) node (locy) {$\ke_2 \mapsto$};
%\matrix(versiony) [version list,column 2/.style={text width=8mm},column 4/.style={text width=7mm}]
%   at ([xshift=\tikzkvspace]locy.east) {
% {a} & $\txid_0$ & {a} & $\txid_{\cl'}^1$ \\
%  {a} & $\left\{\txid_{cl}^{2}\right\}$ & {a} & $\emptyset$\\
%};
%
%\tikzvalue{versiony-1-1}{versiony-2-1}{locy-v0}{$\val_0$};
%\tikzvalue{versiony-1-3}{versiony-2-3}{locy-v1}{$\val_2$};
%
%\end{centertikz}
%\caption{Disallowed by \(\CP\)}
%\label{fig:cp-disallowed-2}
%\end{subfigure}
%
%\begin{subfigure}{\textwidth}
%\begin{centertikz}
%%Location x
%\node(locx) {$\ke_1 \mapsto$};
%
%\matrix(versionx) [version list,column 2/.style={text width=7mm},column 4/.style={text width=7mm}]
%    at ([xshift=\tikzkvspace]locx.east) {
%    {a} & $\txid_0$ & {a} & $\txid_1$\\
%    {a} & $\left\{\txid_2\right\}$ & {a} & $\emptyset$ \\
%};
%\tikzvalue{versionx-1-1}{versionx-2-1}{locx-v0}{$\val_0$};
%\tikzvalue{versionx-1-3}{versionx-2-3}{locx-v1}{$\val_1$};
%
%%Location y
%\path (locx.south) + (0,\tikzkeyspace) node (locy) {$\ke_2 \mapsto$};
%\matrix(versiony) [version list,column 2/.style={text width=7mm},column 4/.style={text width=7mm}]
%   at ([xshift=\tikzkvspace]locy.east) {
% {a} & $\txid_0$ & {a} & $\txid_2$ \\
%  {a} & $\left\{\txid_1\right\}$ & {a} & $\emptyset$\\
%};
%
%\tikzvalue{versiony-1-1}{versiony-2-1}{locy-v0}{$\val_0$};
%\tikzvalue{versiony-1-3}{versiony-2-3}{locy-v1}{$\val_2$};
%\end{centertikz}
%\caption{Disallowed by \(\SER\)}
%\label{fig:ser-disallowed}
%\end{subfigure}
%
%\end{minipage}
%
%&
%\begin{subfigure}{0.55\textwidth}%
%\begin{centertikz}%
%%Location x
%\node(locx) {$\ke_1 \mapsto$};
%
%\matrix(versionx) [version list,column 2/.style={text width=7mm}]
%    at ([xshift=\tikzkvspace]locx.east) {
%    {a} & $\txid_0$ & {a} & $\txid_1$ & {a} & $\txid_2$\\
%    {a} & $\emptyset$ & {a} & $\emptyset$ & {a} & $\emptyset$\\
%};
%\tikzvalue{versionx-1-1}{versionx-2-1}{locx-v0}{$\val_0$};
%\tikzvalue{versionx-1-3}{versionx-2-3}{locx-v1}{$\val_1$};
%\tikzvalue{versionx-1-5}{versionx-2-5}{locx-v1}{$\val_2$};
%%Location y
%\path (locx.south) + (0,\tikzkeyspace) node (locy) {$\ke_2 \mapsto$};
%\matrix(versiony) [version list,column 2/.style={text width=7mm}]
%   at ([xshift=\tikzkvspace]locy.east) {
% {a} & $\txid_0$ & {a} & $\txid_3$ \\
%  {a} & $\left\{\txid_2\right\}$ & {a} & $\emptyset$\\
%};
%
%\tikzvalue{versiony-1-1}{versiony-2-1}{locy-v0}{$\val_0$};
%\tikzvalue{versiony-1-3}{versiony-2-3}{locy-v1}{$\val_1$};
%
%%Location z
%\path (locy.south) + (0,\tikzkeyspace) node (locz) {$\ke_3 \mapsto$};
%\matrix(versionz) [version list,column 2/.style={text width=7mm}]
%   at ([xshift=\tikzkvspace]locz.east) {
% {a} & $\txid_0$ & {a} & $\txid_3$ & {a} & $\txid_4$ \\
%  {a} & $\emptyset$ & {a} & $\emptyset$ & {a} & $\emptyset$\\
%};
%
%\tikzvalue{versionz-1-1}{versionz-2-1}{locz-v0}{$\val_0$};
%\tikzvalue{versionz-1-3}{versionz-2-3}{locz-v1}{$\val_1$};
%\tikzvalue{versionz-1-5}{versionz-2-5}{locz-v2}{$\val_2$};
%
%%Location w
%\path (locz.south) + (0,\tikzkeyspace) node (locw) {$\ke_4 \mapsto$};
%\matrix(versionw) [version list,column 2/.style={text width=7mm}]
%    at ([xshift=\tikzkvspace]locw.east) {
%    {a} & $\txid_0$ & {a} & $\txid_1$ \\
%    {a} & $\{\txid_4\}$ & {a} & $\emptyset$ \\
%};
%\tikzvalue{versionw-1-1}{versionw-2-1}{locw-v0}{$\val_0$};
%\tikzvalue{versionw-1-3}{versionw-2-3}{locw-v1}{$\val_1$};
%\end{centertikz}%
%\caption{Disallowed by \(\SI\)}%
%\label{fig:si-disallowed}%
%\end{subfigure} \\
%\end{tabular}
\hrulefill
\vspace*{-5pt}
\caption{Behaviours disallowed under different consistency models}
\label{fig:anomalies}
\vspace*{-10pt}
\end{figure*}


We now give examples of execution tests in~\cref{fig:execution.tests},
where the associated consistency models for kv-stores correspond to
widely adopted consistency guaranteees for distributed databases.
Following \cite{distrprinciples}, we distinguish between
client- and data-centric consistency models: 
the former constrain the client views; 
the latter impose conditions on the structure of the kv-store.  
In \cref{fig:anomalies} we give illustrative
examples of kv-stores allowed/disallowed by our
consistency models.

\mypar{Monotonic Reads ($\MRd$)}
It states that a client cannot loose information from the view and 
hence read operations can only read increasingly more up-to-date versions. 
This prevents \eg the kv-store of \cref{fig:mr-disallowed},
since client $\cl$ first observes the latest version of $\ke$ in $\txid_{\cl}^{1}$, 
and then observes the older, initial version of $\ke$ in $\txid_{\cl}^{2}$.  
The execution test $\ET_{\MRd}$ ensures that clients  can only extend their views. 

\mypar{Monotonic Writes ($\MW$)}
It states that whenever a transaction observes a version installed by a client $\cl$,
then it observes all previous versions installed by $\cl$. 
This prevents \eg the kv-store of \cref{fig:mw-disallowed}, since 
transaction $\txid'$ observes the second version of $\ke_2$, 
with value $\val_2$ written by client $\cl$, 
but it does not observe the second version of $\ke_1$, 
with value $\val_1$ and previously written by the same client.
The execution test $\ET_{\MW}$  ensures that, prior to executing a transaction,
the set of versions included in the view of the client are write 
prefix-closed with respect to the relation $\PO^?$.

\mypar{Read Your Writes (\RYW)}
It states that a client must always be able to read the versions previously written by the client itself.
This prevents the kv-store in \cref{fig:ryw-disallowed}, 
as the initial version of $\ke$ holds value $0$ 
and client $\cl$ tries to increment the value of $\ke$ by $1$ twice.  
For its first transaction, it reads the initial value $0$ and then installs  a new version with value $1$. 
For its second transaction, since the client need not read its own writes, 
it might read the initial value $0$ again and install a new version with value $1$.
The execution test $\RYW$ ensures that, after committing a transaction, 
the client view includes all the versions it wrote.  

\pg{Below, not well explained at all, I  keep thinking I'm saying
  the same thing again and again.}
\mypar{Write Follows Reads (\WFR)}
%It states that if a client \( \cl \) writes some version $\ver$ in a transaction,
%following  another transaction (or in the same transaction of) who reads of some version $\ver'$, 
%then a transaction may observe version $\ver$ only if it also
%observes $\ver'$. 
It states that, if a transaction observes a version written by a
client $\cl$, then it must also observe the versions previously read by $\cl$ (in $\PO^?$ relation).
This prevents the kv-store of \cref{fig:wfr-disallowed},
since transaction $\txid$ observes a version written by $\cl$ but
not a version previously read by $\cl$.
% the version $\ver_2$ of $\ke_2$ carrying value $\val_2$ written by client $\cl$,
%but the same transaction $\txid$ does not observe the version of $\ke_1$ carrying value $\val_1$, read by $\cl$ prior to writing $\%ver$. 
The execution test $\ET_{\WFR}$  ensures
that a view includes all versions previously read by a client 
if the view already includes a write from that client. 

%\azalea{For \CC, the English and the definition definitely don't match and it is hard to see 
%$\ET_{\MRd} \cap \ET_{\MW} \cap \ET_{\RYW} \cap \ET_{\WFR}$ is indeed the necessary and sufficient condition for the English description. 
%Since we are short on space, I suggest we simply say:
%``The causal consistency guarantee has been defined in the literature~\cite{session2causal} as the conjunction of the four \emph{session guarantees} \MRd, \MW, \RYW\ and \WFR. 
%Analogously, we have defined $\ET_\CC$ as the intersection of the execution tests corresponding to the four session guarantees."
%\sx{Agree}
%}


\mypar{Causal Consistency (\CC)}
%Causal Consistency ensures that if a client observes a version $\ver$, 
%then it also observes the versions on which $\ver$ \emph{depends} \cite{cops} via the
%$\PO^?$  and $\RF_{\hh}$ (defined in \cref{fig:execution_tests}) relations.
%In case of $\PO^?$, this ensures that when a view includes the effects of a transaction by client $\cl$, 
%it also includes the effects of the earlier transactions (in $\PO$ order) of $\cl$. 
%In case of $\RF_{\hh}$, this ensures that when a view includes a transaction,
%it must include all writes that the transaction read from.
%
%
%\pg{Above sentence, I don't know what to write here as I don't understand what it is
%  saying.  For $    \RF_{\hh}$, it means when a view includes a transaction (the versions it write),
%it must include all the writes that the transaction read from.}
%
%A necessary and sufficient condition is to enforce the four session
%guarantees $\MRd, \MW, \RYW$ and $\WFR$ \cite{session2causal}:
%$\ET_{\CC} = \ET_{\MRd} \cap \ET_{\MW} \cap \ET_{\RYW} \cap
%\ET_{\WFR}$.
%
%
The causal consistency guarantee is defined in the literature~\cite{session2causal} as the conjunction (composition) of the four \emph{session guarantees} \(\MRd\), \(\MW\), \(\RYW\) and \(\WFR\):   
$\CMs_{\CC} \eqdef \CMs_{\MRd} \cap \CMs_{\MW} \cap \CMs_{\RYW} \cap \CMs_{\WFR}$. 
Analogously, we have defined $\ET_\CC$ as the conjunction (composition) of the execution tests corresponding to the four session guarantees.
As we discuss in \cref{sec:other_formalisms}, we can do this thanks to the \emph{compositionality} of our execution tests:
the composition of several consistency models is equivalent to the consistency model induced by the composition (intersection) of the corresponding execution tests. 
That is, $\CMs(\ET_\CC) = \CMs(\ET_{\MRd}) \cap \CMs(\ET_{\MW}) \cap
\CMs(\ET_{\RYW}) \cap \CMs(\ET_{\WFR}) = \CMs(\ET_{\MRd} \cap
\ET_{\MW} \cap \ET_{\RYW} \cap \ET_{\WFR})$.



As we discuss in \cref{sec:applications} and prove in the Appendix~\ref{},  the COPS
implementation~\cite{} satisfies
$\ET_{\CC}$. 

\mypar{Update Atomic ($\UA$)}
This model has been proposed in~\cite{framework-concur};
however we are not aware of an implementation that exactly meets  this model.
Nevertheless, many implementations meet a
\emph{strengthening} of $\UA$.
$\UA$ disallows concurrent transactions writing to the same key,
a property known as \emph{write-write conflict freedom}.
This prevents the kv-store of \cref{fig:ua-disallowed},
as $\txid, \txid'$ concurrently increment the initial version of $\ke$ by $1$.
Note that $\UA$ generalises $\RYW$: unlike $\RYW$, $\UA$ does not require $\txid, \txid'$ to be executed by the same client.
The $\ET_\UA$ ensures write-write conflict freedom by allowing a client to write to key $\ke$
only if its view includes all versions of $\ke$.

\mypar{Parallel Snapshot Isolation (\PSI)} 
The guarantees of $\PSI$ have been defined as the conjunction of the guarantees provided by $\CC$ and $\UA$~\cite{framework-concur}.
Analogously, we have defined $\ET_\PSI$ as the intersection of $\ET_\CC$ and $\ET_\UA$. 
Once again, this compositional definition is due to the \emph{compositionality} of our execution tests, as discussed in \cref{sec:other_formalisms}.


\mypar{(Strict) serialisability (\SER)}
Serialisability is the strongest consistency model, requiring that there exist a sequential schedule of transactions. 
The execution test $\ET_{\SER}$ thus allows clients to execute transactions only when 
their view of the kv-store store is complete.
This prevents the kv-store in  \cref{fig:ser-disallowed}: under serialisability either $\txid_1$ or $\txid_2$ commits first.
In the former case when $\txid_1$ commits first, then its write to $\ke_2$ must be included in the view of $\txid_2$, and thus $\txid_2$ should not read the outdated write to $\ke_2$ by $\txid_0$. 
The latter case is analogously prohibited. 
This example is allowed by all the other execution tests in~\cref{fig:execution_tests}.





\mypar{Consistent Prefix ($\CP$)}
\label{para:cp}

When the total order in which transactions commit is known, then 
$\CP$ is described by the following property: 
if a client observes the effects of a transaction $\txid$,
then it must also observe the effects of any transaction that commits before
$\txid$. 
\pg{Above statement cannot be correct. We do not have WW; RW
in CP. }
Our configurations of kv-stores and views only provides {\em
  partial} information about the order in which transactions commit,
but this information is enough to define the consistency models we
seek. 

----the first sentence is not correct, it's in the coments which I
currently cannot see.....

Consider the transaction relations $\RF_{\hh}$, $\VO_{\hh}$ and
$\AD_{\hh}$ defined in Figure~\ref{.}, adapted from well-known
transaction relations associated with dependency graphs~\cite{??}.  If
$(t,t')$ is in $\RF_{\hh}$, $\VO_{\hh}$ or $\PO$ then it {must} be
the case that $t$ commits before the start of $t'$. Let
$\texttt{Rel} =\RF_{\hh} \cup \VO_{\hh} \cup \PO$.  If $(t,t')$ is in
$\texttt{Rel}; \AD_{\hh}$, then it {must} also be the case that
$t$ commits before the start  of $t'$, since a property of
$(t'', t') \in \AD_{\hh}$ is that the start of $t''$ must occur before
$t'$ commits. In fact, the relation
$ ((\PO \cup \RF_{\hh} \cup \VO_{\hh}) ; \AD_{\hh}^?)^{+} \ni (t,t') $
captures that, if transaction $t'$ writes to the kv-store, then the
relation identifies {\em all} the  transactions $t$ that {must} have been written written to the kv-store.
The $\ddagger$ relation, used in $\SI$ below, 
captures the property that if a transation observes 
the effect of a transaction then it observes the effect of all previous
transactions. The $\dagger$ relation is a weaker condition which 
does not use $\RF_{\hh};
\AD_{\hh}$ but is enough to capture $\CP$. 
The execution test $\ET_{\CP}$ is the intersection of $\dagger$ with $\ET_\MRd \cap \ET_\RYW $.

\pg{Somewhere I have the example in a comment, I can't find,
  presumably it's easy when the comments are turned on. Or I've
  commented out. Anyway we need the example explained. The example you
  explained has t3 and t4 in it, so was not correct.}

\pg{In above, how do we  justify the all?}


\pg{I don't know how to explain $\dagger$.}

\pg{Find different notation to $\texttt{Rel} $.}

\pg{Presumably $\dagger$ has nice properties so that the consistency
  model of $\ET_{\CP}$ makes sense.}

 %The interesting
%case is the use of the $\AD_{\hh}$ relation: by itself, this relation
%only identifies reads happening before writes; however, used in the
%relations $\PO ;
%\AD_{\hh}$, it identifies further  writes happening
%before $t$. The $\ddagger$ relation
%\[
%\WTx(\hh(\ke', j)) \toEdge{((\PO ; \AD_{\hh}? )\cup \RF_{\hh} \cup \etVO_{\hh})^{+}} t
%\]
%identifies all the writes that occur before $t$, where $+$ is the
%transitive closure. 



%ensures that the initial view is 
%closed with respect to this $\dagger$  relation, and the updated view 
%increases ($ \ET_\MRd $) and 
%includes the writes associated with the fingerprint using the
%session order ($\ET_\RYW $).

\pg{I do not know how to explain why $WW; RW $ is not there.}

\pg{Here you seem to be talking about SI?
That is, \( \txid_3 \) observes that the update of $\ke_2$ carrying value $\val_2$ happens before the update of $\ke_1$ carrying value $\val_2$,
yet $\txid_{4}$ observes that the update of $\ke_1$ carrying value
$\val_1$ happens before the update of $\ke_2$ carrying value
$\val_2$. }

%\pg{Above, presumably, with Section 5 you have proved somewhere that \[
%\WTx(\hh(\ke', j)) \toEdge{((\PO ; \AD_{\hh}? )\cup \RF_{\hh} \cup \VO_{\hh})^{+}} t
%\]
%identifies all the writes that occur before $t$? Can you reference
%this? All we know from here is that the writes happen before $t$, we
%do not know that we have all of them.

%We coupld perhaps justify transitivity and $\ET_{\CP}$ more. }


\pg{Do we know that the CM relation commutes with union and ;? }

%\pg{Below are my old comments when I had lost it.  I like above much
 % better than below because it's explained in terms of kv-stores
 % alone. Here we are justifying kv-stores, next section we are linking
 % to dependency graphs and abstract execution.}

%Inspired by dependency graph \cite{.....},
%there are minimum observable transactions for each transaction derived from the following:
%\[
 %   \SO  \subseteq  \VIS \qquad
  %  ( ( ( \SO \cup \WR ) ; \RW? )^* \cup \WW ) ; \VIS \subseteq \VIS
%\]

%define earlier so not explained here, get rid of all but essential
%detail as difficult enough, people do not need to be reminded what
%composition is
%where the \( R? \) is the reflexive closure of the relation \( R \) 
%and \( R_1 ; R_2 \defeq \Setcon{(a,b)}{\exsts{c} (a,c) \in R_1 \land
%(c,b) \in R_2 } \) is the composition of the two relation.
%\pg{What are the minimum observable transitions?}
%Given the minimum observable transactions, we can specify $\CP$. 
%\pg{Next sentence, the  English and formal relation $
 % \ET_\RYW $ do not match.}
%First, \( \SO \subseteq \VIS \) means a transaction observes all previous transactions from the same client,
%and it is enforced by \( \ET_\RYW \).
%\pg{At this point, I previously gave up.}
%Then the combination of \( \dagger\) (\cref{fig:execution-tests}) and
%\( \ET_\MRd \) gives us \( ( ( ( \SO \cup \WR ) ; \RW? )^* \cup \WW )
% \VIS \subseteq \VIS \).



%Let consider a client \( \cl \) and the view \( \vi \).
%Assume two transactions \( \txid, \txid' \)  such that \( \txid' \) is in the view \( \vi \) and \( \txid \toEdge{( ( ( \SO \cup \WR ) ; \RW? )^* \cup \WW )} \txid' \).
%If \( \txid' \) is a transaction already observable by some previous transaction from the client \( \cl \), 
%the transaction \( \txid \) must be observable by that time,
%therefore by the \( \ET_\MRd \), the transaction \( \txid \) is in the current view \( \vi \).
%Otherwise, if \( \txid' \) is a transaction that is first time observed by the client \( \cl \),
%the \( \ddagger \) predicate enforces \( \txid \) is also in the view
%\( \vi \).



\mypar{Snapshot Isolation (\SI)}

When the total order in which transactions commit is known then
$\SI = \CP \cap \UA$~\cite{gsi,framework-concur}.  When we just have
the partial order given by kv-stores and views, then the analogous equality does
not hold. For example, the kv-store of \cref{fig:si-disallowed} is
included in both $\CMs(\ET_{\CP})$ and $\CMs(\ET_{\UA})$, but is
disallowed by \SI.  The technical reason for this is that
Theorem~\ref{??}  only holds under certain conditions on
\( \ET_1 \) and \( \ET_2 \)\footnote{%
  This issue also arises with dependency graphs~\cite{.}, see
  Section~\ref{5}.  }, and these conditions are not met by $\ET_{\CP}$
and $\ET_{\UA}$. 
We are able to capture  $\SI$ using  $\ddagger$ which 
states that, if a transation observes a
the effect of a transaction, then it observes the effect of all previous
transactions. The execution test $\ET_{\SI}$ is the intersection of
$\ddagger$, $\ET_\MRd $, $\ET_\RYW $ and $ \ET_{\UA}$.



\pg{In above, presumably you can pinpoint exactly why these conditions are not
  met by $\ET_{\CP}$ and $\ET_{\UA}$? Presumably it is to do with
  dagger. Does this shed light on why we
  need $\ddagger$ not $\dagger$? } 


\pg{Is it the case that the intersection equality holds when using
  $\ddagger$. Presumably so. }

\pg{All the above needs checking, based on our discussion yesterday.}

%Again inspired by~\cite{.}, we can formulate the  $\SI $ property
%without having full knowledge of the total order on transactions.
%The additional component in the relation for $\SI$, compared
%with the relation of $\CP$, is the use of the relation
%$\VO_{\hh}; \AD_{\hh} $....
%\pg{And now I don't know what to say, because this clearly gives more
 % writes than CP before $t$, so I've got the English of CP wrong. This
 % needs to be subtly explained.}
%The execution test $\ET_{\CP}$ ensures that the initial view is 
%closed with respect to this $\ddagger$  relation and 
%sees all the whole history of keys associated with the fingerprint
%($\ET_{\UA}$, and the updated view 
%increases ($ \ET_\MRd $) and 
%includes the writes associated with the fingerprint using the
%session order ($\ET_\RYW $).


%\pg{Below, I had previously given  up. I think the above style is
 % better. Now definitely out of time.}
%We are inspired by the following constraint that has been proven satisfying \( \SI \) \cite{cerone:snapshot}:
%\[
  %  (\SO \cup \WW) \subseteq \VIS \quad  ( (\SO \cup \WW \cup \WR) ; \RW? ) ; \VIS \subseteq \VIS
%\]
%where write-write relation \( (\txid, \txid') \in \WW \) means the transaction \( \txid \) installs a version for a key \( \ke \) following by \( \txid' \) installing a new version for the key \( \ke \).
%The constraint \( \SO \subseteq \VIS \) coincides with \( \ET_\RYW \).
%The \( \WW \subseteq \VIS \) means two transactions cannot concurrently write to the same key,
%which is enforced by \( \ET_\UA \).
%Let consider \( ( (\SO \cup \WW \cup \WR) ; \RW? ) ; \VIS \subseteq \VIS \).
%Similar to the argument we made in Consistent Prefix (\pageref{para:cp}), 
%let assume a client \( \cl \), its view \( \vi \) and two transactions \( \txid, \txid' \) such that 
%\( \txid' \) is in the view \( \vi \)
%and \( \txid \toEdge{(\SO \cup \WW \cup \WR) ; \RW?} \txid' \).
%If \( \txid' \) is observable by any previous transaction of the client \( \cl \),
%then \( \txid \) is also observable before.
%By \( \ET_\MRd\), it is the case \( \txid \) is in the view \( \vi \).
%If \( \txid' \) is a new transaction observed by the client \( \cl \),
%the \( \ddagger \) enforces that \( \txid \) should be included \( \vi \).


As we discuss in \cref{sec:applications} and prove in the Appendix~\ref{LHKJHK},  the Clock-SI
implementation~\cite{Du:2013:CSI:2553409.2553434} satisfies
$\ET_{\SI}$. 

%\subsection{Multi-version Key-value Stores and Views}
\label{sec:mkvs-view}

\subsubsection{Multi-version Key-value Stores} 
We assume a countably infinite set of \emph{client identifiers} $\Clients \defeq \Set{\cl, \cl',\cdots}$. 
We define the set of \emph{transaction identifiers} 
$\TxID \defeq  \Set{\txid_{0}} \uplus \Set{ \txid_{\cl}^{n} \mid \cl \in \Clients \wedge n \geq 0 }$, where 
 $\txid_0$ denotes a designated transaction used for initialisation, 
 and for each $n \in \mathbb{N}$, $\txid_{\cl}^{n}$ identifies a transaction 
 committed by client $\cl$.
% the $n$\textsuperscript{th} transaction of client $\cl$. 
Elements of $\TxID$ are ranged over by $\txid, \txid', \cdots$, 
while subsets of $\TxID$ are ranged over by $\txidset, \txidset', \cdots$. 
We define $\TxID_{0} \defeq \TxID \setminus \{ \txid_0\}$.
As we will see, we assume that each client is bound to a single session, and 
we use the superscript $n$, in a transaction identifier $\txid_{\cl}^{n}$, 
to embed the information about the session order $\PO$ in which clients execute 
transactions:  
%
%The structure of $\TxID$  
%embeds the transaction execution order for each client, or the \emph{session order} $\PO$. 
%More concretely, 
$\PO \defeq \Set{ (\txid, \txid') \mid \exsts{ \cl, n,m } \txid = \txid_{\cl}^{n} \wedge \txid' = \txid_{\cl}^{m} \wedge n < m}$.
%As such, $(\txid, \txid') \in \PO$ denotes that 
%client $\cl$ executes $\txid$ before $\txid'$.
For readability, we often write  $\txid \xrightarrow{\PO} \txid'$ for $(\txid, \txid') \in \PO$.

%Given a set $X$, 
%%then $\powerset{X}$ denotes 
%%the powerset of $X$,
%we write $X^{\ast}$ for the free monoid induced by $X$.
%We next define the notion of \emph{multi-version key-value stores}.


\begin{definition}[Multi-version Key-value Stores]
\label{def:his_heap}
\label{def:mkvs}
Assume a countably infinite set of \emph{keys} $\Keys = \Set{\ke, \ke', \cdots}$, 
and a set of \emph{values} $\Val = \{\val, \val', \cdots\}$.

A \emph{version} is a triple $\ver \in \Versions \defeq \Val \times \TxID \times \powerset{\TxID_{0}}$. 
%The set of versions is denoted by $$.
A \emph{key-value store} is a mapping $\hh: \Keys \rightarrow \Versions^{\ast}$, 
where we recall that $\Versions^{\ast}$ is the free monoid generated by $\Versions$.
%\ac{ The superscript fin over the $\rightharpoonup$ needs to be fixed. You may want to look at the package extpfeil.}
\end{definition}

%For simplicity, we instantiate the set of values as  $\Val \eqdef \Nat \uplus \Keys$,
Among the elements of $\Val$, we distinguish a default value $\val_0 \in \Val$. 
A \emph{version} $\ver = (\val, \txid, \txidset)$ comprises a value $\val$,
and the meta-data of the transactions that accessed it.
Specifically, the \emph{writer} $\txid$ identifies the transaction that wrote version $\ver$, 
and the \emph{readers} $\txidset$ denote the set of transactions that read from $\ver$.
Given a version $\ver = (\val, \txid, \txidset)$, we define $\valueOf(\ver) \defeq \val$,
$\WTx(\ver) \defeq \txid$ and $\RTx(\ver) \defeq \txidset$.
Lists of versions (\ie elements of $\Versions^{\ast}$) are ranged over by $\vilist, \vilist',\cdots$.

%A \emph{multi-version key-value store}, or \emph{kv-store}, 
%is a mapping from keys to lists of versions. 
Given kv-store $\hh$, key $\ke$ and index $i \geq 0$, 
we write $\hh(\ke, i)$ for the $i$-th version (starting from $0$) of $\ke$.
That is, if $\hh(\ke) = \ver_0 \cdots\ver_{n}$, then $\hh(\ke, i) \defeq \ver_{i}$ for $i \leq n$; 
and it is undefined otherwise. 
We write $\lvert \hh(\ke) \rvert$ for the length of $\hh(\ke)$. 

We focus on key-value stores whose consistency model enforces the \emph{atomic visibility} of transactions~\cite{framework-concur}. 
This amounts to requiring that transaction reads at most one version of each key, and similarly 
it writes at most one version for each key. From the point of view of key-value stores, 
these conditions amount to require that \textbf{(i)}
$\fora{\ke i, j} (o \leq i, j \abs{ \hh(\ke) } \land \RTx(\hh(\ke, i)) \cap \RTx(\hh(\ke, j)) \neq \emptyset ) \implies i = j$, 
\textbf{(ii)}
$\fora{\ke, i, j} (0 \leq i,j < \abs{ \hh(\ke) } \wedge \WTx(\hh(\ke, i)) = \WTx(\hh(\ke, j)) ) \implies i = j$. 
We also assume that the list of version for each key has an initial version carrying a default value $\val_0$, 
written by the designated initialisation transaction $\txid_0$: \textbf{(iii)} $\fora{\ke} \hh(\ke, 0) = (\val_0, \txid_0, \stub)$.
Finally, we assume that the state of a key-value store is consistent with 
the session order of clients; a client cannot read a version of a key that has 
been installed by a future transactions within the same session, and 
the order in which versions are installed by a single client must agree 
with its session order: \textbf{(iv)}
$\fora{ \ke, \cl, i,j, n, m} 0 \leq i < j < \abs{\hh(\ke)} 
    \land \txid_{\cl}^{n} = \WTx(\hh(\ke,i)) {} \wedge \txid_{\cl}^{m} \in \Set{\WTx(\hh(\ke,j))} \cup \RTx(\hh(\ke, i))
    \implies n < m $.
%
%Formally, we require the following well-formedness requirement from key-value stores: 
%
%\begin{enumerate}%[label=(\roman*)]
%\item\label{kv:wf.init} 
%    $\hh(\ke, 0) = (\val_0, \txid_0, \stub)$ for $\ke \in \dom(\hh)$, where $\val_0$ is the default value in $\Val$;
%\item\label{kv:wf.onewrite} 
%    transactions write at most one version for each key:
%\[
%\fora{\ke, i,j }
%0 \leq i, j < \abs{ \hh(\ke) }
%\land \WTx(\hh(\ke, i)) = \WTx(\hh(\ke, j))
%\implies i = j 
%\]
%\item\label{kv:wf.oneread} 
%    transactions read at most one version for each key:
%\[
%\fora{\ke, i,j } 
%0 \leq i, j < \abs{ \hh(\ke) }
%\land \RTx(\hh(\ke, i)) \cap \RTx(\hh(\ke, j)) \neq \emptyset 
%\implies i = j
%\]
%\item\label{kv:wf.so} 
%	transactions (of the same client) install different versions of a key in the session order; 
%%    the order in which transactions issued by the same client install different versions for a key $\ke$, 
%%    is consistent with the session order;
%    a client $\cl$ can read versions written by $\cl$ itself only after they have been installed:
%\begin{multline*}
%    \fora{ \ke, \cl, i,j, n, m} 
%    0 \leq i < j < \abs{\hh(\ke)} 
%    \land \txid_{\cl}^{n} = \WTx(\hh(\ke,i)) \\
%    {} \wedge \txid_{\cl}^{m} \in \Set{\WTx(\hh(\ke,j))} \cup \RTx(\hh(\ke, i))
%    \implies n < m
%\end{multline*}
%\end{enumerate}
%
We say that kv-stores that satisfy the conditions \textbf{(i)}-\textbf{(iv)} above are 
\emph{well-formed}.
Henceforth, we will always assume kv-stores tp be well-formed, and we use $\HisHeaps$ to denote 
the set of well-formed kv-stores.

\subsubsection{Views and Configurations}

Key-value stores track the global state of a database. 
However, clients do not need to agree on the portion of 
the state of the database that they observe. Different clients 
may observe different different subsets of versions of the same key.
when executing transactions,
%different \emph{clients} may observe different versions of the same key. 
To keep track of the versions observed by clients,
we introduce the notion of \emph{views} (\cref{def:view}). 

\begin{definition}[Views, configurations]
\label{def:view}
\label{def:cuts}
\label{def:views}
\label{def:configuration}
A \emph{view} of a kv-store $\hh$ is a mapping  
$\vi \in \Views(\hh) \defeq \Keys \to\powerset{\Nat}$ such that:
\begin{align}
    & \fora{ \ke } 
    0 \in \vi(\ke) 
    \wedge \fora{ i \in \vi(\ke) } 
    i < \abs{ \hh(\ke) } 
    \tag{WF}
    \label{eq:view.wf}\\
    %\Set{0} \subseteq \vi(\ke) \subseteq \Setcon{i}{ 0 \leq i < \abs{\mkvs(\ke)}}
    & 
    \fora{ \ke_1,\ke_2, i_1, i_2} 
	i_1 \in \vi(\ke_1) 
	\land \WTx(\hh(\ke_1, i_1)) = \WTx(\hh(\ke_2, i_2)) 
	\implies i_2 \in \vi(\ke_2)
	\tag{Atomic}
	\label{eq:view.atomic}
\end{align}
A \emph{configuration} $\conf \in \Confs$, is a pair $(\hh, \viewFun)$, 
where $\hh \in \HisHeaps$ and
$\viewFun : \Clients \parfinfun \Views(\hh)$. 
\end{definition}
Configurations extend key-value stores with the information of 
the views of each client. In a configuration $\conf = (\hh, \viewFun)$, the view of client 
$\cl$, $\vi = \viewFun(\cl)$ (if defined) determines for each key $\ke \in \Keys$ the sub-list of versions in $\hh$ 
that the client is aware of, or equivalently that it can observe. If $i,j \in \vi(\ke)$ and $i < j$, then the client is 
aware of the fact that $\hh$ contains the versions $\hh(\ke, i)$, $\hh(\ke, j)$, and that $\hh(\ke, j)$ is more 
up-to-date than $\hh(\ke, i)$. The client also observes the information relative to the versions $\hh(\ke, i)$ and 
$\hh(\ke, j)$, i.e. the value they carry and the meta-data relative to writing and reading transactions of such 
versions. 
Equation \eqref{eq:view.wf} in \cref{def:view} is a natural requirement, while \eqref{eq:view.atomic} 
models the atomic visibility of transactions: if a client observes the updates of a transaction $\txid$, then 
it must observe all the updates from $\txid$. 
We let $\Views = \bigcup_{\hh \in \HisHeaps} \Views(\hh)$ be the set of all view. 
Given a kv-store $\hh$ and two views $\vi, \vi' \in \Views(\hh)$, 
we write $\vi \viewleq \vi'$ when $\vi(k) \subseteq \vi'(\ke)$ for all $\ke \in \dom(\hh)$. 
The initial view $\vi_{0}$ is defined as $\vi_{0}(\ke) = \{0\}$ for each $\ke \in \Keys$.
A configuration $\conf_{0} = (\hh_{0}, \viewFun_{0})$ is
\emph{initial} if $\hh_{0}(\ke) = (\val_0, \txid_0, \emptyset)$ for all $\ke \in \Keys$.

Given a configuration $\conf = (\hh, \viewFun)$ and a client $\cl$ for which 
$\viewFun(\cl)$ is defined, the view $\vi(\cl)$ is used to determine a \emph{snapshot}, i.e. 
a mapping from each key to a unique value that the client observes when executing a transaction. 
In general, the snapshot of a transaction is also determined by a \emph{resolution policy} 
cite{}. Throughout this paper, we assume that the database employ the \emph{Last Writer 
Wins} \cite{} resolution policy to determine the snapshot of clients, although generalisation 
to different resolution policies is straightforward.

\begin{definition}[Snapshots]
\label{def:heaps}
\label{def:snapshot}
A snapshot is a mapping from keys to values \( \ss \in \Snapshots  \defeq \Keys \to \Val\).
Given $\hh \in \HisHeaps$ and $\vi \in \Views(\hh)$, the \emph{snapshot} of $\vi$ in 
$\hh$ is defined as $\snapshot(\hh, \vi) \defeq \lambda \ke \ldotp \valueOf(\hh(\ke, \max_{<}(\vi(\ke)))$, 
where we recall that $\max_{<}(\vi(\ke))$ is the maximum element in $\vi(\ke)$ with respect to the natural 
order $<$ over $\mathbb{N}$.
\end{definition}
Given a kv-store $\hh$, a key $\ke$ and a view $\vi$, we often commit an abuse of notation and write 
$\hh(\ke, vi)$ as a shorthand for 
$\hh(\ke, \max_{<}(\vi(\ke))$. Thus, $\snapshot(\hh, \vi) = \lambda \ke \ldotp \valueOf(\hh(\ke, \vi))$. 

\begin{remark}
Because the function $\snapshot(\hh, \vi)$ only selects the last version of $\hh$ comprised 
in $\vi$, one may wonder about the necessity of including multiple versions in the view of a 
key $\ke$.  Here we only point that requiring a view to contain a single version for each key 
would impair the expressiveness of our framework in terms of consistency models that it captures; 
unfortunately, we have to wait until \cref{} before giving more details on this issue.
\end{remark}

%Let $\vi$ be on a key-value store $\hh$; the view $\vi$ determines, 
%for each key $\ke \in \Keys$, the sub-list of versions in $\hh(\ke)$ 
%$\vi(\ke)$ determines the subset of versions that 
%  
%\emph{The set of views} 
%$
%\Views \defeq \bigcup_{\hh \in \HisHeaps} \Views(\hh)
%$.
%A \emph{configuration}, $\conf \in \Confs$, is a pair $(\hh, \viewFun)$, 
%where $\hh \in \HisHeaps$ and
%$\viewFun : \Clients \parfinfun \Views(\hh)$. 
%The $\conf_{0} = (\hh_{0}, \viewFun_{0})$ is an
%\emph{initial configuration} if, $\hh_{0}(\ke) = (\val_0, \txid_0, \emptyset)$ for all $\ke \in \Keys$.
%
%A view of a kv-store $\hh$ is a mapping from the keys in $\hh$ to a non-empty set of natural numbers. 
%For each $\ke \in \dom(\hh)$, $\vi(\ke)$ denotes the indices of versions in $\hh(\ke)$ recorded in $\vi$. 
%As such, when $i \in \vi(\ke)$ then $ i < \abs{ \hh(\ke) }$. 
%Moreover, the initialisation version (at index $0$) must be included in all views. 
%These two properties are captured by \eqref{eq:view.wf} in \cref{def:view} below. 
%Lastly, views cannot observe \emph{partial} effects of a given transaction. 
%That is, if a view includes a version written by a transaction $\txid$, it must include \emph{all} versions written by $\txid$. 
%This is formalised by \eqref{eq:view.atomic} in \cref{def:view} below, and captures the \emph{atomic visibility} of transactions. 


%At any point during execution, the overall state is captured by \emph{a configuration}. 
%A configuration includes a kv-store and a partial mapping from clients to views.
%The view of the client $\cl$ in $\hh$ reflects the set of versions for each key 
%that the client \(\cl \) observes upon executing a transaction. 
%The constraint of \cref{eq:view.atomic} establishes that if a client observes 
%a version of some key written by a transaction $\txid$, then it must observe all the versions of 
%all keys that $\txid$ wrote. 


%\begin{definition}[Views, configurations]
%\label{def:view}
%\label{def:cuts}
%\label{def:views}
%\label{def:configuration}
%A \emph{view} of a kv-store $\hh$ is a mapping  
%$\vi \in \Views(\mkvs) \defeq \dom(\hh) \to\powerset{\Nat}$ such that:
%\begin{align}
%    & \fora{ \ke } 
%    0 \in \vi(\ke) 
%    \wedge \fora{ i \in \vi(\ke) } 
%    i < \abs{ \hh(\ke) } 
%    \tag{WF}
%    \label{eq:view.wf}\\
%    %\Set{0} \subseteq \vi(\ke) \subseteq \Setcon{i}{ 0 \leq i < \abs{\mkvs(\ke)}}
%    & 
%    \fora{ \ke,\ke', i,j} 
%	j \in \vi(\ke) 
%	\land \WTx(\hh(\ke, j)) = \WTx(\hh(\ke', i) 
%	\implies i \in \vi(\ke')
%	\tag{Atomic}
%	\label{eq:view.atomic}
%\end{align}
%
%\emph{The set of views} is
%$
%\Views \defeq \bigcup_{\hh \in \HisHeaps} \Views(\hh)
%$.
%A \emph{configuration}, $\conf \in \Confs$, is a pair $(\hh, \viewFun)$, 
%where $\hh \in \HisHeaps$ and
%$\viewFun : \Clients \parfinfun \Views(\hh)$. 
%The $\conf_{0} = (\hh_{0}, \viewFun_{0})$ is an
%\emph{initial configuration} if, $\hh_{0}(\ke) = (\val_0, \txid_0, \emptyset)$ for all $\ke \in \Keys$.
%\end{definition}



%Given a kv-store $\hh$ and two views $\vi, \vi' \in \Views(\hh)$, 
%we write $\vi \viewleq \vi'$ when $\vi(k) \subseteq \vi'(\ke)$ for all $\ke \in \dom(\hh)$. 
%Also, we commit an abuse of notation and write $\hh(\ke, \vi)$ as a shorthand 
%for $\hh(\ke, \max_{<}(\vi(\ke)))$. 
%Note that such a version always exists as
%$\vi(\ke) \neq \emptyset$ (see \eqref{eq:view.wf} above).

%\subsubsection{Snapshots}
%Transactions are executed with respect to a \emph{snapshot} of a kv-store.
%A snapshot $\h$ is a mapping from keys to values (\cref{def:snapshot}). 
%Given a view $\vi$ of a transaction, a snapshot can be induced 
%by extracting the value of the latest observable version for each key $\ke \in \dom(\hh)$. 
%%Views are used to determine the snapshot in which a transaction 
%%is executed, according to the following definition.


%\begin{definition}[Snapshots]
%\label{def:heaps}
%\label{def:snapshot}
%A snapshot is a mapping from keys to values \( \ss \in \Snapshots  \defeq \Keys \to \Val\).
%Given $\hh \in \HisHeaps$ and $\vi \in \Views(\hh)$, the \emph{snapshot} of $\vi$ in 
%$\hh$ is defined as $\snapshot(\hh, \vi) \defeq \lambda \ke \ldotp \valueOf(\hh(\ke, \max_{<}(\vi(\ke)))$.
%\end{definition}

\paragraph{Fingerprints}
Once the execution of a transaction is completed, its effects are committed to the kv-store. 
The effects of transactions are modelled as a \emph{fingerprint} $\opset$. 
A finger print comprises a set of \emph{operations} $\Ops$: $\opset \subseteq \Ops$. 
An operation is a triple of the form $(l, \ke, \val)$ with $l \in \{\otR, \otW\}$.    
Intuitively, given the fingerprint $\opset$ of a transaction $\txid$, 
$(\otR, \ke, \val) \in \opset$ denotes that 
$\txid$ requested to read key $\ke$ from the kv-store, 
and it fetched a version carrying value $\val$.
Similarly, $(\otW, \ke, \val) \in \opset$ denotes that 
$\txid$ attempted to write value $v$ for key $\ke$. 
A fingerprint includes at most one read operation per key;
this formalises the intuition that, in our setting, 
transactions always read from an atomic snapshot of the kv-store. 
Analogously, a fingerprint includes at most one write operation per key.
%because a client either observes none or all the updates of a transaction.

\begin{definition}[Fingerprints]
The set of \emph{operations} is
$\Ops \defeq \{(l, \ke, \val) \mid$ $ l \in \Set{\otW, \otR} \land \ke \in \Keys \wedge \val \in \Val \}$.
A \emph{fingerprint} $\opset$ is a subset of operations, $\opset \subseteq \Ops$,
such that for all $\ke \in \Keys$ and \( l \in \Set{\otW, \otR} \),
if $(l, \ke, \val_1), (l, \ke, \val_2) \in \opset$, then $\val_1 = \val_2$.
\end{definition}



%\ac{Note that I now require fingerprints to be non-empty sets of transactions. This simplifies a lot the development of 
%the theory of kv-stores, and it fixes a problem that was spotted by Shale, that breaks the compositionality of 
%execution tests (see later). The main reason why we allowed empty fingerprints is that in the semantics, a client can 
%execute a transaction with no access to the memory. In practice, in the semantics we can require that at least 
%one access to the database must be performed in transactions. This can be checked syntactically, and nobody 
%should complain about that. I can put a remark about how this is a natural requirement that, if violated, 
%breaks the compositionality of consistency models.\\ 
%\textbf{Update 02/08/2018}: empty fingerprints are now allowed again. We still had some problems with compositionality, 
%one of which has to do with the fact that we allow the view of a client over some key to move freely after executing a transaction, 
%even if such a key was not accessed by the transaction. Later, I forbid this behaviour by requiring in execution tests that the 
%view of an client for a given key cannot be shifted if the transaction executed by the client did not access such a key.}
%$(\otW, \ke, \val) \in \opset$ means that the transaction writes a new version, carrying value $\val$, for key $\ke$. 




%of $\vi$ by accessing the value of 
%A view $\vi$ in $\hh$ naturally defines a snapshot $\snapshot(\hh, \nu)$
%A MKVS tracks the global state of the system; however, different \emph{clients} may observe different versions of the same key. 
%To model this, we introduce the notion of \emph{views} (\cref{def:views}). 
%A view $V$ reflects the particular version for each key that a client observes upon executing a transaction. 
%%We present an example of views in \cref{fig:hheap-a} with two views: $\client_1$ in red and $\client_2$ in blue.
%More concretely, the view for \( \client_1 \) is given formally as $\vi_1 = \Set{\key{k}_1 \mapsto 1, \key{k}_2 \mapsto 0}$.
%That is, the client with view $\vi_1$ observes the second version (at index 1) of key \( \ke_{1} \) with value $v_1$, and the first version (at index 0) of key \( \ke_2 \) with value $v'_0$.
%%, and 
%%the first version of $\key{k}_2$, carrying value $0$. Similarly, according to its view 
%%$V_2 = [\key{k}_1 \mapsto 2, \key{k}_2 \mapsto 2]$, the client $\txid_2$ observes 
%%in $\hh$the second and most up-to-date version for both $\key{k}_1$ and $\key{k}_2$.
%
%\begin{definition}[Views]
%\label{def:view}
%\label{def:cuts}
%\label{def:views}
%\emph{A view} is a partial finite function from keys to indexes:
%$
%\vi \in \Views \defeq \Addr \parfinfun \Nat 
%%\begin{rclarray}
%%    \vi \in \Views & \defeq & \Addr \parfinfun \Nat 
%%\end{rclarray}
%$.                                                                 
%The \emph{view composition}, $\composeVI: \Views \times \Views \rightharpoonup \Views$ is defined as the standard disjoint function union: $\composeVI \eqdef \uplus$. 
%% \( \vi \composeVI \vi' \defeq \vi \uplus \vi'\) 
%The \emph{unit view}, $\unitVI \in \Views$, is a function with an empty domain: $\unitVI \eqdef \emptyset$. 
%% and the unit is \( \unitVI \defeq \emptyset\).
%The \emph{order relation} on views, $\orderVI: \Views \times \Views$, is defined between two views with the same domain as the point-wise comparison of their indexes for each entry: 
%\[
%\begin{rclarray}
%    \vi \orderVI \vi' & \defiff & \dom(\vi) = \dom(\vi') \land \fora{\ke} \cu(\ke) \leq \cu'(\ke) \\
%\end{rclarray}
%\]
%\end{definition}
%%
%We say view $\vi$ is \emph{older} than view $\vi'$ (or $\vi'$ is \emph{newer} than $\vi$) whenever $\vi \orderVI \vi'$ holds.
%
%
%\mypar{Configurations} A \emph{configuration} comprises an MKVS, and the views associated with clients.
%In \cref{fig:hheap-a} we present an example of a configuration comprising an MKVS and the two views associated with clients $\client_1$ and $\client_2$. 
%We write $\version(\hh, \ke, \vi)$ for $\hh(\ke, \vi(\ke))$; 
%and write $\valueOf(\hh, \ke, \vi)$ as a shorthand for $ \valueOf(\version(\hh, \key{k}, V))$; similarly for $\WTx, \RTx$.
%%we commit an abuse of notation and often write $\valueOf(\hh, \ke, \vi)$ in lieu of $ \valueOf(\version(\hh, \key{k}, V))$, and similarly for $\WTx, \RTx$.
%When $\ver = \version(\hh, \ke, \vi)$, we say that \emph{$\vi$ $\ke$-points to $\ver$ in $\hh$}. 
%When $\ver = \hh(\ke, i)$ for some $0 \leq i \le \vi(\ke)$, we say that \emph{$\vi$ $\ke$-includes $\ver$ in $\hh$}.
%Lastly, we always assume that MKVSs, views, and configurations are well-formed, unless otherwise stated.
%
%
%
%\begin{definition}[Configurations]
%A view $\vi$ is \emph{well-formed with respect to an MKVS} $\mkvs$, written \( \wfV{\mkvs, \vi} \),  iff they have the same domain and every index from $\vi$ is within the range of the corresponding entry in $\mkvs$ and the view is \emph{atomic} with  respect to the key-value store: 
%\[
%\begin{rclarray}
%    \wfV{\mkvs, \vi} & \defeq & \dom(\mkvs) = \dom(\vi) \land \fora{\ke \in \dom(\vi)} 0 \leq \vi(\ke) < \lvert \mkvs(\ke) \rvert \\
%    \pred{atomic}{\vi ,\hh} & \eqdef & \fora{\txid } \exsts{\ke, i} i \leq \vi(\ke) \land \hh(\ke,i) = (\stub, \txid, \stub) \implies \pred{visible}{\txid, \vi, \hh} \\ 
%    \pred{visible}{\txid, \vi, \hh} & \eqdef & \fora{\ke, i} \hh(\ke,i) = (\stub, \txid, \stub) \implies i \leq \vi(\ke) 
%\end{rclarray}
%\]
%%
%\azalea{We need a symbol for this to fill the ???? above. Also ???? below. \sx{Done}}
%A \emph{configuration} $\conf$ is a pair of the form $(\hh, \viewFun)$, where $\hh$ denotes an MKVS, and $\viewFun: \Clients \parfinfun \Views$ is a partial finite function from clients to views. 
%A configuration $\conf = (\hh, \viewFun)$ is \emph{well-formed}, written \( \wfC{\conf}\), iff for all clients $\cl \in \dom(\viewFun)$, the view $\viewFun(\txid)$ is well-formed with respect to $\hh$. 
%%We say that a view $V$ is well-defined with respect to the 
%%MKVS $\hh$ if, $\forall \key{k} \in \ke. 0 < V(\key{k}) \leq 
%%\lvert \hh(\key{k}) \rvert$. 
%%Given a view $V$ that is well-defined 
%%with respect to a 
%
%\end{definition}
%
%\mypar{Snapshots} When a client executes a transaction on the $\mkvs$ MKVS, it extracts a \emph{snapshot} of it via the \( \func{snapshot}{\mkvs, \vi} \) function, extracting the values corresponding to the versions indexed by its view \( \vi \) (\cref{def:snapshot}).
%For instance, for client \( \client_1 \) in \cref{fig:hheap-a}, the $\func{snapshot}{\cdots}$ functions yields a state where key $\ke_1$ carries value $v_1$ and second key \( \ke_2 \) carries value $v'_0$.
%%The concrete state extracted in this way takes the name of the \emph{snapshot} of the transaction.
%%In general, the process of determining the view of a client, hence the snapshot in which such a client executes transactions, is non-deterministic.
%
%\azalea{Before in MKVSs we had values drawn from $\Nat$ in \cref{def:mkvs}. Now we use $\Val$. I think you mean to use $\Val$ in both places? \sx{I would say so} }
%\begin{definition}[Snapshots]
%\label{def:heaps}
%\label{def:snapshot}
%Given the sets of values $\Val$  and keys \( \Addr\)  (\cref{def:mkvs}), the set of \emph{snapshots} is:
%$
%    \h \in \Heaps \eqdef \Addr \parfinfun \Val
%$. 
%%\[
%%\begin{rclarray}
%%    \h \in \Heaps & \eqdef & \Addr \parfinfun \Val
%%\end{rclarray}
%%\]
%The \emph{snapshot composition function}, $\composeH: \Heaps \times \Heaps \parfun \Heaps$, is defined as $\composeH \eqdef \uplus$, where $\uplus$ denotes the standard disjoint function union. The \emph{ snapshot unit element} is $\unitH \eqdef \emptyset$, denoting a function with an empty domain.
%The \emph{partial commutative monoid of snapshots} is $(\Heaps, \composeH, \{\unitH\})$.
%Given an MKVS $\hh$ and a view $\vi$, the snapshot of $\vi$ in $\hh$, written $\snapshot(\hh, \vi) $, is defined as:
%$
%    \snapshot(\hh, \vi) \defeq \lambda \ke \ldotp \valueOf(\hh, \ke, \vi)
%$.
%%\[
%%\begin{rclarray}
%%    \snapshot(\hh, \vi) & \defeq & \lambda \ke \ldotp \valueOf(\hh, \ke, \vi).
%%\end{rclarray}
%%\]
%\end{definition}
%
%\sx{Need some explanation}
%\ac{General Comment on this Section: it is too abstract. We 
%should give either here or in the introduction an example of computation - 
%the write skew program should be okay that helps the reader understanding 
%what's going on. Also, it could be also good to illustrate the notions 
%of execution tests and consistency models.}
%
%\sx{From Andrea: introduce the execution test here with a table, also introduce fingerprint here}


%\subsection{Consistency Models and Execution Tests}
Formally, a \emph{consistency model} $\CMs$ is a set of key-value stores. 
Each $\hh \in \CMs$ represents a possible scenario that 
can be obtained as a result of multiple clients committing transactions. 
To specify consistency models we introduce the notion of \emph{execution tests}. 
\ac{
For example, \emph{serialisability} can be described as the set 
of key-value stores for which it is possible to recover a total schedule of transactions, 
such that each read operation on key $\ke$ fetches its value from the 
most recent write on the same key \cite{??????}.
In this sense, the kv-store $\hh$ from \cref{fig:hheap-a} is not serialisable: 
transaction $\txid_1$ reads the initial version carrying value $\val'_0$ for key $\ke_{2}$, 
and installs a new version of $\ke_{2}$ carrying value $\val_1$. The transaction $\txid_2$ 
reads the initial version carrying value $\val'_0$, and therefore, 
cannot be scheduled after $\txid_1$. Similarly, $\txid_2$ cannot be scheduled after $\txid_1$.
}
\begin{definition}
\label{def:execution.test}
An \emph{execution test} is a set of tuples $\ET \subseteq \HisHeaps \times \Views \times \powerset{\Ops} \times \Views$ 
such that for every element $(\hh, \vi, \opset, \vi') \in \ET$,
\textbf{(i)} for any read operations $(\otR, \ke, \val) \in \opset$ then $\hh(\ke, \max_{<}(\vi(\ke))) = \val$, 
and \textbf{(ii)}  for any key \( \ke \) such that $\vi(\ke) \neq \vi'(\ke)$, 
then $( (\otR, \ke, \_) \in \opset \vee (\otW, \ke, \_)) \in \opset)$.
%\textbf{(iii)} $\forall \opset' \subseteq \opset.\; (\hh, \vi, \opset', \vi') \in \ET$
\end{definition}
%\sx{The definition has a problem that for the subset \( \f'\) the post-view \( \vi' \) might point to an undefined version, I also thing it should satisfy \( \fora{\vi''} \vi \sqsubseteq \vi'' \implies (\hh, \vi, \opset', \vi'') \)}.
%\ac{I removed this condition, as I do not think that it was used anywhere. The new definition requires that 
%you cannot change the view for keys that you do not read nor write.}
%\sx{The current \CP will not satisfy the \textbf{(ii)}. }
Given an execution test $\ET$, 
then $(\hh, \vi, \opset, \vi') \in \ET$ means that 
a client whose view over the key-value store $\hh$ is $\vi$, 
can commit a transaction whose fingerprint is $\opset$;
as a result of this operation, the view of the client must be updated to $\vi'$.
Henceforth, we adopt the more suggestive notation $\ET \vdash (\hh, \vi) \triangleright \opset: \vi'$ 
in lieu of $(\hh, \vi, \opset, \vi') \in \ET$.
Execution tests induce \emph{consistency models} \( \CMs(\ET) \) as defined in \cref{def:reduction,def:cm}.
\begin{definition}[$\ET$-reductions]
\label{def:reduction}
Let $\cl$ be a client and $\opset$ be a fingerprint. 
An \emph{action} $\alpha \in \Act$ has either the form $(\cl, \varepsilon)$, 
or $(\cl, \opset)$. 
Given an execution test $\ET$ the action-labelled relation 
$\xrightarrowtriangle{}_{\ET} \subseteq \Confs \times \Act \times \Confs$ 
is defined as the smallest relation such that:
\begin{itemize}
\item 
    $\forall \vi, \vi', \cl, \hh, \viewFun.\; 
    \viewFun(\cl) = \vi 
    \wedge \vi \sqsubseteq \vi' 
    \implies (\hh, \viewFun) \xrightarrowtriangle{(\cl, \varepsilon)}_{\ET} 
    (\hh, \viewFun\rmto{\cl}{\vi'})$
\item 
    $\begin{array}[t]{@{}l@{}}
        \forall \vi, \vi', \cl, \opset, \hh, \hh', \viewFun.\; 
        \viewFun(\cl) = \vi
        \wedge (\ET \vdash (\hh, \vi) \triangleright \opset: \vi')  \\
        \quad {} \wedge \hh' \in \updateKV(\hh, \vi, \cl, \opset) 
        \implies (\hh, \viewFun) \xrightarrowtriangle{(\cl, \opset)}_{\ET} (\hh', \viewFun\rmto{\cl}{\vi'})
    \end{array}$
\end{itemize}
Such relations take the name of $\ET$-reductions, or simply reductions.
\end{definition}
Given an execution test $\ET$, sequences of $\ET$-reductions of the form $\conf_{0} \xrightarrowtriangle{\alpha_{0}}_{\ET} \cdots 
\xrightarrow{\alpha_{n-1}} \conf_{n}$ take the name of \emph{$\ET$-traces}.
\begin{definition}[Consistency Models]
\label{def:cm}
Given an execution test $\ET$, the set of configurations induced by $\ET$ is given by:
\[
\Confs(\ET) \defeq \Setcon{ \conf}{ \exsts{\conf_0} \conf_0 \text{ is initial } \wedge \conf_0 \xrightarrowtriangle{\stub}_{\ET} \cdots \xrightarrowtriangle{\stub}_{\ET} \conf }
\]
The \emph{consistency model} induced by $\ET$ is:
\( 
\CMs(\ET) \defeq \Setcon{ \hh }{ (\hh, \stub) \in \Confs(\ET) }
\)
\end{definition}
Thus, consistency models are computed from execution tests by closing the set of initial key-value stores with respect to two operations: 
\textbf{(i)} advancing the view of a client, 
and \textbf{(ii)} committing a fingerprint of a transaction. 

Last, for sanity check, consistency models induced by execution tests are monotonic in the following sense.
\begin{proposition}
\label{prop:mono-et}
Let $\ET_1 \subseteq \ET_2$. Then $\CMs(\ET_1) \subseteq \CMs(\ET_2)$.
\end{proposition}
\begin{proof}
    \ifTechReport
    It is sufficient to prove that \(\ET_1 \subseteq \ET_2 \implies \Confs(\ET_1) \subseteq\ Confs(\ET_2) \).
We prove it by induction on the length of the traces, \( n \).

\caseB{n = 0}
We have \( \conf_0 \in \Confs(\ET_1) \) and \( \conf_0 \in \Confs(\ET_2)\).
\caseI(n = i + 1)
Suppose identical traces of \( \ET_1 \) and \( \ET_2 \) respectively with length \( i \).
Let the final configuration be \( \conf_i = ( \mkvs_i, \viewFun_i ) \).
If the next step is a view shift or a step with empty fingerprint, it trivially holds.
If the next step is a step by a client \( \cl \) with fingerprint \( \f \),
we have \( \ET_1 \vdash \mkvs_i, \viewFun_i(\cl) \csat \f : vi' \).
The next configuration from \( \ET_1 \) is \( \conf_{i+1} = (\updateKV{ \mkvs_i, \viewFun_i(\cl), \f, \txid_\cl}) \).
Since \( \ET_1 \subseteq \ET_2 \), so \( \ET_1 \vdash \mkvs_i, \viewFun_i(\cl) \csat \f : vi' \) holds.
It is possible for \( \ET_2 \) to have the exactly same next configuration \( \conf_{n+1}\).

    \else
    See \cref{sec:mono-et}.
    \fi
\end{proof}

