\section{Introduction}
\label{sec:intro}

Transactions are the \emph{de facto} synchronisation mechanism in
modern distributed databases. To achieve scalability and performance,
distributed databases often use weak transactional consistency
guarantees known as \emph{consistency models}.  Many consistency
models were originally invented by engineers using (some quite
informal) definitions specific to particular real-world reference
implementations, \eg
\cite{ramp,si,distrsi,clocksi,cops,PSI-RA,SI-RA,NMSI,PSI}.
More recently, general definitions of consistency model have been
defined independently of particular implementations, either
declaratively using execution graphs \cite{adya,ev_transactions} or
operationally using abstract 
states or execution graphs
\cite{seebelieve,alonetogether,sureshConcur}. 
Our challenge  is to define a general semantics for weak consistency
models with which we can both verify reference implementations \emph{and} analyse 
the behaviour of client programs with respect to a particular consistency
model. 



The declarative approach for defining 
consistency models using execution graphs has been substantially
studied \cite{adya,ev_transactions,framework-concur,SIanalysis,laws}. 
In such graphs, nodes describe the read-write sets of  atomic transactions and edges describe the
known dependencies between transactions.
They capture different consistency models by:
\begin{enumerate*}
	\item constructing  \emph{candidate executions} of the whole program comprising
transactions in which reads may contain arbitrary values; and 
	\item applying the consistency-model \emph{axioms} to rule out candidate executions deemed invalid by the axioms. 
\end{enumerate*}
Such axioms may state, for example, that every read is
validated by a write that has written the read value. 
The most well-known execution graphs are dependency graphs \cite{adya} and abstract
executions \cite{ev_transactions,framework-concur}. 
Dependency graphs tend to be used to analyse  client programs: for
example, 
Fekete \etal\citet{fekete-tods} derived 
a static analysis checker for a particular weak consistency model called
snapshot isolation;  Bernardi and Gotsman \citet{giovanni_concur16}
developed a static analysis checker for several weak consistency
models assuming the so-called snapshot property\footnote{The \emph{snapshot property}, 
also known as \emph{atomic visibility}, states that transactional reads appear to read from an atomic snapshot
of the database and transactional writes appear to commit atomically;  \ie intermediate transactional states are not observable by clients, 
even if the underlying distributed protocol has a more fine-grained behaviour.}; and 
Beillahi \etal\citep{snapshot-isolation-robust-tool} developed a tool based on Lipton's reduction theory~\cite{Lipton-reduction}
for checking robustness\footnote{A particular program (or set of programs) behaves as if the consistency model is serialisability} properties against snapshot isolation.
%that is, a  particular program (or set of programs) behaves as if the consistency model is strong consistency. 
Abstract executions, on the other hand, tend to be used to verify  implementation protocols: for example,
abstract executions are the  standard by which many system engineers
demonstrate that their protocols satisfy  certain
consistency models \cite{cops,NMSI,PSI}. 
Execution graphs provide little information about how the 
state evolves throughout the execution of a program, and 
therefore seem  unsuitable for invariant-based program analysis 
of client programs. 


The operational approach for defining weak consistency models has been
much less studied.  Crooks \etal\citet{seebelieve} introduce
a {\em state-based} trace semantics over distributed kv-stores for
capturing consistency models associated with ANSI/SQL isolation
levels. They show the equivalence of several
implementation-specific definitions of consistency model in the
literature, but  consensus seems to be that it will be difficult to adapt their work to reason about
client programs\cite{givencitationwherethis quoteabout unlikelyis}.  Kaki \etal\citet{alonetogether} provide an
operational semantics over an abstract centralised store, again
focusing on ANSI/SQL isolation levels. They develop a program logic
and prototype tool for reasoning about client programs, but cannot
express fundamental weak consistency models. 
Nagar and Jagannathan~\cite{sureshConcur} introduce an operational semantics based on
abstract-execution graphs, focussing on consistency models for
distributed transactions. They provide robustness results for client
programs using model checking, but their analysis is indirect in that
they move back and forth between abstract executions and dependency
graphs. All these approaches have their merits. However, none 
provide a direct state-based operational semantics for distributed
atomic transactions with which to verify distributed implementations
and analyse client programs using the usual weak consistency models; 
see \cref{sec:newrelated} for further details on this related work. 



We introduce an interleaving operational semantics for describing the
client-observable behaviour of atomic transactions 
updating distributed key-value stores  (\cref{sec:model}). Our semantics is
based on a notion of abstract states comprising a \emph{centralised key-value store} (kv-store) with {multi-versioning} and a \emph{client view}.
Kv-stores are {\em global} in that they record all versions of a key; 
by contrast, client views are {\em partial} in that a client may see only a subset of the versions. 
Our client views are partly inspired by the views in the `promising' C11 semantics~\cite{promises}. 
An execution step depends simply on the abstract state, the read-write set of the atomic transaction, and an \emph{execution test} which
determines if a client with a given view is allowed to commit a transaction. Different execution tests give rise to different
consistency models, 
which we show to be equivalent to well-known
declarative definitions of consistency models based on abstract executions 
(reported here and proven in \cite{shale-phd}) and thus those based on dependency graphs~\cite{laws}. 
Our execution tests are analogous to the commit tests in \cite{seebelieve},
except that \cite{seebelieve} requires analysing the whole trace rather than just the current abstract state. 

As in~\cite{seebelieve,alonetogether,sureshConcur}, we assume that transactions satisfy the
\emph{last-write-wins} resolution policy, a policy widely used in many
real-world distributed kv-stores. 
This means that when a transaction observes several updates to a key, the atomic snapshot contains the
value written by the last
update. We also assume that our transactions satisfy the \emph{snapshot property}. This is a common assumption
in distributed transactional databases; for example, in online shopping applications, a client only sees one snapshot of the database and only
has knowledge that their transaction has successfully committed. The work in \cite{sureshConcur} also assumes the snapshot
property, whereas~\cite{seebelieve} and~\cite{alonetogether} do not as their focus is on ANSI/SQL isolation
levels \cite{si}.
Our execution tests uniformly capture  many well-known consistency
models (\cref{sec:cm}) including 
\emph{causal consistency} \((\CC)\) \citep{ev_transactions,cops,causal-def}, 
\emph{parallel snapshot isolation} \( (\PSI) \) \citep{NMSI,PSI},
\emph{snapshot isolation} \((\SI)\) \citep{si} 
and \emph{serialisability} \((\SER)\) \citep{Papadimitriou-ser}.
%Given these assumptions, we do not know of a consistency model that
%we cannot represent. 
The work in~\cite{sureshConcur} is as expressive as our work here; by contrast,~\cite{seebelieve} is more expressive,  capturing \eg the 
{\em read committed} consistency model~\cite{.}, 
while~\cite{ alonetogether}  is less expressive, capturing $\SI$ but not $\PSI$. 


Using our operational semantics, we verify 
that database protocols satisfy their expected consistency models and
prove invariant properties of client programs under such
consistency models (\cref{sec:applications}).
Specifically, we prove the correctness of two database
protocols using our general definitions: the COPS protocol for fully replicated kv-stores \cite{cops} 
which satisfies $\CC$ (reported in \cref{sec:verify-impl}
and proved in \cite{shale-phd}); 
and the Clock-SI protocol for partitioned kv-stores \cite{clocksi} 
which satisfies $\SI$  (given in \cite{shale-phd}). These results had been previously shown for
specific consistency definitions devised for the specific reference
implementations under consideration.
We also prove invariant properties of library clients (\cref{sec:invariant-client-programs}): the robustness of the single-counter library
against \( \PSI \);  the robustness of the multi-counter library and the
banking library \cite{bank-example-wsi} against \( \SI \); and the
mutual exclusion of a lock library against \( \PSI \). 
We believe our robustness results are the first to take into account client
sessions: with sessions, we show that multiple counters {\em are not} robust against \(\PSI\).
Interestingly, without sessions, Bernardi and Gotsman~\citet{giovanni_concur16} show that multiple counters \emph{are}
robust against \(\PSI\) using static-analysis techniques which are
known not to be applicable to sessions.  
These results indicate that our operational semantics provides an interesting abstract interface
between distributed databases and clients.
This was an important goal for us, resonating with recent work
that does just this for standard shared-memory concurrency \cite{tada,cap,iris,fcsl}. 

\subsection{Related Work} 
\label{sec:newrelated}

Operational semantics for defining weak consistency models for
distributed atomic transactions have hardly been
studied. To our knowledge, the key papers
are~\cite{seebelieve,sureshConcur,alonetogether}. 
We also mention the log-based semantics of Koskinen and Parkinson~\citet{push-pull},
which only focuses on serialisability but has some resonance with our work. 

Crooks \etal\citet{seebelieve} proposed a state-based trace
semantics for describing weak consistency models that employs concepts
similar to our client views and execution tests, called read states and
commit tests respectively.  In their semantics, a one-step trace
reduction is determined by the entire previous history of the trace.
By contrast, our reduction step only depends on the current kv-store
and view.  They capture more consistency models than us, \eg\emph{
  read committed}, because they do not assume the snapshot property
due to their focus on ANSI/SQL isolation levels. They use their semantics to 
demonstrate that 
several definitions of snapshot isolation given in the
literature~\cite{si,lazy-si,geo-si} in fact collapse into one.  They do not verify
protocol implementations and do not prove invariant properties of
client programs.  We believe~\cite{seebelieve} can be used to verify
implementations. We believe it is difficult to use~\cite{seebelieve}
to prove invariant properties of client programs.  Their 
traces contain much information including \eg the total
order on transactions. Not all this information is  observable by a
client: for example,  a client on one replica does not necessarily see
a transaction on another replica. In contrast to \cite{seebelieve} and abstract
exections~\cite{ev_transactions,framework-concur},  our abstract
states  capture the  client-observable history
which is not total. 

Nagar and Jagannathan \cite{sureshConcur} proposed a fine-grained 
interleaving operational semantics on abstract executions, and provide
robustness results for client programs using 
a prototype model-checking tool. 
They do this by converting abstract executions to
dependency graphs and checking the violation of robustness on the
dependency graphs.  We have two concerns with this approach. First, despite 
assuming atomic visibility of transactions, they present a fine-grained
semantics at the level of the individual transactional operations
rather than whole transactions,  in order to capture {\em eventual
  consistency}~\cite{ev_transactions}.  In contrast, our semantics is coarse-grained in that the  interleaving is at the level of whole
transactions, and we instead capture \emph{read atomic}~\cite{ramp}, a variant of \emph{eventual 
consistency}~\cite{} for atomic transactions. 
Second, all the literature that performs client analysis
on abstract executions
\cite{giovanni_concur16,SIanalysis,psi-chopping,laws,sureshConcur},
including the approach of Nagar and Jagannathan,  achieves this indirectly by over-approximating
the consistency-model specifications using dependency graphs. 
It is  unknown how to do this precisely \cite{laws}. 
In contrast, we prove robustness results directly by
analysing the structure of kv-stores, without over-approximation. 
We also give precise reasoning about the mutual exclusion of locks,
which we believe will be difficult to prove using abstract executions.


\label{subsec:cm_examples}

Kaki \etal~\citet{alonetogether} proposed an operational
semantics for SQL transactions over an abstract, centralised,
single-version 
store, with consistency models given by the standard ANSI/SQL
isolation levels \cite{si}. They develop a program logic and prototype
tool for reasoning about client programs, and so can capture invariant
properties of the state. They can express \( \SI \), 
but they do not  capture the weaker
consistency models such as \( \PSI \)
which is an important consistency model for distributed databases.
Kaki \etal have explored \(\PSI\) in
follow-on work~\citet{kakietal,oopsla18} but this work is declarative
not operational. 

Finally,  Koskinen and Parkinson
\citet{push-pull} proposed a log-based semantics for verifying
implementations that satisfy serialisability, based not only on
kv-stores but also on other ADTs. Their work comprises a centralised
global log and partial client-local logs, similar to
our kv-stores and views. Their model focuses on serialisability.
There is no evidence that it can be easily extended to tackle
weaker consistency models.  
