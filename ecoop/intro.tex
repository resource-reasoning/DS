\section{Introduction}
\label{sec:intro}

Transactions are the \emph{de facto} synchronisation mechanism in
modern distributed databases. To achieve scalability and performance,
distributed databases often use weak transactional consistency
guarantees known as \emph{consistency models}.  Many consistency
models were originally invented by engineers using (some quite
informal) definitions specific to particular real-world reference
implementations
\cite{gdur,ramp,CORFU,tango,si,distrsi,clocksi,redblue,rola,cops,PSI-RA,NMSI,PSI,wren}.
More recently, unified definitions of consistency models have been
defined independently of particular implementations, either
declaratively using execution graphs \cite{adya,ev_transactions} or
operationally using state-based semantics or graph-based \cite{seebelieve,alonetogether,sureshConcur}.

A key challenge is to define a unified semantics for consistency
models which provides a \emph{mid-point}, an abtract interface, with which to 
verify distributed implementations \emph{and} analyse the
behaviour of client programs with respect to a particular consistency model. Current
work tends to be suitable  either for verifying implementations or for reasoning
about clients. As far as we are aware, there has been no
previous work on distributed transactions that does both. 
We introduce an operational semantics for directly describing the
client-observable behaviour of atomic transactions 
updating distributed key-value stores (kv-stores).
With our semantics, we can verify distributed implementations and prove
invariant properties of client programs.  

The declarative approach for defining 
consistency models using execution graphs has been substantially
studied \cite{adya,ev_transactions,framework-concur,SIanalysis,laws}. 
In such graphs,  nodes are atomic transactions and edges describe the
known dependencies between transactions.
They capture different consistency models using  a two-step procedure:
\begin{enumerate*}
\item construct \emph{candidate executions} of the whole program comprising
transactions in which reads may contain arbitrary values; and 
\item apply a number of \emph{axioms} on such executions to rule out invalid executions. 
Such axioms may state, for example, that every read is
validated by a write that has written the read value. 
\end{enumerate*}
The most well-known execution graphs are dependency graphs \cite{adya} and abstract
executions \cite{ev_transactions,framework-concur}. 
Dependency graphs tend to be used to analyse  client programs: for
example, 
Fekete et al. \citet{fekete-tods} derived 
a static analysis checker for a particular weak consistency model called
snapshot isolation;  Bernardi and Gotsman \citet{giovanni_concur16}
developed a static analysis checker for several weak consistency
models assuming the so-called snapshot property\footnote{The \emph{snapshot property}, 
also known as \emph{atomic visibility}, states that
the client observes that a transaction reads from an atomic snapshot
of the database and commits atomically;  intermediate states are not observable to other clients, 
even if the underlying distributed protocol has a more fine-grained behaviour.}; and 
Beillahi et al. \citep{snapshot-isolation-robust-tool} developed a tool based on Lipton's reduction \cite{Lipton-reduction}
for checking robustness properties against snapshot isolation: 
that is, a  particular program (or set of programs) behaves as if the consistency model is strong consistency. 
Abstract executions, on the other hand, tend to be used to verify  implementation protocols: for example,
abstract executions are the  standard by which many system engineers
demonstrate that their protocols satisfies  certain
consistency models \cite{cops,NMSI,PSI}
Execution graphs provide little information about how the 
state evolves throughout the execution of a program.
They are therefore unsuitable for invariant-based program analysis 
which requires reasoning about the step-wise execution of a program. 

The operational approach has been much  less studied
\cite{seebelieve,alonetogether,sureshConcur}. Crooks et
al. \citet{seebelieve} provided a \emph{state-based} trace semantics
over a global centralised store, where an execution step involves
analysing the totally-ordered history of states and the read-write set
of the transaction.  They capture a wide range of consistency model,
and demonstrate the equivalence of several implementation-specific
definitions.  However, their approach does not lend itself to
analysing client programs, since observations made by clients require
the total order in which transactions commit which the clients do not
have. Kaki et al. \citet{alonetogether} proposed an operational
semantics for SQL transaction programs over an abstract centralised
store with the consistency models given by the standard ANSI/SQL
isolation levels \cite{si}. They develop a program logic and prototype
tool for reasoning about client programs, and so can capture invariant
properties of the state. They can express snapshot isolation
\cite{si}, but the consensus is that they cannot capture the weaker
consistency models such as parallel snapshot isolation \cite{PSI}
which is an important consistency model for distributed databases.

In constrast, Nagar and Jagannathan \citet{sureshConcur} introduced a 
\emph{graph-based} operational semantics for abstract execution graphs,
and prove the robustness of client programs using model checking. Each
execution step adds a new transaction node to the graph by first
constructing candidate steps and then applying axioms to rule out
invalid candidate steps.  They focus on consistency models with the
snapshot property, but confusingly work with the fine-grained
interleaving of primitive operations.  This results in an unnecessary
explosion of the space of traces obtained by the program.  This
approach is unlikely to be suitable for invariant-based analysis
associated with the state.  Finally, it is worth mentioning that
Koskinen and Parkinson \citet{push-pull} provided a \emph{log-based}
abstract operational semantics for verifying several implementations
of serilisability.  This work does not provide a unified semantics for
defining weaker consistency models.
In \cref{sec:related}, we provide a detailed discussion on \cite{seebelieve,sureshConcur}.

We introduce an interleaving operational semantics for
describing the client-observable behaviour of atomic transactions on
distributed key-value stores (\cref{sec:model}). Our semantics is
based on a notion of abstract state which 
comprises a \emph{centralised key-value store} (kv-store) with
\emph{multi-versioning}, which is {global} in the sense that it
records all the versions of a key, and
\emph{client views}, which are {partial} in the sense that  clients see only a subset of the
versions. Our client views are partly inspired by the views in the specific C11
operational semantics in~\cite{promises}. An execution step depends
simply on the abstract state, the read-write set of the atomic transaction, and an \emph{execution test} which
determines if a client with a given view is allowed to do the
transaction commit. Different execution tests give rise to different consistency models, 
which we show to be equivalent to well-known
declarative definitions of consistency model on abstract executions 
(reported here and proved in \cite{shale-phd}) and hence dependency graphs~\cite{laws}. 
Our execution tests are analogous to the commit tests used in \cite{seebelieve},
except that they require an analysis of the whole trace whereas we require the current abstract state. 

We make the assumption that our transactions satisfy the \emph{snapshot property}.
This assumption is common  in distributed databases: for example, with
on-line shopping application, a client only sees one snapshot of the database and
only has knowledge that their transaction has successfully committed.
We also make the assumption that our transactions satisfy the \emph{last-write-wins} resolution policy,
a widely-used policy in many real-world distributed kv-stores. 
This means that when a transaction observes several updates to a key,
the atomic snapshot contains the value written by the last update.
Our execution tests  uniformly capture  many well-known consistency
models (\cref{sec:cm}) including 
\emph{causal consistency} \((\CC)\) \citep{ev_transactions,cops,causal-def}, 
\emph{parallel snapshot isolation} \( (\PSI) \) \citep{NMSI,PSI},
\emph{snapshot isolation} \((\SI)\) \citep{si} 
and \emph{serialisability} \((\SER)\) \citep{Papadimitriou-ser}.
%Given these assumptions, we do not know of a consistency model that we cannot represent. 

Using our operational semantics, we demonstrate that we can verify
that database protocols satisfy particular consistency models and
prove invariant properties of client programs with respect to such
consistency models (\cref{sec:applications}).
We establish the correctness of two database
protocols: the COPS protocol for  fully replicated kv-stores \cite{cops} 
which satisfies causal consistency (reported in \cref{sec:verify-impl}
and detail in \cite{shale-phd}); 
and the Clock-SI protocol for partitioned kv-stores \cite{clocksi} 
which satisfies snapshot isolation (given in \cite{shale-phd}). 
We prove invariant properties of client programs calling
transactional libraries (\cref{sec:invariant-client-programs}): the robustness of the single-counter library
against PSI;  the robustness of the multi-counter library and the
well-known banking library \cite{bank-example-wsi} against SI; and the
correctness of a program pattern using locks against PSI. 
We believe our robustness results are the first to take into account client
sessions: with sessions, we show that multiple counters {\em are not} robust against \(\PSI\);
interestingly, without sessions, \citet{giovanni_concur16} show that multiple counters \emph{are}
robust against \(\PSI\) using static-analysis techniques which are
known not to be applicable to sessions.  
These results indicate that  our operational semantics provides an interesting \emph{mid-point},
that is, an abtract interface,
between distributed databases and clients.
This was an important goal for us, resonating with recent work
that does just this for standard shared-memory concurrency \cite{tada,cap,iris,fcsl}. 
