\paragraph{Update atomic  (\(\UA\))}
This model disallows concurrent transactions writing to the same key,
a property known as \emph{write-conflict freedom}:  
when two transactions, \( \txid \) and \( \txid'\) write to the same key, one transaction must see the version 
written by the other.
Note that, in distributed systems, 
the resolution policy only determines the commit order of these two transitions 
but provide no information about the starts.
This mean these two transactions may concurrently write to the same key,
depicted in \cref{fig:example-of-concurrent-write}.
Write-conflict freedom is enforced by \(\CanCommit[\UA]\) 
which allows a client to write to key \(\key\) 
only if its view includes all versions of \(\key\).
For example, prior to committing, the view of \( \txid' \) includes versions written by \( \txid \),
hence \( \txid' \) must start after \( \txid \).
Similarly, prior to committing, the view of \( \txid'' \)  includes versions written by \( \txid \) and \( \txid' \), 
which ensures \( \txid'' \) starts after \( \txid \) and \( \txid' \).
In other words, its view is closed with respect to the \(\Inv(\WW)(\key)\) relation 
for all keys \(\key\) written in the fingerprint \(\fp\).
Recall that a view must include the initial version (left-most version) for each key.
\( \UA \) prevents the kv-store in \cref{fig:ua-disallowed}, known as \emph{lost update anomaly},
as \(\txid\) and \(\txid'\) read initial version of \(\key\)
and update it to \( \val_1 \) and \( \val_2 \) respectively.
As client views must include the initial versions, 
once \(\txid\) commits a new version \(\ver\) with value \(\val_1\) to \(\key\)
(the second version in \cref{fig:ua-disallowed}), 
then \(\txid'\) must include \(\ver\) in its view as 
there is a \(\WW\) edge from the initial version to \(\ver\). 
As such, when \(\txid'\) subsequently updates \(\key\), 
it must read from \(\ver\), and not the initial version as depicted in \cref{fig:ua-disallowed}.

\begin{figure}

\begin{tabularx}{\textwidth}{@{}c@{}|@{}X@{}}
\begin{subfigure}{0.394\textwidth}
\centering
\begin{tikzpicture}
\KVMapping{x}{\key}{
    /\val_{0}/\txid_0/\Set{\txid,\boldsymbol{\txid'}}
    , /\val_{1}/\txid/\emptyset
    , /\val_{2}/\boldsymbol{\txid'}/\emptyset
};
\end{tikzpicture}
\caption{Lost update anomaly, disallowed by \(\UA\).  }
\label{fig:ua-disallowed}
\end{subfigure}

& 

\begin{subfigure}{0.594\textwidth}%
\centering
\begin{tikzpicture}%
\KVMapping{x}{\key_1}{
    /\val_0/\txid_0/\emptyset
    , /\val_1/\txid^{1}_{\cl}/\emptyset
    , /\val_2/\txid^{1}_{\cl'}/\Set{\boldsymbol{\txid}}
};
\KVMapping[x]{u}{\key_2}{
    /\val_0/\txid_0/\Set{\boldsymbol{\txid}}
    , /\val_3/\txid^{1}_{\cl}/\emptyset
};
\end{tikzpicture}%
\caption{Allowed by \(\CC\) and \( \UA \) and henceforth \( \WSI\),
but disallowed by \( \PSI \)}
\label{fig:cc-ua-allowed-but-psi}
\end{subfigure}%

\\ \hline

\end{tabularx}

\vspace*{1em}

\begin{subfigure}{\textwidth}
\centering
\begin{tikzpicture}

\node at (0,0) (a) {\(\txid\)};
\node at (4,-1.5) (b) {\(\txid'\)};
\node[anchor=south] at (6,-3) (c) {\(\txid''\)};
%\node at (4.5,0) (c) {\(\txid_1\)};
%\node at (10,-2) (d) {\(\txid_3\)};

\coordinate (a1) at ($(a) + (3,0)$);
\coordinate (b1) at ($(b) + (3,0)$);
\coordinate (c1) at ($(c) + (3,0)$);
%\coordinate (d1) at ($(d) + (3,0)$);

\draw[|-,densely dashed] (a) -- ($(a)!0.5!(a1)$);
\draw[-|] ($(a)!0.5!(a1)$) -- (a1);
\draw[|-,densely dashed] (b) -- ($(b)!0.5!(b1)$);
\draw[-|] ($(b)!0.5!(b1)$) -- (b1);
\draw[|-,densely dashed] (c) -- ($(c)!0.5!(c1)$);
\draw[-|] ($(c)!0.5!(c1)$) -- (c1);

\path[->, thick, densely dashed] (a1) edge node[left, text opacity=1] {$\texttt{view ind.}$} ($(b)+(0.3,0.05)$);
\path[->, thick, densely dashed] (b1) edge node[left, text opacity=1] {$\texttt{view ind.}$} ($(c)+(0.3,0.05)$);
\path[->, thick, densely dashed] (a1) edge[bend right=60] node[left, text opacity=1] {$\texttt{view ind.}$} ($(c)+(0.3,0.05)$);
\path[->, thick] (a1) edge node[above, text opacity=1] {$\WW$} (b1);
\path[->, thick] (b1) edge node[above, text opacity=1] {$\WW$} (c1);
\path[->, thick] (a1) edge[bend left=50] node[above, text opacity=1] {$\WW$} (c1);

\end{tikzpicture}

\caption{An example of concurrent writes with dashed line being able to stretched}
\label{fig:example-of-concurrent-write}

\end{subfigure}


\hrulefill


\caption{Anomalies for \(\UA \) and \( \PSI \)}
\end{figure}

