We were surprise to find a new interesting consistency model using our kv-stores.
This model, which we call \emph{weak snapshot isolation} (\( \WSI \)), is defined by combining \(\CP\) and \(\UA\).
%which we refer to as \emph{weak snapshot isolation} (\(\WSI\)). 
Under \(\WSI\), a client view must be close with respect to relation \( \rel[\WSI] = \rel[\CP] \cup \rel[\UA]\),
in contrast to \( \SI \)-relation \(  \rel[\SI] = \rel[\CP] \cup \rel[\UA] \cup \WW[\kvs] ; \RW[\kvs]\).

\(\WSI\) is stronger than \(\CP\) and \(\UA\) by definition, 
it therefore forbids all the anomalies forbidden by these consistency models, 
for example the long fork (\cref{fig:cp-disallowed}) and the lost update (\cref{fig:ua-disallowed}). 
Moreover, \(\WSI\) is strictly weaker than \(\SI\). 
As such, \(\WSI\) allows all \(\SI\) anomalies such as the write skew (\cref{fig:ser-disallowed}),
but allows behaviour not allowed under \(\SI\) such as that in \cref{fig:si-disallowed}.
We can construct a \((\CP \cap \UA)\)-trace terminating in kv-store \(\kvs\) 
by executing transactions \(\txid_{1}, \txid_{2}, \txid_{3}\) and \(\txid_{4}\) in this order. 
In particular, \(\txid_{4}\) is executed using 
\(\vi = \Mapping{\key_{1} -> \Set{0} | \key_{2} -> \Set{0,1} }\) that is allowed by \( \WSI \)
given that \(\WW[\kvs] ; \RW[\kvs]\ \not\subseteq \rel[\WSI] \).
However, the same trace is not a valid \(\SI\)-trace as explained in \cref{sec:et-si}.

To justify this consistency model in full, it would be useful to explore its implementations. 
Here we focus on the possible benefits of implementing \(\WSI\):
as \(\WSI\) is a weaker consistency model than \(\SI\),
we believe that \(\WSI\) implementations would outperform known \(\SI\) implementations.
Nevertheless, the two consistency models are very similar in that 
many applications that are correct under \(\SI\) are also correct under \(\WSI\). 
We give an example of such an application in \cref{sec:robust-wsi}.

%Under \(\SI\) transaction \(\txid_{4}\) cannot be executed using \(\vi\): 
%\(\txid_{4}\) reads the version of \(\key_2\) written by \(\txid_3\), 
%meaning that \(\vi\) must include the version written by \(\txid_{3}\). 
%Since \((\txid_{2},\txid_{3}) \in \RW[\kvs] \) and \((\txid_{1} ,\txid_2) \in \WW[\kvs]\), 
%then \(\vi\) should contain the version of \(\key_{1}\) written by \(\txid_{1}\), 
%contradicting the fact that \(\txid_{4}\) reads the initial version of \(\key_1\).
