\paragraph{Snapshot isolation} (\(\SI\))
This model can be informally described as:
\begin{enumerate}
\item when a transaction \( \txid \) observes the effect of another transaction \( \txid' \),
then \( \txid \) must observe transactions that commits before \( \txid' \);
\item when two transactions, \( \txid \) and \( \txid'\) write to the same key, one must see the effects by another.
\end{enumerate}
When the total order in which transactions commit is known,  
\emph{snapshot isolation} (\(\SI\)) can be defined compositionally from \(\CP\) and \(\UA\). 
However, as with \(\PSI\), our semantics does not have the total order.
%We cannot define \(\CanCommit[\SI]\) as the conjunction of their associated \(\CanCommit\) predicates. 
%Rather, we define \(\CanCommit[\SI]\) as the closure with respect to \(\rel[\SI]\), 
%which, first, includes \(\rel[\CP] \cup \rel[\UA]\).
We cannot directly define \( \rel[\SI] \) as \(\rel[\CP] \cup \rel[\UA]\).
Additionally, we include \(\WW[\kvs];\RW[\kvs]\) in \(\rel[\SI]\). 
Because when the centralised \(\CP\) implementation (discussed in \cref{sec:et-cp}) 
is strengthened with \emph{write-conflict freedom}, 
then a write-write dependency between two transactions \(\txid\) and \(\txid'\) 
does not only mandate that \(\txid\) commits before \(\txid'\) commits but also before \(\txid'\) starts. 
Consequently, if \((\txid, \txid'') \in \WW[\kvs] ;\RW[\kvs]\), then \(\txid\) must commit before \(\txid''\) commit (\cref{fig:commit-before-ww-rw}).
Therefore in \cref{fig:execution-tests}, \( \CanCommit[\SI] \) is defined as the closure on relation 
\( \rel[\SI] \FuncDef \rel[\UA] \cup \rel[\CP] \cup (\WW[\kvs]; \RW[\kvs]) \).

\begin{figure}

\begin{subfigure}{\textwidth}
\centering
\begin{tikzpicture}%
\KVMapping{x}{\key_1}{
    /\val_0/\txid_0/\Set{\boldsymbol{\txid_4}}
    , /\val_1/\txid_1/\emptyset
    , /\val_2/\txid_2/\emptyset
};
\KVMapping[x]{y}{\key_2}{
    /\val_0/\txid_0/\Set{\txid_2}
    , /\val_3/\txid_3/\Set{\boldsymbol{\txid_4}}
    , /\val_4/\boldsymbol{\txid_4}/\emptyset
};
\end{tikzpicture}


\caption{Allowed by \( \UA \) and \( \CP \) but disallowed by \(\SI\)}%
\label{fig:si-disallowed}%
\end{subfigure}%

\hrulefill

\vspace*{1em}

\begin{subfigure}{\textwidth}
\centering
\begin{tikzpicture}%
\node at (0,0) (a) {\(\txid\)};
\node at (4,-1) (b) {\(\txid'\)};
\node at (2,-2) (c) {\(\txid''\)};

\coordinate (a1) at ($(a) + (3,0)$);
\coordinate (b1) at ($(b) + (5,0)$);
\coordinate (c1) at ($(c) + (4,0)$);

\draw[|-|] (a) -- (a1);
\draw[|-|] (b) -- (b1);
\draw[|-,densely dashed] (c) -- ($(c)!0.5!(c1)$);
\draw[-|] ($(c)!0.5!(c1)$) -- (c1);

\path[->, thick] (a1) edge node[left,text opacity=1] {\(\texttt{view ind.}\)} ($(b)+(0.3,0.05)$);
\path[->, thick] (a1) edge[bend left=40] node[right,text opacity=1] {\(\WW\)} (b1);
\path[->, thick] (a1) edge[bend left=20] node[right,text opacity=1] {\(\texttt{commit before}\)} (c1);
\path[->, thick] ($(b)+(0.3,-0.05)$) edge node[left,text opacity=1] {\(\RW\)} (c1);
\end{tikzpicture}%


\caption{Commit before relation: \( \WW ; \RW\)}%
\label{fig:commit-before-ww-rw}%
\end{subfigure}%

\hrulefill

\caption{Anomaly for \(\SI\) and extra commit before relation for \( \SI \)}%
\end{figure}%

Observe that the kv-store in \cref{fig:si-disallowed} shows an example kv-store 
that satisfies \(\CanCommit[\UA] \land \CanCommit[\CP]\), but not \(\CanCommit[\SI]\).
In \cref{fig:si-disallowed}, transaction \( \txid_4 \) must be the last transaction.
It must include all versions of \( \key_2 \) because of write-conflict freedom.
However, since \( \ToEdge{ \txid_1 | \WW[\kvs] -> \txid_2 | \RW[\kvs] -> \txid_3 } \),
transaction \( \txid_4 \) should read the second version of \( \key_1 \) carrying value \( \val_1 \), 
contradicting the fact that \( \txid_4 \) read the initial version of \( \key_1 \).

