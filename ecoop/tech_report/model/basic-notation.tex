In the definitions, 
the notation \(\MataTypeFont{A} \ni \MataTypeFont{a}\) and \(\MataTypeFont{a} \in \MataTypeFont{A}\) 
denote that the elements of \(\MataTypeFont{A}\) are ranged over
by \(\MataTypeFont{a}\) and its variants such as \(\MataTypeFont{a}', \MataTypeFont{a}_0, \cdots\).
The notation \( \List{\MataTypeFont{A}} \) denotes the set of lists of \( \MataTypeFont{A} \),
and \( \List{\MataTypeFont{a}_0, \cdots, \MataTypeFont{a}_n}\) denotes a list.
Given \( \MataTypeFont{l} \in \List{\MataTypeFont{A}}\), the notation
\( \MataTypeFont{l}\Proj{\idx} \), denoting the \Th{(\idx + 1)} element from the list, is defined by:
\[
    \MataTypeFont{l}\Proj{\idx} \FuncDef
    \begin{cases}
        \MataTypeFont{a}_i & \text{if} \ \MataTypeFont{l} = \List{\MataTypeFont{a}_0, \cdots, \MataTypeFont{a}_i, \cdots}  ,
        \\ \Undef & \ow .
    \end{cases}
\]
Note that index starts from 0.
%Similarly, given a trace \( \MataTypeFont{t} \) of the form of \( \MataTypeFont{s}_0 \to \cdots \to \MataTypeFont{s}_n \),
%the notation \( \MataTypeFont{t}\Proj{\idx} \) denotes the \Th{(\idx + 1)} state defined by
%\[
    %\MataTypeFont{t}\Proj{\idx} \FuncDef
    %\begin{cases}
        %\MataTypeFont{s}_i & \text{if} \ \MataTypeFont{t} = \MataTypeFont{s}_0 \to \cdots \to \MataTypeFont{s}_i \to \cdots ,
        %\\ \Undef & \ow .
    %\end{cases}
%\]
Given two lists \( \MataTypeFont{l},\MataTypeFont{l}' \in \List{\MataTypeFont{A}} \),
the notation \(\MataTypeFont{l} \ListConcat \MataTypeFont{l}' \) denotes the concatenation of the two lists.
%\(\MataTypeFont{l}\) and \(\MataTypeFont{l}'\) defined by:
%\[
    %\List{\MataTypeFont{a}_0, \cdots, \MataTypeFont{a}_n} \ListConcat 
    %\List{\MataTypeFont{a}'_0, \cdots, \MataTypeFont{a}'_m}
    %\FuncDef
    %\List{\MataTypeFont{a}_0, \cdots, \MataTypeFont{a}_n,
        %\MataTypeFont{a}'_0, \cdots, \MataTypeFont{a}'_m } ,
%\]
The notation \( \Abs{ \MataTypeFont{l} }\) denotes the size of the list.
%and the notation \( \MataTypeFont{l}\UpdateFunc{ \idx -> \MataTypeFont{a} }\)
The notation \( \MataTypeFont{l}\UpdateFunc{ \idx -> \MataTypeFont{a} } \) denotes the update of \Th{(\idx + 1)} component to \MataTypeFont{a}.
%denotes the update of the list defined by:
%\[
    %\MataTypeFont{l}\UpdateFunc{ \idx -> \MataTypeFont{a} }
    %\FuncDef \begin{cases}
    %\List{\MataTypeFont{a}_0, \cdots, \MataTypeFont{a}_{\idx-1}
        %, \MataTypeFont{a}, \MataTypeFont{a}_{\idx+1}, \cdots, \MataTypeFont{a}_n}
        %& \If \MataTypeFont{l} = \List{\MataTypeFont{a}_0, \cdots, \MataTypeFont{a}_n} \land 0 \leq \idx \leq n ,
        %\\ \Undef & \ow.
    %\end{cases}
%\]
For a tuple \( \MataTypeFont{p} \), the notation \( \MataTypeFont{p}\Proj{\idx} \) (index \( \idx \) starts from 0)
denotes the \Th{(\idx + 1)} component
and \( \MataTypeFont{p}\UpdateFunc{ \idx -> \MataTypeFont{a} } \) denotes the update of \Th{(\idx + 1)} component to \MataTypeFont{a}.
The notation \( \MataTypeFont{A} \ToTFunc \MataTypeFont{B} \),
\( \MataTypeFont{A} \ToPFunc \MataTypeFont{B} \) and
\( \MataTypeFont{A} \ToPFFunc \MataTypeFont{B} \)
denotes the set of total, partial and partial finite functions 
from \( \MataTypeFont{A} \) to \( \MataTypeFont{B} \) respectively.
For a function \( \MataTypeFont{f} \in \MataTypeFont{A} \ToTFunc \MataTypeFont{B} \), 
(similarly for \( \MataTypeFont{A} \ToPFunc \MataTypeFont{B} \) and \(\MataTypeFont{A} \ToPFFunc \MataTypeFont{B} \) ),
\( \MataTypeFont{a} \in \MataTypeFont{A} \) and \( \MataTypeFont{b} \in \MataTypeFont{B} \),
the notation \( \MataTypeFont{f}\UpdateFunc{ \MataTypeFont{a} -> \MataTypeFont{b} }\) denotes
the update of the function defined by:
\[
    \MataTypeFont{f}\UpdateFunc{ \MataTypeFont{a} -> \MataTypeFont{b} }(\MataTypeFont{a}')
    \FuncDef \begin{cases}
        \MataTypeFont{b} & \text{if} \  \MataTypeFont{a}' = \MataTypeFont{a} ,
        \\  \MataTypeFont{f}(\MataTypeFont{a}') & \ow .
    \end{cases}
\]
