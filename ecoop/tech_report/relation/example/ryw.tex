The execution test \(\et[\RYW]\) is sound with respect to the axiomatic definition
\(\visaxioms[\RYW] = \Set{\lambda \aexec \ldotp \SO_{\aexec} }\) \cite{repldatatypes}.
We pick the following invariant:
\[
    \aexecinv[\RYW](\aexec, \cl) \FuncDef
    \begin{multlined}[t]
    \left( \bigcup_{\Set{\txid[\cl](n) \in \aexec }} \Refl((\Inv(\SO)))(\txid[\cl](n)) \right) 
    \setminus \Set{\txidrd | \txidrd in \aexec \land \Forall{l | \key | \val} (l,\key,\val) \in \aexec(\txid) \implies l = \opR } .
    \end{multlined}
\]

\SOUNDLET{\RYW}{
    \txidsetrd = 
    \left( \bigcup_{\Set{\txid[\cl](n) \in \aexec }} \Refl((\Inv(\SO)))(\txid[\cl](n)) \right) 
    \cap \Set{\txidrd | \txidrd in \aexec \land \Forall{l | \key | \val} (l,\key,\val) \in \aexec(\txid) \implies l = \opR } .
}
\begin{enumerate}
\Case{\(\Forall{\visaxiom \in \visaxioms }
            \Inv(\visaxiom(\aexec'))(\txid) \subseteq \txidset \cup \txidsetrd \)}
    Suppose transactions \( \txid, \txid' \) such that \( \txid,\txid' \in \aexec \) and \( (\txid',\txid) \in \SO \).
    If \( \txid' \) is a read-only transaction, \( \txid' \in \txidsetrd \).
    Otherwise, \( \txid' \) has write, by the definition of \( \aexecinv[\RYW] \), 
    it follows that \( \txid' \in \aexecinv[\RYW](\aexec,\cl) \)
    and therefore \( \txid' \in \txidset \).
\Case{\(\aexecinv(\aexec',\cl) \subseteq \VisTrans(\XToK(\aexec'),\vi') \)}
    Because \( \ToET[\RYW]{\kvs | \vi | \fp | \kvs' | \vi' }\),
    it must be the case that
    \[
        \Forall{\key \in \Keys | \idx \in \Indexs } (\WtOf(\kvs'(\key,\idx)),\txid) \in \Refl(\SO) \implies \idx \in \vi'(\key) 
    \]
    and therefore
    \[
        \Forall{\txid'} \Exists{\key \in \Keys | \val \in \Values} \opW(\key,\val) \in \aexec'(\txid')
        \land (\txid',\txid) \in \SO \implies \txid' \in \VisTrans(\XToK(\aexec'),\vi') .
    \]
    Note that \( \bigcup_{\Set{\txid[\cl](n) \in \aexec }} \Refl((\Inv(\SO)))(\txid[\cl](n)) = \Refl((\Inv(\SO)))(\txid) \).
    Last, we have
    \begin{align*}
        \aexecinv(\aexec',\cl) & = 
            \begin{multlined}[t]
            \left( \bigcup_{\Set{\txid[\cl](n) \in \aexec }} \Inv(\SO)(\txid[\cl](n)) \right) 
            \setminus \Set{\txidrd | \txidrd \in \aexec 
                    \land \Forall{l | \key | \val} (l,\key,\val) \in \aexec(\txid) \implies l = \opR } 
            \end{multlined}
            \\ & = \begin{multlined}[t]
            \left( \Refl((\Inv(\SO)))(\txid) \right) 
            \setminus \Set{\txidrd | \txidrd \in \aexec 
                    \land \Forall{l | \key | \val} (l,\key,\val) \in \aexec(\txid) \implies l = \opR } 
            \end{multlined}
            \\ & \subseteq \VisTrans(\XToK(\aexec'),\vi') 
    \end{align*}
\end{enumerate}

\COMPLETELET{\RYW}
We construct the final view \( \vi'\) depending on whether \( \txid[\cl](n) \) is the last transaction for the client \( \cl \).
\begin{enumerate}
\Case{\( (\txid[\cl](n), \txid') \in \SO \) for \( \txid' \in \aexec \)}
    Let the transaction 
    \( \txid = \Min[\SO](\Set{ \txid' | (\txid[\cl](n), \txid') \in \SO \land \txid' \in \aexec' }) \).
    For this case, let view 
    \( \vi' = \GetView(\aexec, \Refl((\ARInv[\aexec]))(\txid[\cl](\idx)) \cap \VISInv[\aexec](\txid)) \).
    By \( \visaxioms[\RYW] \), it follows that, for any transaction \( \txid' \),
    if \( ( \txid',\txid[\cl](idx) ) \in \Refl(\SO) \), then
    \( \txid' \in \VISInv[\aexec](\txid)) \).
    Since \( \SO \in \AR \), we know that 
    \( \txid' \in \Refl((\ARInv[\aexec]))(\txid[\cl](\idx)) \cap \VISInv[\aexec](\txid)) \).
    Therefore, for any version \( \kvs'(\key,j)\) such that 
    \( ( \WtOf(\kvs'(\key,j)), \txid) \in \Refl(\SO) \),
    then \( j \in \vi'(\key)\).
\Case{\( \neg \left((\txid[\cl](n), \txid') \in \SO \right) \)}
    For this case, let 
    \( \vi' = \GetView(\aexec, \Refl((\ARInv[\aexec]))(\txid[\cl](\idx))) \) be the final view.
    It is easy to see that \( \vi' \) satisfies \( \RYW \). 
\end{enumerate}
