\begin{definition}[Kv-stores to dependency graphs]
\label{def:kv-store-to-dependency-graph}
Given a kv-store \( \kvs \in \KVSs \), 
the dependency graph induced by the kv-store, written \( \KToD(\kvs) \), is defined by:
\[
\KToD(\kvs) \FuncDef (\KToT(\kvs),\SO,\WR[\kvs],\WW[\kvs],\RW[\kvs])
\]
where \( \KToT(\kvs) \) is defined by
\[
\KToT(\kvs) = \lambda \txid \in \kvs \ldotp 
        \begin{multlined}[t]
        \Set{\opW(\key,\val) | \Exists{\idx \in \Indexs} \kvs(\key,\idx) = (\val,\txid,\stub) } \uplus \\ 
        \Set{\opR(\key,\val) | \Exists{\idx \in \Indexs | \txidset \subseteq \TxIDs} 
                        \kvs(\key,\idx) = (\val,\stub,\txidset) \land \txid \in \txidset }.
        \end{multlined}
\]
\end{definition}

Given a kv-store \( \kvs \), the dependency graph induced by the kv-store is given by \( \KToD(\kvs) \).
The auxiliary function \( \KToT(\kvs) \) converts the kv-store \( \kvs\)
to the set of nodes in the dependency graph.
The three dependency relations on the dependency graph 
are exactly the same dependency relations on kv-store defined in \cref{def:dependency-kv-store}.
It is easy to see that 
the dependency graph induced by the kv-store \( \KToD(\kvs) \) is well-formed.
More detail is given in \cref{prop:ktod-well-defined} on page \pageref{sec:proof-kv-to-dep-graph}.


\begin{toappendix}
\label{sec:proof-kv-to-dep-graph}
\begin{proposition}[Well-defined \(\KToD\)]
\label{prop:ktod-well-defined}
Given a kv-store \( \kvs \in \KVSs \), 
the dependency graph induced by the kv-store \( \KToD(\kvs) \) is well-formed,
that is, \( \KToD(\kvs) \in \DependencyGraphs\).
\end{proposition}
\begin{proof}
First we prove the dependency graph \( \KToT(\kvs) \) match the type constraint of the nodes.
Consider a transaction \( \txid \in \kvs \).
\begin{enumerate}
\Case{\( \txid = \txidinit \)}
    By the definition of versions \( \Versions \), \( \txidinit \) cannot in any reader set;
    By well-formedness of \( \kvs \), the initial transaction wrote the first version for each key (\cref{equ:kvs-wf-init-version});
    and it only wrote those versions (snapshot property in \cref{equ:kvs-wf-txid-snapshot-property}).
    Therefore \( \KToT(\kvs)(\txidinit) = \Set{\opW(\key,\valinit) | \key \in \Keys \land \valinit \in \InitVal(\key) } \).
\Case{\( \txid \neq \txidinit \)}
    By snapshot property of \( \kvs \) (\cref{equ:kvs-wf-txid-snapshot-property}), 
    it is easy to see that \( \KToT(\kvs)(\txid) \in \Fingerprints\).
\end{enumerate}
Let \( \WR = \WR[\kvs], \WW = \WW[\kvs] , \RW= \RW[\kvs]\).
Now we prove that the relations are well-formed.
\begin{enumerate}
\Case{write-read dependency \(\WR\)} 
    \begin{enumerate}
    \item Suppose a transaction \( \txid \) such that \( \opR(\key,\stub) \in \KToT(\kvs)(\txid) \).
        There exists a version \( \kvs(\key,\idx) \) for a key \( \key \) and an index \( \idx \), 
        such that \( \txid \in \RsOf(\kvs(\key,\idx))\).
        We have \( (\WtOf(\kvs(\key,\idx)),\txid) \in \WR \) which implies \cref{equ:dgraph-wr-def}.
    \item Suppose transactions \( \txid, \txid'\) such that \( (\txid, \txid') \in \WR \).
        By definition of \( \WR \), there exists a version \( \kvs(\key,\idx) \) 
        for a key \( \key \) and an index \( \idx \), such that \( \txid = \WtOf(\kvs(\key,\idx))\)
        and \( \txid' \in \RsOf(\kvs(\key,\idx))\), which immediately implies \cref{equ:dgraph-wr-minimum}.
    \item Suppose transactions \( \txid, \txid'\) such that \( (\txid, \txid') \in \WR \).
        By definition of \( \WR \), there exists a version \( \kvs(\key,\idx) \) 
        for a key \( \key \) and an index \( \idx \), such that \( \txid = \WtOf(\kvs(\key,\idx))\)
        and \( \txid' \in \RsOf(\kvs(\key,\idx))\).
        Because \(\kvs\) is well-formed (\cref{equ:kvs-wf-so-wr}), it must be the case that \((\txid,\txid') \notin \SO \),
        which implies \cref{equ:dgraph-wr-so}.
    \item Suppose transactions \( \txid, \txid', \txid''\) such that \( (\txid', \txid), (\txid'', \txid)\in \WR \).
        There exist two versions \(\kvs(\key,\idx),\kvs(\key,\idx') \)
        for a key \( \key \) and indexes \( \idx,\idx' \), such that \( \txid \in \RsOf(\kvs(\key,\idx))\)
        and \( \txid \in \RsOf(\kvs(\key,\idx'))\).
        Because \(\kvs\) is well-formed (\cref{equ:kvs-wf-txid-snapshot-property}), 
        it must be the case that \( \idx = \idx' \) which means that \( \txid' = \txid'' = \WtOf(\kvs(\key,\idx)) \)
        and thus \cref{equ:dgraph-wr-unique}.
    \end{enumerate}
\Case{write-write dependency \(\WW\)}
    \begin{enumerate}
    \item Suppose transactions \( \txid, \txid' \) such that 
        \( \opW(\key,\stub) \in \KToT(\kvs)(\txid) \) and \( \opW(\key,\stub) \in \KToT(\kvs)(\txid') \).
        There exist two versions \(\kvs(\key,\idx),\kvs(\key,\idx') \)
        for a key \( \key \) and indexes \( \idx,\idx' \), such that 
        \( \txid = \WtOf(\kvs(\key,\idx))\) and \( \txid' = \WtOf(\kvs(\key,\idx'))\).
        If \( \idx = \idx'\) then \( \txid = \txid' \);
        if \( \idx < \idx' \) or \( \idx > \idx' \),
        then \( (\txid,\txid') \in \WW \) or \( (\txid',\txid) \in \WW \) respectively.
        Thus we prove \cref{equ:dgraph-ww-def}.
    \item Suppose transactions \( \txid, \txid'\) such that \( (\txid, \txid') \in \WW \).
        There exist two versions \(\kvs(\key,\idx),\kvs(\key,\idx') \)
        for a key \( \key \) and indexes \( \idx,\idx' \)
        \( \txid = \WtOf(\kvs(\key,\idx))\) and \( \txid' = \WtOf(\kvs(\key,\idx'))\),
        which immediately implies \cref{equ:dgraph-ww-minimum}.
    \item Suppose transactions \( \txid, \txid'\) such that \( (\txid, \txid') \in \WW \).
        We prove \( \txid' \neq \txidinit \) by contradiction.
        Assume \( \txid' = \txidinit \).
        There exist a version \(\kvs(\key,\idx)\) for a key \( \key \) and an index \( \idx > 0 \),
        such that \( \txidinit = \WtOf(\kvs(\key,\idx))\).
        However, \( \txidinit = \WtOf(\kvs(\key,0)) \) by well-formedness;
        there are two versions for \( \key \) written by \( \txidinit \) 
        which contradicts \cref{equ:kvs-wf-txid-snapshot-property}.
        Therefore \cref{equ:dgraph-ww-init} holds.
    \item Suppose transactions \( \txid, \txid'\) such that \( (\txid, \txid') \in \WW \).
        There exist two versions \(\kvs(\key,\idx),\kvs(\key,\idx') \)
        for a key \( \key \) and indexes \( \idx,\idx' \), such that \( \idx < \idx' \),
        \( \txid = \WtOf(\kvs(\key,\idx))\) and \( \txid' = \WtOf(\kvs(\key,\idx'))\).
        Because \(\kvs\) is well-formed (\cref{equ:kvs-wf-so-ww}), 
        it must be the case that \((\txid',\txid) \notin \SO \),
        which implies \cref{equ:dgraph-ww-so}.
    \item Suppose transactions \( \txid, \txid'\) such that \( (\txid, \txid') \in \WW \).
        We prove \( \txid \neq \txid' \) by contradiction.
        Assume \( \txid = \txid' \).
        There exist two versions \(\kvs(\key,\idx),\kvs(\key,\idx') \)
        for a key \( \key \) and indexes \( \idx,\idx' \), such that
        \( \txid = \WtOf(\kvs(\key,\idx)) = \WtOf(\kvs(\key,\idx'))\);
        this contradicts \cref{equ:kvs-wf-txid-snapshot-property}.
        Thus \cref{equ:dgraph-ww-irreflexive} holds.
    \item Suppose transactions \( \txid, \txid', \txid''\) such that 
        \( (\txid, \txid'),(\txid', \txid'') \in \WW \).
        There exist versions \(\kvs(\key,\idx),\kvs(\key,\idx'),\kvs(\key,\idx'') \)
        for a key \( \key \) and indexes \( \idx,\idx' \), such that \( \idx < \idx' < \idx'' \),
        \( \txid = \WtOf(\kvs(\key,\idx))\), \( \txid' = \WtOf(\kvs(\key,\idx'))\),
        and \( \txid'' = \WtOf(\kvs(\key,\idx''))\).
        By definition of \( \WW \), it must be the case that \( (\txid, \txid'') \in \WW \),
        which implies \cref{equ:dgraph-ww-transitive}.
    \end{enumerate}
\Case{read-write anti-dependency \(\RW\)}
    Suppose transactions \( \txid, \txid'\) such that 
    \begin{Formulae}
    \begin{Formula}
    \txid \neq \txid' \land (\txid, \txid') \in \RW .
    \label{equ:kv-to-dgraph-rw-def}
    \end{Formula}
    \end{Formulae}
    \Cref{equ:kv-to-dgraph-rw-def} holds iff 
    there a transaction \( \txid'' \) and two versions \(\kvs(\key,\idx),\kvs(\key,\idx') \)  such that
    \( \idx < \idx' \), \( \txid \in \RsOf(\kvs(\key,\idx)) \) and \( \txid' = \WtOf(\kvs(\key,\idx')) \);
    this means that 
    \begin{Formulae}
    \begin{Formula}
    (\WtOf(\kvs(\key,\idx)),\txid) \in \WR \land (\WtOf(\kvs(\key,\idx)),\txid') \in \WW \land \txid \neq \txid' .
    \label{equ:kv-to-dgraph-rw-in-kv}
    \end{Formula}
    \end{Formulae}
    Since \cref{equ:kv-to-dgraph-rw-def} implies \cref{equ:kv-to-dgraph-rw-in-kv} and vice versa,
    we have the proof for \cref{equ:dgraph-rw}. \qedhere
\end{enumerate}
\end{proof}
%\begin{proofsketch}
%Given a kv-store \( \kvs \), it is easy to see that
%\[
%\KToT(\kvs) \in \begin{Bracketed} \begin{Bracketed} \TxIDZs \ToPFFunc \Fingerprints 
            %\end{Bracketed} \uplus \Mapping{\txidinit -> \Set{\opW(\key,\valinit) 
                        %| \key \in \Keys }} \end{Bracketed} .
%\]
%Let \( \dgraph = \KToD(\kvs)\).
%By the definition of \( \KToD\),
%we have \( \WR[\kvs] = \WR[\dgraph]\), \( \WW[\kvs] = \WW[\dgraph]\) 
%and \( \RW[\kvs] = \RW[\dgraph]\).
%By the definitions of dependency relations on kv-store in \cref{def:dependency-kv-store},
%it is easy to see that these relations on \( \dgraph \) satisfy 
%\cref{equ:dgraph-wr-minimum,equ:dgraph-wr-def,equ:dgraph-wr-so,equ:dgraph-wr-unique,equ:dgraph-ww-def,equ:dgraph-ww-minimum,equ:dgraph-ww-irreflexive,equ:dgraph-ww-transitive,equ:dgraph-ww-init,equ:dgraph-ww-so,equ:dgraph-rw} 
%in \cref{def:dgraph}.
%The full proof is in \cref{sec:proof-kv-to-dep-graph} on page \pageref{sec:proof-kv-to-dep-graph}.
%\end{proofsketch}
\end{toappendix}
