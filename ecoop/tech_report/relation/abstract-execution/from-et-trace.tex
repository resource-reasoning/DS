We now show how to construct an abstract execution from an \( \ET[\top] \)-trace, via \( \TraceToX(\ettrc) \).
Recall that \( \et[\TOP]\) is defined by \( \CanCommit[\TOP] \PredDef \true \) and  \( \ViewShift[\TOP] \PredDef \true \).
%Recall that, given an abstract execution \( \aexec \), we can recover the trace with the final state being \(\aexec \).

\begin{definition}[{\( \et[\TOP] \)}-traces to abstract executions]
\label{def:kvtrace-aexec}
Given an \( \et[\TOP] \) trace \(\ettrc\),
the set of abstract executions, written \(\TraceToX(\ettrc)\), is defined by:
\begin{align*}
    \TraceToX(\ToRed{\kvsinit | \vienvinit})  
            & \FuncDef \Tuple{\Mapping{\txidinit
                                -> \Set{(\key,\valinit) | \key \in \Keys \land  \valinit \in \InitVal(\key) } }
                        , \emptyset, \emptyset}
    \\ \TraceToX(\ToRed[\TOP]{\ettrc | \lbView{\vi} -> \kvs | \vienv })  
            & \FuncDef \TraceToX(\ettrc)
    \\ \TraceToX(\ToRed[\TOP]{\ettrc | \lbTrans{\fp} -> \kvs | \vienv })
            & \FuncDef \begin{array}[t]{@{}l@{}}
                \CodeFont{let} \ (\kvs',\vienv') = \LastConf(\ettrc),
                \txid = \kvs \setminus \kvs'
                \\ \phantom{\CodeFont{and}} \ 
                    \txidsetrd \subseteq 
                    \Set{ \txid' | \Forall{, \key, \val} 
                            (l,\key,\val) \in \aexec(\txid') \implies l = \opR } , 
                \\ \phantom{\CodeFont{and}} \ \txidset = \VisTrans(\kvs',\vienv(\cl)) \cup \txidsetrd
                \ \AND \aexec \in \TraceToX(\ettrc) 
                \\ \qquad \CodeFont{in} \Set{\UpdateAExec(\aexec, \txidset, \fp, \txid)} .
            \end{array}
\end{align*}
where \( \VisTrans \) is defined in \cref{def:vis-transactions-function}.
\end{definition}

Given an \( \ET[\top] \)-trace \( \ettrc \), function \( \TraceToX \) inductively constructs 
a set of possible abstract executions \( \aexec \):
\begin{enumerate*}
\item a view-shift transition in the trace \( \ettrc \) does not affect \( \aexec \);
and \item a fingerprint transition with a given view \( \vi \)
corresponds to a node of which the visible transactions 
are either the writer of the versions include in the view  \( \vi \),
that is \(  \VisTrans(\kvs',\vienv(\cl)) \), 
or read-only transactions \( \txidsetrd \).
\end{enumerate*}
It is safe to include any numbers of read-only transactions,
because the read-only transactions do not affect the snapshots.
By the construction, the arbitrary order 
in the abstract executions \( \aexec \in \TraceToX(\ettrc) \)  matches 
the commit order in the trace \( \ettrc \).

\begin{toappendix}
\label{sec:et-trace-aexec-well-formed}
\end{toappendix}
\begin{theoremrep}[Well-formed abstract executions of {\( \XToTrace \)}]
\label{prop:et-trace-to-well-formed-aexec}
\label{thm:et-trace-to-well-formed-aexec}
Given an {\(\et[\TOP]\)}-trace \( \ettrc \), an abstract execution \( \aexec \) 
such that \( \aexec \in \TraceToX(\ettrc) \), is a well-formed abstract execution
and \( \LastConf(\ettrc)\Proj{0} = \XToK(\aexec) \).
\end{theoremrep}
\begin{proof}
We prove by induction on the length of the trace \( \ettrc \).
\begin{enumerate}
\CaseBase{\( \ettrc = \ToRed{\kvsinit | \vienvinit } \)}
    By definition of \( \TraceToX \), the only abstract execution \( \Set{\aexec} = \TraceToX(\ettrc) \)
    is defined by \( \aexec = \Tuple{\Mapping{\txidinit
                        -> \Set{(\key,\valinit) | \key \in \Keys \land  \valinit \in \InitVal(\key) } }
        , \emptyset, \emptyset} \);
    it is trivially a well-formed abstract execution and \( \kvsinit = \XToK(\aexec)\).
\CaseInd{\( \ettrc = \ToRed[\TOP]{\ettrc' | \lb -> \kvs | \vienv } \)}
    Consider the label \( \lb \).
    \begin{enumerate}
    \Case{\( \lb = \lbView{\vi}\)}
        By definition of \(\TraceToX\), \( \TraceToX(\ettrc) = \TraceToX(\ettrc') \);
        by \ih, every abstract execution in \TraceToX(\ettrc) is well-formed.
    \Case{\( \lb = \lbTrans{\fp}\)}
        Let configuration \( (\kvs',\vienv') = \LastConf(\ettrc') \) 
        and view \( \vi = \vienv'(\cl)\).
        By \ih, suppose a well-formed abstract execution \( (\txidop',\VIS',\AR') \in \TraceToX(\ettrc') \).
        Let a set of read only transactions \( \txidsetrd \) such that 
        \[
            \txidsetrd \subseteq 
            \Set{ \txid' | \Forall{l | \key \in \Keys | \val \in \Values } 
                    (l,\key,\val) \in \txidop'(\txid') \implies l = \opR } .
        \]
        Let the set of visible transactions \( \txidset = \GetView(\kvs',\vi') \cup \txidsetrd \).
        Assume new transaction \( \txid' = \kvs \setminus \kvs' \).
        By definitions of \( \TraceToX \) and \( \UpdateAExec \), the new abstract execution is defined by
        \[
            \aexec = \Tuple{\txidop \cup \Mapping{\txid' -> \fp }
            , \VIS' \cup \Set{(\txid',\txid) | \txid' \in \txidset }
            , \AR' \cup \Set{(\txid',\txid) | \txid' \in \aexec } } .
        \]
        Let \( \VIS = \VIS' \cup \Set{(\txid',\txid) | \txid' \in \txidset } \)
        and \( \AR = \AR' \cup \Set{(\txid',\txid) | \txid' \in \aexec } \).
        By \ih, we only need to consider 
        the arbitration and visibility edges related to the new transaction \( \txid \).
        It is trivial that the new arbitrary relation \( \AR \) satisfies 
        \cref{equ:aexec-ar-init,equ:aexec-ar-total,equ:aexec-ar-irreflexive,equ:aexec-ar-transitive};
        because the new transition is picked by \( \txid \in \NextTxid(\kvs',\cl)\), 
        \cref{equ:aexec-ar-so} holds.
        Consider the new visibility relation \( \VIS' \cup \Set{(\txid',\txid) | \txid' \in \txidset } \).
        By the definition of \GetView, the initialisation transaction \( \txidinit \in \txidset \),
        and thus \( (\txidinit, \txid) \in \VIS \) which implies \( \cref{equ:aexec-vis-init}\).
        Consider a read operation \( \opR(\key,\val) \in \fp \) for some key \( \key \) and \( \val \).
        By the \( \et[\TOP] \)-trace and \UpdateKV function, 
        it is known that \( \kvs'(\key,\Max[<](\kvs,\vi)) = \val \);
        therefore by the definition of \(\GetView\), the writer
        \( \WtOf(\kvs'(\key,\Max[<](\kvs,\vi))) = \MaxVisTrans((\txidop',\VIS',\AR'),\txidset,\key)\),
        which implies \cref{equ:aexec-vis-external}.
        Because the new transition is picked by \( \txid \in \NextTxid(\kvs',\cl)\), 
        \cref{equ:aexec-vis-so} holds.
        Last, it is trivial that \( \VIS \subseteq \AR \).

        Now we prove \( \kvs = \XToK(\aexec) \) for the new abstract execution \( \aexec \).
        Note that \( \aexec = \UpdateAExec(\aexec,\txidset,\fp,\txid)\) 
        for a fresh transaction identifier \( \txid in \NextAExecTxid(\aexec,\cl) \),
        and \( \kvs = \UpdateKV(\kvs,\vi,\fp,\txid)\) for the view \( \vi = \GetView(\txidset,\kvs) \).
        By \cref{prop:update-aexec-to-udpate-kv}, \(\kvs = \XToK(\aexec) \). \qedhere
    \end{enumerate}
\end{enumerate}
\end{proof}
\begin{proofsketch}
It is trivial by induction on the length of the trace.
For view-shift transition, we directly apply \ih.
For fingerprint transition, the result can be derived from 
the following result (\cref{prop:update-aexec-to-udpate-kv} on page \pageref{sec:update-aexec-kv-store}):
Given:
\begin{enumerate*}
\item a kv-store \( \kvs \);
\item an abstract execution \( \aexec \) such that \( \kvs = \XToK(\aexec)\);
\item a view \( \vi = \GetView(\kvs,\txidset)\) for 
transaction set \( \txidset \subseteq \aexec \) with \( \txidinit \in \txidset \);
and 
\item the next abstract execution 
\( \aexec' = \UpdateAExec(\aexec, \txidset, \fp, \txid) \) for some fingerprint \( \fp \)
and fresh transaction \( \txid \in \NextAExecTxid(\aexec,\cl) \), 
\end{enumerate*}
then \( \UpdateKV(\kvs,\vi,\fp,\txid) = \XToK(\aexec') \).
The full detail is given in \cref{sec:et-trace-aexec-well-formed} on \pageref{sec:et-trace-aexec-well-formed}.
\end{proofsketch}


\begin{toappendix}
\label{sec:update-aexec-kv-store}
\begin{proposition}[Update of abstract execution matching update of kv-store]
\label{prop:update-aexec-to-udpate-kv}
Assume a kv-store \( \kvs \) and an abstract execution \( \aexec \) such that \( \kvs = \XToK(\aexec)\).
Let \( \vi = \GetView(\kvs,\txidset)\) for some view \( \vi \) and 
transaction set \( \txidset \subseteq \aexec \) with \( \txidinit \in \txidset \).
Given the new abstract execution 
\( \aexec' = \UpdateAExec(\aexec, \txidset, \fp, \txid) \) for some fingerprint \( \fp \)
and fresh transaction \( \txid \in \NextAExecTxid(\aexec,\cl) \), 
then \( \UpdateKV(\kvs,\vi,\fp,\txid) = \XToK(\aexec') \).
\end{proposition}
\begin{proof}
Let \( \kvs' = \UpdateKV(\kvs,\vi,\fp,\txid)\).
Note that if the fresh transaction \( \txid \in \NextAExecTxid(\aexec,\cl) \)
then \( \txid \in \NextTxid(\kvs,\cl) \).
By \cref{thm:isomorphism-kvstore-dgraph,def:aexec-dgraph},
it suffice to prove that \( \KToD(\kvs') = \XToD(\aexec')\).
By the definition of \( \KToD \) and \( \XToD \) 
in \cref{def:aexec-dgraph,def:kv-store-to-dependency-graph} respectively,
it is sufficient to prove that:
\begin{multline}
    \KToD(\kvs) = \XToD(\aexec) \implies 
    \KToD(\kvs')\Proj{0} = \KToD(\kvs)\Proj{0}\Mapping{\txid -> \fp} = \XToD(\aexec')\Proj{0}
    \\ {} \land \WR[\kvs'] = \WR[\aexec'] 
    \land \WW[\kvs'] = \WW[\aexec'] .
\end{multline}
Recall that in dependency graphs, \( \RW \) can be derived from \( \WW \)  and \( \WW \).
Consider the three conjunctions.
\begin{enumerate}
\Case{new transaction \( \txid \) and its fingerprint \( \fp \)}
    By definition of \( \UpdateAExec \), it is known that, for any operation \( \op \in \fp\)
    key  \( \key \) and value \( \val \),
    \[
        \begin{multlined}
        ( \op = \opR(\key,\val) \iff \opR(\key,\val) \in \aexec'(\txid) ) 
        \land ( \op = \opW(\key,\val) \iff  \opW(\key,\val) \in \aexec'(\txid) ) .
        \end{multlined}
    \]
    then by definition of \( \XToD \), 
    we know that \[ \XToD(\aexec)\Proj{0}\Mapping{\txid -> \fp} = \XToD(\aexec')\Proj{0} . \]
    Now consider the another side.
    By the definition of \( \UpdateKV \), 
    for any operation \( \op \in \fp \),
    we have the following result for some key \( \key \) and \( \val \):
    \[
        \begin{multlined}
        ( \op = \opR(\key,\val) \iff  \Exists{\idx} \txid \in \RsOf(\kvs'(\key,\idx)) )
        \\ {} \land ( \op = \opW(\key,\val) \iff  \txid = \WtOf(\kvs'(\key,\Abs{\kvs'(\key)} - 1)) ) .
        \end{multlined}
    \]
    Therefore by definition of \( \KToD \),
    \[
        \begin{multlined}
        ( \Exists{\idx} \txid \in \RsOf(\kvs'(\key,\idx)) \iff \opR(\key,\val) \in \KToD(\kvs') )
        \\ {} \land ( \txid = \WtOf(\kvs'(\key,\Abs{\kvs'(\key)} - 1)) \iff \opW(\key,\val) \in \KToD(\kvs') ) ,
        \end{multlined}
    \]
    which means that \( \KToD(\kvs')\Proj{0} = \KToD(\kvs)\Proj{0}\Mapping{\txid -> \fp} \).
    Recall that \( \KToD(\kvs) = \XToD(\aexec)  \); this means \(  \KToD(\kvs') = \XToD(\aexec') \).
\Case{write-read dependency \( \WR \)}
    Because \( \KToD(\kvs) = \XToD(\aexec) \),
    we only need to consider \( \WR \)-edges related to the new transaction \( \txid \).
    Consider any key \( \key \) such that \( \opR(\key, \val) \in \fp \) for some value \(\val \).
    By definition of \( \UpdateAExec \), transaction \( \txid \) must read from 
    a transaction \( \txid' \in \txidset \) 
    who wrote the key \( \key \) and is the latest transaction visible by \( \txid \),
    and therefore by definition of \( \XToD \), 
    \[
        \txid' = \MaxVisTrans(\aexec,\txidset,\key) \iff (\txid',\txid) \in \WR[\aexec'] .
    \]
    By definition of the view \( \vi = \GetView(\aexec,\txidset) \),
    the latest version of \( \key \) must be written by \( \txid \),
    that is,
    \[
        \txid' = \WtOf(\kvs(\key,\Max[<](\vi(\key)) )) .
    \]
    By the definition of \( \UpdateKV \),
    the transaction \( \txid \) must read the version \(\kvs(\key,\Max[<](\vi(\key)) )\),
    and then by definition of \( \KToD \),
    \[
        ( \txid' = \WtOf(\kvs'(\key,\Max[<](\vi(\key)))) \land \txid \in \RsOf(\kvs'(\key,\Max[<](\vi(\key)))) )
        \iff (\txid',\txid) \in \WR[\kvs'] .
    \]
    Given above, we know
    \[
        (\txid',\txid) \in \WR[\aexec'] \iff (\txid',\txid) \in \WR[\kvs']
    \]
    which means that for any new write-read dependency \( \WR[\aexec']\),
    the same edge exists in \( \WR[\kvs']\) and vice versa.
    Since \(\WR[\aexec] = \WR[\kvs]\), we know \( \WR[\aexec'] = \WR[\kvs'] \).
\Case{write-write dependency \( \WW \)}
    Because \( \KToD(\kvs) = \XToD(\aexec) \),
    we only need to consider \( \WW \)-edges related to the new transaction \( \txid \).
    Consider any key \( \key \) such that \( \opW(\key, \val) \in \fp \) for some value \( \val \).
    By the definition of \( UpdateAExec \), the new transaction wrote the key \( \key \),
    and then by definition of \( \XToD \),
    \[
        \opW(\key, \val) \in \aexec'(\txid) \iff 
        \Forall{\txid' \in \aexec | \val' \in \Values}
        (\opW(\key,\val') \in \aexec(\txid') \implies (\txid',\txid) \in \AR[\aexec']) .
    \]
    Since \( \kvs = \XToK(\aexec) \),
    if a transaction wrote the key \( \key \) in \( \aexec \), 
    it must wrote a version of the key \( \key \) in \( \kvs \),
    that is, 
    \[
        \Forall{\txid' \in \aexec | \val' \in \Values }
        \opW(\key,\val') \in \aexec(\txid') \iff  \Exists{\idx \in \Indexs} \txid' = \WtOf(\kvs(\key,\idx)) .
    \]
    By the definition of \( \UpdateKV \), 
    the transaction \( \txid \) wrote the latest version of \( \key \) in the new kv-store \( \kvs \),
    and by the definition of \( \KToD \),
    \[
    \begin{multlined}[t]
        \txid  = \WtOf(\kvs'(\key, \Abs{\kvs'(\key)} - 1 )) \iff {}
            \\ \Forall{\idx \in \Indexs} 
            ( 0 \leq  \idx < \Abs{\kvs'(\key)} - 1 
                        \implies \ToEdge{ \WtOf(\kvs'(\key,\idx)) | \WW[\kvs] -> \txid } )  .
    \end{multlined}
    \]
    Note that \( \Abs{\kvs'(\key)} - 1 = \Abs{\kvs(\key)} \).
    Therefore for any new write-write dependency \( \WW[\aexec']\),
    the same edge exists in \( \WW[\kvs']\) and vice versa.
    Since \(\WW[\aexec] = \WW[\kvs]\), we know \( \WW[\aexec'] = \WW[\kvs'] \). \qedhere
\end{enumerate}
\end{proof}
%\begin{proofsketch}
%We convert \( \UpdateKV(\kvs,\vi,\fp,\txid) \) and \( \XToK(\aexec') \) to dependency graphs, 
%and prove the dependency graphs are the same.
%The full proof is given in \cref{sec:update-aexec-kv-store} on page \pageref{sec:update-aexec-kv-store}.
%\end{proofsketch}
\end{toappendix}

