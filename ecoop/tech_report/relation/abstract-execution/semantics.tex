We introduce an alternative operational semantics on abstract executions.
As such, we can compare the resulting kv-stores and abstract executions with a given program.
The operational semantics on abstract executions is given in \cref{fig:aexec-semantics}.
Each labelled transition is of the form 
\( \ToAexecCmd{\aexec | \clenv | \prog | \lb -> \aexec' | \clenv' | \prog' } \).
The label is either a primitive command, \( \lbPri \),
or a transaction commit step, \( \lbTrans{\txidset, \fp} \),
with the given visible set of transactions \( \txidset \).
The semantics is parametrised by the set of visibility axioms \( \visaxioms \).

\begin{definition}[Abstract execution labels]
The set of abstract execution labels, \( \AexecLabels \ni \lb \), is defined by 
\[
\AexecLabels \TypeDef \Set{ \lbPri | \cl \in \Clients } 
        \cup \Set{ \lbTrans{\txidset, \fp} | \cl \in \Clients 
                        \land \txidset \subseteq \TxIDs \land \fp \in \Fingerprints } .
\]
\end{definition}

\begin{figure}[!th]

\begin{align*}
	\ToAexecCmd*{} & : 
    \begin{multlined}[t]
	( \AbstractExecutions \times \Stacks \times \Commands )
    \times \AexecLabels \times  
	( \AbstractExecutions \times \Stacks \times \Commands )
    \end{multlined}
\end{align*}
\begin{mathpar}
    \inferrule[\rAAtomicTrans]{
        \aexec = (\txidop,\SO,\VIS,\AR)
        \\ \txidinit \in \txidset
        \\ \txidset \subseteq \Dom(\txidop) 
        \\ \snap \in \AexecSnapshot(\aexec, \txidset) 
        \\ \ToTrans{\stk | \snap | \emptyset| \trans  | * -> \stk' | \snap' | \fp | \pskip }
		\\ \txid \in \NextAExecTxid(\aexec, \cl) 
        \\ \aexec' = \UpdateAExec(\aexec, \txidset, \fp, \txid) 
        \\ \Forall{ \visaxiom \in \visaxioms } \Inv(\visaxiom)(\aexec')(\txid) \subseteq \txidset
    }{%
        \ToAexecCmd{ \aexec | \stk | \ptrans{\trans} | \lbTrans{\txidset,\fp} -> \aexec' | \stk' | \pskip }
    }
    \and
    \inferrule[\rAPrimitive]{
	    \stk \ToCmdPri{\cmdpri} \stk'
    }{%
        \ToAexecCmd{ \aexec | \stk | \cmdpri | \lbPri -> \aexec' | \stk' | \pskip } 
    }
    \and
    \inferrule[\rAChoice]{
        \idx \in \Set{1,2}
    }{%
        \ToAexecCmd{ \aexec | \stk | \cmd_{1} \pchoice \cmd_{2} | \lbPri -> \aexec | \stk | \cmd_{\idx} } 
    }
    \and
    \inferrule[\rAIter]{ }{%
        \ToAexecCmd{ \aexec | \stk | \prepeat(\cmd) | \lbPri 
            -> \aexec | \stk | \pskip \pchoice (\cmd \pseq \prepeat(\cmd)) } 
    }
    \and
    \inferrule[\rASeqSkip]{ }{%
        \ToAexecCmd{ \aexec | \stk | \pskip \pseq \cmd | \lbPri -> \aexec | \stk | \cmd } 
    }
    \quad
    \inferrule[\rASeq]{% 
        \ToAexecCmd{ \aexec | \stk | \cmd_{1} | \lb -> \aexec | \stk | \cmd'_{1} } 
    }{%
        \ToAexecCmd{ \aexec | \stk | \cmd_{1} \pseq \cmd_{2} | \lb 
                    -> \aexec | \stk | \cmd'_{1} \pseq \cmd_{2} } 
    }
\end{mathpar}

\hrulefill

\begin{align*}
    \ToAexecProg*{} & : 
    \begin{multlined}[t]
    ( \AbstractExecutions \times \ClientEnvs \times \Programs )
    \times \AexecLabels \times 
    ( \AbstractExecutions \times \ClientEnvs \times \Programs )
    \end{multlined}
\end{align*}
\begin{mathpar}
    \inferrule[\rAProg]{%
        \stk = \clenv(\cl)
        \\ \cmd = \prog(\cl)
        \\ \lb = (\cl, \cdots)
        \\ \ToAexecCmd{ \aexec | \stk | \cmd | \lb -> \aexec | \stk' | \cmd' } 
    }{%
        \ToAexecProg{ \aexec | \clenv | \prog | \lb 
                -> \aexec' | \clenv\UpdateFunc{ \cl -> \stk' } | \prog\UpdateFunc{ \cl -> \cmd' } } 
    }
\end{mathpar}

\hrulefill

\caption{Operational semantics on abstract executions}
\label{fig:aexec-semantics}
\end{figure}


In \cref{fig:aexec-semantics}, except the rule \(\rAAtomicTrans\), other rules are standard 
and mimic the counterparts in \cref{fig:command-semantics,fig:program-semantics}.
The rule \(\rAAtomicTrans\) describes how a client \( \cl \) with the set of visible transactions \( \txidset \),
commits a transaction \( \ptrans{\trans} \), in a similar way as the rule \( \rCAtomicTrans \) in \cref{fig:command-semantics}.
Prior to executing the transactional command \( \trans\),
the client \( \cl \) picks the set of visible transactions \( \txidset \)
(the second and third premises \(\txidinit \in \txidset \land \txidset \subseteq \Dom(\txidop) \)).
This set of transactions determines an initial snapshot
that contains the latest visible value of each key,
captured by the fourth premise \( \snap \in \AexecSnapshot(\aexec, \txidset)\) in \( \rAAtomicTrans \).
The function \( \AexecSnapshot \) models the \emph{last-write-win} policy.
It is straightforward that \( \AexecSnapshot \) is a valid snapshot,
since \( \txidinit \in \txidset \) and \( \txidinit \) initialise all the keys in a well-formed abstract execution.
%in the sense that
%the initial snapshot given by \( \AexecSnapshot(\aexec, \txidset) \)
%contains the last visible value (with respect to \( \AR \)) for each key.

\begin{definition}[Last-write-win resolution policy on abstract executions]
The function,
\( \AexecSnapshot : \AbstractExecutions \times \PowerSet(\TxIDs) \ToTFunc \Snapshots \),
is defined by: for all keys \( \key \in \Keys \),
\[
    \AexecSnapshot(\aexec, \txidset) \FuncDef \lambda \key \ldotp
    \begin{multlined}[t]
        \CodeFont{let} \ \txid = \MaxVisTrans(\aexec,\txidset,\key)
        \CodeFont{and} \ \opW(\key,\val) \in \aexec(\txid) \ \CodeFont{in} \ \val
    \end{multlined}
\]
\end{definition}

\begin{toappendix}
\begin{proposition}[Well-defined last-write-win resolution policy]
\label{prop:well-defined-aexec-snapshot}
Given a well-formed abstract execution \( \aexec \) and a set of transactions \( \txidset \)
that includes the initialisation transaction \( \txidinit \),
the function \( \AexecSnapshot(\aexec, \txidset) \) is defined.
\end{proposition}
\begin{proof}
It is straightforward since transaction \( \txidinit \) initialise all the keys in a well-formed abstract execution.
\end{proof}
\end{toappendix}

The transaction command \( \trans \) is executed with the initial snapshot \( \snap \),
resulting a final fingerprint \( \fp \) where the final snapshot \( \snap' \) is ignored (the fifth premise in \( \rAAtomicTrans\)).
As with \( \rCAtomicTrans \), 
the client \( \cl \) is now ready to commit the transaction with the fingerprint \( \fp \):
\begin{enumerate}
\item pick the next fresh transaction identifier \( \txid \in \NextAExecTxid(\kvs,\cl) \);
\item compute the new abstract execution via \(\aexec' = \UpdateAExec(\aexec, \txidset, \fp, \txid) \); and 
\item check if the \emph{new visibility edges} satisfy the axioms \( \visaxioms \):
    that is, \( \Inv(\visaxiom)(\aexec')(\txid) \subseteq \txidset\) for all \( \visaxiom \in \visaxioms\).
\end{enumerate}

\begin{definition}[Fresh transaction identifiers and abstract execution update]
\label{def:update-aexec}
Given an abstract execution \( \aexec \) and a client \( \cl \),
the set of \emph{next available transaction identifiers}, written \( \NextAExecTxid(\aexec,\cl) \),
is defined by:
\[
    \NextAExecTxid(\aexec,\cl) \FuncDef \Set{ \txid[\cl](n) | \txid[\cl](n) \in \TxIDs 
                \land \Forall{ m \in \Indexs | \txid[\cl](m) \in \aexec } m < n } .
\]
Given a set of visible transactions \( \txidset \subseteq \TxIDs \), a fingerprint \( \fp \in \Fingerprints \),
and a fresh transaction \( \txid \in \NextAExecTxid(\aexec,\cl) \),
the new abstract execution, written \( \UpdateAExec(\aexec,\txidset,\fp,\txid) \),
is defined by:
\[
    \UpdateAExec(\aexec,\txidset,\fp,\txid) \FuncDef 
    \begin{multlined}[t]
    \CodeFont{let} \ (\txidop,\VIS,\AR) = \aexec \ \CodeFont{in} 
    \\ \Tuple{\txidop \cup \Mapping{\txid -> \fp }
            , \VIS \cup \Set{(\txid',\txid) | \txid' \in \txidset }
            , \AR \cup \Set{(\txid',\txid) | \txid' \in \aexec } } .
    \end{multlined}
\]
\end{definition}

\begin{toappendix}
\label{sec:proof-well-defined-update-aexec}
\end{toappendix}
\begin{propositionrep}[Well-defined \(\UpdateAExec\) function]
Given an abstract execution \( \aexec \in \AbstractExecutions \), 
a set of transactions \( \txidset \subseteq \aexec \) with \( \txidinit \in \txidset \),
a fresh transaction identifier \( \txid \in \NextAExecTxid(\kvs,\cl) \) for some client \( \cl \),
and a fingerprint \( \fp \in \Fingerprints \) such that 
\begin{Formulae}
\begin{Formula}
\Forall{\key \in \Keys | \val \in \Values } \opR(\key,\val) \in \fp 
    \implies \MaxVisTrans(\aexec,\txidset,\key) , 
\label{equ:aexec-fp-constrain}
\end{Formula}
\end{Formulae}
the new abstract execution \( \UpdateAExec(\aexec,\txidset,\fp,\txid) \) is a well-formed abstract execution.
\end{propositionrep}
\begin{proof}
Let new abstract execution \( \aexec' = \UpdateAExec(\aexec,\txidset,\fp,\txid) \).
\begin{enumerate}
\Case{well-formed arbitration order \(\AR[\aexec']\)}
    It suffices to only consider the edges related to the new transaction \( \txid \).
    Since the new arbitration order \( \AR[\aexec'] = \AR[\aexec] \cup \Set{(\txid',\txid) | \txid' \in \aexec'} \)
    and \( \txidinit \in \aexec\),
    it is straightforward for \cref{equ:aexec-ar-init,equ:aexec-ar-total}.
    Because \( \txid \) is a fresh transaction such that \( \txid \notin \aexec \), therefore \cref{equ:aexec-ar-irreflexive} holds.
    By the definition of \( \AR[\aexec'] \) that contains edges from any transaction \( \txid'\) from the old abstract execution \( \aexec \) to the new transaction \( \txid \), it follows \cref{equ:aexec-ar-transitive}.
    Last, by definition of \( \NextAExecTxid \),
    the fresh transaction \( \txid =  \txid[\cl](n) \) should be annotated with a bigger number \( n \) than 
    transactions for the same client \( \cl \) in \( \aexec \), that is,
    \( \Forall{m \in \Indexs | \txid[\cl](m) \in \aexec } n > m \);
    this means that \( \SO \subseteq \AR[\aexec'] \) which implies \cref{equ:aexec-ar-so}.
\Case{well-formed visibility relation \(\VIS[\aexec']\)}
    It suffices to only consider the edges related to the new transaction \( \txid \).
    By the definition of \( \VIS[\aexec'] = \VIS[\aexec] \cup \Set{(\txid',\txid) | \txid' \in \txidset}  \) that 
    contains edges from any transaction \( \txid' \) from \( \txidset \) to the new transaction \( \txid \) 
    and \( \txidinit \in \txidset \), 
    it follows \cref{equ:aexec-vis-init}.
    The \cref{equ:aexec-fp-constrain} immediately implies \cref{equ:aexec-vis-external}.
    Since the fresh transaction identifier are pick by  \( \txid \in \NextAExecTxid(\aexec,\cl) \), 
    it is straightforward for \cref{equ:aexec-vis-so}.
    Last, by \( \txidset \subseteq \aexec \), it follows \cref{equ:aexec-ar-vis}. \qedhere
\end{enumerate}
\end{proof}
\begin{proofsketch}
This is straightforward since \(  \txid \in \NextAExecTxid(\aexec,\cl) \) is a fresh transition identifier 
annotated with a greater number than any previous transaction identifiers in the same session.
Therefore the \( \AR \) of the new abstract execution is well-formed.
The fingerprint contains the maximum visible value of each key \( \key \),
if there is a read operation for \( \key \).
Therefore the \( \VIS \) of the new abstract execution is well-formed.
The full proof is given in \cref{sec:proof-well-defined-update-aexec} 
on page \pageref{sec:proof-well-defined-update-aexec}.
\end{proofsketch}


