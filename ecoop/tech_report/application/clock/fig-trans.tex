\begin{figure}

\centering

\begin{subfigure}{\textwidth}
\centering
\begin{tikzpicture}
\OperationsBox[above]{r1}{\clockshard_1,\ \texttt{Local Time:} \ 2}{
    /\texttt{key-value store:},
    /{\Tuple{\key_1 , \valinit, 0} },
    /{},
    /\texttt{Preparation set:},
    /\emptyset
};

\OperationsBox((r1.east) + (2,0.35))[above]{r2}{\clockshard_2,\ \texttt{Local Time:} \ 2}{
    /\texttt{key-value store:},
    /{\Tuple{\key_2 , \valinit, 0} },
    /{},
    /\texttt{Preparation set:},
    /\emptyset
};

\OperationsBox((r1.south) + (-2.8,-1))[above]{t1}{\txid_1,\ \texttt{Snapshot Time:} \ 1}{
    /{\opR(\key_1,\valinit)},
    /{\opR(\key_2,\valinit)},
    /{\opW(\key_2,\val')}
};

\OperationsBox((r2.south) + (-1.7,-1))[above]{t2}{\txid_2,\ \texttt{Snapshot Time:} \ 1}{
    /{\opR(\key_1,\valinit)},
    /{\opR(\key_2,\valinit)}
};
 
\end{tikzpicture}
\caption{The final fingerprint of \( \txid_1 \)}
\label{fig:clock-si-parallel-exec}
\end{subfigure}

\hrulefill

\begin{subfigure}{\textwidth}
\centering
\begin{tikzpicture}
\OperationsBox[above]{r1}{\clockshard_1,\ \texttt{Local Time:} \ 2}{
    /\texttt{key-value store:},
    /{\Tuple{\key_1 , \valinit, 0} },
    /{},
    /\texttt{Preparation set:},
    /\emptyset
};

\OperationsBox((r1.east) + (2,0.35))[above]{r2}{\clockshard_2,\ \texttt{Local Time:} \ 2}{
    /\texttt{key-value store:},
    /{\Tuple{\key_2 , \valinit, 0} },
    /{},
    /\texttt{Preparation set:},
    /{\Tuple{\key_2 , \val', 2} }
};

\OperationsBox((r1.south) + (-2.8,-1))[above]{t1}{\txid_1,\ \texttt{Snapshot Time:} \ 1}{
    /{\opR(\key_1,\valinit)},
    /{\opR(\key_2,\valinit)},
    /{\opW(\key_2,\val')}
};

\OperationsBox((r2.south) + (-1.7,-1))[above]{t2}{\txid_2,\ \texttt{Snapshot Time:} \ 1}{
    /{\opR(\key_1,\valinit)},
    /{\opR(\key_2,\valinit)}
};
 
\end{tikzpicture}
\caption{The preparation phase of \( \txid_1 \)}
\label{fig:clock-si-preparation}
\end{subfigure}

\hrulefill

\begin{subfigure}{\textwidth}
\centering
\begin{tikzpicture}
\OperationsBox[above]{r1}{\clockshard_1,\ \texttt{Local Time:} \ 2}{
    /\texttt{key-value store:},
    /{\Tuple{\key_1 , \valinit, 0} },
    /{},
    /\texttt{Preparation set:},
    /\emptyset
};

\OperationsBox((r1.east) + (2,0.35))[above]{r2}{\clockshard_2,\ \texttt{Local Time:} \ 2}{
    /\texttt{key-value store:},
    /{\Tuple{\key_2 , \valinit, 0} },
    /{},
    /\texttt{Preparation set:},
    /{\Tuple{\key_2 , \val', 2} }
};

\OperationsBox((r1.south) + (-2.8,-1))[above]{t1}{\txid_1,\ \texttt{Commit Time:} \ 2}{
    /{\opR(\key_1,\valinit)},
    /{\opR(\key_2,\valinit)},
    /{\opW(\key_2,\val')}
};

\OperationsBox((r2.south) + (-1.7,-1))[above]{t2}{\txid_2,\ \texttt{Snapshot Time:} \ 1}{
    /{\opR(\key_1,\valinit)},
    /{\opR(\key_2,\valinit)}
};
 
\end{tikzpicture}
\caption{The commit phase of \( \txid_1 \)}
\label{fig:clock-si-preparation}
\end{subfigure}

\hrulefill

\caption{Clock-SI two-phase commit protocol}
\label{fig:clock-si-example}

\end{figure}
