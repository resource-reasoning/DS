In Clock-SI, each shard only guarantees that individual write and read operations be atomic.
Several transactions can execute in the shard in a fine-grained way.

As explained in \cref{sec:clock-si-protocol-description},
a client commits a transaction to an \emph{arbitrary} shard and then awaiting the confirmation.
Upon receiving the request, 
the shard acts as the coordinator of the transaction in that 
it is responsible for fetching value of keys locally or remotely,
and committing any new values to the appropriate shards using a two-phase commits protocol.
The coordinator initialises the transaction runtime, by calling \( \pclockstart \).
The reference implementation for \(  \pclockstart  \) is given in \cref{fig:clock-si-transaction-start}:
\begin{enumerate}
\item the coordinator assigns the shard local time as the \emph{snapshot time} for this transaction (\CodeFont{trans}),
if the coordinator local time is greater than the minimum snapshot time (\CodeFont{clientTime}),
otherwise, the shard postpones the transaction;
\item the coordinator initialises the read-write set (\verb|read-write-set|) 
of the transaction as the empty set;
and \item transaction state is set to be \( \clocksiactive\).
\end{enumerate}

Rule\( \rCLOCKStart\) in \cref{fig:clock-si-transaction-start} captures the semantics of \(  \pclockstart  \).
The runtime transaction, of the form \( \pruntrans{\trans}{\fp,\clockbuffer}{\clocktime,\clockshard} \),
tracks the runtime information of this transaction, 
comprising a snapshot time \( \clocktime \), a coordinator \( \clockshard \),
a fingerprint \( \fp \) and a \emph{preparation buffer} \( \clockbuffer \).

\begin{definition}[Clock-SI runtime commands]
The set of \emph{preparation buffers}, \( \CLOCKBuffers \ni \clockbuffer \),
is defined by \( \CLOCKBuffers \TypeDef \List{\Keys \times \Nat \times \CLOCKtimes } \).
Given the set of commands, \( \cmd \) (\cref{def:command}),
the set of \emph{Clock-SI runtime commands}, \( \CLOCKCommands \ni \clockruncmd \),
is defined:
\[
\clockruncmd ::= \pruntrans{\trans}{\fp,\clockbuffer}{\clocktime,\clockshard} \ | \ \cmd \ | \ \clockruncmd \pseq \cmd 
\]
for some fingerprint \( \fp \), Clock-SI time \( \clocktime \) and shard \( \clockshard \). 
The set of \emph{Clock-SI runtime programs}, \( \CLOCKProgs \ni \clockrunprog\),
is defined by: \( \CLOCKProgs \TypeDef \Clients \ToPFFunc \CLOCKCommands\).
\end{definition}

Each runtime transaction, \( \pruntrans{\trans}{\fp,\clockbuffer}{\clocktime,\clockshard} \), has the following phases:
\begin{enumerate}
    \item the coordinator executes the command of the transaction \( \trans \),
            reading keys locally or remotely and accumulating the effect to the read-write set \( \fp \);
            %where \( \fp \) tracks the read and write operations and \( \clockbuffer \) remains an empty set;
    \item when \( \trans \) is reduced to \( \pskip \),
            the coordinator sends the new values of keys in \( \fp \) to the shards \( R \) that contains these keys,
            and collects the replies that contains receiving times, storing them in \( \clockbuffer \); and
    \item Upon receiving all replies, the coordinator picks the maximum time in \( \clockbuffer \)
            as the actual commit time, and sends the time to \( R \), hence committing the transaction.
\end{enumerate}
Note that, in the Clock-SI protocol,
shards will reject new values of keys if there are conflict writes to these keys.
Although the rejected transaction might restart, it is not observable by clients.
Hence, our operational semantics of Clock-SI focuses on the successful transactions.

Transition in the operational semantics of Clock-SI are annotated with labels.
These labels are useful for manipulating traces in the verification.

\begin{definition}[Clock-SI semantics labels]
The set of \emph{Clock-SI semantics labels}, \( \CLOCKLabels \ni \lb \),
is defined by
\[
    \CLOCKLabels \supseteq \bigcup_{\substack{
                \cl \in \Clients, \clockshard \in \CLOCKShards,
                \\ l \in \Set{\opW, \opR, \opP}, \key \in \Keys,
                \\ \val \in \Values, \clocktime \in \CLOCKTimes
                }}
                \Set{\lbCLOCKStart{\clocktime},\lbCLOCKOp{(l,\key,\val),\clocktime},\\ \lbCLOCKEnd{\clocktime},\lbCLOCKTick{\clocktime}}
\]
\end{definition}

We already see the snapshot label, \( \lbCLOCKStart{\clocktime} \), in \( \rCLOCKStart \),
stating that the client \( \cl \) starts a transaction at time \( \clocktime \) in the coordinator \( \clockshard \).
We give an intuition for the rest labels here and explain them in detail later.
A label of the form \( \lbCLOCKOp{(l,\key,\val),\clocktime} \) means an transaction step:
a transaction with  client \( \cl \) and coordinator \( \clockshard \)
writes \( l = \opW \), reads \( l = \opR \) or prepares \( l = \opP \)
key \( \key \) with value \( \val \).
For write and read label, the time \( \clocktime \) is the snapshot time of the transaction.
For preparation label, the time \( \clocktime \) is the preparation time.
Label \( \lbCLOCKEnd{\clocktime}\) corresponds the final commit transition of a transaction.
Label  \( \lbCLOCKTick{\clocktime} \) corresponds time tick transition 
in which the shard \( \clockshard \) advances its local time.

We now explain how the coordinator executes the command of a transaction.
For any mutation command of the form \( \pmutate{\key}{\expr}\), 
the coordinator calls \( \pclockwrite \) defined in \cref{fig:clock-write-rule}.
The function erase any preview write entry of the key and
adds the new value of the key to the read-write set.
Rule \( \rCLOCKWrite \) in\cref{fig:clock-read-rule} captures the semantics of this function.
The new write operation is added into \( \fp \) via \( \AddOp \).
Recall that the fingerprint combination operation \( \AddOp \) is defined in \cref{def:addop}.

\begin{figure}
\begin{subfigure}{\textwidth}
\begin{mathpar}
\inferrule[\rCLOCKStart]{
    \clocksi(\clockshard) = (\stub, \clocktime')
    \\
    \clocktime' > \clocktime
}{%
    \ToClockCmd{
    \clocksi | \clocktime | \stk
            | \ptrans{\trans}
            | \lbCLOCKStart{\clocktime'} ->
    \clocksi | \clocktime |  \stk
            | \pruntrans{\trans}{ \emptyset, \emptyset }{\clocktime',\clockshard} }
}
\end{mathpar}

\caption{The rules for \( \pclockstart \)}
\label{}
\end{subfigure}

\hrulefill

\begin{subfigure}{\textwidth}
\begin{lstlisting}
transSnapshot( trans, clientTime )
    wait until clientTime < getShardClockTime();
    trans.snapshotTime = getShardClockTime();
    trans.read-write-set = (*\(\emptyset\)*);
    trans.state = active;
\end{lstlisting}

\caption{The reference implement of transaction start}
\label{lst:clock-si-trans-runtime}
\end{subfigure}

\hrulefill

\caption{Clock-SI: transaction start}
\label{fig:clock-si-transaction-start}
\end{figure}


\begin{figure}[!p]

\begin{subfigure}{\textwidth}
\begin{lstlisting}
transWrite( trans, k, v )
    trans.read-write-set = trans.read-write-set \ (w,k,-);
    trans.read-write-set = trans.read-write-set + (w,k,v);
\end{lstlisting}

\caption{The reference implementation of transactional write}
\label{lst:clock-si-write-value}
\end{subfigure}

\hrulefill


\begin{subfigure}{\textwidth}
\begin{mathpar}
\inferrule[\rCLOCKWrite]{
    \key = \EvalE{\expr_1}
    \\ 
    \val = \EvalE{\expr_2}
}{%
    \ToClockCmd{
    \clocksi | \clocktime | \stk
            | \pruntrans{\pmutate{\expr_1}{\expr_2}; \trans}{\fp,\emptyset}{\clocktime',\clockshard} 
            | \lbCLOCKOp{\opW(\key,\val),\clocktime'} ->
    \clocksi | \clocktime |  \stk
            | \pruntrans{\trans}{\fp \AddOp \opW(\key,\val),\emptyset}{\clocktime',\clockshard} }
}
\end{mathpar}

\caption{The semantics of \( \pclockwrite \)}
\label{fig:clock-write-rule}
\end{subfigure}

\hrulefill

\begin{subfigure}{\textwidth}
\begin{lstlisting}
transRead( trans, k )
    if ( (w,k,v) in trans.read-write-set ) return v; (* \label{line:clock-read-from-write} *)
    else if ( (r,k,v) in trans.read-write-set ) return v; (* \label{line:clock-read-from-read} *)
    else {
        v = readFromShard(shard, trans.snapshotTime);
        trans.read-write-set = trans.read-write-set + (r,k,v); (* \label{line:clock-read-write-remote} *)
        return v;
    }

readFromShard(shard, snapshotTime, k)
    wait snapshotTime < getShardClockTime(); (* \label{line:clock-remote-read-wait} *)
    i = 0;
    for ver of getStore() 
        if (ver.key = k and ver.time <= snapshotTime) { 
            wait until ver.state == committed; (* \label{line:clock-remote-read-wait-commit} *)
            if(i <= ver.time) i = ver.time; (* \label{line:clock-remote-latest-version} *)
        }
    return getValue(k,i); (* \label{line:clock-remote-latest-version} *)
\end{lstlisting}

\caption{The reference implementation of transactional read}
\label{lst:clock-si-read-value}
\end{subfigure}

\hrulefill

\begin{subfigure}{\textwidth}
\begin{mathpar}
\inferrule[\rCLOCKReadLocal]{ 
    \key = \EvalE{\expr} 
    \\
    \opW(\key,\val) \in \fp \lor (\opW(\key,\val) \notin \fp \land \opR(\key,\val) \in \fp)
}{%
    \ToClockCmd{
    \clocksi | \clocktime | \stk
            | \pruntrans{\plookup{\var}{\expr}; \trans}{\fp,\emptyset}{\clocktime',\clockshard} 
            | \lbCLOCKOp{\opR(\key,\val),\clocktime'} ->
    \clocksi | \clocktime |  \stk\UpdateFunc{\var -> \val}
            | \pruntrans{\trans}{\fp,\emptyset}{\clocktime',\clockshard} }
}
\and
\inferrule[\rCLOCKReadShard]{ 
    \key = \EvalE{\expr} 
    \\
    \Forall{l,\val} (l,\key,\val) \notin \fp
    \\
    \clockshard' = \ShardOf(\clocksi,\key)
    \\
    (\clockkvs,\clocktime'') = \clocksi(\clockshard')
    \\
    \clocktime'' > \clocktime'
    \\
    \clockverlist = \Set{\clockver' | \Exists{\idx} 
                \clockkvs(\key,\idx) = \clockver'
                \land \TimeOf(\clockver') \leq \clocktime'  }
    \\
    \Forall{\clockver \in \clockverlist} \StateOf(\clockver) = \clocksicommitted
    \\
    (\key,\val,\stub) = \Max(\clockverlist)
}{%
    \ToClockCmd{
    \clocksi | \clocktime | \stk
            | \pruntrans{\plookup{\var}{\expr}; \trans}{\fp,\emptyset}{\clocktime',\clockshard} 
            | \lbCLOCKOp(\clockshard'){\opR(\key,\val),\clocktime'} ->
    \clocksi | \clocktime |  \stk\UpdateFunc{\var -> \val}
            | \pruntrans{\trans}{\fp \AddOp \opR(\key,\val),\emptyset}{\clocktime',\clockshard} }
}
\end{mathpar}

\caption{The semantics of \( \pclockread \)}
\label{fig:clock-read-rule}
\end{subfigure}

\hrulefill

\caption{Clock-SI: transactional write and read}
\label{fig:clock-si-write-value}
\label{fig:clock-si-read-value}
\end{figure}


For any look-up command of the form \( \plookup{\var}{\expr}\),
the coordinator calls \( \pclockread \) defined in \cref{lst:clock-si-read-value}.
If the key has been updated or read in the transaction, 
the coordinator directly reads from the read-write set of the transaction,
that is, \cref{line:clock-read-from-write,line:clock-read-from-read} respectively in \cref{lst:clock-si-read-value}.
Rule \( \rCLOCKReadLocal \) in \cref{fig:clock-read-rule} captures the semantics of local read.
Otherwise, the transaction reads from the shard that contains the key via function call \( \CodeFont{readFromShard}\) defined in \cref{lst:clock-si-read-value}.
Once receiving the read request,
the remote shard, first, will wait until the local time is greater than the transaction snapshot time (\cref{line:clock-remote-read-wait}),
and then wait until all the versions with time-stamps smaller than the snapshot time are committed successfully (\cref{line:clock-remote-read-wait-commit}).
If both conditions are satisfied,
the transaction replies with the latest version before the snapshot time (\cref{line:clock-remote-latest-version}).
The coordinator then will add the version into the read-write set (\cref{line:clock-read-write-remote}).
Rule \( \rCLOCKReadShard \) rule in \cref{fig:clock-read-rule} captures the semantics or remote read.
The premise \(\clocktime'' > \clocktime'\) states that the snapshot time precedes the local shard time.
The premise \( \StateOf(\clockver) = \clocksicommitted  \) states that
any version \( \clockver \in \clockverlist \) that has time-stamp smaller than the snapshot time \( \clocktime' \) have committed successfully.
The last premise fetches the version with the maximum time-stamp in \( \clockverlist\),
that is, \( (\key,\val,\stub) = \Max(\clockverlist) \).
This version is then added into the fingerprint \( \fp \).
Note that, both read and write steps do not change the client local time.

\begin{figure}

\begin{subfigure}{\textwidth}
\begin{lstlisting}
commitTrans( trans )
    prep = [];
    snapshotTime = trans.snapshotTime;
    for (w,k,v) in trans.read-write-set { 
        send ``prepare (k,v,snapshotTime)'' to shardOf(k); (* \label{line:asyn-send-prepare}*)
        wait ``(k,v,t)''; (* \label{line:asyn-wait-prepare}*)
        prep = prep + (k,v,t); (* \label{line:clock-collect-reply}*)
    }

    commitTime = maxTime(prep) (* \label{line:clock-final-commit-time}*)

    for (k,v,t) in prep {
        send ``commit (k,v,commitTime)'' to shardOf(k);
    }

On receive ``prepare (k,v,snapshotTime)'' 
    wait snapshotTime < getShardClockTime(); 
    if noConcurrentWriteSince(k,snapshotTime) { (* \label{line:clock-no-conflict-write}*)
        preparedTime = getShardClockTime(); (* \label{line:clock-assign-prepare-time}*)
        log ``(k,v,preparedTime,prepared)'' in preparation set; (* \label{line:clock-log-prepare-time}*)
        send ``(k,v,preparedTime)''; (* \label{line:clock-echo-prepare-time}*)
    } else {
        // either abort or restart this transaction entirely.
        ...
    }

On receive ``commit (k,v,commitTime)'' 
    insert(k,v,commitTime,committed); (* \label{line:clock-insert-version}*)
\end{lstlisting}

\caption{The reference implementation of transaction commit}
\label{lst:clock-si-commit}
\end{subfigure}

\hrulefill

\begin{subfigure}{\textwidth}
\begin{mathpar}
\inferrule[\rCLOCKPrepare]{ 
    \opW(\key,\val) \in \fp
    \\
    \clockshard = \ShardOf(\clocksi,\key)
    \\
    (\clockkvs,\clocktime'') = \clocksi(\clockshard)
    \\
    \clocktime' < \clocktime''
    \\
    \clockverlist = \clockkvs(\key)
    \\
    \Forall{\idx} 0 \leq \idx < \Abs{\clockverlist} \implies \TimeOf(\clockverlist(\idx)) < \clocktime'
    \\
    \clocksi' = \clocksi\UpdateFunc{\clockshard -> (\clockverlist \ListConcat \List{(\val,\clocktime'',\clocksiprepared)}, \clocktime'') }
}{%
    \ToClockCmd{
    \clocksi | \clocktime | \stk
            | \pruntrans{\pskip}{\fp,\clockbuffer}{\clocktime',\clockshard} 
            | \lbCLOCKOp(\clockshard'){\opP(\key,\val),\clocktime''} -> \ }
   \ToClockCmd{ \clocksi' | \clocktime |  \stk
            | \pruntrans{\pskip}{
                    \fp \setminus \Set{\opW(\key,\val) }
                    , \clockbuffer \uplus \Set{(\key,\Abs{\clockverlist},\clocktime'') }
                }{\clocktime',\clockshard} } 
}
\and
\inferrule[\rCLOCKCommit]{ 
    \Forall{\key,\val} \opW(\key,\val) \notin \fp
    \\
    \clocktime'' = \CLOCKMaxTime(\clockbuffer)
    \\
    \clocksi'' = \CLOCKUpdate(\clocksi,\clockbuffer,\clocktime'')
    %\\
    %\clockshardset = \Set{\ShardOf(\clocksi,\key') | (\key',\stub,\stub) \in \clockbuffer }
}{%
    \ToClockCmd{
    \clocksi | \clocktime | \stk
            | \pruntrans{\pskip}{\fp,\clockbuffer}{\clocktime',\clockshard} 
            | \lbCLOCKEnd{\clocktime''} ->
    \clocksi'' | \clocktime'' |  \stk
            | \pskip }
}
\end{mathpar}

\caption{The semantics of \( \pclockcommit \)}
\label{fig:clock-commit}
\end{subfigure}

\hrulefill

\caption{Clock-SI: transaction commit}
\label{fig:clock-si-commit}
\end{figure}


When the transactional command reaches the end, that is, \( \pskip \),
the coordinator commits all the write operations in the final read-write set 
to the appropriate shards using a two-phase commit protocol.
The reference implementation is given in \cref{lst:clock-si-commit}.
In the first phase, 
for every write operation, \verb|(w,k,v)|, in the write-read set, \verb|trans.read-write-set|,
the coordinator asynchronously sends the new value \( \CodeFont{\val} \),
and the snapshot time \( \CodeFont{snapshotTime} \) of the transaction,
that is, a message of the form ``\(\CodeFont{prepare} \ (\key,\val,\clocktime)\)''
to the shard that contains \( \CodeFont{\key} \) (\cref{line:asyn-send-prepare}), 
and awaiting a reply (\cref{line:asyn-wait-prepare}).
Upon receiving the preparation request, 
the remote shard checks if there is concurrent write since the snapshot time \( \CodeFont{snapshotTime} \).
if there is no conflict write (\cref{line:clock-no-conflict-write}),
the shard assigns the shard local time as the \emph{prepared time} (\cref{line:clock-assign-prepare-time}),
adds the new value \( \CodeFont{\val} \) of \( \CodeFont{\key} \) together with the prepared time to the preparation set (\cref{line:clock-log-prepare-time}), 
and echoes the message but modifies the time to be the prepared time (\cref{line:clock-echo-prepare-time}).
Note that versions in the preparation set are visible to other.
If a transaction \( \txid\) wants to read a key, and 
if there is a prepared version for the key of which the preparation time is smaller 
than the snapshot time of the transaction,
the transaction \( \txid\) must wait until this entry has been committed or aborted.
Also, prepared versions will stop other transactions committing to the same key.

Rule \( \rCLOCKPrepare \) in \cref{fig:clock-commit} captures that
the shard \( \clockshard \) receives a new value \( \val \) of a key \( \key \).
The premise \(\clocktime' < \clocktime'' \) requires 
that the local time of the shard \( \clockshard \) 
be greater than the snapshot time \( \clocktime' \).
The second last premise, \( \TimeOf(\clockverlist(\idx)) < \clocktime' \), states no conflict write: that is, 
existing versions of key \( \key \) must have smaller time-stamp than the snapshot time \( \clocktime' \).
In our semantics of Clock-SI, we do not explicitly have preparation set,
but labels versions with their states.
Hence, we add the version into the store and 
labelled them with the preparation time \( \clocktime'' \) and state tag \( \clocksiprepared \) 
(the last premise of \( \rCLOCKPrepare \)).
Finally, this new version is erased from the fingerprint \( \fp \),
and added into the buffer, updating the buffer to \( \clockbuffer \uplus \Set{(\key,\Abs{\clockverlist},\clocktime'') } \).
Note that the label of this transition tracks the preparation time of this version, 
instead of the snapshot time.

In the second phase of commit,
the coordinator collects all the replies (\cref{line:clock-collect-reply}),
assigns the final commit time \( \CodeFont{commitTime} \) of this transaction 
as the maximum of all the preparation times (\cref{line:clock-final-commit-time}).
The coordinator, then, send a message of the form \verb|commit (k,v,commitTime)| 
to the shards that contain versions of this transaction.
Upon receiving the commit time \(\clocktime'\), 
the shard insert the versions at \(\clocktime'\) to their local stores (\cref{line:clock-insert-version}).
For brevity, the entire second phase 
is captures by one step transition, rule \( \rCLOCKCommit \), in the semantics.
The actual commit time is given by \( \CLOCKMaxTime \) function in the premise of \( \rCLOCKCommit \).

\begin{definition}[\(\CLOCKMaxTime\) function]
Given a set of preparation buffer \( \clockbuffer \),
\( \CLOCKMaxTime \) is defined by 
\( \CLOCKMaxTime(\clockbuffer) \FuncDef 
        \Max(\Set{\clocktime | (\stub,\stub,\clocktime) \in \clockbuffer})\).
\end{definition}

The last premise, \( \CLOCKUpdate(\clocksi,\clockbuffer,\clocktime'') \),
inserts all the preparation versions.
This function alters the time-stamp of versions included in the buffer \( \clockbuffer \)
to the actual commit time \( \clocktime'' \), 
and changes the state tag to  \( \clocksicommitted \).

\begin{definition}[\(\CLOCKUpdate\) function]
Given a Clock-SI machine state \( \clocksi \),
a set of preparation buffer \( \clockbuffer \),
and a time \( \clocktime \),
\( \CLOCKUpdate \) is defined by:
\begin{align*}
\CLOCKUpdate(\clocksi,\emptyset,\clocktime) & \FuncDef \clocksi
\\ \CLOCKUpdate(\clockbuffer \uplus \Set{(\key,\idx,\stub),\clocktime}) 
        & \FuncDef 
        \begin{array}[t]{l}
        \Let \clockshard = \ShardOf(\clocksi,\key), \quad
        (\clockkvs,\clocktime') = \clocksi(\clockshard),
        \\ \AND \clockkvs' = \CLOCKUpdateKVS(\clockkvs,\key,\idx,\clocktime) \In
        \\ \CLOCKUpdate(\clocksi,\clockbuffer)\UpdateFunc{\clockshard -> (\clockkvs',\clocktime') }
        \end{array}
\end{align*}
where \( \CLOCKUpdateKVS \) is defined by:
\[
\CLOCKUpdateKVS(\clockkvs,\key,\idx,\clocktime) 
\FuncDef 
    \begin{array}[t]{l}
    \Let \clockver = \clockkvs(\key), \quad (\val,\stub,\stub) = \clockver(n)
    \\ \AND \clockver' = \clockver\UpdateFunc{\idx -> (\val,\clocktime,\clocksicommitted) }
    \In \clockkvs' = \clockkvs'\UpdateFunc{\key -> \clockver'}  .
    \end{array}
\]
\end{definition}


The operational semantics of Clock-SI is a standard interleaving semantics,
depicted in \cref{fig:clock-si-prog},
where a client takes a step in turn, captured by \( \rCLOCKTrans\).   
In Clock-SI, the shard local times are the physical times.
To model the physical time ticks, 
a shard can arbitrarily advance its local time, captured by \( \rCLOCKShardTick\).

\begin{figure}

\begin{mathpar}
\inferrule[\rCLOCKTrans]{
    \clocktime = \clockclientenv(\cl)
    \\
    \stk = \clenv(\cl)
    \\
    \clockruncmd = \clockrunprog(\cl)
    \\
    \ToClockCmd{ \clocksi | \clocktime | \stk | \clockruncmd | \lb
            -> \clocksi' | \clocktime' | \stk' | \clockruncmd' }
}{%
    \ToClockProg{
     \clocksi | \clockclientenv | \clenv | \clockrunprog | \lb ->
     \clocksi' | \clockclientenv\UpdateFunc{\cl -> \clocktime} 
                | \clenv\UpdateFunc{\cl -> \stk'}
                | \clockrunprog\UpdateFunc{\cl -> \clockruncmd'} }
}
\and
\inferrule[\rCLOCKShardTick]{
    (\clockkvs,\clocktime) = \clocksi(\clockshard)
    \\
    \clocksi' = \clocksi\UpdateFunc{\clockshard -> (\clockkvs, \clocktime + \clocktime') }
}{%
    \ToClockProg{ \clocksi | \clockclientenv | \clenv 
                | \clockrunprog | \lbCLOCKTick{\clocktime + \clocktime'} ->
                        \clocksi' | \clockclientenv | \clenv | \clockrunprog }
}
\end{mathpar}

\hrulefill

\caption{Clock-SI: semantics for programs}
\label{fig:clock-si-prog}
\end{figure}


Note that a commit may fail in the first phase due to conflict writes.
However, our reference operational semantics only models the successful cases,
given that we focus on client observable behaviour.
Therefore the set of Clock-SI traces only contains successful traces 
in that all clients reach the end.

\begin{definition}[Clock-SI traces]
The set of \emph{Clock-SI traces}, \( \CLOCKTraces \ni \clocktrc\),
is defined by
\[
\CLOCKTraces \TypeDef 
    \Set{\ToClockProg{ \Tuple{ \clockconfinit , \clenv_0 } | \prog | \stub | * 
                -> \Tuple{ \clockconf , \clenv } | \prog' } 
            | \Exists{\clockclientenvinit} \clockconfinit \in \CLOCKConfInits
                (\stub,\clockclientenvinit) = \clockconfinit 
                \\ {} \land \Forall{\cl \in \Dom(\prog) }
                \cl \in 
                            \Dom(\clockclientenvinit) \cap \Dom(\clenv) 
                \\ {} \land \prog'(\cl) = \pskip} .
\]
Two traces, \( \clocktrc, \clocktrc'\), are equivalent,
written \( \clocktrc \clocktrceq \clocktrc' \), if and only if
\[
    \clocktrc \clocktrceq \clocktrc' 
    \PredDef 
    \begin{multlined}[t]
        \Forall{\clocksi,\clocksi'} 
        (\clocksi,\stub) = \LastConf(\clocktrc)
        \land (\clocksi',\stub) = \LastConf(\clocktrc') 
        \\ \implies \Dom(\clocksi) = \Dom(\clocksi')
        \land \Forall{\clockshard \in \Dom(\clocksi),\clockkvs,\clockkvs'}
        \\ \left( 
                \clocksi(\clockshard) = (\clockkvs,\stub) 
                \land \clocksi(\clockshard') = (\clockkvs',\stub) 
                \implies \clockkvs = \clockkvs'
            \right) .
    \end{multlined}
\]
\end{definition}

Two machine states are equivalent if for each shard \( \clockshard \),
the key-value stores of \( \clockshard \) have the same state.
Two Clock-SI traces are equivalent if and only if
the final machine states of the two traces are equivalent.
This notation of \emph{trace equivalence} is important for the verification of Clock-SI protocol.
