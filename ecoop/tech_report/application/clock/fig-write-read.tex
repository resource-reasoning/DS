\begin{figure}[!p]

\begin{subfigure}{\textwidth}
\begin{lstlisting}
transWrite( trans, k, v )
    trans.read-write-set = trans.read-write-set \ (w,k,-);
    trans.read-write-set = trans.read-write-set + (w,k,v);
\end{lstlisting}

\caption{The reference implementation of transactional write}
\label{lst:clock-si-write-value}
\end{subfigure}

\hrulefill


\begin{subfigure}{\textwidth}
\begin{mathpar}
\inferrule[\rCLOCKWrite]{
    \key = \EvalE{\expr_1}
    \\ 
    \val = \EvalE{\expr_2}
}{%
    \ToClockCmd{
    \clocksi | \clocktime | \stk
            | \pruntrans{\pmutate{\expr_1}{\expr_2}; \trans}{\fp,\emptyset}{\clocktime',\clockshard} 
            | \lbCLOCKOp{\opW(\key,\val),\clocktime'} ->
    \clocksi | \clocktime |  \stk
            | \pruntrans{\trans}{\fp \AddOp \opW(\key,\val),\emptyset}{\clocktime',\clockshard} }
}
\end{mathpar}

\caption{The semantics of \( \pclockwrite \)}
\label{fig:clock-write-rule}
\end{subfigure}

\hrulefill

\begin{subfigure}{\textwidth}
\begin{lstlisting}
transRead( trans, k )
    if ( (w,k,v) in trans.read-write-set ) return v; (* \label{line:clock-read-from-write} *)
    else if ( (r,k,v) in trans.read-write-set ) return v; (* \label{line:clock-read-from-read} *)
    else {
        v = readFromShard(shard, trans.snapshotTime);
        trans.read-write-set = trans.read-write-set + (r,k,v); (* \label{line:clock-read-write-remote} *)
        return v;
    }

readFromShard(shard, snapshotTime, k)
    wait snapshotTime < getShardClockTime(); (* \label{line:clock-remote-read-wait} *)
    i = 0;
    for ver of getStore() 
        if (ver.key = k and ver.time <= snapshotTime) { 
            wait until ver.state == committed; (* \label{line:clock-remote-read-wait-commit} *)
            if(i <= ver.time) i = ver.time; (* \label{line:clock-remote-latest-version} *)
        }
    return getValue(k,i); (* \label{line:clock-remote-latest-version} *)
\end{lstlisting}

\caption{The reference implementation of transactional read}
\label{lst:clock-si-read-value}
\end{subfigure}

\hrulefill

\begin{subfigure}{\textwidth}
\begin{mathpar}
\inferrule[\rCLOCKReadLocal]{ 
    \key = \EvalE{\expr} 
    \\
    \opW(\key,\val) \in \fp \lor (\opW(\key,\val) \notin \fp \land \opR(\key,\val) \in \fp)
}{%
    \ToClockCmd{
    \clocksi | \clocktime | \stk
            | \pruntrans{\plookup{\var}{\expr}; \trans}{\fp,\emptyset}{\clocktime',\clockshard} 
            | \lbCLOCKOp{\opR(\key,\val),\clocktime'} ->
    \clocksi | \clocktime |  \stk\UpdateFunc{\var -> \val}
            | \pruntrans{\trans}{\fp,\emptyset}{\clocktime',\clockshard} }
}
\and
\inferrule[\rCLOCKReadShard]{ 
    \key = \EvalE{\expr} 
    \\
    \Forall{l,\val} (l,\key,\val) \notin \fp
    \\
    \clockshard' = \ShardOf(\clocksi,\key)
    \\
    (\clockkvs,\clocktime'') = \clocksi(\clockshard')
    \\
    \clocktime'' > \clocktime'
    \\
    \clockverlist = \Set{\clockver' | \Exists{\idx} 
                \clockkvs(\key,\idx) = \clockver'
                \land \TimeOf(\clockver') \leq \clocktime'  }
    \\
    \Forall{\clockver \in \clockverlist} \StateOf(\clockver) = \clocksicommitted
    \\
    (\key,\val,\stub) = \Max(\clockverlist)
}{%
    \ToClockCmd{
    \clocksi | \clocktime | \stk
            | \pruntrans{\plookup{\var}{\expr}; \trans}{\fp,\emptyset}{\clocktime',\clockshard} 
            | \lbCLOCKOp(\clockshard'){\opR(\key,\val),\clocktime'} ->
    \clocksi | \clocktime |  \stk\UpdateFunc{\var -> \val}
            | \pruntrans{\trans}{\fp \AddOp \opR(\key,\val),\emptyset}{\clocktime',\clockshard} }
}
\end{mathpar}

\caption{The semantics of \( \pclockread \)}
\label{fig:clock-read-rule}
\end{subfigure}

\hrulefill

\caption{Clock-SI: transactional write and read}
\label{fig:clock-si-write-value}
\label{fig:clock-si-read-value}
\end{figure}
