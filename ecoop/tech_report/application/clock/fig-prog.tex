\begin{figure}

\begin{mathpar}
\inferrule[\rCLOCKTrans]{
    \clocktime = \clockclientenv(\cl)
    \\
    \stk = \clenv(\cl)
    \\
    \clockruncmd = \clockrunprog(\cl)
    \\
    \ToClockCmd{ \clocksi | \clocktime | \stk | \clockruncmd | \lb
            -> \clocksi' | \clocktime' | \stk' | \clockruncmd' }
}{%
    \ToClockProg{
     \clocksi | \clockclientenv | \clenv | \clockrunprog | \lb ->
     \clocksi' | \clockclientenv\UpdateFunc{\cl -> \clocktime} 
                | \clenv\UpdateFunc{\cl -> \stk'}
                | \clockrunprog\UpdateFunc{\cl -> \clockruncmd'} }
}
\and
\inferrule[\rCLOCKShardTick]{
    (\clockkvs,\clocktime) = \clocksi(\clockshard)
    \\
    \clocksi' = \clocksi\UpdateFunc{\clockshard -> (\clockkvs, \clocktime + \clocktime') }
}{%
    \ToClockProg{ \clocksi | \clockclientenv | \clenv 
                | \clockrunprog | \lbCLOCKTick{\clocktime + \clocktime'} ->
                        \clocksi' | \clockclientenv | \clenv | \clockrunprog }
}
\end{mathpar}

\hrulefill

\caption{Clock-SI: semantics for programs}
\label{fig:clock-si-prog}
\end{figure}
