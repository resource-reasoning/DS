By \cref{thm:clock-si-normalised-trace}, it is enough to only consider annotated normalised traces.
We verify Clock-SI protocol by first converting normalised Clock-SI traces to our kv-store program traces.
We then show that these kv-store  program traces can be obtained under execution test for snapshot isolation \( \et[\PSI] \).

\begin{definition}[Conversion of Clock-SI traces to kv-store  program traces]
Given an annotated normalised Clock-SI trace \( \clockexttrc \),
the kv-store  program trace induced by the trace, written \( \ClockToKVTrace(\clockexttrc) \),
is defined in \cref{fig:def-clock-kv-trace}.
The auxiliary predicate \( \CLOCKAtomic(\clockexttrc,\cl,\clockshard,\clocktime)\),
and the auxiliary functions for extracting static program, fingerprint and view,
that is, \( \ClockStaticProg \), \( \ClockFp\) and \( \ClockView\), are defined in \cref{fig:def-clock-kv-trace}.
\end{definition}

\begin{figure}[hp]
\centering
\rotatebox{270}{\(%
\centering
\begin{aligned}
\COPSToKVS(\copsconfinit,\copsprog) & \FuncDef \kvsinit ,
\\[3pt]
\\ \COPSToKVS(\ToCOPSProg{ \copstrc | \lb -> \cops | \copsctxenv | \copsrunprog })  
            & \FuncDef \CodeFont{let} \ \kvs =  \COPSToKVS(\copstrc) \ \CodeFont{in} 
\\ \multicolumn{2}{l}{\( \qquad \qquad \begin{cases}
               \kvs\UpdateFunc{\key -> \kvs(\key) \ListConcat \List{(\val,\copstxid[\cl][\repl](n,0),\emptyset)}}
                    & \If \lb = \lbCOPSWrite{\opW(\key,\val), \copstxid[\cl][\repl](n,0), \copsctx }  ,
            \\ \kvs\UpdateFunc{\key -> \kvs(\key)\UpdateFunc{\idx
                                    -> \Tuple{\val,\copstxid,\txidset \uplus \copstxid[\cl][\repl](n,m)}}} 
            &  \If \lb = \Tuple{{\lbCOPSRefetch{\opR(\key,\val), \copstxid, \copsctx}}, \copstxid[\cl][\repl](n,m)}  ,
                        \land \kvs(\key)\Proj{\idx} = (\val,\copstxid,\txidset) ,
            \\ \kvs & \ow .
        \end{cases} \) }
\\[7pt]\hline
\\[5pt]
\\ \multicolumn{2}{l}{ \( \COPSToKVTrace(\copsconfinit,\copsprog) \FuncDef 
   \CodeFont{let} \ \stk_0 = \lambda \var \in \ldotp 0 \ \CodeFont{in}
    \ToProg[\TOP]{ \COPSToKVS(\copsconfinit,\copsprog) | \COPSViews(\copsconfinit,\copsprog) 
                | \lambda \cl \in \Dom(\copsprog) \ldotp \stk_0 | \COPSToKVProg(\copsprog) }
    \) }
\\[3pt]
\\ \multicolumn{2}{l}{ \(
            \COPSToKVTrace(\ToCOPSProg{ \copsexttrc | \lb 
                        -> \copsexttrc' | \lb' 
                        -> \cops | \copsctxenv | \copsrunprog  } ) \FuncDef 
                    \begin{array}[t]{@{}l@{}}
                    \CodeFont{let} \ (\kvs,\vienv,\clenv), \prog = \LastConf(\COPSToKVTrace(\copsexttrc)) ,
                    \\ \phantom{\CodeFont{let}} \ \kvs' = \COPSToKVS(\ToCOPSProg{ \copsexttrc | \lb 
                                                                -> \copsexttrc' | \lb' 
                                                                -> \cops | \copsctxenv | \copsprog  }),
                    \\ \phantom{\CodeFont{let}} \ \vienv' = \COPSViews(\cops,\copsctxenv) ,
                    \ \CodeFont{and} \ \prog' = \COPSToKVProg(\copsrunprog)  \ \CodeFont{In}
                \\ \begin{cases}
                    \ToProg[\TOP]{ \COPSToKVTrace(\copsexttrc) | \lbTrans{\vienv(\cl), \fp } -> \kvs' | \vienv' | \clenv | \prog' }
                            & \If \dagger ,
                    \\ \ToProg[\TOP]{ \COPSToKVTrace(\copsexttrc) | \lbTrans{\vienv'(\cl), \fp} -> \kvs' | \vienv' | \clenv' | \prog' }
                            & \If \ddagger ,
                    \\ \COPSToKVTrace(\copsexttrc) & \text{other cases} \ \lb = \lb' \ \text{and} \ \copsexttrc = \emptyset .
                \end{cases} 
                    \end{array} \) }
\\[3pt]
\\ \dagger & \equiv \lb = \lb' = \lbCOPSWrite{\opW(\key,\val), \copstxid, \copsctx } 
                \land \fp = \Set{\opW(\key,\val)}
\\ \ddagger & \equiv 
\begin{multlined}[t]
\lb = \lbCOPSRefetch{\opR(\key_0,\val_0), \copstxid_0, \copsctx_0 },\copstxid 
\land \copsexttrc' = \ToCOPSProg{ \stub 
            | \lbCOPSRefetch{\opR(\key_1,\val_1), \copstxid_1, \copsctx_1 },\copstxid
            -> \cdots  | \lbCOPSRefetch{\opR(\key_n,\val_n), \copstxid_n, \copsctx_n },\copstxid
            -> \stub } 
\\ {} \land \lb' = \lbCOPSFinishRead{ \copsctx' },\copstxid 
\land \fp = \Set{\opR(\key_0,\val_0), \cdots, \opR(\key_n,\val_n)}
\land \clenv' = \clenv\UpdateFunc{\var_0 -> \val_0}\cdots\UpdateFunc{\var_n -> \val_n}
\end{multlined}
\end{aligned}
\)}
\\[3pt]

\hrulefill

\caption{Definitions of \( \COPSToKVS \) and \( \COPSToKVTrace \)}
\label{fig:def-cops-kv-trace}
\end{figure}


Given an annotated normalised  Clock-SI trace \( \clockexttrc \),
We know that the read, write and preparation steps of a transaction 
can only be interfered by time-tick or snapshot steps, captured by predicate \( \CLOCKAtomic \).
The predicate \( \CLOCKAtomic(\clockexttrc',\cl,\clockshard,n) \)  states that 
the trace segment \( \clockexttrc' \) contains all the read, write and preparation steps
of the same transaction from the client \( \cl \) 
with the coordinator \( \clockshard \) and the commit time \( n\).
Function \( \ClockToKVTrace \) constructs a kv-store program trace from \( \clockexttrc \) by inductively converting
each atomic trace segment \( \clockexttrc' \) 
into a single transition of the form \(\lbTrans{\vi,\fp} \).
First, the transactional identifier \( \clocktxid[\cl](n,m) \) in the kv-store trace is annotated 
with the client \( \cl \) that commits the transaction, and 
the commit time \( n \) and snapshot time  \( m \) of the transaction.

\begin{definition}[Clock-SI transaction identifiers]
The set of \emph{Clock-SI transaction identifiers}, \( \ClockTxIDs \ni \clocktxid \),
is defined by:
\[
    \ClockTxIDs \TypeDef \Set{\clocktxid[\cl](n,m) | \cl \in \Clients \land n,m \in \CLOCKTimes} .
\]
Given two identifiers \( \clocktxid[\cl](n,m), \clocktxid[\cl'](n',m') \),
the order \( \clocktxid[\cl](n,m) \clocktxidleq \clocktxid[\cl'](n',m') \) is defined by
\[
    \clocktxid[\cl](n,m) \clocktxidleq \clocktxid[\cl'](n',m') \TypeDef
    n \leq n' \lor (n = n' \land m \leq m') .
\]
\end{definition}

Note that it is useful to annotate the transaction identifiers with the snapshot times,
however, it does not affect the encoding.
Clock-SI identifiers in a store must be unique,
because transactions for a client must have unique commit times.
The detail is given in \cref{prop:clock-si-unique-txid} on page \pageref{sec:proof-clock-si-txid-unique}.

\begin{toappendix}
\label{sec:proof-clock-si-txid-unique}
\begin{proposition}[Clock-SI unique transactional identifiers]
\label{prop:clock-si-unique-txid}
Assume an annotated normalised Clock-SI trace \( \clockexttrc \),
and the kv-store trace induced by the Clock-SI trace \( \progtrc = \ClockToKVTrace(\clockexttrc) \).
Let \( (\clocksi,\clockclientenv,\clenv,\clockrunprog) = \LastConf(\clockexttrc) \)
and \( (\kvs,\vienv,\clenv,\prog) = \LastConf(\progtrc) \).
The transaction identifiers in \( \kvs \) are unique in that:
\begin{Formulae}
& \begin{Formula}
    \nonumber
\end{Formula}
\\ & \begin{Formula}
    \land \Forall{\clockexttrc' \in \clockexttrc | \cl, \clockshard}
    \left( \CLOCKAtomic(\clockexttrc',\cl,\clockshard,n)
    \implies \clocktxid[\cl](n,\stub) \in \kvs \right) 
    \label{equ:clock-si-txid-unique}
\end{Formula}
\\ & \begin{Formula}
    \land \Forall{\clocktxid[\cl](n,m) \in \kvs }
    \Exists{ \clockexttrc' \in \clockexttrc | \cl, \clockshard}
    \\ \left( \CLOCKAtomic(\clockexttrc',\cl,\clockshard,n)
    \land \Forall{ \clockexttrc'' \in \clockexttrc }
    \left( \CLOCKAtomic(\clockexttrc'',\cl,\clockshard,n)
    \implies \clockexttrc' = \clockexttrc'' \right) \right) .
    \label{equ:clock-kv-store-txid-unique}
\end{Formula}
\end{Formulae}
\end{proposition}
\begin{proof}
By the definition of \( \ClockToKVTrace \),
it is easy to see that \cref{equ:clock-si-txid-unique}
and 
\[
    \Forall{\clocktxid[\cl](n,m) \in \kvs }
    \Exists{ \clockexttrc' \in \clockexttrc | \cl, \clockshard}
    \CLOCKAtomic(\clockexttrc',\cl,\clockshard,n) .
\]
We now prove that 
\[
    \Forall{\clocktxid[\cl](n,m) \in \kvs |
    \clockexttrc', \clockexttrc'' \in \clockexttrc | \cl, \clockshard}
    \CLOCKAtomic(\clockexttrc',\cl,\clockshard,n)
    \land \CLOCKAtomic(\clockexttrc'',\cl,\clockshard,n)
    \implies \clockexttrc' = \clockexttrc'' .
\]
The above can be derived from the follows,
\begin{Formulae}*
\begin{Formula}
    \Forall{ \clockexttrc', \clockexttrc'' \in \clockexttrc 
        | \cl \in \Clients | \clockshard, \clockshard' \in \CLOCKShards
        | n,n' \in \Nat }
    \\ \CLOCKAtomic(\clockexttrc',\cl,\clockshard,n)
    \land \CLOCKAtomic(\clockexttrc'',\cl,\clockshard',n')
    \\ \clockexttrc = \ToClockProg{\cdots | \stub -> \clockexttrc' 
            | \stub -> \cdots | \stub -> \clockexttrc' | \stub -> \cdots}
    \implies n < n' .
\end{Formula}
\end{Formulae}
The above can be directly derived by \cref{lem:clock-si-local-time-mono}:
the snapshot time must be strictly greater then the client local time,
and the commit time must be greater or equal to the snapshot time.
Therefore, \cref{equ:clock-kv-store-txid-unique} holds.
\end{proof}
%\begin{proofsketch}
%By \cref{lem:clock-si-local-time-mono}, the snapshot time must be strictly 
%greater then the client local time (\( \rCLOCKStart \)),
%and the commit time must be greater or equal to the snapshot time (\( \rCLOCKCommit \)).
%Then, by the definition of \( \ClockToKVTrace \),
%it is easy to prove \cref{equ:clock-si-txid-unique,equ:clock-kv-store-txid-unique}.
%The full detail is given in \cref{sec:proof-clock-si-txid-unique}
%on page \pageref{sec:proof-clock-si-txid-unique}.
%\end{proofsketch}
\end{toappendix}


\begin{toappendix}
\label{sec:proof-client-local-time-mono}
\begin{lemma}[Monotonic Clock-SI client local times]
\label{lem:clock-si-local-time-mono}
Client local times monotonically increases:
given Clock-SI databases \( \clocksi, \clocksi'\), local time environments \( \clockclientenv, \clockclientenv' \),
local stack environments \( \clenv, \clenv'\), runtime programs \( \clockrunprog, \clockrunprog' \),
and label \( \lb \),
\begin{Formulae}*
& \begin{Formula}
\ToClockProg{\clocksi | \clockclientenv | \clenv | \clockrunprog | \lb 
            -> \clocksi' | \clockclientenv' | \clenv' | \clockrunprog' }
\land \Forall{\cl \in \Dom(\clockclientenv) | \clockshard \in \Dom(\clocksi) | \clocktime }
    \clockclientenv(\cl) \leq \clockclientenv'(\cl)
\\ {} \land \left( 
    \lb = \lbCLOCKStart{\clocktime} \implies 
    \clockclientenv(\cl) < \clockclientenv'(\cl) .
\right)
\end{Formula}
\end{Formulae}
\end{lemma}
\begin{proof}
We perform case analysis on the label \( \lb \).
\begin{enumerate}
\Case{\( \lb = \lbCLOCKStart{\clocktime} \)}
    By the premise of \( \rCLOCKStart \), we have 
    \( \clockclientenv(\cl) < \clockclientenv'(\cl) \).
    Other clients \( \cl' \) such that \( \cl' \neq \cl \), remains unchanged,
    therefore \( \clockclientenv(\cl') = \clockclientenv'(\cl') \).
\Case{\( \lb = \lbCLOCKOp{(l,\key,\val),\clocktime} \) with \( l \in \Set{\opW,\opR,\opP}\)}
    By \( \rCLOCKWrite \), \( \rCLOCKReadLocal \), \( \rCLOCKReadShard \) and \( \rCLOCKPrepare\), 
    the client local environment remains unchanged,
    therefore \( \clockclientenv(\cl) = \clockclientenv'(\cl) \) for all clients \( \cl \).
\Case{\( \lb = \lbCLOCKEnd{\clocktime} \) or \( \lb = \lbCLOCKTick{\clocktime}\)}
    By \( \rCLOCKCommit \) and \( \rCLOCKShardTick \), 
    the client local environment remains unchanged,
    therefore \( \clockclientenv(\cl) = \clockclientenv'(\cl) \) for all clients \( \cl \). \qedhere
\end{enumerate}
\end{proof}
%\begin{proofsketch}
%It is trivial by the rules presented in 
%\cref{fig:clock-si-transaction-start,fig:clock-si-write-value,fig:clock-si-read-value,fig:clock-si-commit}.
%The full detail is given in \cref{sec:proof-client-local-time-mono}
%on page \pageref{sec:proof-client-local-time-mono}
%\end{proofsketch}
\end{toappendix}

The partial client views are extracted from the client local times via function \( \ClockView \).
Given a client \( \cl \), its view, \( \vi = \ClockView(\kvs,\clocktime) \), contains all the versions 
that are committed before its local time, \( \clocktime = \clockclientenv(\cl) \).
The view \( \vi \) is well-formed, because transactions are committed at the same time to all shards.

\begin{proposition}[Well-formed Clock-SI views]
\label{prop:clock-si-view}
Assume an annotated normalised Clock-SI trace \( \clockexttrc \),
and the kv-store trace induced by the Clock-SI trace \( \progtrc = \ClockToKVTrace(\clockexttrc) \).
Let \( (\clocksi,\clockclientenv,\clenv,\clockrunprog) = \LastConf(\clockexttrc) \)
and \( (\kvs,\vienv,\clenv,\prog) = \LastConf(\progtrc) \).
The views in \( \vienv \) are well-formed:
\begin{Formulae}*
& \begin{Formula}
   \Forall{\cl \in \Dom(\vienv) | \vi}
   \vi = \vienv(\cl) \implies \WfView(\kvs,\vi) .
\end{Formula}
\end{Formulae}
\end{proposition}
\begin{proof}
By the definition of \( \CLOCKTraces \), all keys in all shards are initialised at time ZERO,
which implies \cref{equ:view-wf-initial} in \cref{def:views}.
Assume a client \( \cl \) and its view \( \vi = \vienv(\cl) \).
By the definition of \( \ClockView \) that \( \vi \) contains transactions committed
before the client local time \( \clockclientenv(\cl) \), 
it is easy to see that the view is in range, defined in \cref{equ:view-wf-with-in-range} in \cref{def:views}.
Because new versions of a transaction must commit at the same time.
This view includes all or none of the effect of transactions defined in \cref{equ:view-wf-atomic} in \cref{def:views}.
\end{proof}

Given an atomic trace segment \( \clockexttrc' \) in an annotated normalised Clock-SI trace,
The corresponding fingerprint, \( \fp \), is extracted via function \( \ClockFp \).
Given an initial fingerprint \( \fp_0 = \emptyset \), 
function \( \ClockFp(\fp_0,\clockexttrc) \) 
composites the read and write operations in \( \clockexttrc' \) 
using the fingerprint combination operator \( \AddOp \).
This final fingerprint \( \fp = \ClockFp(\fp_0,\clockexttrc) \) is well-formed,
that is,
it contains the first read of each key and the last write of each key.
Detail is given in \cref{prop:well-formed-clock-fp} on page \pageref{sec:proof:clock-fp-well-formed}.

\begin{toappendix}
\label{sec:proof:clock-fp-well-formed}
\begin{proposition}[Well-formed Clock-SI fingerprints]
\label{prop:well-formed-clock-fp}
Assume an atomic trace segment \( \clockexttrc \) such that:
for some \( \clocksi,\clockclientenv,\clenv,\clockrunprog, \trans,\clockbuffer,\clocktime,\clockshard,\cmd\),
\begin{align}
\CLOCKAtomic(\clockexttrc,\cl,\clockshard)
\land (\clocksi,\clockclientenv,\clenv,\clockrunprog) = \clockexttrc\Proj{0}
\land \clockrunprog(\cl) = \pruntrans{\trans}{\emptyset,\clockbuffer}{\clocktime,\clockshard} \pseq \cmd .
\label{equ:clock-init-fp-match-with-trace}
\end{align}
Then, the fingerprint \(\fp = \ClockFp(\emptyset,\clockexttrc)\) satisfies the follows:
for some \( \clocksi',\clockclientenv',\clenv',\clockrunprog', \trans',\clockbuffer',\clocktime,\clockshard,\cmd\),
\begin{align}
(\clocksi',\clockclientenv', \clenv',\clockrunprog') = \LastConf(\clockexttrc)
\land \clockrunprog(\cl) = \pruntrans{\trans'}{\fp,\clockbuffer'}{\clocktime,\clockshard} \pseq \cmd .
\label{equ:clock-fp-match-with-trace}
\end{align}
\end{proposition}
\begin{proof}
Fix the initial fingerprint \( \fp \),
we prove by induction on the length of trace \( \clockexttrc \).
\begin{enumerate}
\CaseBase{\( \clockexttrc = \emptyset \)}
    By the definition,
    we have \( \ClockStaticProg(\emptyset,\clockexttrc) = \emptyset \).
    It is trivial that \( \emptyset \) is a well-formed fingerprint and \cref{equ:clock-fp-match-with-trace} holds.
\CaseInd{\( \clockexttrc = \ToClockProg{\clockexttrc' | \lb -> \clocksi' | \clockclientenv' | \clenv' | \clockrunprog' } \)}
    Given the client \( \cl \),
    assume the fingerprint \( \fp' \) such that 
    \[
        (\clocksi,\clockclientenv,\clenv,\clockrunprog) = \LastConf(\clockexttrc') 
        \land \clockrunprog(\cl) = \pruntrans{\trans}{\fp',\clockbuffer}{\clocktime,\clockshard} \pseq \cmd .
    \]
    and assume fingerprint \( \fp'' \) such that 
    \[ 
        \clockrunprog'(\cl) = \pruntrans{\trans'}{\fp'',\clockbuffer'}{\clocktime,\clockshard} \pseq \cmd 
    \]
    where the free variables are existentially quantified.
    By the \ih, we immediately have \( \ClockStaticProg(\fp,\clockexttrc') = \fp' \).
    We perform case analysis on the label \( \lb \).
    \begin{enumerate}
    \Case{\(\lb = \lbCLOCKOp{\opW(\key,\val),\clocktime,\clocktime'} \)}
        By the rule \( \rCLOCKWrite \),
        we have \( \fp'' = \fp' \AddOp \opW(\key,\val)\).
        By the definition of \( \ClockStaticProg \),
        we have 
        \[ 
        \ClockStaticProg(\fp,\clockexttrc) 
                = \ClockStaticProg(\fp,\clockexttrc') \AddOp \opW(\key,\val) 
                = \fp'  \AddOp \opW(\key,\val) = \fp''
        \]
        which implies \cref{equ:clock-fp-match-with-trace}.
    \Case{\(\lb = \lbCLOCKOp{\opR(\key,\val),\clocktime,\clocktime'} \)}
        Consider if \( \fp' \) contains an entry for the key \( \key \), 
        which corresponds to the two rules \( \rCLOCKReadLocal\) and \( \rCLOCKReadShard\).
        \begin{enumerate} 
        \Case{\( \rCLOCKReadLocal\)}
            In this case, we have \( (l,\key,\val) \in \fp' \) for some \( l \in \Set{\opW,\opR} \) and value \( \val \).
            By the premise of  \( \rCLOCKReadLocal\),
            we have \( \fp' = \fp''\).
            By the definition of \( \ClockStaticProg \),
            we have 
            \[ 
            \ClockStaticProg(\fp,\clockexttrc) = \ClockStaticProg(\fp,\clockexttrc') \AddOp \opR(\key,\val) 
            = \fp' \AddOp \opR(\key,\val) = \fp''
            \]
            which implies \cref{equ:clock-fp-match-with-trace}.
        \Case{\( \rCLOCKReadShard\)}
            In this case, we have \( (l,\key,\val) \notin \fp' \) for all \( l \in \Set{\opW,\opR} \) and value \( \val \).
            By the premise of  \( \rCLOCKReadLocal\),
            we have \( \fp'' = \fp' \AddOp \opR(\key,\val) \).
            By the definition of \( \ClockStaticProg \),
            we have 
            \[ 
            \ClockStaticProg(\fp,\clockexttrc) = \ClockStaticProg(\fp,\clockexttrc') \AddOp \opR(\key,\val) 
            = \fp' \AddOp \opR(\key,\val) = \fp''
            \]
            which implies \cref{equ:clock-fp-match-with-trace}. \qedhere
        \end{enumerate}
    \end{enumerate}
\end{enumerate}
\end{proof}
%\begin{proofsketch}
    %We prove \cref{equ:clock-fp-match-with-trace} by induction on the length of \( \clockexttrc \).
    %In the inductive case, \cref{equ:clock-fp-match-with-trace} can be derived 
    %by \( \rCLOCKWrite \) if the next step is a write,
    %\( \rCLOCKReadLocal\) if the next step is a local read,
    %or \( \rCLOCKReadShard\) if the next step is a read from a shard.
    %The full detail is given in \cref{sec:proof:clock-fp-well-formed}
    %on page \pageref{sec:proof:clock-fp-well-formed}.
%\end{proofsketch}
\end{toappendix}

We then show that:
\begin{enumerate*}
\item every transition in the program kv-store \( \progtrc = \ClockToKVTrace(\clockexttrc)\) is well-formed; and
\item if a version exists in the \( \Last(\progtrc) \), 
then it also exists in  \( \Last(\clockexttrc) \), and vice versa.
\end{enumerate*}                                


\begin{toappendix}
\label{sec:proof-well-formed-clock-si-to-kvs}
\end{toappendix}
\begin{theoremrep}[Well-formed Clock-SI centralised kv-store]
\label{thm:well-formed-clock-trace}
Assume an annotate normalised Clock-SI trace \( \clockexttrc \)
and the kv-store trace induced by the Clock-SI trace \( \progtrc = \ClockToKVTrace(\clockexttrc)\).
Let \( (\kvs,\vienv,\clenv,\prog) = \LastConf(\progtrc) \) and \( (\clocksi,\clockclientenv,\clenv,\clockrunprog) = \LastConf(\clockexttrc) \).
The final kv-store \(\kvs\) is well-formed,
and \( \kvs \) only contains versions that exist in \( \clocksi \), and vice versa,
that is,
\begin{Formulae}
& \begin{Formula}
    \Forall{\key, \idx, \val, \cl,n,m,\txidset} 
    \kvs(\key,\idx) = (\val,\clocktxid[\cl](n,m),\txidset) 
    \implies \Exists{\clockshard,\clockkvs,\idx'} 
    \\ \clockshard = \ShardOf(\clocksi,\key) 
    \land (\clockkvs, \stub) = \clocksi(\clockshard)
    \land \clockkvs(\key,\idx') = (\val,n,\stub) ,
    \label{equ:kvs-to-clock-si}
\end{Formula}
\\ & \begin{Formula}
\Forall{\key,\clockshard,\clockkvs,\idx,\val,n} 
    \clockshard = \ShardOf(\clocksi,\key) 
    \land (\clockkvs, \stub) = \clocksi(\clockshard)
    \land \clockkvs(\key,\idx) = (\val,n,\stub) 
    \implies
    \\ \Exists{\idx', \cl,m,\txidset} 
    \kvs(\key,\idx) = (\val,\clocktxid[\cl](n,m),\txidset)  .
    \label{equ:clock-si-to-kvs}
\end{Formula}
\end{Formulae}
%and any read in \( \kvs \) must have a corresponding read transition in the trace \( \clockexttrc \),
%\begin{Formulae}
%& \begin{Formula}
    %\Forall{\key, \idx,\cl,\cl',n,n',m,m',\val,\txidset} 
    %\kvs(\key,\idx) = \Tuple{\val,\clocktxid[\cl](n,m),\txidset \cup \Set{\clocktxid[\cl'](n',m')}}
    %\implies \Exists{\clockshard} 
    %\\ \clockshard = \ShardOf(\clocksi,\key) 
    %\land \ClockReadFromShard(\clockexttrc,\cl',\clockshard,\key,m',n') 
    %\\ {} \land n = \Set{n'' | \Exists{\idx,\clockkvs} n'' < \clocktime 
    %\land \clocksi(\clockshard) = (\clockkvs,\stub) \land \clockkvs(\key,\idx) = (\stub,n'')} ,
    %\label{equ:kvs-read-to-clock-si-read}
%\end{Formula}
%\\ & \begin{Formula}
    %\Forall{\key,\clockshard,\clocktime,\val,\cl',n,n',m'} 
    %\clockshard = \ShardOf(\clocksi,\key) 
    %\land \ClockReadFromShard(\clockexttrc,\cl',\clockshard,\key,m',n')
    %\\ {} \land n = \Set{n'' | \Exists{\idx,\clockkvs} n'' < \clocktime 
    %\land \clocksi(\clockshard) = (\clockkvs,\stub) \land \clockkvs(\key,\idx) = (\stub,n'')}
    %\\ {} \implies \Exists{\idx,\txidset,\cl} 
    %\kvs(\key,\idx) = \Tuple{\val, \clocktxid[\cl](n,m), \txidset \cup \Set{\clocktxid[\cl'](n',m')}} ,
    %\label{equ:clock-si-read-to-kvs-read}
%\end{Formula}
%\end{Formulae}
%where predicate \( \ClockReadFromShard \) is defined by
%\begin{multline*}
%\ClockReadFromShard(\clockexttrc,\cl,\clockshard,\key,\clocktime,n)
%\PredDef
%\Exists{\val,\clocksi,\clockclientenv,\clenv,\clenv',\clockrunprog,\trans',\fp,\var,\expr}
%\\ \clockexttrc = \ToClockProg{\cdots | \stub
                                %-> \clocksi | \clockclientenv | \clenv 
                                            %| \clockrunprog\UpdateFunc{\cl ->
                                                %\pruntrans{\plookup{\var}{\expr}; \trans}%
                                                            %{\fp,\emptyset}%
                                                            %{\clocktime',\clockshard} } }
                  %\\ \ToClockProg{\ | \lbCLOCKOp{\opR(\key,\val),\clocktime,n}
                                %-> \clocksi | \clockclientenv | \clenv'
                                            %| \clockrunprog\UpdateFunc{\cl -> \pruntrans{\trans}%
                                                            %{\fp \AddOp \opR(\key,\val),\emptyset}%
                                                            %{\clocktime',\clockshard} }  | \stub 
                                %-> \cdots } .
%\end{multline*}
\end{theoremrep}
\begin{proof}
We prove by induction on Clock-SI trace \( \clockexttrc \).
\begin{enumerate}
\CaseBase{\clockexttrc = \ToClockProg{\clocksiinit | \clockclientenvinit | \clenv_0 | \prog_0 }}
    By  the definition of \( \ClockToKVTrace \),
    \( \progtrc = \ToProg{ \kvsinit | \vienvinit | \clenv_0 | \prog_0 } \).
    It is trivial that the initial kv-store \( \kvsinit \) is well-formed,
    and \( \clocksiinit \) and \( \kvsinit \)  satisfy 
    %\cref{equ:kvs-to-clock-si,equ:clock-si-to-kvs,equ:kvs-read-to-clock-si-read,equ:clock-si-read-to-kvs-read}.
    \cref{equ:kvs-to-clock-si,equ:clock-si-to-kvs}.
\CaseInd{ \clockexttrc = \ToClockProg{\clockexttrc' | \lb -> \clockexttrc'' | \lb' 
        -> \clocksi'' | \clockclientenv'' | \clenv'' | \clockrunprog'' } }
    Let \( (\clocksi',\clockclientenv',\clenv',\clockrunprog) = \LastConf(\clockexttrc') \)
    and \( \progtrc' = \ClockToKVTrace(\clockexttrc') \).
    By \ih, the final kv-store \( \kvs' \) such that \( (\kvs',\vienv',\stub,\stub) = \LastConf(\progtrc') \),
    is well-formed,
    and \( \clocksi'\) and \( \kvs' \) satisfy
    %\cref{equ:kvs-to-clock-si,equ:clock-si-to-kvs,equ:kvs-read-to-clock-si-read,equ:clock-si-read-to-kvs-read}.
    \cref{equ:kvs-to-clock-si,equ:clock-si-to-kvs}.
    Now let consider the next step:
    there are two possible cases for \( \clockexttrc''\).
    \begin{enumerate}
    \Cases{\( \lb = \lbCLOCKStart{\clocktime}\) or \( \lb = \lbCLOCKTick{\clocktime} \), and both with \( \clockexttrc'' = \emptyset\)}
        By the definition of \( \ClockToKVTrace \), \( \ClockToKVTrace(\clockexttrc) = \ClockToKVTrace(\clockexttrc') \),
        and by \ih, we have the proof.
    \Cases{\( \lb = \lbCLOCKOp{(l,\key,\val), \clocktime,n}\)}
        Let \( \clockexttrc^* = 
                \ToClockProg{\LastConf(\clockexttrc) | \lb 
                            -> \clockexttrc' | \lb' 
                            -> \clocksi'' | \clockclientenv'' | \clenv'' | \clockrunprog'' } \).
        In this case, we have \( \CLOCKAtomic(\clockexttrc^*,\cl,\clockshard,n) \).
        Let \( \vi = \ClockView(\kvs',\clocktime) \) and \( \fp =  \ClockFp(\emptyset,\clockexttrc^*) \).
        By the definition of \( \progtrc = \ClockToKVTrace(\clockexttrc) \), we have 
        \[
            \progtrc = \ToProg{\progtrc' | \lbTrans{\vi,\fp} 
                                    ->  \kvs''
                                       | \vienv'\UpdateFunc{\cl -> \vi'} 
                                       | \clenv'' 
                                       | \ClockStaticProg(\clockrunprog) }
        \]
        where \( \kvs'' = \UpdateKV(\kvs,\vi,\fp,\clocktxid[cl](n,m)) \) and \( \vi = \ClockView(\kvs'',n) \).
        Note that we have the following results:
        \begin{enumerate}
            \item by \ih, the kv-store \( \kvs'' \) is well-formed;
            \item by \cref{prop:clock-si-view}, the pre-view \( \vi \) is well-formed on \( \kvs'' \);
            \item by \cref{prop:well-formed-clock-fp}, the fingerprint \( \fp \) is well-formed fingerprint; and 
            \item by \cref{prop:clock-si-unique-txid}, the transaction identifier \( \clocktxid[cl](n,m) \) must be a unique 
            and by \cref{lem:clock-si-local-time-mono}, \( \clocktxid[\cl](n,m) \in \NextTxid(\kvs,\cl) \).
        \end{enumerate}
        Given above, by \cref{thm:well-deifned-updatekv},
        we have that \( \kvs'' = \UpdateKV(\kvs,\vi,\fp,\clocktxid[cl](n,m)) \) is well-formed.
        
        Now we prove \cref{equ:kvs-to-clock-si,equ:clock-si-to-kvs}.
        Note that \( \kvs' \) and \( \clocksi' \) satisfy \cref{equ:kvs-to-clock-si,equ:clock-si-to-kvs},
        Consider the new versions in \( \kvs'' \).      
        Take a write operation such that \( \opW(\key,\val) \in \fp \).
        This means that, if and only if, \( \kvs''(\key,\Abs{\kvs''(\key)} - 1 ) = (\val,\clocktxid[\cl](n,m),\emptyset) \).
        By definition of \( \fp = \ClockFp(\emptyset,\clockexttrc^*) \),
        if and only if, there exists two transitions, 
        \( \lbCLOCKOp{\opW(\key,\val),\clocktime,n} \) and \( \lbCLOCKOp{\opW(\key,\val),\clocktime',n} \) 
        for some preparation time \( \clocktime' \) such \( \clocktime' > \clocktime\),
        in the trace \( \clockexttrc^*\).
        By the definition of \( \rCLOCKPrepare \) and \( \rCLOCKCommit \),
        if and only if, we have \( \clockkvs''(\key, \Abs{\clockkvs''(\key)} - 1)  = (\val,n,\clocksicommitted) \)
        for \( (\clockkvs'',\stub) = \clocksi''(\ShardOf(\key))\).
        Thus, \cref{equ:kvs-to-clock-si,equ:clock-si-to-kvs} hold. \qedhere

        %Last, we prove \cref{equ:kvs-read-to-clock-si-read,equ:clock-si-read-to-kvs-read}.
        %Note that \( \kvs' \) and \( \clocksi' \) satisfy 
        %\cref{equ:kvs-read-to-clock-si-read,equ:clock-si-read-to-kvs-read},
        %Consider the new transaction \( \clocktxid[\cl](n,m) \) from \( \cl \) that commits at time \( n\).
        %Assume \( \kvs(\key,\idx) = 
            %\Tuple{\val, \clocktxid[\cl'](n'), \txidset \cup \Set{\clocktxid[\cl](n,m)}} \)
        %for some \( \key, \idx, \val, \txid[\cl](n) ,\txidset \).
        %This means, if and only if, the read operation \( \opR(\key,\val) \in \fp \),
        %by the definition of \( ClockToKVTrace \).
        %By definition of \( \fp = \ClockFp(\emptyset,\clockexttrc^*) \),
        %if and only if, 
        %\[
            %\ClockReadFromShard(\clockexttrc^*,\cl,\clockshard,\key,\clocktime,n)
        %\]
        %and, then if and only if, 
        %\[
            %n' = \Set{n'' | \Exists{\idx,\clockkvs} n'' < \clocktime \land \clocksi(\clockshard) = (\clockkvs,\stub) \land \clockkvs(\key,\idx) = (\stub,n'')}
        %\] 
        %by the premise of \( \rCLOCKReadShard \).
        %Thus, \cref{equ:kvs-read-to-clock-si-read,equ:clock-si-read-to-kvs-read} hold. \qedhere
    \end{enumerate}
\end{enumerate}
\end{proof}
\begin{proofsketch}
We prove this by induction on the trace \( \clockexttrc\).
In the base case, \cref{equ:kvs-to-clock-si,equ:clock-si-to-kvs} trivial hold.
In the inductive case, by the definition of \( \ClockToKVTrace\),
we need to prove that the new kv-store, \(  \kvs' = \UpdateKV(\kvs,\vi,\fp,\clocktxid[cl](n,m)) \), 
as the result of committing a new transaction \( \clocktxid[cl](n,m) \) with fingerprint \( \fp \) and view \( \cl \),
is still well-formed.
Note that we have the following results:
\begin{enumerate}
    \item by \ih, the kv-store \( \kvs \) is well-formed;
    \item the pre-view \( \vi \) is well-formed on \( \kvs \),
    the fingerprint \( \fp \) is well-formed fingerprint,
    and the transaction identifier \( \clocktxid[cl](n,m) \) must be a unique,
    of which details are given in \cref{prop:clock-si-unique-txid,prop:well-formed-clock-fp,prop:clock-si-view}; and 
    \item client local times in Clock-SI monotonically increases (\cref{lem:clock-si-local-time-mono}), hence \( \clocktxid[\cl](n,m) \in \NextTxid(\kvs,\cl) \).
\end{enumerate}
Given above and \cref{thm:well-deifned-updatekv},
the new kv-store \( \kvs' \) is well-formed.
Then, \Cref{equ:kvs-to-clock-si,equ:clock-si-to-kvs} can be derived by the definition of \( \ClockToKVTrace \),
since the function does not create or remove any versions.
The full proof is given in \cref{sec:proof-well-formed-clock-si-to-kvs}
on page \pageref{sec:proof-well-formed-clock-si-to-kvs}.
\end{proofsketch}

