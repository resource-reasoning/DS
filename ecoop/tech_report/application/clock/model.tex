A shard in Clock-SI comprises a local time and all the history versions of a partition of keys.
Each version consists of a value, a time when the version committed
and a version state, which is \( \clocksiprepared \) or \( \clocksicommitted \).
In the semantics of Clock-SI, the shard do not explicitly have the preparation set, 
instead, annotates versions with their states.

\begin{definition}[Clock-SI local times and versions]
The set of \emph{Clock-SI local times}, \( \CLOCKtimes \ni \clocktime \),
is defined by: \( \CLOCKtimes \TypeDef \Nat \).
%Let \( \clocktimeinit = 0\) be the initialisation time.
Assume a set of \emph{version states} \( \CLOCKStates \TypeDef \{\clocksiprepared, \allowbreak \clocksicommitted\}\).
Given the set of values (\cref{def:values}),
the set of \emph{Clock-SI versions}, \( \CLOCKVersions \ni \clockver \),
is defined by: \( \CLOCKVersions \TypeDef \Values \times \CLOCKtimes\times \CLOCKStates  \).
Given a version \( \clockver \), let \( \ValOf(\clockver)\), \( \TimeOf(\clockver) \)
and  \( \StateOf(\clockver)\)
return the first, second and third projections of \( \clockver \) respectively.
Given two versions \( \clockver, \clockver' \),
the order between them, written \(\clockver \clockverleq \clockver' \), is defined by:
\begin{align*}
    \clockver \clockverleq \clockver' 
    \begin{multlined}[t]
    \PredDef
    \Exists{\val,\clocktime,\clocktime',l,l'}
    \clockver = (\val,\clocktime,l)
    \land \clockver = (\val,\clocktime',l')
    \\ {} \land 
    \left( (\clocktime = \clocktime' \land l = l')
    \lor
    (l = \clocksiprepared \land l' = \clocksicommitted \land \clocktime \leq \clocktime') \right).
    \end{multlined}
\end{align*}
\end{definition}
 
The \emph{local key-value store}, \( \clockkvs \), of a shard contains all the history versions,
which is a \emph{partial function} from keys to lists of Clock-SI versions.

\begin{definition}[Clock-SI key-value stores]
The set of \emph{Clock-SI local key-value  stores} or \emph{Clock-SI local store} is defined by:
\[ \CLOCKKVS \TypeDef \Set{\clockkvs \in \Keys \ToPFunc \List{\CLOCKVersions} | \WfCLOCKKvs(\clockkvs)} , \]
where \( \WfCLOCKKvs \) is defined by: for any key \( \key \in \Dom(\clockkvs) \) and indexes \( \idx, \idx'\),
\begin{Formulae}
& \begin{Formula} \clockkvs(\key, 0) = (\valinit,\clocktimeinit,\clocksicommitted) , \label{equ:clock-si-kv-init}
\end{Formula}
\\ & \begin{Formula} 
\Forall{\val, \val' \in \Values | \clocktime, \clocktime' \in \CLOCKtimes | l,l' \in \CLOCKStates}
\\ \clockkvs(\key, \idx) = (\val,\clocktime,l) 
\land \clockkvs(\key, \idx') = (\val',\clocktime',l')
\land \idx < \idx'
\implies \clocktime < \clocktime' . \label{equ:clock-si-kv-mono}
\end{Formula}
\end{Formulae}
The set of \emph{initial Clock-SI key-value stores}, 
\( \CLOCKKVSInit \ni \clockkvsinit \), is defined by:
\[ 
\CLOCKKVSInit \TypeDef
    \Set{\clockkvsinit | \Forall{\key \in \Dom(\clockkvsinit) } 
                    \clockkvsinit(\key) = \List{(\valinit,0,\clocksicommitted)}} .
\]
The order \( \clockkvs \clockkvsleq \clockkvs' \) is defined by:
\[
    \clockkvs \clockkvsleq \clockkvs' \PredDef
    \Forall{\key \in \Keys | \idx \in \Indexs }
    \clockkvs(\key,\idx) \clockverleq \clockkvs'(\key,\idx) .
\]
\end{definition} 

The versions in a Clock-SI key-value store are ordered over their commit times,
captured by the well-formed condition, \( \WfCLOCKKvs \).
In the initial Clock-SI, each key only contains the initial version,
with the initial value \( \valinit \) and time \( 0 \).

Clock-SI is a fully \emph{partitioned},
which means that versions associated with a key are stored in a distinct shard.
This is captured by the definition of Clock-SI machine states, \( \clocksi \),
comprising a function mapping shard identifiers, \( \clockshard \), to their key-value stores and local times.

\begin{definition}[Clock-SI machine states]
Assume a set of \emph{shards identifiers}, \( \CLOCKShards \ni \clockshard \).
The set of \emph{Clock-SI machine states}, \( \CLOCKSI \ni \clocksi \), is defined by:
\[
\CLOCKSI \TypeDef \Set{\clocksi \in \CLOCKShards \ToPFFunc {} 
                    \\ \CLOCKKVS \times \CLOCKtimes |
            \Forall{\clockshard, \clockshard' \in \Dom(\clocksi) 
                        | \clockkvs,\clockkvs' \in \CLOCKKVS} 
            \\ \clockshard \neq \clockshard' 
            \land (\clockkvs,\stub) = \clocksi(\clockshard)
            \land (\clockkvs',\stub) = \clocksi(\clockshard')
            \\ {} \implies \Dom(\clockkvs) \cap \Dom(\clockkvs') = \emptyset } .
\]
The set of \emph{initial Clock-SI machine states}, \( \CLOCKSIInit \ni \clocksiinit\), is defined by:
\[
\CLOCKSIInit \TypeDef \Set{\clocksiinit 
        | \Forall{\clockshard \in \Dom(\clocksiinit)}
                \Exists{\clockkvsinit \in \CLOCKKVSInit}
                \clocksiinit(\clockshard) = \Tuple{\clockkvsinit,0}} .
\]
Given \( \clocksi \), let \( \ShardOf(\clocksi,\key) \) denote the shard containing the key \( \key \).
Let \( \clocksi(\key)\) denote the list of versions associated with the key \( \key \),
defined by:
\(
    \clocksi(\key) \FuncDef \clocksi(\ShardOf(\clocksi,\key))\Proj{0}(\key) .
\)
The order, \( \clocksi \clocksileq \clocksi' \), is defined by:
\begin{multline*}
    \clocksi \clocksileq \clocksi' \PredDef
    \Forall{ \clockshard \in \Dom(\clocksi) | \clockkvs,\clockkvs' \in \CLOCKKVS 
                | \clocktime,\clocktime \in \CLOCKTimes }
    \\ \clocksi(\clockshard) = (\clockkvs,\clocktime) 
       \implies \clocksi'(\clockshard) = (\clockkvs',\clocktime')
       \land \clockkvs \clockkvsleq \clockkvs' 
       \land \clocktime \leq \clocktime' .
\end{multline*}
\end{definition} 

Transactions in a client session are executed sequentially,
yet they might commit to different shards.
Each client maintains a local time, 
initially being zero, that determines the session order between transactions from this client.

\begin{definition}[Clock-SI client environments]
Given the set of client identifiers defined in \cref{def:client-id},
the set of \emph{Clock-SI client environments}, \( \CLOCKClientEnvs \ni \clockclientenv\),
is defined by:
\[
\CLOCKClientEnvs \TypeDef \Set{\clockclientenv \in \Clients \ToPFFunc \CLOCKtimes } .
\]
The set of \emph{initial Clock-SI client environments}, 
\( \CLOCKClientEnvInits \ni \clockclientenvinit\),
is defined by:
\[
    \CLOCKClientEnvInits \TypeDef \Set{\clockclientenvinit 
                | \Forall{\cl \in \Dom(\clockclientenvinit)} \clockclientenvinit(\cl) = 0}.
\]
Given two Clock-SI environment \( \clockclientenv,\clockclientenv'\),
the order, \( \clockclientenv \clockclientenvleq \clockclientenv'\), is defined by:
\[
\clockclientenv \clockclientenvleq \clockclientenv' \PredDef
\Forall{\cl \in \Dom(\clockclientenv)} \clockclientenv(\cl) \leq \clockclientenv'(\cl) .
\]
\end{definition}

The abstract states of Clock-SI semantics comprise the machine states of Clock-SI and client environments.

\begin{definition}[Clock-SI configurations]
the set of \emph{Clock-SI configurations}, \( \CLOCKConfs \ni \clockconf \), is defined by:
\(
\CLOCKConfs \TypeDef \CLOCKSI \times \CLOCKClientEnvs .
\)
and the set of \emph{initial Clock-SI configurations}, \( \CLOCKConfInits \ni \clockconfinit \),
\( 
\CLOCKConfInits \TypeDef \CLOCKSIInit \times \CLOCKClientEnvInits .
\)
\end{definition}
