A transactional library, \(\lib \),
provides a set of transactional operations which can be used by its clients to access the underlying kv-store.
%For simplicity, we model each operation as a single transaction; 
%it does not change any robustness result and is straightforward to extend this to multiple transactions.
For instance, the counter library on a key \(\key\) discussed in \cref{sec:overview} is defined by:
\(\pcounter*(\key) \FuncDef \Set{\pinc*(\var,\key), \pread*(\var,\key) | \var \in \Vars}\).
A program \(\prog\) is a \emph{client program} of \(\lib\), written \( \prog \in \lib\), 
if and only if the transactional calls in \(\prog\) are to operations of \(\lib\).  

\begin{definition}[Invariant of a library]
The invariant of a library \( \lib\)  under execution test \( \et \) is defined by:
\( \EvalET{\lib} \FuncDef \bigcup_{\prog \in \lib} \EvalET{\prog} \).
\end{definition}

A library is \emph{robust} with respect to a weak consistency model,
if, for any program \( \prog \) from the library,
any behaviour obtained by exeucting the program \( \prog \)  under the weak consistency model,
can be obtained by executing it under serialisability.

\begin{definition}[Robustness]
\label{def:robustness}
A library \( \lib \) is \emph{robust} with respect to an execution test \( \et \), if and only if
\( \EvalET{\lib} \subseteq \CMET(\SER) \).
\end{definition}

By the definition of \( \CMET(\SER) \), and the bijection between our kv-stores and dependency graphs,
we prove robustness of a library via proving the invariant on the kv-stores in each transitions.

\begin{theorem}[Serialisibility kv-store]
\label{thm:serialisable-nocycle}
For all kv-stores \(\kvs\), 
\[ 
    \Forall{\txid \in \TxIDs} (\txid,\txid) \notin \Trasi((\WR[\kvs] \cup \SO \cup \WW[\kvs] \cup \RW[\kvs])) \implies 
    \kvs \in \CMET(\SER)
\]
\end{theorem}
\begin{proof}
The result can be derived from 
\begin{enumerate*} 
\item the isomorphism between kv-stores and dependency graphs (\cref{thm:isomorphism-kvstore-dgraph})
and \item a well-known result on dependency graphs \citep{adya} 
stating that a dependency graph is serialisable if and only if the graph is acyclic.
\end{enumerate*}
\end{proof}

Using this theorem, we prove the robustness of a single counter against \(\PSI\).
It is easy to see that single counter library is \emph{not} robust against \( \CC \).
Since the kv-store depicted in \cref{fig:counter_kv_final} is allowed under \( \CC \),
it is obvious not a serialisable kv-store 
due to the cycle \( \ToEdge{ \txid' | \RW -> \txid | \WW -> \txid'} \).

\begin{figure}
\centering
\begin{tikzpicture}

\KVMapping{x}{ \key }{ %
    /0/\txidinit/\Set{\txid_1} \uplus \txidset_0
    , /1/\txid_1/\Set{\txid_2} \uplus \txidset_1
    , /\cdots/\cdots/\cdots
	, /n-1/\txid_{n-1}/\Set{\txid_{n}} \uplus \txidset_{n-1}
	, /n/\txid_{n}/\txidset_{n}
};
\end{tikzpicture}

\hrulefill

\caption{Invariant of the single counter library under \( \PSI \)}
\label{fig:inv-single-counter}
\end{figure}


Now we consider \( \PSI \).
As \(\PSI\) enforces write conflict freedom (\(\UA\)), 
we know that if a transaction \(\txid\) updates \(\key\) (by calling \(\pinc*(\var,\key)\)) 
and writes a new version \(\ver\) to \(\key\), 
then it must have read the version of \(\key\) immediately preceding \(\ver\): that is,
\(\Forall{\txid \in \kvs | \idx \in \Indexs} \txid = \WtOf(\kvs(\key, \idx)) \implies \txid \in \RsOf(\kvs(\key, \idx-1))\). 
Moreover, as \(\PSI\) enforces monotonic reads (\(\MR\)) and causal consistency (\( \CC \))
the order in which clients observe the versions of \(\key\) (by calling \(\pread*(\var,\key)\))
is consistent with the order of versions in \(\kvs(\key)\). 
As such, the kv-stores in \(\EvalET{\pcounter(\key)}[\PSI]\) 
have the invariant depicted in \cref{fig:inv-single-counter} and defined below:
\[
    \kvs(\key) = \List{ \Tuple{0,\txidinit,\txid_1 \uplus \txidset_0}
            , \Tuple{1,\txid_1,\txid_2 \uplus \txidset_1}
            , \cdots
            , \Tuple{n-1,\txid_{n-1},\txid_n \uplus \txidset_{n-1}}
            , \Tuple{n,\txid_n,\txidset_n}} 
\]
where \( \txid_\idx \notin \txidset_j\) for any \( \idx,j\).

We first consider the relation \( \WR[\kvs] \cup \SO \).
Consider two transactions \( \txid,\txid' \) such that \( (\txid,\txid') \in \WR[\kvs] \cup \SO \).
It must be the case that 
\begin{Formulae}
& \begin{Formula}
\Forall{\txid, \txid'} (\txid,\txid') \in \SO
\\ \implies
\Exists{\idx,j}
( \txid = \txid_\idx \implies  
\land \txid' \in \txidset_{\idx - 1}
\land \txid' = \txid_j 
\land \idx < j )
\lor
( \txid \in \txid_{\idx-1}
\land \txid \in \txid_{j-1} 
\land \idx \leq j )
\label{equ:single-robust-so}
\end{Formula}
\\ & \begin{Formula}
\Forall{\txid, \txid'} (\txid,\txid') \in \WR[\kvs]
\implies
\Exists{\idx}
\txid = \txid_\idx 
\land \txid' \in \txidset_{\idx - 1}
\land \txid' = \txid_{\idx + 1}
\label{equ:single-robust-wr}
\end{Formula}
\end{Formulae}
If \( (\txid , \txid') \in \SO \),
then by the \( \ViewShift[\MR \cap \RYW] \), \cref{equ:single-robust-so} must hold.
If \( (\txid , \txid') \in \WR[\kvs] \),
by the definition of \( \WR[\kvs] \), we know that \cref{equ:single-robust-wr} holds.
\Cref{equ:single-robust-so,equ:single-robust-wr} implies 
irereflexivity of the relation \( \Trasi((\WR[\kvs] \cup \SO)) \).
Then by the definitions of \( \WW[\kvs] \) and \( \RW[\kvs] \),
the relation \( \Trasi((\WR[\kvs] \cup \SO \cup \WW[\kvs]  \cup \RW[\kvs])) \) contains no cycle.
This is because transactions contained in 
\( \Trasi((\WR[\kvs] \cup \SO \cup \WW[\kvs] \cup \RW[\kvs])) \) must move forwards 
in term of versions they read or wrote.
The \( \dashrightarrow \) in \cref{fig:relation-single-counter} gives the intuition.

\begin{figure}
\centering
\begin{tikzpicture}

\KVMapping{x}{ \key }{ %
    nonvisible/0/\txidinit/\Set{\txid_1} \uplus \txidset_0
    , nonvisible/1/\txid_1/\Set{\txid_2} \uplus \txidset_1
    , nonvisible/\cdots/\cdots/\cdots
	, nonvisible/n-1/\txid_{n-1}/\Set{\txid_{n}} \uplus \txidset_{n-1}
	, nonvisible/n/\txid_{n}/\txidset_{n}
};

\path[->,dashed,thick]  (Writerx1) edge[bend left=30] (Readersx1)
                        (Readersx1) edge[bend left=30] (Writerx2)
                        (Writerx2) edge[bend left=30] (Readersx2)
                        (Readersx2) edge[bend left=30] (Writerx3) 
                        (Writerx3.east) edge[bend left=30] (Readersx3.east)
                        (Readersx3.east) edge[bend left=30] (Writerx4) 
                        (Writerx4) edge[bend left=30] (Readersx4)
                        (Readersx4) edge[bend left=30] (Writerx5) 
                        (Writerx5.east) edge[bend left=30] (Readersx5.east);
\draw[->,dashed,thick]  ([xshift=-3pt]Readersx1.south) to[out=270, in=180] ([yshift=-15pt]Readersx1.south) 
                    to[out=0, in=270] ([xshift=3pt]Readersx1.south);
\draw[->,dashed,thick]  ([xshift=-3pt]Readersx2.south) to[out=270, in=180] ([yshift=-15pt]Readersx2.south) 
                    to[out=0, in=270] ([xshift=3pt]Readersx2.south);
\draw[->,dashed,thick]  ([xshift=-3pt]Readersx3.south) to[out=270, in=180] ([yshift=-15pt]Readersx3.south) 
                    to[out=0, in=270] ([xshift=3pt]Readersx3.south);
\draw[->,dashed,thick]  ([xshift=-3pt]Readersx4.south) to[out=270, in=180] ([yshift=-15pt]Readersx4.south) 
                    to[out=0, in=270] ([xshift=3pt]Readersx4.south);
\draw[->,dashed,thick]  ([xshift=-3pt]Readersx5.south) to[out=270, in=180] ([yshift=-15pt]Readersx5.south) 
                    to[out=0, in=270] ([xshift=3pt]Readersx5.south);
\end{tikzpicture}

\hrulefill

\caption{The relation \( \WR \cup \SO \cup \WW \cup \RW \) on the invariant in \cref{fig:inv-single-counter}}
\label{fig:relation-single-counter}
\end{figure}

