\citet{bank-example-wsi} present a banking library that they shown to be robust against  \( \SI \).
We show that this library is also robust against \( \WSI \).
The banking example is based on relational databases and has three tables: account, saving and checking.
\begin{center}
\begin{tabular}{c c}
\hline
\multicolumn{2}{c}{\CodeFont{Account}} 
\\ \hline
\underline{\CodeFont{Name}} & \CodeFont{CID}
\\ Alice & 00001 
\\ Bob & 00002 
\\ \( \cdots \) & \( \cdots \) 
\\ \hline
\end{tabular} \qquad 
\begin{tabular}{c c}
\hline
\multicolumn{2}{c}{\CodeFont{Saving}} 
\\ \hline
\underline{\CodeFont{CID}} & \CodeFont{Balance}
\\ 00001 & 33
\\ 00002 & 87 
\\ \( \cdots \) & \( \cdots \)
\\ \hline
\end{tabular} \qquad 
\begin{tabular}{c c}
\hline
\multicolumn{2}{c}{\CodeFont{Checking}} 
\\ \hline
\underline{\CodeFont{CID}} & \CodeFont{Balance}
\\ 00001 & -1
\\ 00002 & 5
\\ \( \cdots \) & \( \cdots \)
\\ \hline
\end{tabular}
\end{center}
The account table maps customer names to customer IDs (\CodeFont{Account(\underline{Name}, CID )}),
the saving table maps customer IDs to their saving balances (\CodeFont{Saving(\underline{CID}, Balance )}) and
the checking table maps customer IDs to their checking balances (\CodeFont{Checking(\underline{CID}, Balance )}).

For simplicity, we encode the saving and checking tables as a kv-store,
and forgo the account table as it is an immutable lookup table.
We model each customer ID as an integer \( n \in \Nat\) and assume that balances are integer values. 
We then define the key associated with customer \(n\) in the checking table as 
\(n_c \FuncDef 2 n\); 
and define the key associated with \(n\) in the saving table as 
\(n_s \FuncDef 2n + 1\). 

The banking library provides five transactional operations for accessing the database,
where the \( \ret \) is special variable that carries the return value or a transaction.
\begin{align*}
    \pbalance(n) & \FuncDef
    \begin{Transaction}
    \plookup{\var}{n_c}; \ 
    \plookup{\var(y)}{n_s}; \ 
    \passign{\ret}{\var + \var(y)} ;
    \end{Transaction} \\
    \pcheck(n,\var(v)) & \FuncDef
    \begin{Transaction}
    \pif(\var(v) \geq 0) \ \{ \
    \plookup{\var}{n_c}; \ 
    \pmutate{n_c}{\var + \var(v)}; \ \}
    \end{Transaction}  \\
    \psave(n,\var(v)) & \FuncDef
    \begin{Transaction}
    \plookup{\var}{n_s}; \ 
    \pif(\var(v) + \var \geq 0) \ \{ \
    \pmutate{n_s}{\var + \var(v)}; \ \}
    \end{Transaction} \\
%
	\pamal(n,n') & \FuncDef
    \begin{Transaction}
    \plookup{\var}{n_s}; \ 
    \plookup{\var(y)}{n_c}; \ 
    \plookup{\var(z)}{n'_c}; 
    \\ \pmutate{n_s}{0}; \ 
    \pmutate{n_c}{0}; \ 
    \pmutate{n'_c}{\var + \var(y) + \var(z)}; 
    \end{Transaction} \\
    \pwritecheck(n,\var(v)) & \FuncDef
    \begin{Transaction}
    \plookup{\var}{n_s}; \ 
    \plookup{\var(y)}{n_c}; 
    \\ \pif(\var + \var(y) < \var(v) ) \ \{ \
    \pmutate{n_c}{\var(y) - \var(v) - 1 }; \ \}
    \\ \pelse \ \{ \
    \pmutate{n_c}{\var(y) - \var(v) }; \ \}
    \\ \pmutate{n_s}{\var}; 
    \end{Transaction}     
\end{align*}

The \( \pbalance(n) \) operation returns the total balance of customer \(n\) in  \( \ret \).
The \( \pcheck(n,\var(v)) \) operation deposits \(\var(v)\) to the checking account of customer \(n\) when \(\var(v) \) is non-negative.
Otherwise the checking account remains unchanged.
When \(\var(v) \geq 0\),  operation \( \psave(n,\var(v)) \) deposits \(\var(v)\) to the saving account of \( n \).
When \(\var(v) < 0\), operation \( \psave(n,\var(v)) \) withdraws \(\var(v)\) from 
the saving account of \(n\) only if the resulting balance is non-negative.
The \( \pamal(n,n') \) operation moves the combined checking and saving funds of consumer \(n\) 
to the checking account of customer \(n'\).
Lastly, \( \pwritecheck(n,\var(v)) \) cashes a cheque of customer \(n\) 
in the amount \(\var(v)\) by deducting \(\var(v)\) from its checking account.
If \(n\) does not hold sufficient funds, that is, the combined checking and saving balance is less than \( \var(v) \), 
customer \(n\) is penalised by deducting one additional pound. 
\citet{bank-example-wsi} argue that to make the banking library robust against \( \SI \),
the \( \pwritecheck(n,\var(v)) \) operation must be strengthened by writing back the balance to the saving account 
(via \(\pmutate{n_s}{\var(x)} \)),
even though the saving balance is unchanged.
The banking library for a set of customers \( N \), written \( \pbank(N) \), is defined by
\[ 
    \pbank(N) \FuncDef 
    \Set{\pbalance(n), \pcheck(n,\var(v)), 
    \\ \psave(n,\var(v)), \pamal(n,n'), 
    \\ \pwritecheck(n,\var(v)) |
    n,n' \in N \land \var(v) \in \Vars }
\]

%The invariant of this banking example satisfies \cref{def:main-body-wsi-safe}:
%all the transactions are either read-only or satisfy strict no blind write.
%By \cref{thm:main-wsi-robust}, it is robust against \( \WSI \).

The banking library is more complex than the multi-counter library discussed in \cref{sec:robust-multiple}.
Nevertheless, all banking transactions are either read-only or
satisfy the strictly-no-blind-writes property; 
the banking library is \WSI-safe.
As such, we can prove its robustness against \(\WSI\) 
in a similar fashion to that of the multi-counter library.

\begin{theorem}[Robustness of the banking application against \( \WSI \)]
The bank library \( \pbank(N) \) is robust against \( \WSI \).
\end{theorem}
\begin{proof}
It is sufficient to prove that any kv-store \( \kvs \) of library \( \pbank(N) \) is \( \WSI \)-safe.
As \(\pbalance(n) \) is read-only, 
it immediately satisfies \cref{equ:wsi-safe-read-only,equ:wsi-safe-write-must-read,equ:wsi-safe-all-write}.
When \(\var(v) \geq 0\), then \(\pcheck(n,\var(v)) \) both reads and writes \( n_c \), and thus preserves  
\cref{equ:wsi-safe-read-only,equ:wsi-safe-write-must-read,equ:wsi-safe-all-write}.
When \(\var(v) \geq 0\), then \(\pcheck(n,\var(v)) \) leaves the kv-store unchanged
\cref{equ:wsi-safe-read-only,equ:wsi-safe-write-must-read,equ:wsi-safe-all-write} are trivially preserved.
Lastly, the
\( \psave(n,\var(v)), \pamal(n,n') \) and \( \pwritecheck(n,\var(v)) \) operations
always read and write the keys they access, 
thus satisfying \cref{equ:wsi-safe-read-only,equ:wsi-safe-write-must-read,equ:wsi-safe-all-write}.
\end{proof}
