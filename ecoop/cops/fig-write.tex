\begin{figure}
\begin{subfigure}{\textwidth}
\begin{lstlisting}
// mixing the client API and system API
'repl' receive 'put(k,v)' request from a client with 'ctx' {
    ltime = inc(repl.local_time);(*\label{line:cops-put-inc-local}*) 
    deps = ctx; (*\label{line:cops-ctx-to-deps}*) 
    txid = (ltime ++ repl.id); (*\label{line:cops-put-id}*)
    list_insert(repl.kv[k],(v, txid, deps)); (*\label{line:cops-put-update-kv}*)
    return txid;
    ctx += (k, txid);  (*\label{line:cops-put-update-ctx}*) 
    asyn_brordcast(k, v, txid, deps); 
}
\end{lstlisting}
\caption{The reference implementation for \pcopsput}
\label{lst:cops-put}
\end{subfigure}

\hrulefill

\begin{subfigure}{\textwidth}
\begin{mathpar}
\inferrule[\rCOPSWrite]{%
    \copsverid = (\repl,\ts+1)
    \\ \copsver = (\val,\copsverid, \copsctx)
    \\ \copskvs' = \COPSInsert(\copskvs, \key, \copsver)
    \\ \copsctx' = \copsctx \uplus \Set{(\key,\copsverid)}
}{%
    \ToCOPSCmd{ 
    \copskvs | \copsctx | \ts | \pcopsput(\key,\val) |
    \lbCOPSWrite{\opW(\key,\val), \copsverid,\copsctx } ->
    \copskvs' | \copsctx' | \ts + 1 | \pskip }
}
\end{mathpar}

\caption{The reference semantics for \pcopsput}
\label{fig:cops-semantics-write}
\end{subfigure}

\hrulefill

\caption{COPS API: \pcopsput}

\end{figure}
