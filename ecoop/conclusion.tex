\section{Conclusions and Future Work}
\label{sec:conclusions}
We have  introduced an interleaving operational semantics for describing the client-observable behaviour of atomic transactions over distributed kv-stores, 
using abstract states comprising global, centralised kv-stores,
partial client views, and transition  steps parametrised by an execution test
which directly captures when a transaction is able to commit on a
state.
Using these execution tests, we  provide a general definition of
consistency model 
and provide example instantiations including \( \CC, \PSI, \SI \) and \( \SER \).
In \cite{shale-phd},  we prove that our definitions are equivalent to the existing definitions in the literature that use execution graphs~\cite{framework-concur}.

We have used our semantics to verify that protocols of real-world
distributed databases satisfy particular consistency models,
\eg that the replicated database COPS \cite{cops} satisfies \( \CC
\), and the partitioned database Clock-SI \cite{clocksi} satisfies \( \SI \). 
These results contrast with those of~\cite{clocksi,cops}, which justify the
correctness of implementations using consistency model definitions that are specific to the implementations. 
We have also proved several invariant properties for clients, 
showing that the clients of several libraries (single-counter,
multi-counter and banking libraries) are robust against the 
appropriate models, 
and showing that certain clients of a lock library satisfy a mutual exclusion property under \( \PSI \), even though they are not robust against $\PSI$. 
We thus believe that our semantics provides an interesting abstract
interface between distributed
implementations and clients.  We plan to validate further the
usefulness of our semantics by verifying other well-known protocols
of distributed databases~\cite{ramp,redblue,eiger,wren}, exploring
robustness results for OLTP workloads such as TPC-C \cite{tpcc} and
RUBiS \cite{rubis}, and exploring other program analysis techniques
such as transaction chopping \cite{psi-chopping,chopping}, invariant
checking \cite{cise,repliss} and program logics \cite{alonetogether}.
We also plan to develop tools to generate litmus tests for implementations and to analyse client programs.

Our work assumes  the
\emph{snapshot property} and the \emph{last-write-wins} policy, common
assumptions in real-world distributed databases. 
Under these assumptions, we are not aware of a consistency
model that we cannot express using our semantics. 
There are consistency models that do not satisfy these assumptions, 
\eg\emph{read committed} \cite{ramp} captured in \cite{seebelieve}. 
In  future, we will explore whether it is possible to weaken our
assumptions to express such weak consistency models. This might be possible by introducing `promises' 
in the style of \cite{promises}. 


There are many resonances between the high-level  behaviour
of distributed systems and the low-level behaviour  of weak
memory. Indeed, our partial client views were 
inspired by the views of  the `promising' C11 semantics in \cite{promises}. 
In future, we plan to explore whether our semantics of atomic transactions can be loosened to describe the more fine-grained behaviour of transactions on weak
memory \cite{PSI-RA,SI-RA,DBLP:conf/pldi/ChongSW18}. 
We are also interested in the work of Doherty \etal\citet{op-semantics-c11-rar}, describing an
operational semantics and a program logic for the release-acquire (RA) fragment of C11, which, interestingly, 
is based on dependency graphs. 
We believe that we can adapt our semantics to model the RA fragment, using simple read-write
primitives rather than atomic transactions and a variant of our definition of causal consistency.%, and would like contrast their approach with ours. 
