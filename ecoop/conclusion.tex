\section{Conclusions and Future Work}
\label{sec:conclusions}

We have introduced an  interleaving operational semantics for
describing the client-observational behaviour of atomic
transactions over distributed kv-stores, with abstract states 
comprising global, centralised kv-stores and  partial
client views, and execution steps parametrised by an execution test
that directly 
captures  when a transaction is able  to commit on a state.
We provide a general
definition of consistency model, parametrised by an execution test,
and explore several example models including  \( \CC, \PSI, \SI \) and \( \SER \).
In \cite{shale-phd}, we prove  that  these models are 
equivalent to the 
well-known consistency models defined on execution graphs.

We have used our semantics to verify that protocols of real-world
distributed databases satisfy particular consistency models: for
example, the replicated database COPS \cite{cops} satisfies \( \CC \); and the
partitioned database Clock-SI \cite{clocksi} satisfies \( \SI \). These results
contrast with the results in the literature, which justify the
correctness of the implementations using definitions of consistency
model specific to the implementations.  We have also proved invariant
properties of clients: robustness properties of clients of the
single-counter, multi-counter and banking libraries; and a
mutual-exclusion property of a program pattern using a lock library
under \( \PSI \), even though it is not robust.  We thus believe that our
semantics provides an interesting \emph{mid-point} between distributed
implementations and clients.  We plan to validate further the
usefulness of our semantics by: verifying other well-known protocols
of distributed databases \cite{ramp,redblue,eiger,wren}; exploring
robustness results for OLTP workloads such as TPC-C \cite{tpcc} and
RUBiS \cite{rubis}; and exploring other program analysis techniques
such as transaction chopping \cite{psi-chopping,chopping}, invariant
checking \cite{cise,repliss} and program logics \cite{alonetogether}.
We also plan to develop tools to generate litmus tests for
implementations and to analyse client programs.


It is common for real-world distributed databases to satisfy the
\emph{snapshot property} and the \emph{last-write-wins} policy. 
Under these assumptions, we do not know of a consistency
model that we cannot express using our semantics. 
There are consistency models that do not satisfy these assumptions,
such as \emph{read committed} \cite{ramp} which is  captured in \cite{seebelieve}. 
In future, we  will explore whether it is possible to weaken our
assumptions to express these 
other 
weak consistency models. This might be possible by introducing promises 
in the style of \cite{promises}. 


There are many resonances between the high-level  behaviour
of distributed systems and the low-level behaviour  of weak
memory. Indeed, our partial client views were 
inspired by the views of  the specific C11 operational semantics
in \cite{promises}. In future, we would like to explore whether our semantics 
of atomic transactions for distributed databases can be loosened to
describe the more fine-grained behaviour of transactions on weak
memory \cite{PSI-RA,DBLP:conf/pldi/ChongSW18}. We are also interested in the
work of 
Doherty et al. \citet{op-semantics-c11-rar}, who have given  an
operational semantics and program logic
for the specific release-acquire fragment of C11 memory model which, interestingly, 
is based on dependency graphs. We believe that we can adapt our
semantics to 
model the release-acquire fragment of C11, using simple read-write
primitives rather than atomic transactions and 
 a variant of our definition of causal consistency, and would like 
contrast their approach with ours. 
