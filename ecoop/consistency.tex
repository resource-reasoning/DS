\section{Consistency Models Using Execution Tests on Kv-stores}
\label{sec:cm}
We define what it means for a kv-store 
to be in a consistent state. Many different consistency models for
distributed databases have 
been proposed in the literature, e.g. 
\cite{si,distrsi,clocksi,rola,cops,PSI-RA,NMSI,PSI},
which capture  different trade-offs 
between  performance and application
correctness.
Example consistency models range from  \emph{serialisability}, a strong
model which only allows kv-stores obtained from a serial execution of transactions
with inevitable performance drawbacks, to \emph{eventual consistency}, a weak model
which imposes few conditions on the structure of kv-stores, leading to
good performance but anomalous behaviours.
We define consistency models for our kv-stores,
by introducing the notion of an
\emph{execution test}, specifying whether a client is allowed to commit a transaction in a given 
kv-store. An execution test $\ET$ induces a consistency model as the set of kv-stores obtained 
by having clients non-deterministically commit transactions, so long as the constraints 
imposed by $\ET$ are satisfied.
We explore a range of execution tests associated with well-known consistency models in the literature. 
In \cite{shale-phd},  we demonstrate that our operational
definitions of consistency models over kv-stores using execution
tests are equivalent to the established declarative definitions of
consistency models over abstract executions \cite{ev_transactions,framework-concur}.

\SpaceAboveDef
\begin{definition}[Execution tests]
\label{def:execution.test}
An \emph{execution test}, $\ET$, is a set of tuples, 
\(\ET \subseteq \MKVSs {\times} \Views {\times} \allowbreak \Fingerprints {\times} \MKVSs {\times} \Views\), 
such that for all \((\mkvs, \vi, \fp, \mkvs', \vi') \!\in\! \ET\): 
\begin{enumerate*}
	\item \(\vi \!\in\!  \allowbreak \Views(\mkvs)\) and \(\vi' \!\in\! \Views(\mkvs')\); 
	\item \(\cancommit \mkvs \vi \fp\); 
	\item \(\vshift \mkvs {\allowbreak \vi} {\mkvs'} {\vi'}\); and 
	\item for all \(\key \!\in\! \mkvs\) and \(\val \!\in\! \Val\), if \((\otR, \key, \val) \!\in\! \fp \) then \(	\mkvs(\key, \max{}_{<}(\vi(\key))) {=} \val   \).
\end{enumerate*}
\end{definition}
\SpaceBelowDef

Intuitively, \((\mkvs, \vi, \fp, \mkvs', \vi') \in \ET\) means that, under the execution test \(\ET\),
a client with initial view \(\vi\) over kv-store \(\mkvs\) can commit a transaction with 
fingerprint \(\fp\) to obtain the resulting kv-store \(\mkvs'\) (given
by \cref{def:updatekv}) while shifting its view
to \(\vi'\). 
Note that the last condition in \cref{def:execution.test} enforces the last-write-wins
policy~\cite{vogels:2009:ec:1435417.1435432}: 
a transaction always reads the most recent writes from the initial view \(\vi\).  

\SpaceAboveDef
\begin{definition}[Consistency models]
\label{def:cm}
The \emph{consistency model} induced by an execution test \(\ET\) is defined as: 
\(
\CMs(\ET) \defeq 
\Set{\mkvs}[ 
\exsts{\mkvs_0,\vienv_0,\clenv,\prog}
(\mkvs_0,\vienv_0,\clenv),\prog \toCMD{\stub}_{\et}^{*} (\mkvs, \stub, \stub),\stub
]
\).
\end{definition}
\SpaceBelowDef

The largest execution test is denoted by \(\ET[\top]\), where for all \(\mkvs, \mkvs', \vi, \vi, \fp\):%

\SpaceAboveMath
\[
	\cancommit[{\ET[\top]}] \mkvs \vi \fp \defiff \mathsf{true}
	\qquad  \text{and} \qquad 
	\vshift[{\ET[\top]}] \mkvs \vi {\mkvs'} {\vi'} \defiff \mathsf{true}
\]%
\SpaceBelowMath

The consistency model induced by \(\ET[\top]\) 
corresponds to the \emph{Read Atomic} model~\cite{ramp}, a variant of \emph{Eventual 
Consistency}~\cite{ev_transactions} for atomic transactions. 
%In what follows we present several examples of execution tests.

\label{subsec:cm_examples}
We present several examples of execution tests which give rise to consistency models on kv-stores.
Recall that the snapshot property and the last-write-wins policy are hard-wired in our framework. 
As such, we can only define consistency models that satisfy these two constraints. 
Although this prohibits interesting consistency models such as \emph{Read Committed},
we can express a large number of consistency models employed by distributed kv-stores.

\mypar{Notation}
Given relations \(\mathsf r, \mathsf r' \subseteq \TypeFont{A} \times \TypeFont{A}\),
we write: \(\mathsf r\rflx\), \(\mathsf r^+\) and \(\mathsf r^*\) for the reflexive, transitive and reflexive-transitive closures of \(\mathsf r\), respectively;
\(\mathsf r^{-1}\) for the inverse of $\mathsf r$;
\(a_1 \toEDGE{\mathsf r} a_2\) for \((a_1, a_2) \in \mathsf r\);
and \( \mathsf r; \mathsf r'\) for \( \Set{(a_1,a_2)}[\exsts{a} (a_1,a) \in \mathsf r\land (a,a_2) \in \mathsf r']\).

Recall that an execution test \(\ET\) is a tuple \((\mkvs,\vi, \fp, \allowbreak \mkvs', \vi')\) such that 
\(\cancommit \mkvs \vi \fp\) and \(\vshift \mkvs \vi {\mkvs'} {\vi'}\) hold (\cref{def:execution.test}).
We proceed with several auxiliary definitions that allow us to define \(\cancommitname\) and \(\vshiftname\) for several consistency models. 



\mypar{Prefix Closure}
The {\em set of visible transactions} of a kv-store \(\mkvs\) and a view \(\vi\) is:
\( \Tx(\mkvs, \vi)  \defeq \Set{\wtOf[\mkvs(\key, i)] }[ i \in \vi(\key)] \).
Given a relation on transactions, \(\rel \subseteq \TxIDs \times \TxIDs\),
a view \(\vi\) is \emph{closed} with respect to a kv-store \(\mkvs\)
and \(\rel\), written \(\closed[\mkvs,\vi,\rel]\),  if and only if 
\[
	\Tx(\mkvs, \vi) = 
	\left( (\rel^*)^{-1} \left( \Tx(\mkvs, \vi) \right) \right) \setminus \Set{\txid }[ \fora{\key \in \kvs ,i} \txid \neq \wtOf[\mkvs(\key,i)] ]
\]
\polish{Question: are  $(\rel^*)^{-1} $ and $(\rel^{-1})^*  $ the
  same? If so, why are we not using the other one which seems easier
  to understand?}

That is, if transaction \(\txid\) is visible in \(\vi\) (\( \txid \in \Tx(\mkvs, \vi) \)),
then all transactions \( \txid'  \) that are \(\rel^*\)-before \(\txid\) (\(\txid' \in (\rel^*)^{-1} \left( \txid \right)\))
and are not read-only \( \txid' \notin \Set{\txid'' }[ \fora{\key,i} \txid'' \neq \wtOf[\mkvs(\key,i)] ] \)
are also visible in \(\vi\) (\( \txid' \in \Tx(\mkvs, \vi) \)).

\mypar{Dependency Relations}
We next define transactional dependency relations for kv-stores.
\Cref{fig:dependencies} illustrates an example kv-store and
its transactional dependency relations.
Given a kv-store \(\mkvs\), a key \(\key\) and 
indexes \(i,j\) such that  \(0 \leq i < j < \Abs{ \mkvs(\key) }\), 
if there exists \(\txid_i, \txidset_i, \txid\) such that 
\(\mkvs(\key, i)  {=} (\stub, \txid_{i}, \txidset_{i})\), \(\mkvs(\key,j) {=} (\stub, \txid_{j}, \stub)\)
and \(\txid \in \txidset_{i}\), 
then for every key \( \key \):

\begin{enumerate} 
\item there is a \emph{Write-Read} dependency from \(\txid_{i}\) to \(\txid\), written \((\txid_{i},\txid) \in \WR[\mkvs](\key)\),
which intuitively means that \( \txid_{i} \) commits before \( \txid \) starts, as depicted in \cref{fig:dependencies-time-line};
\item there is a \emph{Write-Write} dependency from \(\txid_{i}\) to \(\txid_{j}\), 
written \((\txid_{i},\txid_{j}) \in \WW[\mkvs](\key) \),
which intuitively means that \( \txid_{i} \) commits before \( \txid_j \) commits, as depicted in \cref{fig:dependencies-time-line}; and 
\item if \(\txid {\neq} \txid_{j}\), then there is a \emph{Read-Write} anti-dependency from \(\txid\) to \(\txid_{j}\), written \((\txid,\txid_j) \!\in\! \RW[\mkvs](\key)\),
which intuitively means that \( \txid \) starts before \( \txid_j \) commits, as depicted in \cref{fig:dependencies-time-line}.
\end{enumerate}

\begin{figure}[t]
\centering
\begin{tikzpicture}

\node at (0,0) (a) {\(\txid_i\)};
\node at (2,-1) (b) {\(\txid_j\)};
\node at (4.5,0) (c) {\(\txid\)};

\coordinate (a1) at ($(a) + (3,0)$);
\coordinate (b1) at ($(b) + (7,0)$);
\coordinate (c1) at ($(c) + (3,0)$);

\draw[|-,densely dashed] (a) -- ($(a)!0.5!(a1)$);
\draw[-|] ($(a)!0.5!(a1)$) -- (a1);
\draw[|-,densely dashed] (b) -- ($(b)!0.5!(b1)$);
\draw[-|] ($(b)!0.5!(b1)$) -- (b1);
\draw[|-] (c) -- ($(c)!0.5!(c1)$);
\draw[-|,densely dashed] ($(c)!0.5!(c1)$) -- (c1);

\path[->, thick] (a1) edge[bend left=50] node[above, text opacity=1] {$\WR$} ($(c)+(0.3,0.05)$);
\path[->, thick] (a1) edge node[below, text opacity=1] {$\WW$} (b1);
\path[->, thick] (c) edge node[above, text opacity=1] {$\RW$} (b1);

\end{tikzpicture}

\vspace*{-10pt}

\hrulefill

\caption{An example of dependencies between transactions with respect to 
the time line of the starts and commits of these transactions 
(dashed line being able to stretched)}
\label{fig:dependencies-time-line}

\end{figure}


\noindent 
In centralised databases, where there is a global notion of time, 
these dependency relations can be determined by the start and commit time of transaction as in \cref{fig:dependencies-time-line}.
However, in general, there is no global notion of time in distributed databases.
In such settings, the write-read dependency \( \WR \) is induced when a transaction reads from another transaction;
the write-write dependency \( \WW \) is given by the \emph{last-write-wins} resolution policy,
ordering the transactions that write to the same key; and
the read-write anti-dependency \( \RW \) is derived from \( \WR \) and \( \WW \):
if \( (\txid, \txid') \in \WR \) and \( (\txid, \txid'') \in \WW \), then \( (\txid',\txid'') \in \RW \).
We adopt the same names as the dependency relations of dependency graphs \citep{adya}
to underline the similarity.
However, our relations here do \emph{not} depend on those relations in dependency graphs.

We give several definitions of
execution tests using \vshiftname and \cancommitname in \cref{fig:execution_tests}. 

\begin{figure}[t]
\small
\centering
\begin{tabular}{ @{} l @{\hspace{2pt}} || @{\hspace{2pt}} l @{\hspace{2pt}} | @{\hspace{2pt}}  c @{} }
\hline
	\ET 
	& \(\cancommit \mkvs \vi \fp \defeq \closed(\mkvs, \vi, \rel[\et]) \)
    & \(\vshift \mkvs \vi {\mkvs'} {\vi'}\) 
	\\
	\hline
%	
	\MR 
	& \true 
	& \(\vi \viewleq \vi'\)
	\\ \hline  
%
	\RYW
	& \true
	& 
	\( 
    \fora{\txid \in \mkvs' \setminus \mkvs} \fora{\key, i} 
	(\wtOf(\mkvs'(\key, i) ),\txid) \in \SO\rflx \implies i \!\in\! \vi'(\key) 
	\)
	\\ \hline  
	\CC
	& \(\rel[\CC]   \defeq \SO \cup \WR[\mkvs]\) 
	& \(\vshift[\MR \cap \RYW] \mkvs \vi {\mkvs'} {\vi'}\)
	\\ \hline  
%
	\UA 
	& \(\rel[\UA]  \defeq {\textstyle\bigcup_{(\otW, \key, \stub) \in \fp}} \WWInv[\mkvs](\key) \) 
	& \true  
	\\ \hline  
% 
	\PSI
	& \(\rel[\PSI] \defeq \rel[\UA] \cup \rel[\CC] \cup \WW[\mkvs]\) 
	& \(\vshift[\MR \cap \RYW] \mkvs \vi {\mkvs'} {\vi'}\)
	\\ \hline   
%
	\CP 
	& \(\rel[\CP] \defeq \SO;\RWRelf[\mkvs] \cup \WR[\mkvs];\RWRelf[\kvs]  \cup \WW[\mkvs]\) 	
	& \(\vshift[\MR \cap \RYW] \mkvs \vi {\mkvs'} {\vi'}\)
    \\ \hline 
%	
	\SI
	& \( \rel[\SI]  \defeq \rel[\UA] \cup \rel[\CP] \cup (\WW[\mkvs]; \RW[\kvs])\) 
	& \(\vshift[\MR \cap \RYW] \mkvs \vi {\mkvs'} {\vi'}\)
	\\ \hline  
	\SER
	&\(\rel[\SER] \defeq \WWInv[\kvs]\)
	& \true	
	\\ \hline
\end{tabular}
%

\caption{Execution tests of consistency models defined by \( \cancommitname \) and \( \vshiftname \) predicates,
where \(\SO\) is as given in \cref{subsec:kvstores}.}
\label{fig:execution.tests}
\label{fig:execution_tests}
\label{fig:execution-tests}
\end{figure}


\mypar{Monotonic Reads \((\MR)\)}
This consistency model states that,  when committing, a client
cannot lose information in that it can only see increasingly more up-to-date versions from a kv-store.
This prevents, for example, the kv-store of \cref{fig:mr-disallowed},
since client \(\cl\) first reads the latest version of \(\key\) in \(\txid_{\cl}^{1}\), 
and then reads the older, initial version of \(\key\) in \(\txid_{\cl}^{2}\).  
As such, the \(\vshiftname_{\MR}\) predicate in \cref{fig:execution_tests} ensures that clients  can only extend their views. 
When this is the case, clients can \emph{always} commit their transactions, and thus \(\cancommitname_{\MR}\) is simply \(\true\). 

\mypar{Read Your Writes \((\RYW)\)}
This consistency model states that a client must always see all the versions written by the client itself. 
The \(\vshiftname_{\RYW}\) predicate thus states that after executing a transaction, a client 
contains all the versions it wrote in its view. This ensures that such versions will be included in the view of the client 
when committing future transactions.
Note that under \(\RYW\) the kv-store in \cref{fig:ryw-disallowed} is prohibited as
the initial version of \(\key\) holds value \(\val_0\) 
and client \(\cl\) tries to update the value of \(\key\) twice.  
For its first transaction \( \txid_{\cl}^1\), it reads the initial value \(\val_0\) and then writes a new version with value \(\val_1\). 
For its second transaction \( \txid_{\cl}^2\), it reads the initial value \(\val_0\) again and writes a new version with value \(\val_1\).
The \(\vshiftname_{\RYW}\) predicate rules out this example by requiring the client view after committing \(\txid_{\cl}^{1}\) to include the version it wrote.  
When this is the case, clients can always commit their transactions, and thus \(\cancommitname_{\RYW}\) is simply \(\true\).

The \(\MR\) and \(\RYW\) models,  together with the \emph{monotonic
  writes} (\(\MW\)) and \emph{write follows reads} (\(\WFR\)) models,  are collectively known as \emph{session guarantees}. 
Due to space constraints, the definitions associated with \(\MW\) and \(\WFR\) are given in \cite{shale-phd}. 

\begin{figure*}[t]
\newcommand{\SINGLEKV}{0.32\textwidth}
\newcommand{\RIGHTCOL}{0.61\textwidth}
\newcommand{\TWOKV}{0.49\textwidth}
\newcommand{\THREEKV}{0.61\textwidth}
\captionsetup[subfigure]{aboveskip=-5pt, belowskip=0pt}


\begin{tabularx}{\textwidth}{@{} X @{}| c @{}| c @{} }
\hline
\phantom{-} & \phantom{-} & \phantom{-} 
\\[-5pt]
\begin{subfigure}{0.34\textwidth}
\vspace{-6pt}
\begin{centertikz}[.65]

%Location x
\node(locx) {$\key_1 \mapsto$};
\draw pic at ([xshift=\tikzkvspace]locx.east) {vlist={versionx}{%
    /\;\;\;\;\;/\;\;\;\;\;\;\txid_0\;\;\;/\;\;\;\;\Set{\txid_1}\;\;\;\;
    , /\;/\;\;\;\;\txid_2\;\;\;\;/\;\;\;\;\emptyset\;\;\;\;\;
}};

\coordinate (A) at ([xshift=-25,yshift=3]versionx.center);
\coordinate (B) at ([xshift=0,yshift=-17]A.center);
\coordinate (C) at ([xshift=17,yshift=0]A.center);
\coordinate (D) at ([xshift=17,yshift=-17]A.center);
\coordinate (E) at ([xshift=37,yshift=0]C.center);
\coordinate (F) at ([xshift=0,yshift=-5]E.center);

\path[->, thick] ([yshift=5pt]A.center) edge[bend right=75] node[left,fill=white, opacity=.4, text opacity=1] {$\WR$} ([yshift=5pt]B.center)
([yshift=5pt]C.center) edge[bend left=30] node[above,fill=white, opacity=.4, text opacity=1] {$\WW$} ([yshift=5pt]E.center)
([yshift=5pt]D.center) edge[bend right=30] node[below, fill=white, opacity=.4, text opacity=1] {$\RW$} ([yshift=5pt]F.center);

\end{centertikz}%
\caption{Dependencies of kv-stores}
\label{fig:dependencies}
\end{subfigure}
%
&
%
\begin{subfigure}{\SINGLEKV}
\begin{centertikz}

%Location x
\node(locx) {$\key_1 \mapsto$};
\draw pic at ([xshift=\tikzkvspace]locx.east) {vlist={versionx}{%
    /\val_0/\txid_0/\Set{\txid_\cl^2}
    , /\val_1/\txid_1/\Set{\txid_\cl^1}
}};

\end{centertikz}%
\caption{Disallowed by \(\MR\)}
\label{fig:mr-disallowed}
\end{subfigure}
%
&
%
\begin{subfigure}{\SINGLEKV}
\begin{centertikz}%

%Location x
\node(locx) {$\key_1 \mapsto$};
\draw pic at ([xshift=\tikzkvspace]locx.east) {vlist={versionx}{%
    /\val_{0}/\txid_0/\Set{\txid_\cl^1,\txid_\cl^2}
    , /\val_{1}/\txid_\cl^1/\emptyset
    , /\val_{1}/\txid_\cl^2/\emptyset
}};
\end{centertikz}%
\caption{Disallowed by \(\RYW\)}
\label{fig:ryw-disallowed}
\end{subfigure}
\\
\hline
\\[-8pt]
\begin{subfigure}{0.34\textwidth}
\begin{centertikz}

\node(locx) {$\key \mapsto$};
\draw pic at ([xshift=\tikzkvspace]locx.east) {vlist={versionx}{%
    /\val_{0}/\txid_0/\Set{\txid,\txid'}
    , /\val_{1}/\txid/\emptyset
    , /\val_{1}/\txid'/\emptyset
}};

\end{centertikz}
\caption{Disallowed by \(\UA\)}
\vspace*{-10pt}
\label{fig:ua-disallowed}
\end{subfigure}

&

\multicolumn{2}{@{}c@{}}{%
\begin{subfigure}{\THREEKV}
\begin{centertikz}%

%Location x
\node(locx) {$\key_1 \mapsto$};
\draw pic at ([xshift=\tikzkvspace]locx.east) {vlist={versionx}{%
        /\val_0/\txid_0/\Set{\txid}
    , /\val_1/\txid^{1}_{\cl}/\emptyset
}};

%Location y
\path (versionx.east) + (0.75,0) node (locy) {$\key_2 \mapsto$};
\draw pic at ([xshift=\tikzkvspace]locy.east) {vlist={versiony}{%
    /\val_0/\txid_0/\emptyset
    , /\val_2/\txid^{2}_{\cl}/\{\txid^{1}_{\cl'}\}
}};

%Location z
\path (versiony.east) + (0.75,0) node (locz) {$\key_3 \mapsto$};
\draw pic at ([xshift=\tikzkvspace]locz.east) {vlist={versionz}{%
    /\val_0/\txid_0/\emptyset
    , /\val_3/\txid^{2}_{\cl'}/\Set{\txid}
}};

\end{centertikz}%
\caption{Disallowed by \( \CC\)}
\vspace*{-10pt}
\label{fig:wr-wfr-allowed-but-cc}
\end{subfigure}%
}%
\end{tabularx}\\[-1pt]
\begin{tabularx}{\textwidth}{@{} c | X @{}}
\hline
\phantom{-}& \phantom{-} \\[-8pt]
%
\begin{subfigure}{\TWOKV}%
\begin{centertikz}%

%Location x
\node(locx) {$\key_1 \mapsto$};
\draw pic at ([xshift=\tikzkvspace]locx.east) {vlist={versionx}{%
        /\val_0/\txid_0/\emptyset
        , /\val_1/\txid^{1}_{\cl}/\emptyset
        , /\val_2/\txid^{1}_{\cl'}/\Set{\txid}
}};

%Location y
\path (versionx.east) + (1,0) node (locy) {$\key_2 \mapsto$};
\draw pic at ([xshift=\tikzkvspace]locy.east) {vlist={versiony}{%
    /\val_0/\txid_0/\Set{\txid}
    , /\val_3/\txid^{1}_{\cl}/\emptyset
}};

\end{centertikz}%
\caption{Allowed by \(\CC\) and \( \UA \) but not \( \PSI \)}
\label{fig:cc-ua-allowed-but-psi}
\end{subfigure}%
&
\begin{subfigure}{\TWOKV}%
\begin{centertikz}%
%Location x
\node(locx) {$\key_1 \mapsto$};
\draw pic at ([xshift=\tikzkvspace]locx.east) {vlist={versionx}{%
    /\val_0/\txid_0/\big\{\txid_{\cl_2}^2\big\}
    , /\val_1/\txid/\big\{\txid_{\cl_1}^1\big\}
}};

%Location y
\path (versionx.east) + (0.75,0) node (locy) {$\key_2 \mapsto$};
\draw pic at ([xshift=\tikzkvspace]locy.east) {vlist={versiony}{%
    /\val_0/\txid_0/\big\{\txid_{\cl_1}^2\big\}
    , /\val_1/\txid'/\big\{\txid_{\cl_2}^1\big\}
}};
\end{centertikz}%
\caption{Long fork, disallowed by \(\CP\)}
\label{fig:cp-disallowed-2}
\label{fig:cp-disallowed}
\end{subfigure}%
\end{tabularx}\\[-1pt]
\begin{tabularx}{\textwidth}{@{} c | X @{}}
\hline
\phantom{-}& \phantom{-} \\[-8pt]
%
\begin{subfigure}{0.56\textwidth}
\begin{centertikz}%
%Location x
\node(locx) {$\key_1 \mapsto$};
\draw pic at ([xshift=\tikzkvspace]locx.east) {vlist={versionx}{%
    /\val_0/\txid_0/\Set{\txid_4}
    , /\val_1/\txid_1/\emptyset
    , /\val_2/\txid_2/\emptyset
}};

%Location y
\path (versionx.east) + (0.7,0) node (locy) {$\key_2 \mapsto$};
\draw pic at ([xshift=\tikzkvspace]locy.east) {vlist={versiony}{%
    /\val_0/\txid_0/\Set{\txid_2}
    , /\val_3/\txid_3/\Set{\txid_4}
    , /\val_4/\txid_4/\emptyset
}};

\end{centertikz}
\caption{Allowed by \( \UA \) and \( \CP \) but not \(\SI\)}%
\label{fig:si-disallowed}%
\end{subfigure}%
&
\begin{subfigure}{0.43\textwidth}
\begin{centertikz}%

%Location x
\node(locx) {$\key_1 \mapsto$};
\draw pic at ([xshift=\tikzkvspace]locx.east) {vlist={versionx}{%
    /\val_0/\txid_0/\Set{\txid_2}
    , /\val_1/\txid_1/\emptyset
}};

%Location y
\path (versionx.east) + (0.7,0) node (locy) {$\key_2 \mapsto$};
\draw pic at ([xshift=\tikzkvspace]locy.east) {vlist={versiony}{%
    /\val_0/\txid_0/\Set{\txid_1}
    , /\val_2/\txid_2/\emptyset
}};

\end{centertikz}%
\caption{Write skew, disallowed by \(\SER\)}
\label{fig:ser-disallowed}
\end{subfigure}%
\\ \hline
\end{tabularx}
%
\caption{Behaviours disallowed under different consistency models. Sub-figure \ref{fig:dependencies} 
shows the dependencies of transactions in kv-stores (values omitted).  }
\label{fig:anomalies}
\end{figure*}


\polish{I've changed this paragraph, what do you think?} 
We now give the definitions of well-known consistency models in distributed databases,
including \(\CC\) \citep{ev_transactions,cops,causal-def}, 
\( \PSI\) \citep{NMSI,PSI},
\(\SI\) \citep{si} 
and \(\SER\) \citep{Papadimitriou-ser}.
The  $\vshiftname $ relation for these consistency models, given   in
\cref{fig:execution-tests}, 
is simply 
$
\vshiftname_{\MR \cap \RYW}  (\mkvs, u. \mkvs', u')
= \vshiftname_{\MR}(\mkvs, u. \mkvs', u') \cap \vshiftname_{\RYW}
(\mkvs, u. \mkvs', u').
$
%\(\MR\) and \( \RYW \) are straightforward to implement 
%just by the client maintaining  some meta-data about it own history.
The \(\cancommitname_{}\) relation is defined by
\(\cancommit \mkvs \vi \fp \defeq \closed(\mkvs, \vi, \rel[\et]) \)
where $\rel[\et]$ is given for each exection test in
\cref{fig:execution-tests} as a combination of \(\SO\) and the
dependency relations. 
We use two less-known consistency models, 
\emph{update atomic} (\(\UA\)) and \emph{consistent prefix} \( (\CP)
\).
In \citep{giovanni_concur16,cp-def,framework-concur}, 
the definition of \( \SI \) on abstract executions 
can be separated into the conjunction of \(\UA\) and \( \CP\). 
Similarly, the definition of \( \PSI \) on abstract executions can be separated 
into the conjunction of \( \UA \) and \( \CC \) \cite{framework-concur}.
Interestingly, this is not quite the case for the
consistency definitions presented here. 



\mypar{Causal Consistency \((\CC)\)}

This model states that,  if a client view includes 
a version \(\ver\) written by \( \txid \) prior to committing a transaction, 
then it must also include the versions which \(\txid\) observes.
Clearly, \(\txid\) observes all versions that \(\txid\) reads. 
Moreover, \(\txid\) observes all previous transactions from the same client.
This is captured by \(\cancommitname_{\CC}\) in \cref{fig:execution_tests}, 
defined as \(\closed(\mkvs, u, \rel[\CC])\) with \(\rel[\CC] \defeq \SO \cup \WR[\mkvs]\).
For example, the kv-store of \cref{fig:wr-wfr-allowed-but-cc} 
is disallowed by \(\CC\): the \(\key_3\) version with value \(\val_3\) depends on 
the \(\key_1\) version with value \(\val_1\). 
However, \(\txid\) must have been committed by a client
whose view included \(\val_3\) of \( \key_3\), but not \(\val_1\) of \( \key_1\).

\mypar{Update Atomic \((\UA)\)}
This consistency model has been proposed in \citet{framework-concur} 
and implemented in \citet{rola}.
\(\UA\) disallows concurrent transactions writing to the same key,
a property known as \emph{write-conflict freedom}:  
when two transactions write to the same key, one must see the version 
written by the other.
Write-conflict freedom is enforced by \(\cancommitname_{\UA}\) which
allows a client to write to key \(\key\) only if its view includes all
versions of \(\key\): that is,  its view is closed with respect to the \(\WWInv(\key)\) relation for all keys \(\key\) written in the fingerprint \(\fp\).
This prevents the kv-store of \cref{fig:ua-disallowed},
as \(\txid\) and \(\txid'\) concurrently increment the initial version of \(\key\) by \(1\).
As client views must include the initial versions, once \(\txid\) commits a new version \(\ver\) with value \(\val_1\) to \(\key\), then \(\txid'\) must include \(\ver\) in its view as there is a \(\WW\) edge from the initial version to \(\ver\). 
As such, when \(\txid'\) increments \(\key\), it must read from \(\ver\) and not the initial version.% (\cref{fig:ua-disallowed}).

\mypar{Parallel Snapshot Isolation \((\PSI)\)} 
This consistency model states that: 
\begin{enumerate*}
	\item if a client view includes a version \(\ver\) written by \( \txid \) prior to committing a transaction, 
then it must also include the versions that \(\txid\) observes; and
	\item there are no write-conflicts.
\end{enumerate*}

On abstract executions, where there is a total order over transactions,  
\(\PSI\) can be formally defined as the composition of
 \(\CC\) and \(\UA\) \cite{framework-concur}. In constrast, it is not possible to define \(\cancommitname_{\PSI}\) as
the conjunction of the \(\cancommitname_{\CC}\) and
\(\cancommitname_{\UA}\) relations. 
This is for two reasons. 
First, the conjunction would only mandate that \(\vi\) be closed with respect to 
\(\rel[\CC]\) and \(\rel[\UA]\) {individually}, but {not} with respect to their {union}.
Recall that closure is defined in terms of the transitive closure of a given relation 
and thus the closure of \(\rel[\CC]\) and \(\rel[\UA]\) is smaller than the closure of \(\rel[\CC] \cup \rel[\UA]\).
As such, we define \(\cancommitname_{\PSI}\) as closure with respect to \(\rel[\PSI] \) which includes \( \rel[\CC] \cup \rel[\UA]\).
Second, recall that \(\cancommitname_{\CC}\) requires that
if a client view includes 
a version \(\ver\) written by \( \txid' \) prior to committing a transaction, 
then it must also include the versions which \(\txid'\) observes.
For example, the client view of transaction \( \txid \) in \cref{fig:cc-ua-allowed-but-psi}
includes versions written by \( \txidinit\) and \( \txid^{1}_{\cl'} \),
which satisfies \(\cancommitname_{\CC}\).
However, \(\cancommitname_{\UA}\) requires a transaction writing 
to \(\key\) to observe all previous versions of \(\key\):
that is, when write-conflict freedom is enforced, a version \(\ver\) of \( \key \)
written by a transaction, 
for example \( \txid^{1}_{\cl'} \) in in \cref{fig:cc-ua-allowed-but-psi},
observes all previous versions of \(\key\), for example version of \( \txid^{1}_{\cl} \). 
This leads us to include write-write dependencies (\(\WW[\mkvs]\)) in \(\rel[\PSI]\). 
Note that the kv-store in \cref{fig:cc-ua-allowed-but-psi} shows an
example kv-store that satisfies \(\cancommitname_{\CC} \) and \( \cancommitname_{\UA}\), 
but not \(\cancommitname_{\PSI}\). 


\polish{Goodness, this explanation of PSI is so technical, its
  probably too late to do more now.}

\polish{Here and below, you make such a big deal about the total order
  of abstract executions, without ever explaining the point. For
  example, in above, you give two reasons why we do not have the
  conjunction of UA and CP, none of them mention total  order at all. I
  don't know what to do.}

\polish{We had great discussions about how WW was needed for
  transitivity. Here, your explanation is the following which I find 
  incomprehensible: Second, recall that canCommitUA requires a transaction writing to
k to observe all previous versions of k: that is, when write-conflict freedom is enforced, a
version v of k written by a transaction observes all previous versions of k. This leads us to
 include write-write dependencies WW in PSI.}

\mypar{Consistent Prefix \((\CP)\)}
\label{para:cp}
If the total order in which transactions commit is known, then \(\CP\)
can be described as a strengthening of \(\CC\)~\cite{laws}: 
if a client sees the versions written by a transaction \(\txid\),
then it must also see all versions written by transactions that \emph{commit} before \(\txid\). 
Although kv-stores only provide \emph{partial} information about the
order of  transaction commits, 
this is sufficient to formalise $\CP$.

We can approximate the order in which transactions 
commit using \(\WR[\mkvs], \allowbreak \WW[\mkvs], \allowbreak \RW[\mkvs]\) and \(\SO\). 
This approximation is perhaps best understood in terms of an idealised implementation of \(\CP\) on a centralised system,
where the snapshot of a transaction is determined at its \emph{start point} and its effects are made visible to future transactions at its \emph{commit point}.
In this implementation, if \((\txid,\txid') \in \WR\), then 
\(\txid\) must commit before \(\txid'\) starts, and hence before \(\txid'\) commits.
Similarly, if \((\txid,\txid') \in \SO\), then \(\txid\) commits before \(\txid'\) starts, 
and thus before \(\txid'\) commits.
Recall that, if \((\txid'', \txid') \in \RW\),
then \(\txid''\) reads a version that is later overwritten by
\(\txid'\):
that is, \(\txid''\) cannot see the write of \(\txid'\), and thus \(\txid''\) must start before 
\(\txid'\) commits. 
As such, if \(\txid\) commits before \(\txid''\) starts 
(\((\txid, \txid'') \in \WR\) or \((\txid,\txid'') \in \SO\)), 
and \((\txid'', \txid') \in \RW\), then \(\txid\) must commit before 
\(\txid'\) commits. 
In other words, if \((\txid,\txid') \in \WR;\RW\) or \((\txid,\txid') \in \SO;\RW\), then \(\txid\) commits before \(\txid'\).
Finally, if \((\txid,\txid') \in\WW\), then \(\txid\) must commit before \(\txid'\). 
We therefore define \(\rel[\CP] \defeq (\WR[\mkvs]; \RW[\mkvs]\rflx \cup \SO;  \RW[\mkvs]\rflx \cup \WW)\), approximating the order in which transactions commit. 
%
As shown in \citet{laws}, the set \((\rel[\CP]^{+})^{-1}(\txid)\) contains all transactions that must be observed by \(\txid\) under \(\CP\). 
We thus define \(\cancommitname_{\CP}\) by requiring closure with
respect to \(\rel[\CP]\).

%\polish{[16] uses \((\rel[\CP]^{-1})^{+}(\txid)\) , I don't know why
  %we are not using the same (with reflexicity) for prefix xlosure.} 

The $\CP$ model disallows the \emph{long fork anomaly} in \cref{fig:cp-disallowed}, where \(\cl_1\) and \(\cl_2\) observe the updates to \(\key_1\) and \(\key_2\) 
in different orders. 
Assuming without loss of generality that \( \txid_{\cl_1}^{2} \) commits 
before \( \txid_{\cl_2}^{2} \), then \(\cl_2\) sees the \(\key_1\) version with value \(\val_0\) before committing \( \txid_{\cl_2}^{2} \). 
However, as \(\txid {\xrightarrow{\!\WR[\mkvs]\!}} \txid_{\cl_{1}}^{1} 
{\xrightarrow{\!\SO\!}} \txid_{\cl_{1}}^{2} {\xrightarrow{\!\RW\!}} \txid' {\xrightarrow{\!\WR\!}} \txid_{\cl_{2}}^{1} \)
and \( \txid_{\cl_2}^{2} \) must see the versions written by \( \txid_{\cl_2}^{1} \) before committing,
then \( \txid_{\cl_2}^{2} \) must also see the \(\key_1\) version with 
value \(\val_2\), leading to a contradiction.

%\polish{This is a really good write-up of CP, bravo.} 

\mypar{Snapshot Isolation \((\SI)\)}
On abstract executions,  where there is a total order over transactions,  
\(\SI\) can be defined as the composition of \(\CP\) and \(\UA\). 
However, as with \(\PSI\), we cannot define \(\cancommitname_{\SI}\) as the conjunction of their associated \(\cancommitname\) predicates. 
Rather, we define \(\cancommitname_{\SI}\) as closure with respect to \(\rel[\SI]\) which includes \(\rel[\CP] \cup \rel[\UA]\).
Observe that \cref{fig:si-disallowed} shows an example kv-store that
satisfies \(\cancommitname_{\UA} \) and \( \cancommitname_{\CP}\), 
but not \(\cancommitname_{\SI}\).
Additionally, we include \(\WW;\RW\) in \(\rel[\SI]\). 
This is because,  when the centralised \(\CP\) implementation
(discussed before) is strengthened with write-conflict freedom, then a write-write dependency between transactions \(\txid\) and \(\txid'\) 
does not only mandate that \(\txid\) commit before \(\txid'\) commits, but also before \(\txid'\) starts. 
Consequently, if \((\txid, \txid') \in \WW ;\RW\), then \(\txid\) must
commit 
before \(\txid'\) does.

\mypar{(Strict) serialisability \((\SER)\)}
Serialisability is the strongest consistency model in settings that abstract from aborted transactions, 
requiring that transactions execute in a total sequential order. 
The \(\cancommitname_{\SER}\) thus allows clients to commit transactions only when 
their view of the kv-store is complete, \ie the client view is closed with respect to \(\WW^{-1}\).
This requirement prevents the kv-store in \cref{fig:ser-disallowed}: 
if, without loss of generality, \(\txid_1\) commits before \(\txid_2\),
then the client committing \(\txid_2\) must see the \(\key_1\) version written by \(\txid_1\), 
and thus cannot read the outdated value \(\val_0\) for \(\key_1\). 

\mypar{Weak Snapshot Isolation \((\WSI)\): A New Consistency Model} 
\label{sec:new_cm}
Kv-stores and execution tests are useful for investigating new 
consistency models.  
One example is the consistency model induced by combining 
\(\CP\) and \(\UA\), which we refer to as \emph{Weak Snapshot Isolation} (\(\WSI\)). 
Because \(\WSI\) is stronger than \(\CP\) and \(\UA\) by definition, 
it forbids all the  anomalies forbidden by these consistency models, \eg
the long fork (\cref{fig:cp-disallowed}) and the lost update (\cref{fig:ua-disallowed}). 
Moreover, \(\WSI\) is strictly weaker than \(\SI\). 
As such, \(\WSI\) allows all \(\SI\) anomalies, \eg the write skew (\cref{fig:ser-disallowed}), 
and further allows behaviours not allowed under \(\SI\) such as that in \cref{fig:si-disallowed}.
The kv-store \(\mkvs\) is reachable by executing transactions 
\(\txid_{1}, \txid_{2}, \txid_{3}\) and \(\txid_{4}\) in order. 
In particular, \(\txid_{4}\) is executed using \(\vi {=} \{\key_{1} \mapsto \{0\}, \key_{2} \mapsto \{0,1\}\}\). 
However, \(\mkvs\) is not reachable under \(\ET[\SI]\). 
This is because \(\txid_{4}\) cannot be executed using \(\vi\) under \(\SI\): 
\(\txid_{4}\) reads the \(\key_2\) version written by \(\txid_3\);
but as \((\txid_{2},\txid_{3}) \in \RW \) and \((\txid_{1} ,\txid_2) \in \WW\), 
then \(\vi\) should contain the \(\key_{1}\) version written by \(\txid_{1}\), 
contradicting the fact that \(\txid_{4}\) reads the initial version of \(\key_1\).
The two consistency models are very similar in that 
many applications that are correct under \(\SI\) are also correct under \(\WSI\).
We give examples of such applications in \cref{sec:program-analysis}.

\mypar{Correctness of \( \et \)}
Our definitions of consistency models over kv-stores
and client views are equivalent to well-known definitions of
consistency models over abstract executions \cite{framework-concur}, and hence over dependency graphs \cite{laws}.
Given a model $M$ in \cref{fig:execution.tests}, 
let $\CMs(\et[M])$ denote the consistency model induced by execution
test $\et[M]$ of $M$. For example, when $M = \CC$, then $\CMs(\et[\CC])$ denotes the consistency model induced by
execution test $\ET[\CC]$ of \( \CC \). Also, let 
$\CMs(\visaxioms[M])$ denote the  consistency model of $M$ defined on
abstract excutions, 
induced by the set of axioms $\visaxioms[M]$\cite{framework-concur}. For example, when $M =
\CC$, then  $\CMs(\visaxioms[\CC])$ denotes the  consistency mode of
\( \CC \) induced by the \( \CC \)  axioms on abstract executions. 

\SpaceAboveDef
\begin{theorem}
For all consistency models $M$ in \cref{fig:execution.tests}: $ \CMs(\et[M]) = \CMs(\visaxioms[M])$ 
\end{theorem}
\SpaceBelowDef

The full proof is given in \cite{shale-phd}, where we define an \emph{intermediate} operational semantics on
abstract executions parametrised by axioms, and each step corresponds to an atomic transaction.
This is in contrast to \cite{sureshConcur} which defines a more fine-grained operational semantics. 
%This semantics on abstract executions is an intermediate  for proving our definitions of consistency models.
             

