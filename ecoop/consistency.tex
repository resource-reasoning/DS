\section{Consistency Models Using Execution Tests on Kv-stores}
\label{sec:cm}
We define what it means for a kv-store 
to be in a consistent state. Many different consistency models for
distributed databases have 
been proposed in the literature
\cite{gdur,ramp,CORFU,tango,si,distrsi,clocksi,redblue,rola,cops,PSI-RA,NMSI,PSI,wren},
capturing different trade-offs 
between  performance and application
correctness: examples range from  \emph{serialisability}, a strong
consistency model which only allows kv-stores 
obtained  from a serial execution of transactions
with inevitable performance drawbacks, to  \emph{eventual consistency},  a weak consistency model
which imposes few conditions on the structure of a kv-store leading to
good performance but anomalous behaviour.
We define consistency models for our kv-stores,
by introducing the notion of 
\emph{execution test} which specifies  whether a client is allowed to commit a transaction in a given 
kv-store. Each execution test induces a consistency model as the set of kv-stores obtained 
by having clients non-deterministically commit transactions so long as  the constraints 
imposed by the execution test are satisfied.
We explore a range of execution tests  associated with well-known consistency models in the literature. 
In \cite{shale-phd}, we demonstrate that our operational
formulation of  consistency models over kv-stores using execution
tests are  equivalent to the established declarative definitions of
consistency models over abstract executions \cite{ev_transactions,framework-concur}.

\SpaceAboveDef
\begin{definition}[Execution tests]
\label{def:execution.test}
An \emph{execution test} is a set of tuples
\(\ET \subseteq \MKVSs {\times} \Views {\times} \allowbreak \Fingerprints {\times} \MKVSs {\times} \Views\) 
such that, for all \((\mkvs, \vi, \fp, \mkvs', \vi') \in \ET\): 
\begin{enumerate*}
	\item \(\vi {\in}  \allowbreak \Views(\mkvs)\) and \(\vi' \in \Views(\mkvs')\); 
	\item \(\cancommit \mkvs \vi \fp\); 
	\item \(\vshift \mkvs {\allowbreak \vi} {\mkvs'} {\vi'}\); and 
	\item for all \(\key \!\in\! \mkvs\) and \(\val \!\in\! \Val\), if \((\otR, \key, \val) \!\in\! \fp \) then \(	\mkvs(\key, \max{}_{<}(\vi(\key))) {=} \val   \).
\end{enumerate*}
\end{definition}
\SpaceBelowDef

Intuitively, \((\mkvs, \vi, \fp, \mkvs', \vi') \in \ET\) means that, under the execution test \(\ET\),
a client with initial view \(\vi\) over kv-store \(\mkvs\) can commit a transaction with 
fingerprint \(\fp\) to obtain the resulting kv-store \(\mkvs'\) (\cref{def:updatekv}) while shifting its view
to \(\vi'\). We adopt the notation \(\ET \vdash (\mkvs, \vi) \csat \fp: (\mkvs', \vi')\) to capture this intuition. 
Note that the last condition in \cref{def:execution.test} enforces the last-write-wins
policy~\cite{vogels:2009:ec:1435417.1435432}: 
a transaction always reads the most recent writes from the initial view \(\vi\).  

\SpaceAboveDef
\begin{definition}[Consistency model]
\label{def:cm}
The \emph{consistency model} induced by an execution test \(\ET\) is defined as 
\(
\CMs(\ET) \defeq 
\Set{\mkvs}[ 
\exsts{\mkvs_0,\vienv_0,\clenv,\prog}
(\mkvs_0,\vienv_0,\clenv),\prog \toCMD{\stub}_{\et}^{*} (\mkvs, \stub, \stub),\stub
]
\).
\end{definition}
\SpaceBelowDef

The largest execution test is denoted by \(\ET[\top]\), where for all \(\mkvs, \mkvs', \vi, \vi, \fp\):%

\SpaceAboveMath
\[
	\cancommit[{\ET[\top]}] \mkvs \vi \fp \defiff \mathsf{true}
	\qquad  \text{and} \qquad 
	\vshift[{\ET[\top]}] \mkvs \vi {\mkvs'} {\vi'} \defiff \mathsf{true}
\]%
\SpaceBelowMath

The consistency model induced by \(\ET[\top]\) 
corresponds to the \emph{Read Atomic} \cite{ramp}, a variant of \emph{Eventual 
Consistency} \cite{ev_transactions} for atomic transactions. 
We give many examples of execution tests in the following sub-section. 

\label{subsec:cm_examples}
We give several examples of execution tests which give rise to consistency
models on kv-stores.
Recall that the snapshot property and the last write wins are hard-wired into our model. 
This means that we can only define  consistency models that satisfy these two constraints. 
Although this forbids us to express interesting consistency models such as \emph{Read Committed},
we are able to express a large variety of consistency models employed by distributed kv-stores.

\mypar{Notation}
Given relations \(\mathsf r, \mathsf r' \subseteq \TypeFont{A} \times \TypeFont{A}\),
we write:  \(\mathsf r\rflx\), \(\mathsf r^+\) and \(\mathsf r^*\) for its reflexive, transitive and reflexive-transitive closures of \(\mathsf r\), respectively;
\(\mathsf r^{-1}\) for its inverse;
\(a_1 \toEDGE{\mathsf r} a_2\) for \((a_1, a_2) \in \mathsf r\);
and \( \mathsf r; \mathsf r'\) for \( \Set{(a_1,a_2)}[\exsts{a} (a_1,a) \in \mathsf r\land (a,a_2) \in \mathsf r']\).

An execution test \(\ET\) (\cref{def:execution.test})
has the form \((\mkvs,\vi, \fp, \allowbreak \mkvs', \vi')\) 
where  \(\cancommit \mkvs \vi \fp\) and \(\vshift \mkvs \vi {\mkvs'} {\vi'}\). 
We define \(\cancommitname\) and \(\vshiftname\) for several consistency
models, using some auxiliary definitions. 


\mypar{Prefix Closure}
Given a kv-store \(\mkvs\) and a view \(\vi\), the {\em set of visible
transactions} is
\( \Tx(\mkvs, \vi)  \defeq \Set{\wtOf[\mkvs(\key, i)] }[ i \in \vi(\key)] \).
Given a binary relation on transactions, \(\rel \subseteq \TxIDs \times \TxIDs\),
we say that a view \(\vi\) is closed with respect to a kv-store \(\mkvs\) and \(\rel\), written \(\closed[\mkvs,\vi,\rel]\), iff:  
\(
	\Tx(\mkvs, \vi) = 
	\left( (\rel^*)^{-1} \left( \Tx(\mkvs, \vi) \right) \right) \setminus \Set{\txid }[ \fora{\key,i} \txid \neq \wtOf[\mkvs(\key,i)] ] .
\)
That is, if transaction \(\txid\) is visible in \(\vi\) as \( \txid \in \Tx(\mkvs, \vi) \),
then all the transactions \( \txid'  \) that are \(\rel^*\)-before \(\txid\), \ie \(\txid' \in (\rel^*)^{-1} \left( \txid \right)\),
and are not read-only transactions \( \txid' \notin \Set{\txid'' }[ \fora{\key,i} \txid'' \neq \wtOf[\mkvs(\key,i)] ] \),
are also visible in \(\vi\), \ie \( \txid' \in \Tx(\mkvs, \vi) \).

\mypar{Dependency Relations}
We define transaction dependency relations for kv-stores.
\Cref{fig:dependencies} illustrates an example kv-store and
its transaction dependency relations.
Given a kv-store \(\mkvs\), a key \(\key\) and 
indexes \(i,j\) such that  \(0 \leq i < j < \Abs{ \mkvs(\key) }\), 
if there exists \(\txid_i, \txidset_i, \txid\) such that 
\(\mkvs(\key, i)  {=} (\stub, \txid_{i}, \txidset_{i})\), \(\mkvs(\key,j) {=} (\stub, \txid_{j}, \stub)\)
and \(\txid \in \txidset_{i}\), 
then, for every key \( \key \), there is:

\begin{enumerate} 
\item a \emph{Write-Read} dependency from \(\txid_{i}\) to \(\txid\), written \((\txid_{i},\txid) \in \WR[\mkvs](\key)\),
which  intuitively means that \( \txid_{i} \) commits before the start of \( \txid \), depicted in \cref{fig:dependencies-time-line};
\item a \emph{Write-Write} dependency from \(\txid_{i}\) to \(\txid_{j}\), 
written \((\txid_{i},\txid_{j}) \in c\WW[\mkvs](\key) \),
which intuitively means that \( \txid_{i} \) commits before the commit of \( \txid_j \), depicted in \cref{fig:dependencies-time-line}; and 
\item a \emph{Read-Write} anti-dependency from \(\txid\) to \(\txid_{j}\), if 
\(\txid {\neq} \txid_{j}\), written \((\txid,\txid_j) {\in} \RW[\mkvs](\key)\),
which intuitively  means that \( \txid \) starts before the commit of \( \txid_j \), depicted in \cref{fig:dependencies-time-line}.
\end{enumerate}

\begin{figure}[t]
\centering
\begin{tikzpicture}

\node at (0,0) (a) {\(\txid_i\)};
\node at (2,-1) (b) {\(\txid_j\)};
\node at (4.5,0) (c) {\(\txid\)};

\coordinate (a1) at ($(a) + (3,0)$);
\coordinate (b1) at ($(b) + (7,0)$);
\coordinate (c1) at ($(c) + (3,0)$);
%\coordinate (d1) at ($(d) + (3,0)$);

\draw[|-,densely dashed] (a) -- ($(a)!0.5!(a1)$);
\draw[-|] ($(a)!0.5!(a1)$) -- (a1);
\draw[|-,densely dashed] (b) -- ($(b)!0.5!(b1)$);
\draw[-|] ($(b)!0.5!(b1)$) -- (b1);
\draw[|-] (c) -- ($(c)!0.5!(c1)$);
\draw[-|,densely dashed] ($(c)!0.5!(c1)$) -- (c1);
%\draw[|-|] (d) -- (d1);

\path[->, thick] (a1) edge[bend left=50] node[above, text opacity=1] {$\WR$} ($(c)+(0.3,0.05)$);
\path[->, thick] (a1) edge node[below, text opacity=1] {$\WW$} (b1);
\path[->, thick] (c) edge node[above, text opacity=1] {$\RW$} (b1);

\end{tikzpicture}

\vspace*{-10pt}

\hrulefill

\caption{An example of dependencies between transactions with respect to 
the time line of the starts and commits of these transactions 
(dashed line being able to stretched)}
\label{fig:dependencies-time-line}

\end{figure}


\noindent 
In centralised databases, where there is a global time, 
the dependency relations can be determined by the start and commit time of transaction as in \cref{fig:dependencies-time-line}.
However, in general, there is no global time in distributed databases.
In these distributed databases,
the write-read dependency \( \WR \) is directly determined if a transaction reads another transaction.
The write-write dependency \( \WW \) is given by the \emph{last-write-wins} resolution policy,
which determines the order between transactions, if they write to the same key.
The read-write anti-dependency \( \RW \) can be derived from \( \WR \) and \( \WW \) in the sense that
if \( (\txid, \txid') \in \WR \) and \( (\txid, \txid'') \in \WW \), then  \( (\txid',\txid'') \in \RW \).
We adopt the same names as the dependency relations for dependency graphs \citep{adya}
to emphasise the similarity.
However, the relations here do \emph{not} depend on those relations in dependency graphs.

We give several definitions of
execution tests using \vshiftname and \cancommitname in \cref{fig:execution_tests}. 
In \cite{shale-phd},
we show that our definitions correspond to
the well-known declarative definitions of consistency models on abstract executions.

\begin{figure*}[!t]
\small
\begin{tabularx}{\textwidth}{ @{} X r r ||  X  r r @{} }
\hline
Model & Initial \( \txidset \) & Relation \( R \) &
Model & Initial \( \txidset \) & Relation \( R \)
\\
\hline
\MR & \(\clRead[\mkvs,\cl]\) & \( \emptyset \)
&
\UA &   \( \uaWrite[\mkvs,\fp] \) & \( \bigcup_{(\otW,\key,\stub) \in \fp} \WW_{\mkvs}(\key) \)
\\
\MW & \( \emptyset \) & \( \SO\rflx \)
&
\PSI & \( \dagger \cup \uaWrite[\mkvs,\fp] \) & \( \SO \cup \WR_{\mkvs} \cup \WW_{\mkvs}\)
\\
\RYW & \( \clWrite[\mkvs,\cl] \) & \( \emptyset \)
&
\CP & \( \dagger  \)  &\(  \ddagger  \)
\\
\WFR & \( \emptyset \) & \( \WR_{\mkvs} ; \SO\rflx \)
&
$\SI$ & \( \dagger \cup \uaWrite[\mkvs,\fp] \) & \( \ddagger \cup (\WW_\mkvs; \RW_\mkvs) \)
\\
\CC & \( \dagger \)  & \( \SO \cup \WR_{\mkvs} \)
&
\SER & \( \Set{\txid}[\txid \in \mkvs] \) & \( \emptyset \) \\
\hline
\end{tabularx}%
%
\begin{align*}
    \dagger & \equiv 
    \clRead[\mkvs,\cl] \cup \clWrite[\mkvs,\cl] 
    \quad \ddagger 
    \equiv 
    (\SO ; \RW_{\mkvs}\rflx) \cup (\WR_{\mkvs} ; \RW_{\mkvs}\rflx) \cup \WW_\mkvs 
\end{align*}
%
\begin{align*}
    \WR_{\mkvs} &\defeq \bigcup_{\key \in \Keys} \WR_\mkvs(\key)
    & \WR_\mkvs(\key) & \defeq
    \Set{ (\txid, \txid') }[ \exsts{ i }\txid = \wtOf(\mkvs(\key, i)) \land \txid' \in \rsOf(\mkvs(\key, i))]\\
    \WW_{\mkvs} &\defeq  \bigcup_{\key \in \Keys} \WW_\mkvs(\key) 
    & \WW_\mkvs(\key)  & \defeq 
    \Set{ (\txid, \txid') }[ \exsts{ i, j } \txid = \wtOf(\mkvs(\key, i)) \land \txid' = \wtOf(\mkvs(\key, j)) \land i < j]\\
    \RW_{\mkvs} &\defeq   \bigcup_{\key \in \Keys} \RW_\mkvs(\key)  
    & \RW_\mkvs(\key) & \defeq \Set{ (\txid, \txid') }[%
        \exsts{ i, j } \txid \in \rsOf(\mkvs(\key, i)) \land \txid' = \wtOf[\mkvs(\key, j)]%
        \land i < j \land \txid \neq \txid'%
    ]
\end{align*}
%
\begin{align*}
    \lfpTx[\mkvs,\txidset, R] & \defeq \mu X . \txidset \cup R^{-1}(X) \\
    \extRead[\mkvs,\vi,\fp] & \defeq \Set{\wtOf[\mkvs(\key,\max(\vi(\key)))]}[%
        \fora{\key,\val} (\otR,\key,\val) \in \fp 
        \implies 
        \val = \valueOf[\mkvs(\key,\max(\vi(\key)))]%
    ] \\
    \clRead[\mkvs,\cl] & \defeq \Set{\wtOf[\mkvs(\key,i)]}[ \exsts{n} \txid_{\cl}^{n} \in \rsOf[\mkvs(\key,i)] ] 
    \quad \clWrite[\mkvs,\cl] \defeq \Set{\wtOf[\mkvs(\key,i)]}[ \exsts{n} \txid_{\cl}^{n} \in \wtOf[\mkvs(\key,i)] ] \\
    \Tx[\mkvs, \vi] & \defeq 
    \Set{ \wtOf[\mkvs(\key, i)] }[ \key \in \Keys \land i \in \vi(\key) ]
    \quad 
    \getView[\mkvs, \txidset] \defeq 
    \lambda \key. \Set{0} \cup \Set{ i }[\wtOf(\mkvs(\key, i)) \in \txidset] \\
    \uaWrite[\mkvs,\fp] & \defeq \Set{\wtOf[\mkvs(\key,\abs{\mkvs(\key)} - 1)]}[\exsts{\val} (\otW,\key,\val) \in \fp] 
\end{align*}
%
\hrulefill
%
\caption{Execution tests of client-centric (left) and data-centric (right) consistency models, 
with $\SO$ as defined in \cref{subsec:kvstores}. 
All free variables are universally quantified.
}
\label{fig:execution.tests}
\label{fig:execution_tests}
\label{fig:execution-tests}
\end{figure*}



\mypar{Monotonic Reads \((\MR)\)}
This consistency model states that when committing, a client
cannot lose information in that it can only see increasingly more up-to-date versions from a kv-store.
This prevents, for example, the kv-store of \cref{fig:mr-disallowed},
since client \(\cl\) first reads the latest version of \(\key\) in \(\txid_{\cl}^{1}\), 
and then reads the older, initial version of \(\key\) in \(\txid_{\cl}^{2}\).  
As such, the \(\vshiftname_{\MR}\) predicate in \cref{fig:execution_tests} ensures that clients  can only extend their views. 
When this is the case, clients can \emph{always} commit their transactions, and thus \(\cancommitname_{\MR}\) is simply \(\true\). 

\mypar{Read Your Writes \((\RYW)\)}
This consistency model states that a client must always see all the versions written by the client itself. 
The \(\vshiftname_{\RYW}\) predicate thus states that after executing a transaction, a client 
contains all the versions it wrote in its view. This ensures that such versions will be included in the view of the client 
when committing future transactions.
Note that under \(\RYW\) the kv-store in \cref{fig:ryw-disallowed} is prohibited as
the initial version of \(\key\) holds value \(\val_0\) 
and client \(\cl\) tries to update the value of \(\key\) twice.  
For its first transaction \( \txid_{\cl}^1\), it reads the initial value \(\val_0\) and then writes a new version with value \(\val_1\). 
For its second transaction \( \txid_{\cl}^2\), it reads again the initial value \(\val_0\) again and write a new version with value \(\val_1\).
The \(\vshiftname_{\RYW}\) predicate rules out this example by requiring that
the client view, after it commits the transaction  \(\txid_{\cl}^{1}\), includes the version it wrote.  
When this is the case, clients can always commit their transactions, and thus \(\cancommitname_{\RYW}\) is simply \(\true\).

The \(\MR\) and \(\RYW\) models together with \emph{monotonic writes} (\(\MW\)) and \emph{write follows reads} (\(\WFR\)) models  are collectively known as \emph{session guarantees}. 
Due to space constraints, the definitions associated with \(\MW\) and \(\WFR\) are in \cref{sec:full-semantics}. 

\begin{figure*}[t]
\captionsetup[subfigure]{aboveskip=-5pt, belowskip=5pt}
\begin{tabular}{@{} c | c | c @{}}
\hline
\phantom{-}& \phantom{-}& \phantom{-}\\
\begin{subfigure}{0.2\textwidth}
\begin{centertikz}

%Location x
\node(locx) {$\ke_1 \mapsto$};
\draw pic at ([xshift=\tikzkvspace]locx.east) {vlist={versionx}{%
    /$\val_0$/$\txid_0$/$\Set{\txid_\cl^2}$
    , /$\val_1$/$\txid_1$/$\Set{\txid_\cl^1}$
}};

\end{centertikz}\vspace{5pt}%
\caption{Disallowed by \(\MRd\)}
\label{fig:mr-disallowed}
\end{subfigure}
%\quad
&
\begin{subfigure}{0.36\textwidth}
\begin{centertikz}

%Location x

\node(locx) {$\ke_1 \mapsto$};
\draw pic at ([xshift=\tikzkvspace]locx.east) {vlist={versionx}{%
    /$\val_0$/$\txid_0$/$\Set{\txid'}$
    , /$\val_1$/$\txid_\cl^1$/$\emptyset$
}};

%Location y
\path (versionx.east) + (1,0) node (locy) {$\ke_2 \mapsto$};
\draw pic at ([xshift=\tikzkvspace]locy.east) {vlist={versiony}{%
    /$\val_0$/$\txid_0$/$\emptyset$
    , /$\val_2$/$\txid_\cl^2$/$\Set{\txid'}$
}};

\end{centertikz}\vspace{5pt}
\caption{Disallowed by \(\MW\)}
\label{fig:mw-disallowed}
\end{subfigure}
%\quad
&
\begin{subfigure}{0.39\textwidth}
\begin{centertikz}

%Location x
\node(locx) {$\ke_1 \mapsto$};
\draw pic at ([xshift=\tikzkvspace]locx.east) {vlist={versionx}{%
    /$\val_0$/$\txid_0$/$\Set{\txid}$
    , /$\val_1$/$\txid'$/$\Set{\txid_\cl^1}$
}};

%Location y
\path (versionx.east) + (1,0) node (locy) {$\ke_2 \mapsto$};
\draw pic at ([xshift=\tikzkvspace]locy.east) {vlist={versiony}{%
    /$\val_0$/$\txid_0$/$\emptyset$
    , /$\val_2$/$\txid_\cl^2$/$\Set{\txid}$
}};

\end{centertikz}

\vspace{5pt}
\caption{Disallowed by \(\WFR\)}
\label{fig:wfr-disallowed}
\end{subfigure}\\
\hline
\end{tabular}
%
%
%
%
\begin{tabular}{@{} c | c | c @{}}
\phantom{-}& \phantom{-}& \phantom{-}\\
\begin{subfigure}{0.25\textwidth}
\begin{centertikz}%

%Location x
\node(locx) {$\ke_1 \mapsto$};
\draw pic at ([xshift=\tikzkvspace]locx.east) {vlist={versionx}{%
    /$0$/$\txid_0$/$\Set{\txid_\cl^1,\txid_\cl^2}$
    , /$1$/$\txid_\cl^1$/$\emptyset$
    , /$1$/$\txid_\cl^2$/$\emptyset$
}};

\end{centertikz}%
\vspace{5pt}
\caption{Disallowed by \(\RYW\)}
\label{fig:ryw-disallowed}
\end{subfigure}
& 

\begin{subfigure}{0.40\textwidth}
\begin{centertikz}%

%Location x
\node(locx) {$\ke_1 \mapsto$};
\draw pic at ([xshift=\tikzkvspace]locx.east) {vlist={versionx}{%
    /$\val_0$/$\txid_0$/$\Set{\txid_2}$
    , /$\val_1$/$\txid_1$/$\emptyset$
}};

%Location y
\path (versionx.east) + (1,0) node (locy) {$\ke_2 \mapsto$};
\draw pic at ([xshift=\tikzkvspace]locy.east) {vlist={versiony}{%
    /$\val_0$/$\txid_0$/$\Set{\txid_1}$
    , /$\val_2$/$\txid_2$/$\emptyset$
}};

\end{centertikz}%
\vspace{5pt}
\caption{Write skew, disallowed by \(\SER\)}
\label{fig:ser-disallowed}
\end{subfigure}%
&
\begin{subfigure}{0.30\textwidth}
\begin{centertikz}

\node(locx) {$\ke_1 \mapsto$};
\draw pic at ([xshift=\tikzkvspace]locx.east) {vlist={versionx}{%
    /$0$/$\txid_0$/$\Set{\txid,\txid'}$
    , /$1$/$\txid$/$\emptyset$
    , /$1$/$\txid'$/$\emptyset$
}};

\end{centertikz}
\vspace{5pt}
\caption{Lost update, disallowed by \(\UA\)}
\end{subfigure}
\\
\hline
\end{tabular}
%
%
%
%
\phantom{x}\vspace{7pt}
\begin{tabular}{@{} c | c @{}}
\phantom{-}& \phantom{-} \\
\begin{subfigure}{0.42\textwidth}
\begin{centertikz}%
%Location x
\node(locx) {$\ke_1 \mapsto$};
\draw pic at ([xshift=\tikzkvspace]locx.east) {vlist={versionx}{%
    /$\val_0$/$\txid_0$/$\Set{\txid_{\cl_2}^1}$
    , /$\val_1$/$\txid$/$\Set{\txid_{\cl_1}^1}$
}};

%Location y
\path (versionx.east) + (1,0) node (locy) {$\ke_2 \mapsto$};
\draw pic at ([xshift=\tikzkvspace]locy.east) {vlist={versiony}{%
    /$\val_0$/$\txid_0$/$\Set{\txid_{\cl_1}^2}$
    , /$\val_1$/$\txid$/$\Set{\txid_{\cl_2}^2}$
}};

\end{centertikz}%
\vspace{5pt}
\caption{Long fork, disallowed by \(\CP\)}
\label{fig:cp-disallowed-2}
\label{fig:cp-disallowed}
\end{subfigure}
&
\begin{subfigure}{0.542\textwidth}%
\begin{centertikz}%
%Location x
\node(locx) {$\ke_1 \mapsto$};
\draw pic at ([xshift=\tikzkvspace]locx.east) {vlist={versionx}{%
    /$\val_0$/$\txid_0$/$\Set{\txid_4}$
    , /$\val_1$/$\txid_1$/$\emptyset$
    , /$\val_2$/$\txid_2$/$\emptyset$
}};

%Location y
\path (versionx.east) + (1,0) node (locy) {$\ke_2 \mapsto$};
\draw pic at ([xshift=\tikzkvspace]locy.east) {vlist={versiony}{%
    /$\val_0$/$\txid_0$/$\Set{\txid_2}$
    , /$\val_3$/$\txid_3$/$\Set{\txid_4}$
    , /$\val_4$/$\txid_4$/$\emptyset$
}};

%%Location z
%\path (versiony.east) + (1,0) node (locz) {$\ke_3 \mapsto$};
%\matrix(versionz) [version list,column 2/.style={text width=7mm}]
   %at ([xshift=\tikzkvspace]locz.east) {
 %{a} & $\txid_0$ & {a} & $\txid_3$ & {a} & $\txid_4$ \\
  %{a} & $\emptyset$ & {a} & $\emptyset$ & {a} & $\emptyset$\\
%};

%\tikzvalue{versionz-1-1}{versionz-2-1}{locz-v0}{$\val_0$};
%\tikzvalue{versionz-1-3}{versionz-2-3}{locz-v1}{$\val_1$};
%\tikzvalue{versionz-1-5}{versionz-2-5}{locz-v2}{$\val_2$};

%%Location w
%\path (versionz.east) + (1,0) node (locw) {$\ke_4 \mapsto$};
%\matrix(versionw) [version list,column 2/.style={text width=7mm}]
    %at ([xshift=\tikzkvspace]locw.east) {
    %{a} & $\txid_0$ & {a} & $\txid_1$ \\
    %{a} & $\{\txid_4\}$ & {a} & $\emptyset$ \\
%};
%\tikzvalue{versionw-1-1}{versionw-2-1}{locw-v0}{$\val_0$};
%\tikzvalue{versionw-1-3}{versionw-2-3}{locw-v1}{$\val_1$};
\end{centertikz}
\vspace{5pt}
\caption{Allowed by \( \UA \) and \( \CP \) but disallowed by \(\SI\)}%
\label{fig:si-disallowed}%
\end{subfigure} \\
\hline
\end{tabular}
%\begin{tabular}{@{}c @{} c @{}}
%\begin{minipage}{0.4\textwidth}
%\begin{subfigure}{\textwidth}
%\begin{centertikz}
%%Location x
%\node(locx) {$\ke_1 \mapsto$};
%
%\matrix(versionx) [version list,,column 2/.style={text width=8mm},column 4/.style={text width=7mm}]
%    at ([xshift=\tikzkvspace]locx.east) {
%    {a} & $\txid_0$ & {a} & $\txid_{\cl}^{1}$\\
%    {a} & $\left\{\txid_{\cl'}^{2}\right\}$ & {a} & $\emptyset$ \\
%};
%\tikzvalue{versionx-1-1}{versionx-2-1}{locx-v0}{$\val_0$};
%\tikzvalue{versionx-1-3}{versionx-2-3}{locx-v1}{$\val_1$};
%
%%Location y
%\path (locx.south) + (0,\tikzkeyspace) node (locy) {$\ke_2 \mapsto$};
%\matrix(versiony) [version list,column 2/.style={text width=8mm},column 4/.style={text width=7mm}]
%   at ([xshift=\tikzkvspace]locy.east) {
% {a} & $\txid_0$ & {a} & $\txid_{\cl'}^1$ \\
%  {a} & $\left\{\txid_{cl}^{2}\right\}$ & {a} & $\emptyset$\\
%};
%
%\tikzvalue{versiony-1-1}{versiony-2-1}{locy-v0}{$\val_0$};
%\tikzvalue{versiony-1-3}{versiony-2-3}{locy-v1}{$\val_2$};
%
%\end{centertikz}
%\caption{Disallowed by \(\CP\)}
%\label{fig:cp-disallowed-2}
%\end{subfigure}
%
%\begin{subfigure}{\textwidth}
%\begin{centertikz}
%%Location x
%\node(locx) {$\ke_1 \mapsto$};
%
%\matrix(versionx) [version list,column 2/.style={text width=7mm},column 4/.style={text width=7mm}]
%    at ([xshift=\tikzkvspace]locx.east) {
%    {a} & $\txid_0$ & {a} & $\txid_1$\\
%    {a} & $\left\{\txid_2\right\}$ & {a} & $\emptyset$ \\
%};
%\tikzvalue{versionx-1-1}{versionx-2-1}{locx-v0}{$\val_0$};
%\tikzvalue{versionx-1-3}{versionx-2-3}{locx-v1}{$\val_1$};
%
%%Location y
%\path (locx.south) + (0,\tikzkeyspace) node (locy) {$\ke_2 \mapsto$};
%\matrix(versiony) [version list,column 2/.style={text width=7mm},column 4/.style={text width=7mm}]
%   at ([xshift=\tikzkvspace]locy.east) {
% {a} & $\txid_0$ & {a} & $\txid_2$ \\
%  {a} & $\left\{\txid_1\right\}$ & {a} & $\emptyset$\\
%};
%
%\tikzvalue{versiony-1-1}{versiony-2-1}{locy-v0}{$\val_0$};
%\tikzvalue{versiony-1-3}{versiony-2-3}{locy-v1}{$\val_2$};
%\end{centertikz}
%\caption{Disallowed by \(\SER\)}
%\label{fig:ser-disallowed}
%\end{subfigure}
%
%\end{minipage}
%
%&
%\begin{subfigure}{0.55\textwidth}%
%\begin{centertikz}%
%%Location x
%\node(locx) {$\ke_1 \mapsto$};
%
%\matrix(versionx) [version list,column 2/.style={text width=7mm}]
%    at ([xshift=\tikzkvspace]locx.east) {
%    {a} & $\txid_0$ & {a} & $\txid_1$ & {a} & $\txid_2$\\
%    {a} & $\emptyset$ & {a} & $\emptyset$ & {a} & $\emptyset$\\
%};
%\tikzvalue{versionx-1-1}{versionx-2-1}{locx-v0}{$\val_0$};
%\tikzvalue{versionx-1-3}{versionx-2-3}{locx-v1}{$\val_1$};
%\tikzvalue{versionx-1-5}{versionx-2-5}{locx-v1}{$\val_2$};
%%Location y
%\path (locx.south) + (0,\tikzkeyspace) node (locy) {$\ke_2 \mapsto$};
%\matrix(versiony) [version list,column 2/.style={text width=7mm}]
%   at ([xshift=\tikzkvspace]locy.east) {
% {a} & $\txid_0$ & {a} & $\txid_3$ \\
%  {a} & $\left\{\txid_2\right\}$ & {a} & $\emptyset$\\
%};
%
%\tikzvalue{versiony-1-1}{versiony-2-1}{locy-v0}{$\val_0$};
%\tikzvalue{versiony-1-3}{versiony-2-3}{locy-v1}{$\val_1$};
%
%%Location z
%\path (locy.south) + (0,\tikzkeyspace) node (locz) {$\ke_3 \mapsto$};
%\matrix(versionz) [version list,column 2/.style={text width=7mm}]
%   at ([xshift=\tikzkvspace]locz.east) {
% {a} & $\txid_0$ & {a} & $\txid_3$ & {a} & $\txid_4$ \\
%  {a} & $\emptyset$ & {a} & $\emptyset$ & {a} & $\emptyset$\\
%};
%
%\tikzvalue{versionz-1-1}{versionz-2-1}{locz-v0}{$\val_0$};
%\tikzvalue{versionz-1-3}{versionz-2-3}{locz-v1}{$\val_1$};
%\tikzvalue{versionz-1-5}{versionz-2-5}{locz-v2}{$\val_2$};
%
%%Location w
%\path (locz.south) + (0,\tikzkeyspace) node (locw) {$\ke_4 \mapsto$};
%\matrix(versionw) [version list,column 2/.style={text width=7mm}]
%    at ([xshift=\tikzkvspace]locw.east) {
%    {a} & $\txid_0$ & {a} & $\txid_1$ \\
%    {a} & $\{\txid_4\}$ & {a} & $\emptyset$ \\
%};
%\tikzvalue{versionw-1-1}{versionw-2-1}{locw-v0}{$\val_0$};
%\tikzvalue{versionw-1-3}{versionw-2-3}{locw-v1}{$\val_1$};
%\end{centertikz}%
%\caption{Disallowed by \(\SI\)}%
%\label{fig:si-disallowed}%
%\end{subfigure} \\
%\end{tabular}
\hrulefill
\vspace*{-5pt}
\caption{Behaviours disallowed under different consistency models}
\label{fig:anomalies}
\vspace*{-10pt}
\end{figure*}


We now give the definitions of well-known consistency models in distributed databases,
including 
\emph{causal consistency} \((\CC)\) \citep{ev_transactions,cops,causal-def}, 
\emph{parallel snapshot isolation} \( (\PSI) \) \citep{NMSI,PSI},
\emph{snapshot isolation} \((\SI)\) \citep{si} 
and \emph{serialisability} \((\SER)\) \citep{Papadimitriou-ser}.
Researchers \citep{giovanni_concur16,cp-def,framework-concur} proposed that 
the definition of \( \SI \) on abstract executions
can be separated into two different consistency models,
\emph{update atomic} (\(\UA\)) and \emph{consistent prefix} \( (\CP) \).
They also realised that \( \PSI \) can be defined 
as the conjunction of \( \UA \) and \( \CC \) on abstract executions.
Note that, in \cref{fig:execution-tests}, the \( \vshiftname \) for \( \CC \), \(\CP \), \( \PSI \) and \( \SI \)
are all defined as the conjunction of the \(\MR\) and \(\RYW\) session guarantees.
This is because 
many consistency models are originally defined using specific implementation strategies
for tackling certain constraints in distributed databases,
and \(\MR\) and \( \RYW \) are straightforward to implement 
in the sense that each client maintains some meta-data about it own history.

\mypar{Causal Consistency \((\CC)\)}

This model requires that 
if a client view includes 
a version \(\ver\) written by \( \txid \) prior to committing a transaction, 
then it must also include the versions which \(\txid\) observes.
Clearly, \(\txid\) observes all versions that \(\txid\) reads. 
Moreover, \(\txid\) observes other previous transactions from the same client.
This is captured by the \(\cancommitname_{\CC}\) predicate in \cref{fig:execution_tests}, 
defined as \(\closed(\mkvs, u, \rel[\CC])\) with \(\rel[\CC] \defeq \SO \cup \WR[\mkvs]\).
For example, the kv-store of \cref{fig:wr-wfr-allowed-but-cc} 
is disallowed by \(\CC\): the version of key \(\key_3\) carrying value \(\val_3\) depends on 
the version of key \(\key_1\) carrying value \(\val_1\). 
However, transaction \(\txid\) must have been committed by a client
whose view included \(\val_3\) of \( \key_3\), but not \(\val_1\) of \( \key_1\).

\mypar{Update Atomic \((\UA)\)}
This consistency model has been proposed by \citet{framework-concur} 
and implemented by \citet{rola}.
\(\UA\) disallows concurrent transactions writing to the same key,
a property known as \emph{write-conflict freedom}:  
when two transactions write to the same key, one must see the version 
written by the other.
Write-conflict freedom is enforced by \(\cancommitname_{\UA}\) which allows a client to write to key \(\key\) only if its view includes all versions of \(\key\):
that is, its view is closed with respect to the \(\WWInv(\key)\) relation for all keys \(\key\) written in the fingerprint \(\fp\).
This prevents the kv-store of \cref{fig:ua-disallowed},
as \(\txid\) and \(\txid'\) concurrently increment the initial version of \(\key\) by \(1\).
As client views must include the initial versions, once \(\txid\) commits a new version \(\ver\) with value \(\val_1\) to \(\key\), then \(\txid'\) must include \(\ver\) in its view as there is a \(\WW\) edge from the initial version to \(\ver\). 
As such, when \(\txid'\) increments \(\key\), it must read from \(\ver\), and not the initial version (\cref{fig:ua-disallowed}).

\mypar{Parallel Snapshot Isolation \((\PSI)\)} 
This consistency model requires: \begin{enumerate*}
\item 
if a client view includes 
a version \(\ver\) written by \( \txid \) prior to committing a transaction, 
then it must also include the versions which \(\txid\) observes; and
\item there is no write-conflict.
\end{enumerate*}
It is formally defined as the conjunction of the guarantees provided by \(\CC\) and \(\UA\) on 
abstract executions where there is a total order over all transactions \cite{framework-concur}. 
However, we cannot simply define \(\cancommitname_{\PSI}\) as the conjunction of the \(\cancommitname\) predicates for \(\CC\) and \(\UA\). 
This is for two reasons. 
First, their conjunction would only mandate that \(\vi\) be closed with respect to 
\(\rel[\CC]\) and \(\rel[\UA]\) \emph{individually}, but \emph{not} with respect to their \emph{union}.
Recall that closure is defined in terms of the transitive closure of a given relation 
and thus the closure of \(\rel[\CC]\) and \(\rel[\UA]\) is smaller than the closure of \(\rel[\CC] \cup \rel[\UA]\).
As such, we define \(\cancommitname_{\PSI}\) as closure with respect to \(\rel[\PSI] \) that must include \( \rel[\CC] \cup \rel[\UA]\).
Second, recall that \(\cancommitname_{\UA}\) requires that a transaction writing 
to a key \(\key\) must be able to observe all previous versions of \(\key\), \ie all versions of \(\key\). 
That is, when write-conflict freedom is enforced, a version \(\ver\) of a key \( \key \)
written by a transaction observes on all previous versions of \(\key\). 
This leads us to include write-write dependencies (\(\WW[\mkvs]\)) in \(\rel[\PSI]\). 
Note that the kv-store in \cref{fig:cc-ua-allowed-but-psi} shows an example kv-store that satisfies \(\cancommitname_{\CC} \land \cancommitname_{\UA}\), 
but not \(\cancommitname_{\PSI}\).

\mypar{Consistent Prefix \((\CP)\)}
\label{para:cp}
If the total order in which transactions commit is known, \(\CP\)
can be described as a strengthening of \(\CC\): 
if a client sees the versions written by a transaction \(\txid\),
then it must also see all versions written by transactions that \emph{commit} before \(\txid\). 
Although kv-stores only provide \emph{partial} information about the transaction commit order via the dependency relations,
this is sufficient to formalise \emph{Consistent Prefix} \cite{laws}.

In practice, we can approximate the order in which transactions 
commit via the \(\WR[\mkvs], \allowbreak \WW[\mkvs], \allowbreak \RW[\mkvs]\) and \(\SO\)  relations. 
This approximation is best understood in terms of an idealised implementation of \(\CP\) on a centralised system,
where the snapshot of a transaction is determined at its \emph{start point} and its effects are made visible to future transactions at its \emph{commit point}.
With respect to this implementation, if \((\txid,\txid') \in \WR\), then 
\(\txid\) must commit before \(\txid'\) starts, and hence before \(\txid'\) commits.
Similarly, if \((\txid,\txid') \in \SO\), then \(\txid\) commits before \(\txid'\) starts, 
and thus before \(\txid'\) commits.
Recall that, if \((\txid'', \txid') \in \RW\),
then \(\txid''\) reads a version that is later overwritten by \(\txid'\).
That is, \(\txid''\) cannot see the write of \(\txid'\), and thus \(\txid''\) must start before 
\(\txid'\) commits. 
As such, if \(\txid\) commits before \(\txid''\) starts 
(\((\txid, \txid'') \in \WR\) or \((\txid,\txid'') \in \SO\)), 
and \((\txid'', \txid') \in \RW\), then \(\txid\) must commit before 
\(\txid'\) commits. 
In other words, if \((\txid,\txid') \in \WR;\RW\) or \((\txid,\txid') \in \SO;\RW\), then \(\txid\) commits before \(\txid'\).
Finally, if \((\txid,\txid') \in\WW\), then \(\txid\) must commit before \(\txid'\). 
We therefore define \(\rel[\CP] \defeq (\WR[\mkvs]; \RW[\mkvs]\rflx \cup \SO;  \RW[\mkvs]\rflx \cup \WW)\), approximating the order in which transactions commit. 
%
\citet{laws} show that the set \((\rel[\CP]^{-1})^{+}(\txid)\) contains all transactions that must be observed by \(\txid\) under \(\CP\). 
We define \(\cancommitname_{\CP}\) by requiring that the client view be 
closed with respect to \(\rel[\CP]\).

Consistent prefix disallows the \emph{long fork anomaly} shown in \cref{fig:cp-disallowed}, where clients \(\cl_1\) and \(\cl_2\) observe the updates to \(\key_1\) and \(\key_2\) 
in different orders. 
Assuming without loss of generality that \( \txid_{\cl_1}^{2} \) commits 
before \( \txid_{\cl_2}^{2} \), then prior to committing its transaction \(\cl_2\) sees 
the version of \(\key_1\) with value \(\val_0\). 
However, since \(\txid {\xrightarrow{\WR[\mkvs]}} \txid_{\cl_{1}}^{1} 
{\xrightarrow{\SO}} \txid_{\cl_{1}}^{2} {\xrightarrow{\RW}} \txid' {\xrightarrow{\WR}} \txid_{\cl_{2}}^{1} \), 
and prior to commit, \( \txid_{\cl_2}^{2} \) must see versions written by \( \txid_{\cl_2}^{1} \),
then \( \txid_{\cl_2}^{2} \) should also see the version of \(\key_1\) with 
value \(\val_2\), leading to a contradiction.


\mypar{Snapshot Isolation \((\SI)\)}
On abstract executions where there is a total order over transactions,  
\(\SI\) can be defined compositionally from \(\CP\) and \(\UA\). 
However, as with \(\PSI\), we cannot define \(\cancommitname_{\SI}\) as the conjunction of their associated \(\cancommitname\) predicates. 
Rather, we define \(\cancommitname_{\SI}\) as closure with respect to \(\rel[\SI]\), which includes \(\rel[\CP] \cup \rel[\UA]\).
Observe that the kv-store in \cref{fig:si-disallowed} shows an example kv-store that satisfies \(\cancommitname_{\UA} \land \cancommitname_{\CP}\), 
but not \(\cancommitname_{\SI}\).
Additionally, we include \(\WW;\RW\) in \(\rel[\SI]\). 
This is because when the centralised \(\CP\) implementation (discussed before) is strengthened with write-conflict freedom, then a write-write dependency between two transactions \(\txid\) and \(\txid'\) 
does not only mandate that \(\txid\) commits before \(\txid'\) commits but also before \(\txid'\) starts. 
Consequently, if \((\txid, \txid') \in \WW ;\RW\), then \(\txid\) must commit 
before \(\txid'\) commit.

\mypar{(Strict) serialisability \((\SER)\)}
Serialisability is the strongest consistency model 
in any framework that abstracts from aborted transactions, 
requiring that transactions execute in a total sequential order. 
The \(\cancommitname_{\SER}\) thus allows clients to commit transactions only when 
their view of the kv-store is complete, \ie the client view is closed with respect to \(\WW^{-1}\).
This requirement prevents the kv-store in  \cref{fig:ser-disallowed}: 
without loss of generality, suppose that \(\txid_1\) commits before \(\txid_2\). 
Then the client committing \(\txid_2\) must see the version of \(\key_1\) written by \(\txid_1\), 
and thus cannot read the outdated value \(\val_0\) for \(\key_1\). 

\mypar{Weak Snapshot Isolation \((\WSI)\): A New Consistency Model} 
\label{sec:new_cm}
Kv-stores and execution tests are useful for investigating new 
consistency models.  
One example is the consistency model induced by combining 
\(\CP\) and \(\UA\), which we refer to as \emph{Weak Snapshot Isolation} (\(\WSI\)). 
Because \(\WSI\) is stronger than \(\CP\) and \(\UA\) by definition, 
it forbids all the  anomalies forbidden by these consistency models, \eg
the long fork (\cref{fig:cp-disallowed}) and the lost update (\cref{fig:ua-disallowed}). 
Moreover, \(\WSI\) is strictly weaker than \(\SI\). 
As such, \(\WSI\) allows all \(\SI\) anomalies, \eg the write skew (\cref{fig:ser-disallowed}), 
and allows behaviour not allowed under \(\SI\) such as that in \cref{fig:si-disallowed}.
The kv-store \(\mkvs\) is reachable by executing transactions 
\(\txid_{1}, \txid_{2}, \txid_{3}\) and \(\txid_{4}\) in this order. 
In particular, \(\txid_{4}\) is executed using \(\vi {=} \{\key_{1} \mapsto \{0\}, \key_{2} \mapsto \{0,1\}\}\). 
However, the same kv-store is not reachable under \(\ET[\SI]\). 
Under \(\SI\) transaction \(\txid_{4}\) cannot be executed using \(\vi\): 
\(\txid_{4}\) reads the version of \(\key_2\) written by \(\txid_3\), 
but since \((\txid_{2},\txid_{3}) \in \RW \)
and \((\txid_{1} ,\txid_2) \in \WW\), 
then \(\vi\) should contain the version of \(\key_{1}\) written by \(\txid_{1}\), 
contradicting the fact that \(\txid_{4}\) reads the initial version of \(\key_1\).
The two consistency models are very similar in that 
many applications that 
are correct under \(\SI\) are also correct under \(\WSI\).
We give examples of such applications in \cref{sec:program-analysis}.

\mypar{Correctness of \( \et \)}
Our definitions of consistency models over kv-stores
and client views are equivalent to well-known definitions of
consistency models over abstract executions \cite{framework-concur}, and hence dependency graphs \cite{laws}.
Let $\CMs(\et[M])$ denote the consistency model induced by execution test
$\et[M]$ where $M$ is one of the consistency models given in \cref{fig:execution.tests}:
for example, when $M=\CC$ then $\ET[\CC]$ is the execution test of the \( \CC \)
consistency model. Let $\CMs(\visaxioms[M])$ denote the corressponding consistency
model induced by the set of axioms $\visaxioms[M]$: for example, when $M$ is \( \CC \)
then $\CMs(\visaxioms[M])$ denotes the axioms on abstract executions  for \( \CC \). 

\SpaceAboveDef
\begin{theorem}
$ \CMs(\et[M]) = \CMs(\visaxioms[M])$ for all the consistency models $M$
described by \cref{fig:execution.tests}. 
\end{theorem}
\SpaceBelowDef

The proof is given in the technical report. 
In particular, we provide an interleaving operational semantics on
abstract executions parametrised by a set of axioms,  where each step
corresponds to an atomic transaction, 
in  contrast to \cite{sureshConcur} which provides a more fine-grained
operational semantics. 
This semantics on abstract executions is an intermediate  for proving our definitions of consistency models.
             

