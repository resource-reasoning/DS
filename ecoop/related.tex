\section{Related Work} 
\label{sec:related}
Crooks et al. \citet{seebelieve} proposed
a state-based formal framework for weak consistency models 
that employs concepts  similar to views and execution tests, called
read states and commit tests  respectively.
In their semantics, one-step trace reduction is determined by the whole previous history of the trace. 
In contrast, our reduction step only depends on the current kv-store and view.
They shown that several definitions of SI collapse into one.
We believe \citet{seebelieve} can be used to verify implementations,
yet there is no example in their paper.
More importantly,
we share the similar view as in \cite{sureshConcur},
that it is difficult for \citet{seebelieve} to prove invariant properties of client programs,
because their notion of trace includes a large amount of
information, for example, the total order on transactions,
that just would not be observable by a client, 
for example, a client on one replica does not necessarily see a transaction on another replica.
In contrast, our semantics focuses on client observable history, and
we have shown that it is useful to prove invariant properties in \cref{sec:robustness}.

Nagar and Jagannathan \cite{sureshConcur} proposed a fine-grained operational semantics on abstract exactions 
and developed a model-checking tool for the violation of robustness. 
This was achieved by converting abstract executions to
dependency graphs and checking the violation of robustness on the
dependency graphs. This approach has two issues. First, despite 
assuming atomic visibility of transactions, they presents a fine-grained
semantics at the level of the individual transactional operations
rather than whole transactions, introducing unnecessary interleavings
which complicates the client reasoning, for example, increasing the
search space of model-checking tools. 
In contrast, our semantics is coarse-grained in that interleaving is at the level of whole
transactions. 
Second, all the literature that performs client analysis
on abstract executions achieves this indirectly by over-approximating
the consistency-model specifications using dependency graphs
\cite{giovanni_concur16,SIanalysis,psi-chopping,laws,sureshConcur}. 
It is  unknown how to do this precisely \cite{laws}. 
In contrast, we prove robustness results directly by
analysing the structure of kv-stores, without over-approximation. 
We also give precise reasoning about the mutual exclusion of locks,
which we believe will be difficult to prove using abstract executions.

Our semantics shares a similar idea to log-based semantics \cite{push-pull,log-based-op}.
Koskinen and Parkinson
\citet{push-pull}  proposed log-based semantics for 
verifying implementations that satisfy serialisability, based not
only on kv-stores
but  also on other ADTs. Their work comprises a centralised global log 
and partial client local logs, which is a little similar to 
our kv-stores and views. Their model focusses on serialisability, and 
there is no evidence that they could be easily extended to tackle
weaker consistency models.
By contrast, our semantics is parametrised by an execution test which gives rise to 
different consistency model.
