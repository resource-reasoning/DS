\section{Operational Model}
\label{sec:model}

We define an interleaving operational semantics for atomic transactions (\cref{subsec:kv-store-op}) on
abstract states comprising global kv-stores and partial client views (\cref{subsec:kvstores}). 
Our semantics is parametrised by an execution test which induces a
consistency model  (\cref{sec:cm}).

\subsection{Abstract States: Key-Value Stores and Client Views}
\label{subsec:kvstores}
\label{sec:mkvs-view}
The abstract states of our operational semantics comprise 
a  global, centralised kv-store and a partial client view.  A
kv-store comprises key-indexed lists of versions which record
the history of the key with values and meta-data of the
transactions that accessed it: the writer and readers.

We assume a countably infinite set of \emph{client identifiers}\footnote{ We use the notation
 \(\TypeFont{A} \ni a\) to denote that elements of \(\TypeFont{A}\) are ranged over
  by \(a\) and its variants \(a', a_1, \cdots\).},
\(\Clients \ni \cl\).
The set of \emph{transaction identifiers}, \(\TxIDs \ni t\), 
 is defined by
\(\TxIDs \defeq  \Set{\txid_{0}} \uplus \Set{ \txid_{\cl}^{n} \mid \cl \in \Clients \land n \geq 0 }\), 
where  \(\txid_0\) denotes  the  \emph{initialisation transaction}
and \(\txid_{\cl}^{n}\) identifies a transaction committed by client
\(\cl\) with \(n\)  determining  the client session order:
\(\SO \defeq \{ (\txid, \txid') \mid \exsts{ \cl, n,m } \txid = \txid_{\cl}^{n} \land \allowbreak \txid' = \txid_{\cl}^{m} \land \allowbreak n < m\}\).
Subsets of \(\TxIDs\)  are ranged over by \(\txidset, \txidset', \cdots\). 
We let \(\TxIDs_{0} \defeq \TxIDs \setminus \Set{\txid_0}\). 

\SpaceAboveDef
\begin{definition}[Kv-stores]
\label{def:his_heap}
\label{def:mkvs}
Assume a countably infinite set of \emph{keys}, \(\Keys \ni \key\), 
and a countably infinite set of  \emph{values}, \(\Val \ni \val\), 
which includes the keys and an \emph{initialisation value} \(\val_0\).
The set of \emph{versions}, \(\Versions \ni \ver\), is \(\Versions \defeq \Val \times \TxIDs \times \pset{\TxIDs_{0}}\). 
A \emph{kv-store} 
is a function \(\mkvs: \Keys \to \Func{List}(\Versions)\), 
where \(\Func{List}(\Versions) \ni \vilist\) is the set of lists of versions.
\end{definition}
\SpaceBelowDef

Each version has the form 
\(\ver {=} (\val, \txid, \txidset)\), where \(\val\) is
a value, the \emph{writer} \(\txid\) identifies the transaction that
wrote \(\val\),  and the \emph{reader set} \(\txidset\) identifies the
transactions that read \(\val\). We write
\(\valueOf(\ver)\),
\(\wtOf(\ver)\) and \(\rsOf(\ver)\) to project
the components of \(\ver\).
Given a kv-store \(\mkvs\) and a transaction \(\txid\), we write 
\(\txid \in \mkvs\) if \(\txid\) is either the writer or 
one of the readers of a version in \(\mkvs\);
we write \(\lvert \mkvs(\key) \rvert\) for the length of the version
list \(\mkvs(\key)\),
and \(\mkvs(\key, i)\) for the \(i\)\textsuperscript{th} version of
\(\key\) in kv-store $\mkvs$.

We assume that the version list of each key has an initialisation version 
carrying the initialisation value \(\val_0\),  written by the 
initialisation transaction \(\txid_0\) with an initial empty reader set.
We focus on kv-stores whose consistency model satisfies the
\emph{snapshot property}, ensuring that
a transaction reads and writes at most one version for each key:

\SpaceAboveMath
\begin{align}
\fora{\key, i, j} 
\left( \rsOf(\kvs(\key, i)) \cap \rsOf(\kvs(\key, j)) \neq \emptyset \lor
\wtOf(\kvs(\key, i)) = \wtOf(\kvs(\key, j)) \right)
\implies i = j  
\tag{snapshot}
\label{equ:kvs-wf-txid-snapshot-property} 
\end{align}
\SpaceBelowMath

\noindent 
This is a standard assumption for distributed databases, \eg in
\cite{ramp,si,distrsi,clocksi,cops,PSI-RA,NMSI,PSI}.
Finally, we assume that the kv-store agrees with the session order of clients, 
in that a client cannot read a version of a key that has been written by a future transaction within
the same session, and the order in which versions are written by a
client must agree with its session order, \ie
for any \( \key,\idx,j,\txid,\txid' \):

\SpaceAboveMath
\begin{align}
& \txid = \wtOf(\kvs(\key,\idx))
\land \txid' \in \rsOf(\kvs(\key,\idx))
\implies (\txid', \txid) \notin \SO\rflx
\tag{wr-so}
\label{equ:kvs-wf-so-wr}
\\ & \txid = \wtOf(\kvs(\key,\idx))
\land \txid' = \wtOf(\kvs(\key,j))
\land \idx < j
\implies (\txid', \txid) \notin \SO\rflx
\tag{ww-so}
\label{equ:kvs-wf-so-ww}
\end{align}
\SpaceBelowMath

\noindent A kv-store is
\emph{well-formed} if it satisfies these assumptions.
Henceforth, we assume kv-stores are well-formed,
and let \(\MKVSs\) denote the set of well-formed kv-stores.

A global kv-store provides an abstract centralised description of
updates associated with distributed kv-stores that is \emph{complete} in 
that no update has been lost in the description. By contrast, in
both replicated and partitioned distributed databases, a client may
have incomplete information about updates distributed between
machines.  We model this incomplete information by
defining a \emph{client view}, or just \emph{view}, of the kv-store
which provides a \emph{partial} record of the updates observed by a
client. 
We require that a client view be \emph{atomic} in that it can
see either all or none of the updates of a transaction.
This client view was partly inspired 
by the views of the `promising' C11 operational semantics~\cite{promises}.

\SpaceAboveDef
\begin{definition}[Views]
\label{def:view}
\label{def:cuts}
\label{def:views}
A \emph{view} of a kv-store \(\mkvs \in \MKVSs\) is a function
\(\vi \in \Views(\mkvs) \defeq \Keys \to\pset{\Nat}\) such that, for all \(i, i', \key, \key'\):

\SpaceAboveMath
\begin{align*}
    & 
    0 \in \vi(\key) 
    \land (i \in \vi(\key) \implies 0 \leq i < \Abs{ \mkvs(\key) }) 
    \tag{in-range}
    \label{eq:view.wf}\\
    & 
    \begin{array}{@{}l@{}}
	i \in \vi(\key)  
  	\land \wtOf(\mkvs(\key, i)) {=} \wtOf(\mkvs(\key', i'))  
  	\implies i' \in \vi(\key')
    \end{array}
	\tag{atomic}
	\label{eq:view.atomic}
\end{align*}
\SpaceBelowMath

Given two views \(\vi, \vi' \in \Views(\mkvs)\), 
the order between them is defined by \(\vi \viewleq \vi' \defiff \fora{\key \in \dom(\mkvs)} \vi(k) \subseteq \vi'(\key)\).
The set of views is \(\Views \defeq \bigcup_{\mkvs \in \MKVSs} \Views(\mkvs)\).
The \emph{initial view}, \(\vi_{0}\),  is defined by
\(\vi_{0}(\key) = \Set{0}\) for every \(\key \in \Keys\). 
\end{definition}
\SpaceBelowDef

Our operational semantics updates \emph{configurations},  which are pairs
comprising a kv-store and a function describing the
views of a finite set of clients. 

\SpaceAboveDef
\begin{definition}[Configurations]
\label{def:configuration}
A \emph{configuration}, \(\conf \in \Confs \),  is a pair \( (\mkvs, \vienv)\)
with \(\mkvs \in \MKVSs\) and
\(\vienv : \Clients \parfinfun \Views(\mkvs)\). 
The set of \emph{initial configurations}, \(\Confs_0 \subseteq \Confs\),
contains configurations of the form \( (\mkvs_0, \vienv_0)\), where
\(\mkvs_0\) is the initial kv-store defined by
\(\mkvs_{0}(\key)\defeq  (\val_0, \txid_0, \emptyset)\) for
all \(\key \in \Keys\). 
\end{definition}
\SpaceBelowDef


Given a configuration \((\mkvs, \vienv)\) and a client \(\cl\), 
if \(\vi = \vienv(\cl)\) is defined then, for each \(k\),  the
configuration determines the sub-list of versions in \(\mkvs\) that \(\cl\) sees.
If \(i,j \in \vi(\key)\) and \(i < j\), then \(\cl\) sees the values 
carried by versions \(\mkvs(\key, i)\) and  \(\mkvs(\key, j)\), 
and it also sees that the version \(\mkvs(\key, j)\) is more up-to-date than \(\mkvs(\key, i)\). 
It is therefore possible to associate a \emph{snapshot} with the view \(\vi\), 
which identifies, for each key \(\key\), the last version included in the view. 
This definition assumes that the database satisfies the \emph{last-write-wins}
resolution policy, employed by many distributed key-value stores.
However, our formalism can be adapted straightforwardly  to capture other resolution policies. 

\begin{definition}[View Snapshots]
\label{def:snapshot}
Given \(\mkvs \in \MKVSs\) and \(\vi \in \Views(\mkvs)\), the
\emph{view snapshot} of \(\vi\) in 
\(\mkvs\) is a function, 
\(\snapshot[\mkvs, \vi] : \Keys \to \Val\), 
defined by: 

\SpaceAboveMath[-15pt]
\[\snapshot[\mkvs, \vi] \defeq \lambda \key \ldotp \valueOf(\mkvs(\key, \textstyle \max_{<}(\vi(\key))))\]
\SpaceBelowMath[-15pt]

\noindent
where \(\max_{<}(\vi(\key))\) is the maximum element in \(\vi(\key)\) under the natural 
order \(<\) on \(\mathbb{N}\).
\end{definition}
\SpaceBelowDef
When clear from the context, we simply refer to a view snapshot as a {\em snapshot}. 

\subsection{Operational Semantics}
\label{subsec:kv-store-op}

\subparagraph*{Core Programming Language}
We assume a language of expressions built from values \( \val \)
and program variables \( \var \), defined by:
\(\expr ::= \val \mid \var \mid \expr + \expr \mid \cdots\).
The \emph{evaluation} \(\evalE{\expr}\) of expression \(\expr\) is parametric in
the client-local stack \( \stk \):
\(
\evalE{\val} \defeq
\val
\ \
\evalE{\var} \defeq
\stk(\var)
\ \
\evalE{\expr_{1} + \expr_{2}} \defeq
\evalE{\expr_{1}} + \evalE{\expr_{2}}
\ \
\cdots 
\).
A \emph{program} \( \prog \) comprises a finite number of clients,
where each client is associated with a unique identifier \( \cl \in \Clients \), 
and executes a sequential \emph{command} \(\cmd\), defined by:

\SpaceAboveMath
\begin{align*}
\cmd & ::=  
\pskip \!\mid\!
\cmdpri \!\mid\!  
\ptrans{\trans} \!\mid\! 
\cmd \pseq \cmd \!\mid\! 
\cmd \pchoice \cmd \!\mid\! 
\cmd \prepeat
&
 \cmdpri& ::=  
\passign{\var}{\expr} \!\mid\! 
\passume(\expr)
\\
\trans & ::=
\pskip \!\mid\!
\transpri \!\mid\! 
\trans \pseq \trans \!\mid\!
\trans \pchoice \trans \!\mid\!
\trans\prepeat    
&
\transpri & ::= 
\cmdpri \!\mid\!
\plookup{\var}{\expr} \!\mid\!
\pmutate{\expr}{\expr} 
\end{align*}
\SpaceBelowMath
 
\noindent
Sequential commands (\(\cmd\)) comprise \(\pskip\), primitive commands
(\(\cmdpri\)), atomic transactions (\(\ptrans{\trans}\)), and standard
compound constructs: sequential composition (\( ; \)), non-deterministic
choice (\( + \)) and iteration (\( * \)). 
Primitive commands include variable assignment
(\(\passign{\var}{\expr}\)) and assume statements (\(\passume(\expr)\))
which can be used to encode conditionals. They are used for computations based on client-local variables and can hence be invoked
without restriction.  Transactional commands (\(\trans\)) comprises
\(\pskip\), primitive transactional commands (\(\transpri\)), and the
standard compound constructs.  Primitive transactional commands comprise
primitive commands as well as lookup (\(\plookup{\var}{\expr}\)) and mutation
(\(\pmutate{\expr}{\expr}\)) used, respectively, to read and write a
single key to a  kv-store, and can only be invoked within an atomic
transaction.


A \emph{program} \( \prog \) is a finite partial function from client identifiers to sequential
commands.
For clarity, we often write \( \cmd_{1}\ppar \dots \ppar \cmd_{n}\) for a program with \(n\) clients identified by
\(\cl_1 \dots \cl_n\), with each client \(\cl_i\) executing \(\cmd_i\). 
Each client \(\cl_i\) is associated with a client-local \emph{stack}, 
\(\stk_i \in \Stacks \defeq \Vars \to \Val\),  mapping program variables
(ranged over by \(\pvar{x}, \pvar{y}, \cdots\))
to values. 

\mypar{Transactional Semantics} 
In our operational semantics, transactions are executed
\emph{atomically}. It is still possible for an
implementation, \eg COPS \cite{cops}, 
to update the underlying distributed kv-stores while
the transaction is in progress. It just means that, given the
abstractions captured by our global kv-stores and partial client views, 
such an update is modelled as an instantaneous  atomic
update.
Intuitively, given a configuration \(\conf {=} (\mkvs, \vienv)\), 
when a client \(\cl\) executes a transaction \(\ptrans{\trans}\), 
it performs the following steps: 
\begin{enumerate*}
	\item it constructs an initial \emph{snapshot} \(\sn\) of \(\mkvs\) using its view \(\vienv(\cl)\) as described in \cref{def:snapshot};  
	\item it executes \(\trans\) in isolation over \(\sn\) accumulating the effects (the reads and writes) of executing \(\trans\); and
	\item it commits \(\trans\) by incorporating these effects into \(\mkvs\).
\end{enumerate*}

\SpaceAboveDef
\begin{definition}[Transactional snapshots]
\label{def:heaps}
A \emph{transactional snapshot}, \( \sn \in \Snapshots \defeq \Keys \to
\Val\),  is a function from keys to values. 
\end{definition}
\SpaceBelowDef
When clear from the context, we simply refer to a transactional
snapshot  as a {\em snapshot}. 

The rules for transactional commands (\cref{fig:semantics-trans}) are defined using an arbitrary 
transactional snapshot. 
The rules for sequential commands and programs (\cref{fig:semantics-commands}) are defined using a transactional
snapshot given by a view snapshot. 
To capture the effects of executing a transaction \(\trans\) on a snapshot \(\sn\) of kv-store \(\mkvs\), 
we identify a \emph{fingerprint} of \(\trans\) on \(\sn\) which
captures the first values \(\trans\) reads from \(\sn\), and
the last values \(\trans\) writes to \(\sn\) and intends to commit to \(\mkvs\). 
Execution of a transaction in a given configuration  and variable
stack may result in more than one fingerprint due to non-determinism (non-deterministic choice). 

\SpaceAboveDef
\begin{definition}[Fingerprints]
\label{beebop}
\label{def:fingerprint}
Let \( \Ops\) denote the set of \emph{read (\( \otR\)) and write (\(\otW\)) operations} defined by 
\(\Ops \defeq \Set{(l, \key, \val) }[ l \in \Set{\otR, \otW} \land \key \in \Keys \land \val \in \Val ]\).
A \emph{fingerprint} \(\fp\) is a set of operations, \(\fp \subseteq \Ops\),
such that: 
\(\fora{\key \in \Keys, l  \in \Set{\otR, \otW}}
	(l, \key, \val_1), (l, \key, \val_2) \in \fp \implies \val_1 = \val_2\).
\end{definition}
\SpaceBelowDef

\noindent 
A fingerprint contains at most one read operation and at most one write operation for a given key. 
This reflects our assumption regarding transactions that satisfy the snapshot property: reads are taken from a single snapshot of the kv-store;
and only the last write of a transaction to each key is committed to the kv-store.
\begin{figure*}[!t]
\begin{center}
\scalebox{.9}{%
\begin{tabular}{@{}c@{\hspace{10pt}}c@{}}
    \(
    \inferrule[\rl{TPrimitive}]{%
        (\stk, \sn) \toLTS{\transpri} (\stk', \sn') 
       \\
        \op = \Func{op}(\stk, \sn, \transpri)
    }{%
        (\stk, \sn, \fp) , \transpri  \toTRANS   (\stk', \sn', \fp \addO \op) , \pskip 
    }%
    \)
    &
    \(
 \begin{array}{ll}
    \fp \addO (\otR, \key, \val)  
    & \defeq \
    \begin{cases}
        \fp \cup \Set{(\otR, \key, \val)} & \text{if } \fora{l, v'} (l, \key, v') \! \notin \! \fp \\
        \fp &  \text{otherwise} \\
    \end{cases}  \\
    \fp \addO (\otW, \key, \val) 
    & \defeq \
    \left( \fp \! \setminus \! \Set{(\otW, \key, v')}[v' \in \Val] \right)  \!
    \cup \! \Set{(\otW, \key, \val)} \\
    \fp \addO \emptyop  & \defeq \ \fp 
\end{array}
\)
\end{tabular}
}

\hrulefill

\vspace*{5pt}

\noindent
\scalebox{.9}{%
\[
    \begin{array}{@{} r @{\ } c @{\ } l @{\quad} r @{\ } c @{\ } l@{}}
(\stk, \sn)  & \toLTS{\passign{\var}{\expr}}
             & (\stk\rmto{\var}{\evalE{\expr}}, \sn) 
&
(\stk, \sn)  & \toLTS{\passume(\expr)}  
             & (\stk, \sn) \text{ where } \evalE{\expr} \neq 0
\\
(\stk, \sn)  & \toLTS{ \plookup{\var}{\expr} } 
             & (\stk\rmto{\var}{\sn(\evalE{\expr})}, \sn) 
&
(\stk, \sn) & \! \toLTS{\pmutate{\expr_{1}}{\expr_{2}}} \!
            & (\stk, \sn\rmto{\evalE{\expr_{1}}}{\evalE{\expr_{2}}}) \ 
\end{array}
\]%
}

\vspace*{5pt}

\hrulefill

\vspace*{5pt}

\noindent
\scalebox{.9}{%
\[%
\begin{aligned}
    \Func{op}(\stk, \sn, \passign{\var}{\expr}) & \defeq  \emptyop 
    & 
    \Func{op}(\stk, \sn, \passume(\expr)) & \defeq \emptyop 
    \\
    \Func{op}(\stk, \sn,  \plookup{\var}{\expr}) & \defeq (\otR, \evalE{\expr}, \sn(\evalE{\expr})) 
    &
    \Func{op}(\stk,  \sn, \pmutate{\expr_{1}}{\expr_{2}}) & \defeq (\otW, \evalE{\expr_{1}}, \evalE{\expr_{2}})
\end{aligned}
\]%
}
\end{center}

\spaceshrink{-5pt}

\hrulefill

\caption{The semantics of transactional commands}
\label{fig:semantics-trans}
\end{figure*}


The rule for primitive transactional commands, \( \rl{TPrimitive} \),
is given in \cref{fig:semantics-trans}.
The rules for the compound constructs are straightforward and given in \cite{shale-phd}.
The \( \rl{TPrimitive} \) rule updates 
the snapshot and the fingerprint of a transaction: the premise 
\((\stk, \sn) \toLTS{\transpri} (\stk', \sn')\) describes how executing
\(\transpri\) affects the local state (the client stack and the snapshot) of a transaction; 
and the premise \(o =\Func{op}(\stk, \sn, \transpri)\) identifies the operation on the kv-store associated with \(\transpri\), 
where the empty operation \(\emptyop\) is used for those primitive commands that do not
contribute to the fingerprint.

The conclusion of \( \rl{TPrimitive}\) uses the \emph{combination operator} 
\(\addO: \pset{\Ops} \times (\Ops \uplus \Set{\emptyop}) \to \pset{\Ops}\), defined 
in \cref{fig:semantics-trans}, to extend the fingerprint \(\fp\) accumulated with
operation \(o\) associated with \(\transpri\), as
appropriate: it adds a read from \(\key\)  if \(\fp\) 
contains no entry for \(\key\), and it always updates the write for 
\(\key\) to \(\fp\), removing previous writes to \(\key\).

\mypar{Command and Program Semantics}
We give the operational semantics of commands and programs in
\cref{fig:semantics-commands}.  
%
\begin{figure*}[!t]
\begin{center}
\scalebox{.9}{%
\begin{tabular}{@{}c@{\hspace{20pt}}c@{}}
    \(
    \inferrule[\rl{CPrimitive}]{
		\stk \toLTS{\cmdpri} \stk'
    }{
        \cl \vdash 
        ( \mkvs, \vi, \stk ) , \cmdpri 
        \toCMD{(\cl,\iota)}_{\ET} 
        ( \mkvs, \vi, \stk' ) , \pskip
    }%
    \)
    &
    \(
 \begin{array}{r @{\ }c @{\ } l}
        \stk & \toLTS{\passign{\var}{\expr}} & \stk\rmto{\var}{\evalE{\expr}} \\
        \stk & \toLTS{\passume(\expr)} & \stk \ \text{where} \ \evalE{\expr} \neq 0
\end{array}
\)
\\[20pt]
\multicolumn{2}{c}{
    \(
     \inferrule[\rl{CAtomicTrans}]{%
        \vi \viewleq  \vi'' 
        \quad 
        \sn = \snapshot[\mkvs,\vi''] 
        \quad
        (\stk, \sn, \emptyset), \trans \toTRANS^{*}   (\stk', \stub,
  \fp) , \pskip
  \quad
	\cancommit{\mkvs}{\vi''}{\fp}
    \\\\
   \txid \in \nextTxid[\cl, \mkvs] 
\\
     \mkvs' = \updateKV[\mkvs, \vi'', \fp, \txid] 
\\
	\vshift{\mkvs}{\vi''}{\mkvs'}{\vi'}	
    }{%
        \cl \vdash 
        ( \mkvs, \vi, \stk ), \ptrans{\trans} 
        \toCMD{(\cl, \vi'', \fp)}_{\ET}
        (\mkvs',\vi', \stk' ) , \pskip
    }%
    \)
    }


    \\[15pt]
    \multicolumn{2}{c}{
    \(
    \inferrule[\rl{PProg}]{%
	    \vi = \vienv (\cl)
        \\
        \stk = \thdenv(\cl)
        \\
        \cmd = \prog(\cl)
        \\
		\cl \vdash 
		( \mkvs, \vi, \stk ) , \cmd
		\toCMD{\lambda}_{\ET} 
		( \mkvs', \vi', \stk' ) , \cmd'  
    }{%
		\vdash 
		(\mkvs, \vienv, \thdenv ), \prog
		\toCMD{\lambda}_{\ET} 
		( \mkvs', \vienv\rmto{\cl}{\vi'}, \thdenv\rmto{\cl}{\stk'} ) , \prog\rmto{\cl}{\cmd'} ) 
    }%
    \)
    }\\[-10pt]
\end{tabular}
}
\end{center}

\hrulefill

\caption{The semantics of sequential commands and programs}
\label{fig:semantics-commands}
\end{figure*}
%
The command semantics describes transitions of the form
\(\cl \vdash (\mkvs, \vi, \stk), \cmd \ \toCMD{\lambda}_{\ET} \ (\mkvs', \vi', \stk'), \cmd'\)
stating that,  given the kv-store \(\mkvs\), client view \(\vi\) and stack \(\stk\), 
a client \(\cl\) may execute command \(\cmd\) for one step, updating 
the kv-store to \(\mkvs'\), the stack to \(\stk'\), the view to \( \vi' \) and the command to its continuation \(\cmd'\).
The label \(\lambda\) is either of the form \((\cl, \iota)\) denoting that \(\cl\) executed a primitive command
that required no access to \(\mkvs\), 
or \((\cl, \vi'', \fp)\) denoting that \(\cl\) committed an atomic transaction with final fingerprint \(\fp\) under the view \(\vi''\).
The semantics is parametric in the choice of the \emph{execution test}
\(\ET\), which is used to generate the \emph{consistency model} under which a transaction can execute.
In \cref{sec:cm},  we give several examples of execution tests for well-known consistency models.
In \cite{shale-phd},  we prove that the consistency models generated by our execution tests are equivalent to their corresponding existing definitions using abstract executions. 

The rules for compound constructs are straightforward and given in \cite{shale-phd}.
The rule for primitive commands, \(\rl{CPrimitive}\), depends on the 
transition system \(\toLTS{\cmdpri} \subseteq \Stacks \times \Stacks\) 
which describes how the primitive command \(\cmdpri\) affects the stack.
The \rl{CAtomicTrans} rule describes the execution of an atomic 
transaction under the execution test \(\ET\). 


We explain the \rl{CAtomicTrans} rule in detail. 
The first premise states that the current view \(\vi\) of the executing command may be advanced to a newer view \(\vi''\) (see \cref{def:views}). 
Given the new view \(\vi''\), the transaction obtains a snapshot \(\sn\) of the kv-store \(\mkvs\), 
and executes \(\trans\) locally to completion (\(\pskip\)), updating the stack to \(\stk'\), while accumulating the fingerprint \(\fp\), 
as described by the second and third premises of \rl{CAtomicTrans}.
Note that the resulting snapshot is ignored as the effect of the transaction is recorded in the fingerprint \(\fp\). 
The \(\cancommit{\mkvs}{\vi''}{\fp}\) premise ensures that,  under the execution test \(\ET\), 
the final fingerprint \(\fp\) of the transaction is compatible with the (original) kv-store
\(\mkvs\) and the client view \(u''\), and thus the transaction \emph{can commit}. 
Observe that the \cancommitname check is parametric in the execution test \(\ET\).
This is because the conditions checked upon committing depend on the consistency model under which the transaction is to commit. 
In \cref{sec:cm}, we define \cancommitname for several execution tests associated with well-known consistency models.


Client \(\cl\) is  now ready to commit the transaction resulting 
in the kv-store \(\mkvs'\) with the client view \(\vi''\) \emph{shifting} to a new view \(\vi'\) and proceeds as follows: 
\begin{enumerate*}
	\item it picks a fresh transaction identifier \(\txid \in \nextTxid[\cl, \mkvs]\);
	\item computes the new kv-store \(\mkvs' = \updateKV(\mkvs, \vi'',\fp, \txid\); and 
	\item checks if the \emph{view shift} is permitted under \(\ET\) using \(\vshift{\mkvs}{\vi''}{\mkvs'}{\vi'}\). 
\end{enumerate*}
Note that as with \(\cancommitname\), the \(\vshiftname\) check is parametric in the execution test \(\ET\). 
This is because the conditions checked for shifting the client view depend on the consistency model. 
In \cref{sec:cm} we define \(\vshiftname\) for several execution tests associated with well-known consistency models.
The set \(\nextTxid[\cl, \mkvs]\) is given by:
\(
\{ \txid_{\cl}^{n} \mid %
\fora{m} \txid_{\cl}^{m} \in \mkvs \allowbreak \implies \allowbreak m < n \}
\).
The function \(\updateKV[\mkvs, \vi, \fp, \txid]\)
describes how the fingerprint \(\fp\) of transaction \(\txid\) executed under view \(\vi\) updates kv-store \(\mkvs\):
for each read \((\otR, \key, \val) \in \fp\), it adds \(\txid\) 
to the reader set of the last version of \(\key\) in \(\vi\); 
for each write \((\otW, \key, \val)\), it appends a new version \((\val, \txid, \emptyset)\) 
to \(\mkvs(\key)\). 
The function \( \updateKV \) is well-formed, because a fingerprint contains at most one write operation and one read operation for a given key (see \cite{shale-phd} for the full details).


\SpaceAboveDef
\begin{definition}[Transactional update]
\label{eq:updatekv}
\label{def:updatekv}
The function  \(\updateKV[\mkvs, \vi, \fp, \txid]\) is defined as: 

\SpaceAboveMath
\begin{align*}
    \updateKV[\mkvs, \vi, \emptyset, \txid] & \defeq \mkvs
    \\ \updateKV[\mkvs, \vi, \Set{(\otR, \key, \val)} \uplus \fp, \txid]
    & \defeq 
    \begin{multlined}[t]
    \text{let} \ i = \max{}_{<}(\vi(\key)) \ \text{and} \ (\val,\txid',\txidset) = \mkvs(\key,i) \text{ in } 
    \\ \updateKV[\mkvs\rmto{\key}{\mkvs(\key)\rmto{i}{(\val, \txid', \txidset \uplus \Set{\txid})}},\vi, \fp, \txid] 
    \end{multlined}
    \\ \updateKV[\mkvs, \vi, \Set{(\otW, \key, \val)} \uplus \fp, \txid]
    & \defeq \text{let } \mkvs' = \mkvs\rmto{\key}{ \mkvs(\key) \lcat (\val, \txid, \emptyset) } \text{ in } \updateKV[\mkvs', \vi, \fp, \txid] 
\end{align*}
\SpaceBelowMath

\noindent 
where \(\vilist\rmto{i}{\ver} \defeq \ver_0 \lcat \cdots \lcat \ver_{i-1} \lcat \ver \lcat \ver_{i+1} \lcat \cdots \lcat \ver_{n}\) for all version lists \(\vilist = \ver_0 \lcat \cdots \lcat \ver_n\) 
and indexes \(i: 0 \leq i \leq n\).

\end{definition}
\SpaceBelowDef

The last rule, \( \rl{PProg} \) (\cref{fig:semantics-commands}),
captures the execution of a program step 
using a \emph{client environment}, \(\thdenv \in \ThdEnv\), 
which is a function from client identifiers to stacks  associating each client with its stack. 
We assume that the domain of a client environment contains 
the domain of the program throughout the execution: 
\(\dom(\prog) \subseteq \dom(\thdenv)\).
Program transitions are simply defined in terms of the transitions of
their constituent client commands. 
This yields an interleaving semantics for transactions of different clients:  
a client executes a transaction in an atomic step without
interference from the other clients. 
