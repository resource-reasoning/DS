\section{Operational Semantics of Key-Value Stores}

We define a simple programming language for client programs interacting with a kv-store.
Clients may only interact with a kv-store by issuing read and write requests via transactions. 
For simplicity, we abstract away from aborting transactions:
rather than assuming a transaction may abort due to a violation of the consistency guarantees given by the kv-store,
we only allow a transaction to execute if its effects are guaranteed to not violate the consistency model. 
This emulates the setting in which clients always restart a transaction if it aborts.


\subsubsection{Programming Language}

\emph{A program} \( \prog \) comprises a finite number of clients,
where each client is associated with a unique identifier \( \cl \in \Clients \), 
and executes a sequential \emph{command} $\cmd$, given by the following grammar:
\[
\begin{array}{@{} r @{\hspace{2pt}} l @{\hspace{30pt}} r @{\hspace{2pt}} l@{} }
	\cmd ::=  &
        \pskip \mid 
        \ptrans{\trans} \mid 
	    \cmdpri \mid  
        \cmd \pseq \cmd \mid 
        \cmd \pchoice \cmd \mid 
        \cmd \prepeat 
        
   & \cmdpri ::=  &
   		\passign{\txvar}{\expr} \mid 
   		\passume{\expr} \\
%   
	\trans ::= &
        \pskip \mid
        \transpri \mid 
        \trans \pseq \trans \mid
        \trans \pchoice \trans \mid
        \trans\prepeat    
	& \transpri ::= &
   		\cmdpri \mid
        \plookup{\txvar}{\expr} \mid
        \pmutate{\expr}{\expr} 
 \end{array} 
\]
%
Sequential commands include the standard constructs of $\pskip$, sequential composition ($\cmd \pseq \cmd$), non-deterministic choice ($\cmd \pchoice \cmd$), loops $\cmd\prepeat$, 
as well as \emph{transactions} ($\ptrans{\trans}$) and primitive commands. 
Primitive commands include variable assignments ($\passign{\txvar}{\expr}$) and assume statements ($\passume{\expr}$) used to encode conditionals,
and are split into transactional ones (the $\transpri$ clause) 
and non-transactional ones (the $\cmdpri$ clause).
Transactional primitive commands are used for reading and writing to kv-stores and 
can be invoked only within the boundaries of transactions (the $\ptrans{\trans}$ clause).
Non-transactional primitive commands are used for computations over client-local data
and can be invoked without restrictions.

For clarity, we often write \( \cmd_{1}\ppar \dots \ppar \cmd_{n}\) as syntactic sugar 
for a program \( \prog \) with $n$ implicit clients associated with identifiers, 
$\cl_1 \dots \cl_n$, where each client $\cl_i$ executes $\cmd_i$: 
\( \prog = \Set{\cl_{1} \mapsto \cmd_{1}, \dots, \cl_{n} \mapsto \cmd_{n}  }\).

As is standard, we assume a language of expressions built from values ($\val \in \Val$), 
and (client-local) \emph{program variables} $\Vars$, ranged over by $\pvar{x}, \pvar{y}, \cdots$. 
The evaluation $\evalE{\expr}$ of  expression $\expr$ is parametric in the (client-local) \emph{stack} 
$\stk \in \Stacks \defeq \Vars \to \Val$, mapping program variables to values. 
\[
\expr  ::= 
        \val \mid
        \var \mid
        \expr + \expr \mid
        \dots  
\qquad   
\evalE{\val}  =  \val \quad 
\evalE{\var} = \stk(\var)  \quad  
\evalE{\expr_{1} + \expr_{2}}  =  \evalE{\expr_{1}} + \evalE{\expr_{2}} \quad
\dots
\]


\subsubsection{Operational Semantics of Transactions}
Transactional commands in $\transpri$ are associated with a transition system: 
$\toLTS{\transpri} \subseteq (\Stacks \times \Snapshots) \times (\Stacks \times \Snapshots)$, 
describing how the snapshot and stack evolve upon executing $\transpri$:
% \cref{fig:semantics}.
\[
\begin{array}{rcl @{\qquad} rcl}
(\stk, \sn)  & \toLTS{\passign{\var}{\expr}}          & (\stk\rmto{\var}{\evalE{\expr}}, \sn)                  &
(\stk, \sn)  & \toLTS{\passume{\expr}}                & (\stk, \sn) \text{ where } \evalE{\expr} \neq 0        \\
(\stk, \sn)  & \toLTS{\plookup{\var}{\expr}}           & (\stk\rmto{\var}{\sn(\evalE{\expr})}, \sn)              &
(\stk, \sn)  & \toLTS{\pmutate{\expr_{1}}{\expr_{2}}} & (\stk, \sn\rmto{\evalE{\expr_{1}}}{\evalE{\expr_{2}}}) \\
\end{array}                                                                                               
\]

To compute the overall effect of a transaction,
we define a \emph{fingerprint function}, 
$\func{fp}{\stub} : \Stacks \times \Snapshots \times \transpri \rightarrow \Ops \cup \{\varepsilon\}$, extracting the operation of a primitive transactional command: 
%
\[
\begin{array}{rcl @{\quad} rcl}
\func{fp}{\stk, \sn, \passign{\var}{\expr}}          & \defeq & \emptyop                                     &
\func{fp}{\stk, \sn, \passume{\expr}}                & \defeq & \emptyop                                     \\
\func{fp}{\stk, \sn, \plookup{\var}{\expr}}           & \defeq & (\otR, \evalE{\expr}, \sn(\evalE{\expr}))     &
\func{fp}{\stk, \sn, \pmutate{\expr_{1}}{\expr_{2}}} & \defeq & (\otW, \evalE{\expr_{1}}, \evalE{\expr_{2}}) \\
\end{array}
\]
%and a \emph{fingerprint composition operator} \( \addO \).
Note that non-transactional primitive commands are associated with the empty operation $\varepsilon$,
as they only access the local stack and do not access the kv-store.

We further define a \emph{fingerprint composition operator}, \( \addO \), 
incrementally computing the fingerprint of a transaction, after executing each constituent primitive command. 
For instance, when executing $ \ptrans{\trans}$ with $\trans \defeq \transpri^1; \cdots ; \transpri^n$,
the effect of each $\transpri^i$ is calculated via the $\op_i = \func{fp}{-, -, \transpri^i}$ function, 
with the overall fingerprint given as the $\addO$-composition of the constituent effects: $\op_1 \addO \cdots \addO \op_n$. 

For each key $\key$, the composition operator \( \addO \) records
the first value a transaction reads (before a subsequent write) for $\key$, 
and the last value the transaction writes for $\key$.
This is in line with our assumption that transactions read from an atomic snapshot of the kv-store.
In particular, for each key $\key$, 
only the first read from $\key$ fetches its value from the kv-store,
and since clients observe either none or all effects of a transaction, 
only the last write to $\key$ is committed.
%
\begin{align*}
    \fp \addO (\otR, \key, \val)  & \defeq
    \begin{cases}
        \fp \cup \{(\otR, \key, \val)\} & \text{if} \ \fora{l, v'} (l, \key, v') \notin \fp \\
        \fp &  \text{otherwise} \\
    \end{cases} 
    \fp \addO \emptyop  \defeq  \fp \\
    \fp \addO (\otW, \key, \val) & \defeq 
    \left( \fp \setminus \Set{(\otW, \key, \val')}[\val' \in \Val] \right) \cup \Set{(\otW, \key, \val)} 
\end{align*}
\]
%
\ifTechReport
    
\begin{figure*}[!t]
\[
\begin{rclarray}
\toL & : & ((\Stacks \times \Snapshots \times \Fingerprints) \times \Transactions) \times ((\Stacks \times \Snapshots \times \Fingerprints) \times \Transactions)
\end{rclarray}
\]
\begin{mathpar}
    \infer[\rl{TPrimitive}]{%
        (\stk, \h, \fp) , \transpri \ \toL \  (\stk', \h', \fp \addO \op) , \pskip 
    }{%
        (\stk, \h) \toLTS{\transpri} (\stk', \h')
        \\ \op = \func{op}{\stk, \h, \transpri}
    }
    \\\
    \infer[\rl{TChoice}]{%
        (\stk, \h, \fp) , \trans_{1} \pchoice \trans_{2} \ \toL \  (\stk, \h, \fp) , \trans_{i}
    }{
        i \in \Set{1,2}
    }
    \and
    \infer[\rl{TIter}]{%
        (\stk, \h, \fp),  \trans\prepeat \ \toL \  (\stk, \h, \fp), \pskip \pchoice (\trans \pseq \trans\prepeat)
    }{ } 
    \and
    \infer[\rl{TSeqSkip}]{%
        (\stk, \h, \fp), \pskip \pseq \trans \ \toL \  (\stk, \h, \fp), \trans
    }{ }
    \and
    \infer[\rl{TSeq}]{%
        (\stk, \h, \fp), \trans_{1} \pseq \trans_{2} \ \toL \  (\stk', \h', \fp'), \trans_{1}' \pseq \trans_{2}
    }{%
        (\stk, \h, \fp), \trans_{1} \ \toL \  (\stk', \h', \fp'), \trans_{1}'
    }
\end{mathpar}
\hrulefill
\[
\begin{rclarray}
	\toT{}  & : &
    \begin{array}[t]{@{}c@{}}
    \Clients \; \times \;
	\left( ( \HisHeaps \times \Views \times \Stacks ) \times \Commands \right)  
    \; \times\; \ETs \;\times \sort{Labels} \times \;
	\left( ( \HisHeaps \times \Views \times \Stacks ) \times \Commands \right) 
    \end{array}
\end{rclarray}
\]
\begin{mathpar}
    \infer[\rl{CAtomicTrans}]{%
        \cl \vdash 
        ( \mkvs, \vi, \stk ), \ptrans{\trans} \ 
        \toT{(\cl, \vi'', \fp)}_{\ET} \ 
        (\mkvs',\vi', \stk' ) , \pskip
    }{%
		\begin{array}{@{} c @{}}
			\vi \orderVI  \vi''
			\quad \h = \clpsHH{\mkvs,\vi''}
			\quad \txid \in \nextTxId(\cl, \mkvs) \\
			(\stk, \h, \emptyset), \trans \toL^{*}   (\stk', \stub,  \fp) , \pskip
            \quad \mkvs' = \updateKV(\mkvs, \vi'', \fp, \txid) 
            \quad \ET \vdash (\mkvs, \vi'') \csat \f : (\mkvs',\vi')
			%\qquad \ET \vdash (\mkvs, \vi'') \triangleright \fp : \vi'
		\end{array}
    }
    \and
    \infer[\rl{CPrimitive}]{%
    \cl \vdash ( \mkvs, \vi, \stk ) , \cmdpri \ \toT{(\cl,\iota)}_{\ET} \  ( \mkvs, \vi, \stk' ) , \pskip
    }{
        \stk \toLTS{\cmdpri} \stk'
    }
    \and
    \infer[\rl{CChoice}]{%
        \cl \vdash ( \mkvs, \vi, \stk ) , \cmd_{1} \pchoice \cmd_{2} \ \toT{(\cl,\iota)}_{\ET} \  ( \mkvs, \vi, \stk ) , \cmd_{i}
    }{
        i \in \Set{1,2}
    }
    \quad
    \infer[\rl{CIter}]{%
        \cl \vdash ( \mkvs, \vi, \stk ) , \cmd\prepeat \ \toT{(\cl,\iota)}_{\ET} \  ( \mkvs, \vi, \stk ) , \pskip \pchoice (\cmd \pseq \cmd\prepeat)
    }{ }
    \and
    \infer[\rl{CSeqSkip}]{%
        \cl \vdash ( \mkvs, \vi, \stk ) , \pskip \pseq \cmd \ \toT{(\cl,\iota)}_{\ET} \  ( \mkvs, \vi, \stk ) , \cmd
    }{ }
    \quad
    \infer[\rl{CSeq}]{%
        \cl \vdash ( \mkvs, \vi, \stk ) , \cmd_{1} \pseq \cmd_{2} \ \toT{(\cl,\iota)}_{\ET} \ ( \mkvs, \vi', \stk' ) , {\cmd_{1}}' \pseq \cmd_{2}
    }{% 
        \cl \vdash ( \mkvs, \vi, \stk ) , \cmd_{1} \ \toT{(\cl,\iota)}_{\ET} \  ( \mkvs, \vi', \stk' ) , {\cmd_{1}}' 
    }
\end{mathpar}
\vspace*{5pt}

\hrulefill

\[
	\toG{} : 
    ( \Confs \times \ThdEnv \times \Programs) 
    \;\times\; \ETs \;\times \sort{Label} \times \;
    ( \Confs \times \ThdEnv \times \Programs) 
\]
\begin{mathpar}
    \infer[\rl{PProg}]{%
    ( \mkvs, \viewFun, \thdenv ) , \prog \ \toG{\lambda}_{\ET} \  ( \mkvs', \viewFun\rmto{\cl}{\vi'}, \thdenv\rmto{\cl}{\stk'}) , \prog\rmto{\cl}{\cmd'} 
    }{%
        \cl \vdash ( \mkvs, \viewFun(\cl), \thdenv(\cl) ), \prog(\cl), \ \toT{\lambda}_{\ET} \  ( \mkvs', \vi', \stk' ) , \cmd'  
    }
\end{mathpar}
%
\hrulefill
\caption{Operational Semantics on Key-value Store}
\label{fig:transaction_semantics}
\label{def:thread_semantics}
\label{fig:thread_semantics}
\label{def:thread_pool_semantics}
\label{fig:thread_pool_semantics}
\label{def:program_semantics}
\label{fig:program_semantics}
\label{fig:full-semantics}
\end{figure*}

\else
\begin{figure}[t]
\hrulefill
\begin{mathpar}
    \inferrule[\rl{TPrimitive}]{%
        (\stk, \sn) \toLTS{\transpri} (\stk', \sn')
        \\ \op = \func{fp}{\stk, \sn, \transpri}
    }{%
        (\stk, \sn, \fp) , \transpri \ \toTRANS \  (\stk', \sn', \fp \addO \op) , \pskip 
    }
    \\
    \inferrule[\rl{TChoice}]{
        i \in \Set{1,2}
    }{%
        (\stk, \sn, \fp) , \trans_{1} \pchoice \trans_{2} \ \toTRANS \  (\stk, \sn, \fp) , \trans_{i}
    }
    \and
    \inferrule[\rl{TIter}]{ }{%
        (\stk, \sn, \fp),  \trans\prepeat \ \toTRANS \  (\stk, \sn, \fp), \pskip \pchoice (\trans \pseq \trans\prepeat)
    } 
    \and
    \inferrule[\rl{TSeqSkip}]{ }{%
        (\stk, \sn, \fp), \pskip \pseq \trans \ \toTRANS \  (\stk, \sn, \fp), \trans
    }
    \and
    \inferrule[\rl{TSeq}]{%
        (\stk, \sn, \fp), \trans_{1} \ \toTRANS \  (\stk', \sn', \fp'), \trans_{1}'
    }{%
        (\stk, \sn, \fp), \trans_{1} \pseq \trans_{2} \ \toTRANS \  (\stk', \sn', \fp'), \trans_{1}' \pseq \trans_{2}
    }
\end{mathpar}
\hrulefill
\[
    \inferrule[\rl{PCommit}]{%
        \vi \viewleq  \vi''
        \qquad \sn = \snapshot[\mkvs,\vi'']
        \qquad \txid \in \nextTxid(\cl, \mkvs)
        \\\\ (\stk, \sn, \emptyset), \trans \ \toTRANS^{*} \  (\stk', \stub,  \fp) , \pskip
        \\ \ET \vdash (\mkvs, \vi'') \triangleright \fp : \vi'
    }{%
        \cl \vdash ( \mkvs, \vi, \stk ), \ptrans{\trans} \ \toCMD{(\cl, \vi'', \fp)}_{\ET} \ ( \updateKV[\mkvs, \vi, \txid, \fp],\vi', \stk' ) , \pskip
    }
\]
\[
    \inferrule[\rl{PPrimitive}]{
        \stk \toLTS{\cmdpri} \stk'
    }{%
    \cl \vdash ( \mkvs, \vi, \stk ) , \cmdpri \ \toCMD{(\cl,\iota)}_{\ET} \  ( \mkvs, \vi, \stk' ) , \pskip
    }
\]
\hrulefill
\[
    \inferrule[\rl{PProg}]{%
         \cl \vdash ( \mkvs, \vi, \thdenv(\cl) ) , \prog(\cl), \ \toCMD{\lambda}_{\ET} \  ( \mkvs', \vi', \stk' ) , \cmd'  
    }{%
         (\mkvs, \vienv\rmto{\cl}{\vi}, \thdenv ), \prog  \ \toCMD{\lambda}_{\ET} \  ( \mkvs', \vienv\rmto{\cl}{\vi'}, \thdenv\rmto{\cl}{\stk'} ) , \prog\rmto{\cl}{\cmd'} ) 
    }
\]
\hrulefill
\caption{Selected operational semantics rules for transactions (above); sequential commands (middle); and programs (below)}
\label{fig:semantics}
\end{figure}
\fi

In the top part of \cref{fig:semantics} we present a few selected rules of the operational semantics of transactional code (within the transaction boundaries in $\ptrans{}$). 
We refer the reader to \cref{sec:full-semantics} for the full operational semantics.
Transactional rules are of the form $(\stk, \sn, \fp), \toTRANS (\stk', \sn', \fp'), \trans'$, 
stating that when the fingerprint accumulated so far is recorded in $\fp$, 
executing $\trans$ for one step transforms the client-local stack $\stk$ 
and snapshot $\sn$ %(used to fetch the values of read operations),
to $\stk'$ and $\sn'$, respectively, with the extended fingerprint $\fp'$, and continuation $\trans'$.
The \rl{TSeq} rule is in a standard format; 
the \rl{TPrimitive}  models the execution of primitive transactional commands.
When $(\stk, \sn)$ is transformed to $(\stk', \sn')$ via the transition system $\toLTS{\transpri}$,
$\func{fp}{\stk, \sn, \transpri} = \op$, and the fingerprint recorded so far is given by $\fp$, 
then executing $\transpri$ transforms $(\stk, \sn)$ to $(\stk', \sn')$, with the fingerprint updated to $\fp \addO \op$ (to incorporate the effect of $\transpri$), with continuation $\pskip$. 

\subsubsection{Operational Semantics of Commands}
In \cref{fig:semantics} we present a select number of \emph{(sequential) command operational semantics}.
We refer the reader to \cref{sec:full-semantics} for the full operational semantics.
Command transitions are of the form $\cl \vdash (\mkvs, \vi, \stk), \cmd \ \toCMD{\lambda}_{\ET} \ (\mkvs', \vi', \stk') , \cmd'$, 
stating that given the kv-store $\mkvs$, view $\vi$ and stack $\stk$, client $\cl$ may execute command $\cmd$ for one step under $\ET$, update the kv-store to $\mkvs'$, the stack to $\stk'$, and the command to its continuation $\cmd'$, with label $\lambda$.
A transition label $\lambda$ is either
\begin{enumerate*}
	\item of the form $(\cl, \iota)$, denoting that the transition involved 
client-local computation that did not require access to the kv-store (\eg primitive commands $\cmdpri$; or
	\item of the form $(\cl, \vi, \fp)$, denoting that client $\cl$ commits a transaction with fingerprint $\fp$ under the view $\vi$.
\end{enumerate*}
Note that the operational semantics of commands are parametric in the choice of execution test $\ET$, 
and thus the choice of the underlying consistency model.


With the exception of the \rl{PCommit} rule, the remaining command transitions are standard and behave as expected. 
For instance, the \rl{PPrimitive} rule models the execution of a non-transactional primitive command $\cmdpri$, where the $\stk \toLTS{\cmdpri} \stk'$ transition describes how the stack of a client 
evolve upon executing $\cmdpri$:
\[
\begin{array}{rcl @{\qquad} rcl}
\stk  & \toLTS{\passign{\var}{\expr}} & \stk\rmto{\var}{\evalE{\expr}} &
\stk  & \toLTS{\passume{\expr}} & \stk \text{ where } \evalE{\expr} \neq 0 \\
\end{array}                                                                                               
\]
%
%
The \rl{PCommit} rule models the execution of a transaction $\ptrans{\trans}$, under the execution test $\ET$. 
The first premise of \rl{PCommit} states that the current view $\vi$ of the executing command maybe advanced to a newer atomic view $\vi''$ (see \cref{def:views}). 
The semantics only allows to advance the view to latest versions, which corresponds to \emph{monotonic read} \cite{.......}.
Given the new view $\vi''$, the transaction proceeds by obtaining a snapshot $\sn$ of the kv-store $\mkvs$, and executing $\trans$ locally to completion ($\pskip$), updating the stack to $\stk'$, while accumulating the fingerprint $\fp$. Note that the resulting snapshot is ignored (denoted by $\stub$) as the effect of the transaction is recorded in the fingerprint $\fp$. 
%

The transaction is now ready to commit and may propagate its changes to $\mkvs$.
To this end, a \emph{fresh} transaction identifier $\txid \in \nextTxid(\cl, \mkvs)$ (defined shortly) is picked
to identify the completed transaction, and the changes performed by $\txid$ are propagated to $\mkvs$. 
This is done via the $\func{updateMKVS}{\mkvs, \vi, \txid, \fp}$ function (defined shortly) to update $\mkvs$. 
%The fresh identifier $\txid$ is given by the $t \in \nextTxid(\cl, \mkvs)$ function defined shortly.
Once the kv-store is updated, the client subsequently updates its view to $\vi'$ with respect to its fingerprint. 
Lastly, to ensure that the effect of the transaction (its fingerprint  $\fp$) is permitted by the underlying consistency model, 
the last premise requires that the updates be permitted by the execution test $\ET$, \ie \( (\mkvs, \vi'') \csat \fp : \vi'\).

The $\nextTxid(\cl, \mkvs)$ returns the set of transactions identifiers associated with $\cl$ that are fresh with respect to $\mkvs$: 
$\nextTxid(\cl, \mkvs) \defeq \Set{\txid_{\cl}^{n}}[\fora{m} \txid_{\cl}^{m} \in \mkvs \implies m < n ]$.
Note that when $\txid_\cl^n = \nextTxid(\cl, \mkvs)$ in the premise of \rl{PCommit}, then $\txid_\cl^n$ is greater than any transaction identifier 
(with respect to session order $\toEDGE{\SO}$) 
of the form $\txid_{\cl}^{i}$ appearing in $\mkvs$,
as to reflect the fact that $\txid_\cl^n$ is the most recent transaction executed by $\cl$.

%
The \( \updateKV \) function is defined below, where $\lcat$ denotes list concatenation; 
and when $\vilist = \ver_0, \cdots, \ver_n$ and $i=0,\cdots,n$, 
$\vilist\rmto{i}{\ver}$ denotes the updated list 
$\vilist' = \ver_0, \cdots, \ver_{i-1}, \ver, \ver_{i+1}, \cdots, \ver_{n}$. 
%
%
%
%
\begin{align*}         
    \updateKV[\mkvs, \vi, \txid, \emptyset] &\defeq \mkvs \\
    \updateKV[\mkvs, \vi, \txid, \fp \uplus \Set{(\otR, \key, \stub)}] & \defeq 
    \begin{multlined}[t]
        \texttt{let} \ (\val, \txid', \txidset) = \mkvs(\key, \max_{<}(\vi(\key))), \\
        \vilist = \mkvs(\key)\rmto{\max_{<}(\vi(\key))}{(\val, \txid', \txidset \uplus \Set{ \txid })}\\
        \texttt{in} \ \updateKV[\mkvs\rmto{\key}{\vilist}, \vi, \txid, \fp]
    \end{multlined} \\
    \updateKV[\mkvs, \vi, \txid, \fp \uplus \Set{(\otW, \key, \val)}]& \defeq &  
    \begin{multlined}[t]
        \texttt{let } \mkvs' = \mkvs\rmto{\key}{ \mkvs(\key) \lcat (\val, \txid, \emptyset) } \\
        \quad \texttt{in } \updateKV[\mkvs', \vi, \txid, \fp]
    \end{multlined} \\
\end{align*}
%
%
For every read operation $(\otR, \key, \stub)$ in fingerprint $\fp$,
since a transaction reads the values of keys from 
the atomic snapshot determined by the view of the client, 
the version read by $\txid$ for key $\key$ corresponds to $\mkvs(\key, \max_{<}(\vi(\key)))$.
As such, to commit the transaction, 
the reader set of the version $\mkvs(\key, \max_{<}(\vi(\key)))$ is extended with a new reader $\txid_\cl^n$.
Similarly, for every write operation $(\otW, \key, \val)$ in fingerprint $\fp$, 
the list of versions $\mkvs(\key)$ is extended with a new version $(\val, \txid, \emptyset)$, 
denoting that $\txid_\cl^n$ is the writer of this version which has no readers as of yet. 

Note that the assumption that 
versions of a given key are totally ordered and consistent with the order in which 
transactions commit is standard, 
which corresponds to last-write-win resolution policy \cite{adya,framework-concur,seebelieve}. 
Moreover, under the assumption that fingerprints contain at most one read and one write 
operation per key, and the identifier is fresh $\txid \notin \mkvs$%
\footnote{%
We write $\txid \in \mkvs$ when there exists a key 
$\key$ and an index $i=0,\cdots, \lvert \mkvs(\key) \rvert -1$ such that $\txid \in \{\wtOf(\mkvs(\key, i)\} \cup \rsOf(\mkvs(\key, i))$.}, 
the $\updateKV$ is well-defined.

\subsubsection{Operational Semantics of Programs}

The \emph{operational semantics of programs} are given at the bottom of \cref{fig:semantics}. 
Programs transitions are of the form $(\conf,  \thdenv, \prog) \ \toPROG{\lambda}_{\ET} (\conf',  \thdenv', \prog')$,
stating that given the configuration $\conf$ and the \emph{client environment} $\thdenv$, executing program $\prog$ for one step under $\ET$, updates the configuration to $\conf'$, the client environment to $\thdenv'$, and the program to its continuation $\prog'$. 
A \emph{client environment}, $\thdenv \in \ThdEnv$, is a mapping from client identifiers to pairs of stacks and views. 
We assume that the clients in the domain of client environments are are those in the domain of the program throughout the execution: 
$\dom(\thdenv) = \dom(\prog)$.
Program transitions are simply defined in terms of the transitions of their constituent client commands.
This in turn yields the standard interleaving semantics for concurrent programs. 
That is, a client performs a reduction in an atomic step, without affecting other clients.

