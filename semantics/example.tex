\subsection{Examples of Operational Semantics} 
\label{sec:semantics.example}
\label{sec:semantics-example}
To conclude our discussion on the operational semantics, we show in detail one possible computation of a program \( \prog_{1} \) consisting of two transactions executing in parallel:
%The program $\prog_1$ that we consider is illustrated below: 
\[
    \prog_{1} \equiv 
    \begin{session}
        \begin{array}{@{}c || c@{}}
            \txid_{1} : 
            \begin{transaction}
                \pmutate{\vx}{1};\\
            	\pmutate{\vy}{1};
            \end{transaction} &
            \txid_{2} : 
            \begin{transaction}
                \pderef{\pvar{a}}{\vx};\\
            	\pderef{\pvar{b}}{\vy};\\
            	\pifs{\pvar{a}=1 \wedge \pvar{b}=0}\\
            		\quad \passign{\ret}{\sadface}
            	\pife
            \end{transaction}
        \end{array}
    \end{session}
 \]
The \( \pifs{\expr} \cmd_{1} \pifm \cmd_{2} \pife \) is encoded as \( (\passume{\expr} \pseq \cmd_{1}) \pchoice (\neg\passume{\expr} \pseq \cmd_{2} )\).
To recall, we often write \( \cmd_{1} \ppar \cmd_{2} \ppar \dots \ppar \cmd_{n}\) as a syntactic sugar for a program \( \prog \) with implicit unique thread identifiers \( \prog = \Set{\thid_{1} \mapsto \cmd_{1}, \thid_{2} \mapsto \cmd_{2}, \dots, \thid_{n} \mapsto \cmd_{n}  }\).
For better presentation, we annotated transactions with unique identifiers, yet they are allocated dynamically in the semantics.
We also treat the value assigned to the \( \ret \) variable as \emph{returned value}.
Assume the variables \( \vx \) and \( \vy \) refer to two key, and \( \va \) and \( \vb \) are local variables to threads.

The special symbol \(\sadface\), for example the returned value by the transaction $\txid_2$, is to emphasise some undesirable behaviour of a transaction.
In this case, the undesirable behaviour corresponds to the transaction to the right \( \txid_{2} \) observing only one of the updates from \( \txid_{1} \). 
Intuitively, this behaviour violates the constraints that transactions should be executed atomically (further discussed in \cref{......}), we want to show that if no restrictions are placed on the consistency model specification, it is possible for $\prog_1$ to reach a configuration where the second transaction $\txid_2$ returns $\sadface$. 
To illustrate this and also explain the semantics, we instantiate the operation semantics with the most permissive execution tests \( \csatP \), \ie the view after \( \vi' \) at least observes its own writes and no other constraint:
\[
\begin{rclarray}
    (\hh, \vi) \csatP \opset : \vi' & \defeq & \updV{\hh, \vi, \opset} \leq \vi'
\end{rclarray}
\]

\ac{The condition on $V'$ is not really needed.}

\begin{figure}[!t]

\hrule\vspace{5pt}
\begin{center}
\begin{tabular}{@{}c@{}@{}c@{}}
\begin{halfsubfig}
\begin{centertikz}
\begin{pgfonlayer}{foreground}

%Location x
\node(locx) {$\ke_{\vx} \mapsto$};

\matrix(initx) [version list] 
    at ([xshift=\tikzkvspace]locx.east) {
    {a} & $\txid_0$ \\ 
    {a} & $\emptyset$ \\
};  
\tikzvalue{initx-1-1}{initx-2-1}{locx-v0}{0};

%Location y
\path (locx.south) + (0,\tikzkeyspace) node (locy) {$\ke_\vy \mapsto$};
\matrix(inity) [version list] 
    at ([xshift=\tikzkvspace]locy.east) {
    {a} & $\txid_0$ \\
    {a} & $\emptyset$ \\
};
\tikzvalue{inity-1-1}{inity-2-1}{locy-v0}{0};

%blue view - I should  check whether I can use pgfkeys to just declare the list of locations, and then add the view automatically.
\draw[-, blue, very thick, rounded corners=10pt]
([xshift=-3pt, yshift=20pt]locx-v0.north east) node (tid1start) {} -- 
([xshift=-3pt, yshift=-5pt]locy-v0.south east);
 
\path (tid1start) node[anchor=south, rectangle, fill=blue!20, draw=blue, font=\small, inner sep=1pt] {$\thid_1$};

%red view
\draw[-, red, very thick, rounded corners = 10pt]
([xshift=-16pt, yshift=5pt]locx-v0.north east) node (tid2start) {}-- 
([xshift=-16pt, yshift=-5pt]locy-v0.south east) node {};
 
\path (tid2start) node[anchor=south, rectangle, fill=red!20, draw=red, font=\small, inner sep=1pt] {$\thid_2$};

%Stack for threads tid_1 and tid_2

%\draw[-, dashed] let 
   %\p1 = ([xshift=0pt]locy.west),
   %\p2 = ([yshift=-5pt]inity.south),
   %\p3 = ([xshift=10pt]inity.east) in
   %(\x1, \y2) -- (\x3, \y2);
   
%\matrix(stacks) [
   %matrix of nodes,
   %anchor=north, 
   %text=blue, 
   %font=\normalsize, 
   %row 1/.style = {text = blue}, 
   %row 2/.style = {text = red}, 
   %text width= 13mm ] 
   %at ([xshift=-10pt,yshift=-8pt]inity.south) {
   %$\thid_1:$ & $\retvar = 0$\\
   %$\thid_2:$ & $\retvar = 0$\\
   %};
\end{pgfonlayer}
\end{centertikz}
\caption{Initial state}
\label{fig:opsem-example-a}
\end{halfsubfig}
&
\begin{halfsubfig}
\begin{centertikz}

\begin{pgfonlayer}{foreground}
%Uncomment line below for help lines

%Location x
\node(locx) {$\ke_{\vx} \mapsto$};

\matrix(versionx) [version list] 
    at ([xshift=\tikzkvspace]locx.east) { 
    {a} & $\txid_{0}$ &{a} & $\txid_{1}$\\
    {a} & $\emptyset$ & {a} & $\emptyset$ \\
};
\tikzvalue{versionx-1-1}{versionx-2-1}{locx-v0}{0};
\tikzvalue{versionx-1-3}{versionx-2-3}{locx-v1}{1};

%Location y
\path (locx.south) + (0,\tikzkeyspace) node (locy) {$\ke_{\vy} \mapsto$};
\matrix(versiony) [version list]
   at ([xshift=\tikzkvspace]locy.east) {
 {a} & $\txid_0$ & {a} & $\txid_1$\\
  {a} & $\emptyset$ & {a} & $\emptyset$ \\
};
\tikzvalue{versiony-1-1}{versiony-2-1}{locy-v0}{0};
\tikzvalue{versiony-1-3}{versiony-2-3}{locy-v1}{1};

% \draw[-, red, very thick, rounded corners] ([xshift=-5pt, yshift=5pt]locx-v1.north east) |- 
%  ($([xshift=-5pt,yshift=-5pt]locx-v1.south east)!.5!([xshift=-5pt, yshift=5pt]locy-v0.north east)$) -| ([xshift=-5pt, yshift=5pt]locy-v0.south east);

%blue view - I should  check whether I can use pgfkeys to just declare the list of locations, and then add the view automatically.
\draw[-, blue, very thick, rounded corners=10pt]
([xshift=-3pt, yshift=20pt]locx-v1.north east) node (tid1start) {} -- 
([xshift=-3pt, yshift=-5pt]locy-v1.south east);
 
\path (tid1start) node[anchor=south, rectangle, fill=blue!20, draw=blue, font=\small, inner sep=1pt] {$\thid_1$};

%red view
\draw[-, red, very thick, rounded corners = 10pt]
([xshift=-16pt, yshift=5pt]locx-v0.north east) node (tid2start) {}-- 
([xshift=-16pt, yshift=-5pt]locy-v0.south east) node {};
 
\path (tid2start) node[anchor=south, rectangle, fill=red!20, draw=red, font=\small, inner sep=1pt] {$\thid_2$};

%Stack for threads tid_1 and tid_2

%\draw[-, dashed] let 
   %\p1 = ([xshift=0pt]locy.west),
   %\p2 = ([yshift=-5pt]locycells.south),
   %\p3 = ([xshift=10pt]locycells.east) in
   %(\x1, \y2) -- (\x3, \y2);
   
%\matrix(stacks) [
   %matrix of nodes,
   %anchor=north, 
   %text=blue, 
   %font=\normalsize, 
   %row 1/.style = {text = blue}, 
   %row 2/.style = {text = red}, 
   %text width= 13mm ] 
   %at ([xshift=-10pt,yshift=-8pt]locycells.south) {
   %$\thid_1:$ & $\retvar = 0$\\
   %$\thid_2:$ & $\retvar = 0$\\
   %};
\end{pgfonlayer}
\end{centertikz}
\caption{After transaction \( \txid_{1}\)} 
\label{fig:opsem-example-b}
\end{halfsubfig}
\\
\begin{halfsubfig}
\begin{centertikz}

\begin{pgfonlayer}{foreground}
%Uncomment line below for help lines
%\draw[help lines] grid(5,4);

%Location x
\node(locx) {$\ke_{\vx} \mapsto$};

\matrix(versionx) [version list] 
    at ([xshift=\tikzkvspace]locx.east) { 
    {a} & $\txid_{0}$ &{a} & $\txid_{1}$\\
    {a} & $\emptyset$ & {a} & $\emptyset$ \\
};
\tikzvalue{versionx-1-1}{versionx-2-1}{locx-v0}{0};
\tikzvalue{versionx-1-3}{versionx-2-3}{locx-v1}{1};

%Location y
\path (locx.south) + (0,\tikzkeyspace) node (locy) {$\ke_{\vy} \mapsto$};
\matrix(versiony) [version list]
   at ([xshift=\tikzkvspace]locy.east) {
 {a} & $\txid_0$ & {a} & $\txid_1$\\
  {a} & $\emptyset$ & {a} & $\emptyset$ \\
};
\tikzvalue{versiony-1-1}{versiony-2-1}{locy-v0}{0};
\tikzvalue{versiony-1-3}{versiony-2-3}{locy-v1}{1};

%blue view - I should  check whether I can use pgfkeys to just declare the list of locations, and then add the view automatically.
\draw[-, blue, very thick, rounded corners=10pt]
 ([xshift=-3pt, yshift=20pt]locx-v1.north east) node (tid1start) {} -- 
% ([xshift=-2pt, yshift=-5pt]locx-v0.south east) --
% ([xshift=-2pt, yshift=5pt]locy-v0.north east) -- 
 ([xshift=-3pt, yshift=-5pt]locy-v1.south east);
 
 \path (tid1start) node[anchor=south, rectangle, fill=blue!20, draw=blue, font=\small, inner sep=1pt] {$\thid_1$};

%red view
\draw[-, red, very thick, rounded corners = 10pt]
 ([xshift=-16pt, yshift=5pt]locx-v1.north east) node (tid2start) {}-- 
 ([xshift=-16pt, yshift=-5pt]locx-v1.south east) --
 ([xshift=-16pt, yshift=5pt]locy-v0.north east) -- 
 ([xshift=-16pt, yshift=-5pt]locy-v0.south east) node {};
 
\path (tid2start) node[anchor=south, rectangle, fill=red!20, draw=red, font=\small, inner sep=1pt] {$\thid_2$};

%Stack for threads tid_1 and tid_2

%\draw[-, dashed] let 
   %\p1 = ([xshift=0pt]locy.west),
   %\p2 = ([yshift=-5pt]locycells.south),
   %\p3 = ([xshift=10pt]locycells.east) in
   %(\x1, \y2) -- (\x3, \y2);
   
%\matrix(stacks) [
   %matrix of nodes,
   %anchor=north, 
   %text=blue, 
   %font=\normalsize, 
   %row 1/.style = {text = blue}, 
   %row 2/.style = {text = red}, 
   %text width= 13mm ] 
   %at ([xshift=-10pt,yshift=-8pt]locycells.south) {
   %$\thid_1:$ & $\retvar = 0$\\
   %$\thid_2:$ & $\retvar = 0$\\
   %};
\end{pgfonlayer}
\end{centertikz}
\caption{When \( \txid_{2}\) starts}
\label{fig:opsem-example-c}
\end{halfsubfig}
&
\begin{halfsubfig}
\begin{centertikz}

\begin{pgfonlayer}{foreground}
%Uncomment line below for help lines
%\draw[help lines] grid(5,4);

%Location x
\node(locx) {$\ke_{\vx} \mapsto$};

\matrix(versionx) [version list] 
    at ([xshift=\tikzkvspace]locx.east) { 
    {a} & $\txid_{0}$ &{a} & $\txid_{1}$\\
    {a} & $\emptyset$ & {a} & $\Set{\txid_{2}}$ \\
};
\tikzvalue{versionx-1-1}{versionx-2-1}{locx-v0}{0};
\tikzvalue{versionx-1-3}{versionx-2-3}{locx-v1}{1};

%Location y
\path (locx.south) + (0,\tikzkeyspace) node (locy) {$\ke_{\vy} \mapsto$};
\matrix(versiony) [version list]
   at ([xshift=\tikzkvspace]locy.east) {
 {a} & $\txid_0$ & {a} & $\txid_1$\\
  {a} & $\Set{\txid_{2}}$ & {a} & $\emptyset$ \\
};
\tikzvalue{versiony-1-1}{versiony-2-1}{locy-v0}{0};
\tikzvalue{versiony-1-3}{versiony-2-3}{locy-v1}{1};

% \draw[-, red, very thick, rounded corners] ([xshift=-5pt, yshift=5pt]locx-v1.north east) |- 
%  ($([xshift=-5pt,yshift=-5pt]locx-v1.south east)!.5!([xshift=-5pt, yshift=5pt]locy-v0.north east)$) -| ([xshift=-5pt, yshift=5pt]locy-v0.south east);

%blue view - I should  check whether I can use pgfkeys to just declare the list of locations, and then add the view automatically.
\draw[-, blue, very thick, rounded corners=10pt]
 ([xshift=-3pt, yshift=20pt]locx-v1.north east) node (tid1start) {} -- 
% ([xshift=-2pt, yshift=-5pt]locx-v0.south east) --
% ([xshift=-2pt, yshift=5pt]locy-v0.north east) -- 
 ([xshift=-3pt, yshift=-5pt]locy-v1.south east);
 
 \path (tid1start) node[anchor=south, rectangle, fill=blue!20, draw=blue, font=\small, inner sep=1pt] {$\thid_1$};

%red view
\draw[-, red, very thick, rounded corners = 10pt]
 ([xshift=-16pt, yshift=5pt]locx-v1.north east) node (tid2start) {}-- 
 ([xshift=-16pt, yshift=-5pt]locx-v1.south east) --
 ([xshift=-16pt, yshift=5pt]locy-v0.north east) -- 
 ([xshift=-16pt, yshift=-5pt]locy-v0.south east) node {};
 
\path (tid2start) node[anchor=south, rectangle, fill=red!20, draw=red, font=\small, inner sep=1pt] {$\thid_2$};

%Stack for threads tid_1 and tid_2

%\draw[-, dashed] let 
   %\p1 = ([xshift=0pt]locy.west),
   %\p2 = ([yshift=-5pt]locycells.south),
   %\p3 = ([xshift=10pt]locycells.east) in
   %(\x1, \y2) -- (\x3, \y2);
   
%\matrix(stacks) [
   %matrix of nodes,
   %anchor=north, 
   %text=blue, 
   %font=\normalsize, 
   %row 1/.style = {text = blue}, 
   %row 2/.style = {text = red}, 
   %text width= 15mm ] 
   %at ([xshift=-10pt,yshift=-8pt]locycells.south) {
   %$\thid_1:$ & $\retvar = 0$\\
   %$\thid_2:$ & $\retvar = {\Large \frownie}$\\
   %};
   \end{pgfonlayer}
\end{centertikz}
   \caption{Transaction \( \txid_{2}\) returns \( \sadface \)}
    \label{fig:opsem-example-d}
\end{halfsubfig}
\\
\end{tabular}
\end{center}
\hrule\vspace{5pt}
\caption{Graphical Representation of configurations 
obtained through the execution of $\prog_1$.}
\label{fig:opsem.example}
\label{fig:opsem-example}
\end{figure}

Before any computation, the initial configuration for $\prog_1$ is the one in which there are two keys \( \ke_{\vx}\) and \( \ke_{\vy} \) where each key is associated with an initial version with value zero written by an initialisation transaction $\txid_0$, \( \hh_{0} = \Set{\ke_{\vx} \mapsto \List{(0, \txid, \emptyset)}, \ke_{\vy} \mapsto \List{(0, \txid, \emptyset)}} \).
The initial view of each thread points to the initial version of each key, \( \vi^{1}_{0} = \vi^{2}_{0} = \Set{\ke_{\vx} \mapsto 1, \ke_{\vy} \mapsto 1}\).
The two threads have the same initial stack containing two variables \( \vx \) and \( \vy \) referring to the only keys in the key-value store respectively, \ie \( \stk^{1}_{0} = \stk^{2}_{0} = \Set{\vx \mapsto \ke_{\vx}, \vy \mapsto \ke_{\vy}}\).
Therefore the initial configuration is \( (\hh_{0}, \thdenv_{0}, \prog_{1}) \) where \( \thdenv_{0} = \Set{\thid_{1} \mapsto (\stk_{0}^{1}, \vi_{0}^{1}), \thid_{2} \mapsto (\stk_{0}^{2}, \vi_{0}^{2})}\).
\cref{fig:opsem-example-a} gives a graphical representation of the initial configuration.
 
%\ac{
%Before showing the computation of $\prog_1$ that leads to the transaction of 
%$\thid_2$ to return ${\Large \frownie{}}$, we need to introduce some 
%definitions and  notation.
%The initial configuration in which $\prog_1$ is executed is the one in which 
%each location has an initial version written by some initialisation transaction $\tsid_0$, 
%the view of each thread points to the initial version of each location, and all the  
%thread stacks are initialised to $0$. Assuming that the only key in the database 
%are $[\loc_x], [\loc_y]$, and the only variable in the thread stack is $\retvar$, 
%the initial configuration is then given by $\mathcal{C}_0 = (\hh_{0}, \mathsf{Env}_0)$, 
%where $\hh_{0}([\loc_{x}) = \hh_{0}([\loc_y]) =  (0, \tsid_0, \emptyset)$,  
%$\mathsf{Env}_0(\thid_1) = \mathsf{Env}_0(\thid_2) = ([\retvar \mapsto 0], V_0)$, 
%and $V_0([\loc_{x}]) = V_0([\loc_{y}]) = 0$. Here and in the following, we prefer 
%to adopt a more graphical notation for configurations. For example, the initial configuration 
%defined is represented graphically in Figure \ref{fig:opsem.example}(a). 

%In the picture above, the part above the dashed line represents the history heap and 
%view of each thread, while the part below the dashed line contains the thread-stack 
%of each thread. History heaps are represented as mappings to locations to lists of cells, 
%each of which represents a version and has three component: the value of the version 
%to the left, the identifier of the transaction that wrote it to the top right, and the 
%set of transactions that read the version to the bottom right. Vertical lines labelled 
%with thread identifiers are used to represent the views. The position of the view of 
%thread $\thid_1$ relatively to the location $[\loc_x]$ is determined by the version 
%at which the vertical line labelled $\thid_1$ crosses the list of versions for $[\loc_x]$,
 %and similarly for $[\loc_y]$.
%}

We are now ready to show how to derive a computation of $\prog_1$ that violates \emph{atomic visibility} and it will be explained formally in \cref{sec:example-commit-test}.
In the specific computation, we choose to let the transaction $\thid_1$ commit first.
According to rule $\rl{PCommit}$, we need to perform the following steps: 
\sx{Not sure the bullet points work here? Just typesetting. }
\begin{itemize}
\item
Arbitrarily shift the view $\vi_{0}^{1}$ for thread $\thid_1$ to the right as long as it is within the bound of key-value store and obtain a view \(\vi'' \) such that \( \vi'' \geq \vi_{0}^{1} \). 
Because $\hh_0$ contains only one version per key, here the only possibility is that $\vi'' = \vi_{0}^{1}$.
\item
Determine the initial snapshot $\sn = \clpsHH{\hh_0, \vi''}$ for the transaction $\thid_1$.
In this case, we have that $\h = \Set{\loc_x \mapsto 0, \loc_y \mapsto 0}$.
\item 
Given the initial snapshot \( \h \), initially empty fingerprint \( \unitO \) and the stack \( \stk_{0}^{1}\), by the operational semantics for transaction (\cref{fig:transaction_semantics}), after executing the transactional codes \( \pmutate{\vx}{1}; \pmutate{\vy}{1} \), the finial fingerprints include two write operations as $\opset = \Set{(\otW, \ke_{\vx}, 1), (\otW, \ke_{\vy}, 1)}$.
\ac{
\sx{Not sure those details are necessary.}
This amounts to execute the transaction in isolation from the external environment, using the rules in the operational semantics for 
 transactions. Because this execution must match the premiss of Rule $(C-Tx)$,  The code is run using $h$ as the initial heap, $\sigma_0$ as the initial 
 thread stack, $\tau_0 = \lambda_a.0$ as the initial transaction stack, and $\emptyset$ as the 
 initial fingerprint. We only need to apply 
 Rule $(Tx-prim)$ twice, in which case we obtain
 \begin{equation}
\label{eq:tx1}
\begin{array}{lcr}
& \sigma_0 \vdash \left\langle h_0, \_, \emptyset, \begin{array}{l}
\pmutate{\loc_x}{1};\\ \pmutate{\loc_y}{1} \end{array} \right\rangle 
&\rightarrow \\
\rightarrow & 
\left \langle h_0[ [\loc_x] \mapsto 1], \_, \big( \emptyset \oplus \text{WR}\; [\loc_x]: 1 \big), 
\pmutate{\loc_y}{1} \right\rangle &= \\
=&\left \langle h_0[[\loc_x] \mapsto 1], \_, \{\text{WR}\; [\loc_x]: 1\}, 
\pmutate{\loc_y}{1} \right\rangle 
&\rightarrow\\ 
\rightarrow & 
\left\langle h_0[[\loc_x] \mapsto_1, [\loc_y] \mapsto 1], \_, \big( \{\text{WR}\;[\loc_x]:1\} \oplus  (\text{WR}\;[\loc_y]: 1) \big), 
\stub \right\rangle & = \\
=& \left \langle \_, \_, \{\text{WR}\;[\loc_x]: 1, \text{WR}\;[\loc_y]:1 \}, \stub \right\rangle
\end{array}
\end{equation}
Therefore, we conclude $\mathcal{O} = \{\text{WR}\; [\loc_x] : 1, \text{WR}\;[\loc_y]:1\}$.
}
\item 
The transaction throws away the local snapshot and commits the fingerprints to the key-value store.
A fresh transaction identifier \( \txid_{1} \) is picked.
The new key-value store \( \hh_{1} \) is determined by the functions $\hh_{1} = \func{updHisHp}{\hh_0, \vi'',\txid_1, \opset}$ and the lower bound of the new view is given by the function \( \func{updView}{\hh_1, \vi'' ,\opset}\).
\item Last, the local view shift to the right so that it satisfies the execution tests, \( \vi' \geq \func{updView}{\hh_1, \vi'' ,\opset} \land (\hh_{0}, \vi'') \csatP \vi' : \opset \).
In this case, the permissive model does not constraint the view at all.
Therefore the overall final state is in \cref{fig:opsem-example-b}.
\end{itemize}

Next, thread $\thid_2$ executes its own transaction.
Note that the view for \( \thid_2\) still points to initial versions and the semantics allows the view get updated arbitrarily before executing the transaction.
Because the key-value store now has two versions for each key, there are exactly four possible views for the transaction \( \txid_{2} \).
In particular, assume it updates the view for \( \ke_{x} \) but not \( \ke_y \), \ie \( \Set{\ke_\vx \mapsto 1, \ke_\vx \mapsto 0} \) (\cref{fig:opsem-example-c}).
Given the view, the transaction \( \txid_{2} \) will assign \(\sadface\) to the \( \ret\) and this transaction is allowed to commit since the commit test does not stop this (\cref{fig:opsem-example-d}).

%\subsection{Example of Consistency models}
%\label{sec:example-commit-test}
%\ac{This Section is going to become heavy in pictures, which should be organised into figures.}
%In this Section we present different consistency models specifications. 
%For each of them, we give: 
%\begin{itemize}
%\item the intuition of the commit tests for different consistency models, and the formal definitions with respect to the \(\Como\) (\defref{def:consistency-models}).
%%describing the consistency guarantees that schedules of the database should have in plain English, 
%%\item a formal consistency model specification, in the style described in \S \ref{sec:semantics.programs},
%\item examples of litmus tests that, when executed, give rise to the anomalies that are forbidden from the consistency model, 
%\item an explanation of why the consistency model forbids the litmus tests to exhibit the anomaly that should be forbidden. 
%\end{itemize}
%Later, we will show how to compare our consistency models specifications with those already existing in the 
%literature.
%\ac{There is still a long-way to go before proving correspondence with dependency graph specifications, 
%but this should be mentioned here.}

\subsection{Read Atomic} 
\sx{revisit and take out of the updateview }
\sx{This section might take out as the atomic constraint appears before. }
\label{sec:read-atomic}
\label{sec:semantics.example}
\label{sec:semantics-example}
Read atomic (RA) \cite{ramp} is the weakest consistency model among those that enjoy \emph{atomic visibility} \cite{framework-concur}. 
It requires of a transaction to read an atomic snapshot of the database and never observe the partial effects of other transactions.
This is also known as the \emph{all-or-nothing} property: a transaction observes either none or all the updates performed by other transactions. 
To illustrate that, we show in detail one possible computation of a program \( \prog_{1} \) consisting of two transactions executing in parallel if there is no atomic view constraint:
\[
    \prog_{1} \equiv 
    \begin{session}
        \begin{array}{@{}c || c@{}}
            \txid_{1} : 
            \begin{transaction}
                \pmutate{\vx}{1};\\
            	\pmutate{\vy}{1};
            \end{transaction} &
            \txid_{2} : 
            \begin{transaction}
                \pderef{\pvar{a}}{\vx};\\
            	\pderef{\pvar{b}}{\vy};\\
            	\pifs{\pvar{a}=1 \wedge \pvar{b}=0}\\
            		\quad \passign{\ret}{\sadface}
            	\pife
            \end{transaction}
        \end{array}
    \end{session}
 \]
The \( \pifs{\expr} \cmd_{1} \pifm \cmd_{2} \pife \) is encoded as \( (\passume{\expr} \pseq \cmd_{1}) \pchoice (\neg\passume{\expr} \pseq \cmd_{2} )\).
To recall, we often write \( \cmd_{1} \ppar \cmd_{2} \ppar \dots \ppar \cmd_{n}\) as a syntactic sugar for a program \( \prog \) with implicit unique thread identifiers \( \prog = \Set{\thid_{1} \mapsto \cmd_{1}, \thid_{2} \mapsto \cmd_{2}, \dots, \thid_{n} \mapsto \cmd_{n}  }\).
For better presentation, we annotated transactions with unique identifiers, yet they are allocated dynamically in the semantics.
We also treat the value assigned to the \( \ret \) variable as \emph{returned value}.
Assume the variables \( \vx \) and \( \vy \) refer to two key, and \( \va \) and \( \vb \) are local variables to threads.

The special symbol \(\sadface\), for example the returned value by the transaction $\txid_2$, is to emphasise some undesirable behaviour of a transaction.
In this case, the undesirable behaviour corresponds to the transaction to the right \( \txid_{2} \) observing only one of the updates from \( \txid_{1} \). 
%Intuitively, this behaviour violates the constraints that transactions should be executed atomically (further discussed in \cref{......}), we want to show that if no restrictions are placed on the consistency model specification, it is possible for $\prog_1$ to reach a configuration where the second transaction $\txid_2$ returns $\sadface$. 
To illustrate this, we instantiate the operation semantics with the most permissive execution tests \( \csatP \): \( (\hh, \vi) \csatP \opset : \vi'  \defeq  \true \) and we assume there is no \( \predn{atomic} \) constraints for views.

\begin{figure}[!t]
\hrule
\begin{center}
\begin{tabular}{@{}c@{}@{}c@{}}
\begin{halfsubfig}
\begin{centertikz}
\begin{pgfonlayer}{foreground}

%Location x
\node(locx) {$\ke_{\vx} \mapsto$};

\matrix(initx) [version list] 
    at ([xshift=\tikzkvspace]locx.east) {
    {a} & $\txid_0$ \\ 
    {a} & $\emptyset$ \\
};  
\tikzvalue{initx-1-1}{initx-2-1}{locx-v0}{0};

%Location y
\path (locx.south) + (0,\tikzkeyspace) node (locy) {$\ke_\vy \mapsto$};
\matrix(inity) [version list] 
    at ([xshift=\tikzkvspace]locy.east) {
    {a} & $\txid_0$ \\
    {a} & $\emptyset$ \\
};
\tikzvalue{inity-1-1}{inity-2-1}{locy-v0}{0};

%blue view - I should  check whether I can use pgfkeys to just declare the list of locations, and then add the view automatically.
\draw[-, blue, very thick, rounded corners=10pt]
([xshift=-3pt, yshift=20pt]locx-v0.north east) node (tid1start) {} -- 
([xshift=-3pt, yshift=-5pt]locy-v0.south east);
 
\path (tid1start) node[anchor=south, rectangle, fill=blue!20, draw=blue, font=\small, inner sep=1pt] {$\thid_1$};

%red view
\draw[-, red, very thick, rounded corners = 10pt]
([xshift=-16pt, yshift=5pt]locx-v0.north east) node (tid2start) {}-- 
([xshift=-16pt, yshift=-5pt]locy-v0.south east) node {};
 
\path (tid2start) node[anchor=south, rectangle, fill=red!20, draw=red, font=\small, inner sep=1pt] {$\thid_2$};

\end{pgfonlayer}
\end{centertikz}
\caption{Initial state}
\label{fig:opsem-example-a}
\end{halfsubfig}
&
\begin{halfsubfig}
\begin{centertikz}

\begin{pgfonlayer}{foreground}
%Uncomment line below for help lines

%Location x
\node(locx) {$\ke_{\vx} \mapsto$};

\matrix(versionx) [version list] 
    at ([xshift=\tikzkvspace]locx.east) { 
    {a} & $\txid_{0}$ &{a} & $\txid_{1}$\\
    {a} & $\emptyset$ & {a} & $\emptyset$ \\
};
\tikzvalue{versionx-1-1}{versionx-2-1}{locx-v0}{0};
\tikzvalue{versionx-1-3}{versionx-2-3}{locx-v1}{1};

%Location y
\path (locx.south) + (0,\tikzkeyspace) node (locy) {$\ke_{\vy} \mapsto$};
\matrix(versiony) [version list]
   at ([xshift=\tikzkvspace]locy.east) {
 {a} & $\txid_0$ & {a} & $\txid_1$\\
  {a} & $\emptyset$ & {a} & $\emptyset$ \\
};
\tikzvalue{versiony-1-1}{versiony-2-1}{locy-v0}{0};
\tikzvalue{versiony-1-3}{versiony-2-3}{locy-v1}{1};

%blue view - I should  check whether I can use pgfkeys to just declare the list of locations, and then add the view automatically.
\draw[-, blue, very thick, rounded corners=10pt]
([xshift=-3pt, yshift=20pt]locx-v1.north east) node (tid1start) {} -- 
([xshift=-3pt, yshift=-5pt]locy-v1.south east);
 
\path (tid1start) node[anchor=south, rectangle, fill=blue!20, draw=blue, font=\small, inner sep=1pt] {$\thid_1$};

%red view
\draw[-, red, very thick, rounded corners = 10pt]
([xshift=-16pt, yshift=5pt]locx-v0.north east) node (tid2start) {}-- 
([xshift=-16pt, yshift=-5pt]locy-v0.south east) node {};
 
\path (tid2start) node[anchor=south, rectangle, fill=red!20, draw=red, font=\small, inner sep=1pt] {$\thid_2$};

\end{pgfonlayer}
\end{centertikz}
\caption{After transaction \( \txid_{1}\)} 
\label{fig:opsem-example-b}
\end{halfsubfig}
\\
\begin{halfsubfig}
\begin{centertikz}

\begin{pgfonlayer}{foreground}

%Location x
\node(locx) {$\ke_{\vx} \mapsto$};

\matrix(versionx) [version list] 
    at ([xshift=\tikzkvspace]locx.east) { 
    {a} & $\txid_{0}$ &{a} & $\txid_{1}$\\
    {a} & $\emptyset$ & {a} & $\emptyset$ \\
};
\tikzvalue{versionx-1-1}{versionx-2-1}{locx-v0}{0};
\tikzvalue{versionx-1-3}{versionx-2-3}{locx-v1}{1};

%Location y
\path (locx.south) + (0,\tikzkeyspace) node (locy) {$\ke_{\vy} \mapsto$};
\matrix(versiony) [version list]
    at ([xshift=\tikzkvspace]locy.east) {
    {a} & $\txid_0$ & {a} & $\txid_1$\\
    {a} & $\emptyset$ & {a} & $\emptyset$ \\
};
\tikzvalue{versiony-1-1}{versiony-2-1}{locy-v0}{0};
\tikzvalue{versiony-1-3}{versiony-2-3}{locy-v1}{1};

%blue view - I should  check whether I can use pgfkeys to just declare the list of locations, and then add the view automatically.
\draw[-, blue, very thick, rounded corners=10pt]
([xshift=-3pt, yshift=20pt]locx-v1.north east) node (tid1start) {} -- 
([xshift=-3pt, yshift=-5pt]locy-v1.south east);

\path (tid1start) node[anchor=south, rectangle, fill=blue!20, draw=blue, font=\small, inner sep=1pt] {$\thid_1$};

%red view
\draw[-, red, very thick, rounded corners = 10pt]
([xshift=-16pt, yshift=5pt]locx-v1.north east) node (tid2start) {}-- 
([xshift=-16pt, yshift=-5pt]locx-v1.south east) --
([xshift=-16pt, yshift=5pt]locy-v0.north east) -- 
([xshift=-16pt, yshift=-5pt]locy-v0.south east) node {};
 
\path (tid2start) node[anchor=south, rectangle, fill=red!20, draw=red, font=\small, inner sep=1pt] {$\thid_2$};

\end{pgfonlayer}
\end{centertikz}
\caption{When \( \txid_{2}\) starts}
\label{fig:opsem-example-c}
\end{halfsubfig}
&
\begin{halfsubfig}
\begin{centertikz}

\begin{pgfonlayer}{foreground}
%Uncomment line below for help lines
%\draw[help lines] grid(5,4);

%Location x
\node(locx) {$\ke_{\vx} \mapsto$};

\matrix(versionx) [version list] 
    at ([xshift=\tikzkvspace]locx.east) { 
    {a} & $\txid_{0}$ &{a} & $\txid_{1}$\\
    {a} & $\emptyset$ & {a} & $\Set{\txid_{2}}$ \\
};
\tikzvalue{versionx-1-1}{versionx-2-1}{locx-v0}{0};
\tikzvalue{versionx-1-3}{versionx-2-3}{locx-v1}{1};

%Location y
\path (locx.south) + (0,\tikzkeyspace) node (locy) {$\ke_{\vy} \mapsto$};
\matrix(versiony) [version list]
   at ([xshift=\tikzkvspace]locy.east) {
 {a} & $\txid_0$ & {a} & $\txid_1$\\
  {a} & $\Set{\txid_{2}}$ & {a} & $\emptyset$ \\
};
\tikzvalue{versiony-1-1}{versiony-2-1}{locy-v0}{0};
\tikzvalue{versiony-1-3}{versiony-2-3}{locy-v1}{1};


%blue view - I should  check whether I can use pgfkeys to just declare the list of locations, and then add the view automatically.
\draw[-, blue, very thick, rounded corners=10pt]
([xshift=-3pt, yshift=20pt]locx-v1.north east) node (tid1start) {} -- 
([xshift=-3pt, yshift=-5pt]locy-v1.south east);

\path (tid1start) node[anchor=south, rectangle, fill=blue!20, draw=blue, font=\small, inner sep=1pt] {$\thid_1$};

%red view
\draw[-, red, very thick, rounded corners = 10pt]
([xshift=-16pt, yshift=5pt]locx-v1.north east) node (tid2start) {}-- 
([xshift=-16pt, yshift=-5pt]locx-v1.south east) --
([xshift=-16pt, yshift=5pt]locy-v0.north east) -- 
([xshift=-16pt, yshift=-5pt]locy-v0.south east) node {};

\path (tid2start) node[anchor=south, rectangle, fill=red!20, draw=red, font=\small, inner sep=1pt] {$\thid_2$};

\end{pgfonlayer}
\end{centertikz}
\caption{Transaction \( \txid_{2}\) returns \( \sadface \)}
\label{fig:opsem-example-d}
\end{halfsubfig}
\\
\end{tabular}
\end{center}
\hrule
\caption{Graphical Representation of configurations 
obtained through the execution of $\prog_1$.}
\label{fig:opsem.example}
\label{fig:opsem-example}
\end{figure}

Before any computation, the initial configuration for $\prog_1$ is the one in which there are two keys \( \ke_{\vx}\) and \( \ke_{\vy} \) where each key is associated with an initial version with value zero written by an initialisation transaction $\txid_0$, \( \hh_{0} = \Set{\ke_{\vx} \mapsto \List{(0, \txid, \emptyset)}, \ke_{\vy} \mapsto \List{(0, \txid, \emptyset)}} \).
The initial view of each thread points to the initial version of each key, \( \vi^{1}_{0} = \vi^{2}_{0} = \Set{\ke_{\vx} \mapsto 1, \ke_{\vy} \mapsto 1}\).
The two threads have the same initial stack containing two variables \( \vx \) and \( \vy \) referring to the only keys in the key-value store respectively, \ie \( \stk^{1}_{0} = \stk^{2}_{0} = \Set{\vx \mapsto \ke_{\vx}, \vy \mapsto \ke_{\vy}}\).
Therefore the initial configuration is \( (\hh_{0}, \thdenv_{0}, \prog_{1}) \) where \( \thdenv_{0} = \Set{\thid_{1} \mapsto (\stk_{0}^{1}, \vi_{0}^{1}), \thid_{2} \mapsto (\stk_{0}^{2}, \vi_{0}^{2})}\).
\cref{fig:opsem-example-a} gives a graphical representation of the initial configuration.
 
%\ac{
%Before showing the computation of $\prog_1$ that leads to the transaction of 
%$\thid_2$ to return ${\Large \frownie{}}$, we need to introduce some 
%definitions and  notation.
%The initial configuration in which $\prog_1$ is executed is the one in which 
%each location has an initial version written by some initialisation transaction $\tsid_0$, 
%the view of each thread points to the initial version of each location, and all the  
%thread stacks are initialised to $0$. Assuming that the only key in the database 
%are $[\loc_x], [\loc_y]$, and the only variable in the thread stack is $\retvar$, 
%the initial configuration is then given by $\mathcal{C}_0 = (\hh_{0}, \mathsf{Env}_0)$, 
%where $\hh_{0}([\loc_{x}) = \hh_{0}([\loc_y]) =  (0, \tsid_0, \emptyset)$,  
%$\mathsf{Env}_0(\thid_1) = \mathsf{Env}_0(\thid_2) = ([\retvar \mapsto 0], V_0)$, 
%and $V_0([\loc_{x}]) = V_0([\loc_{y}]) = 0$. Here and in the following, we prefer 
%to adopt a more graphical notation for configurations. For example, the initial configuration 
%defined is represented graphically in Figure \ref{fig:opsem.example}(a). 

%In the picture above, the part above the dashed line represents the history heap and 
%view of each thread, while the part below the dashed line contains the thread-stack 
%of each thread. History heaps are represented as mappings to locations to lists of cells, 
%each of which represents a version and has three component: the value of the version 
%to the left, the identifier of the transaction that wrote it to the top right, and the 
%set of transactions that read the version to the bottom right. Vertical lines labelled 
%with thread identifiers are used to represent the views. The position of the view of 
%thread $\thid_1$ relatively to the location $[\loc_x]$ is determined by the version 
%at which the vertical line labelled $\thid_1$ crosses the list of versions for $[\loc_x]$,
 %and similarly for $[\loc_y]$.
%}

We are now ready to show how to derive a computation of $\prog_1$ that violates \emph{atomic visibility} and it will be explained formally in \cref{sec:example-commit-test}.
In the specific computation, we choose to let the transaction $\thid_1$ commit first.
According to rule $\rl{PCommit}$, we need to perform the following steps: 
\sx{Not sure the bullet points work here? Just typesetting. }
\begin{itemize}
\item
Arbitrarily shift the view $\vi_{0}^{1}$ for thread $\thid_1$ to the right as long as it is within the bound of key-value store and obtain a view \(\vi'' \) such that \( \vi'' \geq \vi_{0}^{1} \). 
Because $\hh_0$ contains only one version per key, here the only possibility is that $\vi'' = \vi_{0}^{1}$.
\item
Determine the initial snapshot $\sn = \clpsHH{\hh_0, \vi''}$ for the transaction $\thid_1$.
In this case, we have that $\h = \Set{\loc_x \mapsto 0, \loc_y \mapsto 0}$.
\item 
Given the initial snapshot \( \h \), initially empty fingerprint \( \unitO \) and the stack \( \stk_{0}^{1}\), by the operational semantics for transaction (\cref{fig:transaction_semantics}), after executing the transactional codes \( \pmutate{\vx}{1}; \pmutate{\vy}{1} \), the finial fingerprints include two write operations as $\opset = \Set{(\otW, \ke_{\vx}, 1), (\otW, \ke_{\vy}, 1)}$.
%\ac{
%\sx{Not sure those details are necessary.}
%This amounts to execute the transaction in isolation from the external environment, using the rules in the operational semantics for 
 %transactions. Because this execution must match the premiss of Rule $(C-Tx)$,  The code is run using $h$ as the initial heap, $\sigma_0$ as the initial 
 %thread stack, $\tau_0 = \lambda_a.0$ as the initial transaction stack, and $\emptyset$ as the 
 %initial fingerprint. We only need to apply 
 %Rule $(Tx-prim)$ twice, in which case we obtain
 %\begin{equation}
%\label{eq:tx1}
%\begin{array}{lcr}
%& \sigma_0 \vdash \left\langle h_0, \_, \emptyset, \begin{array}{l}
%\pmutate{\loc_x}{1};\\ \pmutate{\loc_y}{1} \end{array} \right\rangle 
%&\rightarrow \\
%\rightarrow & 
%\left \langle h_0[ [\loc_x] \mapsto 1], \_, \big( \emptyset \oplus \text{WR}\; [\loc_x]: 1 \big), 
%\pmutate{\loc_y}{1} \right\rangle &= \\
%=&\left \langle h_0[[\loc_x] \mapsto 1], \_, \{\text{WR}\; [\loc_x]: 1\}, 
%\pmutate{\loc_y}{1} \right\rangle 
%&\rightarrow\\ 
%\rightarrow & 
%\left\langle h_0[[\loc_x] \mapsto_1, [\loc_y] \mapsto 1], \_, \big( \{\text{WR}\;[\loc_x]:1\} \oplus  (\text{WR}\;[\loc_y]: 1) \big), 
%\stub \right\rangle & = \\
%=& \left \langle \_, \_, \{\text{WR}\;[\loc_x]: 1, \text{WR}\;[\loc_y]:1 \}, \stub \right\rangle
%\end{array}
%\end{equation}
%Therefore, we conclude $\mathcal{O} = \{\text{WR}\; [\loc_x] : 1, \text{WR}\;[\loc_y]:1\}$.
%}
\item 
The transaction throws away the local snapshot and commits the fingerprints to the key-value store.
A fresh transaction identifier \( \txid_{1} \) is picked.
The new key-value store \( \hh_{1} \) is determined by the functions $\hh_{1} = \func{updHisHp}{\hh_0, \vi'',\txid_1, \opset}$ and the lower bound of the new view is given by the function \( \func{updView}{\hh_1, \vi'' ,\opset}\).
\item Last, the local view shift to the right so that it satisfies the execution tests, \( \vi' \geq \func{updView}{\hh_1, \vi'' ,\opset} \land (\hh_{0}, \vi'') \csatP \vi' : \opset \).
In this case, the permissive model does not constraint the view at all.
Therefore the overall final state is in \cref{fig:opsem-example-b}.
\end{itemize}

Next, thread $\thid_2$ executes its own transaction.
Note that the view for \( \thid_2\) still points to initial versions and the semantics allows the view get updated arbitrarily before executing the transaction.
Because the key-value store now has two versions for each key, there are exactly four possible views for the transaction \( \txid_{2} \).
In particular, assume it updates the view for \( \ke_{x} \) but not \( \ke_y \), \ie \( \Set{\ke_\vx \mapsto 1, \ke_\vx \mapsto 0} \) (\cref{fig:opsem-example-c}).
Given the view, the transaction \( \txid_{2} \) will assign \(\sadface\) to the \( \ret\) and this transaction is allowed to commit since the commit test does not stop this (\cref{fig:opsem-example-d}).

%\subsection{Example of Consistency models}
%\label{sec:example-commit-test}
%\ac{This Section is going to become heavy in pictures, which should be organised into figures.}
%In this Section we present different consistency models specifications. 
%For each of them, we give: 
%\begin{itemize}
%\item the intuition of the commit tests for different consistency models, and the formal definitions with respect to the \(\Como\) (\defref{def:consistency-models}).
%%describing the consistency guarantees that schedules of the database should have in plain English, 
%%\item a formal consistency model specification, in the style described in \S \ref{sec:semantics.programs},
%\item examples of litmus tests that, when executed, give rise to the anomalies that are forbidden from the consistency model, 
%\item an explanation of why the consistency model forbids the litmus tests to exhibit the anomaly that should be forbidden. 
%\end{itemize}
%Later, we will show how to compare our consistency models specifications with those already existing in the 
%literature.
%\ac{There is still a long-way to go before proving correspondence with dependency graph specifications, 
%but this should be mentioned here.}







Intuitively, in such a program, it violate the atomic visibility because it is allowed to execute the second transaction \( \trans_2\) when the client $\thid_2$ only observes partial effect from transaction \( \txid_1 \).
%\ac{, which is obtained by removing all the information about $\thid_1$ (view and stack) in Figure \ref{fig:opsem.example}(c).
%\[
%\prog_1 \equiv 
    %\begin{array}{c} 
    %\begin{transaction} 
        %\pderef{\pvar{a}}{\vx};\\
        %\pderef{\pvar{b}}{\vy};\\
        %\pifs{\pvar{a}=1 \wedge \pvar{b}=0}\\
        %\quad \passign{\retvar}{\Large \frownie{}} \\
        %\pife
    %\end{transaction}
%\end{array}
%\]
%}

%\ac{
%To avoid transactions to only observe the partial effects of other transactions, we 
%must ensure that transactional code cannot be executed by a client whose 
%views is up-to-date with respect to some transaction $\tsid$ for some location $[\loc_x]$, 
%but not for some other location $[\loc_y]$. This leads to the following definition.
%}
To avoid a transaction to observe the partial effects of other transactions, it needs to ensure that transactional code cannot be executed by a client whose views is partially up-to-date with respect to some transactions.
This leads to the following definition.
\begin{definition}[Read atomic]
\label{def:readatomic}
%Let $\hh$ be a history heap,$V$ be a view, $[\loc_x]$ be 
%a location and $\nu$ be a version. We say that $V$ $[\loc_x]$-\emph{sees} version 
%$\nu$ if there exists an index $i \leq V([\loc_x])$ such that $V(i) = \nu$. 
%We say that $V$ $[\loc_x]$-\emph{sees} transaction $\tsid$ if 
%$V$ $[\loc_x]$-sees a version $\nu = (\_, \tsid, \_)$. 
Given a view $\vi \in \Views$, a history heap $\hh \in \HisHeaps$, and a transaction identifier $\txid \in \TxID$, the view \emph{sees} the transaction in the history heap, written $\pred{visible}{\txid, \vi, \hh}$, if the view sees all the writes from the transaction,
%We say that $V$ \emph{sees} transaction $\tsid$ in $\hh$, written 
%$\mathsf{Visible}(\tsid, V, \hh)$, iff 
\sx{\( \exsts{i} \) might be enough}
\[
\begin{rclarray}
\pred{visible}{\txid, \vi, \hh} & \eqdef & \fora{\ke, i} \hh(\ke)(i) = (\stub, \txid, \stub) \implies i \leq \vi(\ke).
\end{rclarray}
\]
%\ac{In English: the view is up-to-date with respect to all the updates 
%performed by transaction $\tsid$.

%In English: if the view $V$ is up-to-date with some of the updates performed 
%by $\tsid$, then it must be up-to-date with all the updates performed by $\tsid$. 
%This is the all-or-nothing property.

%In English: Before executing a transaction, either you observe all or none the 
%updates of all other transactions. We may strengthen the consistency model and 
%require that the same property must be satisfied at the end as well, though 
%this is not strictly necessary. In this case the check becomes: 
%\[
%\mathsf{atomic}(\hh, V) \wedge \mathsf{atomic}(\hh, V') \wedge \mathsf{UpdateView}(\hh, V, \mathcal{O}) 
%\sqsubseteq V' \implies (\hh, V) \triangleright_{\mathsf{RA}} \mathcal{O}: V'.
%\]
%}

\sx{consistent has been used in many place, might mislead?}
Then given a history \( \hh \), the view $\vi$ is \emph{consistent} with respect to \emph{atomic visibility}, written $\pred{atomic}{\vi, \hh}$, if the view $\vi$ is up-to-date with some of the updates performed by transaction $\txid$, then it should be up-to-date with all the updates performed by $\txid$,
\[
\begin{rclarray}
\pred{atomic}{\vi ,\hh} & \eqdef & \fora{\txid } \exsts{\ke, i} i \leq \vi(\ke) \land \hh(\ke)(i) = (\stub, \txid, \stub) \implies \pred{visible}{\txid, \vi, \hh}
\end{rclarray}
\]
The commit test for \emph{read atomic} $\csatRA$ is defined as,
\[
\begin{rclarray}
(\hh, \vi) \csatRA \stub: \stub & \defeq & \pred{atomic}{\hh, \vi} 
\end{rclarray}
\]
%written $\mathsf{up-to-date}(\hh, V, \tsid, [\loc_x])$, 
%if either 
%
%\begin{itemize}
%\item for all indexes $i = 0,\cdots, \lvert \hh([\loc_x]) - 1 \rvert$, 
%$\WS(\hh([\loc_x])(i)) \neq \tsid)$, or 
%\item if $\WS(\hh([\loc_x])(i)) = \tsid$ for some $i = 0,\cdots, \lvert \hh([\loc_n]) -1 \rvert$, 
%then $i \leq V([\loc_n])$.
%\end{itemize}
\end{definition}

Suppose that we execute the program $\prog_1$ under read atomic $\comoRA$.
Assume the \( \txid_{1}\) is still committed first as in Section \ref{sec:semantics.example}, yielding the result shown in \cref{fig:opsem-example-b}.
Yet the second transaction \( \txid_{2} \) cannot starts with a view as in \cref{fig:opsem-example-c}, because later when the transaction try to commit, the read atomic commit test will stop it.
%\ac{
%$\langle \mathcal{C_0}, \prog_1 \rangle \xrightarrow{\mathsf{RA}} \langle \mathcal{C}_1, \prog_1' \rangle$, 
%where we recall that $\mathcal{C}_0$, $\mathcal{C}_1$ are depicted in Figure \ref{fig:opsem.exampe}(a), 
%\ref{fig:opsem.example}(b), respectively. 

%It is immediate to observe that the only way in which the execution of transaction $\ptrans{\trans}$ from $\thid_2$ in $\prog_1'$ can return value ${\Large \frownie}$ is the following: 
%\begin{itemize}
%\item first, push the view $V$ of client $\txid_2$ in the configuration 
%$\mathcal{C}_1$ of Figure \ref{fig:opsem.example}(b) to observe the update of location $[\vx]$, but not the update of 
%$[\vy]$. This view is the one labelled with $\txid_2$ in Figure \ref{fig:opsem.example}(c), and we refer 
%to it as $V'$;
%\item then, execute the transaction $\ptrans{\trans}$ in $\thid_2$. 
%\end{itemize}
%}

One may wonder whether it is the case that an execution of a command may get stuck because of a ill-defined execution test.
For example, for a given execution test $\ET$, it may be possible to reach a state of the system, with MKVS $\hh$ and view $\vi$ for client $\cl$ , such that for any $\vi': \vi \sqsubseteq \vi'$ we have that $ \comoRA \not\vdash (\hh, \vi') \csat[] \stub : \stub $.
We show that this is not the case for the execution test $\comoRA$. 

\ac{This does not mean that progress is ensured. A transaction may very well 
not terminate. Rather, this says that the inability of a program to execute a 
transaction does not depend on the execution test itself.}
\sx{what do you mean???}
\begin{proposition}
\label{prop:ra.progress}
The execution test $\comoRA$ \emph{does not hinder progress}: for any MKVS $\mkvs$, views $\vi$, and fingerprint $\opset$, there exist two views $\vi': \vi \sqsubseteq \vi'$, $\vi'': \Vupdate(\hh, \vi', \opset) \sqsubseteq \vi''$ such that $(\hh, \vi) \csatRA \opset : \vi'$.
\end{proposition}

\begin{proof}
For a given MKVS $\hh$, let $\func{up-to-date}{\hh} = \lambda \key{k}. (\lvert \hh \rvert - 1)$ be the view that points to the most recent version of each object. It is immediate to note that $\pred{atomic}{\hh, \func{up-to-date}{\hh}}$, and therefore $(\hh, \vi) \csatRA \opset : \vi'$. 
\end{proof}

\subsection{Transactional Causal Consistency}
\begin{figure}
\centering
\hrule
\begin{tabular}{@{} c @{} c @{} c @{}}
\begin{halfsubfig}
\begin{centertikz}

\begin{pgfonlayer}{foreground}
%Uncomment line below for help lines

%Location x
\node(locx) {$\ke_\vx \mapsto$};

\matrix(versionx) [version list]
    at ([xshift=\tikzkvspace]locx.east) {
    {a} & $\txid_0$ \\
    {a} & $\emptyset$ \\
};
\tikzvalue{versionx-1-1}{versionx-2-1}{locx-v0}{0};

%Location y
\path (locx.south) + (0,\tikzkeyspace) node (locy) {$\ke_{\vy} \mapsto$};
\matrix(versiony) [version list]
    at ([xshift=\tikzkvspace]locy.east) {
    {a} & $\txid_0$ \\
    {a} & $\emptyset$ \\
};
\tikzvalue{versiony-1-1}{versiony-2-1}{locy-v0}{0};

% \draw[-, red, very thick, rounded corners] ([xshift=-5pt, yshift=5pt]locx-v1.north east) |- 
%  ($([xshift=-5pt,yshift=-5pt]locx-v1.south east)!.5!([xshift=-5pt, yshift=5pt]locy-v0.north east)$) -| ([xshift=-5pt, yshift=5pt]locy-v0.south east);

%blue view - I should  check whether I can use pgfkeys to just declare the list of locations, and then add the view automatically.
\draw[-, blue, very thick, rounded corners=10pt]
([xshift=-2pt, yshift=20pt]locx-v0.north east) node (tid1start) {} -- 
([xshift=-2pt, yshift=-5pt]locy-v0.south east);
 
\path (tid1start) node[anchor=south, rectangle, fill=blue!20, draw=blue, font=\small, inner sep=1pt] {$\thid_3$};

%red view
\draw[-, red, very thick, rounded corners = 10pt]
([xshift=-5pt, yshift=5pt]locx-v0.north east) -- 
([xshift=-5pt, yshift=-10pt]locy-v0.south east) node (tid2start) {};
 
\path (tid2start) node[anchor=north, rectangle, fill=red!20, draw=red, font=\small, inner sep=1pt] {$\thid_2$};
 
 %green view
\draw[-, DarkGreen, very thick, rounded corners = 10pt]
([xshift=-16pt, yshift=8pt]locx-v0.north east) node (tid3start) {}-- 
([xshift=-16pt, yshift=-5pt]locy-v0.south east);
 
\path (tid3start) node[anchor=south, rectangle, fill=DarkGreen!20, draw=DarkGreen, font=\small, inner sep=1pt] {$\thid_1$};

\end{pgfonlayer}
\end{centertikz}
\caption{}
\label{fig:cc-exec-a}
\end{halfsubfig}
&
\begin{halfsubfig}
\begin{centertikz}

\begin{pgfonlayer}{foreground}
%Uncomment line below for help lines
%\draw[help lines] grid(5,4);

%Location x
\node(locx) {$\ke_{\vx} \mapsto$};

\matrix(versionx) [version list]
    at ([xshift=\tikzkvspace]locx.east) {
    {a} & $\tsid_0$ & {a} & $\tsid_1$\\
    {a} & $\emptyset$ & {a} & $\emptyset$ \\
};
\tikzvalue{versionx-1-1}{versionx-2-1}{locx-v0}{0};
\tikzvalue{versionx-1-3}{versionx-2-3}{locx-v1}{1};

%Location y
\path (locx.south) + (0,\tikzkeyspace) node (locy) {$\ke_{\vy} \mapsto$};
\matrix(versiony) [version list]
   at ([xshift=\tikzkvspace]locy.east) {
 {a} & $\tsid_0$ \\
   {a} & $\emptyset$ \\
};
\tikzvalue{versiony-1-1}{versiony-2-1}{locy-v0}{0};

%blue view - I should  check whether I can use pgfkeys to just declare the list of locations, and then add the view automatically.
\draw[-, blue, very thick, rounded corners=10pt]
([xshift=-2pt, yshift=20pt]locx-v0.north east) node (tid1start) {} -- 
([xshift=-2pt, yshift=-5pt]locy-v0.south east);
 
\path (tid1start) node[anchor=south, rectangle, fill=blue!20, draw=blue, font=\small, inner sep=1pt] {$\thid_3$};

%red view
\draw[-, red, very thick, rounded corners = 10pt]
([xshift=-5pt, yshift=5pt]locx-v0.north east) -- 
([xshift=-5pt, yshift=-10pt]locy-v0.south east) node (tid2start) {};
 
\path (tid2start) node[anchor=north, rectangle, fill=red!20, draw=red, font=\small, inner sep=1pt] {$\thid_2$};
 
 %green view
\draw[-, DarkGreen, very thick, rounded corners = 10pt]
([xshift=-16pt, yshift=8pt]locx-v1.north east) node (tid3start) {}-- 
([xshift=-16pt, yshift=-5pt]locx-v1.south east) --
([xshift=-16pt, yshift=5pt]locy-v0.north east) -- 
([xshift=-16pt, yshift=-5pt]locy-v0.south east);
 
\path (tid3start) node[anchor=south, rectangle, fill=DarkGreen!20, draw=DarkGreen, font=\small, inner sep=1pt] {$\thid_1$};

\end{pgfonlayer}
\end{centertikz}
\caption{}
\label{fig:cc-exec-b}
\end{halfsubfig}
&
\begin{halfsubfig}
\begin{centertikz}
\begin{pgfonlayer}{foreground}
%Uncomment line below for help lines
%\draw[help lines] grid(5,4);

%Location x
\node(locx) {$\ke_{\vx} \mapsto$};

\matrix(versionx) [version list]
    at ([xshift=\tikzkvspace]locx.east) {
    {a} & $\tsid_0$ & {a} & $\tsid_1$\\
    {a} & $\emptyset$ & {a} & $\emptyset$ \\
};
\tikzvalue{versionx-1-1}{versionx-2-1}{locx-v0}{0};
\tikzvalue{versionx-1-3}{versionx-2-3}{locx-v1}{1};

%Location y
\path (locx.south) + (0,\tikzkeyspace) node (locy) {$\ke_{\vy} \mapsto$};
\matrix(versiony) [version list]
   at ([xshift=\tikzkvspace]locy.east) {
 {a} & $\tsid_0$ \\
   {a} & $\emptyset$ \\
};
\tikzvalue{versiony-1-1}{versiony-2-1}{locy-v0}{0};

%blue view - I should  check whether I can use pgfkeys to just declare the list of locations, and then add the view automatically.
\draw[-, blue, very thick, rounded corners=10pt]
([xshift=-2pt, yshift=20pt]locx-v0.north east) node (tid1start) {} -- 
([xshift=-2pt, yshift=-5pt]locy-v0.south east);
 
\path (tid1start) node[anchor=south, rectangle, fill=blue!20, draw=blue, font=\small, inner sep=1pt] {$\thid_3$};

%red view
\draw[-, red, very thick, rounded corners = 10pt]
([xshift=-5pt, yshift=5pt]locx-v1.north east) -- 
([xshift=-5pt, yshift=-5pt]locx-v1.south east) --
([xshift=-5pt, yshift=3pt]locy-v0.north east) -- 
([xshift=-5pt, yshift=-10pt]locy-v0.south east) node (tid2start) {};
 
\path (tid2start) node[anchor=north, rectangle, fill=red!20, draw=red, font=\small, inner sep=1pt] {$\thid_2$};
 
%green view
\draw[-, DarkGreen, very thick, rounded corners = 10pt]
([xshift=-16pt, yshift=8pt]locx-v1.north east) node (tid3start) {}-- 
([xshift=-16pt, yshift=-5pt]locx-v1.south east) --
([xshift=-16pt, yshift=5pt]locy-v0.north east) -- 
([xshift=-16pt, yshift=-5pt]locy-v0.south east);
 
\path (tid3start) node[anchor=south, rectangle, fill=DarkGreen!20, draw=DarkGreen, font=\small, inner sep=1pt] {$\thid_1$};

\end{pgfonlayer}
\end{centertikz}
\caption{}
\label{fig:cc-exec-c}
\end{halfsubfig}
\\
\begin{halfsubfig}
\begin{centertikz}

\begin{pgfonlayer}{foreground}
%Uncomment line below for help lines
%\draw[help lines] grid(5,4);

%Location x
\node(locx) {$\ke_{\vx} \mapsto$};

\matrix(versionx) [version list]
    at ([xshift=\tikzkvspace]locx.east) {
    {a} & $\tsid_0$ & {a} & $\tsid_1$\\
    {a} & $\emptyset$ & {a} & $\Set{\txid_{2}}$ \\
};
\tikzvalue{versionx-1-1}{versionx-2-1}{locx-v0}{0};
\tikzvalue{versionx-1-3}{versionx-2-3}{locx-v1}{1};

%Location y
\path (locx.south) + (0,\tikzkeyspace) node (locy) {$\ke_{\vy} \mapsto$};
\matrix(versiony) [version list]
    at ([xshift=\tikzkvspace]locy.east) {
    {a} & $\tsid_0$ & {a} & $\tsid_2$ \\
    {a} & $\emptyset$ & {a} & $\emptyset$\\
};
\tikzvalue{versiony-1-1}{versiony-2-1}{locy-v0}{0};
\tikzvalue{versiony-1-3}{versiony-2-3}{locy-v1}{1};

%blue view - I should  check whether I can use pgfkeys to just declare the list of locations, and then add the view automatically.
\draw[-, blue, very thick, rounded corners=10pt]
([xshift=-2pt, yshift=20pt]locx-v0.north east) node (tid1start) {} -- 
([xshift=-2pt, yshift=-5pt]locy-v0.south east);
 
\path (tid1start) node[anchor=south, rectangle, fill=blue!20, draw=blue, font=\small, inner sep=1pt] {$\thid_3$};

%red view
\draw[-, red, very thick, rounded corners = 10pt]
([xshift=-5pt, yshift=5pt]locx-v1.north east) -- 
([xshift=-5pt, yshift=-10pt]locy-v1.south east) node (tid2start) {};
 
\path (tid2start) node[anchor=north, rectangle, fill=red!20, draw=red, font=\small, inner sep=1pt] {$\thid_2$};
 
 %green view
\draw[-, DarkGreen, very thick, rounded corners = 10pt]
([xshift=-16pt, yshift=8pt]locx-v1.north east) node (tid3start) {}-- 
([xshift=-16pt, yshift=-5pt]locx-v1.south east) --
([xshift=-16pt, yshift=5pt]locy-v0.north east) -- 
([xshift=-16pt, yshift=-5pt]locy-v0.south east);
 
\path (tid3start) node[anchor=south, rectangle, fill=DarkGreen!20, draw=DarkGreen, font=\small, inner sep=1pt] {$\thid_1$};

\end{pgfonlayer}
\end{centertikz}
\caption{}
\label{fig:cc-exec-d}
\end{halfsubfig}
&
\begin{halfsubfig}
\begin{centertikz}

\begin{pgfonlayer}{foreground}
%Uncomment line below for help lines
%\draw[help lines] grid(5,4);

%Location x
\node(locx) {$\ke_{\vx} \mapsto$};

\matrix(versionx) [version list]
    at ([xshift=\tikzkvspace]locx.east) {
    {a} & $\tsid_0$ & {a} & $\tsid_1$\\
    {a} & $\emptyset$ & {a} & $\Set{\txid_{2}}$ \\
};
\tikzvalue{versionx-1-1}{versionx-2-1}{locx-v0}{0};
\tikzvalue{versionx-1-3}{versionx-2-3}{locx-v1}{1};

%Location y
\path (locx.south) + (0,\tikzkeyspace) node (locy) {$\ke_{\vy} \mapsto$};
\matrix(versiony) [version list]
    at ([xshift=\tikzkvspace]locy.east) {
    {a} & $\tsid_0$ & {a} & $\tsid_2$ \\
    {a} & $\emptyset$ & {a} & $\emptyset$\\
};
\tikzvalue{versiony-1-1}{versiony-2-1}{locy-v0}{0};
\tikzvalue{versiony-1-3}{versiony-2-3}{locy-v1}{1};

%blue view - I should  check whether I can use pgfkeys to just declare the list of locations, and then add the view automatically.
\draw[-, blue, very thick, rounded corners=10pt]
([xshift=-2pt, yshift=20pt]locx-v0.north east) node (tid1start) {} -- 
([xshift=-2pt, yshift=-5pt]locx-v0.south east) --
([xshift=-2pt, yshift=5pt]locy-v1.north east) -- 
([xshift=-2pt, yshift=-5pt]locy-v1.south east);
 
\path (tid1start) node[anchor=south, rectangle, fill=blue!20, draw=blue, font=\small, inner sep=1pt] {$\thid_3$};

%red view
\draw[-, red, very thick, rounded corners = 10pt]
([xshift=-5pt, yshift=5pt]locx-v1.north east) -- 
([xshift=-5pt, yshift=-10pt]locy-v1.south east) node (tid2start) {};
 
\path (tid2start) node[anchor=north, rectangle, fill=red!20, draw=red, font=\small, inner sep=1pt] {$\thid_2$};
 
%green view
\draw[-, DarkGreen, very thick, rounded corners = 10pt]
([xshift=-16pt, yshift=8pt]locx-v1.north east) node (tid3start) {}-- 
([xshift=-16pt, yshift=-5pt]locx-v1.south east) --
([xshift=-16pt, yshift=5pt]locy-v0.north east) -- 
([xshift=-16pt, yshift=-5pt]locy-v0.south east);
 
\path (tid3start) node[anchor=south, rectangle, fill=DarkGreen!20, draw=DarkGreen, font=\small, inner sep=1pt] {$\thid_1$};

\end{pgfonlayer}
\end{centertikz}
\caption{}
\label{fig:cc-exec-e}
\end{halfsubfig}
&
\begin{halfsubfig}
\begin{centertikz}

\begin{pgfonlayer}{foreground}
%Uncomment line below for help lines
%\draw[help lines] grid(5,4);

%Location x
\node(locx) {$\ke_{\vx} \mapsto$};

\matrix(versionx) [version list]
    at ([xshift=\tikzkvspace]locx.east) {
    {a} & $\tsid_0$ & {a} & $\tsid_1$\\
    {a} & $\Set{\txid_{3}}$ & {a} & $\Set{\txid_{2}}$ \\
};
\tikzvalue{versionx-1-1}{versionx-2-1}{locx-v0}{0};
\tikzvalue{versionx-1-3}{versionx-2-3}{locx-v1}{1};

%Location y
\path (locx.south) + (0,\tikzkeyspace) node (locy) {$\ke_{\vy} \mapsto$};
\matrix(versiony) [version list]
    at ([xshift=\tikzkvspace]locy.east) {
    {a} & $\tsid_0$ & {a} & $\tsid_2$ \\
    {a} & $\emptyset$ & {a} & $\Set{\txid_{3}}$\\
};
\tikzvalue{versiony-1-1}{versiony-2-1}{locy-v0}{0};
\tikzvalue{versiony-1-3}{versiony-2-3}{locy-v1}{1};

%blue view - I should  check whether I can use pgfkeys to just declare the list of locations, and then add the view automatically.
\draw[-, blue, very thick, rounded corners=10pt]
([xshift=-2pt, yshift=20pt]locx-v0.north east) node (tid1start) {} -- 
([xshift=-2pt, yshift=-5pt]locx-v0.south east) --
([xshift=-2pt, yshift=5pt]locy-v1.north east) -- 
([xshift=-2pt, yshift=-5pt]locy-v1.south east);
 
\path (tid1start) node[anchor=south, rectangle, fill=blue!20, draw=blue, font=\small, inner sep=1pt] {$\thid_3$};

%red view
\draw[-, red, very thick, rounded corners = 10pt]
([xshift=-5pt, yshift=5pt]locx-v1.north east) -- 
([xshift=-5pt, yshift=-10pt]locy-v1.south east) node (tid2start) {};
 
\path (tid2start) node[anchor=north, rectangle, fill=red!20, draw=red, font=\small, inner sep=1pt] {$\thid_2$};
 
%green view
\draw[-, DarkGreen, very thick, rounded corners = 10pt]
([xshift=-16pt, yshift=8pt]locx-v1.north east) node (tid3start) {}-- 
([xshift=-16pt, yshift=-5pt]locx-v1.south east) --
([xshift=-16pt, yshift=5pt]locy-v0.north east) -- 
([xshift=-16pt, yshift=-5pt]locy-v0.south east);
 
\path (tid3start) node[anchor=south, rectangle, fill=DarkGreen!20, draw=DarkGreen, font=\small, inner sep=1pt] {$\thid_1$};

\end{pgfonlayer}
\end{centertikz}
\caption{}
\label{fig:cc-exec-f}
\end{halfsubfig}
\end{tabular}
\hrule\vspace{5pt}
\caption{History heaps obtained in a execution of $\prog_2$.}
\label{fig:cc.exec}
\label{fig:cc-exec}
\end{figure}


\sx{Should we give intuition about causal dependencies here ?}
The next consistency model that we are interested is \emph{transactional causal consistency} \cite{cops}. 
Intuitively, it ensures that versions read by transactions are closed with respect to \emph{causal dependencies}. 
Consider for example the following program: 
\[
    \prog_2 \equiv \begin{session}
        \begin{array}{@{}c || c || c@{}}
            \txid_{1} : 
            \begin{transaction}
                \pmutate{\vx}{1};\\
            \end{transaction} &
            \txid_{2} : 
            \begin{transaction} 
                \pderef{\pvar{a}}{\vx};\\
                \pmutate{\vy}{\pvar{a}};\\
            \end{transaction} &
            \txid_{3} :
             \begin{transaction}
               	   \pderef{\pvar{a}}{\vx};\\
               	   \pderef{\pvar{b}}{\vy};\\
               	   \pifs{\pvar{a}=0 \wedge \pvar{b}=1}\\
               			\quad \passign{\retvar}{\sadface}
               		\pife
             \end{transaction}
        \end{array}
    \end{session}
 \]
For the sake of simplicity, we label the code of the three transactions above as $\txid_{1}, \txid_2, \txid_3$ from left to right.
It is easy to see that, if no constraints or even under read atomic, the third transaction $\txid_{3}$ can return ${\sadface{}}$. 
%The same is true even if the consistency model specification $\mathsf{RA}$ is assumed. 
Informally, the return of value ${\sadface}$ by $\txid_3$ can be obtained from the execution outlined below. 
\begin{itemize}
\item The initial configuration of this execution is depicted in \cref{fig:cc-exec-a}.
\item The transaction $\txid_{1}$ executes with the initial view, which points to the initial (and only) version for each location; after this executing the transaction, a new version $( 1, \txid_1, \emptyset )$ is appended to \( \ke_\vx\) (\cref{fig:cc-exec-b}).
\item The second client $\thid_2$ updates its view as to see the last version of $\ke_\vx$ installed by $\thid_1$ (\cref{fig:cc-exec-c}), after which it proceeds to execute transaction $\txid_{2}$. 
This results in a new version with value $1$ to be installed to the end of $\ke_\vy$ (\cref{fig:cc-exec-d}). 
\item Finally, the client $\thid_3$ updates its view so to observe the update of location $\ke_{\vy}$, but not the update of 
location $\ke_{\vx}$ (\cref{fig:cc-exec-e}).
Then, it executes $\txid_{3}$ which will return the value ${\sadface{}}$ (\cref{fig:cc-exec-f}).
\end{itemize}

In the last step, the client to the right commits the transaction $\txid_3$ in a state where its initial view observes the second version of the address $\vy$, which is created by \( \txid_{2} \).
However, because the transaction \( \txid_{2} \) read the second version of \( \ke_\vx \) and create the second version of \( \ke_\vy \), this means the latter depends on the former.
Yet the transaction \( \txid_{3} \) does not read from the second version of \( \ke_\vx \), which is disallowed by \emph{transactional causal consistency}.
Summarising, under transactional causal consistency if a transaction sees updates for a key \( \ke \), it should also observes those keys that \( \ke \) depends on.

%\ac{
%but not the update to address $\vx$.
%However, the update of $[\vy]$ committed by $\txid_{2}$, consisted in copying the value of the update 
%of $\ke_{\vy}$: that is, the update of $\ke_{\vy}$ \textbf{depends} from the update of $\ke_{\vx}$. 
%Summarising, the execution of transaction $\ptrans{\trans_3}$ resulted in a violation of 
%causality: the update of $\ke_{\vy}$ is observed, but not the update of $\ke_{\vx}$ on which 
%it depends.
%}

To formally specify \emph{transactional causal consistency}, we inductively define the set of views that are consistent with respect to a history heap $\hh$. 
The definition below models the fact that, if we start from a causally consistent view, and we wish to update the view for some location $\txid_2$, 

\begin{definition}[Transactional causal dependency]
\label{def:causal}
Given two versions $\ver_{1} = (\val_1, \txid_1, \txidset_1)$, $\ver_2 = (\val_2, \txid_2, \txidset_2)$, $\ver_{1}$ \emph{directly depends on} $\ver_2$, written $\ver_1 \xrightarrow{\ddep} \ver_2$, iff:
\[
\begin{rclarray}
    (\val_1, \txid_1, \txidset_1) \xrightarrow{\ddep} (\val_2, \txid_2, \txidset_2) & \defeq & \txid_1 \in \txidset_2 \lor  \exsts{\txid, \txid'} \txid \in \Set{\txid_1} \cup \txidset_1 \land \txid' \in \Set{\txid_2} \cup \txidset_2 \land \txid \leq \txid'
\end{rclarray}
\]
%\ac{Note to self: the notion of directly depends here has little to do with dependencies 
%between transactions. $\nu_1 \xrightarrow{\mathsf{ddep}} \nu_2$ means that 
%some transaction $\tsid$ touches both versions. However, it reads $\nu_2$ and 
%writes $\nu_1$.}
Given $\hh \in \HisHeaps$, a view \( \vi \) are \emph{causally consistent} with the key-value store $\hh$, $\pred{CCViews}{\hh, \vi}$: 
\[
\begin{rclarray}
    \pred{CCViews}{\hh, \vi} & \defeq & \fora{\ke, \ke', i, j} i \leq \vi(\ke) \land \mkvs(\ke, i) \xrightarrow{\ddep}^{*} \mkvs(\ke, j) \implies j \leq \vi(\ke')
\end{rclarray}
\]
%\begin{itemize} 
%\item the initial view \( \vi_0\)  is in the set, \ie $\vi_0 \in \func{CCViews}{\hh}$ where \( \fora{\ke \in \dom(\hh)} \vi_{0}(\ke) = 1 \),
%\item assume any view $\vi \in \pred{CCViews}{\hh}$ and a new view \( \vi' \) by observing one more version for an address $\vi' = \vi\rmto{\ke}{\vi(\ke) + 1}$, where \( \vi'(\ke) < \left| \hh(\ke) \right| \).
%If some versions directly depend on the version corresponding to \( \vi'(\ke)\) and those versions are aware by \( \vi'\), the new view is included in \( \func{CCViews}{\hh}\),
%\[
%\begin{array}{@{}l}
%\fora{\vi,\vi'} \exsts{\ke}
%\vi \in \func{CCViews}{\hh} 
%\land \ke \in \dom(\vi)
%\land \vi' = \vi\rmto{\ke}{\vi(\ke) + 1}
%\land \vi'(\ke) < \left| \hh(\ke) \right|  \\
%\quad {} \land 
%\begin{B}
%\fora{\ke', i}  
%0 \leq i < \left| \hh(\ke') \right|
%\land \pred{ddep}{\hh(\ke')(i), \hh(\ke)(\vi'(\ke))}
%\implies i \leq \vi(\ke')
%\end{B} \\
%\qquad {} \implies \vi' \in \func{CCViews}{\hh}
%\end{array}
%\]
%\end{itemize}
Causal consistency is stronger than read atomic (\cref{def:readatomic}) by further ensuring that a transaction can only observe a causally consistent state of the database, i.e. the view when the transaction starts is causally consistent with the history heap,
\[
\begin{rclarray}
    (\hh, \vi) \csatCC \opset : \vi' & \defeq & (\hh, \vi) \csatRA \opset : \vi' \land \pred{CCViews}{\hh,\vi}
\end{rclarray}
\]
\end{definition}

%\begin{figure}
%\centering
%\hrule\vspace{5pt}
%\begin{tabular}{@{} c @{} c @{}} 


%\begin{halfsubfig}
%\begin{centertikz}

%\begin{pgfonlayer}{foreground}
%%Uncomment line below for help lines
%%\draw[help lines] grid(5,4);

%%Location x
%\node(locx) {$\ke_{\vx} \mapsto$};

%\matrix(versionx) [version list]
   %at ([xshift=\tikzkvspace]locx.east) {
 %{a} & $\tsid_0$ & {a} & $\tsid_1$\\
  %{a} & $\emptyset$ & {a} & $\Set{\tsid_2}$ \\
%};
%\tikzvalue{versionx-1-1}{versionx-2-1}{locx-v0}{\stub};
%\tikzvalue{versionx-1-3}{versionx-2-3}{locx-v1}{\stub};

%%Location y
%\path (locx.south) + (0,\tikzkeyspace) node (locy) {$\ke_{\vy} \mapsto$};
%\matrix(versiony) [version list]
    %at ([xshift=\tikzkvspace]locy.east) {
    %{a} & $\tsid_0$ & {a} & $\tsid_2$ \\
    %{a} & $\emptyset$ & {a} & $\Set{\tsid_3}$\\
%};

%\tikzvalue{versiony-1-1}{versiony-2-1}{locy-v0}{\stub};
%\tikzvalue{versiony-1-3}{versiony-2-3}{locy-v1}{\stub};

%%Location z

%\path (locy.south) + (0,\tikzkeyspace) node (locz) {$\ke_\vz \mapsto$};
%\matrix(versionz) [version list]
    %at ([xshift=\tikzkvspace]locz.east) {
    %{a} & $\tsid_0$ & {a} & $\tsid_3$ \\
    %{a} & $\emptyset$ & {a} & $\emptyset$\\
%};

%\tikzvalue{versionz-1-1}{versionz-2-1}{locz-v0}{\stub};
%\tikzvalue{versionz-1-3}{versionz-2-3}{locz-v1}{\stub};

%%blue view - I should  check whether I can use pgfkeys to just declare the list of locations, and then add the view automatically.
%\draw[-, blue, very thick, rounded corners=10pt]
%([xshift=-5pt, yshift=5pt]locx-v0.north east) node (tid1start) {} -- 
%([xshift=-5pt, yshift=-5pt]locz-v0.south east);
 
%% \path (tid1start) node[anchor=south, rectangle, fill=blue!20, draw=blue, font=\small, inner sep=1pt] {$\thid_3$};
%\end{pgfonlayer}
%\end{centertikz}
%\caption{}
%\label{fig:cc-view-a}
%\end{halfsubfig}
%&
%\begin{halfsubfig}
%\begin{centertikz}

%\begin{pgfonlayer}{foreground}
%%Uncomment line below for help lines
%%\draw[help lines] grid(5,4);

%%Location x
%\node(locx) {$\ke_{\vx} \mapsto$};

%\matrix(versionx) [version list]
   %at ([xshift=\tikzkvspace]locx.east) {
 %{a} & $\tsid_0$ & {a} & $\tsid_1$\\
  %{a} & $\emptyset$ & {a} & $\Set{\tsid_2}$ \\
%};
%\tikzvalue{versionx-1-1}{versionx-2-1}{locx-v0}{\stub};
%\tikzvalue{versionx-1-3}{versionx-2-3}{locx-v1}{\stub};

%%Location y
%\path (locx.south) + (0,\tikzkeyspace) node (locy) {$\ke_{\vy} \mapsto$};
%\matrix(versiony) [version list]
    %at ([xshift=\tikzkvspace]locy.east) {
    %{a} & $\tsid_0$ & {a} & $\tsid_2$ \\
    %{a} & $\emptyset$ & {a} & $\Set{\tsid_3}$\\
%};

%\tikzvalue{versiony-1-1}{versiony-2-1}{locy-v0}{\stub};
%\tikzvalue{versiony-1-3}{versiony-2-3}{locy-v1}{\stub};

%%Location z

%\path (locy.south) + (0,\tikzkeyspace) node (locz) {$\ke_\vz \mapsto$};
%\matrix(versionz) [version list]
    %at ([xshift=\tikzkvspace]locz.east) {
    %{a} & $\tsid_0$ & {a} & $\tsid_3$ \\
    %{a} & $\emptyset$ & {a} & $\emptyset$\\
%};

%\tikzvalue{versionz-1-1}{versionz-2-1}{locz-v0}{\stub};
%\tikzvalue{versionz-1-3}{versionz-2-3}{locz-v1}{\stub};

%%blue view - I should  check whether I can use pgfkeys to just declare the list of locations, and then add the view automatically.
%\draw[-, blue, very thick, rounded corners=10pt]
%([xshift=-5pt, yshift=5pt]locx-v1.north east) node (tid1start) {} -- 
%([xshift=-5pt, yshift=-5pt]locx-v1.south east) --
%([xshift=-5pt, yshift=5pt]locy-v0.north east) -- 
%([xshift=-5pt, yshift=-5pt]locz-v0.south east);
 
%% \path (tid1start) node[anchor=south, rectangle, fill=blue!20, draw=blue, font=\small, inner sep=1pt] {$\thid_3$};
%\end{pgfonlayer}
%\end{centertikz}
%\caption{}
%\label{fig:cc-view-b}
%\end{halfsubfig}
%\\
%\begin{halfsubfig}
%\begin{centertikz}

%\begin{pgfonlayer}{foreground}
%%Uncomment line below for help lines
%%\draw[help lines] grid(5,4);

%%Location x
%\node(locx) {$\ke_{\vx} \mapsto$};

%\matrix(versionx) [version list]
   %at ([xshift=\tikzkvspace]locx.east) {
 %{a} & $\tsid_0$ & {a} & $\tsid_1$\\
  %{a} & $\emptyset$ & {a} & $\Set{\tsid_2}$ \\
%};
%\tikzvalue{versionx-1-1}{versionx-2-1}{locx-v0}{\stub};
%\tikzvalue{versionx-1-3}{versionx-2-3}{locx-v1}{\stub};

%%Location y
%\path (locx.south) + (0,\tikzkeyspace) node (locy) {$\ke_{\vy} \mapsto$};
%\matrix(versiony) [version list]
    %at ([xshift=\tikzkvspace]locy.east) {
    %{a} & $\tsid_0$ & {a} & $\tsid_2$ \\
    %{a} & $\emptyset$ & {a} & $\Set{\tsid_3}$\\
%};

%\tikzvalue{versiony-1-1}{versiony-2-1}{locy-v0}{\stub};
%\tikzvalue{versiony-1-3}{versiony-2-3}{locy-v1}{\stub};

%%Location z

%\path (locy.south) + (0,\tikzkeyspace) node (locz) {$\ke_\vz \mapsto$};
%\matrix(versionz) [version list]
    %at ([xshift=\tikzkvspace]locz.east) {
    %{a} & $\tsid_0$ & {a} & $\tsid_3$ \\
    %{a} & $\emptyset$ & {a} & $\emptyset$\\
%};

%\tikzvalue{versionz-1-1}{versionz-2-1}{locz-v0}{\stub};
%\tikzvalue{versionz-1-3}{versionz-2-3}{locz-v1}{\stub};

%%blue view - I should  check whether I can use pgfkeys to just declare the list of locations, and then add the view automatically.
%\draw[-, blue, very thick, rounded corners=10pt]
%([xshift=-5pt, yshift=5pt]locx-v1.north east) node (tid1start) {} -- 
%([xshift=-5pt, yshift=-5pt]locy-v1.south east) --
%([xshift=-5pt, yshift=5pt]locz-v0.north east) -- 
%([xshift=-5pt, yshift=-5pt]locz-v0.south east);
 
%% \path (tid1start) node[anchor=south, rectangle, fill=blue!20, draw=blue, font=\small, inner sep=1pt] {$\thid_3$};
%\end{pgfonlayer}
%\end{centertikz}
%\caption{}
%\label{fig:cc-view-c}
%\end{halfsubfig}
%&
%\begin{halfsubfig}
%\begin{centertikz}

%\begin{pgfonlayer}{foreground}
%%Uncomment line below for help lines
%%\draw[help lines] grid(5,4);

%%Location x
%\node(locx) {$\ke_{\vx} \mapsto$};

%\matrix(versionx) [version list]
   %at ([xshift=\tikzkvspace]locx.east) {
 %{a} & $\tsid_0$ & {a} & $\tsid_1$\\
  %{a} & $\emptyset$ & {a} & $\Set{\tsid_2}$ \\
%};
%\tikzvalue{versionx-1-1}{versionx-2-1}{locx-v0}{\stub};
%\tikzvalue{versionx-1-3}{versionx-2-3}{locx-v1}{\stub};

%%Location y
%\path (locx.south) + (0,\tikzkeyspace) node (locy) {$\ke_{\vy} \mapsto$};
%\matrix(versiony) [version list]
    %at ([xshift=\tikzkvspace]locy.east) {
    %{a} & $\tsid_0$ & {a} & $\tsid_2$ \\
    %{a} & $\emptyset$ & {a} & $\Set{\tsid_3}$\\
%};

%\tikzvalue{versiony-1-1}{versiony-2-1}{locy-v0}{\stub};
%\tikzvalue{versiony-1-3}{versiony-2-3}{locy-v1}{\stub};

%%Location z

%\path (locy.south) + (0,\tikzkeyspace) node (locz) {$\ke_\vz \mapsto$};
%\matrix(versionz) [version list]
    %at ([xshift=\tikzkvspace]locz.east) {
    %{a} & $\tsid_0$ & {a} & $\tsid_3$ \\
    %{a} & $\emptyset$ & {a} & $\emptyset$\\
%};

%\tikzvalue{versionz-1-1}{versionz-2-1}{locz-v0}{\stub};
%\tikzvalue{versionz-1-3}{versionz-2-3}{locz-v1}{\stub};

%%blue view - I should  check whether I can use pgfkeys to just declare the list of locations, and then add the view automatically.
%\draw[-, blue, very thick, rounded corners=10pt]
%([xshift=-5pt, yshift=5pt]locx-v1.north east) node (tid1start) {} -- 
%([xshift=-5pt, yshift=-5pt]locz-v1.south east);
 
%% \path (tid1start) node[anchor=south, rectangle, fill=blue!20, draw=blue, font=\small, inner sep=1pt] {$\thid_3$};
%\end{pgfonlayer}
%\end{centertikz}
%\caption{}
%\label{fig:cc-view-d}
%\end{halfsubfig}
%\end{tabular}
%\hrule
%\caption{Building a causally consistent view.}
%\label{fig:cc.view}
%\label{fig:cc-view}
%\end{figure}

%\ac{Note to self: I got this example and the definition wrong several times before getting them
%right. Which means that inductive definition of causally dependent views is not really that 
%intuitive after all...}
%Let consider the history $\hh$ in \cref{fig:cc-view-a}, where $\pred{ddep}{\hh(\ke_\vz)(1),\hh(\ke_{\vy})(1)}$, and $\pred{ddep}{\hh(\ke_{\vy})(1) ,\hh(\ke_{\vx})(1)}$.
%Since the values are irrelevant, we ignore the values.
%We want to find a view $\vi$ that is causal consistent and is up-to-date with the last version of key $[\ke_{\vz}]$, \ie \( \vi(\ke_\vz) = 1\).
%We can construct such a view incrementally by the definition of \( \funcn{CCViews} \) function.

%We start from the initial view $\vi_0$ pointing to initial versions of all keys (\cref{fig:cc-view-a}).
%This view is causally consistent by definition.
%As a first try, one could immediately consider a view $\vi'= \vi_0\rmto{\ke_\vz}{1}$ where the index of version for key $\ke_\vz$ is updated to $1$, but this view is not causally consistent due to \( \pred{ddep}{\hh(\ke_\vz)(1),\hh(\ke_{\vy})(1)} \) and \( \vi'(\ke_\vy) = 0\).
%That is, the version of $\ke_\vz$ observed by $\vi'$ directly depends on the second version of $\ke_\vy$ that is not observed by $\vi'$.
%Similarly one cannot advance the view to observe the latest version of \( \ke_\vy\) without knowing the latest version of \( \ke_\vx\).
%Therefore we can only update the view $V_0$ by including the second version of $\ke_{\vx}$, \ie the new view \(\vi_1 = \vi_0\rmto{\ke_\vx}{1} \) (\cref{fig:cc-view-b}).
%The new view \( \vi_1\) is causally consistent as no version directly depend on the second version of \( \ke_\vx\).
%We can now update the view $\vi_1$ to include the latest version of $\ke_{\vy}$, resulting in the view $\vi_2 = \vi_1\rmto{\ke_{\vy}}{1}$ (\cref{fig:cc-view-c}). 
%The view \( \vi_2\) is causally consistent as the second version of \( \ke_\vx\) is already included in the view.
%Finally, the view includes the latest version of \( \ke_\vz \) (\cref{fig:cc-view-d}).
%The view \( \vi_3\) is causally consistent as the second version of \( \ke_\vy\) is already included in the view.

%\ac{Note to self: here there is something subtle going on. We also need to ensure that dependencies 
%caused by the information flow of the program are tracked down. For example, we could have a 
%transaction returning the value of a location $\ke_{\vx}$, and then another transaction copy such a value 
%into another location $\ke_{\vy}$. There is a notion of dependency between $\ke_{\vx}, \ke_{\vy}$ that 
%is not captured by the notion of \emph{directly depends}. On the other hand, the fact that 
%stacks are client local, and we have per-client view monotonicity, should ensure that also program 
%dependencies are preserved. A definitive proof that $\mathsf{CC}$ is equivalent to caual consistency 
%specified either in terms of abstract executions or dependency graphs, would settle the argument.}

Note that, the view of $\thid_3$ in \cref{fig:cc-exec-e} is not causally consistent.
Because $V(\ke_{\vy}) = 1$, and $\vi(\ke_{\vx}) = 0$. 
However, $\hh(\ke_{\vy},1)$ directly depends on $\hh(\ke_{\vx},1)$, which is not included in the view. 
In general, the only case in which an execution of $\prog_2$ causes transaction $\txid_3$ to return value $\sadface$ is when such a transaction is executed using a snapshot determined by a non-causally consistent view. 
There exists no execution of $\prog_2$ under $\mathsf{CC}$ that causes $\txid_3$ to return value $\sadface$.
%if we execute the program $\prog_2$, illustrated previously in this section, under $\mathsf{CCViews}$,
%it is not possible any more to have the transaction $\ptrans{\trans_3}$ return value ${\sadface}$. 
%This is because, in order for $\ptrans{\trans_3}$ to return such a value, it must be executed in a 
%configuration such as the one of Figure \ref{fig:cc.view}(e) (only the view of $\thid_3$ is relevant here), 
%%state where the view client of $\thid_3$ observes the update to location $\ke_{\vy}$, but not the 
%%update of $\ke_{\vx}$ from which the latter causally depends. 

%\ac{In theory, I can do better, and require that I see only the causal dependencies 
%of what I read. But at the end of the day, who cares?}

%\begin{definition}
%Let $\hh, V$ and $\ke_{\vx}$ be a history heap, a view, and a location, respectively. 
%Given two transactions $\tsid_1, \tsid_2$, we say that $\tsid_2$ write-read depends 
%on $\tsid_1$ via $\ke_{\vx}$, written $\tsid_1 \xrightarrow{\RF(\ke_{\vx})_{\hh}} \tsid_2$, 
%if there exists an index $i = 0,\cdots, \llvert \hh, \rvert -1$ such that $\hh([\loc_n])(x) = 
%(\_, \tsid_1, \mathcal{T})$, and $\tsid_2 \in \mathcal{T}$.
%\end{definition}
%
%\begin{definition}
%Let $\hh$ be a history heap, and $V$ be a view.
%Let $\ke_{\vx}$ be a location, and let 
%$(\_, \tsid, \_) := \hh([\loc_{n}])(V([\loc_{x}]))$. 
%We say that $V$ respects causality for $[\loc_{x}]$ if, 
%whenever $\tsid' \xrightarrow{\RF([\loc_{x}])}_{\tsid}'$, 
%$\hh(\ke_{\vx})(i)$ as follows: 
%\begin{enumerate}
%\end{definition}

\subsection{Update Atomic \( \UA \)}
\label{sec:sound-complete-ua}

Given abstract execution \( \aexec \), we define write-write relation for a key \( \ke \) as the following:
\[ 
    \WW(\aexec,\ke) \defeq \Setcon{(\txid, \txid')}{\txid \toEdge{\AR_\aexec} \txid' \land (\otW,\ke, \stub ) \in \txid \land (\otW,\ke, \stub ) \in \txid'  } 
\]
Then, the notation \( \WW_\aexec \defeq \bigcup_{\ke \in \Keys} \WW(\aexec, \ke) \).
Note that for a kv-store \( \mkvs \) such that \( \mkvs = \mkvs_\aexec \),
by the definition of  \(  \mkvs = \mkvs_\aexec \), 
the following holds:
\[
    \WW_\aexec = \Setcon{(\txid, \txid')}{\exsts{\ke, i,j } \txid = \WTx(\mkvs(\ke, i)) \land \txid' = \WTx(\mkvs(\ke, j)) \land i < j}
\]
Also the \( \WW_\aexec \) coincides with \( \WW_\Gr \) and \( \WW_\mkvs \).

The execution test $\ET_\UA$ is sound with respect to the axiomatic specification \( (\RP_{\LWW}, \Set{\lambda \aexec. \WW_\aexec }) \).
We pick the invariant as \( I( \aexec, \cl ) = \emptyset \), given the fact of no constraint on the final view.
Assume a kv-store $\hh$, an initial and a final view $\vi, \vi'$  a fingerprint $\opset$ 
such that $\ET_{\UA} \vdash (\hh, \vi) \csat \opset: (\hh', \vi')$. 
Also choose an arbitrary $\cl$, a transaction identifier $\txid \in \nextTxId(\hh, \cl)$, 
and an abstract execution $\aexec$ such that $\hh_{\aexec} = \hh$ and 
\( I(\aexec, \cl) =  \emptyset \subseteq \Tx(\hh, \vi) \).
Let \( \aexec' = \extend(\aexec, \txid, \Tx(\mkvs, \vi), \f ) \).
Note that since the invariant is empty set, it remains to prove the following:
\[
    \begin{array}{@{}l@{}}
        \fora{ \txid' } \txid' \toEdge{\WW_{\aexec'}} \txid \implies \txid' \in \Tx(\mkvs, \vi)
    \end{array}
\]
Assume a transaction \( \txid' \) that writes to a key \( \ke \) as \( \txid \), \ie \( \txid' \toEdge{\WW_{\aexec'}} \txid \).
Since that \( \txid' \) is a transaction already existing in \( \mkvs\),
we have \( \WTx(\mkvs(\ke, i)) = \txid' \) for some index \( i \).
By the execution test of \( \UA \), we know \( i \in \vi(\ke) \) therefore \( \txid' \in \Tx(\mkvs, \vi) \).

The execution test $\ET_\UA$ is complete with respect to 
the axiomatic specification \( (\RP_{\LWW}, \Set{\lambda \aexec. \WW_\aexec }) \).
Assume i-\emph{th} transaction \( \txid_i \) in the arbitrary order,
and let view \( \vi_{i} = \getView(\aexec, \VIS^{-1}_{\aexec}(\txid_{i}) ) \).
We also pick any final view such that \( \vi'_{i} \subseteq \getView(\aexec, (\AR^{-1}_{\aexec})?(\txid_{i}) ) \).
Note that there is nothing to prove for \( \vi'_i \),
so it is sufficient to prove the following:
\[
    \fora{\ke} (\otW, \ke, \stub) \in \TtoOp{T}_{\aexec}(\txid_{i}) 
    \implies 
    \fora{j : 0 \leq j < \abs{\mkvs_{\cut(\aexec, i-1)}(\ke)}} j \in \vi_i(\ke)
\]
Let consider a key \( \ke \) that have been overwritten by the transaction \( \txid_i \).
By the constraint of \( \aexec \) that \( \WW_\aexec \subseteq \VIS_\aexec \),
for any transaction \( \txid \) that writes to the same key \( \ke \) and committed before \( \txid_i \), 
they are included in the visible set \(\txid \in \VIS^{-1}_{\aexec}(\txid_{i}) \).
Note that \( \txid \toEdge{\WW_\aexec} \txid_i \implies \txid \toEdge{\AR_\aexec} \txid_i \implies \txid \in \mkvs_{\cut(\aexec,i-1)}\).
Since that the transaction \( \txid \) write to the key \( \ke \),
it means \( \WTx(\mkvs_{\cut(\aexec, i-1)}(\ke,j)) = \txid \) for some index \( j \).
Then by the definition of \( \getView \), we have \( j \in \vi_i(\ke)\).

%\subsection{Consistent Prefix} 
%\begin{figure}
%\begin{center}
%\begin{tabular}{|@{}c|c@{}|}
%\hline
%\begin{tikzpicture}[font=\large]

%\begin{pgfonlayer}{foreground}
%%Uncomment line below for help lines
%%\draw[help lines] grid(5,4);

%%Location x
%\node(locx) at (1,3) {$[\loc_x] \mapsto$};

%\matrix(locxcells) [version list, text width=7mm, anchor=west]
   %at ([xshift=10pt]locx.east) {
 %{a} & $T_0$ \\
  %{a} & $\emptyset$ \\
%};
%\node[version node, fit=(locxcells-1-1) (locxcells-2-1), fill=white, inner sep= 0cm, font=\Large] (locx-v0) {$0$};
%%\node[version node, fit=(locxcells-1-3) (locxcells-2-3), fill=white, inner sep=0cm, font=\Large] (locx-v1) {$1$};

%%Location y
%\path (locx.south) + (0,-1.5) node (locy) {$[\loc_y] \mapsto$};
%\matrix(locycells) [version list, text width=7mm, anchor=west]
   %at ([xshift=10pt]locy.east) {
 %{a} & $T_0$ \\
  %{a} & $\emptyset$ \\
%};
%\node[version node, fit=(locycells-1-1) (locycells-2-1), fill=white, inner sep= 0cm, font=\Large] (locy-v0) {$0$};
%%\node[version node, fit=(locycells-1-3) (locycells-2-3), fill=white, inner sep=0cm, font=\Large] (locy-v1) {$1$};

%% \draw[-, red, very thick, rounded corners] ([xshift=-5pt, yshift=5pt]locx-v1.north east) |- 
%%  ($([xshift=-5pt,yshift=-5pt]locx-v1.south east)!.5!([xshift=-5pt, yshift=5pt]locy-v0.north east)$) -| ([xshift=-5pt, yshift=5pt]locy-v0.south east);

%%blue view - I should  check whether I can use pgfkeys to just declare the list of locations, and then add the view automatically.
%\draw[-, blue, very thick, rounded corners=10pt]
 %([xshift=-3pt, yshift=20pt]locx-v0.north east) node (tid1start) {} -- 
%% ([xshift=-2pt, yshift=-5pt]locx-v0.south east) --
%% ([xshift=-2pt, yshift=5pt]locy-v0.north east) -- 
 %([xshift=-3pt, yshift=-5pt]locy-v0.south east);
 
 %\path (tid1start) node[anchor=south, rectangle, fill=blue!20, draw=blue, font=\small, inner sep=1pt] {$\thid_1$};

%%red view
%\draw[-, red, very thick, rounded corners = 10pt]
 %([xshift=-16pt, yshift=5pt]locx-v0.north east) node (tid2start) {}-- 
%% ([xshift=-8pt, yshift=-5pt]locx-v0.south east) --
%% ([xshift=-8pt, yshift=5pt]locy-v0.north east) -- 
 %([xshift=-16pt, yshift=-5pt]locy-v0.south east) node {};
 
%\path (tid2start) node[anchor=south, rectangle, fill=red!20, draw=red, font=\small, inner sep=1pt] {$\thid_2$};

%%%Stack for threads tid_1 and tid_2
%%
%%\draw[-, dashed] let 
%%   \p1 = ([xshift=0pt]locy.west),
%%   \p2 = ([yshift=-5pt]locycells.south),
%%   \p3 = ([xshift=10pt]locycells.east) in
%%   (\x1, \y2) -- (\x3, \y2);
%%   
%%\matrix(stacks) [
%%   matrix of nodes,
%%   anchor=north, 
%%   text=blue, 
%%   font=\normalsize, 
%%   row 1/.style = {text = blue}, 
%%   row 2/.style = {text = red}, 
%%   text width= 13mm ] 
%%   at ([xshift=-10pt,yshift=-8pt]locycells.south) {
%%   $\thid_1:$ & $\retvar = 0$\\
%%   $\thid_2:$ & $\retvar = 0$\\
%%   };
%\end{pgfonlayer}
%\end{tikzpicture} 
%%
%&
%%
%\begin{tikzpicture}[font=\large]

%\begin{pgfonlayer}{foreground}
%%Uncomment line below for help lines
%%\draw[help lines] grid(5,4);

%%Location x
%\node(locx) at (1,3) {$[\loc_x] \mapsto$};

%\matrix(locxcells) [version list, text width=7mm, anchor=west]
   %at ([xshift=10pt]locx.east) {
 %{a} & $T_0$ & {a} & $\tsid_1$\\
  %{a} & $\emptyset$ & {a} & $\emptyset$ \\
%};
%\node[version node, fit=(locxcells-1-1) (locxcells-2-1), fill=white, inner sep= 0cm, font=\Large] (locx-v0) {$0$};
%\node[version node, fit=(locxcells-1-3) (locxcells-2-3), fill=white, inner sep=0cm, font=\Large] (locx-v1) {$1$};

%%Location y
%\path (locx.south) + (0,-1.5) node (locy) {$[\loc_y] \mapsto$};
%\matrix(locycells) [version list, text width=7mm, anchor=west]
   %at ([xshift=10pt]locy.east) {
 %{a} & $T_0$ \\
  %{a} & $\emptyset$ \\
%};
%\node[version node, fit=(locycells-1-1) (locycells-2-1), fill=white, inner sep= 0cm, font=\Large] (locy-v0) {$0$};
%%\node[version node, fit=(locycells-1-3) (locycells-2-3), fill=white, inner sep=0cm, font=\Large] (locy-v1) {$1$};

%% \draw[-, red, very thick, rounded corners] ([xshift=-5pt, yshift=5pt]locx-v1.north east) |- 
%%  ($([xshift=-5pt,yshift=-5pt]locx-v1.south east)!.5!([xshift=-5pt, yshift=5pt]locy-v0.north east)$) -| ([xshift=-5pt, yshift=5pt]locy-v0.south east);

%%blue view - I should  check whether I can use pgfkeys to just declare the list of locations, and then add the view automatically.
%\draw[-, blue, very thick, rounded corners=10pt]
 %([xshift=-3pt, yshift=20pt]locx-v1.north east) node (tid1start) {} -- 
 %([xshift=-3pt, yshift=-5pt]locx-v1.south east) --
 %([xshift=-3pt, yshift=5pt]locy-v0.north east) -- 
 %([xshift=-3pt, yshift=-5pt]locy-v0.south east);
 
 %\path (tid1start) node[anchor=south, rectangle, fill=blue!20, draw=blue, font=\small, inner sep=1pt] {$\thid_1$};

%%red view
%\draw[-, red, very thick, rounded corners = 10pt]
 %([xshift=-16pt, yshift=5pt]locx-v0.north east) node (tid2start) {}-- 
%% ([xshift=-8pt, yshift=-5pt]locx-v0.south east) --
%% ([xshift=-8pt, yshift=5pt]locy-v0.north east) -- 
 %([xshift=-16pt, yshift=-5pt]locy-v0.south east) node {};
 
%\path (tid2start) node[anchor=south, rectangle, fill=red!20, draw=red, font=\small, inner sep=1pt] {$\thid_2$};

%%%Stack for threads tid_1 and tid_2
%%
%%\draw[-, dashed] let 
%%   \p1 = ([xshift=0pt]locy.west),
%%   \p2 = ([yshift=-5pt]locycells.south),
%%   \p3 = ([xshift=10pt]locycells.east) in
%%   (\x1, \y2) -- (\x3, \y2);
%%   
%%\matrix(stacks) [
%%   matrix of nodes,
%%   anchor=north, 
%%   text=blue, 
%%   font=\normalsize, 
%%   row 1/.style = {text = blue}, 
%%   row 2/.style = {text = red}, 
%%   text width= 13mm ] 
%%   at ([xshift=-10pt,yshift=-8pt]locycells.south) {
%%   $\thid_1:$ & $\retvar = 0$\\
%%   $\thid_2:$ & $\retvar = 0$\\
%%   };
%\end{pgfonlayer}
%\end{tikzpicture}
%\\
%{\small(a)} & {\small(b)}\\
%\hline

%\begin{pgfonlayer}{foreground}
%%Uncomment line below for help lines
%%\draw[help lines] grid(5,4);

%%Location x
%\node(locx) at (1,3) {$[\loc_x] \mapsto$};

%\matrix(locxcells) [version list, text width=7mm, anchor=west]
   %at ([xshift=10pt]locx.east) {
 %{a} & $T_0$ & {a} & $\tsid_1$\\
  %{a} & $\emptyset$ & {a} & $\emptyset$ \\
%};
%\node[version node, fit=(locxcells-1-1) (locxcells-2-1), fill=white, inner sep= 0cm, font=\Large] (locx-v0) {$0$};
%\node[version node, fit=(locxcells-1-3) (locxcells-2-3), fill=white, inner sep=0cm, font=\Large] (locx-v1) {$1$};

%%Location y
%\path (locx.south) + (0,-1.5) node (locy) {$[\loc_y] \mapsto$};
%\matrix(locycells) [version list, text width=7mm, anchor=west]
   %at ([xshift=10pt]locy.east) {
 %{a} & $T_0$ & {a} & $\tsid_2$ \\
  %{a} & $\emptyset$ & {a} & $\emptyset$\\
%};
%\node[version node, fit=(locycells-1-1) (locycells-2-1), fill=white, inner sep= 0cm, font=\Large] (locy-v0) {$0$};
%\node[version node, fit=(locycells-1-3) (locycells-2-3), fill=white, inner sep=0cm, font=\Large] (locy-v1) {$1$};

%% \draw[-, red, very thick, rounded corners] ([xshift=-5pt, yshift=5pt]locx-v1.north east) |- 
%%  ($([xshift=-5pt,yshift=-5pt]locx-v1.south east)!.5!([xshift=-5pt, yshift=5pt]locy-v0.north east)$) -| ([xshift=-5pt, yshift=5pt]locy-v0.south east);

%%blue view - I should  check whether I can use pgfkeys to just declare the list of locations, and then add the view automatically.
%\draw[-, blue, very thick, rounded corners=10pt]
 %([xshift=-3pt, yshift=20pt]locx-v1.north east) node (tid1start) {} -- 
 %([xshift=-3pt, yshift=-5pt]locx-v1.south east) --
 %([xshift=-3pt, yshift=5pt]locy-v0.north east) -- 
 %([xshift=-3pt, yshift=-5pt]locy-v0.south east);
 
 %\path (tid1start) node[anchor=south, rectangle, fill=blue!20, draw=blue, font=\small, inner sep=1pt] {$\thid_1$};

%%red view
%\draw[-, red, very thick, rounded corners = 10pt]
 %([xshift=-16pt, yshift=5pt]locx-v0.north east) node (tid2start) {}-- 
 %([xshift=-16pt, yshift=-5pt]locx-v0.south east) --
 %([xshift=-16pt, yshift=5pt]locy-v1.north east) -- 
 %([xshift=-16pt, yshift=-5pt]locy-v1.south east) node {};
 
%\path (tid2start) node[anchor=south, rectangle, fill=red!20, draw=red, font=\small, inner sep=1pt] {$\thid_2$};

%%%Stack for threads tid_1 and tid_2
%%
%%\draw[-, dashed] let 
%%   \p1 = ([xshift=0pt]locy.west),
%%   \p2 = ([yshift=-5pt]locycells.south),
%%   \p3 = ([xshift=10pt]locycells.east) in
%%   (\x1, \y2) -- (\x3, \y2);
%%   
%%\matrix(stacks) [
%%   matrix of nodes,
%%   anchor=north, 
%%   text=blue, 
%%   font=\normalsize, 
%%   row 1/.style = {text = blue}, 
%%   row 2/.style = {text = red}, 
%%   text width= 13mm ] 
%%   at ([xshift=-10pt,yshift=-8pt]locycells.south) {
%%   $\thid_1:$ & $\retvar = 0$\\
%%   $\thid_2:$ & $\retvar = 0$\\
%%   };
%\end{pgfonlayer}
%\end{tikzpicture}
%%
%&
%%
%\begin{tikzpicture}[font=\large]

%\begin{pgfonlayer}{foreground}
%%Uncomment line below for help lines
%%\draw[help lines] grid(5,4);

%%Location x
%\node(locx) at (1,3) {$[\loc_x] \mapsto$};

%\matrix(locxcells) [version list, text width=7mm, anchor=west]
   %at ([xshift=10pt]locx.east) {
 %{a} & $T_0$ & {a} & $\tsid_1$\\
  %{a} & $\{\tsid_4\}$ & {a} & $\emptyset$ \\
%};
%\node[version node, fit=(locxcells-1-1) (locxcells-2-1), fill=white, inner sep= 0cm, font=\Large] (locx-v0) {$0$};
%\node[version node, fit=(locxcells-1-3) (locxcells-2-3), fill=white, inner sep=0cm, font=\Large] (locx-v1) {$1$};

%%Location y
%\path (locx.south) + (0,-1.5) node (locy) {$[\loc_y] \mapsto$};
%\matrix(locycells) [version list, text width=7mm, anchor=west]
   %at ([xshift=10pt]locy.east) {
 %{a} & $T_0$ & {a} & $\tsid_2$ \\
  %{a} & $\{\tsid_3\}$ & {a} & $\emptyset$\\
%};
%\node[version node, fit=(locycells-1-1) (locycells-2-1), fill=white, inner sep= 0cm, font=\Large] (locy-v0) {$0$};
%\node[version node, fit=(locycells-1-3) (locycells-2-3), fill=white, inner sep=0cm, font=\Large] (locy-v1) {$1$};

%% \draw[-, red, very thick, rounded corners] ([xshift=-5pt, yshift=5pt]locx-v1.north east) |- 
%%  ($([xshift=-5pt,yshift=-5pt]locx-v1.south east)!.5!([xshift=-5pt, yshift=5pt]locy-v0.north east)$) -| ([xshift=-5pt, yshift=5pt]locy-v0.south east);

%%blue view - I should  check whether I can use pgfkeys to just declare the list of locations, and then add the view automatically.
%\draw[-, blue, very thick, rounded corners=10pt]
 %([xshift=-3pt, yshift=20pt]locx-v1.north east) node (tid1start) {} -- 
%% ([xshift=-3pt, yshift=-5pt]locx-v1.south east) --
%% ([xshift=-3pt, yshift=5pt]locy-v0.north east) -- 
 %([xshift=-3pt, yshift=-5pt]locy-v1.south east);
 
 %\path (tid1start) node[anchor=south, rectangle, fill=blue!20, draw=blue, font=\small, inner sep=1pt] {$\thid_1$};

%%red view
%\draw[-, red, very thick, rounded corners = 10pt]
 %([xshift=-16pt, yshift=5pt]locx-v1.north east) node (tid2start) {}-- 
%% ([xshift=-16pt, yshift=-5pt]locx-v0.south east) --
%% ([xshift=-16pt, yshift=5pt]locy-v1.north east) -- 
 %([xshift=-16pt, yshift=-5pt]locy-v1.south east) node {};
 
%\path (tid2start) node[anchor=south, rectangle, fill=red!20, draw=red, font=\small, inner sep=1pt] {$\thid_2$};

%%%Stack for threads tid_1 and tid_2
%%
%%\draw[-, dashed] let 
%%   \p1 = ([xshift=0pt]locy.west),
%%   \p2 = ([yshift=-5pt]locycells.south),
%%   \p3 = ([xshift=10pt]locycells.east) in
%%   (\x1, \y2) -- (\x3, \y2);
%%   
%%\matrix(stacks) [
%%   matrix of nodes,
%%   anchor=north, 
%%   text=blue, 
%%   font=\normalsize, 
%%   row 1/.style = {text = blue}, 
%%   row 2/.style = {text = red}, 
%%   text width= 13mm ] 
%%   at ([xshift=-10pt,yshift=-8pt]locycells.south) {
%%   $\thid_1:$ & $\retvar = 0$\\
%%   $\thid_2:$ & $\retvar = 0$\\
%%   };
%\end{pgfonlayer}
%\end{tikzpicture}
%\\
%{\small(c)} & {\small(d)}\\
%\hline
%\begin{tikzpicture}[font=\large]

%\begin{pgfonlayer}{foreground}
%%Uncomment line below for help lines
%%\draw[help lines] grid(5,4);

%%Location x
%\node(locx) at (1,3) {$[\loc_x] \mapsto$};

%\matrix(locxcells) [version list, text width=7mm, anchor=west]
   %at ([xshift=10pt]locx.east) {
 %{a} & $T_0$ & {a} & $\tsid_1$\\
  %{a} & $\emptyset$ & {a} & $\emptyset$ \\
%};
%\node[version node, fit=(locxcells-1-1) (locxcells-2-1), fill=white, inner sep= 0cm, font=\Large] (locx-v0) {$0$};
%\node[version node, fit=(locxcells-1-3) (locxcells-2-3), fill=white, inner sep=0cm, font=\Large] (locx-v1) {$1$};

%%Location y
%\path (locx.south) + (0,-1.5) node (locy) {$[\loc_y] \mapsto$};
%\matrix(locycells) [version list, text width=7mm, anchor=west]
   %at ([xshift=10pt]locy.east) {
 %{a} & $T_0$ & {a} & $\tsid_2$ \\
  %{a} & $\emptyset$ & {a} & $\emptyset$\\
%};
%\node[version node, fit=(locycells-1-1) (locycells-2-1), fill=white, inner sep= 0cm, font=\Large] (locy-v0) {$0$};
%\node[version node, fit=(locycells-1-3) (locycells-2-3), fill=white, inner sep=0cm, font=\Large] (locy-v1) {$1$};

%% \draw[-, red, very thick, rounded corners] ([xshift=-5pt, yshift=5pt]locx-v1.north east) |- 
%%  ($([xshift=-5pt,yshift=-5pt]locx-v1.south east)!.5!([xshift=-5pt, yshift=5pt]locy-v0.north east)$) -| ([xshift=-5pt, yshift=5pt]locy-v0.south east);

%%blue view - I should  check whether I can use pgfkeys to just declare the list of locations, and then add the view automatically.
%\draw[-, blue, very thick, rounded corners=10pt]
 %([xshift=-3pt, yshift=20pt]locx-v1.north east) node (tid1start) {} -- 
 %([xshift=-3pt, yshift=-5pt]locx-v1.south east) --
 %([xshift=-3pt, yshift=5pt]locy-v0.north east) -- 
 %([xshift=-3pt, yshift=-5pt]locy-v0.south east);
 
 %\path (tid1start) node[anchor=south, rectangle, fill=blue!20, draw=blue, font=\small, inner sep=1pt] {$\thid_1$};

%%red view
%\draw[-, red, very thick, rounded corners = 10pt]
 %([xshift=-16pt, yshift=5pt]locx-v1.north east) node (tid2start) {}-- 
%% ([xshift=-16pt, yshift=-5pt]locx-v0.south east) --
%% ([xshift=-16pt, yshift=5pt]locy-v1.north east) -- 
 %([xshift=-16pt, yshift=-5pt]locy-v1.south east) node {};
 
%\path (tid2start) node[anchor=south, rectangle, fill=red!20, draw=red, font=\small, inner sep=1pt] {$\thid_2$};

%%%Stack for threads tid_1 and tid_2
%%
%%\draw[-, dashed] let 
%%   \p1 = ([xshift=0pt]locy.west),
%%   \p2 = ([yshift=-5pt]locycells.south),
%%   \p3 = ([xshift=10pt]locycells.east) in
%%   (\x1, \y2) -- (\x3, \y2);
%%   
%%\matrix(stacks) [
%%   matrix of nodes,
%%   anchor=north, 
%%   text=blue, 
%%   font=\normalsize, 
%%   row 1/.style = {text = blue}, 
%%   row 2/.style = {text = red}, 
%%   text width= 13mm ] 
%%   at ([xshift=-10pt,yshift=-8pt]locycells.south) {
%%   $\thid_1:$ & $\retvar = 0$\\
%%   $\thid_2:$ & $\retvar = 0$\\
%%   };
%\end{pgfonlayer}
%\end{tikzpicture}
%&
%\begin{tikzpicture}[font=\large]

%\begin{pgfonlayer}{foreground}
%%Uncomment line below for help lines
%%\draw[help lines] grid(5,4);

%%Location x
%\node(locx) at (1,3) {$[\loc_x] \mapsto$};

%\matrix(locxcells) [version list, text width=7mm, anchor=west]
   %at ([xshift=10pt]locx.east) {
 %{a} & $T_0$ & {a} & $\tsid_1$\\
  %{a} & $\emptyset$ & {a} & $\{\tsid_4\}$ \\
%};
%\node[version node, fit=(locxcells-1-1) (locxcells-2-1), fill=white, inner sep= 0cm, font=\Large] (locx-v0) {$0$};
%\node[version node, fit=(locxcells-1-3) (locxcells-2-3), fill=white, inner sep=0cm, font=\Large] (locx-v1) {$1$};

%%Location y
%\path (locx.south) + (0,-1.5) node (locy) {$[\loc_y] \mapsto$};
%\matrix(locycells) [version list, text width=7mm, anchor=west]
   %at ([xshift=10pt]locy.east) {
 %{a} & $T_0$ & {a} & $\tsid_2$ \\
  %{a} & $\{\tsid_3\}$ & {a} & $\emptyset$\\
%};
%\node[version node, fit=(locycells-1-1) (locycells-2-1), fill=white, inner sep= 0cm, font=\Large] (locy-v0) {$0$};
%\node[version node, fit=(locycells-1-3) (locycells-2-3), fill=white, inner sep=0cm, font=\Large] (locy-v1) {$1$};

%% \draw[-, red, very thick, rounded corners] ([xshift=-5pt, yshift=5pt]locx-v1.north east) |- 
%%  ($([xshift=-5pt,yshift=-5pt]locx-v1.south east)!.5!([xshift=-5pt, yshift=5pt]locy-v0.north east)$) -| ([xshift=-5pt, yshift=5pt]locy-v0.south east);

%%blue view - I should  check whether I can use pgfkeys to just declare the list of locations, and then add the view automatically.
%\draw[-, blue, very thick, rounded corners=10pt]
 %([xshift=-3pt, yshift=20pt]locx-v1.north east) node (tid1start) {} -- 
%% ([xshift=-3pt, yshift=-5pt]locx-v1.south east) --
%% ([xshift=-3pt, yshift=5pt]locy-v0.north east) -- 
 %([xshift=-3pt, yshift=-5pt]locy-v1.south east);
 
 %\path (tid1start) node[anchor=south, rectangle, fill=blue!20, draw=blue, font=\small, inner sep=1pt] {$\thid_1$};

%%red view
%\draw[-, red, very thick, rounded corners = 10pt]
 %([xshift=-16pt, yshift=5pt]locx-v1.north east) node (tid2start) {}-- 
%% ([xshift=-16pt, yshift=-5pt]locx-v0.south east) --
%% ([xshift=-16pt, yshift=5pt]locy-v1.north east) -- 
 %([xshift=-16pt, yshift=-5pt]locy-v1.south east) node {};
 
%\path (tid2start) node[anchor=south, rectangle, fill=red!20, draw=red, font=\small, inner sep=1pt] {$\thid_2$};

%%%Stack for threads tid_1 and tid_2
%%
%%\draw[-, dashed] let 
%%   \p1 = ([xshift=0pt]locy.west),
%%   \p2 = ([yshift=-5pt]locycells.south),
%%   \p3 = ([xshift=10pt]locycells.east) in
%%   (\x1, \y2) -- (\x3, \y2);
%%   
%%\matrix(stacks) [
%%   matrix of nodes,
%%   anchor=north, 
%%   text=blue, 
%%   font=\normalsize, 
%%   row 1/.style = {text = blue}, 
%%   row 2/.style = {text = red}, 
%%   text width= 13mm ] 
%%   at ([xshift=-10pt,yshift=-8pt]locycells.south) {
%%   $\thid_1:$ & $\retvar = 0$\\
%%   $\thid_2:$ & $\retvar = 0$\\
%%   };
%\end{pgfonlayer}
%\end{tikzpicture}
%\\
%{\small(e)} & {\small(f)}\\
%\hline
%\end{tabular}
%\end{center}
%\caption{Configurations obtained throughout an execution of 
%$\prog_4$.}
%\label{fig:cp.exec}
%\end{figure}

The next consistency model that we illustrate is \emph{consistent prefix}. 
It ensures that once a thread observes the effects of some transaction $\txid$, it also observes all the transactions that were committed before $\txid$. 
\sx{why different locations}
Another way to express this property is that two different transactions never observe the updates to different addresses in a different order.

Consider the program $\prog_4$ below,
 \[
    \prog_4 \equiv  
    \left( 
    \begin{session} 
        \begin{array}{@{}c || c @{}}
            \begin{session} 
            \txid_1 : 
            \begin{transaction}
                \pmutate{\vx}{1};\\
            \end{transaction}; \\
            
            \txid_3 :
            \begin{transaction}
              	\pderef{\pvar{a}}{\vy};\\
              	\pifs{\pvar{a}=0}\\
                    \quad \passign{\retvar}{\Large \frownie{}} 
                \pife 
            \end{transaction}
            \end{session}
            &
            \begin{session}
            \txid_2 :
            \begin{transaction}
                \pmutate{\vy}{1};\\
            \end{transaction} ; \\
            
            \txid_4 :
            \begin{transaction}
              	\pderef{\pvar{a}}{\vx};\\
              	\pifs{\pvar{a}=0}\\
              		\quad \passign{\retvar}{\Large \frownie{}} 
                \pife
            \end{transaction}
            \end{session}
        \end{array}
    \end{session}
    \right)
 \]

We argue that, under $\mathsf{RA}$, it is possible to obtain an execution 
of program $\prog_4$ where both $\ptrans{\trans_2}$ and $\ptrans{\trans_4}$ 
return value ${\Large \frownie{}}$.
Such an execution can be summarised as follows: 

\begin{itemize}
\item initially, $\thid_1$ executes transaction $\ptrans{\trans_1}$, 
leading to the configuration of Figure \ref{fig:cp.exec}(b), and the program 
$\ptrans{\trans_2} \ppar \ptrans{\trans_3} ; \ptrans{\trans_4}$ to be 
executed, 
\item then $\thid_2$ executes transaction $\ptrans{\trans_3}$, leading 
to the configuration of Figure \ref{fig:cp.exec}(c); the remaining 
program to be executed is $\ptrans{\trans_3} \ppar \ptrans{\trans_4}$, 
\item without changing its view, $\thid_1$ executes transaction $\ptrans{\trans_2}$. 
The code $\trans_2$ is executed using the heap $[ [\loc_x] \mapsto 1, [\loc_y] \mapsto 0]$ 
as a snapshot; this means that, by executing $\trans_2$, the variable $\retvar$ of the thread-local 
stack of $\thid_1$ is set to ${\Large \frownie{}}$. Next, the thread $\thid_2$ executes $\ptrans{\trans_4}$ 
without altering its view. Similarly to the execution of $\ptrans{\trans_2}$ in $\thid_1$, this causes the 
$\retvar$ variable of the thread-local stack of $\thid_2$ to be set to ${\Large \frownie{}}$. At this point, 
the final configuration of the program is the one given in Figure \ref{fig:cp.exec}(d).
\end{itemize}

\sx{What is different update ?}
To avoid different threads to observe different updates in different orders, we impose a constraint known as \emph{consistent prefix}.
\sx{ why centralised? 
in a centralised database, where 
transactions have a start and a commit point, 
}
Assuming a transaction has a start and commit point, \emph{consistent prefix} requires that if a transaction $\txid_1$ observes the effects of another transaction $\txid_2$, then it must observe the effects of any other transaction that committed before $\txid_2$.
In the history heaps framework, transactions are executed in a single step in the semantics,
\sx{
this constraint can be formalised as follows: 
\emph{If a transaction $\tsid_1$ observes the effects of another transaction $\tsid_2$, then it must 
observe the effects of any transaction that committed before $\tsid_2$.}
}
In the history heaps framework, transactions are executed in a single step and the step corresponds to the commit order.
Thus, upon the read atomic (\defref{def:readatomic}), \emph{consistent prefix} requires once a thread commits a transaction, it pushes the view to be up-to-date with the state of the database, so the following transactions from the thread will observe the effects of all transactions committed before \( \txid \) included.
\sx{It is easy to see the definition match the intuition when inside a thread, but not easy to see cross-thread. There is something I feel subtle but dont know what it is and I feel it is actually already included in the definition.}

\sx{ dont understand:
In the history heaps framework, transactions are executed in a single step; 
however, one may think of the order in which transactions execute in our 
operational semantics to be consistent with the order in which 
transactions commit (this correspondence will be made precise later in 
the document, when we will relate executions in our operational semantics 
to abstract executions used in the declarative style for specifications of 
consistency models). By requiring that, after a thread $\thid$ executes 
a transaction $\tsid$, it pushes its own view to be up-to-date with the state of 
the system, we model the fact that any future transaction executed 
by $\thid$ will observe the effects of anything that committed before 
$\tsid$ (included).
}

\begin{defn}[consistent prefix]
\label{def:consistent-prefix}
The \emph{consistent prefix} is stronger than read atomic (\defref{def:readatomic}) by further requiring the view after pushes to the latest for all addresses, 
\[
\begin{rclarray}
    (\hh, \vi) \csat[\mathsf{CP}] \opset: \vi' & \defeq &  (\hh, \vi) \csat[\mathsf{RA}] \opset: \vi' \land \fora{\addr \in \dom(\hh)} \vi'(\addr) = \left| \hh(\addr) \right| \\
\end{rclarray}
\]
\sx{
We say that $(hh, V) \triangleright_{\mathsf{CP}} \mathcal{O}: V'$ 
if and only if $(\hh, V) \triangleright_{\mathsf{RA}} \mathcal{O}: V'$, 
and for any location $[\loc_x]$, $V'([\loc_x]) = \lvert \mathsf{HHeapUpdate}(\hh, V, \mathcal{O}) \rvert -1$. 
}
\end{defn}


\sx{did not read} 
Consider again th program $\prog_4$, ths time to be executed 
under $\mathsf{CP}$. We argue that in this case it is not possible to have both 
threads $\thid_1$ and $\thid_2$ to set the value of $\retvar$ to ${\Large \frownie{}}$. 
The initial configuration in which the program $\prog_4$ is executed is 
the one depicted in Figure \ref{fig:cp.exec}(a). Initially, either thread $\thid_1$ executes 
the code $\ptrans{\trans_1}$, or thread $\thid_2$ executes transaction $\ptrans{\trans_3}$; without 
loss of generality, we consider the former option (the case in which $\thid_2$ executes 
first is symmetric). Upon executing the code $\ptrans{\trans_1}$, we obtain the 
configuration of Figure \ref{fig:cp.exec}(b), with the program $\ptrans{\trans_2} \ppar (\ptrans{\trans_3} ; \ptrans{\trans_4})$ 
to be executed next.
At this point, note that under $\mathsf{CP}$ it is not possible for $\thid_2$ to execute $\ptrans{\trans_3}$ and obtain 
the configuration of Figure \ref{fig:cp.exec}(c) as a result. This is because, in $\mathsf{CP}$, we require the view of $\thid_2$ 
\textbf{after} executing $\ptrans{\trans_3}$ to be up-to-date, i.e. to point to the last location of each version. That is, 
assuming that $\thid_2$ executes $\ptrans{\trans_3}$ next, we obtain the configuration of Figure \ref{fig:cp.exec}(e). 
From this point on, whenever $\thid_2$ will execute transaction $\ptrans{\trans_4}$, it will read value $1$ for 
location $[\loc_x]$, hence it won't be able to set the value of $\retvar$ to ${\Large \frownie{}}$. One possible 
final configuration for the program is given in Figure \ref{fig:cp.exec}(f).

%\begin{definition}
%$(\hh, V, \mathcal{V}) \triangleright_{\mathsf{CP}} \mathcal{O}$ iff 
%$(\hh, V, \mathcal{V}) \triangleright_{\mathsf{RA}} \mathcal{O}$, 
%and for any $V' \in \mathcal{V}$, either $V' \sqsubseteq V$ or 
%$V \sqsubseteq V'$.
%\end{definition}
%\ac{I found that this is a very easy way to encode consistent 
%prefix. In English, a thread can execute a transaction if its view does 
%not cross with the views of any other thread.}

%\ac{Very Important - note to self: It seems that the condition of 
%requiring that views do not cross before executing a transaction 
%does not suffice to model snapshot isolation. In fact, it seems that 
%consistent prefix (when transaction are limited to either one read 
%or one write) coincides with TSO.
%Update, it seems that the condition that I need for Snapshot Isolation 
%(besides write confict detection) is that, after you execute a transaction, 
%you bring your view up-to-date. (So here I have to concede that I was wrong, 
%and the state of the view after you execute a transaction is actually important).\\
%
%}

%\subsection{Parallel Snapshot Isolation and Snapshot Isolation}
\sx{Need some citations here, what is geo-replicated, distributed database vs distributed system}
\emph{Snapshot Isolation} (SI) is a consistency model that has been widely employed in both centralised and distributed databases. 
Because snapshot isolation does not scale well to geo-replicated and distributed systems, a weaker model called \emph{Parallel Snapshot Isolation} (PSI) has been recently proposed. 

%\sx{
%Both SI and PSI can be specified in the history heap framework by combining the consistency models that we have already introduced. 
%In short, SI combines atomic visibility and the snapshot monotonicity property from consistent prefix property (if a transaction $\tsid_1$ 
%observes the effects of another transaction $\tsid_2$, then it also observes 
%the effects of any transaction that committed before), and the write-conflict 
%detection from Update Atomic (two committing transactions do not write 
%concurrently to the same location). In contrast, PSI only requires atomic 
%visibility, causal consistency and write-conflict detection. Formally, we have 
%the following:
%}
Both SI and PSI can be specified in the history heap framework by combining the consistency models that we have already introduced. 
In short, SI combines consistent prefix property and update atomic, while PSI only requires causal consistency and update atomic.
\begin{definition}
The \emph{parallel snapshot isolation} combines causal consistency (\cref{def:causal})  and update atomic (\cref{def:update-atomic}): $\mathsf{PSI} = \mathsf{CC} \cap \mathsf{UA}$.
The {snapshot isolation} combines consistent prefix (\cref{def:consistent-prefix}) and update atomic (\cref{def:update-atomic}): $\mathsf{SI} = \mathsf{CP} \cap \mathsf{UA}$.
\end{definition}



%\subsection{Serialisability \( \SER \)}
\label{sec:sound-complete-ser}

The execution test $\ET_\SER$ is sound with respect to the axiomatic definition 
\[ 
    (\RP_{\LWW}, \Set{\lambda \aexec. \AR })
\]
We pick the invariant as \( I( \aexec, \cl ) = \emptyset \), given the fact of no constraint on the view after update.
Assume a kv-store $\mkvs$, an initial and a final view $\vi, \vi'$  a fingerprint $\fp$ 
such that $\ET_{\SER} \vdash (\mkvs, \vi) \csat \fp: (\mkvs',\vi')$. 
Also choose an arbitrary $\cl$, a transaction identifier $\txid \in \nextTxid(\mkvs, \cl)$, 
and an abstract execution $\aexec$ such that $\mkvs_{\aexec} = \mkvs$ and 
\( I(\aexec, \cl) =  \emptyset \subseteq \Tx[\mkvs, \vi] \).
Let \( \aexec' = \extend[\aexec, \txid, \Tx[\mkvs, \vi], \fp] \).
Note that since the invariant is empty set, it remains to prove there exists a set of read-only transactions \( \txidset_\rd \) such that:
\[
    \begin{array}{@{}l@{}}
        \fora{ \txid' } 
        \txid' \toEDGE{\AR_{\aexec'}} \txid \implies \txid' \in \Tx[\mkvs, \vi] \cup \txidset_\rd
    \end{array}
\]
Since the abstract execution satisfies the constraint for \( \SER \), \ie \( \AR \subseteq \VIS \), we know \( \AR = \VIS \).
Since \( \Tx[\mkvs, \vi]  \) contains all transactions that write at least a key, 
we can pick a \( \txidset_\rd \) such that \( \Tx[\mkvs, \vi] \cup \txidset_\rd = \txidset_\aexec\),
which gives us the proof.


The execution test $\ET_\UA$ is complete with respect to the axiomatic definition \( (\RP_{\LWW}, \Set{\lambda \aexec. \AR_\aexec }) \).
Assume i-\emph{th} transaction \( \txid_i \) in the arbitrary order,
and let view \( \vi_{i} = \getView[\aexec, \VIS^{-1}_{\aexec}(\txid_{i})] \).
We also pick any final view such that \( \vi'_{i} \subseteq \getView[\aexec, (\AR^{-1}_{\aexec})\rflx(\txid_{i})] \).
Note that there is nothing to prove for \( \vi'_i \),
Now we need to prove the following:
\[
    \fora{\key, j}  0 \leq j < \abs{\mkvs_{\cut[\aexec, i-1]}(\key)} \implies j \in \vi_i(\key)
\]
Because \( \VIS^{-1}(\txid_i) = \AR^{-1}(\txid_i) = \Set{\txid }[\txid \in \mkvs_{\cut[\aexec, i-1]} ]\),
so for any key \( \key \) and index \( j \) such that \( 0 \leq j < \abs{\mkvs_{\cut[\aexec, i-1]}(\key)} \),
the j-\emph{th} version of the key contains in the view, \ie \( j \in \vi(\key)\).


