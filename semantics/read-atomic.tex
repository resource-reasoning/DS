\subsection{Read Atomic} 
\sx{revisit and take out of the updateview }
\sx{This section might take out as the atomic constraint appears before. }
\label{sec:read-atomic}
\label{sec:semantics.example}
\label{sec:semantics-example}
Read atomic (RA) \cite{ramp} is the weakest consistency model among those that enjoy \emph{atomic visibility} \cite{framework-concur}. 
It requires of a transaction to read an atomic snapshot of the database and never observe the partial effects of other transactions.
This is also known as the \emph{all-or-nothing} property: a transaction observes either none or all the updates performed by other transactions. 
To illustrate that, we show in detail one possible computation of a program \( \prog_{1} \) consisting of two transactions executing in parallel if there is no atomic view constraint:
\[
    \prog_{1} \equiv 
    \begin{session}
        \begin{array}{@{}c || c@{}}
            \txid_{1} : 
            \begin{transaction}
                \pmutate{\vx}{1};\\
            	\pmutate{\vy}{1};
            \end{transaction} &
            \txid_{2} : 
            \begin{transaction}
                \pderef{\pvar{a}}{\vx};\\
            	\pderef{\pvar{b}}{\vy};\\
            	\pifs{\pvar{a}=1 \wedge \pvar{b}=0}\\
            		\quad \passign{\ret}{\sadface}
            	\pife
            \end{transaction}
        \end{array}
    \end{session}
 \]
The \( \pifs{\expr} \cmd_{1} \pifm \cmd_{2} \pife \) is encoded as \( (\passume{\expr} \pseq \cmd_{1}) \pchoice (\neg\passume{\expr} \pseq \cmd_{2} )\).
To recall, we often write \( \cmd_{1} \ppar \cmd_{2} \ppar \dots \ppar \cmd_{n}\) as a syntactic sugar for a program \( \prog \) with implicit unique thread identifiers \( \prog = \Set{\thid_{1} \mapsto \cmd_{1}, \thid_{2} \mapsto \cmd_{2}, \dots, \thid_{n} \mapsto \cmd_{n}  }\).
For better presentation, we annotated transactions with unique identifiers, yet they are allocated dynamically in the semantics.
We also treat the value assigned to the \( \ret \) variable as \emph{returned value}.
Assume the variables \( \vx \) and \( \vy \) refer to two key, and \( \va \) and \( \vb \) are local variables to threads.

The special symbol \(\sadface\), for example the returned value by the transaction $\txid_2$, is to emphasise some undesirable behaviour of a transaction.
In this case, the undesirable behaviour corresponds to the transaction to the right \( \txid_{2} \) observing only one of the updates from \( \txid_{1} \). 
%Intuitively, this behaviour violates the constraints that transactions should be executed atomically (further discussed in \cref{......}), we want to show that if no restrictions are placed on the consistency model specification, it is possible for $\prog_1$ to reach a configuration where the second transaction $\txid_2$ returns $\sadface$. 
To illustrate this, we instantiate the operation semantics with the most permissive execution tests \( \csatP \): \( (\hh, \vi) \csatP \opset : \vi'  \defeq  \true \) and we assume there is no \( \predn{atomic} \) constraints for views.

\begin{figure}[!t]
\hrule
\begin{center}
\begin{tabular}{@{}c@{}@{}c@{}}
\begin{halfsubfig}
\begin{centertikz}
\begin{pgfonlayer}{foreground}

%Location x
\node(locx) {$\ke_{\vx} \mapsto$};

\matrix(initx) [version list] 
    at ([xshift=\tikzkvspace]locx.east) {
    {a} & $\txid_0$ \\ 
    {a} & $\emptyset$ \\
};  
\tikzvalue{initx-1-1}{initx-2-1}{locx-v0}{0};

%Location y
\path (locx.south) + (0,\tikzkeyspace) node (locy) {$\ke_\vy \mapsto$};
\matrix(inity) [version list] 
    at ([xshift=\tikzkvspace]locy.east) {
    {a} & $\txid_0$ \\
    {a} & $\emptyset$ \\
};
\tikzvalue{inity-1-1}{inity-2-1}{locy-v0}{0};

%blue view - I should  check whether I can use pgfkeys to just declare the list of locations, and then add the view automatically.
\draw[-, blue, very thick, rounded corners=10pt]
([xshift=-3pt, yshift=20pt]locx-v0.north east) node (tid1start) {} -- 
([xshift=-3pt, yshift=-5pt]locy-v0.south east);
 
\path (tid1start) node[anchor=south, rectangle, fill=blue!20, draw=blue, font=\small, inner sep=1pt] {$\thid_1$};

%red view
\draw[-, red, very thick, rounded corners = 10pt]
([xshift=-16pt, yshift=5pt]locx-v0.north east) node (tid2start) {}-- 
([xshift=-16pt, yshift=-5pt]locy-v0.south east) node {};
 
\path (tid2start) node[anchor=south, rectangle, fill=red!20, draw=red, font=\small, inner sep=1pt] {$\thid_2$};

\end{pgfonlayer}
\end{centertikz}
\caption{Initial state}
\label{fig:opsem-example-a}
\end{halfsubfig}
&
\begin{halfsubfig}
\begin{centertikz}

\begin{pgfonlayer}{foreground}
%Uncomment line below for help lines

%Location x
\node(locx) {$\ke_{\vx} \mapsto$};

\matrix(versionx) [version list] 
    at ([xshift=\tikzkvspace]locx.east) { 
    {a} & $\txid_{0}$ &{a} & $\txid_{1}$\\
    {a} & $\emptyset$ & {a} & $\emptyset$ \\
};
\tikzvalue{versionx-1-1}{versionx-2-1}{locx-v0}{0};
\tikzvalue{versionx-1-3}{versionx-2-3}{locx-v1}{1};

%Location y
\path (locx.south) + (0,\tikzkeyspace) node (locy) {$\ke_{\vy} \mapsto$};
\matrix(versiony) [version list]
   at ([xshift=\tikzkvspace]locy.east) {
 {a} & $\txid_0$ & {a} & $\txid_1$\\
  {a} & $\emptyset$ & {a} & $\emptyset$ \\
};
\tikzvalue{versiony-1-1}{versiony-2-1}{locy-v0}{0};
\tikzvalue{versiony-1-3}{versiony-2-3}{locy-v1}{1};

%blue view - I should  check whether I can use pgfkeys to just declare the list of locations, and then add the view automatically.
\draw[-, blue, very thick, rounded corners=10pt]
([xshift=-3pt, yshift=20pt]locx-v1.north east) node (tid1start) {} -- 
([xshift=-3pt, yshift=-5pt]locy-v1.south east);
 
\path (tid1start) node[anchor=south, rectangle, fill=blue!20, draw=blue, font=\small, inner sep=1pt] {$\thid_1$};

%red view
\draw[-, red, very thick, rounded corners = 10pt]
([xshift=-16pt, yshift=5pt]locx-v0.north east) node (tid2start) {}-- 
([xshift=-16pt, yshift=-5pt]locy-v0.south east) node {};
 
\path (tid2start) node[anchor=south, rectangle, fill=red!20, draw=red, font=\small, inner sep=1pt] {$\thid_2$};

\end{pgfonlayer}
\end{centertikz}
\caption{After transaction \( \txid_{1}\)} 
\label{fig:opsem-example-b}
\end{halfsubfig}
\\
\begin{halfsubfig}
\begin{centertikz}

\begin{pgfonlayer}{foreground}

%Location x
\node(locx) {$\ke_{\vx} \mapsto$};

\matrix(versionx) [version list] 
    at ([xshift=\tikzkvspace]locx.east) { 
    {a} & $\txid_{0}$ &{a} & $\txid_{1}$\\
    {a} & $\emptyset$ & {a} & $\emptyset$ \\
};
\tikzvalue{versionx-1-1}{versionx-2-1}{locx-v0}{0};
\tikzvalue{versionx-1-3}{versionx-2-3}{locx-v1}{1};

%Location y
\path (locx.south) + (0,\tikzkeyspace) node (locy) {$\ke_{\vy} \mapsto$};
\matrix(versiony) [version list]
    at ([xshift=\tikzkvspace]locy.east) {
    {a} & $\txid_0$ & {a} & $\txid_1$\\
    {a} & $\emptyset$ & {a} & $\emptyset$ \\
};
\tikzvalue{versiony-1-1}{versiony-2-1}{locy-v0}{0};
\tikzvalue{versiony-1-3}{versiony-2-3}{locy-v1}{1};

%blue view - I should  check whether I can use pgfkeys to just declare the list of locations, and then add the view automatically.
\draw[-, blue, very thick, rounded corners=10pt]
([xshift=-3pt, yshift=20pt]locx-v1.north east) node (tid1start) {} -- 
([xshift=-3pt, yshift=-5pt]locy-v1.south east);

\path (tid1start) node[anchor=south, rectangle, fill=blue!20, draw=blue, font=\small, inner sep=1pt] {$\thid_1$};

%red view
\draw[-, red, very thick, rounded corners = 10pt]
([xshift=-16pt, yshift=5pt]locx-v1.north east) node (tid2start) {}-- 
([xshift=-16pt, yshift=-5pt]locx-v1.south east) --
([xshift=-16pt, yshift=5pt]locy-v0.north east) -- 
([xshift=-16pt, yshift=-5pt]locy-v0.south east) node {};
 
\path (tid2start) node[anchor=south, rectangle, fill=red!20, draw=red, font=\small, inner sep=1pt] {$\thid_2$};

\end{pgfonlayer}
\end{centertikz}
\caption{When \( \txid_{2}\) starts}
\label{fig:opsem-example-c}
\end{halfsubfig}
&
\begin{halfsubfig}
\begin{centertikz}

\begin{pgfonlayer}{foreground}
%Uncomment line below for help lines
%\draw[help lines] grid(5,4);

%Location x
\node(locx) {$\ke_{\vx} \mapsto$};

\matrix(versionx) [version list] 
    at ([xshift=\tikzkvspace]locx.east) { 
    {a} & $\txid_{0}$ &{a} & $\txid_{1}$\\
    {a} & $\emptyset$ & {a} & $\Set{\txid_{2}}$ \\
};
\tikzvalue{versionx-1-1}{versionx-2-1}{locx-v0}{0};
\tikzvalue{versionx-1-3}{versionx-2-3}{locx-v1}{1};

%Location y
\path (locx.south) + (0,\tikzkeyspace) node (locy) {$\ke_{\vy} \mapsto$};
\matrix(versiony) [version list]
   at ([xshift=\tikzkvspace]locy.east) {
 {a} & $\txid_0$ & {a} & $\txid_1$\\
  {a} & $\Set{\txid_{2}}$ & {a} & $\emptyset$ \\
};
\tikzvalue{versiony-1-1}{versiony-2-1}{locy-v0}{0};
\tikzvalue{versiony-1-3}{versiony-2-3}{locy-v1}{1};


%blue view - I should  check whether I can use pgfkeys to just declare the list of locations, and then add the view automatically.
\draw[-, blue, very thick, rounded corners=10pt]
([xshift=-3pt, yshift=20pt]locx-v1.north east) node (tid1start) {} -- 
([xshift=-3pt, yshift=-5pt]locy-v1.south east);

\path (tid1start) node[anchor=south, rectangle, fill=blue!20, draw=blue, font=\small, inner sep=1pt] {$\thid_1$};

%red view
\draw[-, red, very thick, rounded corners = 10pt]
([xshift=-16pt, yshift=5pt]locx-v1.north east) node (tid2start) {}-- 
([xshift=-16pt, yshift=-5pt]locx-v1.south east) --
([xshift=-16pt, yshift=5pt]locy-v0.north east) -- 
([xshift=-16pt, yshift=-5pt]locy-v0.south east) node {};

\path (tid2start) node[anchor=south, rectangle, fill=red!20, draw=red, font=\small, inner sep=1pt] {$\thid_2$};

\end{pgfonlayer}
\end{centertikz}
\caption{Transaction \( \txid_{2}\) returns \( \sadface \)}
\label{fig:opsem-example-d}
\end{halfsubfig}
\\
\end{tabular}
\end{center}
\hrule
\caption{Graphical Representation of configurations 
obtained through the execution of $\prog_1$.}
\label{fig:opsem.example}
\label{fig:opsem-example}
\end{figure}

Before any computation, the initial configuration for $\prog_1$ is the one in which there are two keys \( \ke_{\vx}\) and \( \ke_{\vy} \) where each key is associated with an initial version with value zero written by an initialisation transaction $\txid_0$, \( \hh_{0} = \Set{\ke_{\vx} \mapsto \List{(0, \txid, \emptyset)}, \ke_{\vy} \mapsto \List{(0, \txid, \emptyset)}} \).
The initial view of each thread points to the initial version of each key, \( \vi^{1}_{0} = \vi^{2}_{0} = \Set{\ke_{\vx} \mapsto 1, \ke_{\vy} \mapsto 1}\).
The two threads have the same initial stack containing two variables \( \vx \) and \( \vy \) referring to the only keys in the key-value store respectively, \ie \( \stk^{1}_{0} = \stk^{2}_{0} = \Set{\vx \mapsto \ke_{\vx}, \vy \mapsto \ke_{\vy}}\).
Therefore the initial configuration is \( (\hh_{0}, \thdenv_{0}, \prog_{1}) \) where \( \thdenv_{0} = \Set{\thid_{1} \mapsto (\stk_{0}^{1}, \vi_{0}^{1}), \thid_{2} \mapsto (\stk_{0}^{2}, \vi_{0}^{2})}\).
\cref{fig:opsem-example-a} gives a graphical representation of the initial configuration.
 
%\ac{
%Before showing the computation of $\prog_1$ that leads to the transaction of 
%$\thid_2$ to return ${\Large \frownie{}}$, we need to introduce some 
%definitions and  notation.
%The initial configuration in which $\prog_1$ is executed is the one in which 
%each location has an initial version written by some initialisation transaction $\tsid_0$, 
%the view of each thread points to the initial version of each location, and all the  
%thread stacks are initialised to $0$. Assuming that the only key in the database 
%are $[\loc_x], [\loc_y]$, and the only variable in the thread stack is $\retvar$, 
%the initial configuration is then given by $\mathcal{C}_0 = (\hh_{0}, \mathsf{Env}_0)$, 
%where $\hh_{0}([\loc_{x}) = \hh_{0}([\loc_y]) =  (0, \tsid_0, \emptyset)$,  
%$\mathsf{Env}_0(\thid_1) = \mathsf{Env}_0(\thid_2) = ([\retvar \mapsto 0], V_0)$, 
%and $V_0([\loc_{x}]) = V_0([\loc_{y}]) = 0$. Here and in the following, we prefer 
%to adopt a more graphical notation for configurations. For example, the initial configuration 
%defined is represented graphically in Figure \ref{fig:opsem.example}(a). 

%In the picture above, the part above the dashed line represents the history heap and 
%view of each thread, while the part below the dashed line contains the thread-stack 
%of each thread. History heaps are represented as mappings to locations to lists of cells, 
%each of which represents a version and has three component: the value of the version 
%to the left, the identifier of the transaction that wrote it to the top right, and the 
%set of transactions that read the version to the bottom right. Vertical lines labelled 
%with thread identifiers are used to represent the views. The position of the view of 
%thread $\thid_1$ relatively to the location $[\loc_x]$ is determined by the version 
%at which the vertical line labelled $\thid_1$ crosses the list of versions for $[\loc_x]$,
 %and similarly for $[\loc_y]$.
%}

We are now ready to show how to derive a computation of $\prog_1$ that violates \emph{atomic visibility} and it will be explained formally in \cref{sec:example-commit-test}.
In the specific computation, we choose to let the transaction $\thid_1$ commit first.
According to rule $\rl{PCommit}$, we need to perform the following steps: 
\sx{Not sure the bullet points work here? Just typesetting. }
\begin{itemize}
\item
Arbitrarily shift the view $\vi_{0}^{1}$ for thread $\thid_1$ to the right as long as it is within the bound of key-value store and obtain a view \(\vi'' \) such that \( \vi'' \geq \vi_{0}^{1} \). 
Because $\hh_0$ contains only one version per key, here the only possibility is that $\vi'' = \vi_{0}^{1}$.
\item
Determine the initial snapshot $\sn = \clpsHH{\hh_0, \vi''}$ for the transaction $\thid_1$.
In this case, we have that $\h = \Set{\loc_x \mapsto 0, \loc_y \mapsto 0}$.
\item 
Given the initial snapshot \( \h \), initially empty fingerprint \( \unitO \) and the stack \( \stk_{0}^{1}\), by the operational semantics for transaction (\cref{fig:transaction_semantics}), after executing the transactional codes \( \pmutate{\vx}{1}; \pmutate{\vy}{1} \), the finial fingerprints include two write operations as $\opset = \Set{(\otW, \ke_{\vx}, 1), (\otW, \ke_{\vy}, 1)}$.
%\ac{
%\sx{Not sure those details are necessary.}
%This amounts to execute the transaction in isolation from the external environment, using the rules in the operational semantics for 
 %transactions. Because this execution must match the premiss of Rule $(C-Tx)$,  The code is run using $h$ as the initial heap, $\sigma_0$ as the initial 
 %thread stack, $\tau_0 = \lambda_a.0$ as the initial transaction stack, and $\emptyset$ as the 
 %initial fingerprint. We only need to apply 
 %Rule $(Tx-prim)$ twice, in which case we obtain
 %\begin{equation}
%\label{eq:tx1}
%\begin{array}{lcr}
%& \sigma_0 \vdash \left\langle h_0, \_, \emptyset, \begin{array}{l}
%\pmutate{\loc_x}{1};\\ \pmutate{\loc_y}{1} \end{array} \right\rangle 
%&\rightarrow \\
%\rightarrow & 
%\left \langle h_0[ [\loc_x] \mapsto 1], \_, \big( \emptyset \oplus \text{WR}\; [\loc_x]: 1 \big), 
%\pmutate{\loc_y}{1} \right\rangle &= \\
%=&\left \langle h_0[[\loc_x] \mapsto 1], \_, \{\text{WR}\; [\loc_x]: 1\}, 
%\pmutate{\loc_y}{1} \right\rangle 
%&\rightarrow\\ 
%\rightarrow & 
%\left\langle h_0[[\loc_x] \mapsto_1, [\loc_y] \mapsto 1], \_, \big( \{\text{WR}\;[\loc_x]:1\} \oplus  (\text{WR}\;[\loc_y]: 1) \big), 
%\stub \right\rangle & = \\
%=& \left \langle \_, \_, \{\text{WR}\;[\loc_x]: 1, \text{WR}\;[\loc_y]:1 \}, \stub \right\rangle
%\end{array}
%\end{equation}
%Therefore, we conclude $\mathcal{O} = \{\text{WR}\; [\loc_x] : 1, \text{WR}\;[\loc_y]:1\}$.
%}
\item 
The transaction throws away the local snapshot and commits the fingerprints to the key-value store.
A fresh transaction identifier \( \txid_{1} \) is picked.
The new key-value store \( \hh_{1} \) is determined by the functions $\hh_{1} = \func{updHisHp}{\hh_0, \vi'',\txid_1, \opset}$ and the lower bound of the new view is given by the function \( \func{updView}{\hh_1, \vi'' ,\opset}\).
\item Last, the local view shift to the right so that it satisfies the execution tests, \( \vi' \geq \func{updView}{\hh_1, \vi'' ,\opset} \land (\hh_{0}, \vi'') \csatP \vi' : \opset \).
In this case, the permissive model does not constraint the view at all.
Therefore the overall final state is in \cref{fig:opsem-example-b}.
\end{itemize}

Next, thread $\thid_2$ executes its own transaction.
Note that the view for \( \thid_2\) still points to initial versions and the semantics allows the view get updated arbitrarily before executing the transaction.
Because the key-value store now has two versions for each key, there are exactly four possible views for the transaction \( \txid_{2} \).
In particular, assume it updates the view for \( \ke_{x} \) but not \( \ke_y \), \ie \( \Set{\ke_\vx \mapsto 1, \ke_\vx \mapsto 0} \) (\cref{fig:opsem-example-c}).
Given the view, the transaction \( \txid_{2} \) will assign \(\sadface\) to the \( \ret\) and this transaction is allowed to commit since the commit test does not stop this (\cref{fig:opsem-example-d}).

%\subsection{Example of Consistency models}
%\label{sec:example-commit-test}
%\ac{This Section is going to become heavy in pictures, which should be organised into figures.}
%In this Section we present different consistency models specifications. 
%For each of them, we give: 
%\begin{itemize}
%\item the intuition of the commit tests for different consistency models, and the formal definitions with respect to the \(\Como\) (\defref{def:consistency-models}).
%%describing the consistency guarantees that schedules of the database should have in plain English, 
%%\item a formal consistency model specification, in the style described in \S \ref{sec:semantics.programs},
%\item examples of litmus tests that, when executed, give rise to the anomalies that are forbidden from the consistency model, 
%\item an explanation of why the consistency model forbids the litmus tests to exhibit the anomaly that should be forbidden. 
%\end{itemize}
%Later, we will show how to compare our consistency models specifications with those already existing in the 
%literature.
%\ac{There is still a long-way to go before proving correspondence with dependency graph specifications, 
%but this should be mentioned here.}







Intuitively, in such a program, it violate the atomic visibility because it is allowed to execute the second transaction \( \trans_2\) when the client $\thid_2$ only observes partial effect from transaction \( \txid_1 \).
%\ac{, which is obtained by removing all the information about $\thid_1$ (view and stack) in Figure \ref{fig:opsem.example}(c).
%\[
%\prog_1 \equiv 
    %\begin{array}{c} 
    %\begin{transaction} 
        %\pderef{\pvar{a}}{\vx};\\
        %\pderef{\pvar{b}}{\vy};\\
        %\pifs{\pvar{a}=1 \wedge \pvar{b}=0}\\
        %\quad \passign{\retvar}{\Large \frownie{}} \\
        %\pife
    %\end{transaction}
%\end{array}
%\]
%}

%\ac{
%To avoid transactions to only observe the partial effects of other transactions, we 
%must ensure that transactional code cannot be executed by a client whose 
%views is up-to-date with respect to some transaction $\tsid$ for some location $[\loc_x]$, 
%but not for some other location $[\loc_y]$. This leads to the following definition.
%}
To avoid a transaction to observe the partial effects of other transactions, it needs to ensure that transactional code cannot be executed by a client whose views is partially up-to-date with respect to some transactions.
This leads to the following definition.
\begin{definition}[Read atomic]
\label{def:readatomic}
%Let $\hh$ be a history heap,$V$ be a view, $[\loc_x]$ be 
%a location and $\nu$ be a version. We say that $V$ $[\loc_x]$-\emph{sees} version 
%$\nu$ if there exists an index $i \leq V([\loc_x])$ such that $V(i) = \nu$. 
%We say that $V$ $[\loc_x]$-\emph{sees} transaction $\tsid$ if 
%$V$ $[\loc_x]$-sees a version $\nu = (\_, \tsid, \_)$. 
Given a view $\vi \in \Views$, a history heap $\hh \in \HisHeaps$, and a transaction identifier $\txid \in \TxID$, the view \emph{sees} the transaction in the history heap, written $\pred{visible}{\txid, \vi, \hh}$, if the view sees all the writes from the transaction,
%We say that $V$ \emph{sees} transaction $\tsid$ in $\hh$, written 
%$\mathsf{Visible}(\tsid, V, \hh)$, iff 
\sx{\( \exsts{i} \) might be enough}
\[
\begin{rclarray}
\pred{visible}{\txid, \vi, \hh} & \eqdef & \fora{\ke, i} \hh(\ke)(i) = (\stub, \txid, \stub) \implies i \leq \vi(\ke).
\end{rclarray}
\]
%\ac{In English: the view is up-to-date with respect to all the updates 
%performed by transaction $\tsid$.

%In English: if the view $V$ is up-to-date with some of the updates performed 
%by $\tsid$, then it must be up-to-date with all the updates performed by $\tsid$. 
%This is the all-or-nothing property.

%In English: Before executing a transaction, either you observe all or none the 
%updates of all other transactions. We may strengthen the consistency model and 
%require that the same property must be satisfied at the end as well, though 
%this is not strictly necessary. In this case the check becomes: 
%\[
%\mathsf{atomic}(\hh, V) \wedge \mathsf{atomic}(\hh, V') \wedge \mathsf{UpdateView}(\hh, V, \mathcal{O}) 
%\sqsubseteq V' \implies (\hh, V) \triangleright_{\mathsf{RA}} \mathcal{O}: V'.
%\]
%}

\sx{consistent has been used in many place, might mislead?}
Then given a history \( \hh \), the view $\vi$ is \emph{consistent} with respect to \emph{atomic visibility}, written $\pred{atomic}{\vi, \hh}$, if the view $\vi$ is up-to-date with some of the updates performed by transaction $\txid$, then it should be up-to-date with all the updates performed by $\txid$,
\[
\begin{rclarray}
\pred{atomic}{\vi ,\hh} & \eqdef & \fora{\txid } \exsts{\ke, i} i \leq \vi(\ke) \land \hh(\ke)(i) = (\stub, \txid, \stub) \implies \pred{visible}{\txid, \vi, \hh}
\end{rclarray}
\]
The commit test for \emph{read atomic} $\csatRA$ is defined as,
\[
\begin{rclarray}
(\hh, \vi) \csatRA \stub: \stub & \defeq & \pred{atomic}{\hh, \vi} 
\end{rclarray}
\]
%written $\mathsf{up-to-date}(\hh, V, \tsid, [\loc_x])$, 
%if either 
%
%\begin{itemize}
%\item for all indexes $i = 0,\cdots, \lvert \hh([\loc_x]) - 1 \rvert$, 
%$\WS(\hh([\loc_x])(i)) \neq \tsid)$, or 
%\item if $\WS(\hh([\loc_x])(i)) = \tsid$ for some $i = 0,\cdots, \lvert \hh([\loc_n]) -1 \rvert$, 
%then $i \leq V([\loc_n])$.
%\end{itemize}
\end{definition}

Suppose that we execute the program $\prog_1$ under read atomic $\comoRA$.
Assume the \( \txid_{1}\) is still committed first as in Section \ref{sec:semantics.example}, yielding the result shown in \cref{fig:opsem-example-b}.
Yet the second transaction \( \txid_{2} \) cannot starts with a view as in \cref{fig:opsem-example-c}, because later when the transaction try to commit, the read atomic commit test will stop it.
%\ac{
%$\langle \mathcal{C_0}, \prog_1 \rangle \xrightarrow{\mathsf{RA}} \langle \mathcal{C}_1, \prog_1' \rangle$, 
%where we recall that $\mathcal{C}_0$, $\mathcal{C}_1$ are depicted in Figure \ref{fig:opsem.exampe}(a), 
%\ref{fig:opsem.example}(b), respectively. 

%It is immediate to observe that the only way in which the execution of transaction $\ptrans{\trans}$ from $\thid_2$ in $\prog_1'$ can return value ${\Large \frownie}$ is the following: 
%\begin{itemize}
%\item first, push the view $V$ of client $\txid_2$ in the configuration 
%$\mathcal{C}_1$ of Figure \ref{fig:opsem.example}(b) to observe the update of location $[\vx]$, but not the update of 
%$[\vy]$. This view is the one labelled with $\txid_2$ in Figure \ref{fig:opsem.example}(c), and we refer 
%to it as $V'$;
%\item then, execute the transaction $\ptrans{\trans}$ in $\thid_2$. 
%\end{itemize}
%}

One may wonder whether it is the case that an execution of a command may get stuck because of a ill-defined execution test.
For example, for a given execution test $\ET$, it may be possible to reach a state of the system, with MKVS $\hh$ and view $\vi$ for client $\cl$ , such that for any $\vi': \vi \sqsubseteq \vi'$ we have that $ \comoRA \not\vdash (\hh, \vi') \csat[] \stub : \stub $.
We show that this is not the case for the execution test $\comoRA$. 

\ac{This does not mean that progress is ensured. A transaction may very well 
not terminate. Rather, this says that the inability of a program to execute a 
transaction does not depend on the execution test itself.}
\sx{what do you mean???}
\begin{proposition}
\label{prop:ra.progress}
The execution test $\comoRA$ \emph{does not hinder progress}: for any MKVS $\mkvs$, views $\vi$, and fingerprint $\opset$, there exist two views $\vi': \vi \sqsubseteq \vi'$, $\vi'': \Vupdate(\hh, \vi', \opset) \sqsubseteq \vi''$ such that $(\hh, \vi) \csatRA \opset : \vi'$.
\end{proposition}

\begin{proof}
For a given MKVS $\hh$, let $\func{up-to-date}{\hh} = \lambda \key{k}. (\lvert \hh \rvert - 1)$ be the view that points to the most recent version of each object. It is immediate to note that $\pred{atomic}{\hh, \func{up-to-date}{\hh}}$, and therefore $(\hh, \vi) \csatRA \opset : \vi'$. 
\end{proof}
