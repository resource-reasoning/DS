\subsection{soundness and completeness of the specification}

\subsubsection{\( \MRd \)}

\begin{example}
We now show how to employ the proof technique from \cref{def:et_sound} to prove that the execution 
test $\ET_\MRd$ is sound with respect to the axiomatic specification $(\RP_{\LWW}, \{\lambda \aexec. \VIS_{\aexec} ; \PO_{\aexec} \})$. 
As a consequence of \cref{thm:et_soundness}, it follows that $\CMs(\ET_{\MRd}) \subseteq \{ \hh_{\aexec} \mid \aexec \in \CMa(\RP_{\LWW}, 
\{\lambda \aexec. \VIS_{\aexec} ; \PO_{\aexec})\} \}$. 

As an invariant condition for the execution test $\ET_{\MRd}$, we choose 
\[
I(\aexec, \cl) = \bigcup_{\{\txid_{\cl}^{n} \in \T_{\aexec} \mid n \in \Nat\}} \VIS_{\aexec}^{-1}(\cl).
\]
Let now $\hh, \vi, \vi', \opset$ be such that $\ET_{\MRd} \vdash (\hh, \vi) \triangleright \opset: \vi'$. 
Choose an arbitrary $\cl$, a transaction identifier $\txid \in \nextTxId(\hh, \cl)$, 
an abstract execution $\aexec$ such that $\hh_{\aexec} = \hh$, and a set of 
transactions $\T_{\mathsf{rd}}$ that are read-only in $\aexec$, such that 
\begin{equation}
I(\aexec, \cl) \subseteq \Tx(\hh, \vi) \cup \T_{\mathsf{rd}}.
\label{eq:mr_invariant}
\end{equation}

Let now $\T'_{\mathsf{rd}} = \emptyset$, $T''_{\mathsf{rd}} = \T_{\mathsf{rd}}$. 
Let 
\[ 
\aexec' = \extend(\aexec, \txid, \opset, \Tx(\hh, \vi) \cup \T_{\mathsf{rd}} \cup \T'_{\mathsf{rd}}) = 
\extend(\aexec, \txid, \opset, \Tx(\hh, \vi) \cup \T_{\mathsf{rd}}).\]
We can derive the following facts:
\begin{itemize}
\item $\{\txid' \mid (\txid', \txid) \in \VIS_{\aexec'} ; \PO_{\aexec'} \} \subseteq \Tx(\hh, \vi) \cup \T'_{\mathsf{rd}}$. 
To see why this is true, suppose that $\txid' \xrightarrow{\VIS_{\aexec'}} \txid'' \xrightarrow{\PO_{\aexec'}} \txid$ 
for some $\txid', \txid''$. We show that $\txid' \in I(\aexec, \cl)$, and then \cref{eq:mr_invariant} ensures 
that $\txid' \in \Tx(\hh, \vi) \cup \T_{\mathsf{rd}}$. 

If $\txid'' \xrightarrow{\AR_{\aexec'}} \txid$, {\color{blue} sx: guess it is \( \SO_{\aexec'}\) } then $\txid'' = \txid_{\cl}^{n}$ for some $n \in \Nat$, 
and because $\txid'' \neq \txid$, and $\T_{\aexec'} \setminus \T_{\aexec} = \{ \txid \}$, we also 
have that $\txid'' \in \aexec$. By the definition of $I(\aexec, \cl)$, we have that $\VIS^{-1}_{\aexec}(\cl) \subseteq 
I(\aexec, \cl)$: because $\txid' \xrightarrow{\VIS_{\aexec'}} \txid''$ and $\txid'' \neq \txid$, we have 
that $\txid' \xrightarrow{\VIS_{\aexec}} \txid''$, and therefore $\txid' \in I(\aexec, \cl)$. 

\item $I(\aexec', \cl) \subseteq \Tx(\aexec', \vi') \cup \T''_{\mathsf{rd}} = \Tx(\hh', \vi') \cup \T_{\mathsf{rd}}$. 
In this case, note that because $\ET_{\MRd} \vdash (\hh, \vi) \triangleright \opset: \vi'$, then 
it must be the case that $\vi \viewleq \vi'$. A trivial consequence of this fact is that 
$\Tx(\hh, \vi) \subseteq \Tx(\hh, \vi')$. Also, because $\aexec' = \extend(\aexec, \txid, \Tx(\hh, \vi) \cup \T_{\mathsf{rd}})$, 
we have that $\Tx(\hh_{\aexec}, \vi') = \Tx(\hh_{\aexec}, \vi')$. {\color{blue} sx: guess it should be \( \vi \) instead of \( \vi' \)} {\color{red} to infer this there should 
be a Lemma that states that if $\vi \in \Views(\hh)$, then 
$\Tx(\updateKV(\hh, \vi', \txid, \opset), \vi) = \Tx(\hh, \vi)$.}
Finally, note that $\{\txid_{\cl}^{n} \in \aexec' \mid n \in \Nat\} = 
\{ \txid_{\cl}^{n} \in \T_{\aexec} \mid n \in \Nat\} \cup \txid$, that for any 
$\txid_{\cl}^{n} \in \T_{\aexec}$ we have that $\VIS^{-1}_{\aexec'}(\txid_{\cl}^{n}) = 
\VIS^{-1}_{\aexec}(\txid_{\cl}^{n})$, and that 
$\VIS_{\aexec'}^{-1}(\txid) = \Tx(\hh, \vi) \cup \T_{\mathsf{rd}}$. 
Using all these facts, we obtain 
\[
\begin{array}{lcl}
I(\aexec', \cl) &=& \bigcup_{\{\txid_{\cl}^{n} \in \aexec' \mid n \in \Nat\}} \VIS_{\aexec'}^{-1}(\txid_{\cl}^{n}) \\
&=& \left( \bigcup_{\{\txid_{\cl}^{n} \in \aexec \mid n \in \Nat\}} \VIS_{\aexec}^{-1}(\txid_{\cl}^{n}) \right) \cup \VIS^{-1}_{\aexec'}(\txid)\\
&=& I(\aexec, \cl) \cup (\Tx(\hh, \vi) \cup \T_{\mathsf{rd}})\\
&\stackrel{\eqref{eq:mr_invariant}}{\subseteq}& \Tx(\hh, \vi) \cup \T_{\mathsf{rd}}\\
&=& \Tx(\hh_\aexec, \vi) \cup \T_{\mathsf{rd}}\\
&=& \Tx(\hh_{\aexec'}, \vi) \cup \T_{\mathsf{rd}}\\
&\subseteq& \Tx(\hh_{\aexec'}, \vi') \cup \T_{\mathsf{rd}}.
\end{array}
\]
\end{itemize}
\end{example}

\begin{example}
We show that the execution test $\ET_{\MRd}$ is complete 
with respect to the axiomatic specification $(\RP_{\LWW}, \{\lambda \aexec.(\VIS_{\aexec};\PO_{\aexec})\})$. 
To this end, let $\aexec$ be an abstract execution that satisfies the specification
$\CMa(\RP_{\LWW}, \{\lambda \aexec.(\VIS_{\aexec};\PO_{\aexec})\})$, 
and consider a transaction $\txid \in \T_{\aexec}$. 
Suppose that $\T_{\aexec} = \{\txid_{i}\}_{i=1}^{n}$, and for any $i=1,\cdots, n-1$, $\txid_{i} \xrightarrow{\AR_{\aexec}} 
\txid_{i}$.
Fix $i=1,\cdots,n$, and let $\vi'_{i-1} = \getView(\aexec, \VIS^{-1}_{\aexec}(\txid_{i}))$.
We have two possible cases: 
\begin{itemize}
\item the transaction $\txid'_{i} = \min_{\PO_{\aexec}}\{\txid' \mid \txid_{i} \xrightarrow{\PO_{\aexec}} \txid'\}$ is 
defined. In this case let $\vi_{i} =\getView(\aexec, (\AR^{-1}_{\aexec})?(\txid_{i}) \cap \VIS^{-1}_{\aexec}(\txid'_{i}))$. 
Note that $\txid_{i} \xrightarrow{\PO_{\aexec}} \txid'_{i}$, and because $\aexec \models \VIS_{\aexec} ; \PO_{\aexec}$, 
it follows that $\VIS^{-1}_{\aexec}(\txid_{i}) \subseteq \VIS^{-1}_{\aexec}(\txid'_{i})$. 
We also have that $\VIS^{-1}_{\aexec}(\txid_{i}) \subseteq (\AR^{-1}_{\aexec})?(\txid_{i})$ because of 
the definition of abstract execution. It follows that 
\[
\VIS^{-1}_{\aexec}(\txid_{i}) \subseteq (\AR^{-1}_{\aexec})?(\txid_{i}) \cap \VIS^{-1}_{\aexec}(\txid'_{i}),
\]
Recall that  $\vi'_{i-1} = \getView(\aexec, \VIS^{-1}_{\aexec}(\txid_{i}))$, 
and $\vi_{i} =\getView(\aexec, (\AR^{-1}_{\aexec})?(\txid_{i}) \cap \VIS^{-1}_{\aexec}(\txid'_{i}))$. 
Thus we have that $\vi'_{i-1} \viewleq \vi_{i}$, and therefore $\ET_{\MRd} \vdash (\hh_{\cut(\aexec, i)}, \vi'_{i-1}) 
\triangleright \TtoOp{T}_{\aexec}(\txid_{i}) : \vi_{i}$. 
\item the transaction $\txid'_{i} = \min_{\PO_{\aexec}}\{\txid' \mid \txid_{i} \xrightarrow{\PO_{\aexec}} \txid_{i}\}$ 
is not defined. In this case, let $\vi_{i} = \getView(\aexec, (\AR^{-1}_{\aexec})?(\txid_{i}))$. 
As for the case above, we have that $\vi'_{i-1} \viewleq \vi_{i}$, and therefore 
$\ET_{\MRd} \vdash (\hh_{\cut(\aexec, i)}, \vi'_{i-1}) \triangleright \TtoOp{T}_{\aexec}(\txid_{i}) : \vi_{i}$. 
\end{itemize}
\end{example}

\subsection{\( \MW \)}
