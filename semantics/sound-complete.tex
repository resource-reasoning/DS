\subsection{soundness and completeness of the specification}

\subsubsection{\( \MRd \)}

\label{sec:sound-complete-mr}

We now show how to employ the proof technique from \cref{def:et_sound} to prove that the execution 
test $\ET_\MRd$ is sound with respect to the axiomatic specification $(\RP_{\LWW}, \{\lambda \aexec. \VIS_{\aexec} ; \PO_{\aexec} \})$. 
As a consequence of \cref{thm:et_soundness}, it follows that $\CMs(\ET_{\MRd}) \subseteq \{ \hh_{\aexec} \mid \aexec \in \CMa(\RP_{\LWW}, 
\{\lambda \aexec. \VIS_{\aexec} ; \PO_{\aexec})\} \}$. 

As an invariant condition for the execution test $\ET_{\MRd}$, we choose 
\[
    I(\aexec, \cl) = \left( \bigcup_{\{\txid_{\cl}^{n} \in \T_{\aexec} \mid n \in \Nat\}} \VIS_{\aexec}^{-1}(\txid^n_\cl) \right) \setminus \T_\rd
\]
where \( \T_\rd \) is all the read-only transactions in \( \bigcup\limits_{\{\txid_{\cl}^{n} \in \T_{\aexec} \mid n \in \Nat\}} \VIS_{\aexec}^{-1}(\txid^n_\cl) \) 
Let now $\hh, \vi, \vi', \opset$ be such that $\ET_{\MRd} \vdash (\hh, \vi) \triangleright \opset: \vi'$. 
Choose an arbitrary $\cl$, a transaction identifier $\txid \in \nextTxId(\hh, \cl)$, 
and an abstract execution $\aexec$ such that $\hh_{\aexec} = \hh$ such that 
\begin{equation}
I(\aexec, \cl) \subseteq \Tx(\hh, \vi)
\label{eq:mr_invariant}
\end{equation}
Let 
\[ 
\aexec' = \extend(\aexec, \txid, \opset, \Tx(\hh, \vi) \cup \T_\rd) 
\]
We can derive the following facts:
\begin{itemize}
\item $\{\txid' \mid (\txid', \txid) \in \VIS_{\aexec'} ; \PO_{\aexec'} \} \subseteq \Tx(\hh, \vi) \cup \T_\rd$. 
To see why this is true, suppose that $\txid' \xrightarrow{\VIS_{\aexec'}} \txid'' \xrightarrow{\PO_{\aexec'}} \txid$ 
for some $\txid', \txid''$. We show that $\txid' \in I(\aexec, \cl)$, and then \cref{eq:mr_invariant} ensures 
that $\txid' \in \Tx(\hh, \vi) \cup \T_{\mathsf{rd}}$. 

If $\txid'' \xrightarrow{\AR_{\aexec'}} \txid$, then $\txid'' = \txid_{\cl}^{n}$ for some $n \in \Nat$, 
and because $\txid'' \neq \txid$, and $\T_{\aexec'} \setminus \T_{\aexec} = \{ \txid \}$, we also 
have that $\txid'' \in \aexec$. By the definition of $I(\aexec, \cl)$, we have that $\VIS^{-1}_{\aexec}(\cl) \subseteq 
I(\aexec, \cl)$: because $\txid' \xrightarrow{\VIS_{\aexec'}} \txid''$ and $\txid'' \neq \txid$, we have 
that $\txid' \xrightarrow{\VIS_{\aexec}} \txid''$, and therefore $\txid' \in I(\aexec, \cl)$. 

\item $I(\aexec', \cl) \subseteq \Tx(\aexec', \vi') = \Tx(\hh', \vi')$. 
In this case, because $\ET_{\MRd} \vdash (\hh, \vi) \triangleright \opset: \vi'$, 
then it must be the case that $\vi \viewleq \vi'$. 
A trivial consequence of this fact is that $\Tx(\hh, \vi) \subseteq \Tx(\hh, \vi')$.
Also, because $\aexec' = \extend(\aexec, \txid, \Tx(\hh, \vi) \cup \T_{\mathsf{rd}})$, 
we have that $\Tx(\hh_{\aexec}, \vi) = \Tx(\hh_{\aexec'}, \vi)$. 
{\color{red} to infer this there should be a Lemma that states that if $\vi \in \Views(\hh)$, 
then $\Tx(\updateKV(\hh, \vi', \txid, \opset), \vi) = \Tx(\hh, \vi)$.}
Finally, note that $\{\txid_{\cl}^{n} \in \aexec' \mid n \in \Nat\} = 
\{ \txid_{\cl}^{n} \in \T_{\aexec} \mid n \in \Nat\} \cup \txid$, that for any 
$\txid_{\cl}^{n} \in \T_{\aexec}$ we have that $\VIS^{-1}_{\aexec'}(\txid_{\cl}^{n}) = 
\VIS^{-1}_{\aexec}(\txid_{\cl}^{n})$, and that 
$\VIS_{\aexec'}^{-1}(\txid) = \Tx(\hh, \vi) \cup \T_{\mathsf{rd}}$. 
Using all these facts, we obtain 
\[
\begin{rclarray}
I(\aexec', \cl) &=& \left( \bigcup_{\{\txid_{\cl}^{n} \in \aexec' \mid n \in \Nat\}} \VIS_{\aexec'}^{-1}(\txid_{\cl}^{n}) \right) \setminus \T_\rd \\
                &=& \left( \left( \bigcup_{\{\txid_{\cl}^{n} \in \aexec \mid n \in \Nat\}} \VIS_{\aexec}^{-1}(\txid_{\cl}^{n}) \right) \setminus \T_\rd  \right) \cup \left( \VIS^{-1}_{\aexec'}(\txid) \setminus \T_\rd  \right) \\
&=& I(\aexec, \cl) \cup \Tx(\hh, \vi) \\
&\stackrel{\eqref{eq:mr_invariant}}{\subseteq}& \Tx(\hh, \vi) \\
&=& \Tx(\hh_\aexec, \vi) \\
&=& \Tx(\hh_{\aexec'}, \vi) \\
&\subseteq& \Tx(\hh_{\aexec'}, \vi')
\end{rclarray}
\]
\end{itemize}

We show that the execution test $\ET_{\MRd}$ is complete 
with respect to the axiomatic specification $(\RP_{\LWW}, \{\lambda \aexec.(\VIS_{\aexec};\PO_{\aexec})\})$. 
To this end, let $\aexec$ be an abstract execution that satisfies the specification
$\CMa(\RP_{\LWW}, \{\lambda \aexec.(\VIS_{\aexec};\PO_{\aexec})\})$, 
and consider a transaction $\txid \in \T_{\aexec}$. 
Suppose that $\T_{\aexec} = \{\txid_{i}\}_{i=1}^{n}$, and for any $i=1,\cdots, n-1$,
$\txid_{i} \xrightarrow{\AR_{\aexec}} \txid_{i+1}$.
Fix $i=1,\cdots,n$, and let $\vi'_{i-1} = \getView(\aexec, \VIS^{-1}_{\aexec}(\txid_{i}))$.
We have two possible cases: 
\begin{itemize}
\item the transaction $\txid'_{i} = \min_{\PO_{\aexec}}\{\txid' \mid \txid_{i} \xrightarrow{\PO_{\aexec}} \txid'\}$ is 
defined. In this case let $\vi_{i} =\getView(\aexec, (\AR^{-1}_{\aexec})?(\txid_{i}) \cap \VIS^{-1}_{\aexec}(\txid'_{i}))$. 
Note that $\txid_{i} \xrightarrow{\PO_{\aexec}} \txid'_{i}$, and because $\aexec \models \VIS_{\aexec} ; \PO_{\aexec}$, 
it follows that $\VIS^{-1}_{\aexec}(\txid_{i}) \subseteq \VIS^{-1}_{\aexec}(\txid'_{i})$. 
We also have that $\VIS^{-1}_{\aexec}(\txid_{i}) \subseteq (\AR^{-1}_{\aexec})?(\txid_{i})$ because of 
the definition of abstract execution. It follows that 
\[
\VIS^{-1}_{\aexec}(\txid_{i}) \subseteq (\AR^{-1}_{\aexec})?(\txid_{i}) \cap \VIS^{-1}_{\aexec}(\txid'_{i}),
\]
Recall that  $\vi'_{i-1} = \getView(\aexec, \VIS^{-1}_{\aexec}(\txid_{i}))$, 
and $\vi_{i} =\getView(\aexec, (\AR^{-1}_{\aexec})?(\txid_{i}) \cap \VIS^{-1}_{\aexec}(\txid'_{i}))$. 
Thus we have that $\vi'_{i-1} \viewleq \vi_{i}$, and therefore $\ET_{\MRd} \vdash (\hh_{\cut(\aexec, i)}, \vi'_{i-1}) 
\triangleright \TtoOp{T}_{\aexec}(\txid_{i}) : \vi_{i}$. 
\item the transaction $\txid'_{i} = \min_{\PO_{\aexec}}\{\txid' \mid \txid_{i} \xrightarrow{\PO_{\aexec}} \txid_{i}\}$ 
is not defined. In this case, let $\vi_{i} = \getView(\aexec, (\AR^{-1}_{\aexec})?(\txid_{i}))$. 
As for the case above, we have that $\vi'_{i-1} \viewleq \vi_{i}$, and therefore 
$\ET_{\MRd} \vdash (\hh_{\cut(\aexec, i)}, \vi'_{i-1}) \triangleright \TtoOp{T}_{\aexec}(\txid_{i}) : \vi_{i}$. 
\end{itemize}

\subsection{\( \MW \)}

\label{sec:sound-complete-mw}

The execution test $\ET_\MW$ is sound with respect to the axiomatic specification 
$(\RP_{\LWW}, \Set{\lambda \aexec. \PO_{\aexec} ; \VIS_{\aexec} })$.
We pick the invariant as \( I( \aexec, \cl ) = \emptyset \), given the fact of no constraint on the view after update.
Assume a key-value store \( \mkvs \) and its corresponding abstract execution \( \aexec \) so that \( \mkvs = \mkvs_\aexec \).
Given an update on the key-value store that satisfies \( \MW \), \ie \( \MW \vdash (\mkvs, \vi) \csat \f : \vi' \),
by the \cref{def:et_sound}, it remains to prove that there exists another set of read-only transactions \( \T_{\rd} \) such that
\[
    \begin{array}{@{}l@{}}
        \fora{ \aexec', \txid, \txid' } \aexec' = \extend(\aexec, \txid, \Tx(\mkvs, \vi) \cup \T_\rd, \f ) \land (\txid' ,\txid)  \in \PO_{\aexec'} ; \VIS_{\aexec'}  \\
        \qquad {} \implies \txid' \in \Tx(\mkvs, \vi) \cup \T_\rd
    \end{array}
\]
Initially we take \( \T_\rd = \emptyset \), 
by closing the \( \Tx(\mkvs, \vi) \) with respect to \( \PO_{\aexec'} ; \VIS_{\aexec'} \)
we will add more read-only transactions into the set \( \T_\rd\).
Suppose \( (\txid' ,\txid)  \in \PO_{\aexec'} ; \VIS_{\aexec'} \), 
that is, \( \txid' \toEdge{\SO_{\aexec'}} \txid'' \toEdge{\VIS_{\aexec'}} \txid \).
If the transaction \( \txid'' \) writes to a key.
For the new abstract execution \( \aexec' \), the visible transactions for \( \txid \) must come from \( \Tx(\mkvs, \vi) \cup \T_\rd \).
It means \( \txid'' \in \Tx(\mkvs, \vi) \cup \T_\rd  \),
and then given that \( \txid'' \) is not a read-only transaction, we have \( \txid'' \in \Tx(\mkvs, \vi) \).
Now there are two cases: \textbf{(i)} \( \txid' \) is a read-only transaction, or \textbf{(ii)} \( \txid' \) has at least one write.
For the first case, we include \( \txid' \in \T_{\rd} \).
For the second case, it is easy to see \( \txid' \in \Tx(\mkvs, \vi) \) since \( j \in \vi(\ke) \wedge \WTx(\hh(\ke', i)) \xrightarrow{\PO?} \WTx(\hh(\ke, j)) \implies i \in \vi(\ke') \).
If the transaction \( \txid'' \) is a read-only transaction, 
there must exist a later transaction \( \txid''' \) from the same client that writes to a key,
and the transaction \( \txid''' \) is included in \( \Tx(\mkvs, \vi) \).
Since \( \SO \) is transitive, 
meaning \( \txid' \toEdge{\SO_{\aexec'}} \txid'' \toEdge{\SO_{\aexec'}} \txid''' \toEdge{\VIS_{\aexec'}} \txid \), 
we can prove \( \txid' \in \Tx(\mkvs, \vi) \) or we will include \( \txid' \) in \( \T_\rd \).
Since there are finite transactions from a client in a trace, there must exist a \( \T_\rd \) in the end.


The execution test $\ET_{\MW}$ is complete with respect to 
the axiomatic specification $(\RP_{\LWW}, \{\lambda \aexec.(\PO_{\aexec} ; \VIS_{\aexec})\})$. 
Let $\aexec$ be an abstract execution that satisfies the specification
$\CMa(\RP_{\LWW}, \{\lambda \aexec.(\PO_{\aexec} ; \VIS_{\aexec})\})$.
Suppose that $\T_{\aexec} = \{\txid_{i}\}_{i=1}^{n}$, and for any $i=1,\cdots, n-1$, $\txid_{i} \xrightarrow{\AR_{\aexec}} 
\txid_{i+1}$.
Let views \( \vi_{i} = \getView(\aexec, \VIS^{-1}_{\aexec}(\txid_{i}) ) \) and \( \vi'_{i} \subseteq \getView(\aexec, (\AR^{-1}_{\aexec})?(\txid_{i}) ) \).
It suffices to prove \( \ET_\MW \vdash (\hh_{\cut(\aexec, i-1)}, \vi_i ) \csat  \TtoOp{T}_{\aexec}(\txid_{i}) : \vi'_{i} \).
It means to prove the follows:
\begin{equation}
\label{equ:mw-complete}
\begin{array}{@{}l@{}}
    \fora{i,j,\ke, \ke' } j \in \vi(\ke)
    \wedge \WTx(\hh(\ke', i)) \xrightarrow{\PO?} \WTx(\hh(\ke, j)) 
    \implies i \in \vi(\ke')
\end{array}
\end{equation}
Assume \( j \) and \( \ke' \) such that \( j \in \vi(\ke')\), which means \( \WTx(\hh_{\cut(\aexec, i-1)}(\ke', j)) \in \VIS^{-1}_{\aexec}(\txid_{i}) \).
Now let consider transaction \( \txid \) that commits before \( \txid \) from the same session, \ie \( \txid \toEdge{\SO} \WTx(\hh_{\cut(\aexec, i-1)}(\ke', j)) \).
By the constraint \( \lambda \aexec.(\PO_{\aexec} ; \VIS_{\aexec}) \), the transaction \( \txid \in \VIS^{-1}_{\aexec}(\txid_{i}) \).
It means that in the kv-store \(  \hh_{\cut(\aexec, i-1)} \) every version written by \( \txid \) should be included in the view \( \vi_i \).
Thus we have the proof of \cref{equ:mw-complete}.

\subsection{ \( \RYW \) }

\label{sec:sound-complete-ryw}

The execution test $\ET_\RYW$ is sound with respect to the axiomatic specification 
$(\RP_{\LWW}, \{\lambda \aexec. \PO_{\aexec} \})$.
We pick an invariant for the \( \ET_\RYW \) as the following:
\[
    I(\aexec, \cl) = \left( \bigcup_{\{\txid_{\cl}^{n} \in \T_{\aexec} \mid n \in \Nat\}} (\SO_{\aexec}^{-1})?(\txid^n_\cl) \right) \setminus \T_\rd
\]
where \( \T_\rd \) is all the read-only transactions in \( \bigcup\limits_{\{\txid_{\cl}^{n} \in \T_{\aexec} \mid n \in \Nat\}} (\SO_{\aexec}^{-1})?(\txid^n_\cl) \).
Suppose a kv-store \( \mkvs \), an abstract execution \( \aexec \) and a view \( \vi \)
such that \( \mkvs_\aexec = \mkvs \) and \( I(\aexec, \cl) \subseteq \Tx(\mkvs, \vi) \).
Assume that a fresh transaction identifier \( \txid_\cl^n \) with fingerprint \( \f \) commits to 
the kv-store \( \mkvs \) under view \( \vi \) and the view after is updated to \( \vi' \), that is, 
\( \ET_\RYW \vdash (\mkvs, \vi) \csat \f : \vi' \).
Let a new abstract execution \( \aexec' = \extend(\aexec, \txid_\cl^n, \f, \Tx(\mkvs, \vi) \cup \T_\rd) \).
We need to prove the following:
\begin{gather}
    \fora{\txid} \txid \toEdge{\SO_{\aexec'}} \txid_\cl^n \implies \txid \in \Tx(\mkvs, \vi) \cup \T_\rd \label{equ:ryw-sound-update}\\
    I(\aexec',\cl) \subseteq \Tx(\mkvs_{\aexec'}, \vi')\label{equ:ryw-sound-inv} 
\end{gather}
Assume a transaction \( \txid \) such that \( \txid \toEdge{\SO_{\aexec'}} \txid_\cl^n \).
It immediately implies that \( \txid = \txid_\cl^m\) where \( m < n \) and \( \txid_\cl^m \in \aexec \).
Thus we prove \cref{equ:ryw-sound-update} as \( \txid \in \left( \bigcup_{\{\txid_{\cl}^{n} \in \T_{\aexec} \mid n \in \Nat\}} (\SO_{\aexec}^{-1})?(\txid^n_\cl) \right) \subseteq \Tx(\mkvs,\vi) \cup \T_\rd\).
To prove \cref{equ:ryw-sound-inv}, let \( \T'_\rd = \T_\rd \) if the new transaction \( \txid_\cl^n\) has writes, otherwise \( \T'_\rd = \T_\rd \cup \Set{\txid_\cl^n}\).
First we have \( I(\aexec', \cl) = \left(\bigcup\limits_{\{\txid_{\cl}^{m} \in \T_{\aexec'} \mid m \in \Nat\}} (\SO_{\aexec'}^{-1})?(\txid^m_\cl) \right) \setminus \T'_{\rd} = \left( (\SO_{\aexec'}^{-1})?(\txid^n_\cl) \right) \setminus \T'_\rd \)
since that  \( \txid^n_\cl \) is the latest transaction committed by the client \( \cl \).
For any transaction \( \txid \in (\SO_{\aexec'}^{-1})?(\txid^n_\cl) \) that has write,
because \( \WTx( \mkvs_{\aexec'}(\ke, i) ) \leq \txid \implies i \in \vi'(\ke) \),
then \( \txid \in \Tx(\mkvs_{\aexec'}, \vi') \).
Thus we have the proof of \cref{equ:ryw-sound-inv}.


The execution test $\ET_{\RYW}$ is complete with respect to 
the axiomatic specification $(\RP_{\LWW}, \Set{\lambda \aexec.\PO_{\aexec} })$. 
Let $\aexec$ be an abstract execution that satisfies the specification
$\CMa(\RP_{\LWW}, \Set{\lambda \aexec.\PO_{\aexec} })$.
Suppose that $\T_{\aexec} = \Set{\txid_{i}}_{i=1}^{n}$, and for any $i=1,\cdots, n-1$,
$\txid_{i} \xrightarrow{\AR_{\aexec}} \txid_{i+1}$.
Fix $i=1,\cdots,n$, and let a view $\vi_{i} = \getView(\aexec, \VIS^{-1}_{\aexec}(\txid_{i}))$.
We construct the view after update \( \vi'_i\) depending on whether \( \txid_i \) is the last transaction from the client.
If the transaction \( \txid'_i = \min_{\SO_\aexec}(\Setcon{\txid'}{\txid_i \toEdge{\SO_\aexec} \txid' } ) \)  is defined,
then \( \vi'_i = \getView(\aexec, \T_i) \) where \( \T_i \subseteq (\AR_{\aexec}^{-1})?(\txid_i) \cap \VIS_\aexec^{-1}(\txid'_i) \) for some \( \T_i \).
Given the specification \( \lambda \aexec.\PO_{\aexec} \), 
we know \( \SO_\aexec^{-1}(\txid'_i) \subseteq \VIS_\aexec^{-1}(\txid'_i) \).
We pick \( \T_i = \AR_\aexec^{-1})?(\txid_i) \cap \SO_\aexec^{-1}(\txid'_i) = (\SO_\aexec^{-1})?(\txid_i) \).
Since \( \vi'_i = \getView(\aexec, \T_i) \), therefore \( \ET_\RYW \vdash (\hh_{\cut(\aexec, i-1)}, \vi_i) \csat \TtoOp{T}_{\aexec}(\txid_{i}) : \vi'_{i} \).
If there is no other transaction after \( \txid_i \) from the same client,
we pick \( \vi'_i = \getView(\aexec, \T_i) \) where \( \T_i = (\SO_\aexec^{-1})?(\txid_i) \),
so \( \ET_\RYW \vdash (\hh_{\cut(\aexec, i-1)}, \vi_i) \csat \TtoOp{T}_{\aexec}(\txid_{i}) : \vi'_{i} \).

\subsection{ \( \WFR \) }

\label{sec:sound-complete-wfr}

The write-read relation is defined as 
\( \WR(\aexec, \ke) \defeq \Setcon{ (\txid, \txid') }{ \exsts{\val} (\otW, \ke, \val) \in_\aexec \txid \land (\otR, \ke, \val) \in_\aexec \txid' \land \txid = \max_\AR(\VIS^{-1}(\txid')) }\).
The notation \( \WR_\aexec \) is defined as \( \WR_\aexec \defeq \bigcup\limits_{\ke \in \Keys} \WR(\aexec, \ke) \).

The execution test $\ET_\MW$ is sound with respect to the axiomatic specification 
\( (\RP_{\LWW}, \Set{\lambda \aexec. \WR_\aexec ; (\PO_{\aexec})? ; \VIS_{\aexec} })\).
We pick the invariant as \( I( \aexec, \cl ) = \emptyset \), given the fact of no constraint on the view after update.
Assume a key-value store \( \mkvs \) and its corresponding abstract execution \( \aexec \) so that \( \mkvs = \mkvs_\aexec \).
Given an update on the key-value store that satisfies \( \WFR \), \ie \(\ET_\MW \vdash (\mkvs, \vi) \csat \f : \vi' \),
by the \cref{def:et_sound}, it remains to prove that
\[
    \begin{array}{@{}l@{}}
        \fora{ \aexec', \txid, \txid' } 
        \aexec' = \extend(\aexec, \txid, \Tx(\mkvs, \vi) , \f ) 
        \land (\txid' ,\txid)  \in \WR(\aexec',\ke) ; (\PO_{\aexec'})? ; \VIS_{\aexec'}   \\
        \qquad {} \implies \txid' \in \Tx(\mkvs, \vi) 
    \end{array}
\]
Suppose \( (\txid' ,\txid)  \in \WR(\aexec', \ke) ; (\PO_{\aexec'})? ; \VIS_{\aexec'} \) for some key \( \ke \),
that is, \( \txid' \toEdge{\WR(\aexec', \ke)} \txid'' \toEdge{\SO_{\aexec'}?} \txid''' \toEdge{\VIS_{\aexec'}} \txid \) for some transaction \( \txid''' \).
It immediately implies that \( \txid''' \in \Tx(\mkvs, \vi)  \).
Because \( \txid' \toEdge{\WR(\aexec', \ke)} \txid'' \), there exists an index \( i \) such that \( \mkvs(\ke, i) = (\stub, \txid', \txid'' \cup \stub) \).
By the execution test \( \ET_\WFR \), we have \( \i \in \vi( \ke ) \), thus \( \txid' \in \Tx(\mkvs, \vi ) \).


The execution test $\ET_\WFR$ is complete with respect to the axiomatic specification 
\( (\RP_{\LWW}, \{\lambda \aexec. \WR(\aexec', \ke) ; (\PO_{\aexec'})? ; \VIS_{\aexec'} \})\).
Suppose that $\T_{\aexec} = \Set{\txid_{i}}_{i=1}^{n}$, and for any $i=1,\cdots, n-1$,
$\txid_{i} \xrightarrow{\AR_{\aexec}} \txid_{i+1}$.
Fix $i=1,\cdots,n$, and let a view $\vi_{i} = \getView(\aexec, \VIS^{-1}_{\aexec}(\txid_{i}))$.
It is sufficient to prove the following:
\[
    \fora{\ke, \ke', i, j, \txid'} j \in \vi(\ke) \land \txid' \in \RTx(\hh(\ke', i)) \land \txid' {\xrightarrow{\PO?}} \WTx(\ke, j) ) \implies i \in \vi(\ke')
\]
Given a key \( \ke \) and an index \( j \) such that \( j \in \vi(\ke) \), 
it means that the writer \( \txid \) of the version is visible, \ie \( \txid \in \VIS_{\aexec'}^{-1}(\txid_i) \).
Assume some \( \txid' \) such that \( (\txid', \txid) \in (\PO_{\aexec'})? \) and reads a version of some key \( \mkvs(\ke',i) \).
Therefore, we know the writer of the key \( \txid'' = \WTx(\mkvs(\ke',i)) \) has a write-read edge to \( \txid' \), 
\ie \( (\txid'', \txid') \in \WR(\aexec', \ke)  \).
By the constraint on abstract execution \( \aexec \), we know \( \txid'' \in \VIS^{-1}_{\aexec'}(\txid_{i}) \),
which means \( i \in \vi(\ke')\) by the definition of \( \getView \).

\subsection{\( \CC \)}

\begin{lemma}
    \label{lem:aexec-spec-cc}
    For any abstract execution \( \aexec \) under last-write-win, if it satisfies the following:
    \[
        (\WR_\aexec \cup \SO_\aexec)^{+} ; \VIS_\aexec \subseteq \VIS_\aexec \quad \WR_\aexec \subseteq \VIS_\aexec \quad \SO_\aexec \subseteq \VIS_\aexec
    \]
    There exists a new abstract execution \( \aexec' \) where \( \T_\aexec = \T_{\aexec'} \) and \( \AR_\aexec = \AR_{\aexec'} \).
    Let the visibility relation be \( \VIS_{\aexec'} = \WR_\aexec \cup \SO_\aexec \), so that \( \VIS_{\aexec'} ; \VIS_{\aexec'} \subseteq \VIS_{\aexec'} \).
    Under last-write-win, \( \fora{\txid} \TtoOp{T}_{\aexec}(\txid) = \TtoOp{T}_{\aexec'}(\txid) \).
\end{lemma}
\begin{proof}
    To recall, the write-read relation under a key \( \WR(\aexec, \ke) \) is defined as 
    \( \WR(\aexec, \ke) \defeq \Setcon{ (\txid, \txid') }{ \exsts{\val} (\otW, \ke, \val) \in_\aexec \txid \land (\otR, \ke, \val) \in_\aexec \txid' \land \txid = \max_\AR(\VIS^{-1}(\txid')) }\).
    For any transaction \( \txid \) in the abstract execution \( \aexec\),
    if it read a key, \( (\otR, \ke, \val ) \in_{\aexec} \txid  \),
    since \( \WR_{\aexec} \in \VIS_{\aexec'} \), the same transaction \( \txid \) must read the same value in the new abstract execution \( \aexec' \).
    Note that a transaction can always write to a key, since there is no constraint for write.
    Thus we have the proof.
    
    %Given an \( \aexec \) that satisfies the following
    %\[
        %(\WR_\aexec \cup \SO)_\aexec^{+} ; \VIS_\aexec \subseteq \VIS_\aexec \quad \WR_\aexec \subseteq \VIS_\aexec \quad \SO_\aexec \subseteq \VIS_\aexec
    %\]
    %we add more visibility relation to each transaction following the order of arbitration \( \AR \) until the visibility is transitive.
    %Let \( R_i \) denote the new visibility for transaction \( \txid_i \) such that
    %the visibility relation before (including) the i-\emph{th} transaction \( \txid_i \) is transitive, 
    %thus \( R_i\projection{2} = \Set{\txid_i}\).
    %Let notation \( lr^{\cl}_i \) be a function associated with the client \( \cl \) when i-\emph{th} transaction commits.
    %The function keeps tracks of the most recent read of keys from the client \( \cl \).
    %Note that if a key is not in the domain of the function \( lr^{\cl}_i \), 
    %it means the client \( \cl \) have not read such key.

    %Now initial let \( lr^\cl_0 = \emptyset \) for all possible clients \( \cl \) appear in the \( \aexec \).
    %Suppose that $\T_{\aexec} = \Set{\txid_{i}}_{i=1}^{n}$, and for any $i=1,\cdots, n-1$,
    %$\txid_{i} \xrightarrow{\AR_{\aexec}} \txid_{i+1}$.
    %Let \( \aexec_i = \Cuts(\aexec, i) \) and \( \VIS_i = \WR_{\aexec_i} \cup \SO_{\aexec_i} \cup \bigcup\limits_{ 0 \leq k \leq i} R_i \).
    %For each step, says i-\emph{th} step, we add more visibility \( R_i \) and preserve the following:
    %\begin{gather}
        %\VIS_i ; \VIS_i \subseteq \VIS_i \label{equ:vis-i-transitive} \\
        %\fora{\txid} (\txid,\txid_i) \in R_i \implies (\txid, \txid_i) \in (\WR_i \cup \SO_i)^{+}
        %\label{equ:last-read-correct}
    %\end{gather}
    
    %Let start with \( i = 1 \) and \( R_1 = \emptyset \).
    %Assume it is from client \( \cl \).
    %There is no transaction committed before, so \( \VIS_1 = \emptyset \) and \( \VIS_1 ; \VIS_1 \subseteq \VIS_1 \) as \cref{equ:vis-i-transitive}.
    %For any read \( (\otR, \ke, \val ) \in \txid_1 \), let \( lr^{\cl}_1 = lr^{\cl}_0\rmto{\ke}{\val} \).
    %For other clients \( \cl' \) different from \( \cl \), let \( lr^{\cl'}_1 = lr^{\cl'}_0 \), thus \cref{equ:last-read-correct} holds.

    %Suppose the (i-1)-\emph{th} step satisfies \cref{equ:vis-i-transitive} and \cref{equ:last-read-correct}.
    %Let consider i-\emph{th} step and the transaction \( \txid_i \).
    %We first extend more visibility by closing with respect to \( \WR_i \)
    %and prove those extension does not affect any read from the transaction \( \txid_i \).
    %Then we will do the same for \( \SO_i \).

    %For any read \( (\otR, \ke, \val ) \in \txid_i \),
    %there must be a transaction \( \txid_j \) that \( \txid_j \toEdge{\WR(\aexec_i,\ke), \AR} \txid_i \) and \( j < i \).
    %Let consider all the visible transactions of \( \txid_j \).
    %Assume a transaction \( \txid' \in \VIS_{i-1}^{-1}(\txid_j) \), 
    %thus \( \txid' \in \VIS_{j}^{-1}(\txid_j) = ( \WR_j \cup \SO_j \cup R_j )^{-1}(\txid_j) \).
    %If \( \txid' \in \WR_j^{-1}(\txid_j) \), then \( \txid' \in \WR_i^{-1}(\txid_j) \).
    %Because \( \WR_i \subseteq \VIS_i \) and \( \WR_i ; \VIS_i \subseteq \VIS_i \),
    %we know \( \txid' \in  \VIS_{\aexec_i}^{-1}(\txid_i) \).
    %It is the same when \( \txid' \in \SO_j^{-1}(\txid_j) \).
    %Now if \( \txid' \in R_j^{-1}(\txid_j) \), by \cref{equ:last-read-correct}, \( \WR_i^{-1}(\txid_j) \subseteq \VIS_i\) 
    %and \( (\WR_i \cup \SO_i)^{+} ; \VIS_i \subseteq \VIS_i \), we know \( \)
\end{proof}

The execution test $\ET_\CC$ is sound with respect to the axiomatic specification 
\( (\RP_{\LWW}, \Set{\lambda \aexec. \VIS_{\aexec} ; \VIS_{\aexec}, \lambda \aexec \ldotp \SO_\aexec })\).
By \cref{lem:aexec-spec-cc}, 
It is sufficient to prove soundness with respect to 
\( (\RP_{\LWW}, \Set{\lambda \aexec. ( \SO_{\aexec} \cup \WR_{\aexec} ) ; \VIS_{\aexec}, \lambda \aexec \ldotp \SO_\aexec })\).
We pick an invariant for the \( \ET_\CC \) as the union of those for \( \MRd\) and \( \RYW \) shown in the following:
\[  
\begin{rclarray}
    I_1(\aexec, \cl) & = & \left( \bigcup\limits_{\{\txid_{\cl}^{n} \in \T_{\aexec} \mid n \in \Nat\}} \VIS_{\aexec}^{-1}(\txid^n_\cl) \right) \setminus \T_\rd \\
    I_2(\aexec, \cl) & = & \left( \bigcup\limits_{\{\txid_{\cl}^{n} \in \T_{\aexec} \mid n \in \Nat\}} (\SO_{\aexec}^{-1})?(\txid^n_\cl) \right) \setminus \T_\rd
\end{rclarray}
\]
where \( \T_\rd \) is all the read-only transactions included in both 
\( \left( \bigcup\limits_{\{\txid_{\cl}^{n} \in \T_{\aexec} \mid n \in \Nat\}} \VIS_{\aexec}^{-1}(\txid^n_\cl) \right)\) 
and \( \left( \bigcup\limits_{\{\txid_{\cl}^{n} \in \T_{\aexec} \mid n \in \Nat\}} (\SO_{\aexec}^{-1})?(\txid^n_\cl) \right) \).
Suppose a kv-store \( \mkvs \), an abstract execution \( \aexec \) and a view \( \vi \)
such that \( \mkvs_\aexec = \mkvs \) and \( I_1(\aexec, \cl) \cup I_2(\aexec, \cl) \subseteq \Tx(\mkvs, \vi) \).
Assume that a fresh transaction identifier \( \txid_\cl^n \) with fingerprint \( \f \) commits to 
the kv-store \( \mkvs \) under view \( \vi \) and the view after is updated to \( \vi' \), that is, 
\( \ET_\CC \vdash (\mkvs, \vi) \csat \f : \vi' \).
Let a new abstract execution \( \aexec' = \extend(\aexec, \txid_\cl^n, \f, \Tx(\mkvs, \vi) \cup \T_\rd) \).
We are about to prove there exists an extra set of read-only transaction \( \T'_\rd \) such that:
\begin{gather}
    \fora{\txid} (\txid, \txid_\cl^n) \in \SO_{\aexec'} \implies \txid \in \Tx(\mkvs, \vi) \cup \T_\rd \cup \T'_\rd \label{equ:cc-sound-update-so}\\
    \fora{\txid} (\txid, \txid_\cl^n) \in ( \SO_{\aexec'} \cup \WR_{\aexec'} ) ; \VIS_{\aexec'} \implies \txid \in \Tx(\mkvs, \vi) \cup \T_\rd \cup \T'_\rd \label{equ:cc-sound-update-visvis}\\
    I_1(\aexec',\cl) \cup I_2(\aexec',\cl) \subseteq \Tx(\mkvs_{\aexec'}, \vi') \label{equ:ryw-sound-inv} 
\end{gather}
The invariant \( I_2 \) implies \cref{equ:cc-sound-update-so} as the same as \cref{sec:sound-complete-ryw}.
To prove \cref{equ:cc-sound-update-visvis}, let \( \T'_\rd = \emptyset \) initially,
and more more read-only transactions will be added in \( \T'_\rd \) until the \cref{equ:cc-sound-update-visvis} holds.
Assume transaction \( \txid \) such that \( ((\txid, \txid_\cl^n) \in ( \SO_{\aexec'} \cup \WR_{\aexec'} ) ; \VIS_{\aexec'}) \).
That is, there exists some transaction \( \txid' \) such that
\( \txid \toEdge{( \SO_{\aexec'} \cup \WR_{\aexec'} )}  \txid' \toEdge{\VIS_{\aexec'}} \txid_\cl^n\).
We consider two cases for \( \txid' \): \textbf{(i)} \( \txid' \) is also visible by previous transactions from the same client; 
\textbf{(ii)} \( \txid' \) is a newly visible transaction for the client.
For the first case, it means \( \txid' \toEdge{\VIS_{\aexec'}} \txid_\cl^m\) for some \( m < n \) and then \( \txid_\cl^m \toEdge{\SO} \txid_\cl^n \).
Since \( \txid \toEdge{( \SO_{\aexec'} \cup \WR_{\aexec'} )}  \txid' \) and \( \SO_{\aexec'} \cup \WR_{\aexec'} \subseteq \VIS_{\aexec'} \),
now we have \( \txid \toEdge{ \VIS_{\aexec'} }  \txid'  \).
Because of \( I_1 \), then \( \txid \in I_1 \cup \T_\rd \subseteq \Tx(\mkvs, \vi) \cup \T_\rd \).

For the second case where \( \txid' \) is a newly visible transaction for the client \( \cl \).
Since \( \txid' \VIS_{\aexec'} \txid_\cl^n \) so \( \txid' \in Tx(\mkvs, \vi) \T_\rd \cup \T'_\rd\).
More specifically, \( \txid' \) is not visible for the client before, we know \( \txid' \in Tx(\mkvs, \vi) \cup \T'_\rd\).
If \( \txid' \) writes to some keys, \ie \( \txid' \in Tx(\mkvs, \vi) \), 
by the execution tests for \( \MW \) and \( \WFR \) (the proofs follows \cref{sec:sound-complete-mw} and \cref{sec:sound-complete-wfr}),
the \( \txid \) is either already in \( \Tx(\mkvs, \vi) \), 
or \( \txid \) is a read-only and we include it in \( \T'_\rd \).
If \( \txid' \) is a read-only transaction,
given that \( \T'_\rd \) initially is empty set,
we know there exists a third transaction \( \txid'' \) that at least writes to some key
and it satisfies \( \txid \toEdge{( \SO_{\aexec'} \cup \WR_{\aexec'} )}  \txid' \toEdge{( \SO_{\aexec'} \cup \WR_{\aexec'} )}  \txid'' \toEdge{\VIS_{\aexec'}} \txid_\cl^n \).
Since \( \txid'' \) has a write, it means 
\( \txid \toEdge{( \SO_{\aexec'} \cup \WR_{\aexec'} )}  \txid' \toEdge{( \SO_{\aexec'} )}  \txid'' \toEdge{\VIS_{\aexec'}} \txid_\cl^n \).
Because \( \SO \) is transitive, we have \( \txid \toEdge{( \SO_{\aexec'} \cup \WR_{\aexec'} )} \txid'' \toEdge{\VIS_{\aexec'}} \txid_\cl^n \).
By previous case we already know \( \txid \in \Tx(\mkvs, \vi) \cup \T'_\rd \).

Finally the new abstract execution test preserve the invariant \( I_1 \) and \( I_2 \) because of \( \MW \) and \( \RYW \).
The proofs are the same as those in \cref{sec:sound-complete-mr} and \cref{sec:sound-complete-ryw}.

\subsection{\( \UA \)}
The set of transactions in the abstract execution \( \aexec \)  that write to a key \( \ke \) is defined as \( \WTa(\aexec, \ke) \defeq \Setcon{\txid}{(\otW, \stub, \stub) \in_\aexec \txid} \).
%The notation \( \WTa(\aexec) \) then is defined as \( \WTa(\aexec) \defeq \bigcup\limits_{\ke \in \Keys}  \WTa(\aexec, \ke)\),
%the notation \( [\WTa(\aexec)] \defeq \Setcon{(\txid, \txid)}{\txid \in \WTa(\aexec)}\).
The execution test $\ET_\UA$ is sound with respect to the axiomatic specification 
\[ 
    (\RP_{\LWW}, \Set{\lambda \aexec. \Setcon{(\txid', \txid)}{\exsts{\ke} \txid, \txid' \in \WTa(\aexec, \ke) \land \txid' \toEdge{\AR_\aexec} \txid } })
\]
We pick the invariant as \( I( \aexec, \cl ) = \emptyset \), given the fact of no constraint on the view after update.
Assume a key-value store \( \mkvs \) and its corresponding abstract execution \( \aexec \) so that \( \mkvs = \mkvs_\aexec \).
Given an update on the key-value store that satisfies \( \UA \), \ie \(\ET_\UA \vdash (\mkvs, \vi) \csat \f : \vi' \),
by the \cref{def:et_sound}, it remains to prove that there exists a set of read-only transaction \( \T_{\rd} \) such that
\[
    \begin{array}{@{}l@{}}
        \fora{ \aexec', \txid, \txid', \ke } 
        \aexec' = \extend(\aexec, \txid, \Tx(\mkvs, \vi) \cup \T_\rd, \f ) 
        \land \txid', \txid \in \WTa(\aexec', \ke) \land  \txid' \toEdge{\AR_{\aexec'}} \txid  \\
        \qquad {} \implies \txid' \in \Tx(\mkvs, \vi)
    \end{array}
\]
Given a new transaction \( \txid \) and some transaction \( \txid' \),
assume both of them write to a key \( \ke \), and \( \txid' \toEdge{\AR_{\aexec'}} \txid \).
We have \( \WTx(\mkvs(\ke, i)) = \txid \) for some index \( i \).
By the execution test of \( \UA \), we know \( i \in \vi(\ke) \) therefore \( \txid' \in \Tx(\mkvs, \vi) \).


The execution test $\ET_\UA$ is complete with respect to the axiomatic specification 
\[ 
    (\RP_{\LWW}, \Set{\lambda \aexec. \Setcon{(\txid', \txid)}{\exsts{\ke} \txid, \txid' \in \WTa(\aexec, \ke) \land \txid' \toEdge{\AR_\aexec} \txid } })
\].
Suppose that $\T_{\aexec} = \Set{\txid_{i}}_{i=1}^{n}$, and for any $i=1,\cdots, n-1$,
$\txid_{i} \xrightarrow{\AR_{\aexec}} \txid_{i+1}$.
Fix $i=1,\cdots,n$, and let a view $\vi_{i} = \getView(\aexec, \VIS^{-1}_{\aexec}(\txid_{i}))$.
It is sufficient to prove the following:
\[
    \ET_\UA \vdash ( \mkvs_{\cut(\aexec, i-1)}, \vi_i ) \csat \TtoOp{T}_{\aexec}(\txid_{i}) : \vi'_{i}
\]
for some \( \vi'_i \) such that \( \vi' \subseteq (\AR^{-1}_\aexec)?(\txid_i) \).
By the constraint of \( \aexec \), 
for any transaction \( \txid \) that writes to the same key as \( \txid_i \) and committed before \( \txid_i \), 
they are included in the visible set \(\txid \in \VIS^{-1}_{\aexec}(\txid_{i}) \).
Note that \( \txid \toEdge{\AR_\aexec} \txid_i \iff \txid \in \mkvs_{\cut(\aexec,i-1)}\).
It means 
\[ 
    \fora{\ke, i} \mkvs_{\cut(\aexec,i-1)}(\ke, i) \text{ is defined } \land  (\otW, \ke, \stub) \in \TtoOp{\T}_\aexec(\txid_i) \implies i \in \vi_i(\ke)
\]
thus we have the proof.

\subsection{ \( \SER \)}

The execution test $\ET_\UA$ is sound with respect to the axiomatic specification 
\[ 
    (\RP_{\LWW}, \Set{\lambda \aexec. \AR })
\]
We pick the invariant as \( I( \aexec, \cl ) = \emptyset \), given the fact of no constraint on the view after update.
Assume a key-value store \( \mkvs \) and its corresponding abstract execution \( \aexec \) so that \( \mkvs = \mkvs_\aexec \).
Given an update on the key-value store that satisfies \( \SER \), \ie \(\ET_\SER \vdash (\mkvs, \vi) \csat \f : \vi' \),
by the \cref{def:et_sound}, it remains to prove that there exists a set of read-only transactions \( \T_\rd \) such that
\[
    \begin{array}{@{}l@{}}
        \fora{ \aexec', \txid, \txid', \ke } 
        \aexec' = \extend(\aexec, \txid, \Tx(\mkvs, \vi) \cup \T_\rd, \f ) 
        \land \txid', \txid \in \WTa(\aexec', \ke) \land  \txid' \toEdge{\AR_{\aexec'}} \txid  \\
        \qquad {} \implies \txid' \in \Tx(\mkvs, \vi) \cup \T_\rd
    \end{array}
\]
Since the abstract execution satisfies the constraint for \( \SER \), \ie \( \AR \subseteq \VIS \), we know \( \AR = \VIS \).
Since \( \Tx(\mkvs, \vi)  \) contains all transactions that write at least a key, 
we can pick a \( \T_\rd \) such that \( \Tx(\mkvs, \vi) \cup \T_\rd = \T_\aexec\),
which gives us the proof.

The execution test $\ET_\UA$ is complete with respect to the axiomatic specification \( (\RP_{\LWW}, \Set{\lambda \aexec. \AR_\aexec }) \).
Suppose that $\T_{\aexec} = \Set{\txid_{i}}_{i=1}^{n}$, and for any $i=1,\cdots, n-1$,
$\txid_{i} \xrightarrow{\AR_{\aexec}} \txid_{i+1}$.
Fix $i=1,\cdots,n$, and let a view $\vi_{i} = \getView(\aexec, \VIS^{-1}_{\aexec}(\txid_{i}))$.
It is sufficient to prove the following:
\[
    \ET_\SER \vdash ( \mkvs_{\cut(\aexec, i-1)}, \vi_i ) \csat \TtoOp{T}_{\aexec}(\txid_{i}) : \vi'_{i}
\]
Because \( \VIS^{-1}(\txid_i) = \AR^{-1}(\txid_i) = \Setcon{\txid }{\txid \texttt{ appears in } \mkvs_{\cut(\aexec, i-1)} }\),
so for any key \( \ke \) and index \( i \) such that \( 0 \leq i < \abs{\mkvs_{\cut(\aexec, i-1)}(\ke)} \),
the i-\emph{th} version of the key contains in the view, \ie \( i \in \vi(\ke)\).
