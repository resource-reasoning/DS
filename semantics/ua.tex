\subsection{Update Atomic}
\begin{figure}
\hrule
\begin{tabular}{@{} c c@{}}

\begin{halfsubfig}
\begin{centertikz}

\begin{pgfonlayer}{foreground}
%Uncomment line below for help lines
%\draw[help lines] grid(5,4);

%Location x
\node(locx)  {$\ke_\vx \mapsto$};

\matrix(versionx) [version list]
    at ([xshift=\tikzkvspace]locx.east) {
    {a} & $\txid_0$ \\
    {a} & $\emptyset$ \\
};

\tikzvalue{versionx-1-1}{versionx-2-1}{locx-v0}{0};

%Location y
\path (locx.south) + (0,\tikzkeyspace) node (locf1) {$\ke_{\pv{f1}} \mapsto$};
\matrix(versionf1) [version list]
    at ([xshift=\tikzkvspace]locf1.east) {
    {a} & $\txid_0$ \\
    {a} & $\emptyset$ \\
};
\tikzvalue{versionf1-1-1}{versionf1-2-1}{locf1-v0}{0};

%Location y
\path (locf1.south) + (0,\tikzkeyspace) node (locf2) {$\ke_{\pv{f2}} \mapsto$};
\matrix(versionf2) [version list]
    at ([xshift=\tikzkvspace]locf2.east) {
    {a} & $\txid_0$ \\
    {a} & $\emptyset$ \\
};
\tikzvalue{versionf2-1-1}{versionf2-2-1}{locf2-v0}{0};

% \draw[-, red, very thick, rounded corners] ([xshift=-5pt, yshift=5pt]locx-v1.north east) |- 
%  ($([xshift=-5pt,yshift=-5pt]locx-v1.south east)!.5!([xshift=-5pt, yshift=5pt]locy-v0.north east)$) -| ([xshift=-5pt, yshift=5pt]locy-v0.south east);

%blue view - I should  check whether I can use pgfkeys to just declare the list of locations, and then add the view automatically.
\draw[-, blue, very thick, rounded corners=10pt]
 ([xshift=-2pt, yshift=20pt]locx-v0.north east) node (tid1start) {} -- 
 ([xshift=-2pt, yshift=-5pt]locf2-v0.south east);
 
 \path (tid1start) node[anchor=south, rectangle, fill=blue!20, draw=blue, font=\small, inner sep=1pt] {$\thid_3$};

%red view
\draw[-, red, very thick, rounded corners = 10pt]
 ([xshift=-5pt, yshift=5pt]locx-v0.north east) -- 
 ([xshift=-5pt, yshift=-10pt]locf2-v0.south east) node (tid2start) {};
 
\path (tid2start) node[anchor=north, rectangle, fill=red!20, draw=red, font=\small, inner sep=1pt] {$\thid_2$};
 
 %green view
\draw[-, DarkGreen, very thick, rounded corners = 10pt]
 ([xshift=-16pt, yshift=8pt]locx-v0.north east) node (tid3start) {}-- 
 ([xshift=-16pt, yshift=-5pt]locf2-v0.south east);
 
 \path (tid3start) node[anchor=south, rectangle, fill=DarkGreen!20, draw=DarkGreen, font=\small, inner sep=1pt] {$\thid_1$};

\end{pgfonlayer}
\end{centertikz}%
\caption{Initial configuration}
\label{fig:ua-init}
\end{halfsubfig}
%
&
%
\begin{halfsubfig}
\begin{centertikz}
\begin{pgfonlayer}{foreground}
%Uncomment line below for help lines
%\draw[help lines] grid(5,4);

\node(locx)  {$\ke_\vx \mapsto$};

\matrix(versionx) [version list]
    at ([xshift=\tikzkvspace]locx.east) {
    {a} & $\txid_0$ & {a} & \(\txid_1\)\\
    {a} & $\Set{\txid_1}$ & {a} & \(\emptyset\)\\
};

\tikzvalue{versionx-1-1}{versionx-2-1}{locx-v0}{0};
\tikzvalue{versionx-1-3}{versionx-2-3}{locx-v1}{1};


\path (locx.south) + (0,\tikzkeyspace) node (locf1) {$\ke_{\pv{f1}} \mapsto$};
\matrix(versionf1) [version list]
    at ([xshift=\tikzkvspace]locf1.east) {
    {a} & $\txid_0$ & {a} & $\txid_1$\\
    {a} & $\Set{\txid_1}$ & {a} & $\emptyset$\\
};
\tikzvalue{versionf1-1-1}{versionf1-2-1}{locf1-v0}{0};
\tikzvalue{versionf1-1-3}{versionf1-2-3}{locf1-v1}{1};

%Location y
\path (locf1.south) + (0,\tikzkeyspace) node (locf2) {$\ke_{\pv{f2}} \mapsto$};
\matrix(versionf2) [version list]
    at ([xshift=\tikzkvspace]locf2.east) {
    {a} & $\txid_0$ \\
    {a} & $\emptyset$ \\
};
\tikzvalue{versionf2-1-1}{versionf2-2-1}{locf2-v0}{0};


% \draw[-, red, very thick, rounded corners] ([xshift=-5pt, yshift=5pt]locx-v1.north east) |- 
%  ($([xshift=-5pt,yshift=-5pt]locx-v1.south east)!.5!([xshift=-5pt, yshift=5pt]locy-v0.north east)$) -| ([xshift=-5pt, yshift=5pt]locy-v0.south east);

%blue view - I should  check whether I can use pgfkeys to just declare the list of locations, and then add the view automatically.
\draw[-, blue, very thick, rounded corners=10pt]
 ([xshift=-2pt, yshift=20pt]locx-v0.north east) node (tid1start) {} -- 
 ([xshift=-2pt, yshift=-5pt]locf2-v0.south east);
 
 \path (tid1start) node[anchor=south, rectangle, fill=blue!20, draw=blue, font=\small, inner sep=1pt] {$\thid_3$};

%red view
\draw[-, red, very thick, rounded corners = 10pt]
 ([xshift=-5pt, yshift=5pt]locx-v0.north east) -- 
 ([xshift=-5pt, yshift=-10pt]locf2-v0.south east) node (tid2start) {};
 
\path (tid2start) node[anchor=north, rectangle, fill=red!20, draw=red, font=\small, inner sep=1pt] {$\thid_2$};
 
 %green view
\draw[-, DarkGreen, very thick, rounded corners = 10pt]
 ([xshift=-16pt, yshift=8pt]locx-v1.north east) node (tid3start) {}-- 
 ([xshift=-16pt, yshift=-5pt]locf1-v1.south east) --
 ([xshift=-16pt, yshift=5pt]locf2-v0.north east) -- 
 ([xshift=-16pt, yshift=-5pt]locf2-v0.south east);
 
 \path (tid3start) node[anchor=south, rectangle, fill=DarkGreen!20, draw=DarkGreen, font=\small, inner sep=1pt] {$\thid_1$};

\end{pgfonlayer}
\end{centertikz}
\caption{After \(\txid_1\)}
\label{fig:ua-after-tx1}
\end{halfsubfig}

\\
\begin{subfigure}{0.45\textwidth}
\begin{centertikz}%
\begin{pgfonlayer}{foreground}
%Uncomment line below for help lines
%\draw[help lines] grid(5,4);


\node(locx)  {$\ke_\vx \mapsto$};

\matrix(versionx) [version list, column 2/.style = {text width=14mm}]
    at ([xshift=\tikzkvspace]locx.east) {
    {a} & $\txid_0$ & {a} & $\txid_1$ & {a} & $\txid_2$\\
    {a} & $\Set{\txid_1, \txid_2}$ & {a} & $\emptyset$ & {a} & $\emptyset$\\
};

\tikzvalue{versionx-1-1}{versionx-2-1}{locx-v0}{0};
\tikzvalue{versionx-1-3}{versionx-2-3}{locx-v1}{1};
\tikzvalue{versionx-1-5}{versionx-2-5}{locx-v2}{1};


\path (locx.south) + (0,\tikzkeyspace) node (locf1) {$\ke_{\pv{f1}} \mapsto$};
\matrix(versionf1) [version list]
    at ([xshift=\tikzkvspace]locf1.east) {
    {a} & $\txid_0$ & {a} & $\txid_1$\\
    {a} & $\Set{\txid_1}$ & {a} & $\emptyset$\\
};
\tikzvalue{versionf1-1-1}{versionf1-2-1}{locf1-v0}{0};
\tikzvalue{versionf1-1-3}{versionf1-2-3}{locf1-v1}{1};

%Location y
\path (locf1.south) + (0,\tikzkeyspace) node (locf2) {$\ke_{\pv{f2}} \mapsto$};
\matrix(versionf2) [version list]
    at ([xshift=\tikzkvspace]locf2.east) {
    {a} & $\txid_0$ & {a} & \(\txid_2\) \\
    {a} & $\emptyset$ & {a} & \(\emptyset\) \\
};
\tikzvalue{versionf2-1-1}{versionf2-2-1}{locf2-v0}{0};
\tikzvalue{versionf2-1-3}{versionf2-2-3}{locf2-v1}{1};


% \draw[-, red, very thick, rounded corners] ([xshift=-5pt, yshift=5pt]locx-v1.north east) |- 
%  ($([xshift=-5pt,yshift=-5pt]locx-v1.south east)!.5!([xshift=-5pt, yshift=5pt]locy-v0.north east)$) -| ([xshift=-5pt, yshift=5pt]locy-v0.south east);

%blue view - I should  check whether I can use pgfkeys to just declare the list of locations, and then add the view automatically.
\draw[-, blue, very thick, rounded corners=10pt]
([xshift=-2pt, yshift=20pt]locx-v0.north east) node (tid1start) {} -- 
([xshift=-2pt, yshift=-5pt]locf2-v0.south east);
 
\path (tid1start) node[anchor=south, rectangle, fill=blue!20, draw=blue, font=\small, inner sep=1pt] {$\thid_3$};

%red view
\draw[-, red, very thick, rounded corners = 10pt]
([xshift=-5pt, yshift=5pt]locx-v2.north east) -- 
([xshift=-5pt, yshift=-5pt]locx-v2.south east) --
([xshift=-5pt, yshift=5pt]locf1-v0.north east) -- 
([xshift=-5pt, yshift=-5pt]locf1-v0.south east) --
([xshift=-5pt, yshift=5pt]locf2-v1.north east) -- 
([xshift=-5pt, yshift=-10pt]locf2-v1.south east) node (tid2start) {};

\path (tid2start) node[anchor=north, rectangle, fill=red!20, draw=red, font=\small, inner sep=1pt] {$\thid_2$};
 
 %green view
\draw[-, DarkGreen, very thick, rounded corners = 10pt]
([xshift=-16pt, yshift=8pt]locx-v1.north east) node (tid3start) {}-- 
([xshift=-16pt, yshift=-5pt]locx-v1.south east) --
([xshift=-16pt, yshift=5pt]locf1-v1.north east) -- 
([xshift=-16pt, yshift=-5pt]locf1-v1.south east) --
([xshift=-16pt, yshift=5pt]locf2-v0.north east) -- 
([xshift=-16pt, yshift=-5pt]locf2-v0.south east);

\path (tid3start) node[anchor=south, rectangle, fill=DarkGreen!20, draw=DarkGreen, font=\small, inner sep=1pt] {$\thid_1$};

\end{pgfonlayer}%
\end{centertikz}%
\caption{After \(\txid_2\)}
\label{fig:ua-after-tx2}
\end{subfigure}
%
&
%
\begin{subfigure}{0.45\textwidth}
\begin{centertikz}
\begin{pgfonlayer}{foreground}
%Uncomment line below for help lines
%\draw[help lines] grid(5,4);

\node(locx)  {$\ke_\vx \mapsto$};

\matrix(versionx) [version list, column 2/.style = {text width=14mm}]
    at ([xshift=\tikzkvspace]locx.east) {
    {a} & $\txid_0$ & {a} & $\txid_1$ & {a} & $\txid_2$\\
    {a} & $\Set{\txid_1, \txid_2}$ & {a} & $\emptyset$ & {a} & $\emptyset$\\
};

\tikzvalue{versionx-1-1}{versionx-2-1}{locx-v0}{0};
\tikzvalue{versionx-1-3}{versionx-2-3}{locx-v1}{1};
\tikzvalue{versionx-1-5}{versionx-2-5}{locx-v2}{1};


\path (locx.south) + (0,\tikzkeyspace) node (locf1) {$\ke_{\pv{f1}} \mapsto$};
\matrix(versionf1) [version list]
    at ([xshift=\tikzkvspace]locf1.east) {
    {a} & $\txid_0$ & {a} & $\txid_1$\\
    {a} & $\Set{\txid_1}$ & {a} & $\emptyset$\\
};
\tikzvalue{versionf1-1-1}{versionf1-2-1}{locf1-v0}{0};
\tikzvalue{versionf1-1-3}{versionf1-2-3}{locf1-v1}{1};

%Location y
\path (locf1.south) + (0,\tikzkeyspace) node (locf2) {$\ke_{\pv{f2}} \mapsto$};
\matrix(versionf2) [version list]
    at ([xshift=\tikzkvspace]locf2.east) {
    {a} & $\txid_0$ & {a} & \(\txid_2\) \\
    {a} & $\emptyset$ & {a} & \(\emptyset\) \\
};
\tikzvalue{versionf2-1-1}{versionf2-2-1}{locf2-v0}{0};
\tikzvalue{versionf2-1-3}{versionf2-2-3}{locf2-v1}{1};


% \draw[-, red, very thick, rounded corners] ([xshift=-5pt, yshift=5pt]locx-v1.north east) |- 
%  ($([xshift=-5pt,yshift=-5pt]locx-v1.south east)!.5!([xshift=-5pt, yshift=5pt]locy-v0.north east)$) -| ([xshift=-5pt, yshift=5pt]locy-v0.south east);

%blue view - I should  check whether I can use pgfkeys to just declare the list of locations, and then add the view automatically.
\draw[-, blue, very thick, rounded corners=10pt]
([xshift=-2pt, yshift=20pt]locx-v2.north east) node (tid1start) {} -- 
([xshift=-2pt, yshift=-7pt]locx-v2.south east) --
([xshift=-2pt, yshift=3pt]locf1-v1.north east) -- 
([xshift=-2pt, yshift=-5pt]locf1-v1.south east) --
([xshift=-2pt, yshift=5pt]locf2-v1.north east) -- 
([xshift=-2pt, yshift=-5pt]locf2-v1.south east);

\path (tid1start) node[anchor=south, rectangle, fill=blue!20, draw=blue, font=\small, inner sep=1pt] {$\thid_3$};

%red view
\draw[-, red, very thick, rounded corners = 10pt]
([xshift=-5pt, yshift=5pt]locx-v2.north east) -- 
([xshift=-5pt, yshift=-5pt]locx-v2.south east) --
([xshift=-5pt, yshift=5pt]locf1-v0.north east) -- 
([xshift=-5pt, yshift=-5pt]locf1-v0.south east) --
([xshift=-5pt, yshift=5pt]locf2-v1.north east) -- 
([xshift=-5pt, yshift=-10pt]locf2-v1.south east) node (tid2start) {};

\path (tid2start) node[anchor=north, rectangle, fill=red!20, draw=red, font=\small, inner sep=1pt] {$\thid_2$};
 
 %green view
\draw[-, DarkGreen, very thick, rounded corners = 10pt]
([xshift=-16pt, yshift=8pt]locx-v1.north east) node (tid3start) {}-- 
([xshift=-16pt, yshift=-5pt]locx-v1.south east) --
([xshift=-16pt, yshift=5pt]locf1-v1.north east) -- 
([xshift=-16pt, yshift=-5pt]locf1-v1.south east) --
([xshift=-16pt, yshift=5pt]locf2-v0.north east) -- 
([xshift=-16pt, yshift=-5pt]locf2-v0.south east);

\path (tid3start) node[anchor=south, rectangle, fill=DarkGreen!20, draw=DarkGreen, font=\small, inner sep=1pt] {$\thid_1$};

\end{pgfonlayer}
\end{centertikz}%
\caption{\(\txid_3\) updates the view}
\label{fig:ua-before-tx2}
\end{subfigure} \\
\end{tabular}
\hrule
\caption{An invalid executions under update atomic for $\prog_3$}
\label{fig:cu.exec}
\label{fig:cu-exec}
\end{figure}




\begin{figure}
\hrule
\begin{tabular}{@{} c c@{}}

\begin{subfigure}{0.45\textwidth}
\begin{centertikz}

\begin{pgfonlayer}{foreground}
%Uncomment line below for help lines
%\draw[help lines] grid(5,4);

\node(locx)  {$\ke_\vx \mapsto$};

\matrix(versionx) [version list]
    at ([xshift=\tikzkvspace]locx.east) {
    {a} & $\txid_0$ & {a} & \(\txid_1\)\\
    {a} & $\Set{\txid_1}$ & {a} & \(\emptyset\)\\
};

\tikzvalue{versionx-1-1}{versionx-2-1}{locx-v0}{0};
\tikzvalue{versionx-1-3}{versionx-2-3}{locx-v1}{1};


\path (locx.south) + (0,\tikzkeyspace) node (locf1) {$\ke_{\pv{f1}} \mapsto$};
\matrix(versionf1) [version list]
    at ([xshift=\tikzkvspace]locf1.east) {
    {a} & $\txid_0$ & {a} & $\txid_1$\\
    {a} & $\Set{\txid_1}$ & {a} & $\emptyset$\\
};
\tikzvalue{versionf1-1-1}{versionf1-2-1}{locf1-v0}{0};
\tikzvalue{versionf1-1-3}{versionf1-2-3}{locf1-v1}{1};

%Location y
\path (locf1.south) + (0,\tikzkeyspace) node (locf2) {$\ke_{\pv{f2}} \mapsto$};
\matrix(versionf2) [version list]
    at ([xshift=\tikzkvspace]locf2.east) {
    {a} & $\txid_0$ \\
    {a} & $\emptyset$ \\
};
\tikzvalue{versionf2-1-1}{versionf2-2-1}{locf2-v0}{0};


% \draw[-, red, very thick, rounded corners] ([xshift=-5pt, yshift=5pt]locx-v1.north east) |- 
%  ($([xshift=-5pt,yshift=-5pt]locx-v1.south east)!.5!([xshift=-5pt, yshift=5pt]locy-v0.north east)$) -| ([xshift=-5pt, yshift=5pt]locy-v0.south east);

%blue view - I should  check whether I can use pgfkeys to just declare the list of locations, and then add the view automatically.
\draw[-, blue, very thick, rounded corners=10pt]
([xshift=-2pt, yshift=20pt]locx-v0.north east) node (tid1start) {} -- 
([xshift=-2pt, yshift=-5pt]locf2-v0.south east);

\path (tid1start) node[anchor=south, rectangle, fill=blue!20, draw=blue, font=\small, inner sep=1pt] {$\thid_3$};

%red view
\draw[-, red, very thick, rounded corners = 10pt]
([xshift=-5pt, yshift=5pt]locx-v1.north east) -- 
([xshift=-5pt, yshift=-5pt]locf1-v1.south east) --
([xshift=-5pt, yshift=5pt]locf2-v0.north east) -- 
([xshift=-5pt, yshift=-10pt]locf2-v0.south east) node (tid2start) {};

\path (tid2start) node[anchor=north, rectangle, fill=red!20, draw=red, font=\small, inner sep=1pt] {$\thid_2$};
 
 %green view
\draw[-, DarkGreen, very thick, rounded corners = 10pt]
([xshift=-16pt, yshift=8pt]locx-v1.north east) node (tid3start) {}-- 
([xshift=-16pt, yshift=-5pt]locf1-v1.south east) --
([xshift=-16pt, yshift=5pt]locf2-v0.north east) -- 
([xshift=-16pt, yshift=-5pt]locf2-v0.south east);

\path (tid3start) node[anchor=south, rectangle, fill=DarkGreen!20, draw=DarkGreen, font=\small, inner sep=1pt] {$\thid_1$};

\end{pgfonlayer}
\end{centertikz}%
\caption{\(\thid_2\) updates the view}
\label{fig:ua-thid-2-update-view}
\end{subfigure} 
&
\begin{subfigure}{0.45\textwidth}
\begin{centertikz}

\begin{pgfonlayer}{foreground}
%Uncomment line below for help lines
%\draw[help lines] grid(5,4);

\node(locx)  {$\ke_\vx \mapsto$};

\matrix(versionx) [version list]
    at ([xshift=\tikzkvspace]locx.east) {
    {a} & $\txid_0$ & {a} & $\txid_1$ & {a} & $\txid_2$\\
    {a} & $\Set{\txid_1}$ & {a} & $\Set{\txid_2}$ & {a} & $\emptyset$\\
};

\tikzvalue{versionx-1-1}{versionx-2-1}{locx-v0}{0};
\tikzvalue{versionx-1-3}{versionx-2-3}{locx-v1}{1};
\tikzvalue{versionx-1-5}{versionx-2-5}{locx-v2}{2};


\path (locx.south) + (0,\tikzkeyspace) node (locf1) {$\ke_{\pv{f1}} \mapsto$};
\matrix(versionf1) [version list]
    at ([xshift=\tikzkvspace]locf1.east) {
    {a} & $\txid_0$ & {a} & $\txid_1$\\
    {a} & $\Set{\txid_1}$ & {a} & $\emptyset$\\
};
\tikzvalue{versionf1-1-1}{versionf1-2-1}{locf1-v0}{0};
\tikzvalue{versionf1-1-3}{versionf1-2-3}{locf1-v1}{1};

%Location y
\path (locf1.south) + (0,\tikzkeyspace) node (locf2) {$\ke_{\pv{f2}} \mapsto$};
\matrix(versionf2) [version list]
    at ([xshift=\tikzkvspace]locf2.east) {
    {a} & $\txid_0$ & {a} & \(\txid_2\) \\
    {a} & $\emptyset$ & {a} & \(\emptyset\) \\
};
\tikzvalue{versionf2-1-1}{versionf2-2-1}{locf2-v0}{0};
\tikzvalue{versionf2-1-3}{versionf2-2-3}{locf2-v1}{1};

% \draw[-, red, very thick, rounded corners] ([xshift=-5pt, yshift=5pt]locx-v1.north east) |- 
%  ($([xshift=-5pt,yshift=-5pt]locx-v1.south east)!.5!([xshift=-5pt, yshift=5pt]locy-v0.north east)$) -| ([xshift=-5pt, yshift=5pt]locy-v0.south east);

%blue view - I should  check whether I can use pgfkeys to just declare the list of locations, and then add the view automatically.
\draw[-, blue, very thick, rounded corners=10pt]
([xshift=-2pt, yshift=20pt]locx-v0.north east) node (tid1start) {} -- 
([xshift=-2pt, yshift=-5pt]locf2-v0.south east);

\path (tid1start) node[anchor=south, rectangle, fill=blue!20, draw=blue, font=\small, inner sep=1pt] {$\thid_3$};

%red view
\draw[-, red, very thick, rounded corners = 10pt]
([xshift=-5pt, yshift=5pt]locx-v2.north east) -- 
([xshift=-5pt, yshift=-5pt]locx-v2.south east) --
([xshift=-5pt, yshift=5pt]locf1-v1.north east) -- 
([xshift=-5pt, yshift=-10pt]locf2-v1.south east) node (tid2start) {};

\path (tid2start) node[anchor=north, rectangle, fill=red!20, draw=red, font=\small, inner sep=1pt] {$\thid_2$};
 
 %green view
\draw[-, DarkGreen, very thick, rounded corners = 10pt]
([xshift=-16pt, yshift=8pt]locx-v1.north east) node (tid3start) {}-- 
([xshift=-16pt, yshift=-5pt]locf1-v1.south east) --
([xshift=-16pt, yshift=5pt]locf2-v0.north east) -- 
([xshift=-16pt, yshift=-5pt]locf2-v0.south east);

\path (tid3start) node[anchor=south, rectangle, fill=DarkGreen!20, draw=DarkGreen, font=\small, inner sep=1pt] {$\thid_1$};

\end{pgfonlayer}
\end{centertikz}%
\caption{After \(\txid_2\)}
\label{fig:ua-correct-after-tx2}
\end{subfigure} 
\\
\end{tabular}
\hrule
\caption{A execution of $\prog_3$ without lost-update}
\label{fig:ua-conf-2}
\end{figure}


\ac{This Consistency Model shows why the notion of consistent views must 
depend on the set of operations that need to be executed.}

The next consistency model that we consider is \emph{update atomic}. 
Although we did not find any implementation of this model, it has been proposed in \cite{framework-concur} as a strengthening to Read Atomic to avoid write-write conflicts.
This model states that: \textbf{(i)} transactions satisfy atomic visibility (\cref{def:readatomic}); and \textbf{(ii)} transactions writing to one same keys cannot be executed concurrently.
\sx{ This appears too earlier:
Update Atomic is also needed to specify more sophisticated consistency models, 
such as \emph{Parallel Snapshot Isolation} and \emph{Snapshot Isolation}.}
\ac{Check: Nobi said he was interested in implementing Update Atomic 
at some point, maybe he ended up doing something.}

Programs under update atomic do not exhibit the \emph{lost update} anomaly: two or more transactions update the same address, for example , both increment its value by $1$, but only one of them will be observed by future transactions, for example, only one of the increments takes effect.
To illustrate the \emph{lost-update anomaly}, consider the following program \( \prog_3 \) where two transactions concurrently increment $\vx$ and the third transaction read the value. 
Note that the \( \pvar{f1} \) and \( \pvar{f2} \) are two flags indicating the corresponding transactions has been committed.
\ac{Intuitive behaviour of the litmus test: two transactions concurrently increment $[\loc_x]$. 
 A third transaction observes that the first two transactions have been executed. 
 However, it only observes one of the two increments taking place.
 }
\[
    \prog_3 \equiv \begin{session}
        \begin{array}{@{}c || c || c@{}}
        \txid_1 : 
        \begin{transaction} 
            \pmutate{\pvar{f1}}{1};\\
            \pderef{\pvar{a}}{\vx};\\
            \pmutate{\vx}{a + 1};\\
        \end{transaction} & 
        \txid_2 : 
        \begin{transaction}
            \pmutate{\pvar{f2}}{1};\\
            \pderef{\pvar{a}}{\vx};\\
            \pmutate{\vx}{a + 1};\\
        \end{transaction} &
        \txid_3 : 
        \begin{transaction}
            \pderef{\pvar{a}}{\vx};\\
            \pderef{\pvar{b}}{\pvar{f1}};\\
            \pderef{\pvar{c}}{\pvar{f2}};\\
            \pifs{\pvar{a}=1 \wedge \pvar{b}=1 \wedge \pvar{c} = 1}\\ 
                \quad \passign{\retvar}{\sadface}
            \pife
        \end{transaction}
        \end{array}
    \end{session}
 \]

We consider an execution in which the transactions contained in the code of threads $\thid_1, \thid_2$ both execute on the same snapshot determined by the initial view. 
The initial configuration of the program coincides with the one given in \cref{fig:ua-init}.
After executing the transaction $\txid_1$, the resulting configuration is the one depicted \ref{fig:ua-after-tx1} and then \( \txid_2 \) shown in in \ref{fig:ua-after-tx2}, where both transactions read the initial version for key $\ke_\vx$. 
The third transaction $\txid_3$ choose to update its view to include the most recent version for all the keys (\ref{fig:ua-before-tx3}), then when executing its code, all the keys will have value $1$, and the return variable will be set to ${\sadface}$.

The program $\prog_3$ might exhibit the lost-update anomaly when the second transaction $\txid_2$ starts, its view did not include the most up-to-date version for key $\ke_\vx$ provided that \( \txid_{2}\) will update the key \( \ke_\vx \).
As consequence, the database \emph{lost the update} of a version of \( \ke_\vx \) installed by the transaction $\txid_1$, in a sense that no transaction will observe such the version.
To forbid this anomaly, the \emph{update atomic} requires that if a transaction writes to a key, the transaction should start with a view including the most recent version for the key.

\begin{definition}
\label{def:update-atomic}
\emph{Update atomic} is stronger than then read atomic (\cref{def:readatomic}) by further requiring for all keys written, it should starts with a view including the most recent version for those key:
\[
\begin{rclarray}
(\hh, \vi) \csat[\mathsf{UA}] \opset: \vi' & \defeq &
\begin{array}[t]{@{}l}
(\hh, \vi) \csat[\mathsf{RA}] \opset: \vi' \land \fora{\addr} 
(\otW, \addr, \stub) \in \opset \implies \vi(\addr)  = \left| \hh(\addr) \right| - 1
\end{array} \\
\end{rclarray}
\]
\end{definition}

\begin{proposition}
The execution test $\comoUA$ does not hinder progress. 
For any $\hh, \vi, \opset$, there exist $\vi' : \vi \leq \vi'$ and $\vi'': \Vupdate(\hh, \vi', \opset) \leq \vi''$ such that $(\hh, \vi') \csatUA \opset, \vi''$.
\end{proposition}

The thread $\thid_2$ from the program \( \prog_3\) , under $\mathsf{UA}$, cannot execute the transaction $\txid_2$ starting from the configuration depicted in \cref{fig:ua-after-tx1}.
Because the view of $\thid_2$ does not include the most recent version for key $\ke_\vx$. 
Instead, before executing, $\thid_2$ must update its view to include the most recent version of $\ke_\vx$ (\cref{fig:ua-thid-2-update-view}).
Then the \( \txid_2\) will install a new version for \( \ke_\vx \) with value 2 instead of 1 as shown in \cref{fig:ua-correct-after-tx2}.
There are now three different possible views in which $\thid_3$ can execute its transaction.
First, executing on the initial view, in which the transaction will observe 0 for the three locations, and the transaction will not return value $\sadface$.
Second, executing on the one in which the view of $\thid_3$ for $\ke_\vx$ points to the version $(1, \tsid_1, \Set{\tsid_2})$. 
Because of atomic visibility, it must also includes the most recent version for key $\ke_\pv{f2}$ since it is installed by \( \txid_2 \).
In this case, it will not return \(\sadface \).
Last, executing on the one in which the view of $\thid_3$ for $\ke_\vx$ points to its most recent version $(2, \txid_2, \emptyset)$.
In this case, it will not return \(\sadface \).
