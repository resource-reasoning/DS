
\subsection{Parallel Snapshot Isolation and Snapshot Isolation}
\sx{Need some citations here, what is geo-replicated, distributed database vs distributed system}
\emph{Snapshot Isolation} (SI) is a consistency model that has been widely employed in both centralised and distributed databases. 
Because snapshot isolation does not scale well to geo-replicated and distributed systems, a weaker model called \emph{Parallel Snapshot Isolation} (PSI) has been recently proposed. 

\sx{
Both SI and PSI can be specified in the history heap framework by combining the consistency models that we have already introduced. 
In short, SI combines atomic visibility and the snapshot monotonicity property from consistent prefix property (if a transaction $\tsid_1$ 
observes the effects of another transaction $\tsid_2$, then it also observes 
the effects of any transaction that committed before), and the write-conflict 
detection from Update Atomic (two committing transactions do not write 
concurrently to the same location). In contrast, PSI only requires atomic 
visibility, causal consistency and write-conflict detection. Formally, we have 
the following:
}
Both SI and PSI can be specified in the history heap framework by combining the consistency models that we have already introduced. 
In short, SI combines consistent prefix property and update atomic, while PSI only requires causal consistency and update atomic.
\begin{defn}
The \emph{parallel snapshot isolation} combines causal consistency (\defref{def:causal})  and update atomic (\defref{def:update-atomic}): $\mathsf{PSI} = \mathsf{CC} \cap \mathsf{UA}$.
The {snapshot isolation} combines consistent prefix (\defref{def:consistent-prefix}) and update atomic (\defref{def:update-atomic}): $\mathsf{SI} = \mathsf{CP} \cap \mathsf{UA}$.
\end{defn}

\begin{lem}
SI is stronger than PSI.
\end{lem}

\subsection{Serialisability}
\sx{
Serialisability is the last and strongest consistency 
model that we consider. Informally, under serialisability 
transactions appear to be executed in a sequential order. 
Consider for example the program 
}
\emph{Serialisability} is  the strongest consistency model that we consider. 
Informally, under serialisability transactions appear in a sequential order. 
Consider the following program know as \emph{write skew},
\[
\begin{session}
\begin{array}{@{}c || c@{}}
\begin{transaction}
\pderef{\pvar{a}}{\vy};\\
\pifs{\pvar{a} = 0};\\
    \quad \pmutate{\vx}{1};\\
    \quad \passign{\retvar}{\Large \frownie{}};
\pife
\end{transaction}
&
\begin{transaction}
\pderef{\pvar{a}}{\vx};\\
\pifs{\pvar{a} = 0};\\
\quad \pmutate{\vy}{1};\\
\quad \passign{\retvar}{\Large \frownie{}};
\pife
\end{transaction}
\end{array}
\end{session}
\]


%\begin{figure}
%\begin{center}
%\begin{tabular}{|@{}c|c@{}|}
%\hline
%\begin{tikzpicture}[font=\large]

%\begin{pgfonlayer}{foreground}
%%Uncomment line below for help lines
%%\draw[help lines] grid(5,4);

%%Location x
%\node(locx) at (1,3) {$[\loc_x] \mapsto$};

%\matrix(locxcells) [version list, text width=7mm, anchor=west]
   %at ([xshift=10pt]locx.east) {
 %{a} & $T_0$ \\
  %{a} & $\emptyset$ \\
%};
%\node[version node, fit=(locxcells-1-1) (locxcells-2-1), fill=white, inner sep= 0cm, font=\Large] (locx-v0) {$0$};
%%\node[version node, fit=(locxcells-1-3) (locxcells-2-3), fill=white, inner sep=0cm, font=\Large] (locx-v1) {$1$};

%%Location y
%\path (locx.south) + (0,-1.5) node (locy) {$[\loc_y] \mapsto$};
%\matrix(locycells) [version list, text width=7mm, anchor=west]
   %at ([xshift=10pt]locy.east) {
 %{a} & $T_0$ \\
  %{a} & $\emptyset$ \\
%};
%\node[version node, fit=(locycells-1-1) (locycells-2-1), fill=white, inner sep= 0cm, font=\Large] (locy-v0) {$0$};
%%\node[version node, fit=(locycells-1-3) (locycells-2-3), fill=white, inner sep=0cm, font=\Large] (locy-v1) {$1$};

%% \draw[-, red, very thick, rounded corners] ([xshift=-5pt, yshift=5pt]locx-v1.north east) |- 
%%  ($([xshift=-5pt,yshift=-5pt]locx-v1.south east)!.5!([xshift=-5pt, yshift=5pt]locy-v0.north east)$) -| ([xshift=-5pt, yshift=5pt]locy-v0.south east);

%%blue view - I should  check whether I can use pgfkeys to just declare the list of locations, and then add the view automatically.
%\draw[-, blue, very thick, rounded corners=10pt]
 %([xshift=-3pt, yshift=20pt]locx-v0.north east) node (tid1start) {} -- 
%% ([xshift=-2pt, yshift=-5pt]locx-v0.south east) --
%% ([xshift=-2pt, yshift=5pt]locy-v0.north east) -- 
 %([xshift=-3pt, yshift=-5pt]locy-v0.south east);
 
 %\path (tid1start) node[anchor=south, rectangle, fill=blue!20, draw=blue, font=\small, inner sep=1pt] {$\thid_1$};

%%red view
%\draw[-, red, very thick, rounded corners = 10pt]
 %([xshift=-16pt, yshift=5pt]locx-v0.north east) node (tid2start) {}-- 
%% ([xshift=-8pt, yshift=-5pt]locx-v0.south east) --
%% ([xshift=-8pt, yshift=5pt]locy-v0.north east) -- 
 %([xshift=-16pt, yshift=-5pt]locy-v0.south east) node {};
 
%\path (tid2start) node[anchor=south, rectangle, fill=red!20, draw=red, font=\small, inner sep=1pt] {$\thid_2$};

%%%Stack for threads tid_1 and tid_2
%%
%%\draw[-, dashed] let 
%%   \p1 = ([xshift=0pt]locy.west),
%%   \p2 = ([yshift=-5pt]locycells.south),
%%   \p3 = ([xshift=10pt]locycells.east) in
%%   (\x1, \y2) -- (\x3, \y2);
%%   
%%\matrix(stacks) [
%%   matrix of nodes,
%%   anchor=north, 
%%   text=blue, 
%%   font=\normalsize, 
%%   row 1/.style = {text = blue}, 
%%   row 2/.style = {text = red}, 
%%   text width= 13mm ] 
%%   at ([xshift=-10pt,yshift=-8pt]locycells.south) {
%%   $\thid_1:$ & $\retvar = 0$\\
%%   $\thid_2:$ & $\retvar = 0$\\
%%   };
%\end{pgfonlayer}
%\end{tikzpicture} 
%%
%&
%%
%\begin{tikzpicture}[font=\large]

%\begin{pgfonlayer}{foreground}
%%Uncomment line below for help lines
%%\draw[help lines] grid(5,4);

%%Location x
%\node(locx) at (1,3) {$[\loc_x] \mapsto$};

%\matrix(locxcells) [version list, text width=7mm, anchor=west]
   %at ([xshift=10pt]locx.east) {
 %{a} & $\tsid_0$ & {a} & $\tsid_1$\\
  %{a} & $\emptyset$ & {a} & $\emptyset$ \\
%};
%\node[version node, fit=(locxcells-1-1) (locxcells-2-1), fill=white, inner sep= 0cm, font=\Large] (locx-v0) {$0$};
%\node[version node, fit=(locxcells-1-3) (locxcells-2-3), fill=white, inner sep=0cm, font=\Large] (locx-v1) {$1$};

%%Location y
%\path (locx.south) + (0,-1.5) node (locy) {$[\loc_y] \mapsto$};
%\matrix(locycells) [version list, text width=7mm, anchor=west]
   %at ([xshift=10pt]locy.east) {
 %{a} & $T_0$ \\
  %{a} & $\{\tsid_1\}$ \\
%};
%\node[version node, fit=(locycells-1-1) (locycells-2-1), fill=white, inner sep= 0cm, font=\Large] (locy-v0) {$0$};
%%\node[version node, fit=(locycells-1-3) (locycells-2-3), fill=white, inner sep=0cm, font=\Large] (locy-v1) {$1$};

%% \draw[-, red, very thick, rounded corners] ([xshift=-5pt, yshift=5pt]locx-v1.north east) |- 
%%  ($([xshift=-5pt,yshift=-5pt]locx-v1.south east)!.5!([xshift=-5pt, yshift=5pt]locy-v0.north east)$) -| ([xshift=-5pt, yshift=5pt]locy-v0.south east);

%%blue view - I should  check whether I can use pgfkeys to just declare the list of locations, and then add the view automatically.
%\draw[-, blue, very thick, rounded corners=10pt]
 %([xshift=-3pt, yshift=20pt]locx-v1.north east) node (tid1start) {} -- 
 %([xshift=-3pt, yshift=-5pt]locx-v1.south east) --
 %([xshift=-3pt, yshift=5pt]locy-v0.north east) -- 
 %([xshift=-3pt, yshift=-5pt]locy-v0.south east);
 
 %\path (tid1start) node[anchor=south, rectangle, fill=blue!20, draw=blue, font=\small, inner sep=1pt] {$\thid_1$};

%%red view
%\draw[-, red, very thick, rounded corners = 10pt]
 %([xshift=-16pt, yshift=5pt]locx-v0.north east) node (tid2start) {}-- 
%% ([xshift=-8pt, yshift=-5pt]locx-v0.south east) --
%% ([xshift=-8pt, yshift=5pt]locy-v0.north east) -- 
 %([xshift=-16pt, yshift=-5pt]locy-v0.south east) node {};
 
%\path (tid2start) node[anchor=south, rectangle, fill=red!20, draw=red, font=\small, inner sep=1pt] {$\thid_2$};

%%%Stack for threads tid_1 and tid_2
%%
%%\draw[-, dashed] let 
%%   \p1 = ([xshift=0pt]locy.west),
%%   \p2 = ([yshift=-5pt]locycells.south),
%%   \p3 = ([xshift=10pt]locycells.east) in
%%   (\x1, \y2) -- (\x3, \y2);
%%   
%%\matrix(stacks) [
%%   matrix of nodes,
%%   anchor=north, 
%%   text=blue, 
%%   font=\normalsize, 
%%   row 1/.style = {text = blue}, 
%%   row 2/.style = {text = red}, 
%%   text width= 13mm ] 
%%   at ([xshift=-10pt,yshift=-8pt]locycells.south) {
%%   $\thid_1:$ & $\retvar = 0$\\
%%   $\thid_2:$ & $\retvar = 0$\\
%%   };
%\end{pgfonlayer}
%\end{tikzpicture}
%\\
%{\small(a)} & {\small(b)}\\
%\hline
%\begin{tikzpicture}[font=\large]

%\begin{pgfonlayer}{foreground}
%%Uncomment line below for help lines
%%\draw[help lines] grid(5,4);

%%Location x
%\node(locx) at (1,3) {$[\loc_x] \mapsto$};

%\matrix(locxcells) [version list, text width=7mm, anchor=west]
   %at ([xshift=10pt]locx.east) {
 %{a} & $T_0$ & {a} & $\tsid_1$\\
  %{a} & $\{\tsid_2\}$ & {a} & $\emptyset$ \\
%};
%\node[version node, fit=(locxcells-1-1) (locxcells-2-1), fill=white, inner sep= 0cm, font=\Large] (locx-v0) {$0$};
%\node[version node, fit=(locxcells-1-3) (locxcells-2-3), fill=white, inner sep=0cm, font=\Large] (locx-v1) {$1$};

%%Location y
%\path (locx.south) + (0,-1.5) node (locy) {$[\loc_y] \mapsto$};
%\matrix(locycells) [version list, text width=7mm, anchor=west]
   %at ([xshift=10pt]locy.east) {
 %{a} & $T_0$ & {a} & $\tsid_2$ \\
  %{a} & $\{\tsid_1\}$ & {a} & $\emptyset$\\
%};
%\node[version node, fit=(locycells-1-1) (locycells-2-1), fill=white, inner sep= 0cm, font=\Large] (locy-v0) {$0$};
%\node[version node, fit=(locycells-1-3) (locycells-2-3), fill=white, inner sep=0cm, font=\Large] (locy-v1) {$1$};

%% \draw[-, red, very thick, rounded corners] ([xshift=-5pt, yshift=5pt]locx-v1.north east) |- 
%%  ($([xshift=-5pt,yshift=-5pt]locx-v1.south east)!.5!([xshift=-5pt, yshift=5pt]locy-v0.north east)$) -| ([xshift=-5pt, yshift=5pt]locy-v0.south east);

%%blue view - I should  check whether I can use pgfkeys to just declare the list of locations, and then add the view automatically.
%\draw[-, blue, very thick, rounded corners=10pt]
 %([xshift=-3pt, yshift=20pt]locx-v1.north east) node (tid1start) {} -- 
 %([xshift=-3pt, yshift=-5pt]locx-v1.south east) --
 %([xshift=-3pt, yshift=5pt]locy-v0.north east) -- 
 %([xshift=-3pt, yshift=-5pt]locy-v0.south east);
 
 %\path (tid1start) node[anchor=south, rectangle, fill=blue!20, draw=blue, font=\small, inner sep=1pt] {$\thid_1$};

%%red view
%\draw[-, red, very thick, rounded corners = 10pt]
 %([xshift=-16pt, yshift=5pt]locx-v1.north east) node (tid2start) {}-- 
%% ([xshift=-16pt, yshift=-5pt]locx-v0.south east) --
%% ([xshift=-16pt, yshift=5pt]locy-v1.north east) -- 
 %([xshift=-16pt, yshift=-5pt]locy-v1.south east) node {};
 
%\path (tid2start) node[anchor=south, rectangle, fill=red!20, draw=red, font=\small, inner sep=1pt] {$\thid_2$};

%%%Stack for threads tid_1 and tid_2
%%
%%\draw[-, dashed] let 
%%   \p1 = ([xshift=0pt]locy.west),
%%   \p2 = ([yshift=-5pt]locycells.south),
%%   \p3 = ([xshift=10pt]locycells.east) in
%%   (\x1, \y2) -- (\x3, \y2);
%%   
%%\matrix(stacks) [
%%   matrix of nodes,
%%   anchor=north, 
%%   text=blue, 
%%   font=\normalsize, 
%%   row 1/.style = {text = blue}, 
%%   row 2/.style = {text = red}, 
%%   text width= 13mm ] 
%%   at ([xshift=-10pt,yshift=-8pt]locycells.south) {
%%   $\thid_1:$ & $\retvar = 0$\\
%%   $\thid_2:$ & $\retvar = 0$\\
%%   };
%\end{pgfonlayer}
%\end{tikzpicture}
%%
%&
%%
%\begin{tikzpicture}[font=\large]

%\begin{pgfonlayer}{foreground}
%%Uncomment line below for help lines
%%\draw[help lines] grid(5,4);

%%Location x
%\node(locx) at (1,3) {$[\loc_x] \mapsto$};

%\matrix(locxcells) [version list, text width=7mm, anchor=west]
   %at ([xshift=10pt]locx.east) {
 %{a} & $\tsid_0$ & {a} & $\tsid_1$\\
  %{a} & $\emptyset$ & {a} & $\emptyset$ \\
%};
%\node[version node, fit=(locxcells-1-1) (locxcells-2-1), fill=white, inner sep= 0cm, font=\Large] (locx-v0) {$0$};
%\node[version node, fit=(locxcells-1-3) (locxcells-2-3), fill=white, inner sep=0cm, font=\Large] (locx-v1) {$1$};

%%Location y
%\path (locx.south) + (0,-1.5) node (locy) {$[\loc_y] \mapsto$};
%\matrix(locycells) [version list, text width=7mm, anchor=west]
   %at ([xshift=10pt]locy.east) {
 %{a} & $T_0$ \\
  %{a} & $\{\tsid_1\}$ \\
%};
%\node[version node, fit=(locycells-1-1) (locycells-2-1), fill=white, inner sep= 0cm, font=\Large] (locy-v0) {$0$};
%%\node[version node, fit=(locycells-1-3) (locycells-2-3), fill=white, inner sep=0cm, font=\Large] (locy-v1) {$1$};

%% \draw[-, red, very thick, rounded corners] ([xshift=-5pt, yshift=5pt]locx-v1.north east) |- 
%%  ($([xshift=-5pt,yshift=-5pt]locx-v1.south east)!.5!([xshift=-5pt, yshift=5pt]locy-v0.north east)$) -| ([xshift=-5pt, yshift=5pt]locy-v0.south east);

%%blue view - I should  check whether I can use pgfkeys to just declare the list of locations, and then add the view automatically.
%\draw[-, blue, very thick, rounded corners=10pt]
 %([xshift=-3pt, yshift=20pt]locx-v1.north east) node (tid1start) {} -- 
 %([xshift=-3pt, yshift=-5pt]locx-v1.south east) --
 %([xshift=-3pt, yshift=5pt]locy-v0.north east) -- 
 %([xshift=-3pt, yshift=-5pt]locy-v0.south east);
 
 %\path (tid1start) node[anchor=south, rectangle, fill=blue!20, draw=blue, font=\small, inner sep=1pt] {$\thid_1$};

%%red view
%\draw[-, red, very thick, rounded corners = 10pt]
 %([xshift=-16pt, yshift=5pt]locx-v1.north east) node (tid2start) {}-- 
 %([xshift=-16pt, yshift=-3pt]locx-v1.south east) --
 %([xshift=-16pt, yshift=7pt]locy-v0.north east) -- 
 %([xshift=-16pt, yshift=-5pt]locy-v0.south east) node {};
 
%\path (tid2start) node[anchor=south, rectangle, fill=red!20, draw=red, font=\small, inner sep=1pt] {$\thid_2$};

%%%Stack for threads tid_1 and tid_2
%%
%%\draw[-, dashed] let 
%%   \p1 = ([xshift=0pt]locy.west),
%%   \p2 = ([yshift=-5pt]locycells.south),
%%   \p3 = ([xshift=10pt]locycells.east) in
%%   (\x1, \y2) -- (\x3, \y2);
%%   
%%\matrix(stacks) [
%%   matrix of nodes,
%%   anchor=north, 
%%   text=blue, 
%%   font=\normalsize, 
%%   row 1/.style = {text = blue}, 
%%   row 2/.style = {text = red}, 
%%   text width= 13mm ] 
%%   at ([xshift=-10pt,yshift=-8pt]locycells.south) {
%%   $\thid_1:$ & $\retvar = 0$\\
%%   $\thid_2:$ & $\retvar = 0$\\
%%   };
%\end{pgfonlayer}
%\end{tikzpicture}
%\\
%{\small(c)} & {\small(d)}\\
%\hline
%\multicolumn{2}{|c|}{
%\begin{tikzpicture}[font=\large]

%\begin{pgfonlayer}{foreground}
%%Uncomment line below for help lines
%%\draw[help lines] grid(5,4);

%%Location x
%\node(locx) at (1,3) {$[\loc_x] \mapsto$};

%\matrix(locxcells) [version list, text width=7mm, anchor=west]
   %at ([xshift=10pt]locx.east) {
 %{a} \pgfmatrixnextcell $\tsid_0$ \pgfmatrixnextcell {a} \pgfmatrixnextcell $\tsid_1$\\
  %{a} \pgfmatrixnextcell $\emptyset$ \pgfmatrixnextcell {a} \pgfmatrixnextcell $\{\tsid_2\}$ \\
%};
%\node[version node, fit=(locxcells-1-1) (locxcells-2-1), fill=white, inner sep= 0cm, font=\Large] (locx-v0) {$0$};
%\node[version node, fit=(locxcells-1-3) (locxcells-2-3), fill=white, inner sep=0cm, font=\Large] (locx-v1) {$1$};

%%Location y
%\path (locx.south) + (0,-1.5) node (locy) {$[\loc_y] \mapsto$};
%\matrix(locycells) [version list, text width=7mm, anchor=west]
   %at ([xshift=10pt]locy.east) {
 %{a} \pgfmatrixnextcell $\tsid_0$ \\
  %{a} \pgfmatrixnextcell $\{\tsid_1\}$ \\
%};
%\node[version node, fit=(locycells-1-1) (locycells-2-1), fill=white, inner sep= 0cm, font=\Large] (locy-v0) {$0$};
%%\node[version node, fit=(locycells-1-2) (locycells-1-3), fill=white, inner sep= 0cm, font= \Large] (locy-v0-ws) {$\tsid_0$};
%%\node[version node, fit=(locycells-2-2) (locycells-2-3), fill=white, inner sep=0cm, font=\Large] (locy-v1-rs) {$\{\tsid_1, \tsid_2\}$};

%% \draw[-, red, very thick, rounded corners] ([xshift=-5pt, yshift=5pt]locx-v1.north east) |- 
%%  ($([xshift=-5pt,yshift=-5pt]locx-v1.south east)!.5!([xshift=-5pt, yshift=5pt]locy-v0.north east)$) -| ([xshift=-5pt, yshift=5pt]locy-v0.south east);

%%blue view - I should  check whether I can use pgfkeys to just declare the list of locations, and then add the view automatically.
%\draw[-, blue, very thick, rounded corners=10pt]
 %([xshift=-3pt, yshift=20pt]locx-v1.north east) node (tid1start) {} -- 
 %([xshift=-3pt, yshift=-5pt]locx-v1.south east) --
 %([xshift=-3pt, yshift=5pt]locy-v0.north east) -- 
 %([xshift=-3pt, yshift=-5pt]locy-v0.south east);
 
 %\path (tid1start) node[anchor=south, rectangle, fill=blue!20, draw=blue, font=\small, inner sep=1pt] {$\thid_1$};

%%red view
%\draw[-, red, very thick, rounded corners = 10pt]
 %([xshift=-16pt, yshift=5pt]locx-v1.north east) node (tid2start) {}-- 
 %([xshift=-16pt, yshift=-3pt]locx-v1.south east) --
 %([xshift=-16pt, yshift=7pt]locy-v0.north east) -- 
 %([xshift=-16pt, yshift=-5pt]locy-v0.south east) node {};
 
%\path (tid2start) node[anchor=south, rectangle, fill=red!20, draw=red, font=\small, inner sep=1pt] {$\thid_2$};

%%%Stack for threads tid_1 and tid_2
%%
%%\draw[-, dashed] let 
%%   \p1 = ([xshift=0pt]locy.west),
%%   \p2 = ([yshift=-5pt]locycells.south),
%%   \p3 = ([xshift=10pt]locycells.east) in
%%   (\x1, \y2) -- (\x3, \y2);
%%   
%%\matrix(stacks) [
%%   matrix of nodes,
%%   anchor=north, 
%%   text=blue, 
%%   font=\normalsize, 
%%   row 1/.style = {text = blue}, 
%%   row 2/.style = {text = red}, 
%%   text width= 13mm ] 
%%   at ([xshift=-10pt,yshift=-8pt]locycells.south) {
%%   $\thid_1:$ & $\retvar = 0$\\
%%   $\thid_2:$ & $\retvar = 0$\\
%%   };
%\end{pgfonlayer}
%\end{tikzpicture}
%}\\
%\multicolumn{2}{|c|}{{\small(e)}}\\
%\hline
%\end{tabular}
%\end{center}
%\caption{Graphical representation of configurations obtained by 
%executing $\prog_5$.}
%\label{fig:ser.exec}
%\end{figure}

\sx{did not read }
If we execute the program $\prog_5$ under any of the consistency model presented so far, it is possible to infer 
an execution in which both threads $\thid_1$ and $\thid_2$ set the 
thread-local variable $\retvar$ to value ${\Large \frownie{}}$.
For example, under $\mathsf{SI}$, this can happen as follows: 
\begin{itemize}
\item first, thread $\thid_1$ executes its transactions in the 
configuration of Figure \ref{fig:ser.exec}(a). The snapshot 
under which the transaction is executed is given by $[[\loc_x] \mapsto 0, [\loc_y] \mapsto 0]$, 
hence the execution of the transaction results in the thread-local variable $\retvar$ of $\thid_1$ 
to be set to ${\Large \frownie{}}$, and in the configuration of Figure \ref{fig:ser.exec}(b).
\item Next, thread $\thid_2$ executes its transaction in the initial 
view $[ [\loc_x] \mapsto 0, [\loc_y] \mapsto 0]$. The snapshot under which 
the transaction is executed is again $[[\loc_x \mapsto 0, [\loc_y] \mapsto 0]$. After 
thetransaction has been executed, the thread-local variable $\retvar$ of $\thid_2$ is 
set to ${\Large \frownie{}}$, and the final configuration is the one of Figure \ref{fig:ser.exec}(c). 
Note that, because we are assuming SI as our consistency model, in this configuration 
we require that the view of thread $\thid_2$ points to the last version of each location.
\end{itemize}


To avoid the \emph{write skew} anomaly, it suffices to ensure that, prior to executing a transaction, the view of a thread is always up-to-date for all addresses. 
