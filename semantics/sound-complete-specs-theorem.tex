\subsection{Traces of Programs}
\label{sec:kv-sound-complete-theorem}
In this Section we define the set of $\ET$-traces generated by a program 
$\prog$. Our main goal is that of proving that our semantics is correct, 
meaning that if a program $\prog$ executing under the execution 
test $\ET$ terminates in a state $(\hh, \_)$, then $\hh \in \CMs(\ET)$. 

\begin{definition}
Let $\ET$ be an execution test. For each program $\prog$ and state 
$(\hh, \viewFun, \thdenv)$, we define $\OPtraces(\ET, \prog, \hh, \viewFun, \thdenv)$ 
as the smallest set such that 
\begin{itemize}
\item $(\hh, \viewFun) \in \OPtraces(\ET, \prog, \hh, \viewFun, \thdenv)$, 
\item if $\tr \in \OPtraces(\ET, \prog', \hh', \viewFun',\thdenv')$, 
and $((\hh, \viewFun, \thdenv) , \prog) \toT{(\cl, \iota)}_{\ET} (\hh', \viewFun', \thdenv')$, 
then $\tr \in \OPtraces(\ET, \prog, \hh, \viewFun, \thdenv')$, 
\item if $\tr$ in $\OPtraces(\ET, \prog', \hh', \viewFun', \thdenv')$ and 
\newline $(\hh, \viewFun, \thdenv), \prog) \toT{(\cl, \vi, \opset)} ((\hh', \viewFun', \thdenv'), \prog')$,  
then $(\hh, \viewFun) \xrightarrowtriangle{(\cl, \varepsilon)}_{\ET} (\hh, \viewFun\rmto{\cl}{\vi}) 
\xrightarrowtriangle{(\cl, \opset)}_{\ET} \tr$ in $\OPtraces(\ET, \prog, \hh, \viewFun, \thdenv)$. 
\end{itemize}

The set of traces generated by a program $\prog$ under the execution test $\ET$ is 
then defined as $\OPtraces(\ET, \prog) \triangleq (\OPtraces(\ET, \prog, \hh_{0}, \viewFun_{0}, 
\thdenv_{0})$, where $\viewFun_{0} = \lambda \cl \in \dom(\prog).\lambda \ke.\{0\}$, and 
$\thdenv_{0} = \lambda \cl \in \dom(\prog).\lambda a.0$.

\end{definition}

\begin{proposition}
    \label{prop:program-trace-in-et-trace}
For each program $\prog$ and execution test $\ET$, 
$\OPtraces(\ET, \prog) \subseteq \Confs(\ET)$. 
Furthermore, each $\tr \in \OPtraces(\ET, \prog)$ is in normal form. 
\end{proposition}
\begin{proof}
    First, by the definition of \( \OPtraces \), 
    it only constructs trace in normal form.
    It is easy to prove that for any trace \( \tau \) in \( \OPtraces(\ET, \prog) \), by induction on the trace length,
    the trace is also in \( \Confs(\ET) \).
\end{proof}

\begin{corollary}
If $(\hh_{0}, \viewFun_{0}, \thdenv_{0}), \prog) \toL_{\ET} \cdots \toL_{\ET} 
(\hh, \viewFun, \thdenv, \lambda \cl \in \dom(\prog). \pskip)$, then $\hh \in \CMs(\ET)$.
\end{corollary}
\begin{proof}
    By the definition of \( \OPtraces \), 
    there exists a corresponding trace \( \tau \in \OPtraces(\ET, \prog) \).
    By \cref{prop:program-trace-in-et-trace}, such trace \( \tau \in \Confs(\ET) \),
    therefore \( \mkvs \in \CMs(\ET)\) by definition of \( \CMs(\ET) \).
\end{proof}

\begin{proposition}
For any program $\prog$ and execution test $\ET$, $\OPtraces(\ET, \prog) = \OPtraces(\ET_{\top}, \prog) 
\cap \Confs(\ET)$.
\end{proposition}
\begin{proof}
    It is easy to see \(\OPtraces(\ET, \prog) \subseteq \OPtraces(\ET_\top, \prog) \).
    By \cref{prop:program-trace-in-et-trace}, we know \( \OPtraces(\ET, \prog) \subseteq \Confs(\ET)\).
    Therefore \(  \OPtraces(\ET, \prog) \subseteq \OPtraces(\ET_\top, \prog) \cap \Confs(\ET) \).

    Let consider a trace \( \tau \) in \( \OPtraces(\ET_\top, \prog) \cap \Confs(\ET) \).
    By inductions on the length of trace, 
    every step that commits a new transaction  must satisfy \( \ET \) as \( \tau \in \Confs(\ET) \).
    It also reduce the program \( \prog \) as \( \tau \in \OPtraces(\ET_\top, \prog) \).
    By the definition \( \OPtraces(\ET, \prog) \),
    we can construct the same trace \( \tau \),
    so \( \tau \in \OPtraces(\ET, \prog) \).
\end{proof}

\ac{These results should be easy to prove, either I or somebody else will type up a proof 
at some point.}

\subsection{Adequacy of the Semantics.}
\begin{figure}
\[
    \inferrule[\rl{ACommit}]{
        \T \subseteq \T_{\aexec} \qquad \h \in \RP(\aexec, \T) \qquad
		(\stk, \h, \emptyset), \trans \ \toL^{*} \  (\stk', \stub,  \opset) , \pskip \\\\
		\txid \in \nextTxId(\T_{\aexec}, \cl) \qquad \aexec' = \extend(\aexec, \txid, \T, \opset) \qquad 
		\forall A \in \Ax.\;\{\txid' \mid (\txid', \txid) \in \A(\aexec') \} \subseteq \T
    }{
        \cl \vdash ( \aexec, \stk ), \ptrans{\trans} \ \toA{(\cl, \T, \opset)}_{(\RP, \Ax)} \ ( \aexec', \stk' ) , \pskip
    }
\]
\[
    \inferrule[\rl{ASingleThread}]{%
         \cl \vdash ( \aexec, \thdenv(\thid) ) , \prog(\thid), \ \toA{\lambda}_{(\RP, \Ax)} \  ( \aexec', \stk' ) , \cmd'  
    }{%
         (\aexec, \thdenv ), \prog  \ \toA{\lambda}_{(\RP, \Ax)} \  ( \aexec', \thdenv\rmto{\thid}{\stk'} ) , \prog\rmto{\thid}{\cmd'} ) 
    }
\]
\hrulefill
\caption{Selected Rule of the Semantics for Programs using Abstract Executions.}
\label{fig:aexec.semantics}
\end{figure}
Suppose that a given execution test $\ET$ captures precisely 
a consistency model specified in the axiomatic style, using a set of 
axioms $\Ax$ and a resolution policy $\RP$ over abstract executions.
That is, for any abstract execution $\aexec$ that satisfies 
the axioms $\Ax$ and the resolution policy $\RP$, then $\KVtrace(\ET_{\top}, \aexec) \cap \CMs(\ET) \neq \emptyset$; 
and for any $\tr \in \CMs(\ET)$, there exists an abstract execution 
$\aexec \in \aeset(\tr)$ that satisfies the axioms $\Ax$ and the resolution policy $\RP$. 
\ac{In practice, the functions $\KVtrace(\ET, \aexec) = \KVtrace(\ET_{\top}, \aexec) \cap \CMs(\ET)$, 
and $\aeset(\Ax, \aexec) = \aeset(\ET_{\top}, \aexec) \cap \aeset(\Ax)$, abstracted over the second 
argument and lifted to sets of abstract executions/traces, 
define a Galois Connection between the powerset of abstract executions in the axiomatic specification 
of the consistency model, and the powerset of kv-stores generated by such a consistency model.}
Our main aim in this section consists in proving that, for each program $\prog$, the 
set of kv-stores generated by $\prog$ under $\ET$ corresponds to all the possible kv-stores 
that could be obtained by running $\prog$ on a database that satisfies the axiomatic specification 
$\Ax$. In this sense, we aim to establish that our operational semantics is \emph{adequate}.

To tackle this question, we need to define what is the set of all possible behaviours 
that can be produced by a program $\prog$ under a given consistency model $\CM$, for 
which an axiomatic specification $(\RP, \Ax)$ is known. This in turn requires addressing two orthogonal 
problems: \textbf{(i)} defining the set of all possible behaviours that may be exhibited by a program 
$\prog$, independently of the consistency model; and \textbf{(ii)} defining the set of all possible 
behaviours that are allowed by a given consistency model $\CM$. Then the set of all 
possible behaviours of $\prog$ under $\CM$ is obtained by intersecting the two sets 
above.

The kv-store semantics is intrinsically not expressive enough to tackle problem \textbf{(i)}. 
By \cref{cor:kvtrace2aexec}, only kv-stores arising 
from abstract executions satisfying the last write wins resolution policy can be captured in the kv-store 
framework; instead, we seek to model all the behaviours of a program independently of a consistency 
models, and therefore independently of a resolution policy. 

Our solution requires defining an alternative semantics of programs under weak consistency models, 
based on abstract executions. The operational semantics we propose is parameterised in 
the axiomatic specification $(\RP, \Ax)$ of a consistency model: transitions take the form 
$(\aexec, \Env, \prog) \toA{\_}_{(\RP, \Ax)} (\aexec', \Env', \prog')$. 

In \cref{fig:aexec.semantics} we illustrate two rules of the operational semantics of programs 
based on abstract executions. The complete operational semantics is given in \ref{app:aexec.semantics}. 
\ac{Todo: type down the semantics. This must be done in order to prove the Theorem that 
establishes the correspondence between kv-store semantics and abstract execution semantics.} 
Rule \rl{ACommit} is the abstact execution counterpart of rule \rl{PCommit} for kv-stores, in that 
it models how an abstract execution $\aexec$ evolves when a client wants to execute a transaction whose 
code is $\ptrans{\trans}$. In this rule, $\T$ is the set of transactions of $\aexec$ that are visible to the client 
$\cl$ that wishes to execute $\ptrans{\trans}$. Such a set of transactions is used to determine a snapshot 
$\h \in \RP(\aexec, \T)$ that the client $\cl$ uses to execute the code $\ptrans{\trans}$, and obtain a 
fingerprint $\opset$. This fingerprint is then used to extend abstract execution $\aexec$ with 
a transaction from the set $\nextTxId(\T_{\aexec}, \cl)$. Another rule in \cref{fig:aexec.semantics} 
is Rule \rl{ASinglethread}; the structure of this rule is analogous to \rl{PSingleThread}, and it models 
multi-thread concurrency in an interleaving fashion. All the rules of the abstract operational semantics 
that are not illustrated in \cref{fig:aexec.semantics} have a similar counterpart in the kv-store semantics.

In some sense that is going to be made mathematically precise later, Rule \rl{ACommit} is more general 
than Rule \rl{Pcommit} in the kv-store semantics. In the latter, the snapshot of a transaction is uniquely 
determined from a view of the client, in a way that roughly corresponds to the last write wins policy 
in the abstract execution framework. In contrast, in Rule \rl{ACommit} the snapshot of a transaction 
is chosen non-deterministically from those made available to the client by the resolution policy 
$\RP$ adopted by a weak consistency model, which may not necessarily be $\RP_{\LWW}$. 
One example of resolution policy that we will use in this Section is given by the anarchic resolution policy. 

\begin{definition}
The anarchic resolution policy $\RP_{\anarchic}$ is defined by letting, 
$\RP_{\anarchic}(\_, \_) = \Snapshots$. The \emph{anarchic consistency model} is 
specified axiomatically by the pair $\anarchicCM = (\RP_{\anarchic}, \emptyset)$.
\end{definition}

\begin{example}
Suppose that we want to execute the single-threaded program $\prog$ that maps client 
$\cl$ to the transactional code below:
\[
\begin{session}
%\ptrans{\pmutate{\ke}{\val_2}}; \\
\ptrans{\pderef{\pvar{a}}{\ke}; \\
\pifs{\pvar{a} = \val_{1}} \pmutate{\ke'}{\val_1} \pife}
\end{session}
\]
Suppose that the program is executed under a consistency model that adopts the last write 
wins resolution policy $\RP_{\LWW}$, and with no additional axioms. Then the behaviour of $\prog$ is 
completely deterministic (up-to the choice of transaction identifiers), and the execution of $\prog$ terminates in a 
state corresponding to the abstract execution below: 

\begin{center}
\begin{tikzpicture}[scale=0.85, every node/.style={transform shape}]

\node(t0rx) at (-1,2) {$(\otR, \ke, \val_0)$}; 
%\path (t0wx.south) + (0,-0.2) node[anchor=north] (t0wy) {$(\otW, \ke_2, \val'_0)$};

\begin{pgfonlayer}{background}
\node[background, fit=(t0rx)]  {};

\path(t0.west) node[anchor=east] (t0lbl) {$\txid_{\cl}^{\_}$};
%\path(t1.north) node[anchor=south] (t1lbl) {$\txid_1$};
%\path(t2.south) node[anchor=north] (t2lbl) {$\txid_2$};

%\path[->]
%(t0.north) edge[bend left=70] node[above, yshift=7pt, xshift=-1pt, pos=0.3] {$\RF(\ke_2), \VO(\ke_1)$} (t1.west)
%(t0.south) edge[bend right=70] node[below, yshift=-8pt, xshift=-1pt, pos=0.3] {$\RF(\ke_1), \VO(\ke_2)$} (t2.west)
%([xshift=-8pt]t2.north) edge[bend left=40] node[left] {$\AD(\ke_1)$} ([xshift=-8pt]t1.south) 
%([xshift=8pt]t1.south) edge[bend left=40] node[right] {$\AD(\ke_2)$} ([xshift=8pt]t2.north);
\end{pgfonlayer}
\end{tikzpicture}
\end{center}

However, if we replace the consistency model specification $(\RP_{\LWW}, \emptyset)$ with the 
anarchic one $\anarchicCM$. Because the snapshot in which client $\cl$ executes the 
transactional code above is chosen non-deterministically, 
the program $\prog$ exhibits infinitely many additional behaviours. 
In particular, the program may now terminate in a state corresponding 
to the abstract execution below: 

\begin{center}
\begin{tikzpicture}[scale=0.85, every node/.style={transform shape}]

\node(t0rx) at (-1,2) {$(\otR, \ke, \val_1)$}; 
\path (t0wx.east) + (0,0.2) node[anchor=west] (t0wy) {$(\otW, \ke', \val_1)$};

\begin{pgfonlayer}{background}
\node[background, fit=(t0rx) (t0wy)]  {};

\path(t0.west) node[anchor=east] (t0lbl) {$\txid_{\cl}^{\_}$};
%\path(t1.north) node[anchor=south] (t1lbl) {$\txid_1$};
%\path(t2.south) node[anchor=north] (t2lbl) {$\txid_2$};

%\path[->]
%(t0.north) edge[bend left=70] node[above, yshift=7pt, xshift=-1pt, pos=0.3] {$\RF(\ke_2), \VO(\ke_1)$} (t1.west)
%(t0.south) edge[bend right=70] node[below, yshift=-8pt, xshift=-1pt, pos=0.3] {$\RF(\ke_1), \VO(\ke_2)$} (t2.west)
%([xshift=-8pt]t2.north) edge[bend left=40] node[left] {$\AD(\ke_1)$} ([xshift=-8pt]t1.south) 
%([xshift=8pt]t1.south) edge[bend left=40] node[right] {$\AD(\ke_2)$} ([xshift=8pt]t2.north);
\end{pgfonlayer}
\end{tikzpicture}
\end{center}

It is important to note, however, that the set of abstract executions generated by $\prog$ is still bound 
to the structure of the program itself. For example, executing $\prog$ under the anarchic execution model 
will never lead to an abstract execution with multiple transactions, or to an abstract execution where a transaction 
writes a key other than $\ke'$ is written.
\end{example}

\begin{definition}
The semantics of a program $\prog$ under a consistency model with axiomatic specification 
$(\RP, \Ax)$ is given by 
\[
\interpr{\prog}_{(\RP, \Ax)} = \{ \aexec \mid (\aexec_{0}, \Env_{0}, \prog) \toA{\_}_{(\RP, \Ax)}^{\ast} (\aexec, \_, \prog_{f}) \}, 
\]
where $\Env_{0} = \lambda \cl \in \dom(\prog).\lambda \pvar{x}.0$ and $\prog_{f} = \lambda \cl \in \dom(\prog).\pskip$.
\end{definition}

We define the set of all the possible behaviours of a program $\prog$ to be $\interpr{\prog}_{\anarchic}$. 
The following result supports our claim that this definition is indeed accurate: 

\begin{proposition}
Any abstract execution $\aexec$ satisfies $\anarchicCM$.
\end{proposition}
\begin{proof}
    (I did not find a definition of satisfies, but I guess is first visibility relation satisfies the constrain, 
    and second, there exists a snapshot decided by the visibility relation and resolution policy is consistent with the operations inside transactions.).
    It is trivial since $\anarchicCM$ does not have any constraint for visibility relation,
    and the snapshot for each transaction can be any possible one.
\end{proof}

\begin{example}
One may argue that the axiomatic specification $\anarchicCM$ does not 
truly represent an anarchic consistency model. Consider for example the single-threaded 
program $\prog'$ that associates to a client $\cl$ the following code:
\[
\begin{session}
\ptrans{
\pderef{\pvar{a}}{\ke}; \\
\pderef{\pvar{b}}{\ke};\\
\pifs{\pvar{a} != \pvar{b}} \pmutate{\ke'}{\val_1} \pife}
\end{session}
\]
One would expect that, under a truly anarchic consistency model, it would be possible 
for program $\prog'$ to write the value $\val_1$ for key $\ke'$. However, 
this never happens if $\prog'$ is executed under $\anarchicCM$. This is because 
we embedded into abstract execution the assumption that transactions only read 
at most one value for each key. 

In theory, we could lift this limitation and still retain 
the validity of all the results contained in this report; however, doing so would 
require to work with mathematical structures that are far more complex than 
abstract executions, and we preferred to avoid this issue. 
Furthermore, the constraint that an object is never read twice in transactions is enforced 
at client side in virtually all the implementations {\color{red} to be checked} 
of libraries for accessing kv-stores. When a client first requests to fetch 
the value of some key $\ke$ within a transaction, a local copy of the value fetched is 
saved on the client (typically in an object containing the meta-data of the transaction); 
if a request to read the same key is performed again within the same client, the local 
copy of the value previously fetched for that key is returned, instead of issuing a second 
read request to the kv-store.
\end{example}

As explained above, the set of all possible behaviours exhibited by a program $\prog$ under a 
consistency model $(\RP, \Ax)$ can be defined by intersecting the set of executions 
that $\prog$ exhibits under the anarchic consistency model, with the set of all executions 
allowed by the axiomatic specification $(\RP, \Ax)$. As the next theorem shows, 
this is exactly the set of abstract executions in which $\prog$ terminates, 
when executed under the axiomatic specification $(\RP, \Ax)$.

\begin{theorem}
For any program $\prog$ and axiomatic specification $(\RP, \Ax)$:
\[
\interpr{\prog}_{(\RP, \Ax)} = \interpr{\prog}_{\anarchic} \cap \CMa(\RP, \Ax). 
\]
\end{theorem}
\begin{proof}
AC: The proof was already typed up in the other set of notes, will need to do 
cut, paste and adapt in the appendix.
\end{proof}


\mypar{Comparing the abstract execution and kv-store semantics.}

\begin{proposition}
\label{prop:kv2aexec_transition}
Suppose that $(\hh, \viewFun, \thdenv), \prog \toT{(\cl, \vi, \opset)}_{\ET_{\top}} (\hh', 
\viewFun', \thdenv'), \prog'$. Let $\aexec$ be an abstract execution 
such that $\hh_{\aexec} = \hh$, and let $\T \subseteq \T_{\aexec}$ be a 
set of read-only transactions. Then there exists an abstract execution $\aexec'$ 
such that $\hh_{\aexec'} = \hh'$, and 
\[
(\aexec, \thdenv), \prog \toA{(\cl, \T \cup \Tx(\hh, \vi), \opset)}_{(\RP_{\LWW}, \emptyset)}. 
(\aexec', \thdenv'), \prog'
\]
\end{proposition}
\begin{proof}
See \cref{sec:kv2aexec-transaction}.
\end{proof}

\begin{proposition}
\label{prop:aexec2kv_transition}
Suppose that $(\aexec, \thdenv), \prog \toA{(\cl, \T, \opset)}_{(\RP_{\LWW})} (\aexec', \thdenv'), \prog'$. 
Then for any $\viewFun$ and $\vi \in \Views(\hh_{\aexec})$ such that $\vi \viewleq \getView(\aexec, \T)$, 
we have that 
\[
(\hh_{\aexec}, \viewFun\rmto{\cl}{\vi}, \thdenv), \prog \toA{(\cl, \getView(\aexec, \T), \opset)}_{\ET_{\top}} (\hh_{\aexec'}, \viewFun, \thdenv'), \prog'.
\]
\end{proposition}
\begin{proof}
See \cref{sec:aexec2kv-transaction}.
\end{proof}

\begin{corollary}
For any program $\prog$, 
\[
\interpr{\prog}_{\ET_{\top}} = \{\hh_{\aexec} \mid \aexec \in \interpr{\prog}_{(\RP_{\LWW}, \emptyset)}\}
\]
\end{corollary}
\begin{proof}
    It can be derived by \cref{prop:aexec2kv_transition} and \cref{prop:kv2aexec_transition}.
\end{proof}


\mypar{Putting it all together}
To conclude this Section, we show how all the results illustrated so far 
can be put together to show that the kv-store operational semantics is adequate. 
\begin{definition}
A \emph{client-based invariant condition}, or simply \emph{invariant}, is a 
function $I : \aeset \times \Clients \rightarrow \powerset{\TxID}$ 
such that for any $\cl$ we have that $I(\aeset, \cl) \subseteq \T_{\aexec}$, and 
for any  $\cl'$ such that $\cl' \neq \cl$ we have that 
$I(\extend(\aexec, \txid_{\cl'}^{\cdot}, \_, \_), \cl) = I(\aexec, \cl)$.
%of triples of the form $(\aexec, \cl, \T)$ such that whenever 
%$(\aexec, \cl, \T) \in I$ then $\T \subseteq \aexec$, 
%for any $\cl' \neq \cl$ and transaction $\txid = \txid_{\cl'}^{\cdot}$, 
%then $(\extend(\aexec, \txid, \_, \_)), \cl, \T) \in I$, 
%and for any $\T' : \T \subseteq \T' \subseteq \T_{\aexec}$, 
%$(\aexec, \cl ,\T') \in I$.
\end{definition}
\ac{The idea behind client-based invariant being that $I(\aexec, \cl)$ represents 
the minimal set of transactions that $\cl$ must see in $\aexec$, before 
updating the view and performing a transaction. Such a set of transaction 
roughly correspond to the view of the client before performing a 
sequence of \emph{update view+execute transaction} operations, 
or equivalently from the view obtained after the execution of the 
last transaction from that client.}

%\begin{definition}
%\label{def:et_sound}
%An execution test $\ET$ is sound with respect to an axiomatic 
%specification $(\RP_{\LWW}, \Ax)$ if and only if there exists an 
%invariant condition $I$ such that whenever $\ET \vdash (\hh, \vi) 
%\triangleright \opset: \vi'$, for any $\aexec$ such that 
%$\hh_{\aexec} = \hh$, and for any client $\cl$, and for any 
%transaction identifier $\txid \in \nextTxId(\hh, \cl)$, 
%%and for any $\T \subseteq \T_{\aexec}$ such that 
%%$\getView(\aexec, \T) = \vi$ 
%and $I(\aexec, \cl) \subseteq \Tx(\hh, \vi)$, then  
%there exist a set of read-only transactions $\T_{\mathsf{rd}}$ and 
%a set of transactions 
%%$\T \subseteq \T_{\aexec}$ and 
%$\T' \subseteq \T_{\aexec} \cup \{\txid\}$ 
%\begin{itemize}
%%\item $\getView(\aexec, \T) = \vi$, $I(\aexec, \cl) \subseteq \T$, 
%\item $\forall \A \in \Ax. \{\txid' \mid (\txid', \txid) \in \A(\aexec')\} \subseteq \Tx(\hh, \vi) \cup \T_{\mathsf{rd}}$, 
%where $\aexec' = \extend(\aexec, \txid, \Tx(\hh, \vi), \opset)$ for $\txid \in \nextTxId(\hh, \cl)$, 
%\item $I(\aexec', \cl) \subseteq \T'$, where $\T' = \Tx(\aexec', \vi')$. 
%\end{itemize}
%\end{definition}

\begin{definition}
\label{def:et_sound}
An execution test $\ET$ is sound with respect to an axiomatic 
specification $(\RP_{\LWW}, \Ax)$ if and only if there exists an 
invariant condition $I$ such that whenever $\ET \vdash (\hh, \vi) 
\triangleright \opset: \vi'$, for any $\aexec$ such that 
$\hh_{\aexec} = \hh$, for any client $\cl$, for any 
transaction identifier $\txid \in \nextTxId(\hh, \cl)$, 
%and for any set of transactions $\T_{\rd}$ that are read-only in $\aexec$ 
such that  $I(\aexec, \cl) \subseteq \Tx(\hh, \vi)$, then  
there exist two other sets of transactions $\T_{\rd}$, 
%which are read-only in $\aexec$ and $\aexec' := \extend(\aexec, \txid, 
%\Tx(\hh,\vi) \cup \T_{\mathsf{rd}} \cup \T'_{\mathsf{rd}}, \opset)$, respectively, 
such that 
\begin{itemize}
\item $\forall \A \in \Ax. \Setcon{\txid' }{ (\txid', \txid) \in \A(\aexec')} \subseteq \Tx(\hh, \vi) \cup \T_{\rd}$, 
\item $I(\aexec', \cl) \subseteq \Tx(\mkvs_{\aexec'}, \vi')$
\end{itemize}
\end{definition}

\begin{theorem}
\label{thm:et_soundness}
If $\ET$ is sound with respect to $(\RP_{\LWW}, \Ax)$, then 
\[
    \CMs(\ET) \subseteq \{ \hh \mid \exists \aexec \in \CMa(\RP_{\LWW}, \Ax)).\;\hh_{\aexec} = \hh\}
\].
\end{theorem}
\begin{proof}
    See \cref{sec:thm-et-soundness}.
\end{proof}

\begin{corollary}
\label{cor:et-soundness}
If $\ET$ is sound with respect to $(\RP_{\LWW}, \Ax)$, then 
for any program $\prog$, $\interpr{\prog}_{\ET} \subseteq \{ \hh_{\aexec} \mid \aexec \in \interpr{P}_{(\RP_{\LWW}, \Ax)} \}$.
\end{corollary}
\begin{proof}
See \cref{sec:cor-et-soundness}.
\end{proof}

\begin{definition}
\label{def:et_complete}
An execution test $\ET$ is \emph{complete} with respect 
to an axiomatic specification $(\RP_{\LWW}, \Ax)$ if, for any 
abstract execution $\aexec \in \CMa(\RP_{\LWW}, \Ax)$ 
such that $\AR_{\aexec} = \{(\txid_{i}, \txid_{i+1})\}_{i=1}^{\lvert \T_{\aexec} \rvert - 1}$, 
and for any $i=1,\cdots, n$, there exist two views $\vi_{i}, \vi_{i}'$ such that 
\begin{itemize}
\item $\vi_{i} = \getView(\aexec, \VIS_{\aexec}^{-1}(\txid_{i}))$, 
\item let $\txid_{i} = \txid_{\cl}^{n}$ for some $\cl, n$; if the
transaction $\txid_{i}' = \min_{\PO_{\aexec}}\{\txid' \mid \txid_i \xrightarrow{\PO_{\aexec}} \txid'\}$  
is defined, then $\vi' = \getView(\aexec, \T_{i})$, where $\T_{i} \subseteq (\AR_{\aexec}^{-1})?(\txid_{i}) \cap \VIS_{\aexec}^{-1}(\txid_{i}'))$; 
otherwise $\vi' = \getView(\aexec, \T_{i})$, where $\T_{i} \subseteq (\AR_{\aexec}^{-1})?(\txid_{i})$, 
\item $\ET \vdash (\hh_{\cut(\aexec, i-1)}, \vi_{i}) \triangleright \TtoOp{T}_{\aexec}(\txid_{i}) : \vi_{i}'$.
\end{itemize}
\end{definition}

\begin{theorem}
\label{thm:et_complete}
Let $\ET$ be an execution test that is complete with respect to 
an axiomatic specification $(\RP_{\LWW}, \Ax)$. Then 
$\CMa(\RP_{\LWW}, \Ax) \subseteq \CMs(\ET)$.
\end{theorem}
\begin{proof}
See \cref{sec:et-completeness}.
\end{proof}

%
%\begin{definition}
%An execution test $\ET$ precisely captures an axiomatic specification 
%$(\RP_{\LWW}, \Ax)$ if and only if
%\begin{itemize}
%\item given a kv-store $\hh$ and a view $\vi \in \Views(\hh)$ and a set of operations 
%$\opset$ such that $\ET \vdash \hh, \vi \triangleright \opset : \_$, then for any 
%abstract execution $\aexec$ such that $\hh_{\aexec} = \hh$, there exists a set 
%of transactions $\T = \T' \cup \Tx(\hh, \vi)$, where $\T' \subseteq \T_{\aexec}$ 
%only contains read-only transactions, such that for any $\A \in \Ax.\;\{ \txid' \mid (\txid', \txid) \in \A(\extend(\aexec, \_, \T, \opset)) \times \subseteq \T$, 
%\item given an abstract execution $\aexec$, a transaction identifier $\txid \notin \T_{\aexec}$, a 
%set of operations $\opset$ and a set of transactions $\T$ such that 
%$\forall \A \in \Ax.\{\txid' \mid (\txid', \txid) \in \A(\extend(\aexec, \txid, \T, \opset))\} \subseteq \T$, 
%then $\ET \vdash \hh_{\aexec}, \getView(\aexec, \T) \triangleright \opset : \_$. 
%\end{itemize}
%\end{definition}
%
%\begin{theorem}
%If $\ET$ precisely captures the axiomatic specification $(\RP_{\LWW}, \Ax)$, then 
%\label{thm:prooftechnique_sound}
%\begin{itemize}
%\item $\CMs((\RP_{\LWW}, \Ax)) = \{ \hh_{\aexec} \mid \aexec \in \CMa(\RP_{\LWW}, \Ax)\}$, 
%\item $\forall \prog. \interpr{\prog}_{(\RP_{\LWW}, \Ax)} = \{ \hh_{\aexec} \mid \aexec \in \interpr{\prog}_{\ET} \}$.
%\end{itemize}
%\end{theorem}
