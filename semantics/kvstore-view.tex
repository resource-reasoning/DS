\subsection{Multi-version Key-value Stores and Views}
\label{sec:mkvs-view}

\subsubsection{Key-value Stores} 

We assume a countably infinite set of \emph{keys} $\Keys \defeq \Set{\ke, \ke', \cdots}$, 
a set of \emph{values} $\Val \defeq \{\val, \val', \cdots\}$ which for simplicity we instantiate to be 
$\Nat \uplus \Keys$, a set of clients $\Clients \defeq \Set{\cl, \cl',\cdots}$. 
We also assume a set of transaction identifiers $\TxID \defeq \Set{ \txid_{\cl}^{n} \mid \cl \in \Clients \wedge n \geq 0 } 
\uplus \Set{\txid_{0}}$,
each of which is either a special transaction identifier $\txid_0$, 
or it is indexed by a client identifier and a natural number. 
Elements of $\TxID$ are ranged over by $\txid, \txid', \cdots$, 
while subsets of $\TxID$ are ranged over $\txidset, \txidset', \cdots$. 
We let $\TxID_{0} \defeq \TxID \setminus \{ \txid_0\}$.
The structure of the set $\TxID$  
embeds the order in which transactions are executed by individual clients, or \emph{session order}. 
Specifically, we let $\PO \defeq \Set{ (\txid, \txid') \mid \exsts{ \cl, n,m } \txid = \txid_{\cl}^{n} \wedge \txid' = \txid_{\cl}^{m} \wedge n < m}$; 
$(\txid, \txid') \in \PO$ means that 
some client $\cl$ has executed $\txid$ prior to $\txid'$. For $\PO$ (and in general  
for relations between transaction identifiers) we will often adopt the more graphic notation 
$\txid \xrightarrow{\PO} \txid'$ in lieu of $(\txid, \txid') \in \PO$.

Given a set $X$, then $\powerset{X}$ denotes 
the powerset of $X$, while $X^{\ast}$ is the free monoid induced by $X$.


\begin{definition}[Multi-version Key-value Stores]
\label{def:his_heap}
\label{def:mkvs}
A \emph{version} is a triple $\ver = (\val, \txid, \txidset)$. The set of versions is denoted by $\Versions \defeq \Val \times \TxID \times \powerset{\TxID_{0}}$, 
and a \emph{key-value store} is a mapping $\hh \in \MKVSs \defeq \Keys \rightarrow \Versions^{\ast}$. 
\ac{ The superscript fin over the $\rightharpoonup$ needs to be fixed. You may want to look at the package extpfeil.}
%The set of key-value stores is denoted as $\HisHeaps$.
\end{definition}

\emph{A version} $\ver = (\val, \txid, \txidset)$ consists of a value $\val$, and the meta-data of the transactions 
that accessed the version; specifically, $\txid$ is the identifier of the transaction that wrote such a version, 
and $\txidset$ is the set of identifiers of transactions that read the version.
Given a version $\ver = (\val, \txid, \txidset)$, we let $\valueOf(\ver) \defeq \val$. 
$\WTx(\ver) \defeq \txid$ and $\RTx(\ver) \defeq \txidset$.
Lists of versions, that is elements of $\Versions^{\ast}$, are ranged over by $\vilist, \vilist',\cdots$.

\emph{A multi-version key-value store}, or \emph{kv-store}, 
is a mapping from keys to lists of versions. 
For a given kv-store $\hh$, key $\ke$ and index $i \geq 0$, we use the notation $\hh(\ke, i)$ 
to denote the $i$-th version (starting from $0$) installed for $\ke$; that is, if $\hh(\ke) = \ver_0 \cdots\ver_{n}$, then 
$\hh(\ke, i) \defeq \ver_{i}$ if $i \leq n$, it is undefined otherwise. We also let $\lvert \hh(\ke) \rvert \defeq n +1 $ denote 
the length of $\hh(\ke)$.

{\color{red} 
It will be often convenient to depict key-value stores graphically: an 
example is given by the kv-store $\hh$ depicted in \cref{fig:hheap-a}
{\color{red} (ignore for the moment the vertical lines labelled $\client$ and $\client'$)}. 
It comprises two keys \( \ke_1\) and \( \ke_2 \), 
each of which is associated with two versions carrying values $\val_0$ and $\val_1$, and $\val'_0$ and $\val'_1$, respectively.
The versions of a key are listed in order from left to right. 
We represent each version as a three-cell box, with the left cell storing the value, the top right cell recording the writer, and the bottom right cell recording the readers. 
For example, the version carrying value $\val_0$ in $\ke_1$ has been written by $\txid_0$, and has been read by $\txid_{\cl'}^1$.
}

In this paper we focus on key-value stores whose consistency model enforces the  
\emph{atomic visibility} of transactions \cite{framework-concur}. 
We also assume that in kv-stores, keys with a defined list of versions, have an initial version carrying  a default value $\val_0 \in \Val$, 
written by the special transaction identifies $\txid_0$.
A \emph{well-formed} kv-store $\hh$ requires that:
\begin{enumerate}[(i)]
\item\label{kv:wf.init} for each key $\ke \in \dom(\hh)$, $\hh(\ke, 0) = (\val_0, \txid_0, \stub)$, where $\val_0$ is a default value from $\Val$;
\item\label{kv:wf.onewrite} transactions never write more than one version per key,  
\[
\fora{\ke \in \dom(\hh), i,j : 0 \leq i, j < \abs{ \hh(\ke) }}
\WTx(\hh(\ke, i)) = \WTx(\hh(\ke, j)) \implies i = j 
\]
\item\label{kv:wf.oneread} transactions never read different versions for the same key, 
\[
\fora{\ke \in \dom(\hh), i,j : 0 \leq i, j < \abs{ \hh(\ke) }} 
\RTx(\hh(\ke, i)) \cap \RTx(\hh(\ke, j)) \neq \emptyset \implies i = j
\]
\item\label{kv:wf.so} the order 
in which transactions issued by the same client install different versions for some key $\ke$, is consistent with the order in which 
such transactions have been invoked; similarly, a client can read the version of a key $\ke$ only after it installed it. 
\begin{multline*}
    \fora{ \ke \in \dom(\hh), \cl \in \Clients, i,j: 0 \leq i < j < \abs{\hh(\ke)}, n, m} \\
(\txid_{\cl}^{n} = \WTx(\hh(\ke,i)) \wedge \txid_{\cl}^{m} \in \{\WTx(\hh(\ke,j))\} \cup \RTx(\hh(\ke, i)) \implies n < m.
\end{multline*}
\end{enumerate}
We always assume that kv-stores are well-formed, and let $\HisHeaps$ be the set of well-formed kv-stores.


\begin{figure}
\begin{center}
\hrule
\begin{tabular}{@{}c @{\qquad} c@{}}
\begin{halfsubfig}
\begin{tikzpicture}
\begin{pgfonlayer}{foreground}
%\draw[help lines] grid(5,4);

%Location x
\node(locx) {$\ke_1 \mapsto$};

\matrix(versionx) [version list]
    at ([xshift=\tikzkvspace]locx.east) {
    {a} & $\txid_0$ & {a} & $\txid_{\cl}^{1}$\\
    {a} & $\left\{\txid_{\cl'}^{1}\right\}$ & {a} & $\emptyset$ \\
};
\tikzvalue{versionx-1-1}{versionx-2-1}{locx-v0}{$v_0$};
\tikzvalue{versionx-1-3}{versionx-2-3}{locx-v1}{$v_1$};

%Location y
\path (locx.south) + (0,\tikzkeyspace) node (locy) {$\ke_2 \mapsto$};
\matrix(versiony) [version list]
   at ([xshift=\tikzkvspace]locy.east) {
 {a} & $\txid_0$ & {a} & $\txid_{\cl'}^{1}$ \\
  {a} & $\left\{\txid_{\cl}^1\right\}$ & {a} & $\emptyset$\\
};

\tikzvalue{versiony-1-1}{versiony-2-1}{locy-v0}{$v'_0$};
\tikzvalue{versiony-1-3}{versiony-2-3}{locy-v1}{$v'_1$};

% \draw[-, red, very thick, rounded corners] ([xshift=-5pt, yshift=5pt]locx-v1.north east) |- 
%  ($([xshift=-5pt,yshift=-5pt]locx-v1.south east)!.5!([xshift=-5pt, yshift=5pt]locy-v0.north east)$) -| ([xshift=-5pt, yshift=5pt]locy-v0.south east);

%blue view - I should  check whether I can use pgfkeys to just declare the list of locations, and then add the view automatically.
\draw[-, blue, very thick, rounded corners=10pt]
 ([xshift=-3pt, yshift=20pt]locx-v1.north east) node (tid1start) {} -- 
 ([xshift=-3pt, yshift=-5pt]locx-v1.south east) --
 ([xshift=-3pt, yshift=5pt]locy-v0.north east) -- 
 ([xshift=-3pt, yshift=-5pt]locy-v0.south east);
 
 \path (tid1start) node[anchor=south, rectangle, fill=blue!20, draw=blue, font=\small, inner sep=1pt] {$\client$};

%red view
\draw[-, red, very thick, rounded corners = 10pt]
 ([xshift=-16pt, yshift=5pt]locx-v1.north east) node (tid2start) {}-- 
% ([xshift=-16pt, yshift=-5pt]locx-v0.south east) --
% ([xshift=-16pt, yshift=5pt]locy-v1.north east) -- 
 ([xshift=-16pt, yshift=-5pt]locy-v1.south east) node {};
 
\path (tid2start) node[anchor=south, rectangle, fill=red!20, draw=red, font=\small, inner sep=1pt] {$\client'$};

\end{pgfonlayer}
\end{tikzpicture}
\caption{A configuration with a well-formed kv-store $\hh$ and two views $\vi,\vi'$.}
\label{fig:hheap-a}
\end{halfsubfig}
&

\begin{halfsubfig} 
\begin{center}
\begin{tikzpicture}[scale=0.85, every node/.style={transform shape}]
%\draw[help lines] grid(6,4);

\node(t0wx) at (-1,2) {$(\otW, \ke_1, \val_0)$}; 
\path (t0wx.south) + (0,-0.2) node[anchor=north] (t0wy) {$(\otW, \ke_2, \val'_0)$};
\path (t0wx.north east) + (1,0.5) node[anchor = west] (t1ry) {$(\otR, \ke_2, \val'_0)$}; 
\path (t1ry.east) + (0.2,0) node[anchor = west] (t1wx) {$(\otW, \ke_1, \val_1)$};
\path (t0wy.south east) + (1,-0.5) node[anchor = west] (t2rx) {$(\otR, \ke_1, \val_0)$};
\path (t2rx.east) + (0.2,0) node[anchor = west] (t2wy) {$(\otW, \ke_2, \val'_1$)};

\begin{pgfonlayer}{background}
\node[background, fit=(t0wx) (t0wy)] (t0) {};
\node[background, fit= (t1ry) (t1wx)] (t1) {};
\node[background, fit= (t2rx) (t2wy)] (t2) {};

\path(t0.west) node[anchor=east] (t0lbl) {$\txid_0$};
\path(t1.north) node[anchor=south] (t1lbl) {$\txid_1$};
\path(t2.south) node[anchor=north] (t2lbl) {$\txid_2$};

\path[->]
(t0.north) edge[bend left=70] node[above, yshift=7pt, xshift=-1pt, pos=0.3] {$\RF(\ke_2), \VO(\ke_1)$} (t1.west)
(t0.south) edge[bend right=70] node[below, yshift=-8pt, xshift=-1pt, pos=0.3] {$\RF(\ke_1), \VO(\ke_2)$} (t2.west)
([xshift=-8pt]t2.north) edge[bend left=40] node[left] {$\AD(\ke_1)$} ([xshift=-8pt]t1.south) 
([xshift=8pt]t1.south) edge[bend left=40] node[right] {$\AD(\ke_2)$} ([xshift=8pt]t2.north);
\end{pgfonlayer}

\end{tikzpicture}
\end{center}
\caption{The dependency graph induced by $\hh$.}
\label{fig:hheap-b}
\end{halfsubfig} \\
\end{tabular}
\end{center}
\hrule
\caption{Multi-version key-value stores}
\label{fig:hheap}
\end{figure}

\mypar{Views, Configurations and Snapshots.}
The key-value store tracks the global state, 
but when executing transactions, different \emph{clients} may observe 
different versions of the same key. To keep track of 
the versions they observe, clients are associated with \emph{views} (\cref{def:view}). 

\begin{definition}[Views and configurations]
\label{def:view}
\label{def:cuts}
\label{def:views}
\label{def:configuration}
Given a key-value store $\hh$, \emph{a view} of $\hh$ is a function  
$\vi: \dom(\hh) \to\powerset{\Nat}$ such that  
\[
\forall \ke \in \dom(\hh).\; 0 \in \vi(\ke) \wedge \forall i \in \vi(\ke).\; i < \lvert \hh(\ke) \rvert.
\]
and 
\begin{equation}
\label{eq:view.atomic}
\begin{array}{@{}l@{}}
\fora{ \ke,\ke' \in \dom(\hh), i,j \in \Nat} \\
\quad (j \in \vi(\ke) \wedge \WTx(\hh(\ke, j)) = \WTx(\hh(\ke', i)) \implies i \in \vi(\ke')
\end{array}
\tag{Atomic}
\end{equation}

The set of views of $\hh$ is denoted 
as $\Views(\hh)$, and \emph{the set of views} is defined as:
\[
\Views \defeq \bigcup_{\hh \in \HisHeaps} \Views(\hh)
\]
A \emph{configuration} $\conf$ is a pair $(\hh, \viewFun)$, where $\viewFun: 
\Clients \parfinfun \Views(\hh)$. The configuration $\conf_{0} = (\hh_{0}, \viewFun_{0})$ is 
initial if, for any $\ke$, $\hh_{0}(\ke) = (\val_0, \txid_0, \emptyset)$, for some 
initial value $\val_0$. 
The set of configurations is denoted as $\Confs$.
\end{definition}
Given $\hh \in \HisHeaps$ and two views $\vi, \vi' \in \Views(\hh)$, 
we let $\vi \viewleq \vi'$ if, for any $\ke \in \dom(\hh)$, $\vi(k) \subseteq \vi'(\ke)$. 

\emph{A configuration} includes a kv-store and a partial mapping from clients from clients to views.
The view of the client $\cl$ in $\hh$ reflects the set of versions for each key 
that the client \(\cl \) observes upon executing a transaction. 
The constraint of \cref{eq:view.atomic} establishes that if a client observes 
a version of some key written by a transaction $\txid$, then it must observe all the versions of 
all keys that $\txid$ wrote. This constraint captures the \emph{atomic visibility} of transactions.

{\color{red} We often depict views of clients graphically by drawing client-labelled lines crossing 
versions of key-value stores. A line crossing the $i$-th version of key $\ke$ defines a view 
$\vi$ for client $\cl$, with $\vi(\ke) = i$. One example is given by \cref{fig:hheap-a} where the configuration is
$\conf_0 = (\hh_0, \Set{\cl_1 \mapsto \vi_1, \cl_2 \mapsto \vi_2})$. 
There are two clients, 
$\cl_1$ and $\cl_2$, with views $\vi_1$, and $\vi_2$ respectively. $\vi_1$ crosses $\ke_1$ at its $0$-th 
version, and $\ke_2$ at its $1$-st version. Therefore we have $\vi_1 = \Set{\ke_1 \mapsto 0, \ke_2 \mapsto 1}$. 
Similarly, we have $\vi_2 = \Set{\ke_1 \mapsto 1, \ke_2 \mapsto 0}$. }

Given a kv-store $\hh$, a view $\vi$ and a key $\ke \in \dom(\hh)$, 
we commit an abuse of notation and write $\hh(\ke, \vi)$ as a shorthand 
for $\hh(\ke, \max_{<}(\vi(\ke)))$. Note that such a version is well-defined because 
we are assuming that $\vi(\ke) \neq \emptyset$.
The view $\vi$ naturally induces a \emph{snapshot} 
by extracting the value of the most up-to-date version it observes for each key $\ke \in \dom(\hh)$. 
As we will see presently, transactions are executed relatively 
to a snapshot of a kv-store, which maps each key to a single value.
Views are used to determine the snapshot in which a transaction 
is executed, according to the following definition.
\begin{definition}[Snapshots]
\label{def:heaps}
\label{def:snapshot}
Given $\hh \in \HisHeaps$ and $\vi \in \Views(\hh)$, the \emph{snapshot} of $\vi$ in 
$\hh$ is defined as $\snapshot(\hh, \vi) \defeq \lambda \ke \ldotp \valueOf(\hh(\ke, \max_{<}(\vi(\ke)))$.
\end{definition}

%of $\vi$ by accessing the value of 
%A view $\vi$ in $\hh$ naturally defines a snapshot $\snapshot(\hh, \nu)$
%A MKVS tracks the global state of the system; however, different \emph{clients} may observe different versions of the same key. 
%To model this, we introduce the notion of \emph{views} (\cref{def:views}). 
%A view $V$ reflects the particular version for each key that a client observes upon executing a transaction. 
%%We present an example of views in \cref{fig:hheap-a} with two views: $\client_1$ in red and $\client_2$ in blue.
%More concretely, the view for \( \client_1 \) is given formally as $\vi_1 = \Set{\key{k}_1 \mapsto 1, \key{k}_2 \mapsto 0}$.
%That is, the client with view $\vi_1$ observes the second version (at index 1) of key \( \ke_{1} \) with value $v_1$, and the first version (at index 0) of key \( \ke_2 \) with value $v'_0$.
%%, and 
%%the first version of $\key{k}_2$, carrying value $0$. Similarly, according to its view 
%%$V_2 = [\key{k}_1 \mapsto 2, \key{k}_2 \mapsto 2]$, the client $\txid_2$ observes 
%%in $\hh$the second and most up-to-date version for both $\key{k}_1$ and $\key{k}_2$.
%
%\begin{definition}[Views]
%\label{def:view}
%\label{def:cuts}
%\label{def:views}
%\emph{A view} is a partial finite function from keys to indexes:
%$
%\vi \in \Views \defeq \Addr \parfinfun \Nat 
%%\begin{rclarray}
%%    \vi \in \Views & \defeq & \Addr \parfinfun \Nat 
%%\end{rclarray}
%$.                                                                 
%The \emph{view composition}, $\composeVI: \Views \times \Views \rightharpoonup \Views$ is defined as the standard disjoint function union: $\composeVI \eqdef \uplus$. 
%% \( \vi \composeVI \vi' \defeq \vi \uplus \vi'\) 
%The \emph{unit view}, $\unitVI \in \Views$, is a function with an empty domain: $\unitVI \eqdef \emptyset$. 
%% and the unit is \( \unitVI \defeq \emptyset\).
%The \emph{order relation} on views, $\orderVI: \Views \times \Views$, is defined between two views with the same domain as the point-wise comparison of their indexes for each entry: 
%\[
%\begin{rclarray}
%    \vi \orderVI \vi' & \defiff & \dom(\vi) = \dom(\vi') \land \fora{\ke} \cu(\ke) \leq \cu'(\ke) \\
%\end{rclarray}
%\]
%\end{definition}
%%
%We say view $\vi$ is \emph{older} than view $\vi'$ (or $\vi'$ is \emph{newer} than $\vi$) whenever $\vi \orderVI \vi'$ holds.
%
%
%\mypar{Configurations} A \emph{configuration} comprises an MKVS, and the views associated with clients.
%In \cref{fig:hheap-a} we present an example of a configuration comprising an MKVS and the two views associated with clients $\client_1$ and $\client_2$. 
%We write $\version(\hh, \ke, \vi)$ for $\hh(\ke, \vi(\ke))$; 
%and write $\valueOf(\hh, \ke, \vi)$ as a shorthand for $ \valueOf(\version(\hh, \key{k}, V))$; similarly for $\WTx, \RTx$.
%%we commit an abuse of notation and often write $\valueOf(\hh, \ke, \vi)$ in lieu of $ \valueOf(\version(\hh, \key{k}, V))$, and similarly for $\WTx, \RTx$.
%When $\ver = \version(\hh, \ke, \vi)$, we say that \emph{$\vi$ $\ke$-points to $\ver$ in $\hh$}. 
%When $\ver = \hh(\ke, i)$ for some $0 \leq i \le \vi(\ke)$, we say that \emph{$\vi$ $\ke$-includes $\ver$ in $\hh$}.
%Lastly, we always assume that MKVSs, views, and configurations are well-formed, unless otherwise stated.
%
%
%
%\begin{definition}[Configurations]
%A view $\vi$ is \emph{well-formed with respect to an MKVS} $\mkvs$, written \( \wfV{\mkvs, \vi} \),  iff they have the same domain and every index from $\vi$ is within the range of the corresponding entry in $\mkvs$ and the view is \emph{atomic} with  respect to the key-value store: 
%\[
%\begin{rclarray}
%    \wfV{\mkvs, \vi} & \defeq & \dom(\mkvs) = \dom(\vi) \land \fora{\ke \in \dom(\vi)} 0 \leq \vi(\ke) < \lvert \mkvs(\ke) \rvert \\
%    \pred{atomic}{\vi ,\hh} & \eqdef & \fora{\txid } \exsts{\ke, i} i \leq \vi(\ke) \land \hh(\ke,i) = (\stub, \txid, \stub) \implies \pred{visible}{\txid, \vi, \hh} \\ 
%    \pred{visible}{\txid, \vi, \hh} & \eqdef & \fora{\ke, i} \hh(\ke,i) = (\stub, \txid, \stub) \implies i \leq \vi(\ke) 
%\end{rclarray}
%\]
%%
%\azalea{We need a symbol for this to fill the ???? above. Also ???? below. \sx{Done}}
%A \emph{configuration} $\conf$ is a pair of the form $(\hh, \viewFun)$, where $\hh$ denotes an MKVS, and $\viewFun: \Clients \parfinfun \Views$ is a partial finite function from clients to views. 
%A configuration $\conf = (\hh, \viewFun)$ is \emph{well-formed}, written \( \wfC{\conf}\), iff for all clients $\cl \in \dom(\viewFun)$, the view $\viewFun(\txid)$ is well-formed with respect to $\hh$. 
%%We say that a view $V$ is well-defined with respect to the 
%%MKVS $\hh$ if, $\forall \key{k} \in \ke. 0 < V(\key{k}) \leq 
%%\lvert \hh(\key{k}) \rvert$. 
%%Given a view $V$ that is well-defined 
%%with respect to a 
%
%\end{definition}
%
%\mypar{Snapshots} When a client executes a transaction on the $\mkvs$ MKVS, it extracts a \emph{snapshot} of it via the \( \func{snapshot}{\mkvs, \vi} \) function, extracting the values corresponding to the versions indexed by its view \( \vi \) (\cref{def:snapshot}).
%For instance, for client \( \client_1 \) in \cref{fig:hheap-a}, the $\func{snapshot}{\cdots}$ functions yields a state where key $\ke_1$ carries value $v_1$ and second key \( \ke_2 \) carries value $v'_0$.
%%The concrete state extracted in this way takes the name of the \emph{snapshot} of the transaction.
%%In general, the process of determining the view of a client, hence the snapshot in which such a client executes transactions, is non-deterministic.
%
%\azalea{Before in MKVSs we had values drawn from $\Nat$ in \cref{def:mkvs}. Now we use $\Val$. I think you mean to use $\Val$ in both places? \sx{I would say so} }
%\begin{definition}[Snapshots]
%\label{def:heaps}
%\label{def:snapshot}
%Given the sets of values $\Val$  and keys \( \Addr\)  (\cref{def:mkvs}), the set of \emph{snapshots} is:
%$
%    \h \in \Heaps \eqdef \Addr \parfinfun \Val
%$. 
%%\[
%%\begin{rclarray}
%%    \h \in \Heaps & \eqdef & \Addr \parfinfun \Val
%%\end{rclarray}
%%\]
%The \emph{snapshot composition function}, $\composeH: \Heaps \times \Heaps \parfun \Heaps$, is defined as $\composeH \eqdef \uplus$, where $\uplus$ denotes the standard disjoint function union. The \emph{ snapshot unit element} is $\unitH \eqdef \emptyset$, denoting a function with an empty domain.
%The \emph{partial commutative monoid of snapshots} is $(\Heaps, \composeH, \{\unitH\})$.
%Given an MKVS $\hh$ and a view $\vi$, the snapshot of $\vi$ in $\hh$, written $\snapshot(\hh, \vi) $, is defined as:
%$
%    \snapshot(\hh, \vi) \defeq \lambda \ke \ldotp \valueOf(\hh, \ke, \vi)
%$.
%%\[
%%\begin{rclarray}
%%    \snapshot(\hh, \vi) & \defeq & \lambda \ke \ldotp \valueOf(\hh, \ke, \vi).
%%\end{rclarray}
%%\]
%\end{definition}
%
%\sx{Need some explanation}
%\ac{General Comment on this Section: it is too abstract. We 
%should give either here or in the introduction an example of computation - 
%the write skew program should be okay that helps the reader understanding 
%what's going on. Also, it could be also good to illustrate the notions 
%of execution tests and consistency models.}
%
%\sx{From Andrea: introduce the execution test here with a table, also introduce fingerprint here}

