\section{KV-Stores and Other Formalisms}
\label{sec:other_formalisms}

\begin{figure*}[t]
\centering
\noindent
\begin{subfigure}{0.49\textwidth}
\begin{centertikz}
\draw pic {transaction={t0}{%
        $\otW$/$\key_1$/$\val_0$%
        , $\otW$/$\key_2$/$\val'_0$%
}};
\path(t0.west) node[anchor=east] (t0lbl) {$\txid_0$};

\draw pic at ($(t0.north east) + (1.5,0.6)$) {transaction={t1}{%
        $\otR$/$\key_2$/$\val'_0$%
        , $\otW$/$\key_1$/$\val_1$%
}};
\path(t1.north) node[anchor=south] (t1lbl) {$\txid_1$};

\draw pic at ($(t0.south east) + (1.5,-0.3)$) {transaction={t2}{%
        $\otR$/$\key_1$/$\val_0$%
        , $\otW$/$\key_2$/$\val'_1$%
}};
\path(t2.south) node[anchor=north] (t2lbl) {$\txid_2$};

\path[->]
(t0.north) edge[bend left=30] node[above, yshift=3pt, xshift=-20pt, pos=0.3] {$\WR$} (t1.west)
(t0.south) edge[bend right=30] node[below, yshift=-3pt, xshift=-20pt, pos=0.3] {$\WR$} (t2.west)
([xshift=8pt]t1.south) edge[bend left=20] node[right] {$\RW$} ([xshift=8pt]t2.north)
([xshift=-16pt]t2.north) edge[bend left=20] node[right] {$\RW$} ([xshift=-16pt]t1.south);

\end{centertikz}
\caption{Dependency graph}
\label{fig:dependency-graph}
\end{subfigure}
%
\hfill
%
\begin{subfigure}{0.49\textwidth}
\begin{centertikz}

\draw pic {transaction={t0}{%
        $\otW$/$\key_1$/$\val_0$%
        , $\otW$/$\key_2$/$\val'_0$%
}};
\path(t0.west) node[anchor=east] (t0lbl) {$\txid_0$};

\draw pic at ($(t0.north east) + (1.5,0.6)$) {transaction={t1}{%
        $\otR$/$\key_2$/$\val'_0$%
        , $\otW$/$\key_1$/$\val_1$%
}};
\path(t1.north) node[anchor=south] (t1lbl) {$\txid_1$};

\draw pic at ($(t0.south east) + (1.5,-0.3)$) {transaction={t2}{%
        $\otR$/$\key_1$/$\val_0$%
        , $\otW$/$\key_2$/$\val'_1$%
}};
\path(t2.south) node[anchor=north] (t2lbl) {$\txid_2$};

\path[->]
(t0.north) edge[bend left=30] node[above, yshift=3pt, xshift=-20pt, pos=0.3] {$\VIS, \AR$} (t1.west)
(t0.south) edge[bend right=30] node[below, yshift=-3pt, xshift=-20pt, pos=0.3] {$\VIS, \AR$} (t2.west)
([xshift=8pt]t1.south) edge[bend left=20] node[right] {$\AR$} ([xshift=8pt]t2.north);

\end{centertikz}
\caption{Abstract execution graph}
\label{fig:abstract_execution}
\end{subfigure}

\hrulefill

\caption{The dependency graph (\subref{fig:dependency-graph}) and abstract execution graph (\subref{fig:abstract_execution}) associated with the kv-store in \cref{fig:ser-disallowed}
}
\end{figure*}

We compare kv-stores and execution 
tests to existing declarative formalisms for specifying  
consistency models, based on dependency graphs \cite{adya} 
and abstract executions \cite{framework-concur}. 
We outline our results here, and refer the reader to the 
appendix for more details.


\subsection{Relating KV-Stores and Dependency Graphs}
\label{sec:dep_graphs}
Dependency graphs \cite{adya-icde,adya} are perhaps the most popular 
formalism used for specifying transactional consistency models. 
A dependency graph $\Gr$ is a directed, labelled graph whose 
nodes denote transactions, and whose edges denote \emph{dependencies} between transactions.  
More specifically, nodes are labelled with a transaction identifier and a fingerprint each.
Edges are labelled with a dependency relation $\SO, \WR, WW, RW$, in the 
same spirit of dependencies of transactions in kv-stores \cref{sec:cm}.
An example of dependency graph is given in \cref{fig:dependency-graph}.
%
%\begin{enumerate*}
%    \item a \emph{session order} edge, $\txid_1 \toEDGE{\SO} \txid_2$, 
%	\item a \emph{read dependency} edge, $\txid_1 \toEDGE{\WR} \txid_2$, denotes
%that transaction $\txid_2$ reads a version written by $\txid_1$;
%	\item a \emph{write dependency} edge, $\txid_1 \toEDGE{\WW} \txid_2$, denotes that $\txid_2$ overwrites a version written by $\txid_1$; and 
%	\item an \emph{anti-dependency} edge, $\txid_1 \toEDGE{\RW} \txid_2$, denotes that $\txid_2$ overwrites a version read by $\txid_2$. 
%\end{enumerate*}
We give the formal definition of dependency graphs in \cref{app:depgraphs}.
It is immediate to observe that we can always \emph{extract} a dependency graph  $\Gr$ from a kv-store $\mkvs$.
%we choose the transaction identifiers appearing in $\mkvs$ as the nodes of $\Gr$, 
%and let $\SO$ as defined in \cref{subsec:kvstores}, and $ \WR, \WW, \RW$  
%be as defined in \cref{sec:c}.
For example, \cref{fig:dependency-graph} corresponds to the dependency graph extracted from the kv-store in \cref{fig:counter_kv_final}.
In fact, this construction can be reversed, a fact that we prove formally in \cref{app:depgraphs}.
%we show that this construction can be reversed, thus giving 
%rise to the following result: 
\begin{theorem}
\label{thm:kv_graph_isomorph}
Dependency graphs are isomorphic to kv-stores.
\end{theorem}

Consistency models can be specified by constraining the shape of dependency graph; typically, such constraints require the absence of particular cycles 
from the dependency graphs. 
For example, strict serialisability is defined as the set of acyclic dependency graphs. %where $(\SO \cup \WW \cup \WR \cup \RW)^+$ is acyclic. 
Dependency graph-based specifications can be converted into execution-test-based ones: such execution 
tests prevent committing transactions that result in kv-stores whose associated dependency graph violates  
the constraints imposed by the consistency model.
% the effects of committing a transaction to kv-store $\mkvs$ leads to $\mkvs'$, 
%then check that dependency graph extracted from $\mkvs'$ contains no cycles prohibited by the dependency graph-based specification. 
%
%\mypar{Relating KV-Stores and Abstract Executions}
%\emph{Abstract executions} \cite{ev_transactions,framework-concur} are an alternative formalism for defining consistency models. 
%As with dependency graphs, an abstract execution graph $\aexec$
%is a directed graph with its nodes representing transactions (with each node labelled with a transaction identifier and a set of (read/write) operations), 
%and its edges representing certain relations between transactions. 
%An example abstract execution graph is depicted in \cref{fig:abstract_execution}. 
%Each edge is labelled by either the \emph{visibility} ($\VIS$) or \emph{arbitration} ($\AR$) relation. 
%The $\VIS$ is an irreflexive order on transactions such that $\txid_1 \toEDGE{\VIS} \txid_2$ denotes that the effects (updates) of $\txid_1$ are visible to $\txid_2$. 
%The $\AR$ is a strict total order on transactions such that $\txid_1 \toEDGE{\AR} \txid_2$ denotes that the updates performed by $\txid_2$ are newer than those of $\txid_1$. 
%Moreover, $\AR$ contains $\VIS$ ($\VIS \subseteq \AR$) and agrees with the session order ($\SO \subseteq \AR$).
%Lastly, abstract executions observe the \emph{last-write-wins} policy: 
%a transaction reading $\key$ always fetches the latest visible write ($\VIS$ predecessor) on $\key$.
%We refer the reader to \cref{sec:abstract-execution} for full details.
%
%Following \cite{laws}, we can always \emph{extract} a dependency graph $\Gr_{\aexec}$ from an abstract execution $\aexec$, and thus a kv-store $\mkvs_{\aexec}$ via \cref{thm:kv_graph_isomorph}---see \cref{app:aexec2kv} for the formal details.
%We write  $\mkvs_\aexec$ for the kv-store extracted from $\aexec$ using this construction.  
%Moreover, we show that there is a \emph{Galois connection}
%between $\ET_{\top}$ traces, the weakest possible execution test and abstract executions (\cref{sec:galois-kv-aexec}).
%
%As with dependency graphs, consistency models using abstract executions are defined by constraining the shape of abstract execution graphs via a set of \emph{axioms} $\Ax$, \eg imposing certain conditions on $\VIS$. %the absence of certain cycles.
%All consistency models presented in this paper have an equivalent axiomatic definition based on abstract executions~\cite{framework-concur,laws}. 
%Proving the equivalence of execution test-based and abstract execution-based definitions is non-trivial; 
%however, we have observed that all proofs follow the same \emph{structure}, so long as certain conditions hold. 
%We develop the meta-theory to capture this proof structure.
%Our meta-theory is non-trivial; we refer the reader to \cref{sec:kv2aexec-sound-complete} for the full details. 
%We give an intuitive account of the two conditions required by our meta-theory, and then state our equivalence theorem. 
%
%Given a set of axioms $\Ax$, we define $\CMa(\Ax) \defeq \Set{ \aexec }[ \aexec \text{ sats.} \ \Ax ]$.
%Our first condition is the \emph{soundness} of an execution test against an axiomatic definition.
%An execution test $\ET$ is sound against an axiomatic definition $\Ax$ if:
%for all $n$ and for all \( \ET \)-traces \( \tau \) with \( n \) steps, 
%we can construct $\aexec_0, \cdots, \aexec_n$ and \( \mkvs_i = \mkvs_{\aexec_i} \) such that 
%for each step \( (\mkvs_i, \vi_i) \csat \fp_i : (\mkvs_{i+1}, \vi_{i+1}) \) in \( \tau \),
%the new $\VIS$ edges in \( \aexec_{i+1} \) (those not in $\aexec_i$)
%%, which links transactions included in the view \( \vi_i \) to the new transaction \( \txid_i \),
%satisfy \( \Ax \).
%The formal definition of execution test soundness is given in \cref{sec:kv2aexec-sound-complete}.
%
%Our second condition is the \emph{completeness} of an execution test against an axiomatic definition.
%Let $\txid_i$ denote the $i$\textsuperscript{th} transaction of $\aexec$ in its $\AR$ order, and $\aexec^{i}$ denote the restriction of $\aexec$ to $\txid_1 \cdots \txid_i$. 
%An execution test $\ET$ is complete against an axiomatic definition $\Ax$ if:
%for all abstract executions \( \aexec \) that satisfy \( \Ax \) containing $n$ transactions, 
%all $i \in \Set{1 \cdots n}$, views $\vi_i, \vi'_i$, transactions $\txid'$, and fingerprints $\fp_i$,
%whenever
%\begin{enumerate*}
%	\item $\txid'$ is the immediate $\SO$-successor of $\txid_i$;
%	\item \( \vi_i \) includes all visible transactions of \( \txid_i \); 
%    \item $\vi'_i$ includes all visible transactions, equal or committed before \( \txid_i \), of $\txid'$; and
%	\item $\fp_i$ is the fingerprint of $\txid_i$, 
%\end{enumerate*} 
%then $\ET \vdash (\mkvs_i, \vi_i) \csat \fp_i : (\mkvs_{i+1}, \vi_{i+1})$.
%The formal definition of execution test completeness is given in \cref{sec:kv2aexec-sound-complete}.
%
%Finally, we state our equivalence theorem below (\cref{thm:main-body-et_soundness_completeness}), with its full proof in \cref{sec:kv2aexec-sound-complete}. 
%This theorem ensures that if an execution test is sound and complete against a set of axioms $\Ax$, 
%then the consistency model induced by $\ET$ corresponds to the kv-stores extracted from abstract executions satisfying $\Ax$.
%
%\begin{theorem}
%\label{thm:main-body-et_soundness_completeness}
%For all $\ET, \Ax$, if $\ET$ is sound against $\Ax$, then:
%\(
%\CMs(\ET) \subseteq \Set{ \mkvs_\aexec }[ \aexec \in \CMs(\Ax)]
%\).
%For all $\ET, \Ax$, if $\ET$ is complete against $\Ax$, then:
%\(
%\Set{ \mkvs_\aexec }[ \aexec \in \CMs(\Ax)]  \subseteq \CMs(\ET)
%\).
%\end{theorem} 
%
%In \cref{sec:spec-proof} we apply \cref{thm:main-body-et_soundness_completeness} and show all our definitions in \cref{fig:execution_tests} 
%are sound and complete against (equivalent to) existing axiomatic definitions on abstract executions.
%
%
%\subsection{Andrea's Version}
\subsection{Relating KV-Stores and Abstract Executions}
Abstract executions are another kind of labelled 
graphs: nodes consist of transaction identifiers with the associated fingerprint. %are labelled with a transaction identifier and the associated fingerprint. 
In contrast with dependency graphs, edges in abstract executions are used to 
capture the information that is locally available to one transaction. Each 
edge in abstract executions is labelled either as $\VIS$ or $\AR$: 
\begin{itemize}
\item an edge of the form $\txid \xrightarrow{\VIS} \txid'$:  when transaction 
$\txid'$ is executed, it sees the updates of $\txid$,
\item an edge of the form $\txid \xrightarrow{\AR} \txid'$: updates 
performed by $\txid'$ are newer than those of $\txid$. 
\end{itemize}
% \emph{Abstract executions} \cite{ev_transactions} 
%are directed graph with its nodes representing transactions (with each node labelled with a transaction identifier and a set of (read/write) operations), 
%and its edges representing certain relations between transactions. 
%An example abstract execution graph is depicted in \cref{fig:abstract_execution}. 
%Each node is . 
%Each edge is labelled by either the \emph{visibility} ($\VIS$) or \emph{arbitration} ($\AR$) relation. 
%The $\VIS$ is an irreflexive order on transactions such that $\txid_1 \xrightarrow{\VIS} \txid_2$ denotes that the effects (updates) of $\txid_1$ are visible to $\txid_2$. 
%The $\AR$ is a strict total order on transactions describing the commit order:  $\txid_1 \xrightarrow{\AR} \txid_2$ denotes that $\txid_1$ commits before $\txid_2$. 
%Moreover, $\AR$ contains $\VIS$ ($\VIS \subseteq \AR$) and agrees with the session order: 
%if $\txid_{\cl}^{n} \xrightarrow{\AR}  \txid_{\cl}^{m}$, then $n < m$.
\begin{definition}
\label{def:main-body-absexec}
\label{def:main-body-aexec}
An abstract execution is a triple $\aexec = (\TtoOp{T}, \VIS, \AR)$, where 
 $\TtoOp{T}: \TxID \parfun \pset{\Ops}$ is a partial  
 %$\TtoOp{T}: \TxID_{0} \parfinfun \pset{\Ops}$ is a partial, finite 
function mapping transaction identifiers to the set of operations that they perform,
$\VIS \subseteq \dom(\TtoOp{T}) \times \dom(\TtoOp{T})$ is an irreflexive relation 
such that for any $\txid \in \dom(\TtoOp{T})$, $\txid_{0} \xrightarrow{\VIS} \txid$, 
%called \emph{visibility}, 
$\AR \subseteq \dom(\TtoOp{T}) \times \dom(\TtoOp{T})$ is a strict, total order 
such that $\VIS \subseteq \AR$, %$\min_{\AR}(\dom(\TtoOp{T}) = \txid_{0}$, 
and $\txid_{\cl}^{n} \xrightarrow{\AR} \txid_{\cl}^{m}$ only if $n < m$. 
%Furthermore, for any transaction $\txid \in \dom(\TtoOp{T})$, and $\key \in \Keys$, let 
%$\visibleWrites_{\aexec}(\key, \txid) = \VIS^{-1}(\txid) \cap \{\txid' \mid (\otW,\key, \stub) \in \TtoOp{T}(\txid')\}$; 
%we require that whenever  $(\otR, \key, \val) \in \TtoOp(\txid)$ for some $\key, \val$ 
%and $\txid \in \dom(\TtoOp{T})$, then either $\visibleWrites_{\aexec}(\key, \txid) = \emptyset$ and 
%$\val$ is the default value $\val_{0}$, or the transaction $\txid' = \max_{\AR}(\visibleWrites_{\aexec}(\key, \txid))$ 
%is defined and $(\otR, \key, \val) \in \TtoOp{T}(\txid')$.
\end{definition}
Given an abstract execution $\aexec = (\TtoOp{T}, \VIS, \AR)$ we let $\TtoOp{T}_{\aexec} = \TtoOp{T}$, 
$\VIS_{\aexec} = \VIS$ and $\AR_{\aexec} = \AR$. 
We write $(l, \key, \val) \in_{\aexec} \txid$ a a shorthand for $(l, \key, \val) \in \TtoOp{T}_{\aexec}(\txid)$.
%The set of abstract executions is denoted by $\aeset$.
%Intuitively, $\txid \xrightarrow{\VIS_{\aexec}} \txid'$ means that, at the moment of executing, 
%transaction $\txid'$ sees the updates performed by $\txid$; $\txid \xrightarrow{\AR_{\aexec}} \txid'$ 
%means that the updates performed by $\txid'$ are newer than those performed by $\txid$. 
We are interested in abstract executions that satisfy the \emph{last-write-wins} resolution policy: 
if transaction $\txid$ reads value $\key$ in $\aexec$, then it reads the most up-to-date version among 
the ones written by the transaction it sees. Formally, let $\visibleWrites_{\aexec}(\key, \txid) = \VIS^{-1}_{\aexec}(\txid) \cap 
\{ \txid' \mid (\otW, \key, \stub) \in_{\aexec} \txid'\}$: then $\forall \key, \val.\; (\otR,\key,\val) \in_{\aexec} \txid
\implies (\otW, \key, \val) \in \max_{\AR_{\aexec}}(\visibleWrites_{\aexec}(\key, \txid))$.
We let $\aeset$ be the set of abstract executions that satisfy the last-write-wins policy; henceforth we only consider abstract executions in $\aeset$.
%
%The constraint in \cref{def:main-body-aexec} amounts to require that a transaction reading 
%key $\key$ always fetches the newest write over $\key$ performed in its set of visible transactions, 
%thus modelling the last write wins policy. An example of abstract execution is given in \cref{fig:abstract_execution}.

Following \cite{laws}, it is always possible to extract a dependency graph $\Gr_{\aexec}$ from an abstract 
execution $\aexec$, and therefore a kv-store $\hh_{\aexec}$ (cf. \cref{thm:kv_graph_isomorph}). 
We defer the formal details of such a construction to \cref{app:depgraphs}, 
but we mention here that in such a construction, we draw an edge of the form $\txid \xrightarrow{\RF} \txid'$ 
if $\txid = \max_{\AR_{\aexec}}(\visibleWrites_{\aexec}(\key, \txid'))$ and $(\otR, \key, \stub) \in \TtoOp{T}(\txid')$ for some $\key \in \Keys$, 
while $\txid \rightarrow{\VO} \txid'$ if $(\otW, \key, \stub) \in \TtoOp{T}(\txid), (\otW, \key, \stub) \in \TtoOp{T}(\txid')$ and 
$\txid \xrightarrow{\AR_{\aexec}} \txid'$; edges labelled as $\AD$ can be uniquely determined from edges labelled as 
$\RF$ and $\VO$.
\sx{If we dont have space. we can cut the technical details here}

Specifications of consistency models based on abstract executions are given by imposing 
constraints on the structure of the latter; the set of abstract executions that satisfy those constraints  
is then mapped into kv-stores. 
For example, strict serialisability can be formalised by only considering 
those kv-stores $\hh_{\aexec}$ obtained from abstract executions 
$\aexec$ where $\AR_{\aexec} \subseteq \VIS_{\aexec}$. If no
constraints are placed on the structure of abstract executions, 
the resulting set of kv-stores is equivalent to the \emph{Read Atomic} 
consistency model \cite{ramp}, the cnsistency model induced by the most 
permissive execution test $\CMs(\ET_{\top})$, a fact that we prove shortly.
%a variant of \emph{Eventual Consistency} 
%\cite{ev_transactions} that assumes the last-write-win resolution policy.

Formally, in declarative specifications of consistency models,  
%we require an abstract execution $\aexec$ to satisfy axioms of the 
axioms map abstract executions into constraints of the form
form $\mathcal{R}_{\aexec} \subseteq \VIS_{\aexec}$.
%, where $\mathcal{R}_{\aexec} \subseteq \AR_{\aexec}$. 
Given a set of axioms $\Ax$, we let $\CMs(\Ax) = \{\hh_{\aexec} \mid \aexec \text{ satisfies } \Ax\}$. 
Axiom are functions $\A : \aeset \to \pset{\TxID \times \TxID}$: we say that $\aexec$ satisfies $\A$, if $\A(\aexec) \subseteq \VIS_{\aexec}$; 
for technical reasons, we require that $\forall \aexec.\l \A(\aexec) \subseteq \AR_{\aexec}$.

For example, strict serialisability is specified by the axioms $\A_{1}(\aexec) = \AR_{\aexec}$ 
and $\A_{2}(\aexec) = \SO_{\aexec}$: 
%which require an abstract execution $\aexec$ to satisfy 
serialisable abstract executions $\aexec$ need satisfy
the inequalities $\AR_{\aexec} \subseteq \VIS_{\aexec}$ and $\PO_{\aexec} \subseteq \VIS_{\aexec}$.
All the consistency models presented in this paper have an axiomatic 
specification based on abstract executions \cite{principle-eventual-consistency,surech-session-guarantee,framework-concur,laws}. 

\mypar{Equivalence of Read Atomic and $\CMs(\ET_{\top})$} 
In the rest of this section, we develop the methodologies to show that 
axiomatic specifications using abstract executions are equivalent to those based on execution tests.
We start by showing that the consistency model induced by $\ET_{\top}$, the largest execution test, 
corresponds to \emph{Read Atomic}.
%$\ET_{\top} \vdash (\hh, \vi) \csat \opset, (\mkvs',  \vi')$ for any $\hh, \hh', \vi, \vi', \opset$. 
%We show that there is a close connection between the set $\ET_{\top}$-traces and 
%$\aeset$. 
\begin{theorem}
\label{thm:kvtrace2aexec}
%Given an $\ET_{\top}$-trace $\tau = (\mkvs_{0}, \vienv_{0}) \toET{(\cl_{1}, \alpha_{1})} \cdots \toET{(\cl_{n}, \vienv_{n})} (\mkvs_{n}, \vienv_{n})$, 
%there exists a non-empty set of abstract executions $\execs(\tau) = \{\aexec_i\}_{i = 0}^{n}$ such that, for any $i = 0,\cdots,n$, 
%$\hh_{\aexec_{i}} = \hh_{i}$, and the order in which transactions are executed in $\tau$ is consistent with $\AR_{\aexec_{n}}$. 
Given an $\ET_{\top}$-trace $\tau = (\mkvs_{0}, \vienv_{0}) \toET{(\cl_{0}, \alpha_{0})} \cdots \toET{(\cl_{n-1}, \vienv_{n-1})} (\mkvs_{n}, \vienv_{n})$, 
there exists a non-empty set of abstract executions $\execs(\tau)$ such that, for any $\aexec \in \execs(\tau) = \hh_{\aexec} = \mkvs_{n}$, 
and the order in which transactions are executed in $\tau$ is consistent with $\AR_{\aexec_{n}}$. 
\end{theorem}
\noindent The proof of Theorem \ref{thm:kvtrace2aexec} is highly non-trivial. Here we omit formal details of the proof, 
which can be found in \cref{sec:kvtrace2aexec}, and focus on the construction of the set $\execs(\tau)$ instead.

Let $\tau$ be as defined in \cref{thm:kvtrace2aexec}. The construction of the set $\execs(\tau)$ is incremental. 
We start with the singleton set $\execs_{0}(\tau)$ containing an abstract execution where the initial transactions $\txid_{0}$ writes 
the initial value for each $\key$ in the kv-store: 
%$\execs_{0}(\tau) = \{(\emptyset, \emptyset, \emptyset)\}$. 
$\execs_{0}(\tau) = \{([\txid_{0} \mapsto \{ (\otW, \key, \val_{0} \mid \key \in \Keys)\}], \emptyset, \emptyset)\}$; 
then for $i=0,\cdots, n-1$, we define $\execs_{i+1}(\tau)$ from $\execs_{i}(\tau)$ and the $\ET_{\top}$-reduction 
$(\mkvs_{i}, \vienv_{i} \toET{(\cl_{i}, \alpha_{i})} \mkvs_{i+1}, \tau_{i+1})$. Finally, we define $\execs(\tau) \defeq \execs_{n}(\tau)$.
\ac{Here there is an inconsistency: if we want $\txid_{0}$ to be included in abstract executions, then initially we 
must set $\execs(\tau) = ([\txid_{0} \mapsto \{ (\otW, \key, \val_{0} \mid \key \in \Keys)\}], \emptyset, \emptyset$.} 

Let us show how $\execs_{i+1}(\tau)$ is defined, for $i=0,\cdots,n-1$: 
%inductively from $\execs_{i-1}(\tau)$. 
If the $i$-th action of $\tau$ is a view-shift, i.e. $\alpha_{i} = \varepsilon$, then 
we let $\execs_{i+1}(\tau) = \execs_ {i}(\tau)$. Otherwise, the $i$-th action of $\tau$ consists of client $\cl_{i}$ committing a 
transaction $\txid_{i}$ with fingerprint $\fp_{i}$, and we follow a two step process for constructing $\aexec_{i+1}(\tau)$: first, we extend each of the 
executions in $\aexec_{i+1}(\tau)$ by inserting the mapping $(\txid_{i} \mapsto \fp_{i})$ as the last element of its arbitration order; this operation 
results in an intermediate set of abstract executions, $\execs_{i}^{\mathsf{partial}}(\tau)$. Second, for each 
abstract execution in $\aexec_{\mathsf{partial}} \in \execs_{i}^{\mathsf{partial}}(\tau)$ we determine a set of sets of visible transactions for $\txid_{i}$:  
each of these sets is used to extend the visibility relation of $\aexec_{\mathsf{partial}}$, and the resulting abstract execution  is included in  the final 
set $\aexec_{i+1}(\tau)$. Each visibility set for $\txid_{i}$ must contain at least the writers of the versions contained in $\vienv_{i-1}(\cl_{i})$, 
$\Tx[\hh_{i}, \vienv_{i}]$, because we know that read operations in $\fp_{i}$ fetch their value of a key $\key$ from the 
last version of $\mkvs_{i}$ included in $\vi_{i}$; but it may also contain any other read-only transaction 
(excluded $\txid_{i}$) that is already present in the abstract execution $\aexec_{\mathsf{partial}}$.

%%Formally, consider the $i$-th $\ET_{\top}$-reduction of $\tau$, $(\hh_{i-1}, \vienv_{i-1}) \xrightarrow{(\cl_{i}, \alpha_{i})}_{\ET_{\top}} (\hh_i, \vienv_{i})$, 
%%and assume that $\alpha_{i} = \opset$ for some $\opset$. Let 
%%If $\alpha_{i} = \varepsilon$, then we let $\execs_{i}(\tau) = \execs_{i-1}(\tau)$. Otherwise, 
%%if $\alpha_{i} = \opset$, let 
%$\{\txid_{i}\} := \hh_{i} \setminus \hh_{i-1}$; 
%%we first identify the transaction identifier that is associated to $\opset$ in $\tau$: this 
%%is the only transaction identifier $\txid_{i}$ that appears in $\hh_{i}$ and not in $\hh_{i-1}$; 
%we extend the arbitration order of each abstract execution $\aexec \in \execs_{i}(\tau)$ by appending the transaction identifier 
%$\txid_{i}$ as its last  transaction, and we let $\execs_{i}^{\mathsf{partial}}(\tau)$ be the resulting set of 
%abstract executions. Then, for each abstract execution in this set, we determine a set of suitable 
%local visibility relations for the transaction $\txid_{i}$, i.e. the possible sets of transactions that 
%$\txid_{i}$ sees.  Each of these sets must contain at least the writers of the versions contained $\vienv_{i-1}(\cl_{i})$ in the 
%view of client $\cl_{i}$ in $\vienv_{i-1}$, defined by $\Tx[\hh_{i-1}, \vienv{i-1}]$; but it may also contain any other read-only transaction 
%that is already present in the abstract execution. 
%%Formally, given a kv-store $\hh$ and a view $\vi$, we let 
%%$\Tx[\hh, \vi] \defeq \Setcon{ \WTx(\hh(\key, i)) }{ \key \in \Keys \wedge i \in \vi(\key) }$. 
%That is, for each 
%abstract execution $\aexec \in \execs_{i}^{\mathsf{partial}}$ and each subset $\T_{\mathsf{rd}}$ of 
%read-only transactions in $\aexec$, we extend the visibility relation $\VIS_{\aexec}$ 
%with the pairs $\Set{(\txid, \txid_{i}) }[  \txid \in \Tx[\hh_{i-1}, \vienv{i-1}(\cl_{i})] \cup \txidset_\rd]$, 
%and we include the resulting abstract execution in $\aexec \in \execs_{i}(\tau)$. Finally, we 
%let $\execs(\tau) = \execs_{n}(\tau)$.

%We let $\execs_{i}^{\mathsf{partial}(\tau)} = \{(\TtoOp{T}_{\aexec}[\txid_{i} \mapsto \opset], \VIS_{\aexec}, 
%\AR_{\aexec} \cup \{(\txid, \txid_{i} \mid \txid \in \dom\TtoOp{T}(\aexec)\} \mid \aexec \in \execs_{i}(\tau)\}$. 
%Then, for each $\aexec \in \execs_{i}^{\mathsf{partial}(\tau)}$, we determine the set of transactions that  
In \cref{sec:aexectrace2kv} we show that the reverse construction is possible.
\begin{theorem}
\label{thm:aexec2kvtrace}
Given an abstract execution $\aexec$, there exists a non-empty 
set of $\ET_{\top}$-traces $\{\tau_{i}\}_{i \in I}$ such that, for each $i \in I$, the last configuration of $\tau_{i}$ is 
$(\hh_{\aexec}, \_)$, and $\tau_{i}$ executes transactions in the order established by $\AR_{\aexec}$. 
\end{theorem}
\noindent Theorems \ref{thm:kvtrace2aexec} and \ref{thm:aexec2kvtrace} establish that $\CMs(\ET_{\top})$ is 
sound and complete with respect to the set of abstract executions allowed by the Read Atomic Consistency Model, 
respectively. 

\mypar{Proving the equivalence of execution tests and axiomatic specifications}
The constructions used to convert an abstract execution 
into a  set of $\ET_{\top}$-traces, and vice-versa, are universal:  they can be applied to show the 
equivalence of  the consistency models induced by an arbitrary execution $\CMs(\ET)$ 
and an axiomatic specification over abstract executions. 

We present a proof technique that exploits the construction outlined 
above to establish the soundness of execution-test-based specifications with respect to axiomatic 
ones. Completeness is handled with a similar proof technique, which we illustrate in 
\cref{sec:aexectrace2kv}.
%In \cref{sec:aexectrace2kv} we propose a similar proof technique aimed at 
%for proving the completeness of execution-tests-based specifications with respect to axiomatic ones. 
We have applied the proof techniques to prove the equivalence of execution-test-based specifications from \cref{fig:execution_tests} 
and their respective axiomatic specifications (\cref{app:et_sound_complete}).
 %while Theorem \ref{thm:aexec2kvtrace} proves that $\CMs(\ET_{\top})$ is also 

Suppose that we want to prove that the an execution test $\ET$ induces a consistency model that is sound with respect 
to an axiomatic specification $\Ax$. Given an arbitrary kv-store $\hh \in \CMs(\ET)$, 
we need to find an abstract execution $\aexec$ that satisfies $\Ax$, such that $\hh_{\aexec} = \hh$. 
By definition, if the kv-store $\hh$ is included in $\CMs(\ET)$ then 
we can find an $\ET$-trace $\tau$ (and therefore a $\ET_{\top}$-trace, that terminates in a configuration $(\hh, \stub)$. 
By Theorem \ref{thm:kvtrace2aexec}, it suffices to show that $\execs(\tau)$
%Because 
%any $\ET$-trace is also a $\ET_{\top}$-trace, by \cref{thm:kvtrace2aexec} 
%for any $\aexec \in \execs(\tau)$, $\hh_{\aexec} = \hh$. Therefore, it suffices to prove 
%$\execs(\tau)$ 
contains at least one abstract execution that satisfies $\Ax$, to conclude 
that $\hh \in \CMs(\Ax)$. 

Our proof technique can be summarised as follows: let $\aexec \in \aeset$ satisfy
the axiom $\Ax$, and consider the associated kv-store $\hh_{\aexec}$. Consider 
also a tuple of the form $\ET \vdash (\mkvs_{\aexec}, \vi) \csat \opset: (\mkvs', \vi')$, 
where $\vi, \opset, \mkvs', \vi'$ are arbitrary. We construct an abstract execution 
$\aexec'$ following the same construction adopted for \cref{thm:aexec2kvtrace}:  
we first extend $\aexec$ by appending the unique transaction in $\mkvs' \setminus \mkvs$ 
in its arbitration order, 
%let $\{\txid\} = \mkvs' \setminus \mkvs$, and we extend $\aexec$ by appending $\txid$ 
%as the last transaction in the arbitration order, 
then we select a suitable visibility relation for $\txid$ consisting of $\Tx(\hh_{\aexec}, \vi)$ and 
a set $\T_{\mathsf{read}}$ of read-only transactions; 
%of the form $\Tx(\hh_{\aexec}, \vi) \cup \T_{\mathsf{rd}}$, where $\T_{\mathsf{read}}$ is a 
%set of read-only transactions. Note that, at this step, only the set of read-only transactions 
%must be specified; 
this construction ensures that $\mkvs_{\aexec'} = \mkvs'$. 
Finally, we prove that the resulting abstract execution $\txid'$ satisfies $\Ax$. 

To see why our proof technique guarantees the soundness of $\CMs(\ET)$ with respect 
to $\Ax$, we apply an inductive argument over the sets $\execs_{i}(\tau)$ of an $\ET$-trace $\tau = 
(\mkvs_{0}, \vienv_{0})  \toET{(\cl_{0}, \alpha_{0})} \cdots  \toET{(\cl_{n-1}, \alpha_{n-1})} (\mkvs_{n}, \vienv_{n})$: 
first, we note that the unique abstract execution $\aexec_{0} \in \execs_{0}(\tau)$ trivially satisfies the axioms
$\Ax$: $\Ax(\aexec_{0}) \subseteq \AR_{\aexec_{0}} = \emptyset \subseteq \VIS_{\aexec_{0}}$. 
Then, suppose that there exists an abstract execution $\aexec_{i} \in \execs_{i}(\tau)$ that satisfies 
the axioms $\Ax$. By the soundness of our proof technique, we can construct an abstract execution 
$\aexec_{i+1} \in \execs_{i+1}(\tau)$ that satisfies $\Ax$, and $\mkvs_{\aexec_{i+1}} = \mkvs_{i+1}$. 
By iterating this procedure, we obtain an abstract execution $\aexec_{n}$  that satisfies $\Ax$, and $\hh_{\aexec_{n}} = \hh_{n} = \hh$, 
which is exactly what we wanted to prove.
%
%whenever $\aexec$ satisfies the axioms in
%$\Ax$ and $\ET \vdash (\hh_{\aexec}, \vi) \csat \opset: (\mkvs', \vi')$, 
%then we can always find a set of read-only transactions $\T_{\mathsf{rd}}$ such that 
%by extending $\aexec$ with some transaction $\txid$, appearing last in the arbitration order 
%and whose set of visible transactions is $\Tx(\hh_{\aexec}, \vi) \cup \T_{\mathsf{rd}}$, then 
%the resulting execution also satisfies the axioms in $\Ax$. Assume that we have successfully employed the 
%proof technique for $\ET$, and consider the $\ET$-trace $\tau$ above, which we assume it has 
%$n$ $\ET$-reductions. 
%It is immediate to note that the abstract execution $(\emptyset, \emptyset, \emptyset)$ 
%included in $\execs_{0}(\tau)$ satisfies $\Ax$; furthermore, because we have successfully applied our proof technique to $\ET$, 
%we are ensured that if $\execs_{i}(\tau)$ contains at least one abstract execution $\aexec_{i}$ that satisfies the axioms in $\Ax$, where $i=0,\cdots n-1$. 
%By induction, we conclude that $\execs(\tau)$ contains at least one abstract execution satisfying $\Ax$, and 
%therefore $\hh_{\aexec} \in \CMs(\aexec)$. 

In practice, our proof techniques allows defining an invariant 
%over the set $\Tx(\stub, \stub)$, which 
on the visibility relation of abstract executions, which must be proved to be preserved by $\ET$-reductions. %As we explain in \cref{app:et_sound_complete}, 
Defining the right invariant 
is crucial to prove the soundness of some execution-test-based specifications (see \cref{app:et_sound_complete}).

\sx{Above is about Galois connection between kv-stores under \( \ET_{\top} \) and abstract executions with last write win resolution policy.}


%A semantics under \( \aexec \), where transaction satisfies the specification in the form of \( (\RP,\Ax) \).
%\( \RP \) is a function takes a set of transactions \( \T \) from abstract execution \( \aexec \), 
%and returns  a set of possible snapshots.
%A common one is last-write-win.
%
%\begin{definition}
%\label{def:lww}
%The Last Write Wins resolution policy $\RP_{\LWW}$ is defined as 
%$\RP_{\LWW}(\aexec, \T) \defeq \{\h\}$ where
%\begin{align*}
%    \h & \defeq \lambda \key. \text{let} \ \T_{\key} = ( \T \cap \{\txid \mid (\otW,\key, \stub) \in_{\aexec} \txid\}) \\
%       & \quad \text{in} \
%\begin{cases}
%\val_{0} &\impliedby \T_{\key} =  \emptyset\\
%\val &\impliedby (\otW, \key, \val) \in_{\aexec} \max_{\AR_{\aexec}}(\T_{\key})
%\end{cases}
%\end{align*}
%\end{definition}
%
%\( \Ax \) is a set of constraints on the shape of the abstract execution,
%which defines the minimum visibility relation.
%For example, the minimum visibility for serialisability is a singleton set
%\( \Ax_\SER = \Set{\lambda \aexec \ldotp \AR_\aexec } \),
%which means the minimum visibility should contain the arbitration \( \AR_\aexec \subseteq \VIS_\aexec \).
%In general, if an abstract execution \( \aexec \) satisfies a set of constraints  \( \Ax \),
%if \( \fora{\A \in \Ax} \A(\aexec) \subseteq \VIS_\aexec \).
%
%Through the dependency graph, we can relate key-value stores and abstract executions (\cref{def:main-body-aexec2graph}).
%Note that each abstract execution $\aexec$ determines a key-value store 
%$\hh_{\aexec}$, as per \cref{def:main-body-aexec2graph} and \cref{thm:main-body-kv2graph}. 
%
%\begin{definition}
%\label{def:main-body-aexec2graph}
%Given an abstract execution $\aexec$ that satisfies the last write wins policy,
%the dependency graph $\graphof(\aexec) \defeq (\TtoOp{T}_{\aexec}, \RF_{\aexec}, 
%\VO_{\aexec}, \AD_{\aexec})$ is defined by letting
%\begin{itemize}
%\item $\txid \xrightarrow{\RF_{\aexec}(\key)} \txid'$ if and only if 
%    $\txid = \max_{\AR_{\aexec}}\left( \Setcon{\txid}{(\otW,\key,\stub) \in \txid' \land \txid' \toEdge{\VIS_\aexec} \txid } \right)$, 
%\item $\txid \xrightarrow{\VO_{\aexec}(\key)} \txid'$ if and only 
%$\txid, \txid' \in_{\aexec} (\otW, \;\ke: \stub)$ 
%and $\txid \xrightarrow{\AR_{\aexec}} \txid'$,
%\item $\txid \xrightarrow{\AD_{\aexec}(\key)} \txid'$ if and only if either 
%$(\otR, \key, \stub) \in_{\aexec} \txid, (\otW, \key, \stub) \in_{\aexec} \txid'$ and 
%whenever $\txid'' \xrightarrow{\RF_{\aexec}(\key)} \txid$, 
%then $\txid'' \xrightarrow{\VO_{\aexec}(\key)} \txid'$.
%\end{itemize}
%\end{definition}
%
%Let $\hh$ be the unique kv-store such that $\Gr_{\hh} = \graphof(\aexec)$: then 
%$\hh_{\aexec} = \hh$. 
%Yet this mapping $\hh_{(\stub)}$ is not a bijection, 
%in that several abstract executions may be encoded in the same key-value store. 
%This is because key-value stores abstract away the total arbitration order of transactions.
%
%\sx{
%    explain \( \Tx(\mkvs,\vi) \) and intuition of invariant \( I \):
%    \( \Tx(\mkvs,\vi) \) includes all the writers of the versions \( \mkvs(\key, i)\), 
%    where these versions are included in the view \( i \in \vi(\key)\).
%    In abstract executions, there is  no explicit \( \SO \) relation 
%    but \( \VIS \) and \( \AR \) relations.
%    The invariant \( I(\aexec, \cl) \) is for carrying information 
%    for transactions from the client \( \cl \).
%}


\begin{definition}
\label{def:main-body-et_sound}
An execution test $\ET$ is sound with respect to an axiomatic 
specification $\Ax$ if and only if
there exists an invariant condition $I$ such that, 
for any $\cl, \txid, \vi, \vi', \mkvs, \vi', \fp, \aexec$, if 
%if assuming that
\begin{itemize}
    \item $\ET \vdash (\mkvs, \vi) \csat \fp: (\mkvs',\vi')$, where \( \mkvs' = \updateKV[\mkvs, \vi ,\fp, \txid]\)
%    a client \( \cl \) having an initial view \( \vi \), 
%        commits a transaction \( \txid \) with a fingerprint \( \fp \) and updates the view to \( \vi' \), 
%        which is allowed by \( \ET \) \ie $\ET \vdash (\mkvs, \vi) \csat \fp: (\mkvs',\vi')$ where \( \mkvs' = \updateKV[\mkvs, \vi ,\fp, \txid]\),
    \item  $\mkvs_{\aexec} = \mkvs$ and $I(\aexec, \cl) \subseteq \Tx[\mkvs, \vi]$,
\end{itemize}
then there exist a set of read-only transactions $\txidset_{\rd}$ such that, if $\aexec'$ is obtained by extending $
\aexec$ with $\txid$ appearing last in the arbitration order, and choosing $\VIS^{-1}(\txid) = \Tx[\mkvs, \vi] \cup \txidset_{\rd}$, then 
\begin{itemize}
%    \item the new abstract execution \( \aexec'  = \extend[\aexec,\txid,\Tx[\mkvs,\vi] \cup \txidset_\rd, \fp]\),
%    \item the view \( \vi \) satisfies \( \Ax \), \ie $\fora{ \A \in \Ax } \Set{\txid' }[ (\txid', \txid) \in \A(\aexec')] \subseteq \Tx[\mkvs, \vi] \cup \txidset_{\rd}$, 
%    \item the invariant is preserved, \ie $I(\aexec', \cl) \subseteq \Tx[\mkvs', \vi']$.
%   \\( \aexec'  = \extend[\aexec,\txid,\Tx[\mkvs,\vi] \cup \txidset_\rd, \fp]\),
    \item $\fora{ \A \in \Ax } \Set{\txid' }[ (\txid', \txid) \in \A(\aexec')] \subseteq \Tx[\mkvs, \vi] \cup \txidset_{\rd}$, 
    \item $I(\aexec', \cl) \subseteq \Tx[\mkvs', \vi']$.
\end{itemize}
\end{definition}

\sx{
    No need to formally say \( (\RP_{\LWW}, \Ax) \). We can just say:
    An execution test $\ET$ is sound with respect to an axiomatic 
    definition $\Ax$ under last write win if and only if ...
}

\begin{theorem}
\label{thm:main-body-et_soundness}
If $\ET$ is sound with respect to $(\RP_{\LWW}, \Ax)$, then 
\[
    \CMs(\ET) \subseteq \{ \hh \mid \exists \aexec \in \CMa(\RP_{\LWW}, \Ax)).\;\hh_{\aexec} = \hh\}
\]
\end{theorem}

%Similarly, in \cref{sec:aexectrace2kv} we show that it is possible to 
%construct a set of $\ET$-traces $\{\tau_{i}\}_{i \in I}$ from an abstract execution $\aexec$, such that 
%for any $i \in I$, the last configuration of $\tau_{i}$ has the form $(\hh_{\aexec}, \stub)$, and 
%the order in which transactions are executed in $\tau_{i}$ is consistent with $\AR_{\aexec}$. 
%This gives rise to a proof technique aimed at showing that execution-test based specifications 
%of consistency models are sound and complete with respect with their axiomatic counterparts on abstract executions. 
%More details are outlined in \cref{sec:kv2aexec-sound-complete}. 
%In \cref{app:et_sound_complete} we employ our proof techniques to prove that our specifications (\cref{fig:execution.tests})
%of consistency models using execution tests are sound and complete with respect to previously 
%existing axiomatic specifications based on dependency graphs.
%
%\sx{
%    The sketch prove of this theorem might be too much details.
%}
%
%\begin{definition}
%\label{def:main-body-et_complete}
%An execution test $\ET$ is \emph{complete} with respect 
%to an axiomatic definition $(\RP_{\LWW}, \Ax)$ if, for any abstract execution $\aexec \in \CMa(\RP_{\LWW}, \Ax)$ 
%and index \( i : 1 \leq i < \abs{\txidset_{\aexec}}\) such that \( \txid_{i} \toEDGE{\AR_{\aexec}} \txid_{i+1} \), there exist an initial view $\vi_{i}$ and a final view $\vi_{i}'$ where 
%\begin{itemize}
%\item $\vi_{i} = \getView[\aexec, \VIS_{\aexec}^{-1}(\txid_{i})]$, 
%\item let $\txid_{i} = \txid_{\cl}^{n}$ for some $\cl, n$; 
%    \begin{itemize}
%        \item if the transaction $\txid_{i}' = \min_{\SO_{\aexec}}\Set{\txid' }[ \txid_i \toEDGE{\SO_{\aexec}} \txid']$ is defined, then $\vi' = \getView[\aexec, \txidset_{i}]$ where $\txidset_{i} \subseteq (\AR_{\aexec}^{-1})\rflx(\txid_{i}) \cap \VIS_{\aexec}^{-1}(\txid_{i}'))$; 
%        \item otherwise $\vi' = \getView[\aexec, \txidset_{i}]$ where $\txidset_{i} \subseteq (\AR_{\aexec}^{-1})\rflx(\txid_{i})$, 
%    \end{itemize}
%\item $\ET \vdash (\mkvs_{\cut[\aexec, i-1]}, \vi_{i}) \csat \TtoOp{T}_{\aexec}(\txid_{i}) : (\mkvs_{\cut[\aexec, i]},\vi_{i}')$.
%\end{itemize}
%\end{definition}
%
%\begin{theorem}
%\label{thm:main-body-et_complete}
%Let $\ET$ be an execution test that is complete with respect to 
%an axiomatic specification $(\RP_{\LWW}, \Ax)$. Then 
%$\CMa(\RP_{\LWW}, \Ax) \subseteq \CMs(\ET)$.
%\end{theorem}
%
%\sx{
%    The sketch prove of this theorem might be too much details.
%}
%
%
%\sx{Here we could discuss of applying \cref{def:main-body-et_sound,def:main-body-et_complete} on \( \ET_\PSI \) ( or \( \SI \)) and \( \Ax_\PSI \)  }
%
%





%\sx{
%    START
%}
%
%To specify consistency models, abstract executions constraint resolution policy \( \RP \) and the sharp of abstract execution \( \Ax \) (I believe we could even say more precise --- the minimum visibility relation, and cite andrea's concur paper).
%Resolution policy determines a set of snapshot given a set of observable transactions.
%Most time it is last-write-win which gives a unique single snapshot.
%This corresponds to the \( \snapshot{} \) function in the semantics kv-store.
%The resolution policy part we could (question?) choose to not mentioned, since it is minor and many people always assume last-write-win.
%
%The \( \Ax = \Set{ \A : \aeset \to \pset{\TxID} \times \pset{\TxID}} \) is a set of constraints on visibility relations, \eg \( \Ax = \Set{\lambda \aexec \ldotp \SO_\aexec,  \lambda \aexec \ldotp \VIS_\aexec;\VIS_\aexec} \) means all the abstract executions must satisfy \( \SO \subseteq \VIS , \VIS ; \VIS \subseteq \VIS \).
%
%\begin{definition}
%\label{def:aexec2graph}
%Given an abstract execution $\aexec$ that satisfies the last write wins policy,
%the dependency graph $\graphof(\aexec) \defeq (\TtoOp{T}_{\aexec}, \RF_{\aexec}, 
%\VO_{\aexec}, \AD_{\aexec})$ is defined by letting
%\begin{itemize}
%\item $\txid \xrightarrow{\RF_{\aexec}(\key)} \txid'$ if and only if 
%$\txid = \max_{\AR_{\aexec}}(\visibleWrites_{\aexec}(\key, \txid'))$, 
%\item $\txid \xrightarrow{\VO_{\aexec}(\key)} \txid'$ if and only 
%$\txid, \txid' \in_{\aexec} (\otW, \;\ke: \stub)$ 
%and $\txid \xrightarrow{\AR_{\aexec}} \txid'$,
%\item $\txid \xrightarrow{\AD_{\aexec}(\key)} \txid'$ if and only if either 
%$(\otR, \key, \stub) \in_{\aexec} \txid, (\otW, \key, \stub) \in_{\aexec} \txid'$ and 
%whenever $\txid'' \xrightarrow{\RF_{\aexec}(\key)} \txid$, 
%then $\txid'' \xrightarrow{\VO_{\aexec}(\key)} \txid'$.
%\end{itemize}
%\end{definition}
%Note that each abstract execution $\aexec$ determines a key-value store $\hh_{\aexec}$,
%as a result of \cref{def:aexec2graph} and \cref{thm:kv2graph}. 
%Let $\hh$ be the unique kv-store such that $\Gr_{\hh} = \graphof(\aexec)$, then $\hh_{\aexec} = \hh$. 
%
%A kv-store \( \mkvs \) is compatible with an abstract execution \( \aexec \)
%There is a mapping from 
%\begin{definition}
%\label{def:compatible-main-body}
%Given a key-value store $\hh$,
%an abstract execution $\aexec$ is compatible with $\hh$, written 
%$\aexec \compatible \hh$, if and only if there exists a  mapping 
%$f: \pset{\T_{\aexec}} \rightarrow \Views(\hh)$
%such that  
%\begin{itemize}
%\item for any subset $\T \subseteq \T_{\aexec}$, then $\RP_{\LWW}(\aexec, \T) = \{\snapshot(\hh, f(\T))\}$; 
%\item for any view $\vi \in \Views(\hh)$, there exists a subset $\T \subseteq \T_{\aexec}$ 
%such that $f(\T) = \vi$, and $\RP_{\LWW}(\aexec, \T) = \{\snapshot(\hh_{\aexec}, \vi)\}$.
%\end{itemize}
%\end{definition}
%The function $\getView(\aexec, \T)$ defines the view on \( \mkvs_\aexec \) that corresponds to \( \T \) as the following:
%\[
%\getView(\aexec, \T) \defeq \lambda \key. \{0\} \cup \Setcon{ i }{\WTx(\hh_{\aexec}(\key, i)) \in \T}
%\]
%Inversely, the function \( \Tx(\hh, \vi) \) converts a view to a set of observable transactions:
%\[
%\Tx(\hh, \vi) \defeq \Setcon{ \WTx(\hh(\key, i)) }{ \key \in \Keys \wedge i \in \vi(\key) }
%\]
%Given \( \getView \), \( \Tx \), \cref{def:compatible}, 
%it follows \( \aexec \compatible \hh_{\aexec} \) shown in \cref{thm:aexec2kv.compatible}.
%
%Given a trace with the final kv-store \( \mkvs \), there an abstract execution \( \aexec \) that is compatible with \( \mkvs \).
%First, all transactions in an abstract execution are totally ordered by \( \AR \),
%therefore an abstract execution by itself is a trace.
%The reduction order of a trace of kv-store maps to the \( \AR \) in the abstract execution and vice versa.
%The challenge is to show that
%for each transaction \( \txid \) in the trace and the initial view \( \vi \) of the transaction, the visible transitions \( \Tx(\hh, \vi) \) contains at least those transactions by the constraints \( \Ax \), \ie
%Note that \( \Tx(\hh, \vi) \) do not contain any read-only transactions, 
%thus to be more precise, we mean \( \Tx(\hh, \vi) \) plus some potentially read-only transactions contains those by the constraints \( \Ax \), \ie
%\( \fora{\A\in \Ax, \txid'} (\txid', \txid) \in \A(\aexec) \implies \txid' \in \Tx(\hh, \vi) \lor \txid' \text{is read only} \).
%
%Given an abstract execution \( \aexec \) there is a trace of kv-store with the final kv-store \( \mkvs \) that is compatible.
%Similarly, the \( \AR \) decides the reduction order in the trace.
%The challenge here is that
%for each transaction \( \txid \) with fingerprint \( \f \) in the abstract execution,
%(We should mentioned a node in the graph is the fingerprint)
%We find two views, the initial \( \vi \) and the final \( \vi' \), so \( \ET \vdash (\mkvs, \vi) \csat \f : \vi' \) where the \( \mkvs \) is the kv-store when \( \txid \) commits.
%The initial view \( \vi \) includes all versions written by visible transactions of \( \txid \) from the abstract execution, \ie \( \vi = \getView(\aexec, \VIS_\aexec^{-1}(\txid)) \).
%Since \( \vi' \) gives the minimum views for the next transactions from the same clients,
%it can be computed by including all versions, if exists in \( \mkvs \), written by visible transactions of \( \txid' \) where \( \txid' \toEdge{\SO} \txid \).
%
%\sx{
%    END
%}
%
