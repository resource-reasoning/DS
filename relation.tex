\section{Correct Definitions of Consistency Models}
\label{sec:other_formalisms}

We demonstrate how our kv-stores and execution 
tests relate to existing declarative semantics for specifying  
consistency models, based on abstract executions \cite{framework-concur}. 
We prove our definitions of consistency models using execution tests
are \emph{equivalent} to the definitions on abstract executions.
We give an overview of our results here, and refer the reader to the technical report
for more details.

\emph{Abstract executions} \cite{ev_transactions,framework-concur} are a declarative formalism for defining consistency models. 
An abstract execution graph is a directed graph with its nodes representing transactions 
(with each node labelled with a transaction identifier and a set of read/write operations), 
and its edges representing certain relations between transactions. 
Each edge is labelled by either the \emph{visibility} ($\VIS$) or \emph{arbitration} ($\AR$) relation. 
Visibility is an irreflexive order on transactions such that $(\txid_1,\txid_2) {\in} \VIS$ denotes that the effects (updates) of $\txid_1$ are visible to $\txid_2$. 
Arbitration is a strict total order on transactions such that $(\txid_1,\txid_2) {\in} \AR$ denotes that the updates performed by $\txid_2$ are newer than those of $\txid_1$. 
Moreover, $\AR$ contains $\VIS$ ($\VIS {\subseteq} \AR$) and agrees with the session order.
Lastly, abstract executions observe the \emph{last-write-wins} policy: 
a transaction reading $\key$ always fetches the latest visible write on $\key$.
Consistency models are defined by visibility axioms \( \visaxioms\) 
that impose certain conditions on $\VIS$ and hence the shape of the graphs.
For example, the axioms for \( \CC \) is defined by \( \visaxioms[\CC] \Set{\VIS ; \VIS \subseteq \VIS, \SO \subseteq \VIS}\).
Let \(\CMs(\visaxioms) \) denote 
the set of abstract executions that satisfy \( \visaxioms \).

\SpaceAboveDef
\begin{theorem}
\label{thm:eq-cm-et-and-cm-aexec}
For all the consistency models $M$ in \cref{fig:execution.tests},
$\CMs(\et[M]) = \CMs(\visaxioms[M])$. 
\end{theorem}
\SpaceBelowDef

We show that there is a correspondence between kv-stores and abstract executions.
To do this, we first show that an individual \( \kvs \) can be converted to \( \aexec \) 
and vice versa.
\begin{sx}
KEEP THE OLD TEX
that is, we:
\begin{enumerate*}
	\item establish a bijection between kv-stores and dependency graphs (\cref{app:depgraphs}),
	\label{item:kv-dependency}
another well-known graph-based declarative formalism; and then
	\item following \cite{laws}, we show that we can always \emph{extract} an abstract execution from a dependency graph and therefore from a kv-store by \ref{item:kv-dependency}. 
\end{enumerate*}
\end{sx}
This correspondence is not one-to-one 
as multiple $\AR$ relations may produce equivalent kv-stores.
Moreover, this correspondence result is not enough to prove that 
our definitions using \( \ET \) are equivalent to those using \( \visaxioms \).
This is because \( \ET \) constrains each transition step while \( \visaxioms\) constrains the final graph shape.
We thus propose an operational semantics on abstract executions parametrised by \( \visaxioms \),
where each transition extends the graph with a 
new transaction node and its visibility edges, 
provided that the new graph satisfies \( \visaxioms \).
We prove the operational semantics by showing
that every abstract execution that satisfies \( \visaxioms \),
is reachable in the operational semantics under \( \visaxioms \).

We thus show that, given a program \( \prog \),
a trace of kv-stores under the execution test for a consistency model \( M \), that is \( \et[M] \),
is \emph{equivalent} (\ie \emph{sound} and \emph{complete})
to a trace of abstract executions under the visibility axiom \( \visaxioms[M] \).
Instead of directly working on traces,
we introduce soundness and completeness constructors
that lift certain conditions between \( \et \) and \( \visaxioms \) to the level of traces:
for every transaction \( \txid \) from a client \( \cl \),
the snapshot induced by the view of \( \cl \) when \( \cl \) commits \( \txid \) is the same as
the snapshot induced by the set of visible transactions of \( \txid \), \ie \( \VISInv(\txid) \).
These two constructors then lift the equivalent between individual steps in the operational semantics on
kv-store and abstract executions to the level of traces, and hence \cref{thm:eq-cm-et-and-cm-aexec}.
