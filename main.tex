%% For double-blind review submission, w/o CCS and ACM Reference (max submission space)
\documentclass[acmsmall]{acmart}\settopmatter{printfolios=true,printccs=false,printacmref=false}
%% For double-blind review submission, w/ CCS and ACM Reference
%\documentclass[acmsmall,review,anonymous]{acmart}\settopmatter{printfolios=true}
%% For single-blind review submission, w/o CCS and ACM Reference (max submission space)
%\documentclass[acmsmall,review]{acmart}\settopmatter{printfolios=true,printccs=false,printacmref=false}
%% For single-blind review submission, w/ CCS and ACM Reference
%\documentclass[acmsmall,review]{acmart}\settopmatter{printfolios=true}
%% For final camera-ready submission, w/ required CCS and ACM Reference
%\documentclass[acmsmall]{acmart}\settopmatter{}


%% Journal information
%% Supplied to authors by publisher for camera-ready submission;
%% use defaults for review submission.
\acmJournal{PACMPL}
\acmVolume{1}
\acmNumber{CONF} % CONF = POPL or ICFP or OOPSLA
\acmArticle{1}
\acmYear{2018}
\acmMonth{1}
\acmDOI{} % \acmDOI{10.1145/nnnnnnn.nnnnnnn}
\startPage{1}

%% Copyright information
%% Supplied to authors (based on authors' rights management selection;
%% see authors.acm.org) by publisher for camera-ready submission;
%% use 'none' for review submission.
\setcopyright{none}
%\setcopyright{acmcopyright}
%\setcopyright{acmlicensed}
%\setcopyright{rightsretained}
%\copyrightyear{2018}           %% If different from \acmYear

%% Bibliography style
\bibliographystyle{ACM-Reference-Format}
%% Citation style
%% Note: author/year citations are required for papers published as an
%% issue of PACMPL.
\citestyle{acmauthoryear}   %% For author/year citations

\usepackage{wrapfig}
\usepackage{enumerate}

\newif\ifNonACMMode
\NonACMModefalse

% for theorem, proof, etc.
\usepackage{amsthm}
\theoremstyle{definition}
\newtheorem{thm}{Theorem}[section]
\newtheorem{defn}[thm]{Definition}
\newtheorem{param}[thm]{Parameter}
\newtheorem{lem}[thm]{Lemma}
\newtheorem{prop}[thm]{Proposition}
\newtheorem*{cor}{Corollary}
\newtheoremstyle{case}{}{}{}{}{}{:}{ }{}
\theoremstyle{case}
\newtheorem{case}{Case}

% the inter command for operational semantices
\usepackage{proof}
\usepackage{color}

\usepackage{centernot}

\usepackage{amsmath,amssymb,stmaryrd}
\usepackage{dsfont}

% For the box assertion
\usepackage{varwidth}

\usepackage{hyperref}
\hypersetup{
    colorlinks,
    citecolor=black,
    filecolor=black,
    linkcolor=black,
    urlcolor=black
}
\usepackage[usenames,dvipsnames,svgnames,table]{xcolor}
\usepackage{enumitem}
\usepackage{translang}
\lstset{language=translang}
\usepackage[margin=2cm]{caption}
\usepackage{tikz}
\usetikzlibrary{positioning, shapes, decorations}
\usepackage{bold-extra}
\usepackage{titlesec}

\titleformat{\chapter}{\bfseries\Huge}{\thechapter.}{1ex}{\Huge}

\def\changemargin#1#2{\list{}{\rightmargin#2\leftmargin#1}\item[]}
\let\endchangemargin=\endlist
\usepackage{hhmacros}
\pgfdeclarelayer{main}
\pgfdeclarelayer{background}
\pgfdeclarelayer{foreground}
\pgfsetlayers{background,main,foreground}

\newcommand{\greyness}{gray!40}
\newcommand{\blueness}{cyan!60}

\tikzstyle{background}=[rectangle, draw=black, inner sep=0.2cm, rounded corners=1.2mm]
\tikzstyle{white}=[rectangle, fill=white, inner sep=0.5cm, rounded corners=5mm]

%\tikzstyle{background}=[circle, fill=\greyness,
%                                                inner sep=0.2cm,
%                                                rounded corners=5mm,
%                                                decorate,
%                                                decoration={random steps,
%                                                            segment length=3pt,
%                                                            amplitude=3pt}]

%

 \tikzstyle{hheapcell}=[rectangle, draw=black, inner sep=0.1cm, font=\small]

\tikzstyle{noise}=[circle, thick, minimum size=1.2cm, draw=yellow!85!black, fill=yellow!40, decorate, decoration={random steps, segment length=2pt, amplitude=2pt}]

%\pgfdeclarelayer{background}
%\pgfdeclarelayer{foreground}
%\pgfsetlayers{background,main,foreground}

\tikzstyle{abstract}=[draw, fill=white, text width=5em, text centered, minimum height=2.5em, rounded corners]
    
\tikzstyle{arr}=[draw, ->, thick, color=black]
\tikzstyle{dasharr}=[draw,->,thick,dashed,color=black]

\tikzset{
    version node/.style={
        rectangle,
        draw=black,
        align=center,
        minimum height=5mm,
        text depth=0.5ex,
        text height=2ex,
        inner xsep=0pt,
        outer sep=0pt, 
        font=\footnotesize
    },      
    version list/.style={
        matrix of nodes,
        row sep=-\pgflinewidth,
        column sep=-\pgflinewidth,
        nodes={
            version node
        }
        ,
        execute at empty cell={\node[draw=none]{};},
        text width=5mm,
        anchor=west
    }
}

\newcommand{\tikzvalue}[4]{
    \node[version node, fit=(#1) (#2), fill=white, inner sep=0pt] (#3) {#4}
}
\newcommand{\tikzkvspace}{1.5pt}
\newcommand{\tikzkeyspace}{-1.1}
\newenvironment{halfsubfig}{%
    \begin{subfigure}{0.45\textwidth}
}{%
    \end{subfigure}
}
\newenvironment{onethirdsubfig}{%
    \begin{subfigure}{0.3\textwidth}
}{%
    \end{subfigure}
}
\newenvironment{centertikz}{%
    \begin{center}%
    \begin{tikzpicture}[every node/.style={inner sep=0,outer sep=0},font=\footnotesize]%
}{%
    \end{tikzpicture}%
    \end{center}%
}
\newcommand{\tikzresize}{.8}

\PassOptionsToPackage{svgnames}{xcolor}
\definecolor{DarkGreen}{rgb}{0, 0.5, 0}

\usepackage{mathrsfs}  


%\newcommand*{\newextarrow}[3] {%
%\newcommand*{#1}[2][]{\ext@arrow #2{\arrowfill@#3}{##1}{##2}} }
%
\newextarrow{\xrightarrowtriangle}{{20}{20}{20}{20}}
   {{\relbar}{\relbar}{\rightarrowtriangle}}

%%%%%%%%%%%%%%%%%%%%%% edit mode
\newif\ifCommentEdits
\CommentEditstrue

\newcommand{\pg}[1]{%
\ifComments
\begin{center}
\fbox{%
\begin{minipage}{6.5in} \color{red}
{\bf PG:} {\rm #1}
\end{minipage}
}
\end{center}
\fi
}

\newcommand{\sx}[1]{%
\ifComments
\begin{center}
\fbox{%
\begin{minipage}{6.5in} \color{blue}
{\bf SX:} {\rm #1}
\end{minipage}
}
\end{center}
\fi
}

\definecolor{darkred}{rgb}{0.5, 0, 0}
\newcommand{\azalea}[1]{%
\ifComments
\begin{center}
\fbox{%
\begin{minipage}{6.5in} \color{darkred}
{\bf AR:} {\rm #1}
\end{minipage}
}
\end{center}
\fi
}

\newcommand{\ac}[1]{%
\ifComments
\begin{center}
\fbox{%
\begin{minipage}{6.5in} \color{green}
{\bf SX:} {\rm #1}
\end{minipage}
}
\end{center}
\fi
}

%%%%%%%%%%%%%%%%%%%%%% end edit mode


%% Journal information
%% Supplied to authors by publisher for camera-ready submission;
%% use defaults for review submission.
\acmJournal{PACMPL}
\acmVolume{1}
\acmNumber{CONF} % CONF = POPL or ICFP or OOPSLA
\acmArticle{1}
\acmYear{2018}
\acmMonth{1}
\acmDOI{} % \acmDOI{10.1145/nnnnnnn.nnnnnnn}
\startPage{1}

%% Copyright information
%% Supplied to authors (based on authors' rights management selection;
%% see authors.acm.org) by publisher for camera-ready submission;
%% use 'none' for review submission.
\setcopyright{none}
%\setcopyright{acmcopyright}
%\setcopyright{acmlicensed}
%\setcopyright{rightsretained}
%\copyrightyear{2018}           %% If different from \acmYear

%% Bibliography style
\bibliographystyle{ACM-Reference-Format}
%% Citation style
%% Note: author/year citations are required for papers published as an
%% issue of PACMPL.
\citestyle{acmauthoryear}   %% For author/year citations

\begin{document}



%% Title information
\title{
	Towards a Formal Theory for Clients of Distributed Key-value Stores
    %Operational Semantics and Logic for Weak Consistency in Transactional Systems%
    %: a Multi-version Based Operational Approach
    } 
                                        %% [Short Title] is optional;
                                        %% when present, will be used in
                                        %% header instead of Full Title.
%\titlenote{with title note}             %% \titlenote is optional;
%                                        %% can be repeated if necessary;
%                                        %% contents suppressed with 'anonymous'
%\subtitle{Subtitle}                     %% \subtitle is optional
%\subtitlenote{with subtitle note}       %% \subtitlenote is optional;
                                        %% can be repeated if necessary;
                                        %% contents suppressed with 'anonymous'


%% Author information
%% Contents and number of authors suppressed with 'anonymous'.
%% Each author should be introduced by \author, followed by
%% \authornote (optional), \orcid (optional), \affiliation, and
%% \email.
%% An author may have multiple affiliations and/or emails; repeat the
%% appropriate command.
%% Many elements are not rendered, but should be provided for metadata
%% extraction tools.

%% Author with single affiliation.
\author{Shale Xiong}
%\authornote{with author1 note}          %% \authornote is optional;
                                        %% can be repeated if necessary
%\orcid{nnnn-nnnn-nnnn-nnnn}             %% \orcid is optional
\affiliation{
  \position{Ph.D. Student}
  \department{Department of Computing}              %% \department is recommended
  \institution{Imperial College London}            %% \institution is required
  \streetaddress{Huxley Building}
  \city{London}
  \state{}
  \postcode{SW7 2AZ}
  \country{United Kingdom}                    %% \country is recommended
}
\email{shale.xiong14@imperial.ac.uk}          %% \email is recommended

%% Author with single affiliation.
\author{Andrea Cerone}
%\authornote{with author1 note}          %% \authornote is optional;
                                        %% can be repeated if necessary
%\orcid{nnnn-nnnn-nnnn-nnnn}             %% \orcid is optional
\affiliation{
  \position{Research Associate}
  \department{Department of Computing}              %% \department is recommended
  \institution{Imperial College London}            %% \institution is required
  \streetaddress{Huxley Building}
  \city{London}
  \state{}
  \postcode{SW7 2AZ}
  \country{United Kingdom}                    %% \country is recommended
}
\email{a.cerone@imperial.ac.uk}          %% \email is recommended

\author{Azalea Raad}
%\authornote{with author1 note}          %% \authornote is optional;
                                        %% can be repeated if necessary
%\orcid{nnnn-nnnn-nnnn-nnnn}             %% \orcid is optional
\affiliation{
  \position{Post-doctoral Researcher}
  \department{}              %% \department is recommended
  \institution{Max Planck Institute}            %% \institution is required
  \streetaddress{no idea}
  \city{Kaiserslautern}
  \state{-}
  \postcode{-}
  \country{Germany}                    %% \country is recommended
}
\email{a.raad@mpi-sws.org}          %% \email is recommended

%% Author with single affiliation.
\author{Philippa Gardner}
%\authornote{with author1 note}          %% \authornote is optional;
                                        %% can be repeated if necessary
%\orcid{nnnn-nnnn-nnnn-nnnn}             %% \orcid is optional
\affiliation{
  \position{Professor}
  \department{Department of Computing}              %% \department is recommended
  \institution{Imperial College London}            %% \institution is required
  \streetaddress{Huxley Building}
  \city{London}
  \state{-}
  \postcode{SW7 2AZ}
  \country{United Kingdom}                    %% \country is recommended
}
\email{p.gardner@imperial.ac.uk}          %% \email is recommended

\begin{abstract}
Modern NoSQL databases (e.g. key-value stores) achieve scalability and improve 
latency by weakening the guarantees of distributed transaction 
processing. While the problem of giving a formal specification to 
the consistency models used by such databases has been widely 
studied, formalising the semantics of clients interacting with 
such systems has been largely neglected$^\ast$. 
This paper aims to be a 
first step towards filling this gap. We present a framework 
for capturing the semantics of programs interacting with a 
weakly consistent key-value store whose transactions enjoy atomic visibility. 
Our semantics enjoys 
atomic transaction processing, interleaving concurrency, 
and parameterisation with respect to the consistency model.  
%The latter is captured using the notion of execution tests, which 
%determine when a transaction is allowed  to execute. 
As a main contribution, we prove that our semantics is adequate: 
for each program and consistency model, the semantics captures 
precisely the behaviour that the program exhibits under said consistency model; 
and that specifications of consistency models in our framework 
coincide with previously proposed, axiomatic ones.
%via execution 
%tests are equivalent to the axiomatic specifications which have already 
%been proved to be correct.
As another contribution, we develop a variant of the \emph{Concurrent 
Abstract Predicates} separation logic
%development 
%of a separation logic for clients of key-value stores. We propose 
%a variant of the \emph{Concurrent Abstract Predicates} 
that is tailored to clients of weakly-consistent key-value stores, 
and that is parametric in the specification of consistency models.
%and a multi-version 
%representation of the key-value store to allow concurrent clients to 
%observe different states of the system.  
%We abstract from implementation details of the key-value store, which 
%is represented as a centralised, multi-version system. This allows 
%for concurrent clients to observe different version of the same key, 
%which makes it possible to capture non-serialisable behaviours of 
%programs while still retaining the atomic execution of transactions and 
%interleaving concurrency. 
%Furthermore, our semantics is parameterised by execution tests, which determine 
%when a transaction is allowed to execute.  By changing the execution 
%test of transactions, we capture different consistency models.

\textbf{$^\ast$ Suresh is going to be pissed off a lot by this sentence, but 
let's be honest, his semantics is light years behind ours.}
\ac{The abstract is wayyyy too long. But at least we have a guideline for the paper.}

%We present a a uniform semantics 
%Contents of this set of notes: 
%History heaps. Semantics of Programs 
%running under weak consistency models using history heaps as states. 
%Simulation technique for comparing weak consistency models defined using 
%history heaps. Verification of implementations.
%\textbf{Points following Dagstuhl: Viktor seemed positive about the 
%history heap work. His question was whether the framework is generic 
%enough to capture the protocols that they are developing with Azalea. 
%Alexey's opinion is that the framework may have some use if we 
%manage to prove implementations of protocols correct. 
%I would also like to have Azalea's opinion on a semantics based 
%on history heaps.}
\end{abstract}


%% 2012 ACM Computing Classification System (CSS) concepts
%% Generate at 'http://dl.acm.org/ccs/ccs.cfm'.
\begin{CCSXML}
<ccs2012>
<concept>
<concept_id>10011007.10011006.10011008</concept_id>
<concept_desc>Software and its engineering~General programming languages</concept_desc>
<concept_significance>500</concept_significance>
</concept>
<concept>
<concept_id>10003456.10003457.10003521.10003525</concept_id>
<concept_desc>Social and professional topics~History of programming languages</concept_desc>
<concept_significance>300</concept_significance>
</concept>
</ccs2012>
\end{CCSXML}

\ccsdesc[500]{Software and its engineering~General programming languages}
\ccsdesc[300]{Social and professional topics~History of programming languages}
%% End of generated code


%% Keywords
%% comma separated list
\keywords{keyword1, keyword2, keyword3}  %% \keywords are mandatory in final camera-ready submission


%% \maketitle
%% Note: \maketitle command must come after title commands, author
%% commands, abstract environment, Computing Classification System
%% environment and commands, and keywords command.

\maketitle

\azalea{I have imported the cleveref package! This means that all reference will be printed consistently and we DO NOT need custom names such as \textbackslash fig etc.
Every time you need to refer to something, please write\textbackslash cref\{label\}, \eg \cref{def:mkvs}, and the label (\eg Def.) will be printed correctly. 
These labels can be customised. I have introduced the necessary ones in the macros file. 
}
\newcommand{\RootPath}{.}
\section{Introduction}

In recent times, the area of formal reasoning for concurrent and heap-manipulating programs, has seen a noticeable development towards program logics that can tackle the specification and verification of low-level concurrency in systems. This capability, together with the ubiquity of multithreading in computer programs, allows the formulation of reasoning frameworks around a variety of applications.

Modern database systems make heavy use of concurrency in order to provide a level of performance able to support large scale operations. This leads to an obvious increase in throughput but can cause a lack of consistency in data, which is instead a fundamental requirement for the majority of programs. A number of techniques has been employed in commercial databases to solve the issue and try to give the best of both worlds. Among these, \textit{Two-Phase-Locking} is a blocking approach which resides in the part of the spectrum of solutions where correctness is preferred over performance and that works at the granularity of single database entries. As a consequence, once implemented as part of a complex database system, the algorithm is prone to subtle bugs which might cause the violation of its vital guarantees.

We therefore intend to provide a complete and flexible model of the \textit{Two-Phase-Locking} concurrency control mechanism, as inspired by its real-world use case.
The aim is to derive a specification together with a sound level of abstraction that allows us not to think in terms of the low-level details enforced by the technique.
This leads us to the exploration of formal reasoning about its client usage through a custom program logic which is proven to be sound. The logic framework enables users to prove partial correctness of their programs running in a \textit{Two-Phase-Locking} setting, by only having to reason atomically about blocks of code, without the complexity of concurrent interleavings.

\subsection{Contributions}

The main contributions of this project are listed below, with references to the relevant sections where they are further discussed.
\begin{itemize}
	\item \textbf{mCAP} (Section \ref{sec:mcapModel}, \ref{sec:mcapLogic}, \ref{sec:transLogic})\ \ We reformulate and extend a program logic for concurrent programs, namely CAP \cite{cap}, in order to remove some constraints which are hardcoded in the logic and enable a more flexible reasoning. In fact, we change the underlying model to parametrize both the representation of machine states and of action capabilities. On top of this, we provide a new and cleaner structure for the action model that does not explicitly use interference assertions. We also considerably modify the way environment interference is modelled through the rely/guarantee relations. This is done with the goal of allowing both a thread and the environment to perform multiple shared region updates in one step. It follows that the repartitioning operator also has a new and extended behaviour. At the level of programming language, we leave elementary atomic commands as a parameter to the user of the logic. Finally, we instantiate the mCAP framework into a logic for our particular needs of transactional reasoning.
	
	\item \textbf{\textsc{2pl} Model}\ \ 
	
	\item \textbf{Operational Semantics}\ \
	
	\item \textbf{Semantics Equivalence}
\end{itemize}
\subsection{semantics}
Let \( \repl \in \Repls \) denotes the set of totally ordered replicates.
Each replicate can have multiple clients, and 
each clients can commit a sequence of either read-only transitions or single-write transactions.
To model these, we annotate the transaction identifier with replicate \( \repl \), client \( \cl \), 
local time of the replicate \( n \) and read-only transactions count \( n' \), \ie \( \txidCOPS{\repl}{\cl}{n}{n'} \).
Note that the \( (n, \repl, n') \) can be treated as a single number that \( n \) are the higher bits, 
\( \repl \) the middle bits and \( n' \) the lower bits.
There is a total order among transitions from the same replica and from the same client.
We extend version with the set of all versions it dependencies on, \( \dep \in \pset{\Keys \times \TxID} \).
The function \( \depOf{\ver} \) denotes the dependencies set of the version.
For readability, we annotate view with either a replica, \( \viREPL \), or a client, \( \viCL \).
The view environment is extended with replicas and their views, \( \viewFunCOPS : (\Repls \times \ClientID ) \parfinfun \Views \).
We give the following semantics to capture the behaviours of the code.

\begin{lstlisting}[caption={put},label={lst:simplified-put}]
// mixing the client API and system API
put(repl,k,v,ctx) {

    // Dependency for previous reads and writes
    deps = ctx_to_dep(ctx);(*\label{line:put-ctx-to-deps}*)

    atomic{
        // increase local time.
        inc(repl.local_time);(*\label{line:put-inc-local}*) 

        // appending local kv with a new version.
        list_isnert(repl.kv[k],(v, (local_time + id), deps));(*\label{line:put-update-kv}*)
    }

    // update dependency for writes
    ctx.writers += (k,(local_time + id),deps);(*\label{line:put-update-ctx}*)

    // put in the queue to sync with other replicas
    enqueue(k,v,(current_ver+id),(deps ++ vers));
}
\end{lstlisting}

The client always fetches the version with the maximum writer it can observed for each key,
Which is computed by \( \funcn{getMax} \) function. 
It is different from \( \snapshot \) as \( \snapshot \) fetches the latest version with respect to the position in the list.

\[
    \begin{rclarray}
        \func{getMax}{\mkvsCOPS, \viCL} & \defeq &
        \lambda \ke \ldotp \left( \max_\txid\Setcon{(\val, \txid, \T, \dep)}{\exsts{i} (\val, \txid, \T, \dep) = \mkvsCOPS(\ke, i)} \right)\projection{1}
    \end{rclarray}
\]
\begin{mathpar}
    \inferrule[Put]{%
        ( \stk, \func{getMax}{\mkvsCOPS, \viCL}, \emptyset ), \pmutate{\ke}{\vx} \toL
        ( \stk', \stub, \Set{(\otW, \ke, \val )} ), \pskip
        \\\\
        \dep = \Setcon{(\ke', \txid)}{\exsts{i} i \in \viCL(\ke') \land \txid = \WTx(\mkvsCOPS(\ke', i))} \texttt{ ---> \cref{lst:simplified-put}, \cref{line:put-ctx-to-deps}} 
        \\\\
        \txid = \min\Setcon{%
        \txidCOPS{\repl}{\cl}{n'}{0}
        }{%
            \fora{\ke', i \in \viREPL(\ke'), n} \\
            \quad \txidCOPS{\stub}{\stub}{n}{\stub} = \WTx(\mkvsCOPS(\ke',i)) \\
            \qquad {} \implies n' > n 
        } \texttt{ ---> \cref{lst:simplified-put}, \cref{line:put-inc-local}}
        \\\\
        \mkvsCOPS' = \mkvsCOPS\rmto{\ke}{\mkvsCOPS(\ke) \lcat \List{(\ke, \txid, \emptyset, \dep)}} \texttt{ ---> \cref{lst:simplified-put}, \cref{line:put-update-kv}}
        \\\\
        \viREPL' = \viREPL\rmto{\ke}{\viREPL(\ke) \uplus \Set{\abs{\mkvsCOPS'(\ke)} - 1}} \texttt{ ---> \cref{lst:simplified-put}, \cref{line:put-update-kv}}
        \\\\
        \viCL' = \viCL\rmto{\ke}{\viREPL(\ke) \uplus \Set{\abs{\mkvsCOPS'(\ke)} - 1}} \texttt{ ---> \cref{lst:simplified-put}, \cref{line:put-update-ctx}}
    }{%
    \repl, \cl \vdash 
    \mkvsCOPS, \viREPL, \viCL, \stk, \ptrans{\pmutate{\ke}{\vx};} \toT{}
    \mkvsCOPS', \viREPL', \viCL', \stk', \pskip
    }
\end{mathpar}
The \verb|get_trans| fetches the latest versions from the replica via multiple atomic reads, one for each key.
As a result, a client has a list of candidates \verb|rst|.
Since interleaving might happen, versions might become out-of-date because the replicate receives new versions.
It is not a problem to read old versions as long as they satisfy causal consistency,
\ie if a client read a version \( \ver \), it should at least read all the versions that \( \ver \) depends on.
Thus the algorithm use \verb|ccv| to track the maximum versions the client should fetches,
and re-fetches the \verb|ccv[k]| version from the replica if it is greater than the candidate.

The following is a simplified algorithm by directly taking a list of versions \verb|ccv| satisfies causal consistency constraint,
and then read the versions indicated by \verb|ccv|.
The simplified algorithm is easier to understand.
\begin{lstlisting}[caption={get\_trans},label={lst:get-trans}]
// A simplified version by guessing
// a ccv satisfying dependency constraints
// and then read versions indicated by ccv.
// Note that it is a weaker version of the original code,
// as the original implementation fetches the latest versions
// for keys by a sequence of atomic get_by_version calls
List(Val) get_trans(ks,ctx) {
    take ccv: (*$\forall$*) k (*$\in$*) ks.(*\label{line:get-trans-ccv-1}*)
        (_,_,deps) := get_by_version(k,ccv[k]) (*${}\land \forall$*) dep (*$\in$*) deps.(*\label{line:get-trans-ccv-2}*)
            dep.key (*$\in$*) ks (*$\implies$*) ccv[dep.key] >= dep.ver (*\label{line:get-trans-ccv-3}*)

    for k in ks(*\label{line:get-trans-read-1}*)
        rst[k] = get_by_version(k,ccv[k]);(*\label{line:get-trans-read-2}*)

    // update the ctx
    for (k,ver,deps) in rst(*\label{line:get-trans-update-ctx-1}*)
        ctx.readers += (k,ver,deps);(*\label{line:get-trans-update-ctx-2}*)

    return to_vals(ks);
}                                   
\end{lstlisting}
\begin{mathpar}
    \inferrule[GetTrans]{%
        \viCL \viewleq \viCL' \viewleq \viREPL  \texttt{ ---> \cref{lst:get-trans}, \cref{line:get-trans-update-ctx-1,line:get-trans-update-ctx-2}}
        \\\\
        {\left(\begin{array}{@{}l@{}}
        \fora{i : 1 \leq i \leq j, \ke', m, \ver}  \\
        \quad \ver = \mkvsCOPS(\ke_i, \max(\viCL'(\ke_i)) \land {} \\
        \quad (\ke', \WTx(\mkvsCOPS(\ke', m))) \in \ver\projection{4} \\
        \qquad {} \implies m \in \viCL'(\ke')
        \end{array}\right)} \texttt{ ---> \cref{lst:get-trans}, \cref{line:get-trans-ccv-1,line:get-trans-ccv-2,line:get-trans-ccv-3}}
        \\\\
        \trans =  \pderef{\vx_1}{\ke_1}; \dots; \pderef{\vx_j}{\ke_j};
        \\\\
        ( \stk, \func{getMax}{\mkvsCOPS, \viCL'}, \emptyset ), \trans \toL
        ( \stk', \stub, \f ), \pskip \texttt{ ---> \cref{lst:get-trans}, \cref{line:get-trans-read-1,line:get-trans-read-2}}
        \\\\
        \txidCOPS{\repl}{\cl}{n'}{n} = \max\Setcon{\txidCOPS{\repl}{\cl}{z'}{z}}{\txidCOPS{\repl}{\cl}{z'}{z} \in \mkvsCOPS }
        \\
        \mkvsCOPS' = \updKV{\mkvsCOPS, \viCL', \txidCOPS{\repl}{\cl}{n'}{n+1}, \f} 
    }{%
        \repl, \cl \vdash 
        \mkvsCOPS, \viREPL, \viCL, \stk, \ptrans{\pderef{\vx_1}{\ke_1}; \dots; \pderef{\vx_j}{\ke_j}; } \toT{}
        \mkvsCOPS', \viREPL, \viCL', \stk', \pskip
    }
    \and
    \inferrule[ClientCommit]{%
        \repl, \cl \vdash 
        \mkvsCOPS, \viewFunCOPS(\repl), \viewFunCOPS(\cl), \stk, \prog(\cl) \toT{}
        \mkvsCOPS', \viREPL', \viCL', \stk', \cmd'
    }{%
        \mkvsCOPS, \viewFunCOPS, \thdenv, \prog \toG{}
        \mkvsCOPS', \viewFunCOPS\rmto{\repl}{\viREPL'}\rmto{\cl}{\viCL'}, \thdenv\rmto{\cl}{\stk'}, \prog\rmto{\cl}{\cmd'}
    }
\end{mathpar}
A replica updates its local state only if all the dependencies has been receive.
\begin{lstlisting}[caption={Send and receive},label={lst:send-receive}]
// Syn to other replicas
send() {
    (k,v,ver,deps) := dequeue();
    for id in repls {
        send (k,v,ver,deps) to id;
    }
}

// receive a write message from other replica
on_receive(k,v,ver,deps) {
    // for a single machine
    // the following check immediately holds
    for (k',ver') in deps {
        wait until dep_check(k',ver');(*\label{line:receive-wait}*)
    }

    atomic{
        list_isnert(kv[k],(v,ver,deps));(*\label{line:receive-update-view-1}*)
        (remote_local_time + id) = ver;(*\label{line:receive-update-view-2}*)
        local_time = max(remote_local_time, local_time);(*\label{line:receive-update-view-3}*)
    }
}
\end{lstlisting}
\begin{mathpar}
    \inferrule[sync]{%
        \viREPL = \viewFunCOPS(\repl)\rmto{\ke}{\viewFunCOPS(\repl)(\ke) \uplus i} 
        \texttt{ ---> \cref{lst:send-receive}, \cref{line:receive-update-view-1,line:receive-update-view-2,line:receive-update-view-3}}
        \\\\
        {\left(\begin{array}{@{}l@{}}
        \fora{\ke', m, \ver} 
        \ver = \mkvsCOPS(\ke, i) \land {} \\
        \quad (\ke', \WTx(\mkvsCOPS(\ke', m))) \in \ver\projection{4} \\
        \qquad {} \implies m \in \viREPL'(\ke') 
        \end{array}\right)} \texttt{ ---> \cref{lst:send-receive}, \cref{line:receive-wait}} 
    }{%
        \mkvsCOPS, \viewFunCOPS, \thdenv, \prog \toG{}
        \mkvsCOPS, \viewFunCOPS\rmto{\repl}{\viREPL}, \thdenv, \prog
    }
\end{mathpar}

A view \( \vi \) on key-value store \( \mkvsCOPS \) \emph{agrees} 
with another view \( \vi \) on another key-value store \( \mkvsCOPS' \), if and only
\[
 \func{getMax}{\mkvsCOPS, \vi} = \func{getMax}{\mkvsCOPS', \vi'}
\]



%\begin{theorem}
    %For any trace \( \tr \) of COPS with final configuration \( (\mkvsCOPS, \viewFunCOPS) \), 
    %there exists a trace \( \tr' \) with final configuration \( (\mkvsCOPS', \viewFunCOPS') \) such that 
    %each step of the trace \( \tr' \) commits a transaction with strictly greater transaction identifier than any one appearing in the key-value store:
    %\[
        %\begin{array}{@{}l@{}}
        %(\mkvsCOPS_i, \viewFunCOPS_i) 
        %\toG{} (\mkvsCOPS_{i+1}, \viewFunCOPS_{i+1}) 
        %\land \exsts{\txid} \txid \in \mkvsCOPS_{i+1} 
        %\land \txid \notin \mkvsCOPS_{i+1}
        %\implies \fora{\txid' \in \mkvsCOPS_i} \txid > \txid'
        %\end{array}
    %\]
    %and any replica's view from \( \viewFunCOPS \) agrees with its counterpart from  \( \viewFunCOPS' \):
    %\[
        %\fora{i} 
        %\func{getMax}{\mkvsCOPS, \viewFunCOPS(i)} = \func{getMax}{\mkvsCOPS', \viewFunCOPS'(i)}
    %\]
%\end{theorem}
%\begin{proof}
%\end{proof}

%\begin{lemma}
%\[
    %\begin{array}{@{}l@{}}
    %\fora{\repl, \cl, \mkvsCOPS, \mkvsCOPS', \mkvsCOPS'', \viREPL, \viREPL', \viREPL'', \viCL, \viCL', \viCL'', \stk, \stk', \cmd, \cmd'} \\
    %\quad \func{getMax}{\mkvsCOPS, \viREPL} = \func{getMax}{\mkvsCOPS'', \viREPL''} 
    %\land \func{getMax}{\mkvsCOPS, \viCL} = \func{getMax}{\mkvsCOPS'', \viCL''} \\
    %\qquad \repl, \cl \vdash 
    %\mkvsCOPS, \viREPL, \viCL, \stk, \cmd \toT{}
    %\mkvsCOPS', \viREPL', \viCL', \stk', \cmd' \\
    %\qquad \implies 
    %\exsts{\mkvsCOPS''', \viREPL'''}
    %\mkvsCOPS'', \viREPL'', \viCL'', \stk, \cmd \toT{}
    %\mkvsCOPS''', \viREPL''', \viCL', \stk', \cmd' \\
    %\qquad \func{getMax}{\mkvsCOPS', \viREPL'} = \func{getMax}{\mkvsCOPS''', \viREPL'''} 
    %\land \func{getMax}{\mkvsCOPS', \viCL'} = \func{getMax}{\mkvsCOPS''', \viCL'''} \\
    %\end{array}
%\]
%\end{lemma}
%\begin{proof}
%We perform case analysis.
%\begin{itemize}
    %\item \rl{Put}.
    %We have \( \cmd \equiv \ptrans{\pmutate{\ke}{\vx};} \) for some key \( \ke \) and variable \( \vx \).
    %Suppose key-value stores \(  \mkvsCOPS, \mkvsCOPS', \mkvsCOPS'' \), 
    %replica's views \( \viREPL, \viREPL', \viREPL''\) and client's views \( \viCL, \viCL', \viCL''\) such that
    %\[
    %\begin{array}{@{}l@{}}
    %\func{getMax}{\mkvsCOPS, \viREPL} = \func{getMax}{\mkvsCOPS'', \viREPL''} 
    %\land \func{getMax}{\mkvsCOPS, \viCL} = \func{getMax}{\mkvsCOPS'', \viCL''} \\
    %\qquad \repl, \cl \vdash 
    %\mkvsCOPS, \viREPL, \viCL, \stk, \cmd \toT{}
    %\mkvsCOPS', \viREPL', \viCL', \stk', \cmd' \\
    %\end{array}
    %\]
    %By the premiss of the \rl{Put} rule, the new key-value store
    %\[
        %\mkvsCOPS' = \mkvsCOPS\rmto{\ke}{\mkvsCOPS(\ke) \lcat \List{(\ke, \txid, \emptyset, \dep)}}
    %\]
    %where
    %\[
        %\txid = \min\Setcon{%
            %\txidCOPS{\repl}{\cl}{n'}{0}
        %}{%
            %\fora{\ke', i \in \viREPL(\ke'), n} \\
            %\quad \txidCOPS{\stub}{\stub}{n}{\stub} = \WTx(\mkvsCOPS(\ke',i)) \\
            %\qquad {} \implies n' > n 
        %} 
    %\]
    %and the new views of replica and client are
    %\[   
        %\begin{array}{@{}l@{}}
        %\viREPL' = \viREPL\rmto{\ke}{\viREPL(\ke) \uplus \Set{\abs{\mkvsCOPS'(\ke)} - 1}} \\
        %{} \land \viCL' = \viCL\rmto{\ke}{\viREPL(\ke) \uplus \Set{\abs{\mkvsCOPS'(\ke)} - 1}}
        %\end{array}
    %\]
    %Similarly there exists a new \( \mkvsCOPS''' \) by committing a single-write transaction \( \txid' \) and two new views \( \viREPL''' \) and \( \viCL''' \).
    %This means for those key \( \ke' \) that is different from the key \( \ke \) being overwritten,
    %\begin{equation}
        %\label{equ:get-max-match-all-other-key}
        %\begin{array}{@{}l@{}}
            %\func{getMax}{\mkvsCOPS', \viREPL'}(\ke') = \func{getMax}{\mkvsCOPS''', \viREPL'''}(\ke') \\
            %{} \land \func{getMax}{\mkvsCOPS', \viCL'}(\ke') = \func{getMax}{\mkvsCOPS''', \viCL'''}(\ke') 
        %\end{array}
    %\end{equation}
    %Note that the \( \txid \) is greater than any writers \( \txidCOPS{\repl}{\cl}{n'}{0} \) that can be observed by the \( \viREPL \), so is \( \txid' \).
    %That is,
    %\begin{equation}
        %\label{equ:get-max-match-overwritten-key}
        %\begin{array}{@{}l@{}}
            %\func{getMax}{\mkvsCOPS', \viREPL'}(\ke) = \stk(\vx) = \func{getMax}{\mkvsCOPS''', \viREPL'''}(\ke)  \\
            %{} \land \func{getMax}{\mkvsCOPS', \viCL'}(\ke) = \stk(\vx) = \func{getMax}{\mkvsCOPS''', \viCL'''}(\ke) 
        %\end{array}
    %\end{equation}
    %Combine \cref{equ:get-max-match-all-other-key} and \cref{equ:get-max-match-overwritten-key},
    %we have the proof.

    %\item \rl{GetTrans}.
    %Since the views of replica remain unchanged, so we only need to prove that there exists a new key-value store and a new view \( \viCL''' \) such that
    %\[
        %\func{getMax}{\mkvsCOPS', \viCL'}(\ke) = \stk(\vx) = \func{getMax}{\mkvsCOPS''', \viCL'''}(\ke) 
    %\]
    
%\end{itemize}
%\end{proof}

\begin{lemma}
    \label{lem:client-subset-repl}
    The view of a client is subset of the view of the replica that the client interacts with.
\end{lemma}


\begin{lemma}
    Let ignore the dependencies of versions from \( \mkvsCOPS \).
    Given the initial key-value store \( \mkvsCOPS_0 \), initial views \( \viewFunCOPS_0 \) and some programs \( \prog_0 \), for any \( \mkvsCOPS_i \) and \( \viewFunCOPS_i \)  such that: 
    \[
        \mkvsCOPS_0, \viewFunCOPS_0, \thdenv_0, \prog_0 {\toG{}}^* \mkvsCOPS_i, \viewFunCOPS_i, \thdenv_i, \prog_i
    \]
    The key-value store \( \mkvsCOPS_i \) satisfies the \cref{def:mkvs} and any replica or client view \( \vi \) from \( \viewFunCOPS_i \) is a valid view of the key-value store, \ie \( \vi \in \Views(\mkvsCOPS_i) \).
\end{lemma}
\begin{proof}
    We need to prove the  \( \mkvsCOPS_i \) satisfies the well-formed conditions,
    and any view \( \vi_i \Views(\mkvsCOPS_i) \).
    We prove it by introduction on the length \( i \).
    \begin{itemize}
    \item \caseB{\(i = 0\)}
        It holds trivially since each key only has the initial version \( (\val_0,\txid_0,\emptyset, \emptyset) \).
        Since there is only the initial version for each key, it is easy to see that any view \( \vi_0 \) satisfying the well-formed conditions in \cref{def:views}.
    \item \caseI{\(i > 0\)}
        Suppose it holds when \( i \), let consider \( i + 1 \).
        We perform case analysis on the possible next step:
        \begin{itemize}
            \item \rl{Put}
                Assume the client \( \cl \) of a replica \( \repl \) commits a single-write transaction \( \txid \) that installs a new version for key \( \ke \).
                By the premiss of \rl{Put}, the new transaction identifier \( \txid = \txidCOPS{\repl}{\cl}{n'}{0} \) where for some \( n' \) that is greater than any \( n \) from any writers \( \txidCOPS{\stub}{\stub}{n}{\stub} \) that are observable by the replica \( \repl \).
                Since the new transaction \( \txid = \txidCOPS{\repl}{\cl}{n'}{0} \) is a single-write transaction which is always installed at the end of the list associated to \( \ke \), it is sufficient to prove the following:
                \begin{gather}
                    \fora{j} 0 \leq j < \abs{ \mkvsCOPS_i(\ke) } \implies \WTx(\mkvsCOPS_{i}(\ke, m)) \neq \txid \label{equ:write-trans-unique} \\
                    \fora{j, n} \txidCOPS{\repl}{\cl}{n}{\stub} \in \Set{\WTx(\mkvsCOPS_{i}(\ke,j))} \cup \RTx(\mkvsCOPS_{i}(\ke, j)) \implies n < n' \label{equ:replica-time-monotonic-inc}
                \end{gather}
                By \cref{lem:repl-observe-own}, we know that for any version written \( \ver = \mkvsCOPS_i(\ke, j) \) by the same replica \( \txidCOPS{\repl}{\stub}{\stub}{\stub} = \WTx(\ver) \), such version is included in the replica's view \( j \in \viREPL(\ke) \).
                It implies that first the new transaction identifier is unique \cref{equ:write-trans-unique} and second it is greater than any transactions in the form of \( \txidCOPS{\repl}{\cl}{\stub}{\stub} \) \cref{equ:replica-time-monotonic-inc}.
                Thus the new key-value store \( \mkvsCOPS_{i+1} \) satisfies the well-formed conditions.
                Now let consider the views, especially the views of the replica \( \viREPL' \) and the client \( \viCL' \).
                Since that views \( \vi' \) from different replicas or clients remain unchanged, by \ih they satisfy \( \vi' \in \Views(\mkvsCOPS_{i+1}) \).
                The new view for replica \( \viREPL' = \viREPL\rmto{\ke}{\abs{\mkvsCOPS_{i+1}(\ke)} - 1} \)
                where \( \viREPL \) is the replica's view before updating and the writer of the last version of \( \ke \) is \( \txid \).
                Because \( \txid \) is a single-write transaction, so the new view \( \viREPL' \) still satisfies the atomic read.
                For similar reason, the new view for client \( \viCL' \) till satisfies atomic read.
                Therefore we have \( \viREPL', \viCL' \in \Views(\mkvsCOPS_{i+1}) \).
            \item \rl{GetTrans}

        \end{itemize}
    \end{itemize}
\end{proof}

\begin{lemma}
    \label{lem:repl-observe-own}
    A replica observes all its own transactions.
\end{lemma}

\section{Logic}

\subsection{Rules for Local}

The proof rules are standard except \rl{TRDeref} and \rl{TRMutate}.
The \rl{TRDeref} rule add read fingerprint in finger-tracking set, only if there is no write finger-print.
This is because once a location has been re-written, the rest read are considered as local operations, while the finger-print only records those operations might have effect on global state.
%
\[
    \infer[\rl{TRDeref}]{%
        \judgeL{\expr \fpt{\fp} \lexpr}{ \pderef{\var}{\expr} }{\var \dot= \lexpr \sep \expr \fpt{\addFPR{\fp}} \lexpr }
    }{%
        \var \notin \func{fv}{\expr} &&
        \var \notin \func{fv}{\lexpr}  
    }
\]
 
\[
    \infer[\rl{TRMutate}]{%
        \judgeL{\expr_1 \fpt{\fp} \stub }{ \pmutate{\expr_1}{\expr_2} }{ \expr_1 \fpt{\addFPW{\fp}} \expr_2} 
    }{}
\]

\subsection{Merge}

\begin{defn}[Fingerprint heaps merge]
\label{def:merge-finger-heap}
The \emph{merge of fingerprint heaps}, \( \mergeFP{.}{.} \), is defined as follows:
\[
    \begin{rclarray}
        \mergeFP{\fph_{l}}{\fph_{r}}  & \defeq & \lambda \addr \ldotp 
            \begin{cases}
                \fph_{l}(\addr) & a \in \dom(\fph_{l}) \setminus \dom(\fph_{r})  \\
                \fph_{r}(\addr) & a \in \dom(\fph_{r}) \setminus \dom(\fph_{l}) \\
                \mergeVAL{\fph_{l}(\addr)}{\fph_{r}(\addr)}  & a \in \dom(\fph_{l}) \cap \dom(\fph_{r}) \\
            \end{cases}
    \end{rclarray}
\]
where the \emph{merge of fingerprint heap values} is defined:
\[ \begin{rclarray}
        \mergeVAL{(\val_{l}, \fp_{l})}{(\val_{r}, \fp_{r})} & \defeq &
            \begin{cases}
                (\val_{l}, \fp_{l} \cup \fp_{r} ) & \val_{l} = \val_{r} \land \fpW \notin \fp_{l} \cup \fp_{r} \\
                (\val_{l}, \fp_{l} \cup \fp_{r} ) & \fpW \in \fp_{l} \land \fpW \notin \fp_{r} \\
                (\val_{r}, \fp_{l} \cup \fp_{r} ) & \fpW \notin \fp_{l} \land \fpW \in \fp_{r} \\
            \end{cases}
    \end{rclarray}
\]
\end{defn}

%\sx{
    %Andrea gives a better idea to do this by defining merging fingerprint heaps first.
    %Big thanks. :)))
%}
%\azalea{This is a bit strong! For instance, according to this definition the heaps $\pv x \pt_{\emptyset} 1$ and $\pv x \pt_{\{\fpR\}} 1$ do not agree! Is that what you want?

%Perhaps you can define this as:
%\[
%\begin{rclarray}
	%\agree{\fph_l}{\fph_r} & \defeq  & \ws{\fph_l} \cap \ws{\fph_r} = \emptyset \\
        %&& \land\ \exsts{\h_1, \h_2, \h} \heapOnly{\readOnly{\fph_l}} = \h_1 \composeH \h \land \heapOnly{\readOnly{\fph_r}} = \h \composeFP \h_2 
        %\land \h_1 \composeH \h \composeH \h_2 \isdef
%\end{rclarray}        
%\]
%where
%\[
%\begin{rclarray}
	%\func{ws}{\fph} & \defeq & \myset{\loc}{ \exsts{\fp} \fph(\loc) = (\stub, \fp) \land \fpW \in \fp} \\\\
%%
	%\readOnly{.} & : & \FPHeaps \rightarrow \FPHeaps\\	
	%\readOnly{\fph}(\loc) & \defeq & 
	%\begin{cases}
		%\fph(\loc) & \text{if } \loc \not\in \ws{\fph} \\
		%\text{undefined} & \text{otherwise}
	%\end{cases}\\\\
%%	
	%\heapOnly{.} & : & \FPHeaps \rightarrow \Heaps\\	
	%\heapOnly{\fph}(\loc) & \defeq & 
	%\begin{cases}
		%\val  & \text{if } \exsts{\val} \fph(\loc) = (\val, -) \\
		%\text{undefined} & \text{otherwise}
	%\end{cases}
%\end{rclarray}
%\]
%}

\begin{defn}[Local state merge]
The \emph{merge of local states} is defined as follows, which merges two local states \( \ls_{l} \) and \( \ls_{r} \) with respect to a common initial local state \( \ls \).
\[
    \begin{rclarray}
	\mergeLS{\ls}{\ls_{l}}{\ls_{r}} & \eqdef &
	\myset{
		\left(\fph, \ca_{l}' \composeC \ca_{f} \composeC \ca_{r}' \right) 
	}{
        \fph = \mergeFP{\lsFPH{(\ls_{l})} }{ \lsFPH{(\ls_{r})} } \land \exsts{\ca_{l}, \ca_{r}}\\
		\quad \land\; \lsCAP{\ls} = \ca_{l} \composeC \ca_{f} \composeC \ca_{r}
        \land \lsCAP{(\ls_{l})} = \ca_{l}' \composeC \ca_{f} \composeC \ca_{r}
        \land \lsCAP{(\ls_{r})} = \ca_{l} \composeC \ca_{f} \composeC \ca_{r}'
	}
    \end{rclarray}
\]
where, to recall, the notation \( \lsFPH{(.)} \), and \( \lsCAP{(.)} \) refer to the fingerprint heap and capabilities respectively in a local state.
Then, the \emph{conflict} between two local states is defined as follows:
\[
    \begin{rclarray}
        \conflict{\ls}{\ls_{l}}{\ls_{r}} & \defeq & \mergeLS{\ls}{\ls_{l}}{\ls_{r}} = \emptyset
    \end{rclarray}
\]
\end{defn}

\begin{definition}[Fingerprint worlds]\label{def:fingerprint_worlds}
Given the set of local states $\LStates$ (\defin\ref{def:local_state}) and the set of region identifiers $\RegionID$ (\defin\ref{def:capabilities}), the set of \emph{fingerprint worlds}, $\fpw \in \FPWorlds$, is defined as follows:
%
\[
\begin{rclarray}
	\FPWorlds  & \eqdef  
	& \myset{
		(\ls, \fpgs)
	}{
		(\ls, \fpgs) \in \LStates \times (\RegionID \parfinfun \LStates) \land \wfFW{\ls, \fpgs}
	}
\end{rclarray}
\]
%
with the definitions of the flattening function and the well-formedness predicate lifted as follows:
%
\[
\begin{rclarray}
	\flattenFW{(\ls, \fpgs)}  & \eqdef & \ls \composeLS \prod\limits_{\rid \in \dom(\fpgs)}^{\composeLS} \fpgs(\rid)
\end{rclarray}
\]
%
\[
\begin{rclarray}
	\wfFW{\ls, \fpgs} & \defeq & \exsts{\fph, \ca}\flattenFW{(\ls, \fpgs)} {=} (\fph, \ca) \land\ \dom(\ca) \subseteq \dom(\fpgs) \\
\end{rclarray}
\]
%
The \emph{lift function}, $\liftW{.}: \World \rightarrow \FPWorlds$, is defined as follows:
%
\[
	\liftW{(\lgs, \gs)} \eqdef (\liftLGS{\lgs}, \liftGS{\gs})
\]
%
where
%
\[
\begin{rclarray}
	\liftLGS{(\h, \ca)} = (\fph, \ca)
	& \iffdef 
	& \for{\loc, \val} \h(\loc) = \val \iff \fph(\loc) = (\val, \emptyset) \\
%
	\liftGS{\gs} = \fpgs 
	& \iffdef
	& \for{\rid, \lgs, \inter} \gs(\rid) = (\lgs, \inter) \iff \fpgs(\rid) = (\liftLGS{\lgs'}, \inter) 
\end{rclarray}
\]
%
The \emph{erase function}, $\eraseFW{.}: \FPWorlds \rightarrow \World$, is defined as follows:
%
\[
	\eraseFW{(\ls, \fpgs)} \eqdef (\eraseLS{\ls}, \eraseFGS{\fpgs})
\]
%
where
\[
\begin{rclarray}
	\eraseLS{(\fph, \ca)} = (\h, \ca)
	& \iffdef 
	& \for{\loc, \val} \h(\loc) = \val \iff \fph(\loc) = (\val, \emptyset) \\
%
	\eraseFGS{\fpgs} = \gs 
	& \iffdef
	& \for{\rid, \ls, \inter} \fpgs(\rid) = (\ls, \inter) \Rightarrow \gs(\rid) = (\eraseLS{\ls}, \inter) 
\end{rclarray}
\]
\end{definition}

\begin{definition}[Fingerprint worlds merge]
Given the set of fingerprint worlds $\FPWorlds$ (\defin\ref{def:fingerprint_worlds}), the \emph{merge} function, $\mergeName[\fpw]: \FPWorlds \times \FPWorlds \times \FPWorlds \rightarrow \powerset{\FPWorlds}$, is defined as follows, for all $\fpw,\fpw_{l},\fpw_{r} \in \FPWorlds$:

%
\[
    \begin{rclarray}
	\mergeFW{(\stub, \fpgs)}{(\ls_{l}, \fpgs_{l})}{(\stub, \fpgs_{r})} & \eqdef &
		\Setcon{%
            (\ls_{l}, \fpgs_{p}) 
        }{%
            \fpgs_{p} \in \mergeFGS{\fpgs}{\fpgs_{l}}{\fpgs_{r}}
        } 
    \end{rclarray}
\]
%
where the \( \mergeName[\fpgs] \) is defined as follows:
\[
    \begin{rclarray}
        \mergeFGS{\fpgs}{\fpgs_{l}}{\fpgs_{r}} & \eqdef &
        \begin{cases}
            \emptyset & \text{if} \ \exsts{\rid} \conflict{\fpgs(\rid)}{\fpgs_{l}(\rid)}{\fpgs_{r}(\rid)} \lor \rid \in  \dom(\fpgs_{l}) \cap \dom(\fpgs_{r}) \setminus \dom(\fpgs)  \\
            S & \text{otherwise}
        \end{cases}
    \end{rclarray}
\]
with,
\[
    \begin{rclarray}
	S & = & \myset{\fpgs_{p}}{
		\dom(\fpgs_{p})= \dom(\fpgs_{l}) \cup \dom(\fpgs_{r}) \land \for{\rid}\\
		\quad \rid \in \dom(\fpgs_{l}) \cap \dom(\fpgs_{r}) \implies \fpgs_{p}(\rid) \in \mergeLS{\fpgs(\rid)}{\fpgs_{l}(\rid)}{\fpgs_{r}(\rid)} \\
		\quad \land\ \rid \in \dom(\fpgs_{l}) \setminus \dom(\fpgs_{r}) \implies 	\fpgs_{p}(\rid) = \fpgs_{l}(\rid) \\
		\quad \land\ \rid \in \dom(\fpgs_{r}) \setminus \dom(\fpgs_{l}) \implies 	\fpgs_{p}(\rid) = \fpgs_{r}(\rid)
	}
    \end{rclarray}
\]
\end{definition}

Note that the \( \mergeName[\fpw] \) is not commutative, i.e.\ swapping \( \fpw_{l}\) and \( \fpw_{r}\) might yield different result.

\subsection{Rely and Guarantee}

\begin{definition}[Rely and guarantee]
Given the set of fingerprint worlds $\FPWorlds$ (\defin\ref{def:fingerprint_worlds}), the \emph{update rely} relation, $\relyU: \FPWorlds \times \FPWorlds$, is defined as follows:
%
\[	
    \begin{rclarray}
	\relyU & \eqdef &
	\myset{
		((\ls, \fpgs_{p}), (\ls, \fpgs_{q}))	
	}{
		\exsts{\rid, \ca, \intf, \kap, \ls_{p}, \ls_{q}, \ls_{f}}\\
		\quad \for{\rid'} \rid \ne \rid' \implies \fpgs_{p}(\rid') = \fpgs_{q}(\rid') \\
		\quad \land\ \fpgs_{p}(\rid) = (\ls_{p} \composeLS \ls_{f}, \inter) \land \fpgs_{q}(\rid) = (\ls_{q} \composeLS \ls_{f}, \inter)		 \\
		\quad \land\ ( (\unitFP, \ca) \composeLS \flattenFW{(\ls, \fpgs_{p})} ) \isdef 
		\land \kap \leq \ca(\rid)
		\land (\ls_{p}, \ls_{q}) \in \inter(\kap)
	}
    \end{rclarray}
\]
The \emph{extension rely} relation, $\relyE: \FPWorlds \times \FPWorlds$, is defined as follows:
%
\[	
    \begin{rclarray}
        \relyE \eqdef
        \myset{
            \big((\ls, \fpgs_{p}), (\ls, \fpgs_{q})\big)	
        }{
            \exsts{\rid}
            \dom(\fpgs_{q}) \setminus \dom(\fpgs_{p}) = \{\rid\} \\
            \qquad \land\ \for{\rid'} \rid \ne \rid' \implies \fpgs_{p}(\rid') = \fpgs_{q}(\rid') \\
        }
    \end{rclarray}
\]
The \emph{rely} relation, $\myrely: \FPWorlds \times \FPWorlds$, is defined as follows:
%
\[
    \begin{rclarray}
         \myrely  &\eqdef & \bigcup\limits_{n \in \Nat} R_n \\
    \end{rclarray}
\]
where,
\[
    \begin{rclarray}
        \rely_0 & \eqdef & \closure{(\relyU \cup \relyE)} \\
        R_{n {+} 1} & \eqdef & (R_n \cup \pred{merge\_close}{R_n})^{*} \\
        \pred{merge\_close}{\rely} & \eqdef 
        & \myset{(\fpw, \fpw_{q})}{
            \exsts{\fpw_{l}, \fpw_{r}} (\fpw, \fpw_{l}), (\fpw, \fpw_{r}) \in \rely \land \fpw_{q} \in \mergeFW{\fpw}{\fpw_{l}}{\fpw_{r}}}
    \end{rclarray}
\]
%
The $\closure{(.)}$ denotes the reflexive transitive closure of the relation.
A set of fingerprint worlds $\setworld \subseteq \World$ is \emph{stable}, written $\stable{\setworld}$, if and only if it is closed under the rely relation: 
%
\[
    \begin{rclarray}
        \stable{W} & \eqdef & \for{\world \in \setworld, \fpw'} (\liftW{\world}, \fpw') \in \myrely \implies \eraseFW{\fpw'} \in \setworld
    \end{rclarray}
\]
%
The \emph{update guarantee} relation, $\guarU: \FPWorlds \times \FPWorlds$, is defined as follows:
%
\[	
    \begin{rclarray}
        \guarU & \eqdef &
        \myset{
            ((\ls_{p}, \fpgs_{p}), (\ls_{q}, \fpgs_{q}))	
        }{
            \exsts{\ls_{p}' = \flattenFW{(\ls_{p}, \fpgs_{p})}, \ls_{p}' = \flattenFW{(\ls_{q}, \fpgs_{q})} } \orth{(\lsCAP{(\ls_{p}')})} = \orth{(\lsCAP{(\ls_{q}')})}  \\
            \land 
            \begin{formulea}
                \for{\rid} \fpgs_{p}(\rid) = \fpgs_{q}(\rid) \\
                \lor 
                \begin{formulea}
                    \orth{(\lsFPH{(\ls_{p}')})} = \orth{(\lsFPH{(\ls_{q}')})} 
                    \land \exsts{\rid, \ca, \kap, \intf, \ls_{p}'', \ls_{q}'', \ls_f}\\
                        \quad \for{\rid'} \rid \ne \rid' \implies \fpgs_{p}(\rid') = \fpgs_{q}(\rid') \\
                        \quad \land \fpgs_{p}(\rid) = (\ls_{p}'' \composeLS \ls_f, \inter) \land \fpgs_{q}(\rid) = (\ls_{q}'' \composeLS \ls_f, \inter)		 \\
                        \quad \land (\unitFP, \ca) \leq \ls_{p}
                        \land \kap \leq \ca(\rid)
                        \land (\ls_{p}'', \ls_{q}'') \in \inter(\kap)
                \end{formulea}
            \end{formulea}
        }
    \end{rclarray}
\]
% 
The \emph{extension guarantee} relation, $\guarE: \FPWorlds \times \FPWorlds$, is defined as follows:
%
\[	
    \begin{rclarray}
	\guarE & \eqdef &
	\myset{
		((\ls_{f} \composeLS \ls, \fpgs_{p}), (\ls_{f} \composeLS (\unitFP, \ca), \fpgs_{q}))	
	}{
		\exsts{\rid, \ca'}
		\dom(\fpgs_{q}) = \dom(\fpgs_{p}) \uplus \Set{\rid} \\
		\qquad \land\ \for{\rid'} \rid \ne \rid' \implies \fpgs_{p}(\rid') = \fpgs_{q}(\rid') \\
		\qquad \land\ \fpgs_{q}(\rid) = \ls \composeLS (\unitFP, \ca')
		\land \dom(\ca) = \dom(\ca') = \Set{\rid}
	}
    \end{rclarray}
\]
% 
The \emph{guarantee} relation, $\myguar: \FPWorlds \times \FPWorlds$, is defined as follows:
%
\[
	\myguar \eqdef (\guarU \cup \guarE)^{\scalebox{1.1}{*}}
\]
%
\end{definition}

\subsection{Rules for Global}

The \rl{PRCommit} rule lifts the local effect of transaction \( \trans \) to global level by repartition \( \repartition{\gpre}{\gpost}{\lpre}{\lpost} \).
The repartition stripes off the fingerprints but uses the fingerprints to merge the local effect and the interference.
This is, the environment is allowed to write to locations that are different from the ones by transaction \( \trans \).
%
\[
    \infer[\rl{PRCommit}]{%
        \judgeG{\gpre}{ \ptrans{\trans} }{\gpost}
    }{%
        \judgeL{\lpre}{\trans}{\lpost} &&
        \repartition{\gpre}{\gpost}{\lpre}{\lpost}
    }
\]

\begin{definition}[Repartitioning]
The \emph{repartitioning} is defined as follows:
\[
    \begin{rclarray}
        \mrepartition{\setworld_{p}}{\setworld_{q}}{\setfph_{p}}{\setfph_{q}} & \iffdef &
        \begin{array}[t]{@{} l @{}}
            \for{\world_{p} \in \setworld_{p}} \exsts{\fpw_{p}, \fph_{p}, \fph_{f}}\\
            \quad \fpw_{p} = \liftW{\world_{p}} \land \flattenFW{\fpw_{p}} = (\fph_{p} \composeFP \fph_{f}, \unitC) \land \fph_{p} \in \setfph_{p} \\
            \quad \land\ \for{\fph_{q} \in \setfph_{q}} \exsts{\world_{q}, \fpw_{q}} \\
            \qquad \flattenFW{\fpw_{q}} = (\fph_{q} \composeFP \fph_{f}, \unitC) \land (\fpw_{p}, \fpw_{q}) \in \myguar \land \eraseFW{\fpw_{q}} = \world_{q} \land \world_{q} \in \setworld_{q} \\
            \qquad \land\ \for{\fpw \in \mergeR{\fpw_{p}}{\fpw_{q}}{\myrely}} \eraseFW{\fpw} \in \setworld_{q}
        \end{array}
    \end{rclarray}
\]
with, $\mergeName[\scalebox{.5}{\(\myrely\)}]: \FPWorlds \times \FPWorlds \times \powerset{\FPWorlds \times \FPWorlds} \to \powerset{\FPWorlds}$, defined as follows, for all $\fpw_{p}, \fpw_{q} \in \FPWorlds$:
%
\[
	\mergeR{\fpw_{p}}{\fpw_{q}}{\myrely} \eqdef \bigcup\limits_{\fpw \in \myrely(\fpw_{p})} \mergeFW{\fpw_{p}}{\fpw_{q}}{\fpw}
\]
%
\end{definition}

%\section{Examples\label{sec:example}}

\sx{New observation:
\begin{itemize}
\item We might want to have some assertion to say the initial values of a region.
\item What is the meaning of sequential composition since some consistency model does not have strong session guarantee, maybe the answer is the ``commit'' order.
\item Explain stablisation in a roughly syntactic level.
\end{itemize}
}

\subsection{Single increment and multi-reader.}
\[
    \begin{array}{@{}l@{}}
        \boxass{\V{x} \pt \V{\nat}}{\lrid}{\intass} \\
        \C{Inc} \composeK \C{Inc} \text{ is undefined} \\
        \C{Rd} \text{ is the unit element} \\
    \end{array}
\]
\subsubsection{SER}
\[
    \begin{array}{@{}l@{}}
        \intass : 
        \begin{rclarray}[t]
        \C{Inc} & : & \exsts{\V{m, k}} \Set{(\etR, \V{x}, \V{m}), (\etW, \V{x}, \V{m} + 1)} \mat \V{x} \pt \V{k} \oassto \V{x} \pt \V{k} \\
        \C{Rd}  & : & \exsts{\V{m, k, v}} \Set{(\etR, \V{x}, \V{m})} \mat \V{x} \pt \V{k} \oassto \V{x} \pt \V{k} \\ 
        \end{rclarray} \\
    \end{array}
\]

\[
\begin{session}
\specline{\boxass{\vx \pt 0}{\lrid}{\intass} \sep \cass{\C{Inc}}{\lrid} } \\
\specline{\boxass{\vx \pt 0}{\lrid}{\intass} \sep \cass{\C{Inc}}{\lrid} \sep \cass{\C{Rd}}{\lrid} \sep \cass{\C{Rd}}{\lrid} } \\
\begin{parl}
    \begin{session}
    \specline{\boxass{\vx \pt 0}{\lrid}{\intass} \sep \cass{\C{Inc}}{\lrid} \sep \cass{\C{Rd}}{\lrid} } \\
    \begin{transaction}
        \specline{ \vx \pt 0 } \\
        \pderef{\pvar{a}}{\vx} ; \\
        \specline{ \vx \pt 0 \land \pvar{a} = 0 \sep \Set{(\etR, \vx, 0)} } \\
        \pmutate{\vx}{\pvar{a} + 1} ; \\
        \specline{ \vx \pt 1 \land \pvar{a} = 0 \\
                {} \sep \Set{(\etR, \vx, 0), (\etW, \vx, 1)} } \\
    \end{transaction} \\
    \specline{\boxass{\vx \pt 1}{\lrid}{\intass} \sep \cass{\C{Inc}}{\lrid} \sep \cass{\C{Rd}}{\lrid} } \\
    \begin{transaction}
        \specline{ \vx \pt 1 } \\
        \pderef{\pvar{b}}{\vx} ; \\
        \specline{ \lor \vx \pt 1 \sep \Set{(\etR, \vx, 1)} } \\
    \end{transaction} \\
    \specline{\boxass{\vx \pt 1}{\lrid}{\intass} \sep \cass{\C{Inc}}{\lrid} \sep \cass{\C{Rd}}{\lrid} } \\
    \begin{transaction}
        \specline{ \vx \pt 1 } \\
        \pderef{\pvar{a}}{\vx} ; \\
        \specline{ \vx \pt 1 \land \pvar{a} = 1 \sep \Set{(\etR, \vx, 1)} } \\
        \pmutate{\vx}{\pvar{a} + 1} ; \\
        \specline{ \vx \pt 2 \land \pvar{a} = 1 \\
                {} \sep \Set{(\etR, \vx, 1), (\etW, \vx, 2)} } \\
    \end{transaction} \\
    \specline{\boxass{\vx \pt 2}{\lrid}{\intass} \sep \cass{\C{Inc}}{\lrid} \sep \cass{\C{Rd}}{\lrid} } \\
    \end{session}
    &
    \begin{session}
    \specline{\exsts{\V{n}}\boxass{\vx \pt \V{n}}{\lrid}{\intass} \land \V{n} \geq 0 \sep \cass{\C{Rd}}{\lrid} } \\
    \begin{transaction}
        \specline{ \exsts{ \V{v} } \vx \pt \V{v} { \color{gray} \land \V{v} = \V{n} } } \\
        \pderef{\pvar{c}}{\vx} ; \\
        \specline{ \exsts{ \V{v} } \vx \pt \V{v} \sep \Set{(\etR, \vx, \V{v})} { \color{gray} \land \V{v} = \V{n} } } \\
    \end{transaction} \\
    \specline{\exsts{\V{n}}\boxass{\vx \pt \V{n}}{\lrid}{\intass} \land \V{n} \geq 0 \sep \cass{\C{Rd}}{\lrid} } \\
    \end{session}
\end{parl} \\
\specline{\boxass{\vx \pt 2}{\lrid}{\intass} \sep \cass{\C{Inc}}{\lrid} \sep \cass{\C{Rd}}{\lrid} \sep \cass{\C{Rd}}{\lrid} } \\
\specline{\boxass{\vx \pt 2}{\lrid}{\intass} \sep \cass{\C{Inc}}{\lrid} } \\
\end{session}
\]
\subsubsection{SI/PSI}
\[
    \begin{array}{@{}l@{}}
        \intass : 
        \begin{rclarray}[t]
        \C{Inc} & : & \exsts{\V{m, k}} \Set{(\etR, \V{x}, \V{m}), (\etW, \V{x}, \V{m} + 1)} \mat \V{x} \pt \V{k} \oassto \V{x} \pt \V{k} \\
        \C{Rd}  & : & \exsts{\V{m, k, v}} \Set{(\etR, \V{x}, \V{m})} \mat \V{x} \pt \V{k} \oassto \V{x} \pt \V{v} \land \V{v} \leq \V{k} \\ 
        \end{rclarray} \\
        \C{Inc} \composeK \C{Inc} \text{ is undefined} \\
        \C{Rd} \text{ is the unit element} \\
    \end{array}
\]

\[
\begin{session}
\specline{\boxass{\vx \pt 0}{\lrid}{\intass} \sep \cass{\C{Inc}}{\lrid} } \\
\specline{\boxass{\vx \pt 0}{\lrid}{\intass} \sep \cass{\C{Inc}}{\lrid} \sep \cass{\C{Rd}}{\lrid} \sep \cass{\C{Rd}}{\lrid} } \\
\begin{parl}
    \begin{session}
    \specline{\boxass{\vx \pt 0}{\lrid}{\intass} \sep \cass{\C{Inc}}{\lrid} \sep \cass{\C{Rd}}{\lrid} } \\
    \begin{transaction}
        \specline{ \vx \pt 0 } \\
        \pderef{\pvar{a}}{\vx} ; \\
        \specline{ \vx \pt 0 \land \pvar{a} = 0 \sep \Set{(\etR, \vx, 0)} } \\
        \pmutate{\vx}{\pvar{a} + 1} ; \\
        \specline{ \vx \pt 1 \land \pvar{a} = 0 \\
                {} \sep \Set{(\etR, \vx, 0), (\etW, \vx, 1)} } \\
    \end{transaction} \\
    \specline{\boxass{\vx \pt 1}{\lrid}{\intass} \sep \cass{\C{Inc}}{\lrid} \sep \cass{\C{Rd}}{\lrid} } \\
    \begin{transaction}
        \specline{ {\color{purple} \vx \pt 0} \lor \vx \pt 1 } \\
        \pderef{\pvar{b}}{\vx} ; \\
        \specline{ { \color{purple} \vx \pt 0 \sep \Set{(\etR, \vx, 0)} }  \\
                    {} \lor \vx \pt 1 \sep \Set{(\etR, \vx, 1)} } \\
    \end{transaction} \\
    \specline{\boxass{\vx \pt 1}{\lrid}{\intass} \sep \cass{\C{Inc}}{\lrid} \sep \cass{\C{Rd}}{\lrid} } \\
    \begin{transaction}
        \specline{ \vx \pt 1 } \\
        \pderef{\pvar{a}}{\vx} ; \\
        \specline{ \vx \pt 1 \land \pvar{a} = 1 \sep \Set{(\etR, \vx, 1)} } \\
        \pmutate{\vx}{\pvar{a} + 1} ; \\
        \specline{ \vx \pt 2 \land \pvar{a} = 1 \\
                {} \sep \Set{(\etR, \vx, 1), (\etW, \vx, 2)} } \\
    \end{transaction} \\
    \specline{\boxass{\vx \pt 2}{\lrid}{\intass} \sep \cass{\C{Inc}}{\lrid} \sep \cass{\C{Rd}}{\lrid} } \\
    \end{session}
    &
    \begin{session}
    \specline{\exsts{\V{n}}\boxass{\vx \pt \V{n}}{\lrid}{\intass} \land \V{n} \geq 0 \sep \cass{\C{Rd}}{\lrid} } \\
    \begin{transaction}
        \specline{ \exsts{ \V{v} } \vx \pt \V{v} { \color{gray} \land \V{v} \leq \V{n} } } \\
        \pderef{\pvar{c}}{\vx} ; \\
        \specline{ \exsts{ \V{v} } \vx \pt \V{v} \sep \Set{(\etR, \vx, \V{v})} { \color{gray} \land \V{v} \leq \V{n} } } \\
    \end{transaction} \\
    \specline{\exsts{\V{n}}\boxass{\vx \pt \V{n}}{\lrid}{\intass} \land \V{n} \geq 0 \sep \cass{\C{Rd}}{\lrid} } \\
    \end{session}
\end{parl} \\
\specline{\boxass{\vx \pt 1}{\lrid}{\intass} \sep \cass{\C{Inc}}{\lrid} \sep \cass{\C{Rd}}{\lrid} \sep \cass{\C{Rd}}{\lrid} } \\
\specline{\boxass{\vx \pt 1}{\lrid}{\intass} \sep \cass{\C{Inc}}{\lrid} } \\
\end{session}
\]

\subsubsection{Causal}

\[
    \begin{array}{@{}l@{}}
        \intass : 
        \begin{rclarray}[t]
        \C{Inc} & : & \exsts{\V{m, k}} \Set{(\etR, \V{x}, \V{m}), (\etW, \V{x}, \V{m} + 1)} \mat \V{x} \pt \V{k} \oassto \V{x} \pt \V{v} \\
        \C{Rd}  & : & \exsts{\V{m, k, v}} \Set{(\etR, \V{x}, \V{m})} \mat \V{x} \pt \V{k} \oassto \V{x} \pt \V{v} \\ 
        \end{rclarray} \\
        \C{Inc} \composeK \C{Inc} \text{ is undefined} \\
        \C{Rd} \text{ is the unit element} \\
    \end{array}
\]

\[
\begin{session}
\specline{\boxass{\vx \pt 0}{\lrid}{\intass} \sep \cass{\C{Inc}}{\lrid} } \\
\specline{\boxass{\vx \pt 0}{\lrid}{\intass} \sep \cass{\C{Inc}}{\lrid} \sep \cass{\C{Rd}}{\lrid} \sep \cass{\C{Rd}}{\lrid} } \\
\begin{parl}
    \begin{session}
    \specline{\boxass{\vx \pt 0}{\lrid}{\intass} \sep \cass{\C{Inc}}{\lrid} \sep \cass{\C{Rd}}{\lrid} } \\
    \begin{transaction}
        \specline{ \exsts{ \V{v} } \vx \pt \V{v} } \\
        \pderef{\pvar{a}}{\vx} ; \\
        \specline{ \exsts{ \V{v} } \vx \pt \V{v} \land \pvar{a} = \V{v} \sep \Set{(\etR, \vx, \V{v})} } \\
        \pmutate{\vx}{\pvar{a} + 1} ; \\
        \specline{ \vx \pt \V{v} + 1 \land \pvar{a} = \V{v} \\
                {} \sep \Set{(\etR, \vx, \V{v}), (\etW, \vx, \V{v} + 1)} } \\
    \end{transaction} \\
    \specline{\exsts{\V{n}}\boxass{\vx \pt \V{n}}{\lrid}{\intass} \sep \cass{\C{Inc}}{\lrid} \sep \cass{\C{Rd}}{\lrid} } \\
    \begin{transaction}
        \specline{ \exsts{ \V{v} } \vx \pt \V{v} } \\
        \pderef{\pvar{c}}{\vx} ; \\
        \specline{ \exsts{ \V{v} } \vx \pt \V{v} \sep \Set{(\etR, \vx, \V{v})} } \\
    \end{transaction} \\
    \specline{\exsts{\V{n}}\boxass{\vx \pt \V{n}}{\lrid}{\intass} \sep \cass{\C{Inc}}{\lrid} \sep \cass{\C{Rd}}{\lrid} } \\
    \begin{transaction}
        \specline{ \exsts{ \V{v} } \vx \pt \V{v} } \\
        \pderef{\pvar{a}}{\vx} ; \\
        \specline{ \exsts{ \V{v} } \vx \pt \V{v} \land \pvar{a} = \V{v} \sep \Set{(\etR, \vx, \V{v})} } \\
        \pmutate{\vx}{\pvar{a} + 1} ; \\
        \specline{ \vx \pt \V{v} + 1 \land \pvar{a} = \V{v} \\
                {} \sep \Set{(\etR, \vx, \V{v}), (\etW, \vx, \V{v} + 1)} } \\
    \end{transaction} \\
    \specline{\exsts{\V{n}}\boxass{\vx \pt \V{n}}{\lrid}{\intass} \sep \cass{\C{Inc}}{\lrid} \sep \cass{\C{Rd}}{\lrid} } \\
    \end{session}
    &
    \begin{session}
    \specline{\exsts{\V{n}}\boxass{\vx \pt \V{n}}{\lrid}{\intass} \sep \cass{\C{Rd}}{\lrid} } \\
    \begin{transaction}
        \specline{ \exsts{ \V{v} } \vx \pt \V{v} { \color{gray} \land \V{v}, \V{n} } } \\
        \pderef{\pvar{c}}{\vx} ; \\
        \specline{ \exsts{ \V{v} } \vx \pt \V{v} \sep \Set{(\etR, \vx, \V{v})} { \color{gray} \land \V{v}, \V{n} } } \\
    \end{transaction} \\
    \specline{\exsts{\V{n}}\boxass{\vx \pt \V{n}}{\lrid}{\intass} \sep \cass{\C{Rd}}{\lrid} } \\
    \end{session}
\end{parl} \\
\specline{\exsts{\V{n}}\boxass{\vx \pt \V{n}}{\lrid}{\intass} \sep \cass{\C{Inc}}{\lrid} \sep \cass{\C{Rd}}{\lrid} \sep \cass{\C{Rd}}{\lrid} } \\
\specline{\exsts{\V{n}}\boxass{\vx \pt \V{n}}{\lrid}{\intass} \sep \cass{\C{Inc}}{\lrid} } \\
\end{session}
\]

\subsection{Two associated bank accounts}
\[
    \begin{array}{@{}l@{}}
        \boxass{\V{x} \pt \V{n} \sep \V{y} \pt \V{m} }{\lrid}{\intass} \\
        \C{xx} \composeK \C{xx} \text{ is undefined} \\
        \C{yy} \composeK \C{yy} \text{ is undefined} \\
        \C{Rd} \text{ is the unit} \\
    \end{array}          
\]

\subsubsection{SER}
\[
    \begin{array}{@{}l@{}}
        \intass : 
        \begin{rclarray}[t]
        \C{xx} & : & \exsts{\V{v, k, a, b }} \Setcon{(\etR, \V{x}, \V{v}), (\etR, \V{y}, \V{k}), (\etW, \V{x}, \V{v} - 100)}{\V{v} + \V{k} \geq 100} \\
        & & \qqquad \mat \V{x} \pt \V{a} \sep \V{y} \pt \V{b} \oassto \V{x} \pt \V{a} \sep \V{y} \pt \V{b} \land \V{a} + \V{b} \geq 0 \\
        \C{yy} & : & \exsts{\V{v, k, a, b }} \Setcon{(\etR, \V{x}, \V{v}), (\etR, \V{y}, \V{k}), (\etW, \V{y}, \V{k} - 100)}{\V{v} + \V{k} \geq 100} \\
        & & \qqquad \mat \V{x} \pt \V{a} \sep \V{y} \pt \V{b} \oassto \V{x} \pt \V{a} \sep \V{y} \pt \V{b} \land \V{a} + \V{b} \geq 0 \\
        \C{Rd} & : & \exsts{\V{v, k, a, b }} \Set{(\etR, \V{x}, \V{v}), (\etR, \V{y}, \V{k})} \\
        & & \qqquad \mat \V{x} \pt \V{a} \sep \V{y} \pt \V{b} \oassto \V{x} \pt \V{a} \sep \V{y} \pt \V{b} \land \V{a} + \V{b} \geq 0 \\
        \end{rclarray} \\
    \end{array}
\]

\[
\begin{session}
\specline{ \boxass{ \vx \pt 60 \sep \vy \pt 60 }{\lrid}{\intass} \sep \cass{\C{xx}}{\lrid} \sep \cass{\C{yy}}{\lrid} } \\
\begin{parl}
    \begin{session}
        \specline{ \boxass{ \vx \pt 60 \sep \vy \pt 60 \lor \vx \pt 60 \sep \vy \pt -40 }{\lrid}{\intass} \sep \cass{\C{xx}}{\lrid} } \\
        \begin{transaction}
            \specline{\vx \pt 60 \sep \vy \pt 60 \\ {} \lor \vx \pt 60 \sep \vy \pt -40} \\
            \pderef{\pvar{a}}{\vx}; \\
            \pderef{\pvar{b}}{\vy}; \\
            \pifs{\pvar{a} + \pvar{b} \geq 100} \\
            \quad \pmutate{\vx}{\pvar{a} - 100} ; \\
            \pife \\
            \specline{\vx \pt-40 \sep \vy \pt 60 \sep {} \\
            \Set{(\etR, \vx, 60), (\etR, \vy, 60), (\etW, \vx, -40)} \\ 
            {} \lor \vx \pt 60 \sep \vy \pt -40 \sep {} \\
            \Set{(\etR, \vx, 60), (\etR, \vy, -40)} }
        \end{transaction} \\
        \specline{ \boxass{ \vx \pt -40 \sep \vy \pt 60 \lor \vx \pt 60 \sep \vy \pt -40 }{\lrid}{\intass} \sep \cass{\C{xx}}{\lrid} } \\
    \end{session}
    &
    \begin{session}
        \specline{ \boxass{ \vx \pt -40 \sep \vy \pt 60 \lor \vx \pt 60 \sep \vy \pt 60 }{\lrid}{\intass} \sep \cass{\C{yy}}{\lrid} } \\
        \begin{transaction}
            \pderef{\pvar{a}}{\vx}; \\
            \pderef{\pvar{b}}{\vy}; \\
            \pifs{\pvar{a} + \pvar{b} \geq 100} \\
            \quad \pmutate{\vy}{\pvar{b} - 100} ; \\
            \pife 
        \end{transaction} \\
        \specline{ \boxass{ \vx \pt -40 \sep \vy \pt 60 \lor \vx \pt 60 \sep \vy \pt -40 }{\lrid}{\intass} \sep \cass{\C{yy}}{\lrid} } \\
    \end{session}
\end{parl} \\
\specline{ \boxass{ \vx \pt -40 \sep \vy \pt 60 \lor \vx \pt 60 \sep \vy \pt -40 }{\lrid}{\intass} \sep \cass{\C{xx}}{\lrid} \sep \cass{\C{yy}}{\lrid} } \\
\end{session}
\]

\subsubsection{SI/PSI}
\[
    \begin{array}{@{}l@{}}
        \intass : 
        \begin{rclarray}[t]
        \C{xx} & : & \exsts{\V{v, k, a, b, c }} \Setcon{(\etR, \V{x}, \V{v}), (\etR, \V{y}, \V{k}), (\etW, \V{x}, \V{v} - 100)}{\V{v} + \V{k} \geq 100} \\
        & & \qqquad \mat \V{x} \pt \V{a} \sep \V{y} \pt \V{b} \oassto \V{x} \pt \V{a} \sep \V{y} \pt \V{b} + ( \V{c} \times 100 ) \\
        \C{yy} & : & \exsts{\V{v, k, a, b, c }} \Setcon{(\etR, \V{x}, \V{v}), (\etR, \V{y}, \V{k}), (\etW, \V{y}, \V{k} - 100)}{\V{v} + \V{k} \geq 100} \\
        & & \qqquad \mat \V{x} \pt \V{a} \sep \V{y} \pt \V{b} \oassto \V{x} \pt \V{a} + ( \V{c} \times 100 ) \sep \V{y} \pt \V{b} \\
        \C{Rd} & : & \exsts{\V{v, k, a, b, c, d }} \Set{(\etR, \V{x}, \V{v}), (\etR, \V{y}, \V{k})} \\
        & & \qqquad \mat \V{x} \pt \V{a} \sep \V{y} \pt \V{b} \oassto \V{x} \pt \V{a} + ( \V{c} \times 100 ) \sep \V{y} \pt \V{b} + ( \V{d} \times 100 ) \\
        \end{rclarray} \\
    \end{array}
\]

\[
\begin{session}
\specline{ \boxass{\vx \pt 60 \sep \vy \pt 60 }{\lrid}{\intass} \sep \cass{\C{xx}}{\lrid} \sep \cass{\C{yy}}{\lrid} } \\
\begin{parl}
    \begin{session}
        \specline{ \exsts{ \V{n} } \boxass{\vx \pt 60 \sep \vy \pt 60 - \V{n} \times 100 }{\lrid}{\intass} \sep \cass{\C{xx}}{\lrid} } \\
        \begin{transaction}
            \specline{ \exsts{ \V{n}, \V{k} \geq 0 } \vx \pt 60 + \V{k} \times 100 \sep \vy \pt 60 + \V{n} \times 100 } \\
            \pderef{\pvar{a}}{\vx}; \\
            \pderef{\pvar{b}}{\vy}; \\
            \specline{ \exsts{ \V{n}, \V{k} \geq 0 } \vx \pt 60 + \V{k} \times 100 \sep \vy \pt 60 + \V{n} \times 100 \\
                        {} \sep \Set{ (\etR, \vx, 60 + \V{k} \times 100), (\etR, \vx, 60 + \V{n} \times 100) } \\
                        {} \land \pvar{a} = 60 + \V{k} \times 100 \land \pvar{b} = 60 + \V{n} \times 100 } \\
            \pifs{\pvar{a} + \pvar{b} \geq 100} \\
            \quad \specline{ \exsts{ \V{n}, \V{k} \geq 0 } \vx \pt 60 + \V{k} \times 100 \sep \vy \pt 60 + \V{n} \times 100 \\
                            {} \sep \Set{ (\etR, \vx, 60 + \V{k} \times 100), (\etR, \vx, 60 + \V{n} \times 100) } \\
                            {} \land \pvar{a} = 60 + \V{k} \times 100 \land \pvar{b} = 60 + \V{n} \times 100 \land \V{k} + \V{N} \geq 0} \\
            \quad \pmutate{\vx}{\pvar{a} - 100} ; \\
            \quad \specline{ \exsts{ \V{n}, \V{k} \geq 0 } \vx \pt -40 + \V{k} \times 100 \sep \vy \pt 60 + \V{n} \times 100 \\
                            {} \sep \Set{ (\etR, \vx, 60 + \V{k} \times 100), (\etR, \vx, 60 + \V{n} \times 100), \\ 
                                                    (\etW, \vx, -40 + \V{k} \times 100) } \\
                            {} \land \pvar{a} = 60 + \V{k} \times 100 \land \pvar{b} = 60 + \V{n} \times 100 \land \V{k} + \V{N} \geq 0} \\
            \pife \\
            \comment{Weaken the assertion by } \\
            \comment{throwing away program variables.} \\
            \specline{ \exsts{ \V{n}, \V{k} \geq 0 } \vy \pt 60 + \V{n} \times 100 \\
                        {} \sep 
                        \begin{formulea}
                        \vx \pt -40 + \V{k} \times 100 
                        \land \V{k} + \V{N} \geq 0 \\
                        {} \sep \Set{ (\etR, \vx, 60 + \V{k} \times 100), (\etR, \vx, 60 + \V{n} \times 100), \\
                                                (\etW, \vx, -40 + \V{k} \times 100) } \\
                        \end{formulea} \\
                        {} \lor 
                        \begin{formulea}
                        \vx \pt 60 + \V{k} \times 100 \\ 
                        {} \sep \Set{ (\etR, \vx, 60 + \V{k} \times 100), (\etR, \vx, 60 + \V{n} \times 100) } 
                        \end{formulea}
                    } \\
        \end{transaction} \\
        \comment{To allow the write to be committed,} \\
        \comment{the \V{K} must be 0 } \\
        \specline{ \exsts{ \V{n} } \boxass{ ( \vx \pt 60 \lor \vx \pt -40 ) \sep \vy \pt 60 - \V{n} \times 100 }{\lrid}{\intass} \sep \cass{\C{xx}}{\lrid} } \\
    \end{session}
    &
    \begin{session}
        \specline{ \exsts{ \V{n} } \boxass{\vx \pt 60 - \V{n} \times 100 \sep \vy \pt 60 }{\lrid}{\intass} \sep \cass{\C{yy}}{\lrid} } \\
        \begin{transaction}
            \pderef{\pvar{a}}{\vx}; \\
            \pderef{\pvar{b}}{\vy}; \\
            \pifs{\pvar{a} + \pvar{b} \geq 100} \\
            \quad \pmutate{\vy}{\pvar{b} - 100} ; \\
            \pife 
        \end{transaction} \\
        \specline{ \exsts{ \V{n} } \boxass{\vx \pt 60 - \V{n} \times 100 \sep ( \vy \pt 60 \lor \vy \pt -40 ) }{\lrid}{\intass} \sep \cass{\C{yy}}{\lrid} } \\
    \end{session}
\end{parl} \\
\specline{ \boxass{\vx \pt -40 \sep \vy \pt 60 \lor \vx \pt 60 \sep \vy \pt -40 \lor \vx \pt -40 \sep \vy \pt -40 }{\lrid}{\intass} \sep \cass{\C{xx}}{\lrid} \sep \cass{\C{yy}}{\lrid} } \\
\end{session}
\]

%A dummy bank transfer example that is not serialisable.
%\[
    %\begin{rclarray}
        %\intass(\rid) & = &
        %\begin{cases}
            %\unitelem{} : \exsts{x, m, k} x \fpt{\fp} n \sep \cass{S(m)}{\rid} \transfersto x \fpt{\addFPW{\fp}} n \pm k \sep  \cass{S(m \pm k)}{\rid} \\
            %\unitelem{} : \exsts{x} x \fpt{\fp} n \transfersto x \fpt{\addFPR{\fp}} n 
        %\end{cases} \\
        %S(m) \composeK S(n) & = & S(m+n) \\
        %S(0) & \in & \unitK \\
    %\end{rclarray}
%\]
%A dummy bank transfer example that is serialisable even under snapshot isolation.
%\[
    %\begin{rclarray}
        %\intass & = &
        %\begin{cases}
            %\unitelem{} : \exsts{x, y, m, k} x \fpt{\fp} n \sep  y \fpt{\fp'} m \transfersto x \fpt{\addFPW{\fp}} n - k \sep \fpt{\addFPW{\fp}} m + k  \\
            %\unitelem{} : \exsts{x} x \fptEMP n \transfersto x \fptR n 
        %\end{cases}
    %\end{rclarray}
%\]
%If x write y and if y write x example.
%\[
    %\begin{rclarray}
        %\intass(x,y) & = &
        %\begin{cases}
            %\perm{L} : x \fptEMP 0 \sep y \fptEMP 0 \transfersto x \fptW 1 \sep y \fptR 0 \\
            %\perm{R} : x \fptEMP 0 \sep y \fptEMP 0 \transfersto x \fptR 0 \sep y \fptW 1 \\
        %\end{cases}
    %\end{rclarray}
%\]


\bibliography{bibliography,bibliography2}

%\newpage
\appendix
\subsection{semantics}
Let \( \repl \in \Repls \) denotes the set of totally ordered replicates.
Each replicate can have multiple clients, and 
each clients can commit a sequence of either read-only transitions or single-write transactions.
To model these, we annotate the transaction identifier with replicate \( \repl \), client \( \cl \), 
local time of the replicate \( n \) and read-only transactions count \( n' \), \ie \( \txidCOPS{\repl}{\cl}{n}{n'} \).
Note that the \( (n, \repl, n') \) can be treated as a single number that \( n \) are the higher bits, 
\( \repl \) the middle bits and \( n' \) the lower bits.
There is a total order among transitions from the same replica and from the same client.
We extend version with the set of all versions it dependencies on, \( \dep \in \pset{\Keys \times \TxID} \).
The function \( \depOf{\ver} \) denotes the dependencies set of the version.
For readability, we annotate view with either a replica, \( \viREPL \), or a client, \( \viCL \).
The view environment is extended with replicas and their views, \( \viewFunCOPS : (\Repls \times \ClientID ) \parfinfun \Views \).
We give the following semantics to capture the behaviours of the code.

\begin{lstlisting}[caption={put},label={lst:simplified-put}]
// mixing the client API and system API
put(repl,k,v,ctx) {

    // Dependency for previous reads and writes
    deps = ctx_to_dep(ctx);(*\label{line:put-ctx-to-deps}*)

    atomic{
        // increase local time.
        inc(repl.local_time);(*\label{line:put-inc-local}*) 

        // appending local kv with a new version.
        list_isnert(repl.kv[k],(v, (local_time + id), deps));(*\label{line:put-update-kv}*)
    }

    // update dependency for writes
    ctx.writers += (k,(local_time + id),deps);(*\label{line:put-update-ctx}*)

    // put in the queue to sync with other replicas
    enqueue(k,v,(current_ver+id),(deps ++ vers));
}
\end{lstlisting}

The client always fetches the version with the maximum writer it can observed for each key,
Which is computed by \( \funcn{getMax} \) function. 
It is different from \( \snapshot \) as \( \snapshot \) fetches the latest version with respect to the position in the list.

\[
    \begin{rclarray}
        \func{getMax}{\mkvsCOPS, \viCL} & \defeq &
        \lambda \ke \ldotp \left( \max_\txid\Setcon{(\val, \txid, \T, \dep)}{\exsts{i} (\val, \txid, \T, \dep) = \mkvsCOPS(\ke, i)} \right)\projection{1}
    \end{rclarray}
\]
\begin{mathpar}
    \inferrule[Put]{%
        ( \stk, \func{getMax}{\mkvsCOPS, \viCL}, \emptyset ), \pmutate{\ke}{\vx} \toL
        ( \stk', \stub, \Set{(\otW, \ke, \val )} ), \pskip
        \\\\
        \dep = \Setcon{(\ke', \txid)}{\exsts{i} i \in \viCL(\ke') \land \txid = \WTx(\mkvsCOPS(\ke', i))} \texttt{ ---> \cref{lst:simplified-put}, \cref{line:put-ctx-to-deps}} 
        \\\\
        \txid = \min\Setcon{%
        \txidCOPS{\repl}{\cl}{n'}{0}
        }{%
            \fora{\ke', i \in \viREPL(\ke'), n} \\
            \quad \txidCOPS{\stub}{\stub}{n}{\stub} = \WTx(\mkvsCOPS(\ke',i)) \\
            \qquad {} \implies n' > n 
        } \texttt{ ---> \cref{lst:simplified-put}, \cref{line:put-inc-local}}
        \\\\
        \mkvsCOPS' = \mkvsCOPS\rmto{\ke}{\mkvsCOPS(\ke) \lcat \List{(\ke, \txid, \emptyset, \dep)}} \texttt{ ---> \cref{lst:simplified-put}, \cref{line:put-update-kv}}
        \\\\
        \viREPL' = \viREPL\rmto{\ke}{\viREPL(\ke) \uplus \Set{\abs{\mkvsCOPS'(\ke)} - 1}} \texttt{ ---> \cref{lst:simplified-put}, \cref{line:put-update-kv}}
        \\\\
        \viCL' = \viCL\rmto{\ke}{\viREPL(\ke) \uplus \Set{\abs{\mkvsCOPS'(\ke)} - 1}} \texttt{ ---> \cref{lst:simplified-put}, \cref{line:put-update-ctx}}
    }{%
    \repl, \cl \vdash 
    \mkvsCOPS, \viREPL, \viCL, \stk, \ptrans{\pmutate{\ke}{\vx};} \toT{}
    \mkvsCOPS', \viREPL', \viCL', \stk', \pskip
    }
\end{mathpar}
The \verb|get_trans| fetches the latest versions from the replica via multiple atomic reads, one for each key.
As a result, a client has a list of candidates \verb|rst|.
Since interleaving might happen, versions might become out-of-date because the replicate receives new versions.
It is not a problem to read old versions as long as they satisfy causal consistency,
\ie if a client read a version \( \ver \), it should at least read all the versions that \( \ver \) depends on.
Thus the algorithm use \verb|ccv| to track the maximum versions the client should fetches,
and re-fetches the \verb|ccv[k]| version from the replica if it is greater than the candidate.

The following is a simplified algorithm by directly taking a list of versions \verb|ccv| satisfies causal consistency constraint,
and then read the versions indicated by \verb|ccv|.
The simplified algorithm is easier to understand.
\begin{lstlisting}[caption={get\_trans},label={lst:get-trans}]
// A simplified version by guessing
// a ccv satisfying dependency constraints
// and then read versions indicated by ccv.
// Note that it is a weaker version of the original code,
// as the original implementation fetches the latest versions
// for keys by a sequence of atomic get_by_version calls
List(Val) get_trans(ks,ctx) {
    take ccv: (*$\forall$*) k (*$\in$*) ks.(*\label{line:get-trans-ccv-1}*)
        (_,_,deps) := get_by_version(k,ccv[k]) (*${}\land \forall$*) dep (*$\in$*) deps.(*\label{line:get-trans-ccv-2}*)
            dep.key (*$\in$*) ks (*$\implies$*) ccv[dep.key] >= dep.ver (*\label{line:get-trans-ccv-3}*)

    for k in ks(*\label{line:get-trans-read-1}*)
        rst[k] = get_by_version(k,ccv[k]);(*\label{line:get-trans-read-2}*)

    // update the ctx
    for (k,ver,deps) in rst(*\label{line:get-trans-update-ctx-1}*)
        ctx.readers += (k,ver,deps);(*\label{line:get-trans-update-ctx-2}*)

    return to_vals(ks);
}                                   
\end{lstlisting}
\begin{mathpar}
    \inferrule[GetTrans]{%
        \viCL \viewleq \viCL' \viewleq \viREPL  \texttt{ ---> \cref{lst:get-trans}, \cref{line:get-trans-update-ctx-1,line:get-trans-update-ctx-2}}
        \\\\
        {\left(\begin{array}{@{}l@{}}
        \fora{i : 1 \leq i \leq j, \ke', m, \ver}  \\
        \quad \ver = \mkvsCOPS(\ke_i, \max(\viCL'(\ke_i)) \land {} \\
        \quad (\ke', \WTx(\mkvsCOPS(\ke', m))) \in \ver\projection{4} \\
        \qquad {} \implies m \in \viCL'(\ke')
        \end{array}\right)} \texttt{ ---> \cref{lst:get-trans}, \cref{line:get-trans-ccv-1,line:get-trans-ccv-2,line:get-trans-ccv-3}}
        \\\\
        \trans =  \pderef{\vx_1}{\ke_1}; \dots; \pderef{\vx_j}{\ke_j};
        \\\\
        ( \stk, \func{getMax}{\mkvsCOPS, \viCL'}, \emptyset ), \trans \toL
        ( \stk', \stub, \f ), \pskip \texttt{ ---> \cref{lst:get-trans}, \cref{line:get-trans-read-1,line:get-trans-read-2}}
        \\\\
        \txidCOPS{\repl}{\cl}{n'}{n} = \max\Setcon{\txidCOPS{\repl}{\cl}{z'}{z}}{\txidCOPS{\repl}{\cl}{z'}{z} \in \mkvsCOPS }
        \\
        \mkvsCOPS' = \updKV{\mkvsCOPS, \viCL', \txidCOPS{\repl}{\cl}{n'}{n+1}, \f} 
    }{%
        \repl, \cl \vdash 
        \mkvsCOPS, \viREPL, \viCL, \stk, \ptrans{\pderef{\vx_1}{\ke_1}; \dots; \pderef{\vx_j}{\ke_j}; } \toT{}
        \mkvsCOPS', \viREPL, \viCL', \stk', \pskip
    }
    \and
    \inferrule[ClientCommit]{%
        \repl, \cl \vdash 
        \mkvsCOPS, \viewFunCOPS(\repl), \viewFunCOPS(\cl), \stk, \prog(\cl) \toT{}
        \mkvsCOPS', \viREPL', \viCL', \stk', \cmd'
    }{%
        \mkvsCOPS, \viewFunCOPS, \thdenv, \prog \toG{}
        \mkvsCOPS', \viewFunCOPS\rmto{\repl}{\viREPL'}\rmto{\cl}{\viCL'}, \thdenv\rmto{\cl}{\stk'}, \prog\rmto{\cl}{\cmd'}
    }
\end{mathpar}
A replica updates its local state only if all the dependencies has been receive.
\begin{lstlisting}[caption={Send and receive},label={lst:send-receive}]
// Syn to other replicas
send() {
    (k,v,ver,deps) := dequeue();
    for id in repls {
        send (k,v,ver,deps) to id;
    }
}

// receive a write message from other replica
on_receive(k,v,ver,deps) {
    // for a single machine
    // the following check immediately holds
    for (k',ver') in deps {
        wait until dep_check(k',ver');(*\label{line:receive-wait}*)
    }

    atomic{
        list_isnert(kv[k],(v,ver,deps));(*\label{line:receive-update-view-1}*)
        (remote_local_time + id) = ver;(*\label{line:receive-update-view-2}*)
        local_time = max(remote_local_time, local_time);(*\label{line:receive-update-view-3}*)
    }
}
\end{lstlisting}
\begin{mathpar}
    \inferrule[sync]{%
        \viREPL = \viewFunCOPS(\repl)\rmto{\ke}{\viewFunCOPS(\repl)(\ke) \uplus i} 
        \texttt{ ---> \cref{lst:send-receive}, \cref{line:receive-update-view-1,line:receive-update-view-2,line:receive-update-view-3}}
        \\\\
        {\left(\begin{array}{@{}l@{}}
        \fora{\ke', m, \ver} 
        \ver = \mkvsCOPS(\ke, i) \land {} \\
        \quad (\ke', \WTx(\mkvsCOPS(\ke', m))) \in \ver\projection{4} \\
        \qquad {} \implies m \in \viREPL'(\ke') 
        \end{array}\right)} \texttt{ ---> \cref{lst:send-receive}, \cref{line:receive-wait}} 
    }{%
        \mkvsCOPS, \viewFunCOPS, \thdenv, \prog \toG{}
        \mkvsCOPS, \viewFunCOPS\rmto{\repl}{\viREPL}, \thdenv, \prog
    }
\end{mathpar}

A view \( \vi \) on key-value store \( \mkvsCOPS \) \emph{agrees} 
with another view \( \vi \) on another key-value store \( \mkvsCOPS' \), if and only
\[
 \func{getMax}{\mkvsCOPS, \vi} = \func{getMax}{\mkvsCOPS', \vi'}
\]



%\begin{theorem}
    %For any trace \( \tr \) of COPS with final configuration \( (\mkvsCOPS, \viewFunCOPS) \), 
    %there exists a trace \( \tr' \) with final configuration \( (\mkvsCOPS', \viewFunCOPS') \) such that 
    %each step of the trace \( \tr' \) commits a transaction with strictly greater transaction identifier than any one appearing in the key-value store:
    %\[
        %\begin{array}{@{}l@{}}
        %(\mkvsCOPS_i, \viewFunCOPS_i) 
        %\toG{} (\mkvsCOPS_{i+1}, \viewFunCOPS_{i+1}) 
        %\land \exsts{\txid} \txid \in \mkvsCOPS_{i+1} 
        %\land \txid \notin \mkvsCOPS_{i+1}
        %\implies \fora{\txid' \in \mkvsCOPS_i} \txid > \txid'
        %\end{array}
    %\]
    %and any replica's view from \( \viewFunCOPS \) agrees with its counterpart from  \( \viewFunCOPS' \):
    %\[
        %\fora{i} 
        %\func{getMax}{\mkvsCOPS, \viewFunCOPS(i)} = \func{getMax}{\mkvsCOPS', \viewFunCOPS'(i)}
    %\]
%\end{theorem}
%\begin{proof}
%\end{proof}

%\begin{lemma}
%\[
    %\begin{array}{@{}l@{}}
    %\fora{\repl, \cl, \mkvsCOPS, \mkvsCOPS', \mkvsCOPS'', \viREPL, \viREPL', \viREPL'', \viCL, \viCL', \viCL'', \stk, \stk', \cmd, \cmd'} \\
    %\quad \func{getMax}{\mkvsCOPS, \viREPL} = \func{getMax}{\mkvsCOPS'', \viREPL''} 
    %\land \func{getMax}{\mkvsCOPS, \viCL} = \func{getMax}{\mkvsCOPS'', \viCL''} \\
    %\qquad \repl, \cl \vdash 
    %\mkvsCOPS, \viREPL, \viCL, \stk, \cmd \toT{}
    %\mkvsCOPS', \viREPL', \viCL', \stk', \cmd' \\
    %\qquad \implies 
    %\exsts{\mkvsCOPS''', \viREPL'''}
    %\mkvsCOPS'', \viREPL'', \viCL'', \stk, \cmd \toT{}
    %\mkvsCOPS''', \viREPL''', \viCL', \stk', \cmd' \\
    %\qquad \func{getMax}{\mkvsCOPS', \viREPL'} = \func{getMax}{\mkvsCOPS''', \viREPL'''} 
    %\land \func{getMax}{\mkvsCOPS', \viCL'} = \func{getMax}{\mkvsCOPS''', \viCL'''} \\
    %\end{array}
%\]
%\end{lemma}
%\begin{proof}
%We perform case analysis.
%\begin{itemize}
    %\item \rl{Put}.
    %We have \( \cmd \equiv \ptrans{\pmutate{\ke}{\vx};} \) for some key \( \ke \) and variable \( \vx \).
    %Suppose key-value stores \(  \mkvsCOPS, \mkvsCOPS', \mkvsCOPS'' \), 
    %replica's views \( \viREPL, \viREPL', \viREPL''\) and client's views \( \viCL, \viCL', \viCL''\) such that
    %\[
    %\begin{array}{@{}l@{}}
    %\func{getMax}{\mkvsCOPS, \viREPL} = \func{getMax}{\mkvsCOPS'', \viREPL''} 
    %\land \func{getMax}{\mkvsCOPS, \viCL} = \func{getMax}{\mkvsCOPS'', \viCL''} \\
    %\qquad \repl, \cl \vdash 
    %\mkvsCOPS, \viREPL, \viCL, \stk, \cmd \toT{}
    %\mkvsCOPS', \viREPL', \viCL', \stk', \cmd' \\
    %\end{array}
    %\]
    %By the premiss of the \rl{Put} rule, the new key-value store
    %\[
        %\mkvsCOPS' = \mkvsCOPS\rmto{\ke}{\mkvsCOPS(\ke) \lcat \List{(\ke, \txid, \emptyset, \dep)}}
    %\]
    %where
    %\[
        %\txid = \min\Setcon{%
            %\txidCOPS{\repl}{\cl}{n'}{0}
        %}{%
            %\fora{\ke', i \in \viREPL(\ke'), n} \\
            %\quad \txidCOPS{\stub}{\stub}{n}{\stub} = \WTx(\mkvsCOPS(\ke',i)) \\
            %\qquad {} \implies n' > n 
        %} 
    %\]
    %and the new views of replica and client are
    %\[   
        %\begin{array}{@{}l@{}}
        %\viREPL' = \viREPL\rmto{\ke}{\viREPL(\ke) \uplus \Set{\abs{\mkvsCOPS'(\ke)} - 1}} \\
        %{} \land \viCL' = \viCL\rmto{\ke}{\viREPL(\ke) \uplus \Set{\abs{\mkvsCOPS'(\ke)} - 1}}
        %\end{array}
    %\]
    %Similarly there exists a new \( \mkvsCOPS''' \) by committing a single-write transaction \( \txid' \) and two new views \( \viREPL''' \) and \( \viCL''' \).
    %This means for those key \( \ke' \) that is different from the key \( \ke \) being overwritten,
    %\begin{equation}
        %\label{equ:get-max-match-all-other-key}
        %\begin{array}{@{}l@{}}
            %\func{getMax}{\mkvsCOPS', \viREPL'}(\ke') = \func{getMax}{\mkvsCOPS''', \viREPL'''}(\ke') \\
            %{} \land \func{getMax}{\mkvsCOPS', \viCL'}(\ke') = \func{getMax}{\mkvsCOPS''', \viCL'''}(\ke') 
        %\end{array}
    %\end{equation}
    %Note that the \( \txid \) is greater than any writers \( \txidCOPS{\repl}{\cl}{n'}{0} \) that can be observed by the \( \viREPL \), so is \( \txid' \).
    %That is,
    %\begin{equation}
        %\label{equ:get-max-match-overwritten-key}
        %\begin{array}{@{}l@{}}
            %\func{getMax}{\mkvsCOPS', \viREPL'}(\ke) = \stk(\vx) = \func{getMax}{\mkvsCOPS''', \viREPL'''}(\ke)  \\
            %{} \land \func{getMax}{\mkvsCOPS', \viCL'}(\ke) = \stk(\vx) = \func{getMax}{\mkvsCOPS''', \viCL'''}(\ke) 
        %\end{array}
    %\end{equation}
    %Combine \cref{equ:get-max-match-all-other-key} and \cref{equ:get-max-match-overwritten-key},
    %we have the proof.

    %\item \rl{GetTrans}.
    %Since the views of replica remain unchanged, so we only need to prove that there exists a new key-value store and a new view \( \viCL''' \) such that
    %\[
        %\func{getMax}{\mkvsCOPS', \viCL'}(\ke) = \stk(\vx) = \func{getMax}{\mkvsCOPS''', \viCL'''}(\ke) 
    %\]
    
%\end{itemize}
%\end{proof}

\begin{lemma}
    \label{lem:client-subset-repl}
    The view of a client is subset of the view of the replica that the client interacts with.
\end{lemma}


\begin{lemma}
    Let ignore the dependencies of versions from \( \mkvsCOPS \).
    Given the initial key-value store \( \mkvsCOPS_0 \), initial views \( \viewFunCOPS_0 \) and some programs \( \prog_0 \), for any \( \mkvsCOPS_i \) and \( \viewFunCOPS_i \)  such that: 
    \[
        \mkvsCOPS_0, \viewFunCOPS_0, \thdenv_0, \prog_0 {\toG{}}^* \mkvsCOPS_i, \viewFunCOPS_i, \thdenv_i, \prog_i
    \]
    The key-value store \( \mkvsCOPS_i \) satisfies the \cref{def:mkvs} and any replica or client view \( \vi \) from \( \viewFunCOPS_i \) is a valid view of the key-value store, \ie \( \vi \in \Views(\mkvsCOPS_i) \).
\end{lemma}
\begin{proof}
    We need to prove the  \( \mkvsCOPS_i \) satisfies the well-formed conditions,
    and any view \( \vi_i \Views(\mkvsCOPS_i) \).
    We prove it by introduction on the length \( i \).
    \begin{itemize}
    \item \caseB{\(i = 0\)}
        It holds trivially since each key only has the initial version \( (\val_0,\txid_0,\emptyset, \emptyset) \).
        Since there is only the initial version for each key, it is easy to see that any view \( \vi_0 \) satisfying the well-formed conditions in \cref{def:views}.
    \item \caseI{\(i > 0\)}
        Suppose it holds when \( i \), let consider \( i + 1 \).
        We perform case analysis on the possible next step:
        \begin{itemize}
            \item \rl{Put}
                Assume the client \( \cl \) of a replica \( \repl \) commits a single-write transaction \( \txid \) that installs a new version for key \( \ke \).
                By the premiss of \rl{Put}, the new transaction identifier \( \txid = \txidCOPS{\repl}{\cl}{n'}{0} \) where for some \( n' \) that is greater than any \( n \) from any writers \( \txidCOPS{\stub}{\stub}{n}{\stub} \) that are observable by the replica \( \repl \).
                Since the new transaction \( \txid = \txidCOPS{\repl}{\cl}{n'}{0} \) is a single-write transaction which is always installed at the end of the list associated to \( \ke \), it is sufficient to prove the following:
                \begin{gather}
                    \fora{j} 0 \leq j < \abs{ \mkvsCOPS_i(\ke) } \implies \WTx(\mkvsCOPS_{i}(\ke, m)) \neq \txid \label{equ:write-trans-unique} \\
                    \fora{j, n} \txidCOPS{\repl}{\cl}{n}{\stub} \in \Set{\WTx(\mkvsCOPS_{i}(\ke,j))} \cup \RTx(\mkvsCOPS_{i}(\ke, j)) \implies n < n' \label{equ:replica-time-monotonic-inc}
                \end{gather}
                By \cref{lem:repl-observe-own}, we know that for any version written \( \ver = \mkvsCOPS_i(\ke, j) \) by the same replica \( \txidCOPS{\repl}{\stub}{\stub}{\stub} = \WTx(\ver) \), such version is included in the replica's view \( j \in \viREPL(\ke) \).
                It implies that first the new transaction identifier is unique \cref{equ:write-trans-unique} and second it is greater than any transactions in the form of \( \txidCOPS{\repl}{\cl}{\stub}{\stub} \) \cref{equ:replica-time-monotonic-inc}.
                Thus the new key-value store \( \mkvsCOPS_{i+1} \) satisfies the well-formed conditions.
                Now let consider the views, especially the views of the replica \( \viREPL' \) and the client \( \viCL' \).
                Since that views \( \vi' \) from different replicas or clients remain unchanged, by \ih they satisfy \( \vi' \in \Views(\mkvsCOPS_{i+1}) \).
                The new view for replica \( \viREPL' = \viREPL\rmto{\ke}{\abs{\mkvsCOPS_{i+1}(\ke)} - 1} \)
                where \( \viREPL \) is the replica's view before updating and the writer of the last version of \( \ke \) is \( \txid \).
                Because \( \txid \) is a single-write transaction, so the new view \( \viREPL' \) still satisfies the atomic read.
                For similar reason, the new view for client \( \viCL' \) till satisfies atomic read.
                Therefore we have \( \viREPL', \viCL' \in \Views(\mkvsCOPS_{i+1}) \).
            \item \rl{GetTrans}

        \end{itemize}
    \end{itemize}
\end{proof}

\begin{lemma}
    \label{lem:repl-observe-own}
    A replica observes all its own transactions.
\end{lemma}

\section{Soundness}

\subsection{Compositionality of \( \ET \)}
\label{sec:et-comm}
\label{sec:et-comp}

To make two execution tests \( \ET_1 \) \( \ET_2 \) compositional with respect to to function \( \CMs \),
they need to satisfy \cref{def:conflict-commit,def:noblidwrites,def:et-minimum-footprint,def:et-monotonic-postview}.
For all the definitions we have in \cref{fig:execution.tests},
It is easy to adapt so that they satisfy \cref{def:noblidwrites,def:et-minimum-footprint,def:et-monotonic-postview},
but \( \CP \) and \( \SI \) cannot be adapted so to satisfy \cref{def:conflict-commit}.
Now we can prove compositionality of \( \ET \) (\cref{thm:et-comm}).

\begin{definition}[matching pre-views]
Two executions $\ET_1$ and $\ET_2$ have matching pre-views,

\end{definition}

\begin{theorem}                                                                            
\label{thm:et-comm}                          
Let $\ET_1, \ET_2$ be two execution tests has no blind writes, minimum footprints and monotonic post-views.
If $\ET_1$ is commutative, 
then $\CMs(\ET_1 \cap \ET_2) = \CMs(\ET_1) \cap \CMs(\ET_2)$. 
Furthermore, if $\ET_1, \ET_2$ are commutative, then $\ET_1 \cap \ET_2$ 
is commutative.
\end{theorem}
\begin{proof}
Given the definition of the \( \CMs(.) \) function (\cref{def:cm}), 
it suffices to prove that \( \CMs(\ET_{1} \cap \ET{2}) \subseteq \CMs(\ET_1) \cap \CMs(\ET_2) \)
and \( \CMs(\ET_1) \cap \CMs(\ET_2) \subseteq \CMs(\ET_{1} \cap \ET{2}) \).
The former is proven by the \cref{lem:et12-in-et1-et2} and the later is proven by \cref{lem:et1-et2-in-et12}.
\end{proof}

\begin{lemma}
\label{lem:et12-in-et1-et2}
\( \CMs(\ET_{1} \cap \ET_{2}) \subseteq \CMs(\ET_1) \cap \CMs(\ET_2) \).
\end{lemma}
\begin{proof}
It suffices to prove a stronger result that \( \confOf[\ET_{1} \cap \ET_{2}] \subseteq \confOf[\ET_1] \cap \confOf[\ET_2] \).
By the definition of \confOf (\cref{def:cm}), it suffices to prove for configurations \( \conf_0 \) to \( \conf_n \) 
\begin{equation}
    \label{equ:et12-in-et1-et2}
    \begin{array}{@{}l}
        \conf_0 \in \Confs_0
    \land \conf_0 \toET{\stub}[\ET_1 \cap \ET_2] \cdots \toET{\stub}[\ET_1 \cap \ET_2] \conf_n \implies {} \\
    \quad \conf_0 \toET{\stub}[\ET_{1}] \cdots \toET{\stub}[\ET_{1}] \conf_n \land \conf_0 \toET{\stub}[\ET_{2}] \cdots \toET{\stub}[\ET_{2}] \conf_n 
    \end{array}
\end{equation}
We prove the \cref{equ:et12-in-et1-et2} by induction on the number \( n \).
\begin{itemize}
\item Base case: \(n = 0\). 
The \cref{equ:et12-in-et1-et2} holds when \( n = 0 \), because all initial configurations \( \conf_0 \) are included in the \( \confOf[\ET_1]\) and \( \confOf[\ET_2] \) by the definition of the \( \confOf \) function (\cref{def:cm}).

\item Inductive case: \(n = i+1\). Suppose the \cref{equ:et12-in-et1-et2} holds when \( n = i \) for some \( i \).
Let consider \( n = i + 1 \) and specifically the last step.
For any \( \conf_{i+1} = (\mkvs_{i+1}, \vienv_{i+1}) \) induced by \( \ET_{1} \cap \ET_2 \), 
there exist some client \( \cl \), views \( \vi, \vi' \) and fingerprint \( \fp \) such that:
\[
    \begin{array}{l}
    (\mkvs_i, \vienv_i) \toET{\cl, \fp}[\ET_{1} \cap \ET_{2}] (\mkvs_{i+1}, \vienv_{i+1}) 
    \land \vienv_{i+1} = \vienv_{i}\rmto{\cl}{\vi'} \land (\mkvs_i, \vi, \fp, \vi' ) \in \ET_{1} \cap \ET_{2}
    \end{array}
\]
Thus, it is easy to see that \( \conf_i \toET{\cl, \fp}[\ET_{1}] \conf_{i+1} \) and \( \conf_i \toET{\cl, \fp}[\ET_{2}] \conf_{i+1} \) by the \cref{lem:mono-et}.
\end{itemize}
\end{proof}

\begin{lemma}
\label{lem:mono-et}
If $\conf \toET{\cl, \fp}[\ET] \conf'$ and $\ET \subseteq \ET'$, 
then $\conf \toET{\cl, \fp}[\ET'] \conf'$.
\end{lemma}
\begin{proof}
    Let \((\mkvs, \vienv)  = \conf \), \( (\mkvs', \vienv') = \conf' \) and \( \vi  =\vienv(\cl) \)
    By the definition of  $\conf \toET{\cl, \fp}[\ET] \conf'$ (\cref{def:cm}), we have \(\mkvs' \in \updateKV[\mkvs, \vi, \fp, \cl]\) and  \( \vienv' = \vienv\rmto{\cl}{\vi'} \) for some \( \vi' \) such that \( \ET \vdash (\mkvs, \vi) \csat \fp : (\mkvs',\vi') \).
    Given that \( \ET \subseteq \ET'\), we know \( \ET' \vdash (\mkvs, \vi) \csat \fp : (\mkvs',\vi') \) and so $\conf \toET{\cl, \fp}[\ET'] \conf'$.
\end{proof}

\begin{lemma}
\label{lem:et1-et2-in-et12}
\( \CMs(\ET_1) \cap \CMs(\ET_2) \subseteq \CMs(\ET_{1} \cap \ET_{2}) \).
\end{lemma}
\begin{proof}
    By the definition of \( \CMs\) and \( \confOf \) (\cref{def:cm}), we prove a stronger result that
    for an initial configuration \( \conf_0 \), 
    configurations \( \conf_1 \) to \( \conf_n \) from trace \( \ET_1 \), 
    configurations \( \conf'_1 \) to \( \conf'_m \) from trace \( \ET_2 \),
    \[
    \begin{array}{@{}l}
    \conf_0 \toET{\stub}[\ET_{1}] \conf_1 \toET{\stub}[\ET_{1}] \cdots \toET{\stub}[\ET_{1}] \conf_n 
    \land \conf_0 \toET{\stub}[\ET_{2}] \conf'_1 \toET{\stub}[\ET_{2}]  \cdots \toET{\stub}[\ET_{2}] \conf'_m 
    \land \conf_n\projection{1} = \conf'_m\projection{1} \\
    \end{array}
    \]
    there exists configurations from \( \conf''_1\)  to \( \conf''_k \) from trace \( \ET_1 \cap \ET_2 \):
\begin{equation}
    \label{equ:et1-et2-in-et12}
    \begin{array}{@{}l}
    \conf_0 \toET{\stub}[\ET_1 \cap \ET_2] \conf''_1 \toET{\stub}[\ET_1 \cap \ET_2] \cdots \toET{\stub}[\ET_1 \cap \ET_2] \conf''_k 
    \land \conf_n\projection{1} = \conf'_m\projection{1} = \conf''_k\projection{1}  \\
    \quad {} \land \fora{\cl \in \dom(\conf''_k\projection{2}),\key \in (\conf''_k\projection{1})}
    \conf''_k\projection{2}(\cl)(\key) = \max\Set{\conf_n\projection{2}(\cl)(\key), \conf'_m\projection{2}(\cl)(\key)}
    \end{array}
\end{equation}
We prove \cref{equ:et1-et2-in-et12} by induction on the length \( m \) of the trace of \( \ET_2 \).
\begin{itemize}
    \item \caseB{\(m = 0\)}
We have the trace of \( \ET_1 \):
\begin{equation}
    \label{equ:trace-view-shift-et1}
    \conf_0 \in \Confs_0 \land \conf_0 \toET{\stub}[\ET_1] \dots \toET{\stub}[\ET_1] \conf_n
\end{equation}
for some number \( n \) and configurations from \( \conf_0 \) to \( \conf_n \) and the trace of \( \ET_2 \) with only one configuration:
\begin{equation}
    \label{equ:trace-singleton-et2}
    \conf_0
\end{equation}
By the hypothesis we have \( \conf_0\projection{1} = \conf_n\projection{1} \), which means that all the steps from the trace of \( \ET_1 \) are view shift.
We can pick the trace of \( \ET_1 \) (\cref{equ:trace-view-shift-et1}) as the trace of \( \ET_1 \cap \ET_2 \):
\begin{equation}
    \label{equ:trace-view-shift-et12}
    \conf_0 \toET{\stub}[\ET_1 \cap \ET_2] \dots \toET{\stub}[\ET_1 \cap \ET_2] \conf''_k \land  k = n \land \bigwedge_{ 0 < i \leq k} \conf_i = \conf''_i
\end{equation}
It is easy to see:
\begin{equation}
    \label{equ:max-et1-et2}
    \begin{array}{l}
    \fora{\cl \in \dom(\conf_k\projection{2}), \key \in \dom(\conf_k\projection{1})} 
    \conf_0\projection{2}(\cl)(\key) = \max\Set{\conf_0\projection{2}(\cl)(\key), \conf_n\projection{2}(\cl)(\key)}
\end{array}
\end{equation}
Combine \cref{equ:trace-view-shift-et12} and \cref{equ:max-et1-et2}, we prove the \cref{equ:et1-et2-in-et12}.

\item \caseI{\(m = i + 1\)}
Suppose that \cref{equ:et1-et2-in-et12} holds when \( m = i \).
Let consider \( m = i + 1 \).
We have the trace for \( \ET_1 \):
\begin{equation}
    \conf_0 \toET{\stub}[\ET_{1}] \conf_1 \toET{\stub}[\ET_{1}] \cdots \toET{\stub}[\ET_{1}] \conf'_{n} 
\end{equation}
for some number \( n \) and the configurations from \(\conf_0\) to \( \conf_n \), and the trace of \(\ET_2\):
\begin{equation}
    \conf_0 \toET{\stub}[\ET_{2}] \conf'_1 \toET{\stub}[\ET_{2}] \cdots \toET{\stub}[\ET_{2}] \conf'_{i+1} 
\end{equation}
It is safe to assume these two traces are in normal form by \cref{prop:et.normalform}.
Assume a client \( \cl'_{i} \), views \( \vi'_{i}, \vi'_{i+1} \) and a fingerprint \( \fp'_{i} \) that commit to the second last configuration \( (\mkvs'_i, \vienv'_i) = \conf'_i \) in the trace of \( \ET_2 \) which yields the final configuration \( (\mkvs'_{i+1}, \vienv'_{i+1}) = \conf_{i+1} \):
\begin{equation}
    \label{equ:last-et-2}
    \begin{array}{@{}l @{}}
        (\mkvs'_i, \vienv'_i) \toET{\cl'_{i}, \fp'_{i}}[\ET_2] (\mkvs'_{i+1}, \vienv'_{i+1}) \land \ET_2 \vdash (\mkvs'_i, \vi'_i) \csat \fp'_i  : (\mkvs'_{i+1},\vi'_{i+1}) \\
        \quad {} \land \vi' = \vienv'_i(\cl'_i) \land \vienv'_{i+1} = \vienv'_i\rmto{\cl'_i}{\vi'_{i+1}}
    \end{array}
\end{equation}
There are three cases: \textbf{(i)} \( \fp'_i = \unitO \), \textbf{(i)} \( \fp'_i = \epsilon \) and \textbf{(ii)} \( \fp'_i \neq \unitO \land \fp'_i \neq \epsilon \).
\begin{itemize}
    \item If \( \fp'_i = \epsilon \) or \( \fp'_i = \unitO \), by the \cref{lem:no-effect-for-empty-fingerprint} we know \( \conf'_{i}\projection{1} = \conf'_{i+1}\projection{1}\) from the trace of \( \ET_2 \).
Since \( \conf'_{i+1}\projection{1} = \conf_n\projection{1}\) where \( \conf_n \) is the final configuration of the trace of \( \ET_1 \), we now have \( \conf'_{i}\projection{1} = \conf_n\projection{1}\).
Applying \ih that \cref{equ:et1-et2-in-et12} holds when \( m = i \), so there exist configurations from \( \conf''_1 \) to \( \conf''_k \):
\begin{equation}
    \label{equ:ih-for-k-length}
    \begin{array}{@{}l@{}}
    \quad \conf_0 \toET{\stub}[\ET_1 \cap \ET_2] \conf''_1 \toET{\stub}[\ET_1 \cap \ET_2] \cdots \toET{\stub}[\ET_1 \cap \ET_2] \conf''_k
    \land \conf_n\projection{1} = \conf'_i\projection{1} = \conf''_k\projection{1} \\
    \quad {} \land \fora{\cl \in \dom(\conf''_k\projection{2}),\key \in (\conf''_k\projection{1})} 
    \conf''_k\projection{2}(\cl)(\key) = \max\Set{\conf_n\projection{2}(\cl)(\key), \conf'_i\projection{2}(\cl)(\key)}
\end{array}
\end{equation}
Given the definition of the reduction (\cref{def:reduction}), when \( \fp = \epsilon \) or \( \fp = \unitO \) we know \( \conf'_i\projection{2}(\cl_{i+1}) \sqsubseteq  \conf'_{i+1}\projection{2}(\cl_{i+1})\) thus:
\begin{equation}
    \label{equ:preserve-max-view}
    \begin{array}{l}
    \max\Set{\conf_n\projection{2}(\cl_{i+1}), \conf'_i\projection{2}(\cl_{i+1})} 
    \sqsubseteq \max\Set{\conf_n\projection{2}(\cl_{i+1}), \conf'_{i+1}\projection{2}(\cl_{i+1})} 
    \end{array}
\end{equation}
Therefore \cref{equ:et1-et2-in-et12} holds when \( m = i + 1\) by appending a view shift to the end of the trace in \cref{equ:ih-for-k-length}:
\[
    \begin{array}{@{}l}
    \conf_0 \toET{\stub}[\ET_1 \cap \ET_2] \conf''_1 \toET{\stub}[\ET_1 \cap \ET_2] \cdots
    \toET{\stub}[\ET_1 \cap \ET_2] \conf''_k \toET{\cl_{j}, \epsilon }[\ET_1 \cap \ET_2] \\
    \qquad {} \land \conf''_k\rmto{2}{\conf''_k\projection{2}\rmto{\cl_{i}}{\max\Set{\conf_n\projection{2}(\cl_{i+1}), \conf'_{i+1}\projection{2}(\cl_{i+1})} }}
    \end{array}
\]

    \item If \( \fp'_i \neq \unitO  \land \fp' \neq \epsilon \), by \cref{lem:identical-step} there exists a step \( (\cl_j, \fp_j) \) from the trace of \( \ET_1 \) such that:
\begin{equation}
    \label{equ:j-th-step}
    \begin{array}{l}
    (\mkvs_{j}, \vienv_{j}) \toET{\cl_{j}, \fp_{j}}[\ET_1] (\mkvs_{j + 1}, \vienv_{j + 1}) 
    \land \ET_1 \vdash (\mkvs_{j}, \vi_j) \csat \fp_j : (\mkvs_{j+1},\vienv_{j + 1}(\cl_{j}) ) \land \vi_j = \vienv_{j}(\cl_j)
\end{array}
\end{equation}
for some \( j, \cl_j, \vi_j\) and \( \fp_j \) such that \( 0 \leq  j < n \), \( \cl_j = \cl'_{i}\), \( \fp_j = \fp'_{i}\), and
\[ 
    \fora{\key} (\stub, \key, \stub ) \in \fp_j \implies \vi_j(\key) = \vi'_{i}(\key)
\]
We apply the commutativity of \( \ET_1 \) until the step shown in \cref{equ:j-th-step} is at the end or the second end of the trace of \( \ET_1 \).
Let consider the next two steps, (j+1)-\emph{th} and (j+2)-\emph{th} step.
Since the trace is in normal form, the (j+1)-\emph{th} step is a view shift by a client \( \cl_{j+2} \) and (j+2)-\emph{th} step is a concrete step issued by the same client \( \cl_{j+2} \) under the view \( \vi_{j+2} \):
\begin{equation}
    \label{equ:j-plus-1-th-step}
    \begin{array}{@{}l@{}}
        (\mkvs_{j+1}, \vienv_{j+1}) \toET{\cl_{j+2}, \epsilon}[\ET_1]
        (\mkvs_{j+1}, \vienv_{j+1}\rmto{\cl_{j+2}}{\vi_{j+2}}) \toET{\cl_{j+2}, \fp_{j+2}}[\ET_1] (\mkvs_{j+3}, \vienv_{j+3}) \\
        \qquad \land \ET_1 \vdash (\mkvs_{j+1},\vi_{j+2}) \csat \fp_{j+2} : (\mkvs_{j+3},\vienv_{j+3}(\cl_{j+2}) )
    \end{array}
\end{equation}
It is known that the client  \( \cl_{j+2} \) is different from \( \cl_j \) (\cref{lem:different-cl}) and \( \fp_{j+2} \) writes different keys from \( \fp_j\) (\cref{lem:different-writes}). 
Because \( \cl_j \neq \cl_{j+2} \) we can swap the view shift step shown in \cref{equ:j-plus-1-th-step} before the j-\emph{th} step shown in \cref{equ:j-th-step} which gives the following:
\begin{equation}
    \label{equ:swap-the-view-shift-et1}
    \begin{array}{@{}l@{}}
    (\mkvs_{j}, \vienv_{j}) \toET{\cl_{j+2}, \epsilon}[\ET_1] (\mkvs_{j}, \vienv_{j}\rmto{\cl_{j+2}}{\vi_{j+2}}) \toET{\cl_{j}, \fp_{j}}[\ET_1] \\
    \quad (\mkvs_{j + 1}, \vienv_{j + 1}\rmto{\cl_{j+2}}{\vi_{j+2}}) \toET{\cl_{j+2}, \fp_{j+2}}[\ET_1] (\mkvs_{j+3}, \vienv_{j+3})
    \end{array}
\end{equation}
Now let discuss the (j+2)-\emph{th} step.
Similarly by the \cref{lem:identical-step}, there is a step \((\cl_p, \fp_p)\) from the trace of \( \ET_2 \) such that \( \cl_p = \cl_{j+2}\) and \( \fp_p = \fp_{j+2}\) and \( p < i \).
Note that the last step from \( \ET_2 \), \ie (i+1)-\emph{th} step, is not a view shift therefore the i-\emph{th} step must be a view shift so the p-\emph{th} step must be before  i-\emph{th} step.
This means the fingerprint \( \fp_p \) does not observe any change by (i+1)-\emph{th} step from the trace of \( \ET_2 \).
Therefore \( \vi_{j+2} \) does not observe any change by j-\emph{th} step from the trance of \( \ET_1\), \ie \( \vi_{j+2} \in \Views(\mkvs_j) \).
By \cref{prop:swap-update}, that allows to swap the two adjacent non-conflict steps from \cref{equ:swap-the-view-shift-et1}, \ie the last two steps.
It follows a new kv-stores \( \mkvs'''_{j+2}\) and a new view environment \( \vienv'''_{j+2} \) such that:
\begin{equation}
    \label{equ:swap-step-et1}
    \begin{array}{@{}l@{}}
    (\mkvs_{j}, \vienv_{j}) \toET{\cl_{j+2}, \epsilon}[\ET_1] (\mkvs_{j}, \vienv_{j}\rmto{\cl_{j+2}}{\vi_{j+2}}) \toET{\cl_{j+2}, \fp_{j+2}}[\ET_1] \\
    \quad (\mkvs_{j + 2}''', \vienv_{j + 2}''') \toET{\cl_{j+2}, \fp_{j+2}}[\ET_1] (\mkvs_{j+3}, \vienv_{j+3})
    \end{array}
\end{equation}
In the \cref{equ:swap-step-et1} the j-\emph{th} step moves to the right of (j+2)-\emph{th} step.
We monotonicly move the j-\emph{th} step until it is at the end or the second end of trace of \( \ET_1 \):
\[
    \begin{array}{@{}l}
        \conf_0 \toET{\stub}[\ET_{1}] \cdots \toET{\stub}[\ET_{1}] \conf_{j-1} \toET{\stub}[\ET_{1}]
        \conf'''_{j} \toET{\stub}[\ET_{1}] \dots \toET{\stub}[\ET_{1}] \conf'''_{n-1} \toET{\cl_j, \fp_j }[\ET_{1}] \conf_{n} \lor {} \\
        \conf_0 \toET{\stub}[\ET_{1}] \cdots \toET{\stub}[\ET_{1}] \conf_{j-1} \toET{\stub}[\ET_{1}] 
        \conf'''_{j} \toET{\stub}[\ET_{1}] \dots \toET{\stub}[\ET_{1}] \conf'''_{n-2} \toET{\cl_j, \fp_j }[\ET_{1}] \conf'''_{n-1} \toET{\cl_{n-1}, \epsilon }[\ET_{1}] \conf_{n}  \\ 
    \end{array}
\]
for some new configurations from \( \conf'''_{j}\) to \( \conf'''_{n-1} \).
Note that if it is the second end, the last step must be a view shift step as shown in \cref{equ:new-et-1}.
\begin{itemize}
    \item If the j-\emph{th} step is at the end of the new trace of \( \ET_1 \), we have the trace:
\begin{equation}
    \label{equ:new-et-1}
    \begin{array}{@{}l}
        \conf_0 \toET{\stub}[\ET_{1}] \cdots \toET{\stub}[\ET_{1}] \conf_{j-1} \toET{\stub}[\ET_{1}] 
        \conf'''_{j} \toET{\stub}[\ET_{1}] \dots \toET{\stub}[\ET_{1}] \conf'''_{n-1} \toET{\cl_j, \fp_j }[\ET_{1}] \conf_{n}  \\
    \end{array}
\end{equation}
Given the hypothesis that \( \conf_{n}\projection{1} = \conf'_{i+1}\projection{1} \) and the fact that the last step of the new trace of \( \ET_1 \) (\cref{equ:new-et-1}) and the last step the trace of \( \ET_2 \) (\cref{equ:last-et-2}) are the same step, the kv-stores of the second last configurations the new trace of \( \ET_1 \) (\cref{equ:new-et-1}) and the one from the trace of \( \ET_2 \) (\cref{equ:last-et-2}) are the same \(  \conf'''_{n-1}\projection{1} = \conf'_{i}\projection{1} \).
Then by applying \ih that \cref{equ:et1-et2-in-et12} holds when \( m = i \), there exists a trace of \( \ET_1 \cap \ET_2 \):
\begin{equation}
    \label{equ:ih-for-merge-two-trace}
    \begin{array}{@{}l}
        \conf_0 \toET{\stub}[\ET_1 \cap \ET_2] \dots \toET{\stub}[\ET_1 \cap \ET_2] \conf''_{k-1} 
        \land \conf'''_{n-1}\projection{1} = \conf'_{i}\projection{1} = \conf''_{k-1}\projection{1}  \\
        \quad {} \land \fora{\cl \in \dom(\conf''_{k-1}\projection{2}),\key \in (\conf''_{k-1}\projection{1})} 
        \conf''_{k-1}\projection{2}(\cl)(\key) = \max\Set{\conf_n\projection{2}(\cl)(\key), \conf'_i\projection{2}(\cl)(\key)}
\end{array}
\end{equation}
for some number \( k \) and configurations from \( \conf''_1 \) to \( \conf''_{k-1} \).
By \cref{equ:last-et-2} and \cref{equ:new-et-1}, we have:
\begin{equation}
    \label{equ:et1-et2-csat}
    \begin{array}{@{} l@{}}
        \ET_1 \vdash ( \conf'_{i}\projection{1}, \conf'_{i}\projection{2}(\cl_{i}) )  \csat \fp_{i} : ( \conf'_{i+1}\projection{1}, \conf'_{i+1}\projection{2}(\cl_{i}) ) \\
        \quad {} \land \ET_2 \vdash ( \conf'''_{n-1}\projection{1}, \conf'''_{n-1}\projection{2}(\cl_{i}) )  \csat \fp_{i} : (\conf_{n}\projection{1}, \conf_{n}\projection{2}(\cl_{i}) )
    \end{array}
\end{equation}
First, for any quadraple in \( \ET_1 \) and \( \ET_2 \), it does not constrain the view for keys that are not appear in the fingerprint before update.
That is:
\[
    \begin{array}{@{}l@{}}
    \fora{ \mkvs, \mkvs', \vi, \vi', \vi'', \fp, \key } 
    (\stub, \key, \stub) \in \fp \land \vi(\key) = \vi'(\key) \land (\mkvs, \vi, \fp, \mkvs', \vi'') \in \ET 
    \implies (\mkvs, \vi', \fp, \mkvs', \vi'') \in \ET
    \end{array}
\]
Given above and \cref{equ:ih-for-merge-two-trace}, we can substitute the configurations \( \conf'_{i} \) and  \( \conf'''_{n-1} \) from \cref{equ:et1-et2-csat} by \( \conf''_{k-1}\).
Then,  because for any \( \mkvs, \mkvs', \vi, \vi', \vi'' \) and \( \fp \), if \( (\mkvs, \vi, \fp, \mkvs', \vi' ) \in \ET_1 \) and \( \vi' \sqsubseteq \vi'' \) then \( (\mkvs, \vi, \fp, \mkvs', \vi'' ) \in \ET_1 \), and similarly for \( \ET_2 \).
It means:
\begin{equation}
    \label{equ:et12-csat}
    \begin{array}{l}
    \ET_1 \cap \ET_2 \vdash ( \conf''_{k-1}\projection{1}, \conf''_{k-1}\projection{2}(\cl_{i}) ) \csat
    \fp_{i} : ( \conf_{n}\projection{1}, \max\Set{\conf'_{i+1}\projection{2}(\cl_{i}), \conf_{n}\projection{2}(\cl_{i})} )
    \end{array}
\end{equation}
Therefore the \cref{equ:et1-et2-in-et12} holds when \( m = i + 1\) by appending the shown in \cref{equ:et12-csat} to the end of the trace shown in \cref{equ:ih-for-merge-two-trace}:
\[
\begin{array}{@{}l}
    \conf_0 \toET{\stub}[\ET_1 \cap \ET_2] \dots \toET{\stub}[\ET_1 \cap \ET_2] \conf''_{k-1} \toET{cl_{i}, \fp_{i}}[\ET_1 \cap \ET_2] \\
    \quad \left( \conf_n\projection{1},\conf''_{k-1}\projection{2}\rmto{\cl_{i}}{\max\Set{\conf'_{i+1}\projection{2}(\cl_{i}), \conf_{n}\projection{2}(\cl_{i})} } \right)
\end{array}
\]
    \item If the j-\emph{th} step is the second last step of the new trace of \( \ET_1 \), we have the trace:
\begin{equation}
    \label{equ:new-et-1-with-view-shift-tail}
    \begin{array}{@{}l}
        \conf_0 \toET{\stub}[\ET_{1}] \cdots \toET{\stub}[\ET_{1}] \conf_{j-1} \toET{\stub}[\ET_{1}] 
        \conf'''_{j} \toET{\stub}[\ET_{1}] \dots \\
        \quad {} \toET{\stub}[\ET_{1}] \conf'''_{n-2} \toET{\cl_j, \fp_j }[\ET_{1}] 
        \conf'''_{n-1} \toET{\cl_{n-1}, \epsilon }[\ET_{1}] \conf_{n}  \\ 
    \end{array}
\end{equation}
Since the last step is a view shift, we know \( \conf_n\projection{1} = \conf'''_{n-1}\projection{1}\), and the rest of proof is the same as the case where j-\emph{th} is the last step as shown in \cref{equ:new-et-1}.
\end{itemize}
\end{itemize}
\end{itemize}
\end{proof}

\begin{lemma}[No effect from empty fingerprint and epsilon reduction]
    \label{lem:no-effect-for-empty-fingerprint}
    \label{lem:no-effect-for-view-shift}
    \[
    \fora{\conf, \conf', \cl,\vi} \conf \toET{\cl, \unitO}[\ET] \conf' \lor \conf \toET{\cl, \epsilon}[\ET] \conf' \implies \conf\projection{1} = \conf'\projection{1}
    \]
\end{lemma}
\begin{proof}
    Let \((\mkvs, \vienv)  = \conf \) and \( (\mkvs', \vienv') = \conf' \).
    For the case of empty fingerprint,
    by the definition of  $\conf \toET{\cl, \unitO} \conf'$ (\cref{def:reduction}), we have \(\mkvs' \in \updateKV[\mkvs, \vi, \unitO, \cl]\), and therefore \( \mkvs' = \mkvs \).
    For the case of view shift, by the definition of  $\conf \toET{\cl, \epsilon} \conf'$ (\cref{def:reduction}) it is easy to see \( \mkvs' = \mkvs \).
\end{proof}

We define a \(  \mkvs(\txid) \) function that returns the fingerprint associate with the transaction identifier \( \txid \):
\[
    \begin{rclarray}
        \mkvs(\txid) & \defeq & \Setcon{(\otW, \key, \val)}{\exsts{i} \mkvs(\key)(i) = (\val, \txid, \stub)} \cup  \Setcon{(\otR, \key, \val)}{\exsts{i,\txidset} \mkvs(\key)(i) = (\val, \stub, \txidset) \land \txid \in \txidset}
    \end{rclarray}
\]

\begin{lemma}[Transactions persistence]
    \label{lem:mono-fingerprint}
    \[
        \fora{\ET,\conf,\conf',\txid,\fp} \conf\projection{1}(\txid) = \fp \land \conf \toET{\stub}[\ET] \conf' \implies \conf'\projection{1}(\txid) = \fp
    \]
\end{lemma}
\begin{proof}
    It is easy to prove this by case analysis on the reduction relation.
\end{proof}

\begin{lemma}[Same steps]
\label{lem:identical-step}
Given a trace of \( \ET_1 \) and a trace of \( \ET_2 \),
if the have the same final kv-store,
the trace contains the same concrete steps (free variables are globally quantified):
\[
\begin{array}{@{}l}
    \conf_0 \toET{\cl_1, \fp_1}[\ET_{1}] \cdots \toET{\cl_n, \fp_n}[\ET_{1}] \conf_n \land
    \conf_0 \toET{\cl'_1, \fp'_1}[\ET_{2}] \cdots \toET{\cl'_m, \fp'_m}[\ET_{2}] \conf'_m 
    \land \conf_n\projection{1} = \conf'_m\projection{1} \\
    \quad \implies \fora{i: 0 < i \leq n} 
    \fp_i = \unitO 
    \lor \fp_i = \epsilon 
    \lor \exsts{j: 0 < j \leq m} 
    \cl_i = \cl'_j \land \fp_i = \fp'_j \land ( \fora{\key} (\stub, \key, \stub) \in \fp_i \implies \vi_i(\key) = \vi'_j(\key) )
\end{array}
\]
\end{lemma} 
\begin{proof}
    We prove by contradiction.
    First because \( \mkvs_n = \mkvs'_m \), we know that:
    \begin{equation}
        \label{equ:same-kv-store}
        \fora{\txid, \fp} \mkvs_n(\txid) = \fp \iff \mkvs'_m(\txid) = \fp
    \end{equation}
    Let \(\conf_n = (\mkvs_n,\vienv_n) \) and \(\conf'_m = (\mkvs'_m,\vienv'_m) \).
    Assume a step \( \conf_i \toET{\cl, \fp }[\ET_1] \conf_{i+1} \)  from the trace of \( \ET_1 \) where the transaction identifier is \( \txid \) and \( \fp \neq \unitO \).
    It must have a step from the trace of \( \ET_2 \), which commits some fingerprint via the same transaction identifier  \( \txid \).
    We know \( \mkvs_n(\txid) = \fp \) by \cref{lem:mono-fingerprint}, thus \( \mkvs'_m(\txid) = \fp \) by \cref{equ:same-kv-store}.
    Let assume a key \( \key \) that \( (\stub, \key, \stub) \in \fp \land \vi_i(\key) \neq \vi'_j(\key)\) where \( \vi_i\) and \( \vi'_j\) are the views immediate before the commit of the fingerprint \( \fp \) in traces of \( \ET_1\) and \( \ET_2 \) respectively.
    Since the no blind write assumption, it is safe to assume it is a read operation on the key \( \key \).
    By the definition of the reduction (\cref{def:reduction}) and \cref{lem:mono-fingerprint}, we know \( \func{read}{\mkvs_n(\key)(\vi_i(\key))} \neq \func{read}{\mkvs'_m(\key)(\vi_i(\key))} \), which contradicts with \( \mkvs_n = \mkvs'_m \).
\end{proof}

We define \( \max_\cl(\conf) \) function that returns the most recent transaction identifier for client \( \cl \) in the configuration \( \conf \) 
\[
\begin{rclarray}
    \max_\cl((\mkvs, \vienv)) & \defeq & \max\Setcon{\txid^{n}_\cl}{\txid^{n}_\cl \text{ appear in } \mkvs} \\
\end{rclarray}
\]

\begin{lemma}[Transactions from different clients]
\label{lem:different-cl}
Given a trace of \( \ET_1 \) and a trace of \( \ET_2 \),
if the i-\emph{th} step from \( \ET_1 \) issued by the client \( \cl_i \) 
is the same as the last step from \( \ET_2 \),
then in the trace of \( \ET_1 \) 
there is no concrete step issued by the client \(\cl_i \) after the i-\emph{th} step (free variables are globally quantified):
\[
\begin{array}{@{}l}
    \conf_0 \toET{\cl_1, \fp_1}[\ET_{1}] \cdots \toET{\cl_n, \fp_n}[\ET_{1}] \conf_n 
    \land \conf_0 \toET{\cl'_1, \fp'_1}[\ET_{2}] \cdots \toET{\cl'_m, \fp'_m}[\ET_{2}] \conf'_m 
    \land \conf_n\projection{1} = \conf'_m\projection{1} 
    \land \fp'_m \neq \unitO \\
    \quad {} \land \exsts{i}  
    \cl_i = \cl'_m
    \land \fp_i = \fp'_m 
    \implies \fora{j > i} 
    \fp_j = \epsilon \lor \fp_j = \unitO \lor \cl_i \neq \cl_j
\end{array}
\]
\end{lemma}
\begin{proof}
    We prove by deriving contradiction.
    Assume the last step of the trace of \( \ET_2 \) is:
    \begin{equation}
        \label{equ:last-step-for-cl-et2}
        \conf'_{m-1} \toET{\cl'_m, \fp'_m}[\ET_{2}] \conf'_m
    \end{equation}
    Assume a step of the trace of \( \ET_1 \):
    \begin{equation}
        \label{equ:identical-step-for-cl-et1}
        \conf_{i-1} \toET{\cl_i, \fp_i}[\ET_{1}] \conf_i
    \end{equation}
    where \( \cl_i = \cl'_m \) and \( \fp_i = \fp'_m \).
    Because these two steps (\cref{equ:last-step-for-cl-et2} and \cref{equ:identical-step-for-cl-et1}) are issued by the same transaction identifier,
    we know \( \max{}_{\cl_m}(\conf'_m) = \max{}_{\cl_m}(\conf_i) \).
    Assume that there exists a step from the trace of \( \ET_1 \), says j-\emph{th} step, such that:
    \[
        \conf_{j-1} \toET{\cl_j, \fp_h}[\ET_{1}] \conf_j \land j > i \land \fp_j \neq \unitO \land \cl_i = \cl_j 
    \]
    Therefore we have \( \max{}_{\cl_m}(\conf_j) > \max{}_{\cl_m}(\conf_i) \) by \cref{lem:kv-max-cl}.
    That means \( \max{}_{\cl_m}(\conf_n) > \max{}_{\cl_m}(\conf_j) > \max{}_{\cl_m}(\conf_i) = \max{}_{\cl_m}(\conf'_m) \), which contradicts to \( \conf_n\projection{1} = \conf'_m\projection{1}\).
\end{proof}

\begin{lemma}[Reduction following session order]
\label{lem:kv-max-cl}
\[
\begin{array}{@{}l}
    \fora{\conf, \conf' ,\cl, \fp, \ET}
    \conf \toET{\cl, \fp}  \conf' 
    \land 
    \left( 
        \begin{array}{l}
        \fp \neq \unitO \implies \max{}_\cl(\conf) < \max{}_\cl(\conf') )
        \lor ( \fp = \unitO \implies \max{}_\cl(\conf) = \max{}_\cl(\conf')
        \end{array}
    \right)
\end{array}
\]
\end{lemma}
\begin{proof}
    Assume a step \( (\mkvs, \vienv) \toET{\cl, \fp} (\mkvs', \vienv') \).
    By the definition of \( \toET{\stub}[\ET]\) (\cref{def:reduction}), we know \( \mkvs' \in \updateKV[\mkvs, \vi, \fp, \cl] \).
    The \( \updateKV[\mkvs, \vi, \fp, \cl] \) picks a fresh transaction identifier \( \txid_\cl^{m} \) that is greater than any transaction identifiers \( \txid_\cl^{n} \) in \( \mkvs \) via \( \nextTxid \) function, \ie \( m > n \).
    If the fingerprint \( \fp \) is not empty, the new identifier appears in \( \mkvs' \), so \( \max{}_\cl(\conf) < \max{}_\cl(\conf') \).
    Otherwise  the fingerprint is empty, the new identifier will not appear anywhere in \( \mkvs' \), so \( \max{}_\cl(\conf) = \max{}_\cl(\conf') \). 
\end{proof}

\begin{lemma}[Writing different keys]
\label{lem:different-writes}
Given a trace of \( \ET_1 \) and a trace of \( \ET_2 \),
if the i-\emph{th} step from \( \ET_1 \) that writes to key \( \key \) 
is the same as the last step from \( \ET_2 \),
then in the trace of \( \ET_1 \) 
there is no concrete step writing to the key \(\key\) after the i-\emph{th} step (free variables are globally quantified):
\[
\begin{array}{@{}l}
    \conf_0 \toET{\cl_1, \fp_1}[\ET_{1}] \cdots \toET{\cl_n, \fp_n}[\ET_{1}] \conf_n \land \conf_0 \toET{\cl'_1, \fp'_1}[\ET_{2}] \cdots \toET{\cl'_m, \fp'_m}[\ET_{2}] \conf'_m 
    \land \conf_n\projection{1} = \conf'_m\projection{1} 
    \land \fp'_m \neq \unitO \\
    \quad {} \land \exsts{i} 
    \cl_i = \cl'_m
    \land \fp_i = \fp'_m
    \implies \fora{j > i} \nexists{\key} \ldotp (\otW, \key, \stub) \in \fp_j \cap \fp_i
\end{array}
\]
\end{lemma}
\begin{proof}
    We prove this by deriving contradiction.
    Assume the last step from the trace of \( \ET_2 \):
    \begin{equation}
        \label{equ:last-step-for-write-et2}
        \conf'_{m-1} \toET{\cl'_m, \fp'_m}[\ET_{2}] (\mkvs'_m, \vienv'_m )
    \end{equation}
    Assume the transaction identifier for the \cref{equ:last-step-for-write-et2} is \( \txid \), and by the definition of \( \toET{}[\ET]\) (\cref{def:reduction}) we know:
    \begin{equation}
        \label{equ:write-fingerprint}
        \fora{\key} (\otW, \key, \stub) \in \fp'_m \implies \mkvs'_m(\key)(\lvert\mkvs'_m(\key)\rvert - 1) = (\stub, \txid, \stub)
    \end{equation}
    Assume a step of the trace of \( \ET_1 \) that is issued by the same transaction identifier with the same fingerprint:
    \begin{equation}
        \label{equ:identical-step-for-write-et1}
        \conf_{i-1} \toET{\cl_i, \fp_i}[\ET_{1}] (\mkvs_i, \vienv_i)
    \end{equation}
    where \( \cl_i = \cl'_m \) and \( \fp_i = \fp'_m \).
    Given \cref{equ:write-fingerprint} and \cref{equ:identical-step-for-write-et1}, it follows:
    \[
        \fora{\key} (\otW, \key, \stub) \in \fp_i \implies \mkvs_i(\key)(\lvert\mkvs_i(\key)\rvert - 1) = (\stub, \txid, \stub)
    \]
    Assume a step, says j-\emph{th}, after i-\emph{th} step that writes to the same key:
    \[
        \conf_{j-1} \toET{\cl_j, \fp_j}[\ET_{1}] (\mkvs_j, \vienv_j) 
        \land j > i
        \land \exsts{\key} (\otW, \key, \stub) \in \fp_i \cap \fp_j
    \]
    Therefore, by \cref{lem:unique-writer} we have:
    \[
        \exsts{\key,i} (\otW, \key, \stub) \in \fp_i \cap \fp_j \land \mkvs_j(\key)(i) = \txid \land \mkvs_j(\key)(\lvert \mkvs_j(\key) \rvert - 1) \neq \txid
    \]
    Note that \( \fp_i = \fp'_m\).
    Since the writer of a version cannot be overwritten, for the final configuration of the trace of \( \ET_1 \) \((\mkvs_n, \vienv_n)\), we know:
    \[
        \exsts{\key,i} (\otW, \key, \stub) \in \fp_i \cap \fp_j \land \mkvs_n(\key)(i) = \txid \land \mkvs_n(\key)(\lvert \mkvs_n(\key) \rvert - 1) \neq \txid
    \]
    Last, by \cref{equ:write-fingerprint} and \( \fp_i = \fp'_m\), it follows:
    \[
        \exsts{\key} (\otW, \key, \stub) \in \fp'_m \land \mkvs_n(\key)(\mkvs_n(\key) - 1) \neq \txid \land \mkvs'_m(\key)(\lvert\mkvs'_m(\key)\rvert - 1)\projection{2} = \txid
    \]
    which contradicts with \( \mkvs'_m = \mkvs_n\).
\end{proof}

\begin{lemma}[Version persistence]
    \label{lem:unique-writer}
    \[
    \begin{array}{@{}l}
        \fora{\mkvs, \mkvs',\vienv,\vienv', \cl, \vi, \fp, i} 
        (\mkvs, \vienv) \toET{\cl, \fp}[\ET] (\mkvs', \vienv')
        \land (\otW, \key, \stub) \in \fp  
        \land 0 \leq i < \lvert \mkvs'(\key) \rvert - 1 \\
        \quad {} \implies \mkvs'(\key)(i)\projection{2} \neq \mkvs'(\key)(\lvert \mkvs'(\key) \rvert - 1)\projection{2}
    \end{array}
    \]
\end{lemma}
\begin{proof}
    By the definition of \( \toET{\stub}[\ET] \) (\cref{def:reduction}), the \( \mkvs' \in \updateKV[\mkvs, \vi, \fp, \cl] \).
    Given the definition of \( \updateKV[\mkvs, \vi, \fp, \cl]\), it the picks a fresh transaction identifier \( \txid \) such that does not appear in \( \mkvs \).
    For any write fingerprint \( (\otW, \key, \stub) \in \fp \), a new version is appended to the end of the key \( \key \) and the writer (the second projection) is assigned to be the fresh identifier \( \txid \).
    Thus we have the proof.
\end{proof}

\begin{proposition}
\label{thm:appendix-et-composition-2}
\label{prop:appendix-et-composition-2}
if $\ET_1, \ET_2$ are commutative, then $\ET_1 \cap \ET_2$ is commutative.
\end{proposition}
\begin{proof}
Let \( \ET_{12} = \ET_1 \cap \ET_2 \).
Assume \(\conf_1, \conf_2, \conf_3, \cl, \cl', \vi, \vi', \fp, \fp' \) such that:
\[
    \conf_1 \toET{\cl, \fp}[\ET_{12}] \conf_2 \toET{\cl', \fp'}[\ET_{12}] \conf_3
\]
Therefore, we have:
\[
    \conf_1 \toET{\cl, \fp}[\ET_{1}] \conf_2 \toET{\cl', \fp'}[\ET_{1}] \conf_3 \land 
    \conf_1 \toET{\cl, \fp}[\ET_{2}] \conf_2 \toET{\cl', \fp'}[\ET_{2}] \conf_3
\]
Because \( ET_1 \)  and \( \ET_2 \) are commutative, there exists a configuration \( \conf'_2 \) such that:
\[
    \conf_1 \toET{\cl', \fp'}[\ET_{1}] \conf'_2 \toET{\cl, \fp}[\ET_{1}] \conf_3 \land 
    \conf_1 \toET{\cl', \fp'}[\ET_{2}] \conf'_2 \toET{\cl, \fp}[\ET_{2}] \conf_3
\]
so we have the proof that: 
\[
    \conf_1 \toET{\cl', \fp'}[\ET_{12}] \conf'_2 \toET{\cl, \fp}[\ET_{12}] \conf_3
\]
\end{proof}


\end{document}
