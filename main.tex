%% For double-blind review submission, w/o CCS and ACM Reference (max submission space)
\documentclass[acmsmall,review,anonymous]{acmart}\settopmatter{printfolios=true,printccs=false,printacmref=false}
%% For double-blind review submission, w/ CCS and ACM Reference
%\documentclass[acmsmall,review,anonymous]{acmart}\settopmatter{printfolios=true}
%% For single-blind review submission, w/o CCS and ACM Reference (max submission space)
%\documentclass[acmsmall,review]{acmart}\settopmatter{printfolios=true,printccs=false,printacmref=false}
%% For single-blind review submission, w/ CCS and ACM Reference
%\documentclass[acmsmall,review]{acmart}\settopmatter{printfolios=true}
%% For final camera-ready submission, w/ required CCS and ACM Reference
%\documentclass[acmsmall]{acmart}\settopmatter{}


%% Journal information
%% Supplied to authors by publisher for camera-ready submission;
%% use defaults for review submission.
\acmJournal{PACMPL}
\acmVolume{1}
\acmNumber{CONF} % CONF = POPL or ICFP or OOPSLA
\acmArticle{1}
\acmYear{2018}
\acmMonth{1}
\acmDOI{} % \acmDOI{10.1145/nnnnnnn.nnnnnnn}
\startPage{1}

%% Copyright information
%% Supplied to authors (based on authors' rights management selection;
%% see authors.acm.org) by publisher for camera-ready submission;
%% use 'none' for review submission.
\setcopyright{none}
%\setcopyright{acmcopyright}
%\setcopyright{acmlicensed}
%\setcopyright{rightsretained}
%\copyrightyear{2018}           %% If different from \acmYear

%% Bibliography style
\bibliographystyle{ACM-Reference-Format}
%% Citation style
%% Note: author/year citations are required for papers published as an
%% issue of PACMPL.
\citestyle{acmauthoryear}   %% For author/year citations



\newif\ifNonACMMode
\NonACMModefalse
% for theorem, proof, etc.
\usepackage{amsthm}
\theoremstyle{definition}
\newtheorem{thm}{Theorem}[section]
\newtheorem{defn}[thm]{Definition}
\newtheorem{param}[thm]{Parameter}
\newtheorem{lem}[thm]{Lemma}
\newtheorem{prop}[thm]{Proposition}
\newtheorem*{cor}{Corollary}
\newtheoremstyle{case}{}{}{}{}{}{:}{ }{}
\theoremstyle{case}
\newtheorem{case}{Case}

% the inter command for operational semantices
\usepackage{proof}
\usepackage{color}

\usepackage{centernot}

\usepackage{amsmath,amssymb,stmaryrd}
\usepackage{dsfont}

% For the box assertion
\usepackage{varwidth}

\usepackage{hyperref}
\hypersetup{
    colorlinks,
    citecolor=black,
    filecolor=black,
    linkcolor=black,
    urlcolor=black
}
\usepackage[usenames,dvipsnames,svgnames,table]{xcolor}
\usepackage{enumitem}
\usepackage{translang}
\lstset{language=translang}
\usepackage[margin=2cm]{caption}
\usepackage{tikz}
\usetikzlibrary{positioning, shapes, decorations}
\usepackage{bold-extra}
\usepackage{titlesec}

\titleformat{\chapter}{\bfseries\Huge}{\thechapter.}{1ex}{\Huge}

\def\changemargin#1#2{\list{}{\rightmargin#2\leftmargin#1}\item[]}
\let\endchangemargin=\endlist
\usepackage{hhmacros}
\pgfdeclarelayer{main}
\pgfdeclarelayer{background}
\pgfdeclarelayer{foreground}
\pgfsetlayers{background,main,foreground}

\newcommand{\greyness}{gray!40}
\newcommand{\blueness}{cyan!60}

\tikzstyle{background}=[rectangle, draw=black, inner sep=0.2cm, rounded corners=1.2mm]
\tikzstyle{white}=[rectangle, fill=white, inner sep=0.5cm, rounded corners=5mm]

%\tikzstyle{background}=[circle, fill=\greyness,
%                                                inner sep=0.2cm,
%                                                rounded corners=5mm,
%                                                decorate,
%                                                decoration={random steps,
%                                                            segment length=3pt,
%                                                            amplitude=3pt}]

%

 \tikzstyle{hheapcell}=[rectangle, draw=black, inner sep=0.1cm, font=\small]

\tikzstyle{noise}=[circle, thick, minimum size=1.2cm, draw=yellow!85!black, fill=yellow!40, decorate, decoration={random steps, segment length=2pt, amplitude=2pt}]

%\pgfdeclarelayer{background}
%\pgfdeclarelayer{foreground}
%\pgfsetlayers{background,main,foreground}

\tikzstyle{abstract}=[draw, fill=white, text width=5em, text centered, minimum height=2.5em, rounded corners]
    
\tikzstyle{arr}=[draw, ->, thick, color=black]
\tikzstyle{dasharr}=[draw,->,thick,dashed,color=black]

\tikzset{
    version node/.style={
        rectangle,
        draw=black,
        align=center,
        minimum height=5mm,
        text depth=0.5ex,
        text height=2ex,
        inner xsep=0pt,
        outer sep=0pt, 
        font=\footnotesize
    },      
    version list/.style={
        matrix of nodes,
        row sep=-\pgflinewidth,
        column sep=-\pgflinewidth,
        nodes={
            version node
        }
        ,
        execute at empty cell={\node[draw=none]{};},
        text width=5mm,
        anchor=west
    }
}

\newcommand{\tikzvalue}[4]{
    \node[version node, fit=(#1) (#2), fill=white, inner sep=0pt] (#3) {#4}
}
\newcommand{\tikzkvspace}{1.5pt}
\newcommand{\tikzkeyspace}{-1.1}
\newenvironment{halfsubfig}{%
    \begin{subfigure}{0.45\textwidth}
}{%
    \end{subfigure}
}
\newenvironment{onethirdsubfig}{%
    \begin{subfigure}{0.3\textwidth}
}{%
    \end{subfigure}
}
\newenvironment{centertikz}{%
    \begin{center}%
    \begin{tikzpicture}[every node/.style={inner sep=0,outer sep=0},font=\footnotesize]%
}{%
    \end{tikzpicture}%
    \end{center}%
}
\newcommand{\tikzresize}{.8}

\PassOptionsToPackage{svgnames}{xcolor}
\definecolor{DarkGreen}{rgb}{0, 0.5, 0}


%%%%%%%%%%%%%%%%%%%%%% edit mode
\newif\ifCommentEdits
\CommentEditstrue

\newcommand{\pg}[1]{%
\ifComments
\begin{center}
\fbox{%
\begin{minipage}{6.5in} \color{red}
{\bf PG:} {\rm #1}
\end{minipage}
}
\end{center}
\fi
}

\newcommand{\sx}[1]{%
\ifComments
\begin{center}
\fbox{%
\begin{minipage}{6.5in} \color{blue}
{\bf SX:} {\rm #1}
\end{minipage}
}
\end{center}
\fi
}

\definecolor{darkred}{rgb}{0.5, 0, 0}
\newcommand{\azalea}[1]{%
\ifComments
\begin{center}
\fbox{%
\begin{minipage}{6.5in} \color{darkred}
{\bf AR:} {\rm #1}
\end{minipage}
}
\end{center}
\fi
}

\newcommand{\ac}[1]{%
\ifComments
\begin{center}
\fbox{%
\begin{minipage}{6.5in} \color{green}
{\bf SX:} {\rm #1}
\end{minipage}
}
\end{center}
\fi
}

%%%%%%%%%%%%%%%%%%%%%% end edit mode


%% Journal information
%% Supplied to authors by publisher for camera-ready submission;
%% use defaults for review submission.
\acmJournal{PACMPL}
\acmVolume{1}
\acmNumber{CONF} % CONF = POPL or ICFP or OOPSLA
\acmArticle{1}
\acmYear{2018}
\acmMonth{1}
\acmDOI{} % \acmDOI{10.1145/nnnnnnn.nnnnnnn}
\startPage{1}

%% Copyright information
%% Supplied to authors (based on authors' rights management selection;
%% see authors.acm.org) by publisher for camera-ready submission;
%% use 'none' for review submission.
\setcopyright{none}
%\setcopyright{acmcopyright}
%\setcopyright{acmlicensed}
%\setcopyright{rightsretained}
%\copyrightyear{2018}           %% If different from \acmYear

%% Bibliography style
\bibliographystyle{ACM-Reference-Format}
%% Citation style
%% Note: author/year citations are required for papers published as an
%% issue of PACMPL.
\citestyle{acmauthoryear}   %% For author/year citations

\begin{document}



%% Title information
\title[Short Title]{
    Operational Semantics and Logic for Weak Consistency in Transactional Systems%
    %: a Multi-version Based Operational Approach
    } 
                                        %% [Short Title] is optional;
                                        %% when present, will be used in
                                        %% header instead of Full Title.
\titlenote{with title note}             %% \titlenote is optional;
                                        %% can be repeated if necessary;
                                        %% contents suppressed with 'anonymous'
\subtitle{Subtitle}                     %% \subtitle is optional
\subtitlenote{with subtitle note}       %% \subtitlenote is optional;
                                        %% can be repeated if necessary;
                                        %% contents suppressed with 'anonymous'


%% Author information
%% Contents and number of authors suppressed with 'anonymous'.
%% Each author should be introduced by \author, followed by
%% \authornote (optional), \orcid (optional), \affiliation, and
%% \email.
%% An author may have multiple affiliations and/or emails; repeat the
%% appropriate command.
%% Many elements are not rendered, but should be provided for metadata
%% extraction tools.

%% Author with single affiliation.
\author{Shale Xiong}
%\authornote{with author1 note}          %% \authornote is optional;
                                        %% can be repeated if necessary
%\orcid{nnnn-nnnn-nnnn-nnnn}             %% \orcid is optional
\affiliation{
  \position{Ph.D. Student}
  \department{Department of Computing}              %% \department is recommended
  \institution{Imperial College London}            %% \institution is required
  \streetaddress{Huxley Building}
  \city{London}
  \state{}
  \postcode{SW7 2AZ}
  \country{United Kingdom}                    %% \country is recommended
}
\email{shale.xiong14@imperial.ac.uk}          %% \email is recommended

%% Author with single affiliation.
\author{Andrea Cerone}
%\authornote{with author1 note}          %% \authornote is optional;
                                        %% can be repeated if necessary
%\orcid{nnnn-nnnn-nnnn-nnnn}             %% \orcid is optional
\affiliation{
  \position{Research Associate}
  \department{Department of Computing}              %% \department is recommended
  \institution{Imperial College London}            %% \institution is required
  \streetaddress{Huxley Building}
  \city{London}
  \state{}
  \postcode{SW7 2AZ}
  \country{United Kingdom}                    %% \country is recommended
}
\email{a.cerone@imperial.ac.uk}          %% \email is recommended

\author{Azalea Raad}
%\authornote{with author1 note}          %% \authornote is optional;
                                        %% can be repeated if necessary
%\orcid{nnnn-nnnn-nnnn-nnnn}             %% \orcid is optional
\affiliation{
  \position{Post-doctoral Researcher}
  \department{}              %% \department is recommended
  \institution{Max Planck Institute}            %% \institution is required
  \streetaddress{no idea}
  \city{Kaiserslautern}
  \state{-}
  \postcode{-}
  \country{Germany}                    %% \country is recommended
}
\email{a.raad@mpi-sws.org}          %% \email is recommended

%% Author with single affiliation.
\author{Philippa Gardner}
%\authornote{with author1 note}          %% \authornote is optional;
                                        %% can be repeated if necessary
%\orcid{nnnn-nnnn-nnnn-nnnn}             %% \orcid is optional
\affiliation{
  \position{Professor}
  \department{Department of Computing}              %% \department is recommended
  \institution{Imperial College London}            %% \institution is required
  \streetaddress{Huxley Building}
  \city{London}
  \state{-}
  \postcode{SW7 2AZ}
  \country{United Kingdom}                    %% \country is recommended
}
\email{p.gardner@imperial.ac.uk}          %% \email is recommended

\begin{abstract}
Contents of this set of notes: 
History heaps. Semantics of Programs 
running under weak consistency models using history heaps as states. 
Simulation technique for comparing weak consistency models defined using 
history heaps. Verification of implementations.
\textbf{Points following Dagstuhl: Viktor seemed positive about the 
history heap work. His question was whether the framework is generic 
enough to capture the protocols that they are developing with Azalea. 
Alexey's opinion is that the framework may have some use if we 
manage to prove implementations of protocols correct. 
I would also like to have Azalea's opinion on a semantics based 
on history heaps.}
\end{abstract}


%% 2012 ACM Computing Classification System (CSS) concepts
%% Generate at 'http://dl.acm.org/ccs/ccs.cfm'.
\begin{CCSXML}
<ccs2012>
<concept>
<concept_id>10011007.10011006.10011008</concept_id>
<concept_desc>Software and its engineering~General programming languages</concept_desc>
<concept_significance>500</concept_significance>
</concept>
<concept>
<concept_id>10003456.10003457.10003521.10003525</concept_id>
<concept_desc>Social and professional topics~History of programming languages</concept_desc>
<concept_significance>300</concept_significance>
</concept>
</ccs2012>
\end{CCSXML}

\ccsdesc[500]{Software and its engineering~General programming languages}
\ccsdesc[300]{Social and professional topics~History of programming languages}
%% End of generated code


%% Keywords
%% comma separated list
\keywords{keyword1, keyword2, keyword3}  %% \keywords are mandatory in final camera-ready submission


%% \maketitle
%% Note: \maketitle command must come after title commands, author
%% commands, abstract environment, Computing Classification System
%% environment and commands, and keywords command.

\maketitle

\azalea{I have imported the cleveref package! This means that all reference will be printed consistently and we DO NOT need custom names such as \textbackslash fig etc.
Every time you need to refer to something, please write\textbackslash cref\{label\}, \eg \cref{def:mkvs}, and the label (\eg Def.) will be printed correctly. 
These labels can be customised. I have introduced the necessary ones in the macros file. 
}
\newcommand{\RootPath}{.}
\section{Introduction}

In recent times, the area of formal reasoning for concurrent and heap-manipulating programs, has seen a noticeable development towards program logics that can tackle the specification and verification of low-level concurrency in systems. This capability, together with the ubiquity of multithreading in computer programs, allows the formulation of reasoning frameworks around a variety of applications.

Modern database systems make heavy use of concurrency in order to provide a level of performance able to support large scale operations. This leads to an obvious increase in throughput but can cause a lack of consistency in data, which is instead a fundamental requirement for the majority of programs. A number of techniques has been employed in commercial databases to solve the issue and try to give the best of both worlds. Among these, \textit{Two-Phase-Locking} is a blocking approach which resides in the part of the spectrum of solutions where correctness is preferred over performance and that works at the granularity of single database entries. As a consequence, once implemented as part of a complex database system, the algorithm is prone to subtle bugs which might cause the violation of its vital guarantees.

We therefore intend to provide a complete and flexible model of the \textit{Two-Phase-Locking} concurrency control mechanism, as inspired by its real-world use case.
The aim is to derive a specification together with a sound level of abstraction that allows us not to think in terms of the low-level details enforced by the technique.
This leads us to the exploration of formal reasoning about its client usage through a custom program logic which is proven to be sound. The logic framework enables users to prove partial correctness of their programs running in a \textit{Two-Phase-Locking} setting, by only having to reason atomically about blocks of code, without the complexity of concurrent interleavings.

\subsection{Contributions}

The main contributions of this project are listed below, with references to the relevant sections where they are further discussed.
\begin{itemize}
	\item \textbf{mCAP} (Section \ref{sec:mcapModel}, \ref{sec:mcapLogic}, \ref{sec:transLogic})\ \ We reformulate and extend a program logic for concurrent programs, namely CAP \cite{cap}, in order to remove some constraints which are hardcoded in the logic and enable a more flexible reasoning. In fact, we change the underlying model to parametrize both the representation of machine states and of action capabilities. On top of this, we provide a new and cleaner structure for the action model that does not explicitly use interference assertions. We also considerably modify the way environment interference is modelled through the rely/guarantee relations. This is done with the goal of allowing both a thread and the environment to perform multiple shared region updates in one step. It follows that the repartitioning operator also has a new and extended behaviour. At the level of programming language, we leave elementary atomic commands as a parameter to the user of the logic. Finally, we instantiate the mCAP framework into a logic for our particular needs of transactional reasoning.
	
	\item \textbf{\textsc{2pl} Model}\ \ 
	
	\item \textbf{Operational Semantics}\ \
	
	\item \textbf{Semantics Equivalence}
\end{itemize}
\subsection{semantics}
Let \( \repl \in \Repls \) denotes the set of totally ordered replicates.
Each replicate can have multiple clients, and 
each clients can commit a sequence of either read-only transitions or single-write transactions.
To model these, we annotate the transaction identifier with replicate \( \repl \), client \( \cl \), 
local time of the replicate \( n \) and read-only transactions count \( n' \), \ie \( \txidCOPS{\repl}{\cl}{n}{n'} \).
Note that the \( (n, \repl, n') \) can be treated as a single number that \( n \) are the higher bits, 
\( \repl \) the middle bits and \( n' \) the lower bits.
There is a total order among transitions from the same replica and from the same client.
We extend version with the set of all versions it dependencies on, \( \dep \in \pset{\Keys \times \TxID} \).
The function \( \depOf{\ver} \) denotes the dependencies set of the version.
For readability, we annotate view with either a replica, \( \viREPL \), or a client, \( \viCL \).
The view environment is extended with replicas and their views, \( \viewFunCOPS : (\Repls \times \ClientID ) \parfinfun \Views \).
We give the following semantics to capture the behaviours of the code.

\begin{lstlisting}[caption={put},label={lst:simplified-put}]
// mixing the client API and system API
put(repl,k,v,ctx) {

    // Dependency for previous reads and writes
    deps = ctx_to_dep(ctx);(*\label{line:put-ctx-to-deps}*)

    atomic{
        // increase local time.
        inc(repl.local_time);(*\label{line:put-inc-local}*) 

        // appending local kv with a new version.
        list_isnert(repl.kv[k],(v, (local_time + id), deps));(*\label{line:put-update-kv}*)
    }

    // update dependency for writes
    ctx.writers += (k,(local_time + id),deps);(*\label{line:put-update-ctx}*)

    // put in the queue to sync with other replicas
    enqueue(k,v,(current_ver+id),(deps ++ vers));
}
\end{lstlisting}

The client always fetches the version with the maximum writer it can observed for each key,
Which is computed by \( \funcn{getMax} \) function. 
It is different from \( \snapshot \) as \( \snapshot \) fetches the latest version with respect to the position in the list.

\[
    \begin{rclarray}
        \func{getMax}{\mkvsCOPS, \viCL} & \defeq &
        \lambda \ke \ldotp \left( \max_\txid\Setcon{(\val, \txid, \T, \dep)}{\exsts{i} (\val, \txid, \T, \dep) = \mkvsCOPS(\ke, i)} \right)\projection{1}
    \end{rclarray}
\]
\begin{mathpar}
    \inferrule[Put]{%
        ( \stk, \func{getMax}{\mkvsCOPS, \viCL}, \emptyset ), \pmutate{\ke}{\vx} \toL
        ( \stk', \stub, \Set{(\otW, \ke, \val )} ), \pskip
        \\\\
        \dep = \Setcon{(\ke', \txid)}{\exsts{i} i \in \viCL(\ke') \land \txid = \WTx(\mkvsCOPS(\ke', i))} \texttt{ ---> \cref{lst:simplified-put}, \cref{line:put-ctx-to-deps}} 
        \\\\
        \txid = \min\Setcon{%
        \txidCOPS{\repl}{\cl}{n'}{0}
        }{%
            \fora{\ke', i \in \viREPL(\ke'), n} \\
            \quad \txidCOPS{\stub}{\stub}{n}{\stub} = \WTx(\mkvsCOPS(\ke',i)) \\
            \qquad {} \implies n' > n 
        } \texttt{ ---> \cref{lst:simplified-put}, \cref{line:put-inc-local}}
        \\\\
        \mkvsCOPS' = \mkvsCOPS\rmto{\ke}{\mkvsCOPS(\ke) \lcat \List{(\ke, \txid, \emptyset, \dep)}} \texttt{ ---> \cref{lst:simplified-put}, \cref{line:put-update-kv}}
        \\\\
        \viREPL' = \viREPL\rmto{\ke}{\viREPL(\ke) \uplus \Set{\abs{\mkvsCOPS'(\ke)} - 1}} \texttt{ ---> \cref{lst:simplified-put}, \cref{line:put-update-kv}}
        \\\\
        \viCL' = \viCL\rmto{\ke}{\viREPL(\ke) \uplus \Set{\abs{\mkvsCOPS'(\ke)} - 1}} \texttt{ ---> \cref{lst:simplified-put}, \cref{line:put-update-ctx}}
    }{%
    \repl, \cl \vdash 
    \mkvsCOPS, \viREPL, \viCL, \stk, \ptrans{\pmutate{\ke}{\vx};} \toT{}
    \mkvsCOPS', \viREPL', \viCL', \stk', \pskip
    }
\end{mathpar}
The \verb|get_trans| fetches the latest versions from the replica via multiple atomic reads, one for each key.
As a result, a client has a list of candidates \verb|rst|.
Since interleaving might happen, versions might become out-of-date because the replicate receives new versions.
It is not a problem to read old versions as long as they satisfy causal consistency,
\ie if a client read a version \( \ver \), it should at least read all the versions that \( \ver \) depends on.
Thus the algorithm use \verb|ccv| to track the maximum versions the client should fetches,
and re-fetches the \verb|ccv[k]| version from the replica if it is greater than the candidate.

The following is a simplified algorithm by directly taking a list of versions \verb|ccv| satisfies causal consistency constraint,
and then read the versions indicated by \verb|ccv|.
The simplified algorithm is easier to understand.
\begin{lstlisting}[caption={get\_trans},label={lst:get-trans}]
// A simplified version by guessing
// a ccv satisfying dependency constraints
// and then read versions indicated by ccv.
// Note that it is a weaker version of the original code,
// as the original implementation fetches the latest versions
// for keys by a sequence of atomic get_by_version calls
List(Val) get_trans(ks,ctx) {
    take ccv: (*$\forall$*) k (*$\in$*) ks.(*\label{line:get-trans-ccv-1}*)
        (_,_,deps) := get_by_version(k,ccv[k]) (*${}\land \forall$*) dep (*$\in$*) deps.(*\label{line:get-trans-ccv-2}*)
            dep.key (*$\in$*) ks (*$\implies$*) ccv[dep.key] >= dep.ver (*\label{line:get-trans-ccv-3}*)

    for k in ks(*\label{line:get-trans-read-1}*)
        rst[k] = get_by_version(k,ccv[k]);(*\label{line:get-trans-read-2}*)

    // update the ctx
    for (k,ver,deps) in rst(*\label{line:get-trans-update-ctx-1}*)
        ctx.readers += (k,ver,deps);(*\label{line:get-trans-update-ctx-2}*)

    return to_vals(ks);
}                                   
\end{lstlisting}
\begin{mathpar}
    \inferrule[GetTrans]{%
        \viCL \viewleq \viCL' \viewleq \viREPL  \texttt{ ---> \cref{lst:get-trans}, \cref{line:get-trans-update-ctx-1,line:get-trans-update-ctx-2}}
        \\\\
        {\left(\begin{array}{@{}l@{}}
        \fora{i : 1 \leq i \leq j, \ke', m, \ver}  \\
        \quad \ver = \mkvsCOPS(\ke_i, \max(\viCL'(\ke_i)) \land {} \\
        \quad (\ke', \WTx(\mkvsCOPS(\ke', m))) \in \ver\projection{4} \\
        \qquad {} \implies m \in \viCL'(\ke')
        \end{array}\right)} \texttt{ ---> \cref{lst:get-trans}, \cref{line:get-trans-ccv-1,line:get-trans-ccv-2,line:get-trans-ccv-3}}
        \\\\
        \trans =  \pderef{\vx_1}{\ke_1}; \dots; \pderef{\vx_j}{\ke_j};
        \\\\
        ( \stk, \func{getMax}{\mkvsCOPS, \viCL'}, \emptyset ), \trans \toL
        ( \stk', \stub, \f ), \pskip \texttt{ ---> \cref{lst:get-trans}, \cref{line:get-trans-read-1,line:get-trans-read-2}}
        \\\\
        \txidCOPS{\repl}{\cl}{n'}{n} = \max\Setcon{\txidCOPS{\repl}{\cl}{z'}{z}}{\txidCOPS{\repl}{\cl}{z'}{z} \in \mkvsCOPS }
        \\
        \mkvsCOPS' = \updKV{\mkvsCOPS, \viCL', \txidCOPS{\repl}{\cl}{n'}{n+1}, \f} 
    }{%
        \repl, \cl \vdash 
        \mkvsCOPS, \viREPL, \viCL, \stk, \ptrans{\pderef{\vx_1}{\ke_1}; \dots; \pderef{\vx_j}{\ke_j}; } \toT{}
        \mkvsCOPS', \viREPL, \viCL', \stk', \pskip
    }
    \and
    \inferrule[ClientCommit]{%
        \repl, \cl \vdash 
        \mkvsCOPS, \viewFunCOPS(\repl), \viewFunCOPS(\cl), \stk, \prog(\cl) \toT{}
        \mkvsCOPS', \viREPL', \viCL', \stk', \cmd'
    }{%
        \mkvsCOPS, \viewFunCOPS, \thdenv, \prog \toG{}
        \mkvsCOPS', \viewFunCOPS\rmto{\repl}{\viREPL'}\rmto{\cl}{\viCL'}, \thdenv\rmto{\cl}{\stk'}, \prog\rmto{\cl}{\cmd'}
    }
\end{mathpar}
A replica updates its local state only if all the dependencies has been receive.
\begin{lstlisting}[caption={Send and receive},label={lst:send-receive}]
// Syn to other replicas
send() {
    (k,v,ver,deps) := dequeue();
    for id in repls {
        send (k,v,ver,deps) to id;
    }
}

// receive a write message from other replica
on_receive(k,v,ver,deps) {
    // for a single machine
    // the following check immediately holds
    for (k',ver') in deps {
        wait until dep_check(k',ver');(*\label{line:receive-wait}*)
    }

    atomic{
        list_isnert(kv[k],(v,ver,deps));(*\label{line:receive-update-view-1}*)
        (remote_local_time + id) = ver;(*\label{line:receive-update-view-2}*)
        local_time = max(remote_local_time, local_time);(*\label{line:receive-update-view-3}*)
    }
}
\end{lstlisting}
\begin{mathpar}
    \inferrule[sync]{%
        \viREPL = \viewFunCOPS(\repl)\rmto{\ke}{\viewFunCOPS(\repl)(\ke) \uplus i} 
        \texttt{ ---> \cref{lst:send-receive}, \cref{line:receive-update-view-1,line:receive-update-view-2,line:receive-update-view-3}}
        \\\\
        {\left(\begin{array}{@{}l@{}}
        \fora{\ke', m, \ver} 
        \ver = \mkvsCOPS(\ke, i) \land {} \\
        \quad (\ke', \WTx(\mkvsCOPS(\ke', m))) \in \ver\projection{4} \\
        \qquad {} \implies m \in \viREPL'(\ke') 
        \end{array}\right)} \texttt{ ---> \cref{lst:send-receive}, \cref{line:receive-wait}} 
    }{%
        \mkvsCOPS, \viewFunCOPS, \thdenv, \prog \toG{}
        \mkvsCOPS, \viewFunCOPS\rmto{\repl}{\viREPL}, \thdenv, \prog
    }
\end{mathpar}

A view \( \vi \) on key-value store \( \mkvsCOPS \) \emph{agrees} 
with another view \( \vi \) on another key-value store \( \mkvsCOPS' \), if and only
\[
 \func{getMax}{\mkvsCOPS, \vi} = \func{getMax}{\mkvsCOPS', \vi'}
\]



%\begin{theorem}
    %For any trace \( \tr \) of COPS with final configuration \( (\mkvsCOPS, \viewFunCOPS) \), 
    %there exists a trace \( \tr' \) with final configuration \( (\mkvsCOPS', \viewFunCOPS') \) such that 
    %each step of the trace \( \tr' \) commits a transaction with strictly greater transaction identifier than any one appearing in the key-value store:
    %\[
        %\begin{array}{@{}l@{}}
        %(\mkvsCOPS_i, \viewFunCOPS_i) 
        %\toG{} (\mkvsCOPS_{i+1}, \viewFunCOPS_{i+1}) 
        %\land \exsts{\txid} \txid \in \mkvsCOPS_{i+1} 
        %\land \txid \notin \mkvsCOPS_{i+1}
        %\implies \fora{\txid' \in \mkvsCOPS_i} \txid > \txid'
        %\end{array}
    %\]
    %and any replica's view from \( \viewFunCOPS \) agrees with its counterpart from  \( \viewFunCOPS' \):
    %\[
        %\fora{i} 
        %\func{getMax}{\mkvsCOPS, \viewFunCOPS(i)} = \func{getMax}{\mkvsCOPS', \viewFunCOPS'(i)}
    %\]
%\end{theorem}
%\begin{proof}
%\end{proof}

%\begin{lemma}
%\[
    %\begin{array}{@{}l@{}}
    %\fora{\repl, \cl, \mkvsCOPS, \mkvsCOPS', \mkvsCOPS'', \viREPL, \viREPL', \viREPL'', \viCL, \viCL', \viCL'', \stk, \stk', \cmd, \cmd'} \\
    %\quad \func{getMax}{\mkvsCOPS, \viREPL} = \func{getMax}{\mkvsCOPS'', \viREPL''} 
    %\land \func{getMax}{\mkvsCOPS, \viCL} = \func{getMax}{\mkvsCOPS'', \viCL''} \\
    %\qquad \repl, \cl \vdash 
    %\mkvsCOPS, \viREPL, \viCL, \stk, \cmd \toT{}
    %\mkvsCOPS', \viREPL', \viCL', \stk', \cmd' \\
    %\qquad \implies 
    %\exsts{\mkvsCOPS''', \viREPL'''}
    %\mkvsCOPS'', \viREPL'', \viCL'', \stk, \cmd \toT{}
    %\mkvsCOPS''', \viREPL''', \viCL', \stk', \cmd' \\
    %\qquad \func{getMax}{\mkvsCOPS', \viREPL'} = \func{getMax}{\mkvsCOPS''', \viREPL'''} 
    %\land \func{getMax}{\mkvsCOPS', \viCL'} = \func{getMax}{\mkvsCOPS''', \viCL'''} \\
    %\end{array}
%\]
%\end{lemma}
%\begin{proof}
%We perform case analysis.
%\begin{itemize}
    %\item \rl{Put}.
    %We have \( \cmd \equiv \ptrans{\pmutate{\ke}{\vx};} \) for some key \( \ke \) and variable \( \vx \).
    %Suppose key-value stores \(  \mkvsCOPS, \mkvsCOPS', \mkvsCOPS'' \), 
    %replica's views \( \viREPL, \viREPL', \viREPL''\) and client's views \( \viCL, \viCL', \viCL''\) such that
    %\[
    %\begin{array}{@{}l@{}}
    %\func{getMax}{\mkvsCOPS, \viREPL} = \func{getMax}{\mkvsCOPS'', \viREPL''} 
    %\land \func{getMax}{\mkvsCOPS, \viCL} = \func{getMax}{\mkvsCOPS'', \viCL''} \\
    %\qquad \repl, \cl \vdash 
    %\mkvsCOPS, \viREPL, \viCL, \stk, \cmd \toT{}
    %\mkvsCOPS', \viREPL', \viCL', \stk', \cmd' \\
    %\end{array}
    %\]
    %By the premiss of the \rl{Put} rule, the new key-value store
    %\[
        %\mkvsCOPS' = \mkvsCOPS\rmto{\ke}{\mkvsCOPS(\ke) \lcat \List{(\ke, \txid, \emptyset, \dep)}}
    %\]
    %where
    %\[
        %\txid = \min\Setcon{%
            %\txidCOPS{\repl}{\cl}{n'}{0}
        %}{%
            %\fora{\ke', i \in \viREPL(\ke'), n} \\
            %\quad \txidCOPS{\stub}{\stub}{n}{\stub} = \WTx(\mkvsCOPS(\ke',i)) \\
            %\qquad {} \implies n' > n 
        %} 
    %\]
    %and the new views of replica and client are
    %\[   
        %\begin{array}{@{}l@{}}
        %\viREPL' = \viREPL\rmto{\ke}{\viREPL(\ke) \uplus \Set{\abs{\mkvsCOPS'(\ke)} - 1}} \\
        %{} \land \viCL' = \viCL\rmto{\ke}{\viREPL(\ke) \uplus \Set{\abs{\mkvsCOPS'(\ke)} - 1}}
        %\end{array}
    %\]
    %Similarly there exists a new \( \mkvsCOPS''' \) by committing a single-write transaction \( \txid' \) and two new views \( \viREPL''' \) and \( \viCL''' \).
    %This means for those key \( \ke' \) that is different from the key \( \ke \) being overwritten,
    %\begin{equation}
        %\label{equ:get-max-match-all-other-key}
        %\begin{array}{@{}l@{}}
            %\func{getMax}{\mkvsCOPS', \viREPL'}(\ke') = \func{getMax}{\mkvsCOPS''', \viREPL'''}(\ke') \\
            %{} \land \func{getMax}{\mkvsCOPS', \viCL'}(\ke') = \func{getMax}{\mkvsCOPS''', \viCL'''}(\ke') 
        %\end{array}
    %\end{equation}
    %Note that the \( \txid \) is greater than any writers \( \txidCOPS{\repl}{\cl}{n'}{0} \) that can be observed by the \( \viREPL \), so is \( \txid' \).
    %That is,
    %\begin{equation}
        %\label{equ:get-max-match-overwritten-key}
        %\begin{array}{@{}l@{}}
            %\func{getMax}{\mkvsCOPS', \viREPL'}(\ke) = \stk(\vx) = \func{getMax}{\mkvsCOPS''', \viREPL'''}(\ke)  \\
            %{} \land \func{getMax}{\mkvsCOPS', \viCL'}(\ke) = \stk(\vx) = \func{getMax}{\mkvsCOPS''', \viCL'''}(\ke) 
        %\end{array}
    %\end{equation}
    %Combine \cref{equ:get-max-match-all-other-key} and \cref{equ:get-max-match-overwritten-key},
    %we have the proof.

    %\item \rl{GetTrans}.
    %Since the views of replica remain unchanged, so we only need to prove that there exists a new key-value store and a new view \( \viCL''' \) such that
    %\[
        %\func{getMax}{\mkvsCOPS', \viCL'}(\ke) = \stk(\vx) = \func{getMax}{\mkvsCOPS''', \viCL'''}(\ke) 
    %\]
    
%\end{itemize}
%\end{proof}

\begin{lemma}
    \label{lem:client-subset-repl}
    The view of a client is subset of the view of the replica that the client interacts with.
\end{lemma}


\begin{lemma}
    Let ignore the dependencies of versions from \( \mkvsCOPS \).
    Given the initial key-value store \( \mkvsCOPS_0 \), initial views \( \viewFunCOPS_0 \) and some programs \( \prog_0 \), for any \( \mkvsCOPS_i \) and \( \viewFunCOPS_i \)  such that: 
    \[
        \mkvsCOPS_0, \viewFunCOPS_0, \thdenv_0, \prog_0 {\toG{}}^* \mkvsCOPS_i, \viewFunCOPS_i, \thdenv_i, \prog_i
    \]
    The key-value store \( \mkvsCOPS_i \) satisfies the \cref{def:mkvs} and any replica or client view \( \vi \) from \( \viewFunCOPS_i \) is a valid view of the key-value store, \ie \( \vi \in \Views(\mkvsCOPS_i) \).
\end{lemma}
\begin{proof}
    We need to prove the  \( \mkvsCOPS_i \) satisfies the well-formed conditions,
    and any view \( \vi_i \Views(\mkvsCOPS_i) \).
    We prove it by introduction on the length \( i \).
    \begin{itemize}
    \item \caseB{\(i = 0\)}
        It holds trivially since each key only has the initial version \( (\val_0,\txid_0,\emptyset, \emptyset) \).
        Since there is only the initial version for each key, it is easy to see that any view \( \vi_0 \) satisfying the well-formed conditions in \cref{def:views}.
    \item \caseI{\(i > 0\)}
        Suppose it holds when \( i \), let consider \( i + 1 \).
        We perform case analysis on the possible next step:
        \begin{itemize}
            \item \rl{Put}
                Assume the client \( \cl \) of a replica \( \repl \) commits a single-write transaction \( \txid \) that installs a new version for key \( \ke \).
                By the premiss of \rl{Put}, the new transaction identifier \( \txid = \txidCOPS{\repl}{\cl}{n'}{0} \) where for some \( n' \) that is greater than any \( n \) from any writers \( \txidCOPS{\stub}{\stub}{n}{\stub} \) that are observable by the replica \( \repl \).
                Since the new transaction \( \txid = \txidCOPS{\repl}{\cl}{n'}{0} \) is a single-write transaction which is always installed at the end of the list associated to \( \ke \), it is sufficient to prove the following:
                \begin{gather}
                    \fora{j} 0 \leq j < \abs{ \mkvsCOPS_i(\ke) } \implies \WTx(\mkvsCOPS_{i}(\ke, m)) \neq \txid \label{equ:write-trans-unique} \\
                    \fora{j, n} \txidCOPS{\repl}{\cl}{n}{\stub} \in \Set{\WTx(\mkvsCOPS_{i}(\ke,j))} \cup \RTx(\mkvsCOPS_{i}(\ke, j)) \implies n < n' \label{equ:replica-time-monotonic-inc}
                \end{gather}
                By \cref{lem:repl-observe-own}, we know that for any version written \( \ver = \mkvsCOPS_i(\ke, j) \) by the same replica \( \txidCOPS{\repl}{\stub}{\stub}{\stub} = \WTx(\ver) \), such version is included in the replica's view \( j \in \viREPL(\ke) \).
                It implies that first the new transaction identifier is unique \cref{equ:write-trans-unique} and second it is greater than any transactions in the form of \( \txidCOPS{\repl}{\cl}{\stub}{\stub} \) \cref{equ:replica-time-monotonic-inc}.
                Thus the new key-value store \( \mkvsCOPS_{i+1} \) satisfies the well-formed conditions.
                Now let consider the views, especially the views of the replica \( \viREPL' \) and the client \( \viCL' \).
                Since that views \( \vi' \) from different replicas or clients remain unchanged, by \ih they satisfy \( \vi' \in \Views(\mkvsCOPS_{i+1}) \).
                The new view for replica \( \viREPL' = \viREPL\rmto{\ke}{\abs{\mkvsCOPS_{i+1}(\ke)} - 1} \)
                where \( \viREPL \) is the replica's view before updating and the writer of the last version of \( \ke \) is \( \txid \).
                Because \( \txid \) is a single-write transaction, so the new view \( \viREPL' \) still satisfies the atomic read.
                For similar reason, the new view for client \( \viCL' \) till satisfies atomic read.
                Therefore we have \( \viREPL', \viCL' \in \Views(\mkvsCOPS_{i+1}) \).
            \item \rl{GetTrans}

        \end{itemize}
    \end{itemize}
\end{proof}

\begin{lemma}
    \label{lem:repl-observe-own}
    A replica observes all its own transactions.
\end{lemma}

\section{Logic}

\subsection{Rules for Local}

The proof rules are standard except \rl{TRDeref} and \rl{TRMutate}.
The \rl{TRDeref} rule add read fingerprint in finger-tracking set, only if there is no write finger-print.
This is because once a location has been re-written, the rest read are considered as local operations, while the finger-print only records those operations might have effect on global state.
%
\[
    \infer[\rl{TRDeref}]{%
        \judgeL{\expr \fpt{\fp} \lexpr}{ \pderef{\var}{\expr} }{\var \dot= \lexpr \sep \expr \fpt{\addFPR{\fp}} \lexpr }
    }{%
        \var \notin \func{fv}{\expr} &&
        \var \notin \func{fv}{\lexpr}  
    }
\]
 
\[
    \infer[\rl{TRMutate}]{%
        \judgeL{\expr_1 \fpt{\fp} \stub }{ \pmutate{\expr_1}{\expr_2} }{ \expr_1 \fpt{\addFPW{\fp}} \expr_2} 
    }{}
\]

\subsection{Merge}

\begin{defn}[Fingerprint heaps merge]
\label{def:merge-finger-heap}
The \emph{merge of fingerprint heaps}, \( \mergeFP{.}{.} \), is defined as follows:
\[
    \begin{rclarray}
        \mergeFP{\fph_{l}}{\fph_{r}}  & \defeq & \lambda \addr \ldotp 
            \begin{cases}
                \fph_{l}(\addr) & a \in \dom(\fph_{l}) \setminus \dom(\fph_{r})  \\
                \fph_{r}(\addr) & a \in \dom(\fph_{r}) \setminus \dom(\fph_{l}) \\
                \mergeVAL{\fph_{l}(\addr)}{\fph_{r}(\addr)}  & a \in \dom(\fph_{l}) \cap \dom(\fph_{r}) \\
            \end{cases}
    \end{rclarray}
\]
where the \emph{merge of fingerprint heap values} is defined:
\[ \begin{rclarray}
        \mergeVAL{(\val_{l}, \fp_{l})}{(\val_{r}, \fp_{r})} & \defeq &
            \begin{cases}
                (\val_{l}, \fp_{l} \cup \fp_{r} ) & \val_{l} = \val_{r} \land \fpW \notin \fp_{l} \cup \fp_{r} \\
                (\val_{l}, \fp_{l} \cup \fp_{r} ) & \fpW \in \fp_{l} \land \fpW \notin \fp_{r} \\
                (\val_{r}, \fp_{l} \cup \fp_{r} ) & \fpW \notin \fp_{l} \land \fpW \in \fp_{r} \\
            \end{cases}
    \end{rclarray}
\]
\end{defn}

%\sx{
    %Andrea gives a better idea to do this by defining merging fingerprint heaps first.
    %Big thanks. :)))
%}
%\azalea{This is a bit strong! For instance, according to this definition the heaps $\pv x \pt_{\emptyset} 1$ and $\pv x \pt_{\{\fpR\}} 1$ do not agree! Is that what you want?

%Perhaps you can define this as:
%\[
%\begin{rclarray}
	%\agree{\fph_l}{\fph_r} & \defeq  & \ws{\fph_l} \cap \ws{\fph_r} = \emptyset \\
        %&& \land\ \exsts{\h_1, \h_2, \h} \heapOnly{\readOnly{\fph_l}} = \h_1 \composeH \h \land \heapOnly{\readOnly{\fph_r}} = \h \composeFP \h_2 
        %\land \h_1 \composeH \h \composeH \h_2 \isdef
%\end{rclarray}        
%\]
%where
%\[
%\begin{rclarray}
	%\func{ws}{\fph} & \defeq & \myset{\loc}{ \exsts{\fp} \fph(\loc) = (\stub, \fp) \land \fpW \in \fp} \\\\
%%
	%\readOnly{.} & : & \FPHeaps \rightarrow \FPHeaps\\	
	%\readOnly{\fph}(\loc) & \defeq & 
	%\begin{cases}
		%\fph(\loc) & \text{if } \loc \not\in \ws{\fph} \\
		%\text{undefined} & \text{otherwise}
	%\end{cases}\\\\
%%	
	%\heapOnly{.} & : & \FPHeaps \rightarrow \Heaps\\	
	%\heapOnly{\fph}(\loc) & \defeq & 
	%\begin{cases}
		%\val  & \text{if } \exsts{\val} \fph(\loc) = (\val, -) \\
		%\text{undefined} & \text{otherwise}
	%\end{cases}
%\end{rclarray}
%\]
%}

\begin{defn}[Local state merge]
The \emph{merge of local states} is defined as follows, which merges two local states \( \ls_{l} \) and \( \ls_{r} \) with respect to a common initial local state \( \ls \).
\[
    \begin{rclarray}
	\mergeLS{\ls}{\ls_{l}}{\ls_{r}} & \eqdef &
	\myset{
		\left(\fph, \ca_{l}' \composeC \ca_{f} \composeC \ca_{r}' \right) 
	}{
        \fph = \mergeFP{\lsFPH{(\ls_{l})} }{ \lsFPH{(\ls_{r})} } \land \exsts{\ca_{l}, \ca_{r}}\\
		\quad \land\; \lsCAP{\ls} = \ca_{l} \composeC \ca_{f} \composeC \ca_{r}
        \land \lsCAP{(\ls_{l})} = \ca_{l}' \composeC \ca_{f} \composeC \ca_{r}
        \land \lsCAP{(\ls_{r})} = \ca_{l} \composeC \ca_{f} \composeC \ca_{r}'
	}
    \end{rclarray}
\]
where, to recall, the notation \( \lsFPH{(.)} \), and \( \lsCAP{(.)} \) refer to the fingerprint heap and capabilities respectively in a local state.
Then, the \emph{conflict} between two local states is defined as follows:
\[
    \begin{rclarray}
        \conflict{\ls}{\ls_{l}}{\ls_{r}} & \defeq & \mergeLS{\ls}{\ls_{l}}{\ls_{r}} = \emptyset
    \end{rclarray}
\]
\end{defn}

\begin{definition}[Fingerprint worlds]\label{def:fingerprint_worlds}
Given the set of local states $\LStates$ (\defin\ref{def:local_state}) and the set of region identifiers $\RegionID$ (\defin\ref{def:capabilities}), the set of \emph{fingerprint worlds}, $\fpw \in \FPWorlds$, is defined as follows:
%
\[
\begin{rclarray}
	\FPWorlds  & \eqdef  
	& \myset{
		(\ls, \fpgs)
	}{
		(\ls, \fpgs) \in \LStates \times (\RegionID \parfinfun \LStates) \land \wfFW{\ls, \fpgs}
	}
\end{rclarray}
\]
%
with the definitions of the flattening function and the well-formedness predicate lifted as follows:
%
\[
\begin{rclarray}
	\flattenFW{(\ls, \fpgs)}  & \eqdef & \ls \composeLS \prod\limits_{\rid \in \dom(\fpgs)}^{\composeLS} \fpgs(\rid)
\end{rclarray}
\]
%
\[
\begin{rclarray}
	\wfFW{\ls, \fpgs} & \defeq & \exsts{\fph, \ca}\flattenFW{(\ls, \fpgs)} {=} (\fph, \ca) \land\ \dom(\ca) \subseteq \dom(\fpgs) \\
\end{rclarray}
\]
%
The \emph{lift function}, $\liftW{.}: \World \rightarrow \FPWorlds$, is defined as follows:
%
\[
	\liftW{(\lgs, \gs)} \eqdef (\liftLGS{\lgs}, \liftGS{\gs})
\]
%
where
%
\[
\begin{rclarray}
	\liftLGS{(\h, \ca)} = (\fph, \ca)
	& \iffdef 
	& \for{\loc, \val} \h(\loc) = \val \iff \fph(\loc) = (\val, \emptyset) \\
%
	\liftGS{\gs} = \fpgs 
	& \iffdef
	& \for{\rid, \lgs, \inter} \gs(\rid) = (\lgs, \inter) \iff \fpgs(\rid) = (\liftLGS{\lgs'}, \inter) 
\end{rclarray}
\]
%
The \emph{erase function}, $\eraseFW{.}: \FPWorlds \rightarrow \World$, is defined as follows:
%
\[
	\eraseFW{(\ls, \fpgs)} \eqdef (\eraseLS{\ls}, \eraseFGS{\fpgs})
\]
%
where
\[
\begin{rclarray}
	\eraseLS{(\fph, \ca)} = (\h, \ca)
	& \iffdef 
	& \for{\loc, \val} \h(\loc) = \val \iff \fph(\loc) = (\val, \emptyset) \\
%
	\eraseFGS{\fpgs} = \gs 
	& \iffdef
	& \for{\rid, \ls, \inter} \fpgs(\rid) = (\ls, \inter) \Rightarrow \gs(\rid) = (\eraseLS{\ls}, \inter) 
\end{rclarray}
\]
\end{definition}

\begin{definition}[Fingerprint worlds merge]
Given the set of fingerprint worlds $\FPWorlds$ (\defin\ref{def:fingerprint_worlds}), the \emph{merge} function, $\mergeName[\fpw]: \FPWorlds \times \FPWorlds \times \FPWorlds \rightarrow \powerset{\FPWorlds}$, is defined as follows, for all $\fpw,\fpw_{l},\fpw_{r} \in \FPWorlds$:

%
\[
    \begin{rclarray}
	\mergeFW{(\stub, \fpgs)}{(\ls_{l}, \fpgs_{l})}{(\stub, \fpgs_{r})} & \eqdef &
		\Setcon{%
            (\ls_{l}, \fpgs_{p}) 
        }{%
            \fpgs_{p} \in \mergeFGS{\fpgs}{\fpgs_{l}}{\fpgs_{r}}
        } 
    \end{rclarray}
\]
%
where the \( \mergeName[\fpgs] \) is defined as follows:
\[
    \begin{rclarray}
        \mergeFGS{\fpgs}{\fpgs_{l}}{\fpgs_{r}} & \eqdef &
        \begin{cases}
            \emptyset & \text{if} \ \exsts{\rid} \conflict{\fpgs(\rid)}{\fpgs_{l}(\rid)}{\fpgs_{r}(\rid)} \lor \rid \in  \dom(\fpgs_{l}) \cap \dom(\fpgs_{r}) \setminus \dom(\fpgs)  \\
            S & \text{otherwise}
        \end{cases}
    \end{rclarray}
\]
with,
\[
    \begin{rclarray}
	S & = & \myset{\fpgs_{p}}{
		\dom(\fpgs_{p})= \dom(\fpgs_{l}) \cup \dom(\fpgs_{r}) \land \for{\rid}\\
		\quad \rid \in \dom(\fpgs_{l}) \cap \dom(\fpgs_{r}) \implies \fpgs_{p}(\rid) \in \mergeLS{\fpgs(\rid)}{\fpgs_{l}(\rid)}{\fpgs_{r}(\rid)} \\
		\quad \land\ \rid \in \dom(\fpgs_{l}) \setminus \dom(\fpgs_{r}) \implies 	\fpgs_{p}(\rid) = \fpgs_{l}(\rid) \\
		\quad \land\ \rid \in \dom(\fpgs_{r}) \setminus \dom(\fpgs_{l}) \implies 	\fpgs_{p}(\rid) = \fpgs_{r}(\rid)
	}
    \end{rclarray}
\]
\end{definition}

Note that the \( \mergeName[\fpw] \) is not commutative, i.e.\ swapping \( \fpw_{l}\) and \( \fpw_{r}\) might yield different result.

\subsection{Rely and Guarantee}

\begin{definition}[Rely and guarantee]
Given the set of fingerprint worlds $\FPWorlds$ (\defin\ref{def:fingerprint_worlds}), the \emph{update rely} relation, $\relyU: \FPWorlds \times \FPWorlds$, is defined as follows:
%
\[	
    \begin{rclarray}
	\relyU & \eqdef &
	\myset{
		((\ls, \fpgs_{p}), (\ls, \fpgs_{q}))	
	}{
		\exsts{\rid, \ca, \intf, \kap, \ls_{p}, \ls_{q}, \ls_{f}}\\
		\quad \for{\rid'} \rid \ne \rid' \implies \fpgs_{p}(\rid') = \fpgs_{q}(\rid') \\
		\quad \land\ \fpgs_{p}(\rid) = (\ls_{p} \composeLS \ls_{f}, \inter) \land \fpgs_{q}(\rid) = (\ls_{q} \composeLS \ls_{f}, \inter)		 \\
		\quad \land\ ( (\unitFP, \ca) \composeLS \flattenFW{(\ls, \fpgs_{p})} ) \isdef 
		\land \kap \leq \ca(\rid)
		\land (\ls_{p}, \ls_{q}) \in \inter(\kap)
	}
    \end{rclarray}
\]
The \emph{extension rely} relation, $\relyE: \FPWorlds \times \FPWorlds$, is defined as follows:
%
\[	
    \begin{rclarray}
        \relyE \eqdef
        \myset{
            \big((\ls, \fpgs_{p}), (\ls, \fpgs_{q})\big)	
        }{
            \exsts{\rid}
            \dom(\fpgs_{q}) \setminus \dom(\fpgs_{p}) = \{\rid\} \\
            \qquad \land\ \for{\rid'} \rid \ne \rid' \implies \fpgs_{p}(\rid') = \fpgs_{q}(\rid') \\
        }
    \end{rclarray}
\]
The \emph{rely} relation, $\myrely: \FPWorlds \times \FPWorlds$, is defined as follows:
%
\[
    \begin{rclarray}
         \myrely  &\eqdef & \bigcup\limits_{n \in \Nat} R_n \\
    \end{rclarray}
\]
where,
\[
    \begin{rclarray}
        \rely_0 & \eqdef & \closure{(\relyU \cup \relyE)} \\
        R_{n {+} 1} & \eqdef & (R_n \cup \pred{merge\_close}{R_n})^{*} \\
        \pred{merge\_close}{\rely} & \eqdef 
        & \myset{(\fpw, \fpw_{q})}{
            \exsts{\fpw_{l}, \fpw_{r}} (\fpw, \fpw_{l}), (\fpw, \fpw_{r}) \in \rely \land \fpw_{q} \in \mergeFW{\fpw}{\fpw_{l}}{\fpw_{r}}}
    \end{rclarray}
\]
%
The $\closure{(.)}$ denotes the reflexive transitive closure of the relation.
A set of fingerprint worlds $\setworld \subseteq \World$ is \emph{stable}, written $\stable{\setworld}$, if and only if it is closed under the rely relation: 
%
\[
    \begin{rclarray}
        \stable{W} & \eqdef & \for{\world \in \setworld, \fpw'} (\liftW{\world}, \fpw') \in \myrely \implies \eraseFW{\fpw'} \in \setworld
    \end{rclarray}
\]
%
The \emph{update guarantee} relation, $\guarU: \FPWorlds \times \FPWorlds$, is defined as follows:
%
\[	
    \begin{rclarray}
        \guarU & \eqdef &
        \myset{
            ((\ls_{p}, \fpgs_{p}), (\ls_{q}, \fpgs_{q}))	
        }{
            \exsts{\ls_{p}' = \flattenFW{(\ls_{p}, \fpgs_{p})}, \ls_{p}' = \flattenFW{(\ls_{q}, \fpgs_{q})} } \orth{(\lsCAP{(\ls_{p}')})} = \orth{(\lsCAP{(\ls_{q}')})}  \\
            \land 
            \begin{formulea}
                \for{\rid} \fpgs_{p}(\rid) = \fpgs_{q}(\rid) \\
                \lor 
                \begin{formulea}
                    \orth{(\lsFPH{(\ls_{p}')})} = \orth{(\lsFPH{(\ls_{q}')})} 
                    \land \exsts{\rid, \ca, \kap, \intf, \ls_{p}'', \ls_{q}'', \ls_f}\\
                        \quad \for{\rid'} \rid \ne \rid' \implies \fpgs_{p}(\rid') = \fpgs_{q}(\rid') \\
                        \quad \land \fpgs_{p}(\rid) = (\ls_{p}'' \composeLS \ls_f, \inter) \land \fpgs_{q}(\rid) = (\ls_{q}'' \composeLS \ls_f, \inter)		 \\
                        \quad \land (\unitFP, \ca) \leq \ls_{p}
                        \land \kap \leq \ca(\rid)
                        \land (\ls_{p}'', \ls_{q}'') \in \inter(\kap)
                \end{formulea}
            \end{formulea}
        }
    \end{rclarray}
\]
% 
The \emph{extension guarantee} relation, $\guarE: \FPWorlds \times \FPWorlds$, is defined as follows:
%
\[	
    \begin{rclarray}
	\guarE & \eqdef &
	\myset{
		((\ls_{f} \composeLS \ls, \fpgs_{p}), (\ls_{f} \composeLS (\unitFP, \ca), \fpgs_{q}))	
	}{
		\exsts{\rid, \ca'}
		\dom(\fpgs_{q}) = \dom(\fpgs_{p}) \uplus \Set{\rid} \\
		\qquad \land\ \for{\rid'} \rid \ne \rid' \implies \fpgs_{p}(\rid') = \fpgs_{q}(\rid') \\
		\qquad \land\ \fpgs_{q}(\rid) = \ls \composeLS (\unitFP, \ca')
		\land \dom(\ca) = \dom(\ca') = \Set{\rid}
	}
    \end{rclarray}
\]
% 
The \emph{guarantee} relation, $\myguar: \FPWorlds \times \FPWorlds$, is defined as follows:
%
\[
	\myguar \eqdef (\guarU \cup \guarE)^{\scalebox{1.1}{*}}
\]
%
\end{definition}

\subsection{Rules for Global}

The \rl{PRCommit} rule lifts the local effect of transaction \( \trans \) to global level by repartition \( \repartition{\gpre}{\gpost}{\lpre}{\lpost} \).
The repartition stripes off the fingerprints but uses the fingerprints to merge the local effect and the interference.
This is, the environment is allowed to write to locations that are different from the ones by transaction \( \trans \).
%
\[
    \infer[\rl{PRCommit}]{%
        \judgeG{\gpre}{ \ptrans{\trans} }{\gpost}
    }{%
        \judgeL{\lpre}{\trans}{\lpost} &&
        \repartition{\gpre}{\gpost}{\lpre}{\lpost}
    }
\]

\begin{definition}[Repartitioning]
The \emph{repartitioning} is defined as follows:
\[
    \begin{rclarray}
        \mrepartition{\setworld_{p}}{\setworld_{q}}{\setfph_{p}}{\setfph_{q}} & \iffdef &
        \begin{array}[t]{@{} l @{}}
            \for{\world_{p} \in \setworld_{p}} \exsts{\fpw_{p}, \fph_{p}, \fph_{f}}\\
            \quad \fpw_{p} = \liftW{\world_{p}} \land \flattenFW{\fpw_{p}} = (\fph_{p} \composeFP \fph_{f}, \unitC) \land \fph_{p} \in \setfph_{p} \\
            \quad \land\ \for{\fph_{q} \in \setfph_{q}} \exsts{\world_{q}, \fpw_{q}} \\
            \qquad \flattenFW{\fpw_{q}} = (\fph_{q} \composeFP \fph_{f}, \unitC) \land (\fpw_{p}, \fpw_{q}) \in \myguar \land \eraseFW{\fpw_{q}} = \world_{q} \land \world_{q} \in \setworld_{q} \\
            \qquad \land\ \for{\fpw \in \mergeR{\fpw_{p}}{\fpw_{q}}{\myrely}} \eraseFW{\fpw} \in \setworld_{q}
        \end{array}
    \end{rclarray}
\]
with, $\mergeName[\scalebox{.5}{\(\myrely\)}]: \FPWorlds \times \FPWorlds \times \powerset{\FPWorlds \times \FPWorlds} \to \powerset{\FPWorlds}$, defined as follows, for all $\fpw_{p}, \fpw_{q} \in \FPWorlds$:
%
\[
	\mergeR{\fpw_{p}}{\fpw_{q}}{\myrely} \eqdef \bigcup\limits_{\fpw \in \myrely(\fpw_{p})} \mergeFW{\fpw_{p}}{\fpw_{q}}{\fpw}
\]
%
\end{definition}

\section{Examples\label{sec:example}}

\sx{New observation:
\begin{itemize}
\item We might want to have some assertion to say the initial values of a region.
\item What is the meaning of sequential composition since some consistency model does not have strong session guarantee, maybe the answer is the ``commit'' order.
\item Explain stablisation in a roughly syntactic level.
\end{itemize}
}

\subsection{Single increment and multi-reader.}
\[
    \begin{array}{@{}l@{}}
        \boxass{\V{x} \pt \V{\nat}}{\lrid}{\intass} \\
        \C{Inc} \composeK \C{Inc} \text{ is undefined} \\
        \C{Rd} \text{ is the unit element} \\
    \end{array}
\]
\subsubsection{SER}
\[
    \begin{array}{@{}l@{}}
        \intass : 
        \begin{rclarray}[t]
        \C{Inc} & : & \exsts{\V{m, k}} \Set{(\etR, \V{x}, \V{m}), (\etW, \V{x}, \V{m} + 1)} \mat \V{x} \pt \V{k} \oassto \V{x} \pt \V{k} \\
        \C{Rd}  & : & \exsts{\V{m, k, v}} \Set{(\etR, \V{x}, \V{m})} \mat \V{x} \pt \V{k} \oassto \V{x} \pt \V{k} \\ 
        \end{rclarray} \\
    \end{array}
\]

\[
\begin{session}
\specline{\boxass{\vx \pt 0}{\lrid}{\intass} \sep \cass{\C{Inc}}{\lrid} } \\
\specline{\boxass{\vx \pt 0}{\lrid}{\intass} \sep \cass{\C{Inc}}{\lrid} \sep \cass{\C{Rd}}{\lrid} \sep \cass{\C{Rd}}{\lrid} } \\
\begin{parl}
    \begin{session}
    \specline{\boxass{\vx \pt 0}{\lrid}{\intass} \sep \cass{\C{Inc}}{\lrid} \sep \cass{\C{Rd}}{\lrid} } \\
    \begin{transaction}
        \specline{ \vx \pt 0 } \\
        \pderef{\pvar{a}}{\vx} ; \\
        \specline{ \vx \pt 0 \land \pvar{a} = 0 \sep \Set{(\etR, \vx, 0)} } \\
        \pmutate{\vx}{\pvar{a} + 1} ; \\
        \specline{ \vx \pt 1 \land \pvar{a} = 0 \\
                {} \sep \Set{(\etR, \vx, 0), (\etW, \vx, 1)} } \\
    \end{transaction} \\
    \specline{\boxass{\vx \pt 1}{\lrid}{\intass} \sep \cass{\C{Inc}}{\lrid} \sep \cass{\C{Rd}}{\lrid} } \\
    \begin{transaction}
        \specline{ \vx \pt 1 } \\
        \pderef{\pvar{b}}{\vx} ; \\
        \specline{ \lor \vx \pt 1 \sep \Set{(\etR, \vx, 1)} } \\
    \end{transaction} \\
    \specline{\boxass{\vx \pt 1}{\lrid}{\intass} \sep \cass{\C{Inc}}{\lrid} \sep \cass{\C{Rd}}{\lrid} } \\
    \begin{transaction}
        \specline{ \vx \pt 1 } \\
        \pderef{\pvar{a}}{\vx} ; \\
        \specline{ \vx \pt 1 \land \pvar{a} = 1 \sep \Set{(\etR, \vx, 1)} } \\
        \pmutate{\vx}{\pvar{a} + 1} ; \\
        \specline{ \vx \pt 2 \land \pvar{a} = 1 \\
                {} \sep \Set{(\etR, \vx, 1), (\etW, \vx, 2)} } \\
    \end{transaction} \\
    \specline{\boxass{\vx \pt 2}{\lrid}{\intass} \sep \cass{\C{Inc}}{\lrid} \sep \cass{\C{Rd}}{\lrid} } \\
    \end{session}
    &
    \begin{session}
    \specline{\exsts{\V{n}}\boxass{\vx \pt \V{n}}{\lrid}{\intass} \land \V{n} \geq 0 \sep \cass{\C{Rd}}{\lrid} } \\
    \begin{transaction}
        \specline{ \exsts{ \V{v} } \vx \pt \V{v} { \color{gray} \land \V{v} = \V{n} } } \\
        \pderef{\pvar{c}}{\vx} ; \\
        \specline{ \exsts{ \V{v} } \vx \pt \V{v} \sep \Set{(\etR, \vx, \V{v})} { \color{gray} \land \V{v} = \V{n} } } \\
    \end{transaction} \\
    \specline{\exsts{\V{n}}\boxass{\vx \pt \V{n}}{\lrid}{\intass} \land \V{n} \geq 0 \sep \cass{\C{Rd}}{\lrid} } \\
    \end{session}
\end{parl} \\
\specline{\boxass{\vx \pt 2}{\lrid}{\intass} \sep \cass{\C{Inc}}{\lrid} \sep \cass{\C{Rd}}{\lrid} \sep \cass{\C{Rd}}{\lrid} } \\
\specline{\boxass{\vx \pt 2}{\lrid}{\intass} \sep \cass{\C{Inc}}{\lrid} } \\
\end{session}
\]
\subsubsection{SI/PSI}
\[
    \begin{array}{@{}l@{}}
        \intass : 
        \begin{rclarray}[t]
        \C{Inc} & : & \exsts{\V{m, k}} \Set{(\etR, \V{x}, \V{m}), (\etW, \V{x}, \V{m} + 1)} \mat \V{x} \pt \V{k} \oassto \V{x} \pt \V{k} \\
        \C{Rd}  & : & \exsts{\V{m, k, v}} \Set{(\etR, \V{x}, \V{m})} \mat \V{x} \pt \V{k} \oassto \V{x} \pt \V{v} \land \V{v} \leq \V{k} \\ 
        \end{rclarray} \\
        \C{Inc} \composeK \C{Inc} \text{ is undefined} \\
        \C{Rd} \text{ is the unit element} \\
    \end{array}
\]

\[
\begin{session}
\specline{\boxass{\vx \pt 0}{\lrid}{\intass} \sep \cass{\C{Inc}}{\lrid} } \\
\specline{\boxass{\vx \pt 0}{\lrid}{\intass} \sep \cass{\C{Inc}}{\lrid} \sep \cass{\C{Rd}}{\lrid} \sep \cass{\C{Rd}}{\lrid} } \\
\begin{parl}
    \begin{session}
    \specline{\boxass{\vx \pt 0}{\lrid}{\intass} \sep \cass{\C{Inc}}{\lrid} \sep \cass{\C{Rd}}{\lrid} } \\
    \begin{transaction}
        \specline{ \vx \pt 0 } \\
        \pderef{\pvar{a}}{\vx} ; \\
        \specline{ \vx \pt 0 \land \pvar{a} = 0 \sep \Set{(\etR, \vx, 0)} } \\
        \pmutate{\vx}{\pvar{a} + 1} ; \\
        \specline{ \vx \pt 1 \land \pvar{a} = 0 \\
                {} \sep \Set{(\etR, \vx, 0), (\etW, \vx, 1)} } \\
    \end{transaction} \\
    \specline{\boxass{\vx \pt 1}{\lrid}{\intass} \sep \cass{\C{Inc}}{\lrid} \sep \cass{\C{Rd}}{\lrid} } \\
    \begin{transaction}
        \specline{ {\color{purple} \vx \pt 0} \lor \vx \pt 1 } \\
        \pderef{\pvar{b}}{\vx} ; \\
        \specline{ { \color{purple} \vx \pt 0 \sep \Set{(\etR, \vx, 0)} }  \\
                    {} \lor \vx \pt 1 \sep \Set{(\etR, \vx, 1)} } \\
    \end{transaction} \\
    \specline{\boxass{\vx \pt 1}{\lrid}{\intass} \sep \cass{\C{Inc}}{\lrid} \sep \cass{\C{Rd}}{\lrid} } \\
    \begin{transaction}
        \specline{ \vx \pt 1 } \\
        \pderef{\pvar{a}}{\vx} ; \\
        \specline{ \vx \pt 1 \land \pvar{a} = 1 \sep \Set{(\etR, \vx, 1)} } \\
        \pmutate{\vx}{\pvar{a} + 1} ; \\
        \specline{ \vx \pt 2 \land \pvar{a} = 1 \\
                {} \sep \Set{(\etR, \vx, 1), (\etW, \vx, 2)} } \\
    \end{transaction} \\
    \specline{\boxass{\vx \pt 2}{\lrid}{\intass} \sep \cass{\C{Inc}}{\lrid} \sep \cass{\C{Rd}}{\lrid} } \\
    \end{session}
    &
    \begin{session}
    \specline{\exsts{\V{n}}\boxass{\vx \pt \V{n}}{\lrid}{\intass} \land \V{n} \geq 0 \sep \cass{\C{Rd}}{\lrid} } \\
    \begin{transaction}
        \specline{ \exsts{ \V{v} } \vx \pt \V{v} { \color{gray} \land \V{v} \leq \V{n} } } \\
        \pderef{\pvar{c}}{\vx} ; \\
        \specline{ \exsts{ \V{v} } \vx \pt \V{v} \sep \Set{(\etR, \vx, \V{v})} { \color{gray} \land \V{v} \leq \V{n} } } \\
    \end{transaction} \\
    \specline{\exsts{\V{n}}\boxass{\vx \pt \V{n}}{\lrid}{\intass} \land \V{n} \geq 0 \sep \cass{\C{Rd}}{\lrid} } \\
    \end{session}
\end{parl} \\
\specline{\boxass{\vx \pt 1}{\lrid}{\intass} \sep \cass{\C{Inc}}{\lrid} \sep \cass{\C{Rd}}{\lrid} \sep \cass{\C{Rd}}{\lrid} } \\
\specline{\boxass{\vx \pt 1}{\lrid}{\intass} \sep \cass{\C{Inc}}{\lrid} } \\
\end{session}
\]

\subsubsection{Causal}

\[
    \begin{array}{@{}l@{}}
        \intass : 
        \begin{rclarray}[t]
        \C{Inc} & : & \exsts{\V{m, k}} \Set{(\etR, \V{x}, \V{m}), (\etW, \V{x}, \V{m} + 1)} \mat \V{x} \pt \V{k} \oassto \V{x} \pt \V{v} \\
        \C{Rd}  & : & \exsts{\V{m, k, v}} \Set{(\etR, \V{x}, \V{m})} \mat \V{x} \pt \V{k} \oassto \V{x} \pt \V{v} \\ 
        \end{rclarray} \\
        \C{Inc} \composeK \C{Inc} \text{ is undefined} \\
        \C{Rd} \text{ is the unit element} \\
    \end{array}
\]

\[
\begin{session}
\specline{\boxass{\vx \pt 0}{\lrid}{\intass} \sep \cass{\C{Inc}}{\lrid} } \\
\specline{\boxass{\vx \pt 0}{\lrid}{\intass} \sep \cass{\C{Inc}}{\lrid} \sep \cass{\C{Rd}}{\lrid} \sep \cass{\C{Rd}}{\lrid} } \\
\begin{parl}
    \begin{session}
    \specline{\boxass{\vx \pt 0}{\lrid}{\intass} \sep \cass{\C{Inc}}{\lrid} \sep \cass{\C{Rd}}{\lrid} } \\
    \begin{transaction}
        \specline{ \exsts{ \V{v} } \vx \pt \V{v} } \\
        \pderef{\pvar{a}}{\vx} ; \\
        \specline{ \exsts{ \V{v} } \vx \pt \V{v} \land \pvar{a} = \V{v} \sep \Set{(\etR, \vx, \V{v})} } \\
        \pmutate{\vx}{\pvar{a} + 1} ; \\
        \specline{ \vx \pt \V{v} + 1 \land \pvar{a} = \V{v} \\
                {} \sep \Set{(\etR, \vx, \V{v}), (\etW, \vx, \V{v} + 1)} } \\
    \end{transaction} \\
    \specline{\exsts{\V{n}}\boxass{\vx \pt \V{n}}{\lrid}{\intass} \sep \cass{\C{Inc}}{\lrid} \sep \cass{\C{Rd}}{\lrid} } \\
    \begin{transaction}
        \specline{ \exsts{ \V{v} } \vx \pt \V{v} } \\
        \pderef{\pvar{c}}{\vx} ; \\
        \specline{ \exsts{ \V{v} } \vx \pt \V{v} \sep \Set{(\etR, \vx, \V{v})} } \\
    \end{transaction} \\
    \specline{\exsts{\V{n}}\boxass{\vx \pt \V{n}}{\lrid}{\intass} \sep \cass{\C{Inc}}{\lrid} \sep \cass{\C{Rd}}{\lrid} } \\
    \begin{transaction}
        \specline{ \exsts{ \V{v} } \vx \pt \V{v} } \\
        \pderef{\pvar{a}}{\vx} ; \\
        \specline{ \exsts{ \V{v} } \vx \pt \V{v} \land \pvar{a} = \V{v} \sep \Set{(\etR, \vx, \V{v})} } \\
        \pmutate{\vx}{\pvar{a} + 1} ; \\
        \specline{ \vx \pt \V{v} + 1 \land \pvar{a} = \V{v} \\
                {} \sep \Set{(\etR, \vx, \V{v}), (\etW, \vx, \V{v} + 1)} } \\
    \end{transaction} \\
    \specline{\exsts{\V{n}}\boxass{\vx \pt \V{n}}{\lrid}{\intass} \sep \cass{\C{Inc}}{\lrid} \sep \cass{\C{Rd}}{\lrid} } \\
    \end{session}
    &
    \begin{session}
    \specline{\exsts{\V{n}}\boxass{\vx \pt \V{n}}{\lrid}{\intass} \sep \cass{\C{Rd}}{\lrid} } \\
    \begin{transaction}
        \specline{ \exsts{ \V{v} } \vx \pt \V{v} { \color{gray} \land \V{v}, \V{n} } } \\
        \pderef{\pvar{c}}{\vx} ; \\
        \specline{ \exsts{ \V{v} } \vx \pt \V{v} \sep \Set{(\etR, \vx, \V{v})} { \color{gray} \land \V{v}, \V{n} } } \\
    \end{transaction} \\
    \specline{\exsts{\V{n}}\boxass{\vx \pt \V{n}}{\lrid}{\intass} \sep \cass{\C{Rd}}{\lrid} } \\
    \end{session}
\end{parl} \\
\specline{\exsts{\V{n}}\boxass{\vx \pt \V{n}}{\lrid}{\intass} \sep \cass{\C{Inc}}{\lrid} \sep \cass{\C{Rd}}{\lrid} \sep \cass{\C{Rd}}{\lrid} } \\
\specline{\exsts{\V{n}}\boxass{\vx \pt \V{n}}{\lrid}{\intass} \sep \cass{\C{Inc}}{\lrid} } \\
\end{session}
\]

\subsection{Two associated bank accounts}
\[
    \begin{array}{@{}l@{}}
        \boxass{\V{x} \pt \V{n} \sep \V{y} \pt \V{m} }{\lrid}{\intass} \\
        \C{xx} \composeK \C{xx} \text{ is undefined} \\
        \C{yy} \composeK \C{yy} \text{ is undefined} \\
        \C{Rd} \text{ is the unit} \\
    \end{array}          
\]

\subsubsection{SER}
\[
    \begin{array}{@{}l@{}}
        \intass : 
        \begin{rclarray}[t]
        \C{xx} & : & \exsts{\V{v, k, a, b }} \Setcon{(\etR, \V{x}, \V{v}), (\etR, \V{y}, \V{k}), (\etW, \V{x}, \V{v} - 100)}{\V{v} + \V{k} \geq 100} \\
        & & \qqquad \mat \V{x} \pt \V{a} \sep \V{y} \pt \V{b} \oassto \V{x} \pt \V{a} \sep \V{y} \pt \V{b} \land \V{a} + \V{b} \geq 0 \\
        \C{yy} & : & \exsts{\V{v, k, a, b }} \Setcon{(\etR, \V{x}, \V{v}), (\etR, \V{y}, \V{k}), (\etW, \V{y}, \V{k} - 100)}{\V{v} + \V{k} \geq 100} \\
        & & \qqquad \mat \V{x} \pt \V{a} \sep \V{y} \pt \V{b} \oassto \V{x} \pt \V{a} \sep \V{y} \pt \V{b} \land \V{a} + \V{b} \geq 0 \\
        \C{Rd} & : & \exsts{\V{v, k, a, b }} \Set{(\etR, \V{x}, \V{v}), (\etR, \V{y}, \V{k})} \\
        & & \qqquad \mat \V{x} \pt \V{a} \sep \V{y} \pt \V{b} \oassto \V{x} \pt \V{a} \sep \V{y} \pt \V{b} \land \V{a} + \V{b} \geq 0 \\
        \end{rclarray} \\
    \end{array}
\]

\[
\begin{session}
\specline{ \boxass{ \vx \pt 60 \sep \vy \pt 60 }{\lrid}{\intass} \sep \cass{\C{xx}}{\lrid} \sep \cass{\C{yy}}{\lrid} } \\
\begin{parl}
    \begin{session}
        \specline{ \boxass{ \vx \pt 60 \sep \vy \pt 60 \lor \vx \pt 60 \sep \vy \pt -40 }{\lrid}{\intass} \sep \cass{\C{xx}}{\lrid} } \\
        \begin{transaction}
            \specline{\vx \pt 60 \sep \vy \pt 60 \\ {} \lor \vx \pt 60 \sep \vy \pt -40} \\
            \pderef{\pvar{a}}{\vx}; \\
            \pderef{\pvar{b}}{\vy}; \\
            \pifs{\pvar{a} + \pvar{b} \geq 100} \\
            \quad \pmutate{\vx}{\pvar{a} - 100} ; \\
            \pife \\
            \specline{\vx \pt-40 \sep \vy \pt 60 \sep {} \\
            \Set{(\etR, \vx, 60), (\etR, \vy, 60), (\etW, \vx, -40)} \\ 
            {} \lor \vx \pt 60 \sep \vy \pt -40 \sep {} \\
            \Set{(\etR, \vx, 60), (\etR, \vy, -40)} }
        \end{transaction} \\
        \specline{ \boxass{ \vx \pt -40 \sep \vy \pt 60 \lor \vx \pt 60 \sep \vy \pt -40 }{\lrid}{\intass} \sep \cass{\C{xx}}{\lrid} } \\
    \end{session}
    &
    \begin{session}
        \specline{ \boxass{ \vx \pt -40 \sep \vy \pt 60 \lor \vx \pt 60 \sep \vy \pt 60 }{\lrid}{\intass} \sep \cass{\C{yy}}{\lrid} } \\
        \begin{transaction}
            \pderef{\pvar{a}}{\vx}; \\
            \pderef{\pvar{b}}{\vy}; \\
            \pifs{\pvar{a} + \pvar{b} \geq 100} \\
            \quad \pmutate{\vy}{\pvar{b} - 100} ; \\
            \pife 
        \end{transaction} \\
        \specline{ \boxass{ \vx \pt -40 \sep \vy \pt 60 \lor \vx \pt 60 \sep \vy \pt -40 }{\lrid}{\intass} \sep \cass{\C{yy}}{\lrid} } \\
    \end{session}
\end{parl} \\
\specline{ \boxass{ \vx \pt -40 \sep \vy \pt 60 \lor \vx \pt 60 \sep \vy \pt -40 }{\lrid}{\intass} \sep \cass{\C{xx}}{\lrid} \sep \cass{\C{yy}}{\lrid} } \\
\end{session}
\]

\subsubsection{SI/PSI}
\[
    \begin{array}{@{}l@{}}
        \intass : 
        \begin{rclarray}[t]
        \C{xx} & : & \exsts{\V{v, k, a, b, c }} \Setcon{(\etR, \V{x}, \V{v}), (\etR, \V{y}, \V{k}), (\etW, \V{x}, \V{v} - 100)}{\V{v} + \V{k} \geq 100} \\
        & & \qqquad \mat \V{x} \pt \V{a} \sep \V{y} \pt \V{b} \oassto \V{x} \pt \V{a} \sep \V{y} \pt \V{b} + ( \V{c} \times 100 ) \\
        \C{yy} & : & \exsts{\V{v, k, a, b, c }} \Setcon{(\etR, \V{x}, \V{v}), (\etR, \V{y}, \V{k}), (\etW, \V{y}, \V{k} - 100)}{\V{v} + \V{k} \geq 100} \\
        & & \qqquad \mat \V{x} \pt \V{a} \sep \V{y} \pt \V{b} \oassto \V{x} \pt \V{a} + ( \V{c} \times 100 ) \sep \V{y} \pt \V{b} \\
        \C{Rd} & : & \exsts{\V{v, k, a, b, c, d }} \Set{(\etR, \V{x}, \V{v}), (\etR, \V{y}, \V{k})} \\
        & & \qqquad \mat \V{x} \pt \V{a} \sep \V{y} \pt \V{b} \oassto \V{x} \pt \V{a} + ( \V{c} \times 100 ) \sep \V{y} \pt \V{b} + ( \V{d} \times 100 ) \\
        \end{rclarray} \\
    \end{array}
\]

\[
\begin{session}
\specline{ \boxass{\vx \pt 60 \sep \vy \pt 60 }{\lrid}{\intass} \sep \cass{\C{xx}}{\lrid} \sep \cass{\C{yy}}{\lrid} } \\
\begin{parl}
    \begin{session}
        \specline{ \exsts{ \V{n} } \boxass{\vx \pt 60 \sep \vy \pt 60 - \V{n} \times 100 }{\lrid}{\intass} \sep \cass{\C{xx}}{\lrid} } \\
        \begin{transaction}
            \specline{ \exsts{ \V{n}, \V{k} \geq 0 } \vx \pt 60 + \V{k} \times 100 \sep \vy \pt 60 + \V{n} \times 100 } \\
            \pderef{\pvar{a}}{\vx}; \\
            \pderef{\pvar{b}}{\vy}; \\
            \specline{ \exsts{ \V{n}, \V{k} \geq 0 } \vx \pt 60 + \V{k} \times 100 \sep \vy \pt 60 + \V{n} \times 100 \\
                        {} \sep \Set{ (\etR, \vx, 60 + \V{k} \times 100), (\etR, \vx, 60 + \V{n} \times 100) } \\
                        {} \land \pvar{a} = 60 + \V{k} \times 100 \land \pvar{b} = 60 + \V{n} \times 100 } \\
            \pifs{\pvar{a} + \pvar{b} \geq 100} \\
            \quad \specline{ \exsts{ \V{n}, \V{k} \geq 0 } \vx \pt 60 + \V{k} \times 100 \sep \vy \pt 60 + \V{n} \times 100 \\
                            {} \sep \Set{ (\etR, \vx, 60 + \V{k} \times 100), (\etR, \vx, 60 + \V{n} \times 100) } \\
                            {} \land \pvar{a} = 60 + \V{k} \times 100 \land \pvar{b} = 60 + \V{n} \times 100 \land \V{k} + \V{N} \geq 0} \\
            \quad \pmutate{\vx}{\pvar{a} - 100} ; \\
            \quad \specline{ \exsts{ \V{n}, \V{k} \geq 0 } \vx \pt -40 + \V{k} \times 100 \sep \vy \pt 60 + \V{n} \times 100 \\
                            {} \sep \Set{ (\etR, \vx, 60 + \V{k} \times 100), (\etR, \vx, 60 + \V{n} \times 100), \\ 
                                                    (\etW, \vx, -40 + \V{k} \times 100) } \\
                            {} \land \pvar{a} = 60 + \V{k} \times 100 \land \pvar{b} = 60 + \V{n} \times 100 \land \V{k} + \V{N} \geq 0} \\
            \pife \\
            \comment{Weaken the assertion by } \\
            \comment{throwing away program variables.} \\
            \specline{ \exsts{ \V{n}, \V{k} \geq 0 } \vy \pt 60 + \V{n} \times 100 \\
                        {} \sep 
                        \begin{formulea}
                        \vx \pt -40 + \V{k} \times 100 
                        \land \V{k} + \V{N} \geq 0 \\
                        {} \sep \Set{ (\etR, \vx, 60 + \V{k} \times 100), (\etR, \vx, 60 + \V{n} \times 100), \\
                                                (\etW, \vx, -40 + \V{k} \times 100) } \\
                        \end{formulea} \\
                        {} \lor 
                        \begin{formulea}
                        \vx \pt 60 + \V{k} \times 100 \\ 
                        {} \sep \Set{ (\etR, \vx, 60 + \V{k} \times 100), (\etR, \vx, 60 + \V{n} \times 100) } 
                        \end{formulea}
                    } \\
        \end{transaction} \\
        \comment{To allow the write to be committed,} \\
        \comment{the \V{K} must be 0 } \\
        \specline{ \exsts{ \V{n} } \boxass{ ( \vx \pt 60 \lor \vx \pt -40 ) \sep \vy \pt 60 - \V{n} \times 100 }{\lrid}{\intass} \sep \cass{\C{xx}}{\lrid} } \\
    \end{session}
    &
    \begin{session}
        \specline{ \exsts{ \V{n} } \boxass{\vx \pt 60 - \V{n} \times 100 \sep \vy \pt 60 }{\lrid}{\intass} \sep \cass{\C{yy}}{\lrid} } \\
        \begin{transaction}
            \pderef{\pvar{a}}{\vx}; \\
            \pderef{\pvar{b}}{\vy}; \\
            \pifs{\pvar{a} + \pvar{b} \geq 100} \\
            \quad \pmutate{\vy}{\pvar{b} - 100} ; \\
            \pife 
        \end{transaction} \\
        \specline{ \exsts{ \V{n} } \boxass{\vx \pt 60 - \V{n} \times 100 \sep ( \vy \pt 60 \lor \vy \pt -40 ) }{\lrid}{\intass} \sep \cass{\C{yy}}{\lrid} } \\
    \end{session}
\end{parl} \\
\specline{ \boxass{\vx \pt -40 \sep \vy \pt 60 \lor \vx \pt 60 \sep \vy \pt -40 \lor \vx \pt -40 \sep \vy \pt -40 }{\lrid}{\intass} \sep \cass{\C{xx}}{\lrid} \sep \cass{\C{yy}}{\lrid} } \\
\end{session}
\]

%A dummy bank transfer example that is not serialisable.
%\[
    %\begin{rclarray}
        %\intass(\rid) & = &
        %\begin{cases}
            %\unitelem{} : \exsts{x, m, k} x \fpt{\fp} n \sep \cass{S(m)}{\rid} \transfersto x \fpt{\addFPW{\fp}} n \pm k \sep  \cass{S(m \pm k)}{\rid} \\
            %\unitelem{} : \exsts{x} x \fpt{\fp} n \transfersto x \fpt{\addFPR{\fp}} n 
        %\end{cases} \\
        %S(m) \composeK S(n) & = & S(m+n) \\
        %S(0) & \in & \unitK \\
    %\end{rclarray}
%\]
%A dummy bank transfer example that is serialisable even under snapshot isolation.
%\[
    %\begin{rclarray}
        %\intass & = &
        %\begin{cases}
            %\unitelem{} : \exsts{x, y, m, k} x \fpt{\fp} n \sep  y \fpt{\fp'} m \transfersto x \fpt{\addFPW{\fp}} n - k \sep \fpt{\addFPW{\fp}} m + k  \\
            %\unitelem{} : \exsts{x} x \fptEMP n \transfersto x \fptR n 
        %\end{cases}
    %\end{rclarray}
%\]
%If x write y and if y write x example.
%\[
    %\begin{rclarray}
        %\intass(x,y) & = &
        %\begin{cases}
            %\perm{L} : x \fptEMP 0 \sep y \fptEMP 0 \transfersto x \fptW 1 \sep y \fptR 0 \\
            %\perm{R} : x \fptEMP 0 \sep y \fptEMP 0 \transfersto x \fptR 0 \sep y \fptW 1 \\
        %\end{cases}
    %\end{rclarray}
%\]


\bibliography{bibliography}

%\newpage
%\appendix
%\section{Semantics\label{sec:proof_semantics}}
\subsection{Soundness of the time-stamp semantics}
\begin{lem}[No overwrite]
    \label{lem:no-over-write}
    For a given address and time, if there is a value, threads cannot overwrite the values.
    \[ 
        \for{ \stk, \stk', \tshp, \tshp', \ts, \ts', \iprog, {\iprog}' ,\lb,\addr, \ts } 
        ( \stk, \tshp, \ts ), \iprog \toT{\lb} ( \stk', \tshp', \ts' ), {\iprog}'
        \implies \tshp(\addr)(\ts)\isundef \lor \tshp(\addr)(\ts) = \tshp'(\addr)(\ts)
    \]
\end{lem}
\begin{proof}
    Prove by structural induction on the semantics.
    For base cases \rl{PChoiseLeft}, \rl{PChoiseRight}, \rl{PLoop}, \rl{PSeqSkip}, \rl{PPar} and \rl{Pwait}, they are trivial because those rules do not change time-stamp heap, i.e.\ \( \tshp = \tshp' \).
    For base case \rl{Commit}, the new state \( \tshp' = \commit{\tshp}{\fph_{s}}{\fph_{e}}{\tsid}{\ts_{s}}{\ts_{e}} \) for some \( \fph_{s},\fph_{e}, \tsid, \ts_{s},\ts_{e} \).
    For any addresses updated by the new transaction \( \tsid \), It must pick times \( \ts_{s} \) and \( \ts_{e} \) when the values of those addresses are undefined, because of the constrain \( \cancommitName \), especially the sub-predicate \( \wfhistName \).
    For the inductive case \rl{PSeq}, prove directly by the inductive hypothesis.
\end{proof}

\begin{lem}[Start before end]
    \label{lem:start-before-end}
    \label{lem:read-before-write}
    All the reads/starts operations of a transaction happen before all the writes/ends. 
    Formally,
    \[
        \for{ \stk, \stk', \tshp, \tshp', \ts, \ts', \iprog, {\iprog}' ,\lb } 
        \pred{rw}{\tshp} 
        \land ( \stk, \tshp, \ts ), \iprog \toT{\lb} ( \stk', \tshp', \ts' ), {\iprog}' 
        \implies \pred{rw}{\tshp'}
    \]
    where,
    \[
        \begin{rclarray}
            \pred{rw}{\tshp} & \defeq & 
            \for{ \addr, \addr', \ts, \ts',\etag \in \Set{\etS, \etR}, \etag' \in \Set{\etE,\etW}, \tsid } 
            \tshp(\addr)(\ts) = (\stub, \etag, \tsid) 
            \land \tshp(\addr')(\ts') = (\stub, \etag', \tsid)
            \implies \ts < \ts' 
        \end{rclarray}
    \]
\end{lem}
\begin{proof}
    Prove by structural induction on the semantics.
    For base cases \rl{PChoiseLeft}, \rl{PChoiseRight}, \rl{PLoop}, \rl{PSeqSkip}, \rl{PPar} and \rl{PWait}, they are trivial because those rules do not change time-stamp heap, i.e.\ \( \tshp = \tshp' \).
    For base case \rl{Commit}, the new state \( \tshp' = \commit{\tshp}{\fph_{s}}{\fph_{e}}{\tsid}{\ts_{s}}{\ts_{e}} \) for some \( \fph_{s},\fph_{e}, \tsid, \ts_{s},\ts_{e} \). 
    First by Lemma \ref{lem:no-over-write}, for those transactions in \( \tshp \) it remains the same as in \( \tshp' \).
    Then, the \(\commitName\) function associates \( \etR \) or \( \etS \) to time \( \ts_{s} \), and \( \etW \) or \( \etE \) to \( \ts_{e} \), and \( \ts_{s} < \ts_{e} \), so \( \pred{rw}{\tshp'}\) holds.
    For the inductive case \rl{PSeq}, prove directly by the inductive hypothesis.
\end{proof}

\begin{lem}[Pair of start and end]
    \label{lem:start-end-pair}
    Reads/starts operations are writes/ends operations appear as pairs.
    \[
        \for{ \stk, \stk', \tshp, \tshp', \ts, \ts', \iprog, {\iprog}' ,\lb }
        \pred{pair}{\tshp} 
        \land ( \stk, \tshp, \ts ), \iprog \toT{\lb} ( \stk', \tshp', \ts' ), {\iprog}' 
        \implies \pred{pair}{\tshp'}
    \]
    where,
    \[
        \begin{rclarray}
            \pred{pair}{ \tshp } & \defeq & 
            \for{ \addr, \addr', \ts, \ts',\etag \in \Set{\etS, \etR}, \etag' \in \Set{\etE,\etW}, \tsid } 
            (\tshp(\addr)(\ts) = (\stub, \etag, \tsid) \iff \tshp(\addr')(\ts') = (\stub, \etag', \tsid)  )
        \end{rclarray}
    \]
\end{lem}
\begin{proof}
    Prove by structural induction on the semantics.
    For base cases \rl{PChoiseLeft}, \rl{PChoiseRight}, \rl{PLoop}, \rl{PSeqSkip}, \rl{PPar} and \rl{PWait}, they are trivial because those rules do not change time-stamp heap, i.e.\ \( \tshp = \tshp' \).
    For base case \rl{Commit}, the new state \( \tshp' = \commit{\tshp}{\fph_{s}}{\fph_{e}}{\tsid}{\ts_{s}}{\ts_{e}} \) for some \( \fph_{s},\fph_{e}, \tsid, \ts_{s},\ts_{e} \). 
    First by Lemma \ref{lem:no-over-write}, for those transactions in \( \tshp \) it remains the same as in \( \tshp' \).
    Then the \(\commitName\) function always extends addresses with pairs of operations, so \( \pred{rw}{\tshp'}\) holds.
    For the inductive case \rl{PSeq}, prove directly by the inductive hypothesis.
\end{proof}


\begin{lem}[Start/end at the same time]
    \label{lem:atoic-rw}
    \label{lem:happen-in-same-time}
    All the reads/starts operations among all addresses of a transaction happen in the same time, so do all the writes/ends operations. 
    \[
        \for{ \stk, \stk', \tshp, \tshp', \ts, \ts', \iprog, {\iprog}' ,\lb } 
        \pred{atom}{\tshp} 
        \land ( \stk, \tshp, \ts ), \iprog \toT{\lb} ( \stk', \tshp', \ts' ), {\iprog}' 
        \implies \pred{atom}{\tshp'}
    \]
    where,
    \[
        \begin{rclarray}
        \pred{atom}{\tshp} & \defeq & 
        \for{ \addr, \addr', \ts, \ts', \tsid, \etag, \etag' }
        \tshp(\addr)(\ts) (\stub, \etag, \tsid) 
        \land \tshp(\addr')(\ts') = (\stub, \etag', \tsid) \\
        & & \land (\etag, \etag' \in \Set{\etS, \etR} \lor \etag, \etag' \in \Set{\etE, \etW} ) 
        \implies \ts = \ts'
        \end{rclarray}
    \]
\end{lem}
\begin{proof}
    Prove by structural induction on the semantics.
    For base cases \rl{PChoiseLeft}, \rl{PChoiseRight}, \rl{PLoop}, \rl{PSeqSkip}, \rl{PPar} and \rl{Pwait}, they are trivial because those rules do not change time-stamp heap, i.e.\ \( \tshp = \tshp' \).
    For base case \rl{Commit}, the new state \( \tshp' = \commit{\tshp}{\fph_{s}}{\fph_{e}}{\tsid}{\ts_{s}}{\ts_{e}} \) for some \( \fph_{s},\fph_{e}, \tsid, \ts_{s},\ts_{e} \). 
    First by Lemma \ref{lem:no-over-write}, for those transactions in \( \tshp \) it remains the same as in \( \tshp' \).
    Then the \( \commitName \) function associates all reads/starts operations, \( \etR \) and \(\etS \), of all addresses to the start time \( \ts_{s} \), and writes/ends operations to the end time \( \ts_{e} \), so \( \pred{atom}{\tshp'}\) holds.
    For the inductive case \rl{PSeq}, prove directly by the inductive hypothesis.
\end{proof}

\begin{defn}[Well-form time-stamp heaps]
\label{def:wf-timestamp-heap}
    The well-formedness of time-stamp heap is defined as those satisfying Lemma \ref{lem:start-before-end}, Lemma \ref{lem:start-end-pair} and Lemma \ref{lem:atoic-rw}.
    \[
        \begin{rclarray}
            \wfH{\tshp} & \defeq & \pred{rw}{\tshp} \land \pred{pair}{\tshp} \land \pred{atom}{\tshp} \\
        \end{rclarray}
    \]
\end{defn}

\begin{defn}[Extended Labels]
\label{def:ext-label}
Given the set of transaction labels \( \Translabel \) (\defin \ref{def:label}) and thread identifiers \( \ThreadID \) (\defin \ref{def:thread_semantics}), the set of extended labels \( \ExtTranslabel \defeq ( \Nat \times \TransID \times \Translabel ) \cup \Set{\lbID} \) is defined as follows:
\[
	\ExtTranslabel ::= \lbID \mid (\thid,\lbC{\tsid}) \mid (\thid,\lbF{\thid',\prog}) \mid (\thid,\lbJ{\thid',\ts})
\]
\end{defn}

For brevity, we will write \( \lbC{\thid, \tsid} \) instead of \( ( \thid,\lbC{\tsid}) \), similarly for \( \lbF{\thid, \thid',\prog}\) and \( \lbJ{\thid, \thid',\ts} \).
Also we will use the same meta-variable \( \lb \) to range over extended transaction labels.

\begin{defn}[Traces]
\label{def:traces}
    Assume lifting the operational semantics in \fig \ref{fig:thread_semantics} and \fig \ref{fig:thread_pool_semantics} to cope with extended transaction label (\defin \ref{def:ext-label}), then given an initial time-stamp heap \( \tshp  \TSHeaps \) (\defin \ref{def:timestamp_heaps}) and a thread pool \( \thpl \) (\defin \ref{def:thread_pools}), a trace \( \trc \in \Trace \defeq \powerset{ \ExtTranslabel } \times (\ExtTranslabel \times \ExtTranslabel) \) is a tuple \( (\setlabels,<) \) that satisfies predicate \( \pred{trace}{\tshp,\thpl,\setlabels,<,\nat} \), for some number \( \nat \).
\[
    \begin{rclarray}
        \func{traces}{\tshp,\thpl,0} & \defeq & \Set{(\emptyset, \emptyset,\tshp,\thpl)} \\
        \func{traces}{\tshp,\thpl,\nat} & \defeq & 
        \Setcon{%
            (\setlabels,<,\tshp',\thpl')
        }{%
            (\tshp'', \thpl'') \toG{\lbID} (\tshp', \thpl' ) 
            \land (\setlabels,<,\tshp'',\thpl'') \in \func{trace}{\tshp,\thpl,\setlabels,<,\nat-1}
        } \\
		& & \uplus \Setcon{
				(\setlabels \uplus \Set{\lb}
                , < \uplus \myset{(\lb',\lb)}{ \lb' \in \setlabels}
                , \tshp',\thpl')
			}{ 
	            (\tshp'', \thpl'') \toG{\lb} (\tshp', \thpl' ) 
                \land \lb \neq \lbID \land {} \\
                (\setlabels,<,\tshp'',\thpl'') \in \func{trace}{\tshp,\thpl,\setlabels,<,\nat-1}
			}  \\
    \end{rclarray}
\]
\end{defn}
\sx{
    Not robust enough against the same events happening twice.
    Might fix later or simply assume all labels are unique by tagged a unique identifier. 
    Could add local time also in the label which should be usefully to give more formal prove of program order is inside visibility.}

\begin{defn}[Program Order]
\label{def:po}
Given a trace \( \trc \in \Trace \) (\defin \ref{def:traces}), the \( \funcn{programOrder} : \powerset{\ExtTranslabel} \times (\ExtTranslabel \times \ExtTranslabel) \to \powerset{\Transactions} \times (\Transactions \times \Transactions) \) function is defined as follows:
\[ 
    \begin{rclarray}
        \func{programOrder}{\setlabels, <} & \defeq & ( \settrans, \po ) \\
    \end{rclarray}
\]
where,
\[ 
    \begin{rclarray}
        \settrans & \equiv & \Setcon{\tsid}{\lbC{\thid,\tsid} \in \setlabels } \\
        \po & \equiv & \myset{(\tsid, \tsid')}{\lbC{\thid,\tsid} < \lbC{\thid,\tsid'} } 
        \uplus \myset{(\tsid, \tsid')}{\lbC{\thid,\tsid} < \lbF{\thid,\thid', \stub} < \lbC{\thid',\tsid'} } \\
            & & \uplus \myset{(\tsid, \tsid')}{\lbC{\thid,\tsid} < \lbJ{\thid',\thid, \stub} < \lbC{\thid',\tsid'} }
    \end{rclarray}
\]
\end{defn}

\begin{defn}[Visibility and partial arbitration]
    \label{def:vis-ptlar}
    The \( \funcn{graph}: \TSHeaps \to ( (\Transactions \times \Transactions) \times (\Transactions \times \Transactions) )  \) function convert a time-stamp heap into corresponding \emph{visibility relation} and \emph{partial arbitration relation}, written \( \vis \) and \( \ptlar \) respectively.
    \[
        \begin{rclarray}
            \func{graph}{\tshp} & \defeq & (\vis,\ptlar) \\
        \end{rclarray}
    \]
    where,
    \[
        \begin{rclarray}
            \vis & \equiv & 
            \Setcon{%
                (\tsid,\tsid')
            }{ %
                \exsts{ \addr, \addr', \ts, \ts', \etag \in \Set{\etW,\etE}, \etag' \in \Set{\etR,\etS} } 
                \ts < \ts' \land {} \\
                \tshp(\addr)(\ts) = (\stub, \etag,\tsid)
                \land \tshp(\addr')(\ts') = (\stub, \etag',\tsid') 
			} \\
            \ptlar & \equiv & 
            \Setcon{%
            (\tsid,\tsid')
            }{%
                \exsts{ \addr, \addr', \ts, \ts', \etag, \etag' \in \Set{\etW,\etE} } 
                \ts < \ts' \land {} \\
                \tshp(\addr)(\ts) = (\stub, \etag,\tsid) 
                \land \tshp(\addr')(\ts') = (\stub, \etag',\tsid') 
			}
        \end{rclarray}
    \]
\end{defn}

Note that all transactions in the visibility or partial arbitration relations, for instance \( (\tsid, \tsid') \in \vis \), they must be in the set of transactions given by program order, meaning \( \tsid, \tsid' \in \settrans \).
For brevity, we might only mention and quantify few elements of this tuple but one should think they are quantified as an entire tuple.
For readability, we use \( (\tsid, \tsid') \in \vis \) or \( \tsid \toVIS \tsid' \) interchangeably, and similar for \( \ar \).

\begin{defn}[Partial graph]
\label{def:partial-graph}
Given the well-formedness of time-stamp heaps (\defin \ref{def:wf-timestamp-heap}), program order (\defin \ref{def:po}), and visibility and arbitration relation (\defin \ref{def:vis-ptlar}), the set of \emph{partial graph} is defined as follows, where the ``partial'' means the arbitration relation is partial.
\[
    \begin{rclarray}
    \PGraphs & \defeq & 
    \Setcon{%
        (\settrans, \po, \vis, \ptlar, \tshp)
    }{ 
        \exsts{\tshp', \thpl'} \wfH{\tshp'}
        \land (\setlabels,<,\tshp,\thpl) \in \bigcup\limits_{\nat \in \Nat}\func{traces}{\tshp',\thpl',\nat} \land {} \\
        (\settrans,\po) = \func{programOrder}{\setlabels,<} 
        \land (\vis,\ptlar) = \func{graph}{\tshp} 
    } 
    \end{rclarray}
\]
\end{defn}

\begin{lem}[Separation]
    \label{lem:seperate}
    If two transactions that are not associated by partial arbitration relation \( \ptlar \), they access different addresses but commit at the same time.
    \[
        \begin{array}{@{}l@{}}
            \for{ \tshp, \ptlar, \tsid, \tsid' } (\tsid, \tsid'), (\tsid',\tsid) \notin \ptlar \\
            \qquad \implies \for{ \addr, \addr', \ts, \ts', \etag, \etag \in \Set{\etW, \etE } }
            \tshp(\addr)(\ts) = ( \stub, \etag, \tsid) 
            \land \tshp(\addr')(\ts') = ( \stub, \etag', \tsid') 
            \implies \ts = \ts'
            \land \addr \neq \addr'
        \end{array}
    \]
\end{lem}
\begin{proof}
    Given \( \ptlar \) in \defin \ref{def:partial-graph}, \( \ts = \ts' \) holds, and then \( \addr \neq \addr' \) is derived from Lemma \ref{lem:atoic-rw}.
\end{proof}


\begin{lem}[Semi-acyclic]
    \label{lem:semi-acyclic}
    Both \( \vis \) and \( \ptlar \) are acyclic.
\end{lem}
\begin{proof}
    Proof by contradiction.
    Assume that there is a circle in \( \vis \), which means that \( \bigwedge\limits_{0 \leq i \leq n} \tsid_{i} \toVIS \tsid_{i+1} \land \tsid_{0} = \tsid_{n} \) for some \( \tsid_{0} \) to \( \tsid_{n}\).
    By Lemma \ref{lem:atoic-rw}, each transaction has a start time and a end time, thus let \( \ts^{s}_{i} \) and \( \ts^{e}_{i} \) denote the start time and end time of the transaction \( \tsid_{i} \).
    By Lemma \ref{lem:read-before-write}, we have \( \ts^{s}_{i} < \ts^{e}_{i} \), and  then by \( \vis \) which is defined in \defin \ref{def:vis-ptlar}, we have \( \ts^{e}_{i} < \ts^{s}_{i+1} \).
    Therefore, \( \ts^{s}_{0} < \ts^{e}_{0} < \ts^{s}_{1} < \dots <  \ts^{s}_{n} \), and we have contradiction that \( \ts^{s}_{0} < \ts^{s}_{n} \) with respect to Lemma \ref{lem:atoic-rw}.

    Similarly for \( \ptlar \), we have contradiction that \( \ts^{e}_{0} < \ts^{e}_{1} < \dots  < \ts^{e}_{n} \).
\end{proof}

We can extend partial arbitration relation \( \ptlar \) to a total order, called total arbitration relation or shortly arbitration \( \ar \) by taking the first two transactions that are not connected and then picking a order and take the transitive closure.
To simplify, assume there is a unique initial transactions denoted by \( \tsid_{init} \), where \( (\tsid_{init}, \tsid) \in \ptlar \) for all \( \tsid \).

\begin{defn}[Total graph]
\label{def:tototal}
\label{def:total-graph}
    First, the \( \funcn{toTotal} : \powerset{\Transactions} \times (\Transactions \times \Transactions) \parfun \powerset{\Transactions} \times (\Transactions \times \Transactions) \) function is defined as follows, which convert partial arbitration relation to total one.
    \[
        \begin{rclarray}
            \func{toTotal}{\settrans, \ptlar} & \defeq & 
                \begin{cases}
                    (\settrans, \ptlar) & \text{if} \ \func{firstBranch}{\settrans, \ptlar} = \emptyset\\ 
                    \func{toTotal}{\settrans, ( \ptlar \uplus \Set{(\tsid, \tsid')})^{+}} & \text{if} \ \tsid, \tsid' \in \func{firstBranch}{\settrans, \ptlar} \\    
                \end{cases}
            \\
        \end{rclarray}
    \]
    where, the \( \funcn{firstBranch} : \powerset{\Transactions} \times (\Transactions \times \Transactions) \parfun \powerset{\Transactions} \) function that returns a set that includes the first branched transactions (might be more than two).
    \[
        \begin{rclarray}
            \func{firstBranch}{\settrans, \ptlar} & \defeq &
            \begin{cases}
                \emptyset & \text{if} \ \settrans = \emptyset \\
                \func{firstBranch}{\settrans \setminus \Set{\tsid}} & \text{if} \ \func{dec}{\settrans, \ptlar} = \Set{\tsid} \\
                \func{dec}{\settrans, \ptlar} & \text{if} \ |\func{dec}{\settrans, \ptlar}| > 1 ) \\
                \text{undefined} & \text{otherwise}
            \end{cases} \\
        \end{rclarray}
    \]
    and the \( \funcn{dec} \) function returns all the direct descendants of the first transaction.
    \[
        \begin{rclarray}
            \func{dec}{\settrans, \ptlar} & \defeq &
            \myset{%
                \tsid
            }{%
                \exsts{ \tsid_{init}  } 
                \pred{firstTrans}{\settrans, \ptlar, \tsid_{init}} \land {} \\
                \quad (\tsid_{init}, \tsid) \in \ptlar 
                \land \neg\exsts{ \tsid' } 
                (\tsid_{init}, \tsid'),(\tsid', \tsid) \in \ptlar} \\
                \pred{firstTrans}{\settrans, \ptlar, \tsid} & \defeq & \for{\tsid' \in \settrans } \tsid = \tsid' \lor (\tsid, \tsid') \in \ptlar \\
        \end{rclarray}
    \]
    Now given the set of partial graphs \( \PGraphs \) (\defin \ref{def:partial-graph}), the set of \emph{total graph} \( \TGraphs \) is defined as follows:
    \[
        \begin{rclarray}
            \Setcon{%
                (\settrans, \po, \vis, \ar, \tshp)
            }{
                ( \settrans, \ar ) = \func{toTotal}{\settrans, \ptlar} 
                \land (\settrans, \po, \vis, \ptlar, \tshp) \in \PGraphs
            } \\
        \end{rclarray}
    \]
\end{defn}

\begin{lem}[Visibility]
    \label{lem:visibility}
    Visibility relation is a subset of arbitration relation, i.e.\ \( \vis \subseteq \ar \).
\end{lem}
\begin{proof}
    For all transactions \( \tsid \) and \( \tsid' \), if \( ( \tsid, \tsid' ) \in \vis \), it means that transaction \( \tsid \) commits before \( \tsid' \) starts, so that by Lemma \ref{lem:start-before-end} \( \tsid \) must start before \( \tsid' \) starts.
    This means \( (\tsid, \tsid') \in \ptlar \) (\defin \ref{def:vis-ptlar}), therefore \( (\tsid, \tsid') \in \ar \).
\end{proof}

\begin{lem}[Monotonic time in a thread]
    \label{lem:mono-time-thread}
    The thread's local time monotonically increase.
    This is  
    \[ 
        \for{ \stk, \stk', \tshp, \tshp', \ts, \ts', \iprog, {\iprog}' ,\lb,\addr, \ts } ( \stk, \tshp, \ts ), \iprog \toT{\lb} ( \stk', \tshp', \ts' ), {\iprog}' \implies \ts < \ts'
    \]
\end{lem}
\begin{proof}
    Prove by structural induction on the semantics.
    For base cases \rl{PChoiseLeft}, \rl{PChoiseRight}, \rl{PLoop}, \rl{PSeqSkip} and \rl{PPar}, it is trivial since those rules do not change the time.
    For \rl{Commit}, the premiss enforces that \( \ts' > \ts \) and for \rl{PWait}, \( \ts' = \max\Set{\ts, \stub} \geq \ts\).
    For the inductive case \rl{PSeq}, prove directly by the inductive hypothesis.
\end{proof}

\begin{lem}[Session]
    \label{lem:session}
    \( \po \subseteq \vis \).
\end{lem}   
\begin{proof}
    Prove by case analysis of \( \po \) (\defin \ref{def:po}).
    For \( \tsid \) and \( \tsid' \) where \( \lbC{\thid,\tsid} < \lbC{\thid,\tsid'} \), it means that transaction \( \tsid \) is reduced by the semantics before \( \tsid' \).
    Then by Lemma \ref{lem:mono-time-thread}, the commit time of the first transaction \( \tsid \) is smaller than the start time of the second one \( \tsid' \), so that \( (\tsid, \tsid') \in \vis \).

    For \( \tsid \) and \( \tsid' \) where \( \lbC{\thid,\tsid} < \lbF{\thid,\thid', \stub} < \lbC{\thid',\tsid'} \).
    By Lemma \ref{lem:mono-time-thread}, it means the parent thread \( \thid \) commits transaction \( \tsid \) before the fork of the child thread \( \thid' \), and also the child thread commits transaction \( \tsid' \) after the fork point.
    In the \rl{PFork} rule, the parent thread starts at time \( \ts \) and ends at \( \ts' \), and the child thread initialises with the time \( \ts' \), where in fact \( \ts = \ts' \).
    Therefore, \( (\tsid, \tsid') \in \vis \).

    Similarly, for \( \tsid \) and \( \tsid' \) where \( \lbC{\thid,\tsid} < \lbJ{\thid',\thid, \stub} < \lbC{\thid',\tsid'} \), all the transactions by child thread \( \thid \) must commits before the join point and therefore before all the uncommitted transactions by the parent thread \( \thid' \).
    Therefore, \( (\tsid, \tsid') \in \vis \).
\end{proof}

\sx{The session lemma I feel slightly unhappy, because it seems not very formal.}

\begin{lem}[Semi-prefix]
    \label{lem:semi-prefix}
    \( \ptlar; \vis \subseteq \vis \).
\end{lem}
\begin{proof}
    For all \( \tsid, \tsid', \tsid'' \), if \( (\tsid, \tsid') \in \ptlar \) and \( (\tsid', \tsid'') \in \vis \), by the definitions of \( \po \) and \( \vis \) (\defin \ref{def:vis-ptlar}), transaction \( \tsid' \) commits after \( \tsid \) commits but before \( \tsid'' \) starts.
    There must exist \( \addr, \addr'', \ts, \ts'', \etag \in \Set{\etW, \etE}  \) and \( \etag'' \in \Set{\etR, \etS } \) such that  \( \tshp(\addr)(\ts) = (\stub, \etag, \tsid) \), \( \tshp(\addr'')(\ts'') = (\stub, \etag'', \tsid'') \) and \( \ts < \ts'' \), thus \( ( \tsid, \tsid'' ) \in \vis \).
\end{proof}
\sx{Rework a bit until this point}

\begin{lem}[Prefix]     
    \label{lem:prefix}
    \( \ar; \vis \subseteq \vis \).
\end{lem}
\begin{proof}
    Prove by case analysis of the total arbitration relation \( \ar \).
    For all \( \tsid, \tsid', \tsid'' \) that \( (\tsid, \tsid') \in \ar \) and \( (\tsid', \tsid'') \in \vis \), if \( (\tsid, \tsid') \in \ptlar \), it is proven by Lemma \ref{lem:semi-prefix}.
    If \( ( \tsid, \tsid' ) \notin \ptlar \), it means it is a new edge by the \funcn{toTotal} function from Definition \ref{def:tototal}.
    By the Lemma \ref{lem:seperate}, \( \tsid \) and \( \tsid' \) commit at the same time.
    By the definition of \( \vis \), the commit time of \( \tsid' \) is smaller than the start time of \( \tsid'' \).
    Therefore, the commit time of \( \tsid \) is also smaller than the start time of \( \tsid'' \), so that \( ( \tsid, \tsid'' ) \in \vis \).
\end{proof}

\begin{lem}[No conflict]
    \label{lem:nocoflict}
    Two transactions cannot concurrently write to the same address, this means that one must observe another one.
    This is \( \for{ \addr, \tsid, \tsid' } \tshp(\addr)(\stub) = (\stub, \etW, \tsid) \land  \tshp(\addr)(\stub) = (\stub, \etW, \tsid' ) \implies ((\tsid, \tsid') \in \vis \lor (\tsid', \tsid) \in \vis ))\).
\end{lem}
\begin{proof}
    Prove by contradiction.
    Assume \( (\tsid, \tsid') \notin \vis \land (\tsid', \tsid) \notin \vis \), this intuitively means one transaction is overlapped with another.
    Let \( \ts_{s}, \ts_{e}, \ts'_{s} \) and \( \ts'_{e} \) be the start time and end time of transaction \( \tsid \) and \( \tsid' \) respectively.
    Because of the symmetry,  we can assume that the start time of \( \tsid \) is in between \( \tsid' \), witch means \( \ts'_{s} < \ts_{s} < \ts'_{e} \).
    Now we consider \( \ts_{e} \).
    First note that \( \ts_{e} > \ts_{s} \) by Lemma \ref{lem:start-before-end}, therefore we need to consider two cases \( \ts'_{s} < \ts_{s} < \ts_{e} < \ts'_{e} \) and  \( \ts'_{s} < \ts_{s} < \ts'_{e} < \ts_{e}  \).
    Since both transaction write the same address \( \addr \), those two cases violate the \( \predn{consist} \) requirement in the semantics, so one of the transactions must pick another start and end time.
\end{proof}

\begin{lem}[External]
    \label{lem:ext}
    A transaction should read the last values it can observe.
    This means that for all transaction \( \tsid \) and heap address \( \addr \), if the transaction read a value \( \val \) from the address, i.e.\ \( \exsts{ \ts } \tshp(\addr)(\ts) = (\val, \etR, \tsid) \), the last transaction \( \tsid' \) who writes to the same address and can be observe by \( \tsid \), i.e.\ \( (\tsid, \tsid') \in \vis\), should have written the same value, meaning \( \exsts{ \val', \ts' } \tshp(\addr)(\ts') = (\val', \etW, \tsid') \implies \val' = \val\)
\end{lem}
\begin{proof}
    Given the definition of \( \vis \), we have \( \ts' < \ts \).
    Because transaction \( \tsid' \) is the last one who write to the address, this means that \( \for{ \tsid'', \ts'', \etag'' } \tshp(\addr)(\ts'') = (\stub, \etag'', \tsid'') \implies \etag'' \neq \etW \).
    Thus by the \( \funcn{startstate} \) in the semantics, we have \( \val' = \val \).
\end{proof}

\begin{lem}[Acyclic]
    \label{lem:acyclic}
    Both \( \vis \) and \( \ar \) are acyclic.
\end{lem}
\begin{proof}
    For \( \vis \), it is proven by Lemma \ref{lem:semi-acyclic}.

    To prove \( \ar \) is acyclic, we prove inductively \( \ptlar_{i} \) is acyclic, where \( \ptlar_{0} = \ptlar \), \( \ptlar_{n} = \ar \) and \( \ptlar_{i+1} = (\ptlar_{i} \uplus \Set{(\tsid, \tsid')})^{+} \) for some \( \tsid \) and \( \tsid' \), which follows the process of \funcn{toTotal} from Definition \ref{def:tototal}.
    For base case \( \ptlar_{0} \), it is proven by Lemma \ref{lem:semi-acyclic}.
    Now assume \( \ptlar_{i} \) is acyclic and \( \ptlar_{i+1} = (\ptlar_{i} \uplus \Set{(\tsid, \tsid')})^{+} \) for some \( \tsid \) and \( \tsid' \), we prove by contradiction that \( \ptlar_{i+1} \) is acyclic.
    Given the hypothesis, if there is circle in \( \ptlar_{i+1} \), such circle either contain the edge \( (\tsid, \tsid') \), or can be broken down to a circle containing the edge \( (\tsid, \tsid') \), because of the transitive closure.
    Note that by breaking down to a circle containing \( ( \tsid, \tsid' ) \), the rest edges inside the circle must be in \( \ptlar_{i} \).
    Let assume the circle is \( \tsid_{0}, \tsid_{1}, \dots, \tsid_{i}, \tsid, \tsid', \tsid_{i+1}, \dots, \tsid_{n} \) where \( \tsid_{0} = \tsid_{n} \).
    By the \( \funcn{toTotal} \) function from Definition \ref{def:tototal}, \( \tsid, \tsid' \in \func{firstBranch}{\settrans, \ptlar_{i}}\), which means that for all \( \tsid'' \), \( (\tsid'', \tsid) \in \ptlar_{i} \) if and only if \( ( \tsid'', \tsid') \in \ptlar_{i} \), therefore \( (\tsid'', \tsid) \in \ptlar_{i+1} \) if and only if \( ( \tsid'', \tsid') \in \ptlar_{i+1} \).
    This means that the sequence without \( \tsid \), i.e.\ \( \tsid_{0}, \tsid_{1}, \dots, \tsid_{i}, \tsid', \tsid_{i+1}, \dots, \tsid_{n} \), is still a sequence where two adjacent transactions are connected by \( \ptlar_{i+1} \), and this new sequence is a circle.
    Now all the edges in the new sequence/circle already exist in \( \ptlar_{i} \), which violates our hypothesis.
    Therefore by contradiction, \( \ptlar_{i+1} \) is acyclic.
    Given that, \( \ar \) is acyclic.
\end{proof}

\begin{lem}[Total order]
    \label{lem:totalorder}
    \( \ar \) is a total order.
\end{lem}
\begin{proof}
    By Lemma \ref{lem:acyclic}, \( \ar \) is acyclic, and by \defin \ref{def:tototal}, for all \( \tsid \) and \( \tsid' \), either \( (\tsid, \tsid') \in \ar \) or \( (\tsid', \tsid) \in \ar \).
    Therefore, it is a total order.
\end{proof}

\begin{defn}[Snapshot Isolation Graph]
    \(\sig \in \SIGraphs \)
\end{defn}

\begin{thm}[Soundness of the semantics]
    The thread pool operational semantics $\toG{}$ (\defin\ref{def:thread_pool_semantics}) is sound.
    This means
    \[
        \begin{array}{@{}l@{}}
            \exsts{ f } \for{ \tshp, \thpl, \tshp', \thpl' } (\tshp, \thpl) \toG{\lb} (\tshp', \thpl') \land \exsts{ \sig \in \SIGraphs } f(\tshp, \thpl) = \sig \implies \exsts{ \sig' \in \SIGraphs } f(\tshp', \thpl') = \sig'
        \end{array}
    \]
\end{thm}
\begin{proof}
    Given the \defin \ref{def:traces}, there is a unique corresponding trace \trace, and then by \defin \ref{def:po}, \ref{def:vis-ptlar} and \ref{def:tototal} there exists a \( \sig' \).
    Then, by Lemma \ref{lem:visibility}, \ref{lem:session}, \ref{lem:prefix}, \ref{lem:nocoflict}, \ref{lem:ext}, \ref{lem:totalorder}, the \( \sig' \in \SIGraphs \).
\end{proof}
\sx{Not really right but keep this way for now}

\subsection{Completeness}
\sx{Define a function from a graph to a set of possible \( \fph \) }


%\newpage
%\section{Alternative Semantics\label{sec:alter}}
\subsection{Weaken Atomic}

We reuse some notations in Sect. \ref{sec:semantics}, but redefined the meaning.
The \( \Timestampheap \) now is a partial function from locations to a tuple of \( \Timestamp \) and \( \Val \).
The thread state now is only a local stack and a global stack, and correspondingly, the join label is only parametrised by thread identifier.

\[
    \begin{rclarray}
        \tshp \in \Timestampheap & \defeq & \Loc \parfinfun ( \Timestamp \times  \Val ) \\
        (\stk, \tshp) \in \Threadstate & \defeq & \Stack \times \Timestampheap \\
        \lb \in \Translabel & \defeq & 
              \lbID \quad               |
        \quad \lbS{\tsid} \quad         |
 t      \quad \lbC{\tsid} \quad        |
        \quad \lbF{\thid,\prog} \quad |
        \quad \lbJ{\thid} \\
    \end{rclarray}
\]

The notation \( \ptrans{\cmd}_{\tsid, \tshp} \) indicates that the transaction identifier by \( \tsid \) has been started with the snapshot of heap \( \tshp \), but does not commit so far.

\[
    \begin{syntax}{\iprog}
              \prog \quad                                       |
        \quad \ptrans{\cmd}_{\tsid, \tshp} \pseq \iprog \quad |
        \quad \iprog \pseq \pwait{\thid} \quad 
    \end{syntax}
\]

The main difference of this semantics is, except removing all the local time, that the one-step \rl{Commit} is split into few steps.
Here we assume there is a global functions \(\funcn{freshTransId} \) that return a new transaction identifier each time.

\[
    \infer[\rl{TakeSnapshot}]{%
        ( \stk, \tshp ) , \ptrans{\cmd} \ \toT{\lbS{\tsid}} \ ( \stk, \tshp ) , \ptrans{\cmd}_{\tsid, \tshp}
    }{%
        \begin{array}{c}
            \quad \tsid = \func{freshTransId}{}
        \end{array}
    }
\]

We also redefine some functions and predicates used before.

\[
    \begin{rclarray}
        \func{startstate}{\tshp} & \defeq & \lambda \loc \ldotp \val \ \texttt{where} \ \tshp(\loc) = (\dontcare, \val)\\
        \pred{allowcommit}{\tshp,\tshp',\ws,\rs} & \defeq & \forall \loc \in \ws \ldotp \tshp(\loc) = \tshp'(\loc)\\
        \func{commitTrans}{\tshp,\hp_{e}} & \defeq &
        \lambda \loc \ldotp
        \begin{funcarray}
            ( \tshp(\loc)\projection{1}+1, \hp_{e}(\loc)) & \loc \in \ws \\
            \tshp(\loc) & o.w. \\
        \end{funcarray} \\
    \end{rclarray}
\]

\[
    \infer[\rl{Commit}]{%
        ( \stk, \tshp ) , \ptrans{\cmd}_{\tsid, \tshp'} \ \toT{\lbC{\tsid}} \ ( \stk', \tshp'' ) , \pskip
    }{%
        \begin{array}{c}
            \quad \hp_{s} = \func{startstate}{\tshp'}
            \quad ( \stk, \hp_{s}, \emptyset, \emptyset ) , \cmd \toL^{*} ( \stk', \hp_{e}, \rs, \ws ) , \pskip \\
            \pred{allowcommit}{\tshp,\tshp',\ws,\rs} 
            \quad \tshp'' = \func{commitTrans}{\tshp,\hp_{e}}
        \end{array}
    }
\]

\end{document}
