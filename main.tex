%% For double-blind review submission, w/o CCS and ACM Reference (max submission space)
\documentclass[acmsmall]{acmart}\settopmatter{printfolios=true,printccs=false,printacmref=false}
%% For double-blind review submission, w/ CCS and ACM Reference
%\documentclass[acmsmall,review,anonymous]{acmart}\settopmatter{printfolios=true}
%% For single-blind review submission, w/o CCS and ACM Reference (max submission space)
%\documentclass[acmsmall,review]{acmart}\settopmatter{printfolios=true,printccs=false,printacmref=false}
%% For single-blind review submission, w/ CCS and ACM Reference
%\documentclass[acmsmall,review]{acmart}\settopmatter{printfolios=true}
%% For final camera-ready submission, w/ required CCS and ACM Reference
%\documentclass[acmsmall]{acmart}\settopmatter{}


%% Journal information
%% Supplied to authors by publisher for camera-ready submission;
%% use defaults for review submission.
\acmJournal{PACMPL}
\acmVolume{1}
\acmNumber{CONF} % CONF = POPL or ICFP or OOPSLA
\acmArticle{1}
\acmYear{2018}
\acmMonth{1}
\acmDOI{} % \acmDOI{10.1145/nnnnnnn.nnnnnnn}
\startPage{1}

%% Copyright information
%% Supplied to authors (based on authors' rights management selection;
%% see authors.acm.org) by publisher for camera-ready submission;
%% use 'none' for review submission.
\setcopyright{none}
%\setcopyright{acmcopyright}
%\setcopyright{acmlicensed}
%\setcopyright{rightsretained}
%\copyrightyear{2018}           %% If different from \acmYear

%% Bibliography style
\bibliographystyle{ACM-Reference-Format}
%% Citation style
%% Note: author/year citations are required for papers published as an
%% issue of PACMPL.
\citestyle{acmauthoryear}   %% For author/year citations

\usepackage{wrapfig}
\usepackage{enumerate}

\newif\ifNonACMMode
\NonACMModefalse

% for theorem, proof, etc.
\usepackage{amsthm}
\theoremstyle{definition}
\newtheorem{thm}{Theorem}[section]
\newtheorem{defn}[thm]{Definition}
\newtheorem{param}[thm]{Parameter}
\newtheorem{lem}[thm]{Lemma}
\newtheorem{prop}[thm]{Proposition}
\newtheorem*{cor}{Corollary}
\newtheoremstyle{case}{}{}{}{}{}{:}{ }{}
\theoremstyle{case}
\newtheorem{case}{Case}

% the inter command for operational semantices
\usepackage{proof}
\usepackage{color}

\usepackage{centernot}

\usepackage{amsmath,amssymb,stmaryrd}
\usepackage{dsfont}

% For the box assertion
\usepackage{varwidth}

\usepackage{hyperref}
\hypersetup{
    colorlinks,
    citecolor=black,
    filecolor=black,
    linkcolor=black,
    urlcolor=black
}
\usepackage[usenames,dvipsnames,svgnames,table]{xcolor}
\usepackage{enumitem}
\usepackage{translang}
\lstset{language=translang}
\usepackage[margin=2cm]{caption}
\usepackage{tikz}
\usetikzlibrary{positioning, shapes, decorations}
\usepackage{bold-extra}
\usepackage{titlesec}

\titleformat{\chapter}{\bfseries\Huge}{\thechapter.}{1ex}{\Huge}

\def\changemargin#1#2{\list{}{\rightmargin#2\leftmargin#1}\item[]}
\let\endchangemargin=\endlist
\usepackage{hhmacros}
\pgfdeclarelayer{main}
\pgfdeclarelayer{background}
\pgfdeclarelayer{foreground}
\pgfsetlayers{background,main,foreground}

\newcommand{\greyness}{gray!40}
\newcommand{\blueness}{cyan!60}

\tikzstyle{background}=[rectangle, draw=black, inner sep=0.2cm, rounded corners=1.2mm]
\tikzstyle{white}=[rectangle, fill=white, inner sep=0.5cm, rounded corners=5mm]

%\tikzstyle{background}=[circle, fill=\greyness,
%                                                inner sep=0.2cm,
%                                                rounded corners=5mm,
%                                                decorate,
%                                                decoration={random steps,
%                                                            segment length=3pt,
%                                                            amplitude=3pt}]

%

 \tikzstyle{hheapcell}=[rectangle, draw=black, inner sep=0.1cm, font=\small]

\tikzstyle{noise}=[circle, thick, minimum size=1.2cm, draw=yellow!85!black, fill=yellow!40, decorate, decoration={random steps, segment length=2pt, amplitude=2pt}]

%\pgfdeclarelayer{background}
%\pgfdeclarelayer{foreground}
%\pgfsetlayers{background,main,foreground}

\tikzstyle{abstract}=[draw, fill=white, text width=5em, text centered, minimum height=2.5em, rounded corners]
    
\tikzstyle{arr}=[draw, ->, thick, color=black]
\tikzstyle{dasharr}=[draw,->,thick,dashed,color=black]

\tikzset{
    version node/.style={
        rectangle,
        draw=black,
        align=center,
        minimum height=5mm,
        text depth=0.5ex,
        text height=2ex,
        inner xsep=0pt,
        outer sep=0pt, 
        font=\footnotesize
    },      
    version list/.style={
        matrix of nodes,
        row sep=-\pgflinewidth,
        column sep=-\pgflinewidth,
        nodes={
            version node
        }
        ,
        execute at empty cell={\node[draw=none]{};},
        text width=5mm,
        anchor=west
    }
}

\newcommand{\tikzvalue}[4]{
    \node[version node, fit=(#1) (#2), fill=white, inner sep=0pt] (#3) {#4}
}
\newcommand{\tikzkvspace}{1.5pt}
\newcommand{\tikzkeyspace}{-1.1}
\newenvironment{halfsubfig}{%
    \begin{subfigure}{0.45\textwidth}
}{%
    \end{subfigure}
}
\newenvironment{onethirdsubfig}{%
    \begin{subfigure}{0.3\textwidth}
}{%
    \end{subfigure}
}
\newenvironment{centertikz}{%
    \begin{center}%
    \begin{tikzpicture}[every node/.style={inner sep=0,outer sep=0},font=\footnotesize]%
}{%
    \end{tikzpicture}%
    \end{center}%
}
\newcommand{\tikzresize}{.8}

\PassOptionsToPackage{svgnames}{xcolor}
\definecolor{DarkGreen}{rgb}{0, 0.5, 0}

\usepackage{mathrsfs}  


%\newcommand*{\newextarrow}[3] {%
%\newcommand*{#1}[2][]{\ext@arrow #2{\arrowfill@#3}{##1}{##2}} }
%
\newextarrow{\xrightarrowtriangle}{{20}{20}{20}{20}}
   {{\relbar}{\relbar}{\rightarrowtriangle}}

%\newextarrow{\xrightharpoonup}{{20}{20}{20}{20}}
%   {{\relbar}{\relbar}{\rightharpoonup}}
   
\renewcommand{\parfinfun}{\ensuremath{\xrightharpoonup{\halfMath{fin}}}}

%%%%%%%%%%%%%%%%%%%%%% edit mode
\newif\ifCommentEdits
\CommentEditstrue

\newcommand{\pg}[1]{%
\ifComments
\begin{center}
\fbox{%
\begin{minipage}{6.5in} \color{red}
{\bf PG:} {\rm #1}
\end{minipage}
}
\end{center}
\fi
}

\newcommand{\sx}[1]{%
\ifComments
\begin{center}
\fbox{%
\begin{minipage}{6.5in} \color{blue}
{\bf SX:} {\rm #1}
\end{minipage}
}
\end{center}
\fi
}

\definecolor{darkred}{rgb}{0.5, 0, 0}
\newcommand{\azalea}[1]{%
\ifComments
\begin{center}
\fbox{%
\begin{minipage}{6.5in} \color{darkred}
{\bf AR:} {\rm #1}
\end{minipage}
}
\end{center}
\fi
}

\newcommand{\ac}[1]{%
\ifComments
\begin{center}
\fbox{%
\begin{minipage}{6.5in} \color{green}
{\bf SX:} {\rm #1}
\end{minipage}
}
\end{center}
\fi
}

%%%%%%%%%%%%%%%%%%%%%% end edit mode


%% Journal information
%% Supplied to authors by publisher for camera-ready submission;
%% use defaults for review submission.
\acmJournal{PACMPL}
\acmVolume{1}
\acmNumber{CONF} % CONF = POPL or ICFP or OOPSLA
\acmArticle{1}
\acmYear{2018}
\acmMonth{1}
\acmDOI{} % \acmDOI{10.1145/nnnnnnn.nnnnnnn}
\startPage{1}

%% Copyright information
%% Supplied to authors (based on authors' rights management selection;
%% see authors.acm.org) by publisher for camera-ready submission;
%% use 'none' for review submission.
\setcopyright{none}
%\setcopyright{acmcopyright}
%\setcopyright{acmlicensed}
%\setcopyright{rightsretained}
%\copyrightyear{2018}           %% If different from \acmYear

%% Bibliography style
\bibliographystyle{ACM-Reference-Format}
%% Citation style
%% Note: author/year citations are required for papers published as an
%% issue of PACMPL.
\citestyle{acmauthoryear}   %% For author/year citations

\begin{document}



%% Title information
\title{
	Towards a Formal Theory for Clients of Distributed Key-value Stores
    %Operational Semantics and Logic for Weak Consistency in Transactional Systems%
    %: a Multi-version Based Operational Approach
    } 
                                        %% [Short Title] is optional;
                                        %% when present, will be used in
                                        %% header instead of Full Title.
%\titlenote{with title note}             %% \titlenote is optional;
%                                        %% can be repeated if necessary;
%                                        %% contents suppressed with 'anonymous'
%\subtitle{Subtitle}                     %% \subtitle is optional
%\subtitlenote{with subtitle note}       %% \subtitlenote is optional;
                                        %% can be repeated if necessary;
                                        %% contents suppressed with 'anonymous'


%% Author information
%% Contents and number of authors suppressed with 'anonymous'.
%% Each author should be introduced by \author, followed by
%% \authornote (optional), \orcid (optional), \affiliation, and
%% \email.
%% An author may have multiple affiliations and/or emails; repeat the
%% appropriate command.
%% Many elements are not rendered, but should be provided for metadata
%% extraction tools.

%% Author with single affiliation.
\author{Shale Xiong}
%\authornote{with author1 note}          %% \authornote is optional;
                                        %% can be repeated if necessary
%\orcid{nnnn-nnnn-nnnn-nnnn}             %% \orcid is optional
\affiliation{
  \position{Ph.D. Student}
  \department{Department of Computing}              %% \department is recommended
  \institution{Imperial College London}            %% \institution is required
  \streetaddress{Huxley Building}
  \city{London}
  \state{}
  \postcode{SW7 2AZ}
  \country{United Kingdom}                    %% \country is recommended
}
\email{shale.xiong14@imperial.ac.uk}          %% \email is recommended

%% Author with single affiliation.
\author{Andrea Cerone}
%\authornote{with author1 note}          %% \authornote is optional;
                                        %% can be repeated if necessary
%\orcid{nnnn-nnnn-nnnn-nnnn}             %% \orcid is optional
\affiliation{
  \position{Research Associate}
  \department{Department of Computing}              %% \department is recommended
  \institution{Imperial College London}            %% \institution is required
  \streetaddress{Huxley Building}
  \city{London}
  \state{}
  \postcode{SW7 2AZ}
  \country{United Kingdom}                    %% \country is recommended
}
\email{a.cerone@imperial.ac.uk}          %% \email is recommended

\author{Azalea Raad}
%\authornote{with author1 note}          %% \authornote is optional;
                                        %% can be repeated if necessary
%\orcid{nnnn-nnnn-nnnn-nnnn}             %% \orcid is optional
\affiliation{
  \position{Post-doctoral Researcher}
  \department{}              %% \department is recommended
  \institution{Max Planck Institute}            %% \institution is required
  \streetaddress{no idea}
  \city{Kaiserslautern}
  \state{-}
  \postcode{-}
  \country{Germany}                    %% \country is recommended
}
\email{a.raad@mpi-sws.org}          %% \email is recommended

%% Author with single affiliation.
\author{Philippa Gardner}
%\authornote{with author1 note}          %% \authornote is optional;
                                        %% can be repeated if necessary
%\orcid{nnnn-nnnn-nnnn-nnnn}             %% \orcid is optional
\affiliation{
  \position{Professor}
  \department{Department of Computing}              %% \department is recommended
  \institution{Imperial College London}            %% \institution is required
  \streetaddress{Huxley Building}
  \city{London}
  \state{-}
  \postcode{SW7 2AZ}
  \country{United Kingdom}                    %% \country is recommended
}
\email{p.gardner@imperial.ac.uk}          %% \email is recommended

\begin{abstract}
Modern NoSQL databases (e.g. key-value stores) achieve scalability and improve 
latency by weakening the guarantees of distributed transaction 
processing. While the problem of giving a formal specification to 
the consistency models used by such databases has been widely 
studied, formalising the semantics of clients interacting with 
such systems has been largely neglected$^\ast$. 
This paper aims to be a 
first step towards filling this gap. We present a framework 
for capturing the semantics of programs interacting with a 
weakly consistent key-value store whose transactions enjoy atomic visibility. 
Our semantics enjoys 
atomic transaction processing, interleaving concurrency, 
and parameterisation with respect to the consistency model.  
%The latter is captured using the notion of execution tests, which 
%determine when a transaction is allowed  to execute. 
As a main contribution, we prove that our semantics is adequate: 
for each program and consistency model, the semantics captures 
precisely the behaviour that the program exhibits under said consistency model; 
and that specifications of consistency models in our framework 
coincide with previously proposed, axiomatic ones.
%via execution 
%tests are equivalent to the axiomatic specifications which have already 
%been proved to be correct.
As another contribution, we develop a variant of the \emph{Concurrent 
Abstract Predicates} separation logic
%development 
%of a separation logic for clients of key-value stores. We propose 
%a variant of the \emph{Concurrent Abstract Predicates} 
that is tailored to clients of weakly-consistent key-value stores, 
and that is parametric in the specification of consistency models.
%and a multi-version 
%representation of the key-value store to allow concurrent clients to 
%observe different states of the system.  
%We abstract from implementation details of the key-value store, which 
%is represented as a centralised, multi-version system. This allows 
%for concurrent clients to observe different version of the same key, 
%which makes it possible to capture non-serialisable behaviours of 
%programs while still retaining the atomic execution of transactions and 
%interleaving concurrency. 
%Furthermore, our semantics is parameterised by execution tests, which determine 
%when a transaction is allowed to execute.  By changing the execution 
%test of transactions, we capture different consistency models.

\textbf{$^\ast$ Suresh is going to be pissed off a lot by this sentence, but 
let's be honest, his semantics is light years behind ours.}
\ac{The abstract is wayyyy too long. But at least we have a guideline for the paper.}

%We present a a uniform semantics 
%Contents of this set of notes: 
%History heaps. Semantics of Programs 
%running under weak consistency models using history heaps as states. 
%Simulation technique for comparing weak consistency models defined using 
%history heaps. Verification of implementations.
%\textbf{Points following Dagstuhl: Viktor seemed positive about the 
%history heap work. His question was whether the framework is generic 
%enough to capture the protocols that they are developing with Azalea. 
%Alexey's opinion is that the framework may have some use if we 
%manage to prove implementations of protocols correct. 
%I would also like to have Azalea's opinion on a semantics based 
%on history heaps.}
\end{abstract}


%% 2012 ACM Computing Classification System (CSS) concepts
%% Generate at 'http://dl.acm.org/ccs/ccs.cfm'.
\begin{CCSXML}
<ccs2012>
<concept>
<concept_id>10011007.10011006.10011008</concept_id>
<concept_desc>Software and its engineering~General programming languages</concept_desc>
<concept_significance>500</concept_significance>
</concept>
<concept>
<concept_id>10003456.10003457.10003521.10003525</concept_id>
<concept_desc>Social and professional topics~History of programming languages</concept_desc>
<concept_significance>300</concept_significance>
</concept>
</ccs2012>
\end{CCSXML}

\ccsdesc[500]{Software and its engineering~General programming languages}
\ccsdesc[300]{Social and professional topics~History of programming languages}
%% End of generated code


%% Keywords
%% comma separated listp
\keywords{keyword1, keyword2, keyword3}  %% \keywords are mandatory in final camera-ready submission


%% \maketitle
%% Note: \maketitle command must come after title commands, author
%% commands, abstract environment, Computing Classification System
%% environment and commands, and keywords command.

\maketitle

\azalea{I have imported the cleveref package! This means that all reference will be printed consistently and we DO NOT need custom names such as \textbackslash fig etc.
Every time you need to refer to something, please write\textbackslash cref\{label\}, \eg \cref{def:mkvs}, and the label (\eg Def.) will be printed correctly. 
These labels can be customised. I have introduced the necessary ones in the macros file. 
}
\newcommand{\RootPath}{.}
\section{Introduction}

In recent times, the area of formal reasoning for concurrent and heap-manipulating programs, has seen a noticeable development towards program logics that can tackle the specification and verification of low-level concurrency in systems. This capability, together with the ubiquity of multithreading in computer programs, allows the formulation of reasoning frameworks around a variety of applications.

Modern database systems make heavy use of concurrency in order to provide a level of performance able to support large scale operations. This leads to an obvious increase in throughput but can cause a lack of consistency in data, which is instead a fundamental requirement for the majority of programs. A number of techniques has been employed in commercial databases to solve the issue and try to give the best of both worlds. Among these, \textit{Two-Phase-Locking} is a blocking approach which resides in the part of the spectrum of solutions where correctness is preferred over performance and that works at the granularity of single database entries. As a consequence, once implemented as part of a complex database system, the algorithm is prone to subtle bugs which might cause the violation of its vital guarantees.

We therefore intend to provide a complete and flexible model of the \textit{Two-Phase-Locking} concurrency control mechanism, as inspired by its real-world use case.
The aim is to derive a specification together with a sound level of abstraction that allows us not to think in terms of the low-level details enforced by the technique.
This leads us to the exploration of formal reasoning about its client usage through a custom program logic which is proven to be sound. The logic framework enables users to prove partial correctness of their programs running in a \textit{Two-Phase-Locking} setting, by only having to reason atomically about blocks of code, without the complexity of concurrent interleavings.

\subsection{Contributions}

The main contributions of this project are listed below, with references to the relevant sections where they are further discussed.
\begin{itemize}
	\item \textbf{mCAP} (Section \ref{sec:mcapModel}, \ref{sec:mcapLogic}, \ref{sec:transLogic})\ \ We reformulate and extend a program logic for concurrent programs, namely CAP \cite{cap}, in order to remove some constraints which are hardcoded in the logic and enable a more flexible reasoning. In fact, we change the underlying model to parametrize both the representation of machine states and of action capabilities. On top of this, we provide a new and cleaner structure for the action model that does not explicitly use interference assertions. We also considerably modify the way environment interference is modelled through the rely/guarantee relations. This is done with the goal of allowing both a thread and the environment to perform multiple shared region updates in one step. It follows that the repartitioning operator also has a new and extended behaviour. At the level of programming language, we leave elementary atomic commands as a parameter to the user of the logic. Finally, we instantiate the mCAP framework into a logic for our particular needs of transactional reasoning.
	
	\item \textbf{\textsc{2pl} Model}\ \ 
	
	\item \textbf{Operational Semantics}\ \
	
	\item \textbf{Semantics Equivalence}
\end{itemize}
\subsection{semantics}
Let \( \repl \in \Repls \) denotes the set of totally ordered replicates.
Each replicate can have multiple clients, and 
each clients can commit a sequence of either read-only transitions or single-write transactions.
To model these, we annotate the transaction identifier with replicate \( \repl \), client \( \cl \), 
local time of the replicate \( n \) and read-only transactions count \( n' \), \ie \( \txidCOPS{\repl}{\cl}{n}{n'} \).
Note that the \( (n, \repl, n') \) can be treated as a single number that \( n \) are the higher bits, 
\( \repl \) the middle bits and \( n' \) the lower bits.
There is a total order among transitions from the same replica and from the same client.
We extend version with the set of all versions it dependencies on, \( \dep \in \pset{\Keys \times \TxID} \).
The function \( \depOf{\ver} \) denotes the dependencies set of the version.
For readability, we annotate view with either a replica, \( \viREPL \), or a client, \( \viCL \).
The view environment is extended with replicas and their views, \( \viewFunCOPS : (\Repls \times \ClientID ) \parfinfun \Views \).
We give the following semantics to capture the behaviours of the code.

\begin{lstlisting}[caption={put},label={lst:simplified-put}]
// mixing the client API and system API
put(repl,k,v,ctx) {

    // Dependency for previous reads and writes
    deps = ctx_to_dep(ctx);(*\label{line:put-ctx-to-deps}*)

    atomic{
        // increase local time.
        inc(repl.local_time);(*\label{line:put-inc-local}*) 

        // appending local kv with a new version.
        list_isnert(repl.kv[k],(v, (local_time + id), deps));(*\label{line:put-update-kv}*)
    }

    // update dependency for writes
    ctx.writers += (k,(local_time + id),deps);(*\label{line:put-update-ctx}*)

    // put in the queue to sync with other replicas
    enqueue(k,v,(current_ver+id),(deps ++ vers));
}
\end{lstlisting}

The client always fetches the version with the maximum writer it can observed for each key,
Which is computed by \( \funcn{getMax} \) function. 
It is different from \( \snapshot \) as \( \snapshot \) fetches the latest version with respect to the position in the list.

\[
    \begin{rclarray}
        \func{getMax}{\mkvsCOPS, \viCL} & \defeq &
        \lambda \ke \ldotp \left( \max_\txid\Setcon{(\val, \txid, \T, \dep)}{\exsts{i} (\val, \txid, \T, \dep) = \mkvsCOPS(\ke, i)} \right)\projection{1}
    \end{rclarray}
\]
\begin{mathpar}
    \inferrule[Put]{%
        ( \stk, \func{getMax}{\mkvsCOPS, \viCL}, \emptyset ), \pmutate{\ke}{\vx} \toL
        ( \stk', \stub, \Set{(\otW, \ke, \val )} ), \pskip
        \\\\
        \dep = \Setcon{(\ke', \txid)}{\exsts{i} i \in \viCL(\ke') \land \txid = \WTx(\mkvsCOPS(\ke', i))} \texttt{ ---> \cref{lst:simplified-put}, \cref{line:put-ctx-to-deps}} 
        \\\\
        \txid = \min\Setcon{%
        \txidCOPS{\repl}{\cl}{n'}{0}
        }{%
            \fora{\ke', i \in \viREPL(\ke'), n} \\
            \quad \txidCOPS{\stub}{\stub}{n}{\stub} = \WTx(\mkvsCOPS(\ke',i)) \\
            \qquad {} \implies n' > n 
        } \texttt{ ---> \cref{lst:simplified-put}, \cref{line:put-inc-local}}
        \\\\
        \mkvsCOPS' = \mkvsCOPS\rmto{\ke}{\mkvsCOPS(\ke) \lcat \List{(\ke, \txid, \emptyset, \dep)}} \texttt{ ---> \cref{lst:simplified-put}, \cref{line:put-update-kv}}
        \\\\
        \viREPL' = \viREPL\rmto{\ke}{\viREPL(\ke) \uplus \Set{\abs{\mkvsCOPS'(\ke)} - 1}} \texttt{ ---> \cref{lst:simplified-put}, \cref{line:put-update-kv}}
        \\\\
        \viCL' = \viCL\rmto{\ke}{\viREPL(\ke) \uplus \Set{\abs{\mkvsCOPS'(\ke)} - 1}} \texttt{ ---> \cref{lst:simplified-put}, \cref{line:put-update-ctx}}
    }{%
    \repl, \cl \vdash 
    \mkvsCOPS, \viREPL, \viCL, \stk, \ptrans{\pmutate{\ke}{\vx};} \toT{}
    \mkvsCOPS', \viREPL', \viCL', \stk', \pskip
    }
\end{mathpar}
The \verb|get_trans| fetches the latest versions from the replica via multiple atomic reads, one for each key.
As a result, a client has a list of candidates \verb|rst|.
Since interleaving might happen, versions might become out-of-date because the replicate receives new versions.
It is not a problem to read old versions as long as they satisfy causal consistency,
\ie if a client read a version \( \ver \), it should at least read all the versions that \( \ver \) depends on.
Thus the algorithm use \verb|ccv| to track the maximum versions the client should fetches,
and re-fetches the \verb|ccv[k]| version from the replica if it is greater than the candidate.

The following is a simplified algorithm by directly taking a list of versions \verb|ccv| satisfies causal consistency constraint,
and then read the versions indicated by \verb|ccv|.
The simplified algorithm is easier to understand.
\begin{lstlisting}[caption={get\_trans},label={lst:get-trans}]
// A simplified version by guessing
// a ccv satisfying dependency constraints
// and then read versions indicated by ccv.
// Note that it is a weaker version of the original code,
// as the original implementation fetches the latest versions
// for keys by a sequence of atomic get_by_version calls
List(Val) get_trans(ks,ctx) {
    take ccv: (*$\forall$*) k (*$\in$*) ks.(*\label{line:get-trans-ccv-1}*)
        (_,_,deps) := get_by_version(k,ccv[k]) (*${}\land \forall$*) dep (*$\in$*) deps.(*\label{line:get-trans-ccv-2}*)
            dep.key (*$\in$*) ks (*$\implies$*) ccv[dep.key] >= dep.ver (*\label{line:get-trans-ccv-3}*)

    for k in ks(*\label{line:get-trans-read-1}*)
        rst[k] = get_by_version(k,ccv[k]);(*\label{line:get-trans-read-2}*)

    // update the ctx
    for (k,ver,deps) in rst(*\label{line:get-trans-update-ctx-1}*)
        ctx.readers += (k,ver,deps);(*\label{line:get-trans-update-ctx-2}*)

    return to_vals(ks);
}                                   
\end{lstlisting}
\begin{mathpar}
    \inferrule[GetTrans]{%
        \viCL \viewleq \viCL' \viewleq \viREPL  \texttt{ ---> \cref{lst:get-trans}, \cref{line:get-trans-update-ctx-1,line:get-trans-update-ctx-2}}
        \\\\
        {\left(\begin{array}{@{}l@{}}
        \fora{i : 1 \leq i \leq j, \ke', m, \ver}  \\
        \quad \ver = \mkvsCOPS(\ke_i, \max(\viCL'(\ke_i)) \land {} \\
        \quad (\ke', \WTx(\mkvsCOPS(\ke', m))) \in \ver\projection{4} \\
        \qquad {} \implies m \in \viCL'(\ke')
        \end{array}\right)} \texttt{ ---> \cref{lst:get-trans}, \cref{line:get-trans-ccv-1,line:get-trans-ccv-2,line:get-trans-ccv-3}}
        \\\\
        \trans =  \pderef{\vx_1}{\ke_1}; \dots; \pderef{\vx_j}{\ke_j};
        \\\\
        ( \stk, \func{getMax}{\mkvsCOPS, \viCL'}, \emptyset ), \trans \toL
        ( \stk', \stub, \f ), \pskip \texttt{ ---> \cref{lst:get-trans}, \cref{line:get-trans-read-1,line:get-trans-read-2}}
        \\\\
        \txidCOPS{\repl}{\cl}{n'}{n} = \max\Setcon{\txidCOPS{\repl}{\cl}{z'}{z}}{\txidCOPS{\repl}{\cl}{z'}{z} \in \mkvsCOPS }
        \\
        \mkvsCOPS' = \updKV{\mkvsCOPS, \viCL', \txidCOPS{\repl}{\cl}{n'}{n+1}, \f} 
    }{%
        \repl, \cl \vdash 
        \mkvsCOPS, \viREPL, \viCL, \stk, \ptrans{\pderef{\vx_1}{\ke_1}; \dots; \pderef{\vx_j}{\ke_j}; } \toT{}
        \mkvsCOPS', \viREPL, \viCL', \stk', \pskip
    }
    \and
    \inferrule[ClientCommit]{%
        \repl, \cl \vdash 
        \mkvsCOPS, \viewFunCOPS(\repl), \viewFunCOPS(\cl), \stk, \prog(\cl) \toT{}
        \mkvsCOPS', \viREPL', \viCL', \stk', \cmd'
    }{%
        \mkvsCOPS, \viewFunCOPS, \thdenv, \prog \toG{}
        \mkvsCOPS', \viewFunCOPS\rmto{\repl}{\viREPL'}\rmto{\cl}{\viCL'}, \thdenv\rmto{\cl}{\stk'}, \prog\rmto{\cl}{\cmd'}
    }
\end{mathpar}
A replica updates its local state only if all the dependencies has been receive.
\begin{lstlisting}[caption={Send and receive},label={lst:send-receive}]
// Syn to other replicas
send() {
    (k,v,ver,deps) := dequeue();
    for id in repls {
        send (k,v,ver,deps) to id;
    }
}

// receive a write message from other replica
on_receive(k,v,ver,deps) {
    // for a single machine
    // the following check immediately holds
    for (k',ver') in deps {
        wait until dep_check(k',ver');(*\label{line:receive-wait}*)
    }

    atomic{
        list_isnert(kv[k],(v,ver,deps));(*\label{line:receive-update-view-1}*)
        (remote_local_time + id) = ver;(*\label{line:receive-update-view-2}*)
        local_time = max(remote_local_time, local_time);(*\label{line:receive-update-view-3}*)
    }
}
\end{lstlisting}
\begin{mathpar}
    \inferrule[sync]{%
        \viREPL = \viewFunCOPS(\repl)\rmto{\ke}{\viewFunCOPS(\repl)(\ke) \uplus i} 
        \texttt{ ---> \cref{lst:send-receive}, \cref{line:receive-update-view-1,line:receive-update-view-2,line:receive-update-view-3}}
        \\\\
        {\left(\begin{array}{@{}l@{}}
        \fora{\ke', m, \ver} 
        \ver = \mkvsCOPS(\ke, i) \land {} \\
        \quad (\ke', \WTx(\mkvsCOPS(\ke', m))) \in \ver\projection{4} \\
        \qquad {} \implies m \in \viREPL'(\ke') 
        \end{array}\right)} \texttt{ ---> \cref{lst:send-receive}, \cref{line:receive-wait}} 
    }{%
        \mkvsCOPS, \viewFunCOPS, \thdenv, \prog \toG{}
        \mkvsCOPS, \viewFunCOPS\rmto{\repl}{\viREPL}, \thdenv, \prog
    }
\end{mathpar}

A view \( \vi \) on key-value store \( \mkvsCOPS \) \emph{agrees} 
with another view \( \vi \) on another key-value store \( \mkvsCOPS' \), if and only
\[
 \func{getMax}{\mkvsCOPS, \vi} = \func{getMax}{\mkvsCOPS', \vi'}
\]



%\begin{theorem}
    %For any trace \( \tr \) of COPS with final configuration \( (\mkvsCOPS, \viewFunCOPS) \), 
    %there exists a trace \( \tr' \) with final configuration \( (\mkvsCOPS', \viewFunCOPS') \) such that 
    %each step of the trace \( \tr' \) commits a transaction with strictly greater transaction identifier than any one appearing in the key-value store:
    %\[
        %\begin{array}{@{}l@{}}
        %(\mkvsCOPS_i, \viewFunCOPS_i) 
        %\toG{} (\mkvsCOPS_{i+1}, \viewFunCOPS_{i+1}) 
        %\land \exsts{\txid} \txid \in \mkvsCOPS_{i+1} 
        %\land \txid \notin \mkvsCOPS_{i+1}
        %\implies \fora{\txid' \in \mkvsCOPS_i} \txid > \txid'
        %\end{array}
    %\]
    %and any replica's view from \( \viewFunCOPS \) agrees with its counterpart from  \( \viewFunCOPS' \):
    %\[
        %\fora{i} 
        %\func{getMax}{\mkvsCOPS, \viewFunCOPS(i)} = \func{getMax}{\mkvsCOPS', \viewFunCOPS'(i)}
    %\]
%\end{theorem}
%\begin{proof}
%\end{proof}

%\begin{lemma}
%\[
    %\begin{array}{@{}l@{}}
    %\fora{\repl, \cl, \mkvsCOPS, \mkvsCOPS', \mkvsCOPS'', \viREPL, \viREPL', \viREPL'', \viCL, \viCL', \viCL'', \stk, \stk', \cmd, \cmd'} \\
    %\quad \func{getMax}{\mkvsCOPS, \viREPL} = \func{getMax}{\mkvsCOPS'', \viREPL''} 
    %\land \func{getMax}{\mkvsCOPS, \viCL} = \func{getMax}{\mkvsCOPS'', \viCL''} \\
    %\qquad \repl, \cl \vdash 
    %\mkvsCOPS, \viREPL, \viCL, \stk, \cmd \toT{}
    %\mkvsCOPS', \viREPL', \viCL', \stk', \cmd' \\
    %\qquad \implies 
    %\exsts{\mkvsCOPS''', \viREPL'''}
    %\mkvsCOPS'', \viREPL'', \viCL'', \stk, \cmd \toT{}
    %\mkvsCOPS''', \viREPL''', \viCL', \stk', \cmd' \\
    %\qquad \func{getMax}{\mkvsCOPS', \viREPL'} = \func{getMax}{\mkvsCOPS''', \viREPL'''} 
    %\land \func{getMax}{\mkvsCOPS', \viCL'} = \func{getMax}{\mkvsCOPS''', \viCL'''} \\
    %\end{array}
%\]
%\end{lemma}
%\begin{proof}
%We perform case analysis.
%\begin{itemize}
    %\item \rl{Put}.
    %We have \( \cmd \equiv \ptrans{\pmutate{\ke}{\vx};} \) for some key \( \ke \) and variable \( \vx \).
    %Suppose key-value stores \(  \mkvsCOPS, \mkvsCOPS', \mkvsCOPS'' \), 
    %replica's views \( \viREPL, \viREPL', \viREPL''\) and client's views \( \viCL, \viCL', \viCL''\) such that
    %\[
    %\begin{array}{@{}l@{}}
    %\func{getMax}{\mkvsCOPS, \viREPL} = \func{getMax}{\mkvsCOPS'', \viREPL''} 
    %\land \func{getMax}{\mkvsCOPS, \viCL} = \func{getMax}{\mkvsCOPS'', \viCL''} \\
    %\qquad \repl, \cl \vdash 
    %\mkvsCOPS, \viREPL, \viCL, \stk, \cmd \toT{}
    %\mkvsCOPS', \viREPL', \viCL', \stk', \cmd' \\
    %\end{array}
    %\]
    %By the premiss of the \rl{Put} rule, the new key-value store
    %\[
        %\mkvsCOPS' = \mkvsCOPS\rmto{\ke}{\mkvsCOPS(\ke) \lcat \List{(\ke, \txid, \emptyset, \dep)}}
    %\]
    %where
    %\[
        %\txid = \min\Setcon{%
            %\txidCOPS{\repl}{\cl}{n'}{0}
        %}{%
            %\fora{\ke', i \in \viREPL(\ke'), n} \\
            %\quad \txidCOPS{\stub}{\stub}{n}{\stub} = \WTx(\mkvsCOPS(\ke',i)) \\
            %\qquad {} \implies n' > n 
        %} 
    %\]
    %and the new views of replica and client are
    %\[   
        %\begin{array}{@{}l@{}}
        %\viREPL' = \viREPL\rmto{\ke}{\viREPL(\ke) \uplus \Set{\abs{\mkvsCOPS'(\ke)} - 1}} \\
        %{} \land \viCL' = \viCL\rmto{\ke}{\viREPL(\ke) \uplus \Set{\abs{\mkvsCOPS'(\ke)} - 1}}
        %\end{array}
    %\]
    %Similarly there exists a new \( \mkvsCOPS''' \) by committing a single-write transaction \( \txid' \) and two new views \( \viREPL''' \) and \( \viCL''' \).
    %This means for those key \( \ke' \) that is different from the key \( \ke \) being overwritten,
    %\begin{equation}
        %\label{equ:get-max-match-all-other-key}
        %\begin{array}{@{}l@{}}
            %\func{getMax}{\mkvsCOPS', \viREPL'}(\ke') = \func{getMax}{\mkvsCOPS''', \viREPL'''}(\ke') \\
            %{} \land \func{getMax}{\mkvsCOPS', \viCL'}(\ke') = \func{getMax}{\mkvsCOPS''', \viCL'''}(\ke') 
        %\end{array}
    %\end{equation}
    %Note that the \( \txid \) is greater than any writers \( \txidCOPS{\repl}{\cl}{n'}{0} \) that can be observed by the \( \viREPL \), so is \( \txid' \).
    %That is,
    %\begin{equation}
        %\label{equ:get-max-match-overwritten-key}
        %\begin{array}{@{}l@{}}
            %\func{getMax}{\mkvsCOPS', \viREPL'}(\ke) = \stk(\vx) = \func{getMax}{\mkvsCOPS''', \viREPL'''}(\ke)  \\
            %{} \land \func{getMax}{\mkvsCOPS', \viCL'}(\ke) = \stk(\vx) = \func{getMax}{\mkvsCOPS''', \viCL'''}(\ke) 
        %\end{array}
    %\end{equation}
    %Combine \cref{equ:get-max-match-all-other-key} and \cref{equ:get-max-match-overwritten-key},
    %we have the proof.

    %\item \rl{GetTrans}.
    %Since the views of replica remain unchanged, so we only need to prove that there exists a new key-value store and a new view \( \viCL''' \) such that
    %\[
        %\func{getMax}{\mkvsCOPS', \viCL'}(\ke) = \stk(\vx) = \func{getMax}{\mkvsCOPS''', \viCL'''}(\ke) 
    %\]
    
%\end{itemize}
%\end{proof}

\begin{lemma}
    \label{lem:client-subset-repl}
    The view of a client is subset of the view of the replica that the client interacts with.
\end{lemma}


\begin{lemma}
    Let ignore the dependencies of versions from \( \mkvsCOPS \).
    Given the initial key-value store \( \mkvsCOPS_0 \), initial views \( \viewFunCOPS_0 \) and some programs \( \prog_0 \), for any \( \mkvsCOPS_i \) and \( \viewFunCOPS_i \)  such that: 
    \[
        \mkvsCOPS_0, \viewFunCOPS_0, \thdenv_0, \prog_0 {\toG{}}^* \mkvsCOPS_i, \viewFunCOPS_i, \thdenv_i, \prog_i
    \]
    The key-value store \( \mkvsCOPS_i \) satisfies the \cref{def:mkvs} and any replica or client view \( \vi \) from \( \viewFunCOPS_i \) is a valid view of the key-value store, \ie \( \vi \in \Views(\mkvsCOPS_i) \).
\end{lemma}
\begin{proof}
    We need to prove the  \( \mkvsCOPS_i \) satisfies the well-formed conditions,
    and any view \( \vi_i \Views(\mkvsCOPS_i) \).
    We prove it by introduction on the length \( i \).
    \begin{itemize}
    \item \caseB{\(i = 0\)}
        It holds trivially since each key only has the initial version \( (\val_0,\txid_0,\emptyset, \emptyset) \).
        Since there is only the initial version for each key, it is easy to see that any view \( \vi_0 \) satisfying the well-formed conditions in \cref{def:views}.
    \item \caseI{\(i > 0\)}
        Suppose it holds when \( i \), let consider \( i + 1 \).
        We perform case analysis on the possible next step:
        \begin{itemize}
            \item \rl{Put}
                Assume the client \( \cl \) of a replica \( \repl \) commits a single-write transaction \( \txid \) that installs a new version for key \( \ke \).
                By the premiss of \rl{Put}, the new transaction identifier \( \txid = \txidCOPS{\repl}{\cl}{n'}{0} \) where for some \( n' \) that is greater than any \( n \) from any writers \( \txidCOPS{\stub}{\stub}{n}{\stub} \) that are observable by the replica \( \repl \).
                Since the new transaction \( \txid = \txidCOPS{\repl}{\cl}{n'}{0} \) is a single-write transaction which is always installed at the end of the list associated to \( \ke \), it is sufficient to prove the following:
                \begin{gather}
                    \fora{j} 0 \leq j < \abs{ \mkvsCOPS_i(\ke) } \implies \WTx(\mkvsCOPS_{i}(\ke, m)) \neq \txid \label{equ:write-trans-unique} \\
                    \fora{j, n} \txidCOPS{\repl}{\cl}{n}{\stub} \in \Set{\WTx(\mkvsCOPS_{i}(\ke,j))} \cup \RTx(\mkvsCOPS_{i}(\ke, j)) \implies n < n' \label{equ:replica-time-monotonic-inc}
                \end{gather}
                By \cref{lem:repl-observe-own}, we know that for any version written \( \ver = \mkvsCOPS_i(\ke, j) \) by the same replica \( \txidCOPS{\repl}{\stub}{\stub}{\stub} = \WTx(\ver) \), such version is included in the replica's view \( j \in \viREPL(\ke) \).
                It implies that first the new transaction identifier is unique \cref{equ:write-trans-unique} and second it is greater than any transactions in the form of \( \txidCOPS{\repl}{\cl}{\stub}{\stub} \) \cref{equ:replica-time-monotonic-inc}.
                Thus the new key-value store \( \mkvsCOPS_{i+1} \) satisfies the well-formed conditions.
                Now let consider the views, especially the views of the replica \( \viREPL' \) and the client \( \viCL' \).
                Since that views \( \vi' \) from different replicas or clients remain unchanged, by \ih they satisfy \( \vi' \in \Views(\mkvsCOPS_{i+1}) \).
                The new view for replica \( \viREPL' = \viREPL\rmto{\ke}{\abs{\mkvsCOPS_{i+1}(\ke)} - 1} \)
                where \( \viREPL \) is the replica's view before updating and the writer of the last version of \( \ke \) is \( \txid \).
                Because \( \txid \) is a single-write transaction, so the new view \( \viREPL' \) still satisfies the atomic read.
                For similar reason, the new view for client \( \viCL' \) till satisfies atomic read.
                Therefore we have \( \viREPL', \viCL' \in \Views(\mkvsCOPS_{i+1}) \).
            \item \rl{GetTrans}

        \end{itemize}
    \end{itemize}
\end{proof}

\begin{lemma}
    \label{lem:repl-observe-own}
    A replica observes all its own transactions.
\end{lemma}

\section{Logic}

\subsection{Rules for Local}

The proof rules are standard except \rl{TRDeref} and \rl{TRMutate}.
The \rl{TRDeref} rule add read fingerprint in finger-tracking set, only if there is no write finger-print.
This is because once a location has been re-written, the rest read are considered as local operations, while the finger-print only records those operations might have effect on global state.
%
\[
    \infer[\rl{TRDeref}]{%
        \judgeL{\expr \fpt{\fp} \lexpr}{ \pderef{\var}{\expr} }{\var \dot= \lexpr \sep \expr \fpt{\addFPR{\fp}} \lexpr }
    }{%
        \var \notin \func{fv}{\expr} &&
        \var \notin \func{fv}{\lexpr}  
    }
\]
 
\[
    \infer[\rl{TRMutate}]{%
        \judgeL{\expr_1 \fpt{\fp} \stub }{ \pmutate{\expr_1}{\expr_2} }{ \expr_1 \fpt{\addFPW{\fp}} \expr_2} 
    }{}
\]

\subsection{Merge}

\begin{defn}[Fingerprint heaps merge]
\label{def:merge-finger-heap}
The \emph{merge of fingerprint heaps}, \( \mergeFP{.}{.} \), is defined as follows:
\[
    \begin{rclarray}
        \mergeFP{\fph_{l}}{\fph_{r}}  & \defeq & \lambda \addr \ldotp 
            \begin{cases}
                \fph_{l}(\addr) & a \in \dom(\fph_{l}) \setminus \dom(\fph_{r})  \\
                \fph_{r}(\addr) & a \in \dom(\fph_{r}) \setminus \dom(\fph_{l}) \\
                \mergeVAL{\fph_{l}(\addr)}{\fph_{r}(\addr)}  & a \in \dom(\fph_{l}) \cap \dom(\fph_{r}) \\
            \end{cases}
    \end{rclarray}
\]
where the \emph{merge of fingerprint heap values} is defined:
\[ \begin{rclarray}
        \mergeVAL{(\val_{l}, \fp_{l})}{(\val_{r}, \fp_{r})} & \defeq &
            \begin{cases}
                (\val_{l}, \fp_{l} \cup \fp_{r} ) & \val_{l} = \val_{r} \land \fpW \notin \fp_{l} \cup \fp_{r} \\
                (\val_{l}, \fp_{l} \cup \fp_{r} ) & \fpW \in \fp_{l} \land \fpW \notin \fp_{r} \\
                (\val_{r}, \fp_{l} \cup \fp_{r} ) & \fpW \notin \fp_{l} \land \fpW \in \fp_{r} \\
            \end{cases}
    \end{rclarray}
\]
\end{defn}

%\sx{
    %Andrea gives a better idea to do this by defining merging fingerprint heaps first.
    %Big thanks. :)))
%}
%\azalea{This is a bit strong! For instance, according to this definition the heaps $\pv x \pt_{\emptyset} 1$ and $\pv x \pt_{\{\fpR\}} 1$ do not agree! Is that what you want?

%Perhaps you can define this as:
%\[
%\begin{rclarray}
	%\agree{\fph_l}{\fph_r} & \defeq  & \ws{\fph_l} \cap \ws{\fph_r} = \emptyset \\
        %&& \land\ \exsts{\h_1, \h_2, \h} \heapOnly{\readOnly{\fph_l}} = \h_1 \composeH \h \land \heapOnly{\readOnly{\fph_r}} = \h \composeFP \h_2 
        %\land \h_1 \composeH \h \composeH \h_2 \isdef
%\end{rclarray}        
%\]
%where
%\[
%\begin{rclarray}
	%\func{ws}{\fph} & \defeq & \myset{\loc}{ \exsts{\fp} \fph(\loc) = (\stub, \fp) \land \fpW \in \fp} \\\\
%%
	%\readOnly{.} & : & \FPHeaps \rightarrow \FPHeaps\\	
	%\readOnly{\fph}(\loc) & \defeq & 
	%\begin{cases}
		%\fph(\loc) & \text{if } \loc \not\in \ws{\fph} \\
		%\text{undefined} & \text{otherwise}
	%\end{cases}\\\\
%%	
	%\heapOnly{.} & : & \FPHeaps \rightarrow \Heaps\\	
	%\heapOnly{\fph}(\loc) & \defeq & 
	%\begin{cases}
		%\val  & \text{if } \exsts{\val} \fph(\loc) = (\val, -) \\
		%\text{undefined} & \text{otherwise}
	%\end{cases}
%\end{rclarray}
%\]
%}

\begin{defn}[Local state merge]
The \emph{merge of local states} is defined as follows, which merges two local states \( \ls_{l} \) and \( \ls_{r} \) with respect to a common initial local state \( \ls \).
\[
    \begin{rclarray}
	\mergeLS{\ls}{\ls_{l}}{\ls_{r}} & \eqdef &
	\myset{
		\left(\fph, \ca_{l}' \composeC \ca_{f} \composeC \ca_{r}' \right) 
	}{
        \fph = \mergeFP{\lsFPH{(\ls_{l})} }{ \lsFPH{(\ls_{r})} } \land \exsts{\ca_{l}, \ca_{r}}\\
		\quad \land\; \lsCAP{\ls} = \ca_{l} \composeC \ca_{f} \composeC \ca_{r}
        \land \lsCAP{(\ls_{l})} = \ca_{l}' \composeC \ca_{f} \composeC \ca_{r}
        \land \lsCAP{(\ls_{r})} = \ca_{l} \composeC \ca_{f} \composeC \ca_{r}'
	}
    \end{rclarray}
\]
where, to recall, the notation \( \lsFPH{(.)} \), and \( \lsCAP{(.)} \) refer to the fingerprint heap and capabilities respectively in a local state.
Then, the \emph{conflict} between two local states is defined as follows:
\[
    \begin{rclarray}
        \conflict{\ls}{\ls_{l}}{\ls_{r}} & \defeq & \mergeLS{\ls}{\ls_{l}}{\ls_{r}} = \emptyset
    \end{rclarray}
\]
\end{defn}

\begin{definition}[Fingerprint worlds]\label{def:fingerprint_worlds}
Given the set of local states $\LStates$ (\defin\ref{def:local_state}) and the set of region identifiers $\RegionID$ (\defin\ref{def:capabilities}), the set of \emph{fingerprint worlds}, $\fpw \in \FPWorlds$, is defined as follows:
%
\[
\begin{rclarray}
	\FPWorlds  & \eqdef  
	& \myset{
		(\ls, \fpgs)
	}{
		(\ls, \fpgs) \in \LStates \times (\RegionID \parfinfun \LStates) \land \wfFW{\ls, \fpgs}
	}
\end{rclarray}
\]
%
with the definitions of the flattening function and the well-formedness predicate lifted as follows:
%
\[
\begin{rclarray}
	\flattenFW{(\ls, \fpgs)}  & \eqdef & \ls \composeLS \prod\limits_{\rid \in \dom(\fpgs)}^{\composeLS} \fpgs(\rid)
\end{rclarray}
\]
%
\[
\begin{rclarray}
	\wfFW{\ls, \fpgs} & \defeq & \exsts{\fph, \ca}\flattenFW{(\ls, \fpgs)} {=} (\fph, \ca) \land\ \dom(\ca) \subseteq \dom(\fpgs) \\
\end{rclarray}
\]
%
The \emph{lift function}, $\liftW{.}: \World \rightarrow \FPWorlds$, is defined as follows:
%
\[
	\liftW{(\lgs, \gs)} \eqdef (\liftLGS{\lgs}, \liftGS{\gs})
\]
%
where
%
\[
\begin{rclarray}
	\liftLGS{(\h, \ca)} = (\fph, \ca)
	& \iffdef 
	& \for{\loc, \val} \h(\loc) = \val \iff \fph(\loc) = (\val, \emptyset) \\
%
	\liftGS{\gs} = \fpgs 
	& \iffdef
	& \for{\rid, \lgs, \inter} \gs(\rid) = (\lgs, \inter) \iff \fpgs(\rid) = (\liftLGS{\lgs'}, \inter) 
\end{rclarray}
\]
%
The \emph{erase function}, $\eraseFW{.}: \FPWorlds \rightarrow \World$, is defined as follows:
%
\[
	\eraseFW{(\ls, \fpgs)} \eqdef (\eraseLS{\ls}, \eraseFGS{\fpgs})
\]
%
where
\[
\begin{rclarray}
	\eraseLS{(\fph, \ca)} = (\h, \ca)
	& \iffdef 
	& \for{\loc, \val} \h(\loc) = \val \iff \fph(\loc) = (\val, \emptyset) \\
%
	\eraseFGS{\fpgs} = \gs 
	& \iffdef
	& \for{\rid, \ls, \inter} \fpgs(\rid) = (\ls, \inter) \Rightarrow \gs(\rid) = (\eraseLS{\ls}, \inter) 
\end{rclarray}
\]
\end{definition}

\begin{definition}[Fingerprint worlds merge]
Given the set of fingerprint worlds $\FPWorlds$ (\defin\ref{def:fingerprint_worlds}), the \emph{merge} function, $\mergeName[\fpw]: \FPWorlds \times \FPWorlds \times \FPWorlds \rightarrow \powerset{\FPWorlds}$, is defined as follows, for all $\fpw,\fpw_{l},\fpw_{r} \in \FPWorlds$:

%
\[
    \begin{rclarray}
	\mergeFW{(\stub, \fpgs)}{(\ls_{l}, \fpgs_{l})}{(\stub, \fpgs_{r})} & \eqdef &
		\Setcon{%
            (\ls_{l}, \fpgs_{p}) 
        }{%
            \fpgs_{p} \in \mergeFGS{\fpgs}{\fpgs_{l}}{\fpgs_{r}}
        } 
    \end{rclarray}
\]
%
where the \( \mergeName[\fpgs] \) is defined as follows:
\[
    \begin{rclarray}
        \mergeFGS{\fpgs}{\fpgs_{l}}{\fpgs_{r}} & \eqdef &
        \begin{cases}
            \emptyset & \text{if} \ \exsts{\rid} \conflict{\fpgs(\rid)}{\fpgs_{l}(\rid)}{\fpgs_{r}(\rid)} \lor \rid \in  \dom(\fpgs_{l}) \cap \dom(\fpgs_{r}) \setminus \dom(\fpgs)  \\
            S & \text{otherwise}
        \end{cases}
    \end{rclarray}
\]
with,
\[
    \begin{rclarray}
	S & = & \myset{\fpgs_{p}}{
		\dom(\fpgs_{p})= \dom(\fpgs_{l}) \cup \dom(\fpgs_{r}) \land \for{\rid}\\
		\quad \rid \in \dom(\fpgs_{l}) \cap \dom(\fpgs_{r}) \implies \fpgs_{p}(\rid) \in \mergeLS{\fpgs(\rid)}{\fpgs_{l}(\rid)}{\fpgs_{r}(\rid)} \\
		\quad \land\ \rid \in \dom(\fpgs_{l}) \setminus \dom(\fpgs_{r}) \implies 	\fpgs_{p}(\rid) = \fpgs_{l}(\rid) \\
		\quad \land\ \rid \in \dom(\fpgs_{r}) \setminus \dom(\fpgs_{l}) \implies 	\fpgs_{p}(\rid) = \fpgs_{r}(\rid)
	}
    \end{rclarray}
\]
\end{definition}

Note that the \( \mergeName[\fpw] \) is not commutative, i.e.\ swapping \( \fpw_{l}\) and \( \fpw_{r}\) might yield different result.

\subsection{Rely and Guarantee}

\begin{definition}[Rely and guarantee]
Given the set of fingerprint worlds $\FPWorlds$ (\defin\ref{def:fingerprint_worlds}), the \emph{update rely} relation, $\relyU: \FPWorlds \times \FPWorlds$, is defined as follows:
%
\[	
    \begin{rclarray}
	\relyU & \eqdef &
	\myset{
		((\ls, \fpgs_{p}), (\ls, \fpgs_{q}))	
	}{
		\exsts{\rid, \ca, \intf, \kap, \ls_{p}, \ls_{q}, \ls_{f}}\\
		\quad \for{\rid'} \rid \ne \rid' \implies \fpgs_{p}(\rid') = \fpgs_{q}(\rid') \\
		\quad \land\ \fpgs_{p}(\rid) = (\ls_{p} \composeLS \ls_{f}, \inter) \land \fpgs_{q}(\rid) = (\ls_{q} \composeLS \ls_{f}, \inter)		 \\
		\quad \land\ ( (\unitFP, \ca) \composeLS \flattenFW{(\ls, \fpgs_{p})} ) \isdef 
		\land \kap \leq \ca(\rid)
		\land (\ls_{p}, \ls_{q}) \in \inter(\kap)
	}
    \end{rclarray}
\]
The \emph{extension rely} relation, $\relyE: \FPWorlds \times \FPWorlds$, is defined as follows:
%
\[	
    \begin{rclarray}
        \relyE \eqdef
        \myset{
            \big((\ls, \fpgs_{p}), (\ls, \fpgs_{q})\big)	
        }{
            \exsts{\rid}
            \dom(\fpgs_{q}) \setminus \dom(\fpgs_{p}) = \{\rid\} \\
            \qquad \land\ \for{\rid'} \rid \ne \rid' \implies \fpgs_{p}(\rid') = \fpgs_{q}(\rid') \\
        }
    \end{rclarray}
\]
The \emph{rely} relation, $\myrely: \FPWorlds \times \FPWorlds$, is defined as follows:
%
\[
    \begin{rclarray}
         \myrely  &\eqdef & \bigcup\limits_{n \in \Nat} R_n \\
    \end{rclarray}
\]
where,
\[
    \begin{rclarray}
        \rely_0 & \eqdef & \closure{(\relyU \cup \relyE)} \\
        R_{n {+} 1} & \eqdef & (R_n \cup \pred{merge\_close}{R_n})^{*} \\
        \pred{merge\_close}{\rely} & \eqdef 
        & \myset{(\fpw, \fpw_{q})}{
            \exsts{\fpw_{l}, \fpw_{r}} (\fpw, \fpw_{l}), (\fpw, \fpw_{r}) \in \rely \land \fpw_{q} \in \mergeFW{\fpw}{\fpw_{l}}{\fpw_{r}}}
    \end{rclarray}
\]
%
The $\closure{(.)}$ denotes the reflexive transitive closure of the relation.
A set of fingerprint worlds $\setworld \subseteq \World$ is \emph{stable}, written $\stable{\setworld}$, if and only if it is closed under the rely relation: 
%
\[
    \begin{rclarray}
        \stable{W} & \eqdef & \for{\world \in \setworld, \fpw'} (\liftW{\world}, \fpw') \in \myrely \implies \eraseFW{\fpw'} \in \setworld
    \end{rclarray}
\]
%
The \emph{update guarantee} relation, $\guarU: \FPWorlds \times \FPWorlds$, is defined as follows:
%
\[	
    \begin{rclarray}
        \guarU & \eqdef &
        \myset{
            ((\ls_{p}, \fpgs_{p}), (\ls_{q}, \fpgs_{q}))	
        }{
            \exsts{\ls_{p}' = \flattenFW{(\ls_{p}, \fpgs_{p})}, \ls_{p}' = \flattenFW{(\ls_{q}, \fpgs_{q})} } \orth{(\lsCAP{(\ls_{p}')})} = \orth{(\lsCAP{(\ls_{q}')})}  \\
            \land 
            \begin{formulea}
                \for{\rid} \fpgs_{p}(\rid) = \fpgs_{q}(\rid) \\
                \lor 
                \begin{formulea}
                    \orth{(\lsFPH{(\ls_{p}')})} = \orth{(\lsFPH{(\ls_{q}')})} 
                    \land \exsts{\rid, \ca, \kap, \intf, \ls_{p}'', \ls_{q}'', \ls_f}\\
                        \quad \for{\rid'} \rid \ne \rid' \implies \fpgs_{p}(\rid') = \fpgs_{q}(\rid') \\
                        \quad \land \fpgs_{p}(\rid) = (\ls_{p}'' \composeLS \ls_f, \inter) \land \fpgs_{q}(\rid) = (\ls_{q}'' \composeLS \ls_f, \inter)		 \\
                        \quad \land (\unitFP, \ca) \leq \ls_{p}
                        \land \kap \leq \ca(\rid)
                        \land (\ls_{p}'', \ls_{q}'') \in \inter(\kap)
                \end{formulea}
            \end{formulea}
        }
    \end{rclarray}
\]
% 
The \emph{extension guarantee} relation, $\guarE: \FPWorlds \times \FPWorlds$, is defined as follows:
%
\[	
    \begin{rclarray}
	\guarE & \eqdef &
	\myset{
		((\ls_{f} \composeLS \ls, \fpgs_{p}), (\ls_{f} \composeLS (\unitFP, \ca), \fpgs_{q}))	
	}{
		\exsts{\rid, \ca'}
		\dom(\fpgs_{q}) = \dom(\fpgs_{p}) \uplus \Set{\rid} \\
		\qquad \land\ \for{\rid'} \rid \ne \rid' \implies \fpgs_{p}(\rid') = \fpgs_{q}(\rid') \\
		\qquad \land\ \fpgs_{q}(\rid) = \ls \composeLS (\unitFP, \ca')
		\land \dom(\ca) = \dom(\ca') = \Set{\rid}
	}
    \end{rclarray}
\]
% 
The \emph{guarantee} relation, $\myguar: \FPWorlds \times \FPWorlds$, is defined as follows:
%
\[
	\myguar \eqdef (\guarU \cup \guarE)^{\scalebox{1.1}{*}}
\]
%
\end{definition}

\subsection{Rules for Global}

The \rl{PRCommit} rule lifts the local effect of transaction \( \trans \) to global level by repartition \( \repartition{\gpre}{\gpost}{\lpre}{\lpost} \).
The repartition stripes off the fingerprints but uses the fingerprints to merge the local effect and the interference.
This is, the environment is allowed to write to locations that are different from the ones by transaction \( \trans \).
%
\[
    \infer[\rl{PRCommit}]{%
        \judgeG{\gpre}{ \ptrans{\trans} }{\gpost}
    }{%
        \judgeL{\lpre}{\trans}{\lpost} &&
        \repartition{\gpre}{\gpost}{\lpre}{\lpost}
    }
\]

\begin{definition}[Repartitioning]
The \emph{repartitioning} is defined as follows:
\[
    \begin{rclarray}
        \mrepartition{\setworld_{p}}{\setworld_{q}}{\setfph_{p}}{\setfph_{q}} & \iffdef &
        \begin{array}[t]{@{} l @{}}
            \for{\world_{p} \in \setworld_{p}} \exsts{\fpw_{p}, \fph_{p}, \fph_{f}}\\
            \quad \fpw_{p} = \liftW{\world_{p}} \land \flattenFW{\fpw_{p}} = (\fph_{p} \composeFP \fph_{f}, \unitC) \land \fph_{p} \in \setfph_{p} \\
            \quad \land\ \for{\fph_{q} \in \setfph_{q}} \exsts{\world_{q}, \fpw_{q}} \\
            \qquad \flattenFW{\fpw_{q}} = (\fph_{q} \composeFP \fph_{f}, \unitC) \land (\fpw_{p}, \fpw_{q}) \in \myguar \land \eraseFW{\fpw_{q}} = \world_{q} \land \world_{q} \in \setworld_{q} \\
            \qquad \land\ \for{\fpw \in \mergeR{\fpw_{p}}{\fpw_{q}}{\myrely}} \eraseFW{\fpw} \in \setworld_{q}
        \end{array}
    \end{rclarray}
\]
with, $\mergeName[\scalebox{.5}{\(\myrely\)}]: \FPWorlds \times \FPWorlds \times \powerset{\FPWorlds \times \FPWorlds} \to \powerset{\FPWorlds}$, defined as follows, for all $\fpw_{p}, \fpw_{q} \in \FPWorlds$:
%
\[
	\mergeR{\fpw_{p}}{\fpw_{q}}{\myrely} \eqdef \bigcup\limits_{\fpw \in \myrely(\fpw_{p})} \mergeFW{\fpw_{p}}{\fpw_{q}}{\fpw}
\]
%
\end{definition}

%\section{Examples\label{sec:example}}

\sx{New observation:
\begin{itemize}
\item We might want to have some assertion to say the initial values of a region.
\item What is the meaning of sequential composition since some consistency model does not have strong session guarantee, maybe the answer is the ``commit'' order.
\item Explain stablisation in a roughly syntactic level.
\end{itemize}
}

\subsection{Single increment and multi-reader.}
\[
    \begin{array}{@{}l@{}}
        \boxass{\V{x} \pt \V{\nat}}{\lrid}{\intass} \\
        \C{Inc} \composeK \C{Inc} \text{ is undefined} \\
        \C{Rd} \text{ is the unit element} \\
    \end{array}
\]
\subsubsection{SER}
\[
    \begin{array}{@{}l@{}}
        \intass : 
        \begin{rclarray}[t]
        \C{Inc} & : & \exsts{\V{m, k}} \Set{(\etR, \V{x}, \V{m}), (\etW, \V{x}, \V{m} + 1)} \mat \V{x} \pt \V{k} \oassto \V{x} \pt \V{k} \\
        \C{Rd}  & : & \exsts{\V{m, k, v}} \Set{(\etR, \V{x}, \V{m})} \mat \V{x} \pt \V{k} \oassto \V{x} \pt \V{k} \\ 
        \end{rclarray} \\
    \end{array}
\]

\[
\begin{session}
\specline{\boxass{\vx \pt 0}{\lrid}{\intass} \sep \cass{\C{Inc}}{\lrid} } \\
\specline{\boxass{\vx \pt 0}{\lrid}{\intass} \sep \cass{\C{Inc}}{\lrid} \sep \cass{\C{Rd}}{\lrid} \sep \cass{\C{Rd}}{\lrid} } \\
\begin{parl}
    \begin{session}
    \specline{\boxass{\vx \pt 0}{\lrid}{\intass} \sep \cass{\C{Inc}}{\lrid} \sep \cass{\C{Rd}}{\lrid} } \\
    \begin{transaction}
        \specline{ \vx \pt 0 } \\
        \pderef{\pvar{a}}{\vx} ; \\
        \specline{ \vx \pt 0 \land \pvar{a} = 0 \sep \Set{(\etR, \vx, 0)} } \\
        \pmutate{\vx}{\pvar{a} + 1} ; \\
        \specline{ \vx \pt 1 \land \pvar{a} = 0 \\
                {} \sep \Set{(\etR, \vx, 0), (\etW, \vx, 1)} } \\
    \end{transaction} \\
    \specline{\boxass{\vx \pt 1}{\lrid}{\intass} \sep \cass{\C{Inc}}{\lrid} \sep \cass{\C{Rd}}{\lrid} } \\
    \begin{transaction}
        \specline{ \vx \pt 1 } \\
        \pderef{\pvar{b}}{\vx} ; \\
        \specline{ \lor \vx \pt 1 \sep \Set{(\etR, \vx, 1)} } \\
    \end{transaction} \\
    \specline{\boxass{\vx \pt 1}{\lrid}{\intass} \sep \cass{\C{Inc}}{\lrid} \sep \cass{\C{Rd}}{\lrid} } \\
    \begin{transaction}
        \specline{ \vx \pt 1 } \\
        \pderef{\pvar{a}}{\vx} ; \\
        \specline{ \vx \pt 1 \land \pvar{a} = 1 \sep \Set{(\etR, \vx, 1)} } \\
        \pmutate{\vx}{\pvar{a} + 1} ; \\
        \specline{ \vx \pt 2 \land \pvar{a} = 1 \\
                {} \sep \Set{(\etR, \vx, 1), (\etW, \vx, 2)} } \\
    \end{transaction} \\
    \specline{\boxass{\vx \pt 2}{\lrid}{\intass} \sep \cass{\C{Inc}}{\lrid} \sep \cass{\C{Rd}}{\lrid} } \\
    \end{session}
    &
    \begin{session}
    \specline{\exsts{\V{n}}\boxass{\vx \pt \V{n}}{\lrid}{\intass} \land \V{n} \geq 0 \sep \cass{\C{Rd}}{\lrid} } \\
    \begin{transaction}
        \specline{ \exsts{ \V{v} } \vx \pt \V{v} { \color{gray} \land \V{v} = \V{n} } } \\
        \pderef{\pvar{c}}{\vx} ; \\
        \specline{ \exsts{ \V{v} } \vx \pt \V{v} \sep \Set{(\etR, \vx, \V{v})} { \color{gray} \land \V{v} = \V{n} } } \\
    \end{transaction} \\
    \specline{\exsts{\V{n}}\boxass{\vx \pt \V{n}}{\lrid}{\intass} \land \V{n} \geq 0 \sep \cass{\C{Rd}}{\lrid} } \\
    \end{session}
\end{parl} \\
\specline{\boxass{\vx \pt 2}{\lrid}{\intass} \sep \cass{\C{Inc}}{\lrid} \sep \cass{\C{Rd}}{\lrid} \sep \cass{\C{Rd}}{\lrid} } \\
\specline{\boxass{\vx \pt 2}{\lrid}{\intass} \sep \cass{\C{Inc}}{\lrid} } \\
\end{session}
\]
\subsubsection{SI/PSI}
\[
    \begin{array}{@{}l@{}}
        \intass : 
        \begin{rclarray}[t]
        \C{Inc} & : & \exsts{\V{m, k}} \Set{(\etR, \V{x}, \V{m}), (\etW, \V{x}, \V{m} + 1)} \mat \V{x} \pt \V{k} \oassto \V{x} \pt \V{k} \\
        \C{Rd}  & : & \exsts{\V{m, k, v}} \Set{(\etR, \V{x}, \V{m})} \mat \V{x} \pt \V{k} \oassto \V{x} \pt \V{v} \land \V{v} \leq \V{k} \\ 
        \end{rclarray} \\
        \C{Inc} \composeK \C{Inc} \text{ is undefined} \\
        \C{Rd} \text{ is the unit element} \\
    \end{array}
\]

\[
\begin{session}
\specline{\boxass{\vx \pt 0}{\lrid}{\intass} \sep \cass{\C{Inc}}{\lrid} } \\
\specline{\boxass{\vx \pt 0}{\lrid}{\intass} \sep \cass{\C{Inc}}{\lrid} \sep \cass{\C{Rd}}{\lrid} \sep \cass{\C{Rd}}{\lrid} } \\
\begin{parl}
    \begin{session}
    \specline{\boxass{\vx \pt 0}{\lrid}{\intass} \sep \cass{\C{Inc}}{\lrid} \sep \cass{\C{Rd}}{\lrid} } \\
    \begin{transaction}
        \specline{ \vx \pt 0 } \\
        \pderef{\pvar{a}}{\vx} ; \\
        \specline{ \vx \pt 0 \land \pvar{a} = 0 \sep \Set{(\etR, \vx, 0)} } \\
        \pmutate{\vx}{\pvar{a} + 1} ; \\
        \specline{ \vx \pt 1 \land \pvar{a} = 0 \\
                {} \sep \Set{(\etR, \vx, 0), (\etW, \vx, 1)} } \\
    \end{transaction} \\
    \specline{\boxass{\vx \pt 1}{\lrid}{\intass} \sep \cass{\C{Inc}}{\lrid} \sep \cass{\C{Rd}}{\lrid} } \\
    \begin{transaction}
        \specline{ {\color{purple} \vx \pt 0} \lor \vx \pt 1 } \\
        \pderef{\pvar{b}}{\vx} ; \\
        \specline{ { \color{purple} \vx \pt 0 \sep \Set{(\etR, \vx, 0)} }  \\
                    {} \lor \vx \pt 1 \sep \Set{(\etR, \vx, 1)} } \\
    \end{transaction} \\
    \specline{\boxass{\vx \pt 1}{\lrid}{\intass} \sep \cass{\C{Inc}}{\lrid} \sep \cass{\C{Rd}}{\lrid} } \\
    \begin{transaction}
        \specline{ \vx \pt 1 } \\
        \pderef{\pvar{a}}{\vx} ; \\
        \specline{ \vx \pt 1 \land \pvar{a} = 1 \sep \Set{(\etR, \vx, 1)} } \\
        \pmutate{\vx}{\pvar{a} + 1} ; \\
        \specline{ \vx \pt 2 \land \pvar{a} = 1 \\
                {} \sep \Set{(\etR, \vx, 1), (\etW, \vx, 2)} } \\
    \end{transaction} \\
    \specline{\boxass{\vx \pt 2}{\lrid}{\intass} \sep \cass{\C{Inc}}{\lrid} \sep \cass{\C{Rd}}{\lrid} } \\
    \end{session}
    &
    \begin{session}
    \specline{\exsts{\V{n}}\boxass{\vx \pt \V{n}}{\lrid}{\intass} \land \V{n} \geq 0 \sep \cass{\C{Rd}}{\lrid} } \\
    \begin{transaction}
        \specline{ \exsts{ \V{v} } \vx \pt \V{v} { \color{gray} \land \V{v} \leq \V{n} } } \\
        \pderef{\pvar{c}}{\vx} ; \\
        \specline{ \exsts{ \V{v} } \vx \pt \V{v} \sep \Set{(\etR, \vx, \V{v})} { \color{gray} \land \V{v} \leq \V{n} } } \\
    \end{transaction} \\
    \specline{\exsts{\V{n}}\boxass{\vx \pt \V{n}}{\lrid}{\intass} \land \V{n} \geq 0 \sep \cass{\C{Rd}}{\lrid} } \\
    \end{session}
\end{parl} \\
\specline{\boxass{\vx \pt 1}{\lrid}{\intass} \sep \cass{\C{Inc}}{\lrid} \sep \cass{\C{Rd}}{\lrid} \sep \cass{\C{Rd}}{\lrid} } \\
\specline{\boxass{\vx \pt 1}{\lrid}{\intass} \sep \cass{\C{Inc}}{\lrid} } \\
\end{session}
\]

\subsubsection{Causal}

\[
    \begin{array}{@{}l@{}}
        \intass : 
        \begin{rclarray}[t]
        \C{Inc} & : & \exsts{\V{m, k}} \Set{(\etR, \V{x}, \V{m}), (\etW, \V{x}, \V{m} + 1)} \mat \V{x} \pt \V{k} \oassto \V{x} \pt \V{v} \\
        \C{Rd}  & : & \exsts{\V{m, k, v}} \Set{(\etR, \V{x}, \V{m})} \mat \V{x} \pt \V{k} \oassto \V{x} \pt \V{v} \\ 
        \end{rclarray} \\
        \C{Inc} \composeK \C{Inc} \text{ is undefined} \\
        \C{Rd} \text{ is the unit element} \\
    \end{array}
\]

\[
\begin{session}
\specline{\boxass{\vx \pt 0}{\lrid}{\intass} \sep \cass{\C{Inc}}{\lrid} } \\
\specline{\boxass{\vx \pt 0}{\lrid}{\intass} \sep \cass{\C{Inc}}{\lrid} \sep \cass{\C{Rd}}{\lrid} \sep \cass{\C{Rd}}{\lrid} } \\
\begin{parl}
    \begin{session}
    \specline{\boxass{\vx \pt 0}{\lrid}{\intass} \sep \cass{\C{Inc}}{\lrid} \sep \cass{\C{Rd}}{\lrid} } \\
    \begin{transaction}
        \specline{ \exsts{ \V{v} } \vx \pt \V{v} } \\
        \pderef{\pvar{a}}{\vx} ; \\
        \specline{ \exsts{ \V{v} } \vx \pt \V{v} \land \pvar{a} = \V{v} \sep \Set{(\etR, \vx, \V{v})} } \\
        \pmutate{\vx}{\pvar{a} + 1} ; \\
        \specline{ \vx \pt \V{v} + 1 \land \pvar{a} = \V{v} \\
                {} \sep \Set{(\etR, \vx, \V{v}), (\etW, \vx, \V{v} + 1)} } \\
    \end{transaction} \\
    \specline{\exsts{\V{n}}\boxass{\vx \pt \V{n}}{\lrid}{\intass} \sep \cass{\C{Inc}}{\lrid} \sep \cass{\C{Rd}}{\lrid} } \\
    \begin{transaction}
        \specline{ \exsts{ \V{v} } \vx \pt \V{v} } \\
        \pderef{\pvar{c}}{\vx} ; \\
        \specline{ \exsts{ \V{v} } \vx \pt \V{v} \sep \Set{(\etR, \vx, \V{v})} } \\
    \end{transaction} \\
    \specline{\exsts{\V{n}}\boxass{\vx \pt \V{n}}{\lrid}{\intass} \sep \cass{\C{Inc}}{\lrid} \sep \cass{\C{Rd}}{\lrid} } \\
    \begin{transaction}
        \specline{ \exsts{ \V{v} } \vx \pt \V{v} } \\
        \pderef{\pvar{a}}{\vx} ; \\
        \specline{ \exsts{ \V{v} } \vx \pt \V{v} \land \pvar{a} = \V{v} \sep \Set{(\etR, \vx, \V{v})} } \\
        \pmutate{\vx}{\pvar{a} + 1} ; \\
        \specline{ \vx \pt \V{v} + 1 \land \pvar{a} = \V{v} \\
                {} \sep \Set{(\etR, \vx, \V{v}), (\etW, \vx, \V{v} + 1)} } \\
    \end{transaction} \\
    \specline{\exsts{\V{n}}\boxass{\vx \pt \V{n}}{\lrid}{\intass} \sep \cass{\C{Inc}}{\lrid} \sep \cass{\C{Rd}}{\lrid} } \\
    \end{session}
    &
    \begin{session}
    \specline{\exsts{\V{n}}\boxass{\vx \pt \V{n}}{\lrid}{\intass} \sep \cass{\C{Rd}}{\lrid} } \\
    \begin{transaction}
        \specline{ \exsts{ \V{v} } \vx \pt \V{v} { \color{gray} \land \V{v}, \V{n} } } \\
        \pderef{\pvar{c}}{\vx} ; \\
        \specline{ \exsts{ \V{v} } \vx \pt \V{v} \sep \Set{(\etR, \vx, \V{v})} { \color{gray} \land \V{v}, \V{n} } } \\
    \end{transaction} \\
    \specline{\exsts{\V{n}}\boxass{\vx \pt \V{n}}{\lrid}{\intass} \sep \cass{\C{Rd}}{\lrid} } \\
    \end{session}
\end{parl} \\
\specline{\exsts{\V{n}}\boxass{\vx \pt \V{n}}{\lrid}{\intass} \sep \cass{\C{Inc}}{\lrid} \sep \cass{\C{Rd}}{\lrid} \sep \cass{\C{Rd}}{\lrid} } \\
\specline{\exsts{\V{n}}\boxass{\vx \pt \V{n}}{\lrid}{\intass} \sep \cass{\C{Inc}}{\lrid} } \\
\end{session}
\]

\subsection{Two associated bank accounts}
\[
    \begin{array}{@{}l@{}}
        \boxass{\V{x} \pt \V{n} \sep \V{y} \pt \V{m} }{\lrid}{\intass} \\
        \C{xx} \composeK \C{xx} \text{ is undefined} \\
        \C{yy} \composeK \C{yy} \text{ is undefined} \\
        \C{Rd} \text{ is the unit} \\
    \end{array}          
\]

\subsubsection{SER}
\[
    \begin{array}{@{}l@{}}
        \intass : 
        \begin{rclarray}[t]
        \C{xx} & : & \exsts{\V{v, k, a, b }} \Setcon{(\etR, \V{x}, \V{v}), (\etR, \V{y}, \V{k}), (\etW, \V{x}, \V{v} - 100)}{\V{v} + \V{k} \geq 100} \\
        & & \qqquad \mat \V{x} \pt \V{a} \sep \V{y} \pt \V{b} \oassto \V{x} \pt \V{a} \sep \V{y} \pt \V{b} \land \V{a} + \V{b} \geq 0 \\
        \C{yy} & : & \exsts{\V{v, k, a, b }} \Setcon{(\etR, \V{x}, \V{v}), (\etR, \V{y}, \V{k}), (\etW, \V{y}, \V{k} - 100)}{\V{v} + \V{k} \geq 100} \\
        & & \qqquad \mat \V{x} \pt \V{a} \sep \V{y} \pt \V{b} \oassto \V{x} \pt \V{a} \sep \V{y} \pt \V{b} \land \V{a} + \V{b} \geq 0 \\
        \C{Rd} & : & \exsts{\V{v, k, a, b }} \Set{(\etR, \V{x}, \V{v}), (\etR, \V{y}, \V{k})} \\
        & & \qqquad \mat \V{x} \pt \V{a} \sep \V{y} \pt \V{b} \oassto \V{x} \pt \V{a} \sep \V{y} \pt \V{b} \land \V{a} + \V{b} \geq 0 \\
        \end{rclarray} \\
    \end{array}
\]

\[
\begin{session}
\specline{ \boxass{ \vx \pt 60 \sep \vy \pt 60 }{\lrid}{\intass} \sep \cass{\C{xx}}{\lrid} \sep \cass{\C{yy}}{\lrid} } \\
\begin{parl}
    \begin{session}
        \specline{ \boxass{ \vx \pt 60 \sep \vy \pt 60 \lor \vx \pt 60 \sep \vy \pt -40 }{\lrid}{\intass} \sep \cass{\C{xx}}{\lrid} } \\
        \begin{transaction}
            \specline{\vx \pt 60 \sep \vy \pt 60 \\ {} \lor \vx \pt 60 \sep \vy \pt -40} \\
            \pderef{\pvar{a}}{\vx}; \\
            \pderef{\pvar{b}}{\vy}; \\
            \pifs{\pvar{a} + \pvar{b} \geq 100} \\
            \quad \pmutate{\vx}{\pvar{a} - 100} ; \\
            \pife \\
            \specline{\vx \pt-40 \sep \vy \pt 60 \sep {} \\
            \Set{(\etR, \vx, 60), (\etR, \vy, 60), (\etW, \vx, -40)} \\ 
            {} \lor \vx \pt 60 \sep \vy \pt -40 \sep {} \\
            \Set{(\etR, \vx, 60), (\etR, \vy, -40)} }
        \end{transaction} \\
        \specline{ \boxass{ \vx \pt -40 \sep \vy \pt 60 \lor \vx \pt 60 \sep \vy \pt -40 }{\lrid}{\intass} \sep \cass{\C{xx}}{\lrid} } \\
    \end{session}
    &
    \begin{session}
        \specline{ \boxass{ \vx \pt -40 \sep \vy \pt 60 \lor \vx \pt 60 \sep \vy \pt 60 }{\lrid}{\intass} \sep \cass{\C{yy}}{\lrid} } \\
        \begin{transaction}
            \pderef{\pvar{a}}{\vx}; \\
            \pderef{\pvar{b}}{\vy}; \\
            \pifs{\pvar{a} + \pvar{b} \geq 100} \\
            \quad \pmutate{\vy}{\pvar{b} - 100} ; \\
            \pife 
        \end{transaction} \\
        \specline{ \boxass{ \vx \pt -40 \sep \vy \pt 60 \lor \vx \pt 60 \sep \vy \pt -40 }{\lrid}{\intass} \sep \cass{\C{yy}}{\lrid} } \\
    \end{session}
\end{parl} \\
\specline{ \boxass{ \vx \pt -40 \sep \vy \pt 60 \lor \vx \pt 60 \sep \vy \pt -40 }{\lrid}{\intass} \sep \cass{\C{xx}}{\lrid} \sep \cass{\C{yy}}{\lrid} } \\
\end{session}
\]

\subsubsection{SI/PSI}
\[
    \begin{array}{@{}l@{}}
        \intass : 
        \begin{rclarray}[t]
        \C{xx} & : & \exsts{\V{v, k, a, b, c }} \Setcon{(\etR, \V{x}, \V{v}), (\etR, \V{y}, \V{k}), (\etW, \V{x}, \V{v} - 100)}{\V{v} + \V{k} \geq 100} \\
        & & \qqquad \mat \V{x} \pt \V{a} \sep \V{y} \pt \V{b} \oassto \V{x} \pt \V{a} \sep \V{y} \pt \V{b} + ( \V{c} \times 100 ) \\
        \C{yy} & : & \exsts{\V{v, k, a, b, c }} \Setcon{(\etR, \V{x}, \V{v}), (\etR, \V{y}, \V{k}), (\etW, \V{y}, \V{k} - 100)}{\V{v} + \V{k} \geq 100} \\
        & & \qqquad \mat \V{x} \pt \V{a} \sep \V{y} \pt \V{b} \oassto \V{x} \pt \V{a} + ( \V{c} \times 100 ) \sep \V{y} \pt \V{b} \\
        \C{Rd} & : & \exsts{\V{v, k, a, b, c, d }} \Set{(\etR, \V{x}, \V{v}), (\etR, \V{y}, \V{k})} \\
        & & \qqquad \mat \V{x} \pt \V{a} \sep \V{y} \pt \V{b} \oassto \V{x} \pt \V{a} + ( \V{c} \times 100 ) \sep \V{y} \pt \V{b} + ( \V{d} \times 100 ) \\
        \end{rclarray} \\
    \end{array}
\]

\[
\begin{session}
\specline{ \boxass{\vx \pt 60 \sep \vy \pt 60 }{\lrid}{\intass} \sep \cass{\C{xx}}{\lrid} \sep \cass{\C{yy}}{\lrid} } \\
\begin{parl}
    \begin{session}
        \specline{ \exsts{ \V{n} } \boxass{\vx \pt 60 \sep \vy \pt 60 - \V{n} \times 100 }{\lrid}{\intass} \sep \cass{\C{xx}}{\lrid} } \\
        \begin{transaction}
            \specline{ \exsts{ \V{n}, \V{k} \geq 0 } \vx \pt 60 + \V{k} \times 100 \sep \vy \pt 60 + \V{n} \times 100 } \\
            \pderef{\pvar{a}}{\vx}; \\
            \pderef{\pvar{b}}{\vy}; \\
            \specline{ \exsts{ \V{n}, \V{k} \geq 0 } \vx \pt 60 + \V{k} \times 100 \sep \vy \pt 60 + \V{n} \times 100 \\
                        {} \sep \Set{ (\etR, \vx, 60 + \V{k} \times 100), (\etR, \vx, 60 + \V{n} \times 100) } \\
                        {} \land \pvar{a} = 60 + \V{k} \times 100 \land \pvar{b} = 60 + \V{n} \times 100 } \\
            \pifs{\pvar{a} + \pvar{b} \geq 100} \\
            \quad \specline{ \exsts{ \V{n}, \V{k} \geq 0 } \vx \pt 60 + \V{k} \times 100 \sep \vy \pt 60 + \V{n} \times 100 \\
                            {} \sep \Set{ (\etR, \vx, 60 + \V{k} \times 100), (\etR, \vx, 60 + \V{n} \times 100) } \\
                            {} \land \pvar{a} = 60 + \V{k} \times 100 \land \pvar{b} = 60 + \V{n} \times 100 \land \V{k} + \V{N} \geq 0} \\
            \quad \pmutate{\vx}{\pvar{a} - 100} ; \\
            \quad \specline{ \exsts{ \V{n}, \V{k} \geq 0 } \vx \pt -40 + \V{k} \times 100 \sep \vy \pt 60 + \V{n} \times 100 \\
                            {} \sep \Set{ (\etR, \vx, 60 + \V{k} \times 100), (\etR, \vx, 60 + \V{n} \times 100), \\ 
                                                    (\etW, \vx, -40 + \V{k} \times 100) } \\
                            {} \land \pvar{a} = 60 + \V{k} \times 100 \land \pvar{b} = 60 + \V{n} \times 100 \land \V{k} + \V{N} \geq 0} \\
            \pife \\
            \comment{Weaken the assertion by } \\
            \comment{throwing away program variables.} \\
            \specline{ \exsts{ \V{n}, \V{k} \geq 0 } \vy \pt 60 + \V{n} \times 100 \\
                        {} \sep 
                        \begin{formulea}
                        \vx \pt -40 + \V{k} \times 100 
                        \land \V{k} + \V{N} \geq 0 \\
                        {} \sep \Set{ (\etR, \vx, 60 + \V{k} \times 100), (\etR, \vx, 60 + \V{n} \times 100), \\
                                                (\etW, \vx, -40 + \V{k} \times 100) } \\
                        \end{formulea} \\
                        {} \lor 
                        \begin{formulea}
                        \vx \pt 60 + \V{k} \times 100 \\ 
                        {} \sep \Set{ (\etR, \vx, 60 + \V{k} \times 100), (\etR, \vx, 60 + \V{n} \times 100) } 
                        \end{formulea}
                    } \\
        \end{transaction} \\
        \comment{To allow the write to be committed,} \\
        \comment{the \V{K} must be 0 } \\
        \specline{ \exsts{ \V{n} } \boxass{ ( \vx \pt 60 \lor \vx \pt -40 ) \sep \vy \pt 60 - \V{n} \times 100 }{\lrid}{\intass} \sep \cass{\C{xx}}{\lrid} } \\
    \end{session}
    &
    \begin{session}
        \specline{ \exsts{ \V{n} } \boxass{\vx \pt 60 - \V{n} \times 100 \sep \vy \pt 60 }{\lrid}{\intass} \sep \cass{\C{yy}}{\lrid} } \\
        \begin{transaction}
            \pderef{\pvar{a}}{\vx}; \\
            \pderef{\pvar{b}}{\vy}; \\
            \pifs{\pvar{a} + \pvar{b} \geq 100} \\
            \quad \pmutate{\vy}{\pvar{b} - 100} ; \\
            \pife 
        \end{transaction} \\
        \specline{ \exsts{ \V{n} } \boxass{\vx \pt 60 - \V{n} \times 100 \sep ( \vy \pt 60 \lor \vy \pt -40 ) }{\lrid}{\intass} \sep \cass{\C{yy}}{\lrid} } \\
    \end{session}
\end{parl} \\
\specline{ \boxass{\vx \pt -40 \sep \vy \pt 60 \lor \vx \pt 60 \sep \vy \pt -40 \lor \vx \pt -40 \sep \vy \pt -40 }{\lrid}{\intass} \sep \cass{\C{xx}}{\lrid} \sep \cass{\C{yy}}{\lrid} } \\
\end{session}
\]

%A dummy bank transfer example that is not serialisable.
%\[
    %\begin{rclarray}
        %\intass(\rid) & = &
        %\begin{cases}
            %\unitelem{} : \exsts{x, m, k} x \fpt{\fp} n \sep \cass{S(m)}{\rid} \transfersto x \fpt{\addFPW{\fp}} n \pm k \sep  \cass{S(m \pm k)}{\rid} \\
            %\unitelem{} : \exsts{x} x \fpt{\fp} n \transfersto x \fpt{\addFPR{\fp}} n 
        %\end{cases} \\
        %S(m) \composeK S(n) & = & S(m+n) \\
        %S(0) & \in & \unitK \\
    %\end{rclarray}
%\]
%A dummy bank transfer example that is serialisable even under snapshot isolation.
%\[
    %\begin{rclarray}
        %\intass & = &
        %\begin{cases}
            %\unitelem{} : \exsts{x, y, m, k} x \fpt{\fp} n \sep  y \fpt{\fp'} m \transfersto x \fpt{\addFPW{\fp}} n - k \sep \fpt{\addFPW{\fp}} m + k  \\
            %\unitelem{} : \exsts{x} x \fptEMP n \transfersto x \fptR n 
        %\end{cases}
    %\end{rclarray}
%\]
%If x write y and if y write x example.
%\[
    %\begin{rclarray}
        %\intass(x,y) & = &
        %\begin{cases}
            %\perm{L} : x \fptEMP 0 \sep y \fptEMP 0 \transfersto x \fptW 1 \sep y \fptR 0 \\
            %\perm{R} : x \fptEMP 0 \sep y \fptEMP 0 \transfersto x \fptR 0 \sep y \fptW 1 \\
        %\end{cases}
    %\end{rclarray}
%\]


\bibliography{bibliography,bibliography2}

%\newpage
\appendix
\section{Proof of Theorem \ref{thm:kv2graph}}

\begin{proposition}
Let $\hh$ be a well-formed kv-store. Then $\Gr_{\hh}$ is a well-formed dependency graph.
\end{proposition}

\begin{proof}
Let $\hh$ be a (well-formed) kv-store. We need to show that 
$\Gr_{\hh} = (\TtoOp{T}_{\hh}, \RF_{\hh}, \VO_{\hh}, \AD_{\hh})$ is a dependency graph. 
As a first step, we show that $\Gr_{\hh}$ is a dependency graph, 
i.e. it satisfies all the constraints placed by \cref{def:dgraph}.

\begin{itemize}
\item Let $\txid \in \dom(\TtoOp{T}_{\hh})$, and suppose that $(\otR, \ke, \val) \in \TtoOp{T}_{\hh}(\txid)$. 
We need to prove that either $\val = \val_{0}$, and there exists no $\txid' \in \dom(\TtoOp{T}_{\hh})$ such that 
$\txid' \xrightarrow{\RF_{\hh}(\ke)} \txid$, or $\txid' \xrightarrow{\RF_{\hh}(\ke)} \txid$ for some 
$\txid' \in \dom(\TtoOp{T}_{\hh})$ such that $(\otW, \ke, \val) \in \TtoOp{T}_{\hh}(\txid')$. 
Because $\txid \in \dom(\TtoOp{T}_{\hh})$, the definition of $\Gr_{\hh}$ (and in particular the 
fact that $\TtoOp{T}_{\hh} : \TxID_{0} \rightharpoonup \powerset{\Ops}$) ensures that 
$\txid \neq \txid_0$. Furthermore, because $(\otR, \ke, \val) \in \TtoOp{T}_{\hh}(\txid)$, there 
must exist an index $i: 0 \leq i < \lvert \hh(\ke) \rvert$ such that $\hh(\ke, i) = (\val, \txid', \{\txid \} \cup \_ )$ 
for some $\txid' \in \TxID$. 
We have two possibilities: 
\begin{enumerate}
\item $i = 0$, in which case the hypothesis that $\hh$ is well-formed ensures that $\txid' = \txid_0$, 
and $\val = \val_0$. We also have that there exists no transaction $\txid''$ such that $\txid'' \xrightarrow{\RF_{\hh}(\ke)} \txid$: 
in fact, by \cref{def:kv2graph}, we have that $\txid'' \xrightarrow{\RF(\ke)} \txid$ if and only if there exists an index 
$j: 0 < j < \lvert \hh(\ke) \rvert$ such that $\hh(\ke, j) = (\_, \txid'', \{\txid\} \cup \_)$. However, in this case we would 
have that $0 < j$, and $\txid \in \RTx(\hh(\ke, j)) \cap \RTx(\hh(\ke, 0))$, contradicting the constraint placed 
over well-formed kv-stores that a transaction never reads multiple versions for a key. Therefore, there exists 
no transaction $\txid''$ such that $\txid'' \xrightarrow{\RF_{\hh}(\ke)} \txid$, 
\item $i > 0$; in this case \cref{def:kv2graph} ensures that $\txid' \xrightarrow{\RF_{\hh}(\ke)} \txid$; also, 
because $\hh(\ke, i) = (\val, \txid', \_)$, it must be the case that $(\otW, \ke, \val) \in \TtoOp{T}_{\hh}(\txid')$.
\end{enumerate}
\item Let $\txid \in \dom(\TtoOp{T}_{\hh})$, and suppose that there exist $\txid_1, \txid_2$ such that 
$\txid_{1} \xrightarrow{\RF_{\ke}(\hh)} \txid$, $\txid_{2} \xrightarrow{\RF_{\ke}(\hh)} \txid$. 
By \cref{def:kv2graph}, there exist two indexes $i, j: 0 < i, j < \lvert \hh(\ke) \rvert$, such that 
$\hh(\ke, i) = (\_, \txid_1, \{\txid\} \cup \_)$, $\hh(\ke, j) = (\_, \txid_2, \{\txid\} \cup \_)$. 
We have that $\txid \in \RTx(\hh(\ke, i)) \cap \RTx(\hh(\ke, j))$, i.e. 
$\RTx(\hh(\ke,i)) \cap \RTx(\hh(\ke, j)) \neq \emptyset$. Because we are assuming 
that $\hh$ is well-formed, then it must be the case that $i = j$. This implies that $\txid_1 = \txid_2$.
\item Let $\cl \in \Clients$, $m, n \in \Nat$ and $\ke \in \Keys$ be such that 
$\txid_{\cl}^{n} \xrightarrow{\RF_{\hh}(\ke)} \txid_{\cl}^{m}$.  We prove that 
$n < m$. By \cref{def:kv2graph}, it must be the case that 
there exists an index $i : 0 \leq i < \lvert \hh(\ke) \rvert$ such that $\hh(\ke, i) = 
(\_, \txid_{\cl}^{n}, \{\txid_{\cl}^{m}\} \cup \_)$. Because $\hh$ is well-formed, 
it must be the case that $n < m$.
\item Let $\txid \in \dom(\TtoOp{T}_{\hh})$. We show that $\neg (\txid \xrightarrow{\VO_{\hh}} \txid)$. 
We prove this fact by contradiction: suppose that $\txid \xrightarrow{\VO_{\hh}(\ke)} \txid$ for some key $\ke$. By \cref{def:kv2graph}, 
there must exist two indexes $i,j: 0 < i < j < \lvert \hh(\ke) \rvert$ such that $\txid = \WTx(\hh(\ke,i))$ and 
$\txid = \WTx(\hh(\ke, j))$. Because we are assuming that $\hh$ is well-formed, then it must be the 
case that $i = j$, contradicting the statement that $i < j$. 
\item Let $\txid, \txid'$ be such that $\txid' \xrightarrow{\VO_{\ke}(\hh)} \txid$. 
We must show that  $(\otW, \ke, \_) \in \TtoOp{T}_{\hh}(\txid')$, and $(\otW, \ke, \_) \in \TtoOp{T}_{\hh}(\txid)$.
By \cref{def:kv2graph}, there exist $i, j: 0 < i,j < \lvert \hh(\ke) \rvert$ such that 
$\hh(\ke, i) = (\val', \txid', \_)$ and $\hh(\ke, j) = (\val, \txid, \_)$, for some 
$\val, \val' \in \Val$. \cref{def:kv2graph} also ensures that $(\otW, \ke, \val') \in 
\TtoOp{T}_{\hh}(\txid')$, and $(\otW, \ke, \val) \in \TtoOp{T}_{\hh}(\txid)$.
\item Let $\txid, \txid'$ be such that $(\otW, \ke, \_) \in \TtoOp{T}_{\hh}(\txid)$ 
and $(\otW, \ke, \_) \in \TtoOp{T}_{\hh}(\txid')$. We need to prove that 
either $\txid = \txid', \txid \xrightarrow{\VO_{\hh}(\ke)} \txid'$, or $\txid' \xrightarrow{\VO_{\hh}(\ke)} \txid$. 
By \cref{def:kv2graph} there exist two indexes $i, j: 0 < i,j< \lvert \hh(\ke) \rvert$ such that 
$\hh(\ke, i) = (\_, \txid, \_)$ and $\hh(\ke, j) = (\_, \txid', \_)$. If $i = j$, then $\txid = \txid'$ 
and there is nothing left to prove. Otherwise, suppose without loss of generality that 
$i < j$. Then \cref{def:kv2graph} ensures that $\txid \xrightarrow{\VO_{\hh}(\ke)} \txid'$. 
\item Suppose that $\txid_{\cl}^{m} \xrightarrow{\VO_{\hh}(\ke)} \txid_{\cl}^{n}$ for 
some $\cl \in \Clients$ and $m, n \in \Nat$. We need to show that $m < n$. 
By \cref{def:kv2graph}, because  $\txid_{\cl}^{m} \xrightarrow{\VO_{\hh}(\ke)} \txid_{\cl}^{n}$ 
there exist two indexes $i,j: 0 < i,j < \lvert \hh(\ke) \rvert$ such that $\WTx(\hh(\ke,i)) = \txid_{\cl}^{m}$ 
and $\WTx(\hh(\ke, j)) = \txid_{\cl}^{n}$. From the assumption that $\hh$ is well-formed, it 
follows that $n < m$.
\end{itemize}
\end{proof}

\begin{definition}
Given a dependency graph $\Gr = (\TtoOp{T}, \RF, \VO, \AD)$, we define the kv-store $\hh_{\Gr}$ as follows: 
\begin{enumerate}
\item for any transaction $\txid \in \dom(\TtoOp{T})$ such that $(\otW, \ke, \val) \in \TtoOp{T}(\txid)$, 
let $\T = \{ \txid' \mid \txid \xrightarrow{\RF(\ke)} \txid'\}$, and let $\ver(\txid, \ke) = (\val, \txid, \T)$, 
\item For each key $\ke$, let $\ver_{\ke}^{0} = (\val_0, \txid_0, \T_k^{0})$, where $\T_{k}^{0} = \{ \txid \mid (\otR, \ke, \_) \in 
\TtoOp{T}(\txid) \wedge \forall \txid'.\; \neg( \txid' \xrightarrow{\RF(\ke)} \txid \}$. 
Let also  $\{ \ver_{\ke}^{i} \}_{i = 1}^{n}$ be the ordered set of versions such that, for any 
$i=1,\cdots,n$, $\ver_{\ke}^{i} = \ver(\txid, \ke)$ for some $\txid$ such that $(\otW, \ke, \_) \in \TtoOp{T}(\txid)$, 
and such that for any $i, j: 1 \leq i < j \leq n$, $\WTx(\ver_{\ke}^{i}) \xrightarrow{\VO(\ke)} \WTx(\ver_{\ke}^{j})$. 
Then we let $\hh_{\Gr}= \lambda \ke. \prod_{i=0}^{n} \ver_{\ke}^{i}$.
\end{enumerate}
\end{definition}

\begin{proposition}
For any dependency graph $\Gr = (\TtoOp{T}, \VO, \RF, \AD)$, $\hh_{\Gr}$ is a well-formed kv-store.
\end{proposition}

\begin{proof}
We prove that each of the four constraints required by well-formed kv-stores 
are satisfied by $\hh_{\Gr}$. 
\begin{enumerate}[(i)]
\item For each key $\ke$, $\hh_{\Gr}(\ke, 0) = (\val_0, \txid_0, \_)$. 
By construction, we have that $\hh_{\Gr}(\ke, 0) = \ver_{\ke}^{0} = (\val_0, \txid_0, \_)$. 
\item $\forall \ke \in \Keys.\; \forall i,j: 0 \leq i, j < \lvert \hh_{\Gr}(\ke) \rvert$, 
$\WTx(\hh_{\Gr}(\ke, i)) = \WTx(\hh_{\Gr}(\ke, j)) \implies i = j$.
Let $\ke \in \Keys$, and let $i, j: 0 \leq i,j < \lvert \hh_{\Gr}(\ke) \rvert$ 
be such that $\WTx(\hh_{\Gr}(\ke, i)) = \WTx(\hh_{\Gr}(\ke, j))$. 
Without loss of generality, we can assume that $i \leq j$. 
First, note that if $i = 0$, then $\WTx(\hh_{\Gr}(\ke, i)) = \txid_0$, 
hence it must be the case that $\WTx(\hh_{\Gr}(\ke, j)) = \txid_0$. 
By construction, it is also the case that $\hh_{\Gr}(\ke, j) = \ver_{\ke}^{j}$, 
hence either one of the following is true: 
\begin{enumerate}
\item $j = 0$, in which case there is nothing to prove, or 
\item $j > 0$, and $\hh_{\Gr}(\ke, j) = \ver_{\ke}^{j} = 
\ver(\txid, \ke)$ for some $\txid \in \dom(\TtoOp{T})$. 
We have that $\WTx(\hh_{\Gr}(\ke, j) = \WTx(\ver(\txid, \ke)) = \txid$, 
and because $\txid \in \dom(\TtoOp{T})$, it must be $\txid \neq \txid_0$. 
Contradiction.
\end{enumerate}
Suppose then that $i > 0$. Therefore, it must be the case that $\hh_{\Gr}(\ke, i) = 
\ver_{\ke}^{i} = \ver(\txid, \ke)$ for some $\txid \in \dom(\TtoOp{T})$ such that 
$(\otW, \ke, \_) \in \TtoOp{T}(\txid_{i})$. Similarly, because we are assuming 
that $i \leq j$, we have that $\hh_{\Gr}(\ke, j) = \ver_{\ke}^{j} = \ver(\txid, \ke)$. 
Note that $\hh_{\Gr}(\ke, i) = \hh_{\Gr}(\ke, j)$. Finally, note that if it were 
$i < j$, then by construction we should have that $\txid \xrightarrow{\VO(\ke)} \txid$, 
contradicting the requirement that $\VO(\ke)$ is irreflexive. Therefore, it must 
be $i = j$. 
\item $\forall \ke \in \Keys.\; \forall i,j: 0 \leq i, j < \lvert \hh_{\Gr}(\ke) \rvert$, 
$\RTx(\hh_{\Gr}(\ke, i)) \cap \RTx(\hh_{\Gr}(\ke, j)) \neq \emptyset \implies i = j$. 
Let $\ke \in \Keys$, $i, j: 0 \leq i, j < \lvert \hh_{\Gr}(\ke) \rvert$, 
and $\txid \in \RTx(\hh_{\Gr}(\ke, i)) \cap \RTx(\hh_{\Gr}(\ke, j))$. Without loss 
of generality, suppose that $i \leq j$. We distinguish between two cases: 
\begin{enumerate}
\item $i = 0$; by construction, there exists no $\txid'$ such that 
$\txid' \xrightarrow{\RF(\ke)} \txid$. If it were $j > 0$, then it 
would be the case that $\hh_{\Gr}(\ke, j) = \ver(\txid', \ke)$ for some 
$\txid'$ such that $\txid' \xrightarrow{\RF(\ke)} \txid$; because 
such transaction $\txid'$ does not exist, it cannot be $j > 0$, and 
we are left with the case $j = 0$; in particular, $j = i$. 
\item $i > 0$; by construction, it must be the case that $\hh_{\Gr}(\ke, i) = 
\ver(\txid', \ke)$ for some $\txid'$ such that $\txid' \xrightarrow{\RF(\ke)} \txid$. 
Furthermore, because we are assuming that $i \leq j$, we also have that $j > 0$, 
and  therefore $\hh_{\Gr}(\ke, j) = \ver(\txid'', \ke)$ for some $\txid''$ such that 
$\txid'' \xrightarrow{\RF(\ke)} \txid$. We have that $\txid' \xrightarrow{\RF(\ke)} \txid$, 
and $\txid'' \xrightarrow{\RF(\ke)} \txid$. By definition of dependency graph, this implies 
that $\txid' = \txid''$. We have that $\WTx(\hh_{\Gr}(\ke, i)) = \txid'$, 
$\WTx(\hh_{\Gr}(\ke, j)) = \txid''$, and $\txid' = \txid''$; if it were $i < j$, 
then by construction we would have that $\txid' \xrightarrow{\VO(\ke)} \txid'$, 
contradicting the requirement of dependency graphs that $\VO(\ke)$ is irreflexive. 
Therefore, it must be the case that $i = j$.
\end{enumerate}
\item 
\begin{multline*}
\fora{ \ke \in \dom(\hh), \cl \in \Clients} \fora{ i,j; 0 \leq i < j < \lvert \hh_{\Gr}(\ke) \rvert}
\fora{ n, m \geq 0}\\ (\txid_{\cl}^{n} = \WTx(\hh_{\Gr}(\ke,i)) \wedge \txid_{\cl}^{m} \in \{\WTx(\hh_{\Gr}(\ke,j))\} \cup \RTx(\hh_{\Gr}(\ke, i)) \implies n < m.
\end{multline*}
Let $\ke \in \Keys$, $\cl \in \Clients$, $i, j: 0 \leq i < j < \lvert \hh_{\Gr}(\ke) \rvert$. Let also $n, m \geq 0$. 
First, suppose that $\txid_{\cl}^{n} = \WTx(\hh_{\Gr}(\ke, i)$.
Note that it cannot be $i = 0$, because by construction $\WTx(\hh_{\Gr}(\ke, i)) = \txid_0 \neq \txid_{\cl}^{n}$. 
Therefore, it must be $i > 0$. We prove the following facts: 
\begin{enumerate}
\item if $\txid_{\cl}^{m} = \WTx(\hh_{\Gr}(\ke, j))$, then $n < m$. By construction, 
$\hh_{\Gr}(\ke, i) = \ver(\txid_{\cl}^{n}, \ke)$, and $(\otW, \ke, \_) \in \TtoOp{T}(\txid_{\cl}^{n})$. 
Similarly, $\hh_{\Gr}(\ke, j) = \ver(\txid_{\cl}^{m}, \ke)$, and $(\otW, \ke, \_) \in \TtoOp{T}(\txid_{\cl}^{m})$. 
Because $i < j$, it must be the case that $\txid_{\cl}^{n} = \WTx(\ver(\txid_{\cl}^{n}, \ke) \xrightarrow{\VO(\ke)} 
\WTx(\ver(\txid_{\cl}^{m}, \ke)) = \txid_{\cl}^{m}, \ke)$, and by definition of dependency graph it follows that 
$n < m$, 
\item if $\txid_{\cl}^{m} \in \RTx(\hh_{\Gr}(\ke, i))$, then $n < m$. In this case we have that 
$\txid_{\cl}^{n} \xrightarrow{\RF(\ke)} \txid_{\cl}^{m}$ by construction, hence the definition 
of dependency graph ensures that $n < m$. 
\end{enumerate}
\end{enumerate}
\end{proof}

\begin{proposition}
For any kv-store $\hh$, $\hh_{\Gr_{\hh}} = \hh$.
\end{proposition}

\ac{Needs to be proved, should not be difficult but it will have lots of technical details.}
\subsection{semantics}
Let \( \repl \in \Repls \) denotes the set of totally ordered replicates.
Each replicate can have multiple clients, and 
each clients can commit a sequence of either read-only transitions or single-write transactions.
To model these, we annotate the transaction identifier with replicate \( \repl \), client \( \cl \), 
local time of the replicate \( n \) and read-only transactions count \( n' \), \ie \( \txidCOPS{\repl}{\cl}{n}{n'} \).
Note that the \( (n, \repl, n') \) can be treated as a single number that \( n \) are the higher bits, 
\( \repl \) the middle bits and \( n' \) the lower bits.
There is a total order among transitions from the same replica and from the same client.
We extend version with the set of all versions it dependencies on, \( \dep \in \pset{\Keys \times \TxID} \).
The function \( \depOf{\ver} \) denotes the dependencies set of the version.
For readability, we annotate view with either a replica, \( \viREPL \), or a client, \( \viCL \).
The view environment is extended with replicas and their views, \( \viewFunCOPS : (\Repls \times \ClientID ) \parfinfun \Views \).
We give the following semantics to capture the behaviours of the code.

\begin{lstlisting}[caption={put},label={lst:simplified-put}]
// mixing the client API and system API
put(repl,k,v,ctx) {

    // Dependency for previous reads and writes
    deps = ctx_to_dep(ctx);(*\label{line:put-ctx-to-deps}*)

    atomic{
        // increase local time.
        inc(repl.local_time);(*\label{line:put-inc-local}*) 

        // appending local kv with a new version.
        list_isnert(repl.kv[k],(v, (local_time + id), deps));(*\label{line:put-update-kv}*)
    }

    // update dependency for writes
    ctx.writers += (k,(local_time + id),deps);(*\label{line:put-update-ctx}*)

    // put in the queue to sync with other replicas
    enqueue(k,v,(current_ver+id),(deps ++ vers));
}
\end{lstlisting}

The client always fetches the version with the maximum writer it can observed for each key,
Which is computed by \( \funcn{getMax} \) function. 
It is different from \( \snapshot \) as \( \snapshot \) fetches the latest version with respect to the position in the list.

\[
    \begin{rclarray}
        \func{getMax}{\mkvsCOPS, \viCL} & \defeq &
        \lambda \ke \ldotp \left( \max_\txid\Setcon{(\val, \txid, \T, \dep)}{\exsts{i} (\val, \txid, \T, \dep) = \mkvsCOPS(\ke, i)} \right)\projection{1}
    \end{rclarray}
\]
\begin{mathpar}
    \inferrule[Put]{%
        ( \stk, \func{getMax}{\mkvsCOPS, \viCL}, \emptyset ), \pmutate{\ke}{\vx} \toL
        ( \stk', \stub, \Set{(\otW, \ke, \val )} ), \pskip
        \\\\
        \dep = \Setcon{(\ke', \txid)}{\exsts{i} i \in \viCL(\ke') \land \txid = \WTx(\mkvsCOPS(\ke', i))} \texttt{ ---> \cref{lst:simplified-put}, \cref{line:put-ctx-to-deps}} 
        \\\\
        \txid = \min\Setcon{%
        \txidCOPS{\repl}{\cl}{n'}{0}
        }{%
            \fora{\ke', i \in \viREPL(\ke'), n} \\
            \quad \txidCOPS{\stub}{\stub}{n}{\stub} = \WTx(\mkvsCOPS(\ke',i)) \\
            \qquad {} \implies n' > n 
        } \texttt{ ---> \cref{lst:simplified-put}, \cref{line:put-inc-local}}
        \\\\
        \mkvsCOPS' = \mkvsCOPS\rmto{\ke}{\mkvsCOPS(\ke) \lcat \List{(\ke, \txid, \emptyset, \dep)}} \texttt{ ---> \cref{lst:simplified-put}, \cref{line:put-update-kv}}
        \\\\
        \viREPL' = \viREPL\rmto{\ke}{\viREPL(\ke) \uplus \Set{\abs{\mkvsCOPS'(\ke)} - 1}} \texttt{ ---> \cref{lst:simplified-put}, \cref{line:put-update-kv}}
        \\\\
        \viCL' = \viCL\rmto{\ke}{\viREPL(\ke) \uplus \Set{\abs{\mkvsCOPS'(\ke)} - 1}} \texttt{ ---> \cref{lst:simplified-put}, \cref{line:put-update-ctx}}
    }{%
    \repl, \cl \vdash 
    \mkvsCOPS, \viREPL, \viCL, \stk, \ptrans{\pmutate{\ke}{\vx};} \toT{}
    \mkvsCOPS', \viREPL', \viCL', \stk', \pskip
    }
\end{mathpar}
The \verb|get_trans| fetches the latest versions from the replica via multiple atomic reads, one for each key.
As a result, a client has a list of candidates \verb|rst|.
Since interleaving might happen, versions might become out-of-date because the replicate receives new versions.
It is not a problem to read old versions as long as they satisfy causal consistency,
\ie if a client read a version \( \ver \), it should at least read all the versions that \( \ver \) depends on.
Thus the algorithm use \verb|ccv| to track the maximum versions the client should fetches,
and re-fetches the \verb|ccv[k]| version from the replica if it is greater than the candidate.

The following is a simplified algorithm by directly taking a list of versions \verb|ccv| satisfies causal consistency constraint,
and then read the versions indicated by \verb|ccv|.
The simplified algorithm is easier to understand.
\begin{lstlisting}[caption={get\_trans},label={lst:get-trans}]
// A simplified version by guessing
// a ccv satisfying dependency constraints
// and then read versions indicated by ccv.
// Note that it is a weaker version of the original code,
// as the original implementation fetches the latest versions
// for keys by a sequence of atomic get_by_version calls
List(Val) get_trans(ks,ctx) {
    take ccv: (*$\forall$*) k (*$\in$*) ks.(*\label{line:get-trans-ccv-1}*)
        (_,_,deps) := get_by_version(k,ccv[k]) (*${}\land \forall$*) dep (*$\in$*) deps.(*\label{line:get-trans-ccv-2}*)
            dep.key (*$\in$*) ks (*$\implies$*) ccv[dep.key] >= dep.ver (*\label{line:get-trans-ccv-3}*)

    for k in ks(*\label{line:get-trans-read-1}*)
        rst[k] = get_by_version(k,ccv[k]);(*\label{line:get-trans-read-2}*)

    // update the ctx
    for (k,ver,deps) in rst(*\label{line:get-trans-update-ctx-1}*)
        ctx.readers += (k,ver,deps);(*\label{line:get-trans-update-ctx-2}*)

    return to_vals(ks);
}                                   
\end{lstlisting}
\begin{mathpar}
    \inferrule[GetTrans]{%
        \viCL \viewleq \viCL' \viewleq \viREPL  \texttt{ ---> \cref{lst:get-trans}, \cref{line:get-trans-update-ctx-1,line:get-trans-update-ctx-2}}
        \\\\
        {\left(\begin{array}{@{}l@{}}
        \fora{i : 1 \leq i \leq j, \ke', m, \ver}  \\
        \quad \ver = \mkvsCOPS(\ke_i, \max(\viCL'(\ke_i)) \land {} \\
        \quad (\ke', \WTx(\mkvsCOPS(\ke', m))) \in \ver\projection{4} \\
        \qquad {} \implies m \in \viCL'(\ke')
        \end{array}\right)} \texttt{ ---> \cref{lst:get-trans}, \cref{line:get-trans-ccv-1,line:get-trans-ccv-2,line:get-trans-ccv-3}}
        \\\\
        \trans =  \pderef{\vx_1}{\ke_1}; \dots; \pderef{\vx_j}{\ke_j};
        \\\\
        ( \stk, \func{getMax}{\mkvsCOPS, \viCL'}, \emptyset ), \trans \toL
        ( \stk', \stub, \f ), \pskip \texttt{ ---> \cref{lst:get-trans}, \cref{line:get-trans-read-1,line:get-trans-read-2}}
        \\\\
        \txidCOPS{\repl}{\cl}{n'}{n} = \max\Setcon{\txidCOPS{\repl}{\cl}{z'}{z}}{\txidCOPS{\repl}{\cl}{z'}{z} \in \mkvsCOPS }
        \\
        \mkvsCOPS' = \updKV{\mkvsCOPS, \viCL', \txidCOPS{\repl}{\cl}{n'}{n+1}, \f} 
    }{%
        \repl, \cl \vdash 
        \mkvsCOPS, \viREPL, \viCL, \stk, \ptrans{\pderef{\vx_1}{\ke_1}; \dots; \pderef{\vx_j}{\ke_j}; } \toT{}
        \mkvsCOPS', \viREPL, \viCL', \stk', \pskip
    }
    \and
    \inferrule[ClientCommit]{%
        \repl, \cl \vdash 
        \mkvsCOPS, \viewFunCOPS(\repl), \viewFunCOPS(\cl), \stk, \prog(\cl) \toT{}
        \mkvsCOPS', \viREPL', \viCL', \stk', \cmd'
    }{%
        \mkvsCOPS, \viewFunCOPS, \thdenv, \prog \toG{}
        \mkvsCOPS', \viewFunCOPS\rmto{\repl}{\viREPL'}\rmto{\cl}{\viCL'}, \thdenv\rmto{\cl}{\stk'}, \prog\rmto{\cl}{\cmd'}
    }
\end{mathpar}
A replica updates its local state only if all the dependencies has been receive.
\begin{lstlisting}[caption={Send and receive},label={lst:send-receive}]
// Syn to other replicas
send() {
    (k,v,ver,deps) := dequeue();
    for id in repls {
        send (k,v,ver,deps) to id;
    }
}

// receive a write message from other replica
on_receive(k,v,ver,deps) {
    // for a single machine
    // the following check immediately holds
    for (k',ver') in deps {
        wait until dep_check(k',ver');(*\label{line:receive-wait}*)
    }

    atomic{
        list_isnert(kv[k],(v,ver,deps));(*\label{line:receive-update-view-1}*)
        (remote_local_time + id) = ver;(*\label{line:receive-update-view-2}*)
        local_time = max(remote_local_time, local_time);(*\label{line:receive-update-view-3}*)
    }
}
\end{lstlisting}
\begin{mathpar}
    \inferrule[sync]{%
        \viREPL = \viewFunCOPS(\repl)\rmto{\ke}{\viewFunCOPS(\repl)(\ke) \uplus i} 
        \texttt{ ---> \cref{lst:send-receive}, \cref{line:receive-update-view-1,line:receive-update-view-2,line:receive-update-view-3}}
        \\\\
        {\left(\begin{array}{@{}l@{}}
        \fora{\ke', m, \ver} 
        \ver = \mkvsCOPS(\ke, i) \land {} \\
        \quad (\ke', \WTx(\mkvsCOPS(\ke', m))) \in \ver\projection{4} \\
        \qquad {} \implies m \in \viREPL'(\ke') 
        \end{array}\right)} \texttt{ ---> \cref{lst:send-receive}, \cref{line:receive-wait}} 
    }{%
        \mkvsCOPS, \viewFunCOPS, \thdenv, \prog \toG{}
        \mkvsCOPS, \viewFunCOPS\rmto{\repl}{\viREPL}, \thdenv, \prog
    }
\end{mathpar}

A view \( \vi \) on key-value store \( \mkvsCOPS \) \emph{agrees} 
with another view \( \vi \) on another key-value store \( \mkvsCOPS' \), if and only
\[
 \func{getMax}{\mkvsCOPS, \vi} = \func{getMax}{\mkvsCOPS', \vi'}
\]



%\begin{theorem}
    %For any trace \( \tr \) of COPS with final configuration \( (\mkvsCOPS, \viewFunCOPS) \), 
    %there exists a trace \( \tr' \) with final configuration \( (\mkvsCOPS', \viewFunCOPS') \) such that 
    %each step of the trace \( \tr' \) commits a transaction with strictly greater transaction identifier than any one appearing in the key-value store:
    %\[
        %\begin{array}{@{}l@{}}
        %(\mkvsCOPS_i, \viewFunCOPS_i) 
        %\toG{} (\mkvsCOPS_{i+1}, \viewFunCOPS_{i+1}) 
        %\land \exsts{\txid} \txid \in \mkvsCOPS_{i+1} 
        %\land \txid \notin \mkvsCOPS_{i+1}
        %\implies \fora{\txid' \in \mkvsCOPS_i} \txid > \txid'
        %\end{array}
    %\]
    %and any replica's view from \( \viewFunCOPS \) agrees with its counterpart from  \( \viewFunCOPS' \):
    %\[
        %\fora{i} 
        %\func{getMax}{\mkvsCOPS, \viewFunCOPS(i)} = \func{getMax}{\mkvsCOPS', \viewFunCOPS'(i)}
    %\]
%\end{theorem}
%\begin{proof}
%\end{proof}

%\begin{lemma}
%\[
    %\begin{array}{@{}l@{}}
    %\fora{\repl, \cl, \mkvsCOPS, \mkvsCOPS', \mkvsCOPS'', \viREPL, \viREPL', \viREPL'', \viCL, \viCL', \viCL'', \stk, \stk', \cmd, \cmd'} \\
    %\quad \func{getMax}{\mkvsCOPS, \viREPL} = \func{getMax}{\mkvsCOPS'', \viREPL''} 
    %\land \func{getMax}{\mkvsCOPS, \viCL} = \func{getMax}{\mkvsCOPS'', \viCL''} \\
    %\qquad \repl, \cl \vdash 
    %\mkvsCOPS, \viREPL, \viCL, \stk, \cmd \toT{}
    %\mkvsCOPS', \viREPL', \viCL', \stk', \cmd' \\
    %\qquad \implies 
    %\exsts{\mkvsCOPS''', \viREPL'''}
    %\mkvsCOPS'', \viREPL'', \viCL'', \stk, \cmd \toT{}
    %\mkvsCOPS''', \viREPL''', \viCL', \stk', \cmd' \\
    %\qquad \func{getMax}{\mkvsCOPS', \viREPL'} = \func{getMax}{\mkvsCOPS''', \viREPL'''} 
    %\land \func{getMax}{\mkvsCOPS', \viCL'} = \func{getMax}{\mkvsCOPS''', \viCL'''} \\
    %\end{array}
%\]
%\end{lemma}
%\begin{proof}
%We perform case analysis.
%\begin{itemize}
    %\item \rl{Put}.
    %We have \( \cmd \equiv \ptrans{\pmutate{\ke}{\vx};} \) for some key \( \ke \) and variable \( \vx \).
    %Suppose key-value stores \(  \mkvsCOPS, \mkvsCOPS', \mkvsCOPS'' \), 
    %replica's views \( \viREPL, \viREPL', \viREPL''\) and client's views \( \viCL, \viCL', \viCL''\) such that
    %\[
    %\begin{array}{@{}l@{}}
    %\func{getMax}{\mkvsCOPS, \viREPL} = \func{getMax}{\mkvsCOPS'', \viREPL''} 
    %\land \func{getMax}{\mkvsCOPS, \viCL} = \func{getMax}{\mkvsCOPS'', \viCL''} \\
    %\qquad \repl, \cl \vdash 
    %\mkvsCOPS, \viREPL, \viCL, \stk, \cmd \toT{}
    %\mkvsCOPS', \viREPL', \viCL', \stk', \cmd' \\
    %\end{array}
    %\]
    %By the premiss of the \rl{Put} rule, the new key-value store
    %\[
        %\mkvsCOPS' = \mkvsCOPS\rmto{\ke}{\mkvsCOPS(\ke) \lcat \List{(\ke, \txid, \emptyset, \dep)}}
    %\]
    %where
    %\[
        %\txid = \min\Setcon{%
            %\txidCOPS{\repl}{\cl}{n'}{0}
        %}{%
            %\fora{\ke', i \in \viREPL(\ke'), n} \\
            %\quad \txidCOPS{\stub}{\stub}{n}{\stub} = \WTx(\mkvsCOPS(\ke',i)) \\
            %\qquad {} \implies n' > n 
        %} 
    %\]
    %and the new views of replica and client are
    %\[   
        %\begin{array}{@{}l@{}}
        %\viREPL' = \viREPL\rmto{\ke}{\viREPL(\ke) \uplus \Set{\abs{\mkvsCOPS'(\ke)} - 1}} \\
        %{} \land \viCL' = \viCL\rmto{\ke}{\viREPL(\ke) \uplus \Set{\abs{\mkvsCOPS'(\ke)} - 1}}
        %\end{array}
    %\]
    %Similarly there exists a new \( \mkvsCOPS''' \) by committing a single-write transaction \( \txid' \) and two new views \( \viREPL''' \) and \( \viCL''' \).
    %This means for those key \( \ke' \) that is different from the key \( \ke \) being overwritten,
    %\begin{equation}
        %\label{equ:get-max-match-all-other-key}
        %\begin{array}{@{}l@{}}
            %\func{getMax}{\mkvsCOPS', \viREPL'}(\ke') = \func{getMax}{\mkvsCOPS''', \viREPL'''}(\ke') \\
            %{} \land \func{getMax}{\mkvsCOPS', \viCL'}(\ke') = \func{getMax}{\mkvsCOPS''', \viCL'''}(\ke') 
        %\end{array}
    %\end{equation}
    %Note that the \( \txid \) is greater than any writers \( \txidCOPS{\repl}{\cl}{n'}{0} \) that can be observed by the \( \viREPL \), so is \( \txid' \).
    %That is,
    %\begin{equation}
        %\label{equ:get-max-match-overwritten-key}
        %\begin{array}{@{}l@{}}
            %\func{getMax}{\mkvsCOPS', \viREPL'}(\ke) = \stk(\vx) = \func{getMax}{\mkvsCOPS''', \viREPL'''}(\ke)  \\
            %{} \land \func{getMax}{\mkvsCOPS', \viCL'}(\ke) = \stk(\vx) = \func{getMax}{\mkvsCOPS''', \viCL'''}(\ke) 
        %\end{array}
    %\end{equation}
    %Combine \cref{equ:get-max-match-all-other-key} and \cref{equ:get-max-match-overwritten-key},
    %we have the proof.

    %\item \rl{GetTrans}.
    %Since the views of replica remain unchanged, so we only need to prove that there exists a new key-value store and a new view \( \viCL''' \) such that
    %\[
        %\func{getMax}{\mkvsCOPS', \viCL'}(\ke) = \stk(\vx) = \func{getMax}{\mkvsCOPS''', \viCL'''}(\ke) 
    %\]
    
%\end{itemize}
%\end{proof}

\begin{lemma}
    \label{lem:client-subset-repl}
    The view of a client is subset of the view of the replica that the client interacts with.
\end{lemma}


\begin{lemma}
    Let ignore the dependencies of versions from \( \mkvsCOPS \).
    Given the initial key-value store \( \mkvsCOPS_0 \), initial views \( \viewFunCOPS_0 \) and some programs \( \prog_0 \), for any \( \mkvsCOPS_i \) and \( \viewFunCOPS_i \)  such that: 
    \[
        \mkvsCOPS_0, \viewFunCOPS_0, \thdenv_0, \prog_0 {\toG{}}^* \mkvsCOPS_i, \viewFunCOPS_i, \thdenv_i, \prog_i
    \]
    The key-value store \( \mkvsCOPS_i \) satisfies the \cref{def:mkvs} and any replica or client view \( \vi \) from \( \viewFunCOPS_i \) is a valid view of the key-value store, \ie \( \vi \in \Views(\mkvsCOPS_i) \).
\end{lemma}
\begin{proof}
    We need to prove the  \( \mkvsCOPS_i \) satisfies the well-formed conditions,
    and any view \( \vi_i \Views(\mkvsCOPS_i) \).
    We prove it by introduction on the length \( i \).
    \begin{itemize}
    \item \caseB{\(i = 0\)}
        It holds trivially since each key only has the initial version \( (\val_0,\txid_0,\emptyset, \emptyset) \).
        Since there is only the initial version for each key, it is easy to see that any view \( \vi_0 \) satisfying the well-formed conditions in \cref{def:views}.
    \item \caseI{\(i > 0\)}
        Suppose it holds when \( i \), let consider \( i + 1 \).
        We perform case analysis on the possible next step:
        \begin{itemize}
            \item \rl{Put}
                Assume the client \( \cl \) of a replica \( \repl \) commits a single-write transaction \( \txid \) that installs a new version for key \( \ke \).
                By the premiss of \rl{Put}, the new transaction identifier \( \txid = \txidCOPS{\repl}{\cl}{n'}{0} \) where for some \( n' \) that is greater than any \( n \) from any writers \( \txidCOPS{\stub}{\stub}{n}{\stub} \) that are observable by the replica \( \repl \).
                Since the new transaction \( \txid = \txidCOPS{\repl}{\cl}{n'}{0} \) is a single-write transaction which is always installed at the end of the list associated to \( \ke \), it is sufficient to prove the following:
                \begin{gather}
                    \fora{j} 0 \leq j < \abs{ \mkvsCOPS_i(\ke) } \implies \WTx(\mkvsCOPS_{i}(\ke, m)) \neq \txid \label{equ:write-trans-unique} \\
                    \fora{j, n} \txidCOPS{\repl}{\cl}{n}{\stub} \in \Set{\WTx(\mkvsCOPS_{i}(\ke,j))} \cup \RTx(\mkvsCOPS_{i}(\ke, j)) \implies n < n' \label{equ:replica-time-monotonic-inc}
                \end{gather}
                By \cref{lem:repl-observe-own}, we know that for any version written \( \ver = \mkvsCOPS_i(\ke, j) \) by the same replica \( \txidCOPS{\repl}{\stub}{\stub}{\stub} = \WTx(\ver) \), such version is included in the replica's view \( j \in \viREPL(\ke) \).
                It implies that first the new transaction identifier is unique \cref{equ:write-trans-unique} and second it is greater than any transactions in the form of \( \txidCOPS{\repl}{\cl}{\stub}{\stub} \) \cref{equ:replica-time-monotonic-inc}.
                Thus the new key-value store \( \mkvsCOPS_{i+1} \) satisfies the well-formed conditions.
                Now let consider the views, especially the views of the replica \( \viREPL' \) and the client \( \viCL' \).
                Since that views \( \vi' \) from different replicas or clients remain unchanged, by \ih they satisfy \( \vi' \in \Views(\mkvsCOPS_{i+1}) \).
                The new view for replica \( \viREPL' = \viREPL\rmto{\ke}{\abs{\mkvsCOPS_{i+1}(\ke)} - 1} \)
                where \( \viREPL \) is the replica's view before updating and the writer of the last version of \( \ke \) is \( \txid \).
                Because \( \txid \) is a single-write transaction, so the new view \( \viREPL' \) still satisfies the atomic read.
                For similar reason, the new view for client \( \viCL' \) till satisfies atomic read.
                Therefore we have \( \viREPL', \viCL' \in \Views(\mkvsCOPS_{i+1}) \).
            \item \rl{GetTrans}

        \end{itemize}
    \end{itemize}
\end{proof}

\begin{lemma}
    \label{lem:repl-observe-own}
    A replica observes all its own transactions.
\end{lemma}

\section{Soundness}

For technical reasons, it will be convenient to adopt a reduction strategy for inferring kv-stores induced by an 
execution test: such an execution strategy require that clients only commit transactions with non-empty fingerprints, 
and a client updates its view only immediately before committing a transaction. 
The next proposition states that all kv-stores induced by an execution test $\ET$ can be 
obtained via a sequence of reductions that adhere to the reduction strategy outlined above. 
\begin{definition}
Let $\ET$ be an execution test. The $\ET$-trace
\[
\conf_0 \xrightarrowtriangle{\alpha_0}_{\ET} \conf_1 \xrightarrowtriangle{\alpha_1}_{\ET} \cdots \xrightarrowtriangle{\alpha_{2n}}_{\ET} \conf_{2n + 1}
\]
is in \emph{normal form} if \textbf{(i)} $\conf_0$ is initial, and 
\textbf{(ii)} $\forall i=0,\cdots, n$ there exists a client $\cl_i$ and set of operations $\opset_{i}$ such that 
$\alpha_{2 \cdot i} = (\cl_{i}, \varepsilon)$, and $\alpha_{2 \cdot i + 1}$ is defined and equal to $(\cl_{i}, \opset_{i})$ where \( \f_i \neq \emptyset \).
\end{definition}

\begin{proposition}
\label{prop:et.normalform}
Let $\ET$ be an execution test, and suppose that $\hh \in \CMs(\ET)$. Then there exists a $\ET$-trace  
\[
(\hh_0, \viewFun_0) \xrightarrowtriangle{\stub}_{\ET} \cdots \xrightarrowtriangle{\stub}_{\ET} (\hh_n, \viewFun_{n})
\]
that is in normal form, and such that $\hh_{n} = \hh$.
\end{proposition}
\begin{proof}
    See \cref{sec:normal-form-exist}.
\end{proof}


The \cref{prop:updatekv.comm} captures that
if two different clients $\cl_1$ and $\cl_2$ commit transactions 
whose fingerprints $\opset_1$ and $\opset_2$ do not contain a write 
to the same key, then the order in which the updates are executed is 
not relevant. 
\begin{proposition}
\label{prop:updatekv.comm}
\label{prop:swap-update}
Let $\hh \in \HisHeaps$, $\vi_1, \vi_2 \in \Views(\hh)$ and let $\cl_1, \cl_2 \in \Clients$ 
be such that $\cl_1 \neq \cl_2$. 
Let also $\opset_1, \opset_2 \in \powerset{\Ops}$ be such that 
whenever $(\otW, \ke, \_) \in \opset_1$ for some key $\ke$, then 
$(\otW, \ke, \val) \notin \opset_2$ for all $\val \in \Val$. Then 
\[
\begin{array}{l}
\{ \updateKV(\hh_1, \vi_2, \cl_2, \opset_2) \mid \hh_1 \in \updateKV(\hh, \vi_1, \cl_1, \opset_1)\} = \\
\{ \updateKV(\hh_2, \vi_1, \cl_1, \opset_1) \mid \hh_2 \in \updateKV(\hh, \vi_2, \cl_2, \opset_2)\}\\
\end{array}
\]
\end{proposition}
\begin{proof}
    See \cref{sec:comm-updatekv}.
\end{proof}

A desirable property that one would request from execution 
test is compositionality: the consistency model induced by 
a composite execution test can be recovered from the consistency 
models generated by each execution test: that is, 
\[ 
\forall \ET_1, \ET_2. \CMs(\ET_1 \cap \ET_2) = \CMs(\ET_1) \cap \CMs(\ET_2).
\]
Unfortunately, this property is not satisfied by execution tests in their 
most general setting, as the following example shows: 
\begin{example}
\label{ex:noncompositional.et}
Define the following terms: 
\[
\begin{array}{lcl}
\hh_0 &=& [\ke_1 \mapsto (\val_0, \txid_0, \emptyset) , \ke_2 \mapsto (\val_0, \txid_0, \emptyset)]\\
\hh_1 &=& \big[\ke_1 \mapsto \big( (\val_0, \txid_0, \emptyset) \lcat (\val_2, \txid_{\cl_1}^{1}, \emptyset)\big) , \ke_2 \mapsto (\val_0, \txid_0, \emptyset) \big]\\
\hh_2 &=& \big[\ke_1, \mapsto (\val_0, \txid_0, \emptyset), \ke_2 \mapsto \big( (\val_0, \txid_0, \emptyset) \lcat (\val_2, \txid_{\cl_2}^{1}, \emptyset) \big) \big]\\
\hh_3 &=& \big[\ke_1 \mapsto \big( (\val_0, \txid_0, \emptyset) \lcat (\val_2, \txid_{\cl_1}^{1}, \emptyset)\big), 
                         \ke_2 \mapsto \big( (\val_0, \txid_0, \emptyset) \lcat (\val_2, \txid_{\cl_2}^{1}, \emptyset) \big) \big]\\
&&\\
\vi_0 &=& [\ke_1 \mapsto \{0\}, \ke_2 \mapsto \{0\}]\\
\viewFun_0 &=& [\cl_1 \mapsto \vi_0, \cl_2 \mapsto \vi_0]\\
&&\\
\ET_1 &\vdash& (\hh_0, \vi_0) \triangleright \{(\otW, \ke_1, \val_1)\} : \vi_0\\
\ET_1 &\vdash& (\hh_1, \vi_0) \triangleright \{(\otW, \ke_2, \val_2)\} : \vi_0\\
&&\\
\ET_2 &\vdash& (\hh_0, \vi_0) \triangleright \{(\otW, \ke_2, \val_2)\} : \vi_0\\
\ET_2 &\vdash& (\hh_2, \vi_0) \triangleright \{(\otW, \ke_1, \val_1)\} : \vi_0.
\end{array}
\]
There are no further constraints on $\ET_1, \ET_2$.
For $\ET_1$ and $\ET_2$, we have that 
\[
\begin{array}{l}
(\hh_0, \viewFun_0) \xrightarrowtriangle{(\cl_1, \{(\otW, \ke_1, \val_1)\})}_{\ET_1} 
(\hh_1, \viewFun_0) \xrightarrowtriangle{(\cl_2, \{(\otW, \ke_2, \val_2)\})}_{\ET_1} (\hh_3, \viewFun_0), \\
(\hh_0, \viewFun_0) \xrightarrowtriangle{(\cl_2, \{(\otW, \ke_2, \val_2)\})}_{\ET_2} 
(\hh_2, \viewFun_0) \xrightarrowtriangle{(\cl_1, \{(\otW, \ke_1, \val_1)\})}_{\ET_2} (\hh_3, \viewFun_0).\\
\end{array}
\] 
Therefore, we have that $\hh_3 \in \CMs(\ET_1) \cap \CMs(\ET_2)$. On the other hand, it is immediate 
to observe that $\ET_1 \cap \ET_2 = \emptyset$, and therefore $\hh_3 \notin \CMs(\ET_1 \cap \ET_2)$.
\end{example}
The reason why compositionality fails, for the execution tests of \cref{ex:noncompositional.et}, 
is that both the execution tests $\ET_1, \ET_2$ require that the fingerprints 
$\{(\otW, \ke_1, \_)\}, \{(\otW, \ke_2, \_)\}$ commit in different order: in $\ET_1$, the write to $\ke_1$ must commit 
before the write to $\ke_2$, and vice versa for $\ET_2$. On the other hand, 
because the two fingerprints above do not write to the same key, 
the order in which they are committed should not be relevant: by changing the order 
in which different clients commit such fingerprints to a kv-store, the result stays the same. 
\begin{definition}
Two triples $(\cl_1, \opset_1)$ and $(\cl_2, \opset_2)$ are 
conflicting if either $\cl_1 = \cl_2$, or there exists a key $\ke$ such that 
$(\otW, \ke, \_) \in \opset_1, (\otW, \ke, \_) \in \opset_2$. 

An execution test is $\ET$ is \emph{commutative} if, whenever $(\cl_1, \vi_1, \opset_1)$, 
$(\cl_2, \vi_2, \opset_2)$ are non-conflicting, and $\vi_1, \vi_2 \in \Views(\hh_0)$,  
then for any $\hh_0, \hh', \viewFun, \viewFun'$ we have that 
\[
\begin{array}{lr}
(\hh_0, \viewFun) \xrightarrowtriangle{(\cl_1, \opset_1)}
\_ \xrightarrowtriangle{(\cl_2, \opset_2)}_{\ET} (\hh', \viewFun') &\implies \\
(\hh_0, \viewFun) \xrightarrowtriangle{(\cl_2, \opset_2)}_{\ET} 
\_ \xrightarrowtriangle{(\cl_1, \opset_1)}_{\ET} (\hh', \viewFun')
\end{array}
\]
\end{definition}
\ac{Note that the views are not part of actions anymore. Furthermore, the definition 
of $\ET$-reduction has been changed, so that no view shifts can be made prior to 
committing a transaction, in the reductions above.}

Requiring execution tests to be commutative is a necessary step for ensuring 
that the specification of consistency models are compositional. However, it 
is not sufficient. The next example shows how compositionality fails 
for commutative execution tests. 

\begin{example}
\label{ex:noblindwrites}
For any $n \in \Nat$, let $[n] = \{0,\cdots, n\}$.
Consider the execution tests $\ET_1, \ET_2$ defined below: 
\[
\begin{array}{lcl}
\ET_1 \vdash (\hh, \vi) \triangleright \opset : \vi' &\iff& 
\forall \ke.\;(\otW, \ke, \_) \in \opset \implies \vi(\ke) = \vi'(\ke) = [0]\\
\ET_2 \vdash (\hh, \vi) \triangleright \opset : \vi'(\ke) &\iff& 
\forall \ke. \;(\otW, \ke, \_ ) \in \opset \implies \vi(\ke) = [ \lvert \hh(\ke) \rvert - 1] \wedge \vi'(\ke) = [\lvert \hh(\ke) \rvert ] \\
\end{array}
\]
It is immediate to observe that both $\ET_1$ and $\ET_2$ are commutative. However, 
consider the kv-store $\hh_2 = [\ke \mapsto (\val_0, \txid_0, \emptyset) \lcat (\val_1, \txid_{\cl}^1, \emptyset) \lcat (\val_2, \txid_{\cl}^2, \emptyset)]$. 
We have that $\hh \in \CMs(\ET_1)$ and $\hh \in \CMs(\ET_2)$.
Let in fact $\hh_1 = [\ke \mapsto (\val_0, \txid_0, \emptyset) \lcat (\val_1, \txid_{\cl}^1, \emptyset)]$, $\vi_{i} = [\ke \mapsto [i] ]$.
\ac{Seriously, square brackets are being used everywhere (though all of this is standard notation. Maybe $\langle n \rangle$ for $\{0,\cdots, n\}$ is 
a better notation?
\sx{  \( \ke \mapsto \langle n \rangle \) is cool. }
}
We have the following sequences of reductions: 
\[
\begin{array}{l}
(\hh_0, \vi_0) \xrightarrowtriangle{(\cl, \{(\otW, \ke, \val_1)\})}_{\ET_1} 
(\hh_1, \vi_0) \xrightarrowtriangle{(\cl,\{(\otW, \ke, \val_2)\})}_{\ET_1} (\hh_2, \vi_0)\\
(\hh_0, \vi_0) \xrightarrow{(\cl, \{(\otW, \ke, \val_1)\})}_{\ET_2} (\hh_1, \vi_1) \xrightarrow{(\cl, \{(\otW, \ke, \val_2)\})}_{\ET_2} 
(\hh_2, \vi_2)
\end{array}
\]
On the other hand, we can observe that $\hh_2 \notin \CMs(\ET_1 \cap \ET_2)$. $\ET_1$ allows a client to 
commit a transaction if its view only includes the initial version of each key it writes. $\ET_2$ allows a client 
to commit a transaction when its view include all the versions for each key it writes. In $\ET_1 \cap \ET_2$ 
a client can commit a transaction only if the initial version of each key it writes is also the only version in the kv-store: 
as a result, $\CMs(\ET_1 \cap \ET_2)$ never contains a  kv-stores $\hh$ such that $\hh(\ke) > 1$ for some key $\ke$; 
in particular, $\hh_2 \notin \CMs(\ET_1 \cap \ET_2)$.
\end{example}
\ac{Two possible reasons why compositionality fails: because of blind writes, or because the test $\ET_1$ hinders progress, 
i.e. it is not possible to replace a view with a more up-to-date one to enable progress. We must choose which assumption 
we make on the consistency model.}

One reason why compositionality fails in \cref{ex:noblindwrites} is that the execution tests $\ET_1$ and $\ET_2$ do not contain 
any information about the the views that client $\cl$ used to commit the transactions $\txid_{\cl}^1, \txid_{\cl}^2$. 
To solve this problem we adopt the \emph{no blind writes} assumption, that requires that a client never commits 
a transaction that writes a key, without reading such a key beforehand. Many implementations of consistency models 
in distributed key-value stores respect the no blind writes assumption. 

\begin{definition}
\label{def:noblidwrites}
An execution test $\ET$ has \emph{no blind writes} if, whenever $\ET \vdash (\hh, \vi) \triangleright \opset \cup \{(\otW, \ke, \_)\} : \vi'$, 
then $(\otR, \ke, \_) \in \opset$.
\end{definition}

\begin{definition}
\label{def:et-minimum-footprint}
An execution test $\ET$ has \emph{minimum footprints} if for any \( \hh, \vi, \vi',\vi'', \f \),
\[
\begin{array}{@{}l@{}}
    ( \fora{ \ke} (\stub, \ke, \stub) \in \f \implies \vi(\ke) = \vi'(\ke) ) \land {} \\
    \quad \ET \vdash (\hh, \vi) \triangleright \opset : \vi'' \implies \ET \vdash (\hh, \vi') \triangleright \opset : \vi''
\end{array}
\]
\end{definition}

\begin{definition}
\label{def:et-continuous-postview}
An execution test $\ET$ has \emph{continuous post-views} if for any \( \hh, \vi, \vi',\vi'', \f \),
\[
\begin{array}{@{}l@{}}
    \quad \ET \vdash (\hh, \vi) \triangleright \opset : \vi' \land \vi' \sqsubseteq \vi'' \implies \ET \vdash (\hh, \vi) \triangleright \opset : \vi''
\end{array}
\]
\end{definition}

\begin{theorem}                                                                            
\label{thm:et-comm}                          
Let $\ET_1, \ET_2$ be two execution tests has no blind writes, minimum footprints and continuous post-views.
If $\ET_1$ is commutative, 
then $\CMs(\ET_1 \cap \ET_2) = \CMs(\ET_1) \cap \CMs(\ET_2)$. 
Furthermore, if $\ET_1, \ET_2$ are commutative, then $\ET_1 \cap \ET_2$ 
is commutative.
\end{theorem}
\begin{proof}
    See \cref{sec:et-comm}.
\end{proof}


\subsection{Proof of \cref{prop:updatekv.comm}}
\label{sec:comm-updatekv}

\begin{lemma}
\label{lem:updatekv.explicit}
Let $\hh$ be a kv-store, $\vi \in \Views(\hh)$, $\txid \in \TxID$ and $\opset \in \powerset{\Ops}$. 
Let also $\ke \in \Keys$. Then
\begin{enumerate}
\item\label{item:updatekv.explicit.none} $(\forall \val.\;(\otR, \ke, \val) \notin \opset \wedge (\otW, \ke, \val)) \notin \opset \implies \updateKV(\hh, \vi, \txid, \opset)(\ke) = \hh(\ke)$, 
\item\label{item:updatekv.explicit.rd} $(\otR, \ke, \_) \in \opset \wedge \forall \val.\;( (\otW, \ke, \val) \notin \opset) \implies \updateKV(\hh, \vi, \txid, \opset)(\ke) = 
\text{ let } (\val', \txid', \T') = \hh(\ke, \max(\vi(\ke))) \text{ in } \hh(\ke)\rmto{\max(\vi(\ke))}{(\val', \txid', \T' \cup \{\txid\})}$, 
\item\label{item:updatekv.explicit.wr} $\forall \val.\; (\forall \val'.\;(\otR, \ke, \val) \notin \opset) \wedge (\otW, \ke, \val) \in \opset \implies \updateKV(\hh, \vi, \txid, \opset)(\ke) = 
\hh(\ke) \lcat \List{(\val, \txid, \emptyset)}$, 
\item\label{item:updatekv.explicit.rdwr} $\forall \val.\;(\otR, \ke, \_) \in \opset \wedge (\otW, \ke, \val) \in \opset \implies \updateKV(\hh,\vi,\txid,\opset)(\ke) = 
\mathtt{ let } (\val', \txid', \T') = \hh(\ke, \max(\vi(\ke))) \in \hh(\ke)\rmto{\max(\vi(\ke))}{(\val', \txid', \T' \cup \{\txid\})} \lcat \List{(\val, \txid, \emptyset)}.$
\end{enumerate}
\end{lemma}

\begin{proof}
All the four statements are proved by induction on $\opset$, by keeping the variable $\hh$ universally quantified in the inductive hypothesis. 
Statement \eqref{item:updatekv.explicit.rd} and \eqref{item:updatekv.explicit.wr} requires 
proving \eqref{item:updatekv.explicit.none} first, while Statement \eqref{item:updatekv.explicit.rdwr} requires proving all the other statements. 
Fix then an arbitrary $\ke \in \Keys$.
\begin{enumerate}
	\item 
	Suppose that for any $\val$, $(\otR, \ke, \val) \notin \opset$ and $(\otW, \ke, \val)) \notin \opset$. We prove that $\updateKV(\hh, \vi, \txid, \opset)(\ke) = 
	\hh(\ke)$.
	\begin{itemize}
		\item Base case: $\opset = \emptyset$; in this case we have that 
		\[
		\updateKV(\hh, \vi, \txid, \emptyset)(\ke) \stackrel{\eqref{eq:updatekv}}{=} \hh(\ke).
		\]
		
		\item Suppose that $\opset = \opset' \uplus \{(\otR, \ke', \val')\}$ for some $\ke', \val'$. Because we are assuming that 
		$(\otR, \ke, \val) \notin \opset$ for any $\val \in \Val$, then it must be the case that 
		\begin{equation}
		\label{eq:updatekv.explicit.none.keneqkepRD}
		\ke \neq \ke'.
		\end{equation}
		Also, we have that $(\otR,\ke, \val) \notin \opset'$ and $(\otW, \ke, \val) \notin \opset$ for any $\val \in \Val$. 
		By inductive hypothesis we can assume 
		\begin{equation}
		\forall \hh'.\;\updateKV(\hh', \vi, \txid, \opset')(\ke) = \hh'(\ke).
		\label{eq:updatekv.explicit.none.IHrd}
		\end{equation} 
		Therefore we have 
		\[
		\begin{array}{lr}
		\updateKV(\hh, \vi, \txid, \opset)(\ke) = \updateKV(\hh, \vi, \txid, \opset' \uplus \{(\otR, \ke', \val')\})(\ke) &\stackrel{\eqref{eq:updatekv}}{=}\\ &\\
		\text{let } (\val', \txid', \T') = \hh(\ke', \max(\vi(\ke))) \text{ in } & \\ 
		\;\;\;\;\updateKV(\hh\rmto{\ke'}{\hh(\ke')\rmto{\max(\vi(\ke'))}{(\val', \txid', \T' \cup \{\txid\})}}, \vi, \txid, \opset')(\ke) &			\stackrel{\eqref{eq:updatekv.explicit.none.IHrd}}{=}\\ &\\
		\text{let } (\val', \txid', \T') = \hh(\ke', \max(\vi(\ke'))) \text{ in } \hh\rmto{\ke'}{\hh(\ke')\rmto{\max(\vi(\ke'))}{(\val', \txid', \T' \{\txid\}}}(\ke) &\stackrel{\eqref{eq:updatekv.explicit.none.keneqkepRD}}{=}\\
		\big( \text{let } (\val', \txid', \T') = \hh(\ke', \max(\vi(\ke'))) \text{ in } \hh(\ke) \big) = \hh(\ke)
		\end{array}
		\]
		
		\item Suppose that $\opset = \opset' \uplus \{(\otW, \ke', \val')\}$ for some $\val' \in \Val$. Then it must be the 
		case that 
		\begin{equation}
		\label{eq:updatekv.explicit.none.keneqkepWR}
		\ke \neq \ke'.
		\end{equation}
		Also, we have that $(\otR,\ke, \val) \notin \opset'$ and $(\otW, \ke, \val) \notin \opset$ for any $\val \in \Val$. 
		By inductive hypothesis we can assume 
		\begin{equation}
		\forall \hh'.\;\updateKV(\hh', \vi, \txid, \opset')(\ke) = \hh'(\ke).
		\label{eq:updatekv.explicit.none.IHwr}
		\end{equation}
		Therefore we have 
		\[
		\begin{array}{lr}
		\updateKV(\hh, \ke, \txid, \opset)(\ke) = \updateKV(\hh, \ke, \txid, \opset \uplus \{(\otW, \ke', \val')\})(\ke) &\stackrel{\eqref{eq:updatekv}}{=}\\
		\updateKV(\hh\rmto{\ke'}{\hh(\ke')\lcat \List{(\val', \txid, \emptyset)}}, vi, \txid,\opset)(\ke) &\stackrel{\eqref{eq:updatekv.explicit.none.IHwr}}{=}\\
		\hh\rmto{\ke'}{\hh(\ke') \lcat \List{(\val', \txid, \emptyset)}}, \vi, \txid, \opset)(\ke) & \stackrel{\eqref{eq:updatekv.explicit.none.keneqkepWR}}{=}\\
		\hh(\ke)
		\end{array}
		\]
	\end{itemize}

	\item Suppose now that, $(\otR, \ke, \_) \in \opset$, and $(\otW, \ke, \val) \notin \opset$ for all $\val \in \Val$. 
	Let $(\val, \txid', \T) = \hh(\ke, \max(\vi(\ke)))$. We prove that $\updateKV(\hh, \vi, \txid, \opset)(\ke) = 
	\hh(\ke)\rmto{\vi(\ke)}{(\val, \txid', \T \cup \{\txid\})}$.
		\begin{itemize}
		\item Base case: $\opset = \emptyset$; this case is vacuous, as $(\otR, \ke, \val) \notin \opset$ for all $\val \in \Val$, 
		against the assumption that $(\otR, \ke, \_) \in \opset$. 

		\item Suppose that $\opset = \opset' \cup \{(\otR, \ke', \_)\}$ for some $\ke'$. We have two possible cases: 
			\begin{enumerate}
			\item $\ke = \ke'$, in which case we know that $(\otR, \ke, \val') \notin \opset'$ for all $\val' \in \Val$ because of 
			the assumptions that we make on the structure of $\opset$. By \cref{lem:updatekv.explicit}\eqref{item:updatekv.explicit.none} we have that
			\begin{equation}
			\forall \hh'.\; \updateKV(\hh', \vi, \txid, \opset')(\ke) = \hh'(\ke).
			\label{eq:updatekv.explicit.rd.applynone}
			\end{equation}
			In this case we have that 
			\[
			\begin{array}{lr}
			\updateKV(\hh, \vi, \txid, \opset)(\ke) = \updateKV(\hh,\vi, \txid, \opset' \uplus \{(\otR, \ke, \val')\})(\ke) &\stackrel{\eqref{eq:updatekv}}{=}\\
			\updateKV(\hh\rmto{\ke}{\hh(\ke)\rmto{\max(\vi(\ke))}{(\val, \txid', \T \cup \{\txid\})}}, \vi, \txid, \opset')(\ke) &\stackrel{\eqref{eq:updatekv.explicit.rd.applynone}}{=} \\
			\hh\rmto{\ke}{\hh(\ke)\rmto{\max(\vi(\ke))}{(\val, \txid', \T \cup \{\txid\})}}(\ke) &=\\ 
			\hh(\ke)\rmto{\max(\vi(\ke))}{(\val, \txid', \T \cup \{\txid\})}.
			\end{array}
			\]

			\item \begin{equation}
			\ke \neq \ke'.
			\label{eq:updatekv.explicit.rd.keneqkepRD}
			\end{equation} 
			In this case we know that because $(\otR, \ke, \_) \in \opset$, then 
			it must be $(\otR, \ke, \_) \in \opset'$. We also know that $\forall \val.(\otW, \ke, \val) \notin \opset$. 
			By the inductive hypothesis, we have that 
			\begin{equation}
			\label{eq:updatekv.explicit.rd.IHrd}
			\forall \hh'.\; \updateKV(\hh', \vi, \txid, \opset')(\ke) = \hh'(\ke)\rmto{\max(\vi(\ke))}{(\val, \txid', \T \cup \{(\txid)\})}.
			\end{equation}
			In this case we have 
%			\[
%			\begin{array}{lr}
%			\updateKV(\hh, \vi, \txid, \opset)(\ke) = \updateKV(\hh, \vi, \txid, \opset' \uplus \{(\otR, \ke', \_)\})(\ke) &\stackrel{\eqref{eq:updatekv}}{=}\\
%			\text{let } (\val'', \txid'', \T'') = \hh(\ke', \max(\vi(\ke'))) \text{ in } \updateKV(\hh\rmto{\ke'}{\hh(\ke')\rmto{\max(\vi(\ke'))}{(\val'', \txid'', \T'' \cup \{\txid\})}}, \vi, \txid, \opset)(\ke) 			&
%			\stackrel{\eqref{eq:updatekv.explicit.rd.IHrd}}{=} \\
%			\text{let } (\val'', \txid'', \T'') = \hh(\ke', \max(\vi(\ke'))) \text{ in } &\\
%			\big(\hh\rmto{\ke'}{(\hh(\ke')\rmto{\max(\vi(\ke'))}{(\val'', \txid'', \T'' \cup \{\txid\})}}(\ke)\big) 
%			\rmto{\max(\vi(\ke))}{(\val, \txid', \T' \cup \{(\txid)\})} &\stackrel{\eqref{eq:updatekv.explicit.rd.keneqkepRD}}{=}\\
%			\hh(\ke)\rmto{\max(\vi(\ke))}{(\val, \txid', \T' \cup \{(\txid)\}}
%			\end{array}
%			\]
			\[
			\begin{array}{lr}
			\updateKV(\hh, \vi, \txid, \opset)(\ke) = \updateKV(\hh, \vi, \txid, \opset' \uplus \{(\otR, \ke', \_)\})(\ke) &\stackrel{\eqref{eq:updatekv}}{=}\\
			\updateKV(\hh\rmto{\ke'}{\_}, \vi, \txid, \opset)(\ke) 			&
			\stackrel{\eqref{eq:updatekv.explicit.rd.IHrd}}{=} \\
			\big(\hh\rmto{\ke'}{\_}(\ke)\big) 
			\rmto{\max(\vi(\ke))}{(\val, \txid', \T' \cup \{(\txid)\})} &\stackrel{\eqref{eq:updatekv.explicit.rd.keneqkepRD}}{=}\\
			\hh(\ke)\rmto{\max(\vi(\ke))}{(\val, \txid', \T' \cup \{(\txid)\}}
			\end{array}
			\]
		\end{enumerate}

		\item $\opset = \opset' \uplus \{(\otW, \ke', \val')\}$ for some $\val' \in \Val$. Because $(\otW, \ke, \val) \notin \opset$ 
		for any $\val \in \Val$, it must be the case that 
		\begin{equation}
		\ke \neq \ke'
		\label{eq:updatekv.explicit.rd.keneqkepWR}
		\end{equation}
		Because $(\otR, \ke, \_) \in \opset$, it must also be the case that $(\otR, \ke, \_) \in \opset'$. By the inductive hypothesis, 
		we have that 
		\begin{equation}
		\forall \hh'.\;\updateKV(\hh', \vi,\txid, \opset')(\ke) = \hh(\ke)\rmto{\max(\vi(\ke))}{(\val, \txid', \T \cup \{(\txid)\})}
		\label{eq:updatekv.explicit.rd.IHwr}
		\end{equation}
		It follows that 
%		\[
%		\begin{array}{lr}
%		\updateKV(\hh, \vi, \txid, \opset)(\ke) = \updateKV(\hh, \vi, \txid, \opset' \uplus \{(\otW, \ke', \val')\})(\ke) &\stackrel{\eqref{eq:updatekv}}{=}\\
%		\updateKV(\hh\rmto{\ke'}{\hh(\ke') \lcat \List{(\val', \txid, \emptyset)}}, \vi, \txid, \opset')(\ke) &\stackrel{\eqref{eq:updatekv.explicit.rd.IHwr}}{=}\\
%		\hh(\rmto{\ke'}{\hh(\ke') \lcat \List{(\val', \txid, \emptyset)}}(\ke)\rmto{\vi(\ke)}{(\val, \txid', \T \cup \{\txid\})} &
%		\stackrel{\eqref{eq:updatekv.explicit.rd.keneqkepWR}}{=}\\
%		\hh(\ke)\rmto{\vi(\ke)}{(\val, \txid', \T \cup \{\txid\})}
%		\end{array}
%		\]
		\[
		\begin{array}{lr}
		\updateKV(\hh, \vi, \txid, \opset)(\ke) = \updateKV(\hh, \vi, \txid, \opset' \uplus \{(\otW, \ke', \val')\})(\ke) &\stackrel{\eqref{eq:updatekv}}{=}\\
		\updateKV(\hh\rmto{\ke'}{\_}, \vi, \txid, \opset')(\ke) &\stackrel{\eqref{eq:updatekv.explicit.rd.IHwr}}{=}\\
		\hh(\rmto{\ke'}{\_}(\ke)\rmto{\max(\vi(\ke))}{(\val, \txid', \T \cup \{\txid\})} &
		\stackrel{\eqref{eq:updatekv.explicit.rd.keneqkepWR}}{=}\\
		\hh(\ke)\rmto{\max(\vi(\ke))}{(\val, \txid', \T \cup \{\txid\})}
		\end{array}
		\]
	\end{itemize}
	
	\item Suppose that $(\otW, \ke, \val) \in \opset$ for some $\val \in \Val$, and 
	$(\otR, \ke, \val') \notin \opset$ for any $\val' \in \Val$. We prove that 
	$\updateKV(\hh, \vi, \txid, \opset)(\ke) = \hh(\ke) \lcat \List{(\val, \txid, \emptyset)}$. 
		\begin{itemize}
		\item Base case: $\opset = \emptyset$. This case is vacuous, as we are assuming 
		that $(\otW, \ke, \val) \in \opset$.
		\item Suppose that $\opset = \opset' \uplus \{(\otR, \ke', \_)\}$ for some 
		$\ke'$. Note that, because we are assuming that $\{(\otR, \ke, \val')\} \notin \opset$ 
		for all $\val' \in \Val$, then it must be the case that 
		\begin{equation}
		\ke \neq \ke'.
		\label{eq:updatekv.explicit.wr.kenqkepRD}
		\end{equation}	
		We also have that $\{(\otR, \ke, \val')\} \notin \opset'$ for all $\val' \in \Val$, and 
		$(\otW, \ke, \val) \in \opset'$. By the inductive hypothesis we have that 
		\begin{equation}
		\forall \hh'.\; \updateKV(\hh', \vi, \txid, \opset')(\ke) = \hh'(\ke) \lcat \List{(\val, \txid, \emptyset)}.
		\label{eq:updatekv.explicit.wr.IHrd}
		\end{equation}
		Therefore, we have that 
		\[
		\begin{array}{lr}
		\updateKV(\hh, \vi, \txid, \opset)(\ke) = \updateKV(\hh, \vi, \txid, \opset' \uplus \{(\otR, \ke', \_)\})(\ke) &\stackrel{\eqref{eq:updatekv}}{=}\\
		\updateKV(\hh\rmto{\ke'}{\_}, \vi ,\txid, \opset')(\ke) \stackrel{\eqref{eq:updatekv.explicit.wr.IHrd}}{=} \hh\rmto{\ke'}{\_}(\ke) \lcat \List{(\val, \txid, \emptyset)} 
		& \stackrel{\eqref{eq:updatekv.explicit.wr.kenqkepRD}}{=} \\
		\hh(\ke) \lcat \List{(\val, \txid, \emptyset)}
		\end{array}
		\]
		
		\item Suppose that $\opset = \opset' \uplus \{(\otW, \ke', \val')\}$ 
		for some $\ke'$. We distinguish two possible cases:
			\begin{enumerate}
			\item $\ke = \ke'$. In this case the structure of $\opset$ also imposes that $\val = \val'$, 
			and $(\otW, \ke, \val'') \notin \opset'$ for any $\val'' \in \Val$. Furthermore, we have 
			that $(\otR, \ke, \val'') \notin \opset'$ for any $\val'' \in \Val$. 
			By \cref{lem:updatekv.explicit}\eqref{item:updatekv.explicit.none}, we have that 
			\begin{equation}
			\forall \hh'.\updateKV(\hh', \vi, \txid, \opset')(\ke) = \hh(\ke)
			\label{eq:updatekv.explicit.wr.applynone}
			\end{equation}
			from which it follows 
			\[
			\begin{array}{lr}
			\updateKV(\hh, \vi, \txid, \opset)(\ke) = 
			\updateKV(\hh, \vi, \txid, \opset' \uplus \{(\otW, \ke', \val')\})(\ke) &=\\ 
			\updateKV(\hh, \vi, \txid, \opset' \uplus \{(\otW, \ke, \val)\})(\ke) &\stackrel{\eqref{eq:updatekv}}{=}\\
			\updateKV(\hh\rmto{\ke}{\hh(\ke) \lcat \List{(\val, \txid, \emptyset)}}, \vi, \txid, \opset')(\ke) &\stackrel{\eqref{eq:updatekv.explicit.wr.applynone}}{=}\\
			\hh\rmto{\ke}{\hh(\ke) \lcat \List{(\val, \txid, \emptyset)}}(\ke) = \hh(\ke) \lcat \List{(\val, \txid, \emptyset)}
			\end{array}
			\]
			
			\item 
			\begin{equation}
			\ke \neq \ke'
			\label{eq:updatekv.explicit.wr.keneqkepWR}
			\end{equation}
			In this case we have that, because $(\otW, \ke, \val) \in \opset$, then it must 
			be $(\otW, \ke, \val) \in \opset'$. Furthermore, we also have that $(\otR, \ke, \val'') \notin \opset'$ 
			for any $\val'' \in \Val$. By the inductive hypothesis, we have that 
			\begin{equation}
			\forall \hh'.\; \updateKV(\hh', \vi, \txid, \opset')(\ke) = \hh(\ke) \lcat \List{(\val, \txid, \emptyset)}
			\label{eq:updatekv.explicit.wr.IHwr}
			\end{equation}
			from which it follows 
%			\[
%			\begin{array}{lr}
%			\updateKV(\hh, \vi, \txid, \opset)(\ke) = \updateKV(\hh, \vi, \txid, \opset' \uplus \{(\otW,\ke', \val')\}) &\stackrel{\eqref{eq:updatekv}}{=}\\
%			\updateKV(\hh\rmto{\ke'}{\hh(\ke') \lcat \List{(\val', \txid, \emptyset)}}, \vi, \txid, \opset)(\ke) &\stackrel{\eqref{eq:updatekv.explicit.wr.IHwr}}{=}\\
%			\hh\rmto{\ke'}{\_}(\ke) \lcat \List{(\val, \txid, \emptyset)} \stackrel{\eqref{eq:updatekv.explicit.wr.keneqkepWR}}{=} \hh(\ke) \lcat \List{(\val, \txid, \emptyset)}
%			\end{array}
%			\]
			\[
			\begin{array}{lr}
			\updateKV(\hh, \vi, \txid, \opset)(\ke) = \updateKV(\hh, \vi, \txid, \opset' \uplus \{(\otW,\ke', \val')\}) &\stackrel{\eqref{eq:updatekv}}{=}\\
			\updateKV(\hh\rmto{\ke'}{\_}, \vi, \txid, \opset)(\ke) &\stackrel{\eqref{eq:updatekv.explicit.wr.IHwr}}{=}\\
			\hh\rmto{\ke'}{\_}(\ke) \lcat \List{(\val, \txid, \emptyset)} \stackrel{\eqref{eq:updatekv.explicit.wr.keneqkepWR}}{=} \hh(\ke) \lcat \List{(\val, \txid, \emptyset)}
			\end{array}
			\]
			\end{enumerate}
		\end{itemize}
		
		\item Suppose that $(\otW, \ke, \val) \in \opset$ for some $\val \in \Val$, and $(\otR, \ke, \_) \in \opset$. 
		Let $\hh(\ke, \vi) = (\val', \txid', \T')$. We prove that $\updateKV(\hh, \vi, \txid, \opset)(\ke) = 
		\hh(\ke)\rmto{\vi(\ke)}{(\val', \txid', \T' \cup \{\txid\}} \lcat \List{(\val, \txid, \emptyset)}$ 
		by induction on $\opset$:
			\begin{itemize}
			\item $\opset = \emptyset$; this case is vacuous.
			\item $\opset = \opset' \uplus \{(\otR, \ke', \_)\}$. We distinguish two cases, according to 
			whether $\ke = \ke'$ or $\ke \neq \ke'$. If $\ke = \ke'$, then we know that 
			$(\otW, \ke, \val) \in \opset'$ and $(\otR, \ke, \val'') \notin \opset$ for any $\val'' \in \Val$. 
			By Lemma \cref{lem:updatekv.explicit}\eqref{item:updatekv.explicit.wr} we have that 
			\begin{equation}
			\forall \hh'.\;\updateKV(\hh,\vi,\txid,\opset')(\ke) = \hh(\ke) \lcat \List{(\val, \txid, \emptyset)}
			\label{eq:updateKV.explicit.rdwr.applyWR}
			\end{equation}
			from which it follows that 
			\[
			\begin{array}{lr}
			\updateKV(\hh, \vi, \txid, \opset)(\ke) = \updateKV(\hh, \vi, \txid, \opset' \uplus \{(\otR, \ke', \_)\})(\ke) &=\\
			\updateKV(\hh, \vi, \txid, \opset' \uplus \{(\otR, \ke, \_)\})(\ke) &\stackrel{\eqref{eq:updatekv}}{=}\\
			\updateKV(\hh\rmto{\ke}{\hh(\ke)\rmto{\max(\vi(\ke))}{(\val', \txid', \T' \cup \{\txid\}}}, \vi, \txid, \opset')(\ke) &\stackrel{\eqref{eq:updateKV.explicit.rdwr.applyWR}}{=}\\
			\hh\rmto{\ke}{\hh(\ke)\rmto{\max(\vi(\ke))}{(\val', \txid', \T' \cup \{\txid\}}}(\ke) \lcat \List{(\val, \txid, \emptyset)}\\
			\hh(\ke)\rmto{\max(\vi(\ke))}{(\val' \txid', \T' \cup \{\txid\}} \lcat \List{(\val, \txid, \emptyset)}
			\end{array}
			\]
			If $\ke \neq \ke'$, then we have that both $(\otR, \ke, \_) \in \opset'$ and 
			$(\otW, \ke, \val) \in \opset'$. In this case, by the inductive hypothesis we have that 
			\begin{equation}
			\forall \hh'.\;\updateKV(\hh,\vi,\txid,\opset')(\ke) = \hh'(\ke)\rmto{\max(\vi(\ke))}{(\val', \txid', \T' \cup \{\txid\})} \lcat \List{(\val, \txid, \emptyset)}
			\label{eq:updatekv.explicit.rdwr.IHrd}
			\end{equation}
			from which it follows that 
			\[
			\begin{array}{lr}
			\updateKV(\hh, \vi, \txid, \opset)(\ke) = \updateKV(\hh, \vi, \txid, \opset' \uplus \{(\otR, \ke', \_)\})(\ke) &\stackrel{\eqref{eq:updatekv}}{=}\\
			\updateKV(\hh\rmto{\ke'}{\_}, \vi, \txid, \opset')(\ke) &\stackrel{\eqref{eq:updatekv.explicit.rdwr.IHrd}}{=}\\
			\hh\rmto{\ke'}{\_}(\ke)\rmto{\max(\vi(\ke))}{(\val', \txid', \T' \cup \{\txid\})} \lcat \List{(\val, \txid, \emptyset)} &=\\
			\hh(\ke)\rmto{\max(\vi(\ke))}{(\val', \txid', \T' \cup \{\txid\})} \lcat \List{(\val, \txid, \emptyset)}
			\end{array}
			\]
			
			\item $\opset = \opset' \uplus \{(\otW, \ke'', \val'')\}$ for some $\ke'', \val''$. Again, 
			there are two possible cases to consider. If $\ke = \ke''$, then $\val = \val''$ because of the structure imposed on $\opset$.
			Furthermore, we have that 
			$(\otR, \ke, \_) \in \opset'$ and $(\otW, \ke, \val''') \notin \opset$ for all $\val''' \in \Val$.
			By \cref{lem:updatekv.explicit}\eqref{item:updatekv.explicit.rd} we have that 
			\begin{equation}
			\forall \hh'.\;\updateKV(\hh', \vi, \txid, \opset')(\ke) = \hh'(\ke)\rmto{\max(\vi(\ke))}{(\val', \txid', \T' \cup \{\txid\})}
			\label{eq:updatekv.explicit.rdwr.applyRD}
			\end{equation}
			We have that 
			\[
			\begin{array}{lr}
			\updateKV(\hh,\vi,\txid, \opset)(\ke) = \updateKV(\hh, \vi, \txid, \opset' \cup \{(\otW, \ke'', \val'')\})(\ke) &= \\
			\updateKV(\hh,\vi, \txid, \opset' \cup \{(\otW, \ke, \val)\})(\ke) &\stackrel{\eqref{eq:updatekv}}{=}\\
			\updateKV(\hh\rmto{\ke}{\hh(\ke) \lcat \List{(\val, \txid, \emptyset)}}, \vi, \txid, \opset')(\ke) &\stackrel{\eqref{eq:updatekv.explicit.rdwr.applyRD}}{=}\\
			\hh\rmto{\ke}{\hh(\ke) \lcat \List{(\val, \txid, \emptyset)}}(\ke)\rmto{\max(\vi(\ke))}{(\val', \txid', \T' \cup \{\txid\})} &=\\
			(\hh(\ke) \lcat\List{(\val, \txid, \emptyset)})\rmto{\max(\vi(\ke))}{(\val', \txid', \T' \cup \{(\txid\})} &=\\
			\hh(\ke)\rmto{\max(\vi(\ke))}{(\val',\txid', \T' \cup \{(\txid)\})} \lcat \List{(\val, \txid, \emptyset)}
			\end{array}
			\]
			Finally, if $\ke \neq \ke'$, then we have that $(\otR, \ke, \_) \in \opset'$ and $(\otW, \ke, \val) \in \opset'$. 
			By the inductive hypothesis, we obtain 
			\begin{equation}
			\forall \hh'.\;\updateKV(\hh', \vi, \txid, \opset')(\ke) = \hh'(\ke)\rmto{\max(\vi(\ke))}{(\val', \txid', \T' \cup \{\txid\})} \lcat \List{(\val, \txid, \emptyset)}.
			\label{eq:updatekv.explicit.rdwr.IHwr}
			\end{equation}
			It follows that 
			\[
			\begin{array}{lr}
			\updateKV(\hh, \vi, \txid, \opset)(\ke) = \updateKV(\hh, \vi, \txid, \opset' \uplus \{(\otW, \ke', \_)\}) &\stackrel{\eqref{eq:updatekv}}{=}\\
			\updateKV(\hh\rmto{\ke'}{\_}, \vi, \txid, \opset')(\ke) &\stackrel{\eqref{eq:updatekv.explicit.rdwr.IHwr}}{=}\\
			\hh\rmto{\ke'}{\_}(\ke)\rmto{\max(\vi(\ke))}{(\val', \txid', \T' \cup \{\txid\})} \lcat \List{(\val, \txid, \emptyset)} &=\\
			\hh\rmto{\max(\vi(\ke))}{(\val', \txid', \T' \cup \{\txid\})} \lcat \List{(\val, \txid, \emptyset)}
			\end{array}
			\]
			\end{itemize}
\end{enumerate}
\end{proof}
%\begin{lemma}
%\ac{This Lemma was wrong}
%Let $\hh \in \HisHeaps$, $\vi \in \Views(\hh)$, 
%$\txid \in \TxID$ and $\opset \in \powerset{\Ops}$. 
%Let also $\ke \in \Keys$ be such that 
%$\forall \val \in \Val.\;(\otR, \ke, \val) \notin \opset$. Then 
%for any versions $\ver$, 
%\[
%\updateKV(\hh, \vi, \txid, \opset)\rmto{\ke}{\hh(\ke)\rmto{\vi(\ke)}{\ver}} = 
%\updateKV(\hh\rmto{\ke}{\hh(\ke)\rmto{\vi(\ke)}{\ver}}, \vi, \txid, \opset)
%\] 
%\end{lemma}
%
%\begin{proof}
%By induction on $\opset$. 
%\begin{itemize}
%\item Base case: $\opset = \emptyset$. In this case we have that 
%\[
%\begin{array}{lr}
%\updateKV(\hh, \vi, \txid, \emptyset)\rmto{\ke}{\hh(\ke)\rmto{\vi(\ke)}{\ver}} &=\\
%\hh\rmto{\ke}{\hh(\ke)\rmto{\vi(\ke)}{\ver}} &\\
%&\\
%\updateKV(\hh\rmto{\ke}{\hh(\ke)\rmto{\vi(\ke)}{\ver}}, \vi ,\txid, \opset) &=\\
%\hh\rmto{\ke}{\hh(\ke)\rmto{\vi(\ke)}{\ver}}
%\end{array}
%\]
%and there is nothing left to prove.
%\item Suppose that 
%\begin{equation}
%\label{eq:opset.read.def}
%\opset = \opset' \uplus \{(\otR, \ke', \val')\}. 
%\end{equation} 
%By assumption we have that $\forall \val \in \Val.\;(\otR,\ke, \val) \notin \opset$, which implies that 
%\begin{equation}
%\label{eq:kep.neq.ke}
%\ke' \neq \ke, 
%\end{equation}
%and $\forall \val \in \Val.\; (\otR, \ke, \val) \notin \opset'$. 
%%Also, because we are assuming that $\opset$ contains at most a read operation for key $\ke'$, 
%%it must be the case that $\forall \val \in \Val.\;(\otR, \ke', \val) \notin \opset'$.
%By inductive hypothesis, we have that for any kv-store $\hh'$, then 
%\begin{equation}
%\label{eq:updatekv.rd.extract}
%\updateKV(\hh', \vi, \txid, \opset')\rmto{\ke}{\hh(\ke)\rmto{\vi(\ke)}{\ver}} = 
%\updateKV(\hh'\rmto{\ke}{\hh'(\ke)\rmto{\vi(\ke)}{\ver'}}, \vi, \txid, \opset').
%\end{equation}
%To prove the claim, we will need the following facts, which are trivial to prove: 
%\begin{equation}
%\label{eq:fsubst.same}
%\forall f: X \rightarrow Y.\; \forall x_1,x_2 \in X.\;\forall y \in Y. x_1 \neq x_2 \implies f\rmto{x_2}{y}(x_1) = f(x_1).
%\end{equation}
%\begin{equation}
%\label{eq:fsubst.swap}
%\forall f: X \rightarrow Y. \forall x_1, x_2 \in X.\; \forall y_1, y_2 \in Y. x_1 \neq x_2 \implies f\rmto{x_1}{y_1}\rmto{x_2}{y_2} = 
%f\rmto{x_2}{y_2}\rmto{x_1}{y_1}.
%\end{equation}
%%We then have that 
%%\[
%%\begin{array}{lr}
%%\updateKV(\hh, \vi, \txid, \opset) =
%%\mathtt{ let } (\val, \txid', \T) = \hh(\ke', \vi) \mathtt{ in } &\\  
%% \updateKV(\hh\rmto{\ke'}{\hh(\ke')\rmto{\vi(\ke')}{(\val, \txid', \T \cup \{ t \})}}, \vi, \txid, \opset') &=\\
%% & \\
%% \mathtt{ let } (\val, \txid', \T) = \hh(\ke, \vi) \mathtt{ in } &\\ 
%% \updateKV(\hh, \vi, \txid, \opset')\rmto{\ke'}{\hh(\ke')\rmto{\vi(\ke')}{(\val, \txid', \T \cup \{ t \})}}
%%\end{array}
%%\]
%%from which it follows that \
%We then have that 
%\[
%\begin{array}{lr}
%\updateKV(\hh, \vi, \txid, \opset)\rmto{\ke}{\hh(\ke)\rmto{\vi(\ke)}{\ver}} &\stackrel{\eqref{eq:opset.read.def}}{=}\\ 
%&\\
%\updateKV(\hh, \vi, \txid, \opset' \cup \{(\otR, \ke', \val')\})\rmto{\ke}{\hh(\ke)\rmto{\vi(\ke)}{\ver}} &\stackrel{\eqref{eq:updatekv}}{=}\\ 
%&\\
%\mathtt{ let } (\val, \txid', \T) = \hh(\ke', \vi) \mathtt{ in } & \\
% \updateKV(\hh\rmto{\ke'}{\hh(\ke')\rmto{\vi(\ke')}{(\val, \txid', \T \cup \{ t \})}}, \vi, \txid, \opset')\rmto{\ke}{\hh(\ke)\rmto{\vi(\ke)}{\ver}} &\stackrel{\eqref{eq:updatekv.rd.extract}}{=}\\
%% &\\
%% \mathtt{ let } (\val, \txid', \T) = \hh(\ke', \vi') \mathtt { in } &\\
%% \updateKV(\hh, \vi, \txid, \opset')\rmto{\ke}{\hh(\ke)\rmto{\vi(\ke)}{\ver}}\rmto{\ke'}{\hh(\ke')\rmto{\vi(\ke')}{(\val, \txid', \T \cup \{ t \})}} &=\\
%% &\\
%% \mathtt{ let } (\val, \txid', \T) = \hh(\ke, \vi') \mathtt{ in } &\\
%% \updateKV(\hh\rmto{\ke}{\hh(\ke)\rmto{\vi(\ke)}{\ver}}, \vi, \txid, \opset')\rmto{\ke'}{\hh(\ke')\rmto{\vi(\ke')}{(\val, \txid', \T \cup \{ t \})}} &=\\ 
%&\\
%\mathtt{ let } (\val, \txid', \T) = \hh(\ke', \vi) \mathtt{ in } &\\
% \updateKV(\hh\rmto{\ke'}{\hh(\ke')\rmto{\vi(\ke')}{(\val, \txid', \T \cup \{ t \})}}\rmto{\ke}{\hh(\ke)\rmto{\vi(\ke)}{\ver}}, \vi, \txid, \opset') &\stackrel{\eqref{eq:fsubst.swap}}{=}\\ 
%&\\
%\mathtt{ let } (\val, \txid', \T) = \hh(\ke', \vi) \mathtt{ in } &\\
% \updateKV(\hh\rmto{\ke}{\hh(\ke)\rmto{\vi(\ke)}{\ver}}\rmto{\ke'}{\hh(\ke')\rmto{\vi(\ke')}{(\val, \txid', \T \cup \{ t \})}}, \vi, \txid, \opset') &\stackrel{\eqref{eq:fsubst.same}}{=}\\ 
%&\\
%\mathtt{ let } (\val, \txid', \T) = \hh\rmto{\ke}{\hh(\ke)\rmto{\vi(\ke)}{\ver}}(\ke', \vi) \mathtt{ in } &\\
% \updateKV(\hh\rmto{\ke}{\hh(\ke)\rmto{\vi(\ke)}{\ver}}\rmto{\ke'}{\hh(\ke')\rmto{\vi(\ke')}{(\val, \txid', \T \cup \{ t \})}}, \vi, \txid, \opset') &\stackrel{\eqref{eq:updatekv}}{=}\\ 
%&\\ 
%\updateKV(\hh\rmto{\ke}{\hh(\ke)\rmto{\vi(\ke)}{\ver}}, \vi, \txid, \opset' \cup \{(\otR, \ke', \val')\} &\stackrel{\eqref{eq:opset.read.def}}{=}\\
%&\\
%\updateKV(\hh\rmto{\ke}{\hh(\ke)\rmto{\vi(\ke)}{\ver}}, \vi, \txid, \opset)
%\end{array}
%\]
%\item Suppose that 
%\begin{equation}
%\label{eq:opset.wr.def}
%\opset = \opset' \uplus \{(\otW, \ke', \val)\}
%\end{equation}
%Then we have that 
%\[
%\begin{array}{lr}
%\updateKV(\hh, \vi, \txid, \opset)\rmto{\ke}{\hh(\ke)\rmto{\vi(\ke)}{\ver}} = 
%\updateKV(\hh, \vi, \txid, \opset' \uplus \{(\otW, \ke', \val)\} &=\\
%&\\
%\updateKV(\hh\rmto{\ke}{\hh(\ke) \lcat \List{(\val, \txid, \emptyset)}}, \vi, \txid, \opset')\rmto{\ke}{\hh(\ke)\rmto{\vi(\ke)}{\ver}} &=\\
%&\\
%\updateKV(\hh\rmto{\ke}{\hh(\ke) \lcat \List{(\varl, \txid, \emptyset)}}\rmto{\ke}{\hh(\ke)\rmto{\vi(\ke)}{\ver}}
%\end{array}
%\]
%\end{itemize}
%\end{proof}

In the following, given a version $\ver = (\val, \txid', \T)$ and a set of 
transaction identifiers $\T'$, we let $\ver \oplus \T' = (\val, \txid', \T \cup \T')$. 
Clearly the operator $\oplus$ is commutative over sets of transactions: 
$\forall \ver, \T, \T'.\; (\ver \oplus \T) \oplus \T' = (\ver \oplus \T') \oplus \T = 
\ver \oplus (\T \cup \T')$.

\begin{corollary}
\label{cor:updatekv.singlecell}
Let $\hh$ be a kv-store, $\vi \in \Views(\hh)$, $\txid \in \TxID$ and $\opset \in \powerset{\Ops}$. 
Let also $\ke \in \Keys$. Then 
\begin{enumerate}
\item\label{item:updatekv.singlecell.noview} $\forall i=0,\cdots, \lvert \hh(\ke) \rvert -1.\; i \neq \max(\vi(\ke)) \implies \updateKV(\hh, \vi, \txid, \opset)(\ke, i) = 
\hh(\ke, i)$, 
\item\label{item:updatekv.singlecell.rd} $\forall \val. (\otR, \ke, \_) \in \opset \implies \updateKV(\hh,\vi,\txid, \opset)(\ke, \vi) = \hh(\ke, \max(\vi(\ke))) \oplus \{\txid\}$; 
\item\label{item:updatekv.singlecell.nord} $(\forall \val.(\otR,\ke, \val) \notin \opset) \implies \updateKV(\hh,\vi,\txid, \opset)(\ke,\vi) = \hh(\ke, \max(\vi(\ke)))$;
\item\label{item:updatekv.singlecell.wr} $\forall \val.(\otW, \ke, \val) \in \opset \implies (\lvert \updateKV(\hh,\vi,\txid,\opset)(\ke) \rvert = 
\lvert \hh(\ke) \rvert + 1) \wedge \updateKV(\hh,\vi,\txid,\opset)(\ke, \lvert \hh(\ke) \rvert) = (\val, \txid, \emptyset)$.
\item\label{item:updatekv.singlecell.nowr} $(\forall \val.(\otW, \ke, \val) \notin \opset) \implies \lvert \updateKV(\hh,\vi,\txid,\opset)(\ke) \rvert = \lvert \hh(\ke) \rvert$.
\end{enumerate}
\end{corollary}

\begin{proof}
A simple consequence of \cref{lem:updatekv.explicit}.
\end{proof}
\ac{For the moment I will assume that I have proved this, though it should be a simple consequence of Lemma 
\ref{lem:updatekv.explicit}.}

\begin{proposition}
Let $\hh \in \HisHeaps$, $\vi_1, \vi_2 \in \Views(\hh)$, 
$\txid_1, \txid_2 \in \TxID$. 
Let also $\opset_1, \opset_2$ be such that whenever 
$(\otW, \ke, \_) \in \opset_1$, then $(\otW, \ke, \val) \notin \opset_2$ 
for all $\val \in \Val$. Then we have that
\[
\begin{array}{l}
\mathtt{let } \hh_1 = \updateKV(\hh, \vi_1, \txid_1, \opset_1) \mathtt{ in } \updateKV(\hh_1, \vi_2, \txid_2, \opset_2) = \\
\mathtt{let } \hh_2 = \updateKV(\hh, \vi_2, \txid_2, \opset_2) \mathtt{ in } \updateKV(\hh_2, \vi_1, \txid_1, \opset_1).
\end{array}
\]
\end{proposition}

\begin{proof}
Let $\hh_1 = \updateKV(\hh, \vi_1, \txid_1, \opset_1)$, $\hh_2 = \updateKV(\hh, \vi_2, \txid_2, \opset_2)$. It 
suffices to show that for any $\ke \in \Keys$, $\lvert \updateKV(\hh_1, \vi_2, \txid_2, \opset_2)(\ke) \rvert = \lvert 
\updateKV(\hh_2, \vi_1, \txid_1, \opset_1)(\ke) \rvert$, and for any $i=0,\cdots, \lvert \updateKV(\hh_1, \vi_2, \txid_2, \opset_2)(\ke) \rvert$, 
$\updateKV(\hh_1, \vi_2, \txid_2,\opset_2)(\ke, i) = \updateKV(\hh_2, \vi_1, \txid_1, \opset_1)(\ke_1)$. 
\ac{Note that this proof - and also the proof of the lemma before it- requires proves the equivalence of two functions by proving the equivalence of their values 
for each element in their domains; in other words, this proof uses the axiom of extensionality. I am pretty sure 
that a proof that reduces the two terms to the same normal form exists, though it has way too many details 
and would not be feasible without using a theorem prover.}

First, fix a key $\ke \in \Keys$. Note that if $(\otW, \ke, \_) \in \opset_1$, then 
by Corollary \ref{cor:updatekv.singlecell} we have that $\lvert \updateKV(\hh, \vi_1, \txid_1, \opset_1)(\ke) \rvert = 
\lvert \hh(\ke) \rvert$. Because $\opset_1$ is not conflicting with $\opset_2$, it must be the case 
that $\forall \val.(\otW,\ke,\val) \notin \opset_2$, and therefore by \cref{cor:updatekv.singlecell} 
we have that 
\[
\lvert \updateKV(\hh_1, \vi_2, \txid_2, \opset_2)(\ke) \rvert = \lvert \hh_1(\ke) \rvert = \lvert \hh(\ke) \rvert + 1.
\] 
Similarly, because $\forall \val.(\otW,\ke,\val)\notin \opset_2$ 
and $(\otW,\ke,\_) \in \opset_1$, then 
\[
\lvert \updateKV(\hh_2, \vi_1, \txid_1, \opset_1)(\ke) \rvert = \lvert \hh_2(\ke) \rvert + 1 = 
\lvert \updateKV(\hh, \vi_2, \txid_2, \opset_2)(\ke) \rvert = \lvert \hh(\ke) \rvert + 1.
\]
Therefore, if $(\otW, \ke, \_) \in \opset_1$, we have that 
$\lvert \updateKV(\hh_2, \vi_1, \txid_1, \opset_1)(\ke) \rvert = 
\lvert \updateKV(\hh_1(\ke), \vi_2, \txid_2, \opset_2)(\ke) \rvert$.
Analogously, we can prove that this claim holds also when $(\otW, \ke, \_) \in \opset_2$. 
Finally, if $(\forall \val.(\otW,\ke,\_) \notin \opset_1) \wedge (\forall \val.(\otW,\ke,\val) \notin \opset_2)$, 
then by \cref{cor:updatekv.singlecell} we have that 
\[
\lvert \updateKV(\hh_1, \vi_2, \txid_2, \opset_2)(\ke) \rvert = 
\lvert \hh_1(\ke) \rvert = \lvert \updateKV(\hh, \vi_1, \txid_1, \opset_1)(\ke) \rvert = \lvert \hh(\ke) \rvert,
\]
\[
\lvert \updateKV(\hh_2, \vi_1, \txid_1, \opset_1)(\ke) \rvert = 
\lvert \hh_2(\ke) \rvert = \lvert \updateKV(\hh, \vi_2, \txid_2, \opset_2)(\ke) \rvert = \lvert \hh(\ke) \rvert.
\]
This concludes the proof that, for any $\ke \in \Keys$, $\lvert \updateKV(\hh_1,\vi_2,\txid_2,\opset_2) \rvert = 
\lvert \updateKV(\hh_2,\vi_1,\txid_1,\opset_1) \rvert$.

Next, fix a key $\ke$ and a index $i =0, \cdots, \lvert \hh(\ke) \rvert - 1$. 
We show that $\updateKV(\hh_1,\vi_2,\txid_2,\opset_2)(\ke, i) = \updateKV(\hh_2, \vi_1, \txid_1, \opset_1)(\ke, i)$ 
by performing a case analysis on $\vi_1$: 
\begin{enumerate}
\item $i \neq \max(\vi_1(\ke))$: in this case, by \cref{cor:updatekv.singlecell}\eqref{item:updatekv.singlecell.noview}, 
we have that 
\begin{equation}
\hh_1(\ke, i) = \updateKV(\hh, \vi_1, \txid_1, \opset_1)(\ke, i) = \hh(\ke, i), 
\label{eq:v1.nord.hh1}
\end{equation}
and 
\begin{equation}
\updateKV(\hh_2, \vi_1, \txid_1, \opset_1)(\ke, i) = \hh_2(\ke, i).
\label{eq:v1.nord.uhh2}
\end{equation}
We have three possible sub-cases: 
\begin{enumerate}
\item $i \neq \max(\vi_2(\ke))$: in this case, by \cref{cor:updatekv.singlecell}\eqref{item:updatekv.singlecell.noview} we have that 
\[\updateKV(\hh_1, \vi_2, \txid_2, \opset_2)(\ke, i) = 
\hh_1(\ke, i) \stackrel{\eqref{eq:v1.nord.hh1}}{=} \hh(\ke, i), 
\]
\[
\updateKV(\hh_2, \vi_1, \txid_1, \opset_1) \stackrel{\eqref{eq:v1.nord.uhh2}}{=} \hh_2(\ke,i) = 
\updateKV(\hh, \vi_2, \txid_2, \opset_2)(\ke, i) = \hh(\ke, i).
\]
\item $i = \max(\vi_2(\ke))$, and $(\otR, \ke, \_) \notin \opset_2$. In this case the proof is analogous to the previous case, 
only \cref{cor:updatekv.singlecell}\eqref{item:updatekv.singlecell.nord} needs to be applied in place 
of \cref{cor:updatekv.singlecell}\eqref{item:updatekv.singlecell.noview}.
\item $i = \max(\vi_2(\ke))$, and $(\otR, \ke, \_) \in \opset_2$. In this case we can apply \cref{cor:updatekv.singlecell}\eqref{item:updatekv.singlecell.rd}, 
and deduce that 
\begin{equation}
\updateKV(\hh_1, \vi_2, \txid_2, \opset_2)(\ke, i) = \hh_1(\ke, i) \oplus \{\txid_2\}
\label{eq:v1.nord.v2.rd.uhh1}
\end{equation}
\begin{equation}
\hh_2(\ke, i) = \updateKV(\hh, \vi_2, \txid_2, \opset_2)(\ke, i) = \hh(\ke,i) \oplus \{\txid_2\}
\label{eq:v1.nord.v2.rd.hh2}
\end{equation}
It follows that 
\[
\begin{array}{l}
\updateKV(\hh_1, \vi_2, \txid_2, \opset_2)(\ke, i) \stackrel{\eqref{eq:v1.nord.v2.rd.uhh1}}{=} \hh_1(\ke, i) \oplus \{\txid_2 \} \stackrel{\eqref{eq:v1.nord.hh1}} \hh(\ke, i) \oplus \{\txid_2\}\\
\updateKV(\hh_2,\vi_1, \txid_1,\opset_1)(\ke, i) \stackrel{\eqref{eq:v1.nord.uhh2}}{=} \hh_2(\ke, i) \stackrel{\eqref{eq:v1.nord.v2.rd.hh2}} = \hh(\ke, i) \oplus \{\txid_2\}
\end{array}
\]
\end{enumerate}
\item $i = \max(\vi_1(\ke))$, $(\otR, \ke, \_) \notin \opset_1$. This case is similar to the previous one: we can infer 
that Equations \eqref{eq:v1.nord.hh1} and \eqref{eq:v1.nord.uhh2} are valid in this case using \cref{cor:updatekv.singlecell}
\eqref{item:updatekv.singlecell.nord}, then we can proceed by performing a case analysis on $\vi_2$ and $\opset_2$ as in the previous case.
\item $i = \max(\vi_1(\ke))$, $\otR,\ ke, \_) \in \opset_1$. We can apply \cref{cor:updatekv.singlecell}\eqref{item:updatekv.singlecell.rd} 
to deduce the following: 
\begin{equation}
\hh_1(\ke, i) = \updateKV(\hh, \vi_1, \txid_1, \opset_1)(\ke, i) = \hh(\ke, i) \oplus \{\txid_1\},
\label{eq:v1.rd.hh1}
\end{equation}
\begin{equation}
\updateKV(\hh_2, \vi_1, \txid_1, \opset_1)(\ke, i) = \hh_2(\ke, i) \oplus \{\txid_1\}. 
\label{eq:v1.rd.uhh2}
\end{equation}
We have two different sub-cases to consider: 
\begin{enumerate}
\item $i \neq \max(\vi_2(\ke))$, or $i = \max(\vi_2(\ke))$ with $(\otR,\ke,\_) \notin \opset_2$. In this case, we can apply either 
\cref{cor:updatekv.singlecell}\eqref{item:updatekv.singlecell.noview} (if $i \neq \max(\vi_2(\ke))$ ), or 
\cref{cor:updatekv.singlecell} \eqref{item:updatekv.singlecell.nord} (if $i = \max(\vi_2(\ke))$ and $(\otR, \ke, \_) \notin \opset_2$), 
to obtain 
\[
\updateKV(\hh_1, \vi_2, \txid_2, \opset_2)(\ke, i) = \hh_1(\ke, i) \stackrel{\eqref{eq:v1.rd.hh1}}{=} \hh(\ke, i) \oplus \{\txid_1\},
\]
\[
\updateKV(\hh_2, \vi_1, \txid_1, \opset_1)(\ke, i) \stackrel{\eqref{eq:v1.rd.uhh2}}{=} \hh_2(\ke, i) \oplus \{ \txid_1 \} = 
\hh(\ke, i) \oplus \{ \txid_1 \}.
\]
\item if $i = \max(\vi_2(\ke))$ and $(\otR, \ke, \_) \in \opset_2$, then by \cref{cor:updatekv.singlecell}\eqref{item:updatekv.singlecell.rd} 
we obtain that 
\begin{equation}
\hh_2(\ke, i) = \updateKV(\hh, \vi_2, \txid_2, \opset_2)(\ke, i) = \hh(\ke, \i) \oplus \{\txid_2\},
\label{eq:v1.rd.v2.rd.hh2}
\end{equation}
\begin{equation}
\updateKV(\hh_1, \vi_2, \txid_2, \opset_2)(\ke, i) = \hh_1(\ke, i) \oplus \{ \txid_2\}.
\label{eq:v1.rd.v2.rd.uhh1}
\end{equation}
From these facts it follows that
\[
\begin{array}{l}
\updateKV(\hh_1, \vi_2, \txid_2, \opset_2)(\ke, i) \stackrel{\eqref{eq:v1.rd.v2.rd.uhh1}}{=} \hh_1(\ke, i) \oplus \{\txid_2\} \stackrel{\eqref{eq:v1.rd.hh1}}{=} 
(\hh(\ke, i) \oplus \{\txid_1\}) \oplus \{\txid_2 \} = \hh(\ke, i) \oplus \{\txid_1, \txid_2\}\\
\updateKV(\hh_2, \vi_1, \txid_1, \opset_1)(\ke, i) \stackrel{\eqref{eq:v1.rd.uhh2}}{=} \hh_2(\ke, i) \oplus \{\txid_1\} 
\stackrel{\eqref{eq:v1.rd.v2.rd.hh2}}{=} (\hh(\ke, i) \oplus \{\txid_2\}) \oplus \{\txid_1\} = \hh(\ke, i) \oplus \{\txid_1, \txid_2\}
\end{array}
\]
\end{enumerate}
\end{enumerate}

Next, note that if $\forall \val \in \Val.\;(\otW,\ke,\val) \notin \opset_1 \wedge (\otW, \ke, \val) \notin 
\opset_2$, then $\lvert \updateKV(\hh_1, \vi_2, \txid_2, \opset_2)(\ke) \rvert = \lvert \hh(\ke) \rvert 
= \lvert \updateKV(\hh_2, \vi_1, \txid_1, \opset_1)(\ke) \rvert$. 
Because we have already proved that $\forall i = 0,\cdots, \lvert \hh(\ke) \rvert.\; \updateKV(\hh_1, 
\vi_2, \txid_2, \opset_2)(\ke, i) = \updateKV(\hh_2, \vi_1, \txid_1, \opset_1)(\ke, i)$, it follows 
that $\updateKV(\hh_1, \vi_2, \txid_2, \opset_2)(\ke) = \updateKV(\hh_2,\vi_1,\txid_1,\opset_1)(\ke)$, 
and there is nothing left to prove.

Suppose then that  either $(\otW, \ke, \val) \in \opset_1$ or $(\otW,\ke, \val) \in \opset_2$ 
for some $\val$. Without loss of generality, let $(\otW,\ke,\val) \in \opset_1$ for some $\val \in \Val$; 
because we are assuming that $\opset_1$ does not conflict with $\opset_2$, then 
it must be the case that $\forall \val' \in \Val.\;(\otW,\ke,\val') \notin \opset_2$. 
Using \cref{cor:updatekv.singlecell}\eqref{item:updatekv.singlecell.nowr} and 
\cref{cor:updatekv.singlecell}\eqref{item:updatekv.singlecell.wr}, 
\[
\begin{array}{l}
\updateKV(\hh_1, \vi_2, \txid_2, \opset_2)(\ke, \lvert \hh(\ke) \rvert) = 
\hh_1(\ke, \lvert \hh(\ke) \rvert) = \updateKV(\hh, \vi_1, \txid_1, \opset_1)(\lvert \hh(\ke) \rvert) = (\val, \txid_1, \emptyset)\\
\updateKV(\hh_2, \vi_1, \txid_1,\ opset_1)(\ke, \lvert, \hh(\ke) \rvert) = (\val, \txid_1, \emptyset)
\end{array}
\]
We have now proved that if $(\otW,\ke,\val) \in \opset_1$, then $\lvert \updateKV(\hh_1, \vi_2, \txid_2, \opset_2) \rvert = 
\lvert \updateKV(\hh_2, \vi_1, \txid_1, \opset_1) \rvert$, and for all 
$i=0,\cdots, \lvert \updateKV(\hh_1, \vi_2, \txid_2, \opset_2) \rvert - 1$, 
$\updateKV(\hh_1,\vi_2, \txid_2, \opset_2)(\ke, i) = \updateKV(\hh_2, \vi_1, \txid_1, \opset_1)(\ke, i)$. 
This concludes the proof that $\forall \ke \in \Keys.\updateKV(\hh_1,\vi_2,\txid_2,\opset_2)(\ke) = 
\updateKV(\hh_2,\vi_1,\txid_1,\opset_1)(\ke)$, and therefore 
$\updateKV(\hh_1, \vi_2, \txid_2, \opset_2) = \updateKV(\hh_2, \vi_1,\txid_1,\opset_1)$.
\end{proof}
%
%\begin{proof}
%, $\updateKV(\hh_1, \vi_2, \txid_2, \opset_2)(\ke) = 
%\updateKV(\hh_1,\vi_2, \txid_2,\opset_1)(\ke)$. Fix then $\ke \in \Keys$, and let 
%$(\val_1, \txid_1, \T_1) = \hh(\ke, \vi_1), (\val_2,\txid_2, \T_2) = \hh(\ke, \vi_2)$. We perform a case analysis on $\opset_1$.
%
%\begin{proof}
%\ac{4 cases for $\opset_1$, each of which requires 4 cases for $\opset_2$. 16 sub-cases in total. I hate when 
%this happens.} 
%Let $\hh_1 = \updateKV(\hh, \vi_1, \txid_1, \opset_1)$, $\hh_2 = \updateKV(\hh, \vi_2, \txid_2, \opset_2)$. It 
%suffices to show that for any $\ke \in \Keys$, $\updateKV(\hh_1, \vi_2, \txid_2, \opset_2)(\ke) = 
%\updateKV(\hh_1,\vi_2, \txid_2,\opset_1)(\ke)$. Fix then $\ke \in \Keys$, and let 
%$(\val_1, \txid_1, \T_1) = \hh(\ke, \vi_1), (\val_2,\txid_2, \T_2) = \hh(\ke, \vi_2)$. We perform a case analysis on $\opset_1$.
%\begin{itemize}
%\item Suppose that $\forall \val.\;(\otR, \ke, \val) \notin \opset_1 \wedge (\otW, \ke, \val) \notin \opset_1$. 
%In this case, by \cref{lem:updatekv.explicit}\eqref{item:updatekv.explicit.none} we have that 
%$\hh_1(\ke) = \updateKV(\hh, \vi_1, \txid_1, \opset_1)(\ke) = \hh(\ke)$, 
%and $\updateKV(\hh_2, \vi_1, txid_1, \opset_1)(\ke) = \hh_2(\ke)$. It follows that 
%\[
%\begin{array}{l}
%\updateKV(\hh_1, \vi_2, \txid_2, \opset_2)(\ke) = \updateKV(\hh, \vi_2, \txid_2, \opset_2)(\ke)\\
%\updateKV(\hh_2, \vi_1\ txid_2, \opset_2)(\ke) = \hh_2(\ke) = \updateKV(\hh,\vi_2, \txid_2, \opset_2)(\ke)
%\end{array}
%\]
%\item Suppose now that $(\otR, \ke, \_) \in \opset_1$, and $\forall \val.(\otW,\ke,\val) \notin \opset_1$. 
%By \cref{lem:updatekv.explicit}\eqref{item:updatekv.explicit.rd} we have the following: 
%\begin{align}
%\hh_1(\ke) = \updateKV(\hh, \vi_1, \txid_1, \opset_1)(\ke) = 
%\big( \text{let } (\val_1, \txid_1', \T_1) = \hh(\ke, \vi_1) \text{ in } \hh(\ke)\rmto{\vi_1(\ke)}{(\val_1, \txid_1', \T_1 \cup \{\txid_1\})} \big)
%\label{eq:updatekv.swap.rd11}\\
%\nonumber\updateKV(\hh_2, \vi_1, \txid_1, \opset_1)(\ke) = big( \text{let } (\val_2, \txid_2', \T_2) = \hh_2(\ke, \vi_1) 
%\text{ in } \hh_2(\ke)\rmto{\vi_1(\ke)}{(\val_2, \txid_2', \T_2 \cup \{\txid_1\})} \big) = \\
%\text{let } (\val_2, \txid_2, \T_2) = \hh_2(\ke, \vi_1) \text{ in } \updateKV(\hh, \vi_2, \txid_2, \opset_2)(\ke)\rmto{\vi_1(\ke)}{(\val_2, \txid_2,', \T_2 \cup \{\txid_1\})}
%\label{eq:updatekv.swap.rd12}
%\end{align}
%\ac{The let statement here is necessary (in the second equation, I kept it in the first one for the sake of 
%conformity) because the contents of $\val_2, \txid_2, \T_2$ are going to change according to 
%the values of $\opset_2$ and $\vi_2(\ke)$. Lazy evaluation here helps in not repeating lots of details in 
%the rest of the proof.}
%We now perform a second case analysis on $\opset_2$. 
%\begin{itemize}
%\item $\forall \vi.\;(\otR, \ke, \val) \notin \opset_2 \wedge (\otW, \ke, \val) \notin \opset_2$. 
%By \cref{lem:updatekv.explicit}\eqref{item:updatekv.explicit.none} we have that 
%\begin{align}
%\hh_2(\ke) = \updateKV(\hh, \vi_2, \txid_2, \opset_2)(\ke) = \hh(\ke) 
%\label{eq:updatekv.swap.rd.none1}\\
%\updateKV(\hh_1, \vi_2, \txid_2, \opset_2)(\ke) = \hh_1(\ke)
%\label{eq:updatekv.swap.rd.none2}
%\end{align}
%from which it follows 
%\[
%\begin{array}{lr}
%\updateKV(\hh_1, \vi_2, \txid_2, \opset_2)(\ke) \stackrel{\eqref{eq:updatekv.swap.rd.none2}}{=} 
%\hh_1(\ke) & \stackrel{\eqref{eq:updatekv.swap.rd11}}{=}\\
%\text{let } (\val_1, \txid_1', \T_1) = \hh(\ke, \vi_1) \text{ in } \hh(\ke)\rmto{\vi_1(\ke)}{(\val_1, \txid_1', \T_1 \cup \{\txid_1\})} \big) &\\
%&\\
%\updateKV(\hh_2, \vi_1, \txid_1, \opset_1)(\ke) &\stackrel{\eqref{eq:updatekv.swap.rd12}}{=}\\ 
%\text{let } (\val_2, \txid_2', \T_2) = \hh_2(\ke, \vi_1) \text{ in } \updateKV(\hh, \vi_2, \txid_2, \opset_2)(\ke)\rmto{\vi_1(\ke)}{(\val_2, \txid_2', \T_2 \cup \{\txid_1\})} 
%&\stackrel{\eqref{eq:updatekv.swap.rd.none1}}{=} \\
%\text{let } (\val_2, \txid_2', \T_2) = \hh(\ke, \vi_1) \text{ in } \hh(\ke)\rmto{\vi_1(\ke)}{(\val_2, \txid_2', \T_2 \cup \{\txid_1\})}
%\end{array}
%\]
%
%\item $(\otR, \ke, \_) \in \opset_2 \wedge \forall (\otW, \ke, \val) \notin \opset_2$. 
%By \cref{lem:updatekv.explicit}\eqref{item:updatekv.explicit.rd} we have that 
%\begin{align}
%\nonumber\hh_2(\ke) = \updateKV(\hh, \vi_2, \txid_2) =\\ 
%\text{let } (\val_2, \txid_2', \T_2) = \hh(\ke, \vi_2) \text{ in } \hh(\ke)\rmto{\vi_2(\ke)}{(\val_2, \txid_2', \T_2 \cup \{\txid_2\})}
%\label{eq:updatekv.swap.rd.rd1}\\
%~\\
%\nonumber\updateKV(\hh_1, \vi_2, \txid_2, \opset_2)(\ke) = \\
%\nonumber \text{let } (\val_2, \txid_2', \T_2) = \hh_1(\ke, \vi_2) 
%\text{ in } \hh_1(\ke)\rmto{\vi_2(\ke)}{(\val_2, \txid_2', \T_2 \cup \{\txid_2\})}
%\label{eq:updatekv.swap.rd.rd2}
%\end{align}
%We now distinguish between two cases, according to whether $\vi_1(\ke) = \vi_2(\ke)$ or 
%$\vi_1(\ke) \neq \vi_2(\ke)$. 
%\begin{enumerate} 
%\item If $\vi_1(\ke) = \vi_2(\ke)$, then we have the following: 
%\begin{equation}
%\begin{array}{lr}
%\hh_1(\ke, \vi_2) = \hh_1(\ke, \vi_1) = \hh_1(\ke)(\vi_1(\ke))&\stackrel{\eqref{eq:updatekv.swap.rd11}}{=}\\
%\text{let } (\val_1, \txid_1', \T_1)  = \hh(\ke, \vi_1) \text{ in } 
%\left(\hh(\ke)\rmto{\vi_1(\ke)}{(\val_1, \txid_1', \T_1 \cup \{\txid_1\})}\right)(\vi_1(\ke)) &=\\
%\text{let } (\val_1, \txid_1', \T_1) = \hh(\ke, \vi_1) \text{ in } (\val_1, \txid_1', \T_1 \cup \{\txid_1\})
%\end{array}
%\label{eq:updatekv.swap.rd.rd.aux1}
%\end{equation}
%
%\begin{equation}
%\begin{array}{lr}
%\hh_2(\ke, \vi_1) = \hh_2(\ke, \vi_2) = \hh_2(\ke)(\vi_2(\ke)) &\stackrel{\eqref{eq:updatekv.swap.rd.rd2}}{=}\\
%\text{let } (\val_2, \txid_2', \T_2) = \hh(\ke, \vi_2) \text{ in } \left(\hh(\ke)\rmto{\vi_2(\ke)}{(\val_2, \txid_2', \T_2 \cup \{\txid_2\})}\right)(\vi_2(\ke)) &=\\
%\text{let } (\val_2, \txid_2', \T_2) = \hh(\ke, \vi_2) \text{ in } (\val_2, \txid_2', \T_2 \cup \{\txid_2\})
%\end{array}
%\label{eq:updatekv.swap.rd.rd.aux2}
%\end{equation}
%The two facts above can be used to prove the following:
%\[
%\begin{array}{lr}
%\updateKV(\hh_1, \vi_2, \txid_2, \opset_2)(\ke) \stackrel{\eqref{eq:updatekv.swap.rd.rd2}}{=}
%\big(\text{let } (\val_2, \txid_2', \T_2) = \hh_1(\ke, \vi_2) 
%\text{ in } \hh_1(\ke)\rmto{\vi_2(\ke)}{(\val_2, \txid_2', \T_2 \cup \{\txid_2\})}\big) &\stackrel{\eqref{eq:updatekv.swap.rd.rd.aux1}}{=} \\
%\text{let } (\val_2, \txid_2', \T_2) = ( \text{let } (\val_1, \txid_1', \T_1) = \hh(\ke, \vi_1) \text{in } (\val_1, \txid_1', \T_1 \cup \{\txid_1\}) ) &\\
%\hspace{10pt} \text{ in } \hh_1(\ke)\rmto{\vi_1(\ke)}{(\val_2, \txid_2', \T_2 \cup \{\txid_2\})} &=\\
%\text{let } (\val_1, \txid_1', \T_1) = \hh(\ke, \vi_1) \text{in } \hh_1(\ke)\rmto{\vi_1(\ke)}{(\val_1, \txid_1', \T_2 \cup \{\txid_1, \txid_2\})} &\\
%&\\
%\updateKV(\hh_2, \vi_1, \txid_1, \opset_1)(\ke) &\stackrel{\eqref{eq:updateKV.swap.rd12}}{=}\\
%\text{let } (\val_1, \txid'_1, \T_1) = \hh_2(\ke, \vi_1) \text{ in } \updateKV(\hh, \vi_2, \txid_2, \opset_2)(\ke)\rmto{\vi_1(\ke)}{(\val_1, \txid_1', \T_1 \cup \{\txid_1\})} 
%&\stackrel{\eqref{eq:updatekv.swap.rd.rd.aux2}}{=}\\
%\text{let } (\val_1, \txid'_1, \T_1) = ( \text{let } (\val_2, \txid_2', \T_2) = \hh(\ke, \vi_2) \text{ in } (\val_2, \txid_2', \T_2 \cup \{\txid_2\}) ) \text{ in } &\\
%\updateKV(\hh, \vi_2, \txid_2, \opset_2)(\ke)\rmto{\vi_1(\ke)}{(\val_1, \txid_1', \T_1 \cup \{\txid_1\})} &=\\
%\text{let } (\val_1, \txid'_1, \T_1) = \hh(\ke, \vi_2) \text{ in } &\\
%\updateKV(\hh, \vi_2, \txid_2, \opset_2)(\ke)\rmto{\vi_1(\ke)}{(\val_1, \txid_1', \T_1 \cup \{\txid_1, \txid_2\})} &\stackrel{(\ref{lem:updatekv.explicit}-\ref{item:updatekv.explicit.rd})}{=}\\
%\hh(\ke)\rmto{\vi_2(\ke)}{\_}\rmto{\vi_1(\ke)}{(\val_1, \txid_1', \T_1 \cup \{\txid_1, \txid_2\})} &=\\
%\hh(\ke)\rmto{\vi_1(\ke)}{\_}\rmto{\vi_1(\ke)}{(\val_1, \txid_1', \T_1 \cup \{\txid_1, \txid_2\})} &=\\
%\hh(\ke)\rmto{\vi_1(\ke)}{(\val_1,\txid_1', \T_1, \cup \{(\txid_1, \txid_2\})}
%\end{array}
%\]
%\end{enumerate}
%
%\end{itemize}
%%Let $\hh(\ke, \vi) = (\val_1, \txid_1', \T)$ for 
%%some $\val_1, \txid_1', \T_1$.
%%By \cref{lem:updatekv.explicit}\eqref{item:updatekv.explicit.rd}
%%it follows that 
%%\begin{align}
%%\hh_1(\ke) = \updateKV(\hh,\vi_1,\txid_1,\opset_1)(\ke) = \hh(\ke)\rmto{\vi_1(\ke)}{(\val_1, \txid_1', \T_1 \cup \{\txid_1\})}
%%\label{eq:updatekv.swap.rd11} \\
%%\nonumber \\
%%\nonumber \updateKV(\hh_2, \vi_1, \txid_1, \opset_1)(\ke) = \hh_2(\ke)\rmto{\vi_1(\ke)}{(\val_1, \txid_1', \T_1 \cup \{\txid_1\})} = \\
%%\updateKV(\hh, \vi_2, \txid_2, \opset_2)(\ke)\rmto{\vi_1(\ke)}{(\val_1, \txid_1', \T_1 \cup \{\txid_1\})}
%%\label{eq:updatekv.swap.rd12}
%%\end{align}
%%We perform a second analysis on $\opset_2$ 
%%\begin{itemize}
%%\item $\forall \val.(\otR, \ke, \val) \notin \opset_2 \wedge (\otW,\ke,\val) \notin \opset_2$. 
%%By \cref{lem:updatekv.explicit}\eqref{item:updatekv.explicit.none} we have that 
%%\[
%%\hh_2(\ke) = \updateKV(\hh, \vi_2, \txid_2, \opset_2)(\ke) = \hh(\ke),
%%\]
%%and 
%%\[\updateKV(\hh, \vi_2, \txid_2, \opset_2)(\ke) = \hh(\ke).
%%\] 
%%It follows that 
%%\[
%%\begin{array}{l}
%%\updateKV(\hh_1, \vi_2, \txid_2, \opset_2)(\ke) = \hh_1(\ke) \stackrel{\eqref{eq:updatekv.swap.rd11}}{=} 
%%\hh(\ke)\rmto{\vi_1(\ke)}{(\val_1, \txid_1', \T_1 \cup \{\txid_1\}}\\
%%~\\
%%\updateKV(\hh_2, \vi_1, \txid_1, \opset_1)(\ke) \stackrel{\eqref{eq:updatekv.swap.rd12}}{=} 
%%\updateKV(\hh, \vi_2, \txid_2, \opset_2)(\ke)\rmto{\vi_1(\ke)}{(\val_1, \txid_1', \T_1 \cup \{\txid_1\})} = \\
%%\hh(\ke)\rmto{\vi_1(\ke)}{(\val_1, \txid_1', \T_1 \cup \{\txid_1\})}
%%\end{array}
%%\]
%%\item $(\otR, \ke, \_) \in \opset_2$, and $\forall \val.\;(\otW,\ke,\val) \notin \opset_2)$. 
%%Let $(\val_2, \txid_2', \T_2)$. We need to distinguish between two cases, according to whether 
%%$\vi_1(\ke) = \vi_2(\ke)$ or $\vi_1(\ke) \neq \vi_2(\ke)$. 
%%\begin{enumerate}
%%\item if $\vi_1(\ke) = \vi_2(\ke)$,  recall that by Equation \eqref{eq:updatekv.swap.rd11} $\hh_1(\ke, \vi_2) = 
%%\hh_1(\ke, \vi_1) = (\val_1, \txid_1', \T_1 \cup \{\txid_1\})$. Recall also that $\hh(\ke, \vi_2) = \hh(\ke, \vi_1) = 
%%(\val_1, \txid_1', \T_1)$. 
%%By \cref{lem:updatekv.explicit}\eqref{item:updatekv.explicit.rd} we have the following: 
%%\begin{align}
%%\hh_2(\ke) = \updateKV(\hh, \vi_2, \txid_2, \opset_2)(\ke) = \hh(\ke)\rmto{(\vi_2(\ke)}{(\val_1, \txid_1', \T_1 \cup \{\txid_2\})}
%%\label{eq:updatekv.swap.rd1.rd21.eq}\\
%%\nonumber\\
%%\nonumber \updateKV(\hh_1, \vi_2, \txid_2, \opset_2)(\ke) = \hh_1(\ke)\rmto{(\vi_2(\ke)}{(\val_1, \txid_1', \T_1 \cup \{\txid_1\} \cup \{\txid_2\})} = \\
%%\updateKV(\hh, \vi_1, \txid_1, \opset_1)(\ke)\rmto{(\vi_2(\ke)}{(\val_1,\ txid_1', \T_1 \cup \{\txid_1, \txid_2\})}
%%\label{eq:updatekv.swap.rd1.rd22.eq}
%%\end{align}
%%In this case we have that
%%\[
%%\begin{lr}
%%\updateKV(\hh_1, \vi_2, \txid_2, \opset_2)(\ke) \stackrel{\eqref{eq:updatekv.swap.rd1.rd22}}{=} 
%%\updateKV(\hh, \vi_1, \txid_1,\opset_1)(\ke)\rmto{(\vi_2(\ke)}{(\val_1, \txid_1', \T_1 \cup \{\txid_1, \txid_2\})} &\stackrel{\eqref{eq:updatekv.swap.rd11}}{=}\\
%%(\hh(\ke)\rmto{\vi_1(\ke)}{(\val_1, \txid_1', \T_1 \cup \{\txid_1\})})\rmto{(\vi_2(\ke)}{(\val_1, \txid_1', \T_1 \cup \{\txid_1, \txid_2\})} &=\\
%%\hh(\ke)|rmto{(\vi_2(\ke)}{(\val_1, \txid_1', \T_1 \cup \{\txid_1, \txid_2\})}\\
%%&\\
%%\updateKV(\hh_2, \vi_1, \txid_1,\ opset_1)(\ke) 
%%\end{lr}
%%\]
%%\end{enumerate}
%%
%%be such that $\hh_1(\vi_2(\ke)) = (\val_2, \txid_2', \T_2)$. 
%%By \cref{lem:updatekv.explicit}\eqref{item:updatekv.explicit.rd} we have the following: 
%%
%%We distinguish between two cases, according to whether $\vi_1(\ke) = \vi_2(\ke)$ or $\vi_1(\ke) \neq \vi_2(\ke)$.
%%
%%\end{itemize}
%\end{itemize}
%\end{proof}
%%
%%\begin{proof}
%%By induction on $\opset_1$. 
%%\begin{itemize}
%%\item Base case: $\opset_1 = \emptyset$. Then we have that 
%%\[
%%\begin{array}{lr}
%%\mathtt{let } \hh_1 = \updateKV(\hh, \vi_1, \txid_1, \emptyset) \mathtt{ in } \updateKV(\hh_1, \vi_2, \txid_2, \opset_2) &= \\
%%\mathtt{let } \hh_1 = \hh \mathtt{ in } \updateKV(\hh_1, \vi_2, \txid_2, \opset_2) &= \\
%%\updateKV(\hh, \vi_2, \txid_2, \opset_2)
%%\end{array}
%%\]
%%Similarly, we have that 
%%\[
%%\begin{array}{lr}
%%\mathtt{let } \hh_2 = \updateKV(\hh, \vi_2, \txid_2, \opset_2) \mathtt{ in } \updateKV(\hh_2, \vi_1, \txid_1, \opset_1) &=\\
%%\mathtt{let } \hh_2 = \updateKV(\hh, \vi_2, \txid_2, \opset_2) \mathtt{ in } \hh_2 &=\\
%%\updateKV(\hh, \vi_2, \txid_2, \opset_2)
%%\end{array}
%%\]
%%\item Let $\opset_1 = \opset_1' \uplus \{(\otR, \ke_1, \val_1)\}$, and assume that for all $\hh'$,
%%\begin{equation}
%%\label{eq:updatekv.comm.rd}
%%\begin{array}{l}
%%\mathtt{let } \hh_1' = \updateKV(\hh', \vi_1, \txid_1, \opset'_1) \mathtt{ in } \updateKV(\hh_1, \vi_2, \txid_2, \opset_2) = \\
%%\mathtt{let } \hh_2 = \updateKV(\hh', \vi_2, \txid_2, \opset_2) \mathtt{ in } \updateKV(\hh_2, \vi_1, \txid_1, \opset'_1).
%%\end{array}
%%\end{equation}
%%We perform an inner induction on the structure of $\opset_2$. 
%%\begin{itemize}
%%\item Base case: $\opset_2 = \emptyset$. 
%%In this case we have that 
%%\[
%%\begin{array}{l}
%%\mathtt{let } \hh_1 = \updateKV(\hh, \vi_1, \txid_1, \opset_1) \mathtt{ in } \updateKV(\hh_1, \vi_2, \txid_2, \emptyset) = \\
%%\mathtt{let } \hh_1 = \updateKV(\hh, \vi_1, \txid_1, \opset_1) \mathtt{ in } \hh_1 = \\
%%\updateKV(\hh, \vi_1, \txid_1, \opset_1)
%%\end{array}
%%\]
%%and 
%%\[
%%\begin{array}{l}
%%\mathtt{let } \hh_2 = \updateKV(\hh, \vi_2, \txid_2, \emptyset) \mathtt{ in } \updateKV(\hh_2, \vi_1, \txid_1, \opset_1) = \\
%%\mathtt{let } \hh_2 = \hh \mathtt{ in } \updateKV(\hh, \vi_1, \txid_1, \opset_1) = \\
%%\updateKV(\hh, \vi_1, \txid_1, \opset_1).
%%\end{array}
%%\]
%%There is nothing left to prove in this case.
%%\item Suppose that $\opset_2 = \opset_2' \uplus \{(\otR, \ke_2, \val_2)\}$.
%%\end{itemize}
%%
%%Let $\hh_1 = \updateKV(\hh, \vi_1, \txid_1, \opset_1)$. Let also $(\val, \txid', \T) = \hh(\ke, \vi_1)$, 
%%and $V_{\ke} = \hh_1(\ke)$. By definition, we have that 
%%%\[
%%%\hh_1 = \updateKV(\hh\rmto{\ke}{\V_{ke}\rmto{\vi_1(\ke)}{(\val, \txid', \T \cup \{ \txid_1\}}}, \vi_1, \txid_1, \opset'_1).
%%%\]
%%
%%\end{itemize}
%%\end{proof}

\subsection{Proof of \cref{prop:mono-et} }
\label{sec:mono-et}
It is sufficient to prove that \(\ET_1 \subseteq \ET_2 \implies \Confs(\ET_1) \subseteq\ Confs(\ET_2) \).
We prove it by induction on the length of the traces, \( n \).

\caseB{n = 0}
We have \( \conf_0 \in \Confs(\ET_1) \) and \( \conf_0 \in \Confs(\ET_2)\).
\caseI(n = i + 1)
Suppose identical traces of \( \ET_1 \) and \( \ET_2 \) respectively with length \( i \).
Let the final configuration be \( \conf_i = ( \mkvs_i, \viewFun_i ) \).
If the next step is a view shift or a step with empty fingerprint, it trivially holds.
If the next step is a step by a client \( \cl \) with fingerprint \( \f \),
we have \( \ET_1 \vdash \mkvs_i, \viewFun_i(\cl) \csat \f : vi' \).
The next configuration from \( \ET_1 \) is \( \conf_{i+1} = (\updateKV{ \mkvs_i, \viewFun_i(\cl), \f, \txid_\cl}) \).
Since \( \ET_1 \subseteq \ET_2 \), so \( \ET_1 \vdash \mkvs_i, \viewFun_i(\cl) \csat \f : vi' \) holds.
It is possible for \( \ET_2 \) to have the exactly same next configuration \( \conf_{n+1}\).

\subsection{Proof of \cref{prop:et.normalform}}
\label{sec:normal-form-exist}
\ac{Note to self - Results needed: (i) you can merge two consecutive view shifts from the same client into a single action. (ii) 
you can remove commit of empty-fingerprints from a sequence of reductions - not true for consistency models 
without monotonic reads. (iii) You can push a view 
shift of client $\cl$ - i.e. an action of the form $(\cl, \varepsilon)$ to the right, as long as the action that immediately precedes it has not the form 
$(\cl, \_)$. (Note that you cannot move a view-shift to the right of another view-shift from the same 
client, but you can merge them together and keep pushing the resulting view-shift to the left.}
Throughout this section, we assume that the execution test $\ET$ is fixed.

\begin{lemma}[Absorption]
\label{lem:et.absorb}
If $\conf \xrightarrowtriangle{(\cl, \varepsilon)}_{\ET} \conf' \xrightarrowtriangle{(\cl, \varepsilon)} \conf''$, then 
$\conf \xrightarrowtriangle{(\cl, \varepsilon)}_{\ET} \conf''$.
\end{lemma}

\begin{proof}
Let $\conf = (\hh, \viewFun)$, $\conf' = (\hh', \viewFun')$, $\conf'' = (\hh', \viewFun'')$. 
By \cref{def:reductions} it must be the case that $\hh = \hh'$, and $\viewFun' = \viewFun\rmto{\cl}{\vi'}$ 
for some $\vi' : \vi \sqsubseteq \vi'$. It must also be the case that $\hh' = \hh''$, and $\viewFun'' = \viewFun'\rmto{\cl}{\vi''}$ 
for some $\vi'': \vi' \sqsubseteq \vi''$. Therefore we have that $\hh'' = \hh' = \hh$, and 
$\viewFun'' = \viewFun'\rmto{\cl}{\vi''} = (\viewFun\rmto{\cl}{\vi'})\rmto{\cl}{\vi''} = viewFun\rmto{\cl}{\vi''}$, 
and $\vi \sqsubseteq \vi''$. By \cref{def:reductions}, it follows that 
$\conf = (\hh, \viewFun) \xrightarrowtriangle{(\cl, \varepsilon)} (\hh'', \viewFun'') = \conf''$.
\end{proof}

\begin{lemma}
\label{lem:viewshift.rightmover}
Let $\conf \xrightarrowtriangle{(\cl, \varepsilon)}_{\ET} \conf_1 \xrightarrowtriangle{(\cl', \mu)}_{\ET} \conf'$ 
for some $\conf, \conf_1, \conf''$ and $\cl, \cl'$ such that $\cl' \neq \cl$. 
Then $\conf \xrightarrowtriangle{(\cl', \mu)}_{\ET} \conf_2 \xrightarrowtriangle{(\cl, \varepsilon)}_{\ET} \conf'$ 
\end{lemma}

\begin{proof}
We only consider the case where $\mu = \opset$ for some fingerprint $\opset$. The case where 
$\mu = \varepsilon$ is simpler to prove.
Let $\conf = (\hh, \viewFun)$, $\conf_1 = (\hh_1, \viewFun_1)$, $\conf' = (\hh', \viewFun')$. 
Let also $\vi = \viewFun(\cl)$.
By \cref{def:reductions} we have that $\hh_1 = \hh, \viewFun_1 = \viewFun\rmto{\cl}{\vi_1}$ for 
some $\vi_1: \vi_1 \sqsubseteq \vi_1$. Let $\vi' = \viewFun(\cl')$: then we have that $\viewFun_1(\cl') = 
\vi'$. Because $(\hh_1, \viewFun_1) \xrightarrowtriangle{(\cl', \opset)}_{\ET} (\hh', \viewFun')$, we have that 
$\ET \vdash \hh_1, \vi' \triangleright \opset : \vi''$, where $\vi'' = \viewFun'(\cl')$. Because $\hh_1 = \hh$, 
that means that $\ET \vdash \hh, \vi' \triangleright \opset: \vi''$: by \cref{def:reductions} it follows that 
$(\hh, \viewFun) \xrightarrowtriangle{(\cl', \opset)}_{\ET} (\hh', \viewFun\rmto{\cl'}{\vi''}) 
\xrightarrowtriangle{(\cl, \varepsilon)}_{\ET} (\hh', \viewFun\rmto{\cl'}{\vi''}\rmto{\cl}{\vi_1}) = 
(\hh', \viewFun\rmto{\cl}{\vi_1}\rmto{\cl'}{\vi''}) = (\hh', \viewFun_1\rmto{\cl'}{\vi''}) = 
(\hh', \viewFun')$, as we wanted to prove.
\end{proof}

\mypar{Proof of \cref{prop:et.normalform}}
Let $\hh \in \CMs(\ET)$. By definition, there exists a sequence of reductions 
\begin{equation}
\label{eq:normalform.sequence}
(\hh_{0}, \viewFun_{0}) \xrightarrowtriangle{(\cl_0, \mu_0)}_{\ET} \cdots \xrightarrowtriangle{(\cl_{n-1}, \mu_{n-1})}_{\ET} (\hh_n, \viewFun_{n})
\end{equation}
such that $\hh_{n} = \hh$. Given an index $i = 1,\cdots, n-1$, we say that the action $(\cl_{i}, \mu_{i})$ is \emph{in place} 
if, $\mu_{i} = \opset_{i}$ for some $\opset_{i}$, $\cl_{i-1} = \cl_{i}$, $\mu_{i-1} = \varepsilon$, and if $(\cl_{j}, \mu_{j}) = (\cl_{i}, \varepsilon)$, 
for some  $j = 0,\cdots, i-2$, then there exists $j': j < j' < i$ such that $(\cl_{j'}, \mu_{j'}) = (\cl_i, \opset_{j'})$. An action of the 
form $(\cl_{i}, \mu_{i})$ is \emph{out of place} if it is not in place. 

Given the sequence of reductions in \cref{eq:normalform.sequence}, we show the following: 
\begin{enumerate}
\item if the sequence has no action out of place, then there exists a sequence 
\[
(\hh'_{0}, \viewFun'_{0}) \xrightarrowtriangle{(\cl'_{0}, \mu'_{0})}_{\ET} \cdots \xrightarrowtriangle{(\cl'_{m-1}, \mu'_{m-1})}_{\ET} (\hh'_{m}, \viewFun'_{m})
\]
that is in normal form, and such that $\hh'_{m} = \hh_{n}$, and 
\item if the sequence has $h$ actions out of place, for some $h > 0$, then there exists a sequence 
\[
(\hh'_{0}, \viewFun'_{0}) \xrightarrowtriangle{(\cl'_{0}, \mu'_{0})}_{\ET} \cdots \xrightarrowtriangle{(\cl'_{m-1}, \mu'_{m-1})}_{\ET} (\hh'_{m}, \viewFun'_{m})
\]
that has $h-1$ actions out of place, and such that $\hh'_{m} = \hh_{n}$.
\end{enumerate}
Combining the two facts above, we obtain that if $\hh \in \CMs(\ET)$, then there exists a sequence of reductions in formal form whose final 
configuration is $(\hh, \_)$, as we wanted to prove.

\begin{enumerate}
\item Suppose that the sequence of reductions from \cref{eq:normalform.sequence} has no action out of place. 
Let $i=0,\cdots, n-1$, and consider the greatest index $i=0,\cdots, n-1$ such that  
$\mu_{i} = \varepsilon$, and either $i = n-1$, or 
$\forall \opset.\; (\cl_{i+1}, \mu_{i+1}) \neq (\cl_{i}, \opset)$. 
If such an index does not exist, then the sequence of transitions from \cref{eq:normalform.sequence} is in 
normal form, and there is nothing to prove. Otherwise, note that for any $j = i+1,\cdots, n-1$, 
$\forall \opset.\;(\cl_{j}, \mu_{j}) \neq (\cl_{i}, \opset)$. 

Suppose in fact that there existed 
an index $j = i+1,\cdots, n-1$ such that $(\cl_{j}, \mu_{j}) = (\cl_{i}, \opset_{j})$ for some 
$\opset_{j}$, and without loss of generality assume that $j$ is the smallest such index. This implies that 
there exists no index $j': i < j' < j$ such that $(\cl_{j'}, \mu_{j'}) = (\cl_{i}, \opset_{j'})$ for some 
$\opset_{j'}$. Also, it cannot be $j = i+1$, because we are assuming that $\forall \opset.\;(\cl_{i+1}, \mu_{i+1}) \neq 
(\cl_{i}, \opset)$.  We have that $j \geq i+2$; we also have that  $(\cl_{j}, \mu_{j}) = (\cl_{i}, \opset_{j})$, 
$(\cl_{i}, \mu_{i}) = (\cl_{i}, \varepsilon)$, $\forall j': i < j < j'.\forall \opset.\; (\cl_{j'}, \mu_{j'}) \neq (\cl_{i}, \opset)$. 
By definition, the action $(\cl_{j}, \mu_{j})$ is out of place, contradicting the assumption that the sequence of 
reduction of \cref{eq:normalform.sequence} has no actions out of place.

We have proved that $\forall j = i+1, \cdots, n-1.\;\forall \opset.\;(\cl_{j}, \mu_{j}) \neq (\cl_{i}, \opset)$. 
Also, because we are assuming that $\mu_{i}$ is the greatest index such that $\mu_{i} = \varepsilon$, 
and either $i= n-1$ or $\forall \opset.\;(\cl_{i+1},\mu_{i+1}) \neq (\cl_{i}, \opset)$, 
then $\forall j=i+1,\cdots, n-1\;\forall \mu.\;(\cl_{j}, \mu_{j}) \neq (\cl_{i}, \mu)$. 
Consider the transition 
\[
(\hh_{i}, \viewFun_{i}) \xrightarrowtriangle{(\cl_{i}, \mu_{i})}_{\ET} (\hh_{i+1}, \viewFun_{i+1}).
\]
Let $\vi = \viewFun_{i}(\cl)$. Because $\mu_{i} = \varepsilon$, then it must be the case that 
$\hh_{i} = \hh_{i+1}$, $\viewFun_{i+1} = \viewFun_{i}\rmto{\cl}{\vi'}$ for some $\vi' : \vi \sqsubseteq \vi'$. 
For any $j \geq i$, we have that $\cl_{j} \neq \cl_{i}$. We can replace the transition 
\[
(\hh_{j}, \viewFun_{j}) \xrightarrowtriangle{(\cl_{j}, \mu_{j})}_{\ET} (\hh_{j+1}, \viewFun_{j+1})
\]
with 
\[
(\hh_{j}, \viewFun_{j}\rmto{\cl_{i}}{\vi}) \xrightarrowtriangle{(\cl_{j}, \mu_{j})}_{\ET} (\hh_{j+1}, \viewFun_{j+1}\rmto{\cl_{i}}{\vi}.
\]
It follows that the sequence of transitions 
\[ 
(\hh_{0}, \viewFun_{0}) \xrightarrowtriangle{(\cl_{0}, \mu_{0})}_{\ET} \cdots \xrightarrowtriangle{(\cl_{i-1},\mu_{i-1})} 
(\hh_{i}, \viewFun_{i}) = (\hh_{i+1}, \viewFun_{i+1}\rmto{\cl_{i},\vi}) \xrightarrowtriangle{(\cl_{i+1}, \mu_{i+1})}_{\ET} \cdots 
\xrightarrowtriangle{(\cl_{n-1}, \mu_{n-1})}_{\ET} (\cl_{n}, \viewFun_{n}\rmto{\cl_{i}, \vi})
\]
Note that this sequence has one reduction less than the original sequence from \eqref{eq:normalform.sequence} (specifically, 
the reduction $(\hh_{i}, \viewFun_{i}) \xrightarrowtriangle{(\cl_{i}, \mu_{i})} (\hh_{i+1}, \viewFun_{i+1})$ has 
been removed). We can repeat this procedure until the resulting sequence of reductions has no index $i=0,\cdots, n-1$ such that  
$\mu_{i} = \varepsilon$, and either $i = n-1$, or 
$\forall \opset.\; (\cl_{i+1}, \mu_{i+1}) \neq (\cl_{i}, \opset)$. That is, the resulting sequence of reductions is in normal form, 
and its final configuration is $(\hh_{n}, \_)$.

\item Suppose that the sequence from \cref{eq:normalform.sequence} has $h$ actions out of place, 
where $h > 0$. Let $i$ be the smallest index such that $(\cl_{i}, \mu_{i})$ is out of place. 
This means that either $i = 0$, or $(\cl_{i-1}, \mu_{i-1}) \neq (\cl_{i}, \varepsilon)$, 
or there exists an index $j < i -1 $ such that $(\cl_{j}, \mu_{j}) = (\cl_{i}, \varepsilon)$ 
and, $\forall j': j < j' < i.\;\forall \opset.\;(\cl_{j'}, \mu_{j'}) \neq (\cl_{i}, \opset)$. 
Without loss of generality, we can assume that $i \neq 0$ and $(\cl_{i-1}, \mu_{i-} = (\cl_{i}, \varepsilon)$. 
This is because we can always transform the sequence of reductions of \cref{eq:normalform.sequence} by 
introducing a transition of the form $(\hh_{i}, \viewFun_{i}) \xrightarrowtriangle{(\cl_{i}, \varepsilon)}_{\ET}
(\hh_{i}, \viewFun_{i})$, leading to the sequence of reductions
\[
(\hh_{0}, \viewFun_{0}) \xrightarrowtriangle{(\cl_{0}, \mu_{0})}_{\ET} \cdots \xrightarrowtriangle{(\cl_{i-1}, \mu_{i-1})}_{\ET}
(\hh_{i}, \viewFun_{i}) \xrightarrowtriangle{(\cl_{i}, \varepsilon)}_{\ET} (\hh_{i}, \viewFun_{i}) \xrightarrowtriangle{(\cl_{i+1}, \mu_{i+1})}_{\ET} 
\cdots \xrightarrowtriangle{(\cl_{n-1}, \mu_{n-1})}_{\ET} (\hh_{n}, \viewFun_{n}).
\]

Therefore, it must be the case that there exists an index $j < i-1$ such that $(\cl_{j}, \mu_{j}) = (\cl_{i}, \varepsilon)$, 
and $\forall j': j< j' < i.\;\forall \opset.\;(\cl_{j'}, \mu_{j'}) \neq (\cl_{i}, \opset)$. Let then $j$ be the smallest such index. 
Let $d = (i-1)-j$ be the number of reductions that separate the configuration $(\hh_{j}, \mu_{j})$ from 
$(\hh_{i-1}, \mu_{i-1})$ in \cref{eq:normalform.sequence}. Note that it must be the case that $d > 0$. We show that we can 
construct a sequence of reductions where the distance between these two configurations is reduced to $0$: 
a consequence of this fact is such a sequence of reductions would have exactly $h-1$ actions out of place.
Consider the following fragment in the sequence of reductions from \cref{eq:normalform.sequence}:
\[
(\hh_{j}, \viewFun_{j}) \xrightarrowtriangle{(\cl_{j}, \mu_{j})}_{\ET} (\hh_{j+1}, \viewFun_{j+1}) 
\xrightarrowtriangle{(\cl_{j+1}, \mu_{j+1})}_{\ET} (\hh_{j+2}, \viewFun_{j+2}).
\]
We have two possible cases: 
\begin{itemize}
\item $\cl_{j+1} \neq \cl_{j}$. In this case we can apply \cref{lem:viewshift.rightmover} and infer the sequence of 
reductions 
\[
(\hh_{j}, \viewFun_{j}) \xrightarrowtriangle{(\cl_{j+1}, \mu_{j+1}}_{\ET} (\hh_{j+1}', \viewFun_{j+1}') 
\xrightarrowtriangle{(\cl_{j}, \mu_{j})}_{\ET} (\hh_{j+1}, \viewFun_{j+1}).
\]
which leads to the whole sequence of reductions 
\[
(\hh_{0}, \viewFun_{0}){(\cl_{0}, \mu_{0})}_{\ET} \cdots 
\xrightarrowtriangle{(\cl_{j-1}, \mu_{j-1})}_{\ET} (\cl_{j}, \mu_{j}) \xrightarrowtriangle{(\cl_{j+1}, \mu_{j+1})}_{\ET} 
(\hh'_{j+1}, \viewFun'_{j+1}) \xrightarrowtriangle{(\cl_{j}, \mu_{j})}_{\ET} (\hh_{j+1}, \viewFun_{j+1}) 
\xrightarrowtriangle{(\cl_{j+2}, \mu_{j+2})}_{\ET} \cdots \xrightarrowtriangle{(\cl_{n-1}, \mu_{n-1})} \hh_{n}, \viewFun_{n}.
\]
\item $\cl_{j+1} = \cl_{j}$. In this case we can apply \cref{lem:et.absorb} and infer the reduction 
\[
(\hh_{j}, \viewFun_{j}) \xrightarrowtriangle{(\cl_{j}, \varepsilon)}_{\ET} (\hh_{j+2}, \viewFun_{j+2}),
\]
which leads to the sequence of reductions 
\[
(\hh_{0}, \viewFun_{0}){(\cl_{0}, \mu_{0})}_{\ET} \cdots 
\xrightarrow{(\cl_{j-1}, \mu_{j-1})}_{\ET} (\cl_{j}, \mu_{j}) \xrightarrowtriangle{(\cl_{j}, \varepsilon)}_{\ET} 
(\hh_{j+2}, \viewFun_{j+2}) \xrightarrowtriangle{(\cl_{j+2}, \mu_{j+2})}_{\ET} \cdots 
\xrightarrowtriangle{(\cl_{n-1}, \mu_{n-1})}_{\ET} (\hh_{n}, \viewFun_{n}).
\]
\end{itemize}
In both cases, in the resulting sequence of reductions the number of reductions that separate 
the configuration $(\hh_{j}, \viewFun_{j})$ from $(\hh_{i-1}, \viewFun_{i-1})$ is strictly 
less than $d$. We can repeating applying the procedure outlined above until there are 
no reductions that separate the configuration $(\hh_{j}, \viewFun_{j})$ from 
$(\hh_{i}, \viewFun_{i})$.
\end{enumerate}
\ac{This was more of a proof sketch, rather than a real proof. For the moment it will suffice, though 
I will need to go back at it when all the other results are sorted.}

\subsection{Compositionality of \( \ET \)}
\label{sec:et-comm}
\label{sec:et-comp}

To make two execution tests \( \ET_1 \) \( \ET_2 \) compositional with respect to to function \( \CMs \),
they need to satisfy \cref{def:conflict-commit,def:noblidwrites,def:et-minimum-footprint,def:et-monotonic-postview}.
For all the definitions we have in \cref{fig:execution.tests},
It is easy to adapt so that they satisfy \cref{def:noblidwrites,def:et-minimum-footprint,def:et-monotonic-postview},
but \( \CP \) and \( \SI \) cannot be adapted so to satisfy \cref{def:conflict-commit}.
Now we can prove compositionality of \( \ET \) (\cref{thm:et-comm}).

\begin{definition}[matching pre-views]
Two executions $\ET_1$ and $\ET_2$ have matching pre-views,

\end{definition}

\begin{theorem}                                                                            
\label{thm:et-comm}                          
Let $\ET_1, \ET_2$ be two execution tests has no blind writes, minimum footprints and monotonic post-views.
If $\ET_1$ is commutative, 
then $\CMs(\ET_1 \cap \ET_2) = \CMs(\ET_1) \cap \CMs(\ET_2)$. 
Furthermore, if $\ET_1, \ET_2$ are commutative, then $\ET_1 \cap \ET_2$ 
is commutative.
\end{theorem}
\begin{proof}
Given the definition of the \( \CMs(.) \) function (\cref{def:cm}), 
it suffices to prove that \( \CMs(\ET_{1} \cap \ET{2}) \subseteq \CMs(\ET_1) \cap \CMs(\ET_2) \)
and \( \CMs(\ET_1) \cap \CMs(\ET_2) \subseteq \CMs(\ET_{1} \cap \ET{2}) \).
The former is proven by the \cref{lem:et12-in-et1-et2} and the later is proven by \cref{lem:et1-et2-in-et12}.
\end{proof}

\begin{lemma}
\label{lem:et12-in-et1-et2}
\( \CMs(\ET_{1} \cap \ET_{2}) \subseteq \CMs(\ET_1) \cap \CMs(\ET_2) \).
\end{lemma}
\begin{proof}
It suffices to prove a stronger result that \( \confOf[\ET_{1} \cap \ET_{2}] \subseteq \confOf[\ET_1] \cap \confOf[\ET_2] \).
By the definition of \confOf (\cref{def:cm}), it suffices to prove for configurations \( \conf_0 \) to \( \conf_n \) 
\begin{equation}
    \label{equ:et12-in-et1-et2}
    \begin{array}{@{}l}
        \conf_0 \in \Confs_0
    \land \conf_0 \toET{\stub}[\ET_1 \cap \ET_2] \cdots \toET{\stub}[\ET_1 \cap \ET_2] \conf_n \implies {} \\
    \quad \conf_0 \toET{\stub}[\ET_{1}] \cdots \toET{\stub}[\ET_{1}] \conf_n \land \conf_0 \toET{\stub}[\ET_{2}] \cdots \toET{\stub}[\ET_{2}] \conf_n 
    \end{array}
\end{equation}
We prove the \cref{equ:et12-in-et1-et2} by induction on the number \( n \).
\begin{itemize}
\item Base case: \(n = 0\). 
The \cref{equ:et12-in-et1-et2} holds when \( n = 0 \), because all initial configurations \( \conf_0 \) are included in the \( \confOf[\ET_1]\) and \( \confOf[\ET_2] \) by the definition of the \( \confOf \) function (\cref{def:cm}).

\item Inductive case: \(n = i+1\). Suppose the \cref{equ:et12-in-et1-et2} holds when \( n = i \) for some \( i \).
Let consider \( n = i + 1 \) and specifically the last step.
For any \( \conf_{i+1} = (\mkvs_{i+1}, \vienv_{i+1}) \) induced by \( \ET_{1} \cap \ET_2 \), 
there exist some client \( \cl \), views \( \vi, \vi' \) and fingerprint \( \fp \) such that:
\[
    \begin{array}{l}
    (\mkvs_i, \vienv_i) \toET{\cl, \fp}[\ET_{1} \cap \ET_{2}] (\mkvs_{i+1}, \vienv_{i+1}) 
    \land \vienv_{i+1} = \vienv_{i}\rmto{\cl}{\vi'} \land (\mkvs_i, \vi, \fp, \vi' ) \in \ET_{1} \cap \ET_{2}
    \end{array}
\]
Thus, it is easy to see that \( \conf_i \toET{\cl, \fp}[\ET_{1}] \conf_{i+1} \) and \( \conf_i \toET{\cl, \fp}[\ET_{2}] \conf_{i+1} \) by the \cref{lem:mono-et}.
\end{itemize}
\end{proof}

\begin{lemma}
\label{lem:mono-et}
If $\conf \toET{\cl, \fp}[\ET] \conf'$ and $\ET \subseteq \ET'$, 
then $\conf \toET{\cl, \fp}[\ET'] \conf'$.
\end{lemma}
\begin{proof}
    Let \((\mkvs, \vienv)  = \conf \), \( (\mkvs', \vienv') = \conf' \) and \( \vi  =\vienv(\cl) \)
    By the definition of  $\conf \toET{\cl, \fp}[\ET] \conf'$ (\cref{def:cm}), we have \(\mkvs' \in \updateKV[\mkvs, \vi, \fp, \cl]\) and  \( \vienv' = \vienv\rmto{\cl}{\vi'} \) for some \( \vi' \) such that \( \ET \vdash (\mkvs, \vi) \csat \fp : (\mkvs',\vi') \).
    Given that \( \ET \subseteq \ET'\), we know \( \ET' \vdash (\mkvs, \vi) \csat \fp : (\mkvs',\vi') \) and so $\conf \toET{\cl, \fp}[\ET'] \conf'$.
\end{proof}

\begin{lemma}
\label{lem:et1-et2-in-et12}
\( \CMs(\ET_1) \cap \CMs(\ET_2) \subseteq \CMs(\ET_{1} \cap \ET_{2}) \).
\end{lemma}
\begin{proof}
    By the definition of \( \CMs\) and \( \confOf \) (\cref{def:cm}), we prove a stronger result that
    for an initial configuration \( \conf_0 \), 
    configurations \( \conf_1 \) to \( \conf_n \) from trace \( \ET_1 \), 
    configurations \( \conf'_1 \) to \( \conf'_m \) from trace \( \ET_2 \),
    \[
    \begin{array}{@{}l}
    \conf_0 \toET{\stub}[\ET_{1}] \conf_1 \toET{\stub}[\ET_{1}] \cdots \toET{\stub}[\ET_{1}] \conf_n 
    \land \conf_0 \toET{\stub}[\ET_{2}] \conf'_1 \toET{\stub}[\ET_{2}]  \cdots \toET{\stub}[\ET_{2}] \conf'_m 
    \land \conf_n\projection{1} = \conf'_m\projection{1} \\
    \end{array}
    \]
    there exists configurations from \( \conf''_1\)  to \( \conf''_k \) from trace \( \ET_1 \cap \ET_2 \):
\begin{equation}
    \label{equ:et1-et2-in-et12}
    \begin{array}{@{}l}
    \conf_0 \toET{\stub}[\ET_1 \cap \ET_2] \conf''_1 \toET{\stub}[\ET_1 \cap \ET_2] \cdots \toET{\stub}[\ET_1 \cap \ET_2] \conf''_k 
    \land \conf_n\projection{1} = \conf'_m\projection{1} = \conf''_k\projection{1}  \\
    \quad {} \land \fora{\cl \in \dom(\conf''_k\projection{2}),\key \in (\conf''_k\projection{1})}
    \conf''_k\projection{2}(\cl)(\key) = \max\Set{\conf_n\projection{2}(\cl)(\key), \conf'_m\projection{2}(\cl)(\key)}
    \end{array}
\end{equation}
We prove \cref{equ:et1-et2-in-et12} by induction on the length \( m \) of the trace of \( \ET_2 \).
\begin{itemize}
    \item \caseB{\(m = 0\)}
We have the trace of \( \ET_1 \):
\begin{equation}
    \label{equ:trace-view-shift-et1}
    \conf_0 \in \Confs_0 \land \conf_0 \toET{\stub}[\ET_1] \dots \toET{\stub}[\ET_1] \conf_n
\end{equation}
for some number \( n \) and configurations from \( \conf_0 \) to \( \conf_n \) and the trace of \( \ET_2 \) with only one configuration:
\begin{equation}
    \label{equ:trace-singleton-et2}
    \conf_0
\end{equation}
By the hypothesis we have \( \conf_0\projection{1} = \conf_n\projection{1} \), which means that all the steps from the trace of \( \ET_1 \) are view shift.
We can pick the trace of \( \ET_1 \) (\cref{equ:trace-view-shift-et1}) as the trace of \( \ET_1 \cap \ET_2 \):
\begin{equation}
    \label{equ:trace-view-shift-et12}
    \conf_0 \toET{\stub}[\ET_1 \cap \ET_2] \dots \toET{\stub}[\ET_1 \cap \ET_2] \conf''_k \land  k = n \land \bigwedge_{ 0 < i \leq k} \conf_i = \conf''_i
\end{equation}
It is easy to see:
\begin{equation}
    \label{equ:max-et1-et2}
    \begin{array}{l}
    \fora{\cl \in \dom(\conf_k\projection{2}), \key \in \dom(\conf_k\projection{1})} 
    \conf_0\projection{2}(\cl)(\key) = \max\Set{\conf_0\projection{2}(\cl)(\key), \conf_n\projection{2}(\cl)(\key)}
\end{array}
\end{equation}
Combine \cref{equ:trace-view-shift-et12} and \cref{equ:max-et1-et2}, we prove the \cref{equ:et1-et2-in-et12}.

\item \caseI{\(m = i + 1\)}
Suppose that \cref{equ:et1-et2-in-et12} holds when \( m = i \).
Let consider \( m = i + 1 \).
We have the trace for \( \ET_1 \):
\begin{equation}
    \conf_0 \toET{\stub}[\ET_{1}] \conf_1 \toET{\stub}[\ET_{1}] \cdots \toET{\stub}[\ET_{1}] \conf'_{n} 
\end{equation}
for some number \( n \) and the configurations from \(\conf_0\) to \( \conf_n \), and the trace of \(\ET_2\):
\begin{equation}
    \conf_0 \toET{\stub}[\ET_{2}] \conf'_1 \toET{\stub}[\ET_{2}] \cdots \toET{\stub}[\ET_{2}] \conf'_{i+1} 
\end{equation}
It is safe to assume these two traces are in normal form by \cref{prop:et.normalform}.
Assume a client \( \cl'_{i} \), views \( \vi'_{i}, \vi'_{i+1} \) and a fingerprint \( \fp'_{i} \) that commit to the second last configuration \( (\mkvs'_i, \vienv'_i) = \conf'_i \) in the trace of \( \ET_2 \) which yields the final configuration \( (\mkvs'_{i+1}, \vienv'_{i+1}) = \conf_{i+1} \):
\begin{equation}
    \label{equ:last-et-2}
    \begin{array}{@{}l @{}}
        (\mkvs'_i, \vienv'_i) \toET{\cl'_{i}, \fp'_{i}}[\ET_2] (\mkvs'_{i+1}, \vienv'_{i+1}) \land \ET_2 \vdash (\mkvs'_i, \vi'_i) \csat \fp'_i  : (\mkvs'_{i+1},\vi'_{i+1}) \\
        \quad {} \land \vi' = \vienv'_i(\cl'_i) \land \vienv'_{i+1} = \vienv'_i\rmto{\cl'_i}{\vi'_{i+1}}
    \end{array}
\end{equation}
There are three cases: \textbf{(i)} \( \fp'_i = \unitO \), \textbf{(i)} \( \fp'_i = \epsilon \) and \textbf{(ii)} \( \fp'_i \neq \unitO \land \fp'_i \neq \epsilon \).
\begin{itemize}
    \item If \( \fp'_i = \epsilon \) or \( \fp'_i = \unitO \), by the \cref{lem:no-effect-for-empty-fingerprint} we know \( \conf'_{i}\projection{1} = \conf'_{i+1}\projection{1}\) from the trace of \( \ET_2 \).
Since \( \conf'_{i+1}\projection{1} = \conf_n\projection{1}\) where \( \conf_n \) is the final configuration of the trace of \( \ET_1 \), we now have \( \conf'_{i}\projection{1} = \conf_n\projection{1}\).
Applying \ih that \cref{equ:et1-et2-in-et12} holds when \( m = i \), so there exist configurations from \( \conf''_1 \) to \( \conf''_k \):
\begin{equation}
    \label{equ:ih-for-k-length}
    \begin{array}{@{}l@{}}
    \quad \conf_0 \toET{\stub}[\ET_1 \cap \ET_2] \conf''_1 \toET{\stub}[\ET_1 \cap \ET_2] \cdots \toET{\stub}[\ET_1 \cap \ET_2] \conf''_k
    \land \conf_n\projection{1} = \conf'_i\projection{1} = \conf''_k\projection{1} \\
    \quad {} \land \fora{\cl \in \dom(\conf''_k\projection{2}),\key \in (\conf''_k\projection{1})} 
    \conf''_k\projection{2}(\cl)(\key) = \max\Set{\conf_n\projection{2}(\cl)(\key), \conf'_i\projection{2}(\cl)(\key)}
\end{array}
\end{equation}
Given the definition of the reduction (\cref{def:reduction}), when \( \fp = \epsilon \) or \( \fp = \unitO \) we know \( \conf'_i\projection{2}(\cl_{i+1}) \sqsubseteq  \conf'_{i+1}\projection{2}(\cl_{i+1})\) thus:
\begin{equation}
    \label{equ:preserve-max-view}
    \begin{array}{l}
    \max\Set{\conf_n\projection{2}(\cl_{i+1}), \conf'_i\projection{2}(\cl_{i+1})} 
    \sqsubseteq \max\Set{\conf_n\projection{2}(\cl_{i+1}), \conf'_{i+1}\projection{2}(\cl_{i+1})} 
    \end{array}
\end{equation}
Therefore \cref{equ:et1-et2-in-et12} holds when \( m = i + 1\) by appending a view shift to the end of the trace in \cref{equ:ih-for-k-length}:
\[
    \begin{array}{@{}l}
    \conf_0 \toET{\stub}[\ET_1 \cap \ET_2] \conf''_1 \toET{\stub}[\ET_1 \cap \ET_2] \cdots
    \toET{\stub}[\ET_1 \cap \ET_2] \conf''_k \toET{\cl_{j}, \epsilon }[\ET_1 \cap \ET_2] \\
    \qquad {} \land \conf''_k\rmto{2}{\conf''_k\projection{2}\rmto{\cl_{i}}{\max\Set{\conf_n\projection{2}(\cl_{i+1}), \conf'_{i+1}\projection{2}(\cl_{i+1})} }}
    \end{array}
\]

    \item If \( \fp'_i \neq \unitO  \land \fp' \neq \epsilon \), by \cref{lem:identical-step} there exists a step \( (\cl_j, \fp_j) \) from the trace of \( \ET_1 \) such that:
\begin{equation}
    \label{equ:j-th-step}
    \begin{array}{l}
    (\mkvs_{j}, \vienv_{j}) \toET{\cl_{j}, \fp_{j}}[\ET_1] (\mkvs_{j + 1}, \vienv_{j + 1}) 
    \land \ET_1 \vdash (\mkvs_{j}, \vi_j) \csat \fp_j : (\mkvs_{j+1},\vienv_{j + 1}(\cl_{j}) ) \land \vi_j = \vienv_{j}(\cl_j)
\end{array}
\end{equation}
for some \( j, \cl_j, \vi_j\) and \( \fp_j \) such that \( 0 \leq  j < n \), \( \cl_j = \cl'_{i}\), \( \fp_j = \fp'_{i}\), and
\[ 
    \fora{\key} (\stub, \key, \stub ) \in \fp_j \implies \vi_j(\key) = \vi'_{i}(\key)
\]
We apply the commutativity of \( \ET_1 \) until the step shown in \cref{equ:j-th-step} is at the end or the second end of the trace of \( \ET_1 \).
Let consider the next two steps, (j+1)-\emph{th} and (j+2)-\emph{th} step.
Since the trace is in normal form, the (j+1)-\emph{th} step is a view shift by a client \( \cl_{j+2} \) and (j+2)-\emph{th} step is a concrete step issued by the same client \( \cl_{j+2} \) under the view \( \vi_{j+2} \):
\begin{equation}
    \label{equ:j-plus-1-th-step}
    \begin{array}{@{}l@{}}
        (\mkvs_{j+1}, \vienv_{j+1}) \toET{\cl_{j+2}, \epsilon}[\ET_1]
        (\mkvs_{j+1}, \vienv_{j+1}\rmto{\cl_{j+2}}{\vi_{j+2}}) \toET{\cl_{j+2}, \fp_{j+2}}[\ET_1] (\mkvs_{j+3}, \vienv_{j+3}) \\
        \qquad \land \ET_1 \vdash (\mkvs_{j+1},\vi_{j+2}) \csat \fp_{j+2} : (\mkvs_{j+3},\vienv_{j+3}(\cl_{j+2}) )
    \end{array}
\end{equation}
It is known that the client  \( \cl_{j+2} \) is different from \( \cl_j \) (\cref{lem:different-cl}) and \( \fp_{j+2} \) writes different keys from \( \fp_j\) (\cref{lem:different-writes}). 
Because \( \cl_j \neq \cl_{j+2} \) we can swap the view shift step shown in \cref{equ:j-plus-1-th-step} before the j-\emph{th} step shown in \cref{equ:j-th-step} which gives the following:
\begin{equation}
    \label{equ:swap-the-view-shift-et1}
    \begin{array}{@{}l@{}}
    (\mkvs_{j}, \vienv_{j}) \toET{\cl_{j+2}, \epsilon}[\ET_1] (\mkvs_{j}, \vienv_{j}\rmto{\cl_{j+2}}{\vi_{j+2}}) \toET{\cl_{j}, \fp_{j}}[\ET_1] \\
    \quad (\mkvs_{j + 1}, \vienv_{j + 1}\rmto{\cl_{j+2}}{\vi_{j+2}}) \toET{\cl_{j+2}, \fp_{j+2}}[\ET_1] (\mkvs_{j+3}, \vienv_{j+3})
    \end{array}
\end{equation}
Now let discuss the (j+2)-\emph{th} step.
Similarly by the \cref{lem:identical-step}, there is a step \((\cl_p, \fp_p)\) from the trace of \( \ET_2 \) such that \( \cl_p = \cl_{j+2}\) and \( \fp_p = \fp_{j+2}\) and \( p < i \).
Note that the last step from \( \ET_2 \), \ie (i+1)-\emph{th} step, is not a view shift therefore the i-\emph{th} step must be a view shift so the p-\emph{th} step must be before  i-\emph{th} step.
This means the fingerprint \( \fp_p \) does not observe any change by (i+1)-\emph{th} step from the trace of \( \ET_2 \).
Therefore \( \vi_{j+2} \) does not observe any change by j-\emph{th} step from the trance of \( \ET_1\), \ie \( \vi_{j+2} \in \Views(\mkvs_j) \).
By \cref{prop:swap-update}, that allows to swap the two adjacent non-conflict steps from \cref{equ:swap-the-view-shift-et1}, \ie the last two steps.
It follows a new kv-stores \( \mkvs'''_{j+2}\) and a new view environment \( \vienv'''_{j+2} \) such that:
\begin{equation}
    \label{equ:swap-step-et1}
    \begin{array}{@{}l@{}}
    (\mkvs_{j}, \vienv_{j}) \toET{\cl_{j+2}, \epsilon}[\ET_1] (\mkvs_{j}, \vienv_{j}\rmto{\cl_{j+2}}{\vi_{j+2}}) \toET{\cl_{j+2}, \fp_{j+2}}[\ET_1] \\
    \quad (\mkvs_{j + 2}''', \vienv_{j + 2}''') \toET{\cl_{j+2}, \fp_{j+2}}[\ET_1] (\mkvs_{j+3}, \vienv_{j+3})
    \end{array}
\end{equation}
In the \cref{equ:swap-step-et1} the j-\emph{th} step moves to the right of (j+2)-\emph{th} step.
We monotonicly move the j-\emph{th} step until it is at the end or the second end of trace of \( \ET_1 \):
\[
    \begin{array}{@{}l}
        \conf_0 \toET{\stub}[\ET_{1}] \cdots \toET{\stub}[\ET_{1}] \conf_{j-1} \toET{\stub}[\ET_{1}]
        \conf'''_{j} \toET{\stub}[\ET_{1}] \dots \toET{\stub}[\ET_{1}] \conf'''_{n-1} \toET{\cl_j, \fp_j }[\ET_{1}] \conf_{n} \lor {} \\
        \conf_0 \toET{\stub}[\ET_{1}] \cdots \toET{\stub}[\ET_{1}] \conf_{j-1} \toET{\stub}[\ET_{1}] 
        \conf'''_{j} \toET{\stub}[\ET_{1}] \dots \toET{\stub}[\ET_{1}] \conf'''_{n-2} \toET{\cl_j, \fp_j }[\ET_{1}] \conf'''_{n-1} \toET{\cl_{n-1}, \epsilon }[\ET_{1}] \conf_{n}  \\ 
    \end{array}
\]
for some new configurations from \( \conf'''_{j}\) to \( \conf'''_{n-1} \).
Note that if it is the second end, the last step must be a view shift step as shown in \cref{equ:new-et-1}.
\begin{itemize}
    \item If the j-\emph{th} step is at the end of the new trace of \( \ET_1 \), we have the trace:
\begin{equation}
    \label{equ:new-et-1}
    \begin{array}{@{}l}
        \conf_0 \toET{\stub}[\ET_{1}] \cdots \toET{\stub}[\ET_{1}] \conf_{j-1} \toET{\stub}[\ET_{1}] 
        \conf'''_{j} \toET{\stub}[\ET_{1}] \dots \toET{\stub}[\ET_{1}] \conf'''_{n-1} \toET{\cl_j, \fp_j }[\ET_{1}] \conf_{n}  \\
    \end{array}
\end{equation}
Given the hypothesis that \( \conf_{n}\projection{1} = \conf'_{i+1}\projection{1} \) and the fact that the last step of the new trace of \( \ET_1 \) (\cref{equ:new-et-1}) and the last step the trace of \( \ET_2 \) (\cref{equ:last-et-2}) are the same step, the kv-stores of the second last configurations the new trace of \( \ET_1 \) (\cref{equ:new-et-1}) and the one from the trace of \( \ET_2 \) (\cref{equ:last-et-2}) are the same \(  \conf'''_{n-1}\projection{1} = \conf'_{i}\projection{1} \).
Then by applying \ih that \cref{equ:et1-et2-in-et12} holds when \( m = i \), there exists a trace of \( \ET_1 \cap \ET_2 \):
\begin{equation}
    \label{equ:ih-for-merge-two-trace}
    \begin{array}{@{}l}
        \conf_0 \toET{\stub}[\ET_1 \cap \ET_2] \dots \toET{\stub}[\ET_1 \cap \ET_2] \conf''_{k-1} 
        \land \conf'''_{n-1}\projection{1} = \conf'_{i}\projection{1} = \conf''_{k-1}\projection{1}  \\
        \quad {} \land \fora{\cl \in \dom(\conf''_{k-1}\projection{2}),\key \in (\conf''_{k-1}\projection{1})} 
        \conf''_{k-1}\projection{2}(\cl)(\key) = \max\Set{\conf_n\projection{2}(\cl)(\key), \conf'_i\projection{2}(\cl)(\key)}
\end{array}
\end{equation}
for some number \( k \) and configurations from \( \conf''_1 \) to \( \conf''_{k-1} \).
By \cref{equ:last-et-2} and \cref{equ:new-et-1}, we have:
\begin{equation}
    \label{equ:et1-et2-csat}
    \begin{array}{@{} l@{}}
        \ET_1 \vdash ( \conf'_{i}\projection{1}, \conf'_{i}\projection{2}(\cl_{i}) )  \csat \fp_{i} : ( \conf'_{i+1}\projection{1}, \conf'_{i+1}\projection{2}(\cl_{i}) ) \\
        \quad {} \land \ET_2 \vdash ( \conf'''_{n-1}\projection{1}, \conf'''_{n-1}\projection{2}(\cl_{i}) )  \csat \fp_{i} : (\conf_{n}\projection{1}, \conf_{n}\projection{2}(\cl_{i}) )
    \end{array}
\end{equation}
First, for any quadraple in \( \ET_1 \) and \( \ET_2 \), it does not constrain the view for keys that are not appear in the fingerprint before update.
That is:
\[
    \begin{array}{@{}l@{}}
    \fora{ \mkvs, \mkvs', \vi, \vi', \vi'', \fp, \key } 
    (\stub, \key, \stub) \in \fp \land \vi(\key) = \vi'(\key) \land (\mkvs, \vi, \fp, \mkvs', \vi'') \in \ET 
    \implies (\mkvs, \vi', \fp, \mkvs', \vi'') \in \ET
    \end{array}
\]
Given above and \cref{equ:ih-for-merge-two-trace}, we can substitute the configurations \( \conf'_{i} \) and  \( \conf'''_{n-1} \) from \cref{equ:et1-et2-csat} by \( \conf''_{k-1}\).
Then,  because for any \( \mkvs, \mkvs', \vi, \vi', \vi'' \) and \( \fp \), if \( (\mkvs, \vi, \fp, \mkvs', \vi' ) \in \ET_1 \) and \( \vi' \sqsubseteq \vi'' \) then \( (\mkvs, \vi, \fp, \mkvs', \vi'' ) \in \ET_1 \), and similarly for \( \ET_2 \).
It means:
\begin{equation}
    \label{equ:et12-csat}
    \begin{array}{l}
    \ET_1 \cap \ET_2 \vdash ( \conf''_{k-1}\projection{1}, \conf''_{k-1}\projection{2}(\cl_{i}) ) \csat
    \fp_{i} : ( \conf_{n}\projection{1}, \max\Set{\conf'_{i+1}\projection{2}(\cl_{i}), \conf_{n}\projection{2}(\cl_{i})} )
    \end{array}
\end{equation}
Therefore the \cref{equ:et1-et2-in-et12} holds when \( m = i + 1\) by appending the shown in \cref{equ:et12-csat} to the end of the trace shown in \cref{equ:ih-for-merge-two-trace}:
\[
\begin{array}{@{}l}
    \conf_0 \toET{\stub}[\ET_1 \cap \ET_2] \dots \toET{\stub}[\ET_1 \cap \ET_2] \conf''_{k-1} \toET{cl_{i}, \fp_{i}}[\ET_1 \cap \ET_2] \\
    \quad \left( \conf_n\projection{1},\conf''_{k-1}\projection{2}\rmto{\cl_{i}}{\max\Set{\conf'_{i+1}\projection{2}(\cl_{i}), \conf_{n}\projection{2}(\cl_{i})} } \right)
\end{array}
\]
    \item If the j-\emph{th} step is the second last step of the new trace of \( \ET_1 \), we have the trace:
\begin{equation}
    \label{equ:new-et-1-with-view-shift-tail}
    \begin{array}{@{}l}
        \conf_0 \toET{\stub}[\ET_{1}] \cdots \toET{\stub}[\ET_{1}] \conf_{j-1} \toET{\stub}[\ET_{1}] 
        \conf'''_{j} \toET{\stub}[\ET_{1}] \dots \\
        \quad {} \toET{\stub}[\ET_{1}] \conf'''_{n-2} \toET{\cl_j, \fp_j }[\ET_{1}] 
        \conf'''_{n-1} \toET{\cl_{n-1}, \epsilon }[\ET_{1}] \conf_{n}  \\ 
    \end{array}
\end{equation}
Since the last step is a view shift, we know \( \conf_n\projection{1} = \conf'''_{n-1}\projection{1}\), and the rest of proof is the same as the case where j-\emph{th} is the last step as shown in \cref{equ:new-et-1}.
\end{itemize}
\end{itemize}
\end{itemize}
\end{proof}

\begin{lemma}[No effect from empty fingerprint and epsilon reduction]
    \label{lem:no-effect-for-empty-fingerprint}
    \label{lem:no-effect-for-view-shift}
    \[
    \fora{\conf, \conf', \cl,\vi} \conf \toET{\cl, \unitO}[\ET] \conf' \lor \conf \toET{\cl, \epsilon}[\ET] \conf' \implies \conf\projection{1} = \conf'\projection{1}
    \]
\end{lemma}
\begin{proof}
    Let \((\mkvs, \vienv)  = \conf \) and \( (\mkvs', \vienv') = \conf' \).
    For the case of empty fingerprint,
    by the definition of  $\conf \toET{\cl, \unitO} \conf'$ (\cref{def:reduction}), we have \(\mkvs' \in \updateKV[\mkvs, \vi, \unitO, \cl]\), and therefore \( \mkvs' = \mkvs \).
    For the case of view shift, by the definition of  $\conf \toET{\cl, \epsilon} \conf'$ (\cref{def:reduction}) it is easy to see \( \mkvs' = \mkvs \).
\end{proof}

We define a \(  \mkvs(\txid) \) function that returns the fingerprint associate with the transaction identifier \( \txid \):
\[
    \begin{rclarray}
        \mkvs(\txid) & \defeq & \Setcon{(\otW, \key, \val)}{\exsts{i} \mkvs(\key)(i) = (\val, \txid, \stub)} \cup  \Setcon{(\otR, \key, \val)}{\exsts{i,\txidset} \mkvs(\key)(i) = (\val, \stub, \txidset) \land \txid \in \txidset}
    \end{rclarray}
\]

\begin{lemma}[Transactions persistence]
    \label{lem:mono-fingerprint}
    \[
        \fora{\ET,\conf,\conf',\txid,\fp} \conf\projection{1}(\txid) = \fp \land \conf \toET{\stub}[\ET] \conf' \implies \conf'\projection{1}(\txid) = \fp
    \]
\end{lemma}
\begin{proof}
    It is easy to prove this by case analysis on the reduction relation.
\end{proof}

\begin{lemma}[Same steps]
\label{lem:identical-step}
Given a trace of \( \ET_1 \) and a trace of \( \ET_2 \),
if the have the same final kv-store,
the trace contains the same concrete steps (free variables are globally quantified):
\[
\begin{array}{@{}l}
    \conf_0 \toET{\cl_1, \fp_1}[\ET_{1}] \cdots \toET{\cl_n, \fp_n}[\ET_{1}] \conf_n \land
    \conf_0 \toET{\cl'_1, \fp'_1}[\ET_{2}] \cdots \toET{\cl'_m, \fp'_m}[\ET_{2}] \conf'_m 
    \land \conf_n\projection{1} = \conf'_m\projection{1} \\
    \quad \implies \fora{i: 0 < i \leq n} 
    \fp_i = \unitO 
    \lor \fp_i = \epsilon 
    \lor \exsts{j: 0 < j \leq m} 
    \cl_i = \cl'_j \land \fp_i = \fp'_j \land ( \fora{\key} (\stub, \key, \stub) \in \fp_i \implies \vi_i(\key) = \vi'_j(\key) )
\end{array}
\]
\end{lemma} 
\begin{proof}
    We prove by contradiction.
    First because \( \mkvs_n = \mkvs'_m \), we know that:
    \begin{equation}
        \label{equ:same-kv-store}
        \fora{\txid, \fp} \mkvs_n(\txid) = \fp \iff \mkvs'_m(\txid) = \fp
    \end{equation}
    Let \(\conf_n = (\mkvs_n,\vienv_n) \) and \(\conf'_m = (\mkvs'_m,\vienv'_m) \).
    Assume a step \( \conf_i \toET{\cl, \fp }[\ET_1] \conf_{i+1} \)  from the trace of \( \ET_1 \) where the transaction identifier is \( \txid \) and \( \fp \neq \unitO \).
    It must have a step from the trace of \( \ET_2 \), which commits some fingerprint via the same transaction identifier  \( \txid \).
    We know \( \mkvs_n(\txid) = \fp \) by \cref{lem:mono-fingerprint}, thus \( \mkvs'_m(\txid) = \fp \) by \cref{equ:same-kv-store}.
    Let assume a key \( \key \) that \( (\stub, \key, \stub) \in \fp \land \vi_i(\key) \neq \vi'_j(\key)\) where \( \vi_i\) and \( \vi'_j\) are the views immediate before the commit of the fingerprint \( \fp \) in traces of \( \ET_1\) and \( \ET_2 \) respectively.
    Since the no blind write assumption, it is safe to assume it is a read operation on the key \( \key \).
    By the definition of the reduction (\cref{def:reduction}) and \cref{lem:mono-fingerprint}, we know \( \func{read}{\mkvs_n(\key)(\vi_i(\key))} \neq \func{read}{\mkvs'_m(\key)(\vi_i(\key))} \), which contradicts with \( \mkvs_n = \mkvs'_m \).
\end{proof}

We define \( \max_\cl(\conf) \) function that returns the most recent transaction identifier for client \( \cl \) in the configuration \( \conf \) 
\[
\begin{rclarray}
    \max_\cl((\mkvs, \vienv)) & \defeq & \max\Setcon{\txid^{n}_\cl}{\txid^{n}_\cl \text{ appear in } \mkvs} \\
\end{rclarray}
\]

\begin{lemma}[Transactions from different clients]
\label{lem:different-cl}
Given a trace of \( \ET_1 \) and a trace of \( \ET_2 \),
if the i-\emph{th} step from \( \ET_1 \) issued by the client \( \cl_i \) 
is the same as the last step from \( \ET_2 \),
then in the trace of \( \ET_1 \) 
there is no concrete step issued by the client \(\cl_i \) after the i-\emph{th} step (free variables are globally quantified):
\[
\begin{array}{@{}l}
    \conf_0 \toET{\cl_1, \fp_1}[\ET_{1}] \cdots \toET{\cl_n, \fp_n}[\ET_{1}] \conf_n 
    \land \conf_0 \toET{\cl'_1, \fp'_1}[\ET_{2}] \cdots \toET{\cl'_m, \fp'_m}[\ET_{2}] \conf'_m 
    \land \conf_n\projection{1} = \conf'_m\projection{1} 
    \land \fp'_m \neq \unitO \\
    \quad {} \land \exsts{i}  
    \cl_i = \cl'_m
    \land \fp_i = \fp'_m 
    \implies \fora{j > i} 
    \fp_j = \epsilon \lor \fp_j = \unitO \lor \cl_i \neq \cl_j
\end{array}
\]
\end{lemma}
\begin{proof}
    We prove by deriving contradiction.
    Assume the last step of the trace of \( \ET_2 \) is:
    \begin{equation}
        \label{equ:last-step-for-cl-et2}
        \conf'_{m-1} \toET{\cl'_m, \fp'_m}[\ET_{2}] \conf'_m
    \end{equation}
    Assume a step of the trace of \( \ET_1 \):
    \begin{equation}
        \label{equ:identical-step-for-cl-et1}
        \conf_{i-1} \toET{\cl_i, \fp_i}[\ET_{1}] \conf_i
    \end{equation}
    where \( \cl_i = \cl'_m \) and \( \fp_i = \fp'_m \).
    Because these two steps (\cref{equ:last-step-for-cl-et2} and \cref{equ:identical-step-for-cl-et1}) are issued by the same transaction identifier,
    we know \( \max{}_{\cl_m}(\conf'_m) = \max{}_{\cl_m}(\conf_i) \).
    Assume that there exists a step from the trace of \( \ET_1 \), says j-\emph{th} step, such that:
    \[
        \conf_{j-1} \toET{\cl_j, \fp_h}[\ET_{1}] \conf_j \land j > i \land \fp_j \neq \unitO \land \cl_i = \cl_j 
    \]
    Therefore we have \( \max{}_{\cl_m}(\conf_j) > \max{}_{\cl_m}(\conf_i) \) by \cref{lem:kv-max-cl}.
    That means \( \max{}_{\cl_m}(\conf_n) > \max{}_{\cl_m}(\conf_j) > \max{}_{\cl_m}(\conf_i) = \max{}_{\cl_m}(\conf'_m) \), which contradicts to \( \conf_n\projection{1} = \conf'_m\projection{1}\).
\end{proof}

\begin{lemma}[Reduction following session order]
\label{lem:kv-max-cl}
\[
\begin{array}{@{}l}
    \fora{\conf, \conf' ,\cl, \fp, \ET}
    \conf \toET{\cl, \fp}  \conf' 
    \land 
    \left( 
        \begin{array}{l}
        \fp \neq \unitO \implies \max{}_\cl(\conf) < \max{}_\cl(\conf') )
        \lor ( \fp = \unitO \implies \max{}_\cl(\conf) = \max{}_\cl(\conf')
        \end{array}
    \right)
\end{array}
\]
\end{lemma}
\begin{proof}
    Assume a step \( (\mkvs, \vienv) \toET{\cl, \fp} (\mkvs', \vienv') \).
    By the definition of \( \toET{\stub}[\ET]\) (\cref{def:reduction}), we know \( \mkvs' \in \updateKV[\mkvs, \vi, \fp, \cl] \).
    The \( \updateKV[\mkvs, \vi, \fp, \cl] \) picks a fresh transaction identifier \( \txid_\cl^{m} \) that is greater than any transaction identifiers \( \txid_\cl^{n} \) in \( \mkvs \) via \( \nextTxid \) function, \ie \( m > n \).
    If the fingerprint \( \fp \) is not empty, the new identifier appears in \( \mkvs' \), so \( \max{}_\cl(\conf) < \max{}_\cl(\conf') \).
    Otherwise  the fingerprint is empty, the new identifier will not appear anywhere in \( \mkvs' \), so \( \max{}_\cl(\conf) = \max{}_\cl(\conf') \). 
\end{proof}

\begin{lemma}[Writing different keys]
\label{lem:different-writes}
Given a trace of \( \ET_1 \) and a trace of \( \ET_2 \),
if the i-\emph{th} step from \( \ET_1 \) that writes to key \( \key \) 
is the same as the last step from \( \ET_2 \),
then in the trace of \( \ET_1 \) 
there is no concrete step writing to the key \(\key\) after the i-\emph{th} step (free variables are globally quantified):
\[
\begin{array}{@{}l}
    \conf_0 \toET{\cl_1, \fp_1}[\ET_{1}] \cdots \toET{\cl_n, \fp_n}[\ET_{1}] \conf_n \land \conf_0 \toET{\cl'_1, \fp'_1}[\ET_{2}] \cdots \toET{\cl'_m, \fp'_m}[\ET_{2}] \conf'_m 
    \land \conf_n\projection{1} = \conf'_m\projection{1} 
    \land \fp'_m \neq \unitO \\
    \quad {} \land \exsts{i} 
    \cl_i = \cl'_m
    \land \fp_i = \fp'_m
    \implies \fora{j > i} \nexists{\key} \ldotp (\otW, \key, \stub) \in \fp_j \cap \fp_i
\end{array}
\]
\end{lemma}
\begin{proof}
    We prove this by deriving contradiction.
    Assume the last step from the trace of \( \ET_2 \):
    \begin{equation}
        \label{equ:last-step-for-write-et2}
        \conf'_{m-1} \toET{\cl'_m, \fp'_m}[\ET_{2}] (\mkvs'_m, \vienv'_m )
    \end{equation}
    Assume the transaction identifier for the \cref{equ:last-step-for-write-et2} is \( \txid \), and by the definition of \( \toET{}[\ET]\) (\cref{def:reduction}) we know:
    \begin{equation}
        \label{equ:write-fingerprint}
        \fora{\key} (\otW, \key, \stub) \in \fp'_m \implies \mkvs'_m(\key)(\lvert\mkvs'_m(\key)\rvert - 1) = (\stub, \txid, \stub)
    \end{equation}
    Assume a step of the trace of \( \ET_1 \) that is issued by the same transaction identifier with the same fingerprint:
    \begin{equation}
        \label{equ:identical-step-for-write-et1}
        \conf_{i-1} \toET{\cl_i, \fp_i}[\ET_{1}] (\mkvs_i, \vienv_i)
    \end{equation}
    where \( \cl_i = \cl'_m \) and \( \fp_i = \fp'_m \).
    Given \cref{equ:write-fingerprint} and \cref{equ:identical-step-for-write-et1}, it follows:
    \[
        \fora{\key} (\otW, \key, \stub) \in \fp_i \implies \mkvs_i(\key)(\lvert\mkvs_i(\key)\rvert - 1) = (\stub, \txid, \stub)
    \]
    Assume a step, says j-\emph{th}, after i-\emph{th} step that writes to the same key:
    \[
        \conf_{j-1} \toET{\cl_j, \fp_j}[\ET_{1}] (\mkvs_j, \vienv_j) 
        \land j > i
        \land \exsts{\key} (\otW, \key, \stub) \in \fp_i \cap \fp_j
    \]
    Therefore, by \cref{lem:unique-writer} we have:
    \[
        \exsts{\key,i} (\otW, \key, \stub) \in \fp_i \cap \fp_j \land \mkvs_j(\key)(i) = \txid \land \mkvs_j(\key)(\lvert \mkvs_j(\key) \rvert - 1) \neq \txid
    \]
    Note that \( \fp_i = \fp'_m\).
    Since the writer of a version cannot be overwritten, for the final configuration of the trace of \( \ET_1 \) \((\mkvs_n, \vienv_n)\), we know:
    \[
        \exsts{\key,i} (\otW, \key, \stub) \in \fp_i \cap \fp_j \land \mkvs_n(\key)(i) = \txid \land \mkvs_n(\key)(\lvert \mkvs_n(\key) \rvert - 1) \neq \txid
    \]
    Last, by \cref{equ:write-fingerprint} and \( \fp_i = \fp'_m\), it follows:
    \[
        \exsts{\key} (\otW, \key, \stub) \in \fp'_m \land \mkvs_n(\key)(\mkvs_n(\key) - 1) \neq \txid \land \mkvs'_m(\key)(\lvert\mkvs'_m(\key)\rvert - 1)\projection{2} = \txid
    \]
    which contradicts with \( \mkvs'_m = \mkvs_n\).
\end{proof}

\begin{lemma}[Version persistence]
    \label{lem:unique-writer}
    \[
    \begin{array}{@{}l}
        \fora{\mkvs, \mkvs',\vienv,\vienv', \cl, \vi, \fp, i} 
        (\mkvs, \vienv) \toET{\cl, \fp}[\ET] (\mkvs', \vienv')
        \land (\otW, \key, \stub) \in \fp  
        \land 0 \leq i < \lvert \mkvs'(\key) \rvert - 1 \\
        \quad {} \implies \mkvs'(\key)(i)\projection{2} \neq \mkvs'(\key)(\lvert \mkvs'(\key) \rvert - 1)\projection{2}
    \end{array}
    \]
\end{lemma}
\begin{proof}
    By the definition of \( \toET{\stub}[\ET] \) (\cref{def:reduction}), the \( \mkvs' \in \updateKV[\mkvs, \vi, \fp, \cl] \).
    Given the definition of \( \updateKV[\mkvs, \vi, \fp, \cl]\), it the picks a fresh transaction identifier \( \txid \) such that does not appear in \( \mkvs \).
    For any write fingerprint \( (\otW, \key, \stub) \in \fp \), a new version is appended to the end of the key \( \key \) and the writer (the second projection) is assigned to be the fresh identifier \( \txid \).
    Thus we have the proof.
\end{proof}

\begin{proposition}
\label{thm:appendix-et-composition-2}
\label{prop:appendix-et-composition-2}
if $\ET_1, \ET_2$ are commutative, then $\ET_1 \cap \ET_2$ is commutative.
\end{proposition}
\begin{proof}
Let \( \ET_{12} = \ET_1 \cap \ET_2 \).
Assume \(\conf_1, \conf_2, \conf_3, \cl, \cl', \vi, \vi', \fp, \fp' \) such that:
\[
    \conf_1 \toET{\cl, \fp}[\ET_{12}] \conf_2 \toET{\cl', \fp'}[\ET_{12}] \conf_3
\]
Therefore, we have:
\[
    \conf_1 \toET{\cl, \fp}[\ET_{1}] \conf_2 \toET{\cl', \fp'}[\ET_{1}] \conf_3 \land 
    \conf_1 \toET{\cl, \fp}[\ET_{2}] \conf_2 \toET{\cl', \fp'}[\ET_{2}] \conf_3
\]
Because \( ET_1 \)  and \( \ET_2 \) are commutative, there exists a configuration \( \conf'_2 \) such that:
\[
    \conf_1 \toET{\cl', \fp'}[\ET_{1}] \conf'_2 \toET{\cl, \fp}[\ET_{1}] \conf_3 \land 
    \conf_1 \toET{\cl', \fp'}[\ET_{2}] \conf'_2 \toET{\cl, \fp}[\ET_{2}] \conf_3
\]
so we have the proof that: 
\[
    \conf_1 \toET{\cl', \fp'}[\ET_{12}] \conf'_2 \toET{\cl, \fp}[\ET_{12}] \conf_3
\]
\end{proof}


\end{document}
