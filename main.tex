%% For double-blind review submission, w/o CCS and ACM Reference (max submission space)
\documentclass[acmsmall,review,anonymous]{acmart}\settopmatter{printfolios=true,printccs=false,printacmref=false}
%% For double-blind review submission, w/ CCS and ACM Reference
%\documentclass[acmsmall,review,anonymous]{acmart}\settopmatter{printfolios=true}
%% For single-blind review submission, w/o CCS and ACM Reference (max submission space)
%\documentclass[acmsmall,review]{acmart}\settopmatter{printfolios=true,printccs=false,printacmref=false}
%% For single-blind review submission, w/ CCS and ACM Reference
%\documentclass[acmsmall,review]{acmart}\settopmatter{printfolios=true}
%% For final camera-ready submission, w/ required CCS and ACM Reference
%\documentclass[acmsmall]{acmart}\settopmatter{}


%% Journal information
%% Supplied to authors by publisher for camera-ready submission;
%% use defaults for review submission.
\acmJournal{PACMPL}
\acmVolume{1}
\acmNumber{CONF} % CONF = POPL or ICFP or OOPSLA
\acmArticle{1}
\acmYear{2018}
\acmMonth{1}
\acmDOI{} % \acmDOI{10.1145/nnnnnnn.nnnnnnn}
\startPage{1}

%% Copyright information
%% Supplied to authors (based on authors' rights management selection;
%% see authors.acm.org) by publisher for camera-ready submission;
%% use 'none' for review submission.
\setcopyright{none}
%\setcopyright{acmcopyright}
%\setcopyright{acmlicensed}
%\setcopyright{rightsretained}
%\copyrightyear{2018}           %% If different from \acmYear

%% Bibliography style
\bibliographystyle{ACM-Reference-Format}
%% Citation style
%% Note: author/year citations are required for papers published as an
%% issue of PACMPL.
\citestyle{acmauthoryear}   %% For author/year citations

\usepackage{wrapfig}

\newif\ifNonACMMode
\NonACMModefalse
%**********************************************************************************************************************************
% Note: this file is shared between several documents, please do not delete any macros.
% Maintained by Shale Xiong <sx14@ic.ac.uk>
%**********************************************************************************************************************************

\ifNonESOPMode
%
% for theorem, proof, etc.
%\usepackage{amsthm} 
%
\theoremstyle{definition}
%\newtheorem{definition}[thm]{Definition}
%\newtheorem{lemma}[thm]{Lemma}
%\newtheorem{proposition}[thm]{Proposition}
%\newtheorem{example}[thm]{Example}
\newtheorem{definition}{Definition}[section]
\newtheorem{theorem}{Theorem}[section]
\newtheorem{lemma}{Lemma}[section]
\newtheorem{proposition}{Proposition}[section]
\newtheorem{example}{Example}[section]
%
%
\usepackage{titlesec}
%
\usepackage[titletoc]{appendix}
%
%caption margin
\usepackage[margin=2cm]{caption}
%
\usepackage{bold-extra}
%
\else
%
% to fix the proof without QED from llncs
\let\proof\relax\let\endproof\relax
\usepackage{amsthm}
\fi

% typesetting enum, etc.
\usepackage{enumerate}
\usepackage[inline]{enumitem}

% For url 
\usepackage{url}

% for color 
\usepackage[usenames,dvipsnames,svgnames,table]{xcolor}

\usepackage{hyperref}
\hypersetup{
    colorlinks,
    citecolor=black,
    filecolor=black,
    linkcolor=black,
    urlcolor=black
}


% clearly for the colour needed everywhere
\usepackage{color}

% for bibliography
\usepackage[numbers,sort]{natbib}

% for the line number at the edge
\usepackage{lineno}

% for equation number
\makeatletter
\@addtoreset{equation}{section}
\makeatother
\renewcommand{\theequation}{\arabic{section}.\arabic{equation}}

% For icons
\usepackage{fontawesome}

\usepackage{dsfont}
\usepackage{amsmath}

% the inter command for operational semantices
\usepackage{proof}
%%%%%%%%%%%%%%%%% submission version start from the follows %%%%%%%%%%%%%%%%%%%%%%%


% for xspace comment used in macro
\usepackage{xspace}


\usepackage{centernot}

% sub-figure
\usepackage{subcaption}
\captionsetup{compatibility=false}

% For math font and some commands
\usepackage{amssymb,stmaryrd}
\expandafter\def\csname opt@stmaryrd.sty\endcsname
{only,shortleftarrow,shortrightarrow}
\usepackage{extpfeil}

% For the box assertion
\usepackage{varwidth}

% tikz
\usepackage{tikz}
\usetikzlibrary{positioning, shapes, decorations.pathmorphing, arrows, calc,fit,matrix}
\usepackage{tikz-cd}

% for code 
\usepackage{listings}
\lstset{%
    basicstyle=\footnotesize\ttfamily,
    breaklines=true,
    numberstyle=\scriptsize,
    numbers=left,
    breakatwhitespace=false,
    escapeinside={(*}{*)},
    captionpos=b
}
\renewcommand{\lstlistingname}{Code}

% for the substitute E[e/x]
\usepackage{xfrac}

% better frac
\usepackage{nicefrac}

% For long table, tabularx
\usepackage{ltablex}

% For show the bib entry in the main body.
\usepackage{bibentry}

% box for math env
\usepackage{empheq}

% for better math typesetting 
\usepackage{mathtools}

% for resize the font 
\usepackage{relsize}

% For multirow and multicolumn in math and table
\usepackage{multirow}

% for the smile and sad face 
\usepackage{wasysym}

% typeset rules 
\usepackage{mathpartir}

% for scalebox
\usepackage{graphicx}

%For fig floating next to text
\usepackage{wrapfig}

% For font mathsfs
\usepackage{mathrsfs}  

% for anarchy symbol circledA
\usepackage{marvosym}


% xparse for more powerful macro definition
\usepackage{xparse}

%For reference 
\usepackage{cleveref}


\usepackage{hhmacros}
\pgfdeclarelayer{main}
\pgfdeclarelayer{background}
\pgfdeclarelayer{foreground}
\pgfsetlayers{background,main,foreground}

\newcommand{\greyness}{gray!40}
\newcommand{\blueness}{cyan!60}

\tikzstyle{background}=[rectangle, draw=black, inner sep=0.2cm, rounded corners=1.2mm]
\tikzstyle{white}=[rectangle, fill=white, inner sep=0.5cm, rounded corners=5mm]

%\tikzstyle{background}=[circle, fill=\greyness,
%                                                inner sep=0.2cm,
%                                                rounded corners=5mm,
%                                                decorate,
%                                                decoration={random steps,
%                                                            segment length=3pt,
%                                                            amplitude=3pt}]

%

 \tikzstyle{hheapcell}=[rectangle, draw=black, inner sep=0.1cm, font=\small]

\tikzstyle{noise}=[circle, thick, minimum size=1.2cm, draw=yellow!85!black, fill=yellow!40, decorate, decoration={random steps, segment length=2pt, amplitude=2pt}]

%\pgfdeclarelayer{background}
%\pgfdeclarelayer{foreground}
%\pgfsetlayers{background,main,foreground}

\tikzstyle{abstract}=[draw, fill=white, text width=5em, text centered, minimum height=2.5em, rounded corners]
    
\tikzstyle{arr}=[draw, ->, thick, color=black]
\tikzstyle{dasharr}=[draw,->,thick,dashed,color=black]

\tikzset{
    version node/.style={
        rectangle,
        draw=black,
        align=center,
        minimum height=5mm,
        text depth=0.5ex,
        text height=2ex,
        inner xsep=0pt,
        outer sep=0pt
        %,font=\footnotesize
    },      
    version list/.style={
        matrix of nodes,
        row sep=-\pgflinewidth,
        column sep=-\pgflinewidth,
        nodes={
            version node
        }
        ,
        execute at empty cell={\node[draw=none]{};},
        text width=5mm,
        anchor=west,
        ampersand replacement=\&
    }
}

\newcommand{\tikzvalue}[4]{
    \node[version node, fit=(#1) (#2), fill=white, inner sep=0pt] (#3) {#4}
}
\newcommand{\tikzkvspace}{1.5pt}
\newcommand{\tikzkeyspace}{-1.1}
\newenvironment{halfsubfig}{%
    \begin{subfigure}{0.45\textwidth}
}{%
    \end{subfigure}
}
\newenvironment{onethirdsubfig}{%
    \begin{subfigure}{0.3\textwidth}
}{%
    \end{subfigure}
}
\NewEnviron{centertikz}{%
    \begin{center}%
    \scalebox{.8}{%
    \begin{tikzpicture}[every node/.style={inner sep=0,outer sep=0},font=\large]%
    \BODY%
    \end{tikzpicture}%
    }%
    \end{center}%
}
%\newenvironment{centertikz}{%
    %\begin{center}%
    %\begin{tikzpicture}[every node/.style={inner sep=0,outer sep=0},font=\large]%
%}{%
    %\end{tikzpicture}%
    %\end{center}%
%}

\PassOptionsToPackage{svgnames}{xcolor}
\definecolor{DarkGreen}{rgb}{0, 0.5, 0}


%%%%%%%%%%%%%%%%%%%%%% edit mode
\newif\ifCommentEdits
%\CommentEditstrue

\newcommand{\pg}[1]{%
\ifCommentEdits
    \begin{center}
    \fbox{%
    \begin{minipage}{6.5in} \color{red}
    {\bf PG:} {\rm #1}
    \end{minipage}
    }
    \end{center}
\fi
}

\newcommand{\sx}[1]{%
\ifCommentEdits
    \begin{center}
    \fbox{%
    \begin{minipage}{0.9\textwidth} \color{blue}
    {\bf SX:} {\rm #1}
    \end{minipage}
    }
    \end{center}
\fi
}

\definecolor{darkred}{rgb}{0.5, 0, 0}
\newcommand{\azalea}[1]{%
\ifCommentEdits
    \begin{center}
    \fbox{%
    \begin{minipage}{0.9\textwidth} \color{darkred}
    {\bf AR:} {\rm #1}
    \end{minipage}
    }
    \end{center}
\fi
}

\definecolor{darkblue}{rgb}{0.3,0.0,0.7}
\newcommand{\ac}[1]{%
\ifCommentEdits
    \begin{center}
    \fbox{%
    \begin{minipage}{0.9\textwidth} \color{darkblue}
    {\bf AC:} {\rm #1}
    \end{minipage}
    }
    \end{center}
\fi
}

%%%%%%%%%%%%%%%%%%%%%% end edit mode


%% Journal information
%% Supplied to authors by publisher for camera-ready submission;
%% use defaults for review submission.
\acmJournal{PACMPL}
\acmVolume{1}
\acmNumber{CONF} % CONF = POPL or ICFP or OOPSLA
\acmArticle{1}
\acmYear{2018}
\acmMonth{1}
\acmDOI{} % \acmDOI{10.1145/nnnnnnn.nnnnnnn}
\startPage{1}

%% Copyright information
%% Supplied to authors (based on authors' rights management selection;
%% see authors.acm.org) by publisher for camera-ready submission;
%% use 'none' for review submission.
\setcopyright{none}
%\setcopyright{acmcopyright}
%\setcopyright{acmlicensed}
%\setcopyright{rightsretained}
%\copyrightyear{2018}           %% If different from \acmYear

%% Bibliography style
\bibliographystyle{ACM-Reference-Format}
%% Citation style
%% Note: author/year citations are required for papers published as an
%% issue of PACMPL.
\citestyle{acmauthoryear}   %% For author/year citations

\begin{document}



%% Title information
\title{
	Towards a Formal Theory of Clients of Distributed Key-value Stores
    %Operational Semantics and Logic for Weak Consistency in Transactional Systems%
    %: a Multi-version Based Operational Approach
    } 
                                        %% [Short Title] is optional;
                                        %% when present, will be used in
                                        %% header instead of Full Title.
%\titlenote{with title note}             %% \titlenote is optional;
%                                        %% can be repeated if necessary;
%                                        %% contents suppressed with 'anonymous'
%\subtitle{Subtitle}                     %% \subtitle is optional
%\subtitlenote{with subtitle note}       %% \subtitlenote is optional;
                                        %% can be repeated if necessary;
                                        %% contents suppressed with 'anonymous'


%% Author information
%% Contents and number of authors suppressed with 'anonymous'.
%% Each author should be introduced by \author, followed by
%% \authornote (optional), \orcid (optional), \affiliation, and
%% \email.
%% An author may have multiple affiliations and/or emails; repeat the
%% appropriate command.
%% Many elements are not rendered, but should be provided for metadata
%% extraction tools.

%% Author with single affiliation.
\author{Shale Xiong}
%\authornote{with author1 note}          %% \authornote is optional;
                                        %% can be repeated if necessary
%\orcid{nnnn-nnnn-nnnn-nnnn}             %% \orcid is optional
\affiliation{
  \position{Ph.D. Student}
  \department{Department of Computing}              %% \department is recommended
  \institution{Imperial College London}            %% \institution is required
  \streetaddress{Huxley Building}
  \city{London}
  \state{}
  \postcode{SW7 2AZ}
  \country{United Kingdom}                    %% \country is recommended
}
\email{shale.xiong14@imperial.ac.uk}          %% \email is recommended

%% Author with single affiliation.
\author{Andrea Cerone}
%\authornote{with author1 note}          %% \authornote is optional;
                                        %% can be repeated if necessary
%\orcid{nnnn-nnnn-nnnn-nnnn}             %% \orcid is optional
\affiliation{
  \position{Research Associate}
  \department{Department of Computing}              %% \department is recommended
  \institution{Imperial College London}            %% \institution is required
  \streetaddress{Huxley Building}
  \city{London}
  \state{}
  \postcode{SW7 2AZ}
  \country{United Kingdom}                    %% \country is recommended
}
\email{a.cerone@imperial.ac.uk}          %% \email is recommended

\author{Azalea Raad}
%\authornote{with author1 note}          %% \authornote is optional;
                                        %% can be repeated if necessary
%\orcid{nnnn-nnnn-nnnn-nnnn}             %% \orcid is optional
\affiliation{
  \position{Post-doctoral Researcher}
  \department{}              %% \department is recommended
  \institution{Max Planck Institute}            %% \institution is required
  \streetaddress{no idea}
  \city{Kaiserslautern}
  \state{-}
  \postcode{-}
  \country{Germany}                    %% \country is recommended
}
\email{a.raad@mpi-sws.org}          %% \email is recommended

%% Author with single affiliation.
\author{Philippa Gardner}
%\authornote{with author1 note}          %% \authornote is optional;
                                        %% can be repeated if necessary
%\orcid{nnnn-nnnn-nnnn-nnnn}             %% \orcid is optional
\affiliation{
  \position{Professor}
  \department{Department of Computing}              %% \department is recommended
  \institution{Imperial College London}            %% \institution is required
  \streetaddress{Huxley Building}
  \city{London}
  \state{-}
  \postcode{SW7 2AZ}
  \country{United Kingdom}                    %% \country is recommended
}
\email{p.gardner@imperial.ac.uk}          %% \email is recommended

\begin{abstract}
Modern NoSQL databases (e.g. key-value stores) achieve scalability and improve 
latency by weakening the guarantees of distributed transaction 
processing. While the problem of giving a formal specification of 
the consistency models used by such databases has been widely 
studied, formalising the semantics of clients interacting with 
such systems has been largely neglected$^\ast$. 
This paper aims to be a 
first step towards filling this gap. We present a framework 
for capturing the semantics of programs interacting with a 
weakly consistent key-value store whose transactions enjoy atomic visibility. 
The main features of our semantics can be summarised as follows: 
atomic transaction processing, interleaving concurrency, 
and parameterisation with respect to the consistency model.  
%The latter is captured using the notion of execution tests, which 
%determine when a transaction is allowed  to execute. 
As a main contribution, we prove that our semantics is adequate 
(for each program and consistency model, the semantics captures 
precisely the behaviour that the program exhibits under said consistency model), 
and that specifications of consistency models in our framework 
coincide with previously proposed, axiomatic ones.
%via execution 
%tests are equivalent to the axiomatic specifications which have already 
%been proved to be correct.
As another contribution, we develop a variant of the \emph{Concurrent 
Abstract Predicates} separation logic
%development 
%of a separation logic for clients of key-value stores. We propose 
%a variant of the \emph{Concurrent Abstract Predicates} 
that is tailored to clients of weakly-consistent key-value stores, 
and that is parametric in the specification of consistency models.
%and a multi-version 
%representation of the key-value store to allow concurrent clients to 
%observe different states of the system.  
%We abstract from implementation details of the key-value store, which 
%is represented as a centralised, multi-version system. This allows 
%for concurrent clients to observe different version of the same key, 
%which makes it possible to capture non-serialisable behaviours of 
%programs while still retaining the atomic execution of transactions and 
%interleaving concurrency. 
%Furthermore, our semantics is parameterised by execution tests, which determine 
%when a transaction is allowed to execute.  By changing the execution 
%test of transactions, we capture different consistency models.

\textbf{$^\ast$ Suresh is going to be pissed off a lot by this sentence, but 
let's be honest, his semantics is light years behind ours.}
\ac{The abstract is wayyyy too long. But at least we have a guideline for the paper.}

%We present a a uniform semantics 
%Contents of this set of notes: 
%History heaps. Semantics of Programs 
%running under weak consistency models using history heaps as states. 
%Simulation technique for comparing weak consistency models defined using 
%history heaps. Verification of implementations.
%\textbf{Points following Dagstuhl: Viktor seemed positive about the 
%history heap work. His question was whether the framework is generic 
%enough to capture the protocols that they are developing with Azalea. 
%Alexey's opinion is that the framework may have some use if we 
%manage to prove implementations of protocols correct. 
%I would also like to have Azalea's opinion on a semantics based 
%on history heaps.}
\end{abstract}


%% 2012 ACM Computing Classification System (CSS) concepts
%% Generate at 'http://dl.acm.org/ccs/ccs.cfm'.
\begin{CCSXML}
<ccs2012>
<concept>
<concept_id>10011007.10011006.10011008</concept_id>
<concept_desc>Software and its engineering~General programming languages</concept_desc>
<concept_significance>500</concept_significance>
</concept>
<concept>
<concept_id>10003456.10003457.10003521.10003525</concept_id>
<concept_desc>Social and professional topics~History of programming languages</concept_desc>
<concept_significance>300</concept_significance>
</concept>
</ccs2012>
\end{CCSXML}

\ccsdesc[500]{Software and its engineering~General programming languages}
\ccsdesc[300]{Social and professional topics~History of programming languages}
%% End of generated code


%% Keywords
%% comma separated list
\keywords{keyword1, keyword2, keyword3}  %% \keywords are mandatory in final camera-ready submission


%% \maketitle
%% Note: \maketitle command must come after title commands, author
%% commands, abstract environment, Computing Classification System
%% environment and commands, and keywords command.

\maketitle

\azalea{I have imported the cleveref package! This means that all reference will be printed consistently and we DO NOT need custom names such as \textbackslash fig etc.
Every time you need to refer to something, please write\textbackslash cref\{label\}, \eg \cref{def:mkvs}, and the label (\eg Def.) will be printed correctly. 
These labels can be customised. I have introduced the necessary ones in the macros file. 
}
\newcommand{\RootPath}{.}
\section{Introduction}
Transactions are the \emph{de facto} synchronisation mechanism in modern distributed databases.
To achieve scalability and performance, distributed databases  
often use weak transactional consistency guarantees. 
Much work has been done to formalise the semantics of such consistency guarantees, both
declaratively and operationally.
On the declarative side, several \emph{general} formalisms have been proposed, 
such as dependency graphs~\cite{adya} and abstract executions~\cite{ev_transactions}, to provide a unified
semantics for formulating different consistency models.  
On the operational side, the semantics of \emph{specific} consistency models have
been captured using reference implementations~\cite{si,PSI,PSI-RA}. 
However, unlike declarative approaches, there has been
little work on \emph{general} operational semantics for describing a range
of consistency models, and no work on a general operational semantics
in which to both verify protocols of distributed databases and 
enable the program analysis of clients.

As discussed in \cref{sec:conclusions}, there are several formalisms for a general operational semantics.
%~\cite{sureshConcur,alonetogether,seebelieve}. 
%\citeauthor{alonetogether} 
\cite{alonetogether} propose an operational
semantics over a global, centralised store for reasoning about clients using a program logic; 
they can model several isolation levels, but they cannot
capture consistency models of distributed data-stores, e.g.  
parallel snapshot isolation (\PSI). 
%Moreover, they do not establish the equivalence of their definitions
%of consistency model
%with existing declarative definitions in the literature. 
%\citeauthor{sureshConcur} 
\cite{sureshConcur} propose an operational semantics over abstract executions, 
rather than a concrete centralised store. This semantics captures weaker consistency models
such as \(\PSI\) and has been used to prove the robustness of applications against
a given consistency model. However, although they focus on consistency models with atomic 
visibility (a transaction observes either all or none the updates of another transaction), 
their semantics allows the interleaving of operations in different transactions, resulting in an unnecessarily complicated model.
%However, this semantics cannot model client sessions.
\cite{seebelieve} provide a trace semantics over a global
centralised store, where the behaviour of clients is formalised by the   
observations they can make on the totally ordered history of states of the system, prior to executing a transaction. 
Their framework is tailored at proving the equivalence different specifications of consistency models, 
%but it does little towards proving the correctness of protocols employed by distributed databases. 
but its usefulness  for analysing client programs is not clear, 
%program analysis for clients; in particular, we believe that in their framework the latter task would be 
%difficult,
 given that observations made by clients involve information that is not generally 
available to real world client programs, such as the total order in which transactions commit.
%or to p
%They focus on proving implementations correct. They do not consider program analysis for clients;
%indeed we believe it would be difficult  given their choice to
%keep track of the entire system.

In this paper, we introduce a general operational semantics for describing the
client-observable behaviour of distributed {atomic} transactions
(\cref{sec:overview}, \cref{sec:model}), while successfully abstracting from the 
internal details of protocols of geo-replicated and partitioned databases. In our semantics 
transactions execute atomically, preventing the interleaving of the  
operations they perform: this makes it feasible both to prove interesting properties of client applications 
of the database, and to verify that a distributed protocol correctly implements a consistency model.
Our model comprises a global, centralised
key-value store (kv-store) with {\em multi-versioning},  and
{\em client views}
inspired by the C11 operational semantics
in~\cite{promises}; 
using these mechanisms,
we record all the versions written for each key, and  let clients see only a subset 
of such versions.
Similarly to \cite{seebelieve}, our  operational semantics  is parametric in the notion of {\em
  execution test},  determining if a client with a given view is
allowed to commit a transaction; however, our notion of execution test does not rely on the 
knowledge of the whole history of states preceding a transaction. Using this approach, 
%Different execution tests give rise
%to different consistency models in our semantics. In contrast with \cite{seebelieve}, 
%we restrict observations of clients  to the current state of the system: this leads to 
%execution tests being a more faithful abstraction of the behaviour of real distributed databases protocols.
  we  are able to capture  most of the well-known consistency models in a uniform way (\cref{sec:cm}) using kv-stores and views: e.g.,  causal consistency (\CC), \PSI, snapshot isolation (\SI) and serialisability (\SER); one of our contributions is the development of a general proof technique for proving the correspondence between 
  execution tests and axiomatic specifications of consistency models using abstract executions (\cref{sec:other_formalisms}), 
  which we successfully applied to all the consistency models we consider.
  
  Our framework is thought with \emph{Atomic visibility} 
%  (a transaction 
%  sees either none or all the effects of another transaction) 
  in mind, an in fact we cannot capture popular 
  consistency models such as \emph{Read Committed}. However, because our focus is on protocols and applications employed by distributed databases, 
  most of whose guarantee atomic visibility, we do not find this constraint to not be a severe limitation.
%We define these models using kv-stores and views, 
%provide a correspondence between our kv-stores and dependency graphs, 
%We introduce novel proof techniques for demonstrating that our
%definitions of consistency models 
%are equivalent to existing declarative definitions (\cref{sec:other_formalisms}).
%

We showcase  our semantics by verifying the correctness of two database protocols, 
COPS \cite{cops} and Clock-SI \cite{clocksi}, and by analysing the robustness of simple client applications: in 
particular, we prove the robustness of a single counter against \PSI, and the robustness of multiple counters in \SI  (\cref{sec:applications}). 
%
%For the former, we show that the COPS protocol of a 
%replicated database satisfies our definition of $\CC$ and that the Clock-SI protocol of a partitioned database satisfies our definition of $\SI$.  
%For the latter, we show the robustness of applications against our consistency models: 
%we prove that a transactional library comprising a single counter is robust against $\PSI$; 
%and  that the library with multiple counters is robust against $\SI$, but not $\PSI$.  
%To our knowledge, our robustness results are the first to take into account client sessions.
%Without sessions, multiple counters can be proved to be robust against \(\PSI\) using 
%the static analysis check from \cite{giovanni_concur16}. 
We remark here that we verify protocols and analyse clients in the \emph{same} operational
semantics. By contrast, in existing literature these two tasks are carried out in \emph{different} semantics: for example, protocols are verified using abstract executions;
clients are analysed using dependency graphs; and equivalence results are used to move between the two.

%\mypar{Outline}
%The remainder of this article is organised as follows.
%In \cref{sec:overview} we give an intuitive overview of our ideas. 
%In \cref{sec:model} we present our general operational semantics. 
%In \cref{sec:cm} we show how we encode consistency models in our semantics.
%In \cref{sec:other_formalisms} we relate our formalism to existing declarative formalisms.
%In \cref{sec:applications} we showcase the applications of our semantics.  
%In \cref{sec:conclusions} we discuss related work and conclude. 


\section{semantics\label{sec:semantics}}

Assume that heap and stack are initialised to zero.

\[
    \begin{rclarray}
        \loc \in \Loc & \defeq & \Nat \\
        \val \in \Val & \defeq & \Nat \uplus \Loc \\
        \Var & \defeq & \Set{ \vx, \vy, \dots } \\
        \ts \in \Timestamp & \defeq & \Nat \\
        \hp \in \Heap & \defeq & \Loc \parfun \Val \\
        \stk \in \Stack & \defeq & \Var \to \Val \\
        \rs \in \Readset, \ws \in \Writeset & \defeq & \powerset{\Loc} \\
        \lstt = (\hp, \stk, \rs, \ws ) \in \Localstate & \defeq & \Stack \times \Heap \times \Readset \times \Writeset \\
        \op \in \Operation & \defeq & \Set{\opr, \opw} \\
        \settrans \subseteq \TransID & \defeq & \Set{ \alpha , \beta, \dots } \\
        \tshp \in \Timestampheap & \defeq & \Loc \parfun ( \Timestamp \parfun \Val \times \Operation \times \TransID) \\
        \ThreadID & \defeq & \Set{ i , j, \dots } \\
        (\tshp, \stk, \ts) \in \Threadstate & \defeq & \Timestampheap \times \Stack \times \Timestamp \\
        \tdpl \in \Threadpool & \defeq & \ThreadID \parfun \Stack \times \Timestamp \times \prog \\
        \stt \in \State & \defeq & \Timestampheap \times \Threadpool \\
    \end{rclarray}
\]

No side effect of evaluation of arithmetic expression.

\[
    \begin{syntax}{\texpr}
              \val \quad            |
        \quad \var \quad            |
        \quad \texpr + \texpr \quad |
        \quad \texpr * \texpr \quad |
        \quad \dots 
    \end{syntax}
\]

No side effect of evaluation of boolean expression.

\[
    \begin{rclarray}
        \eval{\val}_{\stk} & \defeq & \val \\
        \eval{\var}_{\stk} & \defeq & \stk(\val) \\
        \eval{\texpr_{1} + \texpr_{2}}_{\stk} & \defeq & \eval{\texpr_{1}}_{\stk} + \eval{\texpr_{2}}_{\stk}   \\
        \eval{\texpr_{1} * \texpr_{2}}_{\stk} & \defeq & \eval{\texpr_{1}}_{\stk} * \eval{\texpr_{2}}_{\stk}  
    \end{rclarray}
\]

\[
    \begin{syntax}{\tbool}
              \true \quad                  |
        \quad \false \quad                 |
        \quad \texpr = \texpr \quad        |
        \quad \texpr < \texpr \quad        |
        \quad \boolnot \tbool \quad        |
        \quad \tbool \booland \tbool \quad |
        \quad \tbool \boolor \tbool \quad  |
        \quad \dots 
    \end{syntax}
\]

\[
    \begin{rclarray}
        \eval{\true}_{\stk}& \defeq & \true \\
        \eval{\false}_{\stk} & \defeq & \false \\
        \eval{\texpr_{1} = \texpr_{2}}_{\stk} & \defeq & \eval{\texpr_{1}}_{\stk} = \eval{\texpr_{2}}_{\stk}   \\
        \eval{\texpr_{1} < \texpr_{2}}_{\stk} & \defeq & \eval{\texpr_{1}}_{\stk} < \eval{\texpr_{2}}_{\stk}   \\
        \eval{\boolnot \tbool}_{\stk} & \defeq & \neg \eval{\tbool}_{\stk} \\
        \eval{\tbool_{1} \booland \tbool_{2}}_{\stk} & \defeq & \eval{\tbool_{1}}_{\stk} \land \eval{\tbool_{2}}_{\stk}  \\
        \eval{\tbool_{1} \boolor \tbool_{2}}_{\stk}& \defeq & \eval{\tbool_{1}}_{\stk} \lor \eval{\tbool_{2}}_{\stk}  
    \end{rclarray}
\]

\[
    \begin{syntax}{\tcmd}
              \tskip \quad                     |
        \quad \tass{\vx}{\texpr} \quad         |
        \quad \tmutate{\texpr}{\texpr} \quad   |
        \quad \tderef{\vx}{\texpr} \quad       |
        \quad \tif{\tbool}{\tcmd}{\tcmd} \quad | \\
              \tloop{\tbool}{\tcmd} \quad      |
        \quad \tcmd \tseq \tcmd
    \end{syntax}
\]

\[
    \begin{rclarray}
        \dontcare, \dontcare, \dontcare, \dontcare, \dontcare \ \localtransfer \ \dontcare, \dontcare, \dontcare, \dontcare, \dontcare & \defeq & \Localstate \times \tcmd \times \Localstate \times \tcmd \\
    \end{rclarray}
\]

\[
    \infer[ass]{%
        \stk, \hp, \rs, \ws, \tass{\var}{\texpr} \ \localtransfer \  \stk \remapsto{\var}{\val}, \hp, \rs, \ws, \tskip
    }{%
    \eval{\texpr}_{\stk} = \val
    }
\]

\[
    \infer[mutate]{%
        \stk, \hp, \rs, \ws, \tmutate{\texpr_{1}}{\texpr_{2}} \ \localtransfer \  \stk, \hp \remapsto{\loc}{\val}, \rs, \ws \cup \Set{\loc}, \tskip
    }{%
        \eval{\texpr_{1}}_{\stk} = \loc \quad 
        \eval{\texpr_{2}}_{\stk} = \val \quad 
        \loc \in \dom(\hp)
    }
\]

\[
    \infer[deref]{%
        \stk, \hp, \rs, \ws, \tderef{\var}{\texpr} \ \localtransfer \  \stk \remapsto{\var}{\val}, \hp, \rs \cup \Set{\loc}, \ws, \tskip
    }{%
        \eval{\texpr}_{\stk} = \loc \quad 
        \val = \hp(\loc) \quad
        \loc \in \dom(\hp)
    }
\]

\[
    \infer[ifelsetrue]{%
        \stk, \hp, \tif{\tbool}{\tcmd_{1}}{\tcmd_{2}} \ \localtransfer \  \stk, \hp, \tcmd_{1}
    }{%
        \eval{\tbool}_{\stk} = \true
    }
\]

\[
    \infer[ifelsefalse]{%
        \stk, \hp, \tif{\tbool}{\tcmd_{1}}{\tcmd_{2}} \ \localtransfer \  \stk, \hp, \tcmd_{2}
    }{%
        \eval{\tbool}_{\stk} = \false
    }
\]

\[
    \infer[whiletrue]{%
        \stk, \hp, \tloop{\tbool} \tcmd \ \localtransfer \  \stk, \hp,  \tcmd \tseq \tloop{\tbool} \tcmd
    }{%
        \eval{\tbool}_{\stk} = \true
    }
\]

\[
    \infer[whilefalse]{%
        \stk, \hp, \tloop{\tbool} \tcmd \ \localtransfer \  \stk, \hp, \tskip
    }{%
        \eval{\tbool}_{\stk} = \false \quad
    }
\]

\[
    \infer[seqskip]{%
        \stk, \hp, \tskip \tseq \tcmd_{2} \ \localtransfer \  \stk, \hp, \tcmd_{2}
    }{%
    }
\]

\[
    \infer[seqnonskip]{%
        \stk, \hp, \tcmd_{1} \tseq \tcmd_{2} \ \localtransfer \  \stk', \hp', \tcmd_{1}' \tseq \tcmd_{2}
    }{%
        \stk, \hp, \tcmd_{1} \ \localtransfer \  \stk', \hp', \tcmd_{1}'
    }
\]

The semantics of transaction are interleaving of start, commit, and restart.

\[
    \begin{syntax}{\prog}
              \pemp \quad               |
        \quad \ptrans{\tcmd} \quad      |
        \quad \prog \pcond \prog \quad  |
        \quad \prept{\prog} \quad       |
        \quad \prog \pseq \prog \quad   |
        \quad \pfork{\var}{\prog} \quad |
        \quad \pjoin{\texpr}   
    \end{syntax}
\]

\[
    \begin{rclarray}
        \prog_{1} \ppar \prog_{2} & \equiv & \pfork{\var}{\prog_{1}} \pseq \prog_{2} \pseq \pjoin{\var} \\
        \tll \in \Translabel & \defeq & 
              \lid \quad                |
              \quad \lfork{\prog} \quad |
        \quad \ljoin{\thid,\ts} \\
        \dontcare, \dontcare, \dontcare, \dontcare \ \threadtransfer{ \dontcare } \ \dontcare, \dontcare, \dontcare, \dontcare & \defeq & \Threadstate \times \prog \times \Translabel \times \Threadstate \times \prog \\
    \end{rclarray}
\]

\[
    \begin{rclarray}
        \func{startstate}(\tshp,\ts) & \defeq & \lambda \loc \ldotp \tshp(\loc)(\max(\Set{\ \ts' \ \middle| \ \ts' \leq \ts \land \tshp(\ts') = (\dontcare,\wop, \dontcare) }))
    \end{rclarray}
\]

\[
    \begin{rclarray}
        \pred{allowcommit}(\tshp,\ws,\rs,\ts_{s},\ts_{e}) & \defeq & 
        \pred{atomicop}(\tshp,\ws,\rs,\ts_{s},\ts_{e}) \land {} \\
        & & \pred{consistent}(\tshp,\ws,\rs,\ts_{s},\ts_{e}) \\
        \pred{atomicop}(\tshp,\ws,\rs,\ts_{s},\ts_{e}) & \defeq  & \forall \loc \in \ws \cup \rs \ldotp \tshp(\loc)(\ts_{s})\undef \land \tshp(\loc)(\ts_{e})\undef \\
        \pred{consistent}(\tshp,\ws,\rs,\ts_{s},\ts_{e}) & \defeq & \forall \ts \in [\ts_{s},\ts_{e}], \loc \in \ws \ldotp \tshp(\loc)(\ts) \neq (\dontcare, \wop, \dontcare) \land {} \\
                                                       & & \exists \ts_{min} = \min(\Set{\ts'' \ \middle| \ \ts'' \geq \ts_{e} \land \tshp(l)(\ts'')\isdef}) \ldotp \\
                                                       & & \ts_{min} \neq \bot \implies \tshp(\loc)(\ts_{min}) = (\dontcare, \wop, \dontcare) \\
        \func{commit}(\tshp,\hp,\ws,\rs,\ts_{s},\ts_{e}) & \defeq &
        \lambda \loc \ldotp
        \begin{funcarray}
            \tshp(\loc) & \loc \notin \ws \cup \rs \\
            \tshp(\loc) \uplus \Set{ \ts_{e} \mapsto (\hp(\loc),\wop,\tsid)} & \loc \in \ws \\
            \tshp(\loc) \uplus \Set{ \ts_{s} \mapsto (\hp(\loc),\rop,\tsid)} & \loc \in \rs \\
        \end{funcarray} \\
        & & \texttt{where} \  \tsid \notin \Set{\tshp(\loc)(\ts)\projection{3} \ \middle| \ \loc \in \dom(\tshp) \land \ts \in \dom(\tshp(\loc))} \\
    \end{rclarray}
\]

\[
    \infer[commit]{%
        \tshp, \stk, \ts, \ptrans{\tcmd} \ \threadtransfer{\lid} \  \tshp', \stk', \ts_{e}, \pemp
    }{%
        \begin{array}{c}
            \ts_{s} \geq \ts \quad \stk, \func{startstate}(\tshp, \ts_{s}), \emptyset, \emptyset \localtransfer^{*} \stk', \hp, \rs, \ws \\
            \pred{allowcommit}(\tshp,\ws,\rs,\ts_{s},\ts_{e}) \quad \ts_{e} > \ts_{s} \quad \tshp' = \func{commit}(\tshp,\hp,\ws,\rs,\ts_{s},\ts_{e})
        \end{array}
    }
\]

\[
    \infer[choiceleft]{%
        \tshp, \stk, \ts, \prog_{1} \pcond \prog_{2} \ \threadtransfer{\lid} \  \tshp, \stk, \ts, \prog_{1}
    }{%
    }
\]

\[
    \infer[choiceright]{%
        \tshp, \stk, \ts, \prog_{1} \pcond \prog_{2} \ \threadtransfer{\lid} \  \tshp, \stk, \ts, \prog_{2}
    }{%
    }
\]

\[
    \infer[norep]{%
        \tshp, \stk, \ts, \prept{\prog} \ \threadtransfer{\lid} \  \tshp, \stk, \ts, \pemp
    }{%
    }
\]

\[
    \infer[rep]{%
        \tshp, \stk, \ts, \prept{\prog} \ \threadtransfer{\lid} \  \tshp, \stk, \ts, \prog \pseq \prept{\prog}
    }{%
    }
\]

\[
    \infer[seqskip]{%
        \tshp, \stk, \ts, \pemp \pseq \prog \ \threadtransfer{\lid} \  \tshp, \stk, \ts, \prog
    }{%
    }
\]

\[
    \infer[seqnoskip]{%
        \tshp, \stk, \ts, \prog_{1} \pseq \prog_{2} \ \threadtransfer{\tll} \  \tshp', \stk', \ts', \prog_{1}' \pseq \prog_{2}
    }{%
        \tshp, \stk, \ts, \prog_{1} \ \threadtransfer{\tll} \  \tshp', \stk', \ts', \prog_{1}' 
    }
\]

\[
    \infer[fork]{%
        \tshp, \stk, \ts, \pfork{\var}{\prog} \ \threadtransfer{\lfork{\thid,\prog}} \  \tshp, \stk\remapsto{\var}{\thid}, \ts, \pemp 
    }{%
    }
\]

\[
    \infer[join]{%
        \tshp, \stk, \ts, \pjoin{\texpr} \ \threadtransfer{\ljoin{\eval{\texpr}_{\stk},\ts'}} \  \tshp, \max\Set{\ts,\ts'}, \pemp 
    }{%
    }
\]

\[
    \begin{rclarray}
        \dontcare, \dontcare \ \globaltransfer{ \dontcare } \ \dontcare, \dontcare & \defeq & \State \times \Translabel \times \State  \\
    \end{rclarray}
\]

\[
    \infer[single]{%
        \tshp, \tdpl \uplus \Set{ \thid \mapsto (\stk, \ts, \prog) } \ \globaltransfer{\tll} \  \tshp', \tdpl \uplus \Set{ \thid \mapsto (\stk', \ts', \prog') }
    }{%
        \tshp, \stk, \ts, \prog \ \threadtransfer{\tll} \  \tshp', \stk', \ts', \prog' 
        \quad \tll \notin \Set{\lfork{\dontcare,\dontcare},\ljoin{\dontcare,\dontcare}}
    }
\]

\[
    \infer[fork]{%
        \tshp, \tdpl \uplus \Set{ \thid \mapsto (\stk, \ts, \prog) } \ \globaltransfer{\lfork{\thid',\prog''}} \  \tshp', \tdpl \uplus \Set{ \thid \mapsto (\stk', \ts', \prog'), \thid' \mapsto (\lambda \var \ldotp 0, \ts', \prog'') }
    }{%
        \tshp, \stk, \ts, \prog \ \threadtransfer{\lfork{\thid',\prog''}} \  \tshp', \stk', \ts', \prog' 
    }
\]

\[
    \infer[join]{%
        \tshp, \tdpl \uplus \Set{ \thid \mapsto (\stk, \ts, \prog), \thid' \mapsto (\stk', \ts'', \pemp) } \ \globaltransfer{\ljoin{\thid',\ts''}} \  \tshp', \tdpl \uplus \Set{ \thid \mapsto (\stk', \ts', \prog')}
    }{%
        \tshp, \stk, \ts, \prog \ \threadtransfer{\ljoin{\thid',\ts''}} \  \tshp', \stk', \ts', \prog' 
    }
\]

\begin{lem}
    A history cannot be overwritten, i.e.\ \( \forall \tshp, \tshp', \loc,\ts \ldotp \tshp, \dontcare \globaltransfer{\dontcare} \tshp', \dontcare \land \tshp(\loc)(\ts)\isdef \implies \tshp(\loc)(\ts) = \tshp'(\loc)(\ts)\)
\end{lem}
\begin{proof}
    From the \( \pred{allowcommit} \).
\end{proof}

\begin{lem}
    \label{lem:read-before-write}
    All the reads of a transaction happen before all the writes. This is 
    \( \forall \tshp, \loc, \loc', \ts, \ts', \tsid \ldotp \tshp(\loc)(\ts) = (\dontcare, \rop, \tsid) \land \tshp(\loc')(\ts') = (\dontcare, \wop, \tsid) \implies \ts < \ts' \).
\end{lem}
\begin{proof}
    From the semantics that \( \ts_{s} < \ts_{e} \).
\end{proof}

\begin{lem}
    \label{lem:atoic-rw}
    All the reads of a transaction happen in the same time, so do all the writes. This is 
    \( \forall \tshp, \loc, \loc', \ts, \ts', \tsid, \op \in \Set{\rop, \wop} \ldotp \tshp(\loc)(\ts) =  \tshp(\loc')(\ts') = (\dontcare, \op, \tsid) \implies \ts = \ts' \).
\end{lem}
\begin{proof}
    From the \( \func{commit} \).
\end{proof}

Now we need to recover \( \vis \) and \( \ar \) from \( \tshp \).
First we need to extend the \( \tshp \) because there are some transactions that only have reads or writes.
We stretch the time by 3, and add extra operation for those transactions.
For a transaction \( \tsid \) that only has reads event, says in time \( \ts \), we add end operations \( (\bot, \tsid, \eop ) \) to the heap cells it reads in time \( (\ts + 1 ) \).
Similarly for a transaction that only has writes event, we add end operations \( (\bot, \tsid, \sop ) \) in time \( (\ts-1) \).

\[
\begin{rclarray}
    \func{stretch}(\tshp) & \defeq & \lambda \loc \ldotp \lambda \ts \ldotp
    \begin{funcarray}
        \tshp(\loc)(\ts') & \ts = 3 * \ts' \\
        \texttt{undef} & o.w. \\
    \end{funcarray} \\
    \func{extend}(\tshp) & \defeq & \lambda \loc \ldotp \tshp(\loc) \uplus \Set{\ts + 1 \mapsto (\bot, \tsid, \eop ) \ \middle| \ \tshp(\loc)(\ts) = (\dontcare, \tsid, \rop) \land \forall \loc', \ts' \ldotp \tshp(\loc')(\ts') = (\dontcare, \tsid, \wop)} \\
                         & & \quad \quad \quad \uplus \Set{\ts - 1 \mapsto (\bot, \tsid, \sop ) \ \middle| \ \tshp(\loc)(\ts) = (\dontcare, \tsid, \wop) \land \forall \loc', \ts' \ldotp \tshp(\loc')(\ts') = (\dontcare, \tsid, \rop)}
\end{rclarray}
\]

\begin{lem}
    After stretching the time by 3, there is no record in time \( 3 * \nat + 1 \) and \( 3 * \nat - 1 \).
    Therefore after extending, there are only \( (\dontcare, \tsid, \eop) \) in time \( 3 * \ts + 1 \) and only \( (\dontcare, \tsid, \sop) \) in time \( 3 * \ts - 1 \).
    This is \( \forall \tshp \ldotp \exists \tshp' = \func{extend} \circ \func{stetch}(\tshp) \ldotp \forall \loc, \ts \ldotp (\tshp'(\loc)(3 * \ts + 1)\isdef \implies \tshp'(\loc)(3 * \ts + 1) = (\dontcare, \dontcare, \eop) ) \land (\tshp'(\loc)(2 * \ts - 1)\isdef \implies \tshp'(\loc)(3 * \ts - 1) = (\dontcare, \dontcare, \sop) ) \).
\end{lem}
\begin{proof}
    trivial.
\end{proof}

\begin{lem}
    \label{lem:start-before-end}
    In the extended heap, all the reads or starts of a transaction happen before all the writes or end. This is 
    \( \forall \tshp, \loc, \loc', \ts, \ts', \op \in \Set{\rop, \sop}, \op' \in \Set{\wop, \eop}, \tsid \ldotp \tshp(\loc)(\ts) = (\dontcare, \op, \tsid) \land \tshp(\loc')(\ts') = (\dontcare, \op', \tsid) \implies \ts < \ts' \).
\end{lem}
\begin{proof}
    From Lemma \ref{lem:read-before-write} and the definition of \func{strech} and \func{extend}.
\end{proof}

\begin{lem}
    \label{lem:happen-in-same-time}
    For an extended heap, all the reads of a transaction happen in the same time, so do all the writes, starts and ends. This is 
    \( \forall \tshp, \loc, \loc', \ts, \ts', \tsid, \op \ldotp \tshp(\loc)(\ts) =  \tshp(\loc')(\ts') = (\dontcare, \op, \tsid) \implies \ts = \ts' \).
\end{lem}
\begin{proof}
    From Lemma \ref{lem:atoic-rw} and the definition of \func{strech} and \func{extend}.
\end{proof}

\begin{lem}
    \label{lem:unique-label}
    A transaction in an extended heap, must have either starts or reads, and either ends or writes.
\end{lem}
\begin{proof}
    From the definition of \func{extend}.
\end{proof}

\[
\begin{rclarray}
    (\settrans, \tvis, \tar) = \func{graph}(\tshp) & \defeq & (\Set{ \tsid \ \middle| \ \forall \loc \ldotp \tshp(\loc) = (\dontcare, \tsid, \dontcare)}, \\
                                                   & & \Set{(\tsid, \tsid') \ \middle| \ 
    \begin{array}{@{}l@{}}
        \exists \loc, \loc', \ts, \ts', \op \in \Set{\wop, \eop}, \op' \in \Set{\rop, \sop} \ldotp \\
        \ts < \ts' \land \tshp(\loc)(\ts) = (\dontcare, \tsid, \op) \land \tshp(\loc')(\ts') = (\dontcare, \tsid', \op')
    \end{array}
}, \\
                                                   & & \Set{(\tsid, \tsid') \ \middle| \ 
    \begin{array}{@{}l@{}}
        \exists \loc, \loc', \ts, \ts', \op, \op' \in \Set{\wop, \eop} \ldotp \\
        \ts < \ts' \land \tshp(\loc)(\ts) = (\dontcare, \tsid, \op) \land \tshp(\loc')(\ts') = (\dontcare, \tsid', \op')
    \end{array}
}, \\
\end{rclarray}
\]

\begin{lem}
    For an extended heap, the corresponding \tvis\ and \tar\ have no circle.
\end{lem}
\begin{proof}
    Assume there is a circle in \(\rvis\), says, \( \tsid_{1} \rvis \tsid_{2} \rvis \dots \rvis \tsid_{n} \rvis \tsid_{n+1} \), where \( \tsid_{1} = \tsid_{n+1} \).
    Therefore, \( \bigwedge\limits_{ 1 \leq i \leq n} \exists \loc, \loc', \ts, \ts', \op \in \Set{\wop, \eop}, \op' \in \Set{\rop, \sop} \ldotp \ts < \ts' \land \tshp(\loc)(\ts) = (\dontcare, \tsid_{i}, \op) \land \tshp(\loc')(\ts') = (\dontcare, \tsid_{i+1}, \op')\).
    By Lemma \ref{lem:unique-label} we can relabel read to start and write to end.
    Then by Lemma \ref{lem:happen-in-same-time}, we can define a list of starts and ends events that is ordered by time: \( \List{ (\tsid_{1},\eop), (\tsid_{2},\sop), (\tsid_{2},\eop), \dots, (\tsid_{n},\eop), (\tsid_{n+1},\sop) } \).
    By the assumption, we have \( \tsid_{1} = \tsid_{n+1} \), thus this contradict Lemma \ref{lem:start-before-end}.

    Similarly for \(\rtar\), the list of write events  \( \List{ (\tsid_{1},\eop), (\tsid_{2},\eop), (\tsid_{2},\eop), \dots, (\tsid_{n},\eop), (\tsid_{n+1},\eop) } \) contradict Lemma \ref{lem:happen-in-same-time}.
\end{proof}

\begin{lem}
    Given a \( \tshp \), the corresponding \((\settrans, \tvis, \tar)\) can be extended to \((\settrans, \vis, \ar)\) so that it is a valid dependency graph of snapshot isolation.
\end{lem}
\begin{proof}
    First, we extend the relations \( \tar \) to a total order \( \ar \).
    Initially, \( \ar \) includes all relations in \( \tar \).
    Given the definition, only if the ends or writes of transactions happen in the same time, those transactions are not ordered by \( \tar \).
    To simplify, we introduce an initial event, i.e.\ \( \forall \tsid \in \settrans \ldotp \tsid_{init} \rar \tsid \).
    From \( \tsid_{init} \), we pick the first two transactions \( \tsid_{1} \) and \( \tsid_{2} \) that are not ordered, this is, \( \forall \tsid, \tsid' \in \Set{\tsid'' \ \middle| \ \tsid'' \rar \tsid_{1} \lor \tsid'' \rar \tsid_{2} } \ldotp \tsid \rar \tsid' \lor \tsid' \rar \tsid \).
    Therefore, there exists an unique \( \tsid_{pre} \) where branching happens, i.e.\ \( \tsid_{pre} \rar \tsid_{1} \land \tsid_{pre} \rar \tsid_{2} \land \nexists \tsid \ldotp \tsid_{pre} \rar \tsid \rar \tsid_{1} \lor \tsid_{pre} \rar \tsid \rar \tsid_{2} \).
\end{proof}

\section{Logic}


\sx{
    Parametrised by CM.

    Generalise CAP by generalise the meaning of box assertions, stable and atomic update rule.

    Two example write skew and long fork.

    Discussion: our first step. For each consistency models, we want to build specific pattern to use the capabilities
    so to provide a more syntactic rules.

    Cite: FCSL(for the idea that recording some history in the assertions), 
    CAP,
    Iris (they have powerful framework, whether it also fits in the weaker consistency world?)
    TaDA
    Relaxed SL (The first separation logic that deal with weaker behaviour)

    Alexey, Reaon about ..... in Dis. Sys. A proof system to verify if 
    a system (a set of operations that might be installed with tokens to specify synchronisation ) satisfies invariant.
}

We present a logic that is parametrised by consistency models.
The logic has syntax that similar to concurrent abstract predicate (CAP),
but we generalise \emph{atomic update rule} so it is parametrised by consistency models.




the meaning \emph{box assertions} so they describe the partial views of key-value stores.
The \emph{stabilisation} by consistency models.
The  now 


We motivate the logic by the \emph{write skew} example under snapshot isolation (\cref{fig:write-skew-si-proof}).
This example distinguishes snapshot isolation (SI) from serialisability.
For serialisability that transactions appear one after another, only one key, \( \vx \) or \( \vy \), will be 1 at the end.
While for SI, transactions take a snapshot when they start, and concurrent transactions can commit as long as they write to different keys.
In \cref{fig:write-skew-si-proof}, both transactions may take snapshots where \( \vx \) and \( \vy \) are 0, and can commit, 
which yields result that \( \vx \) and \( \vy \) are 1.

\begin{figure}[!t]
\hrule
\[
\intass :
\begin{array}[t]{@{} c @{\quad} c @{\quad}  @{} }
\begin{rclarray}[t]
    \CB{L} & : & \vx \fpW 1 \sep \vy \fpR 0 \sep \null \fpA \cass{\CB{L}}{\lrid} \\
\end{rclarray}
&
\begin{rclarray}[t]
    \CB{R} & : & \vx \fpR 0 \sep \vy \fpW 1 \sep \null \fpA \cass{\CB{R}}{\lrid} \\
\end{rclarray}
\\
\begin{rclarray}[t]
    \CB{U} & : & \exsts{\V n} \vx \fpR \V{n} \\
\end{rclarray} 
&
\begin{rclarray}[t]
    \CB{U} & : & \exsts{\V n} \vy \fpR \V{n} \\
\end{rclarray} \\
\end{array}
\]
\[
\CB{L} \composeK \CB{L} \ \text{is undefined} \quad  \CB{R} \composeK \CB{R} \ \text{is undefined} \quad \CB{U} \ \text{is the unit}
\]
\hrule\vspace{5pt}
\[
\begin{session}
{\color{blue}P : } \specline{ \cass{\CB{L}}{\lrid} \sep \cass{\CB{R}}{\lrid} \sep \boxass{\vx \pt 0 \sep \vy \pt 0 }{\lrid}{\intass}  } \\
\begin{parl}
\begin{session}
    {\color{blue}P1 : } \specline{\cass{\CB{L}}{\lrid} \sep 
            \boxass{\vx \pt 0 \sep \vy \pt 0 }{\lrid}{\intass} \\
            {} \lor \boxass{\vx \pt 0 \sep ( \vy \pt 0 \lor \vy \pt 1 ) \sep \cass{\CB{R}}{\lrid} }{\lrid}{\intass} 
    } \\
    \txid_1 : \begin{transaction}
        {\color{blue}p1 : } \specline{\vx \fpI 0 \sep ( \vy \fpI 0 \lor \vy \fpI 1 )} \\
        \pderef{\pvar{b}}{\vy} ; 
        \quad \pifs{\pvar{b} = 0} 
        \pmutate{\vx}{1} ;
        \pife \\
        {\color{blue}q1 : } \specline{\vx \fpW 1 \sep  \vy \fpR 0 \lor {}\\
        \quad \vx \fpI 0 \sep \vy \fpR 1 )} \\
    \end{transaction} \\
    {\color{blue}Q1 : } \specline{ 
            \boxass{ \vx \pt 1 \sep \vy \pt 0 \sep \cass{\CB{L}}{\lrid} }{\lrid}{\intass} \\
            {} \lor \boxass{\vx \pt 1 \sep ( \vy \pt 0 \lor \vy \pt 1 ) \sep {} \\
            \cass{\CB{R}}{\lrid} \sep \cass{\CB{L}}{\lrid} }{\lrid}{\intass}  \\
            {} \lor \cass{\CB{L}}{\lrid} \sep \boxass{\vx \pt 0 \sep \vy \pt 1 \sep \cass{\CB{R}}{\lrid} }{\lrid}{\intass}  \\
    } \\
\end{session}
&
\begin{session}
    {\color{blue}P2 : } \specline{\cass{\CB{R}}{\lrid} \sep 
            \boxass{ ( \vx \pt 0 \sep \vy \pt 0 }{\lrid}{\intass} \\
            {} \lor \boxass{ ( \vx \pt 0 \lor \vx \pt 1 ) \sep \vy \pt 0 \sep \cass{\CB{L}}{\lrid} }{\lrid}{\intass} 
    } \\
    \txid_2 : \begin{transaction}
        {\color{blue}p2 : } \specline{ ( \vx \fpI 0 \lor \vx \fpI 1 ) \sep \vy \fpI 0 )} \\
        \pderef{\pvar{a}}{\vx} ; 
        \quad \pifs{\pvar{a} = 0} 
        \pmutate{\vy}{1} ; 
        \pife \\
        {\color{blue}q2 : } \specline{ \vx \fpR 0 \sep \vy \fpW 1 \lor {} \\
        \quad \vx \fpR 1 \sep \vy \fpI 0 )} \\
    \end{transaction} \\
    {\color{blue}Q2 : } \specline{ 
            \boxass{ \vx \pt 0 \sep \vy \pt 1 \sep \cass{\CB{R}}{\lrid} }{\lrid}{\intass} \\
            {} \lor \boxass{ ( \vx \pt 0 \lor \vx \pt 1 ) \sep \vy \pt 1 \sep {} \\
            \cass{\CB{L}}{\lrid} \sep \cass{\CB{R}}{\lrid} }{\lrid}{\intass}  \\
            {} \lor \cass{\CB{R}}{\lrid} \sep \boxass{\vx \pt 1 \sep \vy \pt 0 \sep \cass{\CB{L}}{\lrid} }{\lrid}{\intass}  \\
    } \\
\end{session}
\end{parl} \\
{\color{blue}Q : } \specline{ 
        \cass{\CB{L}}{\lrid} \sep \boxass{ \vx \pt 0 \sep \vy \pt 1 \sep \cass{\CB{R}}{\lrid} }{\lrid}{\intass} 
        \lor \cass{\CB{R}}{\lrid} \sep \boxass{\vx \pt 1 \sep \vy \pt 0 \sep \cass{\CB{L}}{\lrid} }{\lrid}{\intass}
        \lor \boxass{ \vx \pt 1 \sep \vy \pt 1 \sep \cass{\CB{L}}{\lrid} \sep \cass{\CB{R}}{\lrid} }{\lrid}{\intass}  \\
} \\
\end{session}
\]
\hrule
\caption{Interference (the top) and the sketch proof (the bottom) for write skew under snapshot isolation}
\label{fig:write-skew-si-proof}
\end{figure}

We often use \( \lpre, \lpost \) to denote \emph{transactional assertions} and \( \gpre, \gpost \) for \emph{program assertions}.
The transactional assertions, such as \( p1 \) in \cref{fig:write-skew-si-proof}, 
describe the state of local snapshots for transactions and more importantly the fingerprints.
The fingerprint is the transaction's contribution to the key-value store, that is, \emph{the first read preceding any write} and \emph{last write} of each key.
The transactional assertions for individual keys have the following forms: \( \vx \fpI 0 \), \( \vx \fpR 0\), \( \vx \fpW 0\), and \( \vx \fpRW (0,1) \),
where \( \otR \) and \( \otW \) are read and write labels respectively.
The first three asserts the only key \( \vx \) in the local snapshot has value 0 and it has not been touched (no label), 
has been read (\(\otR\)) and has been written (\(\otW\)) respectively.
The last one asserts that the key \( \vx \) currently has value 1, 
\emph{the first read preceding any write} fetched 0 and the \emph{the last write} updated the key to \( 1 \).
We extend the standard sequential separation logic rules in a ways that the first read to a key adds a read label to the assertion; 
and a write to a key adds a write label to assertion and updates the value.
All the detail is in \cref{sec:reasoning-transaction}.

The program assertions, such as \( P1 \) in \cref{fig:write-skew-si-proof}, 
describe clients' views on key-value stores, together with some capabilities for the purpose of reasoning.
The assertion in the form of \( \boxass{\bar{\lpre}}{\lrid}{\intass}\),
for example \( \boxass{\vx \pt 1 \sep \vy \pt 1}{\lrid}{\intass} \) in \( \gpre_1 \),
is called the \emph{shared region assertions} also know as \emph{boxed assertions},
where \( \lrid \) is a unique \emph{region identifier},  \( \intass \) is \emph{interference},
and \( \lpre \) describes the views on key-value stores.
Region are shareable and indivisible, \ie 
\( \boxass{\bar{\lpre}}{\lrid}{\intass} \sep \boxass{\bar{\lpost}}{\lrid}{\intass} \iff \boxass{\bar{\lpre} \land \bar{\lpost}}{\lrid}{\intass}\).

Interference is a set of actions to specify what transactions are allowed by the region.
Each action has the form \( \kap : \fp \),
where \( \kap \) is the \emph{client-specified capability} and \( \fp \) is the \emph{fingerprint},
describes that if a client holds the capability \( \kap \), 
it is allowed to execute a transaction with the fingerprint \( \fp \).
For instance in \cref{fig:write-skew-si-proof}, 
the \( \CB{L}\) allows a client to read \( \vy \) when it is 0 (\(\vy \fpR 0\)) and write 1 to \( \vx \) (\(\vx \fpW 1\)),
and similarly \( \CB{R} \) allows a client to read \( \vx \) when it is 0 and write 1 to \( \vy \).
The fingerprint \( \fp \) in an action also specifies capabilities transformation.
For example, \( \null \fpA \cass{\CB{L}}{\lrid} \) means that \( \CB{L} \) needs to return to the shared region after the transaction is committed.
Last, the capabilities forms \emph{a partial commutative monoid (PCM)} where \( \composeK \) denotes the composition function.
In the write skew example (\cref{fig:write-skew-si-proof}), 
both \( \CB{L} \) and \( \CB{R} \) are unique as the compositions are undefined and \( \CB{U} \) is the unit.

Interference induces a label transitions system upon states \( (\mkvs, \vi, \ca) \)
where the labels are actions from the interference.
Then the \emph{invariant} of a region \( \func{inv}{\lrid} \) is the set of states reachable from 
the initial states of the regions provided by the function \( \func{init}{\lrid}\).

Given the invariant, a state \( (\mkvs, \vi, \ca) \) satisfies the assertion \( \boxass{\bar{\lpre}}{\lrid}{\intass} \),
if it is in the invariant of the region \( (\mkvs, \vi, \ca) \in \func{inv}{\lrid} \) and the snapshot induced by the view \( \snapshot(\mkvs, \vi) \) and 
the shared capabilities satisfy \( \bar{\lpre} \) (details in \cref{def:prog-assertion}).

Now let discuss the sketch proof in \cref{fig:write-skew-si-proof}, especially the rule for committing a transaction (other rules are standard and can be found in \cref{sec:reasoning-prog})
\[
    \inferrule[\rl{PRCommit}]{%
        \tripleL{\lpre}{\trans}{\lpost} 
        \\ \repartition{\gpre}{\gpost}{\lpre}{\lpost}
        \\\\ \stable{\gpre, \como} 
        \\ \stable{\gpost, \como} 
    }{%
        \tripleG{\gpre}{ \ptrans{\trans} }{\gpost}
    }
\]

An assertion \( \gpre \) is \emph{stable} under execution test \( \ET \), written \( \stable{\gpre, \como} \), if it holds against the interference.
Specifically, stabilisation has two meanings: \textbf{(i)} the environment might interfere and change the state of key-value stores, note that, but not the views of the current client since the views are local; 
\textbf{(ii)} consequentially, the client can advance the views.
For example, \( P1 \) is stable as it describes the states that either the environment does nothing \( \boxass{\vx \pt 0 \sep \vy \pt 0 }{\lrid}{\intass} \),
or the environment has performed the action associated with \( \cass{\CB{R}}{\lrid}\), 
after which the environment cannot do any further action since the capabilities \( \cass{\CB{R}}{\lrid}\) is in the shared region,
\ie \( \boxass{\vx \pt 0 \sep ( \vy \pt 0 \lor \vy \pt 1 ) \sep \cass{\CB{R}}{\lrid} }{\lrid}{\intass}  \).

Let discuss the assertions \( \vy \pt 0  \lor \vy \pt 1  \sep \cass{\CB{R}}{\lrid} \) from \( P1 \).
The environment returns the capability \( \cass{\CB{R}}{\lrid} \), it means the environment commits a transaction that writes 1 to \( vy \) and the capability \( \cass{\CB{R}}{\lrid} \), thus the value \( 0 \) is out-of-date.
Because of stabilisation, the view might be advanced so we get an up-to-date view \( \vy \pt 1 \).

\[
\begin{rclarray}
    \repartition{\gpre}{\gpost}{\lpre}{\lpost} & \defeq & 
    \begin{array}[t]{@{}l@{}}
        \fora{ \w, \mkvs, \vi, \lenv, \stk } 
        \w \in \evalW{\gpre} 
        \land (\mkvs, \vi) = \eraseW{\w}
        \implies  \\
        \quad \exsts{\h_\lpre, \h}
        \h_\lpre \composeH \h = \getSN{\mkvs, \vi}
        \land (\h, \unitO) \in \evalLS{\lpre}  \\
        \quad {} \land \fora{\w', \mkvs', \vi', \stk', \opset, \cl}  \\
        \qquad  (\stub, \opset) \in \evalLS[\lenv, \stk']{\lpost} 
        \land \mkvs' \in \updM{\mkvs, \vi, \cl, \opset} \\
        \qquad {} \land \et \vdash (\mkvs, \vi) \csat \opset : \vi' 
        \land (\mkvs', \vi') = \eraseW{\w'} 
        \land (\w, \w') \in \Guar \\
        \qqqquad \implies \w' \in \evalW[\lenv, \stk']{\gpost}
    \end{array} 
\end{rclarray}                          
\]




Because of atomicity, a transaction takes a snapshot and afterword executes locally on the snapshot and only commits the fingerprint by the end.
For example, \( P1 \) asserts views where \( \vx \) has value 0 and \( \vy \) has value either 0 or 1, the precondition \( p1 \) for transaction \( \txid_1 \) describes exactly the same states, and initial all keys have no fingerprint.
By the end of the transaction, the postcondition \( q1 \) describes that the transaction \( \txid_1 \) either read \( \vy \) with 0 and wrote 1 to \( \vx \),
or only read \( \vy \) with 0 and did nothing to \( \vx \).

To commit a fingerprint under SI, we first need to check the view before the transaction satisfies certain constraints with respect to the fingerprint.
SI requires that if a view observe some transactions, it should observe everything before.
Also if there is a write fingerprint for a key, the view for that key must be up-to-date, so to avoid write-write conflict.
Now given the view on the key-value store \( \gpre \), and we are about to commit the fingerprint given by the transaction \( \lpost \):
\textbf{(i)} for any key that has been over-written, we create a new version with the value;
\textbf{(ii)} for any key that has been read, the view before points to a version containing the same value;
\textbf{(iii)} the view after moves the up-to-date versions for all keys;
and last \textbf{(iv)} we need to check the update is allowed by the local capabilities and transfer the capabilities if necessary.
This will give us the following assertions \emph{before stabilisation}:
\[
\boxass{ \vx \pt 1 \sep \vy \pt 0 \sep \cass{\CB{L}}{\lrid} }{\lrid}{\intass} 
\lor \boxass{\vx \pt 1 \sep \vy \pt 1  \sep \cass{\CB{R}}{\lrid} \sep \cass{\CB{L}}{\lrid} }{\lrid}{\intass}
\lor \cass{\CB{L}}{\lrid} \sep \boxass{\vx \pt 0 \sep \vy \pt 1 \sep \cass{\CB{R}}{\lrid} }{\lrid}{\intass} 
\]
We need to stabilise the assertions.
The environment can commit a transaction that updates \( \vy \) to 1 under the state that satisfies the first disjunction, 
which yields \( \boxass{\vx \pt 1 \sep \vy \pt 0 \sep \cass{\CB{R}}{\lrid} \sep \cass{\CB{L}}{\lrid} }{\lrid}{\intass} \).
Now we have a stable assertions \( Q1 \) shown in \cref{fig:write-skew-si-proof}.

Following similar reasoning, we get \( Q2 \) for right-hand-side client.
The final \( Q \) is  \( Q = Q1 \sep Q2 \) provided that \( \boxass{\bar{\lpre}}{\lrid}{\intass} \sep \boxass{\bar{\lpost}}{\lrid}{\intass} \iff \boxass{\bar{\lpre} \land \bar{\lpost}}{\lrid}{\intass}\).
It means the final views and key-value stores are those come from both \( Q1 \) and \( Q2 \).






Note that unlike serialisability, the view might be out-of-date for SI.
For instance, the assertion \( \boxass{\vx \pt 0 \sep ( \vy \pt 0 \lor \vy \pt 1 ) \sep \cass{\CB{R}}{\lrid} }{\lrid}{\intass}\) from \( P1 \) asserts views where \( \vy \) might point to either 0 or 1.
Since the capability \( \cass{\CB{R}}{\lrid} \) has been returned to the region, the view where \( \vy \) point to 0 is out-of-date.




\begin{figure}[!t]
\hrule
\[
\intass :
\begin{array}[t]{@{} c @{\quad} c @{\quad} c @{\quad} c @{} }
\begin{rclarray}[t]
    \CB{L} & : & \exsts{\V{n}} \vx \fpR \V{n} \sep \vx \fpW \V{n} + 1 \sep \null \fpA \cass{\CB{L}}{\lrid} \\
\end{rclarray}
&
\begin{rclarray}[t]
    \CB{R} & : & \exsts{\V{n}} \vx \fpR \V{n} \sep \vx \fpW \V{n} + 1 \sep \null \fpA \cass{\CB{R}}{\lrid} \\
\end{rclarray} \\
\end{array}
\]
\[
\CB{L} \composeK \CB{L} \ \text{is undefined} \quad  \CB{R} \composeK \CB{R} \ \text{is undefined}
\]
\hrule\vspace{5pt}
\[
\begin{session}
 \specline{ \cass{\CB{L}}{\lrid} \sep \cass{\CB{R}}{\lrid} \sep \boxass{\vx \pt 0}{\lrid}{\intass}  } \\
\begin{parl}
\begin{session}
    \specline{\cass{\CB{L}}{\lrid} \sep 
            \boxass{\vx \pt 0}{\lrid}{\intass} 
            \lor \boxass{\vx \pt 0  \lor \vx \pt 1 \sep \cass{\CB{R}}{\lrid} }{\lrid}{\intass} 
    } \\
    \txid_1 : \begin{transaction}
        \specline{ \vx \fpI 0 \lor \vx \fpI 1 } \\
        \pderef{\pvar{a}}{\vy} ; 
        \pmutate{\vx}{\pv{a} + 1} ; \\
        \specline{ \vx \fpW 2 \lor \vx \fpW 1 } \\
    \end{transaction} \\
    \specline{ 
            \boxass{ \vx \pt 1 \sep \cass{\CB{L}}{\lrid} }{\lrid}{\intass} \lor {} \\
            \boxass{ \vx \pt 1 \lor \vx \pt 2 \sep \cass{\CB{R}}{\lrid} \sep \cass{\CB{L}}{\lrid} }{\lrid}{\intass} \\
    } \\
\end{session}
&
\begin{session}
    \specline{\cass{\CB{R}}{\lrid} \sep 
            \boxass{\vx \pt 0}{\lrid}{\intass} 
            \lor \boxass{\vx \pt 0  \lor \vx \pt 1 \sep \cass{\CB{L}}{\lrid} }{\lrid}{\intass} 
    } \\
    \txid_2 : \begin{transaction}
        \specline{ \vx \fpI 0 \lor \vx \fpI 1 } \\
        \pderef{\pvar{b}}{\vy} ; 
        \pmutate{\vx}{\pv{b} + 1} ; \\
        \specline{ \vx \fpW 2 \lor \vx \fpW 1 } \\
    \end{transaction} \\
    \specline{ 
            \boxass{ \vx \pt 1 \sep \cass{\CB{R}}{\lrid} }{\lrid}{\intass}  \lor {} \\
            \boxass{ \vx \pt 1 \lor \vx \pt 2 \sep \cass{\CB{L}}{\lrid} \sep \cass{\CB{R}}{\lrid} }{\lrid}{\intass} \\
    } \\
\end{session}
\end{parl} \\
    \specline{ 
            \boxass{ \vx \pt 1 \lor \vx \pt 2 \sep \cass{\CB{L}}{\lrid} \sep \cass{\CB{R}}{\lrid} }{\lrid}{\intass} \\
    } \\
\end{session}
\]
\hrule\vspace{5pt}
\[
\begin{session}
 \specline{ \cass{\CB{L}}{\lrid} \sep \cass{\CB{R}}{\lrid} \sep \boxass{\vx \pt 0}{\lrid}{\intass}  } \\
\begin{parl}
\begin{session}
    \specline{\cass{\CB{L}}{\lrid} \sep 
            \boxass{\vx \pt 0}{\lrid}{\intass} 
            \lor \boxass{ \vy \pt 1 \sep \cass{\CB{R}}{\lrid} }{\lrid}{\intass} 
    } \\
    \txid_1 : \begin{transaction}
        \specline{ \vx \fpI 0 \lor \vx \fpI 1 } \\
        \pderef{\pvar{a}}{\vy} ; 
        \pmutate{\vx}{\pv{a} + 1} ; \\
        \specline{ \vx \fpW 2 \lor \vx \fpW 1 } \\
    \end{transaction} \\
    \specline{ 
            \boxass{ \vx \pt 1 \sep \cass{\CB{L}}{\lrid} }{\lrid}{\intass}
            \lor \boxass{ \vx \pt 2 \sep \cass{\CB{R}}{\lrid} \sep \cass{\CB{L}}{\lrid} }{\lrid}{\intass} \\
    } \\
\end{session}
&
\begin{session}
    \specline{\cass{\CB{R}}{\lrid} \sep 
            \boxass{\vx \pt 0}{\lrid}{\intass} 
            \lor \boxass{\vy \pt 1 \sep \cass{\CB{L}}{\lrid} }{\lrid}{\intass} 
    } \\
    \txid_2 : \begin{transaction}
        \specline{ \vx \fpI 0 \lor \vx \fpI 1 } \\
        \pderef{\pvar{b}}{\vy} ; 
        \pmutate{\vx}{\pv{b} + 1} ; \\
        \specline{ \vx \fpW 2 \lor \vx \fpW 1 } \\
    \end{transaction} \\
    \specline{ 
            \boxass{ \vx \pt 1 \sep \cass{\CB{R}}{\lrid} }{\lrid}{\intass}
            \lor \boxass{\vx \pt 2 \sep \cass{\CB{L}}{\lrid} \sep \cass{\CB{R}}{\lrid} }{\lrid}{\intass} \\
    } \\
\end{session}
\end{parl} \\
    \specline{ 
            \boxass{ \vx \pt 2 \sep \cass{\CB{L}}{\lrid} \sep \cass{\CB{R}}{\lrid} }{\lrid}{\intass} \\
    } \\
\end{session}
\]
\hrule
\caption{Sketch proofs for concurrent increments under serialisability (the bottom) and casual consistency (the middle)}
\label{fig:increment-proof}
\end{figure}

The \cref{fig:increment-proof} shows sketch proofs for the same \emph{concurrent increments} under different consistency models, causal consistency (the middle) and serialisability (the bottom).
For causal consistency, a transaction can update keys even if started with an out-of-date view on those keys.
That means, the fingerprint \( \vx \fpW 1\) is allowed to commit when the view for \( \vx \) is out-of-date, \ie \( \vx \pt 0 \).
The same situation is disallowed under serialisability and snapshot isolation.


\begin{figure}[!t]
\hrule
\[
\intass :
\begin{array}[t]{@{} c @{\quad} c @{\quad}  c @{} }
\begin{rclarray}[t]
    \CB{L1} \composeK \CB{L2} & : & \vx \fpW 1 \sep \null \fpA \cass{\CB{L1}}{\lrid} \\
\end{rclarray}
&
\begin{rclarray}[t]
    \CB{R1} \composeK \CB{R2} & : & \vy \fpW 1 \sep \null \fpA \cass{\CB{R1}}{\lrid} \\
\end{rclarray}
&
\begin{rclarray}[t]
    \CB{U} & : & \exsts{\V{n}} \vy \fpR \V{n} \\
\end{rclarray}
\\
\begin{rclarray}[t]
    \CB{L2} & : & \vy \fpR 0 \sep \null \fpA \cass{\CB{L2}}{\lrid} \\
\end{rclarray}
&
\begin{rclarray}[t]
    \CB{R2} & : & \vx \fpR 0 \sep \null \fpA \cass{\CB{R2}}{\lrid} \\
\end{rclarray}
&
\begin{rclarray}[t]
    \CB{U} & : & \exsts{\V{n}} \vx \fpR \V{n} \\
\end{rclarray}
\end{array}
\]


\hrule\vspace{5pt}
\[
\begin{session}
\specline{ \cass{\CB{L1}}{\lrid} \sep \cass{\CB{L2}}{\lrid} \sep \cass{\CB{R1}}{\lrid} \sep \cass{\CB{R2}}{\lrid} \sep \boxass{\vx \pt 0 \sep \vy \pt 0}{\lrid}{\intass}  } \\
\begin{parl}
\begin{session}
    \specline{ 
        \cass{\CB{L1}}{\lrid} \sep \cass{\CB{L2}}{\lrid} \sep \boxass{\vx \pt 0 \sep \vy \pt 0}{\lrid}{\intass} \\
        {} \lor \boxass{\vx \pt 0 \sep ( \vy \pt 0 \lor \vy \pt 1 ) \\ {} \sep \cass{\CB{R1}}{\lrid} 
        \sep ( \cass{\CB{R2}}{\lrid} \lor \assemp ) }{\lrid}{\intass} 
    } \\
    \begin{transaction}
        \pmutate{\vx}{1} ; \\
    \end{transaction} \\
    \specline{ 
        \cass{\CB{L2}}{\lrid} \sep \boxass{\vx \pt 1 \sep \vy \pt 0 \sep \cass{\CB{L1}}{\lrid} }{\lrid}{\intass} \lor {} \\
        \boxass{\vx \pt 1 \sep ( \vy \pt 0 \lor \vy \pt 1 ) \sep \cass{\CB{R1}}{\lrid} 
        \sep \cass{\CB{L1}}{\lrid} }{\lrid}{\intass}  \\
        {} \lor \boxass{\vx \pt 1 \sep  \vy \pt 1 \sep \cass{\CB{R1}}{\lrid} 
        \sep \cass{\CB{R2}}{\lrid} \sep \cass{\CB{L1}}{\lrid} }{\lrid}{\intass} 
    } \\
    \begin{transaction}
        \pderef{\pv{a}}{\vy}; \\
        \pifs{ \pv{a} = 0 } 
        \passign{\pv{f1}}{1} ; 
        \pife
    \end{transaction} \\
    \specline{ 
        \boxass{\vx \pt 1 \sep \vy \pt 0 \sep \cass{\CB{L1}}{\lrid} \sep \cass{\CB{L2}}{\lrid} }{\lrid}{\intass} \\
        {} \lor \boxass{\vx \pt 1 \sep ( \vy \pt 0 \lor \vy \pt 1 ) \\ 
        {} \sep \cass{\CB{R1}}{\lrid} \sep \cass{\CB{L1}}{\lrid} \sep \cass{\CB{L2}}{\lrid} }{\lrid}{\intass}  \lor {} \\
        \cass{\CB{L2}}{\lrid} \sep  \boxass{\vx \pt 1 \sep \vy \pt 1  \sep \cass{\CB{R1}}{\lrid} \\ {}
        \sep ( \cass{\CB{R2}}{\lrid} \lor \assemp ) \sep \cass{\CB{L1}}{\lrid} }{\lrid}{\intass} \\
    } \\
\end{session}
&
\begin{session}
    \specline{ 
        \cass{\CB{R1}}{\lrid} \sep \cass{\CB{R2}}{\lrid} \sep \boxass{\vx \pt 0 \sep \vy \pt 0}{\lrid}{\intass} \\
        {} \lor \boxass{(\vx \pt 0 \lor \vx \pt 1) \sep \vy \pt 0  \\ {} \sep \cass{\CB{L1}}{\lrid} 
        \sep ( \cass{\CB{L2}}{\lrid} \lor \assemp ) }{\lrid}{\intass} 
    } \\
    \begin{transaction}
        \pmutate{\vy}{1} ; \\
    \end{transaction} \\
    \specline{ 
        \cass{\CB{R2}}{\lrid} \sep \boxass{\vx \pt 0 \sep \vy \pt 1 \sep \cass{\CB{R1}}{\lrid} }{\lrid}{\intass} \lor {} \\
        \boxass{(\vx \pt 0 \lor \vx \pt 1) \sep \vy \pt 1 \sep \cass{\CB{L1}}{\lrid} 
        \sep \cass{\CB{R1}}{\lrid} }{\lrid}{\intass} \\
        {} \lor \boxass{\vx \pt 1 \sep \vy \pt 1 \sep \cass{\CB{L1}}{\lrid} 
        \sep \cass{\CB{L2}}{\lrid} \sep \cass{\CB{R1}}{\lrid} }{\lrid}{\intass} \\
    } \\
    \begin{transaction}
        \pderef{\pv{b}}{\vy}; \\
        \pifs{ \pv{b} = 0 } 
        \passign{\pv{f2}}{1} ; 
        \pife
    \end{transaction} \\
    \specline{ 
        \boxass{\vx \pt 0 \sep \vy \pt 1 \sep \cass{\CB{R1}}{\lrid} \sep \cass{\CB{R2}}{\lrid} }{\lrid}{\intass} \\
        {} \lor \boxass{( \vx \pt 0 \lor \vx \pt 1 ) \sep \vy \pt 1 \sep \\ 
        {} \cass{\CB{L1}}{\lrid} \sep \cass{\CB{R1}}{\lrid} \sep \cass{\CB{R2}}{\lrid} }{\lrid}{\intass}  \lor {} \\
        \cass{\CB{R2}}{\lrid} \sep  \boxass{\vx \pt 1 \sep \vy \pt 1  \sep \cass{\CB{L1}}{\lrid} \\ {}
        \sep ( \cass{\CB{L2}}{\lrid} \lor \assemp ) \sep \cass{\CB{R1}}{\lrid} }{\lrid}{\intass} \\
    } \\
\end{session}
\end{parl} \\
\specline{ 
    \cass{\CB{L2}}{\lrid} \sep \boxass{\vx \pt 1 \sep \vy \pt 1 \sep 
    \cass{\CB{L1}}{\lrid} \sep \cass{\CB{R1}}{\lrid} \sep \cass{\CB{R2}}{\lrid} }{\lrid}{\intass} \lor 
    \cass{\CB{R2}}{\lrid} \sep \boxass{\vx \pt 1 \sep \vy \pt 1 \sep 
    \cass{\CB{L1}}{\lrid} \sep \cass{\CB{R1}}{\lrid} \sep \cass{\CB{L2}}{\lrid} }{\lrid}{\intass} \\
    {} \lor \cass{\CB{L2}}{\lrid} \sep \cass{\CB{R2}}{\lrid} \sep \boxass{\vx \pt 1 \sep \vy \pt 1  \sep \cass{\CB{L1}}{\lrid} \sep \cass{\CB{R1}}{\lrid} }{\lrid}{\intass} \\
} \\
\end{session}
\]
\hrule
\caption{Sketch proofs for long fork under snapshot isolation}
\label{fig:long-fork-proof}
\end{figure}

We can also prove the \emph{long fork} (\cref{fig:long-fork-proof}), which distinguish snapshot isolation (SI) from parallel snapshot isolation (PSI) (\cref{fig:long-fork-proof}).
We use the similar pattern by putting back the capabilities to the region once the capabilities has been used.
Note that we encode the program order in interference so that if there is a write, the write must happen before the read.
Let discuss that the postcondition after left client commits the first transaction, especially the case from the second line, \ie \( \boxass{\vx \pt 1 \sep \vy \pt 0  \sep \cass{\CB{R1}}{\lrid} \sep \cass{\CB{L1}}{\lrid} }{\lrid}{\intass} \).
It comes from stabilising \( \boxass{\vx \pt 1 \sep \vy \pt 0 \sep \cass{\CB{L1}}{\lrid} }{\lrid}{\intass} \) without updating  the view.
%Because both \( \cass{\CB{L1}}{\lrid} \) and \( \cass{\CB{L2}}{\lrid} \), it means the environment has updated \( \vy \) to 1 and then read \( \vx \) with value 0 in order before the current client.
%Because the left client updated the \( \vy \) to 1, while the environment successfully read \( \vx \) with a old value 0, the snapshot isolation requires the view must include all transaction before the read, therefore the view must include the write of \( \vy \).


\ifTechReport
\subsection{Reasoning inside transactions}

Recall that a transaction takes a snapshot of the kv-store and commits the fingerprint by the end.
Because of the atomicity, only the \emph{first reads preceding any write} and the \emph{last writes} of keys are contained in the fingerprint.
All the intermediate reads and writes are not observable to other transactions and have no effect on the key-value store.
To capture the state of the local snapshot as well as the fingerprint, 
the \emph{transactional assertions} (\cref{def:local_assertions}) extend normal sequential separate logic assertions with read and/or write labels, 
\eg \( \vx \fpR 0 \), \( \vy \fpW 1 \) and \( \pv{z} \fpRW (0,1) \).


\begin{definition}[Transactional assertions]
\label{def:fingerprint}
\label{def:local_assertions}
\label{def:logical-expr}
Assume a countably infinite set of \emph{logical variables} $\lvar \in \LVar$.
The set of \emph{logical expressions} $\lexpr \in \LExpr$ is defined by the inductive grammar:
\(
\begin{rclarray}
   \lexpr & ::= & \val \mid \lvar \mid \var \mid \lexpr + \lexpr \mid  \dots 
\end{rclarray}
\)
where \(\val \in \Val\)  and \(\var \in \Vars\).
The \emph{logical expression evaluation} function, $\evalLE[(., .)]{.}:\LExpr \times \Stacks \times \LEnv\rightharpoonup \Val$, is defined inductively over the structure of logical expressions,
where the \emph{logical environments} \(\lenv \in \LEnv: \LVar \parfun \Val\) associates logical variables with values:
%
\[
\begin{array}{@{}c@{}}
    \begin{rclarray}
        \evalLE{\val} & \defeq & \val \\
    \end{rclarray}
    \quad
    \begin{rclarray}
        \evalLE{\lvar} & \defeq & \lenv(\lvar) \\
    \end{rclarray}
    \quad
    \begin{rclarray}
        \evalLE{\var} & \defeq & \txstk(\var) \\
    \end{rclarray} 
    \quad
    \begin{rclarray}
        \evalLE{\lexpr_1 + \lexpr_2} & \defeq & \evalLE{\lexpr_1} + \evalLE{\lexpr_2} \\
    \end{rclarray}
    \dots
\end{array}
\]
The set of \emph{transactional assertions}, $\lpre,  \lpost, \fp \in \LAst$, is defined by the following grammars:
\[
\begin{rclarray}
	\lpre, \lpost, \fp & ::= & \False \mid \True \mid \lpre \land \lpost \mid \lpre \lor \lpost \mid \exsts{\lvar} \lpre \mid \lpre \implies \lpost \\
    & & \mid \Emp \mid \lexpr \fpI \lexpr \mid \lexpr \fpR \lexpr \mid \lexpr \fpW \lexpr \mid \lexpr \fpRW (\lexpr, \lexpr)  \mid \lpre \sep \lpost  \\
\end{rclarray}	 
\]
The \emph{transactional assertion interpretation function}, $\evalLS[(.,.)]{.}: \LAst \times \LEnv \times \LAst \parfun \powerset{\Heaps \times \Opsets} $, is defined over the structure of local assertions, where the composition for snapshots \( \composeH \defeq \uplus \) is standard disjointed union on two functions and the composition for fingerprints \( \opset \composeO \opset' \defeq \opset \uplus \opset'\) when they contain different keys, \ie \( \opset\projection{2} \cap \opset'\projection{2} = \emptyset\):
\[
\begin{array}{@{}c @{\qquad} c @{}}
\begin{rclarray}
	\evalLS{\assfalse} & \eqdef & \emptyset \\
	\evalLS{\asstrue} & \defeq & \Heaps \times \Opsets \\
	\evalLS{\lpre \land \lpost} & \defeq & \evalLS{\lpre} \cap \evalLS{\lpost} \\
	\evalLS{\lpre \lor \lpost} & \defeq & \evalLS{\lpre} \cup \evalLS{\lpost} \\
\end{rclarray}
&
\begin{rclarray}
	\evalLS{\exsts{\lvar} \lpre} & \defeq & \bigcup\limits_{\val \in \textnormal{\Val}}\evalLS[\lenv\remapsto{\lvar}{\val}, \stk]{\lpre}  \\
	\evalLS{\lpre \implies \lpost} & \defeq & \Setcon{(\h, \opset)}{(\h , \opset) \in \evalLS{\lpre} \implies (\h , \opset) \in \evalLS{\lpost}}\\
	\evalLS{\assemp} & \defeq & \Set{ ( \unitH, \unitE) }  \\
	\evalLS{ \lexpr_1 \fpI \lexpr_2 } & \defeq & \Set{\left(\Set{\evalLE{\lexpr_1} \mapsto \evalLE{\lexpr_2} }, \unitO\right)} \\
\end{rclarray}
\end{array}
\]
\[
\begin{rclarray}
	\evalLS{ \lexpr_1 \fpR \lexpr_2 } & \defeq & \Set{\left(\Set{\evalLE{\lexpr_1} \mapsto \evalLE{\lexpr_2} }, \Set{(\otR, \evalLE{\lexpr_1},\evalLE{\lexpr_2})}\right)} \\
	\evalLS{ \lexpr_1 \fpW \lexpr_2 } & \defeq & \Set{\left(\Set{\evalLE{\lexpr_1} \mapsto \evalLE{\lexpr_2} }, \Set{(\otW, \evalLE{\lexpr_1},\evalLE{\lexpr_2})}\right)} \\
	\evalLS{ \lexpr_1 \fpRW (\lexpr_2, \lexpr_3) } & \defeq & \Set{\left(\Set{\evalLE{\lexpr_1} \mapsto \evalLE{\lexpr_3} }, \Set{(\otR, \evalLE{\lexpr_1},\evalLE{\lexpr_2}), (\otW, \evalLE{\lexpr_1},\evalLE{\lexpr_3})}\right)} \\
	\evalLS{\lpre \sep \lpost} & \defeq & 
    \Setcon{
        (\h_1 \composeH \h_2, \opset_{1} \composeE \opset_{2})
    }{ 
        (\h_{1},\opset_{1}) \in \evalLS{\lpre} 
        \land (\h_{2}, \opset_{2} ) \in \evalLS{\lpost} 
    } 
\end{rclarray}%
\]
\end{definition}

The \emph{transactional assertions} (\cref{def:local_assertions}) have \( \assfalse \), \(\asstrue \), conjunction \( \land \), disjunction \( \lor \), existential quantification \( \exists \), implication \( \implies  \), empty \( \assemp \), fingerprint assertions \( \stub \stackrel{\stub}{\hookrightarrow} \stub \) and separation conjunction \( \sep \).
They describes the state of local snapshot used by a transaction and more importantly the fingerprint of the transaction.
They are interpreted to pairs of snapshots and fingerprints.

A \emph{fingerprint assertion} describes the possible global effect from a transaction.
It includes the default \(\lexpr_{1} \fpI \lexpr_{2} \), the \emph{first read preceding any write} \( \lexpr_{1} \fpR \lexpr_{2} \), \emph{last write} \( \lexpr_{1} \fpW \lexpr_{2} \) for the key \( \lexpr_{1} \) and the combination of them \( \lexpr_{1} \fpRW \lexpr_{2} \).
The \( \lexpr_{1} \fpI \lexpr_{2} \) means the only key \( \lexpr_{1} \) in the local snapshot has value \( \lexpr_{2} \),
and the key has no associated fingerprint.
The \( \lexpr_{1} \fpR \lexpr_{2} \) means the key has been read before any other write carrying value \( \lexpr_{2} \) and the current value for the key is also \( \lexpr_{2} \).
The \( \lexpr_{1} \fpW \lexpr_{2} \) means the key has been written at least once, and the last written value is \( \lexpr_{2} \).
Because read does not change the state of snapshot, and the write fingerprint corresponds to the last write for the key,
so the state of the snapshot matches the fingerprint for cases \( \lexpr_{1} \fpR \lexpr_{2} \) and  \( \lexpr_{1} \fpW \lexpr_{2} \).
Last, The combined fingerprint \( \lexpr_{1} \fpRW (\lexpr_{2}, \lexpr_{3}) \) means the key has been read and then written at least once, the first read fetched value \( \lexpr_{2} \), and the last written value and the current state of the local snapshot for the key are both \( \lexpr_{3} \).

Other transactional assertions have standard interpretations.
Note that the separation conjunction \( \sep \) asserts two local snapshots and fingerprints when the keys are disjointed.
Observe that program expressions $\Expr$ (\cref{fig:semantics}) are contained in logical expressions $\LExpr$ (\cref{def:local_assertions}), \ie $\Expr \subset \LExpr$. 

The proof rules for transactions (\cref{fig:rule-trans}) are standard except \rl{TRLookup} and \rl{TRMutate}.
The \rl{TRLookup} rule adds read label only if there is no write label.
Because once a key has been written, the following reads are local to the transaction.
However, the local read always needs to match the current state of snapshot.
Especially for the case when the precondition is \( \lexpr \fpRW (\lexpr'', \lexpr') \),
the current state of the key is the last written value \( \lexpr' \).
The\rl{TRMutate} rule changes the state of the key and more importantly adds write label.
For the case when the precondition is \( \lexpr \fpRW (\lexpr'', \lexpr') \), 
the rule changes the value \( \lexpr' \) to the new written value \( \lexpr''' \)
but keeps the old read value \( \lexpr'' \) the same.

\begin{figure}[!t]
\sx{Font for E}
\hrule
\begin{mathpar}
    \inferrule[\rl{TRLookup}]{%
        \var \notin \func{fv}{\lexpr}  
        \\ \lpre \toFP{\otR(\expr, \lexpr)} \lpost
    }{%
        \tripleL{ \lpre }{ \plookup{\var}{\expr} }{\var \dot= \lexpr \sep \lpost\sub{\var}{\lexpr} }
    }
    \and
    \inferrule[\rl{TRMutate}]{
        \lpre \toFP{\otW(\expr_{1},\expr_{2})} \lpost
    }{%
        \tripleL{ \lpre }{ \pmutate{\expr_1}{\expr_2} }{ \lpost } 
    }
    \and
    \inferrule[\rl{TRAss}]{
        \var \notin \func{fv}{\lexpr}
    }{%
        \tripleL{\var \dot= \lexpr }{ \pass{\var}{\expr} }{\var \dot= \expr\sub{\var}{\lexpr} }
    }
    \and
    \inferrule[\rl{TRAssume}]{ }{%
        \tripleL{ \expr \dot\neq 0 }{ \passume{\expr} }{ \expr \dot\neq 0 } 
    }
    \and
    \inferrule[\rl{TRChoice}]{%
        \tripleL{ \lpre }{ \trans_{1} }{ \lpost } 
        \\ \tripleL{ \lpre }{ \trans_{2} }{ \lpost } 
    }{%
        \tripleL{ \lpre }{ \trans_{1} \pchoice \trans_{2} }{ \lpost }
    }
    \and
    \inferrule[\rl{TRSeq}]{%
        \tripleL{ \lpre }{ \trans_{1} }{ \lframe }
        \\ \tripleL{ \lframe }{ \trans_{2} }{ \lpost }
    }{%
        \tripleL{ \lpre }{ \trans_{1} \pseq \trans_{2} }{ \lpost }
    }
    \and
    \inferrule[\rl{TRIter}]{%
        \tripleL{ \lpre }{ \trans }{ \lpre } 
    }{%
        \tripleL{ \lpre }{ \trans\prepeat }{ \lpre }
    }
    \and
    \inferrule[\rl{TRFrame}]{%
        \tripleL{ \lpre }{ \trans }{ \lpost } \and \func{fv}{\lframe} \cup \func{modify}{\trans} = \emptyset
    }{% 
        \tripleL{ \lpre \sep \lframe }{ \trans }{ \lpost \sep \lframe }
    }
\end{mathpar}


\hrule
\[
\begin{array}{@{} c @{\qquad} c @{}}
\begin{rclarray}
    \lexpr \fpI \lexpr' & \toFP{\otR(\lexpr,\lexpr')} & \lexpr \fpR \lexpr' \\
    \lexpr \fpR \lexpr' & \toFP{\otR(\lexpr,\lexpr')} & \lexpr \fpR \lexpr' \\
    \lexpr \fpW \lexpr' & \toFP{\otR(\lexpr,\lexpr')} & \lexpr \fpW \lexpr' \\
    \lexpr \fpRW (\lexpr'', \lexpr') & \toFP{\otR(\lexpr,\lexpr')} & \lexpr \fpRW (\lexpr'', \lexpr') \\
\end{rclarray}
&
\begin{rclarray}
    \lexpr \fpI \lexpr' & \toFP{\otW(\lexpr,\lexpr'')} & \lexpr \fpW \lexpr'' \\
    \lexpr \fpR \lexpr' & \toFP{\otW(\lexpr,\lexpr'')} & \lexpr \fpRW (\lexpr',\lexpr'') \\
    \lexpr \fpW \lexpr' & \toFP{\otW(\lexpr,\lexpr'')} & \lexpr \fpW \lexpr'' \\
    \lexpr \fpRW (\lexpr'', \lexpr') & \toFP{\otW(\lexpr,\lexpr''')} & \lexpr \fpRW (\lexpr'', \lexpr''') \\
\end{rclarray}
\end{array}
\]
\hrule
\caption{The rules for transactions}
\label{fig:rule-trans}
 \end{figure}



\subsection{Reasoning programs}

\emph{Capabilities} (\cref{def:capabilities}) are used to specify the allowed operations on concurrent modules.
Each module is associated with \emph{client-specified capabilities} that forms \emph{a partial commutative monoid (PCM)}.
To recall, \emph{a PCM} is a partially ordered set that is closed under a commutative binary operation \( \compose \) and has a set of identify elements \( \unitelem \).
The client-specified capabilities are lifted to \emph{capability composition function} with their associated region identifiers.
For brevity, we often use \emph{capabilities} for \emph{capability composition function}.
The composition function for \emph{capabilities} \( \ca_l \composeC \ca_r \) is defined as point-wise compositing each region and the units \( \unitC \) are functions where regions map to units of client-specified capabilities.


\begin{definition}[Capabilities]
\label{def:capabilities}
Assume a \emph{partial commutative monoid (PCM)} of \emph{client-specified capabilities} \( (\Kaps, \composeK, \unitK) \) with \( \kap \in \Kaps \), the composition \( \composeK \) the units set \( \unitK \).
Then given a set of \emph{region identifiers} \( \rid \in \RegionID \), 
the \emph{capability composition function} or \emph{capabilities} \( \ca \in \Caps \defeq \RegionID \parfun \Kaps \), where the composition \( \composeC \) is defined as the follows:
\[
    \begin{rclarray}
        (\ca_{l} \composeC \ca_{r})(\rid) & \defeq  &
        \begin{cases}
            \ca_{l}(\rid) \composeK \ca_{r}(\rid) & \rid \in \dom(\ca_{l}) \cap \dom(\ca_{l}) \\
            \ca_{l}(\rid)  & \rid \in \dom(\ca_{l}) \setminus \dom(\ca_{l}) \\
            \ca_{r}(\rid) & \rid \in \dom(\ca_{r}) \setminus \dom(\ca_{l}) \\
            \text{undefined} & \text{otherwise} \\
        \end{cases}
    \end{rclarray}
\]
and the units set \( \unitC \defeq \Setcon{\ca}{\fora{\rid} \ca(\rid) \in \unitK } \) .
A capability assertion is in the form of \( \cass{\kap(\vec{\lvar})}{\lrid} \in \CAst \), where \( \kap(\vec{\lvar}) \) is a token parametrised by logical variables and \( \lrid \) is the region identifiers.
The capability assertion is interpreted to a capability in the model by interpreting all the logical expressions,
\[
\begin{rclarray}
    \evalC{\cass{\kap(\vec{\lvar})}{\lrid}} & \defeq & \Set{\lrid \mapsto \kap(\evalLE{\vec{\lvar}})} \\
\end{rclarray}
\]
\end{definition}

The \emph{capability assertions} are in the form of \( \cass{\kap(\vec{\lvar})}{\lrid} \) where \( \kap(\vec{\lvar}) \) is a syntactic capability and \( \lrid \) is a region identifier
They are interpreted to some capabilities by interpreting the syntactic capabilities \( \kap(\vec{\lvar}) \).:
They are resources that grant abilities to access the module, which we will explain later, or act as ghost resources to provide extra information about the module.

Each region is associated with a \emph{interference} to specify how the region can evolve (\cref{def:invariant-region}).
A action in the interference is in the form of \( \exsts{\vec{\lvar}} \perm{\kap} : \bar{\fp} \) and it says if a client holds the capability \( \perm{\kap} \), it is allowed to commit a transaction that has the \emph{fingerprint with capabilities transformation} \( \bar{\fp} \).
The existential is for binding variables between the capability and the fingerprint assertions.
The \emph{fingerprint with capabilities transformation} are fingerprint assertion (\cref{def:fingerprint}) extended with special one for transferring of capabilities, \ie adding to the shared state \( \fpA \cass{\kap}{\lrid} \), deleting from the shared state \( \fpD \cass{\kap}{\lrid} \) and updating the capabilities \( \cass{\kap(\vec{\lvar})}{\lrid} \fpU \cass{\kap(\vec{\lvar})}{\lrid} \). 

\begin{definition}[Interference]
\label{def:intf}
The \emph{fingerprint with capabilities transformation} is defined by the follows:
\[
\begin{rclarray}    
    \bar{\fp}, \bar{\fp}' & ::= & 
    \lexpr \fpI \lexpr 
    \mid \lexpr \fpR \lexpr 
    \mid \lexpr \fpW \lexpr 
    \mid \lexpr \fpRW (\lexpr, \lexpr) \\
    & & \mid \null \fpA \cass{\kap(\vec{\lvar})}{\lrid}  
    \mid \null \fpD \cass{\kap(\vec{\lvar})}{\lrid} 
    \mid \cass{\kap(\vec{\lvar})}{\lrid} \fpU \cass{\kap(\vec{\lvar})}{\lrid} 
    \mid \bar{\fp} \sep \bar{\fp}'
\end{rclarray}
\] 
Given a logical environment $\lenv \in \LEnv$, a stack $\stk \in \Stacks$ and the fingerprint interpretation function (\cref{def:fingerprint}), the \emph{fingerprint with capabilities transformation} is interpreted through function, $\evalF[(., .)]{.}: \FAst \times \LEnv \times \Stacks \parfun \Heaps \times \Opsets \times \Caps \times \Caps$:
\[
\begin{rclarray}
    \evalF{ \bar{\fp} } & \defeq &
        \Setcon{(\h, \opset, \ca, \ca')}{
            (\h,\opset) \in \evalLS{\bar{\fp}} \land \ca, \ca' \in \unitC
        } \quad \text{where} \ \bar{\fp} \in \LAst \\
    \evalF{\null \fpA \cass{\kap(\vec{\lvar})}{\lrid} } & \defeq & 
        \Setcon{(\unitH, \unitO, \ca, \ca')}{
            \ca = \evalC{\cass{\kap(\vec{\lvar})}{\lrid}} \land \ca' \in \unitC
        } \\
    \evalF{\null \fpD \cass{\kap(\vec{\lvar})}{\lrid} } & \defeq &
        \Setcon{(\unitH, \unitO, \ca, \ca')}{
            \ca \in \unitC \land \ca'  = \evalC{\cass{\kap(\vec{\lvar})}{\lrid}} 
        } \\
    \evalF{\cass{\kap(\vec{\lvar})}{\lrid} \fpU \cass{\kap'(\vec{\lvar}')}{\lrid} } & \defeq &
        \Setcon{(\unitH, \unitO, \ca, \ca')}{
            \ca = \evalC{\cass{\kap(\vec{\lvar})}{\lrid}} \land \ca'  = \evalC{\cass{\kap'(\vec{\lvar}')}{\lrid}} 
        } \\
    \evalF{\fp_{1} \sep \fp_{2}} & \defeq & \Setcon{ ( \h_{1} \composeH \h_{2}, \opset_{1} \composeO \opset_{2}, \ca_{1} \composeC \ca_{2}, \ca'_{1} \composeC \ca'_{2} ) }{(\h_{1}, \opset_{1}, \ca_{1}, \ca'_{1}) \in \evalF{\fp_{1}}  \\ {} \land (\h_{2}, \opset_{2}, \ca_{2}, \ca'_{2}) \in \evalF{\fp_{2}}}\\

\end{rclarray}
\]
The grammar of \emph{interference assertions}, \( \intass \in \IAst \), is defined as the follows:
\[
\begin{rclarray}
	\intass & ::=  & \emptyset \mid \Set{ \exsts{\vec{\lvar}} \perm{\kap} : \fp } \cup \intass 
\end{rclarray}
\]
The interference assertions are interpreted to \emph{interference environments} \( \intf \):
\[
\begin{rclarray}
    \inter \in \Interference & \defeq & \Kaps \to ( \HisHeaps \times \Views \times \Caps ) \times  ( \HisHeaps \times \Views \times \Caps )
\end{rclarray}
\]
The \emph{interference interpretation} function, $\evalI[(., .)]{.}: \IAst \times \LEnv \times \Stacks \to \Interference$, is defined as follows:
\sx{
    Notations are confused, there is different between syntactic \( \kap \) which can be parametrised by logical variables, and client-specified capabilities \( \kap' \).
    Fix the typesetting later.
    } 
\[
\begin{array}{@{}l}
	\evalI{\Set{ \exsts{\vec{\lvar}} \perm{\kap} : \fp } \cup \intass }(\kap') \eqdef \\
    	\quad \left\{ 
            \begin{array}{@{}l @{\qquad} l}
            \multicolumn{2}{@{}l@{}}{
                    \Setcon{
                        \begin{B}
                            (\hh, \vi, \ca_r \composeC \ca_f ), \\ 
                            (\hh',\vi', \ca_f \composeC \ca_a)
                        \end{B}
                    }{ 
                        \exsts{\txid, \opset, \cl} \\
                            \quad ( \stub, \opset, \ca_{a}, \ca_{r} ) \in \evalF[\lenv',\stk]{\fp}   \\
                        \quad {} \land \txid \in \func{nextTxid}{\hh, \cl}  \\
                        \quad {} \land \hh' = \updM{\hh, \vi, \txid, \opset}  \\
                        \quad {} \land \pred{readFrom}{\hh, \vi, \opset} 
                        \land \vi' \geq \vi \\
                    } 
                    \cup \evalI{\intass}(\kap')%
            } \\
            & \text{if there exist a logical environment} \ \lenv' \ \text{by replacing} \ \vec{\lvar} \ \text{with some} \ \vec{\val} \ \text{\ie} \\ 
            & \lenv' = \lenv\rmto{\vec{\lvar}}{\vec{\val}}, \ \text{and under the new logical environment} \ \kap' = \evalI[\lenv', \stk]{\kap} \\
            \evalI{\intass}(\kap') 
            & \text{otherwise} \\
    	    \end{array}
        \right.  \\
\end{array}
\]
The \( \predn{readFrom} \) asserts the fingerprint makes sense with respect to the view:
\[
\begin{rclarray}
    \pred{readFrom}{\hh, \vi, \opset} & \defeq & \fora{\ke, \val} (\otR, \ke, \val) \in \opset \implies \valueOf(\hh(\ke,\vi(\ke))) = \val
\end{rclarray}
\]
\end{definition}


%The interference \( \exsts{\vec{\lvar}} \perm{\kap} : \bar{\lpre} \mat \fp \) says if a thread holds the capability \( \perm{\kap}\) and \emph{the current state of database} satisfies the assertions \( \bar{\lpre} \), the thread is allowed to commit a transaction that has the fingerprint \( \fp \).
%The current state of database refers to the state that all the committed transactions are visible.

%\begin{definition}[Interference]
%\label{def:intf}
%Assume standard separation logic assertion \( \bar{\lpre}\) (the local assertion \( \LAst \) without fingerprint).
%Given the fingerprint assertion \( \fp \in \Fingerprint \) (\defref{def:fingerprint}), the grammar of \emph{interference assertions}, \( \intass \in \IAst \), is defined as the follows,
%\[
%\begin{rclarray}
	%\intass & ::=  &
	%\emptyset \mid \Set{ \perm{\kap} :  \exsts{\vec{\lvar}} \bar{\lpre} \mat \fp } \cup \intass 
%\end{rclarray}
%\]
%The interference assertions are interpreted to a set of \emph{interference environments} that is a function from client-specified capabilities to pairs of history heaps and operations,
%\[
%\begin{rclarray}
    %\inter \in \Interference & \defeq & \Kaps \to \powerset{\HisHeaps \times \Opsets}
%\end{rclarray}
%\]
%Given a logical environment $\lenv \in \LEnv$ and a stack $\stk \in \Stacks$, the \emph{interference interpretation} function, $\evalI[(., .)]{.}: \IAst \times \LEnv \times \Stacks \to \Interference$, is defined as follows,
%%
%\[
%\begin{rclarray}
	%\evalI{\emptyset}(\kap) & \eqdef & \emptyset \\
	%\evalI{\Set{ \perm{\kap} : \exsts{\vec{\lvar}} \bar{\lpre} \mat \fp } \cup \intass }(\kap') & \eqdef &
    %\begin{cases}
    %\Setcon{(\hh, \evalF[\lenv',\stk]{\fp})}{\exsts{\h} \h \in \evalLS[\lenv',\stk]{\bar{\lpre}} \land {} \\ \h = \clpsHH{\hh} } \cup \evalI{\intass}(\kap')  & \kap = \kap' \\
    %\evalI{\intass}(\kap') & \text{ otherwise} \\
    %\end{cases} \\
    %& & \text{where there exists a vector of values \( \vec{\val}\) such that } \lenv' = \lenv\rmto{\vec{\lvar}}{\vec{\val}} \\
%\end{rclarray}
%\] 
%\end{definition}

%We will write \( \intfH(\kap) \)  and \( \intfO(\kap) \) for the first and second projections of all the elements.


%\begin{definition}[Labelled transition system]
%\label{def:labelled-transition-system}
%The labelled transition system is a tuple \( ( \hhset \times \cuset, \opsetset,\toLTS{}, \hhset_{0} \times \cuset_{0}, \como) \) consisting of pairs of history heaps and cuts \( \hhset \times \cuset \), a set of sets of operations \( \opsetset \subseteq \Opsets \), a relation \( \toLTS{} : \HisHeaps \times \Opsets \times \HisHeaps \), a set of initial history heaps and cuts \( \hhset_{0} \times \cuset_{0} \) and the consistency model associated with the transition system \( \como \).
%Assume all the initial abstract executions satisfies the consistency model.
%The relation \( \toLTS{}\) is defined as the follows,
%\[
%\begin{rclarray}
    %(\hh, \cu) \toLTS{\opset} (\hh',\cu') & \defeq &
    %\begin{array}[t]{@{}l}
        %\exsts{\thcu, \thcu', \txid, \thid}
        %\txid \in \fresh{\hh} 
        %\land \hh' = \updM{\hh, \cu, \txid, \opset} 
        %\land \cu' = \updV{\hh', \cu, \opset} \\
        %\quad {} \land ((\hh,\thcu),(\hh',\thcu')) \in \como
        %\land \h = \clpsHH{\hh,\cu} 
        %\land \thcu(\thid) = \cu 
        %\land \thcu'(\thid) = \cu' \\
        %\quad {} \land \fora{\addr,\val} (\otR, \addr, \val)  \in \opset \implies \h(\addr) = \val
    %\end{array}
%\end{rclarray}
%\]
%\end{definition}

%We lift the interference to a invariant.
%The invariant is a labelled transition system that describes how a region evolves providing all the allowed operations.
%Note that the labels are capabilities (with region identifiers) instead of client-specified capabilities, which is only for technical reason.

\begin{definition}[Invariant of a region]
\label{def:invariant-region}
Assume a global function, \( \funcn{init} : \RegionID \to \powerset{ \HisHeaps } \) returning initial key-value stores for regions.
Given the initial states for a region \( \func{init}{\rid}\), the invariant of a region, written \( \func{inv}{\rid, \intf} \), is a set of key-value stores that is closed under the interference \( \intf \):
\[
\begin{array}[t]{@{}l}
    \func{init}{\rid} \in \func{inv}{\rid, \intf} \land 
    \fora{\mkvs, \mkvs'} 
    \exsts{\w, \w', \kap} \\
    \quad \mkvs \in \func{inv}{\rid, \intf} \land  ( \mkvs, \stub ) = \eraseW{\w} \land ( \mkvs', \stub ) = \eraseW{\w'} \land (\w, \w') \in \intf(\kap) \implies \mkvs' \in \func{inv}{\rid, \intf} 
\end{array}
\]
\end{definition}

%For brevity, \( (\hh,\cu) \in \func{inv}{\rid, \intf} \) denotes \( (\hh,\cu) \in \func{inv}{\rid,\intf}\projection{1} \), and similarly \( (\hh,\cu) \toLTS{\opset} (\hh',\cu') \in \func{inv}{\rid, \intf} \).

%\sx{This well form condition allows one to write weaker interference, \eg interference satisfies both SI and SER but to prove the correctness of SER. It is fine since logic only need to be sound?}
%\begin{definition}[Well-form of a region]
%\label{def:well-form-region}
%The well-form condition of the interference, namely \( \pred{wfintf}{\rid, \intf} \) predicate, assertions for any concrete events \( \opset \), the state before the events must be included in the interference.
%\[
%\begin{rclarray}
    %\pred{wfintf}{\rid, \intf} & \defeq & 
    %\begin{array}[t]{@{}l}
        %\fora{\hh, \hh', \opset} 
        %(\hh, \stub) \toLTS{\opset} (\hh',\stub) \in \func{inv}{\rid, \intf} 
        %\implies \exsts{ \kap}
        %(\hh, \opset ) \in \intf( \kap )
    %\end{array} \\
%\end{rclarray}
%\]
%\end{definition}

\begin{definition}[Worlds]
\label{def:world}
Given the set of history heaps $\HisHeaps$ (\cref{def:his_heap}), views \( \Views \) (\cref{def:views}), capabilities \( \Caps\) (\cref{def:capabilities}) and region identifiers \( \RegionID \), the set of \emph{shared states} is \( \SStates \eqdef \RegionID \parfun \HisHeaps \times \Views \times \Caps \times \Interference \).
Each region has its current state and the interference.
The \emph{shared state composition function}, $\composeS: \SStates \times \SStates \parfun \SStates$, is defined as $\composeS \eqdef \composeEq$, where for all domains $\sort M$ and all $m, m' \in \sort M$,
%
\[
\begin{rclarray}
	m \composeEq m' &  \eqdef  &
	\begin{cases}
		m & \text{if } m = m'\\
		\text{undefined} & \text{otherwise}
	\end{cases}
\end{rclarray}
\]
A \emph{world} \( \w \in \World \) is a pair of capabilities \( \ca \) (\cref{def:capabilities}) and a shared state \( \gs \) in which regions are well-formed, \ie (a) they are associated with disjointed part of history heaps; (b) the domain of the view in a region is the same as the domain of the history heap; and (c) the views should not be out of the range of history heaps.
Separately, capabilities from regions and local capabilities are compatible.
These constraints are derived by the clap \(\eraseS{(\ca, \gs)} \neq \emptyset \).
Finally, there is no garbage capability, a capability where the associated region identifier never appear in the shared state.
\[
\begin{rclarray}
	\world \in \World  & \eqdef & 
    \Setcon{
        (\ca, \gs) 
    }{ 
        \ca \in \Caps \land \gs \in \SStates
        \land \exsts{ \ca' } 
        (\stub, \stub, \ca') \in \func{collapse}{\gs}
        \land \dom(\ca \composeC \ca') \subseteq \dom(\gs)  \\
        \quad {} \land \fora{\rid}
        \exsts{\hh, \vi, \intf} 
        \gs(\rid) = (\hh, \vi, \stub, \intf) 
        \land \dom(\hh) = \dom(\vi) 
        \land \mkvs \in \func{inv}{\rid, \intf} \\
        \quad {} \land \fora{ \addr \in \dom(\vi) }
        0 \leq \cu( \addr ) \le \left| \hh(\addr) \right|
    }
\end{rclarray}
\]
where the \(\funcn{collapse} \) function collapses a shared state by erasing the region identifiers:
\[
\begin{rclarray}
    \func{collapse}{\emptyset} & \defeq & \Set{(\unitHH, \unitVI, \unitC )} \\
    \func{collapse}{\Set{\rid \mapsto (\hh, \vi, \ca, \intf)} \uplus \gs } & \defeq & 
        \Setcon{ 
            (\hh \composeHH \hh', \vi \composeCU \vi', \ca \composeC \ca') 
        }{ 
            \land (\hh', \vi', \ca') \in \func{collapse}{\gs} }\\
\end{rclarray}
\] 
% 
The \emph{world composition function}, $\composeW: \World \times \World \parfun \World$, is defined component-wise as: $\composeW \eqdef (\composeC, \composeS)$.
The \emph{world unit set} is $\unitW \eqdef \Setcon{(\ca, \gs)}{(\ca, \gs) \in \World \land \ca \in \unitC}$.
The \emph{partial commutative monoid of worlds} is $(\World, \composeW, \unitW)$.
Because of the well-formedness condition, the function \( \eraseW{.} : \World \to \HisHeaps \times \Views \) collapses a world to \emph{a unique pair of a key-value store and a view}:
\[
\begin{rclarray}
    \eraseW{\w} & \defeq & (\hh, \vi) \text{  where } (\hh, \vi, \stub) \in \func{collapse}{\w\projection{2}}\\
\end{rclarray}
\] 
\end{definition}

\begin{definition}[Assertions]
\label{def:assertion}
Given the set of logical expression \( \lexpr \subseteq \LExpr\), the set of \emph{assertions}, $\gpre, \gpost \in \Ast$, are defined by the following inductive grammar:
\[
\begin{rclarray}
    \bar{\lpre}, \bar{\lpost} & ::= & \False \mid \True \mid \bar{\lpre} \land \bar{\lpost} \mid \bar{\lpre} \lor \bar{\lpost} \mid \exsts{\lvar} \bar{\lpre} \mid \bar{\lpre} \implies \bar{\lpost} \mid \assemp \mid \cass{\kap}{\lrid} \mid \lexpr \pt \lexpr \mid \bar{\lpre} \sep \bar{\lpost} \\
	\gpre , \gpost & ::= & \False \mid \True \mid \gpre \land \gpost \mid \gpre \lor \gpost \mid \exsts{\lvar}\gpre \mid \gpre \implies \gpost \mid \assemp \mid \cass{\kap}{\lrid} \mid \gpre \sep \gpost \mid \boxass{\bar{\lpre}}{\lrid}{\intass}\\
\end{rclarray}
\]
%
where $\lvar, \lrid \in \LVar$, $\lexpr_1, \lexpr_2 \in \LExpr$ (\cref{def:local_assertions}), $\kap \in \Kaps$ (\cref{def:capabilities}) and $\intass \in \IAst$ (\cref{def:intf}).
Given a logical environment $\lenv \in \LEnv$ and a stack $\stk \in \Stacks$, the \emph{assertion interpretation} function, $\evalW[(., .)]{.}: \Ast \times \LEnv \times \Stacks \to \powerset{\World}$, is defined as follows,
%
\[
\begin{rclarray}
	\evalW{\False} & \defeq & \emptyset \\
	\evalW{\True} & \defeq & \World \\
	\evalW{\emp} & \defeq & \unitW \\
	\evalW{\gpre \land \gpost} & \defeq & \evalW{\gpre} \cap \evalW{\gpost} \\
	\evalW{\gpre \lor \gpost} & \defeq & \evalW{\gpre} \cup \evalW{\gpost} \\ 
	\evalW{\exsts{\lvar}  \gpre} & \defeq & \bigcup\limits_{\val \in \textnormal{\Val}} \evalW[\lenv\remapsto{\lvar}{\val}, \stk]{\gpre} \\
	\evalW{\gpre \implies \gpost} & \defeq & \Setcon{\w}{\w \in \evalW{\gpre} \implies \w \in \evalW{\gpost}} \\
	\evalW{\cass{\kap}{\lrid}} & \defeq & \Setcon{ (\Set{\lrid \mapsto \evalI{\kap}}, \gs) }{\gs \in \SStates} \\
	\evalW{ \gpre \sep \gpost } & \defeq & 
	\Setcon{
	   (\world_1 \composeW \world_2) 
    }{
       \world_1 \in \evalW{\gpre} \land \world_2 \in \evalW{\gpost}
	} \\
	\evalW{ \boxass{\bar{\lpre}}{\lrid}{\intass} } & \defeq & 
    \Setcon{
        (\ca, \gs)
    }{         
        \exsts{\hh, \vi, \ca', \intf}
        \ca \in \unitC  \\
        \quad {} \land \intf = \evalI{\intass} 
        \land \gs(\lrid) = (\hh, \vi, \ca', \intf) 
        \land (\hh, \vi, \ca') \in \func{intp}{\bar{\lpre}, \lenv, \stk} 
    } \\
    \\
    \func{intp}{\assfalse,\lenv,\stk} & \defeq & \emptyset \\
    \func{intp}{\asstrue,\lenv,\stk} & \defeq & \HisHeaps \times \Views \times \Caps \\
    \func{intp}{\assemp,\lenv,\stk} & \defeq & \Setcon{ (\unitHH, \unitVI, \ca) }{\ca \in \unitC } \\
    \func{intp}{\bar{\lpre} \land \bar{\lpost},\lenv,\stk} & \defeq & \func{intp}{\bar{\lpre},\lenv,\stk} \cap \func{intp}{\bar{\lpost},\lenv,\stk} \\ 
    \func{intp}{\bar{\lpre} \lor \bar{\lpost},\lenv,\stk} & \defeq & \func{intp}{\bar{\lpre},\lenv,\stk} \cup \func{intp}{\bar{\lpost},\lenv,\stk} \\ 
    \func{intp}{\exsts{\lvar} \bar{\lpre},\lenv,\stk} & \defeq & \bigcup\limits_{\val \in \Val} \func{intp}{\bar{\lpre}, \lenv\rmto{\lvar}{\val}, \stk} \\
    \func{intp}{\bar{\lpre} \implies \bar{\lpost},\lenv,\stk} & \defeq & \Setcon{ (\hh, \vi, \ca) }{ (\hh, \vi, \ca) \in \func{intp}{\bar{\lpre},\lenv,\stk} \implies (\hh, \vi, \ca) \in \func{intp}{\bar{\lpost},\lenv,\stk} }\\
    \func{intp}{\cass{\kap}{\lrid},\lenv,\stk} & \defeq & \Set{ (\unitHH, \unitVI, \Set{\lrid \mapsto \evalC{\kap}}) }\\
    \func{intp}{\lexpr_{1} \pt \luexpr_2,\lenv,\stk} & \defeq & \Setcon{ (\hh, \vi, \ca) }{ \Set{ \evalLE{\lexpr_{1}} \mapsto \evalLE{\luexpr_{1}} } = \clpsHH{\hh, \vi} \land \ca \in \unitC } \\
    \func{intp}{\lexpr_{1} \pt \lexpr_2,\lenv,\stk} & \defeq & 
    \Setcon{ (\hh, \vi, \ca) }{%
        \exsts{\ke = \evalLE{\lexpr_{1}}} \Set{ \ke \mapsto \evalLE{\lexpr_{2}} } = \clpsHH{\hh, \vi} \\ 
        \quad {} \land \vi(\ke) = \lvert \mkvs(\ke) \rvert - 1 \land \ca \in \unitC
    } \\
    \func{intp}{\bar{\lpre} \sep \bar{\lpost},\lenv,\stk} & \defeq & 
    \Setcon{ (\hh \composeHH \hh', \vi \composeVI \vi', \ca \composeC \ca') }{ (\hh, \vi, \ca) \in \func{intp}{\bar{\lpre},\lenv,\stk} \\ {} \land (\hh', \vi', \ca') \in \func{intp}{\bar{\lpost},\lenv,\stk} } \\
\end{rclarray}
\]
\end{definition}



\subsubsection{Rely and Guarantee}

The \emph{rely} and \emph{guarantee} describes the world transformation for the environment and for the current client (\cref{def:rely-guarantee}).
To recall, a world includes local capabilities and a shared state, and a shared state is a client's view for the key-value store.
The \emph{rely} \( \Rely \) is a set of pairs on worlds, describing how the environment can change the state of the key-value store.
Given the local capabilities \( \ca_l \), the environment might own any capabilities \( \ca_e\) that is compatible, \ie \( (\ca_l \composeC \ca_e)\isdef \).
Therefore, the environment can perform actions associated with the their capabilities \( \ca_e \) with their own view \( \vi_e \) to update the key-value store and shared capabilities.
For technical reasons, even though the environment cannot change the view of the current client \( \vi_r\), but it is allowed to arbitrarily shift to the later versions due to the fact that for certain execution tests, the old view might be valid under the new key-value store.

The \emph{guarantee} \( \Guar \) describes the allowed actions for the current client.
The current client can perform actions associated with the local capabilities \( \ca_l \) to update the shared state and the local capabilities.
Yet it should ensure no resource created or deleted by requiring the \emph{orthogonal} of local capabilities and shared capabilities together remains unchanged.
The orthogonal of local capabilities \( \ca \) is a set of capabilities that are compatible with the local capabilities \( \ca \).
This constraint disallow any creation and deletion for capabilities, but it allows to update capabilities.
Note that in the rely and guarantee, it is allowed to update several regions, but each region can be updated at most once.
\begin{definition}[Rely and guarantee]
\label{def:rely-guarantee}
Given the set of worlds $\World$ (\cref{def:world}), the \emph{update rely} relation, $\relyU \subseteq \World \times \World$, is defined as follows:
\sx{In case I get confused again, it is world to world so the shared and local capabilities should always make sense.}
\[	
    \begin{rclarray}
	\relyU & \eqdef &
	\Setcon{
		((\ca_l,\gs), (\ca_l, \gs'))	
	}{
        \exsts{\ca_e}
        (\ca_e \composeC \ca_l) \isdef
        \land \fora{\rid}
        \gs(\rid) = \gs'(\rid) \lor 
        \exsts{\kap, \hh, \hh', \vi_\rid, \vi_{\rid}', \vi_{e}, \vi_{e}', \ca_\rid, \ca_{\rid}', \intf}   \\
        \quad \gs(\rid) = (\hh, \vi_\rid, \ca_\rid, \intf)
        \land \gs'(\rid) = (\hh', \vi_{\rid}', \ca_{\rid}',\intf) \\
        \quad {} \land \kap \sqsubseteq \ca_{e}(\rid) 
        \land ( (\hh, \vi_e, \ca_e), (\hh', \vi_{e}', \ca_{e}') )  \in \intf(\kap)
        \land \vi_{\rid}' \geq \vi_\rid
	} \\
    \end{rclarray}
\]
The \emph{view shift rely} relation $\relyV \subseteq \World \times \World$, is defined as follows:
\[
    \begin{rclarray}
	\relyV & \eqdef &
	\Setcon{
		((\ca_l,\gs), (\ca_l, \gs'))	
	}{
        \exsts{\ca_e}
        (\ca_e \composeC \ca_l) \isdef
        \land \fora{\rid}
        \gs(\rid) = \gs'(\rid) \lor 
        \exsts{\hh, \vi, \vi'\ca, \ca, \intf}   \\
        \quad \gs(\rid) = (\hh, \vi, \ca, \intf)
        \land \gs'(\rid) = (\hh, \vi', \ca, \intf) 
	} \\
    \end{rclarray}
\]
The \emph{rely} relation is transitive closure of updates and view shift: \( \Rely = (\relyU \cup \relyV)^{*} \).
The \emph{guarantee} relation, $\Guar: \World \times \World$, is defined as follows:
\[	
    \begin{rclarray}
	\Guar & \eqdef &
	\Setcon{
		((\ca_l,\gs), (\ca_{l}', \gs'))	
	}{
        \fora{\rid}
        \gs(\rid) = \gs'(\rid) \lor {}
        \exsts{\kap, \hh, \hh', \vi_\rid, \vi_{\rid}', \ca_\rid, \ca_{\rid}', \intf}   \\
        \quad \gs(\rid) = (\hh, \vi_\rid, \ca_\rid,\intf)
        \land \gs'(\rid) = (\hh', \vi_{\rid}', \ca_{\rid}',\intf) 
        \land \kap \sqsubseteq \ca_{l}(\rid)  \\
        \quad {} \land ( (\hh, \vi_\rid, \ca_\rid), (\hh', \vi_{\rid}', \ca_{\rid}') )  \in \intf(\kap)
        \land (\ca_{l} \composeC \ca_\rid)^{\perp} = (\ca_{l}' \composeC \ca_{\rid}')^{\perp}
	} \\
    \end{rclarray}
\]
where for any element \( m \) from its domain \( \sort{M} \), the  \emph{orthogonal} is defined as:
\[
\begin{rclarray}
m^{\perp} & \defeq & \Setcon{m'}{(m \compose{} m')\isdef \land m' \in \sort{M}} 
\end{rclarray}
\]
\end{definition}

The stabilisation says assertions remain true against the environment.
Formally a set of worlds \( \setworld \) is stable under certain execution tests \( \et \) if the set is closed under rely relation under the side conditions: (a) the key-value store transfer is allowed by the execution tests \( (\mkvs, \mkvs') \in \et \);
and (b) the new view under the new key-value store is able to progress.
The first condition says there is at least one view from the environment that can trigger the transformation on the key-value store, and it is allowed by the execution tests.
The second condition is more subtle, as it allows to also update the view to a new view \( \vi' \) so that for all the possible transactions with the fingerprints \( \opset \), if they can execute under the old view, it should be able to execute under the new view.

\begin{definition}[Stable]
\label{def:stable}
A set of worlds $\setworld \subseteq \World$ is \emph{stable}, written $\stable{\setworld, \como}$, if and only if it is closed under the rely relation: 
\[
    \begin{rclarray}
        \stable{\setworld, \como} & \eqdef & 
        \begin{array}[t]{@{}l}
            \fora{\w, \w'} 
            \w \in \setworld 
            \land (\w, \w') \in \Rely  
            \land \exsts{\hh, \hh', \vi, \opset} \\
            \quad (\hh, \vi) = \eraseW{\w}
            \land (\hh', \vi') = \eraseW{\w'} 
            \land (\hh, \hh') \in \como 
            \land \pred{progress}{\w, \et} \\
            \qqquad \implies \w' \in \setworld
        \end{array} \\
    \end{rclarray}
\]
A update history heap update is allowed by consistency model, \ie \( (\hh, \hh') \in \como \), iff there exist some view \( \vi \) and operation set \( \opset \) allowed by the consistency model and the history is updated to \( \hh' \) via them:
\[
    \begin{rclarray}
        (\hh, \hh') \in \como & \eqdef & 
        \begin{array}[t]{@{}l}
            \hh = \hh' \lor 
            \exsts{ \vi, \vi', \opset, \txid, \cl}  \\
            \quad (\hh, \vi) \csat \opset : \vi' 
            \land \txid \in \fresh{\hh, \cl} 
            \land \hh'  = \updM{\hh,\vi, \txid, \opset}
        \end{array}
    \end{rclarray}
\]
A world is able to progress under an execution test \( \et \), iff it is able to execute the empty fingerprint:
\[
    \begin{rclarray}
        \pred{progress}{\w, \et} & \defeq & \exsts{\vi} \eraseW{\w} \csat \unitO :  \vi
    \end{rclarray}
\]
\end{definition}


\subsection{Rules for Global}

The \rl{PRCommit} rule lifts the local effect of transaction \( \trans \) to global level by first converting global state to (local) observable state and then propagating the local fingerprint to the global state.
%The \( \predn{down} \) predicate asserts that the local predicate \( \lpre \) is a over-approximation of the valid observation that is given by the interference.
%The \( \predn{up} \) predicate says the post-condition \( \gpost \) is the result by lifting the local fingerprints \( \fp \) to pre-condition \( \gpre \).


\begin{figure}[t!]
\hrule\vspace{5pt}

\begin{mathpar}
    \inferrule[\rl{PRCommit}]{%
        \tripleL{\lpre}{\trans}{\lpost} 
        \\ \repartition{\gpre}{\gpost}{\lpre}{\lpost}
        \\\\ \stable{\gpre, \como} 
        \\ \stable{\gpost, \como} 
    }{%
        \tripleG{\gpre}{ \ptrans{\trans} }{\gpost}
    }
    \and
    \inferrule[\rl{PRPar}]{%
        \tripleG{ \gpre_{1} }{ \cmd_{1} }{ \gpost_{1} }
        \\ \tripleG{ \gpre_{2} }{ \cmd_{2} }{ \gpost_{2} } 
        \\\\ \stable{\gpre_{1}, \como} 
        \\ \stable{\gpre_{2}, \como} 
    }{%
        \tripleG{ \gpre_{1} \sep \gpre_{2} }{ \cmd_{1} \ppar \cmd_{2} }{ \gpost_{1} \sep \gpost_{2} }
    }
    \and
    \inferrule[\rl{PRAss}]{%
        \thvar \notin \func{fv}{\lexpr} 
    }{%
        \tripleG{\thvar \dot= \lexpr }{ \pass{\thvar}{\expr} }{\thvar \dot= \expr\sub{\thvar}{\lexpr} }
    }
    \and
    \inferrule[\rl{PRAssume}]{ }{%
        \tripleG{ \expr \dot\neq 0 }{ \passume{\expr} }{ \expr \dot\neq 0 } 
    }
    \and
    \inferrule[\rl{PRChoice}]{%
        \tripleG{ \gpre }{ \cmd_{1} }{ \gpost } 
        \\ \tripleG{ \gpre }{ \cmd_{2} }{ \gpost } 
    }{%
        \tripleG{ \gpre }{ \cmd_{1} \pchoice \cmd_{2} }{ \gpost }
    }
    \and
    \inferrule[\rl{PRSeq}]{%
        \tripleG{ \gpre }{ \cmd_{1} }{ \gframe }
        \\ \tripleG{ \gframe }{ \cmd_{2} }{ \gpost }
    }{%
        \tripleG{ \gpre }{ \cmd_{1} \pseq \cmd_{2} }{ \gpost }
    }
    \and
    \inferrule[\rl{PRIter}]{%
        \tripleG{ \gpre }{ \cmd }{ \gpre } 
    }{%
        \tripleG{ \gpre }{ \cmd\prepeat }{ \gpre }
    }
    \and
    \inferrule[\rl{PRFrame}]{%
        \tripleG{ \gpre }{ \cmd }{ \gpost } 
        \\ \stable{\gframe, \como}
        \\ \func{fv}{\gframe} \cap \func{modify}{\cmd} = \emptyset 
    }{%
        \tripleG{ \gpre \sep \gframe }{ \cmd }{ \gpost \sep \gframe }
    }
\end{mathpar}


\hrule\vspace{5pt}
\[
\begin{rclarray}
    \repartition{\gpre}{\gpost}{\lpre}{\lpost} & \defeq & 
    \begin{array}[t]{@{}l@{}}
        \fora{ \w, \lenv, \stk } 
        \w \in \evalW{\gpre} 
        \implies 
        (\getSN{\eraseW{\w}}, \unitO) \in \evalLS{\lpre}  \\
        \quad {} \land \fora{\stk', \txid, \opset, \w', \cl} 
        \txid \in \fresh{\eraseW{\w}\projection{1}, \cl} 
        \land (\stub, \opset) \in \evalF[\lenv, \stk']{\lpost} \\
        \qquad {} \land \eraseW{\w'}\projection{1} = \updM{\eraseW{\w}, \txid, \opset}  \\
        %\land \eraseW{\w}\projection{2} \leq \eraseW{\w'}\projection{2} \\
        \qquad {} \land (\w, \w') \in \Guar  
        \land \eraseW{\w} \csat \opset : \eraseW{\w'}\projection{2}
        \implies \w' \in \evalW[\lenv, \stk']{\gpost}
    \end{array} 
\end{rclarray}                          
\]

\hrule\vspace{5pt}
\caption{The rules for programs}
\label{fig:rule-prog}
\end{figure}

%\azalea{
    %\sx{How to deal with the stack here? As the stack for P and Q are different, just for all quantify two stacks??}
%The quantification seems wrong. Especially, the $\extopset$ needs to be for all quantified, $\h$ needs to be there exist quantified.
%\[
     %\repartition{\gpre}{\gpost}{\lpre}{\lpost} \defeq 
     %\begin{array}[t]{@{}l@{}}
		 %\fora{\w, \hh, \vi, \lenv, \stk, \txid } 
            %\w \in \evalW{\gpre} 
            %\land (\hh, \cu) \in \eraseW{\w}
            %\Rightarrow\\
            %\quad \exsts{\h}
            %\begin{array}[t]{@{} l @{}}
			%\h = \getSN{\hh, \cu}  
                %\land (\h, \unitO) \in \evalLS{\lpre} \\
                %\land\, \fora{\extopset \in \evalF{\lpost}} 
                    %\exsts{\w', \hh', \vi'} \\
                        %\quad \hh' = \updM{\hh, \vi, \txid, \extopset} 
                       %\land \cu' = \updV{\hh, \vi, \extopset} 
                       %\land (\hh',\vi') \in \eraseW{\w'} \\
                        %\quad \land (\w, \w') \in \Guar  
                        %\land (\w, \w') \in \como
                        %\land \w' \in \evalW{\gpost}			
		%\end{array}	
%%		\right)			        	
    %\end{array} 
%\]
%}

%The \( \HHupdate \) and \( \Vupdate \) in the repartition can be replaced by syntactic rules.

%\begin{figure}
%\hrule\vspace{5pt}

%\[
   %\infer[\rl{FInit}]{%
       %\tripleF{ \lexpr \pt \lexpr }{ \lexpr \fpI \lexpr }{ \lexpr \pt \lexpr }
   %}{}
%\]

%\[
   %\infer[\rl{FRead}]{%
       %\tripleF{ \lexpr \pt \lexpr }{ \lexpr \fpR \lexpr }{ \lexpr \pt \lexpr }
   %}{}
%\]

%\[
   %\infer[\rl{FWrite}]{%
       %\tripleF{ \lexpr \pt \lexpr  }{  \lexpr \fpW \lexpr' }{ \lexpr \pt \lexpr' }
   %}{}
%\]

%\[
   %\infer[\rl{FReWrt}]{%
       %\tripleF{ \lexpr \pt \lexpr  }{  \lexpr \fpRW (\lexpr,\lexpr') }{ \lexpr \pt \lexpr' }
   %}{}
%\]

%\[
   %\infer[\rl{FFrame}]{%
       %\tripleF{ \bar{\lpre}_{1} \sep \bar{\lpre}_{2}  }{  \bar{\fp}_{1} \sep \bar{\fp}_{2} }{ \bar{\lpost}_{1} \sep \bar{\lpost}_{2} }
   %}{
       %\tripleF{ \bar{\lpre}_{1} }{ \bar{\fp}_{1} }{ \bar{\lpost}_{1} }
       %&& \tripleF{ \bar{\lpre}_{2}  }{ \bar{\fp}_{2} }{ \bar{\lpost}_{2} }
    %}
%\]


%\hrule\vspace{5pt}
%\caption{Syntactic rule for \( \HHupdate \) and \( \Vupdate \) functions}
%\label{fig:rule-prog}
%\end{figure}


\section{Soundness}


\begin{thm}[Transaction soundness]
\label{thm:transaction-soundness}
Assume the standard lift for transaction interpretation and for fingerprint heaps composite \( \composeFP \), the transaction soundness is as follows:
\[
    \begin{array}{@{}l@{}}
        \for{ \lpre, \trans, \lpost } \tripleL{\lpre}{\trans}{\lpost} \implies \ \tripleSemL{\lpre}{\trans}{\lpost} \\
    \end{array}
\]
where,
\[
    \begin{rclarray}
    \tripleSemL{\lpre}{\trans}{\lpost} & \eqdef &
    \begin{array}[t]{@{}l@{}}
        \for{\lenv, \stk_{p}, \stk_{q}, \fph_{p}, \fph_{q} } 
        \fph_{p} \in \evalLS[\lenv, \stk_{p}]{\lpre}
        \land (\stk_{p}, \fph_{p} ), \trans \toL^{*}  (\stk_{q}, \fph_{q} ), \pskip 
        \implies \fph_{q} \in \evalLS[\lenv, \stk_{q}]{\lpost}
    \end{array}
    \end{rclarray}
\]
\end{thm}
\begin{proof}
Induction on the rules for transactions.

\caseB{\rl{TRSkip}}

We have  \(\trans \equiv \pskip\), \( \lpre \equiv \lpost \equiv \assemp \).
Given the semantics in \fig \ref{fig:thread_semantics}, we have \( \fph_{p} = \fph_{q} = \unitFP \) and \( \stk_{p} = \stk_{q} \), so \( \fph_{q} \in \evalLS[\lenv, \stk_{q}]{\lpost} \).

\caseB{\rl{TRAss}}

We have \(\trans \equiv ( \pass{\var}{\expr} ) \), \( \lpre \equiv ( \var \doteq \lexpr ) \) and \( \lpost \equiv ( \var \doteq \expr\sub{\var}{\lexpr} ) \) for some \( \var, \lexpr, \expr \).
Given the semantics in \fig \ref{fig:thread_semantics}, there exists \( \stk \) such that \( \stk = \stk_{p} \setminus \Set{\var \mapsto \stub} = \stk_{q} \setminus \Set{\var \mapsto \stub} \).
Given the premiss of \rl{TRAss}  in \fig \ref{fig:rule-trans} that \( \var \notin \func{fv}{\lexpr} \), it has \( \evalLE{\lexpr} = \evalLE[\lenv, \stk_{p}]{\lexpr} = \evalLE[\lenv, \stk_{q}]{\lexpr} \), and then must exist \( \val \) so that \( \val = \evalLE{\expr\sub{\var}{\lexpr}} = \evalLE[\lenv, \stk_{q}]{\expr\sub{\var}{\lexpr}} \) and \( \stk_{q} = \stk_{p}\remapsto{\var}{\val} \).
Also because \( \fph_{p} = \fph_{q} = \unitFP \), we have \( \fph_{q} \in \evalLS[\lenv, \stk_{q}]{\lpost} \).

\caseB{\rl{TRDeref}}

We have  \(\trans \equiv ( \pderef{\var}{\expr} ) \), \( \lpre \equiv ( \expr \fpt{\fp} \lexpr ) \) and \( \lpost \equiv ( \var \doteq \lexpr \sep \expr \fpt{\addFPR{\fp}} \lexpr ) \) for some \( \var, \fp, \lexpr, \expr \).
Given the semantics in \fig \ref{fig:thread_semantics}, there exists \( \stk \) such that \( \stk = \stk_{p} \setminus \Set{\var \mapsto \stub} = \stk_{q} \setminus \Set{\var \mapsto \stub} \).
Given the premiss of \rl{TRDeref} in \fig \ref{fig:rule-trans} that \( \var \notin \func{fv}{\lexpr} \), it must exist \( \val \) and \( \addr \) such that \( \val = \evalLE{\lexpr} = \evalLE[\lenv, \stk_{p}]{\lexpr} = \evalLE[\lenv, \stk_{q}]{\lexpr} \), \( \addr = \evalLE{\expr} = \evalLE[\lenv, \stk_{p}]{\expr} = \evalLE[\lenv, \stk_{q}]{\expr} \) and \(  \stk_{q} = \stk_{p}\remapsto{\var}{\val} \).
Also, because \( \lpre \equiv ( \expr \fpt{\fp} \lexpr ) \), we have \( \fph_{p} = \Set{\addr \mapsto (\val, \fp) }\).
Given above and \( \fph_{q} = \fph_{p}\remapsto{\addr}{ ( \val, \addFPR{\fp} ) } \), we have \( \fph_{q} \in \evalLS[\lenv, \stk_{q}]{\lpost} \).

\caseB{ \rl{TRMutate} }

We have \( \trans \equiv (\pmutate{\expr_{1}}{\expr_{2}}) \), \( \lpre \equiv ( \expr_{1} \fpt{\fp} \stub ) \) and \( \lpost \equiv ( \expr_{1} \fpt{\addFPW{\fp}} \expr_{2} ) \), for some \( \expr_{1}, \expr_{2}, \fp \).
Therefore, \( \fph_{p} \in \Setcon{ \addr \mapsto (\val_{p} , \fp) }{ \val_{p} \in \Val } \), where \( \addr = \evalLE[\lenv, \stk_{p}]{\expr_{1}} \).
Given the semantics in \fig \ref{fig:thread_semantics}, we have \( \stk_{p} = \stk_{q} \) and \( \fph_{q} \in \Set{\addr \mapsto ( \evalLE[\lenv, \stk_{q}]{\expr_{2}},  \addFPW{\fp} ) } \), so \( \fph_{q} \in \evalLS[\lenv, \stk_{q}]{\lpost} \).

\caseI{\rl{TRChoice}}

We have  \(\trans \equiv \trans_{1} + \trans_{2} \), where \( \tripleL{\lpre}{\trans_{1}}{\lpost} \) and \( \tripleL{\lpre}{\trans_{2}}{\lpost} \) hold, for some \( \lpre, \lpost, \trans_{1}, \trans_{2} \).
Given the semantics in \fig \ref{fig:thread_semantics}, for any \( \lenv, \stk_{p}, \fph_{p} \) that \( \fph_{p} \in \evalLS[\lenv, \stk_{p}]{\lpre} \), it has either \( ( \stk_{p}, \fph_{p} ), \trans_{1} \pchoice \trans_{2} \toL ( \stk_{p}, \fph_{p} ), \trans_{1} \) or  \( ( \stk_{p}, \fph_{p} ), \trans_{1} \pchoice \trans_{2} \toL ( \stk_{p}, \fph_{p} ), \trans_{2} \).
Since it is symmetric, assume picking \( \trans_{1} \).
Therefore we have \( ( \stk_{p}, \fph_{p} ), \trans_{1} \pchoice \trans_{2} \toL ( \stk_{p}, \fph_{p} ), \trans_{1} \toL^{*} ( \stk_{q}, \fph_{q} ), \pskip \) for some \( \stk_{q} \) and \( \fph_{q} \).
By the \ih and the premiss of \rl{TRChoice} we have \( \tripleSemL{\lpre}{\trans_{1}}{\lpost} \), then we have \( \fph_{q} \in \evalLE[\lenv, \stk_{q}]{\lpost} \).
Symmetrically, if it picks \( \trans_{2} \), it yields the same result.

\caseI{\rl{TRSeq}}

We have \( \trans \equiv \trans_{1} \pseq \trans_{2} \) where \( \tripleL{\lpre}{\trans_{1}}{\lframe} \) and \( \tripleL{\lframe}{\trans_{2}}{\lpost} \) hold, for some \( \lpre, \lframe, \lpost, \trans_{1}, \trans_{2} \).
Given the semantics in \fig \ref{fig:thread_semantics}, for any \( \lenv, \stk_{p}, \fph_{p} \) that \( \fph_{p} \in \evalLS[\lenv, \stk_{p}]{\lpre} \), it has \( ( \stk_{p}, \fph_{p} ), \trans_{1} \pseq \trans_{2} \toL^{*} ( \stk_{r}, \fph_{r} ), \pskip \pseq \trans_{1} \toL ( \stk_{r}, \fph_{r} ), \trans_{1} \toL^{*} ( \stk_{q}, \fph_{q} ), \pskip \) for some \( \stk_{r}, \stk_{q}, \fph_{r}, \fph_{q} \).
By the \ih, we have \( \fph_{r} \in \evalLS[\lenv, \stk_{r}]{\lframe} \), then by the \ih, we have \( \fph_{q} \in \evalLS[\lenv, \stk_{q}]{\lpost} \).

\caseI{\rl{TRLoop}}

Since the triple is only partial correct, meaning that if the transaction \( \trans \) terminates it will reach a state satisfying the post-condition \( \lpost \), it is sufficient to prove the follows,
\[
    \for{\lpre, \trans, \nat > 0} \tripleL{\lpre}{\trans^{\nat}}{\lpre} \implies \ \tripleSemL{\lpre}{\trans^{\nat}}{\lpre} \\
\]
where,
\[
\begin{rclarray}
    \trans^{1} & \defeq  & \trans \\
    \trans^{\nat} & \defeq  & \trans \pseq \trans^{\nat - 1} \\
\end{rclarray}
\]
given the \ih that \(\tripleL{\lpre}{\trans}{\lpre} \implies \ \tripleSemL{\lpre}{\trans}{\lpre} \) holds.

We prove that by induction on the number \( \nat \).
For \( \nat = 1 \), it is proven by the \ih.
For \( \nat > 1 \), assume \( \stk, \fph, \lenv \) that satisfy \( \fph \in \evalLS{\lpre} \). 
Given the semantics in \fig \ref{fig:thread_semantics}, we have \( (\stk, \fph), \trans \pseq \trans^{\nat - 1} \toL^{*} (\stk' \fph'), \trans^{\nat - 1} \) for some \( \stk', \fph' \).
By the \ih that \(\tripleSemL{\lpre}{\trans}{\lpre} \), we have \( \fph' \in \evalLS[\lenv, \stk']{\lpre} \).
Then for any \( \stk'', \fph'' \) that \( (\stk', \fph'), \trans^{\nat - 1} \toL^{*} (\stk'' \fph''), \pskip \), by \ih that \( \tripleSemL{\lpre}{\trans}{\lpre} \), we have \( \fph'' \in \evalLS[\lenv, \stk'']{\lpre} \).
\caseI{\rl{TRFrame}}

We have \( \tripleL{\lpre \sep \lframe }{\trans}{\lpost \sep \lframe} \) and \( \tripleL{\lpre}{\trans}{\lpost} \) for some \( \lpre, \lpost, \lframe, \trans\).
For the precondition, we know \( \fph_{p} \composeFP \fph_{r} \in \evalLS[\lenv, \stk_{p}]{\lpre \sep \lframe} \) where \(  \fph_{p} \in \evalLS[\lenv, \stk_{p}]{\lpre} \) and \( \fph_{r} \in \evalLS[\lenv, \stk_{p}]{\lframe} \) for some \( \fph_{p}, \fph_{r}, \stk_{p}, \lenv \).
By the \( \pred{tsound}{\lpre, \trans, \lpost } \) from \ih, we have \( ( \stk_{p}, \fph_{p} ), \trans \toL^{*} (\stk_{q}, \fph_{q}), \pskip \) and \( \fph_{q} \in \evalLS[\lenv, \stk_{q}]{\lpost}\), for some \( \fph_{q}, \stk_{q} \).
Therefore, it is also true that \( ( \stk_{p}, \fph_{p} \composeLS \fph_{r} ), \trans \toL^{*} (\stk_{q}, \fph_{q} \composeLS \fph_{r}), \pskip \), so that \( \fph_{q} \composeFP \fph_{r} \in \evalLS[\lenv, \stk_{q}]{\lpost \sep \lframe} \).

\end{proof}

\begin{defn}[Conversion to time-stamp heaps]
\label{def:x2tsh}
Given the set of world \( \world \in \World \), fingerprint world \( \fpw \in \FPWorlds \), \( \ls \in \LStates\) the set of \( \tsh \in \TSHeaps \), the overloaded function \( \funcn{x2tsh} : \Set{\World, \FPWorlds, \LStates} \to \powerset{\TSHeaps} \) is defined as follows,
\[
    \begin{rclarray}
        \func{x2tsh}{\world} & \defeq & 
        \Setcon{%
            (\tsh,\ts) 
        }{%
            \exsts{ \h }
            \flattenW{\world} = (\h, \stub) 
            \land \snapshot{\tsh}{\ts}(\addr) = (\val, \stub) 
            \land \for{ \addr } \h(\addr) = \val 
        } \\
        \func{x2tsh}{\fpw} & \defeq & 
        \Setcon{%
            (\tsh,\ts) 
        }{%
            \exsts{ \h }
            \flattenW{\eraseFW{\fpw}} = (\h, \stub) 
            \land \snapshot{\tsh}{\ts}(\addr) = (\val, \stub) 
            \land \for{ \addr } \h(\addr) = \val 
        } \\
        \func{x2tsh}{\ls} & \defeq & 
        \Setcon{%
            (\tsh,\ts) 
        }{%
            \exsts{ \h }
            \ls = (\h, \stub) 
            \land \snapshot{\tsh}{\ts}(\addr) = (\val, \stub) 
            \land \for{ \addr } \h(\addr) = \val 
         }
    \end{rclarray}
\]
where the \( \snapshotName \) function (\fig \ref{fig:thread_semantics}) returns a fingerprint heap corresponding the state at time \( \ts \), and here we match first projection, i.e. the values, with the flattened world.
\end{defn}

\begin{defn}[Semantic triple]
\label{def:semantic-triple}
    The semantic triple \( \tripleSemG{\gpre}{\prog}{\gpost}\) is defined as the follows,
    \[
        \begin{rclarray}
            \tripleSemG{\gpre}{\prog}{\gpost} & \defeq &
            \begin{array}[t]{@{}l@{}}
                \for{\fpw_{p}, \tsh_{p}, \tsh_{q}, \lenv, \stk_{p}, \stk_{q}, \ts_{p}, \ts_{q} }  
                \stable{\gpre}{\intf} \\
                \quad {} \land \eraseFW{\fpw_{p}} \in \evalW[\lenv, \stk_{p}]{\gpre}
                \land (\tsh_{p}, \ts_{p}) \in \func{x2tsh}{\fpw_{p}}
                \land (\stk_{p}, \tsh_{p}, \ts_{p}), \prog \toT{}^{*} (\stk_{q}, \tsh_{q}, \ts_{q}), \pskip \\
                \quad \implies \exsts{ \fpw_{q} } \eraseFW{\fpw_{q}} \in \evalW[\lenv, \stk_{q}]{\gpost}
                \land (\tsh_{q}, \ts_{q}) \in \func{x2tsh}{\fpw_{q}}
                \land \stable{\gpost}{\intf}
            \end{array}
        \end{rclarray}
    \]
\end{defn}


\begin{thm}[Soundness]
The soundness is defined as follows:
\[
    \begin{array}{@{}l@{}}
        \for{\gpre, \gpost, \prog, \intf } \tripleG{\gpre}{\prog}{\gpost} \implies \tripleSemG{\gpre}{\prog}{\gpost}
    \end{array}
\]
\end{thm}
\begin{proof}
Induction on the rules for program \( \prog \).

\caseB{\rl{PRCommit}}
We have \( \tripleG{\gpre}{\ptrans{\trans}}{\gpost} \) given that \( \tripleL{\lpre}{\trans}{\lpost} \), \( \repartition{\gpre}{\gpost}{\lpre}{\lpost} \), \( \stable{\gpre}{\intf} \) and \( \stable{\gpost}{\intf} \) for any \( \trans, \gpre, \gpost, \lpre, \lpost, \inter \). 
For any \( \stk_{p}, \lenv \), let variables \( \world_{p}, \fpw_{p}, \fph_{p}, \fph_{f} \) to satisfy the follows,
\begin{equation}
    \label{equ:def-wwhh}
    \eraseFW{\fpw_{p}} \in \evalW[\lenv, \stk_{p}]{\gpre} 
    \land \flattenFW{\fpw_{p}} = (\fph_{p} \composeFP \fph_{f}, \stub)
    \land \fph_{p} \in \evalLS[\lenv, \stk_{p}]{\lpre}
\end{equation}
Note that we pick the names which are consistent with repartitioning in \defin \ref{def:repartitioning}.
By the soundness of \( \tripleL{\lpre}{\trans}{\lpost} \) (\thmref{thm:transaction-soundness}) and \equref{equ:def-wwhh}, for all \( \fph_{q}, \stk_{q} \), we have the follows,
\begin{equation}
    \label{equ:transaction-soundness}
    (\stk_{p}, \fph_{p} ), \trans \toL^{*}  (\stk_{q}, \fph_{q} ), \pskip 
    \implies \fph_{q} \in \evalLS[\lenv, \stk_{q}]{\lpost}
\end{equation}
Also, given the semantic triple (\defref{def:semantic-triple}) and the operational semantics (\figref{fig:thread_semantics}), let variables \( \tsh_{p}, \tsh_{q}, \ts_{p}, \ts_{q}, \tsid \) satisfy the follows,
\begin{equation}
    \label{equ:commit-current-trans}
    (\stk_{p}, \tsh_{p}, \ts_{p}), \ptrans{\trans} \toT{\lbC{\tsid}} (\stk_{q}, \tsh_{q}, \ts_{q}), \pskip 
\end{equation}
Now we consider addresses being written and read separably.
First, for any address \( \addr_{w} \) being tagged as write or read/write in \( \fph_{q} \), assume the value is \( \val_{w} \).
\begin{equation}
    \label{equ:local-write}
    \exsts{\fp} 
    \fph_{q}(\addr_{w}) = (\val_{w}, \fp)
    \land \fpW \in \fp
\end{equation}
By the \( \commitName \) function in operational semantics (\defref{fig:thread_semantics}), \equref{equ:commit-current-trans}, \equref{equ:transaction-soundness} and \equref{equ:local-write}, we have,
\begin{equation}
    \label{equ:global-write}
    \tsh_{q}(\addr_{w})(\ts_{q}) = (\val_{w}, \etW, \tsid)
\end{equation}
By the repartitioning (\defin \ref{def:repartitioning}), \equref{equ:def-wwhh} and \equref{equ:transaction-soundness}, for any \( \fpw_{q} \) we have,
\begin{equation*}
    \fpw_{q} \in \mergeR{\fpw_{p}}{\fpw_{i}}{\inter} \land \eraseFW{\fpw_{q}} \in \evalW[\lenv, \stk_{q}]{\gpost}
\end{equation*}
Given the \( \mergeName[R] \) function (\defin \ref{def:repartitioning}) that uses several levels of merges until \( \mergeName[\val] \) function (\defin \ref{def:merge-finger-heap}), \equref{equ:transaction-soundness} and \equref{equ:local-write}, we have,
\begin{equation}
    \label{equ:write-remain-the-same}
    \begin{array}{@{}l@{}}
      \for{ \fpw_{q} \in \mergeR{\fpw_{p}}{\fpw_{i}}{\inter} }  
      \eraseFW{\fpw_{q}} \in \evalW[\lenv, \stk_{q}]{\gpost} \\
      \quad {} \land \exsts{\fph} \flattenFW{\fpw_{q}} = (\fph, \stub) \land \fph(\addr_{w}) = \fph_{q}(\addr_{w}) = (\val_{w}, \etW)
    \end{array}
\end{equation}
which matches with Eq. \eqref{equ:global-write}.
Intuitively, because the address being written cannot be merged with others.

Second, we consider addresses \( \addr_{r} \) that are only being read with the value \( \val_{r} \),
\begin{equation}
    \label{equ:local-read}
    \fph_{q}(\addr_{r}) = (\val_{r}, \Set{\fpR})
\end{equation}
By the \( \commitName \) function in operational semantics (\defref{fig:thread_semantics}), \equref{equ:commit-current-trans}, \equref{equ:transaction-soundness} and \equref{equ:local-read}, we have,
\begin{equation}
    \label{equ:global-read}
    \tsh_{q}(\addr_{r})(\ts_{p}) = (\val_{r}, \etR , \tsid)
    \land \tsh_{q}(\addr_{r})(\ts_{q}) = (\val_{r}, \etE , \tsid)
\end{equation}
However, note that,
\[
\neg\left(
    \begin{array}{@{}l@{}}
        \for{\addr_{r}} \exsts{\val_{r}} 
        \tsh_{q}(\addr_{r})(\ts_{p}) = (\val_{r}, \etR, \tsid )
        \land \tsh_{q}(\addr_{r})(\ts_{q}) = (\val_{r}, \etE , \tsid) \\
        \quad \implies \snapshot{\tsh_{q}}{\ts_{q}}(\addr_{r}) = (\val_{r}, \stub)
    \end{array}
\right)
\]
Because there might be other transactions that commits between times \( \ts_{p} \) and \( \ts_{q} \) and writes to some addresses \( \addr_{r} \), which is allowed by the \( \consistentName \) predicate from the operational semantics (\figref{fig:thread_semantics}).
Let \( \tsid_{1} \) to \( \tsid_{\nat} \) be the transactions that commits between times \( \ts_{p} \) and \( \ts_{q} \) and writes to some addresses \( \addr_{r} \) as the follows.
We also assume those transactions are allowed by the \( \relyU \).
\begin{equation}
    \label{equ:concurrent-trans}
    \bigwedge\limits_{1 \leq i \leq \nat} 
    \begin{formulea}
    \exsts{\fpw, \fpw', \ts, \ts', \etag \in \Set{\etS, \etR} } 
    \ts < \ts' 
    \land \ts_{p} < \ts' < \ts_{q} \\
    {} \land \tsh_{q}(\addr_{r})(\ts) = (\stub, \etag, \tsid_{i}) 
    \land \tsh_{q}(\addr_{r})(\ts') = (\stub, \etW, \tsid_{i}) \\
    {} \land \exsts{\ca} \fpw_{p} = (\ca, \stub) 
    \land \fpw = (\ca, \stub) 
    \land \fpw' = (\ca, \stub) \\
    {} \land (\fpw, \fpw') \in \relyU
    \land (\tsh_{q}, \ts) \in \func{x2tsh}{\fpw}
    \land (\tsh_{q}, \ts') \in \func{x2tsh}{\fpw'}
    \end{formulea}
\end{equation}
By the \( \consistentName \) predicate from the operational semantics (\fig \ref{fig:thread_semantics}), we know the first transaction \( \tsid_{1} \) must read the same value as current transaction \( \tsid \).
\begin{equation}
\label{equ:read-the-same-value}   
\exsts{ \ts, \etag } 
\tsh_{q}(\addr_{r})(\ts) = (\val_{r}, \etag, \tsid_{1}) 
\land \etag \in \Set{\etS, \etR}
\end{equation}
Also, if two transactions write to the same addresses, it must be strictly one after another.
\begin{equation}
\label{equ:write-one-after-another}
    \begin{array}{@{}l@{}}
        \bigwedge\limits_{1 \leq i \leq \nat} 
        \exsts{ \ts, \ts', \etag \in \Set{\etS, \etR}, \val } 
        \ts < \ts'
        \land \tsh_{q}(\addr_{r})(\ts_{i}') = (\val, \etW, \tsid_{i - 1}) 
        \land \tsh_{q}(\addr_{r})(\ts_{i}) = (\val, \etag, \tsid_{i})
    \end{array}
\end{equation}
Given the definition of rely (\defref{def:rely-guarantee}), by \equref{equ:global-read}, \equref{equ:concurrent-trans}, \equref{equ:read-the-same-value}, \equref{equ:write-one-after-another}, and then induction on the number \( \nat \), we have,
\begin{equation}
    \label{equ:allowed-by-rely}
    \begin{array}{@{}l@{}}
        \exsts{\fpw, \ts, \etag \in \Set{\etS, \etR}, \ca }
        \tsh_{q}(\addr_{r})(\ts) = (\val_{r}, \etag, \tsid_{i}) 
        \land \fpw_{p} = (\ca, \stub) 
        \land \fpw = (\ca, \stub)  \\
        {} \land \bigwedge\limits_{1 \leq i \leq \nat} 
        \begin{formulea}
        \exsts{ \fpw', \ts' } 
        \ts < \ts' 
        \land \ts_{p} < \ts' < \ts_{q} 
        \land \tsh_{q}(\addr_{r})(\ts') = (\stub, \etW, \tsid_{i}) 
        \land \fpw' = (\ca, \stub) \\
        {} \land (\fpw, \fpw') \in \rely_{i}
        \land (\tsh_{q}, \ts) \in \func{x2tsh}{\fpw}
        \land (\tsh_{q}, \ts') \in \func{x2tsh}{\fpw'}
        \end{formulea}
    \end{array}
\end{equation}
where the \( \rely_{i} \) are defined in \defref{def:rely-guarantee}.
Let looks the first transaction \( \tsid_{1} \) and the last transaction \( \tsid_{\nat} \).
Assume the start state of \( \tsid_{1} \) is \( \fpw_{1} \), the end state of \( \tsid_{\nat} \) is \( \fpw_{\nat} \) and the final value being written to address \( \addr_{r} \) is \( \val_{\nat} \).
By re-writing the \equref{equ:allowed-by-rely}, we have the follows,
\begin{equation}
    \label{equ:first-and-last-concurrent-trans}
    \begin{array}{@{}l@{}}
        \exsts{ \ts_{1}, \ts_{\nat}, \etag \in \Set{\etS, \etR}, \ca }
        \ts_{1} < \ts_{\nat}
        \land \ts_{p} < \ts_{\nat} < \ts_{q}  \\
        \quad {} \land \fpw_{p} = (\ca, \stub) 
        \land \fpw_{1} = (\ca, \stub)  
        \land \fpw_{n} = (\ca, \stub) 
        \land (\fpw_{1}, \fpw_{\nat}) \in \rely_{\nat} \subseteq \myrely \\
        \quad {} \land \tsh_{q}(\addr_{r})(\ts_{1}) = (\val_{r}, \etag, \tsid_{1}) 
        \land \tsh_{q}(\addr_{r})(\ts_{\nat}) = (\val_{\nat}, \etW, \tsid_{\nat})  \\
        \quad {} \land (\tsh_{q}, \ts_{1}) \in \func{x2tsh}{\fpw_{1}}
        \land (\tsh_{q}, \ts_{\nat}) \in \func{x2tsh}{\fpw_{\nat}}
    \end{array}
\end{equation}
Given the  \equref{equ:first-and-last-concurrent-trans}, and the \( \mergeName[R] \) function that is used in repartitioning (\defref{def:repartitioning}), we have,
\begin{equation}
\label{equ:read-can-be-merged}
    \begin{array}{@{}l@{}}
      \exsts{ \fpw_{q} \in \mergeR{\fpw_{p}}{\fpw_{i}}{\inter}, \fph }  
      \eraseFW{\fpw_{q}} \in \evalW[\lenv, \stk_{q}]{\gpost} \\
      \quad {} \land \flattenFW{\fpw_{q}} = (\fph, \stub) \land \fph(\addr_{r}) = \fph_{q}(\addr_{r}) = (\val_{\nat}, \etW)
    \end{array}
\end{equation}

Now combining \equref{equ:write-remain-the-same} and \equref{equ:read-can-be-merged}, we have, 
\begin{equation}
    \eraseFW{\fpw_{q}} \in \evalW[\lenv, \stk_{q}]{\gpost}
    \land (\tsh_{q}, \ts_{q}) \in \func{x2tsh}{\fpw_{q}}  \\
\end{equation}
Then since \( \stable{\gpost}{\intf} \) is proven by the premiss, we have the prove for the \rl{PRCommit}.


\end{proof}










\sx{No use for below}

\begin{defn}[Transaction interpretation]
\label{def:transactions-interpretation}
Given the set of transaction \( \trans \in \Transactions \) (\defin \ref{def:language}) and the operational semantics (\fig \ref{fig:transaction_semantics}), the \emph{transaction interpretation} function \( \intpSQ{.} : \Transactions \to \FPHeaps \to \powerset{\FPHeaps} \) is defined as follows:
\[
    \begin{rclarray}
        \intpSQ{\trans}(\fph) & \defeq & 
            \Setcon{%
                \fph'
            }{%
                \exsts{ \stk, \stk' } (\stk, \fph ), \trans \toL^{*}  (\stk', \fph' ), \pskip
            }\\
    \end{rclarray}
\]
\end{defn}
\sx{probably use the above to prove%
\[
    \tripleL{\lpre}{\trans}{\lpost}
\]%
Simply for less words, maybe?
}
Note that the stack is local and has no side effect to the fingerprint heap, therefore from now we will fix the stack and treat the stack as the same as logical environment.

\begin{defn}[Reification function]
\label{def:reification}
Given the set of assertions \( \gpre \in \Ast \) and time-stamp heap \( \TSHeaps \), the \emph{reification} function \( \reif{.} : \Ast \to \powerset{\TSHeaps} \) is defined as follows:
\[
\begin{rclarray}
    \reif{\gpre} & \defeq & 
    \Setcon{%
        \tsh
    }{%
        \exsts{\lenv, \stk, \h} \world \in \evalW{\gpre} 
        \land (\h, \stub ) = \flattenW{\world}
        \land \tsh \in\func{h2tsh}{\h} 
    }
\end{rclarray}
\]
where the interpretation of assertion \( \evalW{.} \) is defined in \defin \ref{def:assertion}, the world flattening \( \flattenW{.} \) in \defin \ref{def:world} and \( \funcn{h2tsh} \) function in \defin \ref{def:h2tsh}.
\end{defn}

\begin{defn}[Atomic interpretation]
\label{def:atomic-intp}
%Given the set of transactions \( \trans \in \Transactions \) (\defin \ref{def:language}), the set of programs \( \Programs \) and the operational semantics (\fig \ref{fig:thread_semantics}), the \emph{atomic interpretation} function \( \intfATOM{.} : \Atom \to \TSHeaps \to \powerset{\TSHeaps} \) is defined as follows, where \( \Atom \defeq \Setcon{\ptrans{\trans}}{\trans \in \Transactions \land \ptrans{\trans} \in \Programs} \).
\[
    \begin{rclarray}
        \intpSQ{\ptrans{\trans}}(\tsh) & \defeq & 
            \Setcon{%
                \tsh'
            }{%
                \exsts{ \stk, \stk', \ts, \ts' } (\stk, \tsh, \ts ), \ptrans{\trans} \toG{\stub}  (\stk', \tsh', \ts' ), \pskip
            }\\
    \end{rclarray}
\]
\end{defn}

\begin{thm}[Axiom soundness]
Given the set of transactions \( \trans \in \Transactions \) (\defin \ref{def:language}) and the rely relation \( \myrely \) (\defin \ref{def:rely-guarantee}), and assume stardard lift for reification function (\defin \ref{def:reification}) and for atomic interpretation (\defin \ref{def:atomic-intp}), the axiom soundness is defined as follows:
\[
    \begin{array}{@{}l@{}}
        \for{ \trans, \gpre, \gpost } 
        \tripleG{\gpre}{\ptrans{\trans}}{\gpost} \\
        \quad {} \land \for{\world } \intpSQ{\ptrans{\trans}}\left( \reif{\gpre \sep \Set{\world}} \right) \subseteq \left( \reif{\gpost \sep \myrely(\Set{\world})} \right) 
     \end{array}
\]
where \( \Set{\world} \) denotes assertion satisfying \( \world \) under any logical environment and stack, i.e.\ \( \for{\lenv, \stk} \evalW{\Set{\world}} = \Set{\world} \).
\end{thm}
\begin{proof}
Given the reification function, it is sufficient to prove:
\[
    \begin{array}{@{}l@{}}
        \for{ \trans, \gpre, \gpost } 
        \tripleG{\gpre}{\ptrans{\trans}}{\gpost} \\
        \quad {} \land \for{ \stk ,\lenv, \world_{f}, \world_{p} } 
        \exsts{ \world_{q}, \world_{f}' } 
        \world_{p} \in \evalW{\gpre}
        \land \world_{q} \in \evalW{\gpost}
        \land \world_{f}' \in \myrely(\world_{f})  \\
        \quad {} \land \for{ \h_{p}, \tsh_{p} } 
        \exsts{ \h_{q}, \tsh_{q} }
        (\h_{p}, \stub) = \flattenW{\world_{p} \composeW \world_{f}} 
        \land \tsh_{p} \in \func{h2tsh}{\h_{p}}
        \land (\h_{q}, \stub) = \flattenW{\world_{q} \composeW \world_{f}'} 
        \land \tsh_{q} \in \func{h2tsh}{\h_{q}} \\
        \quad \implies \tsh_{q} \in \intpSQ{\ptrans{\trans}} \left( \tsh_{p} \right)
     \end{array}
\]
First we introduce some new variables \( \fpw_{p}, \fph_{p}, \fpw_{q}, \fph_{q} \) that satisfy the follows and whose names are consistent with those in \defin \ref{def:repartitioning}.
\[
\begin{array}{@{}l@{}}
    \exsts{\fph} \\
    \eraseFW{\fpw_{p}} = \world_{p} 
    \land \flattenFW{\fpw_{p}} = (\fph_{p} \composeFP \fph, \unitC)
    \land \for{\addr} \fph_{p}(\addr) = (\stub, \emptyset) \land {} \\
    \eraseFW{\fpw_{q}} = \world_{q} 
    \land \flattenFW{\fpw_{q}} = (\fph_{q} \composeFP \fph, \unitC)
\end{array}
\]
By the definition of repartition (\defin \ref{def:repartitioning})  and transaction soundness (Theorem \ref{thm:transaction-soundness}):
\[
\begin{array}{@{}l@{}}
    \for{\fph} 
    \fph_{q} \composeFP \fph \in \intpSQ{\trans}( \fph_{p} \composeFP \fph) 
\end{array}
\]
\end{proof}

\subsection{Transaction Soundness}

The \cref{thm:transaction-soundness} is the soundness of for transactions, where the most interesting two rules \rl{TRLookup} and \rl{TRMutate} can be derived from the \cref{lem:fingerprint-op}.
Note that the transactional semantics is defined on snapshots as total functions.
Thus in the \cref{thm:transaction-soundness}, it is safe to extend \( h \) with \( \sn'' \)  to make the snapshot a total function, as there is no allocation and deallocation.

\begin{theorem}[Transaction soundness]
\label{thm:transaction-soundness}
The transaction soundness is as follows:
\[
    \begin{array}{@{}l@{}}
        \fora{ \lpre, \trans, \lpost } \tripleL{\lpre}{\trans}{\lpost}
        \implies 
        \fora{\lenv, \stk, \stk', \sn_\lpre, \sn_\lpost, \sn \fp, \fp' }  \\
        \quad (\sn_\lpre, \fp) \in \evalLS[\lenv, \stk]{\lpre}
        \land \dom(\sn) = \Keys \setminus \dom(\sn) \\
        \quad {} \land \vdash (\stk, \sn_\lpre \composeH \sn, \fp ), \trans \toL^{*}  (\stk', \sn_\lpost \composeH \sn, \fp' ), \pskip  \\
        \qqquad \implies (\sn', \fp') \in \evalLS[\lenv, \stk']{\lpost}
    \end{array}
\]
\end{theorem}
\begin{proof}
Induction on the derivations.

\begin{itemize}

\item \caseB{\rl{TRSkip}}
We have \(\trans \equiv \pskip\), \( \lpre \equiv \lpost \equiv \assemp \), thus \( \sn_{p} = \sn_{q} = \unitH \), \( \fp = \fp' \) and \( \stk = \stk' \), and then \( (\unitH,\unitO ) \in \evalLS[\lenv, \stk']{\assemp} \) holds.

\item \caseB{\rl{TRAss}}
We have \(\trans \equiv ( \pass{\var}{\expr} ) \), \( \lpre \equiv ( \var \doteq \lexpr ) \) and \( \lpost \equiv ( \var \doteq \expr\sub{\var}{\lexpr} ) \) 
for some \( \expr, \lexpr \) and \( \var \) such that \( \var \notin \func{fv}{\lexpr}\).
Given the transaction semantics (\cref{fig:thread_semantics}), it has \( \stk' = \stk\rmto{\var}{\val} \) where \( \val = \evalLE[\lenv, \stk]{\expr\sub{\var}{\lexpr}} \).
Since \( \var \notin \func{fv}{\lexpr} \), we know \( \evalLE[\lenv, \stk]{\lexpr} = \evalLE[\lenv, \stk']{\lexpr} \), and then \( \evalLE[\lenv, \stk]{\expr\sub{\var}{\lexpr}} = \evalLE[\lenv, \stk']{\expr\sub{\var}{\lexpr}} \).
This means the assertions related to stack hold even thought the stack changes.
Also because the snapshot and fingerprint remain unchanged, \ie \( \sn = \sn' \) and \( \fp = \fp' \), so we prove \( (\sn', \fp' ) \in \evalLS[\lenv, \stk']{\lpost} \).

\item \caseB{\rl{TRLookup}}
We have \(\trans \equiv ( \plookup{\var}{\expr} ) \) and four cases for pre- and post-conditions defined by the relation \( \toFP{\otR(\expr, \lexpr)}\).
In all the four cases, the stack is updated to \( \stk' = \stk\rmto{\var}{\val} \), yet since \( \var \notin \func{fv}{\lexpr}\), the logical value \( \lexpr \) and new logical address \( \expr\sub{\var}{\lexpr}\) are evaluated to the same value \( \val \) and address \( \key \) as before.
By \cref{lem:appendix-fingerprint-op}, we know that the final state of snapshot and fingerprint satisfy the postcondition.

\item \caseB{ \rl{TRMutate} }
We have  \( \trans \equiv (\pmutate{\expr_{1}}{\expr_{2}}) \) and four cases for pre- and post-conditions defined by the relation \( \toFP{\otW(\expr, \lexpr)}\). 
In all the four cases, the stack remains untouched and  by \cref{lem:appendix-fingerprint-op} the snapshot and fingerprint match the postcondition.

\item \caseI{\rl{TRChoice}}
We have  \(\trans \equiv \trans_{1} + \trans_{2} \), where \( \tripleL{\lpre}{\trans_{1}}{\lpost} \) and \( \tripleL{\lpre}{\trans_{2}}{\lpost} \) hold, for some \( \trans_{1}, \trans_{2}, \lpre, \lpost \).
Given the transaction semantics (\cref{fig:thread_semantics}), it either has 
\[ 
( \stk, \sn_\lpre \composeH \sn, \fp ), \trans_{1} \pchoice \trans_{2} \toL ( \stk, \sn_\lpre \composeH \sn, \fp ), \trans_{1} 
\]
or  
\[ 
( \stk, \sn_\lpre \composeH \sn, \fp ), \trans_{1} \pchoice \trans_{2} \toL ( \stk, \sn_\lpre \composeH \sn, \fp ), \trans_{2} 
\]
Let us pick \( \trans_{1} \) and  assume it can be reduced to \( \pskip \) from the initial state, \ie \( ( \stk, \sn_\lpre \composeH \sn, \fp ), \trans_{1}  \toL^{*} ( \stk', \sn', \fp' ), \pskip \).
By the premiss of the rule \( \tripleL{\lpre}{\trans_{1}}{\lpost} \) and the \ih, 
there exists a \( \sn_\lpost \) such that \( \sn' = \sn_\lpost \composeH \sn \) and  \( (\sn_\lpost, \fp') \in \evalLE[\lenv, \stk']{\lpost} \).
Symmetrically, if we pick \( \trans_{2} \), it gives the same result.

\item \caseI{\rl{TRSeq}}
We have \( \trans \equiv \trans_{1} \pseq \trans_{2} \) where \( \tripleL{\lpre}{\trans_{1}}{\lframe} \) and \( \tripleL{\lframe}{\trans_{2}}{\lpost} \) hold, for some \( \trans_{1}, \trans_{2}, \lpre, \lpost, \lframe \).
Given the transaction semantics (\cref{fig:thread_semantics}), 
it has 
\[
    \begin{array}{l}
    \vdash ( \stk, \sn_\lpre \compose \sn, \fp ), \trans_{1} \pseq \trans_{2} \toL^{*} ( \stk'', \sn'', \fp'' ), \pskip \pseq \trans_{1} \\
    \qqquad \toL ( \stk'', \sn'', \fp'' ), \trans_{1} \toL^{*} ( \stk', \sn', \fp' ), \pskip 
\end{array}
\] 
for a residue \( \sn \) and  some states \( (\stk', \sn', \fp'), (\stk'', \sn'', \fp'') \).
By the \ih that \( \tripleL{\lpre}{\trans_{1}}{\lframe} \) is sound,
there exists a snapshot \( \sn_\lframe \) such that \( \sn'' = \sn_\lframe \composeH \sn \) and \( (\sn_\lframe, \fp'') \in \evalLE[\lenv, \stk'']{\lframe} \).
The elimination of prefix \( \pskip \) does not change the state, so \( (\sn'', \fp'') \in \evalLE[\lenv, \stk'']{\lframe} \).
Then similarly, 
By the \ih that \( \tripleL{\lframe}{\trans_{2}}{\lpost} \) is sound,
there exists a snapshot \( \sn_\lpost \) such that \( \sn'' = \sn_\lpost \composeH \sn \) and \( (\sn_\lpost, \fp') \in \evalLE[\lenv, \stk']{\lpost} \).

\item \caseI{\rl{TRLoop}}
Since the triple is only partial correct, 
meaning that only when the transaction \( \trans \) terminates, it will reach a state satisfying the post-condition \( \lpost \).
It is sufficient to prove the following is sound:
\[
    \fora{\lpre, \trans, \nat > 0} \tripleL{\lpre}{\trans^{\nat}}{\lpre}
\]
where,
\[
    \trans^{0} \defeq  \pskip \qquad
    \trans^{\nat} \defeq  \trans \pseq \trans^{\nat - 1} 
\]

We prove that by induction on the number \( \nat \).
\begin{itemize}
    \item \caseB{\( \nat = 0 \)} It has been proven before \( \triple{\lpre}{\pskip}{\lpre} \).
    \item \caseI{\( \nat > 0 \)} We have 
    \[ 
        \begin{array}{l}
        \vdash (\stk, \sn_\lpre \composeH \sn, \fp), \trans \pseq \trans^{\nat - 1} \toL^{*} \\
        \qqquad (\stk'', \sn'', \fp''), \trans^{\nat - 1} \toL^{*} (\stk', \sn', \fp'), \pskip  
        \end{array}
    \]
    for a residue \( \sn \) and some states \( ( \stk', \sn', \fp' ), ( \stk'', \sn'', \fp'' ) \).
    This is similar to the \rl{PSeq} case.
\end{itemize}

\item \caseI{\rl{TRFrame}}
We need to prove \( \tripleL{\lpre \sep \lframe }{\trans}{\lpost \sep \lframe} \) is sound, 
given the soundness of  \( \tripleL{\lpre}{\trans}{\lpost} \).
Assume snapshots \( \sn_\lpre, \sn_\lpost, \sn_\lframe \), fingerprints \( \fp, \fp', \fp'' \) and stacks\( \stk, \stk' \) 
such that \( ( \sn_\lpre, \fp ) \in \evalLS[\lenv, \stk]{\lpre} \), \( ( \sn_\lframe, \fp' ) \in \evalLS[\lenv, \stk']{\lpost} \) and \( ( \sn_\lpost, \fp'' ) \in \evalLS[\lenv, \stk]{\lframe}\).
Since \( \lpre \sep \lframe \), the component-wise composition is defined, \ie \( (\sn_\lpre \composeH \sn_\lframe, \fp \composeO \fp'') \in \evalLS[\lenv, \stk]{\lpre \sep \lframe} \).
The keys from the snapshots and fingerprints are disjointed.
By the hypothesis that \( \tripleL{\lpre}{\trans}{\lpost} \) is sound, 
we know \( ( \stk, \sn_\lpre \composeH \sn, \fp ), \trans \toL^{*} ( \stk', \sn_\lpost \composeH \sn, \fp' ), \pskip \) for some residues \( \sn \).
The snapshot after the execution contains the same resources as before, that is \( \dom(\sn_\lpre) = \dom(\sn_\lpost) \) and \( \dom(\fp') \subseteq \dom(\sn_\lpost) \).
We know the composition of \( \sn_\lpost \composeH \sn_\lframe \) and \( \fp' \composeO \fp''\) exists, 
so 
\[ 
    ( \stk, \sn_\lpre \composeH \sn_\lframe \composeH \sn', \fp \composeO \fp''), \trans \toL^{*} ( \stk', \sn_\lpost \composeH \sn_\lframe \composeH \sn', \fp' \composeO \fp'' ), \pskip 
\]
for some residues \( \sn' \).
Note that \( \dom(\sn_\lframe \composeH \sn') = \dom(\sn) \).
Finally, because the transaction \( \trans \) does not modify any variable from the frame \( \lframe \), 
the update of stack does not change the evaluation of the frame, 
\( \evalLS[\lenv, \stk]{\lframe} = \evalLS[\lenv, \stk']{\lframe} \) which then gives us the result \( (\sn_\lpost \composeH \sn_\lframe, \fp' \composeO \fp'') \in \evalLS[\lenv, \stk']{\lpost \sep \lframe} \).
\end{itemize}
\end{proof}

\begin{lemma}
\label{lem:fingerprint-op}
\label{lem:appendix-fingerprint-op}
For \( \mathtt{O} \in \Set{\otR, \otW} \), the relation \( \toFP{\mathtt{O}(\key,\val')}\) is sound with respect to the operator \( \addO \):
\[
\begin{array}{@{}l}
    \fora{\lpre, \lpost, \lenv, \stk, \stk', \fp, \fp', \mathtt{O}, \key, \val} \\
    \quad \mathtt{O} \in \Set{\otR, \otW} 
    \land \lpre \toFP{\mathtt{O}(\key,\val)} \lpost
    \land (\stub, \fp) \in \evalLS{\lpre}
    \land (\stub, \fp') \in \evalLS{\lpost} \\
    \qqquad \implies \fp' = \fp \addO (\mathtt{O}, \key, \val)
\end{array}
\]
\end{lemma}
\begin{proof}
\begin{itemize}
    \item In this case, a read operation is added \( \fp' = \Set{(\otR, \key, \val)} \) and it is included in the interpretation of the post condition \( \lpost \equiv \key \fpR \val \).
If \( \lpre \equiv \key \fpR \val \) and \( \lpost \equiv \key \fpR \val \), 
since there is already a read operation in the fingerprint, adding a new read operation does not change the fingerprint, \ie \( \fp' = \Set{(\otR, \key, \val)} \addO (\otR, \key, \val ) = \Set{(\otR, \key, \val)} \).
This is exactly the post-condition.
It is sound for the rest two cases that  when \( \lpre \equiv \key \fpW \val \) and \( \lpost \equiv \key \fpW \val \), and when \( \lpre \equiv \key \fpRW (\lexpr,\val) \) and \( \lpost \equiv \key \fpRW (\lexpr,\val) \), as the fingerprint remains the same in the rest two cases.

\item When \( \mathtt{O} = \otW\), we also have four cases for pre- and post-conditions.
If \( \lpre \equiv \key \fpI \val' \) for some value \( \val' \), a new write operation is added to the initially empty fingerprint, that is, \( \fp' = \Set{(\otW, \key, \val)}\).
This is exactly the post-condition \( \lpost \equiv \key \fpW \val \).
If \( \lpre \equiv \key \fpW \val' \) for some \( \val' \), the fingerprint before execution is \( \fp = \Set{(\otW, \key, \val')}\).
Since the fingerprint only has the last write for the key because of the property of the \( \addO \) operator,
the fingerprint after is \( \fp' = \fp \addO (\otW, \key, \val) = \Set{(\otW, \key, \val)}\) which satisfies the postcondition \( \lpost \equiv \expr_{1} \fpW \expr_{2} \).
The remaining two cases follow the same argument as the fingerprint only have the last write.
\end{itemize}
\end{proof}

\subsection{Program Soundness}
\begin{thm}[Program soundness]
The program soundness is the follows,
\[
    \for{\gpre, \prog, \gpost}
    \tripleG{\gpre}{\prog}{\gpost} 
    \implies 
    \tripleSemG{\gpre}{\prog}{\gpost} 
\]
\end{thm}
\begin{proof}
Induction on the derivations.
\caseB{\rl{PRCommit}}
We have \( \prog \equiv \ptrans{\trans} \).
Because a transaction \( \ptrans{\trans} \) is reduced by one step in the semantics, it is sufficient to prove the follows,
\[
\begin{array}{l}
    \begin{B}
        \stable{\gpre} 
        \land \gpre \snap \bar{\lpre}
        \land \tripleL{\bar{\lpre} \sep \fpEMP}{\trans}{\lpost \sep \fpF}
        \land \rpt{\gpre}{\gpost}{\fp} 
        \land \stable{\gpost}
    \end{B} \\
    \implies 
    \for{\w, \w', \w'', \w''', \hh', \hh'', \cu', \cu'', \thcu', \thcu'', \thid, \lenv, \thstk, \thstk''} \\
    \quad \begin{B}
        \w \in \evalW[\lenv, \thstk]{\gpre} 
        \land (\w, \w') \in \Rely^{*} 
        \land (\hh', \cu') \in \clpsW{\w'}
        \land \thcu'(\thid) = \cu' \\
        {} \land \thid, \func{como}{\w'} \vdash (\thstk, \hh', \thcu'), \ptrans{\trans} 
        \toT{\lbC{\txid}} (\thstk'', \hh'', \thcu''), \pskip  \\
        {} \land \thcu''(\thid) = \cu''
        \land (\hh'', \cu'') \in \clpsW{\w''} 
        \land (\w'', \w''') \in \Rely^{*} 
    \end{B} \\
    \quad \implies  \w''' \in \evalW[\lenv, \thstk'']{\gpost} 
\end{array}
\]
\textbf{Stable pre-condition.} 
Given that \( \stable{\gpre} \), for any world \( \w \) satisfies the pre-condition \( \gpre \) if the world can transfer to another world \( \w' \) through any steps of rely, the new world \( \w' \) also satisfies the pre-condition.
This is, 
\begin{equation}
    \label{equ:stable-pre-condition}
    \for{\w, \w',\lenv, \thstk} 
    \stable{\gpre} 
    \land \w \in \evalW[\lenv, \thstk]{\gpre} 
    \land (\w, \w') \in \Rely^{*}
    \implies \w' \in \evalW[\lenv, \thstk]{\gpre}
\end{equation}
\textbf{Commit.}
Given the \equref{equ:stable-pre-condition}, we want to prove the follows,
\begin{equation}
\label{equ:commit-new-transaction}
    \begin{array}{@{}l}
    \begin{B}
        \gpre \snap \bar{\lpre}
        \land \tripleL{\bar{\lpre} \sep \fpEMP}{\trans}{\lpost \sep \fpF}
        \land \rpt{\gpre}{\gpost}{\fp} 
    \end{B} \\
    \implies 
    \for{\w, \w', \hh, \hh', \cu, \cu', \thcu, \thcu', \thid, \lenv, \thstk, \thstk'} \\
    \quad \begin{B}
        \w \in \evalW[\lenv, \thstk]{\gpre}
        \land (\hh, \cu) \in \clpsW{\w}
        \land \thcu(\thid) = \cu \\
        {} \land \thid, \func{como}{\w} \vdash (\thstk, \hh, \thcu), \ptrans{\trans} 
        \toT{\lbC{\txid}} (\thstk', \hh', \thcu'), \pskip  \\
        {} \land \thcu'(\thid) = \cu'
        \land (\hh', \cu') \in \clpsW{\w'} 
    \end{B} \\
    \quad \implies  \w' \in \evalW[\lenv, \thstk']{\gpost} 
    \end{array}
\end{equation}
For any \( \w, \hh, \cu, \lenv, \thstk \) such that \( \w \in \evalW[\lenv,\thstk]{\gpre} \) and \( (\hh, \cu) \in \clpsW{\w} \), by the predicate \( \gpre \snap \bar{\lpre} \) we know \( \clpsHH{\hh, \cu} \in \evalLS[\lenv,\thstk]{\bar{\lpre}} \), so that,
\begin{equation}
\label{equ:local-pre-condition}
(\clpsHH{\hh, \cu}, \unitO) \in \evalLS[\lenv,\thstk]{\bar{\lpre} \sep \fpEMP}
\end{equation}
Because of the soundness of transaction (\thmref{thm:transaction-soundness}), given a thread stack \( \thstk \) and a logical environment \( \lenv \), if a initial configuration \( (\txstk, \h, \unitO), \trans \) satisfies the pre-condition, \ie \( (\h, \unitO) \in \evalLS[\lenv,\thstk \uplus \txstk]{\bar{\lpre} \sep \fpEMP} \), and a final configuration \( (\txstk', \h', \opset), \pskip \) is semantics reachable from the initial configuration, this final configuration will satisfy the post-condition \( \lpost \sep \fpF \).
This is, for any \( \thstk, \txstk, \txstk', \h, \h', \opset \), they satisfy the follows,
\begin{equation}
\label{equ:local-transaction-sound}
\begin{array}{@{}l}
    (\h, \unitO) \in \evalLS[\lenv,\thstk \uplus \txstk]{\bar{\lpre} \sep \fpEMP}
    \land \thstk \vdash (\txstk, \h, \unitO), \trans \toL (\txstk', \h', \opset), \pskip
    \implies (\h, \opset) \in \evalLS[\lenv,\thstk \uplus \txstk']{\bar{\lpost} \sep \fpF}
\end{array}
\end{equation}
The repartition \( \rpt{\gpre}{\gpost}{\fp} \) requires that any machine state \( \hh' \) by committing the operations \( \opset = \evalF{\fp}\) to an initial state \( \hh \) that satisfies the precondition \( \gpre \), should satisfy the \( \gpost \).
Formally, for any \( \w, \w', \hh, \hh', \cu, \cu', \lenv, \thstk, \txid \), we have,
\begin{equation}
\label{equ:repartition}
\begin{array}{@{}l}
    \begin{B}
        \w \in \evalW{\gpre}
        \land (\hh, \cu) \in \eraseW{\w}
        \land \txid \in \func{fresh}{\hh} \\
        {} \land \hh' = \func{commit}{\hh, \cu, \txid, \evalF{\fp}} 
        \land (\hh',\cu') \in \eraseW{\w'}
        \land (\w, \w') \in \Guar 
    \end{B}
    \implies \w' \in \evalW{\gpost}
\end{array}
\end{equation}
The guarantee says regions that has been updated must not violate their invariants,
\begin{equation}
\begin{array}{@{}l}
\end{array}
\end{equation}
\sx{HERE}

\begin{equation}
\label{equ:reachable}
\begin{array}{@{}l}
    \pred{reachable}{\eraseAEX{\aexecrun'}, \eraseAEX{\aexecrun''}, \txid, \evset, \como} \\
    \quad {} \land \h \in \obsstate{\aexecrun', \txidset, \respo} \land \thstk \vdash (\txstk, \h, \emptyset), \trans \toL^{*} (\txstk', \h', \evset ), \pskip
\end{array}
\end{equation}
\begin{equation}
\label{equ:local-transfer}
    (\h, \unitE) \in \evalLS{\lpre} 
    \land (\h', \unitE) \in \evalLS{\lpost} 
    \land (\unitH, \evset ) \in \evalLS{\fp}
    \land \tripleL{\lpre \sep \emptyset}{\trans}{\lpost \sep \fp}
\end{equation}
Since the new abstract execution \( \eraseAEX{\aexecrun''} \) is reachable from \( \eraseAEX{\aexecrun'} \) provided the set of events \( \evset \) (\equref{equ:reachable}) that is an interpretation of the fingerprint assertion \( \fp \) (\equref{equ:local-transfer}), it means there exists \(  \w'' \) satisfying the following because of the reapportion  \( \rpt{ \gpre }{ \gpost }{ \fp }\),
\[
    \w'' \in \evalW{\gpost} 
    \land \eraseAEX{ \aexecrun'' } \in \clpsW{\w''}
\]
\textbf{Stable post-condition} Similar to the stable pre-condition, we know
\[
    \w''' \in \evalW{\gpost} 
    \land \eraseAEX{ \aexecrun''' } \in \clpsW{\w'''}
\]
\sx{ Below is temporarily  }
\begin{equation}
    \aexecrun'' = \newaexec{\aexecrun', \thid, \evset, \txid, A, \conguar } 
\end{equation}
where \( \evset \) is a set of events corresponding to the code inside transaction, \( \txidset \) is  a set of transactions that are visible by the new transaction \( \txid \) and \( \conguar \) relates to the consistency model deciding whether the events \( \evset \) can happen and can be committed with respect to the visible transactions \( \txidset \).
Since \( \aexecrun'' \) is defined and by the \rl{PCommit} rule (\figref{fig:thread_semantics}), we know the following for some \( \h, \h', \thstk \),
\begin{equation}
    \h \in \obsstate{\aexecrun', \txidset, \respo}
    \land \thstk \vdash (\emptyset, \h, \emptyset) , \trans \ \toL^{*} \  (\txstk, \h', \evset) , \pskip 
\end{equation}
By the transaction soundness theorem (\thmref{thm:transaction-soundness}), there exist \( \lpre, \fp, \lpost \) such that
\[ 
\tripleL{\lpre \sep \emptyset}{\trans}{\lpost \sep \fp} \land (\h, \unitE) \in \evalLS{\lpre \sep \emptyset} \land (\h', \evset) \in \evalLS{\lpost \sep \fp}
\]
By the interpretation of assertions, it has,
\begin{equation}
\label{equ:interpretation-local-post}        
(\h', \unitE) \in \evalLS{\lpost} \land (\unitH, \evset) \in \evalLS{\fp}
\end{equation}
Combining \equref{equ:stable-pre-condition-and-the-graph}, \equref{equ:commit-new-transaction} and \equref{equ:interpretation-local-post}, we need to prove the follows,
\[
    \begin{array}[t]{@{}l@{}}
    \w' \in \evalW{\gpre} 
    \land \eraseAEX{ \aexecrun' } \in \clpsW{\w'}
    \land (\unitH, \evset) \in \evalLS{\fp} \\
    {} \land \aexecrun'' = \newaexec{\aexecrun', \thid, \evset, \txid, A, \conguar } 
    \land \rpt{\gpre}{\gpost}{\fp} \\
    \quad {} \implies 
    \exsts{\w''}
    \w'' \in \evalW{\gpost} 
    \land \eraseAEX{ \aexecrun'' } \in \clpsW{\w''}
    \end{array}
\]
\end{proof}

\begin{lem}
\label{lem:reachable}
Given a runtime abstract execution \( \aexecrun \), if it can reach anther runtime abstract execution \( \aexecrun'\) through the \rl{PCommit}, the former can reach the later statically, \ie
\[
\begin{array}{@{}l}
    \for{\aexecrun, \aexecrun', \thstk, \thstk', \como, \txid, \thid}
    \como, \thid \vdash (\thstk, \aexecrun), \ptrans{\trans}
    \toT{\lbC{\txid}} (\thstk', \aexecrun'), \pskip \\
    \quad {} \implies \pred{reachable}{\eraseAEX{\aexecrun}, \eraseAEX{\aexecrun'}, \txid, \evset, \como}
\end{array}
\]  
where \( \evset \) satisfies the conditions the of the \rl{PRCommit},
\[
    \exsts{\txidset, \h, \h', \txstk, \txstk'} \h \in \obsstate{\aexecrun, \txidset, \respo} \land \thstk \vdash (\txstk, \h, \emptyset), \trans \toL^{*} (\txstk', \h', \evset ), \pskip
\]
\end{lem}
\begin{proof}
It is implied by the soundness of the semantics (\thmref{thm:soundness-semantics}).
\end{proof}

\fi

%\subsection{Local/Transaction}

\begin{definition}[Logical Expressions]
\label{def:logical-expr}
Assume a countably infinite set of \emph{logical variables} $\V x \in \LVar$.
The set of \emph{logical expressions}, $ \lexpr \in \LExpr$ is defined by the following inductive grammar, where \(\val \in \Val\) (\defref{def:program_values}), \(\txvar \in \TxVars\) and \( \thvar \in \ThdVars \) (\defref{def:stacks}),
\[
\begin{rclarray}
   \lexpr & ::= & \val \mid \txvar \mid \thvar \mid \lvar \mid \lexpr + \lexpr \mid \lexpr \times \lexpr \mid \dots 
\end{rclarray}
\]
Assume a set of \emph{logical environments} \(\lenv \in \LEnv: \LVar \parfun \Val\) which associates logical variables with values.
Given a stack $\stk \in \Stacks$ (\defin\ref{def:stacks}) and a logical environment $\lenv \in \LEnv$, the \emph{logical expression evaluation} function, $\evalLE[(., .)]{.}:\LExpr \times \Stacks \times \LEnv\rightharpoonup \Val$, is defined inductively over the structure of logical expressions as follows,
%
\[
    \begin{rclarray}
        \evalLE{\val} & \defeq & \val \\
        \evalLE[\lenv, \thstk \uplus \txstk]{\thvar} & \defeq & \thstk(\thvar) \\
        \evalLE[\lenv, \thstk \uplus \txstk]{\txvar} & \defeq & \txstk(\txvar) \\
        \evalLE{\lvar} & \defeq & \lenv(\lvar) \\
        \evalLE{\lexpr_1 + \lexpr_2} & \defeq & \evalLE{\lexpr_1} + \evalLE{\lexpr_2} \\
        \evalLE{\lexpr_1 \times \lexpr_2} & \defeq & \evalLE{\lexpr_1} \times \evalLE{\lexpr_2} \\
        \dots & \defeq & \dots \\
    \end{rclarray}
\]
Note that the stack \( \stk \) includes transaction variables and thread variables.
\end{definition}

\emph{Fingerprint assertion} or \emph{fingerprint} is a set of tuples in the form of \( (\otag, \lexpr_{1}, \lexpr_{2}) \) where \( \otag \) is either read tag \( \etR \) or write \( \etW \) and the second and third elements are logical assertions representing the address and value respectively.
This assertion is interpreted to a set of transaction events as expected.

\begin{defn}[Fingerprint Assertions]
\label{def:fingerprint}
The \emph{fingerprint assertion} also \emph{fingerprint}, \( \fp \in \FAst \), is defined as the follows, 
\[
\begin{rclarray}
    \fp & \subseteq & \Setcon{ (\otag,\lexpr_{1},\lexpr_{2}) }{ \otag \in \OTags \land \lexpr_{1}, \lexpr_{2} \in \LExpr } \\
\end{rclarray}
\] 
Given a logical environment $\lenv \in \LEnv$ and a stack $\stk \in \Stacks$, the \emph{fingerprint interpretation} function, $\evalF[(., .)]{.}: \FAst \times \LEnv \times \Stacks \parfun \Opsets$, is defined as follows,
\[
\begin{rclarray}
    \evalF{\emptyset} & \defeq & \unitO  \\
    \evalF{\fp \addF (\otag, \lexpr_{1}, \lexpr_{2})} & \defeq & \evalF{\fp} \addO (\otag, \evalLE{\lexpr_{1}}, \evalLE{\lexpr_{2}})
\end{rclarray}
\]
\end{defn}

The local assertions includes normal separation logic assertions and extra fingerprint assertions, which are interpreted as sets of heaps and a set of events respectively.
Notice that the fingerprint assertion cannot be split.

\begin{definition}[Local assertions]
\label{def:local_assertions}
Given the set of logical expressions \( \LExpr \), logical variables \( \LVar \) and fingerprint assertion \( \FAst \), the set of \emph{local assertions}, $\lpre,  \lpost \in \LAst$, is defined inductively by the following grammar, 
\[
\begin{rclarray}
	\lpre, \lpost  & ::= & \False \mid \True \mid \lpre \land \lpost \mid \lpre \lor \lpost \mid \exsts{\lvar} \lpre \mid \lpre \implies \lpost \mid \Emp \mid \lexpr \pt \lexpr \mid \fpF \mid \lpre \sep \lpost  \\
\end{rclarray}	 
\]
Given a logical environment $\lenv \in \LEnv$, the \emph{local interpretation function}, $\evalLS[(.,.)]{.}: \LAst \times \LEnv \times \LAst \parfun \Heaps \times \powerset{ \Events } $, is defined over the structure of local assertions as follows,
\[
\begin{rclarray}
	\evalLS{\assfalse} & \eqdef & \emptyset \\
	\evalLS{\asstrue} & \defeq & \Heaps \times \powerset{ \Events } \\
	\evalLS{\lpre \land \lpost} & \defeq & \evalLS{\lpre} \cap \evalLS{\lpost} \\
	\evalLS{\lpre \lor \lpost} & \defeq & \evalLS{\lpre} \cup \evalLS{\lpost} \\
	\evalLS{\exsts{\lvar} \lpre} & \defeq & \bigcup\limits_{\val \in \textnormal{\Val}}\evalLS[\lenv\remapsto{\lvar}{\val}, \stk]{\lpre}  \\
	\evalLS{\lpre \implies \lpost} & \defeq & \Setcon{\h}{\h \in \evalLS{\lpre} \implies \h \in \evalLS{\lpost}}\\
	\evalLS{\assemp} & \defeq & \Set{ ( \unitH, \unitE) }  \\
	\evalLS{\lexpr_{1} \pt \lexpr_2 } & \defeq & \Set{ (\evalLE{\lexpr_1} \pt \evalLE{\lexpr_2}, \unitE) } \\
	\evalLS{ \fpF } & \defeq & \Set{ (\unitH, \evalF{\fp}) } \\
	\evalLS{\lpre \sep \lpost} & \defeq & 
    \Setcon{
        (\h_1 \composeH \h_2, \evset_{1} \composeE \evset_{2})
    }{ 
        (\h_{1},\evset_{1}) \in \evalLS{\lpre} 
        \land (\h_{2}, \evset_{2} ) \in \evalLS{\lpost} 
    } 
\end{rclarray}
\]
\end{definition}

Observe that program expressions $\Expr$  (\defin\ref{def:language}) are contained in logical expressions $\LExpr$ (\defin\ref{def:local_assertions} above), \ie $\Expr \subset \LExpr$. 
For readability, we will write angle brackets, \eg \( \fpass{(\etR, \vx, 0)} \) instead of curly brackets \( \fpto{\Set{(\etR, \vx, 0)}} \) for fingerprint assertions.

\subsection{Global/Program}

\begin{definition}[Capabilities]
\label{def:capabilities}
Assume a \emph{partial commutative monoid (PCM)} of \emph{client-specified capabilities} \( (\Kaps, \composeK, \unitK) \) with \( \kap \in \Kaps \), the composition \( \composeK \) the units set \( \unitK \).
Then given a set of \emph{region identifiers} \( \rid \in \RegionID \), the \emph{capability composition function} or \emph{capabilities} \( \ca \in \Caps \defeq \RegionID \parfun \Kaps \), where the composition \( \composeC \) is defined as the follows,
\[
    \begin{rclarray}
        (\ca_{l} \composeC \ca_{r})(\rid) & \defeq  &
        \begin{cases}
            \ca_{l}(\rid) \composeK \ca_{r}(\rid) & \rid \in \dom(\ca_{l}) \cap \dom(\ca_{l}) \\
            \ca_{l}(\rid)  & \rid \in \dom(\ca_{l}) \setminus \dom(\ca_{l}) \\
            \ca_{r}(\rid) & \rid \in \dom(\ca_{r}) \setminus \dom(\ca_{l}) \\
            \text{undefined} & \text{otherwise} \\
        \end{cases}
    \end{rclarray}
\]
, and the units set \( \unitC \defeq \Setcon{\ca}{\for{\rid} \ca(\rid) \in \unitK } \) .
\end{definition}

\begin{defn}[Interference]
\label{def:intf}
Given the fingerprint assertion \( \fp \in \Fingerprint \) (\defref{def:fingerprint}) and local assertion \( \lpre \in \LAst \) (\defref{def:local_assertions}), the grammar of \emph{interference assertions}, \( \intass \in \IAst \), is defined as the follows,
\[
\begin{rclarray}
	\intass & ::=  &
	\emptyset \mid \Set{ \perm{\kap} :  \exsts{\vec{\lvar}} \lpre \mat \fp } \cup \intass 
\end{rclarray}
\]
It will be interpreted to a set of \emph{interference environments}, this is,
\[
\begin{rclarray}
    \inter \in \Interference & \defeq & \Kaps \parfun \powerset{\Heaps} \times \Opsets
\end{rclarray}
\]
Given a logical environment $\lenv \in \LEnv$ and a stack $\stk \in \Stacks$, the \emph{interference interpretation} function, $\evalI[(., .)]{.}: \IAst \times \LEnv \times \Stacks \to \Interference$, is defined as follows,
%
\[
\begin{rclarray}
	\evalI{\emptyset}(\kap) & \eqdef & \text{undefined} \\
	\evalI{\Set{ \perm{\kap} : \exsts{\vec{\lvar}} \lpre \mat \fp } \cup \intass }(\kap') & \eqdef &
    \begin{cases}
    (\evalLS[\lenv',\stk]{\lpre}, \evalF[\lenv',\stk]{\fp}) \cup \evalI{\intass}(\kap')  & \kap = \kap' \\
    \evalI{\intass}(\kap') & \text{ otherwise} \\
    \end{cases} \\
    & & \text{where there exists a vector of values \( \vec{\val}\) such that } \lenv' = \lenv\rmto{\vec{\lvar}}{\vec{\val}} \\
\end{rclarray}
\] 
\end{defn}

\begin{defn}[Labelled transition system]
The labelled transition system is a tuple, \( (\aexecset,\actionset,\toLTS{}, \aexecset_{0}, \como) \), consisting of a set of abstract executions \( \aexecset \subseteq \Aexecs \), a set of actions \( \actionset \subseteq \Actions \), a relation \( \toLTS{} : \Aexecs \times \Actions \times \Aexecs \), a set of initial abstract executions \( \aexecset_{0}\) and the consistency model associated with the transition system \( \como \).
Assume all the initial abstract executions satisfies the consistency model.
The relation \( \toLTS{}\) is defined as the follows,
\[
\begin{rclarray}
    \aexec \toLTS{\evset} \aexec' & \defeq &
        \begin{array}[t]{@{}l}
        \exsts{\vis, \po, \ar, \txid } \\
        \quad {} \land \aexec' = (\aexec\prjT \uplus \Set{ \txid \mapsto \evset }, \aexec\prjP \uplus \po, \aexec\prjV \uplus \vis, \aexec\prjA \uplus \ar) \\
        \quad {} \land \vis, \po \subseteq \ar = \Setcon{(\txid', \txid)}{\txid' \in \dom(\aexec\prjT)} 
        \land \aexec' \in \evalCOM{\como}
    \end{array}
\end{rclarray}
\]
\end{defn}
 
\begin{defn}[Invariant a region]
\label{def:invariant-region}
\label{def:world2aexec}
\label{def:state2aexec}
Assume two global functions, \( \funcn{init} : \RegionID \to \powerset{\Aexecs} \) that returns initial abstract executions for regions, and \( \funcn{como} : \RegionID \to \ConsisModels \) that returns the consistency models associated with regions.
Also assume all the initial states for a region satisfy the consistency model, \ie
\[
\for{\rid, \aexec_{0}} \aexec_{0} \in \func{init}{\rid} \implies \aexec_{0} \in \evalCOM{\func{como}{\rid}}
\]
The invariant of a region, namely \( \func{transinv}{\rid, \intf} \), is the labelled transition system where the initial state is \( \func{init}{\rid}\) and all the actions are included in the interference.
\[
\begin{rclarray}
    \func{inv}{\rid, \intf} & \defeq & (\aexecset,\actionset \cup \Set{\unitE},\toLTS{}, \func{init}{\rid}, \func{como}{\rid}) \\
    & & \text{where } \for{\evset} \evset \in \actionset \implies \exsts{\kap} \evset \in \dom(\intf(\kap))
\end{rclarray}
\]
For brevity, \( \aexec \in \func{inv}{\rid, \intf} \) is short-hand for \( \aexec \in \aexecset \), and similarly \( \aexec \toLTS{\evset} \aexec' \in \func{inv}{\rid, \intf} \).
\end{defn}

The empty (unit) event \( \unitE \) in the invariant is a place-holder for other regions.
They might append a concrete event, while the composition of abstract executions composes events point-wise if the two executions have the same structure.

\begin{defn}[Well-form of a region]
\label{def:well-form-region}
The well-form condition of the interference, namely \( \pred{wfintf}{\rid, \intf} \) predicate, assertions for any concrete events \( \evset \), the state before the events must be included in the interference.
\[
\begin{rclarray}
    \pred{wfintf}{\rid, \intf} & \defeq & \for{\aexec, \aexec', \evset} \aexec \toLTS{\evset} \aexec' \in \func{inv}{\rid, \intf} \land ( \evset \neq \unitE \implies \pred{wfabs}{\rid, \intf, \aexec, \evset, \aexec'} ) \\ 
    \pred{wfabs}{\rid, \intf, \aexec, \evset, \aexec'} & \defeq & 
    \begin{array}[t]{@{}l}
        \exsts{ \kap }
        \evset \in ( \dom(\intf( \kap )) ) \land \obsstate{\aexec, \aexec\prjT,\func{como}{\rid}\projection{2}} \subseteq \intf(\kap)(\evset)
    \end{array} \\
\end{rclarray}
\]
Given a region (identifier) \(\rid\), its current state \( \h \), and its interference \( \intf \), the function \(\funcn{r2e} \) returns all the possible abstract executions that are included in the invariant of a region and satisfy the state \( \h \),
\[
\begin{rclarray}
    \func{r2e}{\rid, \h, \intf} & \eqdef & \Setcon{\aexec}{\aexec \in \func{inv}{\rid, \intf} \land \h \in \obsstate{\aexec,\aexec\prjT,\func{como}{\rid}\projection{2}}} \\
\end{rclarray}
\]
A set of heaps \( \hset \) approximates the observation of a region \( \rid \) under state \( \h \), namely \( \pred{approx}{\rid, \h, \hset, \intf} \), when an abstract execution satisfies the state \( \h \) and it can reach a new state by appending a new transaction \( \txid \), the observable state of the new transaction must included in the approximation \( \hset \),
\[
\begin{rclarray}
    \pred{approx}{\rid, \aexec, \hset, \intf} & \eqdef & 
    \begin{array}[t]{@{}l}
    \for{ \aexec' } \aexec \toLTS{\stub} \aexec' \in \func{inv}{\rid, \intf} \\
    \quad {} \land \exsts{ \txid } \txid = \max_{\ar}\Set{\aexec'\prjT}
    \land \obsstate{\aexec,\aexec\prjV^{-1}(\txid),\func{como}{\rid}\projection{2}} \subseteq \hset
    \end{array}
\end{rclarray}
\]
\end{defn}

\begin{definition}[Worlds]
\label{def:world}
Given the set of heaps $\Heaps$ (\defref{def:heaps}) and a set of \emph{region identifiers} \( \rid \in \RegionID \), the set of \emph{shared states} is \( \SStates \eqdef \RegionID \to \Heaps \times \powerset{\Heaps} \times \Interference \).
Each region has its current state, a set of possible initial states for transitions and the interference.
The \emph{shared state composition function}, $\composeS: \SStates \times \SStates \parfun \SStates$, is defined as $\composeS \eqdef \composeEq$, where for all domains $\sort M$ and all $m, m' \in \sort M$,
%
\[
\begin{rclarray}
	m \composeEq m' &  \eqdef  &
	\begin{cases}
		m & \text{if } m = m'\\
		\text{undefined} & \text{otherwise}
	\end{cases}
\end{rclarray}
\]
A \emph{world} \( \w \in \World \) is a pair of a shared state \( \gs \) and capabilities \( \ca \) (\defref{def:capabilities}), where regions are associated with the same consistency model and the collapse of the pair exists, \ie regions are well-form and compatible.
\[
\begin{rclarray}
	\world \in \World  & \eqdef & 
    \Setcon{
        (\ca, \gs) 
    }{ 
        \ca \in \Caps 
        \land \gs \in \SStates
        \land \clpsW{\gs} \neq \emptyset
        \land \dom(\ca) \subseteq \dom(\gs) \\
        \quad {} \land \for{\rid, \rid'}
        \func{como}{\rid} = \func{como}{\rid'} \\
        \quad {} \land \for{\h, \h' }
        \h \in \Set{\gs(\rid)\projection{1}} \cup \gs(\rid)\projection{2}
        \land \h' \in \Set{\gs(\rid')\projection{1}} \cup \gs(\rid')\projection{2} 
        \land ( \h \composeH \h' )\isdef
    }
\end{rclarray}
\]
The function, \( \clpsW{.} : \SStates \parfun \powerset{\Aexecs} \), collapses shared states to sets of abstract executions as the follows,
\[
\begin{rclarray}
    \clpsW{\emptyset} & \defeq & \unitAEX \\
    \clpsW{\Set{\rid \mapsto (\h, \hset, \intf)} \uplus \gs } & \defeq & 
        \Setcon{ \aexec \composeAEX \aexec' }{ \aexec \in \func{r2e}{\rid, \h, \intf} \land \pred{approx}{\rid, \aexec, \hset, \intf} \land \pred{wfintf}{\rid, \intf} \land \aexec' \in \clpsW{\gs} }\\
\end{rclarray}
\] 
% 
The \emph{world composition function}, $\composeW: \World \times \World \parfun \World$, is defined component-wise as: $\composeW \eqdef (\composeC, \composeS)$.
The \emph{world unit set} is $\unitW \eqdef \Setcon{(\ca, \gs)}{(\ca, \gs) \in \World \land \ca \in \unitC}$.
The \emph{partial commutative monoid of worlds} is $(\World, \composeW, \unitW)$.
\end{definition}

\sx{point-wise composition}
\begin{defn}[Invariant of worlds]
Because regions in a well-defined world must disjointed with each other and have the same consistency model, it is easy to lift the invariant of a region to a shared state,
\[
\begin{rclarray}
    \func{inv}{\emptyset} & \defeq & (\aexecset, ... ) \\
    \func{inv}{\Set{\rid \mapsto (\stub, \stub, \intf)} \uplus \gs} & \defeq & \Setcon{\aexec \composeAEX \aexec' }{\aexec \in \func{inv}{\rid, \intf} \land \aexec' \in \func{inv}{\gs}} \\
    \func{transinv}{\emptyset} & = & \Setcon{ ( \aexec , \unitE, \aexec' ) }{\aexec, \aexec' \in \unitAEX } \\
    \func{transinv}{\Set{\rid \mapsto (\stub, \stub, \intf)} \uplus \gs} & = & 
    \Setcon{
        ( \aexec \composeAEX \aexec_{f}, \evset \composeE \evset_{f}, \aexec' \composeAEX \aexec_{f}' ) 
    }{
        (\aexec, \evset, \aexec') \in \func{transinv}{\rid, \intf} \\
        \quad \land (\aexec_{f}, \evset_{f}, \aexec_{f}') \in \func{transinv}{\gs}
    }
\end{rclarray}
\]
\end{defn}

\begin{definition}[Assertions]
\label{def:assertion}
Assume standard separation logic assertion \( \bar{\lpre}, \bar{\lpost }\) (the local assertion \( \LAst \) without fingerprint) and the interpretation function, The set of \emph{assertions}, $\gpre, \gpost \in \Ast$, are defined by the following inductive grammar:
\[
\begin{rclarray}
	\gpre , \gpost & \defeq & \False \mid \True \mid \gpre \land \gpost \mid \gpre \lor \gpost \mid \exsts{\lvar}\gpre \mid \gpre \implies \gpost \mid \assemp \mid \cass{\kap}{\lrid} \mid \gpre \sep \gpost \mid \sptboxass{\bar{\lpre}}{\bar{\lpost}}{\lrid}{\intass}\\
\end{rclarray}
\]
%
where $\lvar, \lrid \in \LVar$, $\lexpr_1, \lexpr_2 \in \LExpr$ (\defin\ref{def:local_assertions}), $\kap \in \Kaps$ (\defin\ref{def:capabilities}) and $\intass \in \IAst$ (\defin\ref{def:intf}).
Given a logical environment $\lenv \in \LEnv$ and a stack $\stk \in \Stacks$, the \emph{assertion interpretation} function, $\evalW[(., .)]{.}: \Ast \times \LEnv \times \Stacks \to \powerset{\World}$, is defined as follows:
%
\[
\begin{rclarray}
	\evalW{\False} & \defeq & \emptyset \\
	\evalW{\True} & \defeq & \World \\
	\evalW{\emp} & \defeq & \unitW \\
	\evalW{\gpre \land \gpost} & \defeq & 
    \Setcon{
        (\ca, \gs)
    }{
        \exsts{\gs_{p}, \gs_{q}} 
        (\ca, \gs_{p}) \in \evalW{\gpre} 
        \land (\ca, \gs_{q}) \in \evalW{\gpost} \\
        \quad {} \land \for{\rid} 
        \exsts{\h, \hset_{p}, \hset_{q}, \intf} 
        \gs(\rid) = (\h, \hset_{p} \cap \hset_{q}, \intf) \\
        \qquad {} \land \gs_{p}(\rid) = (\h, \hset_{p}, \intf)
        \land \gs_{q}(\rid) = (\h, \hset_{q}, \intf)
    } \\
	\evalW{\gpre \lor \gpost} & \defeq & 
    \Setcon{
        (\ca, \gs)
    }{
        \exsts{\gs_{p}, \gs_{q}} 
        (\ca, \gs_{p}) \in \evalW{\gpre} 
        \land (\ca, \gs_{q}) \in \evalW{\gpost} \\
        \quad {} \land \for{\rid} 
        \exsts{\h, \hset_{p}, \hset_{q}, \intf} 
        \gs(\rid) = (\h, \hset_{p} \cup \hset_{q}, \intf) \\
        \qquad {} \land \gs_{p}(\rid) = (\h, \hset_{p}, \intf)
        \land \gs_{q}(\rid) = (\h, \hset_{q}, \intf)
    } \\
	\evalW{\exsts{\lvar}  \gpre} & \defeq & \bigcup\limits_{\val \in \textnormal{\Val}} \evalW[\lenv\remapsto{\lvar}{\val}, \stk]{\gpre} \\
	\evalW{\gpre \implies \gpost} & \defeq & \Setcon{\w}{\w \in \evalW{\gpre} \implies \w \in \evalW{\gpost}} \\
	\evalW{\cass{\kap}{\lrid}} & \defeq & \Setcon{ (\Set{\lrid \mapsto \kap}, \gs) }{\gs \in \SStates} \\
	\evalW{ \gpre \sep \gpost } & \defeq & 
	\Setcon{
	   (\world_1 \composeW \world_2) 
    }{
       \world_1 \in \evalW{\gpre} \land \world_2 \in \evalW{\gpost}
	} \\
	\evalW{ \sptboxass{\bar{\lpre}}{\bar{\lpost}}{\lrid}{\intass} } & \defeq & 
    \Setcon{
        (\ca,\Set{\lrid \mapsto (\h, \hset, \intf)} \uplus \gs)
    }{
        \ca \in \unitC 
        \land \h \in \evalLS{\bar{\lpre}}
        \land \hset = \evalLS{\bar{\lpost}}
        \land \intf  = \evalI{\intass}
    } \\
\end{rclarray}
\]
\end{definition}

We will write \( \boxass{\bar{\lpre}}{\lrid}{\intass} \) as a short-hand for \( \sptboxass{\bar{\lpre}}{\bar{\lpre}}{\lrid}{\intass} \) and \(\expr \pt N\) for \( \exsts{\nat \in N} \expr \pt \nat\) where \( N \subseteq \Val\).


%\subsection{Rules for Local}

The proof rules are standard except \rl{TRDeref} and \rl{TRMutate}.
The \rl{TRDeref} rule add read fingerprint in finger-tracking set, only if there is no write finger-print.
This is because once a location has been re-written, the rest read are considered as local operations, while the finger-print only records those operations might have effect on global state.

\begin{figure}[t]
\hrule\vspace{5pt}
\sx{Need to be careful about the implication, esp. for fingerprint assertion.}
\[
    \infer[\rl{TRSkip}]{%
        \tripleL{\assemp }{ \pskip }{\assemp }
    }{}
\]

\[
    \infer[\rl{TRAss}]{%
        \tripleL{\txvar \dot= \lexpr }{ \passign{\txvar}{\expr} }{\txvar \dot= \expr\sub{\txvar}{\lexpr} }
    }{%
    \txvar \notin \func{fv}[\lexpr]
        && \txvar \in \TxVars  
    }
\]

\[
    \infer[\rl{TRDeref}]{%
        \tripleL{\expr \pt \lexpr \sep \fpF }{ \pderef{\txvar}{\expr} }{\txvar \dot= \lexpr \sep \expr \pt \lexpr \sep \fpto{\assfp'} }
    }{%
    \txvar \notin \func{fv}[\expr]
        && \txvar \notin \func{fv}[\lexpr]
        && \txvar \in \TxVars  
        && \assfp' = \assfp \addF (\otR, \expr, \lexpr)
    }
\]

\[
    \infer[\rl{TRMutate}]{%
        \tripleL{\expr_1 \pt \stub \sep \fpF }{ \pmutate{\expr_1}{\expr_2} }{ \expr_1 \pt \expr_2 \sep \fpto{\assfp'} } 
    }{
        \assfp' = \assfp \addF (\otW, \expr_{1}, \expr_{2})
    }
\]

\[
    \infer[\rl{TRAssume}]{%
        \tripleL{ \expr \doteq 0 }{ \passume{\expr} }{ \expr \doteq 0 } 
    }{}
\]

\[
    \infer[\rl{TRChoice}]{%
        \tripleL{ \lpre }{ \trans_{1} \pchoice \trans_{2} }{ \lpost }
    }{%
        \tripleL{ \lpre }{ \trans_{1} }{ \lpost } && 
        \tripleL{ \lpre }{ \trans_{2} }{ \lpost } 
    }
\]

\[
    \infer[\rl{TRSeq}]{%
        \tripleL{ \lpre }{ \trans_{1} \pseq \trans_{2} }{ \lpost }
    }{%
        \tripleL{ \lpre }{ \trans_{1} }{ \lframe }  && 
        \tripleL{ \lframe }{ \trans_{2} }{ \lpost }
    }
\]

\sx{check the loop invariant with fingerprint}

\[
    \infer[\rl{TRLoop}]{%
        \tripleL{ \lpre }{ \trans\prepeat }{ \lpre }
    }{%
        \tripleL{ \lpre }{ \trans }{ \lpre } 
    }
\]
 
\[
   \infer[\rl{TRFrame}]{%
       \tripleL{ \lpre \sep \lframe }{ \trans }{ \lpost \sep \lframe }
   }{%
       \tripleL{ \lpre }{ \trans }{ \lpost } 
   }
\]
\hrule\vspace{5pt}
\caption{The rules for transactions}
\label{fig:rule-trans}
 \end{figure}

\subsection{Rely and Guarantee}

\begin{definition}[Rely and guarantee]
\label{def:rely-guarantee}
The \( \func{allowed} \) function asserts that a event is allowed by the owned capabilities,
\[
\begin{rclarray}
    \func{allowed}[\fp, \ca, \gs] & \defeq & \fp = \unitE \\
    \func{allowed}[\fp \composeE \fp', \ca, \gs \uplus \Set{\rid \mapsto (\stub, \stub, \intf)}] & \defeq & 
    \exsts{\kap } \kap \sqsubseteq \ca(\rid)
    \land \fp \in \dom(\intf(\kap)) 
    \land \func{allowed}[\fp', \ca, \gs]
\end{rclarray}
\]
Given the set of worlds $\World$ (\defref{def:world}), the \emph{update rely} relation, $\relyU \subseteq \World \times \World$, is defined as follows,
\[	
    \begin{rclarray}
	\relyU & \defeq &
	\myset{
		((\ca, \gs), (\ca, \gs'))	
	}{
        \exsts{\fp,\aexec, \aexec'}  
        \func{allowed}[\fp, \ca, \gs]  \\
        \quad {} \land \aexec \in \clpsW{\gs} 
        \land \aexec' \in \clpsW{\gs'}
        \land (\aexec, \fp, \aexec') \in \func{transinv}[\gs]
	} \\
    \end{rclarray}
\]
The invariant of a shared state is a lift of the invariants of interferences of regions.
The \emph{rely} relation, $\RelyI \defeq \World \times \World$, is defined as follows:
\[
    \begin{rclarray}
         \RelyI &\defeq & \closure{\left(\relyU\right)} \\
    \end{rclarray}
\]
A set of fingerprint worlds $\setworld \subseteq \World$ is \emph{stable}, written $\stable{\setworld}$, if and only if it is closed under the rely relation: 
\[
    \begin{rclarray}
        \stable{\setworld} & \defeq & \fora{\w_{p}, \w_{q}}  \w_{p} \in \setworld \land (\w_{p}, \w_{q}) \in \RelyI \implies \w_{q} \in \setworld
    \end{rclarray}
\]
The \emph{update guarantee} relation, $\guarU: \World \times \World$, is defined as follows:
\[	
    \begin{rclarray}
	\guarU & \defeq &
	\myset{
		((\ca, \gs), (\ca, \gs'))	
	}{
        \exsts{\fp, \ca', \aexec, \aexec'}  
        (\ca' \composeC \ca)\isdef
        \land \func{allowed}[\fp, \ca, \gs]  \\
        \quad {} \land \aexec \in \clpsW{\gs} 
        \land \aexec' \in \clpsW{\gs'}
        \land (\aexec, \fp, \aexec') \in \func{transinv}[\gs]
	} \\
    \end{rclarray}
\]
The \emph{guarantee} relation, $\GuarI \subseteq \World \times \World$, is defined as follows:
\sx{take away of the closure}
\[
	\GuarI \defeq \guarU
\]
\end{definition}

\subsection{Rules for Global}

The \rl{PRCommit} rule lifts the local effect of transaction \( \trans \) to global level by first converting global state to (local) observable state and then propagating the local fingerprint to the global state.
The \( \pred{down} \) predicate asserts that the local predicate \( \lpre \) is a over-approximation of the valid observation that is given by the interference.
The \( \pred{up} \) predicate says the post-condition \( \gpost \) is the result by lifting the local fingerprints \( \assfp \) to pre-condition \( \gpre \).



\begin{figure}[t!]
\hrule\vspace{5pt}

\sx{not right, we want to say a transaction \( \trans \) can be abstracted to a fingerprint \( \assfp \)}

\[
    \infer[\rl{PRCommit}]{%
        \tripleG{\gpre}{ \ptrans{\trans} }{\gpost}
    }{%
        \begin{array}{c}
        \gpre \snap \lpre
        \quad \tripleL{\lpre \sep \fpass{}}{\trans}{\lpost \sep \fpF}
        \quad \rpt{\gpre}{\gpost}{\assfp} \\
        \stable{\gpre} 
        \quad \stable{\gpost} 
        \end{array}
    }
\]

\[
    \infer[\rl{PRAss}]{%
        \tripleG{\var \dot= \lexpr }{ \passign{\var}{\expr} }{\var \dot= \expr\sub{\var}{\lexpr} }
    }{%
        \var \notin \func{fv}[\lexpr]
        && \var \in \ThdVars  
    }
\]

\[
    \infer[\rl{PRAssume}]{%
        \tripleG{ \expr \doteq 0 }{ \passume{\expr} }{ \expr \doteq 0 } 
    }{}
\]

\[
    \infer[\rl{PRChoice}]{%
        \tripleG{ \gpre }{ \prog_{1} \pchoice \prog_{2} }{ \gpost }
    }{%
        \tripleG{ \gpre }{ \prog_{1} }{ \gpost } && 
        \tripleG{ \gpre }{ \prog_{2} }{ \gpost } 
    }
\]

\[
    \infer[\rl{TRSeq}]{%
        \tripleG{ \gpre }{ \prog_{1} \pseq \prog_{2} }{ \gpost }
    }{%
        \tripleG{ \gpre }{ \prog_{1} }{ \gframe }  && 
        \tripleG{ \gframe }{ \prog_{2} }{ \gpost }
    }
\]

\[
    \infer[\rl{TRLoop}]{%
        \tripleG{ \gpre }{ \prog\prepeat }{ \gpre }
    }{%
        \tripleG{ \gpre }{ \prog }{ \gpre } 
    }
\]
 
\[
   \infer[\rl{TRFrame}]{%
       \tripleG{ \gpre \sep \gframe }{ \prog }{ \gpost \sep \gframe }
   }{%
       \tripleG{ \gpre }{ \prog }{ \gpost } 
   }
\]
 
\[
   \infer[\rl{TRPar}]{%
       \tripleG{ \gpre_{1} \sep \gpre_{2} }{ \prog_{1} \ppar \prog_{2} }{ \gpost_{1} \sep \gpost_{2} }
   }{%
       \tripleG{ \gpre_{1} }{ \prog_{1} }{ \gpost_{1} }
       && \tripleG{ \gpre_{2} }{ \prog_{2} }{ \gpost_{2} }
   }
\]

\[
\begin{rclarray}
    \gpre \snap \lpre & \defeq & \fora{ \w, \sn } \w \in \evalW{\gpre} \land \pred{takeinv}{\w, \sn} \implies \sn \in \evalLS{\lpre}\\
    \pred{takeinv}{\gs, \sn} & \defeq & \gs = \emptyset \land \sn = \unitH \\
    \pred{takeinv}{\Set{\rid \mapsto (\stub, \snset, \intf)} \uplus \gs, \sn \composeH \sn'} & \defeq & \sn \in \snset \land \pred{takeinv}{\gs,\sn'}\\
    \rpt{\gpre}{\gpost}{\assfp} & \defeq & 
    \begin{array}[t]{@{}l@{}}
        \fora{\w, \w', \aexec, \aexec'} \\
        \quad \w \in \evalW{\gpre}
        \land \aexec \in \clpsW{\w}
        \land \func{reachable}{\aexec, \aexec', \stub, \evalF{\assfp}, \ET}  \\
        \quad {} \land \aexec' \in \clpsW{\w'}
        \land (\w, \w') \in \Guar 
        \implies \w' \in \evalW{\gpost}
    \end{array} \\
\end{rclarray}                          
\]

\sx{
    For LLW the propagate can be done by syntactically update those addresses that has been written.
}

\hrule\vspace{5pt}
\caption{The rules for programs}
\label{fig:rule-prog}
\end{figure}

Many consistency model use last write win resolution policy, such as snapshot isolation, therefore the repartition \( \rpt{\gpre}{\gpost}{\assfp} \) can be simplified by checking the guarantee and then syntactically propagating the write fingerprints.
Also in practice, many implementation of consistency model assume strong session constraint.

\begin{figure}
\hrule\vspace{5pt}

\[
   \infer[\rl{FRead}]{%
       \tripleF{ \lexpr \pt \lexpr' \mid \lexpr \pt \lexpr'' }{ \Set{(\otR, \lexpr, \lexpr'')} }{ \lexpr \pt \lexpr' \mid \lexpr \pt \lexpr''}
   }{}
\]

\[
   \infer[\rl{FWriteNS}]{%
       \tripleF{ \lexpr \pt \lexpr' \mid \lexpr \pt \lexpr'' }{ \Set{(\otW, \lexpr, \lexpr''')} }{ \lexpr \pt \lexpr''' \mid ( \lexpr \pt \lexpr'' \lor \lexpr \pt \lexpr''') }
   }{}
\]

\[
   \infer[\rl{FWriteS}]{%
       \tripleF{ \lexpr \pt \lexpr' \mid \lexpr \pt \lexpr'' }{ \Set{(\otW, \lexpr, \lexpr''')} }{ \lexpr \pt \lexpr''' \mid \lexpr \pt \lexpr'''}
   }{}
\]

\[
   \infer[\rl{FFrame}]{%
       \tripleF{ \lpre \sep \lframe  \mid \lpre' \sep \lframe' }{ \assfp }{ \lpost \sep \lframe \mid \lpost' \sep \lframe' }
   }{%
       \tripleF{ \lpre \mid \lpre' }{ \assfp }{ \lpost \mid \lpost' }
   }
\]

\[
   \infer[\rl{FContinue}]{%
       \tripleF{ \lpre \sep \lframe  \mid \lpre' \sep \lframe' }{ \assfp  \uplus \assfp' }{ \lpost \sep \lframe \mid \lpost' \sep \lframe' }
   }{%
       \tripleF{ \lpre \sep \lframe  \mid \lpre' \sep \lframe' }{ \assfp }{ \lpost \sep \lframe \mid \lpost' \sep \lframe' }
   }
\]

\[
\begin{rclarray}
    \rpt{\gpre}{\gpost}{\assfp} & \defeq & 
    \begin{array}[t]{@{}l}
    \fora{\w, \w', \sn, \sn', \lpre, \lpre', \lpost, \lpost'}
    \w \in \evalW{\gpre} \\
    \quad {} \land \pred{unbox}{\w\projection{2}, \sn}
    \land \sn \in \evalLS{\lpre} 
    \land \gpre \snap \lpre' \\
    \quad {} \land {} \tripleF{ \lpre \mid \lpre' }{ \assfp }{ \lpost | \lpost'} \\
    \quad {} \land \pred{unbox}{\w'\projection{2},\sn'}
    \land \sn' \in \evalLS{\lpost} 
    \land \gpost \snap \lpost' \\
    \quad \implies \w' \in \evalW{\gpost}
    \end{array} \\
    \pred{unbox}{\gs, \sn} & \defeq & \gs = \emptyset \land \sn = \unitH \\
    \pred{unbox}{\Set{\rid \mapsto (\sn, \stub, \stub)} \uplus \gs, \sn \composeH \sn' } & \defeq & \pred{unbox}{\gs, \sn'} \\
\end{rclarray}
\]

\hrule\vspace{5pt}
\caption{Simplified repartition for last write win}
\label{fig:rule-prog}
\end{figure}



\subsection{Example of Consistency models}
\ac{This Section is going to become heavy in pictures, which should be organised into figures.}
In this Section we present different consistency models specifications. 
For each of them, we give: 
\begin{itemize}
\item the intuition of the commit tests for different consistency models, and the formal definitions with respect to the \(\Como\) (\defref{def:consistency-models}).
%describing the consistency guarantees that schedules of the database should have in plain English, 
%\item a formal consistency model specification, in the style described in \S \ref{sec:semantics.programs},
\item examples of litmus tests that, when executed, give rise to the anomalies that are forbidden from the consistency model, 
\item an explanation of why the consistency model forbids the litmus tests to exhibit the anomaly that should be forbidden. 
\end{itemize}
Later, we will show how to compare our consistency models specifications with those already existing in the 
literature.
\ac{There is still a long-way to go before proving correspondence with dependency graph specifications, 
but this should be mentioned here.}


\subsection{Read Atomic} 
Read atomic (RA) \cite{ramp} is the weakest consistency model among those that enjoy \emph{atomic visibility} \cite{framework-concur}. 
It requires of a transaction to read an atomic snapshot of the database and never observe the partial effects of other transactions.
This is also known as the \emph{all-or-nothing} property: a transaction observes either none or all the updates performed by other transactions. 

\sx{example RAMP, not sure the meaning}
One litmus test that should \textbf{not} be failed in RAMP consists of the program $\prog_1$ from \S \ref{sec:semantics.example}, which we already observed to produce a violation of atomic visibility if no constraints on the consistency model are placed.
\ac{not be failed. Double negation. Bad English.}

\sx{Rephrase}
Intuitively, in such a program, it violate the atomic visibility because we allowed to execute the transaction \( \trans_1^2\) in the thread-local configuration of $\mathcal{C}'$ relative to $\thid_2$, which is obtained by removing all the information about $\thid_1$ (view and stack) in Figure \ref{fig:opsem.example}(c).
\[
\trans_1^2 = \begin{array}{c} 
            \begin{transaction}
            		\pderef{\pvar{a}}{\vx};\\
            		\pderef{\pvar{b}}{\vy};\\
            		\pifs{\pvar{a}=1 \wedge \pvar{b}=0}\\
            			\quad \passign{\retvar}{\Large \frownie{}} \\
                    \pife
             \end{transaction}
     \end{array}
\]

\ac{
To avoid transactions to only observe the partial effects of other transactions, we 
must ensure that transactional code cannot be executed by a thread whose 
views is up-to-date with respect to some transaction $\tsid$ for some location $[\loc_x]$, 
but not for some other location $[\loc_y]$. This leads to the following definition.
}
To avoid a transaction to observe the partial effects of other transactions, we need to ensure that transactional code cannot be executed by a thread whose views is partially up-to-date with respect to some transactions. This leads to the following definition.
\begin{defn}
\label{def:readatomic}
%Let $\hh$ be a history heap,$V$ be a view, $[\loc_x]$ be 
%a location and $\nu$ be a version. We say that $V$ $[\loc_x]$-\emph{sees} version 
%$\nu$ if there exists an index $i \leq V([\loc_x])$ such that $V(i) = \nu$. 
%We say that $V$ $[\loc_x]$-\emph{sees} transaction $\tsid$ if 
%$V$ $[\loc_x]$-sees a version $\nu = (\_, \tsid, \_)$. 
Given a view $\vi \in \Views$, a history heap $\hh \in \HisHeaps$, and a transaction identifier $\txid \in \TxID$, the view \emph{sees} the transaction in the history heap, written $\pred{visible}{\txid, \vi, \hh}$, if the view sees all the writes from the transaction,
%We say that $V$ \emph{sees} transaction $\tsid$ in $\hh$, written 
%$\mathsf{Visible}(\tsid, V, \hh)$, iff 
\sx{\( \exsts{i} \) might be enough}
\[
\begin{rclarray}
\pred{visible}{\txid, \vi, \hh} & \eqdef & \fora{\addr, i} \hh(\addr)(i) = (\stub, \txid, \stub) \implies i \leq \vi(\addr).
\end{rclarray}
\]
\ac{In English: the view is up-to-date with respect to all the updates 
performed by transaction $\tsid$.}

Then given a history heap \( \hh \), the view $V$ is \emph{consistent} with respect to \emph{atomic visibility}, written $\pred{atomic}{\vi, \hh}$, if the view $V$ is up-to-date with some of the updates performed by $\txid$, then it should be up-to-date with all the updates performed by $\txid$,
\[
\begin{rclarray}
\pred{atomic}{\vi ,\hh} & \eqdef & \fora{\txid } \exsts{\addr, i} i \leq \vi(\addr) \land \hh(\addr)(i) = (\stub, \txid, \stub) \implies \pred{vusible}{\txid, \vi, \hh}
\end{rclarray}
\]
\ac{In English: if the view $V$ is up-to-date with some of the updates performed 
by $\tsid$, then it must be up-to-date with all the updates performed by $\tsid$. 
This is the all-or-nothing property.}
%for all location 
%$[\loc_x]$, if there exists an index $i = 0,\cdots, \lvert \hh([\loc_x]) \rvert - 1$, 
%such that $\hh([\loc_x])(i) = (\_, \tsid, \_)$, then $i \leq V([\loc_x])$.

The consistency model specification $\mathsf{RA}$ is defined as the smallest set such that  
\sx{what is the meaning of smallest?}
\[
\pred{atomic}{\hh, \vi} \implies (\hh, \vi) \csat[\mathsf{RA}] \stub: \stub
\]
\ac{In English: Before executing a transaction, either you observe all or none the 
updates of all other transactions. We may strengthen the consistency model and 
require that the same property must be satisfied at the end as well, though 
this is not strictly necessary. In this case the check becomes: 
\[
\mathsf{atomic}(\hh, V) \wedge \mathsf{atomic}(\hh, V') \wedge \mathsf{UpdateView}(\hh, V, \mathcal{O}) 
\sqsubseteq V' \implies (\hh, V) \triangleright_{\mathsf{RA}} \mathcal{O}: V'.
\]
}
%written $\mathsf{up-to-date}(\hh, V, \tsid, [\loc_x])$, 
%if either 
%
%\begin{itemize}
%\item for all indexes $i = 0,\cdots, \lvert \hh([\loc_x]) - 1 \rvert$, 
%$\WS(\hh([\loc_x])(i)) \neq \tsid)$, or 
%\item if $\WS(\hh([\loc_x])(i)) = \tsid$ for some $i = 0,\cdots, \lvert \hh([\loc_n]) -1 \rvert$, 
%then $i \leq V([\loc_n])$.
%\end{itemize}
\end{defn}

\sx{Not sure how to link the explanation from Andrea's document, sort out later }
Suppose that we execute the program $\prog_1$ under the consistency model specification $\mathsf{RA}$.
We can proceed as in Section \ref{sec:semantics.example} to infer the transition 
$\langle \mathcal{C_0}, \prog_1 \rangle \xrightarrow{\mathsf{RA}} \langle \mathcal{C}_1, \prog_1' \rangle$, 
where we recall that $\mathcal{C}_0$, $\mathcal{C}_1$ are depicted in Figure \ref{fig:opsem.exampe}(a), 
\ref{fig:opsem.example}(b), respectively. 

It is immediate to observe that the only way in which the execution of transaction $\ptrans{\trans}$ from $\thid_2$ in $\prog_1'$ can return value ${\Large \frownie}$ is the following: 
\begin{itemize}
\item first, push the view $V$ of thread $\txid_2$ in the configuration 
$\mathcal{C}_1$ of Figure \ref{fig:opsem.example}(b) to observe the update of location $[\vx]$, but not the update of 
$[\vy]$. This view is the one labelled with $\txid_2$ in Figure \ref{fig:opsem.example}(c), and we refer 
to it as $V'$;
\item then, execute the transaction $\ptrans{\trans}$ in $\thid_2$. 
\end{itemize}

\subsection{Causal Consistency}
%\begin{figure}
%\begin{tabular}{|c|c|}
%\hline
%\begin{tikzpicture}[font=\large]

%\begin{pgfonlayer}{foreground}
%%Uncomment line below for help lines
%%\draw[help lines] grid(5,4);

%%Location x
%\node(locx) at (1,3) {$[\loc_x] \mapsto$};

%\matrix(locxcells) [version list, text width=7mm, anchor=west]
   %at ([xshift=10pt]locx.east) {
 %{a} & $T_0$ \\
  %{a} & $\emptyset$ \\
%};
%\node[version node, fit=(locxcells-1-1) (locxcells-2-1), fill=white, inner sep= 0cm, font=\Large] (locx-v0) {$0$};

%%Location y
%\path (locx.south) + (0,-1.5) node (locy) {$[\loc_y] \mapsto$};
%\matrix(locycells) [version list, text width=7mm, anchor=west]
   %at ([xshift=10pt]locy.east) {
 %{a} & $T_0$ \\
  %{a} & $\emptyset$ \\
%};
%\node[version node, fit=(locycells-1-1) (locycells-2-1), fill=white, inner sep= 0cm, font=\Large] (locy-v0) {$0$};

%% \draw[-, red, very thick, rounded corners] ([xshift=-5pt, yshift=5pt]locx-v1.north east) |- 
%%  ($([xshift=-5pt,yshift=-5pt]locx-v1.south east)!.5!([xshift=-5pt, yshift=5pt]locy-v0.north east)$) -| ([xshift=-5pt, yshift=5pt]locy-v0.south east);

%%blue view - I should  check whether I can use pgfkeys to just declare the list of locations, and then add the view automatically.
%\draw[-, blue, very thick, rounded corners=10pt]
 %([xshift=-2pt, yshift=20pt]locx-v0.north east) node (tid1start) {} -- 
%% ([xshift=-2pt, yshift=-5pt]locx-v0.south east) --
%% ([xshift=-2pt, yshift=5pt]locy-v0.north east) -- 
 %([xshift=-2pt, yshift=-5pt]locy-v0.south east);
 
 %\path (tid1start) node[anchor=south, rectangle, fill=blue!20, draw=blue, font=\small, inner sep=1pt] {$\tid_3$};

%%red view
%\draw[-, red, very thick, rounded corners = 10pt]
 %([xshift=-5pt, yshift=5pt]locx-v0.north east) -- 
%% ([xshift=-8pt, yshift=-5pt]locx-v0.south east) --
%% ([xshift=-8pt, yshift=5pt]locy-v0.north east) -- 
 %([xshift=-5pt, yshift=-10pt]locy-v0.south east) node (tid2start) {};
 
%\path (tid2start) node[anchor=north, rectangle, fill=red!20, draw=red, font=\small, inner sep=1pt] {$\tid_2$};
 
 %%green view
%\draw[-, DarkGreen, very thick, rounded corners = 10pt]
 %([xshift=-16pt, yshift=8pt]locx-v0.north east) node (tid3start) {}-- 
%% ([xshift=-15pt, yshift=-5pt]locx-v0.south east) --
%% ([xshift=-15pt, yshift=5pt]locy-v0.north east) -- 
 %([xshift=-16pt, yshift=-5pt]locy-v0.south east);
 
 %\path (tid3start) node[anchor=south, rectangle, fill=DarkGreen!20, draw=DarkGreen, font=\small, inner sep=1pt] {$\tid_1$};

%\end{pgfonlayer}
%\end{tikzpicture}
%&
%\begin{tikzpicture}[font=\large]

%\begin{pgfonlayer}{foreground}
%%Uncomment line below for help lines
%%\draw[help lines] grid(5,4);

%%Location x
%\node(locx) at (1,3) {$[\loc_x] \mapsto$};

%\matrix(locxcells) [version list, text width=7mm, anchor=west]
   %at ([xshift=10pt]locx.east) {
 %{a} & $\tsid_0$ & {a} & $\tsid_1$\\
  %{a} & $\emptyset$ & {a} & $\emptyset$ \\
%};
%\node[version node, fit=(locxcells-1-1) (locxcells-2-1), fill=white, inner sep= 0cm, font=\Large] (locx-v0) {$0$};
%\node[version node, fit=(locxcells-1-3) (locxcells-2-3), fill=white, inner sep=0cm, font=\Large] (locx-v-1) {$1$};
%%Location y
%\path (locx.south) + (0,-1.5) node (locy) {$[\loc_y] \mapsto$};
%\matrix(locycells) [version list, text width=7mm, anchor=west]
   %at ([xshift=10pt]locy.east) {
 %{a} & $\tsid_0$ \\
   %{a} & $\emptyset$ \\
%};
%\node[version node, fit=(locycells-1-1) (locycells-2-1), fill=white, inner sep= 0cm, font=\Large] (locy-v0) {$0$};
%% \draw[-, red, very thick, rounded corners] ([xshift=-5pt, yshift=5pt]locx-v1.north east) |- 
%%  ($([xshift=-5pt,yshift=-5pt]locx-v1.south east)!.5!([xshift=-5pt, yshift=5pt]locy-v0.north east)$) -| ([xshift=-5pt, yshift=5pt]locy-v0.south east);

%%blue view - I should  check whether I can use pgfkeys to just declare the list of locations, and then add the view automatically.
%\draw[-, blue, very thick, rounded corners=10pt]
 %([xshift=-2pt, yshift=20pt]locx-v0.north east) node (tid1start) {} -- 
%% ([xshift=-2pt, yshift=-5pt]locx-v0.south east) --
%% ([xshift=-2pt, yshift=5pt]locy-v0.north east) -- 
 %([xshift=-2pt, yshift=-5pt]locy-v0.south east);
 
 %\path (tid1start) node[anchor=south, rectangle, fill=blue!20, draw=blue, font=\small, inner sep=1pt] {$\tid_3$};

%%red view
%\draw[-, red, very thick, rounded corners = 10pt]
 %([xshift=-5pt, yshift=5pt]locx-v0.north east) -- 
%% ([xshift=-8pt, yshift=-5pt]locx-v0.south east) --
%% ([xshift=-8pt, yshift=5pt]locy-v0.north east) -- 
 %([xshift=-5pt, yshift=-10pt]locy-v0.south east) node (tid2start) {};
 
%\path (tid2start) node[anchor=north, rectangle, fill=red!20, draw=red, font=\small, inner sep=1pt] {$\tid_2$};
 
 %%green view
%\draw[-, DarkGreen, very thick, rounded corners = 10pt]
 %([xshift=-16pt, yshift=8pt]locx-v1.north east) node (tid3start) {}-- 
 %([xshift=-16pt, yshift=-5pt]locx-v1.south east) --
 %([xshift=-16pt, yshift=5pt]locy-v0.north east) -- 
 %([xshift=-16pt, yshift=-5pt]locy-v0.south east);
 
 %\path (tid3start) node[anchor=south, rectangle, fill=DarkGreen!20, draw=DarkGreen, font=\small, inner sep=1pt] {$\tid_1$};

%\end{pgfonlayer}
%\end{tikzpicture}
%\\
%{\small (a)} & {\small (b)}\\
%\hline
%\begin{tikzpicture}[font=\large]

%\begin{pgfonlayer}{foreground}
%%Uncomment line below for help lines
%%\draw[help lines] grid(5,4);

%%Location x
%\node(locx) at (1,3) {$[\loc_x] \mapsto$};

%\matrix(locxcells) [version list, text width=7mm, anchor=west]
   %at ([xshift=10pt]locx.east) {
 %{a} & $\tsid_0$ & {a} & $\tsid_1$\\
  %{a} & $\emptyset$ & {a} & $\emptyset$ \\
%};
%\node[version node, fit=(locxcells-1-1) (locxcells-2-1), fill=white, inner sep= 0cm, font=\Large] (locx-v0) {$0$};
%\node[version node, fit=(locxcells-1-3) (locxcells-2-3), fill=white, inner sep=0cm, font=\Large] (locx-v-1) {$1$};
%%Location y
%\path (locx.south) + (0,-1.5) node (locy) {$[\loc_y] \mapsto$};
%\matrix(locycells) [version list, text width=7mm, anchor=west]
   %at ([xshift=10pt]locy.east) {
 %{a} & $\tsid_0$ \\
   %{a} & $\emptyset$ \\
%};
%\node[version node, fit=(locycells-1-1) (locycells-2-1), fill=white, inner sep= 0cm, font=\Large] (locy-v0) {$0$};
%% \draw[-, red, very thick, rounded corners] ([xshift=-5pt, yshift=5pt]locx-v1.north east) |- 
%%  ($([xshift=-5pt,yshift=-5pt]locx-v1.south east)!.5!([xshift=-5pt, yshift=5pt]locy-v0.north east)$) -| ([xshift=-5pt, yshift=5pt]locy-v0.south east);

%%blue view - I should  check whether I can use pgfkeys to just declare the list of locations, and then add the view automatically.
%\draw[-, blue, very thick, rounded corners=10pt]
 %([xshift=-2pt, yshift=20pt]locx-v0.north east) node (tid1start) {} -- 
%% ([xshift=-2pt, yshift=-5pt]locx-v0.south east) --
%% ([xshift=-2pt, yshift=5pt]locy-v0.north east) -- 
 %([xshift=-2pt, yshift=-5pt]locy-v0.south east);
 
 %\path (tid1start) node[anchor=south, rectangle, fill=blue!20, draw=blue, font=\small, inner sep=1pt] {$\tid_3$};

%%red view
%\draw[-, red, very thick, rounded corners = 10pt]
 %([xshift=-5pt, yshift=5pt]locx-v1.north east) -- 
 %([xshift=-5pt, yshift=-5pt]locx-v1.south east) --
 %([xshift=-5pt, yshift=3pt]locy-v0.north east) -- 
 %([xshift=-5pt, yshift=-10pt]locy-v0.south east) node (tid2start) {};
 
%\path (tid2start) node[anchor=north, rectangle, fill=red!20, draw=red, font=\small, inner sep=1pt] {$\tid_2$};
 
 %%green view
%\draw[-, DarkGreen, very thick, rounded corners = 10pt]
 %([xshift=-16pt, yshift=8pt]locx-v1.north east) node (tid3start) {}-- 
 %([xshift=-16pt, yshift=-5pt]locx-v1.south east) --
 %([xshift=-16pt, yshift=5pt]locy-v0.north east) -- 
 %([xshift=-16pt, yshift=-5pt]locy-v0.south east);
 
 %\path (tid3start) node[anchor=south, rectangle, fill=DarkGreen!20, draw=DarkGreen, font=\small, inner sep=1pt] {$\tid_1$};

%\end{pgfonlayer}
%\end{tikzpicture}
%&
%\begin{tikzpicture}[font=\large]

%\begin{pgfonlayer}{foreground}
%%Uncomment line below for help lines
%%\draw[help lines] grid(5,4);

%%Location x
%\node(locx) at (1,3) {$[\loc_x] \mapsto$};

%\matrix(locxcells) [version list, text width=7mm, anchor=west]
   %at ([xshift=10pt]locx.east) {
 %{a} & $\tsid_0$ & {a} & $\tsid_1$\\
  %{a} & $\emptyset$ & {a} & $\{\tsid_2\}$ \\
%};
%\node[version node, fit=(locxcells-1-1) (locxcells-2-1), fill=white, inner sep= 0cm, font=\Large] (locx-v0) {$0$};
%\node[version node, fit=(locxcells-1-3) (locxcells-2-3), fill=white, inner sep=0cm, font=\Large] (locx-v-1) {$1$};
%%Location y
%\path (locx.south) + (0,-1.5) node (locy) {$[\loc_y] \mapsto$};
%\matrix(locycells) [version list, text width=7mm, anchor=west]
   %at ([xshift=10pt]locy.east) {
 %{a} & $\tsid_0$ & {a} & $\tsid_2$ \\
   %{a} & $\emptyset$ & {a} & $\emptyset$\\
%};
%\node[version node, fit=(locycells-1-1) (locycells-2-1), fill=white, inner sep= 0cm, font=\Large] (locy-v0) {$0$};
%\node[version node, fit=(locycells-1-3) (locycells-2-3), fill=white, inner sep=0cm, font=\Large] (locy-v-1) {$1$};
%% \draw[-, red, very thick, rounded corners] ([xshift=-5pt, yshift=5pt]locx-v1.north east) |- 
%%  ($([xshift=-5pt,yshift=-5pt]locx-v1.south east)!.5!([xshift=-5pt, yshift=5pt]locy-v0.north east)$) -| ([xshift=-5pt, yshift=5pt]locy-v0.south east);

%%blue view - I should  check whether I can use pgfkeys to just declare the list of locations, and then add the view automatically.
%\draw[-, blue, very thick, rounded corners=10pt]
 %([xshift=-2pt, yshift=20pt]locx-v0.north east) node (tid1start) {} -- 
%% ([xshift=-2pt, yshift=-5pt]locx-v0.south east) --
%% ([xshift=-2pt, yshift=5pt]locy-v0.north east) -- 
 %([xshift=-2pt, yshift=-5pt]locy-v0.south east);
 
 %\path (tid1start) node[anchor=south, rectangle, fill=blue!20, draw=blue, font=\small, inner sep=1pt] {$\tid_3$};

%%red view
%\draw[-, red, very thick, rounded corners = 10pt]
 %([xshift=-5pt, yshift=5pt]locx-v1.north east) -- 
%% ([xshift=-5pt, yshift=-5pt]locx-v1.south east) --
%% ([xshift=-5pt, yshift=3pt]locy-v0.north east) -- 
 %([xshift=-5pt, yshift=-10pt]locy-v1.south east) node (tid2start) {};
 
%\path (tid2start) node[anchor=north, rectangle, fill=red!20, draw=red, font=\small, inner sep=1pt] {$\tid_2$};
 
 %%green view
%\draw[-, DarkGreen, very thick, rounded corners = 10pt]
 %([xshift=-16pt, yshift=8pt]locx-v1.north east) node (tid3start) {}-- 
 %([xshift=-16pt, yshift=-5pt]locx-v1.south east) --
 %([xshift=-16pt, yshift=5pt]locy-v0.north east) -- 
 %([xshift=-16pt, yshift=-5pt]locy-v0.south east);
 
 %\path (tid3start) node[anchor=south, rectangle, fill=DarkGreen!20, draw=DarkGreen, font=\small, inner sep=1pt] {$\tid_1$};

%\end{pgfonlayer}
%\end{tikzpicture}\\
%{\small (c)} & {\small (d)} \\
%\hline
%\begin{tikzpicture}[font=\large]

%\begin{pgfonlayer}{foreground}
%%Uncomment line below for help lines
%%\draw[help lines] grid(5,4);

%%Location x
%\node(locx) at (1,3) {$[\loc_x] \mapsto$};

%\matrix(locxcells) [version list, text width=7mm, anchor=west]
   %at ([xshift=10pt]locx.east) {
 %{a} & $\tsid_0$ & {a} & $\tsid_1$\\
  %{a} & $\emptyset$ & {a} & $\{\tsid_2\}$ \\
%};
%\node[version node, fit=(locxcells-1-1) (locxcells-2-1), fill=white, inner sep= 0cm, font=\Large] (locx-v0) {$0$};
%\node[version node, fit=(locxcells-1-3) (locxcells-2-3), fill=white, inner sep=0cm, font=\Large] (locx-v-1) {$1$};
%%Location y
%\path (locx.south) + (0,-1.5) node (locy) {$[\loc_y] \mapsto$};
%\matrix(locycells) [version list, text width=7mm, anchor=west]
   %at ([xshift=10pt]locy.east) {
 %{a} & $\tsid_0$ & {a} & $\tsid_2$ \\
   %{a} & $\emptyset$ & {a} & $\emptyset$\\
%};
%\node[version node, fit=(locycells-1-1) (locycells-2-1), fill=white, inner sep= 0cm, font=\Large] (locy-v0) {$0$};
%\node[version node, fit=(locycells-1-3) (locycells-2-3), fill=white, inner sep=0cm, font=\Large] (locy-v-1) {$1$};
%% \draw[-, red, very thick, rounded corners] ([xshift=-5pt, yshift=5pt]locx-v1.north east) |- 
%%  ($([xshift=-5pt,yshift=-5pt]locx-v1.south east)!.5!([xshift=-5pt, yshift=5pt]locy-v0.north east)$) -| ([xshift=-5pt, yshift=5pt]locy-v0.south east);

%%blue view - I should  check whether I can use pgfkeys to just declare the list of locations, and then add the view automatically.
%\draw[-, blue, very thick, rounded corners=10pt]
 %([xshift=-2pt, yshift=20pt]locx-v0.north east) node (tid1start) {} -- 
 %([xshift=-2pt, yshift=-5pt]locx-v0.south east) --
 %([xshift=-2pt, yshift=5pt]locy-v1.north east) -- 
 %([xshift=-2pt, yshift=-5pt]locy-v1.south east);
 
 %\path (tid1start) node[anchor=south, rectangle, fill=blue!20, draw=blue, font=\small, inner sep=1pt] {$\tid_3$};

%%red view
%\draw[-, red, very thick, rounded corners = 10pt]
 %([xshift=-5pt, yshift=5pt]locx-v1.north east) -- 
%% ([xshift=-5pt, yshift=-5pt]locx-v1.south east) --
%% ([xshift=-5pt, yshift=3pt]locy-v0.north east) -- 
 %([xshift=-5pt, yshift=-10pt]locy-v1.south east) node (tid2start) {};
 
%\path (tid2start) node[anchor=north, rectangle, fill=red!20, draw=red, font=\small, inner sep=1pt] {$\tid_2$};
 
 %%green view
%\draw[-, DarkGreen, very thick, rounded corners = 10pt]
 %([xshift=-16pt, yshift=8pt]locx-v1.north east) node (tid3start) {}-- 
 %([xshift=-16pt, yshift=-5pt]locx-v1.south east) --
 %([xshift=-16pt, yshift=5pt]locy-v0.north east) -- 
 %([xshift=-16pt, yshift=-5pt]locy-v0.south east);
 
 %\path (tid3start) node[anchor=south, rectangle, fill=DarkGreen!20, draw=DarkGreen, font=\small, inner sep=1pt] {$\tid_1$};

%\end{pgfonlayer}
%\end{tikzpicture}
%&
%\begin{tikzpicture}[font=\large]

%\begin{pgfonlayer}{foreground}
%%Uncomment line below for help lines
%%\draw[help lines] grid(5,4);

%%Location x
%\node(locx) at (1,3) {$[\loc_x] \mapsto$};

%\matrix(locxcells) [version list, text width=7mm, anchor=west]
   %at ([xshift=10pt]locx.east) {
 %{a} & $\tsid_0$ & {a} & $\tsid_1$\\
  %{a} & $\{\tsid_3\}$ & {a} & $\{\tsid_2\}$ \\
%};
%\node[version node, fit=(locxcells-1-1) (locxcells-2-1), fill=white, inner sep= 0cm, font=\Large] (locx-v0) {$0$};
%\node[version node, fit=(locxcells-1-3) (locxcells-2-3), fill=white, inner sep=0cm, font=\Large] (locx-v1) {$1$};
%%Location y
%\path (locx.south) + (0,-1.5) node (locy) {$[\loc_y] \mapsto$};
%\matrix(locycells) [version list, text width=7mm, anchor=west]
   %at ([xshift=10pt]locy.east) {
 %{a} & $\tsid_0$ & {a} & $\tsid_2$ \\
   %{a} & $\emptyset$ & {a} & $\{\tsid_3\}$\\
%};
%\node[version node, fit=(locycells-1-1) (locycells-2-1), fill=white, inner sep= 0cm, font=\Large] (locy-v0) {$0$};
%\node[version node, fit=(locycells-1-3) (locycells-2-3), fill=white, inner sep=0cm, font=\Large] (locy-v1) {$1$};
%% \draw[-, red, very thick, rounded corners] ([xshift=-5pt, yshift=5pt]locx-v1.north east) |- 
%%  ($([xshift=-5pt,yshift=-5pt]locx-v1.south east)!.5!([xshift=-5pt, yshift=5pt]locy-v0.north east)$) -| ([xshift=-5pt, yshift=5pt]locy-v0.south east);

%%blue view - I should  check whether I can use pgfkeys to just declare the list of locations, and then add the view automatically.
%\draw[-, blue, very thick, rounded corners=10pt]
 %([xshift=-2pt, yshift=20pt]locx-v0.north east) node (tid1start) {} -- 
 %([xshift=-2pt, yshift=-5pt]locx-v0.south east) --
 %([xshift=-2pt, yshift=5pt]locy-v1.north east) -- 
 %([xshift=-2pt, yshift=-5pt]locy-v1.south east);
 
 %\path (tid1start) node[anchor=south, rectangle, fill=blue!20, draw=blue, font=\small, inner sep=1pt] {$\tid_3$};

%%red view
%\draw[-, red, very thick, rounded corners = 10pt]
 %([xshift=-5pt, yshift=5pt]locx-v1.north east) -- 
%% ([xshift=-5pt, yshift=-5pt]locx-v1.south east) --
%% ([xshift=-5pt, yshift=3pt]locy-v0.north east) -- 
 %([xshift=-5pt, yshift=-10pt]locy-v1.south east) node (tid2start) {};
 
%\path (tid2start) node[anchor=north, rectangle, fill=red!20, draw=red, font=\small, inner sep=1pt] {$\tid_2$};
 
 %%green view
%\draw[-, DarkGreen, very thick, rounded corners = 10pt]
 %([xshift=-16pt, yshift=8pt]locx-v1.north east) node (tid3start) {}-- 
 %([xshift=-16pt, yshift=-5pt]locx-v1.south east) --
 %([xshift=-16pt, yshift=5pt]locy-v0.north east) -- 
 %([xshift=-16pt, yshift=-5pt]locy-v0.south east);
 
 %\path (tid3start) node[anchor=south, rectangle, fill=DarkGreen!20, draw=DarkGreen, font=\small, inner sep=1pt] {$\tid_1$};

%\end{pgfonlayer}
%\end{tikzpicture}
%\\
%{\small (e)} & {\small (f)} \\
%\hline
%\end{tabular}
%\caption{History heaps obtained in a execution of $\prog_2$.}
%\label{fig:cc.exec}
%\end{figure}


\sx{Should we give intuition about causal dependencies here ?}
The next consistency model that we are interested is \emph{transactional causal consistency} \cite{cops}. 
Intuitively, it ensures that versions read by transactions are closed with respect to \emph{causal dependencies}. 
Consider for example the following program: 
\[
    \prog_2 \equiv \begin{session}
        \begin{array}{@{}c || c || c@{}}
            \txid_{1} : 
            \begin{transaction}
                \pmutate{\vx}{1};\\
            \end{transaction} &
            \txid_{2} : 
            \begin{transaction} 
                \pderef{\pvar{a}}{\vx};\\
                \pmutate{\vy}{\pvar{a}};\\
            \end{transaction} &
            \txid_{3} :
             \begin{transaction}
               	   \pderef{\pvar{a}}{\vx};\\
               	   \pderef{\pvar{b}}{\vy};\\
               	   \pifs{\pvar{a}=0 \wedge \pvar{b}=1}\\
               			\quad \passign{\retvar}{\Large \frownie{}}
               		\pife
             \end{transaction}
        \end{array}
    \end{session}
 \]
For the sake of simplicity, we label the code of the three transactions above as $\txid_{1}, \txid_2, \txid_3$ from left to right.
It is easy to see that, if no constraints or even under read atomic, the third transaction $\txid_{3}$ can return ${\Large \frownie{}}$. 
%The same is true even if the consistency model specification $\mathsf{RA}$ is assumed. 
Informally, the return of value ${\Large \frownie{}}$ by $\txid_3$ can be obtained from the execution outlined below. 
\sx{Did not edit, change later}
\begin{itemize}
\item The initial configuration of this execution is depicted in Figure \ref{fig:cc.exec}(a).
\item $\ptrans{\trans_1}$ executes with the initial view, which points to the 
initial (and only) version for each location; after this transaction is 
executed, a new version $\langle 1, T_1, \emptyset \rangle$ is appended 
at the end of $\hh(\loc_{x})$. The resulting history heap is depicted in Figure \ref{fig:cc.exec}(b).
\item next, $\thid_2$ updates its view as to see the version of $\loc_x$ installed by $\thid_1$, after 
which it proceeds to execute $\ptrans{\trans_2}$. This results in a new version with value $1$ 
to be installed for $\loc_y$. The configurations before, and after the execution of $\ptrans{\trans_2}$, 
are depicted in figures \ref{fig:cc.exec}(c) and \ref{fig:cc.exec}(d), respectively.
\item Finally, thread $\thid_3$ updates its view to observe the update of location $[\loc_y]$, but not the update of 
location $[\loc_y]$, before executing transaction $\ptrans{\trans_3}$. The execution of $\ptrans{\trans_3}$ will 
return the value ${\Large \frownie{}}$. The history heaps immediately before and after 
the execution of $\thid_3$, are depicted in figures \ref{fig:cc.exec}(e) and \ref{fig:cc.exec}(f), respectively. 
\end{itemize}

\sx{Change above}

In the last step, the thread to the right commits the transaction $\txid_3$ in a state where its initial view observes the second version of the address $\vy$, which is created by \( \txid_{2} \).
However, because the transaction \( \txid_{2} \) read the second version of \( \vx \) and create the second version of \( \vy \), this means the latter depends on the former.
Yet the transaction \( \txid_{3} \) does not read from the second version of \( \vx \), which is disallowed by \emph{transactional causal consistency}.
Summarising, under transactional causal consistency if a transaction sees updates for an address \( \addr \), it should also observes those addresses that \( \addr \) depends on.

\ac{
but not the update to address $\vx$.
However, the update of $[\vy]$ committed by $\txid_{2}$, consisted in copying the value of the update 
of $[\loc_y]$: that is, the update of $[\loc_y]$ \textbf{depends} from the update of $[\loc_x]$. 
Summarising, the execution of transaction $\ptrans{\trans_3}$ resulted in a violation of 
causality: the update of $[\loc_y]$ is observed, but not the update of $[\loc_x]$ on which 
it depends.
}

\sx{in RA up-to-date view, and here consistent view, not sure are good words, let re-think later}
To formally specify transactional causal consistency, we inductively define the set of views that are consistent with respect to a history heap $\hh$. 
\sx{What do you mean by the word??}
The definition below models the fact that, if we start from a causally consistent view, and we wish to update the view for some location $\txid_2$, 

\begin{defn}
Given two versions $\ver_{1} = (\val_1, \txid_1, \txidset_1)$, $\ver_2 = (\val_2, \txid_2, \txidset_2)$, $\ver_{1}$ \emph{directly depends on} $\ver_2$, written $\pred{ddep}{\ver_{1}, \ver_{2}}$, if $\txid_1 \in \txidset_2$. 
\ac{Note to self: the notion of directly depends here has little to do with dependencies 
between transactions. $\nu_1 \xrightarrow{\mathsf{ddep}} \nu_2$ means that 
some transaction $\tsid$ touches both versions. However, it reads $\nu_2$ and 
writes $\nu_1$.}
Given $\hh \in \HisHeaps$, the set of views that are \emph{causally consistent} with respect to $\hh$, $\func{CCViews}{\hh}$, is defined as the smallest set such that: 
\begin{enumerate} 
\item the initial view \( \vi_0\)  is in the set, \ie $\vi_0 \in \func{CCViews}{\hh}$ where \( \fora{\addr \in \dom(\hh)} \vi_{0}(\addr) = 1 \),
\item assume any view $\vi \in \pred{CCViews}{\hh}$ and a new view \( \vi' \) by observing one more version for an address $\vi' = \vi\rmto{\addr}{\vi(\addr) + 1}$, where \( \vi'(\addr) \leq \left| \hh(\addr) \right| \).
If some versions directly depend on the version corresponding to \( \vi'(\addr)\) and those versions are aware by \( \vi'\), the new view is included in \( \func{CCViews}{\hh}\),
%for some $i: V([\loc_x]) < i \leq \lvert \hh \rvert -1$
%for some $[\loc_x]$ such that $V([\loc_x]) < \lvert \hh([\loc_x]) \rvert - 1$.
%Suppose that 
\[
\begin{array}{@{}l}
\fora{\vi,\vi'} \exsts{\addr}
\vi \in \func{CCViews}{\hh} 
\land \addr \in \dom(\vi)
\land \vi' = \vi\rmto{\addr}{\vi(\addr) + 1}
\land \vi'(\addr) \leq \left| \hh(\addr) \right|  \\
\quad {} \land 
\begin{B}
\fora{\addr', i}  
1 \leq i \leq \left| \hh(\addr') \right|
\land \pred{ddep}{\hh(\addr')(i), \hh(\addr)(\vi'(\addr))}
\implies i \leq \vi(\addr')
\end{B} \\
\qquad {} \implies \vi' \in \func{CCViews}{\hh}
\end{array}
\]
%for any location $[\loc_y]$ and 
%index $j = 0, \cdots, \lvert \hh([\loc_y]) \rvert -1$ such that $\hh([\loc_{x})(V'([\loc_x]))$ 
%directly depends on $\hh([\loc_y])(j)$, then $j \leq V([\loc_y])$. Then 
%$V' \in \mathsf{CCViews}(\hh)$.
% and suppose that $\hh([\loc_{x}])(V'([\loc_x])) = 
%(\_, \tsid, \_ )$ for some $\tsid$. If for all locations $\loc_{y}$ and 
%indexes $j$ such that $\hh([\loc_y])(j) = (\_, \_, \_ \cup \{\tsid\})$, 
%then $j \leq V'([\loc_y])$, then $V' \in \mathsf{CCViews}(\hh)$.
\end{enumerate}
\end{defn}


\bibliography{bibliography,bibliography2}

%\newpage
%\appendix
%\section{Semantics\label{sec:proof_semantics}}
\subsection{Soundness of the time-stamp semantics}
\begin{lem}[No overwrite]
    \label{lem:no-over-write}
    For a given address and time, if there is a value, threads cannot overwrite the values.
    \[ 
        \for{ \stk, \stk', \tshp, \tshp', \ts, \ts', \iprog, {\iprog}' ,\lb,\addr, \ts } 
        ( \stk, \tshp, \ts ), \iprog \toT{\lb} ( \stk', \tshp', \ts' ), {\iprog}'
        \implies \tshp(\addr)(\ts)\undef \lor \tshp(\addr)(\ts) = \tshp'(\addr)(\ts)
    \]
\end{lem}
\begin{proof}
    Prove by structural induction on the semantics.
    For base cases \rl{PChoiseLeft}, \rl{PChoiseRight}, \rl{PLoop}, \rl{PSeqSkip}, \rl{PPar} and \rl{Pwait}, they are trivial because those rules do not change time-stamp heap, i.e.\ \( \tshp = \tshp' \).
    For base case \rl{Commit}, the new state \( \tshp' = \commit{\tshp}{\fph_{s}}{\fph_{e}}{\tsid}{\ts_{s}}{\ts_{e}} \) for some \( \fph_{s},\fph_{e}, \tsid, \ts_{s},\ts_{e} \).
    For any addresses updated by the new transaction \( \tsid \), It must pick times \( \ts_{s} \) and \( \ts_{e} \) when the values of those addresses are undefined, because of the constrain \( \cancommitName \), especially the sub-predicate \( \wfhistName \).
    For the inductive case \rl{PSeq}, prove directly by the inductive hypothesis.
\end{proof}

\begin{lem}[Start before end]
    \label{lem:start-before-end}
    \label{lem:read-before-write}
    All the reads/starts operations of a transaction happen before all the writes/ends. 
    Formally,
    \[
        \for{ \stk, \stk', \tshp, \tshp', \ts, \ts', \iprog, {\iprog}' ,\lb } 
        \pred{rw}{\tshp} 
        \land ( \stk, \tshp, \ts ), \iprog \toT{\lb} ( \stk', \tshp', \ts' ), {\iprog}' 
        \implies \pred{rw}{\tshp'}
    \]
    where,
    \[
        \begin{rclarray}
            \pred{rw}{\tshp} & \defeq & 
            \for{ \addr, \addr', \ts, \ts',\etag \in \Set{\etS, \etR}, \etag' \in \Set{\etE,\etW}, \tsid } 
            \tshp(\addr)(\ts) = (\stub, \etag, \tsid) 
            \land \tshp(\addr')(\ts') = (\stub, \etag', \tsid)
            \implies \ts < \ts' 
        \end{rclarray}
    \]
\end{lem}
\begin{proof}
    Prove by structural induction on the semantics.
    For base cases \rl{PChoiseLeft}, \rl{PChoiseRight}, \rl{PLoop}, \rl{PSeqSkip}, \rl{PPar} and \rl{PWait}, they are trivial because those rules do not change time-stamp heap, i.e.\ \( \tshp = \tshp' \).
    For base case \rl{Commit}, the new state \( \tshp' = \commit{\tshp}{\fph_{s}}{\fph_{e}}{\tsid}{\ts_{s}}{\ts_{e}} \) for some \( \fph_{s},\fph_{e}, \tsid, \ts_{s},\ts_{e} \). 
    First by Lemma \ref{lem:no-over-write}, for those transactions in \( \tshp \) it remains the same as in \( \tshp' \).
    Then, the \(\commitName\) function associates \( \etR \) or \( \etS \) to time \( \ts_{s} \), and \( \etW \) or \( \etE \) to \( \ts_{e} \), and \( \ts_{s} < \ts_{e} \), so \( \pred{rw}{\tshp'}\) holds.
    For the inductive case \rl{PSeq}, prove directly by the inductive hypothesis.
\end{proof}

\begin{lem}[Pair of start and end]
    \label{lem:start-end-pair}
    Reads/starts operations are writes/ends operations appear as pairs.
    \[
        \for{ \stk, \stk', \tshp, \tshp', \ts, \ts', \iprog, {\iprog}' ,\lb }
        \pred{pair}{\tshp} 
        \land ( \stk, \tshp, \ts ), \iprog \toT{\lb} ( \stk', \tshp', \ts' ), {\iprog}' 
        \implies \pred{pair}{\tshp'}
    \]
    where,
    \[
        \begin{rclarray}
            \pred{pair}{ \tshp } & \defeq & 
            \for{ \addr, \addr', \ts, \ts',\etag \in \Set{\etS, \etR}, \etag' \in \Set{\etE,\etW}, \tsid } 
            (\tshp(\addr)(\ts) = (\stub, \etag, \tsid) \iff \tshp(\addr')(\ts') = (\stub, \etag', \tsid)  )
        \end{rclarray}
    \]
\end{lem}
\begin{proof}
    Prove by structural induction on the semantics.
    For base cases \rl{PChoiseLeft}, \rl{PChoiseRight}, \rl{PLoop}, \rl{PSeqSkip}, \rl{PPar} and \rl{PWait}, they are trivial because those rules do not change time-stamp heap, i.e.\ \( \tshp = \tshp' \).
    For base case \rl{Commit}, the new state \( \tshp' = \commit{\tshp}{\fph_{s}}{\fph_{e}}{\tsid}{\ts_{s}}{\ts_{e}} \) for some \( \fph_{s},\fph_{e}, \tsid, \ts_{s},\ts_{e} \). 
    First by Lemma \ref{lem:no-over-write}, for those transactions in \( \tshp \) it remains the same as in \( \tshp' \).
    Then the \(\commitName\) function always extends addresses with pairs of operations, so \( \pred{rw}{\tshp'}\) holds.
    For the inductive case \rl{PSeq}, prove directly by the inductive hypothesis.
\end{proof}


\begin{lem}[Start/end at the same time]
    \label{lem:atoic-rw}
    \label{lem:happen-in-same-time}
    All the reads/starts operations among all addresses of a transaction happen in the same time, so do all the writes/ends operations. 
    \[
        \for{ \stk, \stk', \tshp, \tshp', \ts, \ts', \iprog, {\iprog}' ,\lb } 
        \pred{atom}{\tshp} 
        \land ( \stk, \tshp, \ts ), \iprog \toT{\lb} ( \stk', \tshp', \ts' ), {\iprog}' 
        \implies \pred{atom}{\tshp'}
    \]
    where,
    \[
        \begin{rclarray}
        \pred{atom}{\tshp} & \defeq & 
        \for{ \addr, \addr', \ts, \ts', \tsid, \etag, \etag' }
        \tshp(\addr)(\ts) (\stub, \etag, \tsid) 
        \land \tshp(\addr')(\ts') = (\stub, \etag', \tsid) \\
        & & \land (\etag, \etag' \in \Set{\etS, \etR} \lor \etag, \etag' \in \Set{\etE, \etW} ) 
        \implies \ts = \ts'
        \end{rclarray}
    \]
\end{lem}
\begin{proof}
    Prove by structural induction on the semantics.
    For base cases \rl{PChoiseLeft}, \rl{PChoiseRight}, \rl{PLoop}, \rl{PSeqSkip}, \rl{PPar} and \rl{Pwait}, they are trivial because those rules do not change time-stamp heap, i.e.\ \( \tshp = \tshp' \).
    For base case \rl{Commit}, the new state \( \tshp' = \commit{\tshp}{\fph_{s}}{\fph_{e}}{\tsid}{\ts_{s}}{\ts_{e}} \) for some \( \fph_{s},\fph_{e}, \tsid, \ts_{s},\ts_{e} \). 
    First by Lemma \ref{lem:no-over-write}, for those transactions in \( \tshp \) it remains the same as in \( \tshp' \).
    Then the \( \commitName \) function associates all reads/starts operations, \( \etR \) and \(\etS \), of all addresses to the start time \( \ts_{s} \), and writes/ends operations to the end time \( \ts_{e} \), so \( \pred{atom}{\tshp'}\) holds.
    For the inductive case \rl{PSeq}, prove directly by the inductive hypothesis.
\end{proof}

\begin{defn}[Well-form time-stamp heaps]
\label{def:wf-timestamp-heap}
    The well-formedness of time-stamp heap is defined as those satisfying Lemma \ref{lem:start-before-end}, Lemma \ref{lem:start-end-pair} and Lemma \ref{lem:atoic-rw}.
    \[
        \begin{rclarray}
            \wfH{\tshp} & \defeq & \pred{rw}{\tshp} \land \pred{pair}{\tshp} \land \pred{atom}{\tshp} \\
        \end{rclarray}
    \]
\end{defn}

\begin{defn}[Extended Labels]
\label{def:ext-label}
Given the set of transaction labels \( \Translabel \) (\defin \ref{def:label}) and thread identifiers \( \ThreadID \) (\defin \ref{def:thread_semantics}), the set of extended labels \( \ExtTranslabel \defeq ( \Nat \times \TransID \times \Translabel ) \cup \Set{\lbID} \) is defined as follows:
\[
	\ExtTranslabel ::= \lbID \mid (\thid,\lbC{\tsid}) \mid (\thid,\lbF{\thid',\prog}) \mid (\thid,\lbJ{\thid',\ts})
\]
\end{defn}

For brevity, we will write \( \lbC{\thid, \tsid} \) instead of \( ( \thid,\lbC{\tsid}) \), similarly for \( \lbF{\thid, \thid',\prog}\) and \( \lbJ{\thid, \thid',\ts} \).
Also we will use the same meta-variable \( \lb \) to range over extended transaction labels.

\begin{defn}[Traces]
\label{def:traces}
    Assume lifting the operational semantics in \fig \ref{fig:thread_semantics} and \fig \ref{fig:thread_pool_semantics} to cope with extended transaction label (\defin \ref{def:ext-label}), then given an initial time-stamp heap \( \tshp  \TSHeaps \) (\defin \ref{def:timestamp_heaps}) and a thread pool \( \thpl \) (\defin \ref{def:thread_pools}), a trace \( \trc \in \Trace \defeq \powerset{ \ExtTranslabel } \times (\ExtTranslabel \times \ExtTranslabel) \) is a tuple \( (\setlabels,<) \) that satisfies predicate \( \pred{trace}{\tshp,\thpl,\setlabels,<,\nat} \), for some number \( \nat \).
\[
    \begin{rclarray}
        \func{traces}{\tshp,\thpl,0} & \defeq & \Set{(\emptyset, \emptyset,\tshp,\thpl)} \\
        \func{traces}{\tshp,\thpl,\nat} & \defeq & 
        \Setcon{%
            (\setlabels,<,\tshp',\thpl')
        }{%
            (\tshp'', \thpl'') \toG{\lbID} (\tshp', \thpl' ) 
            \land (\setlabels,<,\tshp'',\thpl'') \in \func{trace}{\tshp,\thpl,\setlabels,<,\nat-1}
        } \\
		& & \uplus \Setcon{
				(\setlabels \uplus \Set{\lb}
                , < \uplus \myset{(\lb',\lb)}{ \lb' \in \setlabels}
                , \tshp',\thpl')
			}{ 
	            (\tshp'', \thpl'') \toG{\lb} (\tshp', \thpl' ) 
                \land \lb \neq \lbID \land {} \\
                (\setlabels,<,\tshp'',\thpl'') \in \func{trace}{\tshp,\thpl,\setlabels,<,\nat-1}
			}  \\
    \end{rclarray}
\]
\end{defn}
\sx{
    Not robust enough against the same events happening twice.
    Might fix later or simply assume all labels are unique by tagged a unique identifier. 
    Could add local time also in the label which should be usefully to give more formal prove of program order is inside visibility.}

\begin{defn}[Program Order]
\label{def:po}
Given a trace \( \trc \in \Trace \) (\defin \ref{def:traces}), the \( \funcn{programOrder} : \powerset{\ExtTranslabel} \times (\ExtTranslabel \times \ExtTranslabel) \to \powerset{\Transactions} \times (\Transactions \times \Transactions) \) function is defined as follows:
\[ 
    \begin{rclarray}
        \func{programOrder}{\setlabels, <} & \defeq & ( \settrans, \po ) \\
    \end{rclarray}
\]
where,
\[ 
    \begin{rclarray}
        \settrans & \equiv & \Setcon{\tsid}{\lbC{\thid,\tsid} \in \setlabels } \\
        \po & \equiv & \myset{(\tsid, \tsid')}{\lbC{\thid,\tsid} < \lbC{\thid,\tsid'} } 
        \uplus \myset{(\tsid, \tsid')}{\lbC{\thid,\tsid} < \lbF{\thid,\thid', \stub} < \lbC{\thid',\tsid'} } \\
            & & \uplus \myset{(\tsid, \tsid')}{\lbC{\thid,\tsid} < \lbJ{\thid',\thid, \stub} < \lbC{\thid',\tsid'} }
    \end{rclarray}
\]
\end{defn}

\begin{defn}[Visibility and partial arbitration]
    \label{def:vis-ptlar}
    The \( \funcn{graph}: \TSHeaps \to ( (\Transactions \times \Transactions) \times (\Transactions \times \Transactions) )  \) function convert a time-stamp heap into corresponding \emph{visibility relation} and \emph{partial arbitration relation}, written \( \vis \) and \( \ptlar \) respectively.
    \[
        \begin{rclarray}
            \func{graph}{\tshp} & \defeq & (\vis,\ptlar) \\
        \end{rclarray}
    \]
    where,
    \[
        \begin{rclarray}
            \vis & \equiv & 
            \Setcon{%
                (\tsid,\tsid')
            }{ %
                \exsts{ \addr, \addr', \ts, \ts', \etag \in \Set{\etW,\etE}, \etag' \in \Set{\etR,\etS} } 
                \ts < \ts' \land {} \\
                \tshp(\addr)(\ts) = (\stub, \etag,\tsid)
                \land \tshp(\addr')(\ts') = (\stub, \etag',\tsid') 
			} \\
            \ptlar & \equiv & 
            \Setcon{%
            (\tsid,\tsid')
            }{%
                \exsts{ \addr, \addr', \ts, \ts', \etag, \etag' \in \Set{\etW,\etE} } 
                \ts < \ts' \land {} \\
                \tshp(\addr)(\ts) = (\stub, \etag,\tsid) 
                \land \tshp(\addr')(\ts') = (\stub, \etag',\tsid') 
			}
        \end{rclarray}
    \]
\end{defn}

Note that all transactions in the visibility or partial arbitration relations, for instance \( (\tsid, \tsid') \in \vis \), they must be in the set of transactions given by program order, meaning \( \tsid, \tsid' \in \settrans \).
For brevity, we might only mention and quantify few elements of this tuple but one should think they are quantified as an entire tuple.
For readability, we use \( (\tsid, \tsid') \in \vis \) or \( \tsid \toVIS \tsid' \) interchangeably, and similar for \( \ar \).

\begin{defn}[Partial graph]
\label{def:partial-graph}
Given the well-formedness of time-stamp heaps (\defin \ref{def:wf-timestamp-heap}), program order (\defin \ref{def:po}), and visibility and arbitration relation (\defin \ref{def:vis-ptlar}), the set of \emph{partial graph} is defined as follows, where the ``partial'' means the arbitration relation is partial.
\[
    \begin{rclarray}
    \PGraphs & \defeq & 
    \Setcon{%
        (\settrans, \po, \vis, \ptlar, \tshp)
    }{ 
        \exsts{\tshp', \thpl'} \wfH{\tshp'}
        \land (\setlabels,<,\tshp,\thpl) \in \bigcup\limits_{\nat \in \Nat}\func{traces}{\tshp',\thpl',\nat} \land {} \\
        (\settrans,\po) = \func{programOrder}{\setlabels,<} 
        \land (\vis,\ptlar) = \func{graph}{\tshp} 
    } 
    \end{rclarray}
\]
\end{defn}

\begin{lem}[Separation]
    \label{lem:seperate}
    If two transactions that are not associated by partial arbitration relation \( \ptlar \), they access different addresses but commit at the same time.
    \[
        \begin{array}{@{}l@{}}
            \for{ \tshp, \ptlar, \tsid, \tsid' } (\tsid, \tsid'), (\tsid',\tsid) \notin \ptlar \\
            \qquad \implies \for{ \addr, \addr', \ts, \ts', \etag, \etag \in \Set{\etW, \etE } }
            \tshp(\addr)(\ts) = ( \stub, \etag, \tsid) 
            \land \tshp(\addr')(\ts') = ( \stub, \etag', \tsid') 
            \implies \ts = \ts'
            \land \addr \neq \addr'
        \end{array}
    \]
\end{lem}
\begin{proof}
    Given \( \ptlar \) in \defin \ref{def:partial-graph}, \( \ts = \ts' \) holds, and then \( \addr \neq \addr' \) is derived from Lemma \ref{lem:atoic-rw}.
\end{proof}


\begin{lem}[Semi-acyclic]
    \label{lem:semi-acyclic}
    Both \( \vis \) and \( \ptlar \) are acyclic.
\end{lem}
\begin{proof}
    Proof by contradiction.
    Assume that there is a circle in \( \vis \), which means that \( \bigwedge\limits_{0 \leq i \leq n} \tsid_{i} \toVIS \tsid_{i+1} \land \tsid_{0} = \tsid_{n} \) for some \( \tsid_{0} \) to \( \tsid_{n}\).
    By Lemma \ref{lem:atoic-rw}, each transaction has a start time and a end time, thus let \( \ts^{s}_{i} \) and \( \ts^{e}_{i} \) denote the start time and end time of the transaction \( \tsid_{i} \).
    By Lemma \ref{lem:read-before-write}, we have \( \ts^{s}_{i} < \ts^{e}_{i} \), and  then by \( \vis \) which is defined in \defin \ref{def:vis-ptlar}, we have \( \ts^{e}_{i} < \ts^{s}_{i+1} \).
    Therefore, \( \ts^{s}_{0} < \ts^{e}_{0} < \ts^{s}_{1} < \dots <  \ts^{s}_{n} \), and we have contradiction that \( \ts^{s}_{0} < \ts^{s}_{n} \) with respect to Lemma \ref{lem:atoic-rw}.

    Similarly for \( \ptlar \), we have contradiction that \( \ts^{e}_{0} < \ts^{e}_{1} < \dots  < \ts^{e}_{n} \).
\end{proof}

We can extend partial arbitration relation \( \ptlar \) to a total order, called total arbitration relation or shortly arbitration \( \ar \) by taking the first two transactions that are not connected and then picking a order and take the transitive closure.
To simplify, assume there is a unique initial transactions denoted by \( \tsid_{init} \), where \( (\tsid_{init}, \tsid) \in \ptlar \) for all \( \tsid \).

\begin{defn}[Total graph]
\label{def:tototal}
\label{def:total-graph}
    First, the \( \funcn{toTotal} : \powerset{\Transactions} \times (\Transactions \times \Transactions) \parfun \powerset{\Transactions} \times (\Transactions \times \Transactions) \) function is defined as follows, which convert partial arbitration relation to total one.
    \[
        \begin{rclarray}
            \func{toTotal}{\settrans, \ptlar} & \defeq & 
                \begin{cases}
                    (\settrans, \ptlar) & \text{if} \ \func{firstBranch}{\settrans, \ptlar} = \emptyset\\ 
                    \func{toTotal}{\settrans, ( \ptlar \uplus \Set{(\tsid, \tsid')})^{+}} & \text{if} \ \tsid, \tsid' \in \func{firstBranch}{\settrans, \ptlar} \\    
                \end{cases}
            \\
        \end{rclarray}
    \]
    where, the \( \funcn{firstBranch} : \powerset{\Transactions} \times (\Transactions \times \Transactions) \parfun \powerset{\Transactions} \) function that returns a set that includes the first branched transactions (might be more than two).
    \[
        \begin{rclarray}
            \func{firstBranch}{\settrans, \ptlar} & \defeq &
            \begin{cases}
                \emptyset & \text{if} \ \settrans = \emptyset \\
                \func{firstBranch}{\settrans \setminus \Set{\tsid}} & \text{if} \ \func{dec}{\settrans, \ptlar} = \Set{\tsid} \\
                \func{dec}{\settrans, \ptlar} & \text{if} \ |\func{dec}{\settrans, \ptlar}| > 1 ) \\
                \text{undefined} & \text{otherwise}
            \end{cases} \\
        \end{rclarray}
    \]
    and the \( \funcn{dec} \) function returns all the direct descendants of the first transaction.
    \[
        \begin{rclarray}
            \func{dec}{\settrans, \ptlar} & \defeq &
            \myset{%
                \tsid
            }{%
                \exsts{ \tsid_{init}  } 
                \pred{firstTrans}{\settrans, \ptlar, \tsid_{init}} \land {} \\
                \quad (\tsid_{init}, \tsid) \in \ptlar 
                \land \neg\exsts{ \tsid' } 
                (\tsid_{init}, \tsid'),(\tsid', \tsid) \in \ptlar} \\
                \pred{firstTrans}{\settrans, \ptlar, \tsid} & \defeq & \for{\tsid' \in \settrans } \tsid = \tsid' \lor (\tsid, \tsid') \in \ptlar \\
        \end{rclarray}
    \]
    Now given the set of partial graphs \( \PGraphs \) (\defin \ref{def:partial-graph}), the set of \emph{total graph} \( \TGraphs \) is defined as follows:
    \[
        \begin{rclarray}
            \Setcon{%
                (\settrans, \po, \vis, \ar, \tshp)
            }{
                ( \settrans, \ar ) = \func{toTotal}{\settrans, \ptlar} 
                \land (\settrans, \po, \vis, \ptlar, \tshp) \in \PGraphs
            } \\
        \end{rclarray}
    \]
\end{defn}

\begin{lem}[Visibility]
    \label{lem:visibility}
    Visibility relation is a subset of arbitration relation, i.e.\ \( \vis \subseteq \ar \).
\end{lem}
\begin{proof}
    For all transactions \( \tsid \) and \( \tsid' \), if \( ( \tsid, \tsid' ) \in \vis \), it means that transaction \( \tsid \) commits before \( \tsid' \) starts, so that by Lemma \ref{lem:start-before-end} \( \tsid \) must start before \( \tsid' \) starts.
    This means \( (\tsid, \tsid') \in \ptlar \) (\defin \ref{def:vis-ptlar}), therefore \( (\tsid, \tsid') \in \ar \).
\end{proof}

\begin{lem}[Monotonic time in a thread]
    \label{lem:mono-time-thread}
    The thread's local time monotonically increase.
    This is  
    \[ 
        \for{ \stk, \stk', \tshp, \tshp', \ts, \ts', \iprog, {\iprog}' ,\lb,\addr, \ts } ( \stk, \tshp, \ts ), \iprog \toT{\lb} ( \stk', \tshp', \ts' ), {\iprog}' \implies \ts < \ts'
    \]
\end{lem}
\begin{proof}
    Prove by structural induction on the semantics.
    For base cases \rl{PChoiseLeft}, \rl{PChoiseRight}, \rl{PLoop}, \rl{PSeqSkip} and \rl{PPar}, it is trivial since those rules do not change the time.
    For \rl{Commit}, the premiss enforces that \( \ts' > \ts \) and for \rl{PWait}, \( \ts' = \max\Set{\ts, \stub} \geq \ts\).
    For the inductive case \rl{PSeq}, prove directly by the inductive hypothesis.
\end{proof}

\begin{lem}[Session]
    \label{lem:session}
    \( \po \subseteq \vis \).
\end{lem}   
\begin{proof}
    Prove by case analysis of \( \po \) (\defin \ref{def:po}).
    For \( \tsid \) and \( \tsid' \) where \( \lbC{\thid,\tsid} < \lbC{\thid,\tsid'} \), it means that transaction \( \tsid \) is reduced by the semantics before \( \tsid' \).
    Then by Lemma \ref{lem:mono-time-thread}, the commit time of the first transaction \( \tsid \) is smaller than the start time of the second one \( \tsid' \), so that \( (\tsid, \tsid') \in \vis \).

    For \( \tsid \) and \( \tsid' \) where \( \lbC{\thid,\tsid} < \lbF{\thid,\thid', \stub} < \lbC{\thid',\tsid'} \).
    By Lemma \ref{lem:mono-time-thread}, it means the parent thread \( \thid \) commits transaction \( \tsid \) before the fork of the child thread \( \thid' \), and also the child thread commits transaction \( \tsid' \) after the fork point.
    In the \rl{PFork} rule, the parent thread starts at time \( \ts \) and ends at \( \ts' \), and the child thread initialises with the time \( \ts' \), where in fact \( \ts = \ts' \).
    Therefore, \( (\tsid, \tsid') \in \vis \).

    Similarly, for \( \tsid \) and \( \tsid' \) where \( \lbC{\thid,\tsid} < \lbJ{\thid',\thid, \stub} < \lbC{\thid',\tsid'} \), all the transactions by child thread \( \thid \) must commits before the join point and therefore before all the uncommitted transactions by the parent thread \( \thid' \).
    Therefore, \( (\tsid, \tsid') \in \vis \).
\end{proof}

\sx{The session lemma I feel slightly unhappy, because it seems not very formal.}

\begin{lem}[Semi-prefix]
    \label{lem:semi-prefix}
    \( \ptlar; \vis \subseteq \vis \).
\end{lem}
\begin{proof}
    For all \( \tsid, \tsid', \tsid'' \), if \( (\tsid, \tsid') \in \ptlar \) and \( (\tsid', \tsid'') \in \vis \), by the definitions of \( \po \) and \( \vis \) (\defin \ref{def:vis-ptlar}), transaction \( \tsid' \) commits after \( \tsid \) commits but before \( \tsid'' \) starts.
    There must exist \( \addr, \addr'', \ts, \ts'', \etag \in \Set{\etW, \etE}  \) and \( \etag'' \in \Set{\etR, \etS } \) such that  \( \tshp(\addr)(\ts) = (\stub, \etag, \tsid) \), \( \tshp(\addr'')(\ts'') = (\stub, \etag'', \tsid'') \) and \( \ts < \ts'' \), thus \( ( \tsid, \tsid'' ) \in \vis \).
\end{proof}
\sx{Rework a bit until this point}

\begin{lem}[Prefix]     
    \label{lem:prefix}
    \( \ar; \vis \subseteq \vis \).
\end{lem}
\begin{proof}
    Prove by case analysis of the total arbitration relation \( \ar \).
    For all \( \tsid, \tsid', \tsid'' \) that \( (\tsid, \tsid') \in \ar \) and \( (\tsid', \tsid'') \in \vis \), if \( (\tsid, \tsid') \in \ptlar \), it is proven by Lemma \ref{lem:semi-prefix}.
    If \( ( \tsid, \tsid' ) \notin \ptlar \), it means it is a new edge by the \funcn{toTotal} function from Definition \ref{def:tototal}.
    By the Lemma \ref{lem:seperate}, \( \tsid \) and \( \tsid' \) commit at the same time.
    By the definition of \( \vis \), the commit time of \( \tsid' \) is smaller than the start time of \( \tsid'' \).
    Therefore, the commit time of \( \tsid \) is also smaller than the start time of \( \tsid'' \), so that \( ( \tsid, \tsid'' ) \in \vis \).
\end{proof}

\begin{lem}[No conflict]
    \label{lem:nocoflict}
    Two transactions cannot concurrently write to the same address, this means that one must observe another one.
    This is \( \for{ \addr, \tsid, \tsid' } \tshp(\addr)(\stub) = (\stub, \etW, \tsid) \land  \tshp(\addr)(\stub) = (\stub, \etW, \tsid' ) \implies ((\tsid, \tsid') \in \vis \lor (\tsid', \tsid) \in \vis ))\).
\end{lem}
\begin{proof}
    Prove by contradiction.
    Assume \( (\tsid, \tsid') \notin \vis \land (\tsid', \tsid) \notin \vis \), this intuitively means one transaction is overlapped with another.
    Let \( \ts_{s}, \ts_{e}, \ts'_{s} \) and \( \ts'_{e} \) be the start time and end time of transaction \( \tsid \) and \( \tsid' \) respectively.
    Because of the symmetry,  we can assume that the start time of \( \tsid \) is in between \( \tsid' \), witch means \( \ts'_{s} < \ts_{s} < \ts'_{e} \).
    Now we consider \( \ts_{e} \).
    First note that \( \ts_{e} > \ts_{s} \) by Lemma \ref{lem:start-before-end}, therefore we need to consider two cases \( \ts'_{s} < \ts_{s} < \ts_{e} < \ts'_{e} \) and  \( \ts'_{s} < \ts_{s} < \ts'_{e} < \ts_{e}  \).
    Since both transaction write the same address \( \addr \), those two cases violate the \( \predn{consist} \) requirement in the semantics, so one of the transactions must pick another start and end time.
\end{proof}

\begin{lem}[External]
    \label{lem:ext}
    A transaction should read the last values it can observe.
    This means that for all transaction \( \tsid \) and heap address \( \addr \), if the transaction read a value \( \val \) from the address, i.e.\ \( \exsts{ \ts } \tshp(\addr)(\ts) = (\val, \etR, \tsid) \), the last transaction \( \tsid' \) who writes to the same address and can be observe by \( \tsid \), i.e.\ \( (\tsid, \tsid') \in \vis\), should have written the same value, meaning \( \exsts{ \val', \ts' } \tshp(\addr)(\ts') = (\val', \etW, \tsid') \implies \val' = \val\)
\end{lem}
\begin{proof}
    Given the definition of \( \vis \), we have \( \ts' < \ts \).
    Because transaction \( \tsid' \) is the last one who write to the address, this means that \( \for{ \tsid'', \ts'', \etag'' } \tshp(\addr)(\ts'') = (\stub, \etag'', \tsid'') \implies \etag'' \neq \etW \).
    Thus by the \( \funcn{startstate} \) in the semantics, we have \( \val' = \val \).
\end{proof}

\begin{lem}[Acyclic]
    \label{lem:acyclic}
    Both \( \vis \) and \( \ar \) are acyclic.
\end{lem}
\begin{proof}
    For \( \vis \), it is proven by Lemma \ref{lem:semi-acyclic}.

    To prove \( \ar \) is acyclic, we prove inductively \( \ptlar_{i} \) is acyclic, where \( \ptlar_{0} = \ptlar \), \( \ptlar_{n} = \ar \) and \( \ptlar_{i+1} = (\ptlar_{i} \uplus \Set{(\tsid, \tsid')})^{+} \) for some \( \tsid \) and \( \tsid' \), which follows the process of \funcn{toTotal} from Definition \ref{def:tototal}.
    For base case \( \ptlar_{0} \), it is proven by Lemma \ref{lem:semi-acyclic}.
    Now assume \( \ptlar_{i} \) is acyclic and \( \ptlar_{i+1} = (\ptlar_{i} \uplus \Set{(\tsid, \tsid')})^{+} \) for some \( \tsid \) and \( \tsid' \), we prove by contradiction that \( \ptlar_{i+1} \) is acyclic.
    Given the hypothesis, if there is circle in \( \ptlar_{i+1} \), such circle either contain the edge \( (\tsid, \tsid') \), or can be broken down to a circle containing the edge \( (\tsid, \tsid') \), because of the transitive closure.
    Note that by breaking down to a circle containing \( ( \tsid, \tsid' ) \), the rest edges inside the circle must be in \( \ptlar_{i} \).
    Let assume the circle is \( \tsid_{0}, \tsid_{1}, \dots, \tsid_{i}, \tsid, \tsid', \tsid_{i+1}, \dots, \tsid_{n} \) where \( \tsid_{0} = \tsid_{n} \).
    By the \( \funcn{toTotal} \) function from Definition \ref{def:tototal}, \( \tsid, \tsid' \in \func{firstBranch}{\settrans, \ptlar_{i}}\), which means that for all \( \tsid'' \), \( (\tsid'', \tsid) \in \ptlar_{i} \) if and only if \( ( \tsid'', \tsid') \in \ptlar_{i} \), therefore \( (\tsid'', \tsid) \in \ptlar_{i+1} \) if and only if \( ( \tsid'', \tsid') \in \ptlar_{i+1} \).
    This means that the sequence without \( \tsid \), i.e.\ \( \tsid_{0}, \tsid_{1}, \dots, \tsid_{i}, \tsid', \tsid_{i+1}, \dots, \tsid_{n} \), is still a sequence where two adjacent transactions are connected by \( \ptlar_{i+1} \), and this new sequence is a circle.
    Now all the edges in the new sequence/circle already exist in \( \ptlar_{i} \), which violates our hypothesis.
    Therefore by contradiction, \( \ptlar_{i+1} \) is acyclic.
    Given that, \( \ar \) is acyclic.
\end{proof}

\begin{lem}[Total order]
    \label{lem:totalorder}
    \( \ar \) is a total order.
\end{lem}
\begin{proof}
    By Lemma \ref{lem:acyclic}, \( \ar \) is acyclic, and by \defin \ref{def:tototal}, for all \( \tsid \) and \( \tsid' \), either \( (\tsid, \tsid') \in \ar \) or \( (\tsid', \tsid) \in \ar \).
    Therefore, it is a total order.
\end{proof}

\begin{defn}[Snapshot Isolation Graph]
    \(\sig \in \SIGraphs \)
\end{defn}

\begin{thm}[Soundness of the semantics]
    The thread pool operational semantics $\toG{}$ (\defin\ref{def:thread_pool_semantics}) is sound.
    This means
    \[
        \begin{array}{@{}l@{}}
            \exsts{ f } \for{ \tshp, \thpl, \tshp', \thpl' } (\tshp, \thpl) \toG{\lb} (\tshp', \thpl') \land \exsts{ \sig \in \SIGraphs } f(\tshp, \thpl) = \sig \implies \exsts{ \sig' \in \SIGraphs } f(\tshp', \thpl') = \sig'
        \end{array}
    \]
\end{thm}
\begin{proof}
    Given the \defin \ref{def:traces}, there is a unique corresponding trace \trace, and then by \defin \ref{def:po}, \ref{def:vis-ptlar} and \ref{def:tototal} there exists a \( \sig' \).
    Then, by Lemma \ref{lem:visibility}, \ref{lem:session}, \ref{lem:prefix}, \ref{lem:nocoflict}, \ref{lem:ext}, \ref{lem:totalorder}, the \( \sig' \in \SIGraphs \).
\end{proof}
\sx{Not really right but keep this way for now}

\subsection{Completeness}
\sx{Define a function from a graph to a set of possible \( \fph \) }


%\newpage
%\section{Alternative Semantics\label{sec:alter}}

where,
\[
    \begin{rclarray}
        \rely_0 & \eqdef & \func{transitive\_close}{\relyU} \\
        \rely_{n+1} & \eqdef & \func{transitive\_close}{ \pred{merge\_close}{\rely_{n}} } \\
        \func{merge\_close}{\rely} & \eqdef 
        & \Setcon{%
            (\fpw, \mergeFW{\fpw_{l}}{\fpw_{r}})
        }{%
            (\fpw, \fpw_{l}), (\fpw, \fpw_{r}) \in \rely 
        } \cup \rely \\
        \func{transitive_close\_close}{\rely} & \eqdef 
        & \Setcon{%
            (\fpw_{p}, \fpw_{q})
        }{%
            \exsts{ \fpw_{m}, \fpw_{m}', \fpw_{q}' } 
            (\fpw_{p}, \fpw_{m}), (\fpw_{m}', \fpw_{q}') \in \rely  \\
            \quad {} \land \eraseFW{\fpw_{m}} =  \eraseFW{\fpw_{m}'} 
        } \cup \rely
    \end{rclarray}
\]
\sx{The fingerprint of \( \fpw_{m} \) does NOT propagate to \( \fpw_{m}'\) hope to fix unsoundness
\[
\begin{array}{l}
1: (0,0,0) \leadsto (1,0,0) \\
2: (1,0,0) \leadsto (1,1,0) \\
3: (0,0,0) \leadsto (0,0,1) \\
4: (0,0,0) \leadsto (2,0,2) \\
\end{array}
\]
It is possible to have \( (0,0,0)  \leadsto (2,1,2) \), 
\[
\begin{array}{l}
1: |---| \\
2: \qqqquad |---| \\
3: \qquad |---| \\
4: \qqquad \qqqquad |---| \\
\end{array}
\]
While if the transitive closure propagate the fingerprint, there is no way to get that.
}

\subsection{Merge}

\begin{defn}[Fingerprint heaps merge]
\label{def:merge-finger-heap}
The \emph{merge of fingerprint heaps}, \( \mergeFPH{.}{.} \), is defined as follows:
\[
    \begin{rclarray}
        \mergeFPH{\fph_{l}}{\fph_{r}}  & \defeq & \lambda \addr \ldotp 
            \begin{cases}
                \fph_{l}(\addr) & a \in \dom(\fph_{l}) \setminus \dom(\fph_{r})  \\
                \fph_{r}(\addr) & a \in \dom(\fph_{r}) \setminus \dom(\fph_{l}) \\
                \mergeVAL{\fph_{l}(\addr)}{\fph_{r}(\addr)}  & a \in \dom(\fph_{l}) \cap \dom(\fph_{r}) \\
            \end{cases}
    \end{rclarray}
\]
where the \emph{merge of fingerprint heap values} is defined:
\[ \begin{rclarray}
        \mergeVAL{(\val_{l}, \fp_{l})}{(\val_{r}, \fp_{r})} & \defeq &
            \begin{cases}
                (\val_{l}, \fp_{l} \cup \fp_{r} ) & \val_{l} = \val_{r} \land \fpW \notin \fp_{l} \cup \fp_{r} \\
                (\val_{l}, \fp_{l} \cup \fp_{r} ) & \fpW \in \fp_{l} \land \fpW \notin \fp_{r} \\
                (\val_{r}, \fp_{l} \cup \fp_{r} ) & \fpW \notin \fp_{l} \land \fpW \in \fp_{r} \\
            \end{cases}
    \end{rclarray}
\]
\end{defn}

\begin{definition}[Fingerprint worlds]
\label{def:fingerprint_worlds}
Given the set of fingerprint heap $\FPHeaps$ (\defref{def:fingerprint_heaps}) and the set of region identifiers $\RegionID$ (\defin\ref{def:capabilities}), the set of \emph{fingerprint worlds}, $\fpw \in \FPWorlds$, is defined as follows:
\[
\begin{rclarray}
	(\ca, \fph) \in \FPWorlds  & \eqdef & \Caps \times \FPHeaps
\end{rclarray}
\]
with the composition function \( \composeFPW \eqdef (\composeC, \composeEq) \) and unit \( \unitFPW  \defeq \Setcon{(\ca, \emptyset)}{\ca \in \unitC }\).
The \emph{erase function}, $\eraseFW{.}: \FPWorlds \rightarrow \World$, is defined as follows:
\[
\begin{rclarray}
	\eraseFW{(\ca, \fph)} & \eqdef & (\ca, \eraseFPH{\fph})
\end{rclarray}
\]
where,
\[
\begin{rclarray}
	\eraseFPH{\fph} & \defeq & \lambda \addr \ldotp \fphVAL(\addr)
\end{rclarray}
\]
The \( \predn{no\_fingerprint} \) is defined as the follows:
\[
\begin{rclarray}
    \pred{no\_fingerprint}{ \ca, \fph } & \defeq & \for{ \addr } \fphFP(\addr) = \emptyset
\end{rclarray}
\]
\end{definition}

\begin{definition}[Fingerprint worlds merge]
Given the set of fingerprint worlds $\FPWorlds$ (\defin\ref{def:fingerprint_worlds}), the \emph{merge} function, $\mergeName[\fpw]: \FPWorlds \times \FPWorlds \parfun \FPWorlds$, is defined as follows, for all $\fpw_{l},\fpw_{r} \in \FPWorlds$:
\[
    \begin{rclarray}
	\mergeFW{(\ca, \fph_{l})}{(\stub, \fph_{r})} & \eqdef & (\ca, \mergeFPH{\fph_{l}}{\fph_{r}}) 
    \end{rclarray}
\]
\end{definition}

Note that the \( \mergeName[\fpw] \) is not commutative, i.e.\ swapping \( \fpw_{l}\) and \( \fpw_{r}\) might yield different result.

\subsection{Timestamp}

We model the global database state as a \emph{timestamp heap}. A timestamp heap is a partial function from addresses to \emph{histories}.
A history is a partial function from timestamps to a set of \emph{events}.
An event is a triple comprising the value read or written, the identifier of the transaction carrying out the event, and an \emph{event tag} denoting a \emph{start event} ($\etS$), an \emph{end} event ($\etE$), a \emph{read} event ($\etR$) or a \emph{write} event ($\etW$).
We model our timestamps, $\ts \in \Timestamp$, as elements of an (uncountably) infinite set with a total order relation $<$. To ensure the availability of an appropriate timestamp, we assume that the timestamp set $\Timestamp$ is \emph{dense}. That is, given any two timestamps $\ts_1$ and $\ts_2$ such that $\ts_1 < \ts_2$, an intermediate timestamp $\ts$ can be found such that $\ts_1 < \ts < \ts_2$.

\begin{defn}[Event tags]
\label{ref:event-tag}
The set of \emph{event tags} is $\etag \in \ETags \eqdef \{\etS, \etE, \etR, \etW \}$.
\end{defn}
%
\begin{defn}[Timestamps]
\label{def:timestamp}
The set of \emph{timestamps}, $\Timestamp$, is a infinitely countable set, and there is a \emph{total order} relation on it, $<\ : \Timestamp \times \Timestamp$, such that:
%
\[
	\for{\ts_1, \ts_2 \in \Timestamp} \ts_1 < \ts_2 \lor \ts_2 < \ts_1
\]
%
This timestamp set $\Timestamp$ is \emph{dense} with respect to the ordering relation $<$. 
That is,
\[
	\for{\ts_1, \ts_2} \ts_1 < \ts_2 
	\implies 
	\exsts{\ts} \ts_1 < \ts < \ts_2	
\]
\end{defn}
%
We write $\ts_1 \leq \ts_2$ as a shorthand for $\ts_1 < \ts_2 \lor \ts_1 = \ts_2$.
We write $\ts_1 \oplus \ts \odot \ts_2 $ for $\ts_1 \oplus \ts \land \ts \odot \ts_2 $, where $\oplus, \odot \in \{<, \leq\}$.
We often use the mirrored symbols and write $\ts_1 > \ts_2$ (resp.~$\ts_1 \geq \ts_2$) for $\ts_2 < \ts_1$ (resp.~$\ts_2 \leq \ts_1$).
Lastly, we use the standard interval notation and write:
%
\[
\begin{rclarray}
	(\ts_1, \ts_2) & \text{for} & \Set{\ts \mid \ts_1 < \ts < \ts_2 } \\
	(\ts_1, \ts_2] & \text{for} & \Set{\ts \mid \ts_1 < \ts \leq \ts_2 } \\
	{[\ts_1, \ts_2)} & \text{for} & \Set{\ts \mid \ts_1 \leq \ts < \ts_2 } \\	
	{[\ts_1, \ts_2]} & \text{for} & \Set{\ts \mid \ts_1 \leq \ts \leq \ts_2 } 
\end{rclarray}
\]
\begin{defn}[Timestamp heaps]
\label{def:timestamp_heaps}
Assume a countably infinite set of \emph{transaction identifiers} $\tsid, \beta \in \TransID$.
Given the set of event tags $\ETags$ (\defin\ref{ref:event-tag}), program values $\Val$ (\defin\ref{def:prgram_values}), program addresses $\Addr$ (\defin\ref{def:prgram_values}) and timestamps \(\Timestamp\) (\defin\ref{def:timestamp}), the set of \emph{timestamp heaps} is defined as $\tshp \in \TSHeaps \eqdef \Addr \parfinfun (\Val \times \TransID \times \ETags)$.
The \emph{timestamp heap composition function}, $\composeTSH: \TSHeaps \times \TSHeaps \parfun \TSHeaps$, is defined as follows, for all $\addr \in \Addr$, where $\uplus$ denotes the standard disjoint function union:
%
\[
	(\tshp_1 \composeTSH \tshp_2)(\addr) \defeq 
	\begin{cases}
		\tshp_1(\addr) \uplus \tshp_2(\addr) & \text{if } \addr \in \dom(\tshp_1) \text{ and } \addr \in \dom(\tshp_2) \\
		\tshp_1(\addr) & \text{if } \addr \in \dom(\tshp_1) \text{ and } \addr \not\in \dom(\tshp_2) \\
		\tshp_2(\addr) & \text{if } \addr \not\in \dom(\tshp_1) \text{ and } \addr \in \dom(\tshp_2) \\
		\text{undefined} & \text{otherwise}
	\end{cases}
\]
%
The \emph{timestamp heap unit element} is $\unitTSH \eqdef \emptyset$, denoting a function with an empty domain.
The \emph{partial commutative monoid of timestamp heaps} is $(\TSHeaps, \composeTSH, \{\unitTSH\})$. 
\end{defn}
 
Whilst the state of the database is given by globally-accessed (by all threads) timestamp heaps, we use \emph{fingerprint heaps} to model a \emph{local snapshot} of the global timestamp heaps made available to transactions during their execution. 
In order to successfully commit a transaction and avoid conflicts, we must track the \emph{fingerprint} of a transaction, namely those addresses read from or written to by the transaction.
As such, we model fingerprint heaps as finite partial maps from addresses to pairs comprising a value and a fingerprint.
The fingerprint of an address may be 1) $\emptyset$ denoting that the address has not been touched;
2) $\Set{\fpR}$ denoting that the address has been read;
3) $\Set{\fpW}$ denoting that the address has been mutated (written to);
and 4) $\Set{\fpR, \fpW}$ denoting that the address has been initially read and subsequently mutated.   
%

\begin{defn}[Thread semantics]
Assume a countably infinite set of thread identifiers $\thid,j \in \ThreadID$.
The set of \emph{intermediate programs}, $\iprog \in \IntermediatePrograms$, is defined by the following grammar:
%
\[
    \iprog ::= \prog \mid \iprog \pseq \pwait{\thid}
\]
%
Given the set of stacks $\Stacks$ (\defin\ref{def:stacks}), timestamp heaps $\TSHeaps$ and timestamps $\Timestamp$ (\defin\ref{def:timestamp}), the \emph{per-thread operational semantics} of programs:
%
\[
	\toT{} : 
	\left((\Stacks \times \TSHeaps \times \Timestamp) \times \IntermediatePrograms \right) 
	\times \Translabel \times  
	\left((\Stacks \times \TSHeaps \times \Timestamp) \times \IntermediatePrograms \right) 
\]
%
is defined in \fig\ref{fig:thread_semantics}.
\end{defn}


\begin{figure}
%
\hrule\vspace{5pt}
%
\[
    \infer[\rl{Commit}]{%
        ( \stk, \tshp, \ts_{s} ) , \ptrans{\trans} \ \toT{\lbC{\tsid}} \ ( \stk', \tshp', \ts_{e} ) , \pskip
    }{%
        \begin{array}{c}
            \ts_s < \ts_e 
            \qquad \fph_s = \snapshot{\tshp}{\ts_s}
            \qquad ( \stk, \fph_s) , \trans \toL^{*} ( \stk', \fph_e) , \pskip \\
            \cancommit{\tshp}{\fph_e}{\ts_s}{\ts_e} 
            \quad \pred{fresh}{\tshp, \tsid}
            \quad \tshp' = \commit{\tshp}{\fph_s}{\fph_e}{\tsid}{\ts_s}{\ts_e}
        \end{array}
    }
\]

\[
    \infer[\rl{PChoiseL}]{%
        ( \stk, \tshp, \ts ) , \prog_{1} \pchoice \prog_{2} \ \toT{\lbID} \  ( \stk, \tshp, \ts ) , \prog_{1}
    }{%
    }
\]

\[
    \infer[\rl{PChoiseR}]{%
        ( \stk, \tshp, \ts ) , \prog_{1} \pchoice \prog_{2} \ \toT{\lbID} \  ( \stk, \tshp, \ts ) , \prog_{2}
    }{%
    }
\]

\[
    \infer[\rl{PLoop}]{%
        ( \stk, \tshp, \ts ) , \prog\prepeat \ \toT{\lbID} \  ( \stk, \tshp, \ts ) , \pskip \pchoice (\prog \pseq \prog\prepeat)
    }{%
    }
\]

\[
    \infer[\rl{PSeqSkip}]{%
        ( \stk, \tshp, \ts ) , \pskip \pseq \iprog \ \toT{\lbID} \  ( \stk, \tshp, \ts ) , \iprog
    }{%
    }
\]

\[
    \infer[\rl{PSeq}]{%
        ( \stk, \tshp, \ts ) , \iprog_{1} \pseq \iprog_{2} \ \toT{\lb} \ ( \stk', \tshp', \ts' ) , {\iprog_{1}}' \pseq \iprog_{2}
    }{%
        ( \stk, \tshp, \ts ) , \iprog_{1} \ \toT{\lb} \  ( \stk', \tshp', \ts' ) , {\iprog_{1}}' 
    }
\]

\[
    \infer[\rl{PPar}]{%
        ( \stk, \tshp, \ts ) , \prog_{1} \ppar \prog_{2} \ \toT{\lbF{\thid, \prog_{2}}} \  ( \stk, \tshp, \ts ) , \prog_{1} \pseq \pwait{\thid}
    }{%
    }
\]

\[
    \infer[\rl{PWait}]{%
        ( \stk, \tshp, \ts ) , \pwait{\thid} \ \toT{\lbJ{\thid,\ts'}} \  ( \stk, \tshp , \max\Set{\ts,\ts'} ) , \pskip 
    }{%
    }
\]

\[
    \infer[\rl{PIncTime}]{%
        ( \stk, \tshp, \ts ) , \iprog \ \toT{\lbID} \  ( \stk, \tshp , \ts' ) , \iprog 
    }{%
       \ts' > \ts
    }
\]
 
where,
\[
\begin{rclarray}
	\snapshot{.}{.} & : & \TSHeaps \times \Timestamp \parfun \FPHeaps \\
%	
	\snapshot{\tshp}{\ts} & \defeq & 
	\lambda \addr \ldotp
	\begin{cases} 
		(\val, \emptyset) & 
		\exsts{\ts' \leq \ts} \tshp(\addr)(\ts') {=} (\val,\etW, \stub) \\
		& \quad {} \land \for{\ts'' \in ( \ts', \ts), \etag} \tshp(\addr)(\ts'') {=} (\stub,\etag,\stub) \implies \etag \in \Set{\etR, \etE}\\
        \text{undefined} & \text{otherwise}
	\end{cases} 
	\vspace{3pt}\\
	\commit{.}{.}{.}{.}{.}{.} & : & \TSHeaps \times \FPHeaps \times \FPHeaps \times \TransID \times \Timestamp \times \Timestamp \to \TSHeaps \\
%	
	\commit{\tshp}{\fph_s}{\fph_e}{\tsid}{\ts_s}{\ts_e} & \defeq &
	\lambda \addr \ldotp
	\begin{cases}
		\tshp(\addr)\remapsto{\ts_s}{(\fphVAL[\fph_s](\addr),\etS,\tsid)}\remapsto{\ts_e}{(\fphVAL[\fph_e](\addr),\etW,\tsid)} 
		& \text{if \;} \fphFP[\fph_e](\addr) {=} \{\fpW\} \\
		\tshp(\addr)\remapsto{\ts_s}{(\fphVAL[\fph_s](\addr),\etR,\tsid)}\remapsto{\ts_e}{(\fphVAL[\fph_e](\addr),\etE,\tsid)} 
		& \text{if \;} \fphFP[\fph_e](\addr) {=} \{\fpR\} \\
		\tshp(\addr)\remapsto{\ts_s}{(\fphVAL[\fph_s](\addr),\etR,\tsid)}\remapsto{\ts_e}{(\fphVAL[\fph_e](\addr),\etW,\tsid)} & \text{if \;} \fphFP[\fph_e](\addr) {=} \{\fpR, \fpW\} \\
		\tshp(\addr)\remapsto{\ts_s}{(\fphVAL[\fph_s](\addr),\etS,\tsid)}\remapsto{\ts_e}{(\fphVAL[\fph_e](\addr),\etE,\tsid)} & \text{otherwise}
	\end{cases} 
	\vspace{3pt}\\
%
%              
	\cancommit{\tshp}{\fph}{\ts_s}{\ts_e} 
	& \defeq & 
	\wfhist{\tshp}{\fph}{\ts_s}{\ts_e} \land \consistent{\tshp}{\fph}{\ts_s}{\ts_e} 
	\vspace{3pt}\\
%
%        
	\wfhist{\tshp}{\fph}{\ts_s}{\ts_e} 
	& \defeq  & 
 	\for{\addr} \for{\ts \in \Set{\ts_{s},\ts_{e}}} \fphFP(\addr) {\ne} \emptyset \implies \tshp(\addr)(\ts)\isundef 
	\vspace{3pt}\\
%
%        
	\consistent{\tshp}{\fph}{\ts_s}{\ts_e}
	& \defeq & 
	\for{\addr} \fpW \in \fphFP(\addr) \implies \\
	&& \quad \for{\ts \in (\ts_s,\ts_e)} \tshp(\addr)(\ts) {\ne} (\stub, \etW, \stub) \\
	&& \quad \land\ \neg\exsts{\tsid, \ts'_s, \ts'_e} \ts'_s < \ts_e \land
       \ts'_e > \ts_e \land \tshp(\addr)(\ts'_s) = (\stub, \stub, \tsid) \land 
       \tshp(\addr)(\ts'_e) = (\stub, \etW, \tsid) \\
	&& \quad \land\ \for{\ts_m} \ts_m {=} \min(\Setcon{\ts'' }{ \ts'' > \ts_{e} \land
       \tshp(\addr)(\ts'')\isdef}) \implies \tshp(\addr)(\ts_{min}) \neq (\stub, \etR, \stub) 
	\vspace{3pt}\\
%
%        
	\pred{fresh}{\tshp, \tsid}  & \defeq & \neg\exsts{\addr, \ts} \tshp(\addr)(\ts) {=} (\stub, \stub, \tsid)
    \end{rclarray}
\]
\hrule\vspace{5pt}
\caption{Per-thread operational semantics}
\end{figure}


In order to formulate the operational semantics of a program $\prog$, we extend the programming language with an auxiliary wait construct, \(\pwait{\thid} \), added to programs as a suffix to denote thread joining points.
Intuitively, the \( \pwait{\thid} \) construct indicates that the current thread is waiting on the thread identified by \( \thid \) to finish its execution and join the current thread. We refer to the programs produced by this extended syntax as \emph{intermediate programs}. This is because the $\pwait{.}$ construct yields additional programs that cannot be written by the clients of the database and is merely used to capture intermediate steps during parallel execution. 
%

We define the per-thread operational semantics of programs with respect to a triple of the form $(\stk, \tshp, \ts)$ comprising a (locally-accessed) stack, a (globally-accessed) timestamp heap, and a (locally-recorded) timestamp. 
Each step of the per-thread operational semantics is decorated with a \emph{label} recording the action taken by the thread. In particular, a label may be $\lbID$, denoting an identity transition; $\lbC{\tsid}$, denoting the committing of transaction $\tsid$; $\lbF{\thid, \prog}$, denoting the forking a new thread $\thid$ the execute program $\prog$; or $\lbJ{\thid, \ts}$, denoting the joining of thread $\thid$ with the local timestamp $\ts$.
  
The per-thread operational semantics of programs is given in \fig\ref{fig:thread_semantics}.
With the exception of the \rl{PCommit}, \rl{PPar} and \rl{PWait}, the remaining rules are straightforward.



\begin{defn}[Fingerprints]
\label{def:fingerprint}
The set of \emph{fingerprints} is $\fp \in \Fingerprint \eqdef \powerset{\Set{\fpR,\fpW}}$.
\end{defn}
 
\begin{defn}[Fingerprint heaps]
\label{def:fingerprint_heaps}
Given the sets of program values $\Val$ (\defin\ref{def:prgram_values}), addresses $\Addr$ (\defin\ref{def:timestamp_heaps}) and fingerprints $\Fingerprint$ (\defin\ref{def:fingerprint}), the set of \emph{fingerprint heaps} is: $\fph \in \FPHeaps \eqdef \Addr \parfinfun (\Val \times \Fingerprint)$.
The \emph{fingerprint heap composition function}, $\composeFPH: \FPHeaps \times \FPHeaps \parfun \FPHeaps$, is defined as $\composeFPH \eqdef \uplus$, where $\uplus$ denotes the standard disjoint function union. The \emph{fingerprint heap unit element} is $\unitFPH \eqdef \emptyset$, denoting a function with an empty domain.
The \emph{partial commutative monoid of fingerprint heaps} is $(\FPHeaps, \composeFPH, \{\unitFPH\})$.  
\end{defn}
 
Given a fingerprint heap $\fph$ and an address $\addr$, we write $\fphVAL(\addr)$ and $\fphFP(\addr)$ for the first and second projections of $\fph(\addr)$, respectively. We write $\unitFPH$ for a fingerprint heap with an empty domain. We write $\fph_1 \uplus \fph_2$ for the standard disjoint function union of $\fph_1$ and $\fph_2$. 
%\ac{Why not define $\fphVAL$ and $\fphFP$ directly as functions from addresses to fingerprints and values directly, respectively?}

We introduce two update functions on fingerprints, $\addFPR{\fp}$ and $\addFPW{\fp}$, for updating a fingerprint $\fp$. Intuitively, the $\addFPW{\fp}$ always \emph{extends} $\fp$ with the $\fpW$ fingerprint. On the other hand, the $\addFPR{\fp}$ extends $\fp$ with $\fpR$ \emph{only if} $\fp$ does not already contain the $\fpW$ fingerprint (i.e.~$\fpW \not\in \fp$). This is to capture the fact that once an address is written to and thus the fingerprint contains $\fpW$, the following reads from the same address are considered local and need not be recorded in the fingerprint.
 
\begin{defn}[Fingerprint extension]
\label{def:fingerprint-extension}
Given the set of fingerprints $\Fingerprint$ (\defin\ref{def:fingerprint}), the \emph{read fingerprint extension} function, $\addFPR{(.)}{.}: \Fingerprint \rightarrow \Fingerprint$, and the \emph{write fingerprint extension} function, $\addFPW{(.)}{.}: \Fingerprint \rightarrow \Fingerprint$, are defined as follows, for all $\fp \in \Fingerprint$:
\[
\begin{rclarray}
	\addFPW{\fp} & \eqdef & \fp \cup \{\fpW\} \\
	\addFPR{\fp} & \eqdef &
	\begin{cases}
		\fp \cup \{\fpR\}  & \text{if } \fpW \not\in \fp \\
		\fp & \text{otherwise}
	\end{cases}
\end{rclarray}	
\]
\end{defn}
%\ac{Can you express heap composition, read and write extension of fingerprints under a single operator?}

\begin{defn}[Read and write sets]
\label{def:rs-ws}
Given the set of fingerprints $\Fingerprint$ (\defin\ref{def:fingerprint}), the \emph{read set} and \emph{write set} function, $\rsName, \wsName: \FPHeaps \to \powerset{\Addr}$ is defined as follows:
\[
\begin{rclarray}
    \ws{\fph} & \defeq & \myset{\loc}{ \exsts{\fp} \fph(\loc) = (\stub, \fp) \land \fpW \in \fp} \\
    \rs{\fph} & \defeq & \myset{\loc}{ \exsts{\fp} \fph(\loc) = (\stub, \fp) \land \fpR \in \fp} \\
\end{rclarray}	
\]
\end{defn}


\subsection{Weaken Atomic}

We reuse some notations in Sect. \ref{sec:semantics}, but redefined the meaning.
The \( \TSHeaps \) now is a partial function from locations to a tuple of \( \Timestamp \) and \( \Val \).
The thread state now is only a local stack and a global stack, and correspondingly, the join label is only parametrised by thread identifier.

\[
    \begin{rclarray}
        \tshp \in \TSHeaps & \defeq & \Loc \parfinfun ( \Timestamp \times  \Val ) \\
        \lb \in \Translabel & \defeq & 
              \lbID \quad               |
        \quad \lbS{\tsid} \quad         |
 t      \quad \lbC{\tsid} \quad        |
        \quad \lbF{\thid,\prog} \quad |
        \quad \lbJ{\thid} \\
    \end{rclarray}
\]

The notation \( \ptrans{\trans}_{\tsid, \tshp} \) indicates that the transaction identifier by \( \tsid \) has been started with the snapshot of heap \( \tshp \), but does not commit so far.

\[
    \begin{rclarray}
        \iprog & ::= &  \prog \mid \mid \ptrans{\trans}_{\tsid, \tshp} \pseq \iprog \mid \iprog \pseq \pwait{\thid} 
    \end{rclarray}
\]

The main difference of this semantics is, except removing all the local time, that the one-step \rl{Commit} is split into few steps.
Here we assume there is a global functions \(\funcn{freshTransId} \) that return a new transaction identifier each time.

\[
    \infer[\rl{TakeSnapshot}]{%
        ( \stk, \tshp ) , \ptrans{\trans} \ \toT{\lbS{\tsid}} \ ( \stk, \tshp ) , \ptrans{\trans}_{\tsid, \tshp}
    }{%
        \begin{array}{c}
            \quad \tsid = \func{freshTransId}{}
        \end{array}
    }
\]

We also redefine some functions and predicates used before.

\[
    \begin{rclarray}
        \func{startstate}{\tshp} & \defeq & \lambda \loc \ldotp (\val, \emptyset) \ \texttt{where} \ \tshp(\loc) = (\stub, \val)\\
        \pred{allowcommit}{\tshp,\tshp', \fph} & \defeq & \forall \loc \ldotp \fpW \in \fphFP[\fph](\loc) \land \tshp(\loc) = \tshp'(\loc)\\
        \func{commitTrans}{\tshp,\fph_{e}} & \defeq &
        \lambda \loc \ldotp
        \begin{cases}
            ( \tshp(\loc)\projection{1}+1, \fphVAL[\fph_{e}](\loc)) & \fpW \in \fphFP[\fph_{e}](\loc)  \\
            \tshp(\loc) & \text{otherwise}
        \end{cases} \\
    \end{rclarray}
\]

\[
    \infer[\rl{Commit}]{%
        ( \stk, \tshp ) , \ptrans{\trans}_{\tsid, \tshp'} \ \toT{\lbC{\tsid}} \ ( \stk', \tshp'' ) , \pskip
    }{%
        \begin{array}{c}
            \quad \fph_{s} = \func{startstate}{\tshp'}
            \quad ( \stk, \fph_{s}) , \trans \toL^{*} ( \stk', \fph_{e}) , \pskip \\
            \pred{allowcommit}{\tshp,\tshp', \fph}
            \quad \tshp'' = \func{commitTrans}{\tshp,\fph_{e}}
        \end{array}
    }
\]

\end{document}
