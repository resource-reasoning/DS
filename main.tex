\documentclass[conference,compsoc]{IEEEtran}

%%%%%%%%%%%%%%%%%%%%%% INCLUDING %%%%%%%%%%%%%%%%%%%%%%%%%
% to switch including packages in env.tex
% all packages
%**********************************************************************************************************************************
% Note: this file is shared between several documents, please do not delete any macros.
% Maintained by Shale Xiong <sx14@ic.ac.uk>
%**********************************************************************************************************************************

\ifNonESOPMode
%
% for theorem, proof, etc.
%\usepackage{amsthm} 
%
\theoremstyle{definition}
%\newtheorem{definition}[thm]{Definition}
%\newtheorem{lemma}[thm]{Lemma}
%\newtheorem{proposition}[thm]{Proposition}
%\newtheorem{example}[thm]{Example}
\newtheorem{definition}{Definition}[section]
\newtheorem{theorem}{Theorem}[section]
\newtheorem{lemma}{Lemma}[section]
\newtheorem{proposition}{Proposition}[section]
\newtheorem{example}{Example}[section]
%
%
\usepackage{titlesec}
%
\usepackage[titletoc]{appendix}
%
%caption margin
\usepackage[margin=2cm]{caption}
%
\usepackage{bold-extra}
%
\else
%
% to fix the proof without QED from llncs
\let\proof\relax\let\endproof\relax
\usepackage{amsthm}
\fi

% typesetting enum, etc.
\usepackage{enumerate}
\usepackage[inline]{enumitem}

% For url 
\usepackage{url}

% for color 
\usepackage[usenames,dvipsnames,svgnames,table]{xcolor}

\usepackage{hyperref}
\hypersetup{
    colorlinks,
    citecolor=black,
    filecolor=black,
    linkcolor=black,
    urlcolor=black
}


% clearly for the colour needed everywhere
\usepackage{color}

% for bibliography
\usepackage[numbers,sort]{natbib}

% for the line number at the edge
\usepackage{lineno}

% for equation number
\makeatletter
\@addtoreset{equation}{section}
\makeatother
\renewcommand{\theequation}{\arabic{section}.\arabic{equation}}

% For icons
\usepackage{fontawesome}

\usepackage{dsfont}
\usepackage{amsmath}

% the inter command for operational semantices
\usepackage{proof}
%%%%%%%%%%%%%%%%% submission version start from the follows %%%%%%%%%%%%%%%%%%%%%%%


% for xspace comment used in macro
\usepackage{xspace}


\usepackage{centernot}

% sub-figure
\usepackage{subcaption}
\captionsetup{compatibility=false}

% For math font and some commands
\usepackage{amssymb,stmaryrd}
\expandafter\def\csname opt@stmaryrd.sty\endcsname
{only,shortleftarrow,shortrightarrow}
\usepackage{extpfeil}

% For the box assertion
\usepackage{varwidth}

% tikz
\usepackage{tikz}
\usetikzlibrary{positioning, shapes, decorations.pathmorphing, arrows, calc,fit,matrix}
\usepackage{tikz-cd}

% for code 
\usepackage{listings}
\lstset{%
    basicstyle=\footnotesize\ttfamily,
    breaklines=true,
    numberstyle=\scriptsize,
    numbers=left,
    breakatwhitespace=false,
    escapeinside={(*}{*)},
    captionpos=b
}
\renewcommand{\lstlistingname}{Code}

% for the substitute E[e/x]
\usepackage{xfrac}

% better frac
\usepackage{nicefrac}

% For long table, tabularx
\usepackage{ltablex}

% For show the bib entry in the main body.
\usepackage{bibentry}

% box for math env
\usepackage{empheq}

% for better math typesetting 
\usepackage{mathtools}

% for resize the font 
\usepackage{relsize}

% For multirow and multicolumn in math and table
\usepackage{multirow}

% for the smile and sad face 
\usepackage{wasysym}

% typeset rules 
\usepackage{mathpartir}

% for scalebox
\usepackage{graphicx}

%For fig floating next to text
\usepackage{wrapfig}

% For font mathsfs
\usepackage{mathrsfs}  

% for anarchy symbol circledA
\usepackage{marvosym}


% xparse for more powerful macro definition
\usepackage{xparse}

%For reference 
\usepackage{cleveref}


% all macros
\usepackage{hhmacros}
% all tikz setting and macros
\pgfdeclarelayer{main}
\pgfdeclarelayer{background}
\pgfdeclarelayer{foreground}
\pgfsetlayers{background,main,foreground}

\newcommand{\greyness}{gray!40}
\newcommand{\blueness}{cyan!60}

\tikzstyle{background}=[rectangle, draw=black, inner sep=0.2cm, rounded corners=1.2mm]
\tikzstyle{white}=[rectangle, fill=white, inner sep=0.5cm, rounded corners=5mm]

%\tikzstyle{background}=[circle, fill=\greyness,
%                                                inner sep=0.2cm,
%                                                rounded corners=5mm,
%                                                decorate,
%                                                decoration={random steps,
%                                                            segment length=3pt,
%                                                            amplitude=3pt}]

%

 \tikzstyle{hheapcell}=[rectangle, draw=black, inner sep=0.1cm, font=\small]

\tikzstyle{noise}=[circle, thick, minimum size=1.2cm, draw=yellow!85!black, fill=yellow!40, decorate, decoration={random steps, segment length=2pt, amplitude=2pt}]

%\pgfdeclarelayer{background}
%\pgfdeclarelayer{foreground}
%\pgfsetlayers{background,main,foreground}

\tikzstyle{abstract}=[draw, fill=white, text width=5em, text centered, minimum height=2.5em, rounded corners]
    
\tikzstyle{arr}=[draw, ->, thick, color=black]
\tikzstyle{dasharr}=[draw,->,thick,dashed,color=black]

\tikzset{
    version node/.style={
        rectangle,
        draw=black,
        align=center,
        minimum height=5mm,
        text depth=0.5ex,
        text height=2ex,
        inner xsep=0pt,
        outer sep=0pt
        %,font=\footnotesize
    },      
    version list/.style={
        matrix of nodes,
        row sep=-\pgflinewidth,
        column sep=-\pgflinewidth,
        nodes={
            version node
        }
        ,
        execute at empty cell={\node[draw=none]{};},
        text width=5mm,
        anchor=west,
        ampersand replacement=\&
    }
}

\newcommand{\tikzvalue}[4]{
    \node[version node, fit=(#1) (#2), fill=white, inner sep=0pt] (#3) {#4}
}
\newcommand{\tikzkvspace}{1.5pt}
\newcommand{\tikzkeyspace}{-1.1}
\newenvironment{halfsubfig}{%
    \begin{subfigure}{0.45\textwidth}
}{%
    \end{subfigure}
}
\newenvironment{onethirdsubfig}{%
    \begin{subfigure}{0.3\textwidth}
}{%
    \end{subfigure}
}
\NewEnviron{centertikz}{%
    \begin{center}%
    \scalebox{.8}{%
    \begin{tikzpicture}[every node/.style={inner sep=0,outer sep=0},font=\large]%
    \BODY%
    \end{tikzpicture}%
    }%
    \end{center}%
}
%\newenvironment{centertikz}{%
    %\begin{center}%
    %\begin{tikzpicture}[every node/.style={inner sep=0,outer sep=0},font=\large]%
%}{%
    %\end{tikzpicture}%
    %\end{center}%
%}

\newcommand{\RootPath}{.}
%%%%%%%%%%%%%%%%%%%%%% END INCLUDING %%%%%%%%%%%%%%%%%%%%%%%%%

%%%%%%%%%%%%%%%%%%%%%% EDIT OPTIONS %%%%%%%%%%%%%%%%%%%%%%%%%
\newif\ifCommentEdits
%\CommentEditstrue
\CommentEditsfalse
\newcommand{\pg}[1]{%
\ifCommentEdits
    \begin{center}
    \fbox{%
    \begin{minipage}{6.5in} \color{red}
    {\bf PG:} {\rm #1}
    \end{minipage}
    }
    \end{center}
\fi
}

\newcommand{\sx}[1]{%
\ifCommentEdits
    \begin{center}
    \fbox{%
    \begin{minipage}{0.9\textwidth} \color{blue}
    {\bf SX:} {\rm #1}
    \end{minipage}
    }
    \end{center}
\fi
}

\definecolor{darkred}{rgb}{0.5, 0, 0}
\newcommand{\azalea}[1]{%
\ifCommentEdits
    \begin{center}
    \fbox{%
    \begin{minipage}{0.9\textwidth} \color{darkred}
    {\bf AR:} {\rm #1}
    \end{minipage}
    }
    \end{center}
\fi
}

\definecolor{darkblue}{rgb}{0.3,0.0,0.7}
\newcommand{\ac}[1]{%
\ifCommentEdits
    \begin{center}
    \fbox{%
    \begin{minipage}{0.9\textwidth} \color{darkblue}
    {\bf AC:} {\rm #1}
    \end{minipage}
    }
    \end{center}
\fi
}

%%%%%%%%%%%%%%%%%%%%%% END EDIT OPTIONS %%%%%%%%%%%%%%%%%%%%%


\newif\ifDoubleColumn
\DoubleColumntrue


\ifCLASSOPTIONcompsoc
  \usepackage[nocompress]{cite}
\else
  \usepackage{cite}
\fi

%\renewcommand{\rmdefault}{ptm}
%\renewcommand{\baselinestretch}{0.9}

\begin{document}

%% Title information
\title{
	Data Consistency in Transactional Storage Systems: A Centralised Approach
    } 


\author{
	\IEEEauthorblockN{
		Shale Xiong$^1$
		\qqqquad Andrea Cerone$^1$
		\qqqquad Azalea Raad$^2$
		\qqqquad Philippa Gardner$^1$\vspace{10pt}
	}
	\IEEEauthorblockA{
		$\begin{array}{c @{\hspace{100pt}} c}
			^1\text{Imperial College London}
			& ^2 \text{MPI-SWS}\\
			\href{
				mailto:shale.xiong14@imperial.ac.uk, a.cerone@imperial.ac.uk, p.gardner@imperial.ac.uk
			}{
				\mathtt{\{shale.xiong14, a.cerone, p.gardner\}@imperial.ac.uk}
			}
			& \href{mailto:azalea@mpi-sws.org}{\mathtt{azalea@mpi{-}sws.org}}
		\end{array}
		$	
	}	
}

\maketitle

\begin{abstract}
    
We introduce an interleaving operational semantics for describing the
client-observable behaviour of atomic transactions on distributed
kv-stores. Our semantics builds on abstract states comprising
centralised, global key-value stores and partial client views.  We provide
operational definitions of consistency models for our key-value stores which
are shown to be equivalent to the well-known declarative definitions
of consistency model for execution graphs. We explore  two
immediate applications of our semantics: specific protocols of 
geo-replicated databases (e.g. COPS) and partitioned databases
(e.g. Clock-SI) can be shown to be correct for a specific consistency
model by embedding them in our centralised semantics; 
programs can be directly shown to have invariant properties such as 
robustness results for libraries  against a weak consistency model.

\end{abstract}

\section{Introduction}
Transactions are the \emph{de facto} synchronisation mechanism in modern distributed databases.
To achieve scalability and performance, distributed databases  
often use weak transactional consistency guarantees. 
Much work has been done to formalise the semantics of such consistency guarantees, both
declaratively and operationally.
On the declarative side, several \emph{general} formalisms have been proposed, 
such as dependency graphs~\cite{adya} and abstract executions~\cite{ev_transactions}, to provide a unified
semantics for formulating different consistency models.  
On the operational side, the semantics of \emph{specific} consistency models have
been captured using reference implementations~\cite{si,PSI,PSI-RA}. 
However, unlike declarative approaches, there has been
little work on \emph{general} operational semantics for describing a range
of consistency models, and no work on a general operational semantics
in which to both verify protocols of distributed databases and 
enable the program analysis of clients.

As discussed in \cref{sec:conclusions}, there are several formalisms for a general operational semantics.
%~\cite{sureshConcur,alonetogether,seebelieve}. 
%\citeauthor{alonetogether} 
\cite{alonetogether} propose an operational
semantics over a global, centralised store for reasoning about clients using a program logic; 
they can model several isolation levels, but they cannot
capture consistency models of distributed data-stores, e.g.  
parallel snapshot isolation (\PSI). 
%Moreover, they do not establish the equivalence of their definitions
%of consistency model
%with existing declarative definitions in the literature. 
%\citeauthor{sureshConcur} 
\cite{sureshConcur} propose an operational semantics over abstract executions, 
rather than a concrete centralised store. This semantics captures weaker consistency models
such as \(\PSI\) and has been used to prove the robustness of applications against
a given consistency model. However, although they focus on consistency models with atomic 
visibility (a transaction observes either all or none the updates of another transaction), 
their semantics allows the interleaving of operations in different transactions, resulting in an unnecessarily complicated model.
%However, this semantics cannot model client sessions.
\cite{seebelieve} provide a trace semantics over a global
centralised store, where the behaviour of clients is formalised by the   
observations they can make on the totally ordered history of states of the system, prior to executing a transaction. 
Their framework is tailored at proving the equivalence different specifications of consistency models, 
%but it does little towards proving the correctness of protocols employed by distributed databases. 
but its usefulness  for analysing client programs is not clear, 
%program analysis for clients; in particular, we believe that in their framework the latter task would be 
%difficult,
 given that observations made by clients involve information that is not generally 
available to real world client programs, such as the total order in which transactions commit.
%or to p
%They focus on proving implementations correct. They do not consider program analysis for clients;
%indeed we believe it would be difficult  given their choice to
%keep track of the entire system.

In this paper, we introduce a general operational semantics for describing the
client-observable behaviour of distributed {atomic} transactions
(\cref{sec:overview}, \cref{sec:model}), while successfully abstracting from the 
internal details of protocols of geo-replicated and partitioned databases. In our semantics 
transactions execute atomically, preventing the interleaving of the  
operations they perform: this makes it feasible both to prove interesting properties of client applications 
of the database, and to verify that a distributed protocol correctly implements a consistency model.
Our model comprises a global, centralised
key-value store (kv-store) with {\em multi-versioning},  and
{\em client views}
inspired by the C11 operational semantics
in~\cite{promises}; 
using these mechanisms,
we record all the versions written for each key, and  let clients see only a subset 
of such versions.
Similarly to \cite{seebelieve}, our  operational semantics  is parametric in the notion of {\em
  execution test},  determining if a client with a given view is
allowed to commit a transaction; however, our notion of execution test does not rely on the 
knowledge of the whole history of states preceding a transaction. Using this approach, 
%Different execution tests give rise
%to different consistency models in our semantics. In contrast with \cite{seebelieve}, 
%we restrict observations of clients  to the current state of the system: this leads to 
%execution tests being a more faithful abstraction of the behaviour of real distributed databases protocols.
  we  are able to capture  most of the well-known consistency models in a uniform way (\cref{sec:cm}) using kv-stores and views: e.g.,  causal consistency (\CC), \PSI, snapshot isolation (\SI) and serialisability (\SER); one of our contributions is the development of a general proof technique for proving the correspondence between 
  execution tests and axiomatic specifications of consistency models using abstract executions (\cref{sec:other_formalisms}), 
  which we successfully applied to all the consistency models we consider.
  
  Our framework is thought with \emph{Atomic visibility} 
%  (a transaction 
%  sees either none or all the effects of another transaction) 
  in mind, an in fact we cannot capture popular 
  consistency models such as \emph{Read Committed}. However, because our focus is on protocols and applications employed by distributed databases, 
  most of whose guarantee atomic visibility, we do not find this constraint to not be a severe limitation.
%We define these models using kv-stores and views, 
%provide a correspondence between our kv-stores and dependency graphs, 
%We introduce novel proof techniques for demonstrating that our
%definitions of consistency models 
%are equivalent to existing declarative definitions (\cref{sec:other_formalisms}).
%

We showcase  our semantics by verifying the correctness of two database protocols, 
COPS \cite{cops} and Clock-SI \cite{clocksi}, and by analysing the robustness of simple client applications: in 
particular, we prove the robustness of a single counter against \PSI, and the robustness of multiple counters in \SI  (\cref{sec:applications}). 
%
%For the former, we show that the COPS protocol of a 
%replicated database satisfies our definition of $\CC$ and that the Clock-SI protocol of a partitioned database satisfies our definition of $\SI$.  
%For the latter, we show the robustness of applications against our consistency models: 
%we prove that a transactional library comprising a single counter is robust against $\PSI$; 
%and  that the library with multiple counters is robust against $\SI$, but not $\PSI$.  
%To our knowledge, our robustness results are the first to take into account client sessions.
%Without sessions, multiple counters can be proved to be robust against \(\PSI\) using 
%the static analysis check from \cite{giovanni_concur16}. 
We remark here that we verify protocols and analyse clients in the \emph{same} operational
semantics. By contrast, in existing literature these two tasks are carried out in \emph{different} semantics: for example, protocols are verified using abstract executions;
clients are analysed using dependency graphs; and equivalence results are used to move between the two.

%\mypar{Outline}
%The remainder of this article is organised as follows.
%In \cref{sec:overview} we give an intuitive overview of our ideas. 
%In \cref{sec:model} we present our general operational semantics. 
%In \cref{sec:cm} we show how we encode consistency models in our semantics.
%In \cref{sec:other_formalisms} we relate our formalism to existing declarative formalisms.
%In \cref{sec:applications} we showcase the applications of our semantics.  
%In \cref{sec:conclusions} we discuss related work and conclude. 


\section{Overview}
\label{sec:overview}

We introduce our centralised operational semantics for describing the
client-observable behaviours of atomic transactions updating  distributed
kv-stores. We show that our interleaving semantics provides 
an abstract interface for both verifying distributed protocols 
and proving invariant properties of client programs.

\mypar{Example} We use a simple transactional library, \(\CodeFont{Counter}(\key)\), to
 introduce our operational semantics.  Clients of this counter library can manipulate the
value of counter \(\key\) via two transactional operations:
\( 
\ctrinc(\key) \defeq 
\begin{transaction}
\plookup{\pv{x}}{\key}; \ 
\pmutate{\key}{\pv{x}{+}1}
\end{transaction}
\)
and
\(
\ctrread(\key) \defeq
\begin{transaction}
\plookup{\pv{x}}{\key}
\end{transaction}
\).
The \( \plookup{\pv{x}}{\key} \) reads the value of \( \key \) in
local variable \( \vx \); and \( \pmutate{\key}{\pv{x}{+}1} \)
writes \( \pv{x}{+}1 \) to \( \key \).  The code of each
operation is wrapped in square brackets, denoting a transaction that 
executes \emph{atomically}.  

Consider a replicated database where a client only interacts with one
replica.
For such a database, the behaviour of the atomic transactions is subtle, 
depending heavily on the particular consistency model under consideration.  
Consider the client program $\prog_{\CodeFont{LU}}$ below:

\SpaceAboveMath
\[ 
\prog_{\CodeFont{LU}} \defeq \quad \cl_1 : \ctrinc(\key) \;|| \; \cl_2: \ctrinc(\key) 
\]
\SpaceBelowMath

where we assume that clients \( \cl_1 \) and \( \cl_2 \) work on
different replicas
and, for simplicity,  each replica has a kv-store with just one key $k$. 
Initially, key \(\key\) holds value \(0\) in all replicas.
Intuitively, as transactions are executed atomically, after both
calls to \(\ctrinc(\key)\) have terminated, the counter should hold value \(2\).
Indeed, this is the only outcome allowed under the 
{\em serialisability} (\(\SER\)) consistency model, 
where transactions appear to execute in a sequential order, one after another.
The implementation of \(\SER\) in distributed kv-stores is known to
come at a
significant performance cost. Implementers are, therefore,  content with
{\em weaker} consistency models
\cite{gdur,ramp,CORFU,tango,si,distrsi,clocksi,redblue,rola,cops,PSI-RA,NMSI,PSI,wren}. 
\polish{Less references.}
For example, if replicas provide no synchronisation mechanism for transactions,
it is possible for both clients to read the same initial value \(0\) for \(\key\) at their
distinct replicas, update it to \(1\), and eventually propagate their updates of \( \key \) to other replicas. 
Thus, both replicas remain unchanged with value  \(1\) for \(\key\).
This weak behaviour is known as the \emph{lost update} anomaly, which
is allowed under the \emph{causal consistency} ($\CC$) model,
but not under \emph{parallel snapshot isolation} ($\PSI$) and \emph{snapshot
  isolation} ($\SI$). 

\begin{figure*}[t]
\centering
\captionsetup[subfigure]{aboveskip=-5pt, belowskip=0pt}
\begin{tabularx}{\textwidth}{@{} c | c |  c | c  | X@{}}
\hline
\phantom{-}& \phantom{-}& \phantom{-}& \phantom{-}\\[-5pt]
\begin{subfigure}{0.16\textwidth}
\centering
\begin{centertikz}
%Location x
\node(locx) {$\key \mapsto$};
\draw pic at ([xshift=\tikzkvspace]locx.east) {vlist={versionx}{%
    /0/\txid_0/\emptyset
}};

\end{centertikz}
\caption{Initial state}
\label{fig:counter_kv_initial}
\end{subfigure}
&
\begin{subfigure}{0.16\textwidth}
\begin{centertikz}

%Location x
\node(locx) {$\key \mapsto$};
\draw pic at ([xshift=\tikzkvspace]locx.east) {vlist={versionx}{%
    /0/\txid_0/\Set{\txid_1}
    , /1/\txid_1 /\emptyset
}};

\end{centertikz}
\caption{the kv-store after \(\txid_1 \)}
\label{fig:counter_kv_first_inc}
\end{subfigure}
&
\begin{subfigure}{0.16\textwidth}
\begin{centertikz}

%Location x
\node(locx) {$\key \mapsto$};
\draw pic at ([xshift=\tikzkvspace]locx.east) {vlist={versionx}{%
    fillbg/0/\txid_0/\Set{\txid_1}
    , /1/\txid_1 /\emptyset
}};

\end{centertikz}
\caption{A view of \( \cl_2 \) with the initialisation version}
\label{fig:counter_kv_view}
\end{subfigure} 
&
\begin{subfigure}{0.16\textwidth}
\begin{centertikz}

%Location x
\node(locx) {$\key \mapsto$};
\draw pic at ([xshift=\tikzkvspace]locx.east) {vlist={versionx}{%
    fillbg/0/\txid_0/\Set{\txid_1 }
    , fillbg/1/\txid_1 /\emptyset
}};

\end{centertikz}
\caption{A view of \( \cl_2 \) with both versions}
\label{fig:counter_kv_view_all}
\end{subfigure} 
&

\begin{subfigure}{0.23\textwidth}
\begin{centertikz}
    
\node(locx) {$\key \mapsto$};
\draw pic at ([xshift=\tikzkvspace]locx.east) {vlist={versionx}{%
    /0/\txid_0/\Set{\txid_1 ,\txid_2}
    , /1/\txid_1 /\emptyset
    , /1/\txid_2 /\emptyset
}};

\end{centertikz}%
\caption{\emph{lost update}: given view in \cref{fig:counter_kv_view},
the kv-store after \( \txid_2 \)}
\label{fig:counter_kv_final}
%\label{fig:ua-disallowed}
\end{subfigure}\\
\hline
\end{tabularx}
\caption{Lost update anomaly: single counter.}
\end{figure*}


\mypar{Centralised Operational Semantics}
Our operational semantics provides  transitions over 
{abstract states},
comprising 
\begin{enumerate*}
	\item a centralised, multi-versioned {\em kv-store}, which is {\em global} in that  it records all the versions written by all its clients, 
	\item a \emph{client view}, which is {\em partial} in that it records only those versions in the kv-store observed by a client, and 
	\item a variable store local to the client. 
\end{enumerate*}
Each transition of our operational semantics either updates the
variable store using a primitive command, or updates the kv-store using an 
atomic transaction. 
The atomic transactions are subject to an {\em execution test}, which
analyses the state to determine if the associated update is allowed under  
the given consistency model. 


We show how the  lost update anomaly in
\(\prog_{\CodeFont{LU}}\) is represented by   our operational semantics.  
A centralised kv-store provides an abstraction of the real-world
replicated key-value store of our example.  It is a function mapping
keys to a {\em version} list, recording {all} the values written to the key
together with information about the transactions that
accessed it. The total order of versions on a key $k$ is always known
due to the resolution policy of the distributed database. \polish{Does
  it have a name?} 
In the \(\prog_{\CodeFont{LU}}\) example, our initial centralised
kv-store comprises a single key \(\key\)  with  one initialisation version \((0, \txid_{0}, \emptyset)\).
This version represents the initialisations in both replicas where \(\key\) holds value \(0\), 
 the version \emph{writer} is the initialising transaction
\(\txid_0\) (this version was written by \(\txid_0\)), 
and  the version \emph{reader set} is empty (no transaction has read this version). 
\Cref{fig:counter_kv_initial} depicts this initial centralised kv-store, with the version
represented as a box sub-divided in three sections: the value \(0\);
the writer \(t_0\); and the reader set \(\emptyset\). 

Suppose that \(\cl_1\) first invokes \(\ctrinc(\key)\) on
\cref{fig:counter_kv_initial}.
It does this by choosing a fresh transaction identifier \(\txid_1\), 
 then reading the initial version
of \(\key\) with value \(0\) 
and writing  a new value \(1\) for \(\key\). 
The resulting kv-store is depicted in \cref{fig:counter_kv_first_inc},
where  the initial version of \(\key\)  has been  updated to reflect that it
has been read by \(\txid_1 \) and a new version with value 1 is installed at
the end of the list. 
Now suppose that client \(\cl_2\) invokes \(\ctrinc(\key)\)  on
\cref{fig:counter_kv_first_inc}.  
As there are now two versions
available for \(\key\), we must determine the version from which
\(\cl_2\) fetches its value.
This is where the partial \emph{client view} comes into play.  Intuitively, a view of
client \(\cl_2\) comprises those versions in the kv-store that are
\emph{visible} to \(\cl_2\); that is, those that can be read by
\(\cl_2\).  If more than one version is visible, then the newest
(right-most) version is selected, modelling the \emph{last-write-wins}
resolution policy used by many distributed key-value stores.
In our example, there are two  candidate views for \(\cl_2\) when running
\(\ctrinc(\key)\) on \cref{fig:counter_kv_first_inc}: 
one containing
only the initial version of \(\key\) as depicted in \cref{fig:counter_kv_view}; and
the other containing both versions of \(\key\) as depicted in \cref{fig:counter_kv_view_all}%
\footnote{As we explain in \cref{sec:mkvs-view}, we always require
the  client view to include the initial version of each key.}.
Given the \(\cl_2\) view in \cref{fig:counter_kv_view},
client \(\cl_2\) chooses a fresh
transaction identifier \(t_2\), reads the initial value \(0\) and writes a
new version with value \(1\), as depicted in \cref{fig:counter_kv_final}. 
Such a kv-store does not contain a
version with value \(2\), despite two increments on \(\key\), producing
the lost update anomaly. 
Had we used the the \(\cl_2\) view in \cref{fig:counter_kv_view_all} instead,
client \(cl_2\) would have read the newest
value \(1\) and written a new version with value \(2\).

The lost update anomaly is allowed under the \(\CC\) consistency
model, 
and disallowed under 
\(\SER\), \(\SI\) and \(\PSI\).  To distinguish these cases, we
use an \emph{execution test} which directly restricts the updates that
are possible at the point where the transaction commits.  A simple  way of
doing this is to require that a client writing a transaction to
\(\key\) have a view containing  {\em all} versions of  \(\key\)
available in the
global state. This prevents the situation
where the view of $cl_2$ is that  given in \cref{fig:counter_kv_view}. 
This execution test corresponds to what is known in the 
literature as \emph{write-conflict freedom} \cite{framework-concur},
which ensures that at most one concurrent transaction can write to a key at any one time. 

The situation becomes more complicated when the library contains multiple counters
where each client can read and increment several counters in one session.
For instance, consider the following client program:

\SpaceAboveMath
\[
    \prog_{\CodeFont{LF}} \defeq 
    \begin{multlined}[t]
    \cl_1 : \ptrans{\plookup{\var}{\key_1} ; \pmutate{\key_1}{\var + 1 }} ; 
                \ptrans{\plookup{\var(y)}{\key_2} ; \pmutate{\key_2}{\var(y) + 1} }
        \\ || \ \cl_2: \ptrans{\plookup{\var}{\key_1} ; \plookup{\var(y)}{\key_2} }
                 || \ \cl_3:  \ptrans{\plookup{\var}{\key_1} ; \plookup{\var(y)}{\key_2} } .
    \end{multlined}
\]
\SpaceBelowMath[-9pt]

where,  for simplicity,
the  kv-store has just the keys $k_1$ and $k_2$ (\cref{fig:overview-sec-long-fork-init}).
Suppose that \(\cl_1\) executes both transactions first,  
writing $1$ to \(\key_1\) and \(\key_2\) using fresh transaction 
identifiers \( \txid_1 \) and \( \txid'_1 \), respectively. 
This results in \(\key_1\) and \(\key_2\) having two versions with
values \(0\) and \(1\) each, as illustrated in \cref{fig:overview-sec-long-fork}. 
Client \(\cl_2\) next executes its transaction, identified by \( \txid_2 \), using a view that 
contains both versions of \(\key_1\) but only the initial version of
\(\key_2\). This means that \(\cl_2\) reads \(1\) for \(\key_1\) and
\(0\) for \(\key_2\);
that is,  \(\cl_2\) observes the increment of \(\key_1\) happening before that of \(\key_2\). 
Symmetrically, \(\cl_3\) executes its transaction, identified by \( \txid_3
\), using a view that contains both versions for \(\key_2\)
but only the initial version of \(\key_1\). 
As such, \(\cl_3\) reads \(0\) for \(\key_1\) and \(1\) for
\(\key_2\);
that is, \(\cl_3\) observes the increment of \(\key_2\) happening before that of  \(\key_1\). 
This behaviour is known as the \emph{long fork} anomaly (\cref{fig:overview-sec-long-fork}). 

\begin{figure}

\begin{tabularx}{\textwidth}{@{} c | c @{} }
\hline
\phantom{-}& \phantom{-}\\[-5pt]
\begin{subfigure}{0.39\textwidth}
\centering
\scalebox{.8}{%
\begin{tikzpicture}%
\KVMapping{x}{\key_1}{
    /0/\txidinit/\emptyset
};
\KVMapping[x]{y}{\key_2}{
    /0/\txidinit/\emptyset
};
\end{tikzpicture}%
}
\caption{Initial kv-store}
\label{fig:overview-sec-long-fork-init}
\end{subfigure}
& \begin{subfigure}{0.58\textwidth}
\centering
\scalebox{.8}{%
\begin{tikzpicture}%
\KVMapping{x}{\key_1}{
    /0/\txidinit/\Set{\txid,\txid_3}
    , /1/\txid/\Set{\txid_2}
};
\KVMapping[x]{y}{\key_2}{
    /0/\txidinit/\Set{\txid',\txid_2}
    , /1/\txid'/\Set{\txid_3}
};
\end{tikzpicture}%
}
\caption{Transactions \( \txid_2 \) and \( \txid_3\)
            observe the update to \( \key_1 \) and \( \key_2 \) 
            in different order (\emph{long fork} anomaly)}
\label{fig:overview-sec-long-fork}
\end{subfigure}
\\
\hline
\end{tabularx} 

\begin{subfigure}{\textwidth}
\centering
\begin{tikzpicture}

\node at (0,0) (a) {\(\txidinit\)};
\node at (2,-0.5) (b) {\(\txid\)};
\node at (4.5,0) (c) {\(\txid_3\)};

\coordinate (a1) at ($(a) + (3,0)$);
\coordinate (b1) at ($(b) + (7,0)$);
\coordinate (c1) at ($(c) + (3,0)$);

\draw[|-,densely dashed] (a) -- ($(a)!0.5!(a1)$);
\draw[-|] ($(a)!0.5!(a1)$) -- (a1);
\draw[|-,densely dashed] (b) -- ($(b)!0.5!(b1)$);
\draw[-|] ($(b)!0.5!(b1)$) -- (b1);
\draw[|-] (c) -- ($(c)!0.5!(c1)$);
\draw[-|,densely dashed] ($(c)!0.5!(c1)$) -- (c1);

\end{tikzpicture}

\caption{An example of dependencies between transactions with respect to 
the time line of the starts and commits of these transactions 
(dashed line being able to stretched)}
\label{fig:overview-dependencies-time-line}

\end{subfigure}

\hrulefill


\caption{Long fork anomaly: multiple counters}
\label{fig:mult-counter}
\end{figure}


The long fork anomaly is disallowed under strong models 
such as \(\SER\) and \(\SI\), 
but is allowed under weaker models such as \(\PSI\) and \(\CC\).
To capture such consistency models and disallow the long fork anomaly 
of \(\prog_{\CodeFont{LF}}\), we must strengthen the execution test associated with the kv-store.
For \(\SER\), we simply strengthen the execution test by ensuring that a client can execute a transaction 
only if its view contains all versions available in the global state.
For \(\SI\), the execution test is more subtle, 
%recovering the order in which 
%updates of versions have been observed by different clients. 
%It ensures that 
requiring that a client view be a set of versions 
that is {\em closed} with respect to the commit order of transactions.
This means that if a client view includes a version written by a transaction \( \txid \),
then it must include all versions written by transactions that committed before \( \txid \).
Our kv-stores do not contain all the information about the commit order.
\polish{I'm not sure about the next sentence, how is it
  over-approximating?} 
However, we have enough information to over-approximate the relevant commit order between transactions:
\begin{enumerate*}
	\item if a transaction, \eg \( \txid_3 \) in \cref{fig:mult-counter},
reads a version written by another transaction, \eg \( \txid_0 \),
then it must start after the commit of the transaction that
wrote the version, \eg \( \txid_3 \) must start after the commit of  \( \txid_0 \)
(\cref{fig:overview-dependencies-time-line}); \polish{And $\txid_1$ must be
  after the commit of $\txid_0$, figure needs changing};
	\item if a transaction writes a newer version of a key, \eg \( \txid_1 \) for \( \key_1 \), 
then  it must commit after the transactions that wrote the previous versions of the key,\eg \( \txidinit \)  (\cref{fig:overview-dependencies-time-line}); and
	\item if a transaction reads an older version of a key, \eg \( \txid_3 \) for \( \key_1 \),
it must start before the commit of all transactions that write the newer versions of \( \key \), \eg \(
\txid_1 \) (\cref{fig:overview-dependencies-time-line}).
\end{enumerate*}

In \cref{sec:cm}, we formally define the execution tests associated with several consistency models on kv-stores and client views. 
In \cite{shale-phd}, we show the equivalence of our operational definitions of consistency
models and the existing declarative definitions based on abstract executions \cite{framework-concur},
and hence those based on dependency graphs \cite{adya}. 

\mypar{Verifying Implementation Protocols} 
The first application of our operational
semantics is to show that  implementation protocols  of distributed
key-value stores satisfy certain consistency models. We do this by
faithfully representing the implementation protocol using our centralised
operational semantics: our abstract states provide a faithful abstraction of replicated and partitioned
databases, and our execution tests provide a faithful abstraction of the synchronisation mechanisms 
enforced by these databases when committing a transaction. 
We verify the correctness of our representation 
using trace refinement. Thus, a distributed protocol
satisfies the particular consistency model associated with the
particular execution
test of our representation. 
We demonstrate that the COPS protocol \citep{cops} for implementing
a replicated database satisfies our definition of $\CC$
(reported in \cref{sec:verify-impl} and proved in \cite{shale-phd}), 
and the Clock-SI protocol \citep{clocksi} for implementing a
partitioned database satisfies our definition of $\SI$
(given in \cite{shale-phd}). Since our definitions of consistency model are equivalent to those
in the literature \cite{shale-phd}, we have demonstrated that COPS and Clock-SI satisfy
the accepted general definitions of the respective consistency models. This contrasts
with the previous results in~\citep{cops} and~\citep{clocksi} which
demonstrated that these protocols satisfy specific consistency models defined for those particular implementations.

\mypar{Proving Invariant Properties of Client Programs} 
The second application of our operational semantics is to prove
invariant properties for transactional libraries (\cref{sec:robustness}).
One well-known  property is \emph{robustness}.
A library is robust if,  for all its client programs \(\prog\) and all kv-stores $\kvs$, 
if $\kvs$ is obtained by executing \(\prog\) under a weak consistency
model, then $\kvs$ can also be obtained under \(\SER\);
that is, the library clients have no observable weak behaviours. 
We prove the robustness of the single
counter library against \(\PSI\), 
and the robustness of a multi-counter library and the  banking library of \citet{bank-example-wsi}
against \(\SI\).
We prove  the robustness results against \(\SI\)
by proving general invariants that guarantee robustness against  a
new model we propose, \( \WSI \), which lies between \(\PSI\)
and  $\SI$. 
As we discuss in \cref{sec:robustness}, although existing techniques in the literature can verify such robustness properties, they typically do so by examining \emph{full traces}.
By contrast, we establish invariant properties at each execution step of our operational semantics, thus allowing a simpler, more compositional proof. 


We also demonstrate the use of  our operational semantics to prove
{library-specific} invariant properties. 
In particular, we show that a lock library is correct against \( \PSI \), in that it satisfies the \emph{mutual exclusion guarantee}, 
even though it is not robust against  \( \PSI \). 
To do this, we  encode this  guarantee as an invariant of the lock
library, establishing  the invariant at each transition step of the
operational semantics. 
By contrast, establishing such library-specific properties using the existing techniques is more difficult. 
\polish{We need to be more precise with this last sentence: is this  about 18 and 24?}
This is because  existing techniques do not directly record the library \emph{state}; 
rather, they record full execution traces, making them less amenable for reasoning about such properties.

\section{Model}

\label{sec:2plMod}

In fact, under the \textsc{2pl} protocol, all accesses to the shared storage cells must be protected by a corresponding lock. This means that any transaction that needs to read or write a particular cell must be granted an appropriate lock for that same cell. The abstract locks we use, behave in a standard way: depending on the mode, they can have zero, one or more \textit{owners}, i.e. threads that have been granted access to the protected cells.

\begin{defn}
	(Lock modes).
	The set of \emph{lock modes}, $\mathsf{Lock}$, is ranged over by $\kappa, \kappa_1, \ldots, \kappa_n$ and defined as:
	\[
		\mathsf{Lock} \triangleq \{ \textsc{u}, \textsc{s}, \textsc{x} \}
	\]
	The associated strict total order, $>$, is defined in the following way:
	\[
		>\ \triangleq \{ (\textsc{x}, \textsc{u}), (\textsc{x}, \textsc{s}), (\textsc{s}, \textsc{u}) \}
	\]
	The total order $\geq$ on the set $\mathsf{Lock}$ is equivalent to $>$ unioned with all of the reflexive pairs.
\end{defn}
We informally refer to each of the lock mode entries as \textit{unlocked}, \textit{shared} and \textit{exclusive} respectively. This reflects the fact that either no transaction is accessing a cell, one or more transactions are allowed to read the cell's content or a single transaction has been given the permission to write to the cell.

The \textsc{2pl} protocol sets a precise constraint on the pattern of acquisition and release of locks. A transaction $\mathds{T}$'s lifecycle is clearly distinguished between two phases. It initially starts executing in the \textit{growing} phase ($\curlywedge$), where it is free to sequentially acquire locks  for any cell it needs. Once it releases one of the locks it is holding, the transaction enters the \textit{shrinking} phase ($\curlyvee$). Here $\mathds{T}$ is denied any new lock acquisitions while it is allowed to gradually release the locks that are still being held.
\begin{defn}
	(Locking phase).
	 The set of \emph{locking phases}, ranged over by $p$, is defined as:
	 \[
	 	\mathsf{Phase} \triangleq \{ \curlywedge, \curlyvee \}
	 \]
\end{defn}
We sometimes refer to the growing and shirinking phase using \textit{acquiring} and \textit{releasing} phase respectively.

At this point we have all of the ingredients to introduce the other main component of our model, the lock manager. This global structure records the status of all locks on storage cells.

\begin{defn}
	(Lock manager).
	A \emph{lock manager} is defined as a total function with a finite domain from storage keys to pairs of transaction identifiers and lock modes.
	\[
		\mathsf{LMan} \triangleq \mathsf{Key} \xrightarrow{\text{fin}} \mathcal{P}(\mathsf{Tid}) \times \mathsf{Lock}
	\]
	The $\mathsf{LMan}$ set is ranged over by $\Phi, \Phi_1, \ldots, \Phi_n$. In order to cope with the default state of locks associated to keys initially absent from the domain of a lock manager, a total function, $\hat{-}(-) : \mathsf{LMan} \times \mathsf{Key} \rightarrow \mathcal{P}(\mathsf{Tid}) \times \mathsf{Lock}$, is defined as:
	\[
		\hat{\Phi}(k)
			\triangleq
		\begin{cases}
			\Phi(k), & \text{if } k \in \pred{dom}{\Phi} \\
			(\emptyset, \textsc{u}), & \text{otherwise}
		\end{cases}
	\]	
\end{defn}
In fact, every cell starts in the unlocked state with no transaction owning it. We say that a lock manager is \textit{empty} and write it as $\Phi = \emptyset$, if and only if all of the keys in its domain are mapped to locks with no owners. Formally:
	\[
		\Phi = \emptyset \iff \forall k \in \pred{dom}{\Phi} \ldotp (\emptyset, \textsc{u}) = \Phi(k)
	\]

We keep track of each transaction's stack and locking phase inside of another global structure which we call the transactions state.
\begin{defn}
	(Transactions state).
	The \emph{state} of all transactions running as part of a program is defined as a total function with a finite domain:
	\[
		\mathsf{TState} \triangleq \mathsf{Tid} \rightharpoonup \mathsf{Stack} \times \mathsf{Phase}
	\]
	Elements of $\mathsf{TState}$ are ranged over by $S, S_1, \ldots, S_n$. Another total function is also defined, $\hat{-}(-) : \mathsf{TState} \times \mathsf{Tid} \rightarrow \mathsf{Stack} \times \mathsf{Phase}$. Such function overrides the original function lookup in order to cope with newly created transactions with an empty stack starting in the growing phase.
	\[
		\hat{S}(\iota)
			\triangleq
		\begin{cases}
			S(\iota), & \text{if } \iota \in \pred{dom}{S} \\
			(\emptyset, \curlywedge), & \text{otherwise}
		\end{cases}
	\]	
\end{defn}
\section{Consistency Models: Kv-stores}
\label{sec:cm}
We define what it means for a kv-store 
to be in a consistent state. Many different consistency models for
distributed databases have 
been proposed in the literature
\cite{principle-eventual-consistency,rola,redblue,PSI,si},
capturing different trade-offs 
between  performance and application
correctness: examples range from  \emph{serialisability}, a strong
consistency model which only allows kv-stores 
obtained  from a serial execution of transactions
with inevitable performance drawbacks, to  \emph{eventual consistency},  a weak consistency model
which imposes few conditions on the structure of a kv-store leading to
good performance but anomalous behaviour.
We define consistency models for our kv-stores,
by introducing the notion of 
\emph{execution test} which specifies  whether a client is allowed to commit a transaction in a given 
kv-store. Each execution test induces a consistency model as the set of kv-stores obtained 
by having clients non-deterministically commit transactions so long as  the constraints 
imposed by the execution test are satisfied.
We explore a range of execution tests  associated with well-known consistency models in the literature. 
In \cref{sec:other_formalisms,app:et-sound-complete},  we demonstrate that our operational
formulation of  consistency models over kv-stores using execution
tests are  equivalent to the established declarative definitions of
consistency models over abstract executions \cite{ev_transactions,framework-concur}.

\spaceshrink{-3pt}
\begin{definition}[Execution tests]
\label{def:execution.test}
An \emph{execution test} is a set of tuples
\(\ET \subseteq \MKVSs {\times} \Views {\times} \Fingerprints {\times} \MKVSs {\times} \Views\) 
such that, for all \((\mkvs, \vi, \fp, \mkvs', \vi') \in \ET\): 
\begin{enumerate*}
	\item \(\vi {\in}  \allowbreak \Views(\mkvs)\) and \(\vi' \in \Views(\mkvs')\); 
	\item \(\cancommit \mkvs \vi \fp\); 
	\item \(\vshift \mkvs {\allowbreak \vi} {\mkvs'} {\vi'}\); and 
	\item for all \(\key \!\in\! \mkvs\) and \(\val \!\in\! \Val\), if \((\otR, \key, \val) \!\in\! \fp \) then \(	\mkvs(\key, \max{}_{<}(\vi(\key))) {=} \val   \).
\end{enumerate*}
\end{definition}
\spaceshrink{-3pt}

%
\noindent 
Intuitively, \((\mkvs, \vi, \fp, \mkvs', \vi') \in \ET\) means that, under the execution test \(\ET\),
a client with initial view \(\vi\) over kv-store \(\mkvs\) can commit a transaction with 
fingerprint \(\fp\) to obtain the resulting kv-store \(\mkvs'\) (\cref{def:updatekv}) while shifting its view
to \(\vi'\). We adopt the notation \(\ET \vdash (\mkvs, \vi) \csat \fp: (\mkvs', \vi')\) to capture this intuition. 
Note that the last condition in \cref{def:execution.test} enforces the last-write-wins
policy~\cite{vogels:2009:ec:1435417.1435432}: 
a transaction always reads the most recent writes from the initial view \(\vi\).  

\spaceshrink{-5pt}
\begin{definition}[Consistency model]
\label{def:cm}
The \emph{consistency model} induced by an execution test \(\ET\) is defined as 
\(
\CMs(\ET) \defeq 
\Set{\mkvs}[ 
\exsts{\conf_0 \in \Confs_0,\prog}
\conf_0,\prog \toCMD{\stub}_{\et}^{*} (\mkvs, \stub),\stub
]
\).
\end{definition}
\spaceshrink{-5pt}

The largest execution test is denoted by \(\ET_{\top}\), where for all \(\mkvs, \mkvs', \vi, \vi, \fp\):%

\spaceshrink{-14pt}
{\displaymathfont
\[
	\cancommit[\ET_{\top}] \mkvs \vi \fp \defiff \mathsf{true}
	\qquad  \text{and} \qquad 
	\vshift[\ET_{\top}] \mkvs \vi {\mkvs'} {\vi'} \defiff \mathsf{true}
\]%
\normalsize}%
\spaceshrink{-17pt}

\noindent
The consistency model induced by \(\ET_{\top}\) 
corresponds to the \emph{Read Atomic} \cite{ramp}, a variant of \emph{Eventual 
Consistency} \cite{ev_transactions} for atomic transactions. 
We give many examples of execution tests in the following sub-section. 

\begin{figure}[t]
%\renewcommand{\true}{\ensuremath{\mathsf{true}}}
\small
\centering
\begin{tabular}{ @{} l @{\hspace{2pt}} || @{\hspace{2pt}} c | @{\hspace{2pt}} l @{\hspace{2pt}} | @{\hspace{2pt}}  c @{} }
\hline
	\ET 
	& $\cancommit \mkvs \vi \fp$
	& Closure Relation (where applicable)
    & $\vshift \mkvs \vi {\mkvs'} {\vi'}$ 
	\\
	\hline
%	
	\MR 
	& \true 
	& 
	& $\vi \viewleq \vi'$
	\\ \hline  
%
	\RYW
	& \true
	& 
	& 
	\protect{$
	\begin{array}[t]{@{} l @{}}
		\fora{\txid \in \mkvs' \setminus \mkvs} \fora{\key, i} \\
		\;\;\wtOf(\mkvs'(\key, i) ) \toEDGE{\!\!\SO\rflx\!\!} \txid \implies i \!\in\! \vi'(\key) 
	\end{array}
	$}
	\\ \hline  
%	
%
%
%	\MW 
%	& \true  
%	& \( \closed(\mkvs, \vi, \SO \cap \WW_\mkvs) \)
%	& \\
%%	
%	\WFR
%	& \true
%	& $\closed(\mkvs, \vi, \rel_{\WFR})$
%	& $\rel_{\WFR}  \defeq \WR_{\mkvs} ; (\SO \cup \RW_\mkvs)\rflx $ \\
%
%	
	\CC
	& $\closed(\mkvs, \vi, \rel_{\CC})$
	& $\rel_{\CC}   \defeq \SO \cup \WR_{\mkvs}$ 
	& $\vshift[\MR \cap \RYW] \mkvs \vi {\mkvs'} {\vi'}$
	\\ \hline  
%
	\UA 
	& $\closed(\mkvs, \vi, \rel_{\UA})$
	& $\rel_{\UA}  \defeq {\textstyle\bigcup_{(\otW, \key, \stub) \in \fp}} \WW^{-1}_{\mkvs}(\key) $ 
	& \true  
	\\ \hline  
% 
	\PSI
	& $\closed(\mkvs, \vi, \rel_{\PSI})$
	& $\rel_{\PSI} \defeq \rel_{\UA} \cup \rel_{\CC} \cup \WW_\mkvs$ 
	& $\vshift[\MR \cap \RYW] \mkvs \vi {\mkvs'} {\vi'}$
	\\ \hline   
%
	\CP 
	& $\closed(\mkvs, \vi, \rel_{\CP})$
	& $\rel_{\CP} \defeq \SO;\RW\rflx_\mkvs \cup \WR_\mkvs;\RW\rflx_\mkvs  \cup \WW_\mkvs$ 	
	& $\vshift[\MR \cap \RYW] \mkvs \vi {\mkvs'} {\vi'}$
    \\ \hline 
%	
	\SI
	& $\closed(\mkvs,\vi, \rel_{\SI})$
	& $  \rel_{\SI}  \defeq \rel_{\UA} \cup \rel_{\CP} \cup (\WW_\mkvs; \RW_\mkvs)$ 
	& $\vshift[\MR \cap \RYW] \mkvs \vi {\mkvs'} {\vi'}$
	\\ \hline  
% 	
%	\SER*
%	& $\vshift[\MR \cap \RYW] \mkvs \vi {\mkvs'} {\vi'}$
%	& $ \closed(\mkvs,\vi, \rel_{\SER^*})$
%	& $\rel_{\SER^*} \defeq \rel_{\UA} \cup \SO \cup \WW_\mkvs \cup \WR_{\mkvs} \cup \RW_\mkvs$ 
%	\\ \hline  
%% 
	\SER
	& $\closed(\mkvs,\vi, \rel_{\SER})$
	&$\rel_{\SER} \defeq \WW^{-1}$
	& \true	
	\\ \hline
\end{tabular}
%
\vspace{0pt}
\caption{Execution tests of well-known consistency models, where \SER* denotes an alternative equivalent $\SER$ specification and $\SO$ is as given in \cref{subsec:kvstores}.
%\( \WR_\mkvs, \WW_\mkvs, \RW_\mkvs\) are given in \cref{subsec:cm_examples}.
}
\label{fig:execution.tests}
\label{fig:execution_tests}
\label{fig:execution-tests}
\end{figure}


\subsection{Example Execution Tests}
\label{subsec:cm_examples}
We give several examples of execution tests which give rise to consistency
models on kv-stores.
Recall that the snapshot property and the last write wins are hard-wired into our model. 
This means that we can only define  consistency models that satisfy these two constraints. 
Although this forbids us to express interesting consistency models such as \emph{Read Committed}, we are able to express a large variety of consistency models employed by distributed kv-stores.

\subsubsection{Notation}
Given relations \(\mathsf r, \mathsf r' \subseteq \sort A \times \sort A\),
we write:  \(\mathsf r\rflx\), \(\mathsf r^+\) and \(\mathsf r^*\) for its reflexive, transitive and reflexive-transitive closures of \(\mathsf r\), respectively;
\(\mathsf r^{-1}\) for its inverse;
\(a_1 \toEDGE{\mathsf r} a_2\) for \((a_1, a_2) \in \mathsf r\);
and \( \mathsf r; \mathsf r'\) for \( \Set{(a_1,a_2)}[\exsts{a} (a_1,a) \in \mathsf r\land (a,a_2) \in \mathsf r']\).

Recall that an  execution test \(\ET\) (\cref{def:execution.test})
has the form \((\mkvs,\vi, \fp, \allowbreak \mkvs', \vi')\) 
where  \(\cancommit \mkvs \vi \fp\) and \(\vshift \mkvs \vi {\mkvs'} {\vi'}\). 
We define \(\cancommitname\) and \(\vshiftname\) for several consistency
models, using some auxiliary definitions. 


\mypar{Prefix Closure}
Given a kv-store \(\mkvs\) and a view \(\vi\), the {\em set of visible
transactions} is
\( \Tx[\mkvs, \vi]  \defeq \Set{\wtOf[\mkvs(\key, i)] }[ i \in \vi(\key)] \).
Given a binary relation on transactions, \(\rel \subseteq \TxID \times \TxID\),
we say that a view \(\vi\) is closed with respect to a kv-store \(\mkvs\) and \(\rel\), written \(\closed[\mkvs,\vi,\rel]\), iff:  

\spaceshrink{-15pt}
{%
\displaymathfont
\begin{align*}
	\closed[\mkvs,\vi,\rel] \defiff
	\Tx(\mkvs, \vi) = 
	\left( (\rel^*)^{-1} \left( \Tx(\mkvs, \vi) \right) \right) \setminus \Set{\txid }[ \fora{\key,i} \txid \neq \wtOf[\mkvs(\key,i)] ]
\end{align*}
}%
\spaceshrink{-17pt}

\noindent
That is, if transaction \(\txid\) is visible in \(\vi\) as \( \txid \in \Tx(\mkvs, \vi) \),
then all the transactions \( \txid'  \) that are \(\rel^*\)-before \(\txid\), \ie \(\txid' \in (\rel^*)^{-1} \left( \txid \right)\),
and are not read-only transactions \( \txid' \notin \Set{\txid'' }[ \fora{\key,i} \txid'' \neq \wtOf[\mkvs(\key,i)] ] \),
are also visible in \(\vi\), \ie \( \txid' \in \Tx(\mkvs, \vi) \).

\mypar{Dependency Relations}
We define transaction dependency relations for kv-stores.
Given a kv-store \(\mkvs\), a key \(\key\) and 
indexes \(i,j\) such that  \(0 \leq i < j < \abs{ \mkvs(\key) }\), 
if there exists \(\txid_i, \T_i, \txid\) such that 
\(\mkvs(\key, i)  {=} (\stub, \txid_{i}, \T_{i})\), \(\mkvs(\key,j) {=} (\stub, \txid_{j}, \stub)\)
and \(\txid \in \T_{i}\), 
then we say that, for every key \( \key \), there is:

\spaceshrink{-7pt}
\begin{enumerate} 
\item a \emph{Write-Read} dependency from \(\txid_{i}\) to \(\txid\), written \((\txid_{i},\txid) \in \WR_{\mkvs}(\key)\);
\item a \emph{Write-Write} dependency from \(\txid_{i}\) to \(\txid_{j}\), 
    written \((\txid_{i},\txid_{j}) \in \WW_{\mkvs}(\key) \); and 
\item a \emph{Read-Write} anti-dependency from \(\txid\) to \(\txid_{j}\), if 
\(\txid {\neq} \txid_{j}\), written \((\txid,\txid_j) {\in} \RW_{\mkvs}(\key)\).
\end{enumerate}
\spaceshrink{-7pt}

\noindent \cref{fig:dependencies} illustrates an example kv-store and
its transaction dependency relations.
We adopt the same names as the dependency relations for dependency graphs \citep{adya}
to emphasis the similarity.
However, the relations here do \emph{not} depend on those in dependency graphs.

We give several definitions of
execution tests using \vshiftname and \cancommitname in \cref{fig:execution_tests}. 
In \cref{sec:other_formalisms,,app:et-sound-complete}, 
we show that our definitions correspond to
the well-known declarative definitions of consistency models on abstract executions.
%the associated consistency models on kv-stores correspond to 
%. We anticipate these results, by labelling the
%execution test with their well-known consistency models.


\subsubsection{Monotonic Reads \((\MR)\)}
This consistency model states that when committing, a client
cannot lose information in that it can only see increasingly more up-to-date versions from a kv-store.
This prevents, for example, the kv-store of \cref{fig:mr-disallowed},
since client \(\cl\) first reads the latest version of \(\key\) in \(\txid_{\cl}^{1}\), 
and then reads the older, initial version of \(\key\) in \(\txid_{\cl}^{2}\).  
As such, the \(\vshiftname_{\MR}\) predicate in \cref{fig:execution_tests} ensures that clients  can only extend their views. 
When this is the case, clients can \emph{always} commit their transactions, and thus \(\cancommitname_{\MR}\) is simply defined as \(\true\). 

\subsubsection{Read Your Writes \((\RYW)\)}
This consistency model states that a client must always see all the versions written by the client itself. 
The \(\vshiftname_{\RYW}\) predicate thus states that after executing a transaction, a client 
contains all the versions it wrote in its view. This ensures that such versions will be included in the view of the client 
when committing future transactions.
Note that under \(\RYW\) the kv-store in \cref{fig:ryw-disallowed} is prohibited as
the initial version of \(\key\) holds value \(\val_0\) 
and client \(\cl\) tries to update the value of \(\key\) twice.  
For its first transaction \( \txid_{\cl}^1\), it reads the initial value \(\val_0\) and then writes a new version with value \(\val_1\). 
For its second transaction \( \txid_{\cl}^2\), it reads again the initial value \(\val_0\) again and write a new version with value \(\val_1\).
The \(\vshiftname_{\RYW}\) predicate rules out this example by requiring that
the client view, after it commits the transaction  \(\txid_{\cl}^{1}\), includes the version it wrote.  
When this is the case, clients can always commit their transactions, and thus \(\cancommitname_{\RYW}\) is simply \(\true\).

The \(\MR\) and \(\RYW\) models together with \emph{monotonic writes} (\(\MW\)) and \emph{write follows reads} (\(\WFR\)) models  are collectively known as \emph{session guarantees}. 
Due to space constraints, the definitions associated with \(\MW\) and \(\WFR\) are in \cref{sec:full-semantics}. 

\begin{figure*}[t]
\newcommand{\LEFTCOL}{0.34\textwidth}
\newcommand{\RIGHTCOL}{0.61\textwidth}
\captionsetup[subfigure]{aboveskip=0pt, belowskip=5pt}


\begin{tabularx}{\textwidth}{@{} c | X @{}}
\hline
\phantom{-}& \phantom{-} \\[-5pt]
\begin{subfigure}{\LEFTCOL}
\begin{centertikz}

%Location x
\node(locx) {$\key_1 \mapsto$};
\draw pic at ([xshift=\tikzkvspace]locx.east) {vlist={versionx}{%
    /\;\;\;\;\;/\;\;\;\;\;\;$\txid_0$\;\;\;/\;\;\;\;$\Set{\txid_1}$\;\;\;\;
    , /\;/\;\;\;\;$\txid_2$\;\;\;\;/\;\;\;\;$\emptyset$\;\;\;\;\;
}};

\coordinate (A) at ([xshift=-25,yshift=3]versionx.center);
\coordinate (B) at ([xshift=0,yshift=-17]A.center);
\coordinate (C) at ([xshift=17,yshift=0]A.center);
\coordinate (D) at ([xshift=17,yshift=-17]A.center);
\coordinate (E) at ([xshift=37,yshift=0]C.center);
\coordinate (F) at ([xshift=0,yshift=-5]E.center);

\path[->, thick] ([yshift=5pt]A.center) edge[bend right=75] node[left,fill=white, opacity=.4, text opacity=1] {$\WR$} ([yshift=5pt]B.center)
([yshift=5pt]C.center) edge[bend left=30] node[above,fill=white, opacity=.4, text opacity=1] {$\WW$} ([yshift=5pt]E.center)
([yshift=5pt]D.center) edge[bend right=30] node[below, fill=white, opacity=.4, text opacity=1] {$\RW$} ([yshift=5pt]F.center);






\end{centertikz}%
\caption{Dependencies of kv-stores}
\label{fig:dependencies}
\end{subfigure}
%\quad
&
\begin{subfigure}{\RIGHTCOL}
\begin{centertikz}

%Location x

\node(locx) {$\key_1 \mapsto$};
\draw pic at ([xshift=\tikzkvspace]locx.east) {vlist={versionx}{%
    /$\val_0$/$\txid_0$/$\Set{\txid}$
    , /$\val_1$/$\txid_{\cl}^1$/$\emptyset$
}};

%Location y
\path (versionx.east) + (1,0) node (locy) {$\key_2 \mapsto$};
\draw pic at ([xshift=\tikzkvspace]locy.east) {vlist={versiony}{%
    /$\val_0$/$\txid_0$/$\emptyset$
    , /$\val_2$/$\txid_\cl^1$/$\emptyset$
    , /$\val_3$/$\txid_\cl^2$/$\Set{\txid}$
}};

\end{centertikz}
\caption{Disallowed by \(\MW\)}
\label{fig:mw-disallowed}
\end{subfigure}
%\quad
\\ \hline
\\[-5pt]
%
\begin{subfigure}{\LEFTCOL}
\begin{centertikz}

%Location x
\node(locx) {$\key_1 \mapsto$};
\draw pic at ([xshift=\tikzkvspace]locx.east) {vlist={versionx}{%
    /$\val_0$/$\txid_0$/$\Set{\txid_\cl^2}$
    , /$\val_1$/$\txid_1$/$\Set{\txid_\cl^1}$
}};

\end{centertikz}%
\caption{Disallowed by \(\MR\)}
\label{fig:mr-disallowed}
\end{subfigure}

&

\begin{subfigure}{\RIGHTCOL}
\begin{centertikz}

%Location x
\node(locx) {$\key_1 \mapsto$};
\draw pic at ([xshift=\tikzkvspace]locx.east) {vlist={versionx}{%
    /$\val_0$/$\txid_0$/$\Set{\txid}$
    , /$\val_1$/$\txid'$/$\emptyset$
}};

%Location y
\path (versionx.east) + (1,0) node (locy) {$\key_2 \mapsto$};
\draw pic at ([xshift=\tikzkvspace]locy.east) {vlist={versiony}{%
    /$\val_0$/$\txid_0$/$\emptyset$
    , /$\val_2$/$\txid'$/$\Set{\txid_\cl^1}$
    , /$\val_3$/$\txid_\cl^2$/$\Set{\txid}$
}};

\end{centertikz}

\caption{Disallowed by \(\WFR\)}
\label{fig:wfr-disallowed}
\end{subfigure}
%
\\ \hline
\\[-5pt]
%
\begin{subfigure}{\LEFTCOL}
\begin{centertikz}%

%Location x
\node(locx) {$\key_1 \mapsto$};
\draw pic at ([xshift=\tikzkvspace]locx.east) {vlist={versionx}{%
    /$\val_{0}$/$\txid_0$/$\Set{\txid_\cl^1,\txid_\cl^2}$
    , /$\val_{1}$/$\txid_\cl^1$/$\emptyset$
    , /$\val_{1}$/$\txid_\cl^2$/$\emptyset$
}};
\end{centertikz}%
\caption{Disallowed by \(\RYW\)}
\label{fig:ryw-disallowed}
\end{subfigure}

&
\begin{subfigure}{\RIGHTCOL}
\begin{centertikz}%

%Location x
\node(locx) {$\key_1 \mapsto$};
\draw pic at ([xshift=\tikzkvspace]locx.east) {vlist={versionx}{%
        /$\val_0$/$\txid_0$/$\Set{\txid}$
    , /$\val_1$/$\txid^{1}_{\cl}$/$\emptyset$
}};

%Location y
\path (versionx.east) + (0.75,0) node (locy) {$\key_2 \mapsto$};
\draw pic at ([xshift=\tikzkvspace]locy.east) {vlist={versiony}{%
    /$\val_0$/$\txid_0$/$\emptyset$
    , /$\val_2$/$\txid^{2}_{\cl}$/$\{\txid^{1}_{\cl'}\}$
}};

%Location z
\path (versiony.east) + (0.75,0) node (locz) {$\key_3 \mapsto$};
\draw pic at ([xshift=\tikzkvspace]locz.east) {vlist={versionz}{%
    /$\val_0$/$\txid_0$/$\emptyset$
    , /$\val_3$/$\txid^{2}_{\cl'}$/$\Set{\txid}$
}};

\end{centertikz}%
\caption{Allowed by \(\MW\) and \( \WFR \) but disallowed by \( \MW + \WFR \)}
\label{fig:wr-wfr-allowed-but-cc}
\end{subfigure}%
\\ 
\hline
%\cline{2-2}
\\[-5pt]
%
\begin{subfigure}{\LEFTCOL}
\begin{centertikz}

\node(locx) {$\key_1 \mapsto$};
\draw pic at ([xshift=\tikzkvspace]locx.east) {vlist={versionx}{%
    /$\val_{0}$/$\txid_0$/$\Set{\txid,\txid'}$
    , /$\val_{1}$/$\txid$/$\emptyset$
    , /$\val_{1}$/$\txid'$/$\emptyset$
}};

\end{centertikz}
\caption{Lost update, disallowed by \(\UA\)}
\label{fig:ua-disallowed}
\end{subfigure}

&
\begin{subfigure}{\RIGHTCOL}%
\begin{centertikz}%

%Location x
\node(locx) {$\key_1 \mapsto$};
\draw pic at ([xshift=\tikzkvspace]locx.east) {vlist={versionx}{%
        /$\val_0$/$\txid_0$/$\emptyset$
        , /$\val_1$/$\txid^{1}_{\cl}$/$\emptyset$
        , /$\val_2$/$\txid^{1}_{\cl'}$/$\Set{\txid}$
}};

%Location y
\path (versionx.east) + (1,0) node (locy) {$\key_2 \mapsto$};
\draw pic at ([xshift=\tikzkvspace]locy.east) {vlist={versiony}{%
    /$\val_0$/$\txid_0$/$\Set{\txid}$
    , /$\val_3$/$\txid^{1}_{\cl}$/$\emptyset$
}};

\end{centertikz}%
\caption{Allowed by \(\CC\) and \( \UA \) but disallowed by \( \PSI \)}
\label{fig:cc-ua-allowed-but-psi}
\end{subfigure}%
\\
\hline
\\[-5pt]
%
\multirow{2}{*}{%
\begin{subfigure}{\LEFTCOL}%
\begin{centertikz}%
%Location x
\node(locx) {$\key_1 \mapsto$};
\draw pic at ([xshift=\tikzkvspace]locx.east) {vlist={versionx}{%
    /$\val_0$/$\txid_0$/$\big\{\txid_{\cl_2}^2\big\}$
    , /$\val_1$/$\txid'$/$\big\{\txid_{\cl_1}^1\big\}$
}};
%Location y
\path (locx) + (0,2) node (locy) {$\key_2 \mapsto$};
\draw pic at ([xshift=\tikzkvspace]locy.east) {vlist={versiony}{%
    /$\val_0$/$\txid_0$/$\big\{\txid_{\cl_1}^2\big\}$
    , /$\val_1$/$\txid$/$\big\{\txid_{\cl_2}^1\big\}$
}};
\end{centertikz}%
\caption{Long fork, disallowed by \(\CP\)}
\label{fig:cp-disallowed-2}
\label{fig:cp-disallowed}
\end{subfigure}%
}%
\\[-13pt]%
&
\begin{subfigure}{\RIGHTCOL}
\begin{centertikz}%

%Location x
\node(locx) {$\key_1 \mapsto$};
\draw pic at ([xshift=\tikzkvspace]locx.east) {vlist={versionx}{%
    /$\val_0$/$\txid_0$/$\Set{\txid_2}$
    , /$\val_1$/$\txid_1$/$\emptyset$
}};

%Location y
\path (versionx.east) + (1,0) node (locy) {$\key_2 \mapsto$};
\draw pic at ([xshift=\tikzkvspace]locy.east) {vlist={versiony}{%
    /$\val_0$/$\txid_0$/$\Set{\txid_1}$
    , /$\val_2$/$\txid_2$/$\emptyset$
}};

\end{centertikz}%
\caption{Write skew, disallowed by \(\SER\)}
\label{fig:ser-disallowed}
\end{subfigure}%
\\
\cline{2-2}
\\[-5pt]
%%%%%%%%% for multirow place holding 
&
\begin{subfigure}{\RIGHTCOL}
\begin{centertikz}%
%Location x
\node(locx) {$\key_1 \mapsto$};
\draw pic at ([xshift=\tikzkvspace]locx.east) {vlist={versionx}{%
    /$\val_0$/$\txid_0$/$\Set{\txid_4}$
    , /$\val_1$/$\txid_1$/$\emptyset$
    , /$\val_2$/$\txid_2$/$\emptyset$
}};

%Location y
\path (versionx.east) + (1,0) node (locy) {$\key_2 \mapsto$};
\draw pic at ([xshift=\tikzkvspace]locy.east) {vlist={versiony}{%
    /$\val_0$/$\txid_0$/$\Set{\txid_2}$
    , /$\val_3$/$\txid_3$/$\Set{\txid_4}$
    , /$\val_4$/$\txid_4$/$\emptyset$
}};

\end{centertikz}
\caption{Allowed by \( \UA \) and \( \CP \) but disallowed by \(\SI\)}%
\label{fig:si-disallowed}%
\end{subfigure}%
%
\\
\hline
\end{tabularx}

\caption{Behaviours disallowed under different consistency models}
\label{fig:anomalies}
\end{figure*}


\subsubsection{Causal Consistency \((\CC)\)}
Causal consistency subsumes the  four session guarantees discussed above. 
As such, the \(\vshiftname_\CC\) predicate is defined as the \emph{conjunction} of their associated \vshiftname predicates.
However, as shown in  \cref{fig:execution_tests}, it is sufficient to define \(\vshiftname_\CC\)
as the conjunction of the \(\MR\) and \(\RYW\) session guarantees alone, where for brevity we 
write \(\vshiftname_{\MR \cap \RYW}\) for  \(\vshiftname_{\MR} \land \vshiftname_{\RYW}\).
This is because 
as we demonstrate in \cref{sec:full-semantics},
the \(\vshiftname_{\MW}\) and \(\vshiftname_{\WFR}\) are defined simply as \( \true \), allowing us to remove them from \(\vshiftname_{\CC}\).

Additionally, \(\CC\) strengthens the session guarantees by requiring that if a client sees a version \(\ver\) prior to committing a transaction, then it must also see the versions 
on which \(\ver\) depends.
If \(\txid\) is the writer of \(\ver\), then 
\(\ver\) clearly depends on all versions that \(\txid\) reads. 
Moreover, if \(\ver\) is, or it depends on, a version \(\ver'\) accessed by 
a client \(\cl\), then it also depends on all versions that were previously 
read or written by \(\cl\). 
This is captured by the \(\cancommitname_{\CC}\) predicate in \cref{fig:execution_tests}, 
defined as \(\closed(\mkvs, u, \rel_{\CC})\) with \(\rel_\CC \defeq \SO \cup \WR_{\mkvs}\).
\azalea{without the definitions of \(\MW\) and \(\WFR\) we no longer can explain why this implies \(\MW\) and \(\WFR\).}
For example, the kv-store of \cref{fig:wr-wfr-allowed-but-cc} 
is disallowed by \(\CC\): the version of key \(\key_3\) carrying value \(\val_3\) depends on 
the version of key \(\key_1\) carrying value \(\val_1\). 
However, transaction \(\txid\) must have been committed by a client
whose view included \(\val_3\) of \( \key_3\), but not \(\val_1\) of \( \key_1\).

\subsubsection{Update Atomic \((\UA)\)}
This consistency model has been proposed by \citet{framework-concur} 
and implemented by \citet{rola}.
\(\UA\) disallows concurrent transactions writing to the same key,
a property known as \emph{write-conflict freedom}:  
when two transactions write to the same key, one must see the version 
written by the other.
Write-conflict-freedom is enforced by \(\cancommitname_{\UA}\) which allows a client to write to key \(\key\) only if its view includes all versions of \(\key\); 
\ie its view is closed with respect to the \(\WW^{-1}(\key)\) relation for all keys \(\key\) written in the fingerprint \(\fp\).
This prevents the kv-store of \cref{fig:ua-disallowed},
as \(\txid\) and \(\txid'\) concurrently increment the initial version of \(\key\) by \(1\).
As client views must include the initial versions, once \(\txid\) commits a new version \(\ver\) with value \(\val_1\) to \(\key\), then \(\txid'\) must include \(\ver\) in its view as there is a \(\WW\) edge from the initial version to \(\ver\). 
As such, when \(\txid'\) % subsequently 
increments \(\key\), it must read from \(\ver\), and not the initial version as depicted in \cref{fig:ua-disallowed}.



\subsubsection{Parallel Snapshot Isolation \((\PSI)\)} 
This consistency model is defined as the conjunction of the guarantees provided by \(\CC\) and \(\UA\)~\cite{framework-concur}. 
As such, the \(\vshiftname_{\PSI}\) predicate is defined as the conjunction of the \(\vshiftname\) predicates for \(\CC\) and \(\UA\).
However, we cannot simply define \(\cancommitname_{\PSI}\) as the conjunction of the \(\cancommitname\) predicates for \(\CC\) and \(\UA\). 
This is for two reasons. 
First, their conjunction would only mandate that \(\vi\) be closed with respect to 
\(\rel_{\CC}\) and \(\rel_{\UA}\) \emph{individually}, but \emph{not} with respect to their \emph{union}.
Recall that closure is defined in terms of the transitive closure of a given relation 
and thus the closure of \(\rel_{\CC}\) and \(\rel_{\UA}\) is smaller than the closure of \(\rel_{\CC} \cup \rel_{\UA}\).
As such, we define \(\cancommitname_{\PSI}\) as closure with respect to \(\rel_{\PSI} \) that must include \( \rel_{\CC} \cup \rel_{\UA}\).
Second, recall that \(\cancommitname_{\UA}\) requires that a transaction writing 
to a key \(\key\) must be able to see all previous versions of \(\key\), \ie all versions of \(\key\). 
That is, when write-conflict-freedom is enforced, a version \(\ver\) of \(\key\) depends on all 
previous versions of \(\key\). 
This observation leads us to include write-write dependencies (\(\WW_{\mkvs}\)) in \(\rel_{\PSI}\). 
Observe that the kv-store in \cref{fig:cc-ua-allowed-but-psi} shows an example kv-store that satisfies \(\cancommitname_{\CC} \land \cancommitname_{\UA}\), 
but not \(\cancommitname_{\PSI}\).

\subsubsection{Consistent Prefix \((\CP)\)}
\label{para:cp}
If the total order in which transactions commit is known, \(\CP\)
can be described as a strengthening of \(\CC\): 
if a client sees the versions written by a transaction \(\txid\),
then it must also see all versions written by transactions that \emph{commit} before \(\txid\). 
Although kv-stores only provide \emph{partial} information about the transaction commit order via the dependency relations,
this is sufficient to formalise \emph{Consistent Prefix} \cite{laws}.

In practice, we can approximate the order in which transactions 
commit via the \(\WR_{\mkvs}, \WW_{\mkvs}, \RW_{\mkvs}\) and \(\SO\)  relations. 
This approximation is best understood in terms of an idealised implementation of \(\CP\) on a centralised system,
where the snapshot of a transaction is determined at its \emph{start point} and its effects are made visible to future transactions at its \emph{commit point}.
With respect to this implementation, if \((\txid,\txid') \in \WR\), then 
\(\txid\) must commit before \(\txid'\) starts, and hence before \(\txid'\) commits.
Similarly, if \((\txid,\txid') \in \SO\), then \(\txid\) commits before \(\txid'\) starts, 
and thus before \(\txid'\) commits.
Recall that, if \((\txid'', \txid') \in \RW\),
then \(\txid''\) reads a version that is later overwritten by \(\txid'\).
That is, \(\txid''\) cannot see the write of \(\txid'\), and thus \(\txid''\) must start before 
\(\txid'\) commits. 
As such, if \(\txid\) commits before \(\txid''\) starts 
(\((\txid, \txid'') \in \WR\) or \((\txid,\txid'') \in \SO\)), 
and \((\txid'', \txid') \in \RW\), then \(\txid\) must commit before 
\(\txid'\) commits. 
In other words, if \((\txid,\txid') \in \WR;\RW\) or \((\txid,\txid') \in \SO;\RW\), then \(\txid\) commits before \(\txid'\).
Finally, if \((\txid,\txid') \in\WW\), then \(\txid\) must commit before \(\txid'\). 
We therefore define \(\rel_{\CP} \defeq (\WR_{\mkvs}; \RW_{\mkvs}\rflx \cup \SO;  \RW_{\mkvs}\rflx \cup \WW)\), approximating the order in which transactions commit. 
%
\citet{laws} show that the set \((\rel_{\CP}^{-1})^{+}(\txid)\) contains all transactions that must be observed by \(\txid\) under \(\CP\). 
We define \(\cancommitname_{\CP}\) by requiring that the client view be 
closed with respect to \(\rel_{\CP}\).

Consistent prefix disallows the \emph{long fork anomaly} shown in \cref{fig:cp-disallowed}, where clients \(\cl_1\) and \(\cl_2\) observe the updates to \(\key_1\) and \(\key_2\) 
in different orders. 
Assuming without loss of generality that \( \txid_{\cl_1}^{2} \) commits 
before \( \txid_{\cl_2}^{2} \), then prior to committing its transaction \(\cl_2\) sees 
the version of \(\key_1\) with value \(\val_0\). 
However, since \(\txid {\xrightarrow{\WR_{\mkvs}}} \txid_{\cl_{1}}^{1} 
{\xrightarrow{\SO}} \txid_{\cl_{1}}^{2} {\xrightarrow{\RW}} \txid' {\xrightarrow{\WR}} \txid_{\cl_{2}}^{1} \), 
and prior to commit, \( \txid_{\cl_2}^{2} \) must see versions written by \( \txid_{\cl_2}^{1} \),
then \( \txid_{\cl_2}^{2} \) should also see the version of \(\key_1\) with 
value \(\val_2\), leading to a contradiction.


\subsubsection{Snapshot Isolation \((\SI)\)}
When the total order in which transactions commit is known,  
\(\SI\) can be defined compositionally from \(\CP\) and \(\UA\). 
As such, \(\vshiftname_{\SI}\) is defined as the conjunction of their associated \(\vshiftname\) predicates. 
However, as with \(\PSI\), we cannot define \(\cancommitname_{\SI}\) as the conjunction of their associated \(\cancommitname\) predicates. 
Rather, we define \(\cancommitname_{\SI}\) as closure with respect to \(\rel_{\SI}\), which includes \(\rel_\CP \cup \rel_{\UA}\).
Observe that the kv-store in \cref{fig:si-disallowed} shows an example kv-store that satisfies \(\cancommitname_{\UA} \land \cancommitname_{\CP}\), 
but not \(\cancommitname_{\SI}\).
Additionally, we include \(\WW;\RW\) in \(\rel_{\SI}\). 
This is because when the centralised \(\CP\) implementation (discussed before) is strengthened with write-conflict-freedom, then a write-write dependency between two transactions \(\txid\) and \(\txid'\) 
does not only mandate that \(\txid\) commits before \(\txid'\) commits but also before \(\txid'\) starts. 
Consequently, if \((\txid, \txid') \in \WW ;\RW\), then \(\txid\) must commit 
before \(\txid'\) commit.

\subsubsection{(Strict) serialisability \((\SER)\)}
Serialisability is the strongest consistency model 
in any framework that abstracts from aborted transactions, 
requiring that transactions execute in a total sequential order. 
The \(\cancommitname_{\SER}\) thus allows clients to commit transactions only when 
their view of the kv-store is complete, \ie the client view is closed with respect to \(\WW^{-1}\).
This requirement prevents the kv-store in  \cref{fig:ser-disallowed}: 
without loss of generality, suppose that \(\txid_1\) commits before \(\txid_2\). 
Then the client committing \(\txid_2\) must see the version of \(\key_1\) written by \(\txid_1\), 
and thus cannot read the outdated value \(\val_0\) for \(\key_1\). 
%This example is allowed by all other execution tests in~\cref{fig:execution_tests}.

\subsubsection{Weak Snapshot Isolation \((\WSI)\): A New Consistency Model} 
\label{sec:new_cm}
Kv-stores and execution tests are useful for investigating new 
consistency models.  
One example is the consistency model induced by combining 
\(\CP\) and \(\UA\), which we refer to as \emph{Weak Snapshot Isolation} (\(\WSI\)). 
To justify this consistency model in full, it would be useful to explore its implementations. 
Here we focus on the benefits of clients on \(\WSI\).
Because \(\WSI\) is stronger than \(\CP\) and \(\UA\) by definition, 
it forbids all the  anomalies forbidden by these consistency models, \eg
the long fork (\cref{fig:cp-disallowed}) and the lost update (\cref{fig:ua-disallowed}). 
Moreover, \(\WSI\) is strictly weaker than \(\SI\). 
As such, \(\WSI\) allows all \(\SI\) anomalies, \eg the write skew (\cref{fig:ser-disallowed}), 
and allows behaviour not allowed under \(\SI\) such as that in \cref{fig:si-disallowed}.
The kv-store \(\mkvs\) is reachable by executing transactions 
\(\txid_{1}, \txid_{2}, \txid_{3}\) and \(\txid_{4}\) in this order. 
In particular, \(\txid_{4}\) is executed using \(\vi {=} \{\key_{1} \mapsto \{0\}, \key_{2} \mapsto \{0,1\}\}\). 
However, the same kv-store is not reachable under \(\ET_{\SI}\). 
Under \(\SI\) transaction \(\txid_{4}\) cannot be executed using \(\vi\): 
\(\txid_{4}\) reads the version of \(\key_2\) written by \(\txid_3\), 
%meaning that \(\vi\) must include the version written by \(\txid_{3}\).
but since \((\txid_{2},\txid_{3}) \in \RW \)
and \((\txid_{1} ,\txid_2) \in \WW\), 
then \(\vi\) should contain the version of \(\key_{1}\) written by \(\txid_{1}\), 
contradicting the fact that \(\txid_{4}\) reads the initial version of \(\key_1\).

As \(\WSI\) is a model weaker than \(\SI\), we believe that \(\WSI\) implementations would outperform known \(\SI\) implementations.
Nevertheless, the two consistency models are very similar in that 
many applications that 
are correct under \(\SI\) are also correct under \(\WSI\). We give examples of such applications in \cref{sec:program-analysis}.



\section{Consistency Models: Dependency Graphs and Abstract Executions}
\label{sec:other_formalisms}


We demonstrate that our consistency models for kv-stores
are equivalent to the declarative consistency models for 
dependency graphs \cite{adya} 
and abstract executions \cite{ev_transactions,framework-concur}. 
We outline our results here, and refer the reader
to \cref{sec:app-abstract-semantics-sound-complete,sec:et-sound-complete-constructor,app:et_sound_complete} for the full details.


\subsection{Relating KV-Stores and Dependency Graphs}
\label{sec:dep_graphs}
Dependency graphs \cite{adya-icde,adya} provide  perhaps the most
well-known 
formalism used for specifying transactional consistency models. 
A dependency graph $\Gr$ is a directed, labelled graph whose 
nodes denote transactions, and whose edges denote \emph{dependencies} between transactions.  
More specifically, nodes are labelled with a transaction identifier
and the fingerprint associated with the  transaction. 
Edges are labelled with a dependency relation $\SO, \WR, \WW, \RW$, in the 
same spirit of dependencies of transactions in kv-stores in \cref{sec:cm}.
An example of dependency graph is given in \cref{fig:dependency-graph}.
%
%\begin{enumerate*}
%    \item a \emph{session order} edge, $\txid_1 \toEDGE{\SO} \txid_2$, 
%	\item a \emph{read dependency} edge, $\txid_1 \toEDGE{\WR} \txid_2$, denotes
%that transaction $\txid_2$ reads a version written by $\txid_1$;
%	\item a \emph{write dependency} edge, $\txid_1 \toEDGE{\WW} \txid_2$, denotes that $\txid_2$ overwrites a version written by $\txid_1$; and 
%	\item an \emph{anti-dependency} edge, $\txid_1 \toEDGE{\RW} \txid_2$, denotes that $\txid_2$ overwrites a version read by $\txid_2$. 
%\end{enumerate*}
We give the formal definition of dependency graphs in \cref{app:depgraphs}.

We can always {extract} a dependency graph  from a kv-store, and vice-versa. 
%we choose the transaction identifiers appearing in $\mkvs$ as the nodes of $\Gr$, 
%and let $\SO$ as defined in \cref{subsec:kvstores}, and $ \WR, \WW, \RW$  
%be as defined in \cref{sec:c}.
For example, \cref{fig:dependency-graph} corresponds to the dependency graph extracted from the kv-store in \cref{fig:ser-disallowed}. 
%In fact, this construction can be reversed; see \cref{app:depgraphs}.
%we show that this construction can be reversed, thus giving 
%rise to the following result: 
\begin{theorem}
\label{thm:kv_graph_isomorph}
Dependency graphs are isomorphic to kv-stores.
\end{theorem}

\begin{figure*}[!t]
\captionsetup[subfigure]{aboveskip=0pt, belowskip=5pt}
\centering
\noindent
\begin{subfigure}{0.49\textwidth}
    \begin{centertikz}[.7]
\draw pic {transaction={t0}{%
        /$\otW$/$\key_1$/$\val_0$%
        , /$\otW$/$\key_2$/$\val'_0$%
}};
\path(t0.west) node[anchor=east] (t0lbl) {$\txid_0$};

\draw pic at ($(t0.north east) + (1.5,0.6)$) {transaction={t1}{%
        /$\otR$/$\key_2$/$\val'_0$%
        , /$\otW$/$\key_1$/$\val_1$%
}};
\path(t1.north) node[anchor=south] (t1lbl) {$\txid_1$};

\draw pic at ($(t0.south east) + (1.5,-0.3)$) {transaction={t2}{%
        /$\otR$/$\key_1$/$\val_0$%
        , /$\otW$/$\key_2$/$\val'_1$%
}};
\path(t2.south) node[anchor=north] (t2lbl) {$\txid_2$};

\path[->]
(t0.north) edge[bend left=30] node[above, yshift=3pt, xshift=-20pt, pos=0.3] {$\WR$} (t1.west)
(t0.south) edge[bend right=30] node[below, yshift=-3pt, xshift=-20pt, pos=0.3] {$\WR$} (t2.west)
([xshift=8pt]t1.south) edge[bend left=20] node[right] {$\RW$} ([xshift=8pt]t2.north)
([xshift=-16pt]t2.north) edge[bend left=20] node[right] {$\RW$} ([xshift=-16pt]t1.south);

\end{centertikz}
\caption{Dependency graph}
\label{fig:dependency-graph}
\end{subfigure}
%
\hfill
%
\begin{subfigure}{0.49\textwidth}
    \begin{centertikz}[.7]

\draw pic {transaction={t0}{%
        /$\otW$/$\key_1$/$\val_0$%
        , /$\otW$/$\key_2$/$\val'_0$%
}};
\path(t0.west) node[anchor=east] (t0lbl) {$\txid_0$};

\draw pic at ($(t0.north east) + (1.5,0.6)$) {transaction={t1}{%
        /$\otR$/$\key_2$/$\val'_0$%
        , /$\otW$/$\key_1$/$\val_1$%
}};
\path(t1.north) node[anchor=south] (t1lbl) {$\txid_1$};

\draw pic at ($(t0.south east) + (1.5,-0.3)$) {transaction={t2}{%
        /$\otR$/$\key_1$/$\val_0$%
        , /$\otW$/$\key_2$/$\val'_1$%
}};
\path(t2.south) node[anchor=north] (t2lbl) {$\txid_2$};

\path[->]
(t0.north) edge[bend left=30] node[above, yshift=3pt, xshift=-20pt, pos=0.3] {$\VIS, \AR$} (t1.west)
(t0.south) edge[bend right=30] node[below, yshift=-3pt, xshift=-20pt, pos=0.3] {$\VIS, \AR$} (t2.west)
([xshift=8pt]t1.south) edge[bend left=20] node[right] {$\AR$} ([xshift=8pt]t2.north);

\end{centertikz}
\caption{Abstract execution}
\label{fig:abstract_execution}
\end{subfigure}

\hrulefill

\caption{The dependency graph (\subref{fig:dependency-graph}) and abstract execution graph (\subref{fig:abstract_execution}) associated with the kv-store in \cref{fig:ser-disallowed}
}
\end{figure*}


Consistency models using dependency graphs can be specified by
constraining the shape of the graphs, typically by requiring the absence of certain cycles.  For example, strict serialisability is defined as
the set of dependency graphs with no cycles. %where $(\SO \cup \WW \cup \WR \cup \RW)^+$ is acyclic.
We can immediately use such constraints to define execution tests on
kv-stores, and hence consistency models for kv-stores. However, to show 
that our consistency models over kv-stores given in~\cref{fig:execution-tests} are equivalent
to existing consistency model definitions using dependency graphs,
we first prove that our models are equivalent
to existing definitions using abstract executions, and then appeal 
to the results of \citet{laws} showing the equivalence between definitions using dependency graphs and those using abstract executions.  


%Note that we use dependency relations in a fundamentally 
%different way from dependency graph specifications \cite{adya}.
%In particular, dependency graph specifications are of a declarative nature: they are specified by placing constraints on the structure of graphs admitted by a consistency model.
%By contrast, our execution tests use dependency relations to define consistency models in a constructive way: they are used to restrict the views of clients before or 
%after committing a transaction. 


%\sx{quoting from Azalea, the following text may put here:\\
%As we will see shortly, we use dependencies in a way fundamentally 
%different from Adya's dependency graphs \cite{adya}, 
%a well-known mathematical structure used for specifying consistency models that is introduced 
%formally in \cref{sec:dep_graphs}.
%Adya's specifications are of a declarative nature: dependency models are specified by 
%placing constraints on the structure of dependency graphs admitted by a consistency model, using the dependency graph counterpart 
%of the relations introduced above. 
%In contrast, execution tests define consistency models in a constructive way. Dependencies 
%over kv-stores are used by execution tests to restrict the views of clients (i.e the sets of versions that clients must see) before or 
%after committing a transaction. The consistency model induced $\ET$ corresponds 
%to the set of kv-stores that can be constructed by allowing clients to commit transactions only 
%when the conditions on such views are respected. This approach is agnostic of the structure 
%of kv-stores contained in $\CMs(\ET)$.}

%Dependency graph-based specifications can be converted into
%execution-test-based ones: such execution tests prevent committing
%transactions that result in kv-stores whose associated dependency
%graph violates the constraints imposed by the consistency model.
% the effects of committing a transaction to kv-store $\mkvs$ leads to $\mkvs'$, 
%then check that dependency graph extracted from $\mkvs'$ contains no cycles prohibited by the dependency graph-based specification. 
%
%\mypar{Relating KV-Stores and Abstract Executions}
%\emph{Abstract executions} \cite{ev_transactions,framework-concur} are an alternative formalism for defining consistency models. 
%As with dependency graphs, an abstract execution graph $\aexec$
%is a directed graph with its nodes representing transactions (with each node labelled with a transaction identifier and a set of (read/write) operations), 
%and its edges representing certain relations between transactions. 
%An example abstract execution graph is depicted in \cref{fig:abstract_execution}. 
%Each edge is labelled by either the \emph{visibility} ($\VIS$) or \emph{arbitration} ($\AR$) relation. 
%The $\VIS$ is an irreflexive order on transactions such that $\txid_1 \toEDGE{\VIS} \txid_2$ denotes that the effects (updates) of $\txid_1$ are visible to $\txid_2$. 
%The $\AR$ is a strict total order on transactions such that $\txid_1 \toEDGE{\AR} \txid_2$ denotes that the updates performed by $\txid_2$ are newer than those of $\txid_1$. 
%Moreover, $\AR$ contains $\VIS$ ($\VIS \subseteq \AR$) and agrees with the session order ($\SO \subseteq \AR$).
%Lastly, abstract executions observe the \emph{last-write-wins} policy: 
%a transaction reading $\key$ always fetches the latest visible write ($\VIS$ predecessor) on $\key$.
%We refer the reader to \cref{sec:abstract-execution} for full details.
%
%Following \cite{laws}, we can always \emph{extract} a dependency graph $\Gr_{\aexec}$ from an abstract execution $\aexec$, and thus a kv-store $\mkvs_{\aexec}$ via \cref{thm:kv_graph_isomorph}---see \cref{app:aexec2kv} for the formal details.
%We write  $\mkvs_\aexec$ for the kv-store extracted from $\aexec$ using this construction.  
%Moreover, we show that there is a \emph{Galois connection}
%between $\ET_{\top}$ traces, the weakest possible execution test and abstract executions (\cref{sec:galois-kv-aexec}).
%
%As with dependency graphs, consistency models using abstract executions are defined by constraining the shape of abstract execution graphs via a set of \emph{axioms} $\Ax$, \eg imposing certain conditions on $\VIS$. %the absence of certain cycles.
%All consistency models presented in this paper have an equivalent axiomatic definition based on abstract executions~\cite{framework-concur,laws}. 
%Proving the equivalence of execution test-based and abstract execution-based definitions is non-trivial; 
%however, we have observed that all proofs follow the same \emph{structure}, so long as certain conditions hold. 
%We develop the meta-theory to capture this proof structure.
%Our meta-theory is non-trivial; we refer the reader to \cref{sec:kv2aexec-sound-complete} for the full details. 
%We give an intuitive account of the two conditions required by our meta-theory, and then state our equivalence theorem. 
%
%Given a set of axioms $\Ax$, we define $\CMa(\Ax) \defeq \Set{ \aexec }[ \aexec \text{ sats.} \ \Ax ]$.
%Our first condition is the \emph{soundness} of an execution test against an axiomatic definition.
%An execution test $\ET$ is sound against an axiomatic definition $\Ax$ if:
%for all $n$ and for all \( \ET \)-traces \( \tau \) with \( n \) steps, 
%we can construct $\aexec_0, \cdots, \aexec_n$ and \( \mkvs_i = \mkvs_{\aexec_i} \) such that 
%for each step \( (\mkvs_i, \vi_i) \csat \fp_i : (\mkvs_{i+1}, \vi_{i+1}) \) in \( \tau \),
%the new $\VIS$ edges in \( \aexec_{i+1} \) (those not in $\aexec_i$)
%%, which links transactions included in the view \( \vi_i \) to the new transaction \( \txid_i \),
%satisfy \( \Ax \).
%The formal definition of execution test soundness is given in \cref{sec:kv2aexec-sound-complete}.
%
%Our second condition is the \emph{completeness} of an execution test against an axiomatic definition.
%Let $\txid_i$ denote the $i$\textsuperscript{th} transaction of $\aexec$ in its $\AR$ order, and $\aexec^{i}$ denote the restriction of $\aexec$ to $\txid_1 \cdots \txid_i$. 
%An execution test $\ET$ is complete against an axiomatic definition $\Ax$ if:
%for all abstract executions \( \aexec \) that satisfy \( \Ax \) containing $n$ transactions, 
%all $i \in \Set{1 \cdots n}$, views $\vi_i, \vi'_i$, transactions $\txid'$, and fingerprints $\fp_i$,
%whenever
%\begin{enumerate*}
%	\item $\txid'$ is the immediate $\SO$-successor of $\txid_i$;
%	\item \( \vi_i \) includes all visible transactions of \( \txid_i \); 
%    \item $\vi'_i$ includes all visible transactions, equal or committed before \( \txid_i \), of $\txid'$; and
%	\item $\fp_i$ is the fingerprint of $\txid_i$, 
%\end{enumerate*} 
%then $\ET \vdash (\mkvs_i, \vi_i) \csat \fp_i : (\mkvs_{i+1}, \vi_{i+1})$.
%The formal definition of execution test completeness is given in \cref{sec:kv2aexec-sound-complete}.
%
%Finally, we state our equivalence theorem below (\cref{thm:main-body-et_soundness_completeness}), with its full proof in \cref{sec:kv2aexec-sound-complete}. 
%This theorem ensures that if an execution test is sound and complete against a set of axioms $\Ax$, 
%then the consistency model induced by $\ET$ corresponds to the kv-stores extracted from abstract executions satisfying $\Ax$.
%
%\begin{theorem}
%\label{thm:main-body-et_soundness_completeness}
%For all $\ET, \Ax$, if $\ET$ is sound against $\Ax$, then:
%\(
%\CMs(\ET) \subseteq \Set{ \mkvs_\aexec }[ \aexec \in \CMs(\Ax)]
%\).
%For all $\ET, \Ax$, if $\ET$ is complete against $\Ax$, then:
%\(
%\Set{ \mkvs_\aexec }[ \aexec \in \CMs(\Ax)]  \subseteq \CMs(\ET)
%\).
%\end{theorem} 
%
%In \cref{sec:spec-proof} we apply \cref{thm:main-body-et_soundness_completeness} and show all our definitions in \cref{fig:execution_tests} 
%are sound and complete against (equivalent to) existing axiomatic definitions on abstract executions.
%
%
%\subsection{Andrea's Version}
\subsection{Relating KV-Stores and Abstract Executions}
We compare our consistency model specifications using execution tests over kv-stores 
with an alternative, axiomatic specification style based on abstract 
executions \cite{framework-concur}, defined shortly. 
Our main contribution here is the development of a general proof technique for proving the equivalence of our execution-test-based specifications and abstract-execution-based specifications.
Our proof technique keeps the proof obligations (conditions that must be satisfied by its user) to a minimum. 
In particular, the user only needs to show that the constraints on client views in execution tests relate to analogous constraints on visibility edges in abstract executions.
We then provide a mapping between the $\ET$-traces to 
$\mkvs$, to a set of abstract executions that satisfy the axiomatic specification corresponding to $\ET$.
Here we use our proof technique to prove that the execution 
tests for serialisability ($\SER$) are equivalent to their 
axiomatic specifications. In \cref{sec:spec-proof} we apply our proof technique 
to show that all the execution tests from \cref{fig:execution.tests} are equivalent 
to their respective axiomatic specifications. 

%\textbf{Main points: need to make sure that these are clear throughout the text.}
%\begin{itemize}
%\item Proof technique for proving the equivalence of specifications using kv-store 
%with those using abstract executions, which is based into mapping $\ET$-traces 
%for a given execution test $\ET$ into sets of abstract executions satisfying the set of axioms $\Ax$.
%\item The core of the proof technique is rooted in the equivalence between the most permissive 
%execution test $\ET$ with Eventual Consistency/Read Atomic (no axioms), which we prove first.
%We show that any $\ET_{\top}$-trace terminating into a kv-store $\mkvs$ can be mapped into a 
%\textbf{maximal} set of abstract executions whose underlying kv-store is $\mkvs$, and that 
%any abstract execution $\aexec$ with underlying kv-store $\mkvs$ can be mapped into a \textbf{maximal} set of 
%$\ET_{\top}$-traces. Both definitions are by induction. 
%I think the maximal part was not stressed before, and it is kind of important. 
%
%\item Our proof technique reduces the proof obligations for proving the equivalence of 
%an execution test $\ET$ with an axiomatic specification $\Ax$ to the commit of a single 
%transaction. Soundness amounts to prove that each kv-store $\mkvs$ generated by $\ET$ 
%can be obtained as the underlying kv-store of some abstract execution that satisfies $\Ax$. 
%To do this, we reuse the construction adopted for the most permissive execution test. 
%
%\end{itemize}

%Abstract executions provide another formalism for specifying
%transactional consistency models. 
\subsubsection{Abstract Executions}
Abstract executions are labelled graphs 
whose nodes comprise transaction identifiers and their fingerprints. 
These nodes may be connected either by a \emph{visibility edge}, $\txid \xrightarrow{\VIS} 
\txid'$, when $\txid'$ sees the updates of $\txid$;  or an \emph{arbitration edge}, $\txid \xrightarrow{\AR} 
\txid'$, when $\txid'$ updates overwrite $\txid$ updates.
\Cref{fig:abstract_execution} depicts an example abstract execution.
%An abstract execution is a directed,
%labelled graph whose nodes consist of transaction identifiers and  their 
%associated fingerprints, 
%and  %are labelled with a transaction identifier and the associated fingerprint. 
% edges labelled  $\VIS$
%or $\AR$ describe  information  which is locally available to  transactions: 
%\begin{itemize}
%\item edge $\txid \xrightarrow{\VIS} \txid'$ denotes that  when transaction 
%$\txid'$ is executed  it sees the updates of $\txid$; and 
%\item edge $\txid \xrightarrow{\AR} \txid'$ denotes  that updates 
%performed by $\txid'$ are newer than those of $\txid$. 
%\end{itemize}
% \emph{Abstract executions} \cite{ev_transactions} 
%are directed graph with its nodes representing transactions (with each node labelled with a transaction identifier and a set of (read/write) operations), 
%and its edges representing certain relations between transactions. 
%An example abstract execution graph is depicted in \cref{fig:abstract_execution}. 
%Each node is . 
%Each edge is labelled by either the \emph{visibility} ($\VIS$) or \emph{arbitration} ($\AR$) relation. 
%The $\VIS$ is an irreflexive order on transactions such that $\txid_1 \xrightarrow{\VIS} \txid_2$ denotes that the effects (updates) of $\txid_1$ are visible to $\txid_2$. 
%The $\AR$ is a strict total order on transactions describing the commit order:  $\txid_1 \xrightarrow{\AR} \txid_2$ denotes that $\txid_1$ commits before $\txid_2$. 
%Moreover, $\AR$ contains $\VIS$ ($\VIS \subseteq \AR$) and agrees with the session order: 
%if $\txid_{\cl}^{n} \xrightarrow{\AR}  \txid_{\cl}^{m}$, then $n < m$.
\begin{definition}[Abstract executions]
\label{def:main-body-absexec}
\label{def:main-body-aexec}
An {\em abstract execution} is a triple $\aexec = (\TtoOp{T}, \VIS, \AR)$, where 
 $\TtoOp{T}: \TxID \parfun \pset{\Ops}$ is a partial  
 %$\TtoOp{T}: \TxID_{0} \parfinfun \pset{\Ops}$ is a partial, finite 
function mapping transaction identifiers to 
fingerprints, with $\TtoOp{T}(\txid_{0}) = \Set{ (\otW, \key, \val_{0}) }[ \key \in \Keys]$, 
$\VIS \subseteq \dom(\TtoOp{T}) \times \dom(\TtoOp{T})$ is an irreflexive relation 
such that, for any $\txid \in \dom(\TtoOp{T})$, $\txid_{0}
\xrightarrow{\VIS} \txid$ for the initial transaction $\txid_0$, and 
%called \emph{visibility}, 
$\AR \subseteq \dom(\TtoOp{T}) \times \dom(\TtoOp{T})$ is a strict, total order 
such that $\VIS \subseteq \AR$, $\min_{\AR}(\dom(\TtoOp{T})) = \txid_{0}$
and $\txid_{\cl}^{n} \xrightarrow{\AR} \txid_{\cl}^{m}$ only if $n < m$. 
%Furthermore, for any transaction $\txid \in \dom(\TtoOp{T})$, and $\key \in \Keys$, let 
%$\visibleWrites_{\aexec}(\key, \txid) = \VIS^{-1}(\txid) \cap \{\txid' \mid (\otW,\key, \stub) \in \TtoOp{T}(\txid')\}$; 
%we require that whenever  $(\otR, \key, \val) \in \TtoOp(\txid)$ for some $\key, \val$ 
%and $\txid \in \dom(\TtoOp{T})$, then either $\visibleWrites_{\aexec}(\key, \txid) = \emptyset$ and 
%$\val$ is the default value $\val_{0}$, or the transaction $\txid' = \max_{\AR}(\visibleWrites_{\aexec}(\key, \txid))$ 
%is defined and $(\otR, \key, \val) \in \TtoOp{T}(\txid')$.
\end{definition}
Given an abstract execution $\aexec = (\TtoOp{T}, \VIS, \AR)$,  we let $\TtoOp{T}_{\aexec} = \TtoOp{T}$, 
$\VIS_{\aexec} = \VIS$ and $\AR_{\aexec} = \AR$. 
We write $(l, \key, \val) \in_{\aexec} \txid$ as a shorthand for $(l, \key, \val) \in \TtoOp{T}_{\aexec}(\txid)$.
%The set of abstract executions is denoted by $\aeset$.
%Intuitively, $\txid \xrightarrow{\VIS_{\aexec}} \txid'$ means that, at the moment of executing, 
%transaction $\txid'$ sees the updates performed by $\txid$; $\txid \xrightarrow{\AR_{\aexec}} \txid'$ 
%means that the updates performed by $\txid'$ are newer than those performed by $\txid$. 
For $\txid \in \T_{\aexec}$, 
%$\txid: (\otR, \key, \val) \in_{\aexec} \txid$, \azalea{I don't know what this notation is saying}
we define its \emph{visible writes} in $\aexec$ as 
$\visibleWrites_{\aexec}(\key, \txid) \defeq \VIS^{-1}_{\aexec}(\txid) \cap 
\{\txid' \mid (\otW, \key, \stub) \in_{\aexec} \txid'\}$. 
An abstract execution $\aexec$ satisfies the \emph{last-writer-wins} policy: if
a transaction \( \txid \) reads key \( \key \), 
it must read from the latest transaction in the arbitration order that is visible to $\txid$ and wrote to key $\key$,
\ie $\forall \txid \in \T_{\aexec}.\; (\otR,\key,\val) \in_{\aexec} \txid 
\implies (\otW, \key, \val) \in \max_{\AR_{\aexec}}(\visibleWrites_{\aexec}(\key, \txid))$.
%We focus on  abstract executions that satisfy the
%\emph{last-writer-wins} resolution policy, which  states that, if transaction $\txid$ reads value $\key$ in $\aexec$, then it reads the most up-to-date version among 
%the ones written by the transactions it sees: formally,  if $\visibleWrites_{\aexec}(\key, \txid) = \VIS^{-1}_{\aexec}(\txid) \cap 
%\{ \txid' \mid (\otW, \key, \stub) \in_{\aexec} \txid'\}$ then $\forall \key, \val.\; (\otR,\key,\val) \in_{\aexec} \txid
%\implies (\otW, \key, \val) \in \max_{\AR_{\aexec}}(\visibleWrites_{\aexec}(\key, \txid))$.
Henceforth we assume that abstract executions satisfy the last-writer-wins policy, 
and we let $\Aexecs$ be the set of all such abstract executions.

%
%The constraint in \cref{def:main-body-aexec} amounts to require that a transaction reading 
%key $\key$ always fetches the newest write over $\key$ performed in its set of visible transactions, 
%thus modelling the last write wins policy. An example of abstract execution is given in \cref{fig:abstract_execution}.
%Consistency models for  abstract executions are specified by
%providing axioms which 

Abstract-execution-based specifications of consistency models constrain the overall structure of abstract executions. 
For most consistency models \cite{laws,framework-concur,sureshConcur}, 
such constraints are over the set of transactions that  \textbf{must} be seen  by 
other transactions. For example, monotonic reads is specified by requiring 
that if a transaction $\txid$ follows another transaction $\txid'$ in the session order, 
then $\txid$ must see all transactions that are seen by $\txid'$.
Serialisability can be specified by requiring that 
%in an abstract execution $\aexec$, 
a transaction $\txid$ see all transactions preceding $\txid$ in the arbitration order.
%form $\mathcal{R}_{\aexec} \subseteq \VIS_{\aexec}$.

Formally, an axiomatic specification $\Ax$ is a set of {\em axioms} $\A : \Aexecs \to \pset{\TxID \times \TxID}$,
where  $\forall \aexec.\;\A(\aexec) \subseteq \AR_{\aexec}$. 
We write $\aexec \models \A$ when $\A(\aexec) \subseteq \VIS_{\aexec}$.
We refer the reader to \cref{sec:abstract-execution} for details about abstract executions.

Returning to the monotonic reads (\MR) example, 
we define $\Ax_{\MRd} \defeq \{\A_{\MRd}\}$, where $\A_{\MRd}(\aexec) \defeq \VIS_{\aexec} ; \SO_{\VIS}$. 
By definition, for a given $\aexec$, $\aexec \models \A_{\MRd}$ if and only 
if $\VIS_{\aexec} ; \SO_{\aexec} \subseteq \VIS_{\aexec}$. 
That is, whenever $\txid'' \xrightarrow{\VIS_{\aexec}} \txid' \xrightarrow{\PO_{\aexec}} \txid $, 
then $\txid'' \xrightarrow{\VIS_{\aexec}} \txid$.
Similarly, for serialisability (\SER) we define $\Ax_{\SER} \defeq \{ \A_{\SER} \}$, where $\A_{\SER}(\aexec) \defeq \AR_{\aexec}$, 
captures the constraint that a transaction $\txid$ must see all transactions
preceding it in the arbitration order.
%The  { consistency
%  model} generated by $\Ax$ is defined by
%$\CMs(\Ax) = \{\aexec \mid \aexec \text{ satisfies } \Ax\}$. 
%Examples of such consistency models include {strict
%  serialisability}  given by  the set of abstract executions $\aexec$ such that
%$\AR_{\aexec} \subseteq \VIS_{\aexec}$ and  $\PO_{\aexec} \subseteq \VIS_{\aexec}$, 
%and {read atomic} ~\cite{ramp} given by  the set of abstract
%executions without constraints. 

Any abstract executions $\aexec$ can be mapped 
into an equivalent dependency graph $\Gr_{\aexec}$ (\citet{laws}), and hence into a kv-store 
$\hh_{\aexec}$ (\cref{thm:kv_graph_isomorph}). 
\sx{straightforward? Maybe better words like: 
we refer the readers to \cite{laws} for mapping an abstract execution $\aexec$ into an
equivalent dependency graph $\Gr_{\aexec}$. } 
We can then use this construction to define the consistency model induced by an abstract-execution-based specification 
$\CMs(\Ax)$ by projecting abstract executions that satisfy the axioms in $\Ax$ to kv-stores: 
$\CMs(\Ax) \defeq \{ \hh_{\aexec} \mid \forall \A \in \Ax. \aexec \models \A \}$. 

In the remainder of this section we develop a proof techniques for showing  
that an execution test $\ET$ and an axiomatic specification $\Ax$ induce the same 
consistency model, \ie$\CMs(\ET) {=} \CMs(\Ax)$. 
Due to space constraints, we focus only on \emph{soundness}, \ie on proving the left-to-right inclusion: $\CMs(\ET) \subseteq \CMs(\Ax)$; 
we describe the other direction in full in \cref{sec:et-sound-complete-constructor}. 
The core of our proof technique lies in the soundness of the most permissive execution 
test, $\CMs(\ET_{\top})$, with respect to the weakest axiomatic specification, given by 
the empty set of axioms, which we prove next.
%lifting this mapping, we can convert a consistency model for abstract
%executions to a consistency model for kv-stores.  
%In~\cref{app:et_sound_complete}, 
%we prove that all the consistency models
%for kv-stores given in~\cref{fig:execution-tests} are equivalent to
% consistency models for abstract
%executions, as  defined
%in the literatures~\cite{principle-eventual-consistency,surech-session-guarantee,framework-concur,laws}.
%This result is based on a 
%generic technique for proving that axiomatic specification $\Ax$ is
%equivalent to an execution test $\ET$, that is,
%$\aexec \in \CMs(\Ax) \Leftrightarrow \mkvs_{\aexec} \in \CMs(\ET)$.
%We describe this generic technique here. We first prove the equivalence of the empty set of axioms, which
%gives rise to the read atomic consistency model for abstract
%executions, and the most permissive execution test $\ET_{\top}$, which
%gives rise to read atomic for kv-stores. Using this result, we 
%demonstrate how to prove the equivalence of axiomatic
%specifications and execution tests in general. 
%
%In \cref{app:et_sound_complete} we prove that these are equivalent to the consistency models 
%defined by the corresponding execution tests. In the following, we focus on proving the 
%equivalence between the axiomatic specification of \emph{Read Atomic}
%%We defer the details of this mapping to \cref{app:depgraphs}.

%
%We defer the formal details of such a construction to \cref{app:depgraphs}, 
%but we mention here that in such a construction, we draw an edge of the form $\txid \xrightarrow{\RF} \txid'$ 
%if $\txid = \max_{\AR_{\aexec}}(\visibleWrites_{\aexec}(\key, \txid'))$ and $(\otR, \key, \stub) \in \TtoOp{T}(\txid')$ for some $\key \in \Keys$, 
%while $\txid \rightarrow{\VO} \txid'$ if $(\otW, \key, \stub) \in \TtoOp{T}(\txid), (\otW, \key, \stub) \in \TtoOp{T}(\txid')$ and 
%$\txid \xrightarrow{\AR_{\aexec}} \txid'$; edges labelled as $\AD$ can be uniquely determined from edges labelled as 
%$\RF$ and $\VO$.
%\sx{If we dont have space. we can cut the technical details here}

%a variant of \emph{Eventual Consistency} 
%\cite{ev_transactions} that assumes the last-write-win resolution policy.

\subsubsection{Equivalence of Read Atomic and $\CMs(\ET_{\top})$} 
\sx{high level comments: read atomic -> snapshot property? }
%We develop the methodology  to show that 
%axiomatic specifications for abstract executions are equivalent to
%execution tests  for kv-stores. 
The weakest axiomatic specification, given by the empty set of 
axioms, corresponds to the \emph{Read Atomic} consistency model \cite{ramp}. 
To prove that the most permissive execution test $\ET_{\top}$ is sound with 
respect to the weakest axiomatic specification, we map $\ET_{\top}$-traces 
terminating in a configuration of the form $(\hh, \stub)$, into the set of 
abstract executions whose underlying kv-store is $\hh$. 
%We show that the  axiomatic specification $\emptyset$ for abstract
%execution $\aexec$  is equivalent to the
%execution test $\ET_{\top}$ for kv-store $\hh_{\aexec}$. The full details are given
%in~\ref{sec:app-abstract-semantics-sound-complete}. 


%$\ET_{\top} \vdash (\hh, \vi) \csat \opset, (\mkvs',  \vi')$ for any $\hh, \hh', \vi, \vi', \opset$. 
%We show that there is a close connection between the set $\ET_{\top}$-traces and 
%$\aeset$. 
\begin{theorem}
\label{thm:kvtrace2aexec}
%Given an $\ET_{\top}$-trace $\tau = (\mkvs_{0}, \vienv_{0}) \toET{(\cl_{1}, \alpha_{1})} \cdots \toET{(\cl_{n}, \vienv_{n})} (\mkvs_{n}, \vienv_{n})$, 
%there exists a non-empty set of abstract executions $\execs(\tau) = \{\aexec_i\}_{i = 0}^{n}$ such that, for any $i = 0,\cdots,n$, 
%$\hh_{\aexec_{i}} = \hh_{i}$, and the order in which transactions are executed in $\tau$ is consistent with $\AR_{\aexec_{n}}$. 
Given an $\ET_{\top}$-trace $\tau$ terminating in $(\hh, \_)$, 
there exists a non-empty set of abstract executions $\execs(\tau)$
such that: $\forall \aexec \in \execs(\tau).\ \hh_{\aexec} = \hh$, 
and the order in which transactions are executed in $\tau$ is consistent with $\AR_{\aexec}$. 
\end{theorem}
\noindent 
The proof of \cref{thm:kvtrace2aexec} is highly
non-trivial, and relies on the following intuition that drives the construction of the set $\execs(\tau)$: 
whenever a client $\cl$ in $\tau$  with view $\vi$ commits a transaction $\txid$, then in all 
abstract executions included in $\execs(\tau)$, transaction $\txid$ must see the writers  
of the versions included in $\vi$, and it never sees the writers of versions not included in $\vi$ (\cref{fig:et-sound-to-aexec}). 
These are defined by the set $\Tx[\hh, \vi]$ (defined in \cref{subsec:cm_examples}). Furthermore, 
abstract executions in $\execs(\tau)$ differ only in the set of read-only transactions (\ie those with no write operations) 
that transactions see. While mapping an $\ET_{\top}$-trace into multiple abstract executions is 
not strictly necessary for proving the soundness of $\ET_{\top}$ with respect 
to the weakest axiomatic specification, it plays a significant role when proving the soundness 
of an arbitrary execution test $\ET$ with respect to its counterpart in axiomatic specifications. 
%The full details of the proof of \cref{thm:kvtrace2aexec} can be found in
%\cref{sec:kvtrace2aexec}.  Here, we  focus on the construction of the set $\execs(\tau)$. 

The definition of $\execs(\tau)$ is by induction on the length of the $\ET_{\top}$-trace
$\tau$. For the base case with $\tau_{0}$ consisting of a single configuration 
$(\hh_{0}, \stub)$, we define $\execs(\tau_{0})$ to contain a single abstract execution with 
a single transaction $\txid_{0}$ that initialises all the keys to the initial value $\val_{0}$:
$\execs(\tau_{0}) \defeq \{ ([\txid_{0} \mapsto \{ (\otW, \key, \val_{0} \mid \key \in \Keys)\}], \emptyset, \emptyset \}$. 
For the inductive case with $\tau {=} \tau' \toET{(\cl, \alpha)} (\hh', \vienv')$, let $(\hh, \vienv)$ be the last 
configuration appearing in $\tau'$. 
%Suppose also that $\aexec(\tau')$ has been already defined. 
If $\alpha {=} \varepsilon$, then $\execs(\tau) \defeq \execs(\tau')$. 
If $\alpha {=} \fp$ for some $\fp$, we first determine the transaction identifier $\txid'$ that was used to commit $\fp$ in $\hh'$, 
the view $\vi' = \vienv'(\cl)$ of the client $\cl$ when committing $\txid'$, the 
set of transactions that $\cl$ must see when committing $\txid'$, given by 
$\Tx[\hh', \vi']$, and the set of read-only transactions $\txidset_{\rd}$ in $\hh'$: 
the latter are those transactions that never appear as writers. 
Then, for all abstract execution $\aexec' \in \execs(\tau')$, we define $\extend[\aexec', \txid', \Tx[\mkvs',\vi'], \fp]$ as the largest set 
such that, whenever $\aexec \in \extend[\aexec', \txid', \Tx[\mkvs',\vi'], \fp]$, then 
\begin{enumerate*}
\item $\TtoOp{T}_{\aexec} = \TtoOp{T}_{\aexec'}\rmto{\txid'}{\fp}$;
\item  for all $\txid' \in \T_{\aexec}$, 
$\txid' \xrightarrow{\AR_{\aexec'}} \txid$; and 
\item if $\txid' \xrightarrow{\VIS_{\aexec}} \txid$, 
then either $\txid \in \Tx[\hh', \vi']$, or $\txid \in \txidset_{\rd}$.  
\end{enumerate*}
Finally, we define $\execs(\tau) \defeq \bigcup_{\aexec \in \execs(\tau')} \extend[\aexec, \txid, \Tx[\mkvs',\vi'], \txidset_{\rd}, \fp]$. 
In \cref{sec:kvtrace2aexec,sec:aexectrace2kv} we give the full details. 
To understand the construction outlined above, we 
illustrate one use of the function $\extend$. The abstract 
execution $\aexec$ to the left of \cref{fig:et-sound-aexec-update} has underlying kv-store $\hh$, 
depicted to the left of \cref{fig:et-sound-kv-store-update}. If a client commits a transaction 
$\txid_{3}$ that reads the last version of $\key_1$, then the resulting kv-store $\hh'$ would be the one 
to the right of \cref{fig:et-sound-kv-store-update}. This commit is simulated by the function 
$\extend[\aexec', \txid_{3}, \Tx[\hh, \vi], \fp]$, where $\vi, \fp$ are the view and fingerprint used to 
update $\hh$ to $\hh'$: the result of this function consists of two abstract executions $\aexec_1, \aexec_2$, 
that only differ in read-only transactions (the right of \cref{fig:et-sound-kv-store-update}).
The visibility edges of $\aexec_1$ are exactly the concrete edges in \cref{fig:et-sound-aexec-update}; however, 
$\aexec_2$ has the extra dashed visibility edge $\txid_{2} \xrightarrow{\VIS} \txid_{3}$. 
Note that $\hh_{\aexec_2} = \hh_{\aexec_3} = \hh'$.

\begin{figure}[t]
\captionsetup[subfigure]{aboveskip=0pt, belowskip=5pt}
\begin{subfigure}{0.95\textwidth}
\begin{centertikz}%
\node(locx) {$\key_1 \mapsto$};
\path (locx.west) + (-1,0) node (k) {$\hh = $};
\draw pic at ([xshift=\tikzkvspace]locx.east) {vlist={versionx}{%
    /$0$/$\txid_0$/$\emptyset$
    , /$1$/$\txid_1$/$\Set{\txid_2}$
}};

\path (versionx.east) + (3,0) node (locy) {$\key_1 \mapsto$};
\draw pic at ([xshift=\tikzkvspace]locy.east) {vlist={versiony}{%
    /$0$/$\txid_0$/$\emptyset$
    , /$1$/$\txid_1$/$\Set{\txid_2, \txid_3}$
}};

\draw[->,
line join=round,
decorate, decoration={
    zigzag,
    segment length=4,
    amplitude=.9,post=lineto,
    post length=2pt
}
] ($(versionx.east) + (0.5,0)$) -- ($(locy.west) + (-0.5,0)$);
\path (versiony.east) + (1,0) node (k1) {$= \hh'$};

\end{centertikz}%
\caption{Commit \( \txid_3 \) in kv-store}
\label{fig:et-sound-kv-store-update}
\end{subfigure}

\hrulefill

\begin{subfigure}{0.95\textwidth}
\begin{centertikz}[.66]%
\draw pic {transaction={t0}{%
        /$\otW$/$\key_1$/$0$%
}};
\path(t0.west) + (0,0) node[anchor=east] (a0) {$\aexec = $};
\path(t0.north) node[anchor=south] (t0lbl) {$\txid_0$};

\draw pic at ($(t0.east) + (1.6,0.15)$) {transaction={t1}{%
        /$\otW$/$\key_1$/$1$%
}};
\path(t1.north) node[anchor=south] (t1lbl) {$\txid_1$};

\draw pic at ($(t1.east) + (1.6,0.15)$) {transaction={t2}{%
        /$\otR$/$\key_1$/$1$%
}};
\path(t2.north) node[anchor=south] (t2lbl) {$\txid_2$};

\path[->]
(t0.east) edge node[above, yshift=0pt, xshift=0pt, pos=0.5] {$\VIS$} (t1.west)
(t0.south) edge[bend right=20] node[above, yshift=0pt, xshift=0pt, pos=0.5] {$\VIS$} (t2.south)
(t1.east) edge node[above, yshift=0pt, xshift=0pt, pos=0.5] {$\VIS$} (t2.west);

\draw pic at ($(t2.east) + (2,0.15)$) {transaction={t00}{%
        /$\otW$/$\key_1$/$0$%
}};
\path(t00.north) node[anchor=south] (t00lbl) {$\txid_0$};

\draw pic at ($(t00.east) + (2,0.15)$) {transaction={t11}{%
        /$\otW$/$\key_1$/$1$%
}};
\path(t11.north) node[anchor=south] (t11lbl) {$\txid_1$};

\draw pic at ($(t11.east) + (2,0.15)$) {transaction={t22}{%
        /$\otR$/$\key_1$/$1$%
}};
\path(t22.north) node[anchor=south] (t22lbl) {$\txid_2$};

\draw pic at ($(t22.east) + (2,0.15)$) {transaction={t3}{%
        /$\otR$/$\key_1$/$1$%
}};
\path(t3.east) + (0,0) node[anchor=west] (a1) {$= \{\aexec_1, \aexec_2\}$};
\path(t3.north) node[anchor=south] (t3lbl) {$\txid_3$};

\path[->]
(t00.east) edge node[above, yshift=0pt, xshift=0pt, pos=0.5] {$\VIS$} (t11.west)
(t00.south) edge[bend right=17] node[above, yshift=0pt, xshift=0pt, pos=0.5] {$\VIS$} (t22.south)
(t00.south) edge[bend right=17] node[above, yshift=0pt, xshift=0pt, pos=0.5] {$\VIS$} (t3.south)
(t11.north east) edge[bend left=28] node[above, yshift=0pt, xshift=0pt, pos=0.5] {$\VIS$} (t3.north west)
(t11.east) edge node[above, yshift=0pt, xshift=0pt, pos=0.5] {$\VIS$} (t22.west);

\path[densely dotted,->]
(t22.east) edge node[above, yshift=0pt, xshift=0pt, pos=0.5] {$\VIS$} (t3.west);

\draw[->,
line join=round,
decorate, decoration={
    zigzag,
    segment length=4,
    amplitude=.9,post=lineto,
    post length=2pt
}
] ($(t2.east) + (0.3,0)$) -- ($(t00.west) + (-0.3,0)$);

\end{centertikz}%
\vspace{-5pt}
\caption{Committing \( \txid_3 \) in abstract executions. For simplicity, arbitration edges have been omitted}
\label{fig:et-sound-aexec-update}
\end{subfigure}

\hrulefill

\label{fig:et-sound-to-aexec}
\caption{Correspondence between committing \( \txid_3 \) in kv-stores and abstract executions. 
The figure to the right denotes a set of abstract executions, which differ in the presence of the dashed visibility edge.}
\end{figure}


%first, for any 
%%For any $\aexec \in \execs(\tau)$, we define a set of abstract execution $\extend[\aexec, 
%%\txid, \Tx[\mkvs',\vi'], \txidset_{\rd}, \fp]$ as follows: 
%%$\aexec''$ as follows: 
%We define $\execs(\tau)$ using the following process: 
%\textbf{Leaving this here: It would be really nice if we could have a picture outlining the construction!}
%\textbf{(i)} we append the transaction $\txid$ at the end of $\AR_{\aexec}$, and by associating the transaction 
%$\txid$ with the fingerprint $\txid$; \textbf{(ii)} for each $\txid' \in \Tx[\hh,\vi]$, we add a visibility edge, 
%\item  we determine the set of read-only transactions in $\hh$: for each of its subsets 
%$\txidset_{\rd}$, we extend $\aexec'$ by adding a visibility edge from each transaction in 
%$\txidset_{\rd}$ to $\txid$; the result is a set of abstract executions, which we denote as 
%$\extend[\aexec,\txid, \fp]$.
%\item The set $\aexec_{\tau}$ is defined by  lifting the 
%construction above from $\ET_{\top}$-traces to the 
%set of $\ET_{\top}$-traces in $\exec(\tau')$: $\exec(\tau) = \bigcup_{\aexec \in \exec(\tau')} 
%\extend[\aexec, \txid, \fp]$.
% \extend[\aexec,\txid,\Tx[\mkvs,\vi] \cup \txidset_\rd, \fp]$

%The proof of \cref{thm:kvtrace2aexec} is highly
%non-trivial. The full details can be found in
%\cref{sec:kvtrace2aexec}.   Here, we  focus on  the construction of the set
%$\execs(\tau)$. 
%Let $\tau$ be an $\ET_{\top}$-trace. The construction of the set $\execs(\tau)$ is incremental. 
%We start with the singleton set $\execs_{0}(\tau)$ containing an abstract execution where the initial transactions $\txid_{0}$ writes 
%the initial value $\val_{0}$ for each $\key$ in the kv-store: 
%$\execs_{0}(\tau) = \{(\emptyset, \emptyset, \emptyset)\}$. 
%$\execs_{0}(\tau) = \{([\txid_{0} \mapsto \{ (\otW, \key, \val_{0} \mid \key \in \Keys)\}], \emptyset, \emptyset)\}$. 
%Then for $i=0,\cdots, n-1$, we define $\execs_{i+1}(\tau)$ from $\execs_{i}(\tau)$ and the $\ET_{\top}$-reduction 
%$(\mkvs_{i}, \vienv_{i} \toET{(\cl_{i}, \alpha_{i})} \mkvs_{i+1}, \vienv_{i+1})$. Finally, we define $\execs(\tau) \defeq \execs_{n}(\tau)$.
%\ac{Here there is an inconsistency: if we want $\txid_{0}$ to be included in abstract executions, then initially we 
%must set $\execs(\tau) = ([\txid_{0} \mapsto \{ (\otW, \key, \val_{0} \mid \key \in \Keys)\}], \emptyset, \emptyset$.} 

%Let us show how $\execs_{i+1}(\tau)$ is defined, for $i=0,\cdots,n-1$: 
%%inductively from $\execs_{i-1}(\tau)$. 
%If the $i$-th action of $\tau$ is a view-shift, i.e. $\alpha_{i} = \varepsilon$, then 
%we let $\execs_{i+1}(\tau) = \execs_{i}(\tau)$. Otherwise, the $i$-th action of $\tau$ consists of client $\cl_{i}$
%committing a transaction $\txid_{i}$ with fingerprint $\fp_{i}$, and we follow a two step process for constructing $\execs_{i+1}(\tau)$: first, we extend each of the 
%executions in $\execs_{i}(\tau)$ by inserting the mapping $(\txid_{i} \mapsto \fp_{i})$ as the last element of its arbitration order; this operation 
%results in an intermediate set of abstract executions, $\execs_{i}^{\mathsf{partial}}(\tau)$. Second, for each 
%abstract execution $\aexec^{\mathsf{partial}} \in \execs_{i}^{\mathsf{partial}}(\tau)$ we determine a set of sets of visible transactions for $\txid_{i}$:  
%each of these sets is used to extend the visibility relation of $\aexec_{\mathsf{partial}}$, and the resulting abstract execution  is included in  the final 
%set $\aexec_{i+1}(\tau)$. Each visibility set for $\txid_{i}$ must contain at least the writers of the versions contained in $\vienv_{i-1}(\cl_{i})$, 
%$\Tx[\hh_{i}, \vienv_{i}]$, because we know that read operations in $\fp_{i}$ fetch their value of a key $\key$ from the 
%last version of $\mkvs_{i}$ included in $\vi_{i}$; but it may also contain any other read-only transaction 
%(excluded $\txid_{i}$) that is already present in the abstract execution $\aexec_{\mathsf{partial}}$.

%\sx{
%I had a go, can someone check:
%If the $i$-th action of $\tau$ is a view-shift, that is $\alpha_{i} = \varepsilon$, then $\execs_{i+1}(\tau) = \execs_{i}(\tau)$.
%Otherwise, the $i$-th action of $\tau$ consists of client $\cl_{i}$ with view \( \vi_i = \vienv_i(\cl_i) \)
%committing a transaction $\txid_{i}$ with fingerprint $\fp_{i}$.
%We construct the next step $\execs_{i+1}(\tau)$ by adding the new transaction with appropriate edges to any abstract execution \( \aexec_i \in \execs_{i}(\tau) \): 
%first, we add the new transaction \( (\txid_{i} \mapsto \fp_{i}) \) to \( \TtoOp{T}_{\aexec_i} \) and
%second, $\txid_{i}$ is appended to the end of \( \AR_{\aexec_i} \) since it is the newest so far;
%these operations result in an intermediate abstract execution, written $\aexec_{i+1}^{\mathsf{partial}}$.
%Last, we construct a set of visible transactions \( \txidset_i \) for $\txid_{i}$
%which is used to extend the visibility relation of $\aexec_{i+1}^{\mathsf{partial}}$.
%This means in the new abstract execution \( \aexec_{i+1} \in \execs_{i+1}(\tau) \), \( \fora{\txid \in \txidset} (\txid,\txid_i) \in \VIS_{\aexec_{i+1}} \).
%Note that \( \txidset_i \) is not unique and
%all of them must contain at least the writers of the versions contained in $\vi_i$, 
%that is, $\Tx[\mkvs_{i}, \vi_{i}]$, but may contain different read-only transactions.
%}

%%Formally, consider the $i$-th $\ET_{\top}$-reduction of $\tau$, $(\hh_{i-1}, \vienv_{i-1}) \xrightarrow{(\cl_{i}, \alpha_{i})}_{\ET_{\top}} (\hh_i, \vienv_{i})$, 
%%and assume that $\alpha_{i} = \opset$ for some $\opset$. Let 
%%If $\alpha_{i} = \varepsilon$, then we let $\execs_{i}(\tau) = \execs_{i-1}(\tau)$. Otherwise, 
%%if $\alpha_{i} = \opset$, let 
%$\{\txid_{i}\} := \hh_{i} \setminus \hh_{i-1}$; 
%%we first identify the transaction identifier that is associated to $\opset$ in $\tau$: this 
%%is the only transaction identifier $\txid_{i}$ that appears in $\hh_{i}$ and not in $\hh_{i-1}$; 
%we extend the arbitration order of each abstract execution $\aexec \in \execs_{i}(\tau)$ by appending the transaction identifier 
%$\txid_{i}$ as its last  transaction, and we let $\execs_{i}^{\mathsf{partial}}(\tau)$ be the resulting set of 
%abstract executions. Then, for each abstract execution in this set, we determine a set of suitable 
%local visibility relations for the transaction $\txid_{i}$, i.e. the possible sets of transactions that 
%$\txid_{i}$ sees.  Each of these sets must contain at least the writers of the versions contained $\vienv_{i-1}(\cl_{i})$ in the 
%view of client $\cl_{i}$ in $\vienv_{i-1}$, defined by $\Tx[\hh_{i-1}, \vienv{i-1}]$; but it may also contain any other read-only transaction 
%that is already present in the abstract execution. 
%%Formally, given a kv-store $\hh$ and a view $\vi$, we let 
%%$\Tx[\hh, \vi] \defeq \Setcon{ \WTx(\hh(\key, i)) }{ \key \in \Keys \wedge i \in \vi(\key) }$. 
%That is, for each 
%abstract execution $\aexec \in \execs_{i}^{\mathsf{partial}}$ and each subset $\T_{\mathsf{rd}}$ of 
%read-only transactions in $\aexec$, we extend the visibility relation $\VIS_{\aexec}$ 
%with the pairs $\Set{(\txid, \txid_{i}) }[  \txid \in \Tx[\hh_{i-1}, \vienv{i-1}(\cl_{i})] \cup \txidset_\rd]$, 
%and we include the resulting abstract execution in $\aexec \in \execs_{i}(\tau)$. Finally, we 
%let $\execs(\tau) = \execs_{n}(\tau)$.

%We let $\execs_{i}^{\mathsf{partial}(\tau)} = \{(\TtoOp{T}_{\aexec}[\txid_{i} \mapsto \opset], \VIS_{\aexec}, 
%\AR_{\aexec} \cup \{(\txid, \txid_{i} \mid \txid \in \dom\TtoOp{T}(\aexec)\} \mid \aexec \in \execs_{i}(\tau)\}$. 
%Then, for each $\aexec \in \execs_{i}^{\mathsf{partial}(\tau)}$, we determine the set of transactions that  

\begin{theorem}
\label{thm:aexec2kvtrace}
Given an abstract execution $\aexec$, there exists a non-empty 
set of $\ET_{\top}$-traces $\{\tau_{i}\}_{i \in I}$ such that, for each $i \in I$, the last configuration of $\tau_{i}$ is 
$(\hh_{\aexec}, \_)$, and $\tau_{i}$ executes transactions in the order established by $\AR_{\aexec}$. 
\end{theorem}
\noindent The proof of \cref{thm:aexec2kvtrace} is given 
in \cref{sec:kvtrace2aexec,sec:aexectrace2kv}. 
\Cref{thm:kvtrace2aexec,thm:aexec2kvtrace} together establish the
equivalence  of $\ET_{\top}$ with the weakest axiomatic specification. 
%$\CMs(\ET_{\top}) = \CMs(\emptyset)$ is sound and complete with respect to the set of abstract executions
%allowed by the read atomic consistency model. 

\subsubsection{Equivalence of axiomatic specifications and execution tests}
We are now ready to present our proof technique for proving the soundness 
of an execution test $\ET$ with respect to an axiomatic specification $\Ax$.
It can be summarised as follows: 
the user considers an arbitrary 
$\aexec: \aexec \models \Ax$ , and a tuple of the form 
$\ET \vdash (\hh_{\aexec}, \vi) \csat \opset: (\hh', \vi')$. 
Then, it constructs a non-empty subset of $\extend[\aexec, \txid, \Tx[\hh, \vi], \fp]$ 
whose elements satisfy the axioms $\Ax$. Note that, because abstract executions in
$\extend[\aexec, \txid, \Tx[\hh, \vi], \fp]$ differ only in visibility edges of the form $\txid_{\mathsf{rd}}
\xrightarrow{\VIS} \txid$, where $\txid_{\mathsf{rd}}$ is a read-only transaction, 
then constructing the set above amounts to identifying a suitable set of 
read-only transactions in $\aexec$.

To see why our proof technique guarantees the soundness of $\ET$ with respect 
to $\Ax$ (\cref{thm:main-body-et_soundness}), we apply an inductive argument over the number of $\ET$-reductions $n$ in a $\ET$-trace $\tau$: 
first, if $n = 0$ then $\tau = (\mkvs_{0}, \vienv_{0})$, and $\aexec_{0} \in \execs(\tau)$ trivially satisfies the axioms
$\Ax$: $\forall \A \in \Ax. \A(\aexec_{0}) \subseteq \AR_{\aexec_{0}} = \emptyset \subseteq \VIS_{\aexec_{0}}$. 
Otherwise, if $\tau = \tau' \xrightarrow{(\cl, \fp)} (\mkvs, \vi)$, 
suppose that there exists an abstract execution $\aexec' \in \execs(\tau')$ that satisfies 
the axioms $\Ax$. If the proof obligations of our proof technique are satisfied, we can construct an abstract execution 
$\aexec \in \extend[\aexec, \txid, \Tx[\hh, \vi], \fp] \subseteq \aexec(\tau)$ that satisfies $\Ax$; furthermore $\hh_{\aexec} = \mkvs$. 

In practice, our proof techniques allows defining a per-client invariant 
on the visibility relation of abstract executions, which must be proved to be preserved by $\ET$-reductions (\cref{def:main-body-et_sound}).
This invariant carries client-specific information that links to \( \vshiftname \) (defined in \cref{sec:cm}) in execution tests.
Defining the right invariant is crucial for proving the soundness of several execution-test-based specifications (see \cref{app:et_sound_complete}).

\begin{definition}
\label{def:main-body-et_sound}
An execution test $\ET$ is sound with respect to an axiomatic 
specification $\Ax$ if and only if
there exists an invariant condition $I$ such that, 
for any $\cl, \txid, \vi, \vi', \mkvs, \vi', \fp, \aexec$, if:
\begin{itemize}
    \item $\ET \vdash (\mkvs, \vi) \csat \fp: (\mkvs',\vi')$, where \( \mkvs' = \updateKV[\mkvs, \vi ,\fp, \txid]\)
    \item  $\mkvs_{\aexec} = \mkvs$ and $I(\aexec, \cl) \subseteq \Tx[\mkvs, \vi]$,
\end{itemize}
then there exists a non-empty subset of $\mathscr{X} \subseteq \extend[\aexec, \txid, \Tx[\hh, \vi], \fp]$ 
such that, for any $\aexec' in \mathscr{X}$, $\aexec \models \Ax$ and $I(\aexec', \cl) \subseteq \Tx[\mkvs', \vi']$.
\end{definition}

\begin{theorem}
\label{thm:main-body-et_soundness}
If $\ET$ is sound with respect to $(\Ax)$, then $\CMs(\ET) \subseteq \CMs(\Ax)$.
\end{theorem}

We conclude this section by outlining how our proof technique can be applied to show that the 
execution test $\ET_{\SER}$ defined for serialisability is sound with respect to the axiomatic 
specification $\Ax_{\SER}$. Let $\aexec$ be an abstract execution such that $\aexec \models \Ax_{\SER}$, 
and suppose that the underlying 
kv-store $\hh_{\aexec}$ is in $\CMs(\ET_{\SER})$. An example is the abstract execution $\aexec$ 
and kv-store $\hh$ to the left of \cref{fig:et-sound-to-aexec}. 
We pick an invariant \( I_{\SER} \) that is always empty since \( \vshiftname_\SER \) is always true.
When a client \(\cl\) commits a transaction $\txid$ 
with fingerprint $\fp$ in $\hh_{\aexec}$ under $\ET_{\SER}$, then its view $\vi$ must contain all the versions stored in $\hh_{\aexec}$ (\( \cancommit[\SER]{\mkvs_\aexec}{\vi}{\fp} \)). 
This means that all the abstract executions \( \aexec'' \) in $\extend[\aexec, \txid, \Tx[\hh_\aexec, \vi], \fp]$ are such that there is a visibility 
edge $\txid' \xrightarrow{\VIS_{\aexec''}} \txid$ where $\txid'$ is a writer transaction in $\hh_{\aexec}$. 
It is easy to see there exists an singleton set \( \Set{\aexec'} \) 
that is a subset of $\extend[\aexec, \txid, \Tx[\hh, \vi], \fp]$ and \( \aexec' \models \Ax_\SER \);
in particular, 
there is an edge $\txid' \xrightarrow{\VIS_{\aexec'}} \txid$ for any transaction $\txid' \in \T_{\aexec'}$.
For example, in \cref{fig:et-sound-to-aexec}, the possible abstract executions are $\aexec_1, \aexec_2$.
We pick $\Set{\aexec_2}$, because the new transaction \( \txid_3 \) in \( \aexec_2 \) sees all previous transactions including  
the visibility edge $\txid_{2} \xrightarrow{\VIS} \txid_{3}$. 
Last, the invariant \( I_\SER(\aexec',\cl) \subseteq \Tx[\hh', \vi] \) given \( I_\SER(\aexec',\cl) = \emptyset \).
%Because $\aexec \models \Ax_{\SER}$ and  
%$\aexec'$ is obtained by extending $\aexec$ with a new transaction $\txid$ that sees all the transactions 
%in $\aexec$, it follows that $\aexec' \models \Ax_{\SER}$.

%Similarly, in \cref{sec:aexectrace2kv} we show that it is possible to 
%construct a set of $\ET$-traces $\{\tau_{i}\}_{i \in I}$ from an abstract execution $\aexec$, such that 
%for any $i \in I$, the last configuration of $\tau_{i}$ has the form $(\hh_{\aexec}, \stub)$, and 
%the order in which transactions are executed in $\tau_{i}$ is consistent with $\AR_{\aexec}$. 
%This gives rise to a proof technique aimed at showing that execution-test based specifications 
%of consistency models are sound and complete with respect with their axiomatic counterparts on abstract executions. 
%More details are outlined in \cref{sec:kv2aexec-sound-complete}. 
%In \cref{app:et_sound_complete} we employ our proof techniques to prove that our specifications (\cref{fig:execution.tests})
%of consistency models using execution tests are sound and complete with respect to previously 
%existing axiomatic specifications based on dependency graphs.
%
%\sx{
%    The sketch prove of this theorem might be too much details.
%}
%
%\begin{definition}
%\label{def:main-body-et_complete}
%An execution test $\ET$ is \emph{complete} with respect 
%to an axiomatic definition $(\RP_{\LWW}, \Ax)$ if, for any abstract execution $\aexec \in \CMa(\RP_{\LWW}, \Ax)$ 
%and index \( i : 1 \leq i < \abs{\txidset_{\aexec}}\) such that \( \txid_{i} \toEDGE{\AR_{\aexec}} \txid_{i+1} \), there exist an initial view $\vi_{i}$ and a final view $\vi_{i}'$ where 
%\begin{itemize}
%\item $\vi_{i} = \getView[\aexec, \VIS_{\aexec}^{-1}(\txid_{i})]$, 
%\item let $\txid_{i} = \txid_{\cl}^{n}$ for some $\cl, n$; 
%    \begin{itemize}
%        \item if the transaction $\txid_{i}' = \min_{\SO_{\aexec}}\Set{\txid' }[ \txid_i \toEDGE{\SO_{\aexec}} \txid']$ is defined, then $\vi' = \getView[\aexec, \txidset_{i}]$ where $\txidset_{i} \subseteq (\AR_{\aexec}^{-1})\rflx(\txid_{i}) \cap \VIS_{\aexec}^{-1}(\txid_{i}'))$; 
%        \item otherwise $\vi' = \getView[\aexec, \txidset_{i}]$ where $\txidset_{i} \subseteq (\AR_{\aexec}^{-1})\rflx(\txid_{i})$, 
%    \end{itemize}
%\item $\ET \vdash (\mkvs_{\cut[\aexec, i-1]}, \vi_{i}) \csat \TtoOp{T}_{\aexec}(\txid_{i}) : (\mkvs_{\cut[\aexec, i]},\vi_{i}')$.
%\end{itemize}
%\end{definition}
%
%\begin{theorem}
%\label{thm:main-body-et_complete}
%Let $\ET$ be an execution test that is complete with respect to 
%an axiomatic specification $(\RP_{\LWW}, \Ax)$. Then 
%$\CMa(\RP_{\LWW}, \Ax) \subseteq \CMs(\ET)$.
%\end{theorem}
%
%\sx{
%    The sketch prove of this theorem might be too much details.
%}
%
%
%\sx{Here we could discuss of applying \cref{def:main-body-et_sound,def:main-body-et_complete} on \( \ET_\PSI \) ( or \( \SI \)) and \( \Ax_\PSI \)  }
%
%





%\sx{
%    START
%}
%
%To specify consistency models, abstract executions constraint resolution policy \( \RP \) and the sharp of abstract execution \( \Ax \) (I believe we could even say more precise --- the minimum visibility relation, and cite andrea's concur paper).
%Resolution policy determines a set of snapshot given a set of observable transactions.
%Most time it is last-write-win which gives a unique single snapshot.
%This corresponds to the \( \snapshot{} \) function in the semantics kv-store.
%The resolution policy part we could (question?) choose to not mentioned, since it is minor and many people always assume last-write-win.
%
%The \( \Ax = \Set{ \A : \aeset \to \pset{\TxID} \times \pset{\TxID}} \) is a set of constraints on visibility relations, \eg \( \Ax = \Set{\lambda \aexec \ldotp \SO_\aexec,  \lambda \aexec \ldotp \VIS_\aexec;\VIS_\aexec} \) means all the abstract executions must satisfy \( \SO \subseteq \VIS , \VIS ; \VIS \subseteq \VIS \).
%
%\begin{definition}
%\label{def:aexec2graph}
%Given an abstract execution $\aexec$ that satisfies the last write wins policy,
%the dependency graph $\graphof(\aexec) \defeq (\TtoOp{T}_{\aexec}, \RF_{\aexec}, 
%\VO_{\aexec}, \AD_{\aexec})$ is defined by letting
%\begin{itemize}
%\item $\txid \xrightarrow{\RF_{\aexec}(\key)} \txid'$ if and only if 
%$\txid = \max_{\AR_{\aexec}}(\visibleWrites_{\aexec}(\key, \txid'))$, 
%\item $\txid \xrightarrow{\VO_{\aexec}(\key)} \txid'$ if and only 
%$\txid, \txid' \in_{\aexec} (\otW, \;\ke: \stub)$ 
%and $\txid \xrightarrow{\AR_{\aexec}} \txid'$,
%\item $\txid \xrightarrow{\AD_{\aexec}(\key)} \txid'$ if and only if either 
%$(\otR, \key, \stub) \in_{\aexec} \txid, (\otW, \key, \stub) \in_{\aexec} \txid'$ and 
%whenever $\txid'' \xrightarrow{\RF_{\aexec}(\key)} \txid$, 
%then $\txid'' \xrightarrow{\VO_{\aexec}(\key)} \txid'$.
%\end{itemize}
%\end{definition}
%Note that each abstract execution $\aexec$ determines a key-value store $\hh_{\aexec}$,
%as a result of \cref{def:aexec2graph} and \cref{thm:kv2graph}. 
%Let $\hh$ be the unique kv-store such that $\Gr_{\hh} = \graphof(\aexec)$, then $\hh_{\aexec} = \hh$. 
%
%A kv-store \( \mkvs \) is compatible with an abstract execution \( \aexec \)
%There is a mapping from 
%\begin{definition}
%\label{def:compatible-main-body}
%Given a key-value store $\hh$,
%an abstract execution $\aexec$ is compatible with $\hh$, written 
%$\aexec \compatible \hh$, if and only if there exists a  mapping 
%$f: \pset{\T_{\aexec}} \rightarrow \Views(\hh)$
%such that  
%\begin{itemize}
%\item for any subset $\T \subseteq \T_{\aexec}$, then $\RP_{\LWW}(\aexec, \T) = \{\snapshot(\hh, f(\T))\}$; 
%\item for any view $\vi \in \Views(\hh)$, there exists a subset $\T \subseteq \T_{\aexec}$ 
%such that $f(\T) = \vi$, and $\RP_{\LWW}(\aexec, \T) = \{\snapshot(\hh_{\aexec}, \vi)\}$.
%\end{itemize}
%\end{definition}
%The function $\getView(\aexec, \T)$ defines the view on \( \mkvs_\aexec \) that corresponds to \( \T \) as the following:
%\[
%\getView(\aexec, \T) \defeq \lambda \key. \{0\} \cup \Setcon{ i }{\WTx(\hh_{\aexec}(\key, i)) \in \T}
%\]
%Inversely, the function \( \Tx(\hh, \vi) \) converts a view to a set of observable transactions:
%\[
%\Tx(\hh, \vi) \defeq \Setcon{ \WTx(\hh(\key, i)) }{ \key \in \Keys \wedge i \in \vi(\key) }
%\]
%Given \( \getView \), \( \Tx \), \cref{def:compatible}, 
%it follows \( \aexec \compatible \hh_{\aexec} \) shown in \cref{thm:aexec2kv.compatible}.
%
%Given a trace with the final kv-store \( \mkvs \), there an abstract execution \( \aexec \) that is compatible with \( \mkvs \).
%First, all transactions in an abstract execution are totally ordered by \( \AR \),
%therefore an abstract execution by itself is a trace.
%The reduction order of a trace of kv-store maps to the \( \AR \) in the abstract execution and vice versa.
%The challenge is to show that
%for each transaction \( \txid \) in the trace and the initial view \( \vi \) of the transaction, the visible transitions \( \Tx(\hh, \vi) \) contains at least those transactions by the constraints \( \Ax \), \ie
%Note that \( \Tx(\hh, \vi) \) do not contain any read-only transactions, 
%thus to be more precise, we mean \( \Tx(\hh, \vi) \) plus some potentially read-only transactions contains those by the constraints \( \Ax \), \ie
%\( \fora{\A\in \Ax, \txid'} (\txid', \txid) \in \A(\aexec) \implies \txid' \in \Tx(\hh, \vi) \lor \txid' \text{is read only} \).
%
%Given an abstract execution \( \aexec \) there is a trace of kv-store with the final kv-store \( \mkvs \) that is compatible.
%Similarly, the \( \AR \) decides the reduction order in the trace.
%The challenge here is that
%for each transaction \( \txid \) with fingerprint \( \f \) in the abstract execution,
%(We should mentioned a node in the graph is the fingerprint)
%We find two views, the initial \( \vi \) and the final \( \vi' \), so \( \ET \vdash (\mkvs, \vi) \csat \f : \vi' \) where the \( \mkvs \) is the kv-store when \( \txid \) commits.
%The initial view \( \vi \) includes all versions written by visible transactions of \( \txid \) from the abstract execution, \ie \( \vi = \getView(\aexec, \VIS_\aexec^{-1}(\txid)) \).
%Since \( \vi' \) gives the minimum views for the next transactions from the same clients,
%it can be computed by including all versions, if exists in \( \mkvs \), written by visible transactions of \( \txid' \) where \( \txid' \toEdge{\SO} \txid \).
%
%\sx{
%    END
%}
%

\section{Applications}
\label{sec:applications}

\subsection{Program Analysis}
\label{sec:program-analysis}
A transactional library is a set of transactions 
$L = \{\ptrans{\trans}_{i}\}_{i \in I}$ that can be invoked 
by clients of a kv-store. A transactional library is said to be 
\emph{robust} against an execution test of $\ET$ if the set of kv-stores 
obtainable by letting clients interact with the kv-store using 
operations from $L$ can also be obtained by if the same 
set of operations were performed under $\ET_{\SER}$.
As an application of our theory, we show how we 
can prove the robustness of some simple applications. 
In the following, we let $\CMs(\ET, L)$ be 
the set of kv-stores that can be obtained if clients 
only invoke operations from $L$.

Elaborating on the examples from \cref{sec:overview}, in this Section 
we prove the robustness of a single counter against $\ET_{\PSI}$, 
and multiple counters against $\ET_{\SI}$. Previous techniques for 
checking the robustness of a transactional library \cite{giovanni_concur16,SIanalysis,laws,sureshConcur} 
are based on static analysis: because the session order of clients cannot be determined at compile 
time, these techniques abstract from sessions. To the best of our knowledge, 
we give the first proofs of robustness of transactional libraries that involve sessions.

Our proof technique for robustness uses the following result, which can be though 
as the kv-store counterpart of another well known result for dependency graphs \cite{adya}:
\begin{theorem}
\label{thm:serialisable_nocycle}
Let $\hh$ be a kv-store. Then $\hh \in \CMs(\ET_{\SER})$ if and only if $(\PO_{\hh} \cup \RF_{\hh} 
\cup \VO_{\hh} \cup \AD_{\hh})^{+}$ is irreflexive.
\end{theorem}
To prove the robustness of a transactional library $L$ against an execution test 
$\ET$, we identify an invariant of kv-stores $\hh$ obtainable by letting clients 
invoke only operations contained in $L$, and prove that the invariant 
implies the irreflexivity of the relation $(\PO_{\hh} \cup \RF_{\hh} \cup \VO_{\hh} \cup 
\AD_{\hh})^{+}$.

\mypar{Robustness of a single counter against $\ET_{\PSI}$}
Fix a key $\ke \in \Keys$, and let $\Counter(\ke) = \{\ctrinc(\ke), 
\ctrread(\ke)\}$. Let $\hh \in \CMs(\ET_{\PSI}, \mathsf{Counter}(\ke))$.
Clients can only write a version for key $\ke$ by invoking the 
operation $\ctrinc(\ke)$; because $\ET_{\PSI}$ 
enforces write-conflict detection freedom, in any key-value store 
$\hh \in \CMs(\ET_{\PSI}, \mathsf{Counter})$, we have that 
for all \( \txid, i > 0\):
$\txid = \WTx(\hh(\ke, i)) \implies \txid \in \RTx(\hh(\ke, i-1))$. 
Furthermore, because 
$\ET_{\PSI}$ satisfies the monotonic reads, 
$\ET_{\PSI} \subseteq \ET_{\MRd}$, the order in which 
a client $\cl$ observes the versions of $\ke$ by invocations of $\ctrread(\ke)$, 
is consistent with the order of the index of such versions in $\hh(\ke)$. 
Formally, we have:
\begin{align*}
\hh(\ke) & = (0, \txid_{0}, \T^{\mathsf{r}}_{0} \cup \{\txid^{\mathsf{w}}_1\}) 
\lcat (1, \txid^{\mathsf{w}}_{1}, \T^{\mathsf{r}}_{1} \cup \{\txid^{\mathsf{w}}_2\}) 
\lcat \cdots  \\
& \qquad \cdots \lcat (n-1, \txid^{\mathsf{w}}_{n-1}, \T^{\mathsf{r}}_{n-1} \cup \Set{\txid^{\mathsf{w}}_n})
\lcat (n, \txid^{\mathsf{w}}_n, \T^{\mathsf{r}}_{n})
\end{align*}
where $\{\txid^{\mathsf{w}}_{i}\}_{i=1}^{n}$ is the 
set of transactions invoking operation $\ctrinc(\ke)$, and $\bigcup_{i=0}^{n} 
\T^{\mathsf{r}}_{i}$ is the set of transactions invoking operation $\ctrread(\ke)$. 
Next, we define a relation $\dashrightarrow$ between
transactions appearing in $\hh$ as the smallest transitive relation that 
satisfies the following constraints: 
\begin{enumerate*}
\item $\txid^{\mathsf{w}}_{i} 
\dashrightarrow \txid$ whenever $\txid \in \T^{\mathsf{r}}_{i}$ 
or $\txid = \txid^{\mathsf{w}}_{i+1}$; 
\item $\txid \dashrightarrow \txid'$ 
for some $\txid \in \T^{\mathsf{r}}_{i}$,
whenever $\txid' = \txid^{\mathsf{w}}_{i+1}$
or $\txid' \in \T^{\mathsf{r}}_{i} \land \txid \xrightarrow{\PO} \txid'$. 
\end{enumerate*}
Because of the structure of $\hh$, the relation $\dashrightarrow$ corresponds to a 
total order over transactions appearing in $\hh$; furthermore, it comprises 
the relations $\PO_{\hh}, \RF_{\hh}, \VO_{\hh}$ and $\AD_{\hh}$, which implies 
the robustness of the library $\Counter(\ke)$ under $\ET_{\PSI}$.


\mypar{Robustness of multiple counters against $\ET_{\SI}$}
While a library comprising a single counter is robust against 
$\ET_{\SI}$, the same does not hold if the library two or more counters. 
In \cref{sec:overview} we have outlined how in this scenario it is 
possible to obtain an anomaly, known as long fork, which cannot be 
replicated under $\ET_{\SER}$. 
This anomaly happens because different clients can observe increments 
over two different counters appearing in a different order. Here we prove 
that, if we strengthen the execution test from $\ET_{\PSI}$ to $\ET_{\SI}$, 
then we recover the robustness of the multiple counter library. For the 
sake of simplicity, here we assume that our library consists of only two 
counters: $\mathsf{Counters} = \Counter(\ke_1) \cup \Counter(\ke_2)$.
To this end, we need the following result for kv-stores included in $\CMs(\ET_{\SI})$, 
which is the kv-store counterpart of a very well known result for snapshot isolation 
\cite{feketeSI,SIanalysis}: 
\begin{theorem}
\label{thm:si_cycles}
Any cycle induced by the relations $\PO_{\hh}, \RF_{\hh}, \VO_{\hh}, \AD_{\hh}$ 
for $\hh \in \CMs(\ET_{\SI})$ has two adjacent $\AD_{\hh}$-edges.
\end{theorem}

Consider a kv-store $\hh \in \CMs(\ET_{\SI}, \mathsf{Counters})$. 
Because $\ET_{\SI} \subseteq \ET_{\PSI}$, each of the list of versions $\hh(\ke_1)$ and 
$\hh(\ke_2)$ must follow the same structure given for kv-stores resulting from clients 
using a single counter under $\ET_{\PSI}$. this structure embeds a total order $\dashrightarrow$ 
over transactions appearing in $\hh(\ke_1)$ and $\hh(\ke_2)$; however, because the same 
client can perform operations over two different counters, it is not the case anymore that 
the relation $\dashrightarrow$ comprises the relation $\SO_{\hh}$. In this case, 
we define a second relation $\twoheadrightarrow$ as
$\txid \twoheadrightarrow \txid'$ if $\txid \in \{\WTx(\hh(\ke_{i}, \_)\} \cup \RTx(\hh(\ke_{i}, \_))$ 
and $\txid' \in \{\WTx(\hh(\ke_{j}, \_)\} \cup \RTx(\hh(\ke_{j}, \_))$ where $i,j \in \Set{1,2} \land j \neq i$,  
and $\txid \xrightarrow{\PO} \txid'$.
Relation $\twoheadrightarrow$ tracks the session order of transactions 
performing operations on different keys. If we let $\rightarrow = (\dashrightarrow \cup \twoheadrightarrow)^{+}$, 
we now have that the relation $\rightarrow$ embeds the relation $(\PO_{\hh} \cup \RF_{\hh} \cup 
\VO_{\hh} \cup \AD_{\hh})^{+}$; furthermore, because the library $\mathsf{Counters}$ does 
not contain any operation that accesses both keys $\ke_1$ and $\ke_2$, 
and because the relation $\dashrightarrow$ is acyclic, 
any cycle in $\rightarrow$ must contain at least two edges from the relation $\twoheadrightarrow$. 
In \cref{app:robustness} we show that in fact the existence of such a cycle implies the 
existence of a cycle in $\rightarrow$ of the form $\txid \dashrightarrow \txid_{a} \twoheadrightarrow 
\txid_{b} \dashrightarrow \txid_{c} \twoheadrightarrow \txid$. However, this scenario cannot 
happen in $\hh$, because otherwise we would have a cycle in the relation $(\PO_{\hh} \cup \RF_{\hh} 
\cup \VO_{\hh} \cup \AD_{\hh})^{+}$ with no adjacent $\AD_{\hh}$-edges (recall that an edge 
labelled as $\twoheadrightarrow$ can only  represent two transactions related by $\PO_{\hh}$, 
thus contradicting \cref{thm:si_cycles}). It follows that the relation $\rightarrow$ is irreflexive, 
and so is $(\PO_{\hh} \cup \RF_{\hh} \cup \VO_{\hh} \cup \AD_{\hh})^+$.

\subsection{Verifying Database Implementations}
\label{sec:verify-impl}
%\sx{
%    Short intro why it is good to use our models
%
%    What are cops and then clock-si.
%    how we verify
%}
Our representation of database state as kv-stores and client views closely matches the representation of database state in many well-known implementations~\cite{??}.
As such, when verifying such implementations, our state-based formalisms is better-suited than the existing graph-based formalisms of dependency graphs and abstract executions. 
In centralised databases, the state of the centralised server corresponds to a kv-store,
and the local snapshots that transactions operate on are extracted from our views.  
In distributed databases, each replica state corresponds to a view. 
Moreover, since most distributed databases are eventual consistent, 
\ie all replicas eventually agree on the database state,
replica states can be collectively encoded as a kv-store.
We verify the correctness of two database implementations in the literature,
COPS~\cite{Lloyd:2011:DSE:2043556.2043593} and Clock-SI~\cite{Du:2013:CSI:2553409.2553434}.

\mypar{COPS}
COPS is a fully replicated database: each replica contains all keys, but their associated values may be out of date.
Database clients operate on a single replica; 
synchronisation between replicas ensures that each replica is in a consistent (albeit out-of-date) state.
The COPS API follows the multiple-readers-single-writer paradigm: 
at any given time, the database can be accessed by either multiple read-only concurrent transactions, or a single writing transaction. 
As such, all versions (on all keys) are totally ordered. 
To ensure causal consistency, each version $\ver$ records a \emph{dependency set}, written $\func{dep}{\ver}$,
tracking the versions on which $\ver$ depends.
A client maintains a context tracking the versions that have been either fetched from, or committed to, other replicas.
%
%When running a read-only transaction, 
%the replica returns a list of versions,  $\vilist = \ver^{1}_{\ke_1}, \dots \ver^{n}_{\ke_n}$,
%such that each $\ver^i_\ke \in \vilist$ is greater than or equal than those in the dependency of $\vilist$.
%That is, $\for{\ke} \for{\ver, \ver_\ke^i \in \vilist} \for{\ver_\ke^j \in \func{dep}{\ver}} i \geq j$. 
%
%%%%%To read multiple keys in a transaction, the replica returns a list of versions \( \vilist = \ver^{i_1}_{\ke_1}, \dots \ver^{i_n}_{\ke_n} \) 
%%%%%that are either equal or greater than any versions in the dependencies,
%%%%%\ie \( \fora{\ke, \ver, \ver^{i}_{\ke}, \ver^{j}_{\ke}} \ver \in \vilist \land \ver^{i}_{\ke} \in \vilist \land \ver^{j}_{\ke} \in \func{dep}{\ver} \implies i \geq j \).
%
%After running a writing transaction, in order to commit a new version to the replica, the client must provide its context. 
%A version is accepted by the replica only once all versions from the dependency have been made known to the same replica,
%otherwise it is put in the pending list.
%
We encode the COPS replicas as a kv-store, and encode client contexts as views.  
%We encode kv-store that includes all versions from all replicas, at any given time.
%A client context then converts to a view on the kv-store, 
%\ie if a version in the context then it is in the view.
We show that the COPS implementation is causally consistent in that it is sound with respect to the $\CMs(\ET_\CC)$
specification, where $\ET_\CC$ is the causal consistency execution test in \cref{fig:execution_tests}.
%We show that the views before and after any COPS transaction,
%satisfy the execution tests for causal consistency in \cref{fig:execution.tests}.
The COPS implementation and our soundness proof are given in \cref{sec:cops}.

\mypar{Clock-SI}
Clock-SI is a \emph{sharded} database implementation for snapshot isolation (SI),
where different shards host disjoint fragments of keys.
As with the original definition of SI, Clock-SI uses timestamps to maintain different versions. 
The clocks on different shards may not agree; however, the difference is assumed to be bounded.
Each key keeps a history of versions, where each version comprises a value and the commit timestamp.
Each transaction $\txid$ is assigned to a designated shard, known as the $\txid$ \emph{coordinator}. 
When $\txid$ starts, it obtains a snapshot and records the snapshot timestamp at its coordinator shard.
Transaction $\txid$ may then read from shards other than its coordinator, so long as the current timestamp at those shards are greater than the recorded snapshot time. 
Once $\txid$ completes, it may commit if  no transaction with a conflicting write has committed since the snapshot time.
%
We encode the shards collectively as a kv-store. 
Each snapshot timestamp is encoded as a view, contains all versions committed before that timestamp.
We show that Clock-SI is sound with respect to the $\CMs(\ET_\SI)$
specification, where $\ET_\SI$ is the snapshot isolation execution test in \cref{fig:execution_tests}.
%We show that the views before and after any COPS transaction,
%satisfy the execution tests for causal consistency in \cref{fig:execution.tests}.
The Clock-SI implementation and our soundness proof are given in \cref{sec:clock-si}.

\section{Conclusion}
 

Kaki et al. proposed an operational semantics of programs 
under different transactional consistency models \cite{alonetogether}. Like us, 
their semantics is parametric in the definition of a consistency model, it 
involves a notion of global state, 
and clients work on a local copy of the state 
before pushing their updates onto the central state. Unlike 
us, their notion of state does not employ multi-versions, 
which prevents the possibility to specify Parallel Snapshot Isolation 
and other weaker consistency models. 
%this makes it impossible only to define weak consistency models 
%where clients observe updates in different orders, such as PSI. 
Furthermore, operations of transactions are interleaved with each 
other: this makes it possible to capture consistency guarantees that 
do not enjoy atomic visiblity, such as read committed, but it comes 
at the price of making the task analysing applications without the aid 
of automatic tools difficult.

Nagar et al. proposed an operational semantics for weak consistency 
models using abstract executions as a state of the system \cite{sureshConcur}. 
Like us, their semantics is parametric in the specification of a consistency model, 
and takes into account very weak consistency models such as PSI. 
The authors also present a static analysis technique for checking the robustness of applications 
under a given consistency model. Unlike us, their work does not take into account 
sessions and session guarantees, thus limiting the robustness results only to databases 
that do not provide sessions. The 
operational semantics presented in this  work also allows for the execution of transactions to be interleaved 
with each other, which we already argued complicates the task of program analysis. 

Crooks et al. proposed a state-based formal framework for weak consistency models, 
which also employs the notion of commit test. Like us, clients use a notion similar 
to the view mechanism to determine the values read for keys in a transaction, 
and a notion of commit test to  determine whether a transaction is allowed to commit. 
Unlike us, their notion of state consists of a mapping from keys to values, and 
commit tests need access to the total order of state changes, to determine whether a transaction can commit.
While approach has been showed to be successful for verifying the correctness of protocols for transactional consistency models, 
the problem of analysing transactional applications is not addressed.
We believe that the requirement for commit test to know the total order 
of state changes hinders the development of program analysis techniques.

Several other works have addressed the issue of robustness of applications in transactional systems. 
Dias et al. \cite{dias-tm} developed a static analysis technique, based on separation logic, 
to check the robustness of applications under SI. Fekete et al. \cite{fekete-tods} derived 
a static analysis check for SI based on dependency graph. Bernardi et al. \cite{giovanni-concur16} 
developed a static analysis check for several consistency models with atomic visibility. 
Cerone et al. \cite{laws} investigated the relationship between abstract 
executions and dependency graph from an algebraic perspective, and applied it to infer 
robustness check for several transactional consistency models.

Our view mechanism has been inspired by the work of Kang et al. \cite{promises}, 
which proposes an operational semantics for programs running under the C11 memory model.

%\ifCLASSOPTIONcompsoc
%  % The Computer Society usually uses the plural form
%  \section*{Acknowledgments}
%\else
%  % regular IEEE prefers the singular form
%  \section*{Acknowledgment}
%\fi
%The authors would like to thank...

\bibliographystyle{IEEEtran}  
\bibliography{bibliography,bibliography2}

%%%%%%%%%%%%%%%%%%%%%%%%%%%%%%%%%%%%%%%%%%%%%%%%%%%%%%%%%%%%%%%%%%%%%%%%
% appendix for submission
%%%%%%%%%%%%%%%%%%%%%%%%%%%%%%%%%%%%%%%%%%%%%%%%%%%%%%%%%%%%%%%%%%%%%%%%
\newpage
\onecolumn
\appendices
\section{Operational Semantics on KV-Stores}
\label{sec:full-semantics}
\subsection{Multi-version Key-value Stores and Views}
\label{sec:mkvs-view}
\ac{I'm going to leave this comment here, I know it's going to be harsh but needs to be said: 
If you are going to change the main text, please be sure that the English is correct, and that the 
writing is consistent! It took me several days to get this in a good shape, and now I have to go over all of it again!}
%\ac{
%We focus on a computational model where multiple client programs can access and update 
%locations in a key-value store using atomic transactions. Transactions in our model execute atomically, 
%though the consistency guarantees that they provide do not necessarily correspond to \emph{serialisability}. 
%This means that, at the moment of executing, a transaction may not observe the most up-to-date value 
%of a location. 

%To overcome this issue, we model the state of the system using \emph{multi-version key-value stores} 
%(MKVSs) and \emph{views}. A 
%MKVS keeps track of all the versions written for any key, as well as the information 
%about the transactions that read and wrote such versions.  Views keep track of the version observed 
%for each key by clients. 
%}

%\sx{ Value as natural number or natural number + key index? Is partial mkvs is a problem here??}


%We often depict MKVSs graphically. 
%One example is given by the MKVS $\hh_0$ of Figure \ref{fig:hheap}(a) (ignore for the moment 
%the straight lines labelled $\txid_1$ and $\txid_2$).

%\ac{\sx{Partial is better for logic}Maybe it's better to keep $\ke$ fixed and say that we look at only a 
%fragment of the key value store. Alternatively, we can go for partial mappings to 
%represent MKVSs, but still avoiding allocation and deallocation of keys.}



%\ac{
%We often depict MKVSs graphically. 
%One example is given by the MKVS $\hh_0$ of Figure \ref{fig:hheap}(a) (ignore for the moment 
%the straight lines labelled $\txid_1$ and $\txid_2$).

%To the left 
%we have the set of keys stored y $\hh_0$, in this case $\key{k}_1$ and 
%$\key{k}_2$. To the right, on the same line of a key, a matrix containing the
 %list of versions stored by such a key in $\hh_0$. Starting from the first column, 
 %each version is represented by two adjacent columns in the matrix: the 
 %column to the left gives the value of the version, while the column to the 
 %right contains the identifier of the transaction that wrote the version to the 
 %top, and the identifiers of the transactions that read such a version to the bottom.
%In the case of $\hh_0$, there are two versions stored for $\key{k}_1$: 
%the first one with value $0$, written by $\txid_0$ and read by $\txid_2$; 
%and the second one with value $1$, written by $\txid_2$ and read by no 
%transaction. 
%}
\azalea{I have changed the values to $v_0$ and $v_1$ (from $0$ and $1$) to help clarify the distinction between indexes and values.
I have also paraphrased, please double check.  }

\subsubsection{Key-value Stores} 

We assume a countably infinite set of \emph{keys} $\Keys \defeq \Set{\ke, \ke', \cdots}$, 
a set of \emph{values} $\Val \defeq \{\val, \val', \cdots\}$ which for simplicity we instantiate to be 
$\Nat \uplus \Keys$, a set of clients $\Clients \defeq \Set{\cl, \cl',\cdots}$. 
We also assume a set of transaction identifiers $\TxID \defeq \Set{ \txid_{\cl}^{n} \mid \cl \in \Clients \wedge n \geq 0 } 
\uplus \Set{\txid_{0}}$,
each of which is either a special transaction identifier $\txid_0$, 
or it is indexed by a client identifier and a natural number. 
Elements of $\TxID$ are ranged over by $\txid, \txid', \cdots$, 
while subsets of $\TxID$ are ranged over $\txidset, \txidset', \cdots$. 
We let $\TxID_{0} \defeq \TxID \setminus \{ \txid_0\}$.
The structure of the set $\TxID$  
embeds the order in which transactions are executed by individual clients, or \emph{session order}. 
Specifically, we let $\PO \defeq \Set{ (\txid, \txid') \mid \exsts{ \cl, n,m } \txid = \txid_{\cl}^{n} \wedge \txid' = \txid_{\cl}^{m} \wedge n < m}$; 
$(\txid, \txid') \in \PO$ means that 
some client $\cl$ has executed $\txid$ prior to $\txid'$. For $\PO$ (and in general  
for relations between transaction identifiers) we will often adopt the more graphic notation 
$\txid \xrightarrow{\PO} \txid'$ in lieu of $(\txid, \txid') \in \PO$.

Given a set $X$, then $\powerset{X}$ denotes 
the powerset of $X$, while $X^{\ast}$ is the free monoid induced by $X$.


\begin{definition}[Multi-version Key-value Stores]
\label{def:his_heap}
\label{def:mkvs}
A \emph{version} is a triple $\ver = (\val, \txid, \txidset)$. The set of versions is denoted by $\Versions \defeq \Val \times \TxID \times \powerset{\TxID_{0}}$, 
and a \emph{key-value store} is a mapping $\hh \in \MKVSs \defeq \Keys \rightarrow \Versions^{\ast}$. 
\ac{ The superscript fin over the $\rightharpoonup$ needs to be fixed. You may want to look at the package extpfeil.}
%The set of key-value stores is denoted as $\HisHeaps$.
\end{definition}

\emph{A version} $\ver = (\val, \txid, \txidset)$ consists of a value $\val$, and the meta-data of the transactions 
that accessed the version; specifically, $\txid$ is the identifier of the transaction that wrote such a version, 
and $\txidset$ is the set of identifiers of transactions that read the version.
Given a version $\ver = (\val, \txid, \txidset)$, we let $\valueOf(\ver) \defeq \val$. 
$\WTx(\ver) \defeq \txid$ and $\RTx(\ver) \defeq \txidset$.
Lists of versions, that is elements of $\Versions^{\ast}$, are ranged over by $\vilist, \vilist',\cdots$.

\emph{A multi-version key-value store}, or \emph{kv-store}, 
is a mapping from keys to lists of versions. 
For a given kv-store $\hh$, key $\ke$ and index $i \geq 0$, we use the notation $\hh(\ke, i)$ 
to denote the $i$-th version (starting from $0$) installed for $\ke$; that is, if $\hh(\ke) = \ver_0 \cdots\ver_{n}$, then 
$\hh(\ke, i) \defeq \ver_{i}$ if $i \leq n$, it is undefined otherwise. We also let $\lvert \hh(\ke) \rvert \defeq n +1 $ denote 
the length of $\hh(\ke)$.

{\color{red} 
It will be often convenient to depict key-value stores graphically: an 
example is given by the kv-store $\hh$ depicted in \cref{fig:hheap-a}
{\color{red} (ignore for the moment the vertical lines labelled $\client$ and $\client'$)}. 
It comprises two keys \( \ke_1\) and \( \ke_2 \), 
each of which is associated with two versions carrying values $\val_0$ and $\val_1$, and $\val'_0$ and $\val'_1$, respectively.
The versions of a key are listed in order from left to right. 
We represent each version as a three-cell box, with the left cell storing the value, the top right cell recording the writer, and the bottom right cell recording the readers. 
For example, the version carrying value $\val_0$ in $\ke_1$ has been written by $\txid_0$, and has been read by $\txid_{\cl'}^1$.
}

In this paper we focus on key-value stores whose consistency model enforces the  
\emph{atomic visibility} of transactions \cite{framework-concur}. 
We also assume that in kv-stores, keys with a defined list of versions, have an initial version carrying  a default value $\val_0 \in \Val$, 
written by the special transaction identifies $\txid_0$.
A \emph{well-formed} kv-store $\hh$ requires that:
\begin{enumerate}[(i)]
\item\label{kv:wf.init} for each key $\ke \in \dom(\hh)$, $\hh(\ke, 0) = (\val_0, \txid_0, \stub)$, where $\val_0$ is a default value from $\Val$;
\item\label{kv:wf.onewrite} transactions never write more than one version per key,  
\[
\fora{\ke \in \dom(\hh), i,j : 0 \leq i, j < \abs{ \hh(\ke) }}
\WTx(\hh(\ke, i)) = \WTx(\hh(\ke, j)) \implies i = j 
\]
\item\label{kv:wf.oneread} transactions never read different versions for the same key, 
\[
\fora{\ke \in \dom(\hh), i,j : 0 \leq i, j < \abs{ \hh(\ke) }} 
\RTx(\hh(\ke, i)) \cap \RTx(\hh(\ke, j)) \neq \emptyset \implies i = j
\]
\item\label{kv:wf.so} the order 
in which transactions issued by the same client install different versions for some key $\ke$, is consistent with the order in which 
such transactions have been invoked; similarly, a client can read the version of a key $\ke$ only after it installed it. 
\begin{multline*}
    \fora{ \ke \in \dom(\hh), \cl \in \Clients, i,j: 0 \leq i < j < \abs{\hh(\ke)}, n, m} \\
(\txid_{\cl}^{n} = \WTx(\hh(\ke,i)) \wedge \txid_{\cl}^{m} \in \{\WTx(\hh(\ke,j))\} \cup \RTx(\hh(\ke, i)) \implies n < m.
\end{multline*}
\end{enumerate}
We always assume that kv-stores are well-formed, and let $\HisHeaps$ be the set of well-formed kv-stores.


\begin{figure}
\begin{center}
\hrule
\begin{tabular}{@{}c @{\qquad} c@{}}
\begin{halfsubfig}
\begin{tikzpicture}
\begin{pgfonlayer}{foreground}
%\draw[help lines] grid(5,4);

%Location x
\node(locx) {$\ke_1 \mapsto$};

\matrix(versionx) [version list]
    at ([xshift=\tikzkvspace]locx.east) {
    {a} & $\txid_0$ & {a} & $\txid_{\cl}^{1}$\\
    {a} & $\left\{\txid_{\cl'}^{1}\right\}$ & {a} & $\emptyset$ \\
};
\tikzvalue{versionx-1-1}{versionx-2-1}{locx-v0}{$v_0$};
\tikzvalue{versionx-1-3}{versionx-2-3}{locx-v1}{$v_1$};

%Location y
\path (locx.south) + (0,\tikzkeyspace) node (locy) {$\ke_2 \mapsto$};
\matrix(versiony) [version list]
   at ([xshift=\tikzkvspace]locy.east) {
 {a} & $\txid_0$ & {a} & $\txid_{\cl'}^{1}$ \\
  {a} & $\left\{\txid_{\cl}^1\right\}$ & {a} & $\emptyset$\\
};

\tikzvalue{versiony-1-1}{versiony-2-1}{locy-v0}{$v'_0$};
\tikzvalue{versiony-1-3}{versiony-2-3}{locy-v1}{$v'_1$};

% \draw[-, red, very thick, rounded corners] ([xshift=-5pt, yshift=5pt]locx-v1.north east) |- 
%  ($([xshift=-5pt,yshift=-5pt]locx-v1.south east)!.5!([xshift=-5pt, yshift=5pt]locy-v0.north east)$) -| ([xshift=-5pt, yshift=5pt]locy-v0.south east);

%blue view - I should  check whether I can use pgfkeys to just declare the list of locations, and then add the view automatically.
\draw[-, blue, very thick, rounded corners=10pt]
 ([xshift=-3pt, yshift=20pt]locx-v1.north east) node (tid1start) {} -- 
 ([xshift=-3pt, yshift=-5pt]locx-v1.south east) --
 ([xshift=-3pt, yshift=5pt]locy-v0.north east) -- 
 ([xshift=-3pt, yshift=-5pt]locy-v0.south east);
 
 \path (tid1start) node[anchor=south, rectangle, fill=blue!20, draw=blue, font=\small, inner sep=1pt] {$\client$};

%red view
\draw[-, red, very thick, rounded corners = 10pt]
 ([xshift=-16pt, yshift=5pt]locx-v1.north east) node (tid2start) {}-- 
% ([xshift=-16pt, yshift=-5pt]locx-v0.south east) --
% ([xshift=-16pt, yshift=5pt]locy-v1.north east) -- 
 ([xshift=-16pt, yshift=-5pt]locy-v1.south east) node {};
 
\path (tid2start) node[anchor=south, rectangle, fill=red!20, draw=red, font=\small, inner sep=1pt] {$\client'$};

\end{pgfonlayer}
\end{tikzpicture}
\caption{A configuration with a well-formed kv-store $\hh$ and two views $\vi,\vi'$.}
\label{fig:hheap-a}
\end{halfsubfig}
&

\begin{halfsubfig} 
\begin{center}
\begin{tikzpicture}[scale=0.85, every node/.style={transform shape}]
%\draw[help lines] grid(6,4);

\node(t0wx) at (-1,2) {$(\otW, \ke_1, \val_0)$}; 
\path (t0wx.south) + (0,-0.2) node[anchor=north] (t0wy) {$(\otW, \ke_2, \val'_0)$};
\path (t0wx.north east) + (1,0.5) node[anchor = west] (t1ry) {$(\otR, \ke_2, \val'_0)$}; 
\path (t1ry.east) + (0.2,0) node[anchor = west] (t1wx) {$(\otW, \ke_1, \val_1)$};
\path (t0wy.south east) + (1,-0.5) node[anchor = west] (t2rx) {$(\otR, \ke_1, \val_0)$};
\path (t2rx.east) + (0.2,0) node[anchor = west] (t2wy) {$(\otW, \ke_2, \val'_1$)};

\begin{pgfonlayer}{background}
\node[background, fit=(t0wx) (t0wy)] (t0) {};
\node[background, fit= (t1ry) (t1wx)] (t1) {};
\node[background, fit= (t2rx) (t2wy)] (t2) {};

\path(t0.west) node[anchor=east] (t0lbl) {$\txid_0$};
\path(t1.north) node[anchor=south] (t1lbl) {$\txid_1$};
\path(t2.south) node[anchor=north] (t2lbl) {$\txid_2$};

\path[->]
(t0.north) edge[bend left=70] node[above, yshift=7pt, xshift=-1pt, pos=0.3] {$\RF(\ke_2), \VO(\ke_1)$} (t1.west)
(t0.south) edge[bend right=70] node[below, yshift=-8pt, xshift=-1pt, pos=0.3] {$\RF(\ke_1), \VO(\ke_2)$} (t2.west)
([xshift=-8pt]t2.north) edge[bend left=40] node[left] {$\AD(\ke_1)$} ([xshift=-8pt]t1.south) 
([xshift=8pt]t1.south) edge[bend left=40] node[right] {$\AD(\ke_2)$} ([xshift=8pt]t2.north);
\end{pgfonlayer}

\end{tikzpicture}
\end{center}
\caption{The dependency graph induced by $\hh$.}
\label{fig:hheap-b}
\end{halfsubfig} \\
\end{tabular}
\end{center}
\hrule
\caption{Multi-version key-value stores}
\label{fig:hheap}
\end{figure}

\mypar{Views, Configurations and Snapshots.}
The key-value store tracks the global state, 
but when executing transactions, different \emph{clients} may observe 
different versions of the same key. To keep track of 
the versions they observe, clients are associated with \emph{views} (\cref{def:view}). 

\begin{definition}[Views and configurations]
\label{def:view}
\label{def:cuts}
\label{def:views}
\label{def:configuration}
Given a key-value store $\hh$, \emph{a view} of $\hh$ is a function  
$\vi: \dom(\hh) \to\powerset{\Nat}$ such that  
\[
\forall \ke \in \dom(\hh).\; 0 \in \vi(\ke) \wedge \forall i \in \vi(\ke).\; i < \lvert \hh(\ke) \rvert.
\]
and 
\begin{equation}
\label{eq:view.atomic}
\begin{array}{@{}l@{}}
\fora{ \ke,\ke' \in \dom(\hh), i,j \in \Nat} \\
\quad (j \in \vi(\ke) \wedge \WTx(\hh(\ke, j)) = \WTx(\hh(\ke', i)) \implies i \in \vi(\ke')
\end{array}
\tag{Atomic}
\end{equation}

\ac{ ALL THE FIGURES THAT USED VIEWS MUST BE DRAWN AGAIN. WE ALSO NEED A 
GRAPHIC FORMALISM FOR THE NEW NOTION OF VIEWS.}

The set of views of $\hh$ is denoted 
as $\Views(\hh)$, and \emph{the set of views} is defined as:
\[
\Views \defeq \bigcup_{\hh \in \HisHeaps} \Views(\hh)
\]
A \emph{configuration} $\conf$ is a pair $(\hh, \viewFun)$, where $\viewFun: 
\Clients \parfinfun \Views(\hh)$. The configuration $\conf_{0} = (\hh_{0}, \viewFun_{0})$ is 
initial if, for any $\ke$, $\hh_{0}(\ke) = (\val_0, \txid_0, \emptyset)$, for some 
initial value $\val_0$. 
The set of configurations is denoted as $\Confs$.
\end{definition}
Given $\hh \in \HisHeaps$ and two views $\vi, \vi' \in \Views(\hh)$, 
we let $\vi \viewleq \vi'$ if, for any $\ke \in \dom(\hh)$, $\vi(k) \subseteq \vi'(\ke)$. 

\emph{A configuration} includes a kv-store and a partial mapping from clients from clients to views.
The view of the client $\cl$ in $\hh$ reflects the set of versions for each key 
that the client \(\cl \) observes upon executing a transaction. 
The constraint of \cref{eq:view.atomic} establishes that if a client observes 
a version of some key written by a transaction $\txid$, then it must observe all the versions of 
all keys that $\txid$ wrote. This constraint captures the \emph{atomic visibility} of transactions.

{\color{red} We often depict views of clients graphically by drawing client-labelled lines crossing 
versions of key-value stores. A line crossing the $i$-th version of key $\ke$ defines a view 
$\vi$ for client $\cl$, with $\vi(\ke) = i$. One example is given by \cref{fig:hheap-a} where the configuration is
$\conf_0 = (\hh_0, \Set{\cl_1 \mapsto \vi_1, \cl_2 \mapsto \vi_2})$. 
There are two clients, 
$\cl_1$ and $\cl_2$, with views $\vi_1$, and $\vi_2$ respectively. $\vi_1$ crosses $\ke_1$ at its $0$-th 
version, and $\ke_2$ at its $1$-st version. Therefore we have $\vi_1 = \Set{\ke_1 \mapsto 0, \ke_2 \mapsto 1}$. 
Similarly, we have $\vi_2 = \Set{\ke_1 \mapsto 1, \ke_2 \mapsto 0}$. }

Given a kv-store $\hh$, a view $\vi$ and a key $\ke \in \dom(\hh)$, 
we commit an abuse of notation and write $\hh(\ke, \vi)$ as a shorthand 
for $\hh(\ke, \max_{<}(\vi(\ke)))$. Note that such a version is well-defined because 
we are assuming that $\vi(\ke) \neq \emptyset$.
The view $\vi$ naturally induces a \emph{snapshot} 
by extracting the value of the most up-to-date version it observes for each key $\ke \in \dom(\hh)$. 
As we will see presently, transactions are executed relatively 
to a snapshot of a kv-store, which maps each key to a single value.
Views are used to determine the snapshot in which a transaction 
is executed, according to the following definition.
\begin{definition}[Snapshots]
\label{def:heaps}
\label{def:snapshot}
Given $\hh \in \HisHeaps$ and $\vi \in \Views(\hh)$, the \emph{snapshot} of $\vi$ in 
$\hh$ is defined as $\snapshot(\hh, \vi) \defeq \lambda \ke \ldotp \valueOf(\hh(\ke, \max_{<}(\vi(\ke)))$.
\end{definition}

%of $\vi$ by accessing the value of 
%A view $\vi$ in $\hh$ naturally defines a snapshot $\snapshot(\hh, \nu)$
%A MKVS tracks the global state of the system; however, different \emph{clients} may observe different versions of the same key. 
%To model this, we introduce the notion of \emph{views} (\cref{def:views}). 
%A view $V$ reflects the particular version for each key that a client observes upon executing a transaction. 
%%We present an example of views in \cref{fig:hheap-a} with two views: $\client_1$ in red and $\client_2$ in blue.
%More concretely, the view for \( \client_1 \) is given formally as $\vi_1 = \Set{\key{k}_1 \mapsto 1, \key{k}_2 \mapsto 0}$.
%That is, the client with view $\vi_1$ observes the second version (at index 1) of key \( \ke_{1} \) with value $v_1$, and the first version (at index 0) of key \( \ke_2 \) with value $v'_0$.
%%, and 
%%the first version of $\key{k}_2$, carrying value $0$. Similarly, according to its view 
%%$V_2 = [\key{k}_1 \mapsto 2, \key{k}_2 \mapsto 2]$, the client $\txid_2$ observes 
%%in $\hh$the second and most up-to-date version for both $\key{k}_1$ and $\key{k}_2$.
%
%\begin{definition}[Views]
%\label{def:view}
%\label{def:cuts}
%\label{def:views}
%\emph{A view} is a partial finite function from keys to indexes:
%$
%\vi \in \Views \defeq \Addr \parfinfun \Nat 
%%\begin{rclarray}
%%    \vi \in \Views & \defeq & \Addr \parfinfun \Nat 
%%\end{rclarray}
%$.                                                                 
%The \emph{view composition}, $\composeVI: \Views \times \Views \rightharpoonup \Views$ is defined as the standard disjoint function union: $\composeVI \eqdef \uplus$. 
%% \( \vi \composeVI \vi' \defeq \vi \uplus \vi'\) 
%The \emph{unit view}, $\unitVI \in \Views$, is a function with an empty domain: $\unitVI \eqdef \emptyset$. 
%% and the unit is \( \unitVI \defeq \emptyset\).
%The \emph{order relation} on views, $\orderVI: \Views \times \Views$, is defined between two views with the same domain as the point-wise comparison of their indexes for each entry: 
%\[
%\begin{rclarray}
%    \vi \orderVI \vi' & \defiff & \dom(\vi) = \dom(\vi') \land \fora{\ke} \cu(\ke) \leq \cu'(\ke) \\
%\end{rclarray}
%\]
%\end{definition}
%%
%We say view $\vi$ is \emph{older} than view $\vi'$ (or $\vi'$ is \emph{newer} than $\vi$) whenever $\vi \orderVI \vi'$ holds.
%
%
%\mypar{Configurations} A \emph{configuration} comprises an MKVS, and the views associated with clients.
%In \cref{fig:hheap-a} we present an example of a configuration comprising an MKVS and the two views associated with clients $\client_1$ and $\client_2$. 
%We write $\version(\hh, \ke, \vi)$ for $\hh(\ke, \vi(\ke))$; 
%and write $\valueOf(\hh, \ke, \vi)$ as a shorthand for $ \valueOf(\version(\hh, \key{k}, V))$; similarly for $\WTx, \RTx$.
%%we commit an abuse of notation and often write $\valueOf(\hh, \ke, \vi)$ in lieu of $ \valueOf(\version(\hh, \key{k}, V))$, and similarly for $\WTx, \RTx$.
%When $\ver = \version(\hh, \ke, \vi)$, we say that \emph{$\vi$ $\ke$-points to $\ver$ in $\hh$}. 
%When $\ver = \hh(\ke, i)$ for some $0 \leq i \le \vi(\ke)$, we say that \emph{$\vi$ $\ke$-includes $\ver$ in $\hh$}.
%Lastly, we always assume that MKVSs, views, and configurations are well-formed, unless otherwise stated.
%
%
%
%\begin{definition}[Configurations]
%A view $\vi$ is \emph{well-formed with respect to an MKVS} $\mkvs$, written \( \wfV{\mkvs, \vi} \),  iff they have the same domain and every index from $\vi$ is within the range of the corresponding entry in $\mkvs$ and the view is \emph{atomic} with  respect to the key-value store: 
%\[
%\begin{rclarray}
%    \wfV{\mkvs, \vi} & \defeq & \dom(\mkvs) = \dom(\vi) \land \fora{\ke \in \dom(\vi)} 0 \leq \vi(\ke) < \lvert \mkvs(\ke) \rvert \\
%    \pred{atomic}{\vi ,\hh} & \eqdef & \fora{\txid } \exsts{\ke, i} i \leq \vi(\ke) \land \hh(\ke,i) = (\stub, \txid, \stub) \implies \pred{visible}{\txid, \vi, \hh} \\ 
%    \pred{visible}{\txid, \vi, \hh} & \eqdef & \fora{\ke, i} \hh(\ke,i) = (\stub, \txid, \stub) \implies i \leq \vi(\ke) 
%\end{rclarray}
%\]
%%
%\azalea{We need a symbol for this to fill the ???? above. Also ???? below. \sx{Done}}
%A \emph{configuration} $\conf$ is a pair of the form $(\hh, \viewFun)$, where $\hh$ denotes an MKVS, and $\viewFun: \Clients \parfinfun \Views$ is a partial finite function from clients to views. 
%A configuration $\conf = (\hh, \viewFun)$ is \emph{well-formed}, written \( \wfC{\conf}\), iff for all clients $\cl \in \dom(\viewFun)$, the view $\viewFun(\txid)$ is well-formed with respect to $\hh$. 
%%We say that a view $V$ is well-defined with respect to the 
%%MKVS $\hh$ if, $\forall \key{k} \in \ke. 0 < V(\key{k}) \leq 
%%\lvert \hh(\key{k}) \rvert$. 
%%Given a view $V$ that is well-defined 
%%with respect to a 
%
%\end{definition}
%
%\mypar{Snapshots} When a client executes a transaction on the $\mkvs$ MKVS, it extracts a \emph{snapshot} of it via the \( \func{snapshot}{\mkvs, \vi} \) function, extracting the values corresponding to the versions indexed by its view \( \vi \) (\cref{def:snapshot}).
%For instance, for client \( \client_1 \) in \cref{fig:hheap-a}, the $\func{snapshot}{\cdots}$ functions yields a state where key $\ke_1$ carries value $v_1$ and second key \( \ke_2 \) carries value $v'_0$.
%%The concrete state extracted in this way takes the name of the \emph{snapshot} of the transaction.
%%In general, the process of determining the view of a client, hence the snapshot in which such a client executes transactions, is non-deterministic.
%
%\azalea{Before in MKVSs we had values drawn from $\Nat$ in \cref{def:mkvs}. Now we use $\Val$. I think you mean to use $\Val$ in both places? \sx{I would say so} }
%\begin{definition}[Snapshots]
%\label{def:heaps}
%\label{def:snapshot}
%Given the sets of values $\Val$  and keys \( \Addr\)  (\cref{def:mkvs}), the set of \emph{snapshots} is:
%$
%    \h \in \Heaps \eqdef \Addr \parfinfun \Val
%$. 
%%\[
%%\begin{rclarray}
%%    \h \in \Heaps & \eqdef & \Addr \parfinfun \Val
%%\end{rclarray}
%%\]
%The \emph{snapshot composition function}, $\composeH: \Heaps \times \Heaps \parfun \Heaps$, is defined as $\composeH \eqdef \uplus$, where $\uplus$ denotes the standard disjoint function union. The \emph{ snapshot unit element} is $\unitH \eqdef \emptyset$, denoting a function with an empty domain.
%The \emph{partial commutative monoid of snapshots} is $(\Heaps, \composeH, \{\unitH\})$.
%Given an MKVS $\hh$ and a view $\vi$, the snapshot of $\vi$ in $\hh$, written $\snapshot(\hh, \vi) $, is defined as:
%$
%    \snapshot(\hh, \vi) \defeq \lambda \ke \ldotp \valueOf(\hh, \ke, \vi)
%$.
%%\[
%%\begin{rclarray}
%%    \snapshot(\hh, \vi) & \defeq & \lambda \ke \ldotp \valueOf(\hh, \ke, \vi).
%%\end{rclarray}
%%\]
%\end{definition}
%
%\sx{Need some explanation}
%\ac{General Comment on this Section: it is too abstract. We 
%should give either here or in the introduction an example of computation - 
%the write skew program should be okay that helps the reader understanding 
%what's going on. Also, it could be also good to illustrate the notions 
%of execution tests and consistency models.}
%
%\sx{From Andrea: introduce the execution test here with a table, also introduce fingerprint here}

\mypar{Relationship between kv-stores and dependency graphs.}
\ac{It could be that this subsection gets moved to a different section, where 
we also relate specifications of consistency models using dependency graphs
with execution tests.}
\emph{Dependency graphs} are a formalism  introduced by Adya to specify 
consistency models of transactional databases \cite{adya}. 
They are directed graphs consisting of transactions as nodes, 
each of which is labelled with a set of read and write operations, 
$\Ops \defeq \Setcon{(\otR, \ke, \val), (\otW, \ke, \val) }{ \ke \in \Keys \wedge \val \in \Val }$
and labelled edges between transactions for specifying how information flows within a computation. 
Specifically, a transaction $\txid$ may read a version for a key $\ke$ that has been written by another transaction $\txid'$ 
(\emph{write-read dependency} \( \WR\)), overwrite a version of $\ke$ written by $\txid'$ (\emph{write-write dependency} \( \WW \)), or 
read a version of $\ke$ that is later overwritten by $\txid'$ (\emph{read-write anti-dependency} \( \RW \)).
\begin{definition}
A \emph{dependency graph} is a quadruple $\Gr = (\TtoOp{T}, \RF, \VO, \AD)$, where
\begin{itemize}
\item $\TtoOp{T}: \TxID_{0} \parfinfun \powerset{\Ops}$ is a partial finite function 
mapping transaction identifiers to the set of operations they perform, where there are at most one read operation and one write operation for each key,
\item $\RF : \Keys \to \pset{\dom(\TtoOp{T}) \times \dom(\TtoOp{T})}$ is a function that 
maps each key $\ke$ into a relation between transactions, such that for any $\txid, \txid_1, \txid_2, 
\ke, \cl, m, n$: 
\begin{itemize}
\item if $(\otR, \ke, \val) \in \TtoOp{T}(\txid)$, either $\val = \val_0$ 
and there exists no $\txid'$ such that $\txid' \xrightarrow{\RF(\ke)} \txid$,  
or there exists $\txid'$ such that $(\otW, \ke, \val) \in \TtoOp{T}(\txid')$, and $\txid' \xrightarrow{\RF(\ke)} \txid$, 
\item if $\txid_1 \xrightarrow{\RF(\ke)} \txid$ and $\txid_2 \xrightarrow{\RF(\ke)} \txid$, then 
$\txid_2 = \txid_1$, 
\item if $\txid_{\cl}^{m} \xrightarrow{\RF(\ke)} \txid_{\cl}^{n}$, then $m < n$.
\end{itemize}
\item $\VO: \Keys \to \pset{\dom(\TtoOp{T}) \times \dom(\TtoOp{T})}$ is a function 
that maps each key into an irreflexive relation between transactions, such that for any $\txid, \txid', \ke, \cl, m, n$, 
\begin{itemize}
\item if $\txid \xrightarrow{\VO(\ke)} \txid'$, then $(\otW, \ke, \_) \in \TtoOp{T}(\txid), (\otW, \ke, \_) \in \TtoOp{T}(\txid')$, 
\item if $(\otW, \ke, \_) \in \TtoOp{T}(\txid), (\otW, \ke, \_) \in \TtoOp{T}(\txid')$, then either $\txid = \txid'$, 
$\txid \xrightarrow{\VO(\ke)} \txid'$, or $\txid' \xrightarrow{\VO(\ke)} \txid$, 
\item if $\txid_{\cl}^{m} \xrightarrow{\RF(\ke)} \txid_{\cl}^{n}$, then $m < n$.
\end{itemize}
\item $\AD: \Keys \to \pset{\dom(\TtoOp{T}) \times \dom(\TtoOp{T})}$ is defined 
by letting $\txid \xrightarrow{\AD(\ke)} \txid'$ if and only if $(\otR, \ke, \_) \in \TtoOp{T}(\txid)$, 
$(\otW, \ke, \_) \in \TtoOp{T}(\txid')$ and 
either there exists no $\txid''$ such that $\txid'' \xrightarrow{\RF(\ke)} \txid$, or 
$\txid'' \xrightarrow{\RF(\ke)} \txid$, $\txid'' \xrightarrow{\VO(\ke)} \txid'$ for 
some $\txid''$.
\end{itemize}
\end{definition}
Given a dependency graph $\Gr = (\TtoOp{T}, \RF, \VO, \AD)$, we often 
commit an abuse of notation and use $\RF$ to denote the relation 
$\bigcup\limits_{\ke \in \Keys} \RF(\ke)$; a similar notation is adopted for $\VO, \AD$. 
It will always be clear from the context whether the symbol $\RF$ refers to a function 
from keys to relations, or to a relation between transactions. 

%A dependency graph $\Gr = (\TtoOp{T}, \RF, \VO, \AD)$ is well-formed if 
%$(\PO \cup \RF \cup \VO)$ is acyclic, i.e. its transitive closure is irreflexive. 
%Henceforth, we always assume that dependency graphs are well-formed, 
%and we let 
We let $\Dgraphs$ be the set of all dependency graphs.
It is always possible to convert a kv-store $\hh$ into a well-formed dependency 
graph. For example, \cref{fig:hheap-b} illustrates the dependency graph constructed 
from the kv-store depicted in \cref{fig:hheap-a}.

\begin{definition}
\label{def:kv2graph}
Given a kv-store $\hh$, the \emph{dependency graph} $\Gr_{\hh} = (\TtoOp{T}_{\hh}, \RF_{\hh}, 
\VO_{\hh}, \AD_{\hh})$ is defined as follows: 
\begin{itemize}
\item for any $\txid \neq \txid_0$, $\TtoOp{T}_{\hh}(\txid)$ is defined if and only if there exists an index $i$ and a key 
$\ke$ such that either $\txid = \WTx(\hh(\ke, i))$, or $\txid \in \RTx(\hh(\ke,i))$; furthermore, 
$(\otW, \ke, \val) \in \TtoOp{T}(\txid)$ if and only 
if $\txid = \WTx(\hh(\ke, i))$ for some $i$, and 
$(\otR, \ke, \val) \in \TtoOp{T}(\txid)$ if and only if $\txid \in \RTx(\hh(\ke, i))$ for some $i$, 
\item $\txid \xrightarrow{\RF(\ke)} \txid'$ if and only if there exists an index $i: 0 < i < \lvert \hh(\ke) \rvert$ 
such that $\txid = \WTx(\hh(\ke, i))$, and $\txid' \in \RTx(\hh(\ke, i))$, 
\item $\txid \xrightarrow{\VO(\ke)} \txid'$ if and only if there exist two indexes $i,j$: $0 < i < j < \lvert \hh(\ke) \rvert$ 
such that $\txid = \WTx(\ke, i)$, $\txid' = \WTx(\ke, j)$, 
\item $\txid \xrightarrow{\AD(\ke)} \txid'$ if and only if there exist two indexes $i,j$: $0 < i < j < \lvert \hh(\ke) \rvert$ 
such that $\txid \in \RTx(\ke, i)$ and $\txid' = \WTx(\ke, j)$.
\end{itemize}
\end{definition}

\begin{theorem}
\label{thm:kv2graph}
The function $\Gr_{(\stub)}$ is a bijection between kv-stores and well-formed dependency graphs.
\end{theorem}
\begin{proof}
See \cref{sec:kv2graph-proof}.
\end{proof}

\ac{Both the formal definition of dependency graph and the function $\graphof$ will need to go in the appendix.}

The full operational semantics is given in \cref{fig:full-semantics}.

\begin{figure*}[!t]
\begin{align*}
\toTRANS & : ((\Stacks \times \Snapshots \times \Fingerprints) 
\\ & \quad \times \Transactions) \times ((\Stacks \times \Snapshots \times \Fingerprints) \times \Transactions)
\end{align*}
\begin{mathpar}
    \inferrule[\rl{TPrimitive}]{%
        (\stk, \sn) \toLTS{\transpri} (\stk', \sn') 
        \\
        \op = \func{op}[\stk, \sn, \transpri]
    }{%
        (\stk, \sn, \fp) , \transpri  \toTRANS   (\stk', \sn', \fp \addO \op) , \pskip 
    }%
    \\
     \inferrule[\rl{TChoice}]{%
		i \in \Set{1,2}
    }{%
		(\stk, \sn, \fp) , \trans_{1} \pchoice \trans_{2}  \toTRANS  (\stk, \sn, \fp) , \trans_{i}
    }
    \and
    \inferrule[\rl{TIter}]{ }{%
        (\stk, \sn, \fp),  \trans\prepeat \toTRANS  (\stk, \sn, \fp), \pskip \pchoice (\trans \pseq \trans\prepeat)
    }%
    \and
    \inferrule[\rl{TSeqSkip}]{ }{%
        (\stk, \sn, \fp), \pskip \pseq \trans \toTRANS  (\stk, \sn, \fp), \trans
    }%
    \and
    \inferrule[\rl{TSeq}]{%
		(\stk, \sn, \fp), \trans_{1} \toTRANS  (\stk', \sn', \fp'), \trans_{1}'
    }{%
		(\stk, \sn, \fp), \trans_{1} \pseq \trans_{2} \toTRANS  (\stk', \sn', \fp'), \trans_{1}' \pseq \trans_{2}
    }%
\end{mathpar}
\hrulefill
\begin{align*}
	\toCMD{}  & : 
    \begin{multlined}[t]
    \Clients \; \times \;
	\left( ( \MKVSs \times \Views \times \Stacks ) \times \Commands \right)  
    \\ \times\; \ETs \;\times \sort{Labels} \times \;
	\left( ( \MKVSs \times \Views \times \Stacks ) \times \Commands \right) 
    \end{multlined}
\end{align*}
\begin{mathpar}
     \inferrule[\rl{CAtomicTrans}]{%
        \vi \viewleq  \vi'' 
        \\
        \sn = \snapshot[\mkvs,\vi''] 
        \\
        (\stk, \sn, \emptyset), \trans \toTRANS^{*}   (\stk', \stub,
  \fp) , \pskip
  \\
   \txid \in \nextTxid[\cl, \mkvs] 
    \\\\
     \mkvs' = \updateKV[\mkvs, \vi'', \fp, \txid] 
\\
	\cancommit{\mkvs}{\vi''}{\fp}
\\
	\vshift{\mkvs}{\vi''}{\mkvs'}{\vi'}	
    }{%
        \cl \vdash 
        ( \mkvs, \vi, \stk ), \ptrans{\trans} 
        \toCMD{(\cl, \vi'', \fp)}_{\ET}
        (\mkvs',\vi', \stk' ) , \pskip
    }%
    \and
     %\inferrule[\rl{CAtomicTrans}]{%
        %\vi \viewleq  \vi'' 
        %\\
        %\sn = \snapshot[\mkvs,\vi''] 
        %\\\\
        %(\stk, \sn, \emptyset), \trans \toTRANS^{*}   (\stk',  \sn',
  %\fp) , \pskip
  %\\\\
    %\cancommit{\mkvs}{\vi''}{\fp}
%\\\\
  %\txid \ \text{is fresh} 
    %\\
     %\mkvs' = \updateKV[\mkvs, \vi'', \fp, \txid] 
%\\\\
    %%\vshift{\mkvs}{\vi''}{\mkvs'}{\vi'}	
%\textsf{view-shift}_{\ET}({\mkvs,}{\vi'',}{\mkvs',}{\vi'})
    %}{%
        %\cl \vdash 
        %( \mkvs, \vi, \stk ), \ptrans{\trans} 
        %\toCMD{(\cl, \vi'', \fp)}_{\ET}
        %(\mkvs',\vi', \stk' ) , \pskip
    %}%
    %\and
    \inferrule[\rl{CPrimitive}]{
		\stk \toLTS{\cmdpri} \stk'
    }{
        \cl \vdash 
        ( \mkvs, \vi, \stk ) , \cmdpri 
        \toCMD{(\cl,\iota)}_{\ET} 
        ( \mkvs, \vi, \stk' ) , \pskip
    }%
    \and
    \inferrule[\rl{CChoice}]{%
        i \in \Set{1,2}
    }{%
        \cl \vdash ( \mkvs, \vi, \stk ) , \cmd_{1} \pchoice \cmd_{2} \ \toCMD{(\cl,\iota)}_{\ET} \  ( \mkvs, \vi, \stk ) , \cmd_{i}
    }
    \and
    \inferrule[\rl{CIter}]{ }{%
        \cl \vdash ( \mkvs, \vi, \stk ) , \cmd\prepeat \ \toCMD{(\cl,\iota)}_{\ET} \  ( \mkvs, \vi, \stk ) , \pskip \pchoice (\cmd \pseq \cmd\prepeat)
    }
    \and
    \inferrule[\rl{CSeqSkip}]{ }{%
        \cl \vdash ( \mkvs, \vi, \stk ) , \pskip \pseq \cmd \ \toCMD{(\cl,\iota)}_{\ET} \  ( \mkvs, \vi, \stk ) , \cmd
    }
    \and
    \inferrule[\rl{CSeq}]{% 
        \cl \vdash ( \mkvs, \vi, \stk ) , \cmd_{1} \ \toCMD{(\cl,\iota)}_{\ET} \  ( \mkvs, \vi', \stk' ) , {\cmd_{1}}' 
    }{%
        \cl \vdash ( \mkvs, \vi, \stk ) , \cmd_{1} \pseq \cmd_{2} \ \toCMD{(\cl,\iota)}_{\ET} \ ( \mkvs, \vi', \stk' ) , {\cmd_{1}}' \pseq \cmd_{2}
    }
\end{mathpar}

\hrulefill

\[
	\toPROG{} : 
    ( \Confs \times \ThdEnv \times \Programs) 
    \;\times\; \ETs \;\times \sort{Label} \times \;
    ( \Confs \times \ThdEnv \times \Programs) 
\]
\begin{mathpar}
    \inferrule[\rl{PProg}]{%
        \cl \vdash ( \mkvs, \vienv(\cl), \thdenv(\cl) ), \prog(\cl), \ \toCMD{\lambda}_{\ET} \  ( \mkvs', \vi', \stk' ) , \cmd'  
    }{%
    ( \mkvs, \vienv, \thdenv ) , \prog \ \toPROG{\lambda}_{\ET} \  ( \mkvs', \vienv\rmto{\cl}{\vi'}, \thdenv\rmto{\cl}{\stk'}) , \prog\rmto{\cl}{\cmd'} 
    }
\end{mathpar}
%
\hrulefill
\caption{Operational Semantics on Key-value Store}
\label{fig:transaction_semantics}
\label{def:thread_semantics}
\label{fig:thread_semantics}
\label{def:thread_pool_semantics}
\label{fig:thread_pool_semantics}
\label{def:program_semantics}
\label{fig:program_semantics}
\label{fig:full-semantics}
\end{figure*}

% composition of kv
\section{Execution Test and Compositionality}
\label{app:compositionality}
\subsection{Sanity Check for \( \ET \)}
\label{sec:mono-et}
\begin{proposition}
\label{prop:mono-et}
if $\ET_1 \subseteq \ET_2$ then $\CMs(\ET_1) \subseteq \CMs(\ET_2)$.
\end{proposition}
\begin{proof}
It is sufficient to prove that \(\ET_1 \subseteq \ET_2 \implies \Confs(\ET_1) \subseteq \Confs(\ET_2) \).
We prove it by induction on the length of the traces, \( i \).

\caseB{i = 0}
We have \( \conf_0 \in \Confs(\ET_1) \) and \( \conf_0 \in \Confs(\ET_2)\).
\caseI{i + 1}
Suppose identical traces of \( \ET_1 \) and \( \ET_2 \) respectively with length \( i \).
Let the final configuration be \( \conf_i = ( \mkvs_i, \viewFun_i ) \).
If the next step is a view shift or a step with empty fingerprint, it trivially holds.
If the next step is a step by a client \( \cl \) with fingerprint \( \fp \),
we have \( \ET_1 \vdash (\mkvs_i, \viewFun_i(\cl)) \csat \fp : (\mkvs_{i+1}, \vi') \),
where \( \mkvs_{i+1} \in \updKV{ \mkvs_i, \viewFun_i(\cl), \fp, \cl} \)
The next configuration from \( \ET_1 \) is \( \conf_{i+1} =  (\mkvs, \viewFun_i\rmto{\cl}{\vi'})\).
Since \( \ET_1 \subseteq \ET_2 \), so \( \ET_2 \vdash (\mkvs_i, \viewFun_i(\cl)) \csat \fp : (\mkvs_{i+1}, \vi') \) holds.
It is possible for \( \ET_2 \) to have the exactly same next configuration \( \conf_{n+1}\).
\end{proof}

\subsection{Normal \( \ET \) Traces}
\label{sec:normal-form-exist}
For technical reasons, it will be convenient to adopt a reduction strategy for inferring kv-stores induced by an 
execution test: such an execution strategy require that clients only commit transactions with non-empty fingerprints, 
and a client updates its view only immediately before committing a transaction. 
The next proposition states that all kv-stores induced by an execution test $\ET$ can be 
obtained via a sequence of reductions that adhere to the reduction strategy outlined above. 
Note that throughout this section, we assume that the execution test $\ET$ is fixed.
\begin{definition}
Let $\ET$ be an execution test. The $\ET$-trace
\[
\conf_0 \xrightarrowtriangle{\alpha_0}_{\ET} \conf_1 \xrightarrowtriangle{\alpha_1}_{\ET} \cdots \xrightarrowtriangle{\alpha_{2n}}_{\ET} \conf_{2n + 1}
\]
is in \emph{normal form} if \textbf{(i)} $\conf_0$ is initial, and 
\textbf{(ii)} $\forall i=0,\cdots, n$ there exists a client $\cl_i$ and set of operations $\opset_{i}$ such that 
$\alpha_{2 \cdot i} = (\cl_{i}, \varepsilon)$, and $\alpha_{2 \cdot i + 1}$ is defined and equal to $(\cl_{i}, \opset_{i})$ where \( \f_i \neq \emptyset \).
\end{definition}

We are about to prove that for any trace, there exists a equivalent normal trace in \cref{prop:et.normalform}.
To do so we first need \cref{lem:et.absorb,plem:viewshift.rightmover}.

\sx{!!!!!!!!!!!!!!}

\begin{lemma}[Absorption]
\label{lem:et.absorb}
If $\conf \xrightarrowtriangle{(\cl, \varepsilon)}_{\ET} \conf' \xrightarrowtriangle{(\cl, \varepsilon)} \conf''$, then 
$\conf \xrightarrowtriangle{(\cl, \varepsilon)}_{\ET} \conf''$.
\end{lemma}

\begin{proof}
Let $\conf = (\hh, \viewFun)$, $\conf' = (\hh', \viewFun')$, $\conf'' = (\hh', \viewFun'')$. 
By \cref{def:reductions} it must be the case that $\hh = \hh'$, and $\viewFun' = \viewFun\rmto{\cl}{\vi'}$ 
for some $\vi' : \vi \sqsubseteq \vi'$. It must also be the case that $\hh' = \hh''$, and $\viewFun'' = \viewFun'\rmto{\cl}{\vi''}$ 
for some $\vi'': \vi' \sqsubseteq \vi''$. Therefore we have that $\hh'' = \hh' = \hh$, and 
$\viewFun'' = \viewFun'\rmto{\cl}{\vi''} = (\viewFun\rmto{\cl}{\vi'})\rmto{\cl}{\vi''} = viewFun\rmto{\cl}{\vi''}$, 
and $\vi \sqsubseteq \vi''$. By \cref{def:reductions}, it follows that 
$\conf = (\hh, \viewFun) \xrightarrowtriangle{(\cl, \varepsilon)} (\hh'', \viewFun'') = \conf''$.
\end{proof}

\begin{lemma}[Commit Swap]
\label{lem:viewshift.rightmover}
Let $\conf \xrightarrowtriangle{(\cl, \varepsilon)}_{\ET} \conf_1 \xrightarrowtriangle{(\cl', \mu)}_{\ET} \conf'$ 
for some $\conf, \conf_1, \conf''$ and $\cl, \cl'$ such that $\cl' \neq \cl$. 
Then $\conf \xrightarrowtriangle{(\cl', \mu)}_{\ET} \conf_2 \xrightarrowtriangle{(\cl, \varepsilon)}_{\ET} \conf'$ 
\end{lemma}

\begin{proof}
We only consider the case where $\mu = \opset$ for some fingerprint $\opset$. The case where 
$\mu = \varepsilon$ is simpler to prove.
Let $\conf = (\hh, \viewFun)$, $\conf_1 = (\hh_1, \viewFun_1)$, $\conf' = (\hh', \viewFun')$. 
Let also $\vi = \viewFun(\cl)$.
By \cref{def:reductions} we have that $\hh_1 = \hh, \viewFun_1 = \viewFun\rmto{\cl}{\vi_1}$ for 
some $\vi_1: \vi_1 \sqsubseteq \vi_1$. Let $\vi' = \viewFun(\cl')$: then we have that $\viewFun_1(\cl') = 
\vi'$. Because $(\hh_1, \viewFun_1) \xrightarrowtriangle{(\cl', \opset)}_{\ET} (\hh', \viewFun')$, we have that 
$\ET \vdash \hh_1, \vi' \triangleright \opset : \vi''$, where $\vi'' = \viewFun'(\cl')$. Because $\hh_1 = \hh$, 
that means that $\ET \vdash \hh, \vi' \triangleright \opset: \vi''$: by \cref{def:reductions} it follows that 
$(\hh, \viewFun) \xrightarrowtriangle{(\cl', \opset)}_{\ET} (\hh', \viewFun\rmto{\cl'}{\vi''}) 
\xrightarrowtriangle{(\cl, \varepsilon)}_{\ET} (\hh', \viewFun\rmto{\cl'}{\vi''}\rmto{\cl}{\vi_1}) = 
(\hh', \viewFun\rmto{\cl}{\vi_1}\rmto{\cl'}{\vi''}) = (\hh', \viewFun_1\rmto{\cl'}{\vi''}) = 
(\hh', \viewFun')$, as we wanted to prove.
\end{proof}

\begin{proposition}[Normal \( \ET \) Traces]
\label{prop:et.normalform}
Let $\ET$ be an execution test, and suppose that $\hh \in \CMs(\ET)$. Then there exists a $\ET$-trace  
\[
(\hh_0, \viewFun_0) \xrightarrowtriangle{\stub}_{\ET} \cdots \xrightarrowtriangle{\stub}_{\ET} (\hh_n, \viewFun_{n})
\]
that is in normal form, and such that $\hh_{n} = \hh$.
\end{proposition}
\begin{proof}
Let $\hh \in \CMs(\ET)$. By definition, there exists a sequence of reductions 
\begin{equation}
\label{eq:normalform.sequence}
(\hh_{0}, \viewFun_{0}) \xrightarrowtriangle{(\cl_0, \mu_0)}_{\ET} \cdots \xrightarrowtriangle{(\cl_{n-1}, \mu_{n-1})}_{\ET} (\hh_n, \viewFun_{n})
\end{equation}
such that $\hh_{n} = \hh$. Given an index $i = 1,\cdots, n-1$, we say that the action $(\cl_{i}, \mu_{i})$ is \emph{in place} 
if, $\mu_{i} = \opset_{i}$ for some $\opset_{i}$, $\cl_{i-1} = \cl_{i}$, $\mu_{i-1} = \varepsilon$, and if $(\cl_{j}, \mu_{j}) = (\cl_{i}, \varepsilon)$, 
for some  $j = 0,\cdots, i-2$, then there exists $j': j < j' < i$ such that $(\cl_{j'}, \mu_{j'}) = (\cl_i, \opset_{j'})$. An action of the 
form $(\cl_{i}, \mu_{i})$ is \emph{out of place} if it is not in place. 

Given the sequence of reductions in \cref{eq:normalform.sequence}, we show the following: 
\begin{enumerate}
\item if the sequence has no action out of place, then there exists a sequence 
\[
(\hh'_{0}, \viewFun'_{0}) \xrightarrowtriangle{(\cl'_{0}, \mu'_{0})}_{\ET} \cdots \xrightarrowtriangle{(\cl'_{m-1}, \mu'_{m-1})}_{\ET} (\hh'_{m}, \viewFun'_{m})
\]
that is in normal form, and such that $\hh'_{m} = \hh_{n}$, and 
\item if the sequence has $h$ actions out of place, for some $h > 0$, then there exists a sequence 
\[
(\hh'_{0}, \viewFun'_{0}) \xrightarrowtriangle{(\cl'_{0}, \mu'_{0})}_{\ET} \cdots \xrightarrowtriangle{(\cl'_{m-1}, \mu'_{m-1})}_{\ET} (\hh'_{m}, \viewFun'_{m})
\]
that has $h-1$ actions out of place, and such that $\hh'_{m} = \hh_{n}$.
\end{enumerate}
Combining the two facts above, we obtain that if $\hh \in \CMs(\ET)$, then there exists a sequence of reductions in formal form whose final 
configuration is $(\hh, \_)$, as we wanted to prove.

\begin{enumerate}
\item Suppose that the sequence of reductions from \cref{eq:normalform.sequence} has no action out of place. 
Let $i=0,\cdots, n-1$, and consider the greatest index $i=0,\cdots, n-1$ such that  
$\mu_{i} = \varepsilon$, and either $i = n-1$, or 
$\forall \opset.\; (\cl_{i+1}, \mu_{i+1}) \neq (\cl_{i}, \opset)$. 
If such an index does not exist, then the sequence of transitions from \cref{eq:normalform.sequence} is in 
normal form, and there is nothing to prove. Otherwise, note that for any $j = i+1,\cdots, n-1$, 
$\forall \opset.\;(\cl_{j}, \mu_{j}) \neq (\cl_{i}, \opset)$. 

Suppose in fact that there existed 
an index $j = i+1,\cdots, n-1$ such that $(\cl_{j}, \mu_{j}) = (\cl_{i}, \opset_{j})$ for some 
$\opset_{j}$, and without loss of generality assume that $j$ is the smallest such index. This implies that 
there exists no index $j': i < j' < j$ such that $(\cl_{j'}, \mu_{j'}) = (\cl_{i}, \opset_{j'})$ for some 
$\opset_{j'}$. Also, it cannot be $j = i+1$, because we are assuming that $\forall \opset.\;(\cl_{i+1}, \mu_{i+1}) \neq 
(\cl_{i}, \opset)$.  We have that $j \geq i+2$; we also have that  $(\cl_{j}, \mu_{j}) = (\cl_{i}, \opset_{j})$, 
$(\cl_{i}, \mu_{i}) = (\cl_{i}, \varepsilon)$, $\forall j': i < j < j'.\forall \opset.\; (\cl_{j'}, \mu_{j'}) \neq (\cl_{i}, \opset)$. 
By definition, the action $(\cl_{j}, \mu_{j})$ is out of place, contradicting the assumption that the sequence of 
reduction of \cref{eq:normalform.sequence} has no actions out of place.

We have proved that $\forall j = i+1, \cdots, n-1.\;\forall \opset.\;(\cl_{j}, \mu_{j}) \neq (\cl_{i}, \opset)$. 
Also, because we are assuming that $\mu_{i}$ is the greatest index such that $\mu_{i} = \varepsilon$, 
and either $i= n-1$ or $\forall \opset.\;(\cl_{i+1},\mu_{i+1}) \neq (\cl_{i}, \opset)$, 
then $\forall j=i+1,\cdots, n-1\;\forall \mu.\;(\cl_{j}, \mu_{j}) \neq (\cl_{i}, \mu)$. 
Consider the transition 
\[
(\hh_{i}, \viewFun_{i}) \xrightarrowtriangle{(\cl_{i}, \mu_{i})}_{\ET} (\hh_{i+1}, \viewFun_{i+1}).
\]
Let $\vi = \viewFun_{i}(\cl)$. Because $\mu_{i} = \varepsilon$, then it must be the case that 
$\hh_{i} = \hh_{i+1}$, $\viewFun_{i+1} = \viewFun_{i}\rmto{\cl}{\vi'}$ for some $\vi' : \vi \sqsubseteq \vi'$. 
For any $j \geq i$, we have that $\cl_{j} \neq \cl_{i}$. We can replace the transition 
\[
(\hh_{j}, \viewFun_{j}) \xrightarrowtriangle{(\cl_{j}, \mu_{j})}_{\ET} (\hh_{j+1}, \viewFun_{j+1})
\]
with 
\[
(\hh_{j}, \viewFun_{j}\rmto{\cl_{i}}{\vi}) \xrightarrowtriangle{(\cl_{j}, \mu_{j})}_{\ET} (\hh_{j+1}, \viewFun_{j+1}\rmto{\cl_{i}}{\vi}.
\]
It follows that the sequence of transitions 
\[ 
(\hh_{0}, \viewFun_{0}) \xrightarrowtriangle{(\cl_{0}, \mu_{0})}_{\ET} \cdots \xrightarrowtriangle{(\cl_{i-1},\mu_{i-1})} 
(\hh_{i}, \viewFun_{i}) = (\hh_{i+1}, \viewFun_{i+1}\rmto{\cl_{i},\vi}) \xrightarrowtriangle{(\cl_{i+1}, \mu_{i+1})}_{\ET} \cdots 
\xrightarrowtriangle{(\cl_{n-1}, \mu_{n-1})}_{\ET} (\cl_{n}, \viewFun_{n}\rmto{\cl_{i}, \vi})
\]
Note that this sequence has one reduction less than the original sequence from \eqref{eq:normalform.sequence} (specifically, 
the reduction $(\hh_{i}, \viewFun_{i}) \xrightarrowtriangle{(\cl_{i}, \mu_{i})} (\hh_{i+1}, \viewFun_{i+1})$ has 
been removed). We can repeat this procedure until the resulting sequence of reductions has no index $i=0,\cdots, n-1$ such that  
$\mu_{i} = \varepsilon$, and either $i = n-1$, or 
$\forall \opset.\; (\cl_{i+1}, \mu_{i+1}) \neq (\cl_{i}, \opset)$. That is, the resulting sequence of reductions is in normal form, 
and its final configuration is $(\hh_{n}, \_)$.

\item Suppose that the sequence from \cref{eq:normalform.sequence} has $h$ actions out of place, 
where $h > 0$. Let $i$ be the smallest index such that $(\cl_{i}, \mu_{i})$ is out of place. 
This means that either $i = 0$, or $(\cl_{i-1}, \mu_{i-1}) \neq (\cl_{i}, \varepsilon)$, 
or there exists an index $j < i -1 $ such that $(\cl_{j}, \mu_{j}) = (\cl_{i}, \varepsilon)$ 
and, $\forall j': j < j' < i.\;\forall \opset.\;(\cl_{j'}, \mu_{j'}) \neq (\cl_{i}, \opset)$. 
Without loss of generality, we can assume that $i \neq 0$ and $(\cl_{i-1}, \mu_{i-} = (\cl_{i}, \varepsilon)$. 
This is because we can always transform the sequence of reductions of \cref{eq:normalform.sequence} by 
introducing a transition of the form $(\hh_{i}, \viewFun_{i}) \xrightarrowtriangle{(\cl_{i}, \varepsilon)}_{\ET}
(\hh_{i}, \viewFun_{i})$, leading to the sequence of reductions
\[
(\hh_{0}, \viewFun_{0}) \xrightarrowtriangle{(\cl_{0}, \mu_{0})}_{\ET} \cdots \xrightarrowtriangle{(\cl_{i-1}, \mu_{i-1})}_{\ET}
(\hh_{i}, \viewFun_{i}) \xrightarrowtriangle{(\cl_{i}, \varepsilon)}_{\ET} (\hh_{i}, \viewFun_{i}) \xrightarrowtriangle{(\cl_{i+1}, \mu_{i+1})}_{\ET} 
\cdots \xrightarrowtriangle{(\cl_{n-1}, \mu_{n-1})}_{\ET} (\hh_{n}, \viewFun_{n}).
\]

Therefore, it must be the case that there exists an index $j < i-1$ such that $(\cl_{j}, \mu_{j}) = (\cl_{i}, \varepsilon)$, 
and $\forall j': j< j' < i.\;\forall \opset.\;(\cl_{j'}, \mu_{j'}) \neq (\cl_{i}, \opset)$. Let then $j$ be the smallest such index. 
Let $d = (i-1)-j$ be the number of reductions that separate the configuration $(\hh_{j}, \mu_{j})$ from 
$(\hh_{i-1}, \mu_{i-1})$ in \cref{eq:normalform.sequence}. Note that it must be the case that $d > 0$. We show that we can 
construct a sequence of reductions where the distance between these two configurations is reduced to $0$: 
a consequence of this fact is such a sequence of reductions would have exactly $h-1$ actions out of place.
Consider the following fragment in the sequence of reductions from \cref{eq:normalform.sequence}:
\[
(\hh_{j}, \viewFun_{j}) \xrightarrowtriangle{(\cl_{j}, \mu_{j})}_{\ET} (\hh_{j+1}, \viewFun_{j+1}) 
\xrightarrowtriangle{(\cl_{j+1}, \mu_{j+1})}_{\ET} (\hh_{j+2}, \viewFun_{j+2}).
\]
We have two possible cases: 
\begin{itemize}
\item $\cl_{j+1} \neq \cl_{j}$. In this case we can apply \cref{lem:viewshift.rightmover} and infer the sequence of 
reductions 
\[
(\hh_{j}, \viewFun_{j}) \xrightarrowtriangle{(\cl_{j+1}, \mu_{j+1}}_{\ET} (\hh_{j+1}', \viewFun_{j+1}') 
\xrightarrowtriangle{(\cl_{j}, \mu_{j})}_{\ET} (\hh_{j+1}, \viewFun_{j+1}).
\]
which leads to the whole sequence of reductions 
\[
(\hh_{0}, \viewFun_{0}){(\cl_{0}, \mu_{0})}_{\ET} \cdots 
\xrightarrowtriangle{(\cl_{j-1}, \mu_{j-1})}_{\ET} (\cl_{j}, \mu_{j}) \xrightarrowtriangle{(\cl_{j+1}, \mu_{j+1})}_{\ET} 
(\hh'_{j+1}, \viewFun'_{j+1}) \xrightarrowtriangle{(\cl_{j}, \mu_{j})}_{\ET} (\hh_{j+1}, \viewFun_{j+1}) 
\xrightarrowtriangle{(\cl_{j+2}, \mu_{j+2})}_{\ET} \cdots \xrightarrowtriangle{(\cl_{n-1}, \mu_{n-1})} \hh_{n}, \viewFun_{n}.
\]
\item $\cl_{j+1} = \cl_{j}$. In this case we can apply \cref{lem:et.absorb} and infer the reduction 
\[
(\hh_{j}, \viewFun_{j}) \xrightarrowtriangle{(\cl_{j}, \varepsilon)}_{\ET} (\hh_{j+2}, \viewFun_{j+2}),
\]
which leads to the sequence of reductions 
\[
(\hh_{0}, \viewFun_{0}){(\cl_{0}, \mu_{0})}_{\ET} \cdots 
\xrightarrow{(\cl_{j-1}, \mu_{j-1})}_{\ET} (\cl_{j}, \mu_{j}) \xrightarrowtriangle{(\cl_{j}, \varepsilon)}_{\ET} 
(\hh_{j+2}, \viewFun_{j+2}) \xrightarrowtriangle{(\cl_{j+2}, \mu_{j+2})}_{\ET} \cdots 
\xrightarrowtriangle{(\cl_{n-1}, \mu_{n-1})}_{\ET} (\hh_{n}, \viewFun_{n}).
\]
\end{itemize}
In both cases, in the resulting sequence of reductions the number of reductions that separate 
the configuration $(\hh_{j}, \viewFun_{j})$ from $(\hh_{i-1}, \viewFun_{i-1})$ is strictly 
less than $d$. We can repeating applying the procedure outlined above until there are 
no reductions that separate the configuration $(\hh_{j}, \viewFun_{j})$ from 
$(\hh_{i}, \viewFun_{i})$.
\end{enumerate}
\ac{This was more of a proof sketch, rather than a real proof. For the moment it will suffice, though 
I will need to go back at it when all the other results are sorted.}
\end{proof}

\subsection{Commutativity \( \updateKV \)}
\label{sec:updatekv-comm}

By \cref{cor:updatekv.singlecell}, then \cref{prop:updatekv.comm} holds, that is,
given two non-conflict commits, it is possible to swap the commit order as shown in \cref{prop:updatekv.comm}.


\begin{proposition}
\label{prop:updatekv.comm}
\label{prop:swap-update}
Let $\mkvs \in \MKVSs$, $\vi_1, \vi_2 \in \Views(\mkvs)$ and let $\cl_1, \cl_2 \in \Clients$ 
be such that $\cl_1 \neq \cl_2$. 
Let also $\fp_1, \fp_2 \in \pset{\Ops}$ be such that 
whenever $(\otW, \key, \stub) \in \fp_1$ for some key $\key$, then 
$(\otW, \key, \val) \notin \fp_2$ for all $\val \in \Val$. Then 
\begin{centermultline}
\Set{ \updateKV[\mkvs_1, \vi_2, \fp_2, \cl_2]}[\mkvs_1 \in \updateKV[\mkvs, \vi_1, \fp_1, \cl_1]] =  \\
\Set{ \updateKV[\mkvs_2, \vi_1, \fp_1, \cl_1]}[\mkvs_2 \in \updateKV[\mkvs, \vi_2, \fp_2, \cl_2]]
\end{centermultline}
\end{proposition}

\begin{proof}
Assume $\mkvs_1 = \updateKV[\mkvs, \vi_1, \fp_1, \txid_1]$, $\mkvs_2 = \updateKV[\mkvs, \vi_2, \fp_2, \txid_2]$. 
It suffices to show that for any key $\key$:
\[\lvert \updateKV[\mkvs_1, \vi_2, \fp_2, \txid_2](\key) \rvert = \lvert 
\updateKV[\mkvs_2, \vi_1, \fp_1, \txid_1](\key) \rvert
\]
and for any index $i$ such that \( 0 \leq i < \lvert \updateKV[\mkvs_1, \vi_2, \fp_2, \txid_2](\key) \rvert \):
\[
\updateKV[\mkvs_1, \vi_2,\fp_2, \txid_2](\key, i) = \updateKV[\mkvs_2, \vi_1, \fp_1, \txid_1](\key_1)
\]

First, fix a key $\key \in \Keys$. Note that if $(\otW, \key, \stub) \in \fp_1$, then 
by Corollary \ref{cor:updatekv.singlecell} we have that $\lvert \updateKV[\mkvs, \vi_1, \fp_1, \txid_1](\key) \rvert = 
\lvert \mkvs(\key) \rvert$. Because $\fp_1$ is not conflicting with $\fp_2$, it must be the case 
that $\fora{ \val } (\otW,\key,\val) \notin \fp_2$, and therefore by \cref{cor:updatekv.singlecell} 
we have that 
\[
\lvert \updateKV[\mkvs_1, \vi_2, \fp_2, \txid_2](\key) \rvert = \lvert \mkvs_1(\key) \rvert = \lvert \mkvs(\key) \rvert + 1.
\] 
Similarly, because $\fora{ \val } (\otW,\key,\val)\notin \fp_2$ 
and $(\otW,\key,\stub) \in \fp_1$, then 
\[
\lvert \updateKV[\mkvs_2, \vi_1, \fp_1, \txid_1](\key) \rvert = \lvert \mkvs_2(\key) \rvert + 1 
= \lvert \updateKV[\mkvs, \vi_2, \fp_2, \txid_2](\key) \rvert = \lvert \mkvs(\key) \rvert + 1
\]
Therefore, if $(\otW, \key, \stub) \in \fp_1$, we have that 
\[ 
\lvert \updateKV[\mkvs_2, \vi_1, \fp_1, \txid_1](\key) \rvert = 
\lvert \updateKV[\mkvs_1(\key), \vi_2, \fp_2, \txid_2](\key) \rvert
\]

Analogously, we can prove that this claim holds also when $(\otW, \key, \stub) \in \fp_2$. 
Finally, if $(\fora{ \val } (\otW,\key,\stub) \notin \fp_1) \land (\fora{ \val } (\otW,\key,\val) \notin \fp_2)$, 
then by \cref{cor:updatekv.singlecell} we have that 
\[
\lvert \updateKV[\mkvs_1, \vi_2, \fp_2, \txid_2](\key) \rvert = 
\lvert \mkvs_1(\key) \rvert = \lvert \updateKV[\mkvs, \vi_1, \fp_1, \txid_1](\key) \rvert = \lvert \mkvs(\key) \rvert
\]
and
\[
\lvert \updateKV[\mkvs_2, \vi_1, \fp_1, \txid_1](\key) \rvert = 
\lvert \mkvs_2(\key) \rvert = \lvert \updateKV[\mkvs, \vi_2, \fp_2, \txid_2](\key) \rvert = \lvert \mkvs(\key) \rvert
\]
This concludes the proof that, for any key $\key \in \Keys$,
\[ \lvert \updateKV[\mkvs_1,\vi_2,\fp_2,\txid_2] \rvert = 
\lvert \updateKV[\mkvs_2,\vi_1,\fp_1,\txid_1] \rvert
\]

Next, fix a key $\key$ and a index $i$ such that $0 \leq i < \abs{ \mkvs(\key) } - 1$. 
We show that:
\[ 
    \updateKV[\mkvs_1,\vi_2,\fp_2,\txid_2](\key, i) = \updateKV[\mkvs_2, \vi_1, \fp_1, \txid_1](\key, i)
\]
by performing a case analysis on $\vi_1$: 
\begin{enumerate}
    \item $i \neq \max_{<}(\vi_1(\key))$. 
In this case, by \cref{cor:updatekv.singlecell}\cref{item:updatekv.singlecell.noview}, 
we have that 
\begin{equation}
\mkvs_1(\key, i) = \updateKV[\mkvs, \vi_1, \fp_1, \txid_1](\key, i) = \mkvs(\key, i)
\label{eq:v1.nord.hh1}
\end{equation}
and 
\begin{equation}
\updateKV[\mkvs_2, \vi_1, \fp_1, \txid_1](\key, i) = \mkvs_2(\key, i)
\label{eq:v1.nord.uhh2}
\end{equation}
Then, we have three possible sub-cases: 
\begin{enumerate}
    \item $i \neq \max_{<}(\vi_2(\key))$: in this case, by \cref{cor:updatekv.singlecell}\cref{item:updatekv.singlecell.noview} we have that 
\[
\updateKV[\mkvs_1, \vi_2, \fp_2, \txid_2](\key, i) = 
\mkvs_1(\key, i) \stackrel{\cref{eq:v1.nord.hh1}}{=} \mkvs(\key, i)
\]
and
\[
\updateKV[\mkvs_2, \vi_1, \fp_1, \txid_1] \stackrel{\cref{eq:v1.nord.uhh2}}{=} \mkvs_2(\key,i) = 
\updateKV[\mkvs, \vi_2, \fp_2, \txid_2](\key, i) = \mkvs(\key, i)
\]
\item $i = \max_{<}(\vi_2(\key))$, and $(\otR, \key, \stub) \notin \fp_2$. In this case the proof is analogous to the previous case, 
only \cref{cor:updatekv.singlecell}\cref{item:updatekv.singlecell.nord} needs to be applied in place 
of \cref{cor:updatekv.singlecell}\cref{item:updatekv.singlecell.noview}.
\item $i = \max_{<}(\vi_2(\key))$, and $(\otR, \key, \stub) \in \fp_2$. In this case we can apply \cref{cor:updatekv.singlecell}\cref{item:updatekv.singlecell.rd}, 
and deduce that 
\begin{equation}
\updateKV[\mkvs_1, \vi_2, \fp_2, \txid_2](\key, i) = \mkvs_1(\key, i) \oplus \Set{\txid_2}
\label{eq:v1.nord.v2.rd.uhh1}
\end{equation}
\begin{equation}
\mkvs_2(\key, i) = \updateKV[\mkvs, \vi_2, \fp_2, \txid_2](\key, i) = \mkvs(\key,i) \oplus \Set{\txid_2}
\label{eq:v1.nord.v2.rd.hh2}
\end{equation}
It follows that 
\[
\begin{array}{l}
\updateKV[\mkvs_1, \vi_2, \fp_2, \txid_2](\key, i) \stackrel{\cref{eq:v1.nord.v2.rd.uhh1}}{=} \mkvs_1(\key, i) \oplus \Set{\txid_2} \stackrel{\cref{eq:v1.nord.hh1}}{=} \mkvs(\key, i) \oplus \Set{\txid_2}\\
\updateKV[\mkvs_2,\vi_1,\fp_1, \txid_1](\key, i) \stackrel{\cref{eq:v1.nord.uhh2}}{=} \mkvs_2(\key, i) \stackrel{\cref{eq:v1.nord.v2.rd.hh2}}{=} \mkvs(\key, i) \oplus \Set{\txid_2}
\end{array}
\]
\end{enumerate}
\item $i = \max_{<}(\vi_1(\key))$, $(\otR, \key, \stub) \notin \fp_1$. This case is similar to the previous one: we can infer 
that Equations \cref{eq:v1.nord.hh1} and \cref{eq:v1.nord.uhh2} are valid in this case using \cref{cor:updatekv.singlecell}
\cref{item:updatekv.singlecell.nord}, then we can proceed by performing a case analysis on $\vi_2$ and $\fp_2$ as in the previous case.
\item $i = \max_{<}(\vi_1(\key))$, $\otR,\ ke, \stub) \in \fp_1$. We can apply \cref{cor:updatekv.singlecell}\cref{item:updatekv.singlecell.rd} 
to deduce the following: 
\begin{equation}
\mkvs_1(\key, i) = \updateKV[\mkvs, \vi_1, \fp_1, \txid_1](\key, i) = \mkvs(\key, i) \oplus \Set{\txid_1}
\label{eq:v1.rd.hh1}
\end{equation}
and
\begin{equation}
\updateKV[\mkvs_2, \vi_1, \fp_1, \txid_1](\key, i) = \mkvs_2(\key, i) \oplus \Set{\txid_1}
\label{eq:v1.rd.uhh2}
\end{equation}
We have two different sub-cases to consider: 
\begin{enumerate}
\item $i \neq \max_{<}(\vi_2(\key))$, or $i = \max_{<}(\vi_2(\key))$ with $(\otR,\key,\stub) \notin \fp_2$. In this case, we can apply either 
\cref{cor:updatekv.singlecell}\cref{item:updatekv.singlecell.noview} (if $i \neq \max_{<}(\vi_2(\key))$ ), or 
\cref{cor:updatekv.singlecell} \cref{item:updatekv.singlecell.nord} (if $i = \max_{<}(\vi_2(\key))$ and $(\otR, \key, \stub) \notin \fp_2$), 
to obtain 
\[
\updateKV[\mkvs_1, \vi_2, \fp_2, \txid_2](\key, i) = \mkvs_1(\key, i) \stackrel{\cref{eq:v1.rd.hh1}}{=} \mkvs(\key, i) \oplus \Set{\txid_1}
\]
and
\[
\updateKV[\mkvs_2, \vi_1, \fp_1, \txid_1](\key, i) \stackrel{\cref{eq:v1.rd.uhh2}}{=} \mkvs_2(\key, i) \oplus \Set{ \txid_1 } = 
\mkvs(\key, i) \oplus \Set{ \txid_1 }
\]
\item if $i = \max_{<}(\vi_2(\key))$ and $(\otR, \key, \stub) \in \fp_2$, then by \cref{cor:updatekv.singlecell}\cref{item:updatekv.singlecell.rd} 
we obtain that 
\begin{equation}
\mkvs_2(\key, i) = \updateKV[\mkvs, \vi_2, \fp_2, \txid_2](\key, i) = \mkvs(\key, i) \oplus \Set{\txid_2}
\label{eq:v1.rd.v2.rd.hh2}
\end{equation}
\begin{equation}
\updateKV[\mkvs_1, \vi_2, \fp_2, \txid_2](\key, i) = \mkvs_1(\key, i) \oplus \Set{ \txid_2}
\label{eq:v1.rd.v2.rd.uhh1}
\end{equation}
From these facts it follows that
\[
\begin{array}{l}
\updateKV[\mkvs_1, \vi_2, \fp_2, \txid_2](\key, i) \stackrel{\cref{eq:v1.rd.v2.rd.uhh1}}{=} 
\mkvs_1(\key, i) \oplus \Set{\txid_2} \stackrel{\cref{eq:v1.rd.hh1}}{=} 
(\mkvs(\key, i) \oplus \Set{\txid_1}) \oplus \Set{\txid_2 } = \mkvs(\key, i) \oplus \Set{\txid_1, \txid_2}\\
\updateKV[\mkvs_2, \vi_1, \fp_1, \txid_1](\key, i) \stackrel{\cref{eq:v1.rd.uhh2}}{=} 
\mkvs_2(\key, i) \oplus \Set{\txid_1} 
\stackrel{\cref{eq:v1.rd.v2.rd.hh2}}{=} (\mkvs(\key, i) \oplus \Set{\txid_2}) \oplus \Set{\txid_1} = \mkvs(\key, i) \oplus \Set{\txid_1, \txid_2}
\end{array}
\]
\end{enumerate}
\end{enumerate}

Next, note that if $\fora{ \val \in \Val } (\otW,\key,\val) \notin \fp_1 \land (\otW, \key, \val) \notin 
\fp_2$, then 
\[
\lvert \updateKV[\mkvs_1, \vi_2, \fp_2, \txid_2](\key) \rvert = \lvert \mkvs(\key) \rvert = 
\lvert \updateKV[\mkvs_2, \vi_1, \fp_1, \txid_1](\key) \rvert
\]
Because we have already proved that 
\[
    \fora{ i = 0,\cdots, \lvert \mkvs(\key) \rvert } \updateKV[\mkvs_1, \vi_2, \fp_2, \txid_2](\key, i) = \updateKV[\mkvs_2, \vi_1, \fp_1, \txid_1](\key, i)
\]
It follows that
\[ 
    \updateKV[\mkvs_1, \vi_2, \fp_2, \txid_2](\key) = \updateKV[\mkvs_2,\vi_1,\fp_1,\txid_1](\key)
\]
and there is nothing left to prove.

Suppose then that  either $(\otW, \key, \val) \in \fp_1$ or $(\otW,\key, \val) \in \fp_2$ 
for some $\val$. Without loss of generality, let $(\otW,\key,\val) \in \fp_1$ for some $\val \in \Val$; 
because we are assuming that $\fp_1$ does not conflict with $\fp_2$, then 
it must be the case that $\fora{ \val' \in \Val } (\otW,\key,\val') \notin \fp_2$. 
Using \cref{cor:updatekv.singlecell}\cref{item:updatekv.singlecell.nowr} and 
\cref{cor:updatekv.singlecell}\cref{item:updatekv.singlecell.wr}, 
\[
\begin{array}{l}
\updateKV[\mkvs_1, \vi_2, \fp_2, \txid_2](\key, \lvert \mkvs(\key) \rvert) = 
\mkvs_1(\key, \lvert \mkvs(\key) \rvert) = \updateKV[\mkvs, \vi_1, \fp_1, \txid_1](\lvert \mkvs(\key) \rvert) = (\val, \txid_1, \emptyset)\\
{} \land \updateKV[\mkvs_2, \vi_1,\fp_1, \txid_1](\key, \lvert, \mkvs(\key) \rvert) = (\val, \txid_1, \emptyset)
\end{array}
\]
We have now proved that if $(\otW,\key,\val) \in \fp_1$, then $\lvert \updateKV[\mkvs_1, \vi_2, \fp_2, \txid_2] \rvert = 
\lvert \updateKV[\mkvs_2, \vi_1, \fp_1, \txid_1] \rvert$, and for all 
$i=0,\cdots, \lvert \updateKV[\mkvs_1, \vi_2, \fp_2, \txid_2] \rvert - 1$, 
$\updateKV[\mkvs_1,\vi_2, \fp_2, \txid_2](\key, i) = \updateKV[\mkvs_2, \vi_1, \fp_1, \txid_1](\key, i)$. 
This concludes the proof that for any key \( \key \), $\updateKV[\mkvs_1,\vi_2,\fp_2,\txid_2](\key) = 
\updateKV[\mkvs_2,\vi_1,\fp_1,\txid_1](\key)$, and therefore 
$\updateKV[\mkvs_1, \vi_2, \fp_2, \txid_2] = \updateKV[\mkvs_2, \vi_1,\fp_1,\txid_1]$.
\end{proof}

\subsection{Counter example for composition}
\begin{example}
\label{ex:noncompositional.et}
Define the following terms: 
\[
\begin{array}{lcl}
\hh_0 &=& [\ke_1 \mapsto (\val_0, \txid_0, \emptyset) , \ke_2 \mapsto (\val_0, \txid_0, \emptyset)]\\
\hh_1 &=& \big[\ke_1 \mapsto \big( (\val_0, \txid_0, \emptyset) \lcat (\val_2, \txid_{\cl_1}^{1}, \emptyset)\big) , \ke_2 \mapsto (\val_0, \txid_0, \emptyset) \big]\\
\hh_2 &=& \big[\ke_1, \mapsto (\val_0, \txid_0, \emptyset), \ke_2 \mapsto \big( (\val_0, \txid_0, \emptyset) \lcat (\val_2, \txid_{\cl_2}^{1}, \emptyset) \big) \big]\\
\hh_3 &=& \big[\ke_1 \mapsto \big( (\val_0, \txid_0, \emptyset) \lcat (\val_2, \txid_{\cl_1}^{1}, \emptyset)\big), 
                         \ke_2 \mapsto \big( (\val_0, \txid_0, \emptyset) \lcat (\val_2, \txid_{\cl_2}^{1}, \emptyset) \big) \big]\\
&&\\
\vi_0 &=& [\ke_1 \mapsto \{0\}, \ke_2 \mapsto \{0\}]\\
\viewFun_0 &=& [\cl_1 \mapsto \vi_0, \cl_2 \mapsto \vi_0]\\
&&\\
\ET_1 &\vdash& (\hh_0, \vi_0) \triangleright \{(\otW, \ke_1, \val_1)\} : \vi_0\\
\ET_1 &\vdash& (\hh_1, \vi_0) \triangleright \{(\otW, \ke_2, \val_2)\} : \vi_0\\
&&\\
\ET_2 &\vdash& (\hh_0, \vi_0) \triangleright \{(\otW, \ke_2, \val_2)\} : \vi_0\\
\ET_2 &\vdash& (\hh_2, \vi_0) \triangleright \{(\otW, \ke_1, \val_1)\} : \vi_0.
\end{array}
\]
There are no further constraints on $\ET_1, \ET_2$.
For $\ET_1$ and $\ET_2$, we have that 
\[
\begin{array}{l}
(\hh_0, \viewFun_0) \xrightarrowtriangle{(\cl_1, \{(\otW, \ke_1, \val_1)\})}_{\ET_1} 
(\hh_1, \viewFun_0) \xrightarrowtriangle{(\cl_2, \{(\otW, \ke_2, \val_2)\})}_{\ET_1} (\hh_3, \viewFun_0), \\
(\hh_0, \viewFun_0) \xrightarrowtriangle{(\cl_2, \{(\otW, \ke_2, \val_2)\})}_{\ET_2} 
(\hh_2, \viewFun_0) \xrightarrowtriangle{(\cl_1, \{(\otW, \ke_1, \val_1)\})}_{\ET_2} (\hh_3, \viewFun_0).\\
\end{array}
\] 
Therefore, we have that $\hh_3 \in \CMs(\ET_1) \cap \CMs(\ET_2)$. On the other hand, it is immediate 
to observe that $\ET_1 \cap \ET_2 = \emptyset$, and therefore $\hh_3 \notin \CMs(\ET_1 \cap \ET_2)$.
\end{example}
The reason why compositionality fails, for the execution tests of \cref{ex:noncompositional.et}, 
is that both the execution tests $\ET_1, \ET_2$ require that the fingerprints 
$\{(\otW, \ke_1, \_)\}, \{(\otW, \ke_2, \_)\}$ commit in different order: in $\ET_1$, the write to $\ke_1$ must commit 
before the write to $\ke_2$, and vice versa for $\ET_2$. On the other hand, 
because the two fingerprints above do not write to the same key, 
the order in which they are committed should not be relevant: by changing the order 
in which different clients commit such fingerprints to a kv-store, the result stays the same. 

Requiring execution tests to be commutative is a necessary step for ensuring 
that the specification of consistency models are compositional. However, it 
is not sufficient. The next example shows how compositionality fails 
for commutative execution tests. 

\begin{example}
\label{ex:noblindwrites}
For any $n \in \Nat$, let $[n] = \{0,\cdots, n\}$.
Consider the execution tests $\ET_1, \ET_2$ defined below: 
\[
\begin{array}{lcl}
\ET_1 \vdash (\hh, \vi) \triangleright \opset : \vi' &\iff& 
\forall \ke.\;(\otW, \ke, \_) \in \opset \implies \vi(\ke) = \vi'(\ke) = [0]\\
\ET_2 \vdash (\hh, \vi) \triangleright \opset : \vi'(\ke) &\iff& 
\forall \ke. \;(\otW, \ke, \_ ) \in \opset \implies \vi(\ke) = [ \lvert \hh(\ke) \rvert - 1] \wedge \vi'(\ke) = [\lvert \hh(\ke) \rvert ] \\
\end{array}
\]
It is immediate to observe that both $\ET_1$ and $\ET_2$ are commutative. However, 
consider the kv-store $\hh_2 = [\ke \mapsto (\val_0, \txid_0, \emptyset) \lcat (\val_1, \txid_{\cl}^1, \emptyset) \lcat (\val_2, \txid_{\cl}^2, \emptyset)]$. 
We have that $\hh \in \CMs(\ET_1)$ and $\hh \in \CMs(\ET_2)$.
Let in fact $\hh_1 = [\ke \mapsto (\val_0, \txid_0, \emptyset) \lcat (\val_1, \txid_{\cl}^1, \emptyset)]$, $\vi_{i} = [\ke \mapsto [i] ]$.
%\ac{Seriously, square brackets are being used everywhere (though all of this is standard notation. Maybe $\langle n \rangle$ for $\{0,\cdots, n\}$ is 
%a better notation?
%\sx{  \( \ke \mapsto \langle n \rangle \) is cool. }
%}
We have the following sequences of reductions: 
\[
\begin{array}{l}
(\hh_0, \vi_0) \xrightarrowtriangle{(\cl, \{(\otW, \ke, \val_1)\})}_{\ET_1} 
(\hh_1, \vi_0) \xrightarrowtriangle{(\cl,\{(\otW, \ke, \val_2)\})}_{\ET_1} (\hh_2, \vi_0)\\
(\hh_0, \vi_0) \xrightarrow{(\cl, \{(\otW, \ke, \val_1)\})}_{\ET_2} (\hh_1, \vi_1) \xrightarrow{(\cl, \{(\otW, \ke, \val_2)\})}_{\ET_2} 
(\hh_2, \vi_2)
\end{array}
\]
On the other hand, we can observe that $\hh_2 \notin \CMs(\ET_1 \cap \ET_2)$. $\ET_1$ allows a client to 
commit a transaction if its view only includes the initial version of each key it writes. $\ET_2$ allows a client 
to commit a transaction when its view include all the versions for each key it writes. In $\ET_1 \cap \ET_2$ 
a client can commit a transaction only if the initial version of each key it writes is also the only version in the kv-store: 
as a result, $\CMs(\ET_1 \cap \ET_2)$ never contains a  kv-stores $\hh$ such that $\hh(\ke) > 1$ for some key $\ke$; 
in particular, $\hh_2 \notin \CMs(\ET_1 \cap \ET_2)$.
\end{example}
\ac{Two possible reasons why compositionality fails: because of blind writes, or because the test $\ET_1$ hinders progress, 
i.e. it is not possible to replace a view with a more up-to-date one to enable progress. We must choose which assumption 
we make on the consistency model.}

One reason why compositionality fails in \cref{ex:noblindwrites} is that the execution tests $\ET_1$ and $\ET_2$ do not contain 
any information about the views that client $\cl$ used to commit the transactions $\txid_{\cl}^1, \txid_{\cl}^2$. 
To solve this problem we adopt the \emph{no blind writes} assumption, that requires that a client never commits 
a transaction that writes a key, without reading such a key beforehand. Many implementations of consistency models 
in distributed key-value stores respect the no blind writes assumption. 

\subsection{Compositionality of \( \ET \)}
\label{sec:et-comm}
\label{sec:et-comp}

To make two execution tests \( \ET_1 \) \( \ET_2 \) compositional with respect to to function \( \CMs \),
they need to satisfy \cref{def:conflict-commit,def:noblidwrites,def:et-minimum-footprint,def:et-continuous-postview}.
For all the specification we have in \cref{fig:execution.tests},
It is easy to adapt so that they satisfy \cref{def:noblidwrites,def:et-minimum-footprint,def:et-continuous-postview},
but not all of them can be adapted so to satisfy \cref{def:conflict-commit}, for example \( \CP \).

\begin{definition}
\label{def:noblidwrites}
An execution test $\ET$ has \emph{no blind writes} if, whenever $\ET \vdash (\hh, \vi) \triangleright \opset \cup \{(\otW, \ke, \_)\} : \vi'$, 
then $(\otR, \ke, \_) \in \opset$.
\end{definition}

\begin{definition}
\label{def:et-minimum-footprint}
An execution test $\ET$ has \emph{minimum footprints} if for any key-value store \( \hh \)
views \( \vi, \vi',\vi''\) and fingerprint \( \f \),
\[
\begin{array}{@{}l@{}}
\ET \vdash (\hh, \vi) \triangleright \opset : \vi''     
\land \fora{ \ke} \left( (\stub, \ke, \stub) \in \f \implies \vi(\ke) = \vi'(\ke) \right) 
\implies \ET \vdash (\hh, \vi') \triangleright \opset : \vi''
\end{array}
\]
\end{definition}

\begin{definition}
\label{def:et-continuous-postview}
An execution test $\ET$ has \emph{continuous post-views} if for any key-value store \( \hh \)
views \( \vi, \vi',\vi''\) and fingerprint \( \f \), 
\[
\begin{array}{@{}l@{}}
    \quad \ET \vdash (\hh, \vi) \triangleright \opset : \vi' \land \vi' \sqsubseteq \vi'' \implies \ET \vdash (\hh, \vi) \triangleright \opset : \vi''
\end{array}
\]
\end{definition}

Now we can prove compositionality of \( \ET \) (\cref{thm:et-comm}).

\begin{theorem}                                                                            
\label{thm:et-comm}                          
Let $\ET_1, \ET_2$ be two execution tests has no blind writes, minimum footprints and continuous post-views.
If $\ET_1$ is commutative, 
then $\CMs(\ET_1 \cap \ET_2) = \CMs(\ET_1) \cap \CMs(\ET_2)$. 
Furthermore, if $\ET_1, \ET_2$ are commutative, then $\ET_1 \cap \ET_2$ 
is commutative.
\end{theorem}
\begin{proof}
Given the definition of the \( \CMs(.) \) function (\cref{def:cm}), 
it suffices to prove that \( \CMs(\ET_{1} \cap \ET{2}) \subseteq \CMs(\ET_1) \cap \CMs(\ET_2) \)
and \( \CMs(\ET_1) \cap \CMs(\ET_2) \subseteq \CMs(\ET_{1} \cap \ET{2}) \).
The former is proven by the \cref{lem:et12-in-et1-et2} and the later is proven by \cref{lem:et1-et2-in-et12}.
\end{proof}

\begin{lemma}
\label{lem:et12-in-et1-et2}
\( \CMs(\ET_{1} \cap \ET_{2}) \subseteq \CMs(\ET_1) \cap \CMs(\ET_2) \).
\end{lemma}
\begin{proof}
It suffices to prove a stronger result that \( \Confs(\ET_{1} \cap \ET_{2}) \subseteq \Confs(\ET_1) \cap \Confs(\ET_2) \).
By the definition of \Confs (\cref{def:cm}), it suffices to prove for configurations \( \conf_0 \) to \( \conf_n \) 
\begin{equation}
    \label{equ:et12-in-et1-et2}
    \begin{array}{@{}l}
    \conf_0 \text{ is initial } 
    \land \conf_0 \xrightarrowtriangle{\stub}_{\ET_1 \cap \ET_2} \cdots \xrightarrowtriangle{\stub}_{\ET_1 \cap \ET_2} \conf_n \implies {} \\
    \quad \conf_0 \xrightarrowtriangle{\stub}_{\ET_{1}} \cdots \xrightarrowtriangle{\stub}_{\ET_{1}} \conf_n \land \conf_0 \xrightarrowtriangle{\stub}_{\ET_{2}} \cdots \xrightarrowtriangle{\stub}_{\ET_{2}} \conf_n 
    \end{array}
\end{equation}
We prove the \cref{equ:et12-in-et1-et2} by induction on the number \( n \).
\begin{itemize}
\item Base case: \(n = 0\). 
The \cref{equ:et12-in-et1-et2} holds when \( n = 0 \), because all initial configurations \( \conf_0 \) are included in the \( \Confs(\ET_1)\) and \( \Confs(\ET_2) \) by the definition of the \( \Confs \) function (\cref{def:cm}).

\item Inductive case: \(n = i+1\). Suppose the \cref{equ:et12-in-et1-et2} holds when \( n = i \) for some \( i \).
Let consider \( n = i + 1 \) and specifically the last step.
For any \( \conf_{i+1} = (\mkvs_{i+1}, \viewFun_{i+1}) \) induced by \( \ET_{1} \cap \ET_2 \), 
there exist some client \( \cl \), views \( \vi, \vi' \) and fingerprint \( \f \) such that:
\[
    \begin{array}{l}
    (\mkvs_i, \viewFun_i) \xrightarrowtriangle{\cl, \opset}_{\ET_{1} \cap \ET_{2}} (\mkvs_{i+1}, \viewFun_{i+1}) 
    \land \viewFun_{i+1} = \viewFun_{i}\rmto{\cl}{\vi'} \land (\mkvs_i, \vi, \f, \vi' ) \in \ET_{1} \cap \ET_{2}
    \end{array}
\]
Thus, it is easy to see that \( \conf_i \xrightarrowtriangle{\cl, \opset}_{\ET_{1}} \conf_{i+1} \) and \( \conf_i \xrightarrowtriangle{\cl, \opset}_{\ET_{2}} \conf_{i+1} \) by the \cref{lem:mono-et}.
\end{itemize}
\end{proof}

\begin{lemma}
\label{lem:mono-et}
If $\conf \xrightarrowtriangle{\cl, \opset}_{\ET} \conf'$ and $\ET \subseteq \ET'$, 
then $\conf \xrightarrowtriangle{\cl, \opset}_{\ET'} \conf'$.
\end{lemma}
\begin{proof}
    Let \((\mkvs, \viewFun)  = \conf \), \( (\mkvs', \viewFun') = \conf' \) and \( \vi  =\viewFun(\cl) \)
    By the definition of  $\conf \xrightarrowtriangle{\cl, \opset}_{\ET} \conf'$ (\cref{def:cm}), we have \(\mkvs' \in \updKV{\mkvs, \vi, \cl, \f}\) and  \( \viewFun' = \viewFun\rmto{\cl}{\vi'} \) for some \( \vi' \) such that \( \ET \vdash (\mkvs, \vi) \csat \f : \vi' \).
    Given that \( \ET \subseteq \ET'\), we know \( \ET' \vdash (\mkvs, \vi) \csat \f : \vi' \) and so $\conf \xrightarrowtriangle{\cl, \opset}_{\ET'} \conf'$.
\end{proof}

\begin{lemma}
\label{lem:et1-et2-in-et12}
\( \CMs(\ET_1) \cap \CMs(\ET_2) \subseteq \CMs(\ET_{1} \cap \ET_{2}) \).
\end{lemma}
\begin{proof}
    By the definition of \( \CMs\) and \( \Confs\) (\cref{def:cm}), we prove a stronger result that
    for an initial configuration \( \conf_0 \), 
    configurations \( \conf_1 \) to \( \conf_n \) from trace \( \ET_1 \), 
    configurations \( \conf'_1 \) to \( \conf'_m \) from trace \( \ET_2 \),
    \[
    \begin{array}{@{}l}
    \conf_0 \xrightarrowtriangle{\stub}_{\ET_{1}} \conf_1 \xrightarrowtriangle{\stub}_{\ET_{1}} \cdots \xrightarrowtriangle{\stub}_{\ET_{1}} \conf_n 
    \land \conf_0 \xrightarrowtriangle{\stub}_{\ET_{2}} \conf'_1 \xrightarrowtriangle{\stub}_{\ET_{2}}  \cdots \xrightarrowtriangle{\stub}_{\ET_{2}} \conf'_m 
    \land \conf_n\projection{1} = \conf'_m\projection{1} \\
    \end{array}
    \]
    there exists configurations from \( \conf''_1\)  to \( \conf''_k \) from trace \( \ET_1 \cap \ET_2 \):
\begin{equation}
    \label{equ:et1-et2-in-et12}
    \begin{array}{@{}l}
    \conf_0 \xrightarrowtriangle{\stub}_{\ET_1 \cap \ET_2} \conf''_1 \xrightarrowtriangle{\stub}_{\ET_1 \cap \ET_2} \cdots \xrightarrowtriangle{\stub}_{\ET_1 \cap \ET_2} \conf''_k 
    \land \conf_n\projection{1} = \conf'_m\projection{1} = \conf''_k\projection{1}  \\
    \quad {} \land \fora{\cl \in \dom(\conf''_k\projection{2}),\ke \in (\conf''_k\projection{1})}
    \conf''_k\projection{2}(\cl)(\ke) = \max\Set{\conf_n\projection{2}(\cl)(\ke), \conf'_m\projection{2}(\cl)(\ke)}
    \end{array}
\end{equation}
We prove \cref{equ:et1-et2-in-et12} by induction on the length \( m \) of the trace of \( \ET_2 \).
\begin{itemize}
    \item \caseB{\(m = 0\)}
We have the trace of \( \ET_1 \):
\begin{equation}
    \label{equ:trace-view-shift-et1}
    \conf_0 \text{ is initial } \land \conf_0 \toET{\stub}{\ET_1} \dots \toET{\stub}{\ET_1} \conf_n
\end{equation}
for some number \( n \) and configurations from \( \conf_0 \) to \( \conf_n \) and the trace of \( \ET_2 \) with only one configuration:
\begin{equation}
    \label{equ:trace-singleton-et2}
    \conf_0
\end{equation}
By the hypothesis we have \( \conf_0\projection{1} = \conf_n\projection{1} \), which means that all the steps from the trace of \( \ET_1 \) are view shift.
We can pick the trace of \( \ET_1 \) (\cref{equ:trace-view-shift-et1}) as the trace of \( \ET_1 \cap \ET_2 \):
\begin{equation}
    \label{equ:trace-view-shift-et12}
    \conf_0 \toET{\stub}{\ET_1 \cap \ET_2} \dots \toET{\stub}{\ET_1 \cap \ET_2} \conf''_k \land  k = n \land \bigwedge_{ 0 < i \leq k} \conf_i = \conf''_i
\end{equation}
It is easy to see:
\begin{equation}
    \label{equ:max-et1-et2}
    \begin{array}{l}
    \fora{\cl \in \dom(\conf_k\projection{2}), \ke \in \dom(\conf_k\projection{1})} 
    \conf_0\projection{2}(\cl)(\ke) = \max\Set{\conf_0\projection{2}(\cl)(\ke), \conf_n\projection{2}(\cl)(\ke)}
\end{array}
\end{equation}
Combine \cref{equ:trace-view-shift-et12} and \cref{equ:max-et1-et2}, we prove the \cref{equ:et1-et2-in-et12}.

\item \caseI{\(m = i + 1\)}
Suppose that \cref{equ:et1-et2-in-et12} holds when \( m = i \).
Let consider \( m = i + 1 \).
We have the trace for \( \ET_1 \):
\begin{equation}
    \conf_0 \xrightarrowtriangle{\stub}_{\ET_{1}} \conf_1 \xrightarrowtriangle{\stub}_{\ET_{1}} \cdots \xrightarrowtriangle{\stub}_{\ET_{1}} \conf'_{n} 
\end{equation}
for some number \( n \) and the configurations from \(\conf_0\) to \( \conf_n \), and the trace of \(\ET_2\):
\begin{equation}
    \conf_0 \xrightarrowtriangle{\stub}_{\ET_{2}} \conf'_1 \xrightarrowtriangle{\stub}_{\ET_{2}} \cdots \xrightarrowtriangle{\stub}_{\ET_{2}} \conf'_{i+1} 
\end{equation}
It is safe to assume these two traces are in normal form by \cref{prop:et.normalform}.
Assume a client \( \cl'_{i} \), views \( \vi'_{i}, \vi'_{i+1} \) and a fingerprint \( \opset'_{i} \) that commit to the second last configuration \( (\mkvs'_i, \viewFun'_i) = \conf'_i \) in the trace of \( \ET_2 \) which yields the final configuration \( (\mkvs'_{i+1}, \viewFun'_{i+1}) = \conf_{i+1} \):
\begin{equation}
    \label{equ:last-et-2}
    \begin{array}{@{}l @{}}
    (\mkvs'_i, \viewFun'_i) \toET{\cl'_{i}, \opset'_{i}}{\ET_2} (\mkvs'_{i+1}, \viewFun'_{i+1}) \land \ET_2 \vdash (\mkvs'_i, \vi'_i) \csat \f'_i  : \vi'_{i+1} 
    \land \vi' = \viewFun'_i(\cl'_i) \land \viewFun'_{i+1} = \viewFun'_i\rmto{\cl'_i}{\vi'_{i+1}}
    \end{array}
\end{equation}
There are three cases: \textbf{(i)} \( \f'_i = \unitO \), \textbf{(i)} \( \f'_i = \epsilon \) and \textbf{(ii)} \( \f'_i \neq \unitO \land \f'_i \neq \epsilon \).
\begin{itemize}
    \item If \( \f'_i = \epsilon \) or \( \f'_i = \unitO \), by the \cref{lem:no-effect-for-empty-fingerprint} we know \( \conf'_{i}\projection{1} = \conf'_{i+1}\projection{1}\) from the trace of \( \ET_2 \).
Since \( \conf'_{i+1}\projection{1} = \conf_n\projection{1}\) where \( \conf_n \) is the final configuration of the trace of \( \ET_1 \), we now have \( \conf'_{i}\projection{1} = \conf_n\projection{1}\).
Applying \ih that \cref{equ:et1-et2-in-et12} holds when \( m = i \), so there exist configurations from \( \conf''_1 \) to \( \conf''_k \):
\begin{equation}
    \label{equ:ih-for-k-length}
    \begin{array}{@{}l@{}}
    \quad \conf_0 \xrightarrowtriangle{\stub}_{\ET_1 \cap \ET_2} \conf''_1 \xrightarrowtriangle{\stub}_{\ET_1 \cap \ET_2} \cdots \xrightarrowtriangle{\stub}_{\ET_1 \cap \ET_2} \conf''_k
    \land \conf_n\projection{1} = \conf'_i\projection{1} = \conf''_k\projection{1} \\
    \quad {} \land \fora{\cl \in \dom(\conf''_k\projection{2}),\ke \in (\conf''_k\projection{1})} 
    \conf''_k\projection{2}(\cl)(\ke) = \max\Set{\conf_n\projection{2}(\cl)(\ke), \conf'_i\projection{2}(\cl)(\ke)}
\end{array}
\end{equation}
Given the definition of the reduction (\cref{def:reduction}), when \( \f = \epsilon \) or \( \f = \unitO \) we know \( \conf'_i\projection{2}(\cl_{i+1}) \sqsubseteq  \conf'_{i+1}\projection{2}(\cl_{i+1})\) thus:
\begin{equation}
    \label{equ:preserve-max-view}
    \begin{array}{l}
    \max\Set{\conf_n\projection{2}(\cl_{i+1}), \conf'_i\projection{2}(\cl_{i+1})} 
    \sqsubseteq \max\Set{\conf_n\projection{2}(\cl_{i+1}), \conf'_{i+1}\projection{2}(\cl_{i+1})} 
    \end{array}
\end{equation}
Therefore \cref{equ:et1-et2-in-et12} holds when \( m = i + 1\) by appending a view shift to the end of the trace in \cref{equ:ih-for-k-length}:
\[
    \begin{array}{@{}l}
    \conf_0 \xrightarrowtriangle{\stub}_{\ET_1 \cap \ET_2} \conf''_1 \xrightarrowtriangle{\stub}_{\ET_1 \cap \ET_2} \cdots
    \xrightarrowtriangle{\stub}_{\ET_1 \cap \ET_2} \conf''_k \toET{\cl_{j}, \epsilon }{\ET_1 \cap \ET_2} \\
    \qquad {} \land \conf''_k\rmto{2}{\conf''_k\projection{2}\rmto{\cl_{i}}{\max\Set{\conf_n\projection{2}(\cl_{i+1}), \conf'_{i+1}\projection{2}(\cl_{i+1})} }}
    \end{array}
\]

    \item If \( \f'_i \neq \unitO  \land \f' \neq \epsilon \), by \cref{lem:identical-step} there exists a step \( (\cl_j, \f_j) \) from the trace of \( \ET_1 \) such that:
\begin{equation}
    \label{equ:j-th-step}
    \begin{array}{l}
    (\mkvs_{j}, \viewFun_{j}) \xrightarrowtriangle{\cl_{j}, \opset_{j}}_{\ET_1} (\mkvs_{j + 1}, \viewFun_{j + 1}) 
    \land \ET \vdash (\mkvs_{j}, \vi_j) \csat \f_j : \viewFun_{j + 1}(\cl_{j}) \land \vi_j = \viewFun_{j}(\cl_j)
\end{array}
\end{equation}
for some \( j, \cl_j, \vi_j\) and \( \f_j \) such that \( 0 \leq  j < n \), \( \cl_j = \cl'_{i}\), \( \f_j = \f'_{i}\), and
\[ 
    \fora{\ke} (\stub, \ke, \stub ) \in \f_j \implies \vi_j(\ke) = \vi'_{i}(\ke)
\]
We apply the commutativity of \( \ET_1 \) until the step shown in \cref{equ:j-th-step} is at the end or the second end of the trace of \( \ET_1 \).
Let consider the next two steps, (j+1)-\emph{th} and (j+2)-\emph{th} step.
Since the trace is in normal form, the (j+1)-\emph{th} step is a view shift by a client \( \cl_{j+2} \) and (j+2)-\emph{th} step is a concrete step issued by the same client \( \cl_{j+2} \) under the view \( \vi_{j+2} \):
\begin{equation}
    \label{equ:j-plus-1-th-step}
    \begin{array}{@{}l@{}}
        (\mkvs_{j+1}, \viewFun_{j+1}) \xrightarrowtriangle{\cl_{j+2}, \epsilon}_{\ET_1}
        (\mkvs_{j+1}, \viewFun_{j+1}\rmto{\cl_{j+2}}{\vi_{j+2}}) \toET{\cl_{j+2}, \f_{j+2}}{\ET_1} (\mkvs_{j+3}, \viewFun_{j+3}) \\
        \qquad \land \ET \vdash (\mkvs_{j+1},\vi_{j+2}) \csat \f_{j+2} : \viewFun_{j+2}(\cl_{j+2}) 
    \end{array}
\end{equation}
It is known that the client  \( \cl_{j+2} \) is different from \( \cl_j \) (\cref{lem:different-cl}) and \( \f_{j+2} \) writes different keys from \( \f_j\) (\cref{lem:different-writes}). 
Because \( \cl_j \neq \cl_{j+2} \) we can swap the view shift step shown in \cref{equ:j-plus-1-th-step} before the j-\emph{th} step shown in \cref{equ:j-th-step} which gives the following:
\begin{equation}
    \label{equ:swap-the-view-shift-et1}
    \begin{array}{@{}l@{}}
    (\mkvs_{j}, \viewFun_{j}) \toET{\cl_{j+2}, \epsilon}{\ET_1} (\mkvs_{j}, \viewFun_{j}\rmto{\cl_{j+2}}{\vi_{j+2}}) \toET{\cl_{j}, \opset_{j}}{\ET_1} \\
    \quad (\mkvs_{j + 1}, \viewFun_{j + 1}\rmto{\cl_{j+2}}{\vi_{j+2}}) \toET{\cl_{j+2}, \f_{j+2}}{\ET_1} (\mkvs_{j+3}, \viewFun_{j+3})
    \end{array}
\end{equation}
Now let discuss the (j+2)-\emph{th} step.
Similarly by the \cref{lem:identical-step}, there is a step \((\cl_p, \f_p)\) from the trace of \( \ET_2 \) such that \( \cl_p = \cl_{j+2}\) and \( \f_p = \f_{j+2}\) and \( p < i \).
Note that the last step from \( \ET_2 \), \ie (i+1)-\emph{th} step, is not a view shift therefore the i-\emph{th} step must be a view shift so the p-\emph{th} step must be before  i-\emph{th} step.
This means the fingerprint \( \f_p \) does not observe any change by (i+1)-\emph{th} step from the trace of \( \ET_2 \).
Therefore \( \vi_{j+2} \) does not observe any change by j-\emph{th} step from the trance of \( \ET_1\), \ie \( \vi_{j+2} \in \Views(\mkvs_j) \).
By \cref{prop:swap-update}, that allows to swap the two adjacent non-conflict steps from \cref{equ:swap-the-view-shift-et1}, \ie the last two steps.
It follows a new kv-stores \( \mkvs'''_{j+2}\) and a new view environment \( \viewFun'''_{j+2} \) such that:
\begin{equation}
    \label{equ:swap-step-et1}
    \begin{array}{@{}l@{}}
    (\mkvs_{j}, \viewFun_{j}) \toET{\cl_{j+2}, \epsilon}{\ET_1} (\mkvs_{j}, \viewFun_{j}\rmto{\cl_{j+2}}{\vi_{j+2}}) \toET{\cl_{j+2}, \opset_{j+2}}{\ET_1} \\
    \quad (\mkvs_{j + 2}''', \viewFun_{j + 2}''') \toET{\cl_{j+2}, \f_{j+2}}{\ET_1} (\mkvs_{j+3}, \viewFun_{j+3})
    \end{array}
\end{equation}
In the \cref{equ:swap-step-et1} the j-\emph{th} step moves to the right of (j+2)-\emph{th} step.
We continuously move the j-\emph{th} step until it is at the end or the second end of trace of \( \ET_1 \):
\[
    \begin{array}{@{}l}
        \conf_0 \xrightarrowtriangle{\stub}_{\ET_{1}} \cdots \xrightarrowtriangle{\stub}_{\ET_{1}} \conf_{j-1} \toET{\stub}{\ET_{1}} 
        \conf'''_{j} \toET{\stub}{\ET_{1}} \dots \toET{\stub}{\ET_{1}} \conf'''_{n-1} \toET{\cl_j, \f_j }{\ET_{1}} \conf_{n} \lor {} \\
        \conf_0 \xrightarrowtriangle{\stub}_{\ET_{1}} \cdots \xrightarrowtriangle{\stub}_{\ET_{1}} \conf_{j-1} \toET{\stub}{\ET_{1}} 
        \conf'''_{j} \toET{\stub}{\ET_{1}} \dots \toET{\stub}{\ET_{1}} \conf'''_{n-2} \toET{\cl_j, \f_j }{\ET_{1}} \conf'''_{n-1} \toET{\cl_{n-1}, \epsilon }{\ET_{1}} \conf_{n}  \\ 
    \end{array}
\]
for some new configurations from \( \conf'''_{j}\) to \( \conf'''_{n-1} \).
Note that if it is the second end, the last step must be a view shift step as shown in \cref{equ:new-et-1}.
\begin{itemize}
    \item If the j-\emph{th} step is at the end of the new trace of \( \ET_1 \), we have the trace:
\begin{equation}
    \label{equ:new-et-1}
    \begin{array}{@{}l}
        \conf_0 \xrightarrowtriangle{\stub}_{\ET_{1}} \cdots \xrightarrowtriangle{\stub}_{\ET_{1}} \conf_{j-1} \toET{\stub}{\ET_{1}} 
        \conf'''_{j} \toET{\stub}{\ET_{1}} \dots \toET{\stub}{\ET_{1}} \conf'''_{n-1} \toET{\cl_j, \f_j }{\ET_{1}} \conf_{n}  \\
    \end{array}
\end{equation}
Given the hypothesis that \( \conf_{n}\projection{1} = \conf'_{i+1}\projection{1} \) and the fact that the last step of the new trace of \( \ET_1 \) (\cref{equ:new-et-1}) and the last step the trace of \( \ET_2 \) (\cref{equ:last-et-2}) are the same step, the kv-stores of the second last configurations the new trace of \( \ET_1 \) (\cref{equ:new-et-1}) and the one from the trace of \( \ET_2 \) (\cref{equ:last-et-2}) are the same \(  \conf'''_{n-1}\projection{1} = \conf'_{i}\projection{1} \).
Then by applying \ih that \cref{equ:et1-et2-in-et12} holds when \( m = i \), there exists a trace of \( \ET_1 \cap \ET_2 \):
\begin{equation}
    \label{equ:ih-for-merge-two-trace}
    \begin{array}{@{}l}
        \conf_0 \toET{\stub}{\ET_1 \cap \ET_2} \dots \toET{\stub}{\ET_1 \cap \ET_2} \conf''_{k-1} 
        \land \conf'''_{n-1}\projection{1} = \conf'_{i}\projection{1} = \conf''_{k-1}\projection{1}  \\
        \quad {} \land \fora{\cl \in \dom(\conf''_{k-1}\projection{2}),\ke \in (\conf''_{k-1}\projection{1})} 
        \conf''_{k-1}\projection{2}(\cl)(\ke) = \max\Set{\conf_n\projection{2}(\cl)(\ke), \conf'_i\projection{2}(\cl)(\ke)}
\end{array}
\end{equation}
for some number \( k \) and configurations from \( \conf''_1 \) to \( \conf''_{k-1} \).
By \cref{equ:last-et-2} and \cref{equ:new-et-1}, we have:
\begin{equation}
    \label{equ:et1-et2-csat}
    \begin{array}{@{} l@{}}
    \ET_1 \vdash ( \conf'_{i}\projection{1}, \conf'_{i}\projection{2}(\cl_{i}) )  \csat \f_{i}, \conf'_{i+1}\projection{2}(\cl_{i})  
    \land \ET_2 \vdash ( \conf'''_{n-1}\projection{1}, \conf'''_{n-1}\projection{2}(\cl_{i}) )  \csat \f_{i}, \conf_{n}\projection{2}(\cl_{i})
    \end{array}
\end{equation}
First, for any quadraple in \( \ET_1 \) and \( \ET_2 \), it does not constrain the view for keys that are not appear in the fingerprint before update.
That is:
\[
    \begin{array}{@{}l@{}}
    \for{ \mkvs, \vi, \vi', \vi'', \f, \ke } 
    (\stub, \ke, \stub) \in \f \land \vi(\ke) = \vi'(\ke) \land (\mkvs, \vi, \f, \vi'') \in \ET 
    \implies (\mkvs, \vi', \f, \vi'') \in \ET
    \end{array}
\]
Given above and \cref{equ:ih-for-merge-two-trace}, we can substitute the configurations \( \conf'_{i} \) and  \( \conf'''_{n-1} \) from \cref{equ:et1-et2-csat} by \( \conf''_{k-1}\).
Then,  because for any \( \mkvs, \vi, \vi', \vi'' \) and \( \f \), if \( (\mkvs, \vi, \f, \vi' ) \in \ET_1 \) and \( \vi' \sqsubseteq \vi'' \) then \( (\mkvs, \vi, \f, \vi'' ) \in \ET_1 \), and similarly for \( \ET_2 \).
It means:
\begin{equation}
    \label{equ:et12-csat}
    \begin{array}{l}
    \ET_1 \cap \ET_2 \vdash ( \conf''_{k-1}\projection{1}, \conf''_{k-1}\projection{2}(\cl_{i}) ) \csat
    \f_{i}, \max\Set{\conf'_{i+1}\projection{2}(\cl_{i}), \conf_{n}\projection{2}(\cl_{i})}
    \end{array}
\end{equation}
Therefore the \cref{equ:et1-et2-in-et12} holds when \( m = i + 1\) by appending the shown in \cref{equ:et12-csat} to the end of the trace shown in \cref{equ:ih-for-merge-two-trace}:
\[
\begin{array}{@{}l}
    \conf_0 \toET{\stub}{\ET_1 \cap \ET_2} \dots \toET{\stub}{\ET_1 \cap \ET_2} \conf''_{k-1} \toET{cl_{i}, \f_{i}}{\ET_1 \cap \ET_2} \\
    \quad \left( \conf_n\projection{1},\conf''_{k-1}\projection{2}\rmto{\cl_{i}}{\max\Set{\conf'_{i+1}\projection{2}(\cl_{i}), \conf_{n}\projection{2}(\cl_{i})} } \right)
\end{array}
\]
    \item If the j-\emph{th} step is the second last step of the new trace of \( \ET_1 \), we have the trace:
\begin{equation}
    \label{equ:new-et-1-with-view-shift-tail}
    \begin{array}{@{}l}
        \conf_0 \xrightarrowtriangle{\stub}_{\ET_{1}} \cdots \xrightarrowtriangle{\stub}_{\ET_{1}} \conf_{j-1} \toET{\stub}{\ET_{1}} 
        \conf'''_{j} \toET{\stub}{\ET_{1}} \dots \toET{\stub}{\ET_{1}} \conf'''_{n-2} \toET{\cl_j, \f_j }{\ET_{1}} 
        \conf'''_{n-1} \toET{\cl_{n-1}, \epsilon }{\ET_{1}} \conf_{n}  \\ 
    \end{array}
\end{equation}
Since the last step is a view shift, we know \( \conf_n\projection{1} = \conf'''_{n-1}\projection{1}\), and the rest of proof is the same as the case where j-\emph{th} is the last step as shown in \cref{equ:new-et-1}.
\end{itemize}
\end{itemize}
\end{itemize}
\end{proof}

\begin{lemma}[No effect from empty fingerprint and epsilon reduction]
    \label{lem:no-effect-for-empty-fingerprint}
    \label{lem:no-effect-for-view-shift}
    \[
    \fora{\conf, \conf', \cl,\vi} \conf \toET{\cl, \unitO}{\ET} \conf' \lor \conf \toET{\cl, \epsilon}{\ET} \conf' \implies \conf\projection{1} = \conf'\projection{1}
    \]
\end{lemma}
\begin{proof}
    Let \((\mkvs, \viewFun)  = \conf \) and \( (\mkvs', \viewFun') = \conf' \).
    For the case of empty fingerprint,
    by the definition of  $\conf \xrightarrowtriangle{\cl, \unitO}_{\ET} \conf'$ (\cref{def:reduction}), we have \(\mkvs' \in \updKV{\mkvs, \vi, \cl, \unitO}\), and therefore \( \mkvs' = \mkvs \).
    For the case of view shift, by the definition of  $\conf \xrightarrowtriangle{\cl, \epsilon}_{\ET} \conf'$ (\cref{def:reduction}) it is easy to see \( \mkvs' = \mkvs \).
\end{proof}

We define a \(  \mkvs(\txid) \) function that returns the fingerprint associate with the transaction identifier \( \txid \):
\[
    \begin{rclarray}
        \mkvs(\txid) & \defeq & \Setcon{(\otW, \ke, \val)}{\exsts{i} \mkvs(\ke)(i) = (\val, \txid, \stub)} \cup  \Setcon{(\otR, \ke, \val)}{\exsts{i,\txidset} \mkvs(\ke)(i) = (\val, \stub, \txidset) \land \txid \in \txidset}
    \end{rclarray}
\]

\begin{lemma}[Transactions persistence]
    \label{lem:mono-fingerprint}
    \[
        \fora{\ET,\conf,\conf',\txid,\f} \conf\projection{1}(\txid) = \f \land \conf \toET{\stub}{\ET} \conf' \implies \conf'\projection{1}(\txid) = \f
    \]
\end{lemma}
\begin{proof}
    It is easy to prove this by case analysis on the reduction relation.
\end{proof}

\begin{lemma}[Same steps]
\label{lem:identical-step}
Given a trace of \( \ET_1 \) and a trace of \( \ET_2 \),
if the have the same final key-value store,
the trace contains the same concrete steps (free variables are globally quantified):
\[
\begin{array}{@{}l}
    \conf_0 \xrightarrowtriangle{\cl_1, \f_1}_{\ET_{1}} \cdots \xrightarrowtriangle{\cl_n, \f_n}_{\ET_{1}} \conf_n \land
    \conf_0 \xrightarrowtriangle{\cl'_1, \f'_1}_{\ET_{2}} \cdots \xrightarrowtriangle{\cl'_m, \f'_m}_{\ET_{2}} \conf'_m 
    \land \conf_n\projection{1} = \conf'_m\projection{1} \\
    \quad \implies \fora{i: 0 < i \leq n} 
    \f_i = \unitO 
    \lor \f_i = \epsilon 
    \lor \exsts{j: 0 < j \leq m} 
    \cl_i = \cl'_j \land \f_i = \f'_j \land ( \fora{\ke} (\stub, \ke, \stub) \in \f_i \implies \vi_i(\ke) = \vi'_j(\ke) )
\end{array}
\]
\end{lemma} 
\begin{proof}
    We prove by contradiction.
    First because \( \mkvs_n = \mkvs'_m \), we know that:
    \begin{equation}
        \label{equ:same-kv-store}
        \fora{\txid, \f} \mkvs_n(\txid) = \f \iff \mkvs'_m(\txid) = \f
    \end{equation}
    Let \(\conf_n = (\mkvs_n,\viewFun_n) \) and \(\conf'_m = (\mkvs'_m,\viewFun'_m) \).
    Assume a step \( \conf_i \toET{\cl, \f }{\ET_1} \conf_{i+1} \)  from the trace of \( \ET_1 \) where the transaction identifier is \( \txid \) and \( \f \neq \unitO \).
    It must have a step from the trace of \( \ET_2 \), which commits some fingerprint via the same transaction identifier  \( \txid \).
    We know \( \mkvs_n(\txid) = \f \) by \cref{lem:mono-fingerprint}, thus \( \mkvs'_m(\txid) = \f \) by \cref{equ:same-kv-store}.
    Let assume a key \( \ke \) that \( (\stub, \ke, \stub) \in \f \land \vi_i(\ke) \neq \vi'_j(\ke)\) where \( \vi_i\) and \( \vi'_j\) are the views immediate before the commit of the fingerprint \( \f \) in traces of \( \ET_1\) and \( \ET_2 \) respectively.
    Since the no blind write assumption, it is safe to assume it is a read operation on the key \( \ke \).
    By the definition of the reduction (\cref{def:reduction}) and \cref{lem:mono-fingerprint}, we know \( \func{read}{\mkvs_n(\ke)(\vi_i(\ke))} \neq \func{read}{\mkvs'_m(\ke)(\vi_i(\ke))} \), which contradicts with \( \mkvs_n = \mkvs'_m \).
\end{proof}

We define \( \max_\cl(\conf) \) function that returns the most recent transaction identifier for client \( \cl \) in the configuration \( \conf \) 
\[
\begin{rclarray}
    \max_\cl((\mkvs, \viewFun)) & \defeq & \max\Setcon{\txid^{n}_\cl}{\txid^{n}_\cl \text{ appear in } \mkvs} \\
\end{rclarray}
\]

\begin{lemma}[Transactions from different clients]
\label{lem:different-cl}
Given a trace of \( \ET_1 \) and a trace of \( \ET_2 \),
if the i-\emph{th} step from \( \ET_1 \) issued by the client \( \cl_i \) 
is the same as the last step from \( \ET_2 \),
then in the trace of \( \ET_1 \) 
there is no concrete step issued by the client \(\cl_i \) after the i-\emph{th} step (free variables are globally quantified):
\[
\begin{array}{@{}l}
    \conf_0 \xrightarrowtriangle{\cl_1, \f_1}_{\ET_{1}} \cdots \xrightarrowtriangle{\cl_n, \f_n}_{\ET_{1}} \conf_n 
    \land \conf_0 \xrightarrowtriangle{\cl'_1, \f'_1}_{\ET_{2}} \cdots \xrightarrowtriangle{\cl'_m, \f'_m}_{\ET_{2}} \conf'_m 
    \land \conf_n\projection{1} = \conf'_m\projection{1} 
    \land \f'_m \neq \unitO \\
    \quad {} \land \exsts{i}  
    \cl_i = \cl'_m
    \land \f_i = \f'_m 
    \implies \fora{j > i} 
    \f_j = \epsilon \lor \f_j = \unitO \lor \cl_i \neq \cl_j
\end{array}
\]
\end{lemma}
\begin{proof}
    We prove by deriving contradiction.
    Assume the last step of the trace of \( \ET_2 \) is:
    \begin{equation}
        \label{equ:last-step-for-cl-et2}
        \conf'_{m-1} \xrightarrowtriangle{\cl'_m, \f'_m}_{\ET_{2}} \conf'_m
    \end{equation}
    Assume a step of the trace of \( \ET_1 \):
    \begin{equation}
        \label{equ:identical-step-for-cl-et1}
        \conf_{i-1} \xrightarrowtriangle{\cl_i, \f_i}_{\ET_{1}} \conf_i
    \end{equation}
    where \( \cl_i = \cl'_m \) and \( \f_i = \f'_m \).
    Because these two steps (\cref{equ:last-step-for-cl-et2} and \cref{equ:identical-step-for-cl-et1}) are issued by the same transaction identifier,
    we know \( \max{}_{\cl_m}(\conf'_m) = \max{}_{\cl_m}(\conf_i) \).
    Assume that there exists a step from the trace of \( \ET_1 \), says j-\emph{th} step, such that:
    \[
        \conf_{j-1} \xrightarrowtriangle{\cl_j, \f_h}_{\ET_{1}} \conf_j \land j > i \land \f_j \neq \unitO \land \cl_i = \cl_j 
    \]
    Therefore we have \( \max{}_{\cl_m}(\conf_j) > \max{}_{\cl_m}(\conf_i) \) by \cref{lem:kv-max-cl}.
    That means \( \max{}_{\cl_m}(\conf_n) > \max{}_{\cl_m}(\conf_j) > \max{}_{\cl_m}(\conf_i) = \max{}_{\cl_m}(\conf'_m) \), which contradicts to \( \conf_n\projection{1} = \conf'_m\projection{1}\).
\end{proof}

\begin{lemma}[Reduction following session order]
\label{lem:kv-max-cl}
\[
\begin{array}{@{}l}
    \fora{\conf, \conf' ,\cl, \f, \ET}
    \conf \xrightarrowtriangle{\cl, \f}_{\ET}  \conf' 
    \land 
    \left( 
        \begin{array}{l}
        \f \neq \unitO \implies \max{}_\cl(\conf) < \max{}_\cl(\conf') )
        \lor ( \f = \unitO \implies \max{}_\cl(\conf) = \max{}_\cl(\conf')
        \end{array}
    \right)
\end{array}
\]
\end{lemma}
\begin{proof}
    Assume a step \( (\mkvs, \viewFun) \xrightarrowtriangle{\cl, \f}_{\ET} (\mkvs', \viewFun') \).
    By the definition of \( \toET{\stub}{\ET}\) (\cref{def:reduction}), we know \( \mkvs' \in \updKV{\mkvs, \vi, \cl, \f} \).
    The \( \updKV{\mkvs, \vi, \cl, \f} \) picks a fresh transaction identifier \( \txid_\cl^{m} \) that is greater than any transaction identifiers \( \txid_\cl^{n} \) in \( \mkvs \) via \( \nextTxId \) function, \ie \( m > n \).
    If the fingerprint \( \f \) is not empty, the new identifier appears in \( \mkvs' \), so \( \max{}_\cl(\conf) < \max{}_\cl(\conf') \).
    Otherwise  the fingerprint is empty, the new identifier will not appear anywhere in \( \mkvs' \), so \( \max{}_\cl(\conf) = \max{}_\cl(\conf') \). 
\end{proof}

\begin{lemma}[Writing different keys]
\label{lem:different-writes}
Given a trace of \( \ET_1 \) and a trace of \( \ET_2 \),
if the i-\emph{th} step from \( \ET_1 \) that writes to key \( \ke \) 
is the same as the last step from \( \ET_2 \),
then in the trace of \( \ET_1 \) 
there is no concrete step writing to the key \(\ke\) after the i-\emph{th} step (free variables are globally quantified):
\[
\begin{array}{@{}l}
    \conf_0 \xrightarrowtriangle{\cl_1, \f_1}_{\ET_{1}} \cdots \xrightarrowtriangle{\cl_n, \f_n}_{\ET_{1}} \conf_n \land \conf_0 \xrightarrowtriangle{\cl'_1, \f'_1}_{\ET_{2}} \cdots \xrightarrowtriangle{\cl'_m, \f'_m}_{\ET_{2}} \conf'_m 
    \land \conf_n\projection{1} = \conf'_m\projection{1} 
    \land \f'_m \neq \unitO \\
    \quad {} \land \exsts{i} 
    \cl_i = \cl'_m
    \land \f_i = \f'_m
    \implies \fora{j > i} \nexists{\ke} \ldotp (\otW, \ke, \stub) \in \f_j \cap \f_i
\end{array}
\]
\end{lemma}
\begin{proof}
    We prove this by deriving contradiction.
    Assume the last step from the trace of \( \ET_2 \):
    \begin{equation}
        \label{equ:last-step-for-write-et2}
        \conf'_{m-1} \xrightarrowtriangle{\cl'_m, \f'_m}_{\ET_{2}} (\mkvs'_m, \viewFun'_m )
    \end{equation}
    Assume the transaction identifier for the \cref{equ:last-step-for-write-et2} is \( \txid \), and by the definition of \( \toET{}{\ET}\) (\cref{def:reduction}) we know:
    \begin{equation}
        \label{equ:write-fingerprint}
        \fora{\ke} (\otW, \ke, \stub) \in \f'_m \implies \mkvs'_m(\ke)(\lvert\mkvs'_m(\ke)\rvert - 1) = (\stub, \txid, \stub)
    \end{equation}
    Assume a step of the trace of \( \ET_1 \) that is issued by the same transaction identifier with the same fingerprint:
    \begin{equation}
        \label{equ:identical-step-for-write-et1}
        \conf_{i-1} \xrightarrowtriangle{\cl_i, \f_i}_{\ET_{1}} (\mkvs_i, \viewFun_i)
    \end{equation}
    where \( \cl_i = \cl'_m \) and \( \f_i = \f'_m \).
    Given \cref{equ:write-fingerprint} and \cref{equ:identical-step-for-write-et1}, it follows:
    \[
        \fora{\ke} (\otW, \ke, \stub) \in \f_i \implies \mkvs_i(\ke)(\lvert\mkvs_i(\ke)\rvert - 1) = (\stub, \txid, \stub)
    \]
    Assume a step, says j-\emph{th}, after i-\emph{th} step that writes to the same key:
    \[
        \conf_{j-1} \xrightarrowtriangle{\cl_j, \f_j}_{\ET_{1}} (\mkvs_j, \viewFun_j) 
        \land j > i
        \land \exsts{\ke} (\otW, \ke, \stub) \in \f_i \cap \f_j
    \]
    Therefore, by \cref{lem:unique-writer} we have:
    \[
        \exsts{\ke,i} (\otW, \ke, \stub) \in \f_i \cap \f_j \land \mkvs_j(\ke)(i) = \txid \land \mkvs_j(\ke)(\lvert \mkvs_j(\ke) \rvert - 1) \neq \txid
    \]
    Note that \( \f_i = \f'_m\).
    Since the writer of a version cannot be overwritten, for the final configuration of the trace of \( \ET_1 \) \((\mkvs_n, \viewFun_n)\), we know:
    \[
        \exsts{\ke,i} (\otW, \ke, \stub) \in \f_i \cap \f_j \land \mkvs_n(\ke)(i) = \txid \land \mkvs_n(\ke)(\lvert \mkvs_n(\ke) \rvert - 1) \neq \txid
    \]
    Last, by \cref{equ:write-fingerprint} and \( \f_i = \f'_m\), it follows:
    \[
        \exsts{\ke} (\otW, \ke, \stub) \in \f'_m \land \mkvs_n(\ke)(\mkvs_n(\ke) - 1) \neq \txid \land \mkvs'_m(\ke)(\lvert\mkvs'_m(\ke)\rvert - 1)\projection{2} = \txid
    \]
    which contradicts with \( \mkvs'_m = \mkvs_n\).
\end{proof}

\begin{lemma}[Version persistence]
    \label{lem:unique-writer}
    \[
    \begin{array}{@{}l}
        \fora{\mkvs, \mkvs',\viewFun,\viewFun', \cl, \vi, \f, i} 
        (\mkvs, \viewFun) \toET{\cl, \f}{\ET} (\mkvs', \viewFun')
        \land (\otW, \ke, \stub) \in \f  
        \land 0 \leq i < \lvert \mkvs'(\ke) \rvert - 1 \\
        \quad {} \implies \mkvs'(\ke)(i)\projection{2} \neq \mkvs'(\ke)(\lvert \mkvs'(\ke) \rvert - 1)\projection{2}
    \end{array}
    \]
\end{lemma}
\begin{proof}
    By the definition of \( \toET{\stub}{\ET} \) (\cref{def:reduction}), the \( \mkvs' \in \updKV{\mkvs, \vi, \cl, \f} \).
    Given the definition of \( \updKV{\mkvs, \vi, \cl, \f}\), it the picks a fresh transaction identifier \( \txid \) such that does not appear in \( \mkvs \).
    For any write fingerprint \( (\otW, \ke, \stub) \in \f \), a new version is appended to the end of the key \( \ke \) and the writer (the second projection) is assigned to be the fresh identifier \( \txid \).
    Thus we have the proof.
\end{proof}

\begin{proposition}
\label{thm:appendix-et-composition-2}
\label{prop:appendix-et-composition-2}
if $\ET_1, \ET_2$ are commutative, then $\ET_1 \cap \ET_2$ is commutative.
\end{proposition}
\begin{proof}
Let \( \ET_{12} = \ET_1 \cap \ET_2 \).
Assume \(\conf_1, \conf_2, \conf_3, \cl, \cl', \vi, \vi', \opset, \opset' \) such that:
\[
    \conf_1 \xrightarrowtriangle{\cl, \vi, \opset}_{\ET_{12}} \conf_2 \xrightarrowtriangle{\cl', \vi', \opset'}_{\ET_{12}} \conf_3
\]
Therefore, we have:
\[
    \conf_1 \xrightarrowtriangle{\cl, \vi, \opset}_{\ET_{1}} \conf_2 \xrightarrowtriangle{\cl', \vi', \opset'}_{\ET_{1}} \conf_3 \land 
    \conf_1 \xrightarrowtriangle{\cl, \vi, \opset}_{\ET_{2}} \conf_2 \xrightarrowtriangle{\cl', \vi', \opset'}_{\ET_{2}} \conf_3
\]
Because \( ET_1 \)  and \( \ET_2 \) are commutative, there exists a configuration \( \conf'_2 \) such that:
\[
    \conf_1 \xrightarrowtriangle{\cl', \vi', \opset'}_{\ET_{1}} \conf'_2 \xrightarrowtriangle{\cl, \vi, \opset}_{\ET_{1}} \conf_3 \land 
    \conf_1 \xrightarrowtriangle{\cl', \vi', \opset'}_{\ET_{2}} \conf'_2 \xrightarrowtriangle{\cl, \vi, \opset}_{\ET_{2}} \conf_3
\]
so we have the proof that: 
\[
    \conf_1 \xrightarrowtriangle{\cl', \vi', \opset'}_{\ET_{12}} \conf'_2 \xrightarrowtriangle{\cl, \vi, \opset}_{\ET_{12}} \conf_3
\]
\end{proof}


\subsection{Example: \( \CP \) and \( \SI \) is not commutative}
\label{sec:comm-counter-cp-si}
We should the following counter example why \( \CP \) and \( \SI \) are not commutative.
Let consider an initial kv-store \( \mkvs \) (some writers are omitted):
\begin{centertikz}
%Location x
\node(locx) {$\key_1 \mapsto$};
\draw pic at ([xshift=\tikzkvspace]locx.east) {vlist={versionx}{%
    /$0$/$\stub$/$\emptyset$
    , /$1$/$\txid$/$\emptyset$
}};

%Location y
\path (versionx.east) + (1,0) node (locy) {$\key_2 \mapsto$};
\draw pic at ([xshift=\tikzkvspace]locy.east) {vlist={versiony}{%
    /$2$/$\stub$/$\emptyset$
    , /$3$/$\stub$/$\emptyset$
}};
\end{centertikz}
Assume two clients, \( \cl_1 \) and \( \cl_2 \), want to read the two keys \( \key_1 \) and \( \key_2 \).
Let assume the first client \( cl_1 \) initially has view 
\( \vi_1 = \Set{\key_1 \mapsto \Set{0}, \key_2 \mapsto \Set{0}} \)
With the view, we have the fingerprint for the client \( \fp_1 = \Set{(\otR, \key_1,0), (\otR, \key_2,2)} \),
which leads to the final kv-store \( \mkvs_1 \):
\begin{centertikz}
%Location x
\node(locx) {$\key_1 \mapsto$};
\draw pic at ([xshift=\tikzkvspace]locx.east) {vlist={versionx}{%
    /$0$/$\stub$/$\Set{\txid_1}$
    , /$1$/$\txid$/$\emptyset$
}};

%Location y
\path (versionx.east) + (1,0) node (locy) {$\key_2 \mapsto$};
\draw pic at ([xshift=\tikzkvspace]locy.east) {vlist={versiony}{%
    /$2$/$\stub$/$\Set{\txid_1}$
    , /$3$/$\stub$/$\emptyset$
}};
\end{centertikz}
Assume the view remains the same afterwards.
It is easy to see \( \ET_\CP \vdash (\mkvs,\vi_1) \csat \fp_1 : (\mkvs_1,\vi_1)\), and also for \( \ET_\SI \).

Now for the second client \( \cl_2 \), assume the view 
\( \vi_2 = \Set{\key_1 \mapsto \Set{0,1}, \key_2 \mapsto \Set{0}} \),
which leads to the fingerprint \( \fp_2 = \Set{(\otR, \key_1,1), (\otR, \key_2,2)} \),
and the kv-store \( \mkvs_2 \):
\begin{centertikz}
%Location x
\node(locx) {$\key_1 \mapsto$};
\draw pic at ([xshift=\tikzkvspace]locx.east) {vlist={versionx}{%
    /$0$/$\stub$/$\Set{\txid_1}$
    , /$1$/$\txid$/$\Set{\txid_2}$
}};

%Location y
\path (versionx.east) + (1,0) node (locy) {$\key_2 \mapsto$};
\draw pic at ([xshift=\tikzkvspace]locy.east) {vlist={versiony}{%
    /$2$/$\stub$/$\Set{\txid_1,\txid_2}$
    , /$3$/$\stub$/$\emptyset$
}};
\end{centertikz}
It is trivial that \( \ET_\CP \vdash (\mkvs_1,\vi_2) \csat \fp_2 : (\mkvs_2,\vi_2)\), and also for \( \ET_\SI \).

Commutative allows to swap two fingerprints from different clients and still yields the same kv-store.
It is not the case here.
If we let \( \fp_2 \) commits first we have the following kv-store \( \mkvs'_2 \):
\begin{centertikz}
%Location x
\node(locx) {$\key_1 \mapsto$};
\draw pic at ([xshift=\tikzkvspace]locx.east) {vlist={versionx}{%
    /$0$/$\stub$/$\emptyset$
    , /$1$/$\txid$/$\Set{\txid_2}$
}};

%Location y
\path (versionx.east) + (1,0) node (locy) {$\key_2 \mapsto$};
\draw pic at ([xshift=\tikzkvspace]locy.east) {vlist={versiony}{%
    /$2$/$\stub$/$\Set{\txid_2}$
    , /$3$/$\stub$/$\emptyset$
}};
\end{centertikz}
Over the \( \mkvs'_2 \), the view \( \vi_1 \) is no longer valid because in \( \ET_\CP \):
\[
0 \in \vi_1(\key_2)  
\wedge \wtOf(\mkvs(\key_1, 1)) \toEDGE{(((\SO \cup \WR_{\mkvs}) ; \RW_{\mkvs}^?) \cup \WW_{\mkvs})^{+}} \wtOf(\mkvs(\key_2, 0)) 
\]
but \( i \notin \vi_1(\key_1)\).
It is similar for \( \ET_\SI \).


% depenedent graph relates kv
\section{Relations to Dependency Graphs}
\label{app:depgraphs}
\label{sec:dependent-graph}
\emph{Dependency graphs} were introduced by Adya to define consistency models of transactional databases \cite{adya}. 
They are directed graphs consisting of transactions as nodes, 
each of which is labelled with transaction identifier and a set of read and write operations,
and labelled edges between transactions for describing how information flows between nodes. 
Specifically, a transaction $\txid$ reads a version for a key $\key$ that has been written by another transaction $\txid'$ 
(\emph{write-read dependency} \( \WR\)), overwrites a version of $\key$ written by $\txid'$ (\emph{write-write dependency} \( \WW \)),
or reads a version of $\key$ that is later overwritten by $\txid'$ (\emph{read-write anti-dependency} \( \RW \)). 
Note that we have named dependencies in kv-stores after the labelled edges of dependency graph. 
The main result of this Section shows that kv-stores are in fact isomorphic to dependency graphs, 
and dependencies in a kv-store naturally translates into a labelled edge in the associated dependency graph.

\begin{definition}
\label{def:dgraph}
A \emph{dependency graph} is a quadruple $\Gr = (\TtoOp{T}, \WR, \WW, \RW)$, where
\begin{itemize}
\item 
    $\TtoOp{T}: \TxID \parfun \pset{\Ops}$ is a partial mapping from transaction identifiers 
    to the set of operations, where there are at most one read operation and one write operation per key, 
    and such that $\TtoOp{T}(\txid_{0}) = \{(\otW, \key, \val_{0} \mid \key \in \Keys \}$; furthermore, 
    $\txid_{0} \in \dom(\TtoOp{T}$, and $\TtoOp{T}(\txid_{0}) = \{(\otW, \key, \val_{0}) \mid \key \in \Keys\}$, 
\item
    $\WR : \Keys \to \pset{\dom(\TtoOp{T}) \times \dom(\TtoOp{T})}$ is a function that 
maps each key $\key$ into a relation between transactions, such that for any $\txid, \txid_1, \txid_2, 
\key, \cl, m, n$: 
\begin{itemize}
\item if $(\otR, \key, \val) \in \TtoOp{T}(\txid)$, 
%either $\val = \val_0$ 
%and there exists no $\txid'$ such that $\txid' \toEDGE{\WR(\key)} \txid$,  
%or 
there exists $\txid' \neq \txid$ such that $(\otW, \key, \val) \in \TtoOp{T}(\txid')$, and $\txid' \toEDGE{\WR(\key)} \txid$, 
\item if $\txid_1 \toEDGE{\WR(\key)} \txid$ and $\txid_2 \toEDGE{\WR(\key)} \txid$, then 
$\txid_1 = \txid_2$.
\item if $\txid_{\cl}^{m} \toEDGE{\WR(\key)} \txid_{\cl}^{n}$, then $m < n$.
\end{itemize}
\item $\VO: \Keys \to \pset{\dom(\TtoOp{T}) \times \dom(\TtoOp{T})}$ is a function 
that maps each key into an irreflexive relation between transactions, such that for any $\txid, \txid', \key, \cl, m, n$, 
\begin{itemize}
\item if $\txid \toEDGE{\WW(\key)} \txid'$, then $(\otW, \key, \_) \in \TtoOp{T}(\txid), (\otW, \key, \_) \in \TtoOp{T}(\txid')$, 
\item if $(\otW, \key, \_) \in \TtoOp{T}(\txid), (\otW, \key, \_) \in \TtoOp{T}(\txid')$, then either $\txid = \txid'$, 
$\txid \toEDGE{\WW(\key)} \txid'$, or $\txid' \toEDGE{\WW(\key)} \txid$; furthermore, if $\txid = \txid_{0}$, 
then it must be the case that $\txid \toEDGE{(\WW(\key))} \txid'$,
\item if $\txid_{\cl}^{m} \toEDGE{\WW(\key)} \txid_{\cl}^{n}$, then $m < n$, 
\end{itemize}
\item $\AD: \Keys \to \pset{\dom(\TtoOp{T}) \times \dom(\TtoOp{T})}$ is defined 
by letting $\txid \toEDGE{\RW(\key)} \txid'$ if and only if $\txid'' \toEDGE{\WR(\key)} \txid$, 
$\txid'' \toEDGE{\WW(\key)} \txid'$ for some $\txid''$.
%$(\otR, \key, \_) \in \TtoOp{T}(\txid)$, 
%$(\otW, \key, \_) \in \TtoOp{T}(\txid')$ and 
%either there exists no $\txid''$ such that $\txid'' \toEDGE{\WR(\key)} \txid$, or 
%$\txid'' \toEDGE{\WR(\key)} \txid$, $\txid'' \toEDGE{\WW(\key)} \txid'$ for 
%some $\txid''$.
\end{itemize}
Let $\Dgraphs$ be the set of all dependency graphs.
\end{definition}

Given a dependency graph $\Gr = (\TtoOp{T}, \WR, \WW, \RW)$, we 
let $\WR_{\Gr} = \WR$, and similarly for $\WW$ and $\RW$. We also let $
\T_{\Gr} = \dom(\TtoOp{T})$, and write $(l, \key, \val) \in_{\Gr} \txid$ if 
$(;, \key, \val) \in \TtoOp{\Gr}(\txid)$. We
often 
commit an abuse of notation and use $\WR$ to denote the relation 
$\bigcup_{\key \in \Keys} \WR(\key)$; a similar notation is adopted for $\WW, \RW$. 
It will always be clear from the context whether the symbol $\WR$ refers to a function 
from keys to relations, or to a relation between transactions. 

As stated above, kv-stores are isomorphic to dependency graphs. The proof 
of this result is the topic of this Section. 

\begin{theorem}
\label{thm:kv2graph}
There is a one-to-one map between kv-stores and dependency graphs.
\end{theorem}
The proof structure of \cref{thm:kv2graph} is standard in its nature. 
We first how to encode a kv-store into a dependency graph. Then we 
show how to encode a dependency graph into a kv-store. Finally, 
we prove that the two constructions are one the inverse of the other: 
if we convert a kv-store $\mkvs$ into a dependency graph $\Gr_{\mkvs}$, 
then back to a kv-store $\mkvs_{\Gr_{\mkvs}}$, we obtain the initial kv-store.

To convert a kv-store $\mkvs$ into a dependency graph, we first define how 
to extract a fingerprint of a transaction identifier $\txid$ appearing in $\mkvs$:
\begin{definition}
\label{def:mkvs_fingerprint}
Let $\mkvs$ be a kv-store. For any transaction identifier $\txid$, we define 
$\fp_{\mkvs}(\txid)$ to be the smallest set such that whenever 
$\mkvs(\key, \_) = (\val, \txid, \_)$ then $(\otW, \key, \val) \in \txid$, and 
whenever $\mkvs(\key, \_) = (\val, \_, \{\txid\} \cup \_)$, then $(\otR, \key, \val) \in \txid$. 
\end{definition}
\begin{proposition}
\label{prop:mkvs_fp_welldefined}
For any $\mkvs, \txid$, the fingerprint $\fp_{\mkvs}(\txid)$ is well defined. 
That is, whenever $(\otW,\key,\val_1), (\otW,\key,\val_2) \in \fp_{\mkvs}(\txid)$, 
then $\val_1 = \val_2$, and whenever $(\otR, \key, \val_1), (\otR,\key, \val_2) \in \fp_{\mkvs}(\txid)$, 
then $\val_1 = \val_2$.
\end{proposition}

\begin{proof}
Suppose that $(\otW, \key, \val_1), (\otW,\key,\val_2) \in \fp_{\mkvs}(\txid)$ for some $key, 
\val_1, \val_2$. That is, there exist two indexes $i_1, i_2$ such that 
$\mkvs(\key, i_1) = (\val_1, \txid, \_)$, and $\mkvs(\key, i_2) = (\val_2, \txid, \_)$. 
That is, $\wtOf(\mkvs(\key, i_1)) = \wtOf(\mkvs(\key, i_2))$, and it follows 
from \cref{def:mkvs-appendix} that $i_1 = i_2$. In particular, this implies that $\val_1 = \val_2$. 

A similar argument can be used to prove that if $(\otR, \key, \val_1), (\otR,\key, \val_2) \in \fp_{\mkvs}(\txid)$, 
then $\val_1 = \val_2$. In this case, in fact, we have that there exist two indexes $i_1, i_2$ such that 
$\mkvs(\key, i_1) = (\val_1, \_,\{\txid\} \cup \_)$, and $\mkvs(\key, i_2) = (\val_2, \_, \{\txid\} \cup \_)$. 
Equivalently, $\txid \in \rsOf(\mkvs(\txid, i_1)) \cap \rsOf(\mkvs(\txid, i_2))$, and from 
\cref{def:mkvs-appendix} it must be the case that $i_1 = i_2$, hence $\val_1 = \val_2$.
\end{proof}

Using \cref{def:mkvs_fingerprint}, conerting a kv-store $\mkvs$  into a dependency graph is immediate, as the following 
definition shows: 

\begin{definition}
\label{def:kv2graph}
Given a kv-store $\mkvs$, the \emph{dependency graph} $\Gr_{\mkvs} = (\TtoOp{T}_{\mkvs}, \WR_{\mkvs}, 
\WW_{\mkvs}, \RW_{\mkvs})$ is defined by letting  $\TtoOp{T}_{\mkvs}(\txid)$ be defined if and only if
$\fp_{\mkvs}(\txid) \neq \emptyset$, in which case we let $\TtoOp{T}_{\mkvs}(\txid) = \fp_{\mkvs}(\txid)$. 
The relations $\WR_{\mkvs}, \WW_{\mkvs}, \RW_{\mkvs}$ are inherited directly from the transactional 
dependencies defined for $\mkvs$.
%here exists an index $i$ and a key 
%$\key$ such that either $\txid = \wtOf(\mkvs(\key, i))$, or $\txid \in \rsOf(\mkvs(\key,i))$; furthermore, 
%$(\otW, \key, \val) \in \TtoOp{T}(\txid)$ if and only 
%if $\txid = \wtOf(\mkvs(\key, i))$ for some $i$, and 
%$(\otR, \key, \val) \in \TtoOp{T}(\txid)$ if and only if $\txid \in \rsOf(\mkvs(\key, i))$ for some $i$, 
%\item $ if and only if $\txid \toEDGE{\WR_{\mkvs}(\key)} \txid'$: 
%recall that this means that there exists an index $i: 0 \leq i < \lvert \mkvs(\key) \rvert$ 
%such that $\txid = \wtOf(\mkvs(\key, i))$, and $\txid' \in \rsOf(\mkvs(\key, i))$, 
%\item $\txid \toEDGE{\WW(\key)} \txid'$ if and only if $\txid \toEDGE{}
%there exist two indexes $i,j$: $0 \leq i < j < \lvert \mkvs(\key) \rvert$ 
%such that $\txid = \wtOf(\key, i)$, $\txid' = \wtOf(\key, j)$, 
%\item $\txid \toEDGE{\RW(\key)} \txid'$ if and only if there exist two indexes $i,j$: $0 \leq i < j < \lvert \mkvs(\key) \rvert$ 
%such that $\txid \in \rsOf(\key, i)$ and $\txid' = \wtOf(\key, j)$.
%\end{itemize}
\end{definition}

\begin{definition}
\label{def:dependency-to-kv-store}
Given a dependency graph $\Gr = (\TtoOp{T}, \WR, \WW, \RW)$, we define the kv-store $\mkvs_{\Gr}$ as follows: 
\begin{enumerate}
\item for any transaction $\txid \in \dom(\TtoOp{T})$ such that $(\otW, \key, \val) \in \TtoOp{T}(\txid)$, 
    let $\txidset = \Set{ \txid' }[ \txid \toEDGE{\WR(\key)} \txid']$, and let $\ver(\txid, \key) = (\val, \txid, \txidset)$, 
\item For each key $\key$, let $\ver_{\key}^{0} = (\val_0, \txid_0, \txidset_k^{0})$, where $\txidset_{k}^{0} = \Set{ \txid }[ (\otR, \key, \stub) \in 
\TtoOp{T}(\txid) \land \fora{ \txid' } \neg( \txid' \toEDGE{\WR(\key)} \txid ]$. 
Let also $\Set{ \ver_{\key}^{i} }_{i = 1}^{n}$ be the ordered set of versions such that, for any 
$i=1,\cdots,n$, $\ver_{\key}^{i} = \ver(\txid, \key)$ for some $\txid$ such that $(\otW, \key, \_) \in \TtoOp{T}(\txid)$, 
and such that for any $i, j: 1 \leq i < j \leq n$, $\wtOf(\ver_{\key}^{i}) \toEDGE{\WW(\key)} \wtOf(\ver_{\key}^{j})$. 
Then we let $\mkvs_{\Gr}= \lambda \key. \prod_{i=0}^{n} \ver_{\key}^{i}$.
\end{enumerate}
\end{definition}

\begin{proposition}
\label{prop:well-formed-kv-store-to-dependency}
Let $\mkvs$ be a well-formed kv-store. Then $\Gr_{\mkvs}$ is a well-formed dependency graph.
\end{proposition}

\begin{proof}
Let $\mkvs$ be a (well-formed) kv-store. We need to show that 
$\Gr_{\mkvs} = (\TtoOp{T}_{\mkvs}, \WR_{\mkvs}, \WW_{\mkvs}, \RW_{\mkvs})$ is a dependency graph. 
As a first step, we show that $\Gr_{\mkvs}$ is a dependency graph, 
i.e. it satisfies all the constraints placed by \cref{def:dgraph}.

\begin{itemize}
\item Let $\txid \in \dom(\TtoOp{T}_{\mkvs})$, and suppose that $(\otR, \key, \val) \in \TtoOp{T}_{\mkvs}(\txid)$. 
We need to prove that there exists a transaction $\txid' \in \dom(\TtoOp{T}_{\mkvs})$ such
%either $\val = \val_{0}$, and there exists no $\txid' \in \dom(\TtoOp{T}_{\mkvs})$ such that 
%$\txid' \toEDGE{\WR_{\mkvs}(\key)} \txid$, or $\txid' \toEDGE{\WR_{\mkvs}(\key)} \txid$ for some 
%$\txid' \in \dom(\TtoOp{T}_{\mkvs})$ such 
%that $(\otW, \key, \val) \in \TtoOp{T}_{\mkvs}(\txid')$. 
%Because $\txid \in \dom(\TtoOp{T}_{\mkvs})$, the definition of $\Gr_{\mkvs}$ (and in particular the 
%fact that $\TtoOp{T}_{\mkvs} : \TxID \rightharpoonup \pset{\Ops}$) ensures that 
%$\txid \neq \txid_0$. Furthermore, 
Because $(\otR, \key, \val) \in \TtoOp{T}_{\mkvs}(\txid)$, there 
must exist an index $i: 0 \leq i < \lvert \mkvs(\key) \rvert$ such that $\mkvs(\key, i) = (\val, \txid', \Set{\txid } \cup \_ )$ 
for some $\txid' \in \TxID$.  In this case we have that $\txid' \toEDGE{\WR_{\mkvs}(\key)} \txid$, 
and by \cref{def:mkvs_fingerprint} we have that $(\otW, \key, \val) \in_{\Gr_{\mkvs}} \txid'$.
%We have two possibilities: 
%\begin{enumerate}
%\item $i = 0$, in which case the hypothesis that $\mkvs$ is well-formed ensures that $\txid' = \txid_0$, 
%and $\val = \val_0$. We also have that there exists no transaction $\txid''$ such that $\txid'' \toEDGE{\WR_{\mkvs}(\key)} \txid$: 
%in fact, by \cref{def:kv2graph}, we have that $\txid'' \toEDGE{\WR(\key)} \txid$ if and only if there exists an index 
%$j: 0 < j < \lvert \mkvs(\key) \rvert$ such that $\mkvs(\key, j) = (\_, \txid'', \Set{\txid} \cup \_)$. However, in this case we would 
%have that $0 < j$, and $\txid \in \rsOf(\mkvs(\key, j)) \cap \rsOf(\mkvs(\key, 0))$, contradicting the constraint placed 
%over well-formed kv-stores that a transaction never reads multiple versions for a key. Therefore, there exists 
%no transaction $\txid''$ such that $\txid'' \toEDGE{\WR_{\mkvs}(\key)} \txid$, 
%\item $i > 0$; in this case \cref{def:kv2graph} ensures that $\txid' \toEDGE{\WR_{\mkvs}(\key)} \txid$; also, 
%because $\mkvs(\key, i) = (\val, \txid', \_)$, it must be the case that $(\otW, \key, \val) \in \TtoOp{T}_{\mkvs}(\txid')$.
%\end{enumerate}
\item Let $\txid \in \dom(\TtoOp{T}_{\mkvs})$, and suppose that there exist $\txid_1, \txid_2$ such that 
$\txid_{1} \toEDGE{\WR_{\key}(\mkvs)} \txid$, $\txid_{2} \toEDGE{\WR_{\key}(\mkvs)} \txid$. 
By \cref{def:kv2graph}, there exist two indexes $i, j: 0 \leq i, j < \lvert \mkvs(\key) \rvert$, such that 
$\mkvs(\key, i) = (\_, \txid_1, \Set{\txid} \cup \_)$, $\mkvs(\key, j) = (\_, \txid_2, \Set{\txid} \cup \_)$. 
We have that $\txid \in \rsOf(\mkvs(\key, i)) \cap \rsOf(\mkvs(\key, j))$, i.e. 
$\rsOf(\mkvs(\key,i)) \cap \rsOf(\mkvs(\key, j)) \neq \emptyset$. Because we are assuming 
that $\mkvs$ is well-formed, then it must be the case that $i = j$. This implies that $\txid_1 = \txid_2$.
\item Let $\cl \in \Clients$, $m, n \in \Nat$ and $\key \in \Keys$ be such that 
$\txid_{\cl}^{n} \toEDGE{\WR_{\mkvs}(\key)} \txid_{\cl}^{m}$.  We prove that 
$n < m$. By \cref{def:kv2graph}, it must be the case that 
there exists an index $i : 0 \leq i < \lvert \mkvs(\key) \rvert$ such that $\mkvs(\key, i) = 
(\_, \txid_{\cl}^{n}, \Set{\txid_{\cl}^{m}} \cup \_)$. Because $\mkvs$ is well-formed, 
it must be the case that $n < m$.
\item Let $\txid \in \dom(\TtoOp{T}_{\mkvs})$. We show that $\neg (\txid \toEDGE{\WW_{\mkvs}} \txid)$. 
We prove this fact by contradiction: suppose that $\txid \toEDGE{\WW_{\mkvs}(\key)} \txid$ for some key $\key$. By \cref{def:kv2graph}, 
there must exist two indexes $i,j: 0 \leq i < j < \lvert \mkvs(\key) \rvert$ such that $\txid = \wtOf(\mkvs(\key,i))$ and 
$\txid = \wtOf(\mkvs(\key, j))$. Because we are assuming that $\mkvs$ is well-formed, then it must be the 
case that $i = j$, contradicting the statement that $i < j$. 
\item Let $\txid, \txid'$ be such that $\txid' \toEDGE{\WW_{\key}(\mkvs)} \txid$. 
We must show that  $(\otW, \key, \_) \in \TtoOp{T}_{\mkvs}(\txid')$, and $(\otW, \key, \_) \in \TtoOp{T}_{\mkvs}(\txid)$.
By \cref{def:kv2graph}, there exist $i, j: 0 \leq i,j < \lvert \mkvs(\key) \rvert$ such that 
$\mkvs(\key, i) = (\val', \txid', \_)$ and $\mkvs(\key, j) = (\val, \txid, \_)$, for some 
$\val, \val' \in \Val$. \cref{def:kv2graph} also ensures that $(\otW, \key, \val') \in 
\TtoOp{T}_{\mkvs}(\txid')$, and $(\otW, \key, \val) \in \TtoOp{T}_{\mkvs}(\txid)$.
\item Let $\txid, \txid'$ be such that $(\otW, \key, \_) \in \TtoOp{T}_{\mkvs}(\txid)$ 
and $(\otW, \key, \_) \in \TtoOp{T}_{\mkvs}(\txid')$. We need to prove that 
either $\txid = \txid', \txid \toEDGE{\WW_{\mkvs}(\key)} \txid'$, or $\txid' \toEDGE{\WW_{\mkvs}(\key)} \txid$. 
By \cref{def:kv2graph} there exist two indexes $i, j: 0 < i,j< \lvert \mkvs(\key) \rvert$ such that 
$\mkvs(\key, i) = (\_, \txid, \_)$ and $\mkvs(\key, j) = (\_, \txid', \_)$. If $i = j$, then $\txid = \txid'$ 
and there is nothing left to prove. Otherwise, suppose without loss of generality that 
$i < j$. Then \cref{def:kv2graph} ensures that $\txid \toEDGE{\WW_{\mkvs}(\key)} \txid'$. 
\item Suppose that $\txid_{\cl}^{m} \toEDGE{\WW_{\mkvs}(\key)} \txid_{\cl}^{n}$ for 
some $\cl \in \Clients$ and $m, n \in \Nat$. We need to show that $m < n$. 
By \cref{def:kv2graph}, because  $\txid_{\cl}^{m} \toEDGE{\WW_{\mkvs}(\key)} \txid_{\cl}^{n}$ 
there exist two indexes $i,j: 0 < i,j < \lvert \mkvs(\key) \rvert$ such that $\wtOf(\mkvs(\key,i)) = \txid_{\cl}^{m}$ 
and $\wtOf(\mkvs(\key, j)) = \txid_{\cl}^{n}$. From the assumption that $\mkvs$ is well-formed, it 
follows that $n < m$.
\end{itemize}
\end{proof}

Next, we show how to convert a dependency graph $\Gr$ into a kv-store $\mkvs$. 
The main idea is that any transaction $\txid \in \T_{\Gr}$ induces a set of versions, and 
for each key $\key$, the write-write-dependency order $\WW_{\Gr}(\key)$ determines 
the order of these versions in $\mkvs_{\Gr}$. 

\begin{definition}
\label{def:kv-store-to-dependency-graph}
Let $\Gr$ be a dependency graph. Given a key $\key$, let $n_{\key}$, 
$\{\val_{i}^{\key}\}_{i=0}^{n_{\key}}$
$\{\txid^{\key}_{i}\}_{i=0}^{n_{\key}}$ be such that 
$\{\txid^{\key}_{i}\}_{i=0}^{n_{\key}}= \{ \txid \mid (\otW, \key, \val_{i}^{\key}) \in_{\Gr} \txid \}$, 
where the index set $\{1,\cdots,n_{\key}\}$  is chosen to be consistent
with $\WW_{\Gr}(\key)$: that is, $\txid_{i} \xrightarrow{\WW(\key)} \txid_{j}$ if 
and only if $i < j$. Given a key $\key$ and an index $i=1,\cdots, n_{\key}$, we also 
let $\T_{i}^{\key} = \{ \txid \mid \txid_{i}^{\key} \xrightarrow{\WR(\key)}\} \txid$. Note that 
this set is possibly empty. Finally, we let $\mkvs_{\Gr}$ be such that, for any $\key \in \Keys$, 
$\lvert \mkvs_{\Gr}(\key) \rvert = n_{\key}$, and for any $i=0,\cdots,n$, $\mkvs_{\Gr}(\key,i) = 
(\val_{i}^{\key}, \txid_{i}^{\key}, \T_{i}^{\key})$.
\end{definition}

\begin{proposition}
\label{prop:dependency-to-kv-store}
For any dependency graph $\Gr$, $\mkvs_{\Gr}$ is a (well-formed) kv-store.
\end{proposition}

\begin{proof}
We show that $\mkvs_{\Gr}$ satisfies all the constraints fromf \cref{def:mkvs-appendix}. Throughout 
the proof, we adopt the same notation of \cref{def:kv-store-to-dependency-graph}.

Let $\key \in \Keys$, and let $i,j$ be such that $\rsOf(\mkvs_{\Gr}(\key,i)) \cap \rsOf(\mkvs_{\Gr}(\key,j)) \neq \emptyset$, 
that is there exists a transaction $\txid \in \rsOf(\mkvs_{\Gr}(\key, i)) \cap \rsOf(\mkvs_{\Gr}(\key,j))$. We show that $i = j$. 
By definition, $\rsOf(\mkvs_{\Gr}(\key, i)) = \T_{\key}^{i}$, and $\rsOf(\mkvs_{\Gr}(\key, j)) = \T_{\key}^{j}$. 
\cref{def:kv-store-to-dependency-graph} ensures that $\txid^{\key}_{i} \xrightarrow{\WR_{\Gr}(\key)} \txid$, 
and $\txid^{\key}_{j} \xrightarrow{\WR_{\Gr}(\key)} \txid$. By definition of dependency graph, it must be the 
case that $\txid^{\key}_{i} = \txid^{\key}_{j}$, and because the order of writers 
transactions in versions in $\mkvs_{\Gr}(\key)$ 
is defined to be consistent with $\WW_{\Gr}(\key)$, then it must also be the case that $i = j$. 

Suppose no that $\key, i, j$ are such that $\wtOf(\mkvs_{\Gr}(\key, i)) = \wtOf(\mkvs_{\Gr}(\key, j))$. By 
definition $\wtOf(\mkvs_{\Gr}(\key, i)) = \txid_{i}^{\key}$, and $\wtOf(\mkvs_{\Gr}(\key, j) = \txid^{\key}_{j}$. 
That is, $\txid_{i}^{\key} = \txid_{j}^{\key}$. Because the order of writer transactions in $\mkvs_{\Gr}(\key)$ 
is consistent with $\WW_{\Gr}(\key)$, we also have that $i = j$.

Next, note that for any key $\key$, $\txid^{\key}_{0} = \txid_{0}$. In fact, because $\txid_{0} \in_{\Gr} 
(\otW, \key, \val_{0})$, we have that $\txid_{0} = \txid^{\key}_{i}$ for some $i=0,\cdots, n_{\key}$.
Also, because whenever $\txid$ is such that $(\otW, \key, \_) \in_{\Gr} \txid$, then 
it must be the case that $\txid_{0} \xrightarrow{\WW(\key)} \txid$, then it must be 
the case that $i = 0$. It follows that, for any $\key \in \Keys$, $\mkvs_{\Gr}(\key, 0) = (\val_{0}, \txid_{0}, \_)$.

Finally, suppose that $\txid_{\cl}^{n} = \wtOf(\mkvs_{\Gr}(\key, i))$, $\txid_{\cl}^{m}  = \wtOf(\mkvs_{\Gr}(\key, j))$ 
for some $i, j$ such that $i < j$. In this case we have that $\txid_{\cl}^{n} = \txid_{i}^{\key}$, $\txid_{\cl}^{m} = 
\txid_{j}^{\key}$, and because $i < j$ it must be the case that $\txid_{\cl}^{n} \xrightarrow{\WW_{\Gr}(\key)} 
\txid_{\cl}^{m}$. The definition of dependency graph ensures then that it must $n < m$. A similar 
argument shows that, if $\txid_{\cl}^{n} \in \wtOf(\mkvs_{\Gr}(\key, i)), \txid_{cl}^{m} \in \rsOf(\mkvs_{\Gr}(\key, i))$, 
then it must be the case that $\txid_{\cl}^{n} \xrightarrow{\WR_{\Gr}(\key)} \txid_{\cl}^{m}$, and therefore 
$n < m$.
\end{proof}

Finally, we need to show that the two constructions outlined in \cref{def:kv-store-to-dependency-graph} and 
\cref{def:dependency-to-kv-store} are one the inverse of the other.

\begin{proposition}
\label{prop:kv-store-back-and-forth}
For any kv-store $\mkvs$, $\mkvs_{\Gr_{\mkvs}} = \mkvs$.
\end{proposition}

\begin{proof}
We prove that for any $\key \in \Keys$, 
$\mkvs(\key) = \mkvs_{\Gr_{\mkvs}}(\key)$.

Let then $\key \in \Keys$, and suppose that $\mkvs(\key) = (\val_{0}, \txid_{0}, \T_{0}) \cdots (\val_{n}, \txid_{n}, \T_{n})$. 
By construction, in $(\otW, \key, \val_{i}) \in_{\Gr_{\mkvs}} \txid_{i}$, and whenever there is a transaction 
$\txid$ such that $(\otW, \key, \txid) \in_{\Gr_{\mkvs}} \txid$, then $\txid = \txid_{i}$ for some $i=0,\cdots, n$. 
In particular, we have that $\txid_{0} \xrightarrow{\WW_{\mkvs}(\key)} \cdots \xrightarrow{\WW_{\mkvs}(\key)} \txid_{n}$ 
completely characterises the write-write-dependency relation $\WW_{\mkvs}(\key)$ over $\mkvs_{\Gr}$ 
 (recall that, by \cref{def:kv-store-to-dependency-graph}, $\WR_{\Gr_{\mkvs}} = \WR_{\mkvs}$).
By definition, we have that $\mkvs_{\Gr_{\mkvs}} = (\val_{0}, \txid_{0}, \T'_{0}) \cdots (\val_{n}, \txid_{n}, \T'_{n})$. 

It remains to prove that, for any $i=0,\cdots, n$, $\T'_{i} = \T_{i}$.
For any $i=0,\cdots, n$, and transaction $\txid \in \T_{i}$, \cref{def:kv-store-to-dependency-graph} ensures that 
$\txid_{i} \xrightarrow{\WR_{\mkvs}} \txid$, 
and by \cref{def:dependency-to-kv-store} it must be the case that $\txid \in \T'_{i}$.
Furthermore, if $\txid' \in \T'_{i}$, then from \cref{def:dependency-to-kv-store} it must be 
the case that $\txid_{i} \xrightarrow{\WR_{\Gr_{mkvs}}(\key)} \txid'$, 
or equivalently $\txid_{i} \xrightarrow{\WR_{\mkvs}(\key)} \txid'_{i}$.
(\cref{def:kv-store-to-dependency-graph}). Then it must be the case that $\txid' \in \T_{i}$. 
\end{proof}

\begin{proposition}
\label{prop:dependency-back-and-forth}
For any dependency graph $\Gr$, $\Gr_{\mkvs_{\Gr}} = \Gr$. 
\end{proposition}

\begin{proof}
The proof of this claim is similar to \cref{prop:kv-store-back-and-forth}, and therefore omitted.
\end{proof}

%\begin{proof}
%    We prove that given any a well-formed kv-store \( \mkvs \), then $\Gr_{\mkvs}$ is a well-formed dependency graph in \cref{prop:well-formed-kv-store-to-dependency},
%    and given any \( \Gr \), then  $\mkvs_{\Gr}$ is a well-formed kv-store in \cref{prop:well-formed-dependency-to-kv-store}.
%    Then we prove the bijection that $\mkvs_{\Gr_{\mkvs}} = \mkvs$ in \cref{prop:bijection:mkvs-dgraph}.
%\end{proof}
%
%
%
%\begin{proposition}
%\label{prop:well-formed-kv-store-to-dependency}
%Let $\mkvs$ be a well-formed kv-store. Then $\Gr_{\mkvs}$ is a well-formed dependency graph.
%\end{proposition}
%
%\begin{proof}
%Let $\mkvs$ be a (well-formed) kv-store. We need to show that 
%$\Gr_{\mkvs} = (\TtoOp{T}_{\mkvs}, \WR_{\mkvs}, \WW_{\mkvs}, \RW_{\mkvs})$ is a dependency graph. 
%As a first step, we show that $\Gr_{\mkvs}$ is a dependency graph, 
%i.e. it satisfies all the constraints placed by \cref{def:dgraph}.
%
%\begin{itemize}
%\item Let $\txid \in \dom(\TtoOp{T}_{\mkvs})$, and suppose that $(\otR, \key, \val) \in \TtoOp{T}_{\mkvs}(\txid)$. 
%We need to prove that either $\val = \val_{0}$, and there exists no $\txid' \in \dom(\TtoOp{T}_{\mkvs})$ such that 
%$\txid' \toEDGE{\WR_{\mkvs}(\key)} \txid$, or $\txid' \toEDGE{\WR_{\mkvs}(\key)} \txid$ for some 
%$\txid' \in \dom(\TtoOp{T}_{\mkvs})$ such that $(\otW, \key, \val) \in \TtoOp{T}_{\mkvs}(\txid')$. 
%Because $\txid \in \dom(\TtoOp{T}_{\mkvs})$, the definition of $\Gr_{\mkvs}$ (and in particular the 
%fact that $\TtoOp{T}_{\mkvs} : \TxID_{0} \rightharpoonup \pset{\Ops}$) ensures that 
%$\txid \neq \txid_0$. Furthermore, because $(\otR, \key, \val) \in \TtoOp{T}_{\mkvs}(\txid)$, there 
%must exist an index $i: 0 \leq i < \lvert \mkvs(\key) \rvert$ such that $\mkvs(\key, i) = (\val, \txid', \Set{\txid } \cup \_ )$ 
%for some $\txid' \in \TxID$. 
%We have two possibilities: 
%\begin{enumerate}
%\item $i = 0$, in which case the hypothesis that $\mkvs$ is well-formed ensures that $\txid' = \txid_0$, 
%and $\val = \val_0$. We also have that there exists no transaction $\txid''$ such that $\txid'' \toEDGE{\WR_{\mkvs}(\key)} \txid$: 
%in fact, by \cref{def:kv2graph}, we have that $\txid'' \toEDGE{\WR(\key)} \txid$ if and only if there exists an index 
%$j: 0 < j < \lvert \mkvs(\key) \rvert$ such that $\mkvs(\key, j) = (\_, \txid'', \Set{\txid} \cup \_)$. However, in this case we would 
%have that $0 < j$, and $\txid \in \rsOf(\mkvs(\key, j)) \cap \rsOf(\mkvs(\key, 0))$, contradicting the constraint placed 
%over well-formed kv-stores that a transaction never reads multiple versions for a key. Therefore, there exists 
%no transaction $\txid''$ such that $\txid'' \toEDGE{\WR_{\mkvs}(\key)} \txid$, 
%\item $i > 0$; in this case \cref{def:kv2graph} ensures that $\txid' \toEDGE{\WR_{\mkvs}(\key)} \txid$; also, 
%because $\mkvs(\key, i) = (\val, \txid', \_)$, it must be the case that $(\otW, \key, \val) \in \TtoOp{T}_{\mkvs}(\txid')$.
%\end{enumerate}
%\item Let $\txid \in \dom(\TtoOp{T}_{\mkvs})$, and suppose that there exist $\txid_1, \txid_2$ such that 
%$\txid_{1} \toEDGE{\WR_{\key}(\mkvs)} \txid$, $\txid_{2} \toEDGE{\WR_{\key}(\mkvs)} \txid$. 
%By \cref{def:kv2graph}, there exist two indexes $i, j: 0 < i, j < \lvert \mkvs(\key) \rvert$, such that 
%$\mkvs(\key, i) = (\_, \txid_1, \Set{\txid} \cup \_)$, $\mkvs(\key, j) = (\_, \txid_2, \Set{\txid} \cup \_)$. 
%We have that $\txid \in \rsOf(\mkvs(\key, i)) \cap \rsOf(\mkvs(\key, j))$, i.e. 
%$\rsOf(\mkvs(\key,i)) \cap \rsOf(\mkvs(\key, j)) \neq \emptyset$. Because we are assuming 
%that $\mkvs$ is well-formed, then it must be the case that $i = j$. This implies that $\txid_1 = \txid_2$.
%\item Let $\cl \in \Clients$, $m, n \in \Nat$ and $\key \in \Keys$ be such that 
%$\txid_{\cl}^{n} \toEDGE{\WR_{\mkvs}(\key)} \txid_{\cl}^{m}$.  We prove that 
%$n < m$. By \cref{def:kv2graph}, it must be the case that 
%there exists an index $i : 0 \leq i < \lvert \mkvs(\key) \rvert$ such that $\mkvs(\key, i) = 
%(\_, \txid_{\cl}^{n}, \Set{\txid_{\cl}^{m}} \cup \_)$. Because $\mkvs$ is well-formed, 
%it must be the case that $n < m$.
%\item Let $\txid \in \dom(\TtoOp{T}_{\mkvs})$. We show that $\neg (\txid \toEDGE{\WW_{\mkvs}} \txid)$. 
%We prove this fact by contradiction: suppose that $\txid \toEDGE{\WW_{\mkvs}(\key)} \txid$ for some key $\key$. By \cref{def:kv2graph}, 
%there must exist two indexes $i,j: 0 < i < j < \lvert \mkvs(\key) \rvert$ such that $\txid = \wtOf(\mkvs(\key,i))$ and 
%$\txid = \wtOf(\mkvs(\key, j))$. Because we are assuming that $\mkvs$ is well-formed, then it must be the 
%case that $i = j$, contradicting the statement that $i < j$. 
%\item Let $\txid, \txid'$ be such that $\txid' \toEDGE{\WW_{\key}(\mkvs)} \txid$. 
%We must show that  $(\otW, \key, \_) \in \TtoOp{T}_{\mkvs}(\txid')$, and $(\otW, \key, \_) \in \TtoOp{T}_{\mkvs}(\txid)$.
%By \cref{def:kv2graph}, there exist $i, j: 0 < i,j < \lvert \mkvs(\key) \rvert$ such that 
%$\mkvs(\key, i) = (\val', \txid', \_)$ and $\mkvs(\key, j) = (\val, \txid, \_)$, for some 
%$\val, \val' \in \Val$. \cref{def:kv2graph} also ensures that $(\otW, \key, \val') \in 
%\TtoOp{T}_{\mkvs}(\txid')$, and $(\otW, \key, \val) \in \TtoOp{T}_{\mkvs}(\txid)$.
%\item Let $\txid, \txid'$ be such that $(\otW, \key, \_) \in \TtoOp{T}_{\mkvs}(\txid)$ 
%and $(\otW, \key, \_) \in \TtoOp{T}_{\mkvs}(\txid')$. We need to prove that 
%either $\txid = \txid', \txid \toEDGE{\WW_{\mkvs}(\key)} \txid'$, or $\txid' \toEDGE{\WW_{\mkvs}(\key)} \txid$. 
%By \cref{def:kv2graph} there exist two indexes $i, j: 0 < i,j< \lvert \mkvs(\key) \rvert$ such that 
%$\mkvs(\key, i) = (\_, \txid, \_)$ and $\mkvs(\key, j) = (\_, \txid', \_)$. If $i = j$, then $\txid = \txid'$ 
%and there is nothing left to prove. Otherwise, suppose without loss of generality that 
%$i < j$. Then \cref{def:kv2graph} ensures that $\txid \toEDGE{\WW_{\mkvs}(\key)} \txid'$. 
%\item Suppose that $\txid_{\cl}^{m} \toEDGE{\WW_{\mkvs}(\key)} \txid_{\cl}^{n}$ for 
%some $\cl \in \Clients$ and $m, n \in \Nat$. We need to show that $m < n$. 
%By \cref{def:kv2graph}, because  $\txid_{\cl}^{m} \toEDGE{\WW_{\mkvs}(\key)} \txid_{\cl}^{n}$ 
%there exist two indexes $i,j: 0 < i,j < \lvert \mkvs(\key) \rvert$ such that $\wtOf(\mkvs(\key,i)) = \txid_{\cl}^{m}$ 
%and $\wtOf(\mkvs(\key, j)) = \txid_{\cl}^{n}$. From the assumption that $\mkvs$ is well-formed, it 
%follows that $n < m$.
%\end{itemize}
%\end{proof}
%
%
%
%\begin{proposition}
%\label{prop:well-formed-dependency-to-kv-store}
%For any dependency graph $\Gr = (\TtoOp{T}, \WW, \WR, \RW)$, $\mkvs_{\Gr}$ is a well-formed kv-store.
%\end{proposition}
%
%\begin{proof}
%We prove that each of the four constraints required by well-formed kv-stores 
%are satisfied by $\mkvs_{\Gr}$. 
%\begin{enumerate}[label=(\roman*)]
%\item For each key $\key$, $\mkvs_{\Gr}(\key, 0) = (\val_0, \txid_0, \_)$. 
%By construction, we have that $\mkvs_{\Gr}(\key, 0) = \ver_{\key}^{0} = (\val_0, \txid_0, \_)$. 
%\item $\fora{ \key \in \Keys, i,j: 0 \leq i, j < \abs{ \mkvs_{\Gr}(\key) } }
%\wtOf[\mkvs_{\Gr}(\key, i)] = \wtOf[\mkvs_{\Gr}(\key, j)] \implies i = j$.
%Let $\key \in \Keys$, and let $i, j: 0 \leq i,j < \lvert \mkvs_{\Gr}(\key) \rvert$ 
%be such that $\wtOf(\mkvs_{\Gr}(\key, i)) = \wtOf(\mkvs_{\Gr}(\key, j))$. 
%Without loss of generality, we can assume that $i \leq j$. 
%First, note that if $i = 0$, then $\wtOf(\mkvs_{\Gr}(\key, i)) = \txid_0$, 
%hence it must be the case that $\wtOf(\mkvs_{\Gr}(\key, j)) = \txid_0$. 
%By construction, it is also the case that $\mkvs_{\Gr}(\key, j) = \ver_{\key}^{j}$, 
%hence either one of the following is true: 
%\begin{enumerate}
%\item $j = 0$, in which case there is nothing to prove, or 
%\item $j > 0$, and $\mkvs_{\Gr}(\key, j) = \ver_{\key}^{j} = 
%\ver(\txid, \key)$ for some $\txid \in \dom(\TtoOp{T})$. 
%We have that $\wtOf(\mkvs_{\Gr}(\key, j) = \wtOf(\ver(\txid, \key)) = \txid$, 
%and because $\txid \in \dom(\TtoOp{T})$, it must be $\txid \neq \txid_0$. 
%Contradiction.
%\end{enumerate}
%Suppose then that $i > 0$. Therefore, it must be the case that $\mkvs_{\Gr}(\key, i) = 
%\ver_{\key}^{i} = \ver(\txid, \key)$ for some $\txid \in \dom(\TtoOp{T})$ such that 
%$(\otW, \key, \_) \in \TtoOp{T}(\txid_{i})$. Similarly, because we are assuming 
%that $i \leq j$, we have that $\mkvs_{\Gr}(\key, j) = \ver_{\key}^{j} = \ver(\txid, \key)$. 
%Note that $\mkvs_{\Gr}(\key, i) = \mkvs_{\Gr}(\key, j)$. Finally, note that if it were 
%$i < j$, then by construction we should have that $\txid \toEDGE{\WW(\key)} \txid$, 
%contradicting the requirement that $\WW(\key)$ is irreflexive. Therefore, it must 
%be $i = j$. 
%\item $\fora{ \key \in \Keys,i,j: 0 \leq i, j < \abs{ \mkvs_{\Gr}(\key) } }
%\rsOf[\mkvs_{\Gr}(\key, i)] \cap \rsOf[\mkvs_{\Gr}(\key, j)] \neq \emptyset \implies i = j$. 
%Let $\key \in \Keys$, $i, j: 0 \leq i, j < \lvert \mkvs_{\Gr}(\key) \rvert$, 
%and $\txid \in \rsOf(\mkvs_{\Gr}(\key, i)) \cap \rsOf(\mkvs_{\Gr}(\key, j))$. Without loss 
%of generality, suppose that $i \leq j$. We distinguish between two cases: 
%\begin{enumerate}
%\item $i = 0$; by construction, there exists no $\txid'$ such that 
%$\txid' \toEDGE{\WR(\key)} \txid$. If it were $j > 0$, then it 
%would be the case that $\mkvs_{\Gr}(\key, j) = \ver(\txid', \key)$ for some 
%$\txid'$ such that $\txid' \toEDGE{\WR(\key)} \txid$; because 
%such transaction $\txid'$ does not exist, it cannot be $j > 0$, and 
%we are left with the case $j = 0$; in particular, $j = i$. 
%\item $i > 0$; by construction, it must be the case that $\mkvs_{\Gr}(\key, i) = 
%\ver(\txid', \key)$ for some $\txid'$ such that $\txid' \toEDGE{\WR(\key)} \txid$. 
%Furthermore, because we are assuming that $i \leq j$, we also have that $j > 0$, 
%and  therefore $\mkvs_{\Gr}(\key, j) = \ver(\txid'', \key)$ for some $\txid''$ such that 
%$\txid'' \toEDGE{\WR(\key)} \txid$. We have that $\txid' \toEDGE{\WR(\key)} \txid$, 
%and $\txid'' \toEDGE{\WR(\key)} \txid$. By definition of dependency graph, this implies 
%that $\txid' = \txid''$. We have that $\wtOf(\mkvs_{\Gr}(\key, i)) = \txid'$, 
%$\wtOf(\mkvs_{\Gr}(\key, j)) = \txid''$, and $\txid' = \txid''$; if it were $i < j$, 
%then by construction we would have that $\txid' \toEDGE{\WW(\key)} \txid'$, 
%contradicting the requirement of dependency graphs that $\WW(\key)$ is irreflexive. 
%Therefore, it must be the case that $i = j$.
%\end{enumerate}
%\item Suppose the following holds:
%\[
%\begin{array}{l}
%\fora{ \key \in \dom(\mkvs), \cl \in \Clients} \fora{ i,j; 0 \leq i < j < \lvert \mkvs_{\Gr}(\key) \rvert}
%\fora{ n, m \geq 0}\\
%\quad (\txid_{\cl}^{n} = \wtOf(\mkvs_{\Gr}(\key,i)) \land \txid_{\cl}^{m} \in \Set{\wtOf(\mkvs_{\Gr}(\key,j))} \cup \rsOf(\mkvs_{\Gr}(\key, i)) \implies n < m
%\end{array}
%\]
%Let $\key \in \Keys$, $\cl \in \Clients$, $i, j: 0 \leq i < j < \lvert \mkvs_{\Gr}(\key) \rvert$. Let also $n, m \geq 0$. 
%First, suppose that $\txid_{\cl}^{n} = \wtOf(\mkvs_{\Gr}(\key, i)$.
%Note that it cannot be $i = 0$, because by construction $\wtOf(\mkvs_{\Gr}(\key, i)) = \txid_0 \neq \txid_{\cl}^{n}$. 
%Therefore, it must be $i > 0$. We prove the following facts: 
%\begin{enumerate}
%\item if $\txid_{\cl}^{m} = \wtOf(\mkvs_{\Gr}(\key, j))$, then $n < m$. By construction, 
%$\mkvs_{\Gr}(\key, i) = \ver(\txid_{\cl}^{n}, \key)$, and $(\otW, \key, \_) \in \TtoOp{T}(\txid_{\cl}^{n})$. 
%Similarly, $\mkvs_{\Gr}(\key, j) = \ver(\txid_{\cl}^{m}, \key)$, and $(\otW, \key, \_) \in \TtoOp{T}(\txid_{\cl}^{m})$. 
%Because $i < j$, it must be the case that $\txid_{\cl}^{n} = \wtOf(\ver(\txid_{\cl}^{n}, \key) \toEDGE{\WW(\key)} 
%\wtOf(\ver(\txid_{\cl}^{m}, \key)) = \txid_{\cl}^{m}, \key)$, and by definition of dependency graph it follows that 
%$n < m$, 
%\item if $\txid_{\cl}^{m} \in \rsOf(\mkvs_{\Gr}(\key, i))$, then $n < m$. In this case we have that 
%$\txid_{\cl}^{n} \toEDGE{\WR(\key)} \txid_{\cl}^{m}$ by construction, hence the definition 
%of dependency graph ensures that $n < m$. 
%\end{enumerate}
%\end{enumerate}
%\end{proof}
%
%\begin{proposition}
%    \label{prop:bijection:mkvs-dgraph}
%For any kv-store $\mkvs$, $\mkvs_{\Gr_{\mkvs}} = \mkvs$.
%\end{proposition}
%\begin{proof}
%The conversions from kv-store to dependency graph (\cref{def:kv2graph}) and vice versa (\cref{def:dependency-to-kv-store}) are based per key.
%Those conversions are well-formed by \cref{prop:well-formed-kv-store-to-dependency} and \cref{prop:well-formed-dependency-to-kv-store}.
%It is sufficient to fix a key \( \key \) and to prove \( \mkvs_{\Gr_{\mkvs}}(\key) = \mkvs(\key) \).
%We prove \( \mkvs_{\Gr_{\mkvs}}(\key) = \mkvs(\key) \) by induction on the length of \( \mkvs(\key) \).
%
%\begin{itemize}
%    \item \caseB{\abs{\mkvs(\key)} = 1}
%Let \( \mkvs(\key) = (\val_0, \txid_0, \txidset_0 ) \) for some transactions \( \txidset_0 \) that read the initial value \( \val_0 \).
%Given the definition of \( \Gr_\mkvs \) (\cref{def:kv2graph}), we know that \( (\otR, \key, \val_0 ) \in \TtoOp{T}(\txid) \) for all \( \txid \in \txidset_0 \) and \( \WW(\key) = \WR(\key) = \RW(\key) = \emptyset  \).
%Given the definition of \( \mkvs_{\Gr_\mkvs}\) (\cref{def:dependency-to-kv-store}), it is easy to see \( \mkvs_{\Gr_\mkvs}(\key) = \mkvs(\key) \).
%
%    \item \caseI{\abs{\mkvs(\key)}= m + 1 }
%Suppose \( \mkvs_{\Gr_\mkvs}(\key) = \mkvs(\key) \) when \( |\mkvs(\key)| = m \) and let consider  \( |\mkvs(\key)| = m + 1 \).
%Let 
%\[
%\mkvs(\key) = (\val_0, \txid_0, \txidset_0 ) \lcat \dots \lcat (\val_m, \txid_m, \txidset_m ) \lcat (\val_{m+1}, \txid_{m+1}, \txidset_{m+1} ) 
%\]
%We now discuss the \( \WW(\key) \), \( \WR(\key) \) and \( \RW(\key) \) relations in \( \Gr_\mkvs(\key) \) and the corresponding versions in \( \mkvs_{\Gr_\mkvs}(\key) \).
%\begin{itemize}
%    \item For any \( (\txid, \txid') \in \WW(\key) \), there are two cases: \( \txid' \neq \txid_{m+1} \) and \( \txid' = \txid_{m+1} \).
%    If \( \txid' \neq \txid_{m+1} \), then \( \txid = \txid_i \) and \( \txid = \txid_j \) for some \( i \) and \( j \) such that \( 0 < i < j < m + 1 \) by the definition of \( \Gr_\mkvs \) (\cref{def:kv2graph}).
%    By the \ih, we have \( \wtOf(\mkvs_{\Gr_\mkvs}(\key,i))  = \txid_i \) and \( \wtOf(\mkvs_{\Gr_\mkvs}(\key,i))  = \txid_j \).
%    If \( \txid' = \txid_{m + 1} \), then \( \txid = \txid_i \) for some \( i \) such that \( 0 \leq i < m + 1 \).
%    By the definition of  \( \mkvs_{\Gr_\mkvs}(\key) \) (\cref{def:dependency-to-kv-store}), the order of versions is the same as the order of \( \RW(\key) \).
%    That means the version \( (\val_{m+1}, \txid_{m+1}, \stub ) \) is the last one, \ie (m + 1)-\emph{th}, in the \( \mkvs_{\Gr_\mkvs}(\key) \).
%    Combine the two cases above, we know:
%    \begin{equation}
%        \label{equ:ww-back-to-ww}
%        \fora{i : 0 \leq i \leq m + 1} \exsts{\txidset} \mkvs_{\Gr_\mkvs}(\key, i) = (\val_i, \txid_i, \txidset)
%    \end{equation}
%    
%    \item For any \( (\txid, \txid' ) \in \WR(\key) \).
%    Assume \( \txid = \txid_i \) for some \( i \) that \( 0 < i \leq m + 1\).
%    Given the definition of \( \Gr_\mkvs \) (\cref{def:kv2graph}), it muse be that \( \txid' \in \txidset_i \)
%    By the definition of  \( \mkvs_{\Gr_\mkvs}(\key) \) and \cref{equ:ww-back-to-ww}, it follows:
%    \begin{equation}
%        \label{equ:k-to-kgk}
%        \fora{i : 0 \leq i \leq m + 1} \mkvs_{\Gr_\mkvs}(\key, i) = (\val_i, \txid_i, \txidset_i)
%    \end{equation}
%\end{itemize}
%\end{itemize}
%The \cref{equ:k-to-kgk} implies \( \mkvs(\key) = \mkvs_{\Gr_\mkvs}(\key) \) and then \( \mkvs = \mkvs_{\Gr_\mkvs} \).
%\end{proof}

% abstract execution semantics
\section{Operational Semantics of Abstract Executions}
\subsection{Operational Semantics of Abstract Executions}

Our solution requires defining an alternative semantics of programs under weak consistency models, 
based on abstract executions.
The operational semantics we propose is parametrised in the axiomatic specification $(\RP, \Ax)$ of a consistency model:
transitions take the form $(\aexec, \Env, \prog) \toA{\_}_{(\RP, \Ax)} (\aexec', \Env', \prog')$. 

\begin{figure}[t]
\[
\begin{rclarray}
	\toA{}  & : &
    \begin{array}[t]{@{}c@{}}
    \ClientID \; \times \;
	\left( ( \Aexecs \times \Stacks ) \times \Commands \right)  \\
    \; \times\; \Como \; \times \; \sort{Label} \;\times 
	\left( ( \Aexecs \times \Stacks ) \times \Commands \right) 
    \end{array}
\end{rclarray}
\]
\begin{mathpar}
    \inferrule[\rl{ACommit}]{
        \T \subseteq \T_{\aexec} \qquad \h \in \RP(\aexec, \T) \qquad
		(\stk, \h, \emptyset), \trans \ \toL^{*} \  (\stk', \stub,  \opset) , \pskip \\\\
		\txid \in \nextTxId(\T_{\aexec}, \cl) \qquad \aexec' = \extend(\aexec, \txid, \T, \opset) \qquad 
		\forall A \in \Ax.\;\{\txid' \mid (\txid', \txid) \in \A(\aexec') \} \subseteq \T
    }{
    \cl \vdash ( \aexec, \stk ), \ptrans{\trans} \ \toA{(\cl, \T,\f)}_{(\RP, \Ax)} \ ( \aexec', \stk' ) , \pskip
    }
    \and
    \inferrule[\rl{APrimitive}]{
        \stk \toLTS{\cmdpri} \stk'
    }{%
    \cl \vdash ( \aexec, \stk ) , \cmdpri \ \toA{(\cl,\iota)}_{\como} \  ( \aexec, \stk' ) , \pskip
    }
    \and
    \inferrule[\rl{AChoice}]{
        i \in \Set{1,2}
    }{%
        \cl \vdash ( \aexec, \stk ) , \cmd_{1} \pchoice \cmd_{2} \ \toA{(\cl,\iota)}_{\como} \  ( \aexec, \stk ) , \cmd_{i}
    }
    \quad
    \inferrule[\rl{AIter}]{ }{%
        \cl \vdash ( \aexec, \stk ) , \cmd\prepeat \ \toA{(\cl,\iota)}_{\como} \  ( \aexec, \stk ) , \pskip \pchoice (\cmd \pseq \cmd\prepeat)
    }
    \and
    \inferrule[\rl{ASeqSkip}]{ }{%
        \cl \vdash ( \aexec, \stk ) , \pskip \pseq \cmd \ \toA{(\cl,\iota)}_{\como} \  ( \aexec, \stk ) , \cmd
    }
    \quad
    \inferrule[\rl{ASeq}]{% 
        \cl \vdash ( \aexec, \stk ) , \cmd_{1} \ \toA{(\cl,\iota)}_{\como} \  ( \aexec, \stk' ) , {\cmd_{1}}' 
    }{%
        \cl \vdash ( \aexec,\stk ) , \cmd_{1} \pseq \cmd_{2} \ \toA{(\cl,\iota)}_{\como} \ ( \aexec, \stk' ) , {\cmd_{1}}' \pseq \cmd_{2}
    }
\end{mathpar}

\hrulefill

\[
	\toA{} : 
    ( \Aexecs \times \ThdEnv \times \Programs) 
    \;\times\; \Como \; \times \sort{Label} \times \;
    ( \Aexecs \times \ThdEnv \times \Programs) 
\]
\[
    \inferrule[\rl{ASingleThread}]{%
         \cl \vdash ( \aexec, \thdenv(\thid) ) , \prog(\thid), \ \toA{\lambda}_{(\RP, \Ax)} \  ( \aexec', \stk' ) , \cmd'  
    }{%
         (\aexec, \thdenv ), \prog  \ \toA{\lambda}_{(\RP, \Ax)} \  ( \aexec', \thdenv\rmto{\thid}{\stk'} ) , \prog\rmto{\thid}{\cmd'} ) 
    }
\]
\hrulefill
\caption{Operational Semantics on Abstract Executions}
\label{fig:aexec.semantics}
\end{figure}

In \cref{fig:aexec.semantics} we all rules of the operational semantics of programs based on abstract executions. 
Rule \rl{ACommit} is the abstact execution counterpart of rule \rl{PCommit} for kv-stores, 
in that it models how an abstract execution $\aexec$ evolves when a client wants to execute a transaction whose code is $\ptrans{\trans}$. 
In this rule, $\T$ is the set of transactions of $\aexec$ that are visible to the client $\cl$ that wishes to execute $\ptrans{\trans}$.
Such a set of transactions is used to determine a snapshot $\h \in \RP(\aexec, \T)$ that 
the client $\cl$ uses to execute the code $\ptrans{\trans}$, and obtain a fingerprint $\opset$. 
This fingerprint is then used to extend abstract execution $\aexec$ with a transaction from the set $\nextTxId(\T_{\aexec}, \cl)$.
Another rule in \cref{fig:aexec.semantics} is Rule \rl{ASinglethread}; 
the structure of this rule is analogous to \rl{PSingleThread}, and it models multi-thread concurrency in an interleaving fashion. 
All the rest rules of the abstract operational semantics in \cref{fig:aexec.semantics} have a similar counterpart in the key-value store semantics.

In some sense that is going to be made mathematically precise later, Rule \rl{ACommit} is more general 
than Rule \rl{Pcommit} in the kv-store semantics. In the latter, the snapshot of a transaction is uniquely 
determined from a view of the client, in a way that roughly corresponds to the last write wins policy 
in the abstract execution framework. In contrast, in Rule \rl{ACommit} the snapshot of a transaction 
is chosen non-deterministically from those made available to the client by the resolution policy 
$\RP$ adopted by a weak consistency model, which may not necessarily be $\RP_{\LWW}$. 
One example of resolution policy that we will use in this Section is given by the anarchic resolution policy. 


\subsection{Anarchic Model}

To justify this semantics capture all the possible abstract executions under certain consistency model,
we introduce anarchic model, and by \cref{thm:consistency-intersect-anarchic},
we are confident that the operational semantics capture all possible behaviours.


\begin{definition}
The anarchic resolution policy $\RP_{\anarchic}$ is defined by letting, 
$\RP_{\anarchic}(\_, \_) = \Snapshots$. The \emph{anarchic consistency model} is 
specified axiomatically by the pair $\anarchicCM = (\RP_{\anarchic}, \emptyset)$.
\end{definition}

\begin{example}
Suppose that we want to execute the single-threaded program $\prog$ that maps client 
$\cl$ to the transactional code below:
\[
\begin{session}
%\ptrans{\pmutate{\ke}{\val_2}}; \\
\ptrans{\pderef{\pvar{a}}{\ke}; \\
\pifs{\pvar{a} = \val_{1}} \pmutate{\ke'}{\val_1} \pife}
\end{session}
\]
Suppose that the program is executed under a consistency model that adopts the last write 
wins resolution policy $\RP_{\LWW}$, and with no additional axioms. Then the behaviour of $\prog$ is 
completely deterministic (up-to the choice of transaction identifiers), and the execution of $\prog$ terminates in a 
state corresponding to the abstract execution below: 

\begin{center}
\begin{tikzpicture}[scale=0.85, every node/.style={transform shape}]

\node(t0rx) at (-1,2) {$(\otR, \ke, \val_0)$}; 
%\path (t0wx.south) + (0,-0.2) node[anchor=north] (t0wy) {$(\otW, \ke_2, \val'_0)$};

\begin{pgfonlayer}{background}
\node[background, fit=(t0rx)]  {};

\path(t0.west) node[anchor=east] (t0lbl) {$\txid_{\cl}^{\_}$};
%\path(t1.north) node[anchor=south] (t1lbl) {$\txid_1$};
%\path(t2.south) node[anchor=north] (t2lbl) {$\txid_2$};

%\path[->]
%(t0.north) edge[bend left=70] node[above, yshift=7pt, xshift=-1pt, pos=0.3] {$\RF(\ke_2), \VO(\ke_1)$} (t1.west)
%(t0.south) edge[bend right=70] node[below, yshift=-8pt, xshift=-1pt, pos=0.3] {$\RF(\ke_1), \VO(\ke_2)$} (t2.west)
%([xshift=-8pt]t2.north) edge[bend left=40] node[left] {$\AD(\ke_1)$} ([xshift=-8pt]t1.south) 
%([xshift=8pt]t1.south) edge[bend left=40] node[right] {$\AD(\ke_2)$} ([xshift=8pt]t2.north);
\end{pgfonlayer}
\end{tikzpicture}
\end{center}

However, if we replace the consistency model specification $(\RP_{\LWW}, \emptyset)$ with the 
anarchic one $\anarchicCM$. Because the snapshot in which client $\cl$ executes the 
transactional code above is chosen non-deterministically, 
the program $\prog$ exhibits infinitely many additional behaviours. 
In particular, the program may now terminate in a state corresponding 
to the abstract execution below: 

\begin{center}
\begin{tikzpicture}[scale=0.85, every node/.style={transform shape}]

\node(t0rx) at (-1,2) {$(\otR, \ke, \val_1)$}; 
\path (t0wx.east) + (0,0.2) node[anchor=west] (t0wy) {$(\otW, \ke', \val_1)$};

\begin{pgfonlayer}{background}
\node[background, fit=(t0rx) (t0wy)]  {};

\path(t0.west) node[anchor=east] (t0lbl) {$\txid_{\cl}^{\_}$};
%\path(t1.north) node[anchor=south] (t1lbl) {$\txid_1$};
%\path(t2.south) node[anchor=north] (t2lbl) {$\txid_2$};

%\path[->]
%(t0.north) edge[bend left=70] node[above, yshift=7pt, xshift=-1pt, pos=0.3] {$\RF(\ke_2), \VO(\ke_1)$} (t1.west)
%(t0.south) edge[bend right=70] node[below, yshift=-8pt, xshift=-1pt, pos=0.3] {$\RF(\ke_1), \VO(\ke_2)$} (t2.west)
%([xshift=-8pt]t2.north) edge[bend left=40] node[left] {$\AD(\ke_1)$} ([xshift=-8pt]t1.south) 
%([xshift=8pt]t1.south) edge[bend left=40] node[right] {$\AD(\ke_2)$} ([xshift=8pt]t2.north);
\end{pgfonlayer}
\end{tikzpicture}
\end{center}

It is important to note, however, that the set of abstract executions generated by $\prog$ is still bound 
to the structure of the program itself. For example, executing $\prog$ under the anarchic execution model 
will never lead to an abstract execution with multiple transactions, or to an abstract execution where a transaction 
writes a key other than $\ke'$ is written.
\end{example}

\begin{definition}
The semantics of a program $\prog$ under a consistency model with axiomatic specification 
$(\RP, \Ax)$ is given by 
\[
\interpr{\prog}_{(\RP, \Ax)} = \{ \aexec \mid (\aexec_{0}, \Env_{0}, \prog) \toA{\_}_{(\RP, \Ax)}^{\ast} (\aexec, \_, \prog_{f}) \}, 
\]
where $\Env_{0} = \lambda \cl \in \dom(\prog).\lambda \pvar{x}.0$ and $\prog_{f} = \lambda \cl \in \dom(\prog).\pskip$.
\end{definition}

We define the set of all the possible behaviours of a program $\prog$ to be $\interpr{\prog}_{\anarchic}$. 
The following result supports our claim that this definition is indeed accurate: 


\begin{example}
One may argue that the axiomatic specification $\anarchicCM$ does not 
truly represent an anarchic consistency model. Consider for example the single-threaded 
program $\prog'$ that associates to a client $\cl$ the following code:
\[
\begin{session}
\ptrans{
\pderef{\pvar{a}}{\ke}; \\
\pderef{\pvar{b}}{\ke};\\
\pifs{\pvar{a} != \pvar{b}} \pmutate{\ke'}{\val_1} \pife}
\end{session}
\]
One would expect that, under a truly anarchic consistency model, it would be possible 
for program $\prog'$ to write the value $\val_1$ for key $\ke'$. However, 
this never happens if $\prog'$ is executed under $\anarchicCM$. This is because 
we embedded into abstract execution the assumption that transactions only read 
at most one value for each key. 

In theory, we could lift this limitation and still retain 
the validity of all the results contained in this report; however, doing so would 
require to work with mathematical structures that are far more complex than 
abstract executions, and we preferred to avoid this issue. 
Furthermore, the constraint that an object is never read twice in transactions is enforced 
at client side in virtually all the implementations  
of libraries for accessing kv-stores. When a client first requests to fetch 
the value of some key $\ke$ within a transaction, a local copy of the value fetched is 
saved on the client (typically in an object containing the meta-data of the transaction); 
if a request to read the same key is performed again within the same client, the local 
copy of the value previously fetched for that key is returned, instead of issuing a second 
read request to the kv-store.
\end{example}

As explained above, the set of all possible behaviours exhibited by a program $\prog$ under a 
consistency model $(\RP, \Ax)$ can be defined by intersecting the set of executions 
that $\prog$ exhibits under the anarchic consistency model, with the set of all executions 
allowed by the axiomatic specification $(\RP, \Ax)$. As the next theorem shows, 
this is exactly the set of abstract executions in which $\prog$ terminates, 
when executed under the axiomatic specification $(\RP, \Ax)$.

\begin{theorem}
\label{thm:consistency-intersect-anarchic}
For any program $\prog$ and axiomatic specification $(\RP, \Ax)$:
\[
\interpr{\prog}_{(\RP, \Ax)} = \interpr{\prog}_{\anarchic} \cap \CMa(\RP, \Ax)
\]
\end{theorem}
\begin{proof}
    It is easy to see \( \interpr{\prog}_{(\RP, \Ax)} \subseteq \CMa(\RP, \Ax) \) so that 
    \( \interpr{\prog}_{(\RP, \Ax)} \subseteq \interpr{\prog}_{\anarchic} \cap \CMa(\RP, \Ax) \).
    Given \cref{prop:aexec-semantics-mono}, we know \( \interpr{\prog}_{\anarchic} \subseteq \interpr{\prog}_{(\RP, \Ax)} \).
    Then, by the definition of \( \interpr{\prog}_{\stub} \), it follows
    \( \interpr{\prog}_{\anarchic} \cap \CMa(\RP, \Ax) \subseteq \interpr{\prog}_{(\RP, \Ax)} \).
\end{proof}


\begin{proposition}
\label{prop:aexec-semantics-mono}
Let define \( (\RP_1, \Ax_1) \sqsubseteq (\RP_2, \Ax_2) \) as the following:
\[
    \fora{\aexec,\T} \RP_2{\aexec,\T} \subseteq  \RP_1{\aexec,\T}
\]
and,
\[
    \fora{\aexec} \bigcup\limits{\A_1 \in \Ax_1}\A_1(\aexec) \subseteq \bigcup\limits{\A_2 \in \Ax_2}\A_2(\aexec)
\]
then
\[ \interpr{\prog}_{(\RP, \Ax)} \subseteq \interpr{\prog}_{\anarchic} \]
\end{proposition}
\begin{proof}
    Since \( \anarchic \sqsubseteq (\RP, \Ax) \),
    We prove stronger result that for \( (\RP_1, \Ax_1) \sqsubseteq  (\RP_2, \Ax_2)\),
    the following  hold:
    \[
        \begin{array}{@{}l@{}}
            \fora{\cl, \aexec, \aexec', \prog, \prog', \stk, \stk'}  \\
            \quad \cl \vdash ( \aexec, \stk ), \prog \toA{\stub}_{(\RP_1, \Ax_1)} ( \aexec', \stk' ), \prog' \\
            \quad \implies \cl \vdash ( \aexec, \stk ), \prog \toA{\stub}_{(\RP_2, \Ax_2)} ( \aexec', \stk' ), \prog' \\
        \end{array}
    \]
    We prove it by induction on the derivations.
    The only interesting case is the \rl{PCommit} rule.
    Given an initial runtime abstract execution \( \aexec \),
    a set of observable transactions \( \T \),
    a new transaction identifier \( \txid \),
    by the \rl{ACommit} rule, it is sufficient to prove, 
    first, all the snapshot under the stronger consistency model is also a valid snapshot under the weaker one:
    \begin{equation}
        \label{equ:obs-state-included}
        \fora{\aexec,\T} \RP_2{\aexec,\T} \subseteq  \RP_1{\aexec,\T}
    \end{equation}
    and second if it is valid  to commit a new transition with the observable set \( \T \) under stronger consistency model,
    it is able to do so under weaker consistency model:
    \begin{equation}
        \label{equ:consis-both-exist}
        \bigcup\limits{\A_1 \in \Ax_1}\A_1(\aexec')^{-1}(\txid) \subseteq \bigcup\limits{\A_2 \in \Ax_2}\A_2(\aexec')^{-1}(\txid)
    \end{equation}
    The \cref{equ:obs-state-included,equ:consis-both-exist} can be proven by \( (\RP_1, \Ax_1) \sqsubseteq  (\RP_2, \Ax_2) \).

    \caseB{\rl{PAssign}, \rl{PAssume}, \rl{PChoice}, \rl{PLoop}, \rl{PSeqSkipS}, \rl{PPar}, \rl{PWait}}
    These base cases do not depend on the consistency model, so they trivial hold because of the hypothesis.
    \caseI{\rl{PSeq}}
    It is proved directly by applying the \ih
\end{proof}

% abstract execution relates kv
\section{Relationship between kv-stores and abstract execution}
\subsection{Key-value Store to Abstract Executions}
\label{app:aexec2kv}
\label{sec:thm:aexec2kv-compatible-proof}

We introduce the definition of the dependency graph induced an abstract execution:

\begin{definition}
\label{def:aexec2graph}
Given an abstract execution $\aexec$ that satisfies the last write wins policy,
the dependency graph $\graphof(\aexec) \defeq (\TtoOp{T}_{\aexec}, \RF_{\aexec}, 
\VO_{\aexec}, \AD_{\aexec})$ is defined by letting
\begin{itemize}
\item $\txid \xrightarrow{\RF_{\aexec}(\ke)} \txid'$ if and only if 
$\txid = \max_{\AR_{\aexec}}(\visibleWrites_{\aexec}(\ke, \txid'))$, 
\item $\txid \xrightarrow{\VO_{\aexec}(\ke)} \txid'$ if and only 
$\txid, \txid' \in_{\aexec} (\otW, \;\ke, \stub)$ 
and $\txid \xrightarrow{\AR_{\aexec}} \txid'$,
\item $\txid \xrightarrow{\AD_{\aexec}(\ke)} \txid'$ if and only if either 
$(\otR, \ke, \stub) \in_{\aexec} \txid, (\otW, \ke, \stub) \in_{\aexec} \txid'$ and 
whenever $\txid'' \xrightarrow{\RF_{\aexec}(\ke)} \txid$, 
then $\txid'' \xrightarrow{\VO_{\aexec}(\ke)} \txid'$.
\end{itemize}
\end{definition}

Note that each abstract execution $\aexec$ determines a key-value store $\hh_{\aexec}$,
as a result of \cref{def:aexec2graph} and \cref{thm:kv2graph}. 
Let $\hh$ be the unique kv-store such that $\Gr_{\hh} = \graphof(\aexec)$, then $\hh_{\aexec} = \hh$. 
As we will discuss later in this Section,
this mapping $\hh_{(\stub)}$ is NOT a bijection, 
in that several abstract executions may be encoded in the same key-value store.
Because key-value stores abstract away the total arbitration order of transactions.

Upon the relation \( \hh_{\aexec} = \hh \),
there is a deeper link between key-value store plus views and abstract exertions.
This notion, named \emph{compatibility}, bases on the intuition that 
clients can make observations over key-value stores and abstract executions, in terms of snapshots.

In key-value stores, observations are snapshots induced by views. 
While in abstract executions, observations correspond to the snapshots induced by the visible transactions.
Note that it is under the condition that the abstract execution adopts $\RP_{\LWW}$ resolution policy.
This approach is analogous to the one used by operation contexts in \cite{repldatatypes}.
Thus, a key-value store $\hh$ is \emph{compatible} with an abstract execution $\aexec$, written \( \mkvs \simeq \aexec \)
if any observation made on $\hh$ can be replicated by an observation made on $\aexec$, and vice-versa. 

\begin{definition}
\label{def:compatible}
Given a key-value store $\hh$,
an abstract execution $\aexec$ is compatible with $\hh$, written 
$\aexec \compatible \hh$, if and only if there exists a  mapping 
$f: \powerset{\T_{\aexec}} \rightarrow \Views(\hh)$
such that  
\begin{itemize}
\item for any subset $\T \subseteq \T_{\aexec}$, then $\RP_{\LWW}(\aexec, \T) = \{\snapshot(\hh, f(\T))\}$; 
\item for any view $\vi \in \Views(\hh)$, there exists a subset $\T \subseteq \T_{\aexec}$ 
such that $f(\T) = \vi$, and $\RP_{\LWW}(\aexec, \T) = \{\snapshot(\hh_{\aexec}, \vi)\}$.
\end{itemize}
\end{definition}

The function $\getView(\aexec, \T)$ defines the view on \( \mkvs_\aexec \) that corresponds to \( \T \) as the following:
\[
    \getView(\aexec, \T) \defeq \lambda \ke. \{0\} \cup \Setcon{ i }{\WTx(\hh_{\aexec}(\ke, i)) \in \T}
\]
Inversely, the function \( \Tx(\hh, \vi) \) converts a view to a set of observable transactions:
\[
\Tx(\hh, \vi) \defeq \Setcon{ \WTx(\hh(\ke, i)) }{ \ke \in \Keys \wedge i \in \vi(\ke) }
\]
Given \( \getView \), \( \Tx \), \cref{def:compatible}, 
it follows \( \aexec \compatible \hh_{\aexec} \) shown in \cref{thm:aexec2kv.compatible}.

\begin{theorem}
\label{thm:aexec2kv.compatible}
For any abstract execution $\aexec$ that satisfies the last write wins policy, $\aexec \compatible \hh_{\aexec}$.
\end{theorem}
\begin{proof}
Given the function $\getView(\aexec, \cdot )$ from $\powerset{\T_{\aexec}}$ to $\Views(\hh_{\aexec})$,
we prove it satisfies the constraint of \cref{def:compatible}.
Fix a set of transitions \( \T \).
By the \cref{prop:getview.valid}, the view $\getView(\aexec, \T )$  on \( \mkvs_\aexec \) is a valid view,
that is, \( \getView(\aexec, \T ) \in \Views(\mkvs_\aexec) \).
Given that it is a valid view, the \cref{prop:compatible.aexec2kv} proves:
\begin{equation}
    \label{equ:visible-trans-to-view}
    \RP_{\LWW}(\aexec, \T) = \Set{\snapshot(\hh_{\aexec}, \getView(\aexec, \T))} 
\end{equation}

The another way round is more subtle,
because \( \T \) contains any read only transaction.
By \cref{prop:getview.tx}, it is safe to erase read only transactions from \( \T \),
when calculating the view \( \getView(\aexec, \T ) \).
Last, by \cref{prop:compatible.kv2aexec}, we prove the following:
\begin{equation}
    \label{equ:view-to-visible-trans}
    \RP_{\LWW}(\aexec, \T) = \snapshot(\hh_{\aexec}, \vi)
\end{equation}
By \cref{equ:visible-trans-to-view} and \cref{equ:view-to-visible-trans},
it follows \( \aexec \compatible \hh_{\aexec} \).
\end{proof}

\begin{proposition}[Valid views]
\label{prop:getview.valid}
For any abstract execution $\aexec$, and $\T \subseteq \T_{\aexec}$, 
$\getView(\aexec, \T) \in \Views(\hh_{\aexec})$.
\end{proposition}
\begin{proof}
Assume an abstract execution $\aexec$, a set of transactions $\T \subseteq \T_{\aexec}$, and a key \( \ke \).
By the definition of $\getView(\aexec, \T)$, 
then $0 \in \getView(\aexec, \T)(\ke)$, and 
$0 \leq i < \abs{ \hh_{\aexec}(\ke) }$ for any index \( i \) such that $i \in \getView(\aexec, \T)(\ke)$.
Therefore we only need to prove that $\getView(\aexec, \T)$ satisfies \eqref{eq:view.atomic}.
Let $j \in \getView(\aexec, \T)(\ke)$ for some key $\ke$, and let $\txid = 
\WTx(\hh_{\aexec}(\ke, j))$. Let also $\ke', i$ be such that 
$\WTx(\hh_{\aexec}(\ke', i)) = \txid$. We need to show that 
$i \in \getView(\aexec, \T)(\ke')$. Note that it $\txid = \txid_{0}$ 
then $\WTx(\hh_{\aexec}(\ke', i)) = \txid$ only if $i = 0$, and 
$0 \in \getView(\aexec, \T)(\ke')$ by definition. 
Let then $\txid \neq \txid_{0}$. Because $\WTx(\hh_{\aexec}(\ke, j)) = \txid$ 
and $j \in \getView(\aexec, \T)$, then it must be the case that $\txid \in \T$. 
Also, because $\WTx(\hh_{\aexec}(\ke', i)) = \txid$, then $(\otW, \ke, \stub) \in 
\TtoOp{T}_{\aexec}(\txid)$. It follows that there exists an index $i' \in \getView(\aexec, \txid)(\ke')$ 
such that $\WTx(\hh_{\aexec}(\ke', i')) = \txid$. By definition of 
$\hh_{\aexec}$, if $\WTx(\hh_{\aexec}(\ke', i')) = \txid$, then it must 
be $i' = i$, and therefore $i \in \getView(\aexec, \txid)(\ke')$.
\end{proof}


\begin{proposition}[Visible transactions to views]
\label{prop:compatible.aexec2kv}
For any subset $\T \subseteq \T_{\aexec}$, $\RP_{\LWW}(\aexec, \T) = \{\snapshot(\hh_{\aexec}, \getView(\aexec, \T))\}$.
\end{proposition}

\begin{proof}
Fix $\T \subseteq \aexec$, and let $\{\hh\} = \RP_{\LWW}(\aexec, \T)$. We prove that, for any $\ke \in \Keys$, 
$\hh(\ke) = \snapshot(\getView(\aexec, \T))(\ke)$. There are two different cases: 
\begin{enumerate}
\item $\T \cap \{ \txid \mid (\otW, \ke, \stub) \in_{\aexec} \txid \} = \emptyset$. 
In this case $\hh(\ke) = \val_0$. 
We know that $\graphof(\aexec)$ satisfies all the constraints required by the definition of dependency graph 
(\cite{laws}). Together with \cref{thm:kv2graph} it follows that $\hh_{\aexec}(\ke, 0) = (\val_0, \txid_0, \stub)$.
We prove that $\getView(\aexec, \T)(\ke) = \{0\}$, 
hence 
\[ 
\snapshot(\hh_{\aexec}, \getView(\aexec, \T))(\ke) = \valueOf(\hh_{\aexec}(\ke, 0)) = \val_{0}
\]
Note that whenever $(\otW, \ke, \stub) \in_{\aexec} \txid$ for some $\txid$, then 
$\txid \notin \T$. Therefore, whenever $(\val, \txid, \stub) = \hh_{\aexec}(\ke, i)$ for some $i \geq 0$, then 
$\txid \notin \T$.
\[
\getView(\aexec, \T)(\ke) = \{0\} \cup \{i \mid \WTx(\hh_{\aexec}(\ke, i)) \in \T)\} = \{0\} \cup \emptyset = \{0\}
\]
\item Suppose now that $\T \cap \{ \txid \mid (\otW, \ke, \stub) \in_{\aexec} \txid \} \neq \emptyset$. 
Let then $\txid = \max_{\AR_{\aexec}}(\T \cap \{\txid \mid (\otW, \ke, \stub) \in_{\aexec} \txid\})$. 
Then $(\otW, \ke, \val) \in_{\aexec} \txid$ for some $\val \in \Val$. Furthermore, $\RP_{\LWW}(\aexec, \T)(\ke) = \val$.
By definition, $\txid' \in \T \cap \{ \txid \mid (\otW, \ke, \stub) \in_{\aexec} \txid\}$, 
then either $\txid' = \txid$ or $\txid' \xrightarrow{\AR_{\aexec}} \txid$. The definition of 
$\graphof(\aexec)$ gives that $\txid' \xrightarrow{\VO_{\aexec}(\ke)} \txid$. 
Because $(\otW, \ke, \val) \in_{\aexec} \txid$, then there exists an index 
$i \geq 0$ such that $\hh_{\aexec}(\ke, i) = (\val, \txid, \stub)$. Furthermore, 
whenever $\WTx(\ke, j) = \txid'$ for some $\txid'$ and $j > i$, then it must 
be the case that $\txid \xrightarrow{\VO_{\aexec}(\ke)} \txid'$, and because 
$\VO_{\aexec}(\ke)$ is transitive and irreflexive, it must be that  
$\neg( \txid' \xrightarrow{\VO_{\aexec}(\ke)} \txid)$ and $\txid \neq \txid'$: this implies that 
$\txid' \notin \T$. It follows that $\max(\getView(\aexec, \T)(\ke)) = i$, hence 
$\snapshot(\hh_{\aexec}, \getView(\aexec, \T)) = \valueOf(\hh_{\aexec}(\ke, i)) = \val$.
\end{enumerate}
\end{proof}

\begin{proposition}[Read-only transactions erasing]
\label{prop:getview.tx}
Let $\vi \in \Views(\hh_{\aexec})$, and let $\T \subseteq \T_{\aexec}$ be a 
set of read-only transactions in $\aexec$. Then 
$\getView(\aexec, \T \cup \Tx(\hh_{\aexec}, \vi)) = \vi$. 
\end{proposition}

\begin{proof}
Fix a key $\ke$. Suppose that $i \in \getView(\aexec, \T \cup \Tx(\hh_{\aexec}, \vi))(\ke)$. 
By definition, $\hh_{\aexec}(\ke, j) = (\stub, \txid, \stub)$ for some $\txid \in \T \cup \Tx(\hh_{\aexec}, \vi)$. 
Because $\T$ only contains read-only transactions, by definition of $\hh_{\aexec}$ there exists 
no index $j$ such that $\hh_{\aexec}(\ke, j) = (\stub, \txid', \stub)$ for some $\txid' \in \T$, 
hence it must be the case that $\txid \in \Tx(\hh_{\aexec}, \vi)$. By definition of $\Tx$, 
this is possible only if there exist a key $\ke'$ and an index $j$ such that $\hh_{\aexec}(\ke', \vi) = (\stub, \txid, \stub)$. 
Because $\vi$ is atomic by definition, and because $\hh_{\aexec}(\ke, i) = (\stub, \txid, \stub)$, then we have that $i \in \vi(\ke)$. 

Now suppose that $i \in \vi(\ke)$, and let $\hh_{\aexec}(\ke, i) = (\stub, \txid, \stub)$ for some $\txid$. 
This implies that $(\otW, \ke, \stub) \in_{\aexec} \txid$.
By definition $\txid \in \Tx(\hh_{\aexec}, \vi)$, hence $\txid \in \T \cup \Tx(\hh_{\aexec}, \vi))$. 
Because $\txid \in \T \cup \Tx(\hh_{\aexec}, \vi)$, then for any key $\ke'$ such that 
$(\otW, \ke', \stub) \in_{\aexec} \txid$, there exists an index $j \in \getView(\aexec, \T \cup \Tx(\hh_{\aexec}, \vi))$ 
$\hh(\ke', j) = (\stub, \txid, \stub)$; because kv-stores only allow a transaction to write at most one version 
per key, then the index $j$ is uniquely determined. In particular, we know that $(\otW, \ke, \stub) \in_{\aexec} \txid$, 
and $\hh_{\aexec}(\ke, \i) = (\stub, \txid, \stub)$, from which it follows that $i \in \getView(\aexec, \T \cup \Tx(\hh_{\aexec}, \vi))(\ke)$.
\end{proof}


\begin{proposition}[Views to visible transactions]
\label{prop:compatible.kv2aexec}
Given a view $\vi \in \Views(\hh_{\aexec})$, there exists $\T \subseteq \T_{\aexec}$ 
such that $\getView(\aexec, \T) = \vi$, and $\RP_{\LWW}(\aexec, \T) = \snapshot(\hh_{\aexec}, \vi)$.
\end{proposition}

\begin{proof}
We only need to prove that, for any $\vi \in \Views(\hh_{\aexec})$, there exists $\T \subseteq \T_{\aexec}$ such 
that $\getView(\aexec, \T) = \vi$. Then it follows from \cref{prop:compatible.aexec2kv} that 
$\RP_{\LWW}(\aexec, \T) = \snapshot(\hh_{\aexec}, \vi)$. 
It suffices to choose $\T = \bigcup\limits_{\ke \in \Keys}(\Setcon{ \WTx(\hh_{\aexec}(\ke, i)) }{ i > 0 
\wedge i \in \vi(\ke)})$.
Fix a key $\ke$, and let $i \in \vi(\ke)$. We prove that $i \in \getView(\aexec, \T)$. 
If $i = 0$, then $i \in \getView(\aexec, \T)$ by definition. 
Therefore, assume that $i > 0$. Let $\txid = \WTx(\hh_{\aexec}(\ke, i))$.
It must be the case that $\txid \in \T$ and $i \in \getView(\aexec, \T)(\ke)$.

Next, suppose that $i \in \getView(\aexec, \T)(\ke)$. We prove that $i \in \vi(\ke)$.
Note that if $i = 0$, then $i \in \vi(\ke)$ because of the 
definition of views. Let then $i > 0$. Because $i \in \getView(\aexec, \T)(\ke)$, we have that 
$\WTx(\hh_{\aexec}(\ke, i)) \in \T$.  Let $\txid = \WTx(\hh_{\aexec}(\ke, i))$. Because $i > 0$, 
it must be the case that $\txid \neq \txid_0$.
By definition, $\txid \in \T$ only if there 
exists an index $j$ and key $\ke'$, possibly different from $\ke$, such that $\WTx(\hh_{\aexec}(\ke', j)) = \txid$ and $j \in \vi(\ke')$. 
Because $\txid \neq \txid_0$ we have that $j > 0$. Finally, because $\vi$ is atomic by definition, $j \in \vi(\ke')$
$\WTx(\hh_{\aexec}(\ke', j)) = \txid = \WTx(\hh_{\aexec}(\ke, i))$, then it must be the case 
that $i \in \vi(\ke)$, which concludes the proof.
\end{proof}


\subsection{Key-value Store Traces to Abstract Execution Traces}
\label{sec:kvtrace2aexec}

To prove our specification using execution test on key-value stores 
is sound and complete with respect with the axiomatic specification on abstract execution (\cref{sec:kv-sound-complete-proof}),
we need to prove trace equivalent between these two models.

In this section, we only consider the trace that does not involve \( \prog \) but only committing fingerprint and view shift.
In \cref{sec:kv-sound-complete-proof}, we will go further and discuss the trace installed with \( \prog \).

Similar to \(\anarchic\), let $\ET_\top$ be the most permissive execution test.
That is $\ET_\top \vdash (\hh, \vi) \triangleright \opset: \vi'$ 
such that whenever $\vi(\ke) \neq \vi'(\ke)$ then either $(\otW, \ke, \_) \in \opset$ or $(\otR, \ke, \_) \in \opset$.
%{\color{red} I forgot this last constraint in the latest version of the document, definitions 
%and proofs of theorems that follow must be re-factored to take the constraint into account.}
We will relate $\ET_{\top}$-traces to abstract executions that satisfy the last write wins resolution policy, \ie \( (\RP_\LWW, \emptyset) \).

To bridge $\ET_{\top}$-traces to abstract executions, 
The \aeset(\tr) function converse the trace of \( \ET_\top \) to set of possible abstract executions (\cref{def:kvtrace2aexec}).
In fact, for any trace \( \tr \) and abstract execution $\aexec \in \aeset(\tr)$, 
the last configuration of $\tr$ is $(\hh_{\aexec}, \_)$ (\cref{prop:kvtrace2aexec}).
We often use \( \aexec_\tr \) for \( \aexec \in \aeset(\tr) \).

\begin{definition}
\label{def:kvtrace2aexec}
Given a key-value store $\hh$, a view $\vi$, 
an initial abstract execution $\aexec_0 = ( [ ], \emptyset, \emptyset)$, 
an abstract execution $\aexec$, a set of transactions  
$\T \subseteq \T_{\aexec}$, a transaction identifier $\txid$ and a set of operations $\opset$,
the \( \extend \)  function defined as the follows:
\[
\begin{rclarray}
\extend(\aexec, \txid, \T, \opset) & \defeq &
\begin{cases}
\text{undefined} & \text{if }  \txid = \txid_{0}\\
\left(\TtoOp{T}_{\aexec} \uplus \Set{\txid \mapsto \opset}, \VIS', \AR' \right) & \text{if } \dagger \\
\end{cases} \\
\dagger & \equiv &  
\begin{array}[t]{@{}l}
\txid = \txid_{\cl}^{n}
\wedge \VIS' = \VIS_{\aexec} \uplus \Setcon{(\txid', \txid) }{ \txid \in \T}  \\
{} \wedge \AR' = \AR_{\aexec} \uplus \Setcon{(\txid', \txid) }{ \txid' \in \T_{\aexec}}
\end{array}
\end{rclarray}
\]
Given a $\ET_{\top}$ trace $\tr$, let $\lastConf(\tr)$ be the last configuration appearing in $\tr$.
The set of abstract executions $\aeset(\tr)$ is defined as the smallest set such that:
\begin{itemize}
\item $\aexec_{0} \in \aeset((\hh_{0}, \viewFun_{0}))$, 
\item if $\aexec \in \aeset(\tr)$, then $\aexec \in \aeset\left(\tr \xrightarrowtriangle{(\cl, \varepsilon)}_{\ET_{\top}} (\hh, \viewFun) \right)$, 
\item if $\aexec \in \aeset(\tr)$, then $\aexec \in \aeset\left(\tr \xrightarrowtriangle{(\cl, \emptyset)}_{\ET_{\top}} (\hh, \viewFun) \right)$, 
\item 
    let $(\hh', \viewFun') = \lastConf(\tr)$; 
    if $\aexec \in \aeset(\tr)$, $\opset \neq \emptyset$,
    and $\T = \Tx(\hh, \viewFun'(\cl)) \cup \T_\rd$ where \( \T_\rd \) is a set of \emph{read-only transactions}
    such that $(\otW, \ke, \val) \notin_{\aexec} \txid'$ for all keys \( \ke \) and values \( \val \) and transactions \( \txid' \in \T_\rd\),
    and if the transaction $\txid$ is the transaction appearing in $\lastConf(\tr)$ but not in $\hh$, 
    then $\extend(\aexec, \txid, \T, \opset) \in \aeset\left(\tr \xrightarrowtriangle{(\cl, \opset)}_{\ET_{\top}} (\hh, \viewFun) \right)$.
\end{itemize}
\end{definition}

\begin{proposition}[Trace of \( \ET \) to abstract executions]
\label{prop:kvtrace2aexec}
For any $\ET_{\top}$-trace $\tr$, 
the abstract execution $\aexec \in \aeset(\tr)$ satisfies the last write wins policy,
and $(\hh_{\aexec}, \stub) = \lastConf(\tr)$.
\end{proposition}
\begin{proof}
Fix a $\ET_{\top}$-trace $\tau$. 
We prove by induction on the number of transitions $n$ in $\tau$. 
\begin{itemize}
\item \caseB{$n = 0$}
It means $\tr = (\hh_{0}, \_)$.
It follows from \cref{def:kvtrace2aexec} that $\aexec_{\tau} = ([], \emptyset, \emptyset)$. 
This triple satisfies the constraints of \cref{def:aexec}, as well as the resolution policy $\RP_{\LWW}$. 
It is also immediate to see that $\graphof(\aexec) = ([], \emptyset, \emptyset, \emptyset)$.
In particular, $\T_{\graphof(\aexec)} = \emptyset$, 
and the only kv-store $\hh$ such that $\T_{\Gr_{\hh}} = \emptyset$ 
is given by $\hh = \hh_{0}$. 
By definition, $\hh_{\aexec_{\tr}} = \hh_{0}$, as we wanted to prove.

\item \caseI{$n > 0$} In this case, we have that $\tr = \tr' \xrightarrowtriangle{(\cl, \mu)} (\hh, \viewFun)$ 
for some $\cl, \mu, \hh, \viewFun$. The $\ET_{\top}$-trace $\tau'$ contains exactly $n-1$ transitions, 
so that by induction we can assume that $\aexec_{\tau'}$ is a valid abstract execution that satisfies 
$\RP_{\LWW}$. and $\lastConf(\tau') = (\hh_{\aexec_{\tau'}}, \viewFun')$ for some $\viewFun'$. 

We perform a case analysis on $\mu$. 
If $\mu = \varepsilon$, then it follows that $\hh = \hh_{\aexec_{\tau'}}$, 
and $\aexec_{\tau} = \aexec_{\tau'}$ by \cref{def:kvtrace2aexec}. 
Then by the inductive hypothesis $\aexec_{\tau}$ is an abstract execution that satisfies $\RP_{\LWW}$,
$\lastConf(\tau) = (\hh, \_)$, and $\hh_{\aexec_{\tau}} = \hh_{\aexec_{\tau'}} = \hh$, 
and there is nothing left to prove. 

Suppose now that $\mu = \opset$, for some $\opset$. In this case we have that  
$\hh \in \updateKV(\hh_{\aexec_{\tau'}}, \viewFun'(\cl), \cl, \opset)$. Note that if 
$\opset = \emptyset$, then $\hh = \hh_{\aexec_{\tau'}}$ and $\aexec_{\tau} = \aexec_{\tau'}$. 
By the inductive hypothesis, $\aexec_{\tau}$ is an abstract execution that satisfies 
$\RP_{\LWW}$, and $\hh = \hh_{\aexec_{\tau'}} = \hh_{\aexec_{\tau}}$. 
Assume then that $\opset \neq \emptyset$. By definition, $\hh = \updateKV(\hh_{\aexec_{\tau'}}, 
\viewFun'(\cl), \txid, \opset)$ for some $\txid \in \nextTxId(\cl, \hh_{\aexec_{\tau}})$. It follows that $\txid$ 
is the unique transaction such that $\txid \notin \hh_{\aexec_{\tau'}}$, and $\txid \in \hh$ 
(the fact that $\txid \in \hh$ follows from the assumption that $\opset \neq \emptyset$). Let 
$\T = \Tx(\hh_{\aexec_{\tau'}}, \viewFun'(\cl))$; then $\aexec_{\tau} = \extend(\hh_{\aexec_{\tau'}}, \txid, \T, \opset)$. 
Note that $\aexec_{\tau}$ satisfies the constraints of abstract execution required by \cref{def:aexec}:
\begin{itemize}
\item  Because $\txid \in \nextTxId(\cl, \hh_{\aexec_{\tau}})$, it must be the case that $\txid = \txid_{\cl}^{m}$ for some 
$m \geq 1$; we have that $\TtoOp{T}_{\aexec_{\tau}} = \TtoOp{T}_{\aexec_{\tau'}}\rmto{\txid_{\cl}^{m}}{\opset}$, 
from which it follows that 
\[
\T_{\aexec_{\tau}} = \dom(\TtoOp{T}_{\aexec_{\tau}}) = \dom(\TtoOp{T}_{\aexec_{\tau'}}) \cup 
\{\txid_{\cl}^{m} \} = \T_{\aexec_{\tau'}} \cup \{\txid_{\cl}^{m} \}
\]
By inductive hypothesis, $\txid_0 \notin \T_{\aexec_{\tau'}}$, and therefore $\txid_{0} \notin 
\T_{\aexec_{\tau'}} \cup \{\txid_{\cl}^{m} \} = \T_{\aexec}$.

\item \( \VIS_{\aexec_{\tau}} \subseteq \AR_{\aexec_{\tau}} \).
    Let $(\txid' ,\txid'') \in \VIS_{\aexec_{\tau}}$. Then either $\txid'' = \txid_{\cl}^{m}$ and $\txid' \in \T$, or $(\txid', \txid'') \in 
\VIS_{\aexec_{\tau'}}$. In the former case, we have that $(\txid', \txid_{\cl}^{m}) \in \AR_{\aexec_{\tau}}$ by definition; 
in the latter case, we have that $(\txid', \txid'') \in \AR_{\aexec_{\tau'}}$ because $\aexec_{\tau'}$ is a valid 
abstract execution by inductive hypothesis, and therefore $(\txid', \txid'') \in \AR_{\aexec_{\tau}}$ by definition. 
This concludes the proof that $\VIS_{\aexec_{\tau}} \subseteq \AR_{\aexec_{\tau}}$. 
\item \( \VIS_{\aexec_\tr} \) is irreflexive.
Assume $(\txid', \txid'') \in \VIS_{\aexec_{\tau}}$, then either 
$(\txid' \txid'') \in \VIS_{\aexec_{\tau'}}$, and because $\VIS_{\aexec_{\tau'}}$ is irreflexive by the inductive hypothesis, 
then $\txid' \neq \txid''$; 
or $\txid'' = \txid_{\cl}^{m}$, $\txid' \in \T \subseteq \T_{\aexec_{\tau'}}$, 
and because $\txid_{\cl}^{m} \notin \hh_{\aexec_{\tau'}}$, then $\txid' \neq \txid_{\cl}^{m}$. 

\item $\AR_{\aexec_{\tau}}$ is total. Let $(\txid', \txid'') \in \T_{\aexec_{\tau}}$. 
Suppose that $\txid' \neq \txid''$.
\begin{enumerate}
\item If $\txid' \neq 
\txid_{\cl}^{m}$, $\txid'' \neq \txid_{\cl}^{m}$, then it must be the case that $\txid', \txid'' \in \T_{\aexec_{\tau'}}$; 
this is because we have already argued that $\T_{\aexec_{\tau}} = \T_{\aexec_{\tau'}} \cup \{\txid_{\cl}^{m}\}$. 
By the inductive hypothesis, we have that either $(\txid', \txid'') \in \AR_{\aexec_{\tau'}}$, or 
$(\txid'', \txid') \in \AR_{\aexec_{\tau'}}$. Because $\AR_{\aexec_{\tau'}} \subseteq \AR_{\aexec_{\tau}}$, 
then either $(\txid', \txid'') \in \AR_{\aexec_{\tau'}}$ or $(\txid'', \txid') \in \AR_{\aexec_{\tau}}$. 
\item if $\txid'' = \txid_{\cl}^{m}$, then it must be $\txid' \in \T_{\aexec_{\tau'}}$. By definition, 
$(\txid', \txid_{\cl}^{m}) \in \AR_{\aexec_{\tau}}$. Similarly, if $\txid' = \txid_{\cl}^{m}$, we 
can prove that $(\txid'', \txid_{\cl}^{m}) \in \AR_{\aexec_{\tau}}$.
\end{enumerate}
\item  $\AR_{\aexec_{\tau}}$ is irreflexive. It follows is the same as the one of $\VIS_{\aexec_{\tau}}$.
\item \( \AR_{\aexec_{\tau}} \) is transitive.
Assume $(\txid', \txid'') \in \AR_{\aexec_{\tau}}$ and $(\txid'', \txid''') \in \AR_{\aexec_{\tau}}$. 
Note that it must be the case that $\txid', \txid'' \in \T_{\aexec_{\tau'}}$ by the definition of 
$\AR_{\aexec}$, and in particular $(\txid', \txid'') \in \AR_{\aexec_{\tau'}}$. 
For $\txid'''$, we have two possible cases. 
\begin{enumerate}
\item Either $\txid''' \in \T_{\aexec_{\tau}}$, from 
which it follows that $(\txid'', \txid''') \in \AR_{\aexec_{\tau'}}$; because
of $\AR_{\aexec_{\tau'}}$ is transitive by the inductive hypothesis, then 
$(\txid', \txid''') \in \AR_{\aexec_{\tau'}}$, and therefore $(\txid' ,\txid''') \in 
\AR_{\aexec_{\tau}}$.
\item Or $\txid''' = \txid_{\cl}^{m}$, and because $\txid' \in \T_{\aexec_{\tau'}}$, then 
$(\txid', \txid_{\cl}^{m}) \in \AR_{\aexec_{\tau}}$ by definition. 
\end{enumerate}
\item \( \SO_{\aexec_{\tau}} \subseteq \AR_{\aexec_{\tau}} \).
Let $\cl'$ be a client such that $(\txid_{\cl'}^{i}, \txid_{\cl'}^{j}) \in \AR_{\aexec_{\tau}}$. 
If $\cl' \neq \cl$, then it must be the case that $\txid_{\cl'}^{i}, \txid_{\cl'}^{j} \in \T_{\aexec_{\tau'}}$, 
and therefore $(\txid_{\cl'}^{i}, \txid_{\cl'}^{j}) \in \AR_{\aexec_{\tau'}}$. By the inductive hypothesis, 
it follows that $i < j$. If $\cl' = \cl$, then by definition of $\AR_{\aexec_{\tau}}$ it must be  $i \neq m$. 
If $j \neq m$ we can proceed as in the previous case to prove that $i < j$. If $j = m$, then 
note that $\txid_{\cl}^{i} \in \T_{\aexec_{\tau}}$ only if $\txid_{\cl}^{i} \in \hh_{\aexec_{tau'}}$. 
Because $\txid_{\cl}^{m} \in \nextTxId(\hh_{\aexec_{\tau'}}, \cl)$, then we have that $i < m$, 
as we wanted to prove.
\end{itemize}

Next, we prove that $\aexec_{\tau}$ satisfies the last write wins policy. 
Let $\txid' \in \T_{\aexec_{\tau}}$, and suppose that $(\otR, \ke, \val) \in_{\aexec_{\tau}} \txid'$. 
\begin{itemize} 
\item If $\txid' \neq \txid$, then we have that $\txid \in \T_{\aexec_{\tau'}}$. We also have that 
$\VIS^{-1}_{\aexec_{\tau}}(\txid') = \VIS^{-1}_{\aexec_{\tau'}}(\txid')$, $\AR^{-1}_{\aexec_{\tau}}(\txid') 
= \AR^{-1}_{\aexec_{\tau'}}(\txid')$; finally, for any $\txid'' \in \T_{\aexec_{\tau'}}$, 
$(\otW, \ke, \val') \in_{\aexec_{\tau}} \txid''$ if and only if $(\otW, \ke, \val') \in_{\aexec_{\tau'}} 
\txid''$. Therefore, let $\txid_{r} = \max_{\AR_{\aexec_{\tau}}}(\VIS^{-1}_{\aexec_{\tau}}(\txid') \cap 
\{\txid'' \mid (\otW, \ke, \_) \in_{\aexec_{\tau}} \txid''\})$. We have that $\txid_{r} = \max_{\AR_{\aexec_{\tau'}}}(\VIS^{-1}_{\aexec_{\tau'}}(\txid) 
\cap \{ \txid'' \mid (\otW, \ke, \_ \in_{\aexec_{\tau'}} \txid''\})$, and because $\aexec_{\tau'}$ satisfies the last write 
wins resolution policy,then $(\otW, \ke, \val) \in_{\aexec_{\tau'}} \txid_{r}$. This also implies that 
$(\otW, \ke, \va) \in_{\aexec_{\tau}} \txid_{r}$. 

\item Now, suppose that $\txid' = \txid$. Suppose that $(\otR, \ke, \val) \in_{\aexec_{\tau}} \txid'$. 
By definition, we have that $(\otR, \ke, \val) \in \opset$. Recall that $\tau = \tau' \xrightarrow{(\cl, \opset)}_{\ET_{\top}} (\hh, \viewFun)$, 
and $\lastConf(\tau') = (\hh_{\aexec_{\tau'}}, \viewFun')$ for some $\viewFun'$. 
That is, 
\[
    (\hh_{\aexec_{\tau'}}, \viewFun') \xrightarrowtriangle{(\cl, \opset)}_{\ET_{\top}} (\hh, \viewFun)
\]
which in turn implies that $\ET_{\top} \vdash (\hh_{\aexec_{\tau'}}, \viewFun'(\cl)) \triangleright \opset : \viewFun(\cl)$. 
Let then $r = \max\{i \mid  i \in \viewFun'(\cl)(\ke)\}$. 
By definition of execution test, and because $(\otR, \ke, \val) \in \opset$, then it must be the case that 
$\hh_{\aexec_{\tau'}}(\ke, r) = (\val, \txid'', \_)$ for some $\txid''$. 

We now prove that 
$\txid'' = \max_{\AR_{\aexec_{\tau}}}(\VIS^{-1}_{\aexec_{\tau}}(\txid) \cap \{ \txid'' \mid (\otW, \ke, \_) \in_{\aexec_{\tau}} \txid''\})$. 
First we have
\[ 
\begin{array}{l}
\VIS^{-1}_{\aexec_{\tau}}(\txid) = 
\quad \Tx(\hh_{\aexec_{\tau'}}, \viewFun'(\cl)) = \\
\qquad \{\WTx(\hh_{\aexec_{\tau'}}(\ke',  i)) \mid \ke' \in \Keys \wedge  i \in \viewFun'(\cl)(\ke')\}
\end{array}
\]
Note that $r \in \viewFun'(\cl)(\ke)$, and $\txid'' = \WTx(\hh_{\aexec_{\tau'}}(\ke, r))$. 
Therefore, $\txid'' \in \VIS^{-1}_{\aexec_{\tau}}(\txid)$. 
Because $\hh = \updateKV(\hh_{\aexec_{\tau'}}, \viewFun'(\cl), \txid, \opset)$, it 
must be the case that $\WTx(\hh(\ke, r)) = \txid''$. Also, because $\WTx(\hh_{\aexec_{\tau'}}(\ke, r)) = \txid''$, 
then $(\otW, \ke, \_) \in_{\aexec_{\tau''}} \txid''$, or equivalently $(\otW, \ke, \_) \in \TtoOp{T}_{\aexec_{\tau'}}(\txid'')$. 
We have already proved that $\VIS_{\aexec_{\tau}}$ is irreflexive, hence it must be the case that $\txid'' \neq \txid$. 
In particular, because $\aexec_{\tau} = \extend(\aexec_{\tau'}, \txid, \_, \_)$, then we have that 
$\TtoOp{T}_{\aexec_{\tau}}(\txid'') = \TtoOp{T}_{\aexec_{\tau'}}\rmto{\txid}{\opset}(\txid'') = 
\TtoOp{T}_{\aexec_{\tau'}}(\txid'')$, hence $(\otW, \ke, \_) \in \TtoOp{T}_{\aexec_{\tau}}(\txid'')$. Equivalently, 
$(\otW, \ke, \_) \in_{\aexec_{\tau}} \txid''$. We have proved that $\txid'' \in \VIS^{-1}_{\aexec_{\tau}}(\txid)$, 
and $(\otW, \ke, \_) \in_{\aexec_{\tau}} \txid''$. 

Now let $\txid'''$ be such that $\txid''' \in \VIS^{-1}_{\aexec_{\tau}}(\txid)$, and $(\otW, \ke \_) \in_{\aexec_{\tau}} \txid'''$. 
Note that $\txid''' \neq \txid$ because $\VIS_{\aexec_{\tau}}$ is irreflexive.
We show that either $\txid''' = \txid''$, or $\txid''' \xrightarrow{\AR_{\aexec_{\tau}}} \txid''$. 
Because $\txid''' \in \VIS^{-1}_{\aexec_{\tau}}(\txid)$, then there exists a key $\ke'$ and an index $i \in \viewFun'(\cl)$ 
such that $\WTx(\hh_{\aexec_{\tau'}}(\ke', i)) = \txid'''$. Because $(\otW, \ke, \_) \in_{\aexec_{\tau}} \txid'''$, 
and because $\txid''' \neq \txid$, then $(\otW, \ke, \_) \in_{\aexec_{\tau'}} \txid'''$, and therefore there exists 
an index $j$ such that $\WTx(\hh_{\aexec_{\tau'}}(\ke, j)) = \txid'''$. We have that $\WTx(\hh_{\aexec_{\tau'}}(\ke, j) = 
\WTx(\hh_{\aexec_{\tau'}}(\ke', i))$, and $i \in \viewFun'(\cl)$. By \cref{eq:view.atomic}, it must be $j \in \viewFun'(\cl)$. 
Note that $r = \max\{i \mid i \in \viewFun'(\cl)\}$, hence we have that $j \leq r$. If $j = r$, then $\txid''' = \txid''$ and 
there is nothing left to prove. If $j < r$, then we have that $(\txid''', \txid'') \in \AR_{\aexec_{\tau'}}$, and 
therefore $(\txid''', \txid'') \in \AR_{\aexec_{\tau}}$.
\end{itemize}
Finally, we need to prove that $\hh = \hh_{\aexec_{\tau}}$.
Recall $\hh = \updateKV(\hh_{\aexec_{\tau'}}, \viewFun'(\cl), \txid, \opset)$, 
and $\aexec_{\tau} = \extend(\aexec_{\tau'}, \txid, \Tx(\hh_{\aexec_{\tau'}}, \viewFun'(\cl), \opset)$. 
The result follows then from \cref{prop:extend.update.sameop}. 
\end{itemize}
\end{proof}

\sx{NEVER USED LEMMA}
\begin{lemma}
\label{lem:updatekv.preserveviews}
Let $\hh$ be a kv-store, and let $\txid \notin \hh$. For any $\vi, \vi' \in \Views(\hh)$, 
view $\vi'$ and set of operations $\opset$, then $\vi \in \updateKV(\hh, \txid, \vi', \opset)$.
\end{lemma}

\begin{proof}
Immediate from the definition of $\updateKV$. The fact that $\txid \notin \hh$ ensures that 
$\vi$ satisfies \eqref{eq:view.atomic} with respect to $\updateKV(\hh, \txid, \vi', \opset)$.
\end{proof}


\sx{NEVER USED LEMMA}
\begin{lemma}
\label{lem:cut.views}
Let $\aexec$ be an abstract execution, with 
$\T_{\aexec} = \{\txid_{i}\}_{i = 1}^{n}$ for 
$n = \lvert \T_{\aexec} \rvert$, with $\txid_{i} \xrightarrow{\AR_{\aexec}} \txid_{i+1}$ 
for all $i=0,\cdots, n-1$.  Let $\T_{0} = \emptyset$, and for $i=1,\cdots, n$, Let $\T_{i} \subseteq \AR^{-1}?(\txid_{i})$. 
Then $\getView(\aexec, \T_{i}) \in \Views(\hh_{\cut(\aexec, i)})$.
\end{lemma}

\begin{proof}
By induction on $i$. 
\begin{itemize}
\item If $i = 0$, then $\T_{0} = \emptyset$, and $\getView(\aexec, \T_{0}) = \lambda \ke. \{0 \}$. 
We also have that $\hh_{\cut(\aexec, 0)} = \lambda \ke. \List{(\val_0, \txid_{0}, \emptyset)}$, hence 
it is immediate to see that $\getView(\aexec, \T_{0}) \in \Views(\hh_{\cut(\aexec, 0)})$.

\item Let $i = i'+1$, and suppose that for any $\T \subseteq (\AR_{\aexec}^{-1}?)(\txid_{i'})$, 
$\getView(\aexec, \T) \in \Views(\hh_{\cut(\aexec, i)})$. Consider the set $\T_{i}$.
Note that, because of \cref{prop:extend.update.sameop}, we have that
\[
\begin{array}{ll}
\hh_{\cut(\aexec, i)} &=\\ 
\hh_{\extend(\cut(\aexec, i'), \txid_{i}, \VIS^{-1}_{\aexec}(\txid_{i}), \TtoOp{T}_{\aexec}(\txid_{i})} &= \\
\updateKV(\hh_{\cut(\aexec, i')}, \txid_{i}, \getView(\VIS^{-1}_{\aexec}(\txid_{i}), \TtoOp{T}_{\aexec}(\txid_{i})
\end{array}
\]
We consider two possibilities.
\begin{enumerate}
\item $\txid_{i} \notin \T_{i}$, in which case $\T_{i} \subseteq (\AR_{\aexec}^{-1})?(\txid_{i'})$, 
and from the inductive hypothesis we get $\getView(\aexec, \T_{i}) \in \Views(\hh_{\cut(\aexec, i')})$. 
Note that, from \ref{lem:cut.explicit} it must be the case that $\hh_{\cut, \aexec, i')}$ only contains 
the transactions identifiers from $\txid_{1}$ to $\txid_{i'}$; in particular, it does not contain 
$\txid_{i}$. Because $\hh_{\cut(\aexec, i)} = \updateKV(\hh_{\cut(\aexec, i')}, \txid_{i}, \_, \_)$, 
then by \ref{lem:updatekv.preserveviews} we have that  
$\getView(\aexec, \txid_{i}) \in \Views(\hh_{\cut(\aexec, i)})$.

\item $\txid \in \T_{i}$. Note that for any key $\ke$ such that 
$(\otW, \ke, \_) \notin \TtoOp{T}_{\aexec}(\txid_{i})$, then 
$\getView(\aexec, \T_{i})(\ke) = \getView(\aexec, \T_{i} \setminus \{(\txid_{i})\})(\ke)$; 
and for any key $\ke$ such that $(\otW, \ke, \_) \in \TtoOp{T}_{\aexec}(\txid_{i})$, 
then $\getView(\aexec, \T_{i}(\ke) = \getView(\aexec, \T_{i} \setminus \{\txid_{i}\})(\ke) 
\cup \{j \mid \WTx(\hh_{\aexec}(\ke, i)) = \txid_{j}\}$. Note that in this last case the index 
$j$ must be such that $j < \lvert \hh_{cut(\aexec, i)} \rvert - 1$, because we know that 
$\txid_{i} \in \hh_{\cut(\aexec, i)}$. It follows from this fact and the inductive hypothesis, 
that $\getView(\aexec, \T_{i}) \in \Views(\cut(\aexec, i))$.
\ac{This is a really loose proof sketch. The reason for an incomplete proof is 
that i got bored while typing it.} 
\end{enumerate}
\end{itemize}
\end{proof}

\begin{proposition}[\( \extend \) matching \( \updateKV\)]
\label{prop:extend.update.sameop}
Given an abstract execution $\aexec$, a set of transactions $\T \subseteq \T_{\aexec}$,
a transaction $\txid \notin \T_{\aexec}$, and a fingerprint $\opset \subseteq \powerset{\Ops}$,
if the new abstract execution $\aexec' = \extend(\aexec, \T, \txid, \opset)$,
and the view $\vi = \getView(\hh_{\aexec}, \T)$,
then $\updateKV(\hh_{\aexec}, \vi, \txid, \opset) = \hh_{\aexec'}$.
\end{proposition}

\begin{proof}
Let $\Gr = \Gr_{\updateKV(\hh_{\aexec}, \vi, \txid, \opset)}$, $\Gr' = \graphof(\aexec')$. 
Note that $\hh_{\aexec'}$ is the unique kv-store such that $\Gr_{\hh_{\aexec'}} = \graphof(\aexec') = \Gr'$. 
It suffices to prove that $\Gr = \Gr'$. Because the function $\Gr_{\cdot}$ is injective, it follows that 
$\updateKV(\hh_{\aexec}, \vi, \txid, \opset) = \hh_{\aexec'}$, as we wanted to prove.  

The proof is a consequence of \cref{lem:graph.extend} and \cref{lem:graph.update}. 
Consider the dependency graph $\Gr_{\hh_{\aexec}}$.
Recall that $\hh_{\aexec}$ is the unique key-value store such that $\Gr_{\hh_{\aexec}} = \graphof(\aexec)$. 
We prove that $\TtoOp{T}_{\Gr} = \TtoOp{T}_{\Gr'}$, $\RF_{\Gr} = \RF_{\Gr'}$ and 
$\VO_{\Gr} = \VO_{\Gr'}$ (from the last two it follows that $\AD_{\Gr} = \AD_{\Gr'}$). 
\begin{itemize}
\item It is easy to see $\TtoOp{T}_{\Gr} = \TtoOp{T}_{\Gr'}$.

\item $\RF_{\Gr} = \RF_{\Gr'}$.
Let \( \mkvs  = \mkvs_\aexec \).
Suppose that $\txid' \xrightarrow{\RF_{\Gr}(\ke)} \txid''$ for some $\txid', \txid''$. 
By \cref{lem:graph.update} we have that either $\txid' \xrightarrow{\RF_{\Gr_\hh}(\ke)} \txid''$, 
or $\txid'' = \txid$, $(\otR, \ke, \_) \in \opset$, $\txid' = \max_{\VO_{\Gr_\hh}(\ke)}\{\WTx(\ke, i) \mid i \in \vi(\ke)\}$. 

\begin{itemize}
\item If $\txid' \xrightarrow{\RF_{\Gr_\hh}(\ke)} \txid''$, then because 
$\Gr_\mkvs = \graphof(\aexec)$, we have that $\txid' \xrightarrow{\RF_{\graphof(\aexec)}(\ke)} \txid''$. 
Recall that $\Gr' = \graphof(\extend(\aexec, \T, \txid, \opset))$, hence by \cref{lem:graph.extend} 
we obtain that $\txid' \xrightarrow{\RF_{\Gr'}(\ke)} \txid''$. 

\item If $\txid'' = \txid$, $(\otR, \ke, \_) \in \opset$, and $\txid' = \max_{\VO_{\Gr_\mkvs}(\ke)} \{ \WTx(\hh_{\aexec}(\ke, i)) \mid i \in \vi(\ke)\}$, 
then we also have that $\txid' = \max_{\VO_{\graphof(\aexec)}(\ke)} (\T \cap \{ \txid''' \mid (\otW, \ke, \_) \in_{\aexec} \txid'''\}) $. 
This is because of the assumption that 
\[
\begin{array}{l}
\{\WTx(\hh_{\aexec}(\ke, i)) \mid i \in \vi(\ke)\}  \\
\quad = \{\WTx(\hh_{\aexec}(\ke', i)) \mid \ke' \in \Keys \wedge i \in \vi(\ke')\} \cap \{\WTx(\hh_{\aexec}(\ke, \_)\} \\
\quad = \Tx(\hh_{\aexec}, \vi) \cap \{\WTx(\hh_{\aexec}(\ke, \_)\}  \\
\quad = \T \cap \{(\txid''' \mid (\otW, \ke, \_) \in_{\aexec} \txid''')\}
\end{array}
\]
Again, it follows from \cref{lem:graph.extend} that $\txid' \xrightarrow{\RF_{\Gr'}(\ke)} \txid''$. 
\end{itemize}
\item \( \VO_{\Gr} = \VO_{\Gr'}\). The \( \VO_{\Gr} = \VO_{\Gr'} \) follows the similar reasons as $\RF_{\Gr} = \RF_{\Gr'}$.
\end{itemize}
\end{proof}

\begin{lemma}[Graph extension]
\label{lem:graph.extend}
Let $\aexec$ be an abstract execution, 
%$\cl$,  be a client such that $\txid \in \nextTxid(\cl, \T_{\aexec})$, 
$\txid \notin \T_{\aexec} \cup \{\txid_0\}$ be a transaction identifier $\T_{\aexec}$, and $\opset \in \Tx$. 
Let $\T \subseteq \T_{\aexec}$ be a set of transaction identifiers.
%\ac{$\nextTxId$ has been defined only for kv-stores, it must be defined for sets of transactions.}
Define $\Gr := \graphof(\aexec), \Gr' := \graphof(\extend(\aexec, \txid, \T, \opset))$. 
We have the following: 
\begin{enumerate}
\item for any $\txid' \in \T_{\Gr'}$, either $\txid' \in \T_{\Gr}$ and $\TtoOp{T}_{\Gr}(\txid') = \TtoOp{T}_{\Gr'}(\txid')$, 
or $\txid' = \txid$ and $\TtoOp{T}_{\Gr'}(\txid) = \opset$.
\item $\txid' \xrightarrow{\RF_{\Gr'}(\ke)} \txid''$ if and only if either 
$\txid' \xrightarrow{\RF_{\Gr}(\ke)_{\Gr}} \txid''$, or $(\otR, \ke, \_) \in \opset$, $\txid'' = \txid$ and 
$\txid' = \max_{\VO_{\Gr}(\ke)}(\T)$, 
\item $\txid' \xrightarrow{\VO_{\Gr'}(\ke)} \txid''$ if and only if 
either $\txid' \xrightarrow{\VO_{\Gr}(\ke)} \txid''$, or $(\otW, \ke, \_) \in \opset$, $\txid'' = \txid$, 
and $(\otW, \ke, \_) \in_{\Gr} \txid'$.
\end{enumerate}
\end{lemma}

\begin{proof}
Fix a key $\ke$. Let $\aexec' = \extend(\aexec, \txid, \T, \opset)$. Recall that $\Gr' = \graphof(\aexec')$.

\begin{enumerate}
\item By definition of $\extend$, and 
because $\txid \notin \T_{\aexec}$, we have that 
$\T_{\aexec'} = \T_{\aexec} \uplus \{\txid\}$. Furthermore, $\TtoOp{T}_{\aexec'}(\txid) = \opset$, 
from which it follows that $\TtoOp{T}_{\Gr'}(\txid) = \opset$.
For all $\txid' \in \T_{\aexec}$, we have that $\TtoOp{T}_{\aexec'}(\txid') = 
\TtoOp{T}_{\aexec}(\txid') = \TtoOp{T}_{\Gr}(\txid')$.
\item
Suppose that $\txid' \xrightarrow{\RF(\ke)_{\Gr}} \txid''$ for some $\txid', \txid'' \in \T_{\Gr}$. 
By definition, $(\otR, \ke, \_) \in_{\aexec} \txid''$,  
and $\txid' = \max_{\AR_{\aexec}}(\VIS_{\aexec}^{-1}(\txid'') \cap \{\txid''' \mid (\otW, \ke, \_) \in_{\aexec} \txid'''\})$. 
Because $\txid'' \in \T_{\Gr} = \T_{\aexec}$, it follows that $\txid'' \neq \txid$. By definition, 
$\VIS^{-1}_{\aexec'}(\txid'') = \VIS^{-1}_{\aexec}(\txid)$: also, whenever 
$\txid_{a}, \txid_{b} \in \VIS^{-1}_{\aexec'}(\txid)$ we have that $\txid_{a}, \txid_{b} \in \T_{\aexec}$, 
and therefore if $\txid_{a} \xrightarrow{\AR_{\aexec'}} \txid_{b}$, then it must be the case 
that $\txid_{a} \xrightarrow{\AR_{\aexec}} \txid_b$; also, $\TtoOp{T}_{\aexec}(\txid_{a}) = \TtoOp{T}_{\aexec'}(\txid_{a})$. 
As a consequence, we have that 
\[\max_{\AR_{\aexec'}}(\VIS^{-1}_{\aexec'}(\txid) \cap \{ \txid''' \mid (\otW, \ke, \_) \in_{\aexec'} \txid'''\}) = 
\max_{\AR_{\aexec}}(\VIS^{-1}_{\aexec}(\txid) \cap \{ \txid''' \mid (\otW, \ke, \_) \in_{\aexec} \txid'''\}) = \txid', \] 
and therefore $\txid' \xrightarrow{\RF_{\Gr'}} \txid$. 

Suppose now that $(\otR,\ke, \_) \in \opset$, and $\txid' = \max_{\VO(\ke)_{\Gr}}(\T)$. 
By Definition, $\txid' = \max_{\AR_{\aexec}}(\T) \cap \{ \txid''' \mid (\otW, \ke, \_) \in_{\aexec} \txid'''\}$, 
Also, $\T = \VIS^{-1}_{\aexec'}(\txid)$, and because $\T \subseteq \T_{\aexec}$, we have 
that for any $\txid_{a}, \txid_{b}$, if $\txid_{a} \xrightarrow{\AR_{\aexec}} \txid_{b}$, 
then $\txid_{a} \xrightarrow{\AR_{\aexec'}} \txid_{b}$; and $\TtoOp{T}_{\aexec'}(\txid_{a}) = 
TtoOp{T}_{\aexec}(\txid_a)$. Therefore, 
\[
\txid' = \max_{\AR_{\aexec'}}(\VIS^{-1}_{\aexec'}(\txid) \cap \{ \txid''' \mid (\otW, \ke, \_) \in_{\aexec'} \txid'''\}, 
\] 
from which it follows that $\txid' \xrightarrow{\RF_{\Gr'}(\ke)}\txid$.

Now, suppose that $\txid' \xrightarrow{\RF_{\Gr'}(\ke)} \txid''$ for some $\txid', \txid'' \in \T_{\Gr'} = 
\T_{\aexec'}$. We have that $ (\otR, \ke, \_) \in_{\aexec'} \txid''$, 
$(\otW, \ke, \_) \in_{\aexec'} \txid'$, and $\txid'' = \max_{\AR_{\aexec'}}(\VIS_{\aexec'}^{-1}(\txid'') 
\cap \{ \txid''' \mid (\otW, \ke, \_) \in_{\aexec'} \txid'''\}$. 
We also have that $\T_{\aexec'} = \T_{\aexec} \uplus \{\txid\}$. We perform a case 
analysis on $\txid''$. 

If $\txid'' = \txid$, then by definition of $\extend$ we have that 
$\VIS^{-1}_{\aexec'}(\txid) = \T$. Note that $\T \subseteq \T_{\aexec}$, so that 
for any $\txid_{a}, \txid_{b} \in \T_{\aexec}$, we have that $\txid_{a} \xrightarrow{\AR_{\aexec'}} \txid_{b}$ 
if and only if $\txid_{a} \xrightarrow{\AR_{\aexec}} \txid_{b}$, 
and $(\otW, \ke, \val) \in_{\aexec'} \txid_{a}$ if and only if $(\otW, \ke, \val) \in_{\aexec} \txid_{a}$. 
Thus, $\txid' = \max_{\AR_{\aexec}}(\T 
\cap \{\txid''' \mid (\otW, \ke, \_) \in_{\aexec} \txid'''\}) = \max_{\VO_{\Gr}(\ke)}(\T)$. 

If $\txid'' \in \T_{\aexec}$, then it is the case that 
$\txid' = \max_{\AR_{\aexec'}}(\VIS^{-1}_{\aexec'}(\txid'') \cap \{ \txid''' \mid (\otW, \ke, \_) \in_{\aexec'} \txid'''\}$. 
Similarly to the case above, we can prove that $\VIS^{-1}_{\aexec'}(\txid'') = \VIS^{-1}_{\aexec}(\txid)$, 
for any $\txid_{a}, \txid_{b} \in \VIS^{-1}_{\aexec}(\txid)$, $(\otW, \ke, \val) \in_{\aexec'} \txid_{a}$ 
implies $(\otW, \ke, \val) \in_{\aexec} \txid_{a}$, and $\txid_{a} \xrightarrow{\AR_{\aexec'}} \txid_{b}$ 
implies $\txid_{a} \xrightarrow{\AR_{\aexec}} \txid_{b}$, from which it follows that 
$\txid' = \max_{\AR_{\aexec}}(\VIS^{-1}_{\aexec}(\txid'') \cap \{ \txid''' \mid (\otW, \ke \_) \in_{\aexec} \txid'''\})$, 
and therefore $\txid' \xrightarrow{\RF_{\Gr}(\ke)} \txid''$.

\item Suppose that $\txid' \xrightarrow{\VO_{\Gr}(\ke)} \txid''$ for some $\txid', \txid'' \in \T_{\aexec}$. 
Then $(\otW,\ke,\_) \in_{\aexec} \txid', (\otW, \ke, \_) \in_{\aexec} \txid''$, and $\txid' \xrightarrow{\AR_{\aexec}} \txid''$. 
By definition of $\extend$, it follows that $\txid' \xrightarrow{\AR_{\aexec'}} \txid''$, and because 
$\txid', \txid'' \in \T_{\aexec}$, hence $\txid', \txid'' \neq \txid$, then 
$(\otW,\ke, \_) \in_{\aexec'} \txid'$, $(\otW, \ke, \_) \in_{\aexec'} \txid''$. By definition, 
we have that $\txid' \xrightarrow{\VO_{\aexec'}(\ke)} \txid''$.

Suppose that $(otW, \ke, \_) \in_{\aexec} \txid'$, $(\otW, \ke, \_) \in \opset$. Because $\txid' \in \T_{\aexec}$, 
we have that $\txid' \neq \txid$, hence $(\otW, \ke, \_) \in_{\aexec' }\txid'$. By definition, 
$\TtoOp{T}_{\aexec'}(\txid) = \opset$, hence $(\otW, \ke, \_) \in_{\aexec'} \txid$. Finally, 
the definition of $\extend$ ensures that $\txid' \xrightarrow{\AR_{\aexec'}} \txid$. Combining 
these three facts together, we obtain that  
$\txid' \xrightarrow{\VO_{\Gr'}(\ke)} \txid$. 

Now, suppose that $\txid' \xrightarrow{\VO_{\Gr'}(\ke)} \txid''$ for some $\txid', \txid'' \in \T_{\aexec}$. 
Then $\txid' \xrightarrow{\AR_{\aexec'}} \txid''$, $(\otW, \ke, \_) \in_{\aexec'} \txid'$, $(\otW, \ke, \_) 
\in_{\aexec'} \txid''$. 
Recall that $\T_{\Gr'} = \T_{\aexec'} = \T_{\aexec} \uplus \{ \txid \}$. We perform a case analysis on $\txid''$. 

If $\txid'' = \txid$, then the definition of $\extend$ ensures that $\txid' \xrightarrow{\AR_{\aexec'}} \txid$ 
implies that $\txid \in \T_{\aexec}$, hence $\txid' \neq \txid$. 
Together with $(\otW, \ke, \_) \in_{\aexec'} 
\txid'$, this leads to $(\otW, \ke, \_) \in_{\aexec} \txid'$. 

If $\txid'' \in \T_{\aexec}$, then $\txid'' \neq \txid$. The definition of $\extend$ ensures that $\txid' \xrightarrow{\AR_{\aexec}} \txid''$. 
This implies that $\txid', \txid'' \in \T_{\aexec}$, hence $\txid', \txid'' \neq \txid$, and $\TtoOp{T}_{\aexec'}(\txid') = \TtoOp{T}_{\aexec}(\txid')$, 
$\TtoOp{T}_{\aexec'}(\txid'') = \TtoOp{T}_{\aexec}(\txid'')$. It follows that $(\otW, \ke, \_) \in_{\aexec} \txid'$, 
$(\otW, \ke, \_) \in_{\aexec} \txid''$, and therefore $\txid' \xrightarrow{\VO_{\Gr}}(\ke) \txid''$.

\end{enumerate}
\end{proof}


\begin{lemma}[Graph update]
\label{lem:graph.update}
Let $\hh$ be a kv-store, and $\vi \in \Views(\hh)$. Let $\txid \notin \hh$, and 
$\opset \subseteq \powerset{\Ops}$, and let $\hh' = \updateKV(\hh, \vi, \txid, \opset)$. 
Let $\Gr = \Gr_{\hh}$, $\Gr' = \Gr_{\hh'}$; then for all $\txid', \txid'' \in \T_{\Gr'}$ and keys $\ke$, 
\begin{itemize}
\item $\TtoOp{T}_{\Gr'} = \TtoOp{T}_{\Gr}\rmto{\txid}{\opset}$, 
\item $\txid' \xrightarrow{\RF_{\Gr'}(\ke)} \txid''$ if and only if either 
$\txid' \xrightarrow{\RF_{\Gr}(\ke)} \txid''$, or $(\otR, \ke, \_) \in \opset$ and 
$\txid' = \max_{\VO_{\Gr}(\ke)}(\{\WTx(\hh(\ke, i)) \mid i \in \vi(\ke)\})$, 
\item $\txid' \xrightarrow{\VO_{\Gr'}(\ke)} \txid''$ if and only if either 
$\txid' \xrightarrow{\VO_{\Gr}(\ke)} \txid''$, or $(\otW, \ke, \_) \in \opset$ 
and $\txid' = \WTx(\hh(\ke, \_))$. 
\end{itemize}

\end{lemma}

\begin{proof}
Fix $\ke \in \Keys$. Because $\txid \notin \hh$, then $\txid \notin \T_{\Gr}$, 
and by definition of $update$ we obtain that $\{\txid' \mid \txid' \in \hh'\} = 
\{\txid' \mid \txid' \in \hh\} \cup \{\txid\}$. It follows that $\T_{\Gr'} = \T_{\Gr} \uplus \{\txid \}$.

\begin{enumerate}
\item Suppose that $(\otR, \ke, \val) \in_{\Gr} \txid'$. By definition, 
there exists an index $i = 0,\cdots, \lvert \hh(\ke) \rvert - 1$ such 
that $\hh(\ke, i) = (\val, \_, \{\txid'\} \cup \_)$. Because $\hh' = \updateKV(\hh, \vi, \txid, \opset)$, 
it is immediate to observe that $\hh'(\ke, i) = (\val, \_, \{\txid'\} \cup \_)$, and therefore 
$(\otR,\ke, \val) \in_{\Gr'} \txid'$. Conversely, note that if $(\otR, \ke, \val) \in_{\Gr'} \txid$, 
then there exists an index $i = 0,\cdots, \lvert \hh'(\ke) \rvert - 1$ such that 
$\hh'(\ke, i) = (\val, \_, \{\txid'\} \cup \_)$. As a simple consequence of \cref{cor:updatekv.singlecell} 
it follows that it must be the case that $i \leq \lvert \hh(\ke) \rvert - 1$, and because 
$\txid' \neq \txid$, we have that $\hh(\ke, i) = (\val, \_, \{\txid'\} \cup \_)$. Therefore 
$(\otR, \ke, \val) \in_{\Gr} \txid'$. 

Similarly, if $(\otW, \ke, \val) \in_{\Gr} \txid'$, 
then there exists an index $i=0,\cdots, \lvert \hh(\ke) \rvert - 1$ such that 
$\hh(\ke, i) = (\val, \txid', \val)$. It follows that $\hh'(\ke, i) = (\val, \txid', \_)$, hence 
$(\otW, \ke, \val) \in_{\Gr'} \txid'$. If $(\otW, \ke, \val) \in \opset$, then we 
have from \cref{cor:updatekv.singlecell} that $\hh'(\ke, \lvert \hh'(\ke) \rvert - 1) = (\val, \txid', \_)$, 
hence $(\otW, \ke, \val) \in_{\Gr'} \txid'$. 
Conversely, if $(\otW, \ke, \val) \in_{\Gr'} \txid'$, then there exists an index 
$i = 0, \cdots, \lvert \hh'(\ke) \rvert - 1$ such that $\hh(\ke, i) = (\val, \txid', \_)$. 
We have two possible cases: either $i < \lvert \hh'(\ke, i) \rvert - 1$, leading to  
$\txid' \neq \txid$ and $\hh(\ke, i) = (\val, \txid', \_)$, or equivalently 
$(\otR,\ke, \val) \in_{\Gr} \txid'$; or $i = \lvert \hh'(\ke, i) \rvert - 1$, 
leading to $\txid' = \txid$, and $\hh(\ke, i) = (\val, \txid, \emptyset)$ 
for some $\val$ such that $(\otW, \ke, \val) \in \opset$. 

Putting together the facts above, we obtain that $\TtoOp{T}_{\Gr'} = 
\TtoOp{T}_{\Gr}\rmto{\txid}{\opset}$, as we wanted to prove.

\item Suppose that $\txid' \xrightarrow{\RF_{\Gr}(\ke)} \txid''$. 
By definition, there exists an index $i = 0,\cdots, \lvert \hh(\ke) \rvert - 1$ 
such that $\hh(\ke, i) = (\_, \txid', \{\txid''\} \cup \_)$. It is immediate 
to observer, from the definition of $\updateKV$, that $\hh'(\ke, i) = (\_, \txid', \{\txid''\} \cup \_)$, 
and therefore $\txid' \xrightarrow{\RF_{\Gr'}(\ke)} \txid''$. 

Next, suppose that $(\otR, \ke, \_) \in \opset$, and $\txid' = \max_{\VO_{\Gr}(\ke)}(\{\WTx(\hh(\ke, i)) \mid i \in \vi(\ke)\}$. 
By Definition, $\hh(\ke, i) = (\_, \txid', \_)$, where $i = \max(\vi(\ke))$. This is because 
$\txid' \rightarrow{\VO_{\Gr}(\ke)} \txid''$ if and only if $\txid' = \WTx(\hh(\ke, j_1)), \txid'' = 
\WTx(\hh(\ke, j_2))$ for some $j_1, j_2$ such that $j_1 < j_2$. 
The definition of $\updateKV$ now ensures that $\hh'(\ke, i) = (\_, \txid', \{\txid \} \cup \_)$, 
from which it follows that $\txid' \xrightarrow{\RF_{\Gr'}(\ke)} \txid$.

Conversely, suppose that $\txid' \xrightarrow{\RF_{\Gr'}(\ke)} \txid''$. 
Recall that $\T_{\Gr'} = \T_{\Gr} \cup \{ \txid \}$, hence either 
$\txid'' \in \T_{\Gr}$ or $\txid'' = \txid$. 

If $\txid'' = \txid$, then it must be the case that there exists an index $i = 0,\cdots, \lvert \hh'(\ke) \rvert - 1$ 
such that $\hh'(\ke, i) = (\_, \txid', \{\txid \} \cup \_)$. Note that if $\hh'(\ke, \lvert \hh'(\ke) \rvert -1)$ is 
defined, then it must be the case that $\hh'(\ke, \lvert \hh'(\ke) \rvert -1) = (\_, \txid, \emptyset)$, 
hence it must be the case that $i < \lvert \hh'(\ke) \rvert - 1$. Because $\txid \notin \hh$, 
then by the definition of $\updateKV$ it must be the case that $(\otR, \ke, \_) \in \opset$, 
$\hh(\ke, i) = (\_, \txid', \_)$ and $i = \max(\vi(\ke))$; this also implies that $\txid' = 
\max_{\VO(\ke)}\{\WTx(\hh(\ke, i)) \mid i \in \vi(\ke)\}$. 

If $\txid'' \in \T_{\Gr}$, then  it must be the case that $\txid'' \neq \txid$. 
In this case, it also must exist an index $i = 0,\cdots, \lvert \hh'(\ke) \rvert - 1$ 
such that $\hh'(\ke, i) = (\_, \txid', \{\txid''\} \cup \_)$. As in the previous 
case, we note that $i < \lvert \hh'(\ke) \rvert - 1$, which together 
with the fact that $\txid'' \neq \txid$ leads to $\hh(\ke, i) = (\_, \txid', \{\txid''\} \cup \_)$. 
It follows that $\txid' \xrightarrow{\RF_{\Gr}(\ke)} \txid''$.

\item Suppose that $\txid' \xrightarrow{\VO_{\Gr}(\ke)} \txid''$. 
By definition, there exist two indexes $i, j$ such that 
$\hh(\ke, i) = (\_, \txid', \_)$, $\hh(\ke, j) = (\_, \txid'', \_)$ 
and $i < j$. The definition of $\updateKV$ ensures that 
$\hh'(\ke, i) = (\_, \txid', \_)$, $\hh'(\ke, j) = (\_, \txid'', \_)$, 
and because $i < j$ we obtain that $\txid' \xrightarrow{\VO_{\Gr'}(\ke)} \txid''$. 

Suppose that $(\otW, \ke, \_) \in \opset$. Then $\hh'(\ke, \lvert \hh(\ke) \rvert) = (\_, \txid, \_)$.
Let $\txid' \in \T_{\Gr}$; by definition there exists an index $i = 0,\cdots, \lvert \hh(\ke) \rvert$ 
such that $\hh(\ke, i) = (\_, \txid', \_)$. It follows that $\hh'(\ke, i) = (\_, \txid', \_)$, and 
because $i < \lvert \hh(\ke) \rvert$, then we have that $\txid' \xrightarrow{\VO_{\Gr'}(\ke)} \txid$. 

Conversely, suppose that $\txid' \xrightarrow{\VO_{\Gr'}(\ke)} \txid''$. Because 
$\T_{\Gr'} = \T_{\Gr} \cup \{ \txid \}$, we have two possibilities. Either $\txid'' = \txid$, 
or $\txid'' \in \T_{\Gr}$. 

If $\txid'' = \txid$, then it must be the case that $(\otW, \ke, \_) \in_{\Gr'} \txid$, 
or equivalently there exists an index $i=0,\cdots, \lvert \hh'(\ke) \rvert -1 $ such that 
$\hh'(\ke, i) = (\_, \txid, \_)$. Because $\txid \notin \hh$, and because for any 
$i = 0, \cdots, \lvert \hh(\ke) \rvert - 1$, $\hh'(\ke, i) = (\_, \txid, \_) \implies 
\hh(\ke, i) = (\_, \txid, \_)$, then it necessarily has to be $i = \hh'(\ke) \rvert - 1$. 
According to the definition of $\updateKV$, this is possible only if $(\otW,\ke, \_) \in \opset$. 
Finally, note that because $\txid' \xrightarrow{\VO_{\Gr'}(\ke)} \txid$, then 
there exists an index $j < \lvert \hh'(\ke, i) \rvert - 1$ such that 
$\hh'(\ke, j) = (\_, \txid' ,\_)$. The fact that $j < \lvert \hh'(\ke, i) \rvert - 1$ 
From \cref{cor:updatekv.singlecell} we obtain that $\hh(\ke, j) = (\_, \txid', \_)$, 
or equivalently $\txid' = \WTx(\hh(\ke, \_))$. 

If $\txid'' \in \T_{\Gr}$, then there exist two indexes $i,j$ such that 
$j < \lvert \hh'(\ke, j) \rvert - 1$, $\hh'(\ke, j) = (\_, \txid'', \_)$, 
$i < j$, and $\hh'(\ke, i) = (\_, \txid', \_)$. It is immediate to observe 
that $\hh(\ke, i) = (\_, \txid', \_)$, $\hh(\ke, j) = (\_, \txid'', \_)$, 
from which $\txid' \xrightarrow{\VO_{\Gr}(\ke)} \txid''$ follows. 

\end{enumerate}
\end{proof}





Next, we show to construct, given an abstract execution $\aexec$, 
a set of $\ET_{\top}$-traces $\KVtrace(\ET_{\top}, \aexec)$ in normal form such that for any 
$\tr \in \KVtrace(\ET_{\top}, \aexec)$, $\lastConf(\tr) = (\hh_{\aexec}, \_)$. 
To define the function $\KVtrace(\ET_{\top}, \_)$ formally, 
we first provide a principle to reason about abstract executions inductively in \cref{def:aexec.inductive}. 

\begin{definition}
\label{def:aexec.inductive}
Let $\aexec$ be an abstract execution, let $n = \lvert \T_{\aexec} \rvert$, and let 
$\{\txid_{i}\}_{i=1}^{n} \subseteq \T_{\aexec}$ be such that for any $i=1,\cdots,n-1$, 
$\txid_{i} \xrightarrow{\AR_{\aexec}} \txid_{i+1}$. 
For $i = 0,\cdots, n-1$, define 
\[
\begin{array}{lll}
\cut(\aexec, 0) & \defeq & ([], \emptyset, \emptyset)\\
\cut(\aexec , i+1) & \defeq & \extend(\cut(\aexec, i), \txid_{i+1}, \VIS^{-1}_{\aexec}(\txid_{i+1}), \TtoOp{T}_{\aexec}(\txid_{i+1}))
\end{array}
\]
\end{definition}

\begin{proposition}
\label{prop:aexec.inductive}
For any abstract execution $\aexec$, $\aexec = \cut(\aexec, \lvert \T_{\aexec} \rvert)$.
\end{proposition}
\begin{proof}
    This is now an instantiation of \cref{lem:cut.explicit}, choosing $i = \lvert \T_{\aexec} \rvert$. \qed
\end{proof}

\begin{lemma}
\label{lem:cut.explicit}
For any abstract execution $\aexec$, and index $i: i \leq j \leq \lvert \T_{\aexec} \rvert$, 
let $\T_{\aexec} = \{\txid_{i}\}_{i=1}^{n}$ be such that $\txid_{1} \xrightarrow{\AR_{\aexec}}
\txid_{n}$. Then $\cut(\aexec, i) = \aexec_{i}$, where 
\[
\begin{array}{lcr}
\TtoOp{T}_{\aexec_{i}}(\txid) &=& 
\begin{cases}
\TtoOp{T}_{\aexec}(\txid) &\impliedby \exists j \leq i.\; \txid = \txid_{j}\\
\text{undefined} &\impliedby \text{otherwise}\\
\end{cases}
\\
\VIS_{\aexec_{i}} &=& \{ (\txid, \txid') \in \T_{\aexec_{i}} \mid \txid \xrightarrow{\VIS_{\aexec}} \txid'\}\\
\AR_{\aexec_{i}} &=& \{ (\txid, \txid') \in \T_{\aexec_{i}} \mid \txid \xrightarrow{\AR_{\aexec}} \txid'\}
\end{array}
\]
\end{lemma}

\begin{proof}
Fix an abstract execution $\aexec$. We prove the claim by induction on $i = \lvert \T_{\aexec} \rvert$.
\begin{itemize}
\item Base case: $i = 0$. Then note that $\TtoOp{T}_{\aexec'} = [], \VIS_{\aexec'} = \emptyset$, 
$\AR_{\aexec'} = \emptyset$, which leads to $\aexec' = \cut(\aexec, 0)$. 
\item Let $i = i' + 1$, and assume that $\cut(\aexec, i') = \aexec_{i'}$. 
We prove the following: 
\begin{enumerate}
\item $\TtoOp{T}_{\cut(\aexec, i)} = \TtoOp{T}_{\aexec_i}$. 
By definition, 
\[
\begin{array}{lr}
\TtoOp{T}_{\cut(\aexec,i)} = \TtoOp{T}_{\cut(\aexec, i')}\rmto{\txid_{i}}{\TtoOp{T}_{\aexec}(\txid_{i})} &= \\
\TtoOp{T}_{\aexec_{i'}}\rmto{\txid_{i}}{\TtoOp{T}_{\aexec}}(\txid_{i}) = \TtoOp{T}_{\aexec_{i}}
\end{array}
\]
\item $\VIS_{\cut(\aexec, i)} = \VIS_{\aexec_{i}}$. 
Note that, by inductive hypothesis, $\T_{\cut(\aexec, i')} = \T_{\aexec_{i'}} = \{\txid_{j}\}_{j=1}^{i'}$. 
We have that  
\[
\begin{array}{lr}
\VIS_{\cut(\aexec, i)} = \VIS_{\cut(\aexec, i')} \cup \{(\txid_j, \txid_{i}) \in \VIS_{\aexec} \mid j = 1,\cdots, i'\} &= \\
\VIS_{\aexec_{i'}} \cup \{(\txid_{j}, \txid_{i}) \in \VIS_{\aexec} \mid j=1,\cdots, i'\} = 
\{(\txid_{j'}, \txid_{j}) \in \VIS_{\aexec} \mid j', j = 0,\cdots, i'\} \cup \{(\txid_{j}, \txid_{i} \in \VIS_{\aexec} \mid j=1,\cdots, i'\} &=\\
\{(\txid_{j'}, \txid_{j} \in \VIS_{\aexec} \mid j',j = 0,\cdots, i\} = \VIS_{\aexec_{i}}.
\end{array}
\]
\item $\AR_{\cut(\aexec, i)} = \AR_{\aexec_{i}}$. This can be proved in the same way 
as the item above. 
\end{enumerate}
\end{itemize}
\end{proof}

Given above, the conversion from abstract execution tests to \( \ET \) traces is in \cref{def:aexec2kvtrace}.

\begin{definition}
\label{def:aexec2kvtrace}
Given an abstract execution $\aexec$, a client $\cl$ and an integer $i=0,\cdots, \lvert \aexec \rvert$, 
define $\nextTx(\aexec, \cl, i) \defeq \min_{\AR_{\aexec}}\{\txid_{\cl}^{j} \mid \txid_{\cl}^{n} \notin \T_{\cut(\aexec, i)}\}$. 
Note that $\nextTx(\aexec, \cl, i)$ could be undefined. 
Let $\Clients(\aexec) \defeq \{\cl \mid \exists n.\;\txid_{\cl}^{n} \in \T_{\aexec}\}$.

Given an abstract execution $\aexec$ and an integer $i = 0,\cdots, \lvert \T_{\aexec} \rvert$, let 
$\KVtrace(\ET_{\top}, \aexec, i)$ be the smallest set such that 
\begin{itemize}
\item 
$(\hh_{0}, \lambda \cl \in \Clients(\aexec). \lambda \ke.\{0\}) \in \KVtrace(\ET_{\top}, \aexec, 0)$, 
\item suppose that $\tr \in \KVtrace(\ET_{\top}, \aexec, i)$ for some $i = 0,\cdots, \lvert \T_{\aexec} \rvert - 1$.  
Let
\begin{itemize} 
\item $\txid = \min_{\AR_{\aexec}}(\T_{\aexec} \setminus T_{\cut(\aexec, i)})$, 
\item  $\cl, n$ be such that $\txid = \txid_{\cl}^{n}$, 
\item  $\vi = \getView(\aexec, \VIS^{-1}_{\aexec}(\txid_{\cl}^{n}))$, 
\item $\vi' = \getView(\aexec, \T)$, where $\T$ is an arbitrary subset of $\T_{\aexec}$ if 
$\nextTx(\aexec, \cl, i+1)$ is undefined, or is such that 
$\T \subseteq (\AR_{\aexec}^{-1})?(\txid) \cap \VIS^{-1}_{\aexec}(\nextTx(\cl, i+1))$, 
\item $\opset = \TtoOp{T}_{\aexec}(\txid)$, 
\item $(\hh_{\tr}, \viewFun_{\tr}) = \lastConf(\tr)$, 
\item $\hh = \updateKV(\hh_{\tr}, \vi, \txid, \opset)$.
\end{itemize}
Then
\[
\big( \tr \xrightarrowtriangle{(\cl, \varepsilon)}_{\ET_{\top}} (\hh_{\tr}, \viewFun_{\tr}\rmto{\cl}{\vi}) 
\xrightarrowtriangle{(\cl, \opset)}_{\ET_{\top}} (\hh, \viewFun_{\tr}\rmto{\cl}{\vi'}) \big) \in \KVtrace(\ET_{\top}, \aexec, i+1)
\]
\end{itemize}

Finally, we let $\KVtrace(\ET_{\top}, \aexec) \defeq \KVtrace(\ET_{\top}, \aexec, \lvert \T_\aexec \rvert)$.
\end{definition}

\begin{proposition}
\label{prop:aexec2kvtrace}
Let $\aexec$ be an abstract execution that satisfies $\RP_{\LWW}$, 
and let $\tr \in \KVtrace(\ET_{\top}, \aexec)$. Then $\lastConf(\tr) = (\hh_{\aexec}, \_)$ and $\hh_{\aexec} \in \CMs(\ET_{\top})$. 
\end{proposition}
\begin{proof}
Let $\aexec$ be an abstract execution that satisfies the last write wins policy. 
Let $n = \lvert \T_{\aexec} \rvert$. Fix $i =0,\cdots, n$, 
and let $\tr \in \KVtrace(\ET_{\top}, \aexec, i)$. We prove, by 
induction on $i$, that $\tr \in \CMs(\ET_{\top})$, and 
$\lastConf(\tr) = (\hh_{(\cut(\aexec, i)}, \_)$. 
Then the result follows from  \cref{prop:aexec.inductive}.

\begin{itemize}
\item Case $i = 0$. By definition, $\tr = (\kappa_{0}, \viewFun_{0})$, 
where $\viewFun_{0} = \lambda \cl \in \Clients(\aexec). \lambda \ke.\{0\}$. 
Clearly, we have that $\tr \in CMs(\ET_{\top})$. 
\item Case $i = i'+1$. Let $\txid_{i} = \min_{\AR_{\aexec}}(\T_{\aexec} \setminus \T_{\cut(\aexec, i')})$, 
and suppose that $\txid_{i} = \txid_{\cl}^{m}$ for some client $\cl$ and index $m$. 
Fix $\vi = \getView(\aexec, \VIS_{\aexec}^{-1}(\txid_{i}))$, and  $\opset = \TtoOp{T}_{\aexec}(\txid_{i})$, . 
Then there exists a trace $\tr' \in \KVtrace(\ET_{\top}, \aexec, i')$ and a set 
$\T$ such that: 
\begin{enumerate}
\item if $\nextTx(\cl, \aexec, i)$ is undefined then $\T \subseteq \T_{\aexec}$, otherwise 
$\T \subseteq \VIS^{-1}_{\aexec}(\nextTx(\cl, \aexec, i)) \cap (\AR_{\aexec}^{-1})?(\txid_{i})$, 
\item
\[
\tr = \tr' \xrightarrowtriangle{(\cl, \varepsilon)} (\hh_{\tr'}, \viewFun_{\tr'}\rmto{\cl}{\vi}) \xrightarrowtriangle{(\cl, \opset)} 
(\hh,  \viewFun_{\tr'}\rmto{\cl}{\vi'}),
\]
where $(\hh_{\tr'}, \viewFun_{\tr'}) = \lastConf(\tr')$, and $\hh = \updateKV(\hh_{\tr'}, \vi, \txid_{i}, \opset)$, 
and $\vi' = \getView(\aexec, \T)$.
\end{enumerate}
By inductive hypothesis, we may assume that $\tr' \in \CMs(\ET_{\top})$, and $\hh_{\tr'} = \hh_{\cut(\aexec, i')}$. 
We prove the following facts: 
\begin{enumerate}
\item $\hh = \hh_{\extend(\cut(\aexec, i)}$: because of 
\ref{prop:extend.update.sameop} and \cref{prop:aexec.inductive} 
we obtain 
\[
\begin{array}{lr}
\hh = \updateKV(\hh_{\tr'}, \vi, \txid_{i}, \opset) &=\\ 
\updateKV(\hh_{\cut(\aexec, i')}, \getView(\aexec, \VIS^{-1}_{\aexec}(\txid_{i}), \txid_{i}, \TtoOp{T}_{\aexec}(\txid_{i})) &=\\
\hh_{\extend(\cut(\aexec, i'), \VIS^{-1}_{\aexec}(\txid_{i}), \txid_{i}, \TtoOp{T}_{\aexec}(\txid_{i})) }&=\\
\hh_{\extend(\cut(\aexec, i))}.
\end{array}
\]

\item $(\hh_{\tr'}, \viewFun_{\tr'}) \xrightarrowtriangle{(\cl, \varepsilon)}_{\ET_{\top}} (\hh_{\tr'}, \viewFun_{\tr'}\rmto{\cl}{\vi})$. 
To this end, it suffices to prove that for any key $\ke$, $\viewFun_{\tr'}(\cl) \viewleq \vi$. 
By \cref{lem:cut.explicit} we have that $\T_{\cut(\aexec, i')} = \{\txid_{j}\}_{j=1}^{i'}$, for 
some $\txid_{1},\cdots, \txid_{i'}$ such that whenever $1 \leq j < j' \leq i'$, then 
$\txid_{j} \xrightarrow{\AR_{\aexec}} \txid_{j'}$. We consider two possible cases: 

\begin{itemize}
\item For all $j =1,\cdots, i'$, for all $h \in \nat$, $\txid_{j} \neq \txid_{\cl}^{h}$. In 
this case we have that no transition contained in $\tr'$ has the form 
$(_, \_) \xrightarrowtriangle{(\cl, \_)} (\_, \_)$, from which it is possible to infer 
that  $\viewFun_{\tr'}(\cl) = \lambda \ke. \{0\}$. Because $\vi = \getView(\aexec, \VIS^{-1}_{\aexec}(\txid_{i}))$, 
then by definition we have that $0 \in \vi(\ke)$ for all keys $\ke \in \Keys$. It follows that 
$\viewFun_{\tr'}(\cl) \viewleq \vi$. 
\item There exists an index $j = 1,\cdots, n$ and an integer $h \in \nat$ such that $\txid_{j} = \txid_{\cl}^{h}$. 
Without loss of generality, let $j$ be the largest such index. It follows that the last transition in $\tr'$ of the form 
$(\_, \_) \xrightarrow{(\cl, \opset_{j})} (\_, \viewFun_{\mathsf{pre}})$ is such that $\viewFun_{\mathsf{pre}}(\cl) = 
\getView(\aexec, \T_{\mathsf{pre}})$, for some $\T_{\mathsf{pre}} \subseteq \VIS^{-1}_{\aexec}(\txid_{i}) \cap 
(\AR^{-1}_{\aexec})?(\txid_{j})$. This is because $\nextTx(\cl, \aexec, j)$  is defined and equal to $\txid_{i}$. 
Furthermore, because the trace $\tr'$ is in normal form by construction, in $\tr'$ a transition of the form 
$(\_, \_) \xrightarrowtriangle{(\cl, \varepsilon)}_{\ET_{\top}} (\_, \_)$ is always followed by a transition of the form
$(\_, \_) \xrightarrowtriangle{(\cl, \opset')}_{\ET_{\top}} (\_, \_)$. Because we are assuming that the last transition where client 
$\cl$ executes a transaction in $\tr'$ has the form $(\_, \_) \xrightarrowtriangle{(\cl, \opset_{j})}_{\ET_{\top}} (\_, \viewFun_{\mathsf{pre}})$, 
then the latter is also the last transition for client $\cl$ in $\tr'$ (i.e. including both execution of transactions and view updates). 
It follows that $\viewFun_{\tr'}(\cl) = \viewFun_{\mathsf{pre}}(\cl)$, and in particular 
$\viewFun_{\tr'}(\cl) = \getView(\aexec, \T_{\mathsf{pre}})$. By definition, 
$\T_{\mathsf{pre}} \subseteq  \VIS^{-1}_{\aexec}(\txid_{i}) \cap (\AR^{-1}_{\aexec})?(\txid_{j}) 
\subseteq \VIS^{-1}_{\aexec}(\txid_{i})$. By  \cref{lem:getView.monotone}, 
we have that $\viewFun_{\tr'}(\cl) = \getView(\aexec, \T_{\mathsf{pre}}) \viewleq 
\getView(\aexec, \VIS^{-1}_{\aexec}(\txid_{i})) = \vi$, as we wanted to prove.
\ac{Note: this is more a sketch, rather than a real proof. A Proposition giving an explicit form to the 
structure of any $\tr \in \KVtrace(\ET_{\top}, \aexec)$ would be helpful for a more rigorous proof here.}
\end{itemize}


\item $(\hh_{\tr'}, \viewFun_{\tr'}\rmto{\cl}{\vi}) \xrightarrowtriangle{(\cl, \opset)}_{\ET_{\top}} 
(\hh,  \viewFun_{\tr'}\rmto{\cl}{\vi'})$. It suffices to show that $\ET_{\top} \vdash (\hh_{\tr'}, \vi) 
\triangleright \opset: \vi'$. That is, it suffices to show that $\vi \in \Views(\hh_{\tr'})$, 
$\vi' \in \Views(\hh)$, and whenever $(\otR, \ke, \val) \in \opset$, then 
$\max_{<}(\vi(\ke)) = (\val, \_, \_)$. The first two facts are a consequence of 
\ref{lem:cut.views}, the fact that $\hh_{\tr'} = \hh_{\cut(\aexec, i')}$, and the fact that 
$\hh_{\cut(\aexec, i)}$. The fact that if $(\otR, \ke, \val) \in \opset$ then 
$\max_{<}(\vi(\ke) = (\val, \_, \_)$ follows from the fact that $\aexec$ satisfies 
the last write wins policy, and the fact that $\vi = \getView(\VIS^{-1}_{\aexec}(\txid_{i})$.
\ac{Again, the proof is really loose here, mostly because I got bored.}
\end{enumerate} 

\end{itemize}
\end{proof}


\begin{lemma}
\label{lem:getView.monotone}
Let $\aexec$ be an abstract execution, and let $\T_1 \subseteq \T_2 \subseteq \T_{\aexec}$. 
Then $\getView(\aexec, \T_1) \viewleq \getView(\aexec, \T_2)$.
\end{lemma}
\begin{proof}
Fix $\ke \in \Keys$. By definition  
\[
\begin{array}{lr}
\getView(\aexec, \T_1)(\ke) = \{0\} \cup \{ i \mid i = 1,\cdots, \lvert \hh_{\aexec}(\ke) \rvert - 1 \wedge \WTx(\hh_{\aexec}(\ke, i)) \in \T_1\} &\subseteq\\
\{0\} \cup \{i \mid i=1,\cdots, \lvert \hh_{\aexec}(\ke) \rvert - 1 \wedge \WTx(\hh_{\aexec}(\ke, i)) \in \T_2\} & = \\
\getView(\aexec, \T_2)(\ke).
\end{array}
\]
It follows that  $\getView(\aexec, \T_1) \viewleq \getView(\aexec, \T_2)$.
\end{proof}

Now we can prove \cref{thm:kvtrace2aexec}.
The last statement in \cref{thm:kvtrace2aexec} implies that there is a \emph{Galois connection}
between the set of $\ET_{\top}$-traces, and the set of abstract executions that satisfy the 
last write wins policy. The lower and upper adjoints of this connection are the 
lifting of the functions $\aeset(\cdot)$ and $\KVtrace(\cdot)$ to sets of $\ET_{\top}$-traces 
and abstract executions, respectively. Note however that these two sets are not isomorphic: 
when converting a set of abstract executions into kv-traces, we abstract away the 
pairs $\txid \xrightarrow{\VIS} \txid'$ in the visibility relation where $\txid$ is a read-only 
transaction. When converting a $\ET_{\top}$-trace into a set of abstract executions, 
we (partially) lose the information about the view that a clients has, immediately after it executed a transaction.

\begin{theorem}
\label{thm:kvtrace2aexec}
Let $\tr$ be a $\ET_{\top}$-trace; there exists a set of abstract executions $\aeset({\tr})$ 
such that, for any $\hh \in \aeset(\tr)$, $\lastConf(\tr) = (\hh, \_)$.
Let $\aexec$ be an abstract execution that satisfies the last write wins resolution policy. 
There exists a set of $\ET_{\top}$-traces $\KVtrace({\aexec})$ in normal form, 
such that for any $\tr \in \KVtrace({\aexec})$, $\lastConf(\tr) = (\hh_{\aexec}, \_)$. 
\end{theorem}
\begin{proof}
    For readability the proof is in \cref{sec:kvtrace2aexec}.
\end{proof}

\begin{corollary} 
\label{cor:kvtrace2aexec}
$\CMs(\ET_{\top}) = \{\hh_{\aexec} \mid \aexec \text{ satisfies } \RP_{\LWW}\}$.
\end{corollary}





\subsection{Abstract Execution Traces to KV-Store Traces}
\label{sec:aexectrace2kv}

We show to construct, given an abstract execution $\aexec$, 
a set of $\ET_{\top}$-traces $\KVtrace(\ET_{\top}, \aexec)$ in normal form such that for any 
$\tr \in \KVtrace(\ET_{\top}, \aexec)$, the trace \( \tr \) satisfies $\lastConf(\tr) = (\mkvs_{\aexec}, \stub)$. 
%To define the function $\KVtrace(\ET_{\top}, \stub)$ formally, 
We first define the \( \cut(\aexec,n) \) function in \cref{def:aexec.inductive} 
which gives the prefix of the first \( n \) transactions of the abstract execution \( \aexec \).
The  \( \cut(\aexec,n) \) function is very useful for later discussion.

\begin{definition}
\label{def:aexec.inductive}
Let $\aexec$ be an abstract execution, let $n = \lvert \txidset_{\aexec} \rvert$, and let 
$\Set{\txid_{i}}_{i=1}^{n} \subseteq \txidset_{\aexec}$ be such that $\txid_{i} \toEDGE{\AR_{\aexec}} \txid_{i+1}$. 
The \emph{cut} of the first \( n \) transactions from an abstract execution \( \aexec \) is defined as the follows:
\[
\begin{rclarray}
\cut(\aexec, 0) & \defeq & ([], \emptyset, \emptyset)\\
\cut(\aexec , i+1) & \defeq & \extend(\cut(\aexec, i), \txid_{i+1}, \VIS^{-1}_{\aexec}(\txid_{i+1}), \TtoOp{T}_{\aexec}(\txid_{i+1}))
\end{rclarray}
\]
\end{definition}

\begin{proposition}[Well-defined \( \cut \)]
\label{prop:aexec.inductive}
For any abstract execution $\aexec$, $\aexec = \cut(\aexec, \lvert \txidset_{\aexec} \rvert)$.
\end{proposition}
\begin{proof}
    This is an instantiation of \cref{lem:cut.explicit} by choosing $i = \lvert \txidset_{\aexec} \rvert$. 
\end{proof}

\begin{lemma}[Prefix]
\label{lem:cut.explicit}
For any abstract execution $\aexec$, and index $i: i \leq j \leq \lvert \txidset_{\aexec} \rvert$, 
if $\txidset_{\aexec} = \Set{\txid_{i}}_{i=1}^{n}$ be such that 
$\txid_{i} \toEDGE{\AR_{\aexec}} \txid_{i+1}$, 
then $\cut(\aexec, i) = \aexec_{i}$ where 
\[
\begin{rclarray}
\TtoOp{T}_{\aexec_{i}}(\txid) &=& 
\begin{cases}
\TtoOp{T}_{\aexec}(\txid) & \text{if } \exists j \leq i.\; \txid = \txid_{j}\\
\text{undefined} & \text{otherwise}\\
\end{cases} \\
\VIS_{\aexec_{i}} &=& \Setcon{ (\txid, \txid') \in \txidset_{\aexec_{i}} }{ \txid \toEDGE{\VIS_{\aexec}} \txid'} \\
\AR_{\aexec_{i}} &=& \Setcon{ (\txid, \txid') \in \txidset_{\aexec_{i}} }{ \txid \toEDGE{\AR_{\aexec}} \txid'}
\end{rclarray}
\]
\end{lemma}

\begin{proof}
Fix an abstract execution $\aexec$. We prove by induction on $i = \lvert \txidset_{\aexec} \rvert$.
\begin{itemize}
\item \caseB{$i = 0$} Then $\TtoOp{T}_{\aexec'} = [], \VIS_{\aexec'} = \emptyset$, 
$\AR_{\aexec'} = \emptyset$, which leads to $\aexec' = \cut(\aexec, 0)$. 
\item \caseI{$i = i' + 1$} 
Assume that $\cut(\aexec, i') = \aexec_{i'}$. 
We prove the following: 
\begin{itemize}
\item $\TtoOp{T}_{\cut(\aexec, i)} = \TtoOp{T}_{\aexec_i}$. 
By definition, 
\[
    \begin{array}{l}
\TtoOp{T}_{\cut(\aexec,i)} = \TtoOp{T}_{\cut(\aexec, i')}\rmto{\txid_{i}}{\TtoOp{T}_{\aexec}(\txid_{i})} 
\TtoOp{T}_{\aexec_{i'}}\rmto{\txid_{i}}{\TtoOp{T}_{\aexec}}(\txid_{i}) = \TtoOp{T}_{\aexec_{i}}
\end{array}
\]
\item $\VIS_{\cut(\aexec, i)} = \VIS_{\aexec_{i}}$. 
Note that, by inductive hypothesis, $\txidset_{\cut(\aexec, i')} = \txidset_{\aexec_{i'}} = \Set{\txid_{j}}_{j=1}^{i'}$. 
We have that  
\[
\begin{array}{l}
    \VIS_{\cut(\aexec, i)}
    \begin{array}[t]{l}
    = \VIS_{\cut(\aexec, i')} \cup \Setcon{(\txid_j, \txid_{i}) \in \VIS_{\aexec} }{ j = 1,\cdots, i'} \\ 
    = \VIS_{\aexec_{i'}} \cup \{(\txid_{j}, \txid_{i}) \in \VIS_{\aexec} \mid j=1,\cdots, i'\} \\ 
    = \Setcon{(\txid_{j'}, \txid_{j}) \in \VIS_{\aexec} }{ j', j = 0,\cdots, i'} \cup \Setcon{(\txid_{j}, \txid_{i} \in \VIS_{\aexec} }{ j=1,\cdots, i'} \\
    = \Setcon{(\txid_{j'}, \txid_{j} \in \VIS_{\aexec}  }{ j',j = 0,\cdots, i} \\
    = \VIS_{\aexec_{i}}
    \end{array}
\end{array}
\]
\item $\AR_{\cut(\aexec, i)} = \AR_{\aexec_{i}}$. It follows the same way 
as the above. 
\end{itemize}
\end{itemize}
\end{proof}

Let $\Clients(\aexec) \defeq \Setcon{\cl }{ \exists n.\;\txid_{\cl}^{n} \in \txidset_{\aexec}}$.
Given an abstract execution $\aexec$, a client $\cl$ and an index $i : 0 \leq i < \abs{\txidset_\aexec}$,
the function $\nextTxid[\aexec, \cl, i] \defeq \min_{\AR_{\aexec}} \Setcon{\txid_{\cl}^{j} }{ \txid_{\cl}^{n} \notin \txidset_{\cut(\aexec, i)}}$. 
Note that $\nextTxid[\aexec, \cl, i]$ could be undefined. 
The conversion from abstract execution tests to \( \ET \) traces is in \cref{def:aexec2kvtrace}.

\begin{definition}
\label{def:aexec2kvtrace}
Given an abstract execution $\aexec$ and an index $i : 0 \leq i < \abs{\txidset_\aexec}$, 
the function $\KVtrace(\ET_{\top}, \aexec, i)$ is defined as the smallest set such that:
\begin{itemize}
\item 
$(\mkvs_{0}, \lambda \cl \in \Clients(\aexec). \lambda \key.\Set{0}) \in \KVtrace(\ET_{\top}, \aexec, 0)$, 
\item suppose that $\tr \in \KVtrace(\ET_{\top}, \aexec, i)$ for some $i$.  
Let
\begin{itemize} 
\item $\txid = \min_{\AR_{\aexec}}(\txidset_{\aexec} \setminus T_{\cut(\aexec, i)})$, 
\item  $\cl, n$ be such that $\txid = \txid_{\cl}^{n}$, 
\item  $\vi = \getView(\aexec, \VIS^{-1}_{\aexec}(\txid_{\cl}^{n}))$, 
\item $\vi' = \getView(\aexec, \txidset)$, where $\txidset$ is an arbitrary subset of $\txidset_{\aexec}$ if 
$\nextTxid[\aexec, \cl, i+1]$ is undefined, or is such that 
$\txidset \subseteq (\AR_{\aexec}^{-1})\rflx(\txid) \cap \VIS^{-1}_{\aexec}(\nextTxid[\cl, i+1])$, 
\item $\fp = \TtoOp{T}_{\aexec}(\txid)$, 
\item $(\mkvs_{\tr}, \vienv_{\tr}) = \lastConf(\tr)$, and
\item $\mkvs = \updateKV(\mkvs_{\tr}, \vi, \fp, \txid)$.
\end{itemize}
Then
\[
\left( 
\begin{array}{l}
\tr \xrightarrowtriangle{(\cl, \varepsilon)}_{\ET_{\top}} (\mkvs_{\tr}, \vienv_{\tr}\rmto{\cl}{\vi}) 
\xrightarrowtriangle{(\cl, \fp)}_{\ET_{\top}} (\mkvs, \vienv_{\tr}\rmto{\cl}{\vi'}) 
\end{array}
\right) \in \KVtrace(\ET_{\top}, \aexec, i+1)
\]
\end{itemize}
Last, the function $\KVtrace(\ET_{\top}, \aexec) \defeq \KVtrace(\ET_{\top}, \aexec, \lvert \txidset_\aexec \rvert)$.
\end{definition}

\begin{proposition}[Abstract executions to trace \( \ET_\top \)]
\label{prop:aexec2kvtrace}
Given an abstract execution $\aexec$ satisfying $\RP_{\LWW}$, 
and a trace $\tr \in \KVtrace(\ET_{\top}, \aexec)$,
then $\lastConf(\tr) = (\mkvs_{\aexec}, \stub)$ and $\mkvs_{\aexec} \in \CMs(\ET_{\top})$. 
\end{proposition}
\begin{proof}
Let $\aexec$ be an abstract execution that satisfies the last write wins policy. 
Let $n = \lvert \txidset_{\aexec} \rvert$. Fix $i =0,\cdots, n$, 
and let $\tr \in \KVtrace(\ET_{\top}, \aexec, i)$. We prove, by 
induction on $i$, that $\tr \in \CMs(\ET_{\top})$, and 
$\lastConf(\tr) = (\mkvs_{(\cut(\aexec, i)}, \stub)$. 
Then the result follows from  \cref{prop:aexec.inductive}.

\begin{itemize}
\item \caseB{$i = 0$} By definition, $\tr = (\mkvs_{0}, \vienv_{0})$, 
where $\vienv_{0} = \lambda \cl \in \Clients(\aexec). \lambda \key.\Set{0}$. 
Clearly, we have that $\tr \in \CMs(\ET_{\top})$. 
\item \caseI{$i = i'+1$} Let $\txid_{i} = \min_{\AR_{\aexec}}(\txidset_{\aexec} \setminus \txidset_{\cut(\aexec, i')})$, 
and suppose that $\txid_{i} = \txid_{\cl}^{m}$ for some client $\cl$ and index $m$. 
Fix $\vi = \getView(\aexec, \VIS_{\aexec}^{-1}(\txid_{i}))$, and  $\fp = \TtoOp{T}_{\aexec}(\txid_{i})$.
We prove that there exists a trace $\tr' \in \KVtrace(\ET_{\top}, \aexec, i')$ and a set 
$\txidset$ such that: 
\begin{enumerate}
\item if $\nextTxid[\cl, \aexec, i]$ is undefined then $\txidset \subseteq \txidset_{\aexec}$, otherwise 
\[
    \txidset \subseteq \VIS^{-1}_{\aexec}(\nextTxid[\cl, \aexec, i]) \cap (\AR_{\aexec}^{-1})\rflx(\txid_{i})
\]
\item the new trace \( \tr \) such that
\[
\tr = \tr' \xrightarrowtriangle{(\cl, \varepsilon)} (\mkvs_{\tr'}, \vienv_{\tr'}\rmto{\cl}{\vi}) \xrightarrowtriangle{(\cl, \fp)} 
(\mkvs,  \vienv_{\tr'}\rmto{\cl}{\vi'})
\]
where $(\mkvs_{\tr'}, \vienv_{\tr'}) = \lastConf(\tr')$, and $\mkvs = \updateKV(\mkvs_{\tr'}, \vi, \fp, \txid_{i})$, 
and $\vi' = \getView(\aexec, \txidset)$.
\end{enumerate}
By inductive hypothesis, we may assume that $\tr' \in \CMs(\ET_{\top})$, and $\mkvs_{\tr'} = \mkvs_{\cut(\aexec, i')}$. 
We prove the following facts: 
\begin{enumerate}
\item $\mkvs = \mkvs_{\extend(\cut(\aexec, i))}$. 
Because of \cref{prop:extend.update.sameop} and \cref{prop:aexec.inductive},
we obtain 
\[
\begin{array}{l}
\mkvs = \updateKV(\mkvs_{\tr'}, \vi, \fp, \txid_{i}) \\
\quad = \updateKV(\mkvs_{\cut(\aexec, i')}, \getView(\aexec, \VIS^{-1}_{\aexec}(\txid_{i}), \TtoOp{T}_{\aexec}(\txid_{i}), \txid_{i}) \\
\quad = \mkvs_{\extend(\cut(\aexec, i'), \VIS^{-1}_{\aexec}(\txid_{i}), \txid_{i}, \TtoOp{T}_{\aexec}(\txid_{i})) } \\
\quad = \mkvs_{\extend(\cut(\aexec, i))}
\end{array}
\]

\item $(\mkvs_{\tr'}, \vienv_{\tr'}) \xrightarrowtriangle{(\cl, \varepsilon)}_{\ET_{\top}} (\mkvs_{\tr'}, \vienv_{\tr'}\rmto{\cl}{\vi})$. 
It suffices to prove that $\vienv_{\tr'}(\cl) \viewleq \vi$ for any key $\key$.
By \cref{lem:cut.explicit} we have that $\txidset_{\cut(\aexec, i')} = \Set{\txid_{j}}_{j=1}^{i'}$, for 
some $\txid_{1},\cdots, \txid_{i'}$ such that whenever $1 \leq j < j' \leq i'$, then 
$\txid_{j} \toEDGE{\AR_{\aexec}} \txid_{j'}$. We consider two possible cases: 

\begin{itemize}
\item For all $j : 1 \leq j \leq i'$, and $h \in \Nat$, then $\txid_{j} \neq \txid_{\cl}^{h}$.
In this case we have that no transition contained in $\tr'$ has the form 
$(\stub, \stub) \xrightarrowtriangle{(\cl, \stub)} (\stub, \stub)$, from which it is possible to infer 
that  $\vienv_{\tr'}(\cl) = \lambda \key. \Set{0}$. Because $\vi = \getView(\aexec, \VIS^{-1}_{\aexec}(\txid_{i}))$, 
then by definition we have that $0 \in \vi(\key)$ for all keys $\key \in \Keys$. It follows that 
$\vienv_{\tr'}(\cl) \viewleq \vi$. 

\item There exists an index $j : 1 \leq j \leq i'$ and an integer $h \in \Nat$ such that $\txid_{j} = \txid_{\cl}^{h}$. 
Without loss of generality, let $j$ be the largest such index. 
It follows that the last transition in $\tr'$ of the form $(\stub, \stub) \toEDGE{(\cl, \fp_{j})} (\stub, \vienv_{\mathsf{pre}})$ 
is such that $\vienv_{\mathsf{pre}}(\cl) = \getView(\aexec, \txidset_{\mathsf{pre}})$, 
for some $\txidset_{\mathsf{pre}} \subseteq \VIS^{-1}_{\aexec}(\txid_{i}) \cap (\AR^{-1}_{\aexec})\rflx(\txid_{j})$.
This is because $\nextTxid[\cl, \aexec, j]$  is defined and equal to $\txid_{i}$. 
Furthermore, because the trace $\tr'$ is in normal form by construction, 
in $\tr'$ a transition of the form $(\stub, \stub) \xrightarrowtriangle{(\cl, \varepsilon)}_{\ET_{\top}} (\stub, \stub)$ 
is always followed by a transition of the form $(\stub, \stub) \xrightarrowtriangle{(\cl, \fp')}_{\ET_{\top}} (\stub, \stub)$. 
Because we assume that the last transition where client $\cl$ executes a transaction in $\tr'$ 
has the form $(\stub, \stub) \xrightarrowtriangle{(\cl, \fp_{j})}_{\ET_{\top}} (\stub, \vienv_{\mathsf{pre}})$, 
then the latter is also the last transition for client $\cl$ in $\tr'$ 
(i.e. including both execution of transactions and view updates). 
It follows that $\vienv_{\tr'}(\cl) = \vienv_{\mathsf{pre}}(\cl)$, and in particular 
$\vienv_{\tr'}(\cl) = \getView(\aexec, \txidset_{\mathsf{pre}})$. By definition, 
$\txidset_{\mathsf{pre}} \subseteq  \VIS^{-1}_{\aexec}(\txid_{i}) \cap (\AR^{-1}_{\aexec})\rflx(\txid_{j}) 
\subseteq \VIS^{-1}_{\aexec}(\txid_{i})$. By  \cref{lem:getView.monotone}, 
we have that $\vienv_{\tr'}(\cl) = \getView(\aexec, \txidset_{\mathsf{pre}}) \viewleq 
\getView(\aexec, \VIS^{-1}_{\aexec}(\txid_{i})) = \vi$, as we wanted to prove.
\ac{Note: this is more a sketch, rather than a real proof. A Proposition giving an explicit form to the 
structure of any $\tr \in \KVtrace(\ET_{\top}, \aexec)$ would be helpful for a more rigorous proof here.}
\end{itemize}


\item $(\mkvs_{\tr'}, \vienv_{\tr'}\rmto{\cl}{\vi}) \xrightarrowtriangle{(\cl, \fp)}_{\ET_{\top}} (\mkvs,  \vienv_{\tr'}\rmto{\cl}{\vi'})$. 
    It suffices to show that $\ET_{\top} \vdash (\mkvs_{\tr'}, \vi) \csat \fp: (\mkvs,\vi')$. 
That is, it suffices to show that $\vi \in \Views(\mkvs_{\tr'})$, $\vi' \in \Views(\mkvs)$, 
and whenever $(\otR, \key, \val) \in \fp$, then $\max_{<}(\vi(\key)) = (\val, \stub, \stub)$. 
The first two facts are a consequence of \cref{lem:cut.views}, $\mkvs_{\tr'} = \mkvs_{\cut(\aexec, i')}$, and  $\mkvs_{\cut(\aexec, i)}$. 
The last one that if $(\otR, \key, \val) \in \fp$ then $\max_{<}(\vi(\key)) = (\val, \stub, \stub)$ follows the fact that 
$\aexec$ satisfies the last write wins policy and the fact that $\vi = \getView(\VIS^{-1}_{\aexec}(\txid_{i}))$.
\ac{Again, the proof is really loose here, mostly because I got bored.}
\end{enumerate} 

\end{itemize}
\end{proof}

\begin{lemma}[Monotonic \( \getView \)]
\label{lem:getView.monotone}
Let $\aexec$ be an abstract execution, and let $\txidset_1 \subseteq \txidset_2 \subseteq \txidset_{\aexec}$. 
Then $\getView(\aexec, \txidset_1) \viewleq \getView(\aexec, \txidset_2)$.
\end{lemma}
\begin{proof}
Fix $\key \in \Keys$. By definition  
\[
\begin{array}{l}
    \getView(\aexec, \txidset_1)(\key) = \Set{0} \cup \Setcon{ i }{\wtOf(\mkvs_{\aexec}(\key, i)) \in \txidset_1} \\
    \quad \subseteq \Set{0} \cup \Setcon{i }{ \wtOf(\mkvs_{\aexec}(\key, i)) \in \txidset_2} \\
\quad = \getView(\aexec, \txidset_2)(\key)
\end{array}
\]
then it follows that  $\getView(\aexec, \txidset_1) \viewleq \getView(\aexec, \txidset_2)$.
\end{proof}

\begin{lemma}[Valid view on cut of abstract execution]
\label{lem:cut.views}
Let $\aexec$ be an abstract execution, with $\txidset_{\aexec} = \Set{\txid_{i}}_{i = 1}^{n}$ for 
$n = \lvert \txidset_{\aexec} \rvert$, and \( i : 0 \leq i < n\) such that $\txid_{i} \toEDGE{\AR_{\aexec}} \txid_{i+1}$.
Assuming $\txidset_{0} = \emptyset$, and $\txidset_{i} \subseteq (\AR^{-1})\rflx(\txid_{i})$ for $i : 0 \leq i \leq n$,
then $\getView(\aexec, \txidset_{i}) \in \Views(\mkvs_{\cut(\aexec, i)})$.
\end{lemma}

\begin{proof}
We prove by induction on the index $i$. 
\begin{itemize}
\item \caseB{$i = 0$} It follows $\txidset_{0} = \emptyset$, and $\getView(\aexec, \txidset_{0}) = \lambda \key. \Set{0}$. 
We also have that $\mkvs_{\cut(\aexec, 0)} = \lambda \key. \List{(\val_0, \txid_{0}, \emptyset)}$, hence 
it is immediate to see that $\getView(\aexec, \txidset_{0}) \in \Views(\mkvs_{\cut(\aexec, 0)})$.

\item \caseI{$i = i'+1$}
Suppose that for any $\txidset \subseteq (\AR_{\aexec}^{-1})\rflx(\txid_{i'})$, 
then $\getView(\aexec, \txidset) \in \Views(\mkvs_{\cut(\aexec, i)})$. 
Let consider the set $\txidset_{i}$.
Note that, because of \cref{prop:extend.update.sameop}, we have that
\[
\begin{array}{l}
\mkvs_{\cut(\aexec, i)} =
\mkvs_{\extend(\cut(\aexec, i'), \txid_{i}, \VIS^{-1}_{\aexec}(\txid_{i}), \TtoOp{T}_{\aexec}(\txid_{i})} 
= \updateKV(\mkvs_{\cut(\aexec, i')}, \getView(\VIS^{-1}_{\aexec}(\txid_{i})), \TtoOp{T}_{\aexec}(\txid_{i}), \txid_{i})
\end{array}
\]
There are two possibilities:
\begin{itemize}
\item $\txid_{i} \notin \txidset_{i}$, where case $\txidset_{i} \subseteq (\AR_{\aexec}^{-1})\rflx(\txid_{i'})$.
From the inductive hypothesis we get $\getView(\aexec, \txidset_{i}) \in \Views(\mkvs_{\cut(\aexec, i')})$. 
Note that $\mkvs_{\cut(\aexec, i')}$ only contains the transactions identifiers from $\txid_{1}$ to $\txid_{i'}$;
in particular, it does not contain $\txid_{i}$. 
Because $\mkvs_{\cut(\aexec, i)} = \updateKV(\mkvs_{\cut(\aexec, i')}, \stub, \stub, \txid_{i})$, 
then by \cref{lem:updatekv.preserveviews} we have that $\getView(\aexec, \txid_{i}) \in \Views(\mkvs_{\cut(\aexec, i)})$.

\item $\txid \in \txidset_{i}$. Note that for any key $\key$ such that 
$(\otW, \key, \stub) \notin \TtoOp{T}_{\aexec}(\txid_{i})$, then 
$\getView(\aexec, \txidset_{i})(\key) = \getView(\aexec, \txidset_{i} \setminus \Set{\txid_{i}})(\key)$; 
and for any key $\key$ such that $(\otW, \key, \stub) \in \TtoOp{T}_{\aexec}(\txid_{i})$, 
then $\getView(\aexec, \txidset_{i}(\key) = \getView(\aexec, \txidset_{i} \setminus \Set{\txid_{i}})(\key) 
\cup \Setcon{j }{ \wtOf(\mkvs_{\aexec}(\key, i)) = \txid_{j}}$. 
In the last case, the index $j$ must be such that $j < \lvert \mkvs_{\cut(\aexec, i)} \rvert - 1$, 
because we know that $\txid_{i} \in \mkvs_{\cut(\aexec, i)}$. 
It follows from this fact and the inductive hypothesis, 
that $\getView(\aexec, \txidset_{i}) \in \Views(\mkvs_{\cut(\aexec, i)})$.
\ac{This is a loose proof sketch.} 
\end{itemize}
\end{itemize}
\end{proof}

\begin{lemma}[\(\updateKV \) preserving views]
\label{lem:updatekv.preserveviews}
Given a kv-store $\mkvs$, a transactions $\txid \notin \mkvs$, views $\vi, \vi' \in \Views(\mkvs)$, 
and set of operations $\fp$, then $\vi \in \updateKV(\mkvs, \vi', \fp, \txid)$.
\end{lemma}

\begin{proof}
Immediate from the definition of $\updateKV$. Note that $\txid \notin \mkvs$ ensures that 
$\vi$ still satisfies \eqref{eq:view.atomic} with respect to the new kv-store $\updateKV(\mkvs, \vi', \fp, \txid)$.
\end{proof}

\subsection{Galois connection}
\label{sec:galois-kv-aexec}
Now we can prove \cref{thm:kvtrace2aexec}, which we rephrase in \cref{thm:kvtrace2aexec.app}.
The last statement in \cref{thm:kvtrace2aexec} implies that there is a \emph{Galois connection}
between the set of $\ET_{\top}$-traces, and the set of abstract executions that satisfy the 
last write wins policy. The lower and upper adjoints of this connection are the 
lifting of the functions $\aeset(\cdot)$ and $\KVtrace(\cdot)$ to sets of $\ET_{\top}$-traces 
and abstract executions, respectively. However, these two sets are not isomorphic: 
when converting a set of abstract executions into kv-traces, we abstract away the 
pairs $\txid \toEDGE{\VIS} \txid'$ in the visibility relation where $\txid$ is a read-only transaction.
When converting a $\ET_{\top}$-trace into a set of abstract executions, 
we (partially) lose the information about the views of clients immediately after it executed a transaction.

\begin{theorem}[Galois connection between kv-store trace and abstract execution]
\label{thm:kvtrace2aexec.app}
Given a $\ET_{\top}$-trace $\tr$, there exists a set of abstract executions $\aeset({\tr})$ 
such that $\lastConf(\tr) = (\mkvs, \_)$ for any $\mkvs \in \aeset(\tr)$.
Given an abstract execution $\aexec$ satisfying the last write wins resolution policy,
there exists a set of $\ET_{\top}$-traces $\KVtrace({\aexec})$ in normal form
such that $\lastConf(\tr) = (\mkvs_{\aexec}, \_)$ for any $\tr \in \KVtrace({\aexec})$.
\end{theorem}
\begin{proof}
    It can be derived from \cref{prop:kvtrace2aexec} and \cref{prop:aexec2kvtrace}.
\end{proof}

\begin{corollary} 
\label{cor:kvtrace2aexec}
$\CMs(\ET_{\top}) = \Set{\mkvs_{\aexec} }[ \aexec \text{ satisfies } \RP_{\LWW}]$.
\end{corollary}
\begin{proof}
    It follows by \cref{thm:kvtrace2aexec.app}.
\end{proof}


% adequate between kv and abstract execution
\section{The Sound and Complete Constructors of the KV-Store Semantics with Respect to Abstract Executions}
\label{sec:kv-sound-complete-proof}

In this Section we first define the set of $\ET$-traces generated by a program $\prog$. 
Then we prove correctness our semantics on kv-stores,
meaning that if a program $\prog$ executing under the execution 
test $\ET$ terminates in a state $(\hh, \_)$, then $\hh \in \CMs(\ET)$. 

\subsection{Traces of Programs under KV-Stores}
\label{sec:kv-sound-complete-theorem}

The \( \Ptraces(\ET, \prog) \) is the set of all possible traces generated by the program \( \prog \)
starting from the initial configuration \( ( \mkvs_0, \vienv_0 ) \).

\begin{definition}
Given an execution test $\ET$ a program $\prog$ and a state 
$(\mkvs, \vienv, \thdenv)$, the  $\Ptraces(\ET, \prog, \mkvs, \vienv, \thdenv)$ function
is defined as the smallest set such that:
\begin{itemize}
\item $(\mkvs, \vienv) \in \Ptraces(\ET, \prog, \mkvs, \vienv, \thdenv)$
\item if $\tr \in \Ptraces(\ET, \prog', \mkvs', \vienv',\thdenv')$
and $((\mkvs, \vienv, \thdenv) , \prog) \toCMD{(\cl, \iota)}_{\ET} (\mkvs', \vienv', \thdenv')$, 
then $\tr \in \Ptraces(\ET, \prog, \mkvs, \vienv, \thdenv')$
\item if $\tr \in \Ptraces(\ET, \prog', \mkvs', \vienv', \thdenv')$ and 
$(\mkvs, \vienv, \thdenv), \prog) \toCMD{(\cl, \vi, \fp)} ((\mkvs', \vienv', \thdenv'), \prog')$,  
then 
\[
\left( 
\begin{array}{l}
(\mkvs, \vienv) \toET{(\cl, \varepsilon)}
(\mkvs, \vienv\rmto{\cl}{\vi}) \toET{(\cl, \fp)} \tr 
\end{array}
\right) \in \Ptraces(\ET, \prog, \mkvs, \vienv, \thdenv)
\]
\end{itemize}
The set of traces generated by a program $\prog$ under the execution test $\ET$ is 
then defined as $\Ptraces(\ET, \prog) \defeq \Ptraces(\ET, \prog, \mkvs_{0}, \vienv_{0}, \thdenv_{0})$, 
where $\vienv_{0} = \lambda \cl \in \dom(\prog).\lambda \key.\Set{0}$, and 
$\thdenv_{0} = \lambda \cl \in \dom(\prog).\lambda a.0$.
\end{definition}


\begin{proposition}
\label{prop:program-trace-in-et-trace}
For any program $\prog$ and execution test $\ET$, 
$\Ptraces(\ET, \prog) \subseteq \confOf[\ET]$ and $\tr \in \Ptraces(\ET, \prog)$ is in normal form. 
\end{proposition}
\begin{proof}
    First, by the definition of \( \Ptraces \), 
    it only constructs trace in normal form.
    It is easy to prove that for any trace \( \tau \) in \( \Ptraces(\ET, \prog) \), by induction on the trace length,
    the trace is also in \( \confOf[\ET] \).
\end{proof}

\begin{corollary}
If a trace in the following form
\[
    (\mkvs_{0}, \vienv_{0}, \thdenv_{0}), \prog) \toTRANS_{\ET} \cdots \toTRANS_{\ET} 
    (\mkvs, \vienv, \thdenv, \lambda \cl \in \dom(\prog). \pskip)
\]
then $\mkvs \in \CMs(\ET)$.
\end{corollary}
\begin{proof}
    By the definition of \( \Ptraces \), 
    there exists a corresponding trace \( \tau \in \Ptraces(\ET, \prog) \).
    By \cref{prop:program-trace-in-et-trace}, such trace \( \tau \in \confOf[\ET] \),
    therefore \( \mkvs \in \CMs(\ET)\) by definition of \( \CMs(\ET) \).
\end{proof}

Similar to \( \interpr{\prog}_{(\RP,\Ax)} \) (\cref{def:axiom-to-prog}), the function \( \interpr{\prog}_{\ET} \) is defined as the following:
\[
    \interpr{\prog}_{\ET} = \Set{ \mkvs }[ (\mkvs_{0}, \vienv_0 \thdenv_{0} ), \prog \toPROG{\stub}_{\ET}^{\ast} (\mkvs, \stub, \prog_{f}) ]
\]
where $\thdenv_{0} = \lambda \cl \in \dom(\prog).\lambda \vx.0$ and $\prog_{f} = \lambda \cl \in \dom(\prog).\pskip$.

\begin{proposition}
    \label{thm:consistency-intersect-permissive}
    For any program $\prog$ and execution test $\ET$:
    \( \interpr{\prog}_{\ET} = \interpr{\prog}_{\ET_\top}  \cap \CMs(\ET) \).
\end{proposition}
\begin{proof}
    We prove a stronger result that for any program $\prog$ and execution test $\ET$, $\Ptraces(\ET, \prog) = \Ptraces(\ET_{\top}, \prog) \cap \confOf[\ET]$.
    It is easy to see \(\Ptraces(\ET, \prog) \subseteq \Ptraces(\ET_\top, \prog) \).
    By \cref{prop:program-trace-in-et-trace}, we know \( \Ptraces(\ET, \prog) \subseteq \confOf[\ET]\).
    Therefore \(  \Ptraces(\ET, \prog) \subseteq \Ptraces(\ET_\top, \prog) \cap \confOf[\ET] \).

    Let consider a trace \( \tau \) in \( \Ptraces(\ET_\top, \prog) \cap \confOf[\ET] \).
    By inductions on the length of trace, 
    every step that commits a new transaction  must satisfy \( \ET \) as \( \tau \in \confOf[\ET] \).
    It also reduce the program \( \prog \) since \( \tau \in \Ptraces(\ET_\top, \prog) \).
    By the definition \( \Ptraces(\ET, \prog) \), we can construct the same trace \( \tau \) so that \( \tau \in \Ptraces(\ET, \prog) \).
\end{proof}

\section{The Soundness and Completeness of the Key-value Semantics with Respect to Abstract Executions}

\label{sec:kv-sound-complete-proof}
In this Section we first define the set of $\ET$-traces generated by a program $\prog$. 
Then we prove correctness our semantics on key-value store,
meaning that if a program $\prog$ executing under the execution 
test $\ET$ terminates in a state $(\hh, \_)$, then $\hh \in \CMs(\ET)$. 

\subsection{Traces of Programs under KV-Stores}
\label{sec:kv-sound-complete-theorem}

The \( \Ptraces(\ET, \prog) \) is the set of all possible traces generated by the program \( \prog \)
starting from the initial configuration \( ( \mkvs_0, \vienv_0 ) \).

\begin{definition}
Given an execution test $\ET$ a program $\prog$ and a state 
$(\mkvs, \vienv, \thdenv)$, the  $\Ptraces(\ET, \prog, \mkvs, \vienv, \thdenv)$ function
is defined as the smallest set such that:
\begin{itemize}
\item $(\mkvs, \vienv) \in \Ptraces(\ET, \prog, \mkvs, \vienv, \thdenv)$
\item if $\tr \in \Ptraces(\ET, \prog', \mkvs', \vienv',\thdenv')$
and $((\mkvs, \vienv, \thdenv) , \prog) \toCMD{(\cl, \iota)}_{\ET} (\mkvs', \vienv', \thdenv')$, 
then $\tr \in \Ptraces(\ET, \prog, \mkvs, \vienv, \thdenv')$
\item if $\tr \in \Ptraces(\ET, \prog', \mkvs', \vienv', \thdenv')$ and 
$(\mkvs, \vienv, \thdenv), \prog) \toCMD{(\cl, \vi, \fp)} ((\mkvs', \vienv', \thdenv'), \prog')$,  
then 
\[
\left( 
\begin{array}{l}
(\mkvs, \vienv) \toET{(\cl, \varepsilon)}
(\mkvs, \vienv\rmto{\cl}{\vi}) \toET{(\cl, \fp)} \tr 
\end{array}
\right) \in \Ptraces(\ET, \prog, \mkvs, \vienv, \thdenv)
\]
\end{itemize}
The set of traces generated by a program $\prog$ under the execution test $\ET$ is 
then defined as $\Ptraces(\ET, \prog) \defeq \Ptraces(\ET, \prog, \mkvs_{0}, \vienv_{0}, \thdenv_{0})$, 
where $\vienv_{0} = \lambda \cl \in \dom(\prog).\lambda \key.\Set{0}$, and 
$\thdenv_{0} = \lambda \cl \in \dom(\prog).\lambda a.0$.
\end{definition}


\begin{proposition}
\label{prop:program-trace-in-et-trace}
For any program $\prog$ and execution test $\ET$, 
$\Ptraces(\ET, \prog) \subseteq \confOf[\ET]$ and $\tr \in \Ptraces(\ET, \prog)$ is in normal form. 
\end{proposition}
\begin{proof}
    First, by the definition of \( \Ptraces \), 
    it only constructs trace in normal form.
    It is easy to prove that for any trace \( \tau \) in \( \Ptraces(\ET, \prog) \), by induction on the trace length,
    the trace is also in \( \confOf[\ET] \).
\end{proof}

\begin{corollary}
If a trace in the following form
\[
    (\mkvs_{0}, \vienv_{0}, \thdenv_{0}), \prog) \toTRANS_{\ET} \cdots \toTRANS_{\ET} 
    (\mkvs, \vienv, \thdenv, \lambda \cl \in \dom(\prog). \pskip)
\]
then $\mkvs \in \CMs(\ET)$.
\end{corollary}
\begin{proof}
    By the definition of \( \Ptraces \), 
    there exists a corresponding trace \( \tau \in \Ptraces(\ET, \prog) \).
    By \cref{prop:program-trace-in-et-trace}, such trace \( \tau \in \confOf[\ET] \),
    therefore \( \mkvs \in \CMs(\ET)\) by definition of \( \CMs(\ET) \).
\end{proof}

Similar to \( \interpr{\prog}_{(\RP,\Ax)} \) (\cref{def:axiom-to-prog}), the function \( \interpr{\prog}_{\ET} \) is defined as the following:
\[
    \interpr{\prog}_{\ET} = \Set{ \mkvs }[ (\mkvs_{0}, \vienv_0 \thdenv_{0} ), \prog \toPROG{\stub}_{\ET}^{\ast} (\mkvs, \stub, \prog_{f}) ]
\]
where $\thdenv_{0} = \lambda \cl \in \dom(\prog).\lambda \vx.0$ and $\prog_{f} = \lambda \cl \in \dom(\prog).\pskip$.

\begin{proposition}
    \label{thm:consistency-intersect-permissive}
    For any program $\prog$ and execution test $\ET$:
    \( \interpr{\prog}_{\ET} = \interpr{\prog}_{\ET_\top}  \cap \CMs(\ET) \).
\end{proposition}
\begin{proof}
    We prove a stronger result that for any program $\prog$ and execution test $\ET$, $\Ptraces(\ET, \prog) = \Ptraces(\ET_{\top}, \prog) \cap \confOf[\ET]$.
    It is easy to see \(\Ptraces(\ET, \prog) \subseteq \Ptraces(\ET_\top, \prog) \).
    By \cref{prop:program-trace-in-et-trace}, we know \( \Ptraces(\ET, \prog) \subseteq \confOf[\ET]\).
    Therefore \(  \Ptraces(\ET, \prog) \subseteq \Ptraces(\ET_\top, \prog) \cap \confOf[\ET] \).

    Let consider a trace \( \tau \) in \( \Ptraces(\ET_\top, \prog) \cap \confOf[\ET] \).
    By inductions on the length of trace, 
    every step that commits a new transaction  must satisfy \( \ET \) as \( \tau \in \confOf[\ET] \).
    It also reduce the program \( \prog \) since \( \tau \in \Ptraces(\ET_\top, \prog) \).
    By the definition \( \Ptraces(\ET, \prog) \), we can construct the same trace \( \tau \) so that \( \tau \in \Ptraces(\ET, \prog) \).
\end{proof}

\section{The Soundness and Completeness of the Key-value Semantics with Respect to Abstract Executions}

\label{sec:kv-sound-complete-proof}
In this Section we first define the set of $\ET$-traces generated by a program $\prog$. 
Then we prove correctness our semantics on key-value store,
meaning that if a program $\prog$ executing under the execution 
test $\ET$ terminates in a state $(\hh, \_)$, then $\hh \in \CMs(\ET)$. 

\subsection{Traces of Programs under KV-Stores}
\label{sec:kv-sound-complete-theorem}

The \( \Ptraces(\ET, \prog) \) is the set of all possible traces generated by the program \( \prog \)
starting from the initial configuration \( ( \mkvs_0, \vienv_0 ) \).

\begin{definition}
Given an execution test $\ET$ a program $\prog$ and a state 
$(\mkvs, \vienv, \thdenv)$, the  $\Ptraces(\ET, \prog, \mkvs, \vienv, \thdenv)$ function
is defined as the smallest set such that:
\begin{itemize}
\item $(\mkvs, \vienv) \in \Ptraces(\ET, \prog, \mkvs, \vienv, \thdenv)$
\item if $\tr \in \Ptraces(\ET, \prog', \mkvs', \vienv',\thdenv')$
and $((\mkvs, \vienv, \thdenv) , \prog) \toCMD{(\cl, \iota)}_{\ET} (\mkvs', \vienv', \thdenv')$, 
then $\tr \in \Ptraces(\ET, \prog, \mkvs, \vienv, \thdenv')$
\item if $\tr \in \Ptraces(\ET, \prog', \mkvs', \vienv', \thdenv')$ and 
$(\mkvs, \vienv, \thdenv), \prog) \toCMD{(\cl, \vi, \fp)} ((\mkvs', \vienv', \thdenv'), \prog')$,  
then 
\[
\left( 
\begin{array}{l}
(\mkvs, \vienv) \toET{(\cl, \varepsilon)}
(\mkvs, \vienv\rmto{\cl}{\vi}) \toET{(\cl, \fp)} \tr 
\end{array}
\right) \in \Ptraces(\ET, \prog, \mkvs, \vienv, \thdenv)
\]
\end{itemize}
The set of traces generated by a program $\prog$ under the execution test $\ET$ is 
then defined as $\Ptraces(\ET, \prog) \defeq \Ptraces(\ET, \prog, \mkvs_{0}, \vienv_{0}, \thdenv_{0})$, 
where $\vienv_{0} = \lambda \cl \in \dom(\prog).\lambda \key.\Set{0}$, and 
$\thdenv_{0} = \lambda \cl \in \dom(\prog).\lambda a.0$.
\end{definition}


\begin{proposition}
\label{prop:program-trace-in-et-trace}
For any program $\prog$ and execution test $\ET$, 
$\Ptraces(\ET, \prog) \subseteq \confOf[\ET]$ and $\tr \in \Ptraces(\ET, \prog)$ is in normal form. 
\end{proposition}
\begin{proof}
    First, by the definition of \( \Ptraces \), 
    it only constructs trace in normal form.
    It is easy to prove that for any trace \( \tau \) in \( \Ptraces(\ET, \prog) \), by induction on the trace length,
    the trace is also in \( \confOf[\ET] \).
\end{proof}

\begin{corollary}
If a trace in the following form
\[
    (\mkvs_{0}, \vienv_{0}, \thdenv_{0}), \prog) \toTRANS_{\ET} \cdots \toTRANS_{\ET} 
    (\mkvs, \vienv, \thdenv, \lambda \cl \in \dom(\prog). \pskip)
\]
then $\mkvs \in \CMs(\ET)$.
\end{corollary}
\begin{proof}
    By the definition of \( \Ptraces \), 
    there exists a corresponding trace \( \tau \in \Ptraces(\ET, \prog) \).
    By \cref{prop:program-trace-in-et-trace}, such trace \( \tau \in \confOf[\ET] \),
    therefore \( \mkvs \in \CMs(\ET)\) by definition of \( \CMs(\ET) \).
\end{proof}

Similar to \( \interpr{\prog}_{(\RP,\Ax)} \) (\cref{def:axiom-to-prog}), the function \( \interpr{\prog}_{\ET} \) is defined as the following:
\[
    \interpr{\prog}_{\ET} = \Set{ \mkvs }[ (\mkvs_{0}, \vienv_0 \thdenv_{0} ), \prog \toPROG{\stub}_{\ET}^{\ast} (\mkvs, \stub, \prog_{f}) ]
\]
where $\thdenv_{0} = \lambda \cl \in \dom(\prog).\lambda \vx.0$ and $\prog_{f} = \lambda \cl \in \dom(\prog).\pskip$.

\begin{proposition}
    \label{thm:consistency-intersect-permissive}
    For any program $\prog$ and execution test $\ET$:
    \( \interpr{\prog}_{\ET} = \interpr{\prog}_{\ET_\top}  \cap \CMs(\ET) \).
\end{proposition}
\begin{proof}
    We prove a stronger result that for any program $\prog$ and execution test $\ET$, $\Ptraces(\ET, \prog) = \Ptraces(\ET_{\top}, \prog) \cap \confOf[\ET]$.
    It is easy to see \(\Ptraces(\ET, \prog) \subseteq \Ptraces(\ET_\top, \prog) \).
    By \cref{prop:program-trace-in-et-trace}, we know \( \Ptraces(\ET, \prog) \subseteq \confOf[\ET]\).
    Therefore \(  \Ptraces(\ET, \prog) \subseteq \Ptraces(\ET_\top, \prog) \cap \confOf[\ET] \).

    Let consider a trace \( \tau \) in \( \Ptraces(\ET_\top, \prog) \cap \confOf[\ET] \).
    By inductions on the length of trace, 
    every step that commits a new transaction  must satisfy \( \ET \) as \( \tau \in \confOf[\ET] \).
    It also reduce the program \( \prog \) since \( \tau \in \Ptraces(\ET_\top, \prog) \).
    By the definition \( \Ptraces(\ET, \prog) \), we can construct the same trace \( \tau \) so that \( \tau \in \Ptraces(\ET, \prog) \).
\end{proof}

\section{The Soundness and Completeness of the Key-value Semantics with Respect to Abstract Executions}

\label{sec:kv-sound-complete-proof}
In this Section we first define the set of $\ET$-traces generated by a program $\prog$. 
Then we prove correctness our semantics on key-value store,
meaning that if a program $\prog$ executing under the execution 
test $\ET$ terminates in a state $(\hh, \_)$, then $\hh \in \CMs(\ET)$. 

\input{\RootPath/adequate/et-prog-trace.tex}
\input{\RootPath/adequate/adequate.tex}
\input{\RootPath/adequate/sound-complete-judgement.tex}

\section{The Soundness and Completeness of Execution Tests}

We now show using \cref{def:et_sound,def:et_complete} to prove the soundness and completeness of execution tests with respect to axiomatic specification.
It is sufficient to match these two definition, 
then by \cref{cor:et-soundness,cor:et-completeness} we have \( \CMs(\ET) = \Setcon{\mkvs_\aexec}{\aexec \in \CMa(\RP_\LWW,\Ax)} \).

\label{sec:kv-sound-complete-proof}
\label{sec:spec-proof}

\input{\RootPath/spec-s-c/mr.tex}
\input{\RootPath/spec-s-c/mw.tex}
\input{\RootPath/spec-s-c/ryw.tex}
\input{\RootPath/spec-s-c/wfr.tex}
\input{\RootPath/spec-s-c/cc.tex}
\input{\RootPath/spec-s-c/ua.tex}
\input{\RootPath/spec-s-c/cp.tex}
\input{\RootPath/spec-s-c/psi.tex}
\input{\RootPath/spec-s-c/si.tex}
\input{\RootPath/spec-s-c/ser.tex}

\subsection{Soundness and Completeness Constructor}
\label{sec:kv2aexec-sound-complete}

We now show how all the results illustrated so far 
can be put together to show that the kv-store operational semantics 
is sound and complete with respect to abstract execution operational semantics.

\subsubsection{Soundness}
Recall that in the abstract execution operational semantics,
a client \( \cl \) loses information of the visible transactions immediately after it commits a transaction.
Yet such information is indirectly presented when the next transaction from the same client is committed.
To define the soundness judgement (\cref{def:et_sound}), we introduce a notation of \emph{invariant} ({def:invariant-for-clients})
to encore constraints on the visible transactions after each commit.

\ac{The idea behind client-based invariant being that $I(\aexec, \cl)$ represents 
the minimal set of transactions that $\cl$ must see in $\aexec$, before 
updating the view and performing a transaction. Such a set of transaction 
roughly correspond to the view of the client before performing a 
sequence of \emph{update view+execute transaction} operations, 
or equivalently from the view obtained after the execution of the 
last transaction from that client.}

\begin{definition}[Invariant for clients]
\label{def:invariant-for-clients}
A \emph{client-based invariant condition}, or simply \emph{invariant}, is a 
function $I : \Aexecs \times \Clients \rightarrow \powerset{\TxID}$ 
such that for any $\cl$ we have that $I(\aexec, \cl) \subseteq \T_{\aexec}$, and 
for any  $\cl'$ such that $\cl' \neq \cl$ we have that 
$I(\extend(\aexec, \txid_{\cl'}^{\cdot}, \stub, \stub), \cl) = I(\aexec, \cl)$.
\end{definition}



\begin{definition}[Soundness judgement]
\label{def:et_sound}
An execution test $\ET$ is sound with respect to an axiomatic 
definition $(\RP_{\LWW}, \Ax)$ if and only if
there exists an invariant condition $I$ such that 
if assuming that
\begin{itemize}
    \item a client \( \cl \) having an initial view \( \vi \), 
        commits a transaction \( \txid \) with a fingerprint \( \fp \) and updates the view to \( \vi' \), 
        which is allowed by \( \ET \) \ie $\ET \vdash (\mkvs, \vi) \csat \fp: (\mkvs',\vi')$ where \( \mkvs' = \updKV{\mkvs, \vi ,\fp, \txid}\),
    \item a $\aexec$ such that $\hh_{\aexec} = \mkvs$ and $I(\aexec, \cl) \subseteq \Tx(\mkvs, \vi)$,
\end{itemize}
then there exist a set of read-only transactions $\T_{\rd}$ such that 
\begin{itemize}
\item the view \( \vi \) satisfies \( \Ax \), \ie $\forall \A \in \Ax. \Setcon{\txid' }{ (\txid', \txid) \in \A(\aexec')} \subseteq \Tx(\mkvs, \vi) \cup \T_{\rd}$, 
\item the invariant is preserved, \ie $I(\aexec', \cl) \subseteq \Tx(\mkvs', \vi')$ for some \( \aexec' \) that \( \mkvs' = \hh_{\aexec'}\)
\end{itemize}
\end{definition}

\begin{theorem}[Soundness]
\label{thm:et_soundness}
If $\ET$ is sound with respect to $(\RP_{\LWW}, \Ax)$, then 
\[
    \CMs(\ET) \subseteq \Setcon{ \mkvs }{ \exists \aexec \in \CMa(\RP_{\LWW}, \Ax)).\;\hh_{\aexec} = \mkvs}
\]
\end{theorem}
\begin{proof}
Let $\ET$ be an execution test that is sound with respect to an 
axiomatic definition $(\RP_{\LWW}, \Ax)$. Let $I$ be 
the invariant that satisfies \cref{def:et_sound}. 
Let consider an $\ET$-trace $\tr$.
Because of \cref{prop:et.normalform}, we can assume that $\tr$ is in normal form. 
Without lose generality, we can also assume that the trace does not have transitions labelled as $(\stub, \emptyset)$.
Thus we have that the following trace \( \tr \):
\[
\begin{rclarray}
\tr & = & (\hh_{0}, \viewFun_{0}) \xrightarrowtriangle{(\cl_{0}, \varepsilon)} (\hh_{0}, \viewFun_{0}') 
\xrightarrowtriangle{(\cl_{0}, \fp_{0})} 
(\hh_1, \viewFun_{1}) \xrightarrowtriangle{(\cl_1, \varepsilon)}  \cdots
\xrightarrowtriangle{(\cl_{n-1}, \fp_{n-1})} (\hh_{n}, \viewFun_{n}).
\end{rclarray}
\]
For any $i : 0 \leq i \leq n$, let $\tr_{i}$ be the prefix of $\tr$ that 
contains only the first $2i$ transitions. 
Clearly $\tr_{i}$ is a valid $\ET$-trace, and it is also a $\ET_{\top}$-trace. 
By \cref{prop:kvtrace2aexec}, 
any abstract execution $\aexec_{i} \in \aeset(\tr_{i})$ satisfies the last write wins policy. 
We show by induction on $i$ that we can always find 
an abstract execution $\aexec_{i} \in \aeset(\tr_{i})$ such that $\aexec_i \models \Ax$ and $I(\aexec_{i}, \cl) \subseteq \T^{i}_{\cl}$
for any client $\cl$ and set of transactions 
$\T^{i}_{\cl} = \Tx(\aexec_{i}, \viewFun_{i}(\cl)) \cup \T^{i}_\rd$, 
and read-only transactions $\T_\rd^{i}$ in $\aexec_{i}$.
If so, because $\aexec_{i}$ satisfies the last write wins policy,
then it must be the case that $\aexec_{i} \models (\RP_{\LWW}, \Ax)$. 
Then by choosing $i = n$, we will obtain that $\aexec_{n} \models (\RP_{\LWW}, \Ax)$. 
Last, by \cref{prop:kvtrace2aexec}, $\hh_{\aexec_{n}} = \hh_{n}$, and there is nothing left to prove.
Now let prove such $\aexec_{i} \in \aeset(\tr_{i})$ always exists.

\caseB{$i = 0$} 
Let $\aexec_{0}$ be the only abstract execution included in $\aeset(\tr_{0})$, 
that is $\aexec_{0} = ([], \emptyset, \emptyset)$. 
For any $\A \in \Ax$, it must be the case that 
$\A(\aexec_{0}) \subseteq \T_{\aexec_{0}} = \emptyset$, 
hence the inequation $\A(\aexec_{0}) \subseteq \VIS_{\aexec_{0}}$ is trivially satisfies.
Furthermore, for the client invariant $I$ we also require that $I(\aexec_{0}, \stub) \subseteq \T_{\aexec_{0}} = \emptyset$; 
for any client $\cl$ we can choose $\T_{\cl}^{0} = \Tx(\viewFun_{0}(\cl)) \cup \emptyset = \emptyset$. 
Therefore $I(\aexec_{0}, \cl) = \emptyset \subseteq \emptyset = \T_{\cl}^{0}$.

\caseI{$i' = i + 1$ where $i < n$}
By the inductive hypothesis, there exists an abstract execution $\aexec_i$ such that  
\begin{itemize}
\item $\aexec_{i} \models \A$ for all $\A \in \Ax$, and 
\item $I(\aexec, \cl) \subseteq \T_{\cl}^{i}$ for any client $\cl$ and set of transactions $\T_{\cl}^{i} = \Tx(\hh_{i}, \viewFun_{i}(\cl))$.
\end{itemize}

We have two transitions to check, the view shift and committing a transaction.
\begin{itemize}
\item the view shift transition $(\hh_{i}, \viewFun_{i}) \xrightarrowtriangle{(\cl_{i}, \varepsilon)} (\hh_{i}, \viewFun'_{i})$. 
By definition, it must be the case that $\viewFun'_{i} = \viewFun_{i}\rmto{\cl}{\vi'_{i}}$ 
for some $\vi'_{i}$ such that $\viewFun_{i}(\cl) \viewleq \vi'_{i}$.
Let $(\T_{\cl}^{i})' = \Tx(\hh_{i}, \vi'_{i})$; then we have 
\(
\T_{\cl}^{i} = \Tx(\hh_{i}, \viewFun_{i}(\cl)) \subseteq \Tx(\hh_{i}, \vi'_{i}) = (\T_{\cl}^{i})'
\)
As a consequence, $I(\aexec, \cl) \subseteq \T_{\cl}^{i} \subseteq (\T_{\cl}^{i})'$.

\item the commit transaction transition $(\hh_{i}, \viewFun_{i}') \xrightarrowtriangle{(\cl_{i}, \fp_{i})}_{\ET} 
(\hh_{i+1}, \viewFun_{i+1})$.
A necessary condition for this transition 
to appear in $\tr$ is that $\ET \vdash (\hh_{i}, \viewFun(\cl)) \triangleright \fp_{i}: (\mkvs_{i+1},\viewFun_{i+1}(\cl))$. 
Because $I$ is the invariant to derive that $\ET$ is sound with respect to $\Ax$, 
and because $I(\aexec_{i}, \cl_{i}) \subseteq (\T^{i}_{\cl})'$, 
then by \cref{def:et_sound} we have the following:
\begin{itemize}
\item there exists a set of read-only transactions $\T_\rd$ 
    such that 
    \[
        \Setcon{\txid' }{ (\txid', \txid_{(\cl, i)}) \in \A(\aexec_{i+1})} \subseteq (\T^{i}_{\cl})' \cup \T_\rd
    \]
where 
$\txid_{(\cl, i)} \in \nextTxid(\hh_{i}, \cl)$
and $\aexec_{i+1} = \extend(\aexec_{i}, \txid_{(\cl, i)}, (\T^{i}_{\cl})' \cup \T_\rd, \fp_{i})$,
\item  $I(\aexec_{i+1}, \cl) \subseteq \Tx(\hh_{i+1}, \viewFun_{i+1}(\cl))$.
\end{itemize} 
Because $\aexec_{i} \in \aeset(\tr_{i})$, by definition of $\aeset(\stub)$ we have that 
$\aexec_{i+1} \in \aeset(\tr)$ (under the assumption that $\fp_{i} \neq \emptyset$), 
and because $\lastConf(\tr_{i+1}) = (\hh_{i+1}, \stub)$, then $\hh_{\aexec_{i+1}} = \hh_{i+1}$. 

Now we need to check if \( \aexec_{i+1} \) satisfies \( \Ax\) and the invariant \( I \) is preserved.
\begin{itemize}
\item $\A(\aexec_{i+1}) \subseteq \VIS_{\aexec}^{i+1}$ for any $\A \in \Ax$.
Fix $\A \in \Ax$ and $(\txid', \txid) \in \A(\aexec_{i+1})$. 
Because $\aexec_{i+1} = \extend(\aexec_{i}, \txid_{(\cl, i)}, (\T_{\cl}^{i})' \cup \T_\rd, \fp_{i}$), 
we distinguish between two cases.
\begin{itemize}
\item If $\txid = \txid_{(\cl, i)}$, then it must be the case that $\txid' \in (\T^{i}_{\cl})' \cup \T_\rd$, 
and by definition of $\extend(\stub)$ we have that $(\txid' ,\txid_{(\cl, i)}) \in \VIS_{\aexec_{i+1}}$. 
\item If $\txid \neq \txid_{(\cl, i)}$, then we have that $\txid, \txid' \in \T_{\aexec_{i}}$. 
Because $\aexec_{i}$ and $\aexec_{i+1}$ agree on $\T_{\aexec_{i}}$, then $(\txid', \txid) \in \A(\aexec_{i})$.
Because $\aexec_{i} \models \A$, then $(\txid', \txid) \in \VIS_{\aexec_{i}}$. 
By definition of $\extend$, it follows that $(\txid', \txid) \in \VIS_{\aexec_{i+1}}$.
\end{itemize}

\item Finally, we show the invariant is preserved.
Fix a client $\cl'$. 
\begin{itemize}
\item If $\cl' = \cl$, then we have already proved that 
$I(\aexec_{i+1}, \cl) \subseteq \T_{\cl}^{i+1}$. 
\item if $\cl' \neq \cl$, then note that $\viewFun_{i}(\cl') = \viewFun'_{i}(\cl') = \viewFun_{i+1}(\cl')$, 
and in particular $(\T^{\cl'}_{i})' = \Tx(\aexec_{i}, \viewFun'_{i}(\cl')) = \Tx(\aexec_{i+1}, \viewFun_{i+1}(\cl')) =  \T_{\cl'}^{i+1}$.
By the inductive hypothesis we know that $I(\aexec_{i}, \cl) \subseteq \T_{\cl'}^{i}$, 
and by the definition of invariant, we have $I(\aexec_{i+1}, \cl) \subseteq \T_{\cl'}^{i} = \T_{\cl'}^{i+1}$. 
\end{itemize}
\end{itemize}
\end{itemize}
\end{proof}

\begin{corollary}
\label{cor:et-soundness}
If $\ET$ is sound with respect to $(\RP_{\LWW}, \Ax)$, then 
for any program $\prog$, $\interpr{\prog}_{\ET} \subseteq \Setcon{ \hh_{\aexec} }{ \aexec \in \interpr{\prog}_{(\RP_{\LWW}, \Ax)} }$.
\end{corollary}
\begin{proof}
\[
\begin{rclarray}
\interpr{\prog}_{\ET} 
& \stackrel{\cref{thm:consistency-intersect-permissive}}{=} & 
\interpr{\prog}_{\ET_\top} \cap \CMs(\ET) \\
& \stackrel{\cref{cor:kvtrace2aexec}}{=} & 
\Setcon{\hh_{\aexec} }{ \aexec \text{ satisfies } \RP_{\LWW}} \cap \CMs(\ET) \\
& \stackrel{\cref{thm:et_soundness}}{\subseteq} & 
\Setcon{\hh_{\aexec} }{ \aexec \text{ satisfies } \RP_{\LWW} \land \aexec \in \CMa(\RP_{\LWW}, \Ax) } \\
& \stackrel{\cref{thm:consistency-intersect-anarchic}}{=} &
\Setcon{ \hh_{\aexec} }{ \aexec \in \interpr{\prog}_{(\RP_{\LWW}, \Ax)} }
\end{rclarray}
\]
\end{proof}

\subsubsection{Completeness}
The Completeness judgement is in \cref{def:et_complete}.
Given a transaction \( \txid_i \) from client \( \cl \), it converts the visible transactions \( \VIS_{\aexec}^{-1}(\txid_{i}) \) into view  and such view should satisfy the \( \ET \).
Note that \( \aexec \) does not contain precise information about final view after update,
yet the visible transactions of the immediate next transaction from the same client \( \cl \) include those information.

\begin{definition}
\label{def:et_complete}
An execution test $\ET$ is \emph{complete} with respect 
to an axiomatic definition $(\RP_{\LWW}, \Ax)$ if, for any abstract execution $\aexec \in \CMa(\RP_{\LWW}, \Ax)$ 
and index \( i : 1 \leq i < \abs{\T_{\aexec}}\) such that \( \txid_{i} \toEdge{\AR_{\aexec}} \txid_{i+1} \), there exist an initial view $\vi_{i}$ and a final view $\vi_{i}'$ where 
\begin{itemize}
\item $\vi_{i} = \getView(\aexec, \VIS_{\aexec}^{-1}(\txid_{i}))$, 
\item let $\txid_{i} = \txid_{\cl}^{n}$ for some $\cl, n$; 
    \begin{itemize}
        \item if the transaction $\txid_{i}' = \min_{\PO_{\aexec}}\Setcon{\txid' }{ \txid_i \xrightarrow{\PO_{\aexec}} \txid'}$ is defined, then $\vi' = \getView(\aexec, \T_{i})$ where $\T_{i} \subseteq (\AR_{\aexec}^{-1})?(\txid_{i}) \cap \VIS_{\aexec}^{-1}(\txid_{i}'))$; 
        \item otherwise $\vi' = \getView(\aexec, \T_{i})$ where $\T_{i} \subseteq (\AR_{\aexec}^{-1})?(\txid_{i})$, 
    \end{itemize}
\item $\ET \vdash (\hh_{\cut(\aexec, i-1)}, \vi_{i}) \csat \TtoOp{T}_{\aexec}(\txid_{i}) : (\hh_{\cut(\aexec, i)},\vi_{i}')$.
\end{itemize}
\end{definition}

\begin{theorem}
\label{thm:et_complete}
Let $\ET$ be an execution test that is complete with respect to an axiomatic definition $(\RP_{\LWW}, \Ax)$. 
Then $\CMa(\RP_{\LWW}, \Ax) \subseteq \CMs(\ET)$.
\end{theorem}
\begin{proof}
Fix an abstract execution $\aexec \in \CMa(\RP_{\LWW}, \Ax)$. 
For any \(i : 1 \leq i < \abs{\T_\aexec} \), suppose that \( \txid_i \) that is the i-\emph{th} transaction follows the arbitrary order, \ie $\txid_{i} \xrightarrow{\AR_{\aexec}} \txid_{i+1}$ 
and let $\cl_{i}$ be the client of the i-\emph{th} step, \ie $\txid_{i} = \txid_{\cl_{i}}^{\stub}$.
Because $\ET$ is complete with respect to $(\RP_{\LWW}, \Ax)$, 
for any step $i$ we can find an initial views $\vi_i$,and a final view $\vi'_{i}$ such that 
\begin{itemize}
\item $\vi_i = \getView(\aexec, \VIS^{-1}_{\aexec}(\txid_{i}))$, 
\item there exists a set of transactions $\T_{i}$ such that $\getView(\aexec, \T_{i}) = \vi'_{i}$, and 
either $\min_{\PO_{\aexec}}\Setcon{\txid' }{ \txid_{i} \xrightarrow{\PO_{\aexec}} \txid'}$ is 
is defined and $\T_{i} \subseteq (\AR_{\aexec}^{-1})?(\txid_{i}) \cap \VIS^{-1}_{\aexec}(\txid')$, 
or $\T_{i} \subseteq (\AR_{\aexec}^{-1}?(\txid_{i})$, 
\item $\ET \vdash (\hh_{\cut(\aexec, i-1)}, \vi_i) \triangleright \TtoOp{T}_{\aexec}(\txid_{i}): (\hh_{\cut(\aexec, i)}, \vi'_{i})$.
\end{itemize}
Given above, let $\hh_{i} = \cut(\aexec, i)$ and $\fp_{i} = \TtoOp{T}_{\aexec}(\txid_{i})$. Define the views for clients as 
\[
\viewFun_{0} = \lambda \cl \in \Setcon{\cl' }{ \exsts{ \txid \in \T_{\aexec} } \txid = \txid_{\cl'}} \ldotp \lambda \key \ldotp \Set{0}
\quad \viewFun'_{i-1} = \viewFun_{i}\rmto{\cl_{i}}{ \vi_i}
\quad \viewFun_{i} = \viewFun'_{i-1}\rmto{\cl_{i} }{\vi'_{i}}
\]
and the ke-stores as
\[
\hh_{0} = \lambda \key.(\val_{0}, \txid_{0}, \emptyset)
\quad \hh_{i} = \updateKV(\hh_{i-1}, \vi_i, \fp_{i}, \txid_{i})
\]
Now by \cref{prop:aexec2kvtrace} we have that the following sequence of $\ET_{\top}$-reductions 
\[
\begin{array}{l}
(\hh_{0}, \viewFun_{0}) \xrightarrowtriangle{(\cl_{1}, \varepsilon)}_{\ET_{\top}} (\hh_{0}, \viewFun'_{0}) 
\xrightarrowtriangle{(\cl_{1}, \fp_{1})}_{\ET_{\top}} (\hh_{1}, \viewFun_{1}) 
\xrightarrowtriangle{(\cl_{2}, \varepsilon)}_{\ET_{\top}} 
\cdots \xrightarrowtriangle{(\cl_{n}, \fp_{n})}_{\ET_{\top}} (\hh_{n}, \viewFun_{n})
\end{array}
\]
Note that $\hh_{i} = \hh_{\cut(\aexec, i)}$. 
Because $\ET \vdash ( \hh_{\cut(\aexec,i-1)}, \vi_i ) \csat \fp_{i} : (\hh_{i}, \vi'_{i})$, 
or equivalently $\ET \vdash ( \hh_{\cut(\aexec, i-1)}, \viewFun'_{i-1}(\cl_{i}) ) \csat \fp_{i} : ( \hh_{\cut(\aexec, i-1)}, \viewFun_{i}(\cl_{i}) )$, therefore 
\[
\begin{array}{l}
(\hh_{0}, \viewFun_{0}) \xrightarrowtriangle{(\cl_{1}, \varepsilon)}_{\ET} (\hh_{0}, \viewFun'_{0}) 
\xrightarrowtriangle{(\cl_{1}, \fp_{1})}_{\ET} (\hh_{1}, \viewFun_{1})
\xrightarrowtriangle{(\cl_{2}, \varepsilon)}_{\ET} 
\cdots \xrightarrowtriangle{(\cl_{n}, \fp_{n})}_{\ET} (\hh_{n}, \viewFun_{n})
\end{array}
\]
It follows that $\hh_{n} \in \CMs(\ET)$ then $\hh_{n} = \hh_{\cut(\aexec, n)} = \hh_{\aexec}$, and there is nothing left to prove.
\end{proof}

\begin{corollary}
\label{cor:et-completeness}
If $\ET$ is complete with respect to $(\RP_{\LWW}, \Ax)$, then 
for any program $\prog$, $\Setcon{ \hh_{\aexec} }{ \aexec \in \interpr{\prog}_{(\RP_{\LWW}, \Ax)} } \subseteq \interpr{\prog}_{\ET}$.
\end{corollary}
\begin{proof}
\[
\begin{rclarray}
    \Setcon{ \hh_{\aexec} }{ \aexec \in \interpr{\prog}_{(\RP_{\LWW}, \Ax)} }
& \stackrel{\cref{thm:consistency-intersect-anarchic}}{=} &
\Setcon{\hh_{\aexec} }{ \aexec \text{ satisfies } \RP_{\LWW} \land \aexec \in \CMa(\RP_{\LWW}, \Ax) } \\
& \stackrel{\cref{thm:et_complete}}{\subseteq} & 
\Set{\hh_{\aexec} }[ \aexec \text{ satisfies } \RP_{\LWW}] \cap \CMs(\ET) \\
& \stackrel{\cref{cor:kvtrace2aexec}}{=} & 
\interpr{\prog}_{\ET_\top} \cap \CMs(\ET) \\
& \stackrel{\cref{thm:consistency-intersect-permissive}}{=} & 
\interpr{\prog}_{\ET} 
\end{rclarray}
\]
\end{proof}


\section{The Soundness and Completeness of Execution Tests}

We now show using \cref{def:et_sound,def:et_complete} to prove the soundness and completeness of execution tests with respect to axiomatic specification.
It is sufficient to match these two definition, 
then by \cref{cor:et-soundness,cor:et-completeness} we have \( \CMs(\ET) = \Setcon{\mkvs_\aexec}{\aexec \in \CMa(\RP_\LWW,\Ax)} \).

\label{sec:kv-sound-complete-proof}
\label{sec:spec-proof}

\emph{Monotonic read} (\( \MR \)) \citep{session-guarantee,repldatatypes} states that after committing a transaction, 
a client cannot lose information in that 
it can only see increasingly more versions from a kv-store.
This prevents, for example, the kv-store in \cref{fig:mr-disallowed},
since client \(\cl\) first reads the latest version of \(\key\) in \(\txid_{\cl}^{1}\), 
and then reads the older, initial version of \(\key\) in \(\txid_{\cl}^{2}\).  
As such, the \(\ViewShift[\MR]\) predicate in \cref{fig:execution-tests} 
ensures that clients can only extend their views,
that is, \( \vi \vileq \vi' \) for views \( \vi, \vi'\) before and after committing.
When this is the case, clients can then \emph{always} commit their transactions,
and thus \(\CanCommit[\MR]\) is simply defined as \(\true\). 

\subsection{Monotonic Write \( \MW \)}
\label{sec:sound-complete-mw}

The execution test $\ET_\MW$ is sound with respect to the axiomatic specification 
$(\RP_{\LWW}, \Set{\lambda \aexec. \PO_{\aexec} ; \VIS_{\aexec} })$.
We pick the invariant as empty set given the fact of no constraint on the view after update:
\[ 
    I( \aexec, \cl ) = \emptyset 
\]
Assume a key-value store $\hh$, an initial and a final view $\vi, \vi'$  a fingerprint $\opset$ 
such that $\ET_{\MW} \vdash (\hh, \vi) \csat \opset: (\hh',\vi')$. 
Also choose an arbitrary $\cl$, a transaction identifier $\txid \in \nextTxId(\hh, \cl)$, 
and an abstract execution $\aexec$ such that $\hh_{\aexec} = \hh$ and 
\( I(\aexec, \cl) =  \emptyset \subseteq \Tx(\hh, \vi) \).
Let \( \aexec' = \extend(\aexec, \txid, \Tx(\mkvs, \vi) \cup \T_\rd, \f ) \).
Note that since the invariant  is empty set, it remains to prove that there exists a set of read-only transactions \( \T_\rd \) such that:
\[
    \begin{array}{@{}l@{}}
        \fora{ \txid' }  (\txid' ,\txid)  \in \PO_{\aexec'} ; \VIS_{\aexec'}
        \implies \txid' \in \Tx(\mkvs, \vi) \cup \T_\rd
    \end{array}
\]
Initially we take \( \T_\rd = \emptyset \), 
and by closing the \( \Tx(\mkvs, \vi) \) with respect to the relation \( \PO_{\aexec'} ; \VIS_{\aexec'} \),
we will add more read-only transactions into the set \( \T_\rd\).
Suppose \( (\txid' ,\txid)  \in \PO_{\aexec'} ; \VIS_{\aexec'} \), 
that is, \( \txid' \toEdge{\SO_{\aexec'}} \txid'' \toEdge{\VIS_{\aexec'}} \txid \).
We perform a case analysis on if \( \txid'' \) has write:
\begin{itemize}
\item If the transaction \( \txid'' \) writes to a key.
For the new abstract execution \( \aexec' \), the visible transactions for \( \txid \) must come from \( \Tx(\mkvs, \vi) \cup \T_\rd \).
It means \( \txid'' \in \Tx(\mkvs, \vi) \cup \T_\rd  \).
Then given that \( \txid'' \) is not a read-only transaction, we have \( \txid'' \in \Tx(\mkvs, \vi) \).
Now there are two cases:
\begin{itemize}
    \item if \( \txid' \) is a read-only transaction, we include \( \txid' \in \T_{\rd} \).
    \item if \( \txid' \) has at least one write, it is easy to see \( \txid' \in \Tx(\mkvs, \vi) \) since \( j \in \vi(\ke) \wedge \WTx(\hh(\ke', i)) \xrightarrow{\PO?} \WTx(\hh(\ke, j)) \implies i \in \vi(\ke') \).
\end{itemize}
\item If the transaction \( \txid'' \in \T_\rd \) is a read-only transaction, 
since \( \T_\rd \) is initial empty, there must exist a later transaction \( \txid''' \) from the same client that writes to a key,
and such transaction \( \txid''' \) is included in \( \Tx(\mkvs, \vi) \):
\[
    \txid' \toEdge{\SO_{\aexec'}} \txid'' 
    \toEdge{\SO_{\aexec'}} \txid''' \toEdge{\VIS_{\aexec'}} \txid 
    \land \txid''' \in \Tx(\mkvs,\vi)
\]
Since \( \SO \) is transitive, 
therefore \( \txid' \toEdge{\SO_{\aexec'}} \txid''' \toEdge{\VIS_{\aexec'}} \txid \),
which we have already proven \( \txid' \in \Tx(\mkvs, \vi) \) or we will include \( \txid' \) in \( \T_\rd \).
Since there are finite transactions from a client in a trace, there must exist a \( \T_\rd \) in the end.
\end{itemize}


The execution test $\ET_{\MW}$ is complete with respect to 
the axiomatic specification $(\RP_{\LWW}, \Set{\lambda \aexec.(\PO_{\aexec} ; \VIS_{\aexec})})$. 
Let $\aexec$ be an abstract execution that satisfies the specification
$\CMa(\RP_{\LWW}, \Set{\lambda \aexec.(\PO_{\aexec} ; \VIS_{\aexec})})$, 
and consider a transaction $\txid \in \T_{\aexec}$. 
Assume i-\emph{th} transaction \( \txid_i \) in the arbitrary order,
and let view \( \vi_{i} = \getView(\aexec, \VIS^{-1}_{\aexec}(\txid_{i}) ) \).
We also pick any final view such that \( \vi'_{i} \subseteq \getView(\aexec, (\AR^{-1}_{\aexec})?(\txid_{i}) ) \).
It suffices to prove \( \ET_\MW \vdash (\hh_{\cut(\aexec, i-1)}, \vi_i ) \csat  \TtoOp{T}_{\aexec}(\txid_{i}) : (\hh_{\cut(\aexec, i-1)}, \vi'_{i}) \).
It means to prove the follows:
\begin{equation}
\label{equ:mw-complete}
\begin{array}{@{}l@{}}
    \fora{j,m,\ke, \ke' } j \in \vi(\ke)  
    \wedge \WTx(\hh_{\cut(\aexec, i-1)}(\ke', m)) \xrightarrow{\PO?} \WTx(\hh_{\cut(\aexec, i-1)}(\ke, j))  
    \implies m \in \vi(\ke')
\end{array}
\end{equation}
Assume \( j \) and \( \ke' \) such that \( j \in \vi(\ke')\), which means \( \WTx(\hh_{\cut(\aexec, i-1)}(\ke', j)) \in \VIS^{-1}_{\aexec}(\txid_{i}) \).
Now let consider transaction \( \txid \) that commits before \( \txid \) from the same session, \ie \( \txid \toEdge{\SO} \WTx(\hh_{\cut(\aexec, i-1)}(\ke, j)) \).
By the constraint \( \lambda \aexec.(\PO_{\aexec} ; \VIS_{\aexec}) \), the transaction \( \txid \in \VIS^{-1}_{\aexec}(\txid_{i}) \).
It means that in the kv-store \(  \hh_{\cut(\aexec, i-1)} \) every version written by \( \txid =  \WTx(\hh_{\cut(\aexec, i-1)}(\ke', m)) \) should be included in the view \( m \in \vi_i(\ke') \).
Thus we have the proof of \cref{equ:mw-complete}.

The execution test \(\et[\RYW]\) is sound with respect to the axiomatic definition
\(\visaxioms[\RYW] = \Set{\lambda \aexec \ldotp \SO_{\aexec} }\) \cite{repldatatypes}.
We pick the following invariant:
\[
    \aexecinv[\RYW](\aexec, \cl) \FuncDef
    \begin{multlined}[t]
    \left( \bigcup_{\Set{\txid[\cl](n) \in \aexec }} \Refl((\Inv(\SO)))(\txid[\cl](n)) \right) 
    \setminus \Set{\txidrd | \txidrd in \aexec \land \Forall{l | \key | \val} (l,\key,\val) \in \aexec(\txid) \implies l = \opR } .
    \end{multlined}
\]

\SOUNDLET{\RYW}{
    \txidsetrd = 
    \left( \bigcup_{\Set{\txid[\cl](n) \in \aexec }} \Refl((\Inv(\SO)))(\txid[\cl](n)) \right) 
    \cap \Set{\txidrd | \txidrd in \aexec \land \Forall{l | \key | \val} (l,\key,\val) \in \aexec(\txid) \implies l = \opR } .
}
\begin{enumerate}
\Case{\(\Forall{\visaxiom \in \visaxioms }
            \Inv(\visaxiom(\aexec'))(\txid) \subseteq \txidset \cup \txidsetrd \)}
    Suppose transactions \( \txid, \txid' \) such that \( \txid,\txid' \in \aexec \) and \( (\txid',\txid) \in \SO \).
    If \( \txid' \) is a read-only transaction, \( \txid' \in \txidsetrd \).
    Otherwise, \( \txid' \) has write, by the definition of \( \aexecinv[\RYW] \), 
    it follows that \( \txid' \in \aexecinv[\RYW](\aexec,\cl) \)
    and therefore \( \txid' \in \txidset \).
\Case{\(\aexecinv(\aexec',\cl) \subseteq \VisTrans(\XToK(\aexec'),\vi') \)}
    Because \( \ToET[\RYW]{\kvs | \vi | \fp | \kvs' | \vi' }\),
    it must be the case that
    \[
        \Forall{\key \in \Keys | \idx \in \Indexs } (\WtOf(\kvs'(\key,\idx)),\txid) \in \Refl(\SO) \implies \idx \in \vi'(\key) 
    \]
    and therefore
    \[
        \Forall{\txid'} \Exists{\key \in \Keys | \val \in \Values} \opW(\key,\val) \in \aexec'(\txid')
        \land (\txid',\txid) \in \SO \implies \txid' \in \VisTrans(\XToK(\aexec'),\vi') .
    \]
    Note that \( \bigcup_{\Set{\txid[\cl](n) \in \aexec }} \Refl((\Inv(\SO)))(\txid[\cl](n)) = \Refl((\Inv(\SO)))(\txid) \).
    Last, we have
    \begin{align*}
        \aexecinv(\aexec',\cl) & = 
            \begin{multlined}[t]
            \left( \bigcup_{\Set{\txid[\cl](n) \in \aexec }} \Inv(\SO)(\txid[\cl](n)) \right) 
            \setminus \Set{\txidrd | \txidrd \in \aexec 
                    \land \Forall{l | \key | \val} (l,\key,\val) \in \aexec(\txid) \implies l = \opR } 
            \end{multlined}
            \\ & = \begin{multlined}[t]
            \left( \Refl((\Inv(\SO)))(\txid) \right) 
            \setminus \Set{\txidrd | \txidrd \in \aexec 
                    \land \Forall{l | \key | \val} (l,\key,\val) \in \aexec(\txid) \implies l = \opR } 
            \end{multlined}
            \\ & \subseteq \VisTrans(\XToK(\aexec'),\vi') 
    \end{align*}
\end{enumerate}

\COMPLETELET{\RYW}
We construct the final view \( \vi'\) depending on whether \( \txid[\cl](n) \) is the last transaction for the client \( \cl \).
\begin{enumerate}
\Case{\( (\txid[\cl](n), \txid') \in \SO \) for \( \txid' \in \aexec \)}
    Let the transaction 
    \( \txid = \Min[\SO](\Set{ \txid' | (\txid[\cl](n), \txid') \in \SO \land \txid' \in \aexec' }) \).
    For this case, let view 
    \( \vi' = \GetView(\aexec, \Refl((\ARInv[\aexec]))(\txid[\cl](\idx)) \cap \VISInv[\aexec](\txid)) \).
    By \( \visaxioms[\RYW] \), it follows that, for any transaction \( \txid' \),
    if \( ( \txid',\txid[\cl](idx) ) \in \Refl(\SO) \), then
    \( \txid' \in \VISInv[\aexec](\txid)) \).
    Since \( \SO \in \AR \), we know that 
    \( \txid' \in \Refl((\ARInv[\aexec]))(\txid[\cl](\idx)) \cap \VISInv[\aexec](\txid)) \).
    Therefore, for any version \( \kvs'(\key,j)\) such that 
    \( ( \WtOf(\kvs'(\key,j)), \txid) \in \Refl(\SO) \),
    then \( j \in \vi'(\key)\).
\Case{\( \neg \left((\txid[\cl](n), \txid') \in \SO \right) \)}
    For this case, let 
    \( \vi' = \GetView(\aexec, \Refl((\ARInv[\aexec]))(\txid[\cl](\idx))) \) be the final view.
    It is easy to see that \( \vi' \) satisfies \( \RYW \). 
\end{enumerate}

The execution test \(\et[\WFR]\) is sound with respect to the axiomatic definition 
\(\visaxioms[\WFR] \Set{\lambda \aexec \ldotp \WR[\aexec] ; \Refl((\SO \cap \RW[\aexec] )) ; \VIS[\aexec] })\) 
\citep{surech-session-guarantee}.
By picking the invariant as \( I( \aexec, \cl ) = \emptyset \), the soundness and completeness
can be derived from \cref{thm:view-vis-relation} in a similar way as the proofs for \( \MW \).

The wildly used definition on abstract executions for causal consistency is that 
\( \VIS \) is transitive and \( \SO \in \VIS \).
Yet it is for the sack of elegant definition,
while there is a equivalent minimum visibility relation (\cref{thm:cc-visaxioms}) defined by 
\( \visaxioms[\CC] \FuncDef \Set{ \lambda \aexec \ldotp (\WR[\aexec] \cup \SO) ; \VIS[\aexec] \subseteq \VIS[\aexec] , 
                                    \lambda \aexec \ldotp \SO \subseteq \VIS[\aexec]} \),
where \( \WR[\aexec] \) is defined in \cref{def:aexec-dgraph}.

\begin{theorem}[Minimum visibility relation for (\texorpdfstring{\CC}{\texttt{CC}})]
\label{thm:cc-visaxioms}
For two abstract executions \( \aexec,\aexec' \),
the following constrain on visibility,
\begin{Formulae}
\begin{Formula}
    (\WR[\aexec] \cup \SO) ; \VIS[\aexec] \subseteq \VIS[\aexec] \land \SO \subseteq \VIS[\aexec]
    \label{equ:kvstore-cc-spec}
\end{Formula}
\end{Formulae}
is equivalent to
\begin{Formulae}
\begin{Formula}
    \VIS[\aexec'] ; \VIS[\aexec'] \subseteq \VIS[\aexec'] \land \SO \subseteq \VIS[\aexec']
    \label{equ:aexec-cc-spec}
\end{Formula}
\end{Formulae}
in that 
\(
    \Forall{\txid \in \TxIDs | \fp } \left( \fp = \aexec(\txid) \iff \fp = \aexec'(\txid) \right)
    \land \AR[\aexec] = \AR[\aexec'] .
\)
\end{theorem}
\begin{proof}
For an abstract execution \( \aexec \) that satisfies \cref{equ:kvstore-cc-spec},
by \cref{lem:aexec-spec-cc}, there exists \( \aexec' \) that satisfies \cref{equ:aexec-cc-spec}.
Assume an abstract execution \( \aexec' \) that satisfies \cref{equ:aexec-cc-spec}.
Since \( \WR[\aexec'] \subseteq \VIS[\aexec']\) by the definition of \( \WR[\aexec']\),
thus \( \aexec' \) satisfies \cref{equ:kvstore-cc-spec}.
\end{proof}

\begin{toappendix}
\begin{lemma}[Minimum visibility relation for (\texorpdfstring{\CC}{\texttt{CC}})]
\label{lem:aexec-spec-cc}
For any abstract execution \( \aexec \), if it satisfies \( \visaxioms[\CC] \),
there exists a new abstract execution \( \aexec' \) such that \( \SO \in \VIS[\aexec]\) and
\begin{Formulae}
\begin{Formula}
    \Forall{\txid \in \TxIDs | \fp } \left( \fp = \aexec(\txid) \iff \fp = \aexec'(\txid) \right)
    \land \AR[\aexec] = \AR[\aexec'] \land \VIS[\aexec'] ; \VIS[\aexec'] \subseteq \VIS[\aexec'] .
    \label{equ:aexec-spec-cc}
\end{Formula}
\end{Formulae}
\end{lemma}
\begin{proof}
We erase some visibility relation for each transaction following 
the arbitration order \( \AR \) until the visibility is transitive.
Intuitively, the final visibility relation is exactly \( \Trasi((\WR[\aexec] \cup \SO)) \).
Assume the \Th{\idx} transaction \( \txid_\idx \)  with respect to the arbitration order.
Let \( \rel[\idx] \) be a new visibility for the transaction \( \txid_\idx \) such that
\( {\rel[\idx]}\Proj{2} = \Set{\txid_\idx}\) for all indexes \( \idx \)
and the union of visibility relations \( \bigcup_{0 \leq j \leq \idx } \rel[\idx] \) is transitive.
We preserve that, for each index \( \idx \), cut of abstract execution \( \aexec' =  \AexecCut(\aexec, \idx) \)
and visibility relation \( \VIS' = \bigcup_{0 \leq j \leq \idx } \rel[j] \),
the following invariant holds:
\begin{Formulae}
& \begin{Formula} 
    \VIS' ; \VIS' \subseteq \VIS'  ,
    \label{equ:cc-vis-idx-transitive} 
\end{Formula}
\\ & \begin{Formula}
    \Forall{ \txid \in \aexec } (\txid,\txid_i) \in \rel[\idx] \implies (\txid, \txid_i) \in (\WR[\aexec'] \cup \SO) .
    \label{equ:cc-vis-idx-minimum}
\end{Formula}
\end{Formulae}
We prove the above by induction on the number \( \idx \).
\begin{enumerate}
\CaseBase{\( \idx = 0 \)}
    By the definition of \( \AexecCut \), we know that \(\aexecinit = \AexecCut(\aexec,0) \)
    and \cref{equ:cc-vis-idx-transitive,equ:cc-vis-idx-minimum} trivially hold.
\CaseInd{\( \idx > 0 \)}
    Suppose that, for the \Th{(\idx-1)} step,
    the abstract execution \( \aexec'' =  \AexecCut(\aexec, \idx - 1) \)
    and the visibility relation \( \VIS'' = \bigcup_{0 \leq j \leq \idx-1 } \rel[j] \) 
    satisfy \cref{equ:cc-vis-idx-transitive,equ:cc-vis-idx-minimum}.
    Let consider \Th{\idx} step, the transaction \( \txid_i \),
    the cut \( \aexec' =  \AexecCut(\aexec, \idx) \)
    and the visibility relation \( \VIS' = \bigcup_{0 \leq j \leq \idx } \rel[j] \).
    Initially we take \( \rel \) as an empty set.
    First, we include \( \Set{(\txid,\txid_i) | (\txid,\txid_i) \in \WR[\aexec]} \) to \( \rel \)
    and, by the definition of \( \WR[\aexec]\), 
    it trivially does not affect any read operation for the transaction \( \txid_i \).
    Then we do the same for \( \SO \) as that 
    we include \( \Set{(\txid,\txid_i) | (\txid,\txid_i) \in \SO} \) to \( \rel \).
    Note that \( \SO \) cannot affect any read operation for the transaction \( \txid_i \) neither,
    otherwise it contradicts to that \( \SO \subseteq \VIS[\aexec] \) and the definition of \( \WR[\aexec] \).

    For relations \( \rel' = \rel ; \bigcup_{0 \leq j \leq \idx-1 } \rel[j] \) and then \( \rel[\idx] = \rel \cup \rel' \),
    it easy to see that \( \rel \in \VIS[\aexec]\) and, then by \ih, \( \rel' \in \VIS[\aexec] \).
    We prove that the \( \rel[\idx] \) does not affect any read operation for the transaction \( \txid_i \)
    by contradiction.
    Assume distinct transactions \( \txid,\txid' \) such that
    \( \ToEdge{\txid'' | \rel \cup \rel' -> \txid' | \rel \cup \rel' -> \txid_i } \),
    and immediately  by the definition of \( \rel \) and \( \rel' \),
    then \( \ToEdge{\txid'' | \rel' -> \txid' | \rel -> \txid_i } \).
    Assume that \( \txid'' \) change the read operation for a key \( \key \) in \( \txid_i \).
    This means that there exists a transaction \( \txid^* \) such that
    \( (\txid^*,\txid_i) \in \WR[\aexec](\key)\) and \( (\txid^*,\txid'') \in \AR[\aexec] \),
    where the latter implies that \( (\txid'',\txid_i) \in \WR[\aexec](\key) \);
    there is a contradiction and thus 
    \( \rel[\idx] \) does not affect any read operation for the transaction \( \txid_i \).

    We now prove that \cref{equ:cc-vis-idx-transitive,equ:cc-vis-idx-minimum} still hold.
    \begin{enumerate}
    \Case{\cref{equ:cc-vis-idx-transitive}}
        Assume a relation \( \rel^* = \bigcup_{0 \leq j \leq \idx-1 } \rel[j] \) 
        and transactions \( \txid, \txid',\txid'' \) such that 
        \[
            \ToEdge{\txid | \rel^* \cup \rel[\idx] -> \txid' | \rel^* \cup \rel[\idx] -> \txid'' } .
        \]
        If \( \ToEdge{\txid | \rel^*  -> \txid' | \rel^*  -> \txid'' } \), 
        then by \ih, \( \ToEdge{\txid | \rel^*  -> \txid'' } \).
        Note that \( \ToEdge{\txid | \rel[\idx]  -> \txid' | \rel^*  -> \txid'' } \) cannot happen,
        because it contradict to that \( \txid' = \txid_i\) and \( (\txid'',\txid_i) \in \AR[\aexec] \).
        Thus consider \( \ToEdge{\txid | \rel^*  -> \txid' | \rel[\idx]  -> \txid'' } \).
        It must be the case that \( \txid'' = \txid_i \) and by the definition of \( \rel[\idx] \),
        we know that \( \ToEdge{\txid | \rel[\idx]  -> \txid'' } \).
    \Case{\cref{equ:cc-vis-idx-minimum}}
        By the construction, \cref{equ:cc-vis-idx-minimum} hold. \qedhere
    \end{enumerate}
\end{enumerate}
\end{proof}
\end{toappendix}

We pick the invariant as \( \aexecinv[\CC] = \aexecinv[\MR] \cup \aexecinv[\RYW]  \).
\SOUNDLET{\CC}{ \txidsetrd \supseteq
\begin{multlined}[t]
\left( \bigcup_{\Set{\txid[\cl](\idx) | \txid[\cl](\idx) \in \aexec}} 
\VISInv[\aexec](\txid[\cl](\idx)) \cup \Refl((\Inv(\SO)))(\txid[\cl](\idx)) \right) 
\setminus \Set{\txid' | \Forall{l | \key | \val } (l,\key,\val) \in \aexec(\txid') \implies l = \opR } .
\end{multlined} }
Assume 
\[ 
\txidsetrd' = 
\begin{multlined}[t]
\left( \bigcup_{\Set{\txid[\cl](\idx) | \txid[\cl](\idx) \in \aexec}} 
\VISInv[\aexec](\txid[\cl](\idx)) \cup \Refl((\Inv(\SO)))(\txid[\cl](\idx)) \right) 
\setminus \Set{\txid' | \Forall{l | \key | \val } (l,\key,\val) \in \aexec(\txid') \implies l = \opR } .
\end{multlined} 
\]
and \( \txidsetrd'' = \txidsetrd \setminus \txidsetrd' \).
By the definition of soundness, we prove the following result
\begin{Formulae}
& \begin{Formula}
\Inv(\SO)(\txid) \subseteq \txidset \cup \txidsetrd'
\label{equ:cc-so-vis}
\end{Formula}
\\ & \begin{Formula}
\Inv((( \WR[\aexec'] \cup \SO ) ; \VIS[\aexec'] )) (\txid) \subseteq \txidset \cup \txidsetrd' \cup \txidsetrd''
\label{equ:cc-vis-transitive}
\end{Formula}
\\ & \begin{Formula}
\aexecinv[\CC](\aexec',\cl) \subseteq \VisTrans(\XToK(\aexec'),\vi')
\label{equ:cc-inv-preserve}
\end{Formula}
\end{Formulae}
\Cref{equ:cc-so-vis} can be proven in the same way as in \cref{sec:sound-complete-mr}
We now prove \cref{equ:cc-vis-transitive}.
Initially we take \( \txidsetrd'' \) to be an empty set.
Note that \(\VISInv[\aexec'](\txid) = \txidset \cup \txidsetrd' \cup \txidsetrd'' \).
By \cref{thm:view-vis-relation,equ:view-close-to-aexec}, there exists \( \txidsetrd'' \) such that
\( \txidset \cup \txidsetrd'' \) is closed under \( \WR[\aexec'] \cup \SO \).
Now consider a transaction \( \txidrd \in \txidsetrd' \) and
assume a transaction \( \txid' \) such that \( \ToEdge{ \txid' | \WR[\aexec'] \cup \SO -> \txidrd } \).
There are two cases depending on \( \txidrd \).
\begin{enumerate}
\Case{\( \ToEdge{\txidrd | \VIS[\aexec'] -> \txid'' | \SO -> \txid} \) for some \( \txid'' \)}
    For this case, we have
    \begin{align*}
    \ToEdge{\txid' | \WR[\aexec'] \cup \SO -> \txidrd | \VIS[\aexec'] -> \txid'' | \SO -> \txid }
    & 
    \implies \ToEdge{\txid' | \WR[\aexec] \cup \SO -> \txidrd | \VIS[\aexec] -> \txid'' | \SO -> \txid }
    \\ & \implies \ToEdge{\txid' | \VIS[\aexec] -> \txid'' | \SO -> \txid } .
    \end{align*}
    By \( \aexecinv[\MR]\), we know that \(\txid' \in \aexecinv[\MR] \cup \txidsetrd' \).
\Case{\( \ToEdge{\txidrd | \SO -> \txid} \)}
    For this case, we have
    \begin{align*}
    \ToEdge{\txid' | \WR[\aexec'] \cup \SO -> \txidrd | \SO -> \txid }
    & 
    \implies \ToEdge{\txid' | \WR[\aexec] \cup \SO -> \txidrd | \SO -> \txid }
    \\ & \implies \ToEdge{\txid' | \VIS[\aexec] -> \txid'' | \SO -> \txid } .
    \end{align*}
    By \( \aexecinv[\MR]\), we know that \(\txid' \in \aexecinv[\MR] \cup \txidsetrd' \).
\end{enumerate}
Last, \cref{equ:cc-inv-preserve}can be proven in the same way as in \cref{sec:sound-complete-mr,sec:sound-complete-ryw}.

\COMPLETELET{\CC}
By \cref{thm:cc-visaxioms},
it is sufficient to prove with respect to the following visibility axioms,
\( \visaxioms[\CC]' \FuncDef \Set{ \lambda \aexec \ldotp  \VIS[\aexec] ; \VIS[\aexec] \subseteq \VIS[\aexec] , 
                                    \lambda \aexec \ldotp \SO \subseteq \VIS[\aexec]} \).
By the definition of \( \et[\CC] \), we prove \( \CanCommit[\CC]\) and \( \ViewShift[\MR \cup \RYW]\) respectively.
Since \( (\WR[\aexec] \cup \SO) ; \VIS[\aexec]  \subseteq \VIS[\aexec] ; \VIS[\aexec] \subseteq \VIS[\aexec] \),
then \( \CanCommit[\CC]\) can be derived from \cref{thm:view-vis-relation,equ:aexec-close-to-view}
and \( \ViewShift[\RYW] \) can be proven in the same way as in \cref{sec:sound-complete-ryw}.
By \( \VIS[\aexec] ; \SO \subseteq \VIS[\aexec] ; \VIS[\aexec] \subseteq \VIS[\aexec]  \),
\( \ViewShift[\MR] \) can be proven in the same way as in \cref{sec:sound-complete-mr}.



\subsection{Update Atomic}
\begin{figure}
\hrule
\begin{tabular}{@{} c c@{}}

\begin{halfsubfig}
\begin{centertikz}

\begin{pgfonlayer}{foreground}
%Uncomment line below for help lines
%\draw[help lines] grid(5,4);

%Location x
\node(locx)  {$\ke_\vx \mapsto$};

\matrix(versionx) [version list]
    at ([xshift=\tikzkvspace]locx.east) {
    {a} & $\txid_0$ \\
    {a} & $\emptyset$ \\
};

\tikzvalue{versionx-1-1}{versionx-2-1}{locx-v0}{0};

%Location y
\path (locx.south) + (0,\tikzkeyspace) node (locf1) {$\ke_{\pv{f1}} \mapsto$};
\matrix(versionf1) [version list]
    at ([xshift=\tikzkvspace]locf1.east) {
    {a} & $\txid_0$ \\
    {a} & $\emptyset$ \\
};
\tikzvalue{versionf1-1-1}{versionf1-2-1}{locf1-v0}{0};

%Location y
\path (locf1.south) + (0,\tikzkeyspace) node (locf2) {$\ke_{\pv{f2}} \mapsto$};
\matrix(versionf2) [version list]
    at ([xshift=\tikzkvspace]locf2.east) {
    {a} & $\txid_0$ \\
    {a} & $\emptyset$ \\
};
\tikzvalue{versionf2-1-1}{versionf2-2-1}{locf2-v0}{0};

% \draw[-, red, very thick, rounded corners] ([xshift=-5pt, yshift=5pt]locx-v1.north east) |- 
%  ($([xshift=-5pt,yshift=-5pt]locx-v1.south east)!.5!([xshift=-5pt, yshift=5pt]locy-v0.north east)$) -| ([xshift=-5pt, yshift=5pt]locy-v0.south east);

%blue view - I should  check whether I can use pgfkeys to just declare the list of locations, and then add the view automatically.
\draw[-, blue, very thick, rounded corners=10pt]
 ([xshift=-2pt, yshift=20pt]locx-v0.north east) node (tid1start) {} -- 
 ([xshift=-2pt, yshift=-5pt]locf2-v0.south east);
 
 \path (tid1start) node[anchor=south, rectangle, fill=blue!20, draw=blue, font=\small, inner sep=1pt] {$\thid_3$};

%red view
\draw[-, red, very thick, rounded corners = 10pt]
 ([xshift=-5pt, yshift=5pt]locx-v0.north east) -- 
 ([xshift=-5pt, yshift=-10pt]locf2-v0.south east) node (tid2start) {};
 
\path (tid2start) node[anchor=north, rectangle, fill=red!20, draw=red, font=\small, inner sep=1pt] {$\thid_2$};
 
 %green view
\draw[-, DarkGreen, very thick, rounded corners = 10pt]
 ([xshift=-16pt, yshift=8pt]locx-v0.north east) node (tid3start) {}-- 
 ([xshift=-16pt, yshift=-5pt]locf2-v0.south east);
 
 \path (tid3start) node[anchor=south, rectangle, fill=DarkGreen!20, draw=DarkGreen, font=\small, inner sep=1pt] {$\thid_1$};

\end{pgfonlayer}
\end{centertikz}%
\caption{Initial configuration}
\label{fig:ua-init}
\end{halfsubfig}
%
&
%
\begin{halfsubfig}
\begin{centertikz}
\begin{pgfonlayer}{foreground}
%Uncomment line below for help lines
%\draw[help lines] grid(5,4);

\node(locx)  {$\ke_\vx \mapsto$};

\matrix(versionx) [version list]
    at ([xshift=\tikzkvspace]locx.east) {
    {a} & $\txid_0$ & {a} & \(\txid_1\)\\
    {a} & $\Set{\txid_1}$ & {a} & \(\emptyset\)\\
};

\tikzvalue{versionx-1-1}{versionx-2-1}{locx-v0}{0};
\tikzvalue{versionx-1-3}{versionx-2-3}{locx-v1}{1};


\path (locx.south) + (0,\tikzkeyspace) node (locf1) {$\ke_{\pv{f1}} \mapsto$};
\matrix(versionf1) [version list]
    at ([xshift=\tikzkvspace]locf1.east) {
    {a} & $\txid_0$ & {a} & $\txid_1$\\
    {a} & $\Set{\txid_1}$ & {a} & $\emptyset$\\
};
\tikzvalue{versionf1-1-1}{versionf1-2-1}{locf1-v0}{0};
\tikzvalue{versionf1-1-3}{versionf1-2-3}{locf1-v1}{1};

%Location y
\path (locf1.south) + (0,\tikzkeyspace) node (locf2) {$\ke_{\pv{f2}} \mapsto$};
\matrix(versionf2) [version list]
    at ([xshift=\tikzkvspace]locf2.east) {
    {a} & $\txid_0$ \\
    {a} & $\emptyset$ \\
};
\tikzvalue{versionf2-1-1}{versionf2-2-1}{locf2-v0}{0};


% \draw[-, red, very thick, rounded corners] ([xshift=-5pt, yshift=5pt]locx-v1.north east) |- 
%  ($([xshift=-5pt,yshift=-5pt]locx-v1.south east)!.5!([xshift=-5pt, yshift=5pt]locy-v0.north east)$) -| ([xshift=-5pt, yshift=5pt]locy-v0.south east);

%blue view - I should  check whether I can use pgfkeys to just declare the list of locations, and then add the view automatically.
\draw[-, blue, very thick, rounded corners=10pt]
 ([xshift=-2pt, yshift=20pt]locx-v0.north east) node (tid1start) {} -- 
 ([xshift=-2pt, yshift=-5pt]locf2-v0.south east);
 
 \path (tid1start) node[anchor=south, rectangle, fill=blue!20, draw=blue, font=\small, inner sep=1pt] {$\thid_3$};

%red view
\draw[-, red, very thick, rounded corners = 10pt]
 ([xshift=-5pt, yshift=5pt]locx-v0.north east) -- 
 ([xshift=-5pt, yshift=-10pt]locf2-v0.south east) node (tid2start) {};
 
\path (tid2start) node[anchor=north, rectangle, fill=red!20, draw=red, font=\small, inner sep=1pt] {$\thid_2$};
 
 %green view
\draw[-, DarkGreen, very thick, rounded corners = 10pt]
 ([xshift=-16pt, yshift=8pt]locx-v1.north east) node (tid3start) {}-- 
 ([xshift=-16pt, yshift=-5pt]locf1-v1.south east) --
 ([xshift=-16pt, yshift=5pt]locf2-v0.north east) -- 
 ([xshift=-16pt, yshift=-5pt]locf2-v0.south east);
 
 \path (tid3start) node[anchor=south, rectangle, fill=DarkGreen!20, draw=DarkGreen, font=\small, inner sep=1pt] {$\thid_1$};

\end{pgfonlayer}
\end{centertikz}
\caption{After \(\txid_1\)}
\label{fig:ua-after-tx1}
\end{halfsubfig}

\\
\begin{subfigure}{0.45\textwidth}
\begin{centertikz}%
\begin{pgfonlayer}{foreground}
%Uncomment line below for help lines
%\draw[help lines] grid(5,4);


\node(locx)  {$\ke_\vx \mapsto$};

\matrix(versionx) [version list, column 2/.style = {text width=14mm}]
    at ([xshift=\tikzkvspace]locx.east) {
    {a} & $\txid_0$ & {a} & $\txid_1$ & {a} & $\txid_2$\\
    {a} & $\Set{\txid_1, \txid_2}$ & {a} & $\emptyset$ & {a} & $\emptyset$\\
};

\tikzvalue{versionx-1-1}{versionx-2-1}{locx-v0}{0};
\tikzvalue{versionx-1-3}{versionx-2-3}{locx-v1}{1};
\tikzvalue{versionx-1-5}{versionx-2-5}{locx-v2}{1};


\path (locx.south) + (0,\tikzkeyspace) node (locf1) {$\ke_{\pv{f1}} \mapsto$};
\matrix(versionf1) [version list]
    at ([xshift=\tikzkvspace]locf1.east) {
    {a} & $\txid_0$ & {a} & $\txid_1$\\
    {a} & $\Set{\txid_1}$ & {a} & $\emptyset$\\
};
\tikzvalue{versionf1-1-1}{versionf1-2-1}{locf1-v0}{0};
\tikzvalue{versionf1-1-3}{versionf1-2-3}{locf1-v1}{1};

%Location y
\path (locf1.south) + (0,\tikzkeyspace) node (locf2) {$\ke_{\pv{f2}} \mapsto$};
\matrix(versionf2) [version list]
    at ([xshift=\tikzkvspace]locf2.east) {
    {a} & $\txid_0$ & {a} & \(\txid_2\) \\
    {a} & $\emptyset$ & {a} & \(\emptyset\) \\
};
\tikzvalue{versionf2-1-1}{versionf2-2-1}{locf2-v0}{0};
\tikzvalue{versionf2-1-3}{versionf2-2-3}{locf2-v1}{1};


% \draw[-, red, very thick, rounded corners] ([xshift=-5pt, yshift=5pt]locx-v1.north east) |- 
%  ($([xshift=-5pt,yshift=-5pt]locx-v1.south east)!.5!([xshift=-5pt, yshift=5pt]locy-v0.north east)$) -| ([xshift=-5pt, yshift=5pt]locy-v0.south east);

%blue view - I should  check whether I can use pgfkeys to just declare the list of locations, and then add the view automatically.
\draw[-, blue, very thick, rounded corners=10pt]
([xshift=-2pt, yshift=20pt]locx-v0.north east) node (tid1start) {} -- 
([xshift=-2pt, yshift=-5pt]locf2-v0.south east);
 
\path (tid1start) node[anchor=south, rectangle, fill=blue!20, draw=blue, font=\small, inner sep=1pt] {$\thid_3$};

%red view
\draw[-, red, very thick, rounded corners = 10pt]
([xshift=-5pt, yshift=5pt]locx-v2.north east) -- 
([xshift=-5pt, yshift=-5pt]locx-v2.south east) --
([xshift=-5pt, yshift=5pt]locf1-v0.north east) -- 
([xshift=-5pt, yshift=-5pt]locf1-v0.south east) --
([xshift=-5pt, yshift=5pt]locf2-v1.north east) -- 
([xshift=-5pt, yshift=-10pt]locf2-v1.south east) node (tid2start) {};

\path (tid2start) node[anchor=north, rectangle, fill=red!20, draw=red, font=\small, inner sep=1pt] {$\thid_2$};
 
 %green view
\draw[-, DarkGreen, very thick, rounded corners = 10pt]
([xshift=-16pt, yshift=8pt]locx-v1.north east) node (tid3start) {}-- 
([xshift=-16pt, yshift=-5pt]locx-v1.south east) --
([xshift=-16pt, yshift=5pt]locf1-v1.north east) -- 
([xshift=-16pt, yshift=-5pt]locf1-v1.south east) --
([xshift=-16pt, yshift=5pt]locf2-v0.north east) -- 
([xshift=-16pt, yshift=-5pt]locf2-v0.south east);

\path (tid3start) node[anchor=south, rectangle, fill=DarkGreen!20, draw=DarkGreen, font=\small, inner sep=1pt] {$\thid_1$};

\end{pgfonlayer}%
\end{centertikz}%
\caption{After \(\txid_2\)}
\label{fig:ua-after-tx2}
\end{subfigure}
%
&
%
\begin{subfigure}{0.45\textwidth}
\begin{centertikz}
\begin{pgfonlayer}{foreground}
%Uncomment line below for help lines
%\draw[help lines] grid(5,4);

\node(locx)  {$\ke_\vx \mapsto$};

\matrix(versionx) [version list, column 2/.style = {text width=14mm}]
    at ([xshift=\tikzkvspace]locx.east) {
    {a} & $\txid_0$ & {a} & $\txid_1$ & {a} & $\txid_2$\\
    {a} & $\Set{\txid_1, \txid_2}$ & {a} & $\emptyset$ & {a} & $\emptyset$\\
};

\tikzvalue{versionx-1-1}{versionx-2-1}{locx-v0}{0};
\tikzvalue{versionx-1-3}{versionx-2-3}{locx-v1}{1};
\tikzvalue{versionx-1-5}{versionx-2-5}{locx-v2}{1};


\path (locx.south) + (0,\tikzkeyspace) node (locf1) {$\ke_{\pv{f1}} \mapsto$};
\matrix(versionf1) [version list]
    at ([xshift=\tikzkvspace]locf1.east) {
    {a} & $\txid_0$ & {a} & $\txid_1$\\
    {a} & $\Set{\txid_1}$ & {a} & $\emptyset$\\
};
\tikzvalue{versionf1-1-1}{versionf1-2-1}{locf1-v0}{0};
\tikzvalue{versionf1-1-3}{versionf1-2-3}{locf1-v1}{1};

%Location y
\path (locf1.south) + (0,\tikzkeyspace) node (locf2) {$\ke_{\pv{f2}} \mapsto$};
\matrix(versionf2) [version list]
    at ([xshift=\tikzkvspace]locf2.east) {
    {a} & $\txid_0$ & {a} & \(\txid_2\) \\
    {a} & $\emptyset$ & {a} & \(\emptyset\) \\
};
\tikzvalue{versionf2-1-1}{versionf2-2-1}{locf2-v0}{0};
\tikzvalue{versionf2-1-3}{versionf2-2-3}{locf2-v1}{1};


% \draw[-, red, very thick, rounded corners] ([xshift=-5pt, yshift=5pt]locx-v1.north east) |- 
%  ($([xshift=-5pt,yshift=-5pt]locx-v1.south east)!.5!([xshift=-5pt, yshift=5pt]locy-v0.north east)$) -| ([xshift=-5pt, yshift=5pt]locy-v0.south east);

%blue view - I should  check whether I can use pgfkeys to just declare the list of locations, and then add the view automatically.
\draw[-, blue, very thick, rounded corners=10pt]
([xshift=-2pt, yshift=20pt]locx-v2.north east) node (tid1start) {} -- 
([xshift=-2pt, yshift=-7pt]locx-v2.south east) --
([xshift=-2pt, yshift=3pt]locf1-v1.north east) -- 
([xshift=-2pt, yshift=-5pt]locf1-v1.south east) --
([xshift=-2pt, yshift=5pt]locf2-v1.north east) -- 
([xshift=-2pt, yshift=-5pt]locf2-v1.south east);

\path (tid1start) node[anchor=south, rectangle, fill=blue!20, draw=blue, font=\small, inner sep=1pt] {$\thid_3$};

%red view
\draw[-, red, very thick, rounded corners = 10pt]
([xshift=-5pt, yshift=5pt]locx-v2.north east) -- 
([xshift=-5pt, yshift=-5pt]locx-v2.south east) --
([xshift=-5pt, yshift=5pt]locf1-v0.north east) -- 
([xshift=-5pt, yshift=-5pt]locf1-v0.south east) --
([xshift=-5pt, yshift=5pt]locf2-v1.north east) -- 
([xshift=-5pt, yshift=-10pt]locf2-v1.south east) node (tid2start) {};

\path (tid2start) node[anchor=north, rectangle, fill=red!20, draw=red, font=\small, inner sep=1pt] {$\thid_2$};
 
 %green view
\draw[-, DarkGreen, very thick, rounded corners = 10pt]
([xshift=-16pt, yshift=8pt]locx-v1.north east) node (tid3start) {}-- 
([xshift=-16pt, yshift=-5pt]locx-v1.south east) --
([xshift=-16pt, yshift=5pt]locf1-v1.north east) -- 
([xshift=-16pt, yshift=-5pt]locf1-v1.south east) --
([xshift=-16pt, yshift=5pt]locf2-v0.north east) -- 
([xshift=-16pt, yshift=-5pt]locf2-v0.south east);

\path (tid3start) node[anchor=south, rectangle, fill=DarkGreen!20, draw=DarkGreen, font=\small, inner sep=1pt] {$\thid_1$};

\end{pgfonlayer}
\end{centertikz}%
\caption{\(\txid_3\) updates the view}
\label{fig:ua-before-tx2}
\end{subfigure} \\
\end{tabular}
\hrule
\caption{An invalid executions under update atomic for $\prog_3$}
\label{fig:cu.exec}
\label{fig:cu-exec}
\end{figure}




\begin{figure}
\hrule
\begin{tabular}{@{} c c@{}}

\begin{subfigure}{0.45\textwidth}
\begin{centertikz}

\begin{pgfonlayer}{foreground}
%Uncomment line below for help lines
%\draw[help lines] grid(5,4);

\node(locx)  {$\ke_\vx \mapsto$};

\matrix(versionx) [version list]
    at ([xshift=\tikzkvspace]locx.east) {
    {a} & $\txid_0$ & {a} & \(\txid_1\)\\
    {a} & $\Set{\txid_1}$ & {a} & \(\emptyset\)\\
};

\tikzvalue{versionx-1-1}{versionx-2-1}{locx-v0}{0};
\tikzvalue{versionx-1-3}{versionx-2-3}{locx-v1}{1};


\path (locx.south) + (0,\tikzkeyspace) node (locf1) {$\ke_{\pv{f1}} \mapsto$};
\matrix(versionf1) [version list]
    at ([xshift=\tikzkvspace]locf1.east) {
    {a} & $\txid_0$ & {a} & $\txid_1$\\
    {a} & $\Set{\txid_1}$ & {a} & $\emptyset$\\
};
\tikzvalue{versionf1-1-1}{versionf1-2-1}{locf1-v0}{0};
\tikzvalue{versionf1-1-3}{versionf1-2-3}{locf1-v1}{1};

%Location y
\path (locf1.south) + (0,\tikzkeyspace) node (locf2) {$\ke_{\pv{f2}} \mapsto$};
\matrix(versionf2) [version list]
    at ([xshift=\tikzkvspace]locf2.east) {
    {a} & $\txid_0$ \\
    {a} & $\emptyset$ \\
};
\tikzvalue{versionf2-1-1}{versionf2-2-1}{locf2-v0}{0};


% \draw[-, red, very thick, rounded corners] ([xshift=-5pt, yshift=5pt]locx-v1.north east) |- 
%  ($([xshift=-5pt,yshift=-5pt]locx-v1.south east)!.5!([xshift=-5pt, yshift=5pt]locy-v0.north east)$) -| ([xshift=-5pt, yshift=5pt]locy-v0.south east);

%blue view - I should  check whether I can use pgfkeys to just declare the list of locations, and then add the view automatically.
\draw[-, blue, very thick, rounded corners=10pt]
([xshift=-2pt, yshift=20pt]locx-v0.north east) node (tid1start) {} -- 
([xshift=-2pt, yshift=-5pt]locf2-v0.south east);

\path (tid1start) node[anchor=south, rectangle, fill=blue!20, draw=blue, font=\small, inner sep=1pt] {$\thid_3$};

%red view
\draw[-, red, very thick, rounded corners = 10pt]
([xshift=-5pt, yshift=5pt]locx-v1.north east) -- 
([xshift=-5pt, yshift=-5pt]locf1-v1.south east) --
([xshift=-5pt, yshift=5pt]locf2-v0.north east) -- 
([xshift=-5pt, yshift=-10pt]locf2-v0.south east) node (tid2start) {};

\path (tid2start) node[anchor=north, rectangle, fill=red!20, draw=red, font=\small, inner sep=1pt] {$\thid_2$};
 
 %green view
\draw[-, DarkGreen, very thick, rounded corners = 10pt]
([xshift=-16pt, yshift=8pt]locx-v1.north east) node (tid3start) {}-- 
([xshift=-16pt, yshift=-5pt]locf1-v1.south east) --
([xshift=-16pt, yshift=5pt]locf2-v0.north east) -- 
([xshift=-16pt, yshift=-5pt]locf2-v0.south east);

\path (tid3start) node[anchor=south, rectangle, fill=DarkGreen!20, draw=DarkGreen, font=\small, inner sep=1pt] {$\thid_1$};

\end{pgfonlayer}
\end{centertikz}%
\caption{\(\thid_2\) updates the view}
\label{fig:ua-thid-2-update-view}
\end{subfigure} 
&
\begin{subfigure}{0.45\textwidth}
\begin{centertikz}

\begin{pgfonlayer}{foreground}
%Uncomment line below for help lines
%\draw[help lines] grid(5,4);

\node(locx)  {$\ke_\vx \mapsto$};

\matrix(versionx) [version list]
    at ([xshift=\tikzkvspace]locx.east) {
    {a} & $\txid_0$ & {a} & $\txid_1$ & {a} & $\txid_2$\\
    {a} & $\Set{\txid_1}$ & {a} & $\Set{\txid_2}$ & {a} & $\emptyset$\\
};

\tikzvalue{versionx-1-1}{versionx-2-1}{locx-v0}{0};
\tikzvalue{versionx-1-3}{versionx-2-3}{locx-v1}{1};
\tikzvalue{versionx-1-5}{versionx-2-5}{locx-v2}{2};


\path (locx.south) + (0,\tikzkeyspace) node (locf1) {$\ke_{\pv{f1}} \mapsto$};
\matrix(versionf1) [version list]
    at ([xshift=\tikzkvspace]locf1.east) {
    {a} & $\txid_0$ & {a} & $\txid_1$\\
    {a} & $\Set{\txid_1}$ & {a} & $\emptyset$\\
};
\tikzvalue{versionf1-1-1}{versionf1-2-1}{locf1-v0}{0};
\tikzvalue{versionf1-1-3}{versionf1-2-3}{locf1-v1}{1};

%Location y
\path (locf1.south) + (0,\tikzkeyspace) node (locf2) {$\ke_{\pv{f2}} \mapsto$};
\matrix(versionf2) [version list]
    at ([xshift=\tikzkvspace]locf2.east) {
    {a} & $\txid_0$ & {a} & \(\txid_2\) \\
    {a} & $\emptyset$ & {a} & \(\emptyset\) \\
};
\tikzvalue{versionf2-1-1}{versionf2-2-1}{locf2-v0}{0};
\tikzvalue{versionf2-1-3}{versionf2-2-3}{locf2-v1}{1};

% \draw[-, red, very thick, rounded corners] ([xshift=-5pt, yshift=5pt]locx-v1.north east) |- 
%  ($([xshift=-5pt,yshift=-5pt]locx-v1.south east)!.5!([xshift=-5pt, yshift=5pt]locy-v0.north east)$) -| ([xshift=-5pt, yshift=5pt]locy-v0.south east);

%blue view - I should  check whether I can use pgfkeys to just declare the list of locations, and then add the view automatically.
\draw[-, blue, very thick, rounded corners=10pt]
([xshift=-2pt, yshift=20pt]locx-v0.north east) node (tid1start) {} -- 
([xshift=-2pt, yshift=-5pt]locf2-v0.south east);

\path (tid1start) node[anchor=south, rectangle, fill=blue!20, draw=blue, font=\small, inner sep=1pt] {$\thid_3$};

%red view
\draw[-, red, very thick, rounded corners = 10pt]
([xshift=-5pt, yshift=5pt]locx-v2.north east) -- 
([xshift=-5pt, yshift=-5pt]locx-v2.south east) --
([xshift=-5pt, yshift=5pt]locf1-v1.north east) -- 
([xshift=-5pt, yshift=-10pt]locf2-v1.south east) node (tid2start) {};

\path (tid2start) node[anchor=north, rectangle, fill=red!20, draw=red, font=\small, inner sep=1pt] {$\thid_2$};
 
 %green view
\draw[-, DarkGreen, very thick, rounded corners = 10pt]
([xshift=-16pt, yshift=8pt]locx-v1.north east) node (tid3start) {}-- 
([xshift=-16pt, yshift=-5pt]locf1-v1.south east) --
([xshift=-16pt, yshift=5pt]locf2-v0.north east) -- 
([xshift=-16pt, yshift=-5pt]locf2-v0.south east);

\path (tid3start) node[anchor=south, rectangle, fill=DarkGreen!20, draw=DarkGreen, font=\small, inner sep=1pt] {$\thid_1$};

\end{pgfonlayer}
\end{centertikz}%
\caption{After \(\txid_2\)}
\label{fig:ua-correct-after-tx2}
\end{subfigure} 
\\
\end{tabular}
\hrule
\caption{A execution of $\prog_3$ without lost-update}
\label{fig:ua-conf-2}
\end{figure}


\ac{This Consistency Model shows why the notion of consistent views must 
depend on the set of operations that need to be executed.}

The next consistency model that we consider is \emph{update atomic}. 
Although we did not find any implementation of this model, it has been proposed in \cite{framework-concur} as a strengthening to Read Atomic to avoid write-write conflicts.
This model states that: \textbf{(i)} transactions satisfy atomic visibility (\cref{def:readatomic}); and \textbf{(ii)} transactions writing to one same keys cannot be executed concurrently.
\sx{ This appears too earlier:
Update Atomic is also needed to specify more sophisticated consistency models, 
such as \emph{Parallel Snapshot Isolation} and \emph{Snapshot Isolation}.}
\ac{Check: Nobi said he was interested in implementing Update Atomic 
at some point, maybe he ended up doing something.}

Programs under update atomic do not exhibit the \emph{lost update} anomaly: two or more transactions update the same address, for example , both increment its value by $1$, but only one of them will be observed by future transactions, for example, only one of the increments takes effect.
To illustrate the \emph{lost-update anomaly}, consider the following program \( \prog_3 \) where two transactions concurrently increment $\vx$ and the third transaction read the value. 
Note that the \( \pvar{f1} \) and \( \pvar{f2} \) are two flags indicating the corresponding transactions has been committed.
\ac{Intuitive behaviour of the litmus test: two transactions concurrently increment $[\loc_x]$. 
 A third transaction observes that the first two transactions have been executed. 
 However, it only observes one of the two increments taking place.
 }
\[
    \prog_3 \equiv \begin{session}
        \begin{array}{@{}c || c || c@{}}
        \txid_1 : 
        \begin{transaction} 
            \pmutate{\pvar{f1}}{1};\\
            \pderef{\pvar{a}}{\vx};\\
            \pmutate{\vx}{a + 1};\\
        \end{transaction} & 
        \txid_2 : 
        \begin{transaction}
            \pmutate{\pvar{f2}}{1};\\
            \pderef{\pvar{a}}{\vx};\\
            \pmutate{\vx}{a + 1};\\
        \end{transaction} &
        \txid_3 : 
        \begin{transaction}
            \pderef{\pvar{a}}{\vx};\\
            \pderef{\pvar{b}}{\pvar{f1}};\\
            \pderef{\pvar{c}}{\pvar{f2}};\\
            \pifs{\pvar{a}=1 \wedge \pvar{b}=1 \wedge \pvar{c} = 1}\\ 
                \quad \passign{\retvar}{\sadface}
            \pife
        \end{transaction}
        \end{array}
    \end{session}
 \]

We consider an execution in which the transactions contained in the code of threads $\thid_1, \thid_2$ both execute on the same snapshot determined by the initial view. 
The initial configuration of the program coincides with the one given in \cref{fig:ua-init}.
After executing the transaction $\txid_1$, the resulting configuration is the one depicted \ref{fig:ua-after-tx1} and then \( \txid_2 \) shown in in \ref{fig:ua-after-tx2}, where both transactions read the initial version for key $\ke_\vx$. 
The third transaction $\txid_3$ choose to update its view to include the most recent version for all the keys (\ref{fig:ua-before-tx3}), then when executing its code, all the keys will have value $1$, and the return variable will be set to ${\sadface}$.

The program $\prog_3$ might exhibit the lost-update anomaly when the second transaction $\txid_2$ starts, its view did not include the most up-to-date version for key $\ke_\vx$ provided that \( \txid_{2}\) will update the key \( \ke_\vx \).
As consequence, the database \emph{lost the update} of a version of \( \ke_\vx \) installed by the transaction $\txid_1$, in a sense that no transaction will observe such the version.
To forbid this anomaly, the \emph{update atomic} requires that if a transaction writes to a key, the transaction should start with a view including the most recent version for the key.

\begin{definition}
\label{def:update-atomic}
\emph{Update atomic} is stronger than then read atomic (\cref{def:readatomic}) by further requiring for all keys written, it should starts with a view including the most recent version for those key:
\[
\begin{rclarray}
(\hh, \vi) \csat[\mathsf{UA}] \opset: \vi' & \defeq &
\begin{array}[t]{@{}l}
(\hh, \vi) \csat[\mathsf{RA}] \opset: \vi' \land \fora{\addr} 
(\otW, \addr, \stub) \in \opset \implies \vi(\addr)  = \left| \hh(\addr) \right| - 1
\end{array} \\
\end{rclarray}
\]
\end{definition}

\begin{proposition}
The execution test $\comoUA$ does not hinder progress. 
For any $\hh, \vi, \opset$, there exist $\vi' : \vi \leq \vi'$ and $\vi'': \Vupdate(\hh, \vi', \opset) \leq \vi''$ such that $(\hh, \vi') \csatUA \opset, \vi''$.
\end{proposition}

The thread $\thid_2$ from the program \( \prog_3\) , under $\mathsf{UA}$, cannot execute the transaction $\txid_2$ starting from the configuration depicted in \cref{fig:ua-after-tx1}.
Because the view of $\thid_2$ does not include the most recent version for key $\ke_\vx$. 
Instead, before executing, $\thid_2$ must update its view to include the most recent version of $\ke_\vx$ (\cref{fig:ua-thid-2-update-view}).
Then the \( \txid_2\) will install a new version for \( \ke_\vx \) with value 2 instead of 1 as shown in \cref{fig:ua-correct-after-tx2}.
There are now three different possible views in which $\thid_3$ can execute its transaction.
First, executing on the initial view, in which the transaction will observe 0 for the three locations, and the transaction will not return value $\sadface$.
Second, executing on the one in which the view of $\thid_3$ for $\ke_\vx$ points to the version $(1, \tsid_1, \Set{\tsid_2})$. 
Because of atomic visibility, it must also includes the most recent version for key $\ke_\pv{f2}$ since it is installed by \( \txid_2 \).
In this case, it will not return \(\sadface \).
Last, executing on the one in which the view of $\thid_3$ for $\ke_\vx$ points to its most recent version $(2, \txid_2, \emptyset)$.
In this case, it will not return \(\sadface \).

\subsection{Consistency Prefix \( \CP \) }
\label{sec:sound-complete-cp}

Given abstract execution \( \aexec \), we define read-write read-write relation:
\[
    \RW(\aexec,\ke) \defeq \Setcon{(\txid, \txid')}{\txid \toEdge{\AR_\aexec} \txid' \land (\otR,\ke, \stub ) \in \txid \land (\otW,\ke, \stub ) \in \txid'  } 
\]
It is easy to see \( \RW(\aexec,\ke) \)  can be derived from \( \WW(\aexec,\ke) \) and \( \WR(\aexec, \ke ) \) as the following:
\[
    \RW(\aexec,\ke) = \Setcon{(\txid, \txid')}{ \exsts{\txid'' } (\txid'', \txid) \in \WR(\aexec, \ke) \land (\txid'', \txid') \in \WW(\aexec, \ke) }
\]
Then, the notation \( \RW_\aexec \defeq \bigcup\limits_{\ke \in \Keys} \RW(\aexec, \ke) \).
Note that for a key-value store \( \mkvs \) such that \( \mkvs = \mkvs_\aexec \),
by the definition of  \(  \mkvs = \mkvs_\aexec \), 
the following holds:
\[
    \RW_\aexec = \Setcon{(\txid, \txid')}{\exsts{\ke, i,j } \txid \in \RTx(\mkvs(\ke, i)) \land \txid' = \WTx(\mkvs(\ke, j)) \land i < j}
\]
The \( \RW_\aexec \) also coincides with \( \RW_\Gr \) and \( \RW_\mkvs \).


An abstract execution \( \aexec \) satisfies consistency prefix (\(\CP\)), 
if it satisfies \( \AR_\aexec ; \VIS_\aexec \subseteq \VIS_\aexec \) and \( \SO_\aexec \subseteq \VIS_\aexec \).
Given the specification, there is a corresponding specification on dependency graph by solve the following inequalities:
\[
    \begin{array}{@{}l@{}}
        \WR \subseteq \VIS \\
        \WW \subseteq \AR \\
        \VIS \subseteq \AR \\
        \VIS ; \RW \subseteq \AR \\
        \AR ; \AR \subseteq \AR  \\
        \SO \subseteq \VIS \\
        \AR ; \VIS \subseteq \VIS
    \end{array}
\]
By solving the inequalities the visibility and arbitration relations are:
\[
    \begin{rclarray}
        \AR & \defeq & \left( (\SO \cup \WR ) ; \RW? \cup \WW \cup R \right)^+ \\
        \VIS & \defeq & \left( (\SO \cup \WR ) ; \RW? \cup \WW \cup R \right)^* ; (\SO \cup \WR )
    \end{rclarray}
\]
for some relation \( R \subseteq \AR \).
When \( R = \emptyset \), it is the smallest solution therefore the minimum visibility required.

\sx{A bit verbal}
\begin{lemma}
    \label{lem:cp-eauiv-spec}
    For any abstract execution \( \aexec \),
    if it satisfies 
    \[
        \left( (\SO \cup \WR ) ; \RW? \cup \WW \right)^* ; \VIS_\aexec \subseteq \VIS_\aexec 
        \qquad \SO_\aexec \subseteq \VIS_\aexec
    \]
    then there exists a new \( \aexec' \) such that \( \T_\aexec = \T_{\aexec'} \), 
    under last-write-win \( \TtoOp{T}_{\aexec}(\txid) = \TtoOp{T}_{\aexec'}(\txid) \) for all transactions \( \txid \),
    and the relations satisfy the following:
    \[ 
        \AR_{\aexec'} ; \VIS_{\aexec'} \subseteq \VIS_{\aexec'}  \qquad \SO_{\aexec'} \subseteq \VIS_{\aexec'}
    \]
    and vice versa.
\end{lemma}
\begin{proof}
    Assume abstract execution \( \aexec' \) that
    satisfies \( \AR_{\aexec'} ; \VIS_{\aexec'} \subseteq \VIS_{\aexec'} \)
    and  \( \SO_{\aexec'} \subseteq \VIS_{\aexec'} \).
    We already show that:
\[
    \begin{rclarray}
        \AR_{\aexec'} & = & \left( (\SO_\aexec \cup \WR_\aexec ) ; \RW_\aexec? \cup \WW_\aexec \cup R \right)^+ \\
        \VIS_{\aexec'} & = & \left( (\SO_\aexec \cup \WR_\aexec ) ; \RW_\aexec? \cup \WW_\aexec \cup R \right)^* ; (\SO_\aexec \cup \WR_\aexec )
    \end{rclarray}
\]
for some relation \( R \subseteq \AR_{\aexec'} \).
If we take \( R  = \emptyset \), we have the proof for:
\[
        \SO \subseteq \VIS_\aexec \qquad 
        \left( (\SO_\aexec \cup \WR_\aexec ) ; \RW_\aexec? \cup \WW_\aexec \right)^* ; \VIS_\aexec \subseteq \VIS_\aexec
\]
For another way, we pick the \( R \) that extends
\( \left( (\SO_\aexec \cup \WR_\aexec ) ; \RW_\aexec? \cup \WW_\aexec \cup R \right)^+ \) 
to a total order.
\end{proof}

By \cref{lem:cp-eauiv-spec} to prove soundness and completeness of \( \ET_\CP \), it is sufficient to use the specification:
\[
    (\RP_{\LWW}, \Set{\lambda \aexec. \left( (\SO \cup \WR ) ; \RW? \cup \WW \right)^* ; \VIS_\aexec, \lambda \aexec \ldotp \SO_\aexec }) 
\]

For the soundness, we pick the invariant as the following:
\[  
\begin{rclarray}
    I_1(\aexec, \cl) & = & \left( \bigcup\limits_{\{\txid_{\cl}^{i} \in \T_{\aexec} \mid i \in \Nat\}} \VIS_{\aexec}^{-1}(\txid^i_\cl) \right) \setminus \T_\rd \\
    I_2(\aexec, \cl) & = & \left( \bigcup\limits_{\{\txid_{\cl}^{i} \in \T_{\aexec} \mid i \in \Nat\}} (\SO_{\aexec}^{-1})?(\txid^i_\cl) \right) \setminus \T_\rd
\end{rclarray}
\]
where \( \T_\rd \) is all the read-only transactions included in both 
\( \left( \bigcup\limits_{\{\txid_{\cl}^{i} \in \T_{\aexec} \mid i \in \Nat\}} \VIS_{\aexec}^{-1}(\txid^i_\cl) \right)\) 
and \( \left( \bigcup\limits_{\{\txid_{\cl}^{i} \in \T_{\aexec} \mid i \in \Nat\}} (\SO_{\aexec}^{-1})?(\txid^i_\cl) \right) \).
Assume a key-value store $\hh$, an initial and a final view $\vi, \vi'$  a fingerprint $\opset$ 
such that $\ET_{\CP} \vdash (\hh, \vi) \triangleright \opset: \vi'$. 
Also choose an arbitrary $\cl$, a transaction identifier $\txid_\cl^n \in \nextTxId(\hh, \cl)$, 
and an abstract execution $\aexec$ such that $\hh_{\aexec} = \hh$ and 
\( I_1(\aexec, \cl) \cup I_2(\aexec, \cl) \subseteq \Tx(\hh, \vi) \).
Let a new abstract execution \( \aexec' = \extend(\aexec, \txid_\cl^n, \f, \Tx(\mkvs, \vi) \cup \T_\rd) \).
We are about to prove that there exists an extra set of read-only transaction \( \T'_\rd \) such that:
\begin{gather}
    \fora{\txid} (\txid, \txid_\cl^n) \in \SO_{\aexec'} \implies \txid \in \Tx(\mkvs, \vi) \cup \T_\rd \cup \T'_\rd \label{equ:cp-sound-update-so}\\
    \begin{array}{l}
    \fora{\txid} (\txid, \txid_\cl^n) \in \left( (\SO_{\aexec'} \cup \WR_{\aexec'} ) ; \RW_{\aexec'}? \cup \WW_{\aexec'} \right)^* ; \VIS_{\aexec'} \\
    \qqquad \implies \txid \in \Tx(\mkvs, \vi) \cup \T_\rd \cup \T'_\rd 
    \end{array}
    \label{equ:cp-sound-update-arvis}\\
    I_1(\aexec',\cl) \cup I_2(\aexec',\cl) \subseteq \Tx(\mkvs_{\aexec'}, \vi') \label{equ:cp-sound-inv} 
\end{gather}
\begin{itemize}
\item the invariant \( I_2 \) implies the \cref{equ:cp-sound-update-so} where the proof is the same as \( \RYW \) in \cref{sec:sound-complete-ryw}.

\item For \cref{equ:cp-sound-update-arvis}, it is sufficient to prove one step inclusion, \ie
\[
    \begin{array}{l}
    \fora{\txid} (\txid, \txid_\cl^n) \in \left( (\SO_{\aexec'} \cup \WR_{\aexec'} ) ; \RW_{\aexec'}? \cup \WW_{\aexec'} \right) ; \VIS_{\aexec'} \\
    \qqquad \implies \txid \in \Tx(\mkvs, \vi) \cup \T_\rd \cup \T'_\rd 
\end{array}
\]
To prove above, let \( \T'_\rd \) initially be empty set.
We will add more read-only transactions until it satisfies \cref{equ:cp-sound-update-arvis}.
Assume a transaction \( \txid \) such that 
\( (\txid, \txid_\cl^n) \in \left( (\SO_{\aexec'} \cup \WR_{\aexec'} ) ; \RW_{\aexec'}? \cup \WW_{\aexec'} \right) ; \VIS_{\aexec'}\).
There exists a transaction \( \txid' \) such that \( \txid \toEdge{(\SO_{\aexec'} \cup \WR_{\aexec'} ) ; \RW_{\aexec'}? \cup \WW_{\aexec'}} \txid' \toEdge{\VIS_{\aexec'}}  \txid_\cl^n \).
It follows \( \txid'  \in \Tx(\mkvs, \vi) \cup \T_\rd \cup \T'_\rd  \).
Note that \( \txid \) and \( \txid' \) must exist in the abstract execution \( \aexec \) before update.
There are two cases: \( \txid' \) writes to at least a key; or \( \txid' \) is a read-only transaction.
\begin{itemize}
    \item
    If \( \txid' \) writes to at least a key, then \( \txid' \in \Tx(\mkvs, \vi)\).
    %\[
        %\begin{rclarray}
            %\func{RW^{-1}}{\mkvs, \ke, i} & \defeq & \Setcon{\txid}{\exsts{ j \leq i } \txid \in \RTx(\mkvs(\ke,j))} \\
            %\ddagger & \equiv &
            %\begin{array}[t]{@{}l@{}}
                %\fora{\ke, \ke', i, j, m, \txid, \txid', \txid''} \\
                %\left( \begin{array}{@{}l@{}}
                %i \in \vi(\ke) 
                %\land \txid \in \Set{\WTx(\mkvs(\ke,i))} \cup \func{RW^{-1}}{\mkvs, \ke, i} \land {} \\
                %\quad \left(
                    %\begin{array}{@{}l @{}}
                        %\left( \begin{array}{@{}l@{}}
                                %\txid' \in \func{SO^{-1}}{\txid} \land {} \\
                                %\txid' \in \Set{\WTx(\mkvs(\ke',j))} \cup  \RTx(\mkvs(\ke',j))
                        %\end{array} \right)  \lor {} \\
                        %\left( \begin{array}{@{}l@{}}
                                %\txid \in \RTx(\mkvs(\ke',j)) \land \txid' = \WTx(\mkvs(\ke',j))
                        %\end{array} \right)
                        %\end{array} \right) 
                    %\end{array}
                    %\right)  \\
                    %{} \lor \left( \begin{array}{@{}l@{}}
                            %i \in \vi(\ke) \land \ke = \ke' \land j < i
                    %\end{array} \right) \\
                    %\qquad \implies j \in \vi(\ke') 
            %\end{array} \\
        %\end{rclarray}
    %\]
    %We link the conditions in \( \ddagger \) to relation:
    %\begin{itemize}
        %\item \( \RW_\aexec\). Assume a key \( \ke \),  an index \( i \) and the writer \( \txid  = \WTx(\mkvs(\ke,i))\),
    %then \( \txid' \in \RW^{-1}(\mkvs, \ke, i)\) if and only if \( \txid' \toEdge{\RW_\aexec} \txid\).
        %\item \( \SO_\aexec\). The transaction identifiers encode the \( \SO_\aexec \).
        %That is, \( \txid' \in \SO^{-1}(\txid)\) if and only if \(\txid' \toEdge{\SO_\aexec} \txid \).
        %\item  \( \WR_\aexec \). It is easy to see \( \txid \in \RTx(\mkvs(\ke',j)) \land \txid' = \WTx(\mkvs(\ke',j)) \) if and only if \( \txid' \toEdge{\WR_\aexec} \txid \).
        %\item \( \WW_\aexec \). The write-write relation describes the order of write operations for a key which corresponds the version orders in key-value store.
        %That is, \( \txid' = \WTx(\mkvs(\ke,j)) \land \txid = \WTx(\mkvs(\ke,i)) \land j < i\) if and only if
        %\( \txid' \toEdge{\WW_\aexec} \txid\).
    %\end{itemize}
    %Let assume \( \txid' \) writes to i-\emph{th} version a key \( \ke \).
    %Given above and 
    %\[ \txid \toEdge{(\SO_{\aexec'} \cup \WR_{\aexec'} ) ; \RW_{\aexec'}? \cup \WW_{\aexec'}} \txid' \toEdge{\VIS_{\aexec'}}  \txid_\cl^n \] we can substitute and rewrite the \( \ddagger \) as the following:
    %\begin{gather}
        %\begin{array}{@{}l@{}}
            %\fora{\txid'',\ke',j}
            %\WTx(\mkvs(\ke,i)) = \txid' \land {} \\
            %\left( \begin{array}{@{}l@{}}
            %\txid'' \toEdge{\RW_{\aexec'}?} \txid' \land {} \\
            %\quad \left(
                %\begin{array}{@{}l @{}}
                    %\left( \begin{array}{@{}l@{}}
                            %\txid \toEdge{\SO_{\aexec'}} \txid'' \land 
                            %\txid \in \Set{\WTx(\mkvs(\ke',j))} \cup  \RTx(\mkvs(\ke',j))
                    %\end{array} \right)  \\
                    %{} \lor 
                    %\left( \begin{array}{@{}l@{}}
                            %\txid \toEdge{\WR_{\aexec'}} \txid'' \land \txid = \WTx(\mkvs(\ke',j))
                    %\end{array} \right)
                    %\end{array} \right) 
                %\end{array}
                %\right)  \\
                %{} \lor \left( \begin{array}{@{}l@{}}
                        %\txid \toEdge{\WW_{\aexec'}} \txid'' \land \txid = \WTx(\mkvs(\ke',j))
                %\end{array} \right) \\
                %\qquad \implies j \in \vi(\ke') 
        %\end{array} 
        %\label{equ:cp-dagger}
    %\end{gather}
    Now we perform case analysis if \( \txid \) is a read-only transaction.
    \begin{itemize}
        \item if \( \txid \) has write, we prove \( \txid \in \Tx(\mkvs, \vi)\).
        Recall the \( \ddagger \) is defined as the following:
        \begin{equation}
        \label{equ:cp-dagger}
        \ddagger  \equiv 
            \fora{\ke, \ke', i, j}
                i \in \vi(\ke)  \wedge \WTx(\hh(\ke', j)) \toEdge{(((\PO \cup \RF_{\hh}) ; \AD_{\hh}?) \cup \VO_{\hh})^{+}} \WTx(\hh(\ke, i))
            \implies j \in \vi(\ke')  
        \end{equation}
        Since \( \WR_\mkvs \), \( \WW_\mkvs \) and \( \RW_\mkvs \) coincide with
        \( \WR_\aexec \), \( \WW_\aexec \) and \( \RW_\aexec \) respectively.
        Also because \( \txid \) write to at least one key,
        it is easy to see there exists some version \( \ke'',m\) such that 
        \( \txid = \WTx(\mkvs(\ke'',m))\) and \( m \in \vi(\ke'')\).
        By definition of \( \Tx \), it follows \( \txid \in \Tx(\mkvs, \vi) \).
        %Therefore by the definition of \( \Tx \), then \( \txid \in \VIS^{-1}(\txid_\cl^n)\).
        \item if \( \txid \) is a read-only transaction, we add it into \( \T'_\rd \).
    \end{itemize}
    \item 
    if \( \txid' \) is a read-only transaction, then either \( \txid' \in \T_\rd \) or \( \txid' \in \T'_\rd \).
    More specifically we have three cases: \textbf{(i)} \( \txid' \in \bigcup\limits_{\{\txid_{\cl}^{i} \in \T_{\aexec} \mid i \in \Nat\}} \VIS_{\aexec}^{-1}(\txid^i_\cl) \), \textbf{(ii)} \( \txid' \in \bigcup\limits_{\{\txid_{\cl}^{i} \in \T_{\aexec} \mid i \in \Nat\}} (\SO_{\aexec}^{-1})?(\txid^i_\cl) \) or \textbf{(iii)} \( \txid' \in \T'_\rd\).
    \begin{itemize}
        \item
        Assume \( \txid' \in \bigcup\limits_{\{\txid_{\cl}^{i} \in \T_{\aexec} \mid i \in \Nat\}} \VIS_{\aexec}^{-1}(\txid^i_\cl) \).
        It means \( \txid' \) is visible for some previous transaction \( \txid_\cl^m \) (\( m < n \)) from the same client \( cl \), 
        \ie 
        \[ 
            \txid \toEdge{(\SO_{\aexec'} \cup \WR_{\aexec'} ) ; \RW_{\aexec'}? \cup \WW_{\aexec'}} \txid' \toEdge{\VIS_{\aexec'}}  \txid_\cl^m 
        \]
        Note that all the edges before \( \txid_\cl^m \) must exist in \( \aexec \).
        Since \( \aexec \) satisfies the \( \left( (\SO \cup \WR ) ; \RW? \cup \WW \right)^* ; \VIS_\aexec \subseteq \VIS_\aexec \),
        we have \( \txid \toEdge{\VIS_{\aexec'}} \txid_\cl^m \) and then \( \txid \in \bigcup\limits_{\{\txid_{\cl}^{i} \in \T_{\aexec} \mid i \in \Nat\}} \VIS_{\aexec}^{-1}(\txid^i_\cl)\).
        By the invariant \( I_1 \), it means \( \txid \in \Tx(\mkvs, \vi) \cup \T_\rd \).
    \item \( \txid' \in \bigcup\limits_{\{\txid_{\cl}^{i} \in \T_{\aexec} \mid i \in \Nat\}} \SO_{\aexec}^{-1}(\txid^i_\cl) \).
    Since \( \txid' \) is a read-only transaction, 
    the edges can be simplified to \( \txid \toEdge{(\SO_{\aexec'} \cup \WR_{\aexec'} )} \txid' \toEdge{\SO_{\aexec'}}  \txid_\cl^n \).
    Given that \( \SO \) is transitive, then  either \( \txid \toEdge{\SO_{\aexec'}} \txid_\cl^n \) or \( \txid \toEdge{\WR_{\aexec'} } \txid' \toEdge{\SO_{\aexec'}}  \txid_\cl^n \).
    \begin{itemize}
        \item \( \txid \toEdge{\SO_{\aexec'}} \txid_\cl^n \).
            It follows \( \txid \in \bigcup\limits_{\{\txid_{\cl}^{i} \in \T_{\aexec} \mid i \in \Nat\}} \SO_{\aexec}^{-1}(\txid^i_\cl) = \Tx(\mkvs, \vi) \cup \T_\rd \).
        \item \( \txid \toEdge{\WR_{\aexec'} } \txid' \toEdge{\SO_{\aexec'}}  \txid_\cl^n \).
            The \( \WR \) edge must exists in \( \aexec \).
            Because \( \WR_\aexec \subseteq \VIS_\aexec \) then  \( \txid \toEdge{\VIS_{\aexec} } \txid' \toEdge{\SO_{\aexec'}}  \txid_\cl^n  \).
            It means 
            \[ 
                \txid \in \bigcup\limits_{\{\txid_{\cl}^{i} \in \T_{\aexec} \mid i \in \Nat\}} \VIS_{\aexec}^{-1}(\txid^i_\cl) = \Tx(\mkvs, \vi) \cup \T_\rd 
            \]
    \end{itemize}
    \item 
    Last, \( \txid' \in \T'_\rd \).
    Since \( \T'_\rd \) initially is empty set, there exists another write transaction \( \txid'' \) such that:
    \[
        \txid \toEdge{(\SO_{\aexec'} \cup \WR_{\aexec'} ) ; \RW_{\aexec'}? \cup \WW_{\aexec'}} \txid' \toEdge{(\SO_{\aexec'} \cup \WR_{\aexec'} ) ; \RW_{\aexec'}? \cup \WW_{\aexec'}} \txid'' \toEdge{\VIS_{\aexec'}}  \txid_\cl^n
    \]
    %Given that \( \txid' \) is a read-only and \( \txid'' \) has write, the edges can be simplified:
    %\[
        %\txid \toEdge{(\SO_{\aexec'} \cup \WR_{\aexec'} )} \txid' \toEdge{\SO_{\aexec'} ; \RW_{\aexec'}?} \txid'' \toEdge{\VIS_{\aexec'}}  \txid_\cl^n
    %\]
    %Because transitivity of  \( \SO \), we have the following two cases:
    %\[
        %\begin{array}{@{}l@{}}
            %\txid \toEdge{ \WR_{\aexec'} } \txid' \toEdge{\SO_{\aexec'} ; \RW_{\aexec'}?} \txid'' \toEdge{\VIS_{\aexec'}}  \txid_\cl^n \\
            %\txid \toEdge{\SO_{\aexec'} ; \RW_{\aexec'}?} \txid'' \toEdge{\VIS_{\aexec'}}  \txid_\cl^n 
        %\end{array}
    %\]
    %\( \txid \toEdge{ \WR_{\aexec'} } \txid' \toEdge{\SO_{\aexec'} ; \RW_{\aexec'}?} \txid'' \toEdge{\VIS_{\aexec'}}  \txid_\cl^n \).
        If \( \txid \) has write, by \cref{equ:cp-dagger} then \( \txid \in \Tx(\mkvs,\vi) \).
        Otherwise if \( \txid \) is a read only transaction, we add it into \( \T'_\rd \).
            %\( \txid \toEdge{\SO_{\aexec'} ; \RW_{\aexec'}?} \txid'' \toEdge{\VIS_{\aexec'}}  \txid_\cl^n \).
            %Similarly by \cref{equ:cp-dagger}, either \( \txid \in \Tx(\mkvs,\vi) \)  or we add it into \( \T'_\rd \).
    \end{itemize}
\end{itemize}

\item Since \( \CP \) satisfies \( \RYW \) and \( \MRd \), thus invariants \( I_1 \) and  \( I_2 \) are preserved after update.

\end{itemize}

    
For completeness, we prove the three parts of the execution test separately.
\begin{itemize}
\item Since \( \SO_\aexec \subseteq \VIS_\aexec  \), the prove for \( \ET_\RYW \) is the as in \cref{sec:sound-complete-mr}.
\item For any \( \VIS_\aexec \)  satisfies the constraint for \( \CP \), by \cref{lem:cp-eauiv-spec} it satisfies that 
\[
    \VIS \defeq \left( (\SO \cup \WR ) ; \RW? \cup \WW \cup R \right)^* ; (\SO \cup \WR )
\]
for some relation \( R \).
It means \( \VIS_\aexec ; \SO_\aexec \subseteq \VIS_\aexec \).
Therefore it is complete with respect to \( \ET_\MRd \).

\item Let consider the \( \ddagger \).
Assume i-\emph{th} transaction \( \txid_i \) in the arbitrary order,
and let view \( \vi_{i} = \getView(\aexec, \VIS^{-1}_{\aexec}(\txid_{i}) ) \).
We also pick any final view such that \( \vi'_{i} \subseteq \getView(\aexec, (\AR^{-1}_{\aexec})?(\txid_{i}) ) \).
Note that there is nothing to prove for \( \vi'_i \) since the \( \ddagger \) does not constrain the \( \vi'_i \).
Recall the \( \ddagger \):
\[
\ddagger  \equiv 
        \fora{\ke, \ke', m, j}
             m \in \vi(\ke)  \wedge \WTx(\hh(\ke', j)) \toEdge{(((\PO \cup \RF_{\hh}) ; \AD_{\hh}?) \cup \VO_{\hh})^{+}} \WTx(\hh(\ke, m))
         \implies j \in \vi(\ke')  
\]
Assume \( j \in \vi_i(\ke) \) for some key \(\ke \) and index \( i \).
It means the writer of the version is visible by the transaction \( \txid_i\),
\ie \( \WTx(\mkvs(\ke,i)) \in \VIS^{-1}_{\aexec}(\txid_{i}) \).
Let the \( \mkvs = \mkvs_{\cut(\aexec, i-1)} \).
We need to prove the following:
\begin{gather}
    \label{equ:cp-complete-arvis}
    \begin{array}{@{}l@{}}
        \fora{\ke, \ke', m, j, \txid, \txid'} 
        m \in \vi(\ke) 
        \land \WTx(\mkvs(\ke,m)) \in \VIS_\aexec^{-1}(\txid_i) \\
        \quad {} \land \WTx(\hh(\ke', j)) \toEdge{(((\PO \cup \RF_{\hh}) ; \AD_{\hh}?) \cup \VO_{\hh})^{+}} \WTx(\hh(\ke, m)) \\
            \qquad \implies \WTx(\mkvs(\ke',j)) \in \VIS_\aexec^{-1}(\txid_i)
    \end{array}
\end{gather}
%Note that \( \txid \in \Set{\WTx(\mkvs(\ke,i))} \cup \func{RW^{-1}}{\mkvs, \ke, i} \) 
%means \( \txid \toEdge{\RW_{\aexec}?} \WTx(\mkvs(\ke,i)) \),
%the formulae \(\left( \begin{array}{@{}l@{}} \txid \in \RTx(\mkvs(\ke',j)) \land \txid' = \WTx(\mkvs(\ke',j)) \end{array} \right) \) 
%means \( \txid \toEdge{\WR_\aexec} \txid' \),
%and \( \left( \begin{array}{@{}l@{}} \txid = \WTx(\mkvs(\ke',m)) \land \txid' = \WTx(\mkvs(\ke',j)) \land m > j \end{array} \right) \) 
%means \( \txid \toEdge{\WW_\aexec} \txid' \).
%Given all the correspondence, the \cref{equ:cp-complete-arvis} holds if the following holds:
%\[
    %\begin{rclarray}
        %\begin{array}[t]{@{}l@{}}
            %\fora{\ke, \ke', i, j, \txid, \txid'} \\
            %\left( \begin{array}{@{}l@{}}
            %i \in \vi(\ke) 
            %\land \WTx(\mkvs(\ke,i)) \in \VIS_\aexec^{-1}(\txid_i) \\
            %{} \land \txid' = \WTx(\mkvs(\ke',j))
            %\land \txid \toEdge{\RW_{\aexec}?} \WTx(\mkvs(\ke,i)) \land {} \\
            %\left(
                %\begin{array}{@{}l @{}}
                    %\txid' \toEdge{\WR_\aexec ; \SO_\aexec}\txid \lor
                    %\txid' \toEdge{\SO_\aexec}\txid \lor
                    %\txid' \toEdge{\WR_\aexec}\txid
                    %\end{array} \right) 
                %\end{array}
                %\right)  \\
                %{} \lor \txid' \toEdge{\WW_\aexec} \WTx(\mkvs(\ke,i)) \\
                %\qquad \implies \txid' \in \VIS_\aexec^{-1}(\txid_i)
        %\end{array} \\
    %\end{rclarray}
%\]
%Then the above holds, if the following holds:
%\begin{gather}
    %\label{equ:cp-complete-arvis-2}
    %\begin{rclarray}
        %\begin{array}[t]{@{}l@{}}
            %\fora{\ke, i, \txid} \\
            %\left( \begin{array}{@{}l@{}}
            %i \in \vi(\ke) 
            %\land \WTx(\mkvs(\ke,i)) \in \VIS_\aexec^{-1}(\txid_i) \\
            %{} \land \txid \toEdge{( (\WR_\aexec; \SO_\aexec) \cup \SO_\aexec \cup \WR_\aexec) ; \RW_{\aexec}? \cup \WW_\aexec} \WTx(\mkvs(\ke,i)) 
                %\end{array}
                %\right)  \\
                %\qquad \implies \txid \in \VIS_\aexec^{-1}(\txid_i)
        %\end{array} \\
    %\end{rclarray}
%\end{gather}
Since \( \WR_\mkvs \), \( \WW_\mkvs \) and \( \RW_\mkvs \) coincide with
\( \WR_\aexec \), \( \WW_\aexec \) and \( \RW_\aexec \) respectively,
and \( \left( (\SO \cup \WR ) ; \RW? \cup \WW \right)^* ; \VIS_\aexec \subseteq \VIS_\aexec \),
It implies \cref{equ:cp-complete-arvis}.
\end{itemize}

\subsection{Parallel Snapshot Isolation \(\PSI\)}
\label{sec:sound-complete-psi}

The axiomatic definition for \( \PSI \) is 
\[ 
    (\RP_{\LWW}, \Set{\lambda \aexec. \VIS_{\aexec} ; \VIS_{\aexec}, \lambda \aexec \ldotp \SO_\aexec, \lambda \aexec. \WW_\aexec })
\]
Given the definition, there is a corresponding definition on dependency graph by solve the following inequalities:
\[
    \begin{array}{@{}l@{}}
        \WR \subseteq \VIS \\
        \WW \subseteq \VIS \\
        \SO \subseteq \VIS \\
        \VIS ; \VIS \subseteq \VIS 
    \end{array}
\]
We have \( \VIS = (\WR \cup \WW \cup \SO \cup R)^{+} \) for some \( R \subseteq \AR \).
Thus, there exist a minimum visibility such that 
\[ 
    (\RP_{\LWW}, \Set{\lambda \aexec. (\WR_{\aexec} \cup \WW_{\aexec} \cup \SO ) ; \VIS_{\aexec}, \lambda \aexec \ldotp \SO_\aexec, \lambda \aexec. \WW_\aexec })
\]

To prove soundness, we pick an invariant for the \( \ET_\PSI \) as the union of those for \( \MR\) and \( \RYW \) shown in the following:
\begin{align*}
    I_1(\aexec, \cl) & =  \left( \bigcup_{\Set{\txid_{\cl}^{n} \in \txidset_{\aexec} }[ n \in \Nat ]} \VIS_{\aexec}^{-1}(\txid^n_\cl) \right) \setminus \txidset_\rd \\
    I_2(\aexec, \cl) & =  \left( \bigcup_{\Set{\txid_{\cl}^{n} \in \txidset_{\aexec} }[ n \in \Nat ]} (\SO_{\aexec}^{-1})\rflx(\txid^n_\cl) \right) \setminus \txidset_\rd
\end{align*}
where \( \txidset_\rd \) is all the read-only transactions included in both 
\( \left( \bigcup_{\Set{\txid_{\cl}^{n} \in \txidset_{\aexec} }[ n \in \Nat ]} \VIS_{\aexec}^{-1}(\txid^n_\cl) \right)\) 
and \( \left( \bigcup_{\Set{\txid_{\cl}^{n} \in \txidset_{\aexec} }[ n \in \Nat ]} (\SO_{\aexec}^{-1})\rflx(\txid^n_\cl) \right) \).
Assume a kv-store $\mkvs$, an initial and a final view $\vi, \vi'$  a fingerprint $\fp$ 
such that $\ET_{\PSI} \vdash (\mkvs, \vi) \csat \fp: (\mkvs',\vi')$. 
Also choose an arbitrary $\cl$, a transaction identifier $\txid_\cl^n \in \nextTxid(\mkvs, \cl)$, 
and an abstract execution $\aexec$ such that $\mkvs_{\aexec} = \mkvs$ and 
\( I_1(\aexec, \cl) \cup I_2(\aexec, \cl) \subseteq \Tx[\mkvs, \vi] \).
We are about to prove there exists an extra set of read-only transactions \( \txidset'_\rd \) such that
the new abstract execution \( \aexec' = \extend[\aexec, \txid_\cl^n, \fp, \Tx[\mkvs, \vi] \cup \txidset_\rd \cup \txidset'_\rd] \) and:
\begin{gather}
    \fora{\txid} (\txid, \txid_\cl^n) \in \SO_{\aexec'} \implies \txid \in \Tx[\mkvs, \vi] \cup \txidset_\rd \cup \txidset'_\rd \label{equ:psi-sound-update-so}\\
    \fora{\txid} (\txid, \txid_\cl^n) \in \WW_{\aexec'} \implies \txid \in \Tx[\mkvs, \vi] \cup \txidset_\rd \cup \txidset'_\rd \label{equ:psi-sound-update-ua}\\
    \fora{\txid} (\txid, \txid_\cl^n) \in ( \SO_{\aexec'} \cup \WR_{\aexec'} \cup \WW_{\aexec'} )^{+} ; \VIS_{\aexec'} \implies \txid \in \Tx[\mkvs, \vi] \cup \txidset_\rd \cup \txidset'_\rd \label{equ:psi-sound-update-closure}\\
    I_1(\aexec',\cl) \cup I_2(\aexec',\cl) \subseteq \Tx[\mkvs_{\aexec'}, \vi'] \label{equ:psi-sound-inv} 
\end{gather}
\begin{itemize}
\item The invariant \( I_2 \) implies \cref{equ:psi-sound-update-so} as the same as \( \RYW \) in \cref{sec:sound-complete-ryw}.
\item Since \( \PSI \) also satisfies \( \UA \), the \cref{equ:si-sound-update-ww} can be proven as the same as \( \UA \) in \cref{sec:sound-complete-ua}.
\item \cref{equ:psi-sound-update-closure}.
    Note that \( (\txid, \txid_\cl^n) \in ( \SO_{\aexec'} \cup \WR_{\aexec'} \cup \WW_{\aexec'}); \VIS_{\aexec'} \implies (\txid, \txid_\cl^n) \in ( \SO_{\aexec} \cup \WR_{\aexec}  \cup \WW_{\aexec} ) ; \VIS_{\aexec'}\).
    Also, recall that \( \SO_\aexec = \SO_\mkvs \), \( \WR_\aexec = \WR_\mkvs \) and  \( \WW_\aexec = \WW_\mkvs \).
    Let \( \txidset'_\rd = \lfpTx[\mkvs,\vi,\SO_{\mkvs} \cup \WR_{\mkvs} \cup \WW_{\mkvs}] \). 
    This means that \( \aexec' = \extend[\aexec, \txid_\cl^n, \fp, \lfpTx[\mkvs, \vi, \SO_{\mkvs} \cup \WR_{\mkvs}] \cup \txidset_\rd ] \).
    Let assume \( \txid \toEDGE{\SO_{\mkvs} \cup \WR_{\mkvs} \cup \WW_{\mkvs}} \txid' \) and \( \txid' \in \lfpTx[\mkvs, \vi, \SO_{\mkvs} \cup \WR_{\mkvs}] \cup \txidset_\rd \).
    We have two possible cases:
    \begin{itemize}
        \item If \( \txid' \in \lfpTx[\mkvs, \vi, \SO_{\mkvs} \cup \WR_{\mkvs} \cup \WW_{\mkvs}] \), by  \cref{thm:view-vis-relation}, we know \( \txid \in \lfpTx[\mkvs, \vi, \SO_{\mkvs} \cup \WR_{\mkvs} \cup \WW_{\mkvs}] \).
        \item If \( \txid' \in \txidset_\rd \), there are two cases:
        \begin{itemize}
            \item \( \txid' \in  \left( \bigcup_{\Set{\txid_{\cl}^{n} \in \txidset_{\aexec} }[ n \in \Nat ]} \VIS_{\aexec}^{-1}(\txid^n_\cl) \right) \).
                Since \( \txid' \) is a read-only transaction, it means \( \txid \toEDGE{\SO_{\mkvs} \cup \WR_{\mkvs} } \txid' \).
                By the property of \( \aexec \) (before update) that \( \SO \cup \WR_\aexec \in \VIS_\aexec \), it is known that \( \txid \in \left( \bigcup_{\Set{\txid_{\cl}^{n} \in \txidset_{\aexec} }[ n \in \Nat ]} \VIS_{\aexec}^{-1}(\txid^n_\cl) \right) \), that is, \( \txid \in \Tx[\mkvs,\vi] \cup \txidset_\rd\).

            \item \( \txid' \in  \left( \bigcup_{\Set{\txid_{\cl}^{n} \in \txidset_{\aexec} }[ n \in \Nat ]} \SO_{\aexec}^{-1}(\txid^n_\cl) \right) \).
                Given that \( \txid' \) is a read only transaction, we know \( \txid \in (\SO \cup \WR_\aexec)^{-1} \left( \bigcup_{\Set{\txid_{\cl}^{n} \in \txidset_{\aexec} }[ n \in \Nat ]} \SO_{\aexec}^{-1}(\txid^n_\cl) \right) \).
                By the property of \( \aexec \) (before update) that \( \SO \cup \WR_\aexec \in \VIS_\aexec \),
                it follows:
                \begin{align*}
                    \txid & \in VIS_\aexec^{-1} \left( \bigcup_{\Set{\txid_{\cl}^{n} \in \txidset_{\aexec} }[ n \in \Nat ]} \SO_{\aexec}^{-1}(\txid^n_\cl) \right) \\
                          & = \left( \bigcup_{\Set{\txid_{\cl}^{n} \in \txidset_{\aexec} }[ n \in \Nat ]} \VIS_{\aexec}^{-1}(\txid^n_\cl) \right)  \\
                          & = \Tx[\mkvs,\vi] \cup \txidset_\rd
                \end{align*}
                
        \end{itemize}
    \end{itemize}
\item Finally the new abstract execution preserves the invariant \( I_1 \) and \( I_2 \) 
because  \( \CC \) satisfies \( \MW \) and \( \RYW \).
\end{itemize}

Given that \( \VIS_\aexec = (\WR_\aexec \cup \WW_\aexec \cup \SO_\aexec \cup R)^{+} \),
we know \( \VIS_\aexec ; \SO_\aexec \subseteq \VIS_\aexec \).
First the completeness follows \( \MR \) in \cref{sec:sound-complete-mr}, \( \RYW \) in \cref{sec:sound-complete-ryw} and  \( \UA \) in \cref{sec:sound-complete-ua}.
Similarly, by \cref{lem:aexec-spec-cc},

An abstract execution \( \aexec \) satisfies snapshot isolation (\(\SI\)), 
if it satisfies
\( \{\lambda \aexec. \AR[\aexec] ; \VIS[\aexec], \lambda \aexec \ldotp \SO, \allowbreak \lambda \aexec. \WW[\aexec] \}) \),
which is intersection of \( \CP \) and \( \UA \) on abstract executions \citep{SIanalysis}.
\citet{SIanalysis} also proposed the minimum visibility relation that gives rise of the following equivalent definition
\[
    \visaxioms[\SI] \FuncDef
    \Set{\lambda \aexec. \left( (\WR[\aexec]  \cup \SO \cup \WW[\aexec] ) ; \Refl(\RW[\aexec]) \right) ; \VIS[\aexec]
            ,\lambda \aexec \ldotp \SO, \lambda \aexec \ldotp \WW[\aexec] }  .
\]

The execution test \( \et[\CP] \) is sound with respect to the axiomatic definition \( \visaxioms[\SI] \)
We pick the invariant \( \aexecinv[\CP] = \aexecinv[\RYW]\).
\SOUNDLET{\CP}{ \txidsetrd \supseteq
\begin{multlined}[t]
\left( \bigcup_{\Set{\txid[\cl](\idx) | \txid[\cl](\idx) \in \aexec}} 
\VISInv[\aexec](\txid[\cl](\idx)) \cup \Refl((\Inv(\SO)))(\txid[\cl](\idx)) \right) 
\setminus \Set{\txid' | \Forall{l | \key | \val } (l,\key,\val) \in \aexec(\txid') \implies l = \opR } .
\end{multlined} }
Assume 
\[ 
\txidsetrd' = 
\begin{multlined}[t]
\left( \bigcup_{\Set{\txid[\cl](\idx) | \txid[\cl](\idx) \in \aexec}} 
\VISInv[\aexec](\txid[\cl](\idx)) \cup \Refl((\Inv(\SO)))(\txid[\cl](\idx)) \right) 
\setminus \Set{\txid' | \Forall{l | \key | \val } (l,\key,\val) \in \aexec(\txid') \implies l = \opR } .
\end{multlined} 
\]
and \( \txidsetrd'' = \txidsetrd \setminus \txidsetrd' \).
By the definition of soundness, we prove the following result:
\begin{Formulae}
& \begin{Formula}
    \Inv(\SO)(\txid) \subseteq \txidset \cup \txidsetrd' 
    \label{equ:si-sound-update-so}
\end{Formula}
\\ & \begin{Formula}
    \Inv(\WW)(\txid) \subseteq \txidset 
    \label{equ:si-sound-update-ua}
\end{Formula}
\\ & \begin{Formula}
    \Inv(\left( (\WR[\aexec] \cup \SO \cup \WW[\aexec] ) ; \Refl(\RW[\aexec]) \right)) (\txid) 
            \subseteq \txidset \cup \txidsetrd' \cup \txidsetrd''
     \label{equ:si-sound-update-closure}
\end{Formula}
\\ & \begin{Formula}
    \aexecinv[\PSI](\aexec',\cl) \subseteq \VisTrans(\XToK(\aexec'),\vi')
    \label{equ:si-inv-preserve}
\end{Formula}
\end{Formulae}
\Cref{equ:si-sound-update-so,equ:si-sound-update-ua} 
can be proven in the same way as in \cref{sec:sound-complete-mr,sec:sound-complete-ua} respectively.
We now prove \cref{equ:si-sound-update-closure}.
Initially we take \( \txidsetrd'' \) to be an empty set.
Note that \(\VISInv[\aexec'](\txid) = \txidset \cup \txidsetrd' \cup \txidsetrd'' \).
By \cref{thm:view-vis-relation,equ:view-close-to-aexec}, there exists \( \txidsetrd'' \) such that
\( \txidset \cup \txidsetrd'' \) is closed under \( \left( (\WR[\aexec] \cup \SO \cup \WW[\aexec] ) ; \Refl(\RW[\aexec]) \right)\).
Now consider a transaction \( \txidrd \in \txidsetrd' \) and
assume a transaction \( \txid' \) such that \( \ToEdge{ \txid' | (\WR[\aexec] \cup \SO \cup \WW[\aexec] ) ; \Refl(\RW[\aexec]) -> \txidrd } \).
Since \( \txidrd \) is a read-only transaction, thus
\( \ToEdge{ \txid' |  (\SO \cup \WR[\aexec] )  -> \txidrd } \)
and the rest proof is exactly the same as as in \cref{sec:sound-complete-cc}.
Last, \cref{equ:si-inv-preserve} can be proven in the same way as in \cref{sec:sound-complete-mr,sec:sound-complete-ryw}.

The execution test $\et[\SI]$ is complete with respect to the axiomatic definition \( \visaxioms[\SI] \).
By \citet{SIanalysis}, it suffices to prove completeness with respect to the following definition,
\[
\Set{\lambda \aexec \ldotp \AR[\aexec] ; \VIS[\aexec], \lambda \aexec \ldotp \SO , \lambda \aexec \ldotp \WW[\aexec] } .
\]
\COMPLETELET{\SI}
By the definition of \( \et[\SI]\), we prove \( \CanCommit[\SI] \), \( \ViewShift[\MR]\) and \( \ViewShift[\RYW]\) respectively.
Recall that \( \CanCommit[\SI] = \PreClosed(\kvs,\vi,\rel[\UA] \cup \left( (\WR[\aexec] \cup \SO \cup \WW[\aexec] ) ; \Refl(\RW[\aexec]) \right)) \).
It is easy to see that 
\[
\begin{multlined}
\PreClosed(\kvs,\vi,\rel[\UA] \cup \WR[\kvs] \cup \SO \cup \WW[\kvs]) \iff {}
    \\ \PreClosed(\kvs,\vi,\rel[\UA]) 
    \land \PreClosed(\kvs,\vi,\left( (\WR[\aexec] \cup \SO \cup \WW[\aexec] ) ; \Refl(\RW[\aexec]) \right)) . 
\end{multlined}
\]
The predicate \( \PreClosed(\kvs,\vi,\rel[\UA]) \) can be proven in the same way as in \cref{sec:sound-complete-ua}
Because
\begin{align*}
\left( (\WR[\aexec] \cup \SO \cup \WW[\aexec]) ; \Refl(\RW[\aexec])  \right) ; \VIS[\aexec] 
        & \subseteq \left( \VIS[\aexec] ; \Refl(\RW[\aexec]) \right) ;  \VIS[\aexec]
        & \subseteq  \AR[\aexec] ; \VIS[\aexec] \subseteq \VIS[\aexec] .
\end{align*}
Then \( \PreClosed(\kvs,\vi,\left( (\WR[\aexec] \cup \SO \cup \WW[\aexec] ) ; \Refl(\RW[\aexec]) \right)) \)
can be derived from \cref{thm:view-vis-relation,equ:aexec-close-to-view}.
The predicate \( \ViewShift[\RYW] \) can be proven in the same way as in \cref{sec:sound-complete-ryw}.
Since \( \VIS[\aexec] ; \SO \subseteq \AR[\aexec] ; \VIS[\aexec] \subseteq \VIS[\aexec] \)
\( \ViewShift[\MR] \) can be proven in the same way as in \cref{sec:sound-complete-mr}.

\paragraph{Strict serialisability (\(\SER\))}  
This model is the strongest consistency model
in any framework that abstracts from aborted transactions, 
requiring that transactions execute in a total sequential order.
The \(\CanCommit[\SER]\) thus allows a client to commit a transaction only 
when the client view on the kv-store is complete
in that the view is closed with respect to \(\WWInv[\kvs]\). 
This requirement prevents the kv-store in  \cref{fig:ser-disallowed}.
Without loss of generality, suppose that \(\txid\) commits before \(\txid'\),
then the client committing \(\txid'\) must see the version of \(\key_1\) written by \(\txid\), 
and thus cannot read the outdated value \(\val_0\) for \(\key_1\). 
This example, known as \emph{write skew anomaly}, 
is allowed by all other execution tests in \cref{fig:execution-tests}.

\begin{figure}
\centering
\begin{tikzpicture}%
\KVMapping{x}{\key_1}{
    /\val_0/\txid_0/\Set{\boldsymbol{\txid'}}
    , /\val_1/\txid/\emptyset
};
\KVMapping[x]{y}{\key_2}{
    /\val_0/\txid_0/\Set{\txid}
    , /\val_2/\boldsymbol{\txid'}/\emptyset
};
\end{tikzpicture}%

\hrulefill

\caption{Write skew anomaly, disallowed by \(\SER\)}
\label{fig:ser-disallowed}
\end{figure}%


\subsection{Soundness and Completeness Constructor}
\label{sec:kv2aexec-sound-complete}

We now show how all the results illustrated so far 
can be put together to show that the kv-store operational semantics 
is sound and complete with respect to abstract execution operational semantics.

\subsubsection{Soundness}
Recall that in the abstract execution operational semantics,
a client \( \cl \) loses information of the visible transactions immediately after it commits a transaction.
Yet such information is indirectly presented when the next transaction from the same client is committed.
To define the soundness judgement (\cref{def:et_sound}), we introduce a notation of \emph{invariant} ({def:invariant-for-clients})
to encore constraints on the visible transactions after each commit.

\ac{The idea behind client-based invariant being that $I(\aexec, \cl)$ represents 
the minimal set of transactions that $\cl$ must see in $\aexec$, before 
updating the view and performing a transaction. Such a set of transaction 
roughly correspond to the view of the client before performing a 
sequence of \emph{update view+execute transaction} operations, 
or equivalently from the view obtained after the execution of the 
last transaction from that client.}

\begin{definition}[Invariant for clients]
\label{def:invariant-for-clients}
A \emph{client-based invariant condition}, or simply \emph{invariant}, is a 
function $I : \Aexecs \times \Clients \rightarrow \powerset{\TxID}$ 
such that for any $\cl$ we have that $I(\aexec, \cl) \subseteq \T_{\aexec}$, and 
for any  $\cl'$ such that $\cl' \neq \cl$ we have that 
$I(\extend(\aexec, \txid_{\cl'}^{\cdot}, \stub, \stub), \cl) = I(\aexec, \cl)$.
\end{definition}



\begin{definition}[Soundness judgement]
\label{def:et_sound}
An execution test $\ET$ is sound with respect to an axiomatic 
definition $(\RP_{\LWW}, \Ax)$ if and only if
there exists an invariant condition $I$ such that 
if assuming that
\begin{itemize}
    \item a client \( \cl \) having an initial view \( \vi \), 
        commits a transaction \( \txid \) with a fingerprint \( \fp \) and updates the view to \( \vi' \), 
        which is allowed by \( \ET \) \ie $\ET \vdash (\mkvs, \vi) \csat \fp: (\mkvs',\vi')$ where \( \mkvs' = \updKV{\mkvs, \vi ,\fp, \txid}\),
    \item a $\aexec$ such that $\hh_{\aexec} = \mkvs$ and $I(\aexec, \cl) \subseteq \Tx(\mkvs, \vi)$,
\end{itemize}
then there exist a set of read-only transactions $\T_{\rd}$ such that 
\begin{itemize}
\item the view \( \vi \) satisfies \( \Ax \), \ie $\forall \A \in \Ax. \Setcon{\txid' }{ (\txid', \txid) \in \A(\aexec')} \subseteq \Tx(\mkvs, \vi) \cup \T_{\rd}$, 
\item the invariant is preserved, \ie $I(\aexec', \cl) \subseteq \Tx(\mkvs', \vi')$ for some \( \aexec' \) that \( \mkvs' = \hh_{\aexec'}\)
\end{itemize}
\end{definition}

\begin{theorem}[Soundness]
\label{thm:et_soundness}
If $\ET$ is sound with respect to $(\RP_{\LWW}, \Ax)$, then 
\[
    \CMs(\ET) \subseteq \Setcon{ \mkvs }{ \exists \aexec \in \CMa(\RP_{\LWW}, \Ax)).\;\hh_{\aexec} = \mkvs}
\]
\end{theorem}
\begin{proof}
Let $\ET$ be an execution test that is sound with respect to an 
axiomatic definition $(\RP_{\LWW}, \Ax)$. Let $I$ be 
the invariant that satisfies \cref{def:et_sound}. 
Let consider an $\ET$-trace $\tr$.
Because of \cref{prop:et.normalform}, we can assume that $\tr$ is in normal form. 
Without lose generality, we can also assume that the trace does not have transitions labelled as $(\stub, \emptyset)$.
Thus we have that the following trace \( \tr \):
\[
\begin{rclarray}
\tr & = & (\hh_{0}, \viewFun_{0}) \xrightarrowtriangle{(\cl_{0}, \varepsilon)} (\hh_{0}, \viewFun_{0}') 
\xrightarrowtriangle{(\cl_{0}, \fp_{0})} 
(\hh_1, \viewFun_{1}) \xrightarrowtriangle{(\cl_1, \varepsilon)}  \cdots
\xrightarrowtriangle{(\cl_{n-1}, \fp_{n-1})} (\hh_{n}, \viewFun_{n}).
\end{rclarray}
\]
For any $i : 0 \leq i \leq n$, let $\tr_{i}$ be the prefix of $\tr$ that 
contains only the first $2i$ transitions. 
Clearly $\tr_{i}$ is a valid $\ET$-trace, and it is also a $\ET_{\top}$-trace. 
By \cref{prop:kvtrace2aexec}, 
any abstract execution $\aexec_{i} \in \aeset(\tr_{i})$ satisfies the last write wins policy. 
We show by induction on $i$ that we can always find 
an abstract execution $\aexec_{i} \in \aeset(\tr_{i})$ such that $\aexec_i \models \Ax$ and $I(\aexec_{i}, \cl) \subseteq \T^{i}_{\cl}$
for any client $\cl$ and set of transactions 
$\T^{i}_{\cl} = \Tx(\aexec_{i}, \viewFun_{i}(\cl)) \cup \T^{i}_\rd$, 
and read-only transactions $\T_\rd^{i}$ in $\aexec_{i}$.
If so, because $\aexec_{i}$ satisfies the last write wins policy,
then it must be the case that $\aexec_{i} \models (\RP_{\LWW}, \Ax)$. 
Then by choosing $i = n$, we will obtain that $\aexec_{n} \models (\RP_{\LWW}, \Ax)$. 
Last, by \cref{prop:kvtrace2aexec}, $\hh_{\aexec_{n}} = \hh_{n}$, and there is nothing left to prove.
Now let prove such $\aexec_{i} \in \aeset(\tr_{i})$ always exists.

\caseB{$i = 0$} 
Let $\aexec_{0}$ be the only abstract execution included in $\aeset(\tr_{0})$, 
that is $\aexec_{0} = ([], \emptyset, \emptyset)$. 
For any $\A \in \Ax$, it must be the case that 
$\A(\aexec_{0}) \subseteq \T_{\aexec_{0}} = \emptyset$, 
hence the inequation $\A(\aexec_{0}) \subseteq \VIS_{\aexec_{0}}$ is trivially satisfies.
Furthermore, for the client invariant $I$ we also require that $I(\aexec_{0}, \stub) \subseteq \T_{\aexec_{0}} = \emptyset$; 
for any client $\cl$ we can choose $\T_{\cl}^{0} = \Tx(\viewFun_{0}(\cl)) \cup \emptyset = \emptyset$. 
Therefore $I(\aexec_{0}, \cl) = \emptyset \subseteq \emptyset = \T_{\cl}^{0}$.

\caseI{$i' = i + 1$ where $i < n$}
By the inductive hypothesis, there exists an abstract execution $\aexec_i$ such that  
\begin{itemize}
\item $\aexec_{i} \models \A$ for all $\A \in \Ax$, and 
\item $I(\aexec, \cl) \subseteq \T_{\cl}^{i}$ for any client $\cl$ and set of transactions $\T_{\cl}^{i} = \Tx(\hh_{i}, \viewFun_{i}(\cl))$.
\end{itemize}

We have two transitions to check, the view shift and committing a transaction.
\begin{itemize}
\item the view shift transition $(\hh_{i}, \viewFun_{i}) \xrightarrowtriangle{(\cl_{i}, \varepsilon)} (\hh_{i}, \viewFun'_{i})$. 
By definition, it must be the case that $\viewFun'_{i} = \viewFun_{i}\rmto{\cl}{\vi'_{i}}$ 
for some $\vi'_{i}$ such that $\viewFun_{i}(\cl) \viewleq \vi'_{i}$.
Let $(\T_{\cl}^{i})' = \Tx(\hh_{i}, \vi'_{i})$; then we have 
\(
\T_{\cl}^{i} = \Tx(\hh_{i}, \viewFun_{i}(\cl)) \subseteq \Tx(\hh_{i}, \vi'_{i}) = (\T_{\cl}^{i})'
\)
As a consequence, $I(\aexec, \cl) \subseteq \T_{\cl}^{i} \subseteq (\T_{\cl}^{i})'$.

\item the commit transaction transition $(\hh_{i}, \viewFun_{i}') \xrightarrowtriangle{(\cl_{i}, \fp_{i})}_{\ET} 
(\hh_{i+1}, \viewFun_{i+1})$.
A necessary condition for this transition 
to appear in $\tr$ is that $\ET \vdash (\hh_{i}, \viewFun(\cl)) \triangleright \fp_{i}: (\mkvs_{i+1},\viewFun_{i+1}(\cl))$. 
Because $I$ is the invariant to derive that $\ET$ is sound with respect to $\Ax$, 
and because $I(\aexec_{i}, \cl_{i}) \subseteq (\T^{i}_{\cl})'$, 
then by \cref{def:et_sound} we have the following:
\begin{itemize}
\item there exists a set of read-only transactions $\T_\rd$ 
    such that 
    \[
        \Setcon{\txid' }{ (\txid', \txid_{(\cl, i)}) \in \A(\aexec_{i+1})} \subseteq (\T^{i}_{\cl})' \cup \T_\rd
    \]
where 
$\txid_{(\cl, i)} \in \nextTxid(\hh_{i}, \cl)$
and $\aexec_{i+1} = \extend(\aexec_{i}, \txid_{(\cl, i)}, (\T^{i}_{\cl})' \cup \T_\rd, \fp_{i})$,
\item  $I(\aexec_{i+1}, \cl) \subseteq \Tx(\hh_{i+1}, \viewFun_{i+1}(\cl))$.
\end{itemize} 
Because $\aexec_{i} \in \aeset(\tr_{i})$, by definition of $\aeset(\stub)$ we have that 
$\aexec_{i+1} \in \aeset(\tr)$ (under the assumption that $\fp_{i} \neq \emptyset$), 
and because $\lastConf(\tr_{i+1}) = (\hh_{i+1}, \stub)$, then $\hh_{\aexec_{i+1}} = \hh_{i+1}$. 

Now we need to check if \( \aexec_{i+1} \) satisfies \( \Ax\) and the invariant \( I \) is preserved.
\begin{itemize}
\item $\A(\aexec_{i+1}) \subseteq \VIS_{\aexec}^{i+1}$ for any $\A \in \Ax$.
Fix $\A \in \Ax$ and $(\txid', \txid) \in \A(\aexec_{i+1})$. 
Because $\aexec_{i+1} = \extend(\aexec_{i}, \txid_{(\cl, i)}, (\T_{\cl}^{i})' \cup \T_\rd, \fp_{i}$), 
we distinguish between two cases.
\begin{itemize}
\item If $\txid = \txid_{(\cl, i)}$, then it must be the case that $\txid' \in (\T^{i}_{\cl})' \cup \T_\rd$, 
and by definition of $\extend(\stub)$ we have that $(\txid' ,\txid_{(\cl, i)}) \in \VIS_{\aexec_{i+1}}$. 
\item If $\txid \neq \txid_{(\cl, i)}$, then we have that $\txid, \txid' \in \T_{\aexec_{i}}$. 
Because $\aexec_{i}$ and $\aexec_{i+1}$ agree on $\T_{\aexec_{i}}$, then $(\txid', \txid) \in \A(\aexec_{i})$.
Because $\aexec_{i} \models \A$, then $(\txid', \txid) \in \VIS_{\aexec_{i}}$. 
By definition of $\extend$, it follows that $(\txid', \txid) \in \VIS_{\aexec_{i+1}}$.
\end{itemize}

\item Finally, we show the invariant is preserved.
Fix a client $\cl'$. 
\begin{itemize}
\item If $\cl' = \cl$, then we have already proved that 
$I(\aexec_{i+1}, \cl) \subseteq \T_{\cl}^{i+1}$. 
\item if $\cl' \neq \cl$, then note that $\viewFun_{i}(\cl') = \viewFun'_{i}(\cl') = \viewFun_{i+1}(\cl')$, 
and in particular $(\T^{\cl'}_{i})' = \Tx(\aexec_{i}, \viewFun'_{i}(\cl')) = \Tx(\aexec_{i+1}, \viewFun_{i+1}(\cl')) =  \T_{\cl'}^{i+1}$.
By the inductive hypothesis we know that $I(\aexec_{i}, \cl) \subseteq \T_{\cl'}^{i}$, 
and by the definition of invariant, we have $I(\aexec_{i+1}, \cl) \subseteq \T_{\cl'}^{i} = \T_{\cl'}^{i+1}$. 
\end{itemize}
\end{itemize}
\end{itemize}
\end{proof}

\begin{corollary}
\label{cor:et-soundness}
If $\ET$ is sound with respect to $(\RP_{\LWW}, \Ax)$, then 
for any program $\prog$, $\interpr{\prog}_{\ET} \subseteq \Setcon{ \hh_{\aexec} }{ \aexec \in \interpr{\prog}_{(\RP_{\LWW}, \Ax)} }$.
\end{corollary}
\begin{proof}
\[
\begin{rclarray}
\interpr{\prog}_{\ET} 
& \stackrel{\cref{thm:consistency-intersect-permissive}}{=} & 
\interpr{\prog}_{\ET_\top} \cap \CMs(\ET) \\
& \stackrel{\cref{cor:kvtrace2aexec}}{=} & 
\Setcon{\hh_{\aexec} }{ \aexec \text{ satisfies } \RP_{\LWW}} \cap \CMs(\ET) \\
& \stackrel{\cref{thm:et_soundness}}{\subseteq} & 
\Setcon{\hh_{\aexec} }{ \aexec \text{ satisfies } \RP_{\LWW} \land \aexec \in \CMa(\RP_{\LWW}, \Ax) } \\
& \stackrel{\cref{thm:consistency-intersect-anarchic}}{=} &
\Setcon{ \hh_{\aexec} }{ \aexec \in \interpr{\prog}_{(\RP_{\LWW}, \Ax)} }
\end{rclarray}
\]
\end{proof}

\subsubsection{Completeness}
The Completeness judgement is in \cref{def:et_complete}.
Given a transaction \( \txid_i \) from client \( \cl \), it converts the visible transactions \( \VIS_{\aexec}^{-1}(\txid_{i}) \) into view  and such view should satisfy the \( \ET \).
Note that \( \aexec \) does not contain precise information about final view after update,
yet the visible transactions of the immediate next transaction from the same client \( \cl \) include those information.

\begin{definition}
\label{def:et_complete}
An execution test $\ET$ is \emph{complete} with respect 
to an axiomatic definition $(\RP_{\LWW}, \Ax)$ if, for any abstract execution $\aexec \in \CMa(\RP_{\LWW}, \Ax)$ 
and index \( i : 1 \leq i < \abs{\T_{\aexec}}\) such that \( \txid_{i} \toEdge{\AR_{\aexec}} \txid_{i+1} \), there exist an initial view $\vi_{i}$ and a final view $\vi_{i}'$ where 
\begin{itemize}
\item $\vi_{i} = \getView(\aexec, \VIS_{\aexec}^{-1}(\txid_{i}))$, 
\item let $\txid_{i} = \txid_{\cl}^{n}$ for some $\cl, n$; 
    \begin{itemize}
        \item if the transaction $\txid_{i}' = \min_{\PO_{\aexec}}\Setcon{\txid' }{ \txid_i \xrightarrow{\PO_{\aexec}} \txid'}$ is defined, then $\vi' = \getView(\aexec, \T_{i})$ where $\T_{i} \subseteq (\AR_{\aexec}^{-1})?(\txid_{i}) \cap \VIS_{\aexec}^{-1}(\txid_{i}'))$; 
        \item otherwise $\vi' = \getView(\aexec, \T_{i})$ where $\T_{i} \subseteq (\AR_{\aexec}^{-1})?(\txid_{i})$, 
    \end{itemize}
\item $\ET \vdash (\hh_{\cut(\aexec, i-1)}, \vi_{i}) \csat \TtoOp{T}_{\aexec}(\txid_{i}) : (\hh_{\cut(\aexec, i)},\vi_{i}')$.
\end{itemize}
\end{definition}

\begin{theorem}
\label{thm:et_complete}
Let $\ET$ be an execution test that is complete with respect to an axiomatic definition $(\RP_{\LWW}, \Ax)$. 
Then $\CMa(\RP_{\LWW}, \Ax) \subseteq \CMs(\ET)$.
\end{theorem}
\begin{proof}
Fix an abstract execution $\aexec \in \CMa(\RP_{\LWW}, \Ax)$. 
For any \(i : 1 \leq i < \abs{\T_\aexec} \), suppose that \( \txid_i \) that is the i-\emph{th} transaction follows the arbitrary order, \ie $\txid_{i} \xrightarrow{\AR_{\aexec}} \txid_{i+1}$ 
and let $\cl_{i}$ be the client of the i-\emph{th} step, \ie $\txid_{i} = \txid_{\cl_{i}}^{\stub}$.
Because $\ET$ is complete with respect to $(\RP_{\LWW}, \Ax)$, 
for any step $i$ we can find an initial views $\vi_i$,and a final view $\vi'_{i}$ such that 
\begin{itemize}
\item $\vi_i = \getView(\aexec, \VIS^{-1}_{\aexec}(\txid_{i}))$, 
\item there exists a set of transactions $\T_{i}$ such that $\getView(\aexec, \T_{i}) = \vi'_{i}$, and 
either $\min_{\PO_{\aexec}}\Setcon{\txid' }{ \txid_{i} \xrightarrow{\PO_{\aexec}} \txid'}$ is 
is defined and $\T_{i} \subseteq (\AR_{\aexec}^{-1})?(\txid_{i}) \cap \VIS^{-1}_{\aexec}(\txid')$, 
or $\T_{i} \subseteq (\AR_{\aexec}^{-1}?(\txid_{i})$, 
\item $\ET \vdash (\hh_{\cut(\aexec, i-1)}, \vi_i) \triangleright \TtoOp{T}_{\aexec}(\txid_{i}): (\hh_{\cut(\aexec, i)}, \vi'_{i})$.
\end{itemize}
Given above, let $\hh_{i} = \cut(\aexec, i)$ and $\fp_{i} = \TtoOp{T}_{\aexec}(\txid_{i})$. Define the views for clients as 
\[
\viewFun_{0} = \lambda \cl \in \Setcon{\cl' }{ \exsts{ \txid \in \T_{\aexec} } \txid = \txid_{\cl'}} \ldotp \lambda \key \ldotp \Set{0}
\quad \viewFun'_{i-1} = \viewFun_{i}\rmto{\cl_{i}}{ \vi_i}
\quad \viewFun_{i} = \viewFun'_{i-1}\rmto{\cl_{i} }{\vi'_{i}}
\]
and the ke-stores as
\[
\hh_{0} = \lambda \key.(\val_{0}, \txid_{0}, \emptyset)
\quad \hh_{i} = \updateKV(\hh_{i-1}, \vi_i, \fp_{i}, \txid_{i})
\]
Now by \cref{prop:aexec2kvtrace} we have that the following sequence of $\ET_{\top}$-reductions 
\[
\begin{array}{l}
(\hh_{0}, \viewFun_{0}) \xrightarrowtriangle{(\cl_{1}, \varepsilon)}_{\ET_{\top}} (\hh_{0}, \viewFun'_{0}) 
\xrightarrowtriangle{(\cl_{1}, \fp_{1})}_{\ET_{\top}} (\hh_{1}, \viewFun_{1}) 
\xrightarrowtriangle{(\cl_{2}, \varepsilon)}_{\ET_{\top}} 
\cdots \xrightarrowtriangle{(\cl_{n}, \fp_{n})}_{\ET_{\top}} (\hh_{n}, \viewFun_{n})
\end{array}
\]
Note that $\hh_{i} = \hh_{\cut(\aexec, i)}$. 
Because $\ET \vdash ( \hh_{\cut(\aexec,i-1)}, \vi_i ) \csat \fp_{i} : (\hh_{i}, \vi'_{i})$, 
or equivalently $\ET \vdash ( \hh_{\cut(\aexec, i-1)}, \viewFun'_{i-1}(\cl_{i}) ) \csat \fp_{i} : ( \hh_{\cut(\aexec, i-1)}, \viewFun_{i}(\cl_{i}) )$, therefore 
\[
\begin{array}{l}
(\hh_{0}, \viewFun_{0}) \xrightarrowtriangle{(\cl_{1}, \varepsilon)}_{\ET} (\hh_{0}, \viewFun'_{0}) 
\xrightarrowtriangle{(\cl_{1}, \fp_{1})}_{\ET} (\hh_{1}, \viewFun_{1})
\xrightarrowtriangle{(\cl_{2}, \varepsilon)}_{\ET} 
\cdots \xrightarrowtriangle{(\cl_{n}, \fp_{n})}_{\ET} (\hh_{n}, \viewFun_{n})
\end{array}
\]
It follows that $\hh_{n} \in \CMs(\ET)$ then $\hh_{n} = \hh_{\cut(\aexec, n)} = \hh_{\aexec}$, and there is nothing left to prove.
\end{proof}

\begin{corollary}
\label{cor:et-completeness}
If $\ET$ is complete with respect to $(\RP_{\LWW}, \Ax)$, then 
for any program $\prog$, $\Setcon{ \hh_{\aexec} }{ \aexec \in \interpr{\prog}_{(\RP_{\LWW}, \Ax)} } \subseteq \interpr{\prog}_{\ET}$.
\end{corollary}
\begin{proof}
\[
\begin{rclarray}
    \Setcon{ \hh_{\aexec} }{ \aexec \in \interpr{\prog}_{(\RP_{\LWW}, \Ax)} }
& \stackrel{\cref{thm:consistency-intersect-anarchic}}{=} &
\Setcon{\hh_{\aexec} }{ \aexec \text{ satisfies } \RP_{\LWW} \land \aexec \in \CMa(\RP_{\LWW}, \Ax) } \\
& \stackrel{\cref{thm:et_complete}}{\subseteq} & 
\Set{\hh_{\aexec} }[ \aexec \text{ satisfies } \RP_{\LWW}] \cap \CMs(\ET) \\
& \stackrel{\cref{cor:kvtrace2aexec}}{=} & 
\interpr{\prog}_{\ET_\top} \cap \CMs(\ET) \\
& \stackrel{\cref{thm:consistency-intersect-permissive}}{=} & 
\interpr{\prog}_{\ET} 
\end{rclarray}
\]
\end{proof}


\section{The Soundness and Completeness of Execution Tests}

We now show using \cref{def:et_sound,def:et_complete} to prove the soundness and completeness of execution tests with respect to axiomatic specification.
It is sufficient to match these two definition, 
then by \cref{cor:et-soundness,cor:et-completeness} we have \( \CMs(\ET) = \Setcon{\mkvs_\aexec}{\aexec \in \CMa(\RP_\LWW,\Ax)} \).

\label{sec:kv-sound-complete-proof}
\label{sec:spec-proof}

\emph{Monotonic read} (\( \MR \)) \citep{session-guarantee,repldatatypes} states that after committing a transaction, 
a client cannot lose information in that 
it can only see increasingly more versions from a kv-store.
This prevents, for example, the kv-store in \cref{fig:mr-disallowed},
since client \(\cl\) first reads the latest version of \(\key\) in \(\txid_{\cl}^{1}\), 
and then reads the older, initial version of \(\key\) in \(\txid_{\cl}^{2}\).  
As such, the \(\ViewShift[\MR]\) predicate in \cref{fig:execution-tests} 
ensures that clients can only extend their views,
that is, \( \vi \vileq \vi' \) for views \( \vi, \vi'\) before and after committing.
When this is the case, clients can then \emph{always} commit their transactions,
and thus \(\CanCommit[\MR]\) is simply defined as \(\true\). 

\subsection{Monotonic Write \( \MW \)}
\label{sec:sound-complete-mw}

The execution test $\ET_\MW$ is sound with respect to the axiomatic specification 
$(\RP_{\LWW}, \Set{\lambda \aexec. \PO_{\aexec} ; \VIS_{\aexec} })$.
We pick the invariant as empty set given the fact of no constraint on the view after update:
\[ 
    I( \aexec, \cl ) = \emptyset 
\]
Assume a key-value store $\hh$, an initial and a final view $\vi, \vi'$  a fingerprint $\opset$ 
such that $\ET_{\MW} \vdash (\hh, \vi) \csat \opset: (\hh',\vi')$. 
Also choose an arbitrary $\cl$, a transaction identifier $\txid \in \nextTxId(\hh, \cl)$, 
and an abstract execution $\aexec$ such that $\hh_{\aexec} = \hh$ and 
\( I(\aexec, \cl) =  \emptyset \subseteq \Tx(\hh, \vi) \).
Let \( \aexec' = \extend(\aexec, \txid, \Tx(\mkvs, \vi) \cup \T_\rd, \f ) \).
Note that since the invariant  is empty set, it remains to prove that there exists a set of read-only transactions \( \T_\rd \) such that:
\[
    \begin{array}{@{}l@{}}
        \fora{ \txid' }  (\txid' ,\txid)  \in \PO_{\aexec'} ; \VIS_{\aexec'}
        \implies \txid' \in \Tx(\mkvs, \vi) \cup \T_\rd
    \end{array}
\]
Initially we take \( \T_\rd = \emptyset \), 
and by closing the \( \Tx(\mkvs, \vi) \) with respect to the relation \( \PO_{\aexec'} ; \VIS_{\aexec'} \),
we will add more read-only transactions into the set \( \T_\rd\).
Suppose \( (\txid' ,\txid)  \in \PO_{\aexec'} ; \VIS_{\aexec'} \), 
that is, \( \txid' \toEdge{\SO_{\aexec'}} \txid'' \toEdge{\VIS_{\aexec'}} \txid \).
We perform a case analysis on if \( \txid'' \) has write:
\begin{itemize}
\item If the transaction \( \txid'' \) writes to a key.
For the new abstract execution \( \aexec' \), the visible transactions for \( \txid \) must come from \( \Tx(\mkvs, \vi) \cup \T_\rd \).
It means \( \txid'' \in \Tx(\mkvs, \vi) \cup \T_\rd  \).
Then given that \( \txid'' \) is not a read-only transaction, we have \( \txid'' \in \Tx(\mkvs, \vi) \).
Now there are two cases:
\begin{itemize}
    \item if \( \txid' \) is a read-only transaction, we include \( \txid' \in \T_{\rd} \).
    \item if \( \txid' \) has at least one write, it is easy to see \( \txid' \in \Tx(\mkvs, \vi) \) since \( j \in \vi(\ke) \wedge \WTx(\hh(\ke', i)) \xrightarrow{\PO?} \WTx(\hh(\ke, j)) \implies i \in \vi(\ke') \).
\end{itemize}
\item If the transaction \( \txid'' \in \T_\rd \) is a read-only transaction, 
since \( \T_\rd \) is initial empty, there must exist a later transaction \( \txid''' \) from the same client that writes to a key,
and such transaction \( \txid''' \) is included in \( \Tx(\mkvs, \vi) \):
\[
    \txid' \toEdge{\SO_{\aexec'}} \txid'' 
    \toEdge{\SO_{\aexec'}} \txid''' \toEdge{\VIS_{\aexec'}} \txid 
    \land \txid''' \in \Tx(\mkvs,\vi)
\]
Since \( \SO \) is transitive, 
therefore \( \txid' \toEdge{\SO_{\aexec'}} \txid''' \toEdge{\VIS_{\aexec'}} \txid \),
which we have already proven \( \txid' \in \Tx(\mkvs, \vi) \) or we will include \( \txid' \) in \( \T_\rd \).
Since there are finite transactions from a client in a trace, there must exist a \( \T_\rd \) in the end.
\end{itemize}


The execution test $\ET_{\MW}$ is complete with respect to 
the axiomatic specification $(\RP_{\LWW}, \Set{\lambda \aexec.(\PO_{\aexec} ; \VIS_{\aexec})})$. 
Let $\aexec$ be an abstract execution that satisfies the specification
$\CMa(\RP_{\LWW}, \Set{\lambda \aexec.(\PO_{\aexec} ; \VIS_{\aexec})})$, 
and consider a transaction $\txid \in \T_{\aexec}$. 
Assume i-\emph{th} transaction \( \txid_i \) in the arbitrary order,
and let view \( \vi_{i} = \getView(\aexec, \VIS^{-1}_{\aexec}(\txid_{i}) ) \).
We also pick any final view such that \( \vi'_{i} \subseteq \getView(\aexec, (\AR^{-1}_{\aexec})?(\txid_{i}) ) \).
It suffices to prove \( \ET_\MW \vdash (\hh_{\cut(\aexec, i-1)}, \vi_i ) \csat  \TtoOp{T}_{\aexec}(\txid_{i}) : (\hh_{\cut(\aexec, i-1)}, \vi'_{i}) \).
It means to prove the follows:
\begin{equation}
\label{equ:mw-complete}
\begin{array}{@{}l@{}}
    \fora{j,m,\ke, \ke' } j \in \vi(\ke)  
    \wedge \WTx(\hh_{\cut(\aexec, i-1)}(\ke', m)) \xrightarrow{\PO?} \WTx(\hh_{\cut(\aexec, i-1)}(\ke, j))  
    \implies m \in \vi(\ke')
\end{array}
\end{equation}
Assume \( j \) and \( \ke' \) such that \( j \in \vi(\ke')\), which means \( \WTx(\hh_{\cut(\aexec, i-1)}(\ke', j)) \in \VIS^{-1}_{\aexec}(\txid_{i}) \).
Now let consider transaction \( \txid \) that commits before \( \txid \) from the same session, \ie \( \txid \toEdge{\SO} \WTx(\hh_{\cut(\aexec, i-1)}(\ke, j)) \).
By the constraint \( \lambda \aexec.(\PO_{\aexec} ; \VIS_{\aexec}) \), the transaction \( \txid \in \VIS^{-1}_{\aexec}(\txid_{i}) \).
It means that in the kv-store \(  \hh_{\cut(\aexec, i-1)} \) every version written by \( \txid =  \WTx(\hh_{\cut(\aexec, i-1)}(\ke', m)) \) should be included in the view \( m \in \vi_i(\ke') \).
Thus we have the proof of \cref{equ:mw-complete}.

The execution test \(\et[\RYW]\) is sound with respect to the axiomatic definition
\(\visaxioms[\RYW] = \Set{\lambda \aexec \ldotp \SO_{\aexec} }\) \cite{repldatatypes}.
We pick the following invariant:
\[
    \aexecinv[\RYW](\aexec, \cl) \FuncDef
    \begin{multlined}[t]
    \left( \bigcup_{\Set{\txid[\cl](n) \in \aexec }} \Refl((\Inv(\SO)))(\txid[\cl](n)) \right) 
    \setminus \Set{\txidrd | \txidrd in \aexec \land \Forall{l | \key | \val} (l,\key,\val) \in \aexec(\txid) \implies l = \opR } .
    \end{multlined}
\]

\SOUNDLET{\RYW}{
    \txidsetrd = 
    \left( \bigcup_{\Set{\txid[\cl](n) \in \aexec }} \Refl((\Inv(\SO)))(\txid[\cl](n)) \right) 
    \cap \Set{\txidrd | \txidrd in \aexec \land \Forall{l | \key | \val} (l,\key,\val) \in \aexec(\txid) \implies l = \opR } .
}
\begin{enumerate}
\Case{\(\Forall{\visaxiom \in \visaxioms }
            \Inv(\visaxiom(\aexec'))(\txid) \subseteq \txidset \cup \txidsetrd \)}
    Suppose transactions \( \txid, \txid' \) such that \( \txid,\txid' \in \aexec \) and \( (\txid',\txid) \in \SO \).
    If \( \txid' \) is a read-only transaction, \( \txid' \in \txidsetrd \).
    Otherwise, \( \txid' \) has write, by the definition of \( \aexecinv[\RYW] \), 
    it follows that \( \txid' \in \aexecinv[\RYW](\aexec,\cl) \)
    and therefore \( \txid' \in \txidset \).
\Case{\(\aexecinv(\aexec',\cl) \subseteq \VisTrans(\XToK(\aexec'),\vi') \)}
    Because \( \ToET[\RYW]{\kvs | \vi | \fp | \kvs' | \vi' }\),
    it must be the case that
    \[
        \Forall{\key \in \Keys | \idx \in \Indexs } (\WtOf(\kvs'(\key,\idx)),\txid) \in \Refl(\SO) \implies \idx \in \vi'(\key) 
    \]
    and therefore
    \[
        \Forall{\txid'} \Exists{\key \in \Keys | \val \in \Values} \opW(\key,\val) \in \aexec'(\txid')
        \land (\txid',\txid) \in \SO \implies \txid' \in \VisTrans(\XToK(\aexec'),\vi') .
    \]
    Note that \( \bigcup_{\Set{\txid[\cl](n) \in \aexec }} \Refl((\Inv(\SO)))(\txid[\cl](n)) = \Refl((\Inv(\SO)))(\txid) \).
    Last, we have
    \begin{align*}
        \aexecinv(\aexec',\cl) & = 
            \begin{multlined}[t]
            \left( \bigcup_{\Set{\txid[\cl](n) \in \aexec }} \Inv(\SO)(\txid[\cl](n)) \right) 
            \setminus \Set{\txidrd | \txidrd \in \aexec 
                    \land \Forall{l | \key | \val} (l,\key,\val) \in \aexec(\txid) \implies l = \opR } 
            \end{multlined}
            \\ & = \begin{multlined}[t]
            \left( \Refl((\Inv(\SO)))(\txid) \right) 
            \setminus \Set{\txidrd | \txidrd \in \aexec 
                    \land \Forall{l | \key | \val} (l,\key,\val) \in \aexec(\txid) \implies l = \opR } 
            \end{multlined}
            \\ & \subseteq \VisTrans(\XToK(\aexec'),\vi') 
    \end{align*}
\end{enumerate}

\COMPLETELET{\RYW}
We construct the final view \( \vi'\) depending on whether \( \txid[\cl](n) \) is the last transaction for the client \( \cl \).
\begin{enumerate}
\Case{\( (\txid[\cl](n), \txid') \in \SO \) for \( \txid' \in \aexec \)}
    Let the transaction 
    \( \txid = \Min[\SO](\Set{ \txid' | (\txid[\cl](n), \txid') \in \SO \land \txid' \in \aexec' }) \).
    For this case, let view 
    \( \vi' = \GetView(\aexec, \Refl((\ARInv[\aexec]))(\txid[\cl](\idx)) \cap \VISInv[\aexec](\txid)) \).
    By \( \visaxioms[\RYW] \), it follows that, for any transaction \( \txid' \),
    if \( ( \txid',\txid[\cl](idx) ) \in \Refl(\SO) \), then
    \( \txid' \in \VISInv[\aexec](\txid)) \).
    Since \( \SO \in \AR \), we know that 
    \( \txid' \in \Refl((\ARInv[\aexec]))(\txid[\cl](\idx)) \cap \VISInv[\aexec](\txid)) \).
    Therefore, for any version \( \kvs'(\key,j)\) such that 
    \( ( \WtOf(\kvs'(\key,j)), \txid) \in \Refl(\SO) \),
    then \( j \in \vi'(\key)\).
\Case{\( \neg \left((\txid[\cl](n), \txid') \in \SO \right) \)}
    For this case, let 
    \( \vi' = \GetView(\aexec, \Refl((\ARInv[\aexec]))(\txid[\cl](\idx))) \) be the final view.
    It is easy to see that \( \vi' \) satisfies \( \RYW \). 
\end{enumerate}

The execution test \(\et[\WFR]\) is sound with respect to the axiomatic definition 
\(\visaxioms[\WFR] \Set{\lambda \aexec \ldotp \WR[\aexec] ; \Refl((\SO \cap \RW[\aexec] )) ; \VIS[\aexec] })\) 
\citep{surech-session-guarantee}.
By picking the invariant as \( I( \aexec, \cl ) = \emptyset \), the soundness and completeness
can be derived from \cref{thm:view-vis-relation} in a similar way as the proofs for \( \MW \).

The wildly used definition on abstract executions for causal consistency is that 
\( \VIS \) is transitive and \( \SO \in \VIS \).
Yet it is for the sack of elegant definition,
while there is a equivalent minimum visibility relation (\cref{thm:cc-visaxioms}) defined by 
\( \visaxioms[\CC] \FuncDef \Set{ \lambda \aexec \ldotp (\WR[\aexec] \cup \SO) ; \VIS[\aexec] \subseteq \VIS[\aexec] , 
                                    \lambda \aexec \ldotp \SO \subseteq \VIS[\aexec]} \),
where \( \WR[\aexec] \) is defined in \cref{def:aexec-dgraph}.

\begin{theorem}[Minimum visibility relation for (\texorpdfstring{\CC}{\texttt{CC}})]
\label{thm:cc-visaxioms}
For two abstract executions \( \aexec,\aexec' \),
the following constrain on visibility,
\begin{Formulae}
\begin{Formula}
    (\WR[\aexec] \cup \SO) ; \VIS[\aexec] \subseteq \VIS[\aexec] \land \SO \subseteq \VIS[\aexec]
    \label{equ:kvstore-cc-spec}
\end{Formula}
\end{Formulae}
is equivalent to
\begin{Formulae}
\begin{Formula}
    \VIS[\aexec'] ; \VIS[\aexec'] \subseteq \VIS[\aexec'] \land \SO \subseteq \VIS[\aexec']
    \label{equ:aexec-cc-spec}
\end{Formula}
\end{Formulae}
in that 
\(
    \Forall{\txid \in \TxIDs | \fp } \left( \fp = \aexec(\txid) \iff \fp = \aexec'(\txid) \right)
    \land \AR[\aexec] = \AR[\aexec'] .
\)
\end{theorem}
\begin{proof}
For an abstract execution \( \aexec \) that satisfies \cref{equ:kvstore-cc-spec},
by \cref{lem:aexec-spec-cc}, there exists \( \aexec' \) that satisfies \cref{equ:aexec-cc-spec}.
Assume an abstract execution \( \aexec' \) that satisfies \cref{equ:aexec-cc-spec}.
Since \( \WR[\aexec'] \subseteq \VIS[\aexec']\) by the definition of \( \WR[\aexec']\),
thus \( \aexec' \) satisfies \cref{equ:kvstore-cc-spec}.
\end{proof}

\begin{toappendix}
\begin{lemma}[Minimum visibility relation for (\texorpdfstring{\CC}{\texttt{CC}})]
\label{lem:aexec-spec-cc}
For any abstract execution \( \aexec \), if it satisfies \( \visaxioms[\CC] \),
there exists a new abstract execution \( \aexec' \) such that \( \SO \in \VIS[\aexec]\) and
\begin{Formulae}
\begin{Formula}
    \Forall{\txid \in \TxIDs | \fp } \left( \fp = \aexec(\txid) \iff \fp = \aexec'(\txid) \right)
    \land \AR[\aexec] = \AR[\aexec'] \land \VIS[\aexec'] ; \VIS[\aexec'] \subseteq \VIS[\aexec'] .
    \label{equ:aexec-spec-cc}
\end{Formula}
\end{Formulae}
\end{lemma}
\begin{proof}
We erase some visibility relation for each transaction following 
the arbitration order \( \AR \) until the visibility is transitive.
Intuitively, the final visibility relation is exactly \( \Trasi((\WR[\aexec] \cup \SO)) \).
Assume the \Th{\idx} transaction \( \txid_\idx \)  with respect to the arbitration order.
Let \( \rel[\idx] \) be a new visibility for the transaction \( \txid_\idx \) such that
\( {\rel[\idx]}\Proj{2} = \Set{\txid_\idx}\) for all indexes \( \idx \)
and the union of visibility relations \( \bigcup_{0 \leq j \leq \idx } \rel[\idx] \) is transitive.
We preserve that, for each index \( \idx \), cut of abstract execution \( \aexec' =  \AexecCut(\aexec, \idx) \)
and visibility relation \( \VIS' = \bigcup_{0 \leq j \leq \idx } \rel[j] \),
the following invariant holds:
\begin{Formulae}
& \begin{Formula} 
    \VIS' ; \VIS' \subseteq \VIS'  ,
    \label{equ:cc-vis-idx-transitive} 
\end{Formula}
\\ & \begin{Formula}
    \Forall{ \txid \in \aexec } (\txid,\txid_i) \in \rel[\idx] \implies (\txid, \txid_i) \in (\WR[\aexec'] \cup \SO) .
    \label{equ:cc-vis-idx-minimum}
\end{Formula}
\end{Formulae}
We prove the above by induction on the number \( \idx \).
\begin{enumerate}
\CaseBase{\( \idx = 0 \)}
    By the definition of \( \AexecCut \), we know that \(\aexecinit = \AexecCut(\aexec,0) \)
    and \cref{equ:cc-vis-idx-transitive,equ:cc-vis-idx-minimum} trivially hold.
\CaseInd{\( \idx > 0 \)}
    Suppose that, for the \Th{(\idx-1)} step,
    the abstract execution \( \aexec'' =  \AexecCut(\aexec, \idx - 1) \)
    and the visibility relation \( \VIS'' = \bigcup_{0 \leq j \leq \idx-1 } \rel[j] \) 
    satisfy \cref{equ:cc-vis-idx-transitive,equ:cc-vis-idx-minimum}.
    Let consider \Th{\idx} step, the transaction \( \txid_i \),
    the cut \( \aexec' =  \AexecCut(\aexec, \idx) \)
    and the visibility relation \( \VIS' = \bigcup_{0 \leq j \leq \idx } \rel[j] \).
    Initially we take \( \rel \) as an empty set.
    First, we include \( \Set{(\txid,\txid_i) | (\txid,\txid_i) \in \WR[\aexec]} \) to \( \rel \)
    and, by the definition of \( \WR[\aexec]\), 
    it trivially does not affect any read operation for the transaction \( \txid_i \).
    Then we do the same for \( \SO \) as that 
    we include \( \Set{(\txid,\txid_i) | (\txid,\txid_i) \in \SO} \) to \( \rel \).
    Note that \( \SO \) cannot affect any read operation for the transaction \( \txid_i \) neither,
    otherwise it contradicts to that \( \SO \subseteq \VIS[\aexec] \) and the definition of \( \WR[\aexec] \).

    For relations \( \rel' = \rel ; \bigcup_{0 \leq j \leq \idx-1 } \rel[j] \) and then \( \rel[\idx] = \rel \cup \rel' \),
    it easy to see that \( \rel \in \VIS[\aexec]\) and, then by \ih, \( \rel' \in \VIS[\aexec] \).
    We prove that the \( \rel[\idx] \) does not affect any read operation for the transaction \( \txid_i \)
    by contradiction.
    Assume distinct transactions \( \txid,\txid' \) such that
    \( \ToEdge{\txid'' | \rel \cup \rel' -> \txid' | \rel \cup \rel' -> \txid_i } \),
    and immediately  by the definition of \( \rel \) and \( \rel' \),
    then \( \ToEdge{\txid'' | \rel' -> \txid' | \rel -> \txid_i } \).
    Assume that \( \txid'' \) change the read operation for a key \( \key \) in \( \txid_i \).
    This means that there exists a transaction \( \txid^* \) such that
    \( (\txid^*,\txid_i) \in \WR[\aexec](\key)\) and \( (\txid^*,\txid'') \in \AR[\aexec] \),
    where the latter implies that \( (\txid'',\txid_i) \in \WR[\aexec](\key) \);
    there is a contradiction and thus 
    \( \rel[\idx] \) does not affect any read operation for the transaction \( \txid_i \).

    We now prove that \cref{equ:cc-vis-idx-transitive,equ:cc-vis-idx-minimum} still hold.
    \begin{enumerate}
    \Case{\cref{equ:cc-vis-idx-transitive}}
        Assume a relation \( \rel^* = \bigcup_{0 \leq j \leq \idx-1 } \rel[j] \) 
        and transactions \( \txid, \txid',\txid'' \) such that 
        \[
            \ToEdge{\txid | \rel^* \cup \rel[\idx] -> \txid' | \rel^* \cup \rel[\idx] -> \txid'' } .
        \]
        If \( \ToEdge{\txid | \rel^*  -> \txid' | \rel^*  -> \txid'' } \), 
        then by \ih, \( \ToEdge{\txid | \rel^*  -> \txid'' } \).
        Note that \( \ToEdge{\txid | \rel[\idx]  -> \txid' | \rel^*  -> \txid'' } \) cannot happen,
        because it contradict to that \( \txid' = \txid_i\) and \( (\txid'',\txid_i) \in \AR[\aexec] \).
        Thus consider \( \ToEdge{\txid | \rel^*  -> \txid' | \rel[\idx]  -> \txid'' } \).
        It must be the case that \( \txid'' = \txid_i \) and by the definition of \( \rel[\idx] \),
        we know that \( \ToEdge{\txid | \rel[\idx]  -> \txid'' } \).
    \Case{\cref{equ:cc-vis-idx-minimum}}
        By the construction, \cref{equ:cc-vis-idx-minimum} hold. \qedhere
    \end{enumerate}
\end{enumerate}
\end{proof}
\end{toappendix}

We pick the invariant as \( \aexecinv[\CC] = \aexecinv[\MR] \cup \aexecinv[\RYW]  \).
\SOUNDLET{\CC}{ \txidsetrd \supseteq
\begin{multlined}[t]
\left( \bigcup_{\Set{\txid[\cl](\idx) | \txid[\cl](\idx) \in \aexec}} 
\VISInv[\aexec](\txid[\cl](\idx)) \cup \Refl((\Inv(\SO)))(\txid[\cl](\idx)) \right) 
\setminus \Set{\txid' | \Forall{l | \key | \val } (l,\key,\val) \in \aexec(\txid') \implies l = \opR } .
\end{multlined} }
Assume 
\[ 
\txidsetrd' = 
\begin{multlined}[t]
\left( \bigcup_{\Set{\txid[\cl](\idx) | \txid[\cl](\idx) \in \aexec}} 
\VISInv[\aexec](\txid[\cl](\idx)) \cup \Refl((\Inv(\SO)))(\txid[\cl](\idx)) \right) 
\setminus \Set{\txid' | \Forall{l | \key | \val } (l,\key,\val) \in \aexec(\txid') \implies l = \opR } .
\end{multlined} 
\]
and \( \txidsetrd'' = \txidsetrd \setminus \txidsetrd' \).
By the definition of soundness, we prove the following result
\begin{Formulae}
& \begin{Formula}
\Inv(\SO)(\txid) \subseteq \txidset \cup \txidsetrd'
\label{equ:cc-so-vis}
\end{Formula}
\\ & \begin{Formula}
\Inv((( \WR[\aexec'] \cup \SO ) ; \VIS[\aexec'] )) (\txid) \subseteq \txidset \cup \txidsetrd' \cup \txidsetrd''
\label{equ:cc-vis-transitive}
\end{Formula}
\\ & \begin{Formula}
\aexecinv[\CC](\aexec',\cl) \subseteq \VisTrans(\XToK(\aexec'),\vi')
\label{equ:cc-inv-preserve}
\end{Formula}
\end{Formulae}
\Cref{equ:cc-so-vis} can be proven in the same way as in \cref{sec:sound-complete-mr}
We now prove \cref{equ:cc-vis-transitive}.
Initially we take \( \txidsetrd'' \) to be an empty set.
Note that \(\VISInv[\aexec'](\txid) = \txidset \cup \txidsetrd' \cup \txidsetrd'' \).
By \cref{thm:view-vis-relation,equ:view-close-to-aexec}, there exists \( \txidsetrd'' \) such that
\( \txidset \cup \txidsetrd'' \) is closed under \( \WR[\aexec'] \cup \SO \).
Now consider a transaction \( \txidrd \in \txidsetrd' \) and
assume a transaction \( \txid' \) such that \( \ToEdge{ \txid' | \WR[\aexec'] \cup \SO -> \txidrd } \).
There are two cases depending on \( \txidrd \).
\begin{enumerate}
\Case{\( \ToEdge{\txidrd | \VIS[\aexec'] -> \txid'' | \SO -> \txid} \) for some \( \txid'' \)}
    For this case, we have
    \begin{align*}
    \ToEdge{\txid' | \WR[\aexec'] \cup \SO -> \txidrd | \VIS[\aexec'] -> \txid'' | \SO -> \txid }
    & 
    \implies \ToEdge{\txid' | \WR[\aexec] \cup \SO -> \txidrd | \VIS[\aexec] -> \txid'' | \SO -> \txid }
    \\ & \implies \ToEdge{\txid' | \VIS[\aexec] -> \txid'' | \SO -> \txid } .
    \end{align*}
    By \( \aexecinv[\MR]\), we know that \(\txid' \in \aexecinv[\MR] \cup \txidsetrd' \).
\Case{\( \ToEdge{\txidrd | \SO -> \txid} \)}
    For this case, we have
    \begin{align*}
    \ToEdge{\txid' | \WR[\aexec'] \cup \SO -> \txidrd | \SO -> \txid }
    & 
    \implies \ToEdge{\txid' | \WR[\aexec] \cup \SO -> \txidrd | \SO -> \txid }
    \\ & \implies \ToEdge{\txid' | \VIS[\aexec] -> \txid'' | \SO -> \txid } .
    \end{align*}
    By \( \aexecinv[\MR]\), we know that \(\txid' \in \aexecinv[\MR] \cup \txidsetrd' \).
\end{enumerate}
Last, \cref{equ:cc-inv-preserve}can be proven in the same way as in \cref{sec:sound-complete-mr,sec:sound-complete-ryw}.

\COMPLETELET{\CC}
By \cref{thm:cc-visaxioms},
it is sufficient to prove with respect to the following visibility axioms,
\( \visaxioms[\CC]' \FuncDef \Set{ \lambda \aexec \ldotp  \VIS[\aexec] ; \VIS[\aexec] \subseteq \VIS[\aexec] , 
                                    \lambda \aexec \ldotp \SO \subseteq \VIS[\aexec]} \).
By the definition of \( \et[\CC] \), we prove \( \CanCommit[\CC]\) and \( \ViewShift[\MR \cup \RYW]\) respectively.
Since \( (\WR[\aexec] \cup \SO) ; \VIS[\aexec]  \subseteq \VIS[\aexec] ; \VIS[\aexec] \subseteq \VIS[\aexec] \),
then \( \CanCommit[\CC]\) can be derived from \cref{thm:view-vis-relation,equ:aexec-close-to-view}
and \( \ViewShift[\RYW] \) can be proven in the same way as in \cref{sec:sound-complete-ryw}.
By \( \VIS[\aexec] ; \SO \subseteq \VIS[\aexec] ; \VIS[\aexec] \subseteq \VIS[\aexec]  \),
\( \ViewShift[\MR] \) can be proven in the same way as in \cref{sec:sound-complete-mr}.



\subsection{Update Atomic}
\begin{figure}
\hrule
\begin{tabular}{@{} c c@{}}

\begin{halfsubfig}
\begin{centertikz}

\begin{pgfonlayer}{foreground}
%Uncomment line below for help lines
%\draw[help lines] grid(5,4);

%Location x
\node(locx)  {$\ke_\vx \mapsto$};

\matrix(versionx) [version list]
    at ([xshift=\tikzkvspace]locx.east) {
    {a} & $\txid_0$ \\
    {a} & $\emptyset$ \\
};

\tikzvalue{versionx-1-1}{versionx-2-1}{locx-v0}{0};

%Location y
\path (locx.south) + (0,\tikzkeyspace) node (locf1) {$\ke_{\pv{f1}} \mapsto$};
\matrix(versionf1) [version list]
    at ([xshift=\tikzkvspace]locf1.east) {
    {a} & $\txid_0$ \\
    {a} & $\emptyset$ \\
};
\tikzvalue{versionf1-1-1}{versionf1-2-1}{locf1-v0}{0};

%Location y
\path (locf1.south) + (0,\tikzkeyspace) node (locf2) {$\ke_{\pv{f2}} \mapsto$};
\matrix(versionf2) [version list]
    at ([xshift=\tikzkvspace]locf2.east) {
    {a} & $\txid_0$ \\
    {a} & $\emptyset$ \\
};
\tikzvalue{versionf2-1-1}{versionf2-2-1}{locf2-v0}{0};

% \draw[-, red, very thick, rounded corners] ([xshift=-5pt, yshift=5pt]locx-v1.north east) |- 
%  ($([xshift=-5pt,yshift=-5pt]locx-v1.south east)!.5!([xshift=-5pt, yshift=5pt]locy-v0.north east)$) -| ([xshift=-5pt, yshift=5pt]locy-v0.south east);

%blue view - I should  check whether I can use pgfkeys to just declare the list of locations, and then add the view automatically.
\draw[-, blue, very thick, rounded corners=10pt]
 ([xshift=-2pt, yshift=20pt]locx-v0.north east) node (tid1start) {} -- 
 ([xshift=-2pt, yshift=-5pt]locf2-v0.south east);
 
 \path (tid1start) node[anchor=south, rectangle, fill=blue!20, draw=blue, font=\small, inner sep=1pt] {$\thid_3$};

%red view
\draw[-, red, very thick, rounded corners = 10pt]
 ([xshift=-5pt, yshift=5pt]locx-v0.north east) -- 
 ([xshift=-5pt, yshift=-10pt]locf2-v0.south east) node (tid2start) {};
 
\path (tid2start) node[anchor=north, rectangle, fill=red!20, draw=red, font=\small, inner sep=1pt] {$\thid_2$};
 
 %green view
\draw[-, DarkGreen, very thick, rounded corners = 10pt]
 ([xshift=-16pt, yshift=8pt]locx-v0.north east) node (tid3start) {}-- 
 ([xshift=-16pt, yshift=-5pt]locf2-v0.south east);
 
 \path (tid3start) node[anchor=south, rectangle, fill=DarkGreen!20, draw=DarkGreen, font=\small, inner sep=1pt] {$\thid_1$};

\end{pgfonlayer}
\end{centertikz}%
\caption{Initial configuration}
\label{fig:ua-init}
\end{halfsubfig}
%
&
%
\begin{halfsubfig}
\begin{centertikz}
\begin{pgfonlayer}{foreground}
%Uncomment line below for help lines
%\draw[help lines] grid(5,4);

\node(locx)  {$\ke_\vx \mapsto$};

\matrix(versionx) [version list]
    at ([xshift=\tikzkvspace]locx.east) {
    {a} & $\txid_0$ & {a} & \(\txid_1\)\\
    {a} & $\Set{\txid_1}$ & {a} & \(\emptyset\)\\
};

\tikzvalue{versionx-1-1}{versionx-2-1}{locx-v0}{0};
\tikzvalue{versionx-1-3}{versionx-2-3}{locx-v1}{1};


\path (locx.south) + (0,\tikzkeyspace) node (locf1) {$\ke_{\pv{f1}} \mapsto$};
\matrix(versionf1) [version list]
    at ([xshift=\tikzkvspace]locf1.east) {
    {a} & $\txid_0$ & {a} & $\txid_1$\\
    {a} & $\Set{\txid_1}$ & {a} & $\emptyset$\\
};
\tikzvalue{versionf1-1-1}{versionf1-2-1}{locf1-v0}{0};
\tikzvalue{versionf1-1-3}{versionf1-2-3}{locf1-v1}{1};

%Location y
\path (locf1.south) + (0,\tikzkeyspace) node (locf2) {$\ke_{\pv{f2}} \mapsto$};
\matrix(versionf2) [version list]
    at ([xshift=\tikzkvspace]locf2.east) {
    {a} & $\txid_0$ \\
    {a} & $\emptyset$ \\
};
\tikzvalue{versionf2-1-1}{versionf2-2-1}{locf2-v0}{0};


% \draw[-, red, very thick, rounded corners] ([xshift=-5pt, yshift=5pt]locx-v1.north east) |- 
%  ($([xshift=-5pt,yshift=-5pt]locx-v1.south east)!.5!([xshift=-5pt, yshift=5pt]locy-v0.north east)$) -| ([xshift=-5pt, yshift=5pt]locy-v0.south east);

%blue view - I should  check whether I can use pgfkeys to just declare the list of locations, and then add the view automatically.
\draw[-, blue, very thick, rounded corners=10pt]
 ([xshift=-2pt, yshift=20pt]locx-v0.north east) node (tid1start) {} -- 
 ([xshift=-2pt, yshift=-5pt]locf2-v0.south east);
 
 \path (tid1start) node[anchor=south, rectangle, fill=blue!20, draw=blue, font=\small, inner sep=1pt] {$\thid_3$};

%red view
\draw[-, red, very thick, rounded corners = 10pt]
 ([xshift=-5pt, yshift=5pt]locx-v0.north east) -- 
 ([xshift=-5pt, yshift=-10pt]locf2-v0.south east) node (tid2start) {};
 
\path (tid2start) node[anchor=north, rectangle, fill=red!20, draw=red, font=\small, inner sep=1pt] {$\thid_2$};
 
 %green view
\draw[-, DarkGreen, very thick, rounded corners = 10pt]
 ([xshift=-16pt, yshift=8pt]locx-v1.north east) node (tid3start) {}-- 
 ([xshift=-16pt, yshift=-5pt]locf1-v1.south east) --
 ([xshift=-16pt, yshift=5pt]locf2-v0.north east) -- 
 ([xshift=-16pt, yshift=-5pt]locf2-v0.south east);
 
 \path (tid3start) node[anchor=south, rectangle, fill=DarkGreen!20, draw=DarkGreen, font=\small, inner sep=1pt] {$\thid_1$};

\end{pgfonlayer}
\end{centertikz}
\caption{After \(\txid_1\)}
\label{fig:ua-after-tx1}
\end{halfsubfig}

\\
\begin{subfigure}{0.45\textwidth}
\begin{centertikz}%
\begin{pgfonlayer}{foreground}
%Uncomment line below for help lines
%\draw[help lines] grid(5,4);


\node(locx)  {$\ke_\vx \mapsto$};

\matrix(versionx) [version list, column 2/.style = {text width=14mm}]
    at ([xshift=\tikzkvspace]locx.east) {
    {a} & $\txid_0$ & {a} & $\txid_1$ & {a} & $\txid_2$\\
    {a} & $\Set{\txid_1, \txid_2}$ & {a} & $\emptyset$ & {a} & $\emptyset$\\
};

\tikzvalue{versionx-1-1}{versionx-2-1}{locx-v0}{0};
\tikzvalue{versionx-1-3}{versionx-2-3}{locx-v1}{1};
\tikzvalue{versionx-1-5}{versionx-2-5}{locx-v2}{1};


\path (locx.south) + (0,\tikzkeyspace) node (locf1) {$\ke_{\pv{f1}} \mapsto$};
\matrix(versionf1) [version list]
    at ([xshift=\tikzkvspace]locf1.east) {
    {a} & $\txid_0$ & {a} & $\txid_1$\\
    {a} & $\Set{\txid_1}$ & {a} & $\emptyset$\\
};
\tikzvalue{versionf1-1-1}{versionf1-2-1}{locf1-v0}{0};
\tikzvalue{versionf1-1-3}{versionf1-2-3}{locf1-v1}{1};

%Location y
\path (locf1.south) + (0,\tikzkeyspace) node (locf2) {$\ke_{\pv{f2}} \mapsto$};
\matrix(versionf2) [version list]
    at ([xshift=\tikzkvspace]locf2.east) {
    {a} & $\txid_0$ & {a} & \(\txid_2\) \\
    {a} & $\emptyset$ & {a} & \(\emptyset\) \\
};
\tikzvalue{versionf2-1-1}{versionf2-2-1}{locf2-v0}{0};
\tikzvalue{versionf2-1-3}{versionf2-2-3}{locf2-v1}{1};


% \draw[-, red, very thick, rounded corners] ([xshift=-5pt, yshift=5pt]locx-v1.north east) |- 
%  ($([xshift=-5pt,yshift=-5pt]locx-v1.south east)!.5!([xshift=-5pt, yshift=5pt]locy-v0.north east)$) -| ([xshift=-5pt, yshift=5pt]locy-v0.south east);

%blue view - I should  check whether I can use pgfkeys to just declare the list of locations, and then add the view automatically.
\draw[-, blue, very thick, rounded corners=10pt]
([xshift=-2pt, yshift=20pt]locx-v0.north east) node (tid1start) {} -- 
([xshift=-2pt, yshift=-5pt]locf2-v0.south east);
 
\path (tid1start) node[anchor=south, rectangle, fill=blue!20, draw=blue, font=\small, inner sep=1pt] {$\thid_3$};

%red view
\draw[-, red, very thick, rounded corners = 10pt]
([xshift=-5pt, yshift=5pt]locx-v2.north east) -- 
([xshift=-5pt, yshift=-5pt]locx-v2.south east) --
([xshift=-5pt, yshift=5pt]locf1-v0.north east) -- 
([xshift=-5pt, yshift=-5pt]locf1-v0.south east) --
([xshift=-5pt, yshift=5pt]locf2-v1.north east) -- 
([xshift=-5pt, yshift=-10pt]locf2-v1.south east) node (tid2start) {};

\path (tid2start) node[anchor=north, rectangle, fill=red!20, draw=red, font=\small, inner sep=1pt] {$\thid_2$};
 
 %green view
\draw[-, DarkGreen, very thick, rounded corners = 10pt]
([xshift=-16pt, yshift=8pt]locx-v1.north east) node (tid3start) {}-- 
([xshift=-16pt, yshift=-5pt]locx-v1.south east) --
([xshift=-16pt, yshift=5pt]locf1-v1.north east) -- 
([xshift=-16pt, yshift=-5pt]locf1-v1.south east) --
([xshift=-16pt, yshift=5pt]locf2-v0.north east) -- 
([xshift=-16pt, yshift=-5pt]locf2-v0.south east);

\path (tid3start) node[anchor=south, rectangle, fill=DarkGreen!20, draw=DarkGreen, font=\small, inner sep=1pt] {$\thid_1$};

\end{pgfonlayer}%
\end{centertikz}%
\caption{After \(\txid_2\)}
\label{fig:ua-after-tx2}
\end{subfigure}
%
&
%
\begin{subfigure}{0.45\textwidth}
\begin{centertikz}
\begin{pgfonlayer}{foreground}
%Uncomment line below for help lines
%\draw[help lines] grid(5,4);

\node(locx)  {$\ke_\vx \mapsto$};

\matrix(versionx) [version list, column 2/.style = {text width=14mm}]
    at ([xshift=\tikzkvspace]locx.east) {
    {a} & $\txid_0$ & {a} & $\txid_1$ & {a} & $\txid_2$\\
    {a} & $\Set{\txid_1, \txid_2}$ & {a} & $\emptyset$ & {a} & $\emptyset$\\
};

\tikzvalue{versionx-1-1}{versionx-2-1}{locx-v0}{0};
\tikzvalue{versionx-1-3}{versionx-2-3}{locx-v1}{1};
\tikzvalue{versionx-1-5}{versionx-2-5}{locx-v2}{1};


\path (locx.south) + (0,\tikzkeyspace) node (locf1) {$\ke_{\pv{f1}} \mapsto$};
\matrix(versionf1) [version list]
    at ([xshift=\tikzkvspace]locf1.east) {
    {a} & $\txid_0$ & {a} & $\txid_1$\\
    {a} & $\Set{\txid_1}$ & {a} & $\emptyset$\\
};
\tikzvalue{versionf1-1-1}{versionf1-2-1}{locf1-v0}{0};
\tikzvalue{versionf1-1-3}{versionf1-2-3}{locf1-v1}{1};

%Location y
\path (locf1.south) + (0,\tikzkeyspace) node (locf2) {$\ke_{\pv{f2}} \mapsto$};
\matrix(versionf2) [version list]
    at ([xshift=\tikzkvspace]locf2.east) {
    {a} & $\txid_0$ & {a} & \(\txid_2\) \\
    {a} & $\emptyset$ & {a} & \(\emptyset\) \\
};
\tikzvalue{versionf2-1-1}{versionf2-2-1}{locf2-v0}{0};
\tikzvalue{versionf2-1-3}{versionf2-2-3}{locf2-v1}{1};


% \draw[-, red, very thick, rounded corners] ([xshift=-5pt, yshift=5pt]locx-v1.north east) |- 
%  ($([xshift=-5pt,yshift=-5pt]locx-v1.south east)!.5!([xshift=-5pt, yshift=5pt]locy-v0.north east)$) -| ([xshift=-5pt, yshift=5pt]locy-v0.south east);

%blue view - I should  check whether I can use pgfkeys to just declare the list of locations, and then add the view automatically.
\draw[-, blue, very thick, rounded corners=10pt]
([xshift=-2pt, yshift=20pt]locx-v2.north east) node (tid1start) {} -- 
([xshift=-2pt, yshift=-7pt]locx-v2.south east) --
([xshift=-2pt, yshift=3pt]locf1-v1.north east) -- 
([xshift=-2pt, yshift=-5pt]locf1-v1.south east) --
([xshift=-2pt, yshift=5pt]locf2-v1.north east) -- 
([xshift=-2pt, yshift=-5pt]locf2-v1.south east);

\path (tid1start) node[anchor=south, rectangle, fill=blue!20, draw=blue, font=\small, inner sep=1pt] {$\thid_3$};

%red view
\draw[-, red, very thick, rounded corners = 10pt]
([xshift=-5pt, yshift=5pt]locx-v2.north east) -- 
([xshift=-5pt, yshift=-5pt]locx-v2.south east) --
([xshift=-5pt, yshift=5pt]locf1-v0.north east) -- 
([xshift=-5pt, yshift=-5pt]locf1-v0.south east) --
([xshift=-5pt, yshift=5pt]locf2-v1.north east) -- 
([xshift=-5pt, yshift=-10pt]locf2-v1.south east) node (tid2start) {};

\path (tid2start) node[anchor=north, rectangle, fill=red!20, draw=red, font=\small, inner sep=1pt] {$\thid_2$};
 
 %green view
\draw[-, DarkGreen, very thick, rounded corners = 10pt]
([xshift=-16pt, yshift=8pt]locx-v1.north east) node (tid3start) {}-- 
([xshift=-16pt, yshift=-5pt]locx-v1.south east) --
([xshift=-16pt, yshift=5pt]locf1-v1.north east) -- 
([xshift=-16pt, yshift=-5pt]locf1-v1.south east) --
([xshift=-16pt, yshift=5pt]locf2-v0.north east) -- 
([xshift=-16pt, yshift=-5pt]locf2-v0.south east);

\path (tid3start) node[anchor=south, rectangle, fill=DarkGreen!20, draw=DarkGreen, font=\small, inner sep=1pt] {$\thid_1$};

\end{pgfonlayer}
\end{centertikz}%
\caption{\(\txid_3\) updates the view}
\label{fig:ua-before-tx2}
\end{subfigure} \\
\end{tabular}
\hrule
\caption{An invalid executions under update atomic for $\prog_3$}
\label{fig:cu.exec}
\label{fig:cu-exec}
\end{figure}




\begin{figure}
\hrule
\begin{tabular}{@{} c c@{}}

\begin{subfigure}{0.45\textwidth}
\begin{centertikz}

\begin{pgfonlayer}{foreground}
%Uncomment line below for help lines
%\draw[help lines] grid(5,4);

\node(locx)  {$\ke_\vx \mapsto$};

\matrix(versionx) [version list]
    at ([xshift=\tikzkvspace]locx.east) {
    {a} & $\txid_0$ & {a} & \(\txid_1\)\\
    {a} & $\Set{\txid_1}$ & {a} & \(\emptyset\)\\
};

\tikzvalue{versionx-1-1}{versionx-2-1}{locx-v0}{0};
\tikzvalue{versionx-1-3}{versionx-2-3}{locx-v1}{1};


\path (locx.south) + (0,\tikzkeyspace) node (locf1) {$\ke_{\pv{f1}} \mapsto$};
\matrix(versionf1) [version list]
    at ([xshift=\tikzkvspace]locf1.east) {
    {a} & $\txid_0$ & {a} & $\txid_1$\\
    {a} & $\Set{\txid_1}$ & {a} & $\emptyset$\\
};
\tikzvalue{versionf1-1-1}{versionf1-2-1}{locf1-v0}{0};
\tikzvalue{versionf1-1-3}{versionf1-2-3}{locf1-v1}{1};

%Location y
\path (locf1.south) + (0,\tikzkeyspace) node (locf2) {$\ke_{\pv{f2}} \mapsto$};
\matrix(versionf2) [version list]
    at ([xshift=\tikzkvspace]locf2.east) {
    {a} & $\txid_0$ \\
    {a} & $\emptyset$ \\
};
\tikzvalue{versionf2-1-1}{versionf2-2-1}{locf2-v0}{0};


% \draw[-, red, very thick, rounded corners] ([xshift=-5pt, yshift=5pt]locx-v1.north east) |- 
%  ($([xshift=-5pt,yshift=-5pt]locx-v1.south east)!.5!([xshift=-5pt, yshift=5pt]locy-v0.north east)$) -| ([xshift=-5pt, yshift=5pt]locy-v0.south east);

%blue view - I should  check whether I can use pgfkeys to just declare the list of locations, and then add the view automatically.
\draw[-, blue, very thick, rounded corners=10pt]
([xshift=-2pt, yshift=20pt]locx-v0.north east) node (tid1start) {} -- 
([xshift=-2pt, yshift=-5pt]locf2-v0.south east);

\path (tid1start) node[anchor=south, rectangle, fill=blue!20, draw=blue, font=\small, inner sep=1pt] {$\thid_3$};

%red view
\draw[-, red, very thick, rounded corners = 10pt]
([xshift=-5pt, yshift=5pt]locx-v1.north east) -- 
([xshift=-5pt, yshift=-5pt]locf1-v1.south east) --
([xshift=-5pt, yshift=5pt]locf2-v0.north east) -- 
([xshift=-5pt, yshift=-10pt]locf2-v0.south east) node (tid2start) {};

\path (tid2start) node[anchor=north, rectangle, fill=red!20, draw=red, font=\small, inner sep=1pt] {$\thid_2$};
 
 %green view
\draw[-, DarkGreen, very thick, rounded corners = 10pt]
([xshift=-16pt, yshift=8pt]locx-v1.north east) node (tid3start) {}-- 
([xshift=-16pt, yshift=-5pt]locf1-v1.south east) --
([xshift=-16pt, yshift=5pt]locf2-v0.north east) -- 
([xshift=-16pt, yshift=-5pt]locf2-v0.south east);

\path (tid3start) node[anchor=south, rectangle, fill=DarkGreen!20, draw=DarkGreen, font=\small, inner sep=1pt] {$\thid_1$};

\end{pgfonlayer}
\end{centertikz}%
\caption{\(\thid_2\) updates the view}
\label{fig:ua-thid-2-update-view}
\end{subfigure} 
&
\begin{subfigure}{0.45\textwidth}
\begin{centertikz}

\begin{pgfonlayer}{foreground}
%Uncomment line below for help lines
%\draw[help lines] grid(5,4);

\node(locx)  {$\ke_\vx \mapsto$};

\matrix(versionx) [version list]
    at ([xshift=\tikzkvspace]locx.east) {
    {a} & $\txid_0$ & {a} & $\txid_1$ & {a} & $\txid_2$\\
    {a} & $\Set{\txid_1}$ & {a} & $\Set{\txid_2}$ & {a} & $\emptyset$\\
};

\tikzvalue{versionx-1-1}{versionx-2-1}{locx-v0}{0};
\tikzvalue{versionx-1-3}{versionx-2-3}{locx-v1}{1};
\tikzvalue{versionx-1-5}{versionx-2-5}{locx-v2}{2};


\path (locx.south) + (0,\tikzkeyspace) node (locf1) {$\ke_{\pv{f1}} \mapsto$};
\matrix(versionf1) [version list]
    at ([xshift=\tikzkvspace]locf1.east) {
    {a} & $\txid_0$ & {a} & $\txid_1$\\
    {a} & $\Set{\txid_1}$ & {a} & $\emptyset$\\
};
\tikzvalue{versionf1-1-1}{versionf1-2-1}{locf1-v0}{0};
\tikzvalue{versionf1-1-3}{versionf1-2-3}{locf1-v1}{1};

%Location y
\path (locf1.south) + (0,\tikzkeyspace) node (locf2) {$\ke_{\pv{f2}} \mapsto$};
\matrix(versionf2) [version list]
    at ([xshift=\tikzkvspace]locf2.east) {
    {a} & $\txid_0$ & {a} & \(\txid_2\) \\
    {a} & $\emptyset$ & {a} & \(\emptyset\) \\
};
\tikzvalue{versionf2-1-1}{versionf2-2-1}{locf2-v0}{0};
\tikzvalue{versionf2-1-3}{versionf2-2-3}{locf2-v1}{1};

% \draw[-, red, very thick, rounded corners] ([xshift=-5pt, yshift=5pt]locx-v1.north east) |- 
%  ($([xshift=-5pt,yshift=-5pt]locx-v1.south east)!.5!([xshift=-5pt, yshift=5pt]locy-v0.north east)$) -| ([xshift=-5pt, yshift=5pt]locy-v0.south east);

%blue view - I should  check whether I can use pgfkeys to just declare the list of locations, and then add the view automatically.
\draw[-, blue, very thick, rounded corners=10pt]
([xshift=-2pt, yshift=20pt]locx-v0.north east) node (tid1start) {} -- 
([xshift=-2pt, yshift=-5pt]locf2-v0.south east);

\path (tid1start) node[anchor=south, rectangle, fill=blue!20, draw=blue, font=\small, inner sep=1pt] {$\thid_3$};

%red view
\draw[-, red, very thick, rounded corners = 10pt]
([xshift=-5pt, yshift=5pt]locx-v2.north east) -- 
([xshift=-5pt, yshift=-5pt]locx-v2.south east) --
([xshift=-5pt, yshift=5pt]locf1-v1.north east) -- 
([xshift=-5pt, yshift=-10pt]locf2-v1.south east) node (tid2start) {};

\path (tid2start) node[anchor=north, rectangle, fill=red!20, draw=red, font=\small, inner sep=1pt] {$\thid_2$};
 
 %green view
\draw[-, DarkGreen, very thick, rounded corners = 10pt]
([xshift=-16pt, yshift=8pt]locx-v1.north east) node (tid3start) {}-- 
([xshift=-16pt, yshift=-5pt]locf1-v1.south east) --
([xshift=-16pt, yshift=5pt]locf2-v0.north east) -- 
([xshift=-16pt, yshift=-5pt]locf2-v0.south east);

\path (tid3start) node[anchor=south, rectangle, fill=DarkGreen!20, draw=DarkGreen, font=\small, inner sep=1pt] {$\thid_1$};

\end{pgfonlayer}
\end{centertikz}%
\caption{After \(\txid_2\)}
\label{fig:ua-correct-after-tx2}
\end{subfigure} 
\\
\end{tabular}
\hrule
\caption{A execution of $\prog_3$ without lost-update}
\label{fig:ua-conf-2}
\end{figure}


\ac{This Consistency Model shows why the notion of consistent views must 
depend on the set of operations that need to be executed.}

The next consistency model that we consider is \emph{update atomic}. 
Although we did not find any implementation of this model, it has been proposed in \cite{framework-concur} as a strengthening to Read Atomic to avoid write-write conflicts.
This model states that: \textbf{(i)} transactions satisfy atomic visibility (\cref{def:readatomic}); and \textbf{(ii)} transactions writing to one same keys cannot be executed concurrently.
\sx{ This appears too earlier:
Update Atomic is also needed to specify more sophisticated consistency models, 
such as \emph{Parallel Snapshot Isolation} and \emph{Snapshot Isolation}.}
\ac{Check: Nobi said he was interested in implementing Update Atomic 
at some point, maybe he ended up doing something.}

Programs under update atomic do not exhibit the \emph{lost update} anomaly: two or more transactions update the same address, for example , both increment its value by $1$, but only one of them will be observed by future transactions, for example, only one of the increments takes effect.
To illustrate the \emph{lost-update anomaly}, consider the following program \( \prog_3 \) where two transactions concurrently increment $\vx$ and the third transaction read the value. 
Note that the \( \pvar{f1} \) and \( \pvar{f2} \) are two flags indicating the corresponding transactions has been committed.
\ac{Intuitive behaviour of the litmus test: two transactions concurrently increment $[\loc_x]$. 
 A third transaction observes that the first two transactions have been executed. 
 However, it only observes one of the two increments taking place.
 }
\[
    \prog_3 \equiv \begin{session}
        \begin{array}{@{}c || c || c@{}}
        \txid_1 : 
        \begin{transaction} 
            \pmutate{\pvar{f1}}{1};\\
            \pderef{\pvar{a}}{\vx};\\
            \pmutate{\vx}{a + 1};\\
        \end{transaction} & 
        \txid_2 : 
        \begin{transaction}
            \pmutate{\pvar{f2}}{1};\\
            \pderef{\pvar{a}}{\vx};\\
            \pmutate{\vx}{a + 1};\\
        \end{transaction} &
        \txid_3 : 
        \begin{transaction}
            \pderef{\pvar{a}}{\vx};\\
            \pderef{\pvar{b}}{\pvar{f1}};\\
            \pderef{\pvar{c}}{\pvar{f2}};\\
            \pifs{\pvar{a}=1 \wedge \pvar{b}=1 \wedge \pvar{c} = 1}\\ 
                \quad \passign{\retvar}{\sadface}
            \pife
        \end{transaction}
        \end{array}
    \end{session}
 \]

We consider an execution in which the transactions contained in the code of threads $\thid_1, \thid_2$ both execute on the same snapshot determined by the initial view. 
The initial configuration of the program coincides with the one given in \cref{fig:ua-init}.
After executing the transaction $\txid_1$, the resulting configuration is the one depicted \ref{fig:ua-after-tx1} and then \( \txid_2 \) shown in in \ref{fig:ua-after-tx2}, where both transactions read the initial version for key $\ke_\vx$. 
The third transaction $\txid_3$ choose to update its view to include the most recent version for all the keys (\ref{fig:ua-before-tx3}), then when executing its code, all the keys will have value $1$, and the return variable will be set to ${\sadface}$.

The program $\prog_3$ might exhibit the lost-update anomaly when the second transaction $\txid_2$ starts, its view did not include the most up-to-date version for key $\ke_\vx$ provided that \( \txid_{2}\) will update the key \( \ke_\vx \).
As consequence, the database \emph{lost the update} of a version of \( \ke_\vx \) installed by the transaction $\txid_1$, in a sense that no transaction will observe such the version.
To forbid this anomaly, the \emph{update atomic} requires that if a transaction writes to a key, the transaction should start with a view including the most recent version for the key.

\begin{definition}
\label{def:update-atomic}
\emph{Update atomic} is stronger than then read atomic (\cref{def:readatomic}) by further requiring for all keys written, it should starts with a view including the most recent version for those key:
\[
\begin{rclarray}
(\hh, \vi) \csat[\mathsf{UA}] \opset: \vi' & \defeq &
\begin{array}[t]{@{}l}
(\hh, \vi) \csat[\mathsf{RA}] \opset: \vi' \land \fora{\addr} 
(\otW, \addr, \stub) \in \opset \implies \vi(\addr)  = \left| \hh(\addr) \right| - 1
\end{array} \\
\end{rclarray}
\]
\end{definition}

\begin{proposition}
The execution test $\comoUA$ does not hinder progress. 
For any $\hh, \vi, \opset$, there exist $\vi' : \vi \leq \vi'$ and $\vi'': \Vupdate(\hh, \vi', \opset) \leq \vi''$ such that $(\hh, \vi') \csatUA \opset, \vi''$.
\end{proposition}

The thread $\thid_2$ from the program \( \prog_3\) , under $\mathsf{UA}$, cannot execute the transaction $\txid_2$ starting from the configuration depicted in \cref{fig:ua-after-tx1}.
Because the view of $\thid_2$ does not include the most recent version for key $\ke_\vx$. 
Instead, before executing, $\thid_2$ must update its view to include the most recent version of $\ke_\vx$ (\cref{fig:ua-thid-2-update-view}).
Then the \( \txid_2\) will install a new version for \( \ke_\vx \) with value 2 instead of 1 as shown in \cref{fig:ua-correct-after-tx2}.
There are now three different possible views in which $\thid_3$ can execute its transaction.
First, executing on the initial view, in which the transaction will observe 0 for the three locations, and the transaction will not return value $\sadface$.
Second, executing on the one in which the view of $\thid_3$ for $\ke_\vx$ points to the version $(1, \tsid_1, \Set{\tsid_2})$. 
Because of atomic visibility, it must also includes the most recent version for key $\ke_\pv{f2}$ since it is installed by \( \txid_2 \).
In this case, it will not return \(\sadface \).
Last, executing on the one in which the view of $\thid_3$ for $\ke_\vx$ points to its most recent version $(2, \txid_2, \emptyset)$.
In this case, it will not return \(\sadface \).

\subsection{Consistency Prefix \( \CP \) }
\label{sec:sound-complete-cp}

Given abstract execution \( \aexec \), we define read-write read-write relation:
\[
    \RW(\aexec,\ke) \defeq \Setcon{(\txid, \txid')}{\txid \toEdge{\AR_\aexec} \txid' \land (\otR,\ke, \stub ) \in \txid \land (\otW,\ke, \stub ) \in \txid'  } 
\]
It is easy to see \( \RW(\aexec,\ke) \)  can be derived from \( \WW(\aexec,\ke) \) and \( \WR(\aexec, \ke ) \) as the following:
\[
    \RW(\aexec,\ke) = \Setcon{(\txid, \txid')}{ \exsts{\txid'' } (\txid'', \txid) \in \WR(\aexec, \ke) \land (\txid'', \txid') \in \WW(\aexec, \ke) }
\]
Then, the notation \( \RW_\aexec \defeq \bigcup\limits_{\ke \in \Keys} \RW(\aexec, \ke) \).
Note that for a key-value store \( \mkvs \) such that \( \mkvs = \mkvs_\aexec \),
by the definition of  \(  \mkvs = \mkvs_\aexec \), 
the following holds:
\[
    \RW_\aexec = \Setcon{(\txid, \txid')}{\exsts{\ke, i,j } \txid \in \RTx(\mkvs(\ke, i)) \land \txid' = \WTx(\mkvs(\ke, j)) \land i < j}
\]
The \( \RW_\aexec \) also coincides with \( \RW_\Gr \) and \( \RW_\mkvs \).


An abstract execution \( \aexec \) satisfies consistency prefix (\(\CP\)), 
if it satisfies \( \AR_\aexec ; \VIS_\aexec \subseteq \VIS_\aexec \) and \( \SO_\aexec \subseteq \VIS_\aexec \).
Given the specification, there is a corresponding specification on dependency graph by solve the following inequalities:
\[
    \begin{array}{@{}l@{}}
        \WR \subseteq \VIS \\
        \WW \subseteq \AR \\
        \VIS \subseteq \AR \\
        \VIS ; \RW \subseteq \AR \\
        \AR ; \AR \subseteq \AR  \\
        \SO \subseteq \VIS \\
        \AR ; \VIS \subseteq \VIS
    \end{array}
\]
By solving the inequalities the visibility and arbitration relations are:
\[
    \begin{rclarray}
        \AR & \defeq & \left( (\SO \cup \WR ) ; \RW? \cup \WW \cup R \right)^+ \\
        \VIS & \defeq & \left( (\SO \cup \WR ) ; \RW? \cup \WW \cup R \right)^* ; (\SO \cup \WR )
    \end{rclarray}
\]
for some relation \( R \subseteq \AR \).
When \( R = \emptyset \), it is the smallest solution therefore the minimum visibility required.

\sx{A bit verbal}
\begin{lemma}
    \label{lem:cp-eauiv-spec}
    For any abstract execution \( \aexec \),
    if it satisfies 
    \[
        \left( (\SO \cup \WR ) ; \RW? \cup \WW \right)^* ; \VIS_\aexec \subseteq \VIS_\aexec 
        \qquad \SO_\aexec \subseteq \VIS_\aexec
    \]
    then there exists a new \( \aexec' \) such that \( \T_\aexec = \T_{\aexec'} \), 
    under last-write-win \( \TtoOp{T}_{\aexec}(\txid) = \TtoOp{T}_{\aexec'}(\txid) \) for all transactions \( \txid \),
    and the relations satisfy the following:
    \[ 
        \AR_{\aexec'} ; \VIS_{\aexec'} \subseteq \VIS_{\aexec'}  \qquad \SO_{\aexec'} \subseteq \VIS_{\aexec'}
    \]
    and vice versa.
\end{lemma}
\begin{proof}
    Assume abstract execution \( \aexec' \) that
    satisfies \( \AR_{\aexec'} ; \VIS_{\aexec'} \subseteq \VIS_{\aexec'} \)
    and  \( \SO_{\aexec'} \subseteq \VIS_{\aexec'} \).
    We already show that:
\[
    \begin{rclarray}
        \AR_{\aexec'} & = & \left( (\SO_\aexec \cup \WR_\aexec ) ; \RW_\aexec? \cup \WW_\aexec \cup R \right)^+ \\
        \VIS_{\aexec'} & = & \left( (\SO_\aexec \cup \WR_\aexec ) ; \RW_\aexec? \cup \WW_\aexec \cup R \right)^* ; (\SO_\aexec \cup \WR_\aexec )
    \end{rclarray}
\]
for some relation \( R \subseteq \AR_{\aexec'} \).
If we take \( R  = \emptyset \), we have the proof for:
\[
        \SO \subseteq \VIS_\aexec \qquad 
        \left( (\SO_\aexec \cup \WR_\aexec ) ; \RW_\aexec? \cup \WW_\aexec \right)^* ; \VIS_\aexec \subseteq \VIS_\aexec
\]
For another way, we pick the \( R \) that extends
\( \left( (\SO_\aexec \cup \WR_\aexec ) ; \RW_\aexec? \cup \WW_\aexec \cup R \right)^+ \) 
to a total order.
\end{proof}

By \cref{lem:cp-eauiv-spec} to prove soundness and completeness of \( \ET_\CP \), it is sufficient to use the specification:
\[
    (\RP_{\LWW}, \Set{\lambda \aexec. \left( (\SO \cup \WR ) ; \RW? \cup \WW \right)^* ; \VIS_\aexec, \lambda \aexec \ldotp \SO_\aexec }) 
\]

For the soundness, we pick the invariant as the following:
\[  
\begin{rclarray}
    I_1(\aexec, \cl) & = & \left( \bigcup\limits_{\{\txid_{\cl}^{i} \in \T_{\aexec} \mid i \in \Nat\}} \VIS_{\aexec}^{-1}(\txid^i_\cl) \right) \setminus \T_\rd \\
    I_2(\aexec, \cl) & = & \left( \bigcup\limits_{\{\txid_{\cl}^{i} \in \T_{\aexec} \mid i \in \Nat\}} (\SO_{\aexec}^{-1})?(\txid^i_\cl) \right) \setminus \T_\rd
\end{rclarray}
\]
where \( \T_\rd \) is all the read-only transactions included in both 
\( \left( \bigcup\limits_{\{\txid_{\cl}^{i} \in \T_{\aexec} \mid i \in \Nat\}} \VIS_{\aexec}^{-1}(\txid^i_\cl) \right)\) 
and \( \left( \bigcup\limits_{\{\txid_{\cl}^{i} \in \T_{\aexec} \mid i \in \Nat\}} (\SO_{\aexec}^{-1})?(\txid^i_\cl) \right) \).
Assume a key-value store $\hh$, an initial and a final view $\vi, \vi'$  a fingerprint $\opset$ 
such that $\ET_{\CP} \vdash (\hh, \vi) \triangleright \opset: \vi'$. 
Also choose an arbitrary $\cl$, a transaction identifier $\txid_\cl^n \in \nextTxId(\hh, \cl)$, 
and an abstract execution $\aexec$ such that $\hh_{\aexec} = \hh$ and 
\( I_1(\aexec, \cl) \cup I_2(\aexec, \cl) \subseteq \Tx(\hh, \vi) \).
Let a new abstract execution \( \aexec' = \extend(\aexec, \txid_\cl^n, \f, \Tx(\mkvs, \vi) \cup \T_\rd) \).
We are about to prove that there exists an extra set of read-only transaction \( \T'_\rd \) such that:
\begin{gather}
    \fora{\txid} (\txid, \txid_\cl^n) \in \SO_{\aexec'} \implies \txid \in \Tx(\mkvs, \vi) \cup \T_\rd \cup \T'_\rd \label{equ:cp-sound-update-so}\\
    \begin{array}{l}
    \fora{\txid} (\txid, \txid_\cl^n) \in \left( (\SO_{\aexec'} \cup \WR_{\aexec'} ) ; \RW_{\aexec'}? \cup \WW_{\aexec'} \right)^* ; \VIS_{\aexec'} \\
    \qqquad \implies \txid \in \Tx(\mkvs, \vi) \cup \T_\rd \cup \T'_\rd 
    \end{array}
    \label{equ:cp-sound-update-arvis}\\
    I_1(\aexec',\cl) \cup I_2(\aexec',\cl) \subseteq \Tx(\mkvs_{\aexec'}, \vi') \label{equ:cp-sound-inv} 
\end{gather}
\begin{itemize}
\item the invariant \( I_2 \) implies the \cref{equ:cp-sound-update-so} where the proof is the same as \( \RYW \) in \cref{sec:sound-complete-ryw}.

\item For \cref{equ:cp-sound-update-arvis}, it is sufficient to prove one step inclusion, \ie
\[
    \begin{array}{l}
    \fora{\txid} (\txid, \txid_\cl^n) \in \left( (\SO_{\aexec'} \cup \WR_{\aexec'} ) ; \RW_{\aexec'}? \cup \WW_{\aexec'} \right) ; \VIS_{\aexec'} \\
    \qqquad \implies \txid \in \Tx(\mkvs, \vi) \cup \T_\rd \cup \T'_\rd 
\end{array}
\]
To prove above, let \( \T'_\rd \) initially be empty set.
We will add more read-only transactions until it satisfies \cref{equ:cp-sound-update-arvis}.
Assume a transaction \( \txid \) such that 
\( (\txid, \txid_\cl^n) \in \left( (\SO_{\aexec'} \cup \WR_{\aexec'} ) ; \RW_{\aexec'}? \cup \WW_{\aexec'} \right) ; \VIS_{\aexec'}\).
There exists a transaction \( \txid' \) such that \( \txid \toEdge{(\SO_{\aexec'} \cup \WR_{\aexec'} ) ; \RW_{\aexec'}? \cup \WW_{\aexec'}} \txid' \toEdge{\VIS_{\aexec'}}  \txid_\cl^n \).
It follows \( \txid'  \in \Tx(\mkvs, \vi) \cup \T_\rd \cup \T'_\rd  \).
Note that \( \txid \) and \( \txid' \) must exist in the abstract execution \( \aexec \) before update.
There are two cases: \( \txid' \) writes to at least a key; or \( \txid' \) is a read-only transaction.
\begin{itemize}
    \item
    If \( \txid' \) writes to at least a key, then \( \txid' \in \Tx(\mkvs, \vi)\).
    %\[
        %\begin{rclarray}
            %\func{RW^{-1}}{\mkvs, \ke, i} & \defeq & \Setcon{\txid}{\exsts{ j \leq i } \txid \in \RTx(\mkvs(\ke,j))} \\
            %\ddagger & \equiv &
            %\begin{array}[t]{@{}l@{}}
                %\fora{\ke, \ke', i, j, m, \txid, \txid', \txid''} \\
                %\left( \begin{array}{@{}l@{}}
                %i \in \vi(\ke) 
                %\land \txid \in \Set{\WTx(\mkvs(\ke,i))} \cup \func{RW^{-1}}{\mkvs, \ke, i} \land {} \\
                %\quad \left(
                    %\begin{array}{@{}l @{}}
                        %\left( \begin{array}{@{}l@{}}
                                %\txid' \in \func{SO^{-1}}{\txid} \land {} \\
                                %\txid' \in \Set{\WTx(\mkvs(\ke',j))} \cup  \RTx(\mkvs(\ke',j))
                        %\end{array} \right)  \lor {} \\
                        %\left( \begin{array}{@{}l@{}}
                                %\txid \in \RTx(\mkvs(\ke',j)) \land \txid' = \WTx(\mkvs(\ke',j))
                        %\end{array} \right)
                        %\end{array} \right) 
                    %\end{array}
                    %\right)  \\
                    %{} \lor \left( \begin{array}{@{}l@{}}
                            %i \in \vi(\ke) \land \ke = \ke' \land j < i
                    %\end{array} \right) \\
                    %\qquad \implies j \in \vi(\ke') 
            %\end{array} \\
        %\end{rclarray}
    %\]
    %We link the conditions in \( \ddagger \) to relation:
    %\begin{itemize}
        %\item \( \RW_\aexec\). Assume a key \( \ke \),  an index \( i \) and the writer \( \txid  = \WTx(\mkvs(\ke,i))\),
    %then \( \txid' \in \RW^{-1}(\mkvs, \ke, i)\) if and only if \( \txid' \toEdge{\RW_\aexec} \txid\).
        %\item \( \SO_\aexec\). The transaction identifiers encode the \( \SO_\aexec \).
        %That is, \( \txid' \in \SO^{-1}(\txid)\) if and only if \(\txid' \toEdge{\SO_\aexec} \txid \).
        %\item  \( \WR_\aexec \). It is easy to see \( \txid \in \RTx(\mkvs(\ke',j)) \land \txid' = \WTx(\mkvs(\ke',j)) \) if and only if \( \txid' \toEdge{\WR_\aexec} \txid \).
        %\item \( \WW_\aexec \). The write-write relation describes the order of write operations for a key which corresponds the version orders in key-value store.
        %That is, \( \txid' = \WTx(\mkvs(\ke,j)) \land \txid = \WTx(\mkvs(\ke,i)) \land j < i\) if and only if
        %\( \txid' \toEdge{\WW_\aexec} \txid\).
    %\end{itemize}
    %Let assume \( \txid' \) writes to i-\emph{th} version a key \( \ke \).
    %Given above and 
    %\[ \txid \toEdge{(\SO_{\aexec'} \cup \WR_{\aexec'} ) ; \RW_{\aexec'}? \cup \WW_{\aexec'}} \txid' \toEdge{\VIS_{\aexec'}}  \txid_\cl^n \] we can substitute and rewrite the \( \ddagger \) as the following:
    %\begin{gather}
        %\begin{array}{@{}l@{}}
            %\fora{\txid'',\ke',j}
            %\WTx(\mkvs(\ke,i)) = \txid' \land {} \\
            %\left( \begin{array}{@{}l@{}}
            %\txid'' \toEdge{\RW_{\aexec'}?} \txid' \land {} \\
            %\quad \left(
                %\begin{array}{@{}l @{}}
                    %\left( \begin{array}{@{}l@{}}
                            %\txid \toEdge{\SO_{\aexec'}} \txid'' \land 
                            %\txid \in \Set{\WTx(\mkvs(\ke',j))} \cup  \RTx(\mkvs(\ke',j))
                    %\end{array} \right)  \\
                    %{} \lor 
                    %\left( \begin{array}{@{}l@{}}
                            %\txid \toEdge{\WR_{\aexec'}} \txid'' \land \txid = \WTx(\mkvs(\ke',j))
                    %\end{array} \right)
                    %\end{array} \right) 
                %\end{array}
                %\right)  \\
                %{} \lor \left( \begin{array}{@{}l@{}}
                        %\txid \toEdge{\WW_{\aexec'}} \txid'' \land \txid = \WTx(\mkvs(\ke',j))
                %\end{array} \right) \\
                %\qquad \implies j \in \vi(\ke') 
        %\end{array} 
        %\label{equ:cp-dagger}
    %\end{gather}
    Now we perform case analysis if \( \txid \) is a read-only transaction.
    \begin{itemize}
        \item if \( \txid \) has write, we prove \( \txid \in \Tx(\mkvs, \vi)\).
        Recall the \( \ddagger \) is defined as the following:
        \begin{equation}
        \label{equ:cp-dagger}
        \ddagger  \equiv 
            \fora{\ke, \ke', i, j}
                i \in \vi(\ke)  \wedge \WTx(\hh(\ke', j)) \toEdge{(((\PO \cup \RF_{\hh}) ; \AD_{\hh}?) \cup \VO_{\hh})^{+}} \WTx(\hh(\ke, i))
            \implies j \in \vi(\ke')  
        \end{equation}
        Since \( \WR_\mkvs \), \( \WW_\mkvs \) and \( \RW_\mkvs \) coincide with
        \( \WR_\aexec \), \( \WW_\aexec \) and \( \RW_\aexec \) respectively.
        Also because \( \txid \) write to at least one key,
        it is easy to see there exists some version \( \ke'',m\) such that 
        \( \txid = \WTx(\mkvs(\ke'',m))\) and \( m \in \vi(\ke'')\).
        By definition of \( \Tx \), it follows \( \txid \in \Tx(\mkvs, \vi) \).
        %Therefore by the definition of \( \Tx \), then \( \txid \in \VIS^{-1}(\txid_\cl^n)\).
        \item if \( \txid \) is a read-only transaction, we add it into \( \T'_\rd \).
    \end{itemize}
    \item 
    if \( \txid' \) is a read-only transaction, then either \( \txid' \in \T_\rd \) or \( \txid' \in \T'_\rd \).
    More specifically we have three cases: \textbf{(i)} \( \txid' \in \bigcup\limits_{\{\txid_{\cl}^{i} \in \T_{\aexec} \mid i \in \Nat\}} \VIS_{\aexec}^{-1}(\txid^i_\cl) \), \textbf{(ii)} \( \txid' \in \bigcup\limits_{\{\txid_{\cl}^{i} \in \T_{\aexec} \mid i \in \Nat\}} (\SO_{\aexec}^{-1})?(\txid^i_\cl) \) or \textbf{(iii)} \( \txid' \in \T'_\rd\).
    \begin{itemize}
        \item
        Assume \( \txid' \in \bigcup\limits_{\{\txid_{\cl}^{i} \in \T_{\aexec} \mid i \in \Nat\}} \VIS_{\aexec}^{-1}(\txid^i_\cl) \).
        It means \( \txid' \) is visible for some previous transaction \( \txid_\cl^m \) (\( m < n \)) from the same client \( cl \), 
        \ie 
        \[ 
            \txid \toEdge{(\SO_{\aexec'} \cup \WR_{\aexec'} ) ; \RW_{\aexec'}? \cup \WW_{\aexec'}} \txid' \toEdge{\VIS_{\aexec'}}  \txid_\cl^m 
        \]
        Note that all the edges before \( \txid_\cl^m \) must exist in \( \aexec \).
        Since \( \aexec \) satisfies the \( \left( (\SO \cup \WR ) ; \RW? \cup \WW \right)^* ; \VIS_\aexec \subseteq \VIS_\aexec \),
        we have \( \txid \toEdge{\VIS_{\aexec'}} \txid_\cl^m \) and then \( \txid \in \bigcup\limits_{\{\txid_{\cl}^{i} \in \T_{\aexec} \mid i \in \Nat\}} \VIS_{\aexec}^{-1}(\txid^i_\cl)\).
        By the invariant \( I_1 \), it means \( \txid \in \Tx(\mkvs, \vi) \cup \T_\rd \).
    \item \( \txid' \in \bigcup\limits_{\{\txid_{\cl}^{i} \in \T_{\aexec} \mid i \in \Nat\}} \SO_{\aexec}^{-1}(\txid^i_\cl) \).
    Since \( \txid' \) is a read-only transaction, 
    the edges can be simplified to \( \txid \toEdge{(\SO_{\aexec'} \cup \WR_{\aexec'} )} \txid' \toEdge{\SO_{\aexec'}}  \txid_\cl^n \).
    Given that \( \SO \) is transitive, then  either \( \txid \toEdge{\SO_{\aexec'}} \txid_\cl^n \) or \( \txid \toEdge{\WR_{\aexec'} } \txid' \toEdge{\SO_{\aexec'}}  \txid_\cl^n \).
    \begin{itemize}
        \item \( \txid \toEdge{\SO_{\aexec'}} \txid_\cl^n \).
            It follows \( \txid \in \bigcup\limits_{\{\txid_{\cl}^{i} \in \T_{\aexec} \mid i \in \Nat\}} \SO_{\aexec}^{-1}(\txid^i_\cl) = \Tx(\mkvs, \vi) \cup \T_\rd \).
        \item \( \txid \toEdge{\WR_{\aexec'} } \txid' \toEdge{\SO_{\aexec'}}  \txid_\cl^n \).
            The \( \WR \) edge must exists in \( \aexec \).
            Because \( \WR_\aexec \subseteq \VIS_\aexec \) then  \( \txid \toEdge{\VIS_{\aexec} } \txid' \toEdge{\SO_{\aexec'}}  \txid_\cl^n  \).
            It means 
            \[ 
                \txid \in \bigcup\limits_{\{\txid_{\cl}^{i} \in \T_{\aexec} \mid i \in \Nat\}} \VIS_{\aexec}^{-1}(\txid^i_\cl) = \Tx(\mkvs, \vi) \cup \T_\rd 
            \]
    \end{itemize}
    \item 
    Last, \( \txid' \in \T'_\rd \).
    Since \( \T'_\rd \) initially is empty set, there exists another write transaction \( \txid'' \) such that:
    \[
        \txid \toEdge{(\SO_{\aexec'} \cup \WR_{\aexec'} ) ; \RW_{\aexec'}? \cup \WW_{\aexec'}} \txid' \toEdge{(\SO_{\aexec'} \cup \WR_{\aexec'} ) ; \RW_{\aexec'}? \cup \WW_{\aexec'}} \txid'' \toEdge{\VIS_{\aexec'}}  \txid_\cl^n
    \]
    %Given that \( \txid' \) is a read-only and \( \txid'' \) has write, the edges can be simplified:
    %\[
        %\txid \toEdge{(\SO_{\aexec'} \cup \WR_{\aexec'} )} \txid' \toEdge{\SO_{\aexec'} ; \RW_{\aexec'}?} \txid'' \toEdge{\VIS_{\aexec'}}  \txid_\cl^n
    %\]
    %Because transitivity of  \( \SO \), we have the following two cases:
    %\[
        %\begin{array}{@{}l@{}}
            %\txid \toEdge{ \WR_{\aexec'} } \txid' \toEdge{\SO_{\aexec'} ; \RW_{\aexec'}?} \txid'' \toEdge{\VIS_{\aexec'}}  \txid_\cl^n \\
            %\txid \toEdge{\SO_{\aexec'} ; \RW_{\aexec'}?} \txid'' \toEdge{\VIS_{\aexec'}}  \txid_\cl^n 
        %\end{array}
    %\]
    %\( \txid \toEdge{ \WR_{\aexec'} } \txid' \toEdge{\SO_{\aexec'} ; \RW_{\aexec'}?} \txid'' \toEdge{\VIS_{\aexec'}}  \txid_\cl^n \).
        If \( \txid \) has write, by \cref{equ:cp-dagger} then \( \txid \in \Tx(\mkvs,\vi) \).
        Otherwise if \( \txid \) is a read only transaction, we add it into \( \T'_\rd \).
            %\( \txid \toEdge{\SO_{\aexec'} ; \RW_{\aexec'}?} \txid'' \toEdge{\VIS_{\aexec'}}  \txid_\cl^n \).
            %Similarly by \cref{equ:cp-dagger}, either \( \txid \in \Tx(\mkvs,\vi) \)  or we add it into \( \T'_\rd \).
    \end{itemize}
\end{itemize}

\item Since \( \CP \) satisfies \( \RYW \) and \( \MRd \), thus invariants \( I_1 \) and  \( I_2 \) are preserved after update.

\end{itemize}

    
For completeness, we prove the three parts of the execution test separately.
\begin{itemize}
\item Since \( \SO_\aexec \subseteq \VIS_\aexec  \), the prove for \( \ET_\RYW \) is the as in \cref{sec:sound-complete-mr}.
\item For any \( \VIS_\aexec \)  satisfies the constraint for \( \CP \), by \cref{lem:cp-eauiv-spec} it satisfies that 
\[
    \VIS \defeq \left( (\SO \cup \WR ) ; \RW? \cup \WW \cup R \right)^* ; (\SO \cup \WR )
\]
for some relation \( R \).
It means \( \VIS_\aexec ; \SO_\aexec \subseteq \VIS_\aexec \).
Therefore it is complete with respect to \( \ET_\MRd \).

\item Let consider the \( \ddagger \).
Assume i-\emph{th} transaction \( \txid_i \) in the arbitrary order,
and let view \( \vi_{i} = \getView(\aexec, \VIS^{-1}_{\aexec}(\txid_{i}) ) \).
We also pick any final view such that \( \vi'_{i} \subseteq \getView(\aexec, (\AR^{-1}_{\aexec})?(\txid_{i}) ) \).
Note that there is nothing to prove for \( \vi'_i \) since the \( \ddagger \) does not constrain the \( \vi'_i \).
Recall the \( \ddagger \):
\[
\ddagger  \equiv 
        \fora{\ke, \ke', m, j}
             m \in \vi(\ke)  \wedge \WTx(\hh(\ke', j)) \toEdge{(((\PO \cup \RF_{\hh}) ; \AD_{\hh}?) \cup \VO_{\hh})^{+}} \WTx(\hh(\ke, m))
         \implies j \in \vi(\ke')  
\]
Assume \( j \in \vi_i(\ke) \) for some key \(\ke \) and index \( i \).
It means the writer of the version is visible by the transaction \( \txid_i\),
\ie \( \WTx(\mkvs(\ke,i)) \in \VIS^{-1}_{\aexec}(\txid_{i}) \).
Let the \( \mkvs = \mkvs_{\cut(\aexec, i-1)} \).
We need to prove the following:
\begin{gather}
    \label{equ:cp-complete-arvis}
    \begin{array}{@{}l@{}}
        \fora{\ke, \ke', m, j, \txid, \txid'} 
        m \in \vi(\ke) 
        \land \WTx(\mkvs(\ke,m)) \in \VIS_\aexec^{-1}(\txid_i) \\
        \quad {} \land \WTx(\hh(\ke', j)) \toEdge{(((\PO \cup \RF_{\hh}) ; \AD_{\hh}?) \cup \VO_{\hh})^{+}} \WTx(\hh(\ke, m)) \\
            \qquad \implies \WTx(\mkvs(\ke',j)) \in \VIS_\aexec^{-1}(\txid_i)
    \end{array}
\end{gather}
%Note that \( \txid \in \Set{\WTx(\mkvs(\ke,i))} \cup \func{RW^{-1}}{\mkvs, \ke, i} \) 
%means \( \txid \toEdge{\RW_{\aexec}?} \WTx(\mkvs(\ke,i)) \),
%the formulae \(\left( \begin{array}{@{}l@{}} \txid \in \RTx(\mkvs(\ke',j)) \land \txid' = \WTx(\mkvs(\ke',j)) \end{array} \right) \) 
%means \( \txid \toEdge{\WR_\aexec} \txid' \),
%and \( \left( \begin{array}{@{}l@{}} \txid = \WTx(\mkvs(\ke',m)) \land \txid' = \WTx(\mkvs(\ke',j)) \land m > j \end{array} \right) \) 
%means \( \txid \toEdge{\WW_\aexec} \txid' \).
%Given all the correspondence, the \cref{equ:cp-complete-arvis} holds if the following holds:
%\[
    %\begin{rclarray}
        %\begin{array}[t]{@{}l@{}}
            %\fora{\ke, \ke', i, j, \txid, \txid'} \\
            %\left( \begin{array}{@{}l@{}}
            %i \in \vi(\ke) 
            %\land \WTx(\mkvs(\ke,i)) \in \VIS_\aexec^{-1}(\txid_i) \\
            %{} \land \txid' = \WTx(\mkvs(\ke',j))
            %\land \txid \toEdge{\RW_{\aexec}?} \WTx(\mkvs(\ke,i)) \land {} \\
            %\left(
                %\begin{array}{@{}l @{}}
                    %\txid' \toEdge{\WR_\aexec ; \SO_\aexec}\txid \lor
                    %\txid' \toEdge{\SO_\aexec}\txid \lor
                    %\txid' \toEdge{\WR_\aexec}\txid
                    %\end{array} \right) 
                %\end{array}
                %\right)  \\
                %{} \lor \txid' \toEdge{\WW_\aexec} \WTx(\mkvs(\ke,i)) \\
                %\qquad \implies \txid' \in \VIS_\aexec^{-1}(\txid_i)
        %\end{array} \\
    %\end{rclarray}
%\]
%Then the above holds, if the following holds:
%\begin{gather}
    %\label{equ:cp-complete-arvis-2}
    %\begin{rclarray}
        %\begin{array}[t]{@{}l@{}}
            %\fora{\ke, i, \txid} \\
            %\left( \begin{array}{@{}l@{}}
            %i \in \vi(\ke) 
            %\land \WTx(\mkvs(\ke,i)) \in \VIS_\aexec^{-1}(\txid_i) \\
            %{} \land \txid \toEdge{( (\WR_\aexec; \SO_\aexec) \cup \SO_\aexec \cup \WR_\aexec) ; \RW_{\aexec}? \cup \WW_\aexec} \WTx(\mkvs(\ke,i)) 
                %\end{array}
                %\right)  \\
                %\qquad \implies \txid \in \VIS_\aexec^{-1}(\txid_i)
        %\end{array} \\
    %\end{rclarray}
%\end{gather}
Since \( \WR_\mkvs \), \( \WW_\mkvs \) and \( \RW_\mkvs \) coincide with
\( \WR_\aexec \), \( \WW_\aexec \) and \( \RW_\aexec \) respectively,
and \( \left( (\SO \cup \WR ) ; \RW? \cup \WW \right)^* ; \VIS_\aexec \subseteq \VIS_\aexec \),
It implies \cref{equ:cp-complete-arvis}.
\end{itemize}

\subsection{Parallel Snapshot Isolation \(\PSI\)}
\label{sec:sound-complete-psi}

The axiomatic definition for \( \PSI \) is 
\[ 
    (\RP_{\LWW}, \Set{\lambda \aexec. \VIS_{\aexec} ; \VIS_{\aexec}, \lambda \aexec \ldotp \SO_\aexec, \lambda \aexec. \WW_\aexec })
\]
Given the definition, there is a corresponding definition on dependency graph by solve the following inequalities:
\[
    \begin{array}{@{}l@{}}
        \WR \subseteq \VIS \\
        \WW \subseteq \VIS \\
        \SO \subseteq \VIS \\
        \VIS ; \VIS \subseteq \VIS 
    \end{array}
\]
We have \( \VIS = (\WR \cup \WW \cup \SO \cup R)^{+} \) for some \( R \subseteq \AR \).
Thus, there exist a minimum visibility such that 
\[ 
    (\RP_{\LWW}, \Set{\lambda \aexec. (\WR_{\aexec} \cup \WW_{\aexec} \cup \SO ) ; \VIS_{\aexec}, \lambda \aexec \ldotp \SO_\aexec, \lambda \aexec. \WW_\aexec })
\]

To prove soundness, we pick an invariant for the \( \ET_\PSI \) as the union of those for \( \MR\) and \( \RYW \) shown in the following:
\begin{align*}
    I_1(\aexec, \cl) & =  \left( \bigcup_{\Set{\txid_{\cl}^{n} \in \txidset_{\aexec} }[ n \in \Nat ]} \VIS_{\aexec}^{-1}(\txid^n_\cl) \right) \setminus \txidset_\rd \\
    I_2(\aexec, \cl) & =  \left( \bigcup_{\Set{\txid_{\cl}^{n} \in \txidset_{\aexec} }[ n \in \Nat ]} (\SO_{\aexec}^{-1})\rflx(\txid^n_\cl) \right) \setminus \txidset_\rd
\end{align*}
where \( \txidset_\rd \) is all the read-only transactions included in both 
\( \left( \bigcup_{\Set{\txid_{\cl}^{n} \in \txidset_{\aexec} }[ n \in \Nat ]} \VIS_{\aexec}^{-1}(\txid^n_\cl) \right)\) 
and \( \left( \bigcup_{\Set{\txid_{\cl}^{n} \in \txidset_{\aexec} }[ n \in \Nat ]} (\SO_{\aexec}^{-1})\rflx(\txid^n_\cl) \right) \).
Assume a kv-store $\mkvs$, an initial and a final view $\vi, \vi'$  a fingerprint $\fp$ 
such that $\ET_{\PSI} \vdash (\mkvs, \vi) \csat \fp: (\mkvs',\vi')$. 
Also choose an arbitrary $\cl$, a transaction identifier $\txid_\cl^n \in \nextTxid(\mkvs, \cl)$, 
and an abstract execution $\aexec$ such that $\mkvs_{\aexec} = \mkvs$ and 
\( I_1(\aexec, \cl) \cup I_2(\aexec, \cl) \subseteq \Tx[\mkvs, \vi] \).
We are about to prove there exists an extra set of read-only transactions \( \txidset'_\rd \) such that
the new abstract execution \( \aexec' = \extend[\aexec, \txid_\cl^n, \fp, \Tx[\mkvs, \vi] \cup \txidset_\rd \cup \txidset'_\rd] \) and:
\begin{gather}
    \fora{\txid} (\txid, \txid_\cl^n) \in \SO_{\aexec'} \implies \txid \in \Tx[\mkvs, \vi] \cup \txidset_\rd \cup \txidset'_\rd \label{equ:psi-sound-update-so}\\
    \fora{\txid} (\txid, \txid_\cl^n) \in \WW_{\aexec'} \implies \txid \in \Tx[\mkvs, \vi] \cup \txidset_\rd \cup \txidset'_\rd \label{equ:psi-sound-update-ua}\\
    \fora{\txid} (\txid, \txid_\cl^n) \in ( \SO_{\aexec'} \cup \WR_{\aexec'} \cup \WW_{\aexec'} )^{+} ; \VIS_{\aexec'} \implies \txid \in \Tx[\mkvs, \vi] \cup \txidset_\rd \cup \txidset'_\rd \label{equ:psi-sound-update-closure}\\
    I_1(\aexec',\cl) \cup I_2(\aexec',\cl) \subseteq \Tx[\mkvs_{\aexec'}, \vi'] \label{equ:psi-sound-inv} 
\end{gather}
\begin{itemize}
\item The invariant \( I_2 \) implies \cref{equ:psi-sound-update-so} as the same as \( \RYW \) in \cref{sec:sound-complete-ryw}.
\item Since \( \PSI \) also satisfies \( \UA \), the \cref{equ:si-sound-update-ww} can be proven as the same as \( \UA \) in \cref{sec:sound-complete-ua}.
\item \cref{equ:psi-sound-update-closure}.
    Note that \( (\txid, \txid_\cl^n) \in ( \SO_{\aexec'} \cup \WR_{\aexec'} \cup \WW_{\aexec'}); \VIS_{\aexec'} \implies (\txid, \txid_\cl^n) \in ( \SO_{\aexec} \cup \WR_{\aexec}  \cup \WW_{\aexec} ) ; \VIS_{\aexec'}\).
    Also, recall that \( \SO_\aexec = \SO_\mkvs \), \( \WR_\aexec = \WR_\mkvs \) and  \( \WW_\aexec = \WW_\mkvs \).
    Let \( \txidset'_\rd = \lfpTx[\mkvs,\vi,\SO_{\mkvs} \cup \WR_{\mkvs} \cup \WW_{\mkvs}] \). 
    This means that \( \aexec' = \extend[\aexec, \txid_\cl^n, \fp, \lfpTx[\mkvs, \vi, \SO_{\mkvs} \cup \WR_{\mkvs}] \cup \txidset_\rd ] \).
    Let assume \( \txid \toEDGE{\SO_{\mkvs} \cup \WR_{\mkvs} \cup \WW_{\mkvs}} \txid' \) and \( \txid' \in \lfpTx[\mkvs, \vi, \SO_{\mkvs} \cup \WR_{\mkvs}] \cup \txidset_\rd \).
    We have two possible cases:
    \begin{itemize}
        \item If \( \txid' \in \lfpTx[\mkvs, \vi, \SO_{\mkvs} \cup \WR_{\mkvs} \cup \WW_{\mkvs}] \), by  \cref{thm:view-vis-relation}, we know \( \txid \in \lfpTx[\mkvs, \vi, \SO_{\mkvs} \cup \WR_{\mkvs} \cup \WW_{\mkvs}] \).
        \item If \( \txid' \in \txidset_\rd \), there are two cases:
        \begin{itemize}
            \item \( \txid' \in  \left( \bigcup_{\Set{\txid_{\cl}^{n} \in \txidset_{\aexec} }[ n \in \Nat ]} \VIS_{\aexec}^{-1}(\txid^n_\cl) \right) \).
                Since \( \txid' \) is a read-only transaction, it means \( \txid \toEDGE{\SO_{\mkvs} \cup \WR_{\mkvs} } \txid' \).
                By the property of \( \aexec \) (before update) that \( \SO \cup \WR_\aexec \in \VIS_\aexec \), it is known that \( \txid \in \left( \bigcup_{\Set{\txid_{\cl}^{n} \in \txidset_{\aexec} }[ n \in \Nat ]} \VIS_{\aexec}^{-1}(\txid^n_\cl) \right) \), that is, \( \txid \in \Tx[\mkvs,\vi] \cup \txidset_\rd\).

            \item \( \txid' \in  \left( \bigcup_{\Set{\txid_{\cl}^{n} \in \txidset_{\aexec} }[ n \in \Nat ]} \SO_{\aexec}^{-1}(\txid^n_\cl) \right) \).
                Given that \( \txid' \) is a read only transaction, we know \( \txid \in (\SO \cup \WR_\aexec)^{-1} \left( \bigcup_{\Set{\txid_{\cl}^{n} \in \txidset_{\aexec} }[ n \in \Nat ]} \SO_{\aexec}^{-1}(\txid^n_\cl) \right) \).
                By the property of \( \aexec \) (before update) that \( \SO \cup \WR_\aexec \in \VIS_\aexec \),
                it follows:
                \begin{align*}
                    \txid & \in VIS_\aexec^{-1} \left( \bigcup_{\Set{\txid_{\cl}^{n} \in \txidset_{\aexec} }[ n \in \Nat ]} \SO_{\aexec}^{-1}(\txid^n_\cl) \right) \\
                          & = \left( \bigcup_{\Set{\txid_{\cl}^{n} \in \txidset_{\aexec} }[ n \in \Nat ]} \VIS_{\aexec}^{-1}(\txid^n_\cl) \right)  \\
                          & = \Tx[\mkvs,\vi] \cup \txidset_\rd
                \end{align*}
                
        \end{itemize}
    \end{itemize}
\item Finally the new abstract execution preserves the invariant \( I_1 \) and \( I_2 \) 
because  \( \CC \) satisfies \( \MW \) and \( \RYW \).
\end{itemize}

Given that \( \VIS_\aexec = (\WR_\aexec \cup \WW_\aexec \cup \SO_\aexec \cup R)^{+} \),
we know \( \VIS_\aexec ; \SO_\aexec \subseteq \VIS_\aexec \).
First the completeness follows \( \MR \) in \cref{sec:sound-complete-mr}, \( \RYW \) in \cref{sec:sound-complete-ryw} and  \( \UA \) in \cref{sec:sound-complete-ua}.
Similarly, by \cref{lem:aexec-spec-cc},

An abstract execution \( \aexec \) satisfies snapshot isolation (\(\SI\)), 
if it satisfies
\( \{\lambda \aexec. \AR[\aexec] ; \VIS[\aexec], \lambda \aexec \ldotp \SO, \allowbreak \lambda \aexec. \WW[\aexec] \}) \),
which is intersection of \( \CP \) and \( \UA \) on abstract executions \citep{SIanalysis}.
\citet{SIanalysis} also proposed the minimum visibility relation that gives rise of the following equivalent definition
\[
    \visaxioms[\SI] \FuncDef
    \Set{\lambda \aexec. \left( (\WR[\aexec]  \cup \SO \cup \WW[\aexec] ) ; \Refl(\RW[\aexec]) \right) ; \VIS[\aexec]
            ,\lambda \aexec \ldotp \SO, \lambda \aexec \ldotp \WW[\aexec] }  .
\]

The execution test \( \et[\CP] \) is sound with respect to the axiomatic definition \( \visaxioms[\SI] \)
We pick the invariant \( \aexecinv[\CP] = \aexecinv[\RYW]\).
\SOUNDLET{\CP}{ \txidsetrd \supseteq
\begin{multlined}[t]
\left( \bigcup_{\Set{\txid[\cl](\idx) | \txid[\cl](\idx) \in \aexec}} 
\VISInv[\aexec](\txid[\cl](\idx)) \cup \Refl((\Inv(\SO)))(\txid[\cl](\idx)) \right) 
\setminus \Set{\txid' | \Forall{l | \key | \val } (l,\key,\val) \in \aexec(\txid') \implies l = \opR } .
\end{multlined} }
Assume 
\[ 
\txidsetrd' = 
\begin{multlined}[t]
\left( \bigcup_{\Set{\txid[\cl](\idx) | \txid[\cl](\idx) \in \aexec}} 
\VISInv[\aexec](\txid[\cl](\idx)) \cup \Refl((\Inv(\SO)))(\txid[\cl](\idx)) \right) 
\setminus \Set{\txid' | \Forall{l | \key | \val } (l,\key,\val) \in \aexec(\txid') \implies l = \opR } .
\end{multlined} 
\]
and \( \txidsetrd'' = \txidsetrd \setminus \txidsetrd' \).
By the definition of soundness, we prove the following result:
\begin{Formulae}
& \begin{Formula}
    \Inv(\SO)(\txid) \subseteq \txidset \cup \txidsetrd' 
    \label{equ:si-sound-update-so}
\end{Formula}
\\ & \begin{Formula}
    \Inv(\WW)(\txid) \subseteq \txidset 
    \label{equ:si-sound-update-ua}
\end{Formula}
\\ & \begin{Formula}
    \Inv(\left( (\WR[\aexec] \cup \SO \cup \WW[\aexec] ) ; \Refl(\RW[\aexec]) \right)) (\txid) 
            \subseteq \txidset \cup \txidsetrd' \cup \txidsetrd''
     \label{equ:si-sound-update-closure}
\end{Formula}
\\ & \begin{Formula}
    \aexecinv[\PSI](\aexec',\cl) \subseteq \VisTrans(\XToK(\aexec'),\vi')
    \label{equ:si-inv-preserve}
\end{Formula}
\end{Formulae}
\Cref{equ:si-sound-update-so,equ:si-sound-update-ua} 
can be proven in the same way as in \cref{sec:sound-complete-mr,sec:sound-complete-ua} respectively.
We now prove \cref{equ:si-sound-update-closure}.
Initially we take \( \txidsetrd'' \) to be an empty set.
Note that \(\VISInv[\aexec'](\txid) = \txidset \cup \txidsetrd' \cup \txidsetrd'' \).
By \cref{thm:view-vis-relation,equ:view-close-to-aexec}, there exists \( \txidsetrd'' \) such that
\( \txidset \cup \txidsetrd'' \) is closed under \( \left( (\WR[\aexec] \cup \SO \cup \WW[\aexec] ) ; \Refl(\RW[\aexec]) \right)\).
Now consider a transaction \( \txidrd \in \txidsetrd' \) and
assume a transaction \( \txid' \) such that \( \ToEdge{ \txid' | (\WR[\aexec] \cup \SO \cup \WW[\aexec] ) ; \Refl(\RW[\aexec]) -> \txidrd } \).
Since \( \txidrd \) is a read-only transaction, thus
\( \ToEdge{ \txid' |  (\SO \cup \WR[\aexec] )  -> \txidrd } \)
and the rest proof is exactly the same as as in \cref{sec:sound-complete-cc}.
Last, \cref{equ:si-inv-preserve} can be proven in the same way as in \cref{sec:sound-complete-mr,sec:sound-complete-ryw}.

The execution test $\et[\SI]$ is complete with respect to the axiomatic definition \( \visaxioms[\SI] \).
By \citet{SIanalysis}, it suffices to prove completeness with respect to the following definition,
\[
\Set{\lambda \aexec \ldotp \AR[\aexec] ; \VIS[\aexec], \lambda \aexec \ldotp \SO , \lambda \aexec \ldotp \WW[\aexec] } .
\]
\COMPLETELET{\SI}
By the definition of \( \et[\SI]\), we prove \( \CanCommit[\SI] \), \( \ViewShift[\MR]\) and \( \ViewShift[\RYW]\) respectively.
Recall that \( \CanCommit[\SI] = \PreClosed(\kvs,\vi,\rel[\UA] \cup \left( (\WR[\aexec] \cup \SO \cup \WW[\aexec] ) ; \Refl(\RW[\aexec]) \right)) \).
It is easy to see that 
\[
\begin{multlined}
\PreClosed(\kvs,\vi,\rel[\UA] \cup \WR[\kvs] \cup \SO \cup \WW[\kvs]) \iff {}
    \\ \PreClosed(\kvs,\vi,\rel[\UA]) 
    \land \PreClosed(\kvs,\vi,\left( (\WR[\aexec] \cup \SO \cup \WW[\aexec] ) ; \Refl(\RW[\aexec]) \right)) . 
\end{multlined}
\]
The predicate \( \PreClosed(\kvs,\vi,\rel[\UA]) \) can be proven in the same way as in \cref{sec:sound-complete-ua}
Because
\begin{align*}
\left( (\WR[\aexec] \cup \SO \cup \WW[\aexec]) ; \Refl(\RW[\aexec])  \right) ; \VIS[\aexec] 
        & \subseteq \left( \VIS[\aexec] ; \Refl(\RW[\aexec]) \right) ;  \VIS[\aexec]
        & \subseteq  \AR[\aexec] ; \VIS[\aexec] \subseteq \VIS[\aexec] .
\end{align*}
Then \( \PreClosed(\kvs,\vi,\left( (\WR[\aexec] \cup \SO \cup \WW[\aexec] ) ; \Refl(\RW[\aexec]) \right)) \)
can be derived from \cref{thm:view-vis-relation,equ:aexec-close-to-view}.
The predicate \( \ViewShift[\RYW] \) can be proven in the same way as in \cref{sec:sound-complete-ryw}.
Since \( \VIS[\aexec] ; \SO \subseteq \AR[\aexec] ; \VIS[\aexec] \subseteq \VIS[\aexec] \)
\( \ViewShift[\MR] \) can be proven in the same way as in \cref{sec:sound-complete-mr}.

\paragraph{Strict serialisability (\(\SER\))}  
This model is the strongest consistency model
in any framework that abstracts from aborted transactions, 
requiring that transactions execute in a total sequential order.
The \(\CanCommit[\SER]\) thus allows a client to commit a transaction only 
when the client view on the kv-store is complete
in that the view is closed with respect to \(\WWInv[\kvs]\). 
This requirement prevents the kv-store in  \cref{fig:ser-disallowed}.
Without loss of generality, suppose that \(\txid\) commits before \(\txid'\),
then the client committing \(\txid'\) must see the version of \(\key_1\) written by \(\txid\), 
and thus cannot read the outdated value \(\val_0\) for \(\key_1\). 
This example, known as \emph{write skew anomaly}, 
is allowed by all other execution tests in \cref{fig:execution-tests}.

\begin{figure}
\centering
\begin{tikzpicture}%
\KVMapping{x}{\key_1}{
    /\val_0/\txid_0/\Set{\boldsymbol{\txid'}}
    , /\val_1/\txid/\emptyset
};
\KVMapping[x]{y}{\key_2}{
    /\val_0/\txid_0/\Set{\txid}
    , /\val_2/\boldsymbol{\txid'}/\emptyset
};
\end{tikzpicture}%

\hrulefill

\caption{Write skew anomaly, disallowed by \(\SER\)}
\label{fig:ser-disallowed}
\end{figure}%


\subsection{Soundness and Completeness Constructor}
\label{sec:kv2aexec-sound-complete}

We now show how all the results illustrated so far 
can be put together to show that the kv-store operational semantics 
is sound and complete with respect to abstract execution operational semantics.

\subsubsection{Soundness}
Recall that in the abstract execution operational semantics,
a client \( \cl \) loses information of the visible transactions immediately after it commits a transaction.
Yet such information is indirectly presented when the next transaction from the same client is committed.
To define the soundness judgement (\cref{def:et_sound}), we introduce a notation of \emph{invariant} ({def:invariant-for-clients})
to encore constraints on the visible transactions after each commit.

\ac{The idea behind client-based invariant being that $I(\aexec, \cl)$ represents 
the minimal set of transactions that $\cl$ must see in $\aexec$, before 
updating the view and performing a transaction. Such a set of transaction 
roughly correspond to the view of the client before performing a 
sequence of \emph{update view+execute transaction} operations, 
or equivalently from the view obtained after the execution of the 
last transaction from that client.}

\begin{definition}[Invariant for clients]
\label{def:invariant-for-clients}
A \emph{client-based invariant condition}, or simply \emph{invariant}, is a 
function $I : \Aexecs \times \Clients \rightarrow \powerset{\TxID}$ 
such that for any $\cl$ we have that $I(\aexec, \cl) \subseteq \T_{\aexec}$, and 
for any  $\cl'$ such that $\cl' \neq \cl$ we have that 
$I(\extend(\aexec, \txid_{\cl'}^{\cdot}, \stub, \stub), \cl) = I(\aexec, \cl)$.
\end{definition}



\begin{definition}[Soundness judgement]
\label{def:et_sound}
An execution test $\ET$ is sound with respect to an axiomatic 
definition $(\RP_{\LWW}, \Ax)$ if and only if
there exists an invariant condition $I$ such that 
if assuming that
\begin{itemize}
    \item a client \( \cl \) having an initial view \( \vi \), 
        commits a transaction \( \txid \) with a fingerprint \( \fp \) and updates the view to \( \vi' \), 
        which is allowed by \( \ET \) \ie $\ET \vdash (\mkvs, \vi) \csat \fp: (\mkvs',\vi')$ where \( \mkvs' = \updKV{\mkvs, \vi ,\fp, \txid}\),
    \item a $\aexec$ such that $\hh_{\aexec} = \mkvs$ and $I(\aexec, \cl) \subseteq \Tx(\mkvs, \vi)$,
\end{itemize}
then there exist a set of read-only transactions $\T_{\rd}$ such that 
\begin{itemize}
\item the view \( \vi \) satisfies \( \Ax \), \ie $\forall \A \in \Ax. \Setcon{\txid' }{ (\txid', \txid) \in \A(\aexec')} \subseteq \Tx(\mkvs, \vi) \cup \T_{\rd}$, 
\item the invariant is preserved, \ie $I(\aexec', \cl) \subseteq \Tx(\mkvs', \vi')$ for some \( \aexec' \) that \( \mkvs' = \hh_{\aexec'}\)
\end{itemize}
\end{definition}

\begin{theorem}[Soundness]
\label{thm:et_soundness}
If $\ET$ is sound with respect to $(\RP_{\LWW}, \Ax)$, then 
\[
    \CMs(\ET) \subseteq \Setcon{ \mkvs }{ \exists \aexec \in \CMa(\RP_{\LWW}, \Ax)).\;\hh_{\aexec} = \mkvs}
\]
\end{theorem}
\begin{proof}
Let $\ET$ be an execution test that is sound with respect to an 
axiomatic definition $(\RP_{\LWW}, \Ax)$. Let $I$ be 
the invariant that satisfies \cref{def:et_sound}. 
Let consider an $\ET$-trace $\tr$.
Because of \cref{prop:et.normalform}, we can assume that $\tr$ is in normal form. 
Without lose generality, we can also assume that the trace does not have transitions labelled as $(\stub, \emptyset)$.
Thus we have that the following trace \( \tr \):
\[
\begin{rclarray}
\tr & = & (\hh_{0}, \viewFun_{0}) \xrightarrowtriangle{(\cl_{0}, \varepsilon)} (\hh_{0}, \viewFun_{0}') 
\xrightarrowtriangle{(\cl_{0}, \fp_{0})} 
(\hh_1, \viewFun_{1}) \xrightarrowtriangle{(\cl_1, \varepsilon)}  \cdots
\xrightarrowtriangle{(\cl_{n-1}, \fp_{n-1})} (\hh_{n}, \viewFun_{n}).
\end{rclarray}
\]
For any $i : 0 \leq i \leq n$, let $\tr_{i}$ be the prefix of $\tr$ that 
contains only the first $2i$ transitions. 
Clearly $\tr_{i}$ is a valid $\ET$-trace, and it is also a $\ET_{\top}$-trace. 
By \cref{prop:kvtrace2aexec}, 
any abstract execution $\aexec_{i} \in \aeset(\tr_{i})$ satisfies the last write wins policy. 
We show by induction on $i$ that we can always find 
an abstract execution $\aexec_{i} \in \aeset(\tr_{i})$ such that $\aexec_i \models \Ax$ and $I(\aexec_{i}, \cl) \subseteq \T^{i}_{\cl}$
for any client $\cl$ and set of transactions 
$\T^{i}_{\cl} = \Tx(\aexec_{i}, \viewFun_{i}(\cl)) \cup \T^{i}_\rd$, 
and read-only transactions $\T_\rd^{i}$ in $\aexec_{i}$.
If so, because $\aexec_{i}$ satisfies the last write wins policy,
then it must be the case that $\aexec_{i} \models (\RP_{\LWW}, \Ax)$. 
Then by choosing $i = n$, we will obtain that $\aexec_{n} \models (\RP_{\LWW}, \Ax)$. 
Last, by \cref{prop:kvtrace2aexec}, $\hh_{\aexec_{n}} = \hh_{n}$, and there is nothing left to prove.
Now let prove such $\aexec_{i} \in \aeset(\tr_{i})$ always exists.

\caseB{$i = 0$} 
Let $\aexec_{0}$ be the only abstract execution included in $\aeset(\tr_{0})$, 
that is $\aexec_{0} = ([], \emptyset, \emptyset)$. 
For any $\A \in \Ax$, it must be the case that 
$\A(\aexec_{0}) \subseteq \T_{\aexec_{0}} = \emptyset$, 
hence the inequation $\A(\aexec_{0}) \subseteq \VIS_{\aexec_{0}}$ is trivially satisfies.
Furthermore, for the client invariant $I$ we also require that $I(\aexec_{0}, \stub) \subseteq \T_{\aexec_{0}} = \emptyset$; 
for any client $\cl$ we can choose $\T_{\cl}^{0} = \Tx(\viewFun_{0}(\cl)) \cup \emptyset = \emptyset$. 
Therefore $I(\aexec_{0}, \cl) = \emptyset \subseteq \emptyset = \T_{\cl}^{0}$.

\caseI{$i' = i + 1$ where $i < n$}
By the inductive hypothesis, there exists an abstract execution $\aexec_i$ such that  
\begin{itemize}
\item $\aexec_{i} \models \A$ for all $\A \in \Ax$, and 
\item $I(\aexec, \cl) \subseteq \T_{\cl}^{i}$ for any client $\cl$ and set of transactions $\T_{\cl}^{i} = \Tx(\hh_{i}, \viewFun_{i}(\cl))$.
\end{itemize}

We have two transitions to check, the view shift and committing a transaction.
\begin{itemize}
\item the view shift transition $(\hh_{i}, \viewFun_{i}) \xrightarrowtriangle{(\cl_{i}, \varepsilon)} (\hh_{i}, \viewFun'_{i})$. 
By definition, it must be the case that $\viewFun'_{i} = \viewFun_{i}\rmto{\cl}{\vi'_{i}}$ 
for some $\vi'_{i}$ such that $\viewFun_{i}(\cl) \viewleq \vi'_{i}$.
Let $(\T_{\cl}^{i})' = \Tx(\hh_{i}, \vi'_{i})$; then we have 
\(
\T_{\cl}^{i} = \Tx(\hh_{i}, \viewFun_{i}(\cl)) \subseteq \Tx(\hh_{i}, \vi'_{i}) = (\T_{\cl}^{i})'
\)
As a consequence, $I(\aexec, \cl) \subseteq \T_{\cl}^{i} \subseteq (\T_{\cl}^{i})'$.

\item the commit transaction transition $(\hh_{i}, \viewFun_{i}') \xrightarrowtriangle{(\cl_{i}, \fp_{i})}_{\ET} 
(\hh_{i+1}, \viewFun_{i+1})$.
A necessary condition for this transition 
to appear in $\tr$ is that $\ET \vdash (\hh_{i}, \viewFun(\cl)) \triangleright \fp_{i}: (\mkvs_{i+1},\viewFun_{i+1}(\cl))$. 
Because $I$ is the invariant to derive that $\ET$ is sound with respect to $\Ax$, 
and because $I(\aexec_{i}, \cl_{i}) \subseteq (\T^{i}_{\cl})'$, 
then by \cref{def:et_sound} we have the following:
\begin{itemize}
\item there exists a set of read-only transactions $\T_\rd$ 
    such that 
    \[
        \Setcon{\txid' }{ (\txid', \txid_{(\cl, i)}) \in \A(\aexec_{i+1})} \subseteq (\T^{i}_{\cl})' \cup \T_\rd
    \]
where 
$\txid_{(\cl, i)} \in \nextTxid(\hh_{i}, \cl)$
and $\aexec_{i+1} = \extend(\aexec_{i}, \txid_{(\cl, i)}, (\T^{i}_{\cl})' \cup \T_\rd, \fp_{i})$,
\item  $I(\aexec_{i+1}, \cl) \subseteq \Tx(\hh_{i+1}, \viewFun_{i+1}(\cl))$.
\end{itemize} 
Because $\aexec_{i} \in \aeset(\tr_{i})$, by definition of $\aeset(\stub)$ we have that 
$\aexec_{i+1} \in \aeset(\tr)$ (under the assumption that $\fp_{i} \neq \emptyset$), 
and because $\lastConf(\tr_{i+1}) = (\hh_{i+1}, \stub)$, then $\hh_{\aexec_{i+1}} = \hh_{i+1}$. 

Now we need to check if \( \aexec_{i+1} \) satisfies \( \Ax\) and the invariant \( I \) is preserved.
\begin{itemize}
\item $\A(\aexec_{i+1}) \subseteq \VIS_{\aexec}^{i+1}$ for any $\A \in \Ax$.
Fix $\A \in \Ax$ and $(\txid', \txid) \in \A(\aexec_{i+1})$. 
Because $\aexec_{i+1} = \extend(\aexec_{i}, \txid_{(\cl, i)}, (\T_{\cl}^{i})' \cup \T_\rd, \fp_{i}$), 
we distinguish between two cases.
\begin{itemize}
\item If $\txid = \txid_{(\cl, i)}$, then it must be the case that $\txid' \in (\T^{i}_{\cl})' \cup \T_\rd$, 
and by definition of $\extend(\stub)$ we have that $(\txid' ,\txid_{(\cl, i)}) \in \VIS_{\aexec_{i+1}}$. 
\item If $\txid \neq \txid_{(\cl, i)}$, then we have that $\txid, \txid' \in \T_{\aexec_{i}}$. 
Because $\aexec_{i}$ and $\aexec_{i+1}$ agree on $\T_{\aexec_{i}}$, then $(\txid', \txid) \in \A(\aexec_{i})$.
Because $\aexec_{i} \models \A$, then $(\txid', \txid) \in \VIS_{\aexec_{i}}$. 
By definition of $\extend$, it follows that $(\txid', \txid) \in \VIS_{\aexec_{i+1}}$.
\end{itemize}

\item Finally, we show the invariant is preserved.
Fix a client $\cl'$. 
\begin{itemize}
\item If $\cl' = \cl$, then we have already proved that 
$I(\aexec_{i+1}, \cl) \subseteq \T_{\cl}^{i+1}$. 
\item if $\cl' \neq \cl$, then note that $\viewFun_{i}(\cl') = \viewFun'_{i}(\cl') = \viewFun_{i+1}(\cl')$, 
and in particular $(\T^{\cl'}_{i})' = \Tx(\aexec_{i}, \viewFun'_{i}(\cl')) = \Tx(\aexec_{i+1}, \viewFun_{i+1}(\cl')) =  \T_{\cl'}^{i+1}$.
By the inductive hypothesis we know that $I(\aexec_{i}, \cl) \subseteq \T_{\cl'}^{i}$, 
and by the definition of invariant, we have $I(\aexec_{i+1}, \cl) \subseteq \T_{\cl'}^{i} = \T_{\cl'}^{i+1}$. 
\end{itemize}
\end{itemize}
\end{itemize}
\end{proof}

\begin{corollary}
\label{cor:et-soundness}
If $\ET$ is sound with respect to $(\RP_{\LWW}, \Ax)$, then 
for any program $\prog$, $\interpr{\prog}_{\ET} \subseteq \Setcon{ \hh_{\aexec} }{ \aexec \in \interpr{\prog}_{(\RP_{\LWW}, \Ax)} }$.
\end{corollary}
\begin{proof}
\[
\begin{rclarray}
\interpr{\prog}_{\ET} 
& \stackrel{\cref{thm:consistency-intersect-permissive}}{=} & 
\interpr{\prog}_{\ET_\top} \cap \CMs(\ET) \\
& \stackrel{\cref{cor:kvtrace2aexec}}{=} & 
\Setcon{\hh_{\aexec} }{ \aexec \text{ satisfies } \RP_{\LWW}} \cap \CMs(\ET) \\
& \stackrel{\cref{thm:et_soundness}}{\subseteq} & 
\Setcon{\hh_{\aexec} }{ \aexec \text{ satisfies } \RP_{\LWW} \land \aexec \in \CMa(\RP_{\LWW}, \Ax) } \\
& \stackrel{\cref{thm:consistency-intersect-anarchic}}{=} &
\Setcon{ \hh_{\aexec} }{ \aexec \in \interpr{\prog}_{(\RP_{\LWW}, \Ax)} }
\end{rclarray}
\]
\end{proof}

\subsubsection{Completeness}
The Completeness judgement is in \cref{def:et_complete}.
Given a transaction \( \txid_i \) from client \( \cl \), it converts the visible transactions \( \VIS_{\aexec}^{-1}(\txid_{i}) \) into view  and such view should satisfy the \( \ET \).
Note that \( \aexec \) does not contain precise information about final view after update,
yet the visible transactions of the immediate next transaction from the same client \( \cl \) include those information.

\begin{definition}
\label{def:et_complete}
An execution test $\ET$ is \emph{complete} with respect 
to an axiomatic definition $(\RP_{\LWW}, \Ax)$ if, for any abstract execution $\aexec \in \CMa(\RP_{\LWW}, \Ax)$ 
and index \( i : 1 \leq i < \abs{\T_{\aexec}}\) such that \( \txid_{i} \toEdge{\AR_{\aexec}} \txid_{i+1} \), there exist an initial view $\vi_{i}$ and a final view $\vi_{i}'$ where 
\begin{itemize}
\item $\vi_{i} = \getView(\aexec, \VIS_{\aexec}^{-1}(\txid_{i}))$, 
\item let $\txid_{i} = \txid_{\cl}^{n}$ for some $\cl, n$; 
    \begin{itemize}
        \item if the transaction $\txid_{i}' = \min_{\PO_{\aexec}}\Setcon{\txid' }{ \txid_i \xrightarrow{\PO_{\aexec}} \txid'}$ is defined, then $\vi' = \getView(\aexec, \T_{i})$ where $\T_{i} \subseteq (\AR_{\aexec}^{-1})?(\txid_{i}) \cap \VIS_{\aexec}^{-1}(\txid_{i}'))$; 
        \item otherwise $\vi' = \getView(\aexec, \T_{i})$ where $\T_{i} \subseteq (\AR_{\aexec}^{-1})?(\txid_{i})$, 
    \end{itemize}
\item $\ET \vdash (\hh_{\cut(\aexec, i-1)}, \vi_{i}) \csat \TtoOp{T}_{\aexec}(\txid_{i}) : (\hh_{\cut(\aexec, i)},\vi_{i}')$.
\end{itemize}
\end{definition}

\begin{theorem}
\label{thm:et_complete}
Let $\ET$ be an execution test that is complete with respect to an axiomatic definition $(\RP_{\LWW}, \Ax)$. 
Then $\CMa(\RP_{\LWW}, \Ax) \subseteq \CMs(\ET)$.
\end{theorem}
\begin{proof}
Fix an abstract execution $\aexec \in \CMa(\RP_{\LWW}, \Ax)$. 
For any \(i : 1 \leq i < \abs{\T_\aexec} \), suppose that \( \txid_i \) that is the i-\emph{th} transaction follows the arbitrary order, \ie $\txid_{i} \xrightarrow{\AR_{\aexec}} \txid_{i+1}$ 
and let $\cl_{i}$ be the client of the i-\emph{th} step, \ie $\txid_{i} = \txid_{\cl_{i}}^{\stub}$.
Because $\ET$ is complete with respect to $(\RP_{\LWW}, \Ax)$, 
for any step $i$ we can find an initial views $\vi_i$,and a final view $\vi'_{i}$ such that 
\begin{itemize}
\item $\vi_i = \getView(\aexec, \VIS^{-1}_{\aexec}(\txid_{i}))$, 
\item there exists a set of transactions $\T_{i}$ such that $\getView(\aexec, \T_{i}) = \vi'_{i}$, and 
either $\min_{\PO_{\aexec}}\Setcon{\txid' }{ \txid_{i} \xrightarrow{\PO_{\aexec}} \txid'}$ is 
is defined and $\T_{i} \subseteq (\AR_{\aexec}^{-1})?(\txid_{i}) \cap \VIS^{-1}_{\aexec}(\txid')$, 
or $\T_{i} \subseteq (\AR_{\aexec}^{-1}?(\txid_{i})$, 
\item $\ET \vdash (\hh_{\cut(\aexec, i-1)}, \vi_i) \triangleright \TtoOp{T}_{\aexec}(\txid_{i}): (\hh_{\cut(\aexec, i)}, \vi'_{i})$.
\end{itemize}
Given above, let $\hh_{i} = \cut(\aexec, i)$ and $\fp_{i} = \TtoOp{T}_{\aexec}(\txid_{i})$. Define the views for clients as 
\[
\viewFun_{0} = \lambda \cl \in \Setcon{\cl' }{ \exsts{ \txid \in \T_{\aexec} } \txid = \txid_{\cl'}} \ldotp \lambda \key \ldotp \Set{0}
\quad \viewFun'_{i-1} = \viewFun_{i}\rmto{\cl_{i}}{ \vi_i}
\quad \viewFun_{i} = \viewFun'_{i-1}\rmto{\cl_{i} }{\vi'_{i}}
\]
and the ke-stores as
\[
\hh_{0} = \lambda \key.(\val_{0}, \txid_{0}, \emptyset)
\quad \hh_{i} = \updateKV(\hh_{i-1}, \vi_i, \fp_{i}, \txid_{i})
\]
Now by \cref{prop:aexec2kvtrace} we have that the following sequence of $\ET_{\top}$-reductions 
\[
\begin{array}{l}
(\hh_{0}, \viewFun_{0}) \xrightarrowtriangle{(\cl_{1}, \varepsilon)}_{\ET_{\top}} (\hh_{0}, \viewFun'_{0}) 
\xrightarrowtriangle{(\cl_{1}, \fp_{1})}_{\ET_{\top}} (\hh_{1}, \viewFun_{1}) 
\xrightarrowtriangle{(\cl_{2}, \varepsilon)}_{\ET_{\top}} 
\cdots \xrightarrowtriangle{(\cl_{n}, \fp_{n})}_{\ET_{\top}} (\hh_{n}, \viewFun_{n})
\end{array}
\]
Note that $\hh_{i} = \hh_{\cut(\aexec, i)}$. 
Because $\ET \vdash ( \hh_{\cut(\aexec,i-1)}, \vi_i ) \csat \fp_{i} : (\hh_{i}, \vi'_{i})$, 
or equivalently $\ET \vdash ( \hh_{\cut(\aexec, i-1)}, \viewFun'_{i-1}(\cl_{i}) ) \csat \fp_{i} : ( \hh_{\cut(\aexec, i-1)}, \viewFun_{i}(\cl_{i}) )$, therefore 
\[
\begin{array}{l}
(\hh_{0}, \viewFun_{0}) \xrightarrowtriangle{(\cl_{1}, \varepsilon)}_{\ET} (\hh_{0}, \viewFun'_{0}) 
\xrightarrowtriangle{(\cl_{1}, \fp_{1})}_{\ET} (\hh_{1}, \viewFun_{1})
\xrightarrowtriangle{(\cl_{2}, \varepsilon)}_{\ET} 
\cdots \xrightarrowtriangle{(\cl_{n}, \fp_{n})}_{\ET} (\hh_{n}, \viewFun_{n})
\end{array}
\]
It follows that $\hh_{n} \in \CMs(\ET)$ then $\hh_{n} = \hh_{\cut(\aexec, n)} = \hh_{\aexec}$, and there is nothing left to prove.
\end{proof}

\begin{corollary}
\label{cor:et-completeness}
If $\ET$ is complete with respect to $(\RP_{\LWW}, \Ax)$, then 
for any program $\prog$, $\Setcon{ \hh_{\aexec} }{ \aexec \in \interpr{\prog}_{(\RP_{\LWW}, \Ax)} } \subseteq \interpr{\prog}_{\ET}$.
\end{corollary}
\begin{proof}
\[
\begin{rclarray}
    \Setcon{ \hh_{\aexec} }{ \aexec \in \interpr{\prog}_{(\RP_{\LWW}, \Ax)} }
& \stackrel{\cref{thm:consistency-intersect-anarchic}}{=} &
\Setcon{\hh_{\aexec} }{ \aexec \text{ satisfies } \RP_{\LWW} \land \aexec \in \CMa(\RP_{\LWW}, \Ax) } \\
& \stackrel{\cref{thm:et_complete}}{\subseteq} & 
\Set{\hh_{\aexec} }[ \aexec \text{ satisfies } \RP_{\LWW}] \cap \CMs(\ET) \\
& \stackrel{\cref{cor:kvtrace2aexec}}{=} & 
\interpr{\prog}_{\ET_\top} \cap \CMs(\ET) \\
& \stackrel{\cref{thm:consistency-intersect-permissive}}{=} & 
\interpr{\prog}_{\ET} 
\end{rclarray}
\]
\end{proof}


% proofs of et
\section{The Soundness and Completeness of Execution Tests}
\label{app:et_sound_complete}
We now show using \cref{def:et_sound,def:et_complete} to prove the soundness and completeness of execution tests with respect to axiomatic specification.
It is sufficient to match these two definition, 
then by \cref{cor:et-soundness,cor:et-completeness} we have \( \CMs(\ET) = \Setcon{\mkvs_\aexec}{\aexec \in \CMa(\RP_\LWW,\Ax)} \).

\label{sec:kv-sound-complete-proof}
\label{sec:spec-proof}

\emph{Monotonic read} (\( \MR \)) \citep{session-guarantee,repldatatypes} states that after committing a transaction, 
a client cannot lose information in that 
it can only see increasingly more versions from a kv-store.
This prevents, for example, the kv-store in \cref{fig:mr-disallowed},
since client \(\cl\) first reads the latest version of \(\key\) in \(\txid_{\cl}^{1}\), 
and then reads the older, initial version of \(\key\) in \(\txid_{\cl}^{2}\).  
As such, the \(\ViewShift[\MR]\) predicate in \cref{fig:execution-tests} 
ensures that clients can only extend their views,
that is, \( \vi \vileq \vi' \) for views \( \vi, \vi'\) before and after committing.
When this is the case, clients can then \emph{always} commit their transactions,
and thus \(\CanCommit[\MR]\) is simply defined as \(\true\). 

\subsection{Monotonic Write \( \MW \)}
\label{sec:sound-complete-mw}

The execution test $\ET_\MW$ is sound with respect to the axiomatic specification 
$(\RP_{\LWW}, \Set{\lambda \aexec. \PO_{\aexec} ; \VIS_{\aexec} })$.
We pick the invariant as empty set given the fact of no constraint on the view after update:
\[ 
    I( \aexec, \cl ) = \emptyset 
\]
Assume a key-value store $\hh$, an initial and a final view $\vi, \vi'$  a fingerprint $\opset$ 
such that $\ET_{\MW} \vdash (\hh, \vi) \csat \opset: (\hh',\vi')$. 
Also choose an arbitrary $\cl$, a transaction identifier $\txid \in \nextTxId(\hh, \cl)$, 
and an abstract execution $\aexec$ such that $\hh_{\aexec} = \hh$ and 
\( I(\aexec, \cl) =  \emptyset \subseteq \Tx(\hh, \vi) \).
Let \( \aexec' = \extend(\aexec, \txid, \Tx(\mkvs, \vi) \cup \T_\rd, \f ) \).
Note that since the invariant  is empty set, it remains to prove that there exists a set of read-only transactions \( \T_\rd \) such that:
\[
    \begin{array}{@{}l@{}}
        \fora{ \txid' }  (\txid' ,\txid)  \in \PO_{\aexec'} ; \VIS_{\aexec'}
        \implies \txid' \in \Tx(\mkvs, \vi) \cup \T_\rd
    \end{array}
\]
Initially we take \( \T_\rd = \emptyset \), 
and by closing the \( \Tx(\mkvs, \vi) \) with respect to the relation \( \PO_{\aexec'} ; \VIS_{\aexec'} \),
we will add more read-only transactions into the set \( \T_\rd\).
Suppose \( (\txid' ,\txid)  \in \PO_{\aexec'} ; \VIS_{\aexec'} \), 
that is, \( \txid' \toEdge{\SO_{\aexec'}} \txid'' \toEdge{\VIS_{\aexec'}} \txid \).
We perform a case analysis on if \( \txid'' \) has write:
\begin{itemize}
\item If the transaction \( \txid'' \) writes to a key.
For the new abstract execution \( \aexec' \), the visible transactions for \( \txid \) must come from \( \Tx(\mkvs, \vi) \cup \T_\rd \).
It means \( \txid'' \in \Tx(\mkvs, \vi) \cup \T_\rd  \).
Then given that \( \txid'' \) is not a read-only transaction, we have \( \txid'' \in \Tx(\mkvs, \vi) \).
Now there are two cases:
\begin{itemize}
    \item if \( \txid' \) is a read-only transaction, we include \( \txid' \in \T_{\rd} \).
    \item if \( \txid' \) has at least one write, it is easy to see \( \txid' \in \Tx(\mkvs, \vi) \) since \( j \in \vi(\ke) \wedge \WTx(\hh(\ke', i)) \xrightarrow{\PO?} \WTx(\hh(\ke, j)) \implies i \in \vi(\ke') \).
\end{itemize}
\item If the transaction \( \txid'' \in \T_\rd \) is a read-only transaction, 
since \( \T_\rd \) is initial empty, there must exist a later transaction \( \txid''' \) from the same client that writes to a key,
and such transaction \( \txid''' \) is included in \( \Tx(\mkvs, \vi) \):
\[
    \txid' \toEdge{\SO_{\aexec'}} \txid'' 
    \toEdge{\SO_{\aexec'}} \txid''' \toEdge{\VIS_{\aexec'}} \txid 
    \land \txid''' \in \Tx(\mkvs,\vi)
\]
Since \( \SO \) is transitive, 
therefore \( \txid' \toEdge{\SO_{\aexec'}} \txid''' \toEdge{\VIS_{\aexec'}} \txid \),
which we have already proven \( \txid' \in \Tx(\mkvs, \vi) \) or we will include \( \txid' \) in \( \T_\rd \).
Since there are finite transactions from a client in a trace, there must exist a \( \T_\rd \) in the end.
\end{itemize}


The execution test $\ET_{\MW}$ is complete with respect to 
the axiomatic specification $(\RP_{\LWW}, \Set{\lambda \aexec.(\PO_{\aexec} ; \VIS_{\aexec})})$. 
Let $\aexec$ be an abstract execution that satisfies the specification
$\CMa(\RP_{\LWW}, \Set{\lambda \aexec.(\PO_{\aexec} ; \VIS_{\aexec})})$, 
and consider a transaction $\txid \in \T_{\aexec}$. 
Assume i-\emph{th} transaction \( \txid_i \) in the arbitrary order,
and let view \( \vi_{i} = \getView(\aexec, \VIS^{-1}_{\aexec}(\txid_{i}) ) \).
We also pick any final view such that \( \vi'_{i} \subseteq \getView(\aexec, (\AR^{-1}_{\aexec})?(\txid_{i}) ) \).
It suffices to prove \( \ET_\MW \vdash (\hh_{\cut(\aexec, i-1)}, \vi_i ) \csat  \TtoOp{T}_{\aexec}(\txid_{i}) : (\hh_{\cut(\aexec, i-1)}, \vi'_{i}) \).
It means to prove the follows:
\begin{equation}
\label{equ:mw-complete}
\begin{array}{@{}l@{}}
    \fora{j,m,\ke, \ke' } j \in \vi(\ke)  
    \wedge \WTx(\hh_{\cut(\aexec, i-1)}(\ke', m)) \xrightarrow{\PO?} \WTx(\hh_{\cut(\aexec, i-1)}(\ke, j))  
    \implies m \in \vi(\ke')
\end{array}
\end{equation}
Assume \( j \) and \( \ke' \) such that \( j \in \vi(\ke')\), which means \( \WTx(\hh_{\cut(\aexec, i-1)}(\ke', j)) \in \VIS^{-1}_{\aexec}(\txid_{i}) \).
Now let consider transaction \( \txid \) that commits before \( \txid \) from the same session, \ie \( \txid \toEdge{\SO} \WTx(\hh_{\cut(\aexec, i-1)}(\ke, j)) \).
By the constraint \( \lambda \aexec.(\PO_{\aexec} ; \VIS_{\aexec}) \), the transaction \( \txid \in \VIS^{-1}_{\aexec}(\txid_{i}) \).
It means that in the kv-store \(  \hh_{\cut(\aexec, i-1)} \) every version written by \( \txid =  \WTx(\hh_{\cut(\aexec, i-1)}(\ke', m)) \) should be included in the view \( m \in \vi_i(\ke') \).
Thus we have the proof of \cref{equ:mw-complete}.

The execution test \(\et[\RYW]\) is sound with respect to the axiomatic definition
\(\visaxioms[\RYW] = \Set{\lambda \aexec \ldotp \SO_{\aexec} }\) \cite{repldatatypes}.
We pick the following invariant:
\[
    \aexecinv[\RYW](\aexec, \cl) \FuncDef
    \begin{multlined}[t]
    \left( \bigcup_{\Set{\txid[\cl](n) \in \aexec }} \Refl((\Inv(\SO)))(\txid[\cl](n)) \right) 
    \setminus \Set{\txidrd | \txidrd in \aexec \land \Forall{l | \key | \val} (l,\key,\val) \in \aexec(\txid) \implies l = \opR } .
    \end{multlined}
\]

\SOUNDLET{\RYW}{
    \txidsetrd = 
    \left( \bigcup_{\Set{\txid[\cl](n) \in \aexec }} \Refl((\Inv(\SO)))(\txid[\cl](n)) \right) 
    \cap \Set{\txidrd | \txidrd in \aexec \land \Forall{l | \key | \val} (l,\key,\val) \in \aexec(\txid) \implies l = \opR } .
}
\begin{enumerate}
\Case{\(\Forall{\visaxiom \in \visaxioms }
            \Inv(\visaxiom(\aexec'))(\txid) \subseteq \txidset \cup \txidsetrd \)}
    Suppose transactions \( \txid, \txid' \) such that \( \txid,\txid' \in \aexec \) and \( (\txid',\txid) \in \SO \).
    If \( \txid' \) is a read-only transaction, \( \txid' \in \txidsetrd \).
    Otherwise, \( \txid' \) has write, by the definition of \( \aexecinv[\RYW] \), 
    it follows that \( \txid' \in \aexecinv[\RYW](\aexec,\cl) \)
    and therefore \( \txid' \in \txidset \).
\Case{\(\aexecinv(\aexec',\cl) \subseteq \VisTrans(\XToK(\aexec'),\vi') \)}
    Because \( \ToET[\RYW]{\kvs | \vi | \fp | \kvs' | \vi' }\),
    it must be the case that
    \[
        \Forall{\key \in \Keys | \idx \in \Indexs } (\WtOf(\kvs'(\key,\idx)),\txid) \in \Refl(\SO) \implies \idx \in \vi'(\key) 
    \]
    and therefore
    \[
        \Forall{\txid'} \Exists{\key \in \Keys | \val \in \Values} \opW(\key,\val) \in \aexec'(\txid')
        \land (\txid',\txid) \in \SO \implies \txid' \in \VisTrans(\XToK(\aexec'),\vi') .
    \]
    Note that \( \bigcup_{\Set{\txid[\cl](n) \in \aexec }} \Refl((\Inv(\SO)))(\txid[\cl](n)) = \Refl((\Inv(\SO)))(\txid) \).
    Last, we have
    \begin{align*}
        \aexecinv(\aexec',\cl) & = 
            \begin{multlined}[t]
            \left( \bigcup_{\Set{\txid[\cl](n) \in \aexec }} \Inv(\SO)(\txid[\cl](n)) \right) 
            \setminus \Set{\txidrd | \txidrd \in \aexec 
                    \land \Forall{l | \key | \val} (l,\key,\val) \in \aexec(\txid) \implies l = \opR } 
            \end{multlined}
            \\ & = \begin{multlined}[t]
            \left( \Refl((\Inv(\SO)))(\txid) \right) 
            \setminus \Set{\txidrd | \txidrd \in \aexec 
                    \land \Forall{l | \key | \val} (l,\key,\val) \in \aexec(\txid) \implies l = \opR } 
            \end{multlined}
            \\ & \subseteq \VisTrans(\XToK(\aexec'),\vi') 
    \end{align*}
\end{enumerate}

\COMPLETELET{\RYW}
We construct the final view \( \vi'\) depending on whether \( \txid[\cl](n) \) is the last transaction for the client \( \cl \).
\begin{enumerate}
\Case{\( (\txid[\cl](n), \txid') \in \SO \) for \( \txid' \in \aexec \)}
    Let the transaction 
    \( \txid = \Min[\SO](\Set{ \txid' | (\txid[\cl](n), \txid') \in \SO \land \txid' \in \aexec' }) \).
    For this case, let view 
    \( \vi' = \GetView(\aexec, \Refl((\ARInv[\aexec]))(\txid[\cl](\idx)) \cap \VISInv[\aexec](\txid)) \).
    By \( \visaxioms[\RYW] \), it follows that, for any transaction \( \txid' \),
    if \( ( \txid',\txid[\cl](idx) ) \in \Refl(\SO) \), then
    \( \txid' \in \VISInv[\aexec](\txid)) \).
    Since \( \SO \in \AR \), we know that 
    \( \txid' \in \Refl((\ARInv[\aexec]))(\txid[\cl](\idx)) \cap \VISInv[\aexec](\txid)) \).
    Therefore, for any version \( \kvs'(\key,j)\) such that 
    \( ( \WtOf(\kvs'(\key,j)), \txid) \in \Refl(\SO) \),
    then \( j \in \vi'(\key)\).
\Case{\( \neg \left((\txid[\cl](n), \txid') \in \SO \right) \)}
    For this case, let 
    \( \vi' = \GetView(\aexec, \Refl((\ARInv[\aexec]))(\txid[\cl](\idx))) \) be the final view.
    It is easy to see that \( \vi' \) satisfies \( \RYW \). 
\end{enumerate}

The execution test \(\et[\WFR]\) is sound with respect to the axiomatic definition 
\(\visaxioms[\WFR] \Set{\lambda \aexec \ldotp \WR[\aexec] ; \Refl((\SO \cap \RW[\aexec] )) ; \VIS[\aexec] })\) 
\citep{surech-session-guarantee}.
By picking the invariant as \( I( \aexec, \cl ) = \emptyset \), the soundness and completeness
can be derived from \cref{thm:view-vis-relation} in a similar way as the proofs for \( \MW \).

The wildly used definition on abstract executions for causal consistency is that 
\( \VIS \) is transitive and \( \SO \in \VIS \).
Yet it is for the sack of elegant definition,
while there is a equivalent minimum visibility relation (\cref{thm:cc-visaxioms}) defined by 
\( \visaxioms[\CC] \FuncDef \Set{ \lambda \aexec \ldotp (\WR[\aexec] \cup \SO) ; \VIS[\aexec] \subseteq \VIS[\aexec] , 
                                    \lambda \aexec \ldotp \SO \subseteq \VIS[\aexec]} \),
where \( \WR[\aexec] \) is defined in \cref{def:aexec-dgraph}.

\begin{theorem}[Minimum visibility relation for (\texorpdfstring{\CC}{\texttt{CC}})]
\label{thm:cc-visaxioms}
For two abstract executions \( \aexec,\aexec' \),
the following constrain on visibility,
\begin{Formulae}
\begin{Formula}
    (\WR[\aexec] \cup \SO) ; \VIS[\aexec] \subseteq \VIS[\aexec] \land \SO \subseteq \VIS[\aexec]
    \label{equ:kvstore-cc-spec}
\end{Formula}
\end{Formulae}
is equivalent to
\begin{Formulae}
\begin{Formula}
    \VIS[\aexec'] ; \VIS[\aexec'] \subseteq \VIS[\aexec'] \land \SO \subseteq \VIS[\aexec']
    \label{equ:aexec-cc-spec}
\end{Formula}
\end{Formulae}
in that 
\(
    \Forall{\txid \in \TxIDs | \fp } \left( \fp = \aexec(\txid) \iff \fp = \aexec'(\txid) \right)
    \land \AR[\aexec] = \AR[\aexec'] .
\)
\end{theorem}
\begin{proof}
For an abstract execution \( \aexec \) that satisfies \cref{equ:kvstore-cc-spec},
by \cref{lem:aexec-spec-cc}, there exists \( \aexec' \) that satisfies \cref{equ:aexec-cc-spec}.
Assume an abstract execution \( \aexec' \) that satisfies \cref{equ:aexec-cc-spec}.
Since \( \WR[\aexec'] \subseteq \VIS[\aexec']\) by the definition of \( \WR[\aexec']\),
thus \( \aexec' \) satisfies \cref{equ:kvstore-cc-spec}.
\end{proof}

\begin{toappendix}
\begin{lemma}[Minimum visibility relation for (\texorpdfstring{\CC}{\texttt{CC}})]
\label{lem:aexec-spec-cc}
For any abstract execution \( \aexec \), if it satisfies \( \visaxioms[\CC] \),
there exists a new abstract execution \( \aexec' \) such that \( \SO \in \VIS[\aexec]\) and
\begin{Formulae}
\begin{Formula}
    \Forall{\txid \in \TxIDs | \fp } \left( \fp = \aexec(\txid) \iff \fp = \aexec'(\txid) \right)
    \land \AR[\aexec] = \AR[\aexec'] \land \VIS[\aexec'] ; \VIS[\aexec'] \subseteq \VIS[\aexec'] .
    \label{equ:aexec-spec-cc}
\end{Formula}
\end{Formulae}
\end{lemma}
\begin{proof}
We erase some visibility relation for each transaction following 
the arbitration order \( \AR \) until the visibility is transitive.
Intuitively, the final visibility relation is exactly \( \Trasi((\WR[\aexec] \cup \SO)) \).
Assume the \Th{\idx} transaction \( \txid_\idx \)  with respect to the arbitration order.
Let \( \rel[\idx] \) be a new visibility for the transaction \( \txid_\idx \) such that
\( {\rel[\idx]}\Proj{2} = \Set{\txid_\idx}\) for all indexes \( \idx \)
and the union of visibility relations \( \bigcup_{0 \leq j \leq \idx } \rel[\idx] \) is transitive.
We preserve that, for each index \( \idx \), cut of abstract execution \( \aexec' =  \AexecCut(\aexec, \idx) \)
and visibility relation \( \VIS' = \bigcup_{0 \leq j \leq \idx } \rel[j] \),
the following invariant holds:
\begin{Formulae}
& \begin{Formula} 
    \VIS' ; \VIS' \subseteq \VIS'  ,
    \label{equ:cc-vis-idx-transitive} 
\end{Formula}
\\ & \begin{Formula}
    \Forall{ \txid \in \aexec } (\txid,\txid_i) \in \rel[\idx] \implies (\txid, \txid_i) \in (\WR[\aexec'] \cup \SO) .
    \label{equ:cc-vis-idx-minimum}
\end{Formula}
\end{Formulae}
We prove the above by induction on the number \( \idx \).
\begin{enumerate}
\CaseBase{\( \idx = 0 \)}
    By the definition of \( \AexecCut \), we know that \(\aexecinit = \AexecCut(\aexec,0) \)
    and \cref{equ:cc-vis-idx-transitive,equ:cc-vis-idx-minimum} trivially hold.
\CaseInd{\( \idx > 0 \)}
    Suppose that, for the \Th{(\idx-1)} step,
    the abstract execution \( \aexec'' =  \AexecCut(\aexec, \idx - 1) \)
    and the visibility relation \( \VIS'' = \bigcup_{0 \leq j \leq \idx-1 } \rel[j] \) 
    satisfy \cref{equ:cc-vis-idx-transitive,equ:cc-vis-idx-minimum}.
    Let consider \Th{\idx} step, the transaction \( \txid_i \),
    the cut \( \aexec' =  \AexecCut(\aexec, \idx) \)
    and the visibility relation \( \VIS' = \bigcup_{0 \leq j \leq \idx } \rel[j] \).
    Initially we take \( \rel \) as an empty set.
    First, we include \( \Set{(\txid,\txid_i) | (\txid,\txid_i) \in \WR[\aexec]} \) to \( \rel \)
    and, by the definition of \( \WR[\aexec]\), 
    it trivially does not affect any read operation for the transaction \( \txid_i \).
    Then we do the same for \( \SO \) as that 
    we include \( \Set{(\txid,\txid_i) | (\txid,\txid_i) \in \SO} \) to \( \rel \).
    Note that \( \SO \) cannot affect any read operation for the transaction \( \txid_i \) neither,
    otherwise it contradicts to that \( \SO \subseteq \VIS[\aexec] \) and the definition of \( \WR[\aexec] \).

    For relations \( \rel' = \rel ; \bigcup_{0 \leq j \leq \idx-1 } \rel[j] \) and then \( \rel[\idx] = \rel \cup \rel' \),
    it easy to see that \( \rel \in \VIS[\aexec]\) and, then by \ih, \( \rel' \in \VIS[\aexec] \).
    We prove that the \( \rel[\idx] \) does not affect any read operation for the transaction \( \txid_i \)
    by contradiction.
    Assume distinct transactions \( \txid,\txid' \) such that
    \( \ToEdge{\txid'' | \rel \cup \rel' -> \txid' | \rel \cup \rel' -> \txid_i } \),
    and immediately  by the definition of \( \rel \) and \( \rel' \),
    then \( \ToEdge{\txid'' | \rel' -> \txid' | \rel -> \txid_i } \).
    Assume that \( \txid'' \) change the read operation for a key \( \key \) in \( \txid_i \).
    This means that there exists a transaction \( \txid^* \) such that
    \( (\txid^*,\txid_i) \in \WR[\aexec](\key)\) and \( (\txid^*,\txid'') \in \AR[\aexec] \),
    where the latter implies that \( (\txid'',\txid_i) \in \WR[\aexec](\key) \);
    there is a contradiction and thus 
    \( \rel[\idx] \) does not affect any read operation for the transaction \( \txid_i \).

    We now prove that \cref{equ:cc-vis-idx-transitive,equ:cc-vis-idx-minimum} still hold.
    \begin{enumerate}
    \Case{\cref{equ:cc-vis-idx-transitive}}
        Assume a relation \( \rel^* = \bigcup_{0 \leq j \leq \idx-1 } \rel[j] \) 
        and transactions \( \txid, \txid',\txid'' \) such that 
        \[
            \ToEdge{\txid | \rel^* \cup \rel[\idx] -> \txid' | \rel^* \cup \rel[\idx] -> \txid'' } .
        \]
        If \( \ToEdge{\txid | \rel^*  -> \txid' | \rel^*  -> \txid'' } \), 
        then by \ih, \( \ToEdge{\txid | \rel^*  -> \txid'' } \).
        Note that \( \ToEdge{\txid | \rel[\idx]  -> \txid' | \rel^*  -> \txid'' } \) cannot happen,
        because it contradict to that \( \txid' = \txid_i\) and \( (\txid'',\txid_i) \in \AR[\aexec] \).
        Thus consider \( \ToEdge{\txid | \rel^*  -> \txid' | \rel[\idx]  -> \txid'' } \).
        It must be the case that \( \txid'' = \txid_i \) and by the definition of \( \rel[\idx] \),
        we know that \( \ToEdge{\txid | \rel[\idx]  -> \txid'' } \).
    \Case{\cref{equ:cc-vis-idx-minimum}}
        By the construction, \cref{equ:cc-vis-idx-minimum} hold. \qedhere
    \end{enumerate}
\end{enumerate}
\end{proof}
\end{toappendix}

We pick the invariant as \( \aexecinv[\CC] = \aexecinv[\MR] \cup \aexecinv[\RYW]  \).
\SOUNDLET{\CC}{ \txidsetrd \supseteq
\begin{multlined}[t]
\left( \bigcup_{\Set{\txid[\cl](\idx) | \txid[\cl](\idx) \in \aexec}} 
\VISInv[\aexec](\txid[\cl](\idx)) \cup \Refl((\Inv(\SO)))(\txid[\cl](\idx)) \right) 
\setminus \Set{\txid' | \Forall{l | \key | \val } (l,\key,\val) \in \aexec(\txid') \implies l = \opR } .
\end{multlined} }
Assume 
\[ 
\txidsetrd' = 
\begin{multlined}[t]
\left( \bigcup_{\Set{\txid[\cl](\idx) | \txid[\cl](\idx) \in \aexec}} 
\VISInv[\aexec](\txid[\cl](\idx)) \cup \Refl((\Inv(\SO)))(\txid[\cl](\idx)) \right) 
\setminus \Set{\txid' | \Forall{l | \key | \val } (l,\key,\val) \in \aexec(\txid') \implies l = \opR } .
\end{multlined} 
\]
and \( \txidsetrd'' = \txidsetrd \setminus \txidsetrd' \).
By the definition of soundness, we prove the following result
\begin{Formulae}
& \begin{Formula}
\Inv(\SO)(\txid) \subseteq \txidset \cup \txidsetrd'
\label{equ:cc-so-vis}
\end{Formula}
\\ & \begin{Formula}
\Inv((( \WR[\aexec'] \cup \SO ) ; \VIS[\aexec'] )) (\txid) \subseteq \txidset \cup \txidsetrd' \cup \txidsetrd''
\label{equ:cc-vis-transitive}
\end{Formula}
\\ & \begin{Formula}
\aexecinv[\CC](\aexec',\cl) \subseteq \VisTrans(\XToK(\aexec'),\vi')
\label{equ:cc-inv-preserve}
\end{Formula}
\end{Formulae}
\Cref{equ:cc-so-vis} can be proven in the same way as in \cref{sec:sound-complete-mr}
We now prove \cref{equ:cc-vis-transitive}.
Initially we take \( \txidsetrd'' \) to be an empty set.
Note that \(\VISInv[\aexec'](\txid) = \txidset \cup \txidsetrd' \cup \txidsetrd'' \).
By \cref{thm:view-vis-relation,equ:view-close-to-aexec}, there exists \( \txidsetrd'' \) such that
\( \txidset \cup \txidsetrd'' \) is closed under \( \WR[\aexec'] \cup \SO \).
Now consider a transaction \( \txidrd \in \txidsetrd' \) and
assume a transaction \( \txid' \) such that \( \ToEdge{ \txid' | \WR[\aexec'] \cup \SO -> \txidrd } \).
There are two cases depending on \( \txidrd \).
\begin{enumerate}
\Case{\( \ToEdge{\txidrd | \VIS[\aexec'] -> \txid'' | \SO -> \txid} \) for some \( \txid'' \)}
    For this case, we have
    \begin{align*}
    \ToEdge{\txid' | \WR[\aexec'] \cup \SO -> \txidrd | \VIS[\aexec'] -> \txid'' | \SO -> \txid }
    & 
    \implies \ToEdge{\txid' | \WR[\aexec] \cup \SO -> \txidrd | \VIS[\aexec] -> \txid'' | \SO -> \txid }
    \\ & \implies \ToEdge{\txid' | \VIS[\aexec] -> \txid'' | \SO -> \txid } .
    \end{align*}
    By \( \aexecinv[\MR]\), we know that \(\txid' \in \aexecinv[\MR] \cup \txidsetrd' \).
\Case{\( \ToEdge{\txidrd | \SO -> \txid} \)}
    For this case, we have
    \begin{align*}
    \ToEdge{\txid' | \WR[\aexec'] \cup \SO -> \txidrd | \SO -> \txid }
    & 
    \implies \ToEdge{\txid' | \WR[\aexec] \cup \SO -> \txidrd | \SO -> \txid }
    \\ & \implies \ToEdge{\txid' | \VIS[\aexec] -> \txid'' | \SO -> \txid } .
    \end{align*}
    By \( \aexecinv[\MR]\), we know that \(\txid' \in \aexecinv[\MR] \cup \txidsetrd' \).
\end{enumerate}
Last, \cref{equ:cc-inv-preserve}can be proven in the same way as in \cref{sec:sound-complete-mr,sec:sound-complete-ryw}.

\COMPLETELET{\CC}
By \cref{thm:cc-visaxioms},
it is sufficient to prove with respect to the following visibility axioms,
\( \visaxioms[\CC]' \FuncDef \Set{ \lambda \aexec \ldotp  \VIS[\aexec] ; \VIS[\aexec] \subseteq \VIS[\aexec] , 
                                    \lambda \aexec \ldotp \SO \subseteq \VIS[\aexec]} \).
By the definition of \( \et[\CC] \), we prove \( \CanCommit[\CC]\) and \( \ViewShift[\MR \cup \RYW]\) respectively.
Since \( (\WR[\aexec] \cup \SO) ; \VIS[\aexec]  \subseteq \VIS[\aexec] ; \VIS[\aexec] \subseteq \VIS[\aexec] \),
then \( \CanCommit[\CC]\) can be derived from \cref{thm:view-vis-relation,equ:aexec-close-to-view}
and \( \ViewShift[\RYW] \) can be proven in the same way as in \cref{sec:sound-complete-ryw}.
By \( \VIS[\aexec] ; \SO \subseteq \VIS[\aexec] ; \VIS[\aexec] \subseteq \VIS[\aexec]  \),
\( \ViewShift[\MR] \) can be proven in the same way as in \cref{sec:sound-complete-mr}.



\subsection{Update Atomic}
\begin{figure}
\hrule
\begin{tabular}{@{} c c@{}}

\begin{halfsubfig}
\begin{centertikz}

\begin{pgfonlayer}{foreground}
%Uncomment line below for help lines
%\draw[help lines] grid(5,4);

%Location x
\node(locx)  {$\ke_\vx \mapsto$};

\matrix(versionx) [version list]
    at ([xshift=\tikzkvspace]locx.east) {
    {a} & $\txid_0$ \\
    {a} & $\emptyset$ \\
};

\tikzvalue{versionx-1-1}{versionx-2-1}{locx-v0}{0};

%Location y
\path (locx.south) + (0,\tikzkeyspace) node (locf1) {$\ke_{\pv{f1}} \mapsto$};
\matrix(versionf1) [version list]
    at ([xshift=\tikzkvspace]locf1.east) {
    {a} & $\txid_0$ \\
    {a} & $\emptyset$ \\
};
\tikzvalue{versionf1-1-1}{versionf1-2-1}{locf1-v0}{0};

%Location y
\path (locf1.south) + (0,\tikzkeyspace) node (locf2) {$\ke_{\pv{f2}} \mapsto$};
\matrix(versionf2) [version list]
    at ([xshift=\tikzkvspace]locf2.east) {
    {a} & $\txid_0$ \\
    {a} & $\emptyset$ \\
};
\tikzvalue{versionf2-1-1}{versionf2-2-1}{locf2-v0}{0};

% \draw[-, red, very thick, rounded corners] ([xshift=-5pt, yshift=5pt]locx-v1.north east) |- 
%  ($([xshift=-5pt,yshift=-5pt]locx-v1.south east)!.5!([xshift=-5pt, yshift=5pt]locy-v0.north east)$) -| ([xshift=-5pt, yshift=5pt]locy-v0.south east);

%blue view - I should  check whether I can use pgfkeys to just declare the list of locations, and then add the view automatically.
\draw[-, blue, very thick, rounded corners=10pt]
 ([xshift=-2pt, yshift=20pt]locx-v0.north east) node (tid1start) {} -- 
 ([xshift=-2pt, yshift=-5pt]locf2-v0.south east);
 
 \path (tid1start) node[anchor=south, rectangle, fill=blue!20, draw=blue, font=\small, inner sep=1pt] {$\thid_3$};

%red view
\draw[-, red, very thick, rounded corners = 10pt]
 ([xshift=-5pt, yshift=5pt]locx-v0.north east) -- 
 ([xshift=-5pt, yshift=-10pt]locf2-v0.south east) node (tid2start) {};
 
\path (tid2start) node[anchor=north, rectangle, fill=red!20, draw=red, font=\small, inner sep=1pt] {$\thid_2$};
 
 %green view
\draw[-, DarkGreen, very thick, rounded corners = 10pt]
 ([xshift=-16pt, yshift=8pt]locx-v0.north east) node (tid3start) {}-- 
 ([xshift=-16pt, yshift=-5pt]locf2-v0.south east);
 
 \path (tid3start) node[anchor=south, rectangle, fill=DarkGreen!20, draw=DarkGreen, font=\small, inner sep=1pt] {$\thid_1$};

\end{pgfonlayer}
\end{centertikz}%
\caption{Initial configuration}
\label{fig:ua-init}
\end{halfsubfig}
%
&
%
\begin{halfsubfig}
\begin{centertikz}
\begin{pgfonlayer}{foreground}
%Uncomment line below for help lines
%\draw[help lines] grid(5,4);

\node(locx)  {$\ke_\vx \mapsto$};

\matrix(versionx) [version list]
    at ([xshift=\tikzkvspace]locx.east) {
    {a} & $\txid_0$ & {a} & \(\txid_1\)\\
    {a} & $\Set{\txid_1}$ & {a} & \(\emptyset\)\\
};

\tikzvalue{versionx-1-1}{versionx-2-1}{locx-v0}{0};
\tikzvalue{versionx-1-3}{versionx-2-3}{locx-v1}{1};


\path (locx.south) + (0,\tikzkeyspace) node (locf1) {$\ke_{\pv{f1}} \mapsto$};
\matrix(versionf1) [version list]
    at ([xshift=\tikzkvspace]locf1.east) {
    {a} & $\txid_0$ & {a} & $\txid_1$\\
    {a} & $\Set{\txid_1}$ & {a} & $\emptyset$\\
};
\tikzvalue{versionf1-1-1}{versionf1-2-1}{locf1-v0}{0};
\tikzvalue{versionf1-1-3}{versionf1-2-3}{locf1-v1}{1};

%Location y
\path (locf1.south) + (0,\tikzkeyspace) node (locf2) {$\ke_{\pv{f2}} \mapsto$};
\matrix(versionf2) [version list]
    at ([xshift=\tikzkvspace]locf2.east) {
    {a} & $\txid_0$ \\
    {a} & $\emptyset$ \\
};
\tikzvalue{versionf2-1-1}{versionf2-2-1}{locf2-v0}{0};


% \draw[-, red, very thick, rounded corners] ([xshift=-5pt, yshift=5pt]locx-v1.north east) |- 
%  ($([xshift=-5pt,yshift=-5pt]locx-v1.south east)!.5!([xshift=-5pt, yshift=5pt]locy-v0.north east)$) -| ([xshift=-5pt, yshift=5pt]locy-v0.south east);

%blue view - I should  check whether I can use pgfkeys to just declare the list of locations, and then add the view automatically.
\draw[-, blue, very thick, rounded corners=10pt]
 ([xshift=-2pt, yshift=20pt]locx-v0.north east) node (tid1start) {} -- 
 ([xshift=-2pt, yshift=-5pt]locf2-v0.south east);
 
 \path (tid1start) node[anchor=south, rectangle, fill=blue!20, draw=blue, font=\small, inner sep=1pt] {$\thid_3$};

%red view
\draw[-, red, very thick, rounded corners = 10pt]
 ([xshift=-5pt, yshift=5pt]locx-v0.north east) -- 
 ([xshift=-5pt, yshift=-10pt]locf2-v0.south east) node (tid2start) {};
 
\path (tid2start) node[anchor=north, rectangle, fill=red!20, draw=red, font=\small, inner sep=1pt] {$\thid_2$};
 
 %green view
\draw[-, DarkGreen, very thick, rounded corners = 10pt]
 ([xshift=-16pt, yshift=8pt]locx-v1.north east) node (tid3start) {}-- 
 ([xshift=-16pt, yshift=-5pt]locf1-v1.south east) --
 ([xshift=-16pt, yshift=5pt]locf2-v0.north east) -- 
 ([xshift=-16pt, yshift=-5pt]locf2-v0.south east);
 
 \path (tid3start) node[anchor=south, rectangle, fill=DarkGreen!20, draw=DarkGreen, font=\small, inner sep=1pt] {$\thid_1$};

\end{pgfonlayer}
\end{centertikz}
\caption{After \(\txid_1\)}
\label{fig:ua-after-tx1}
\end{halfsubfig}

\\
\begin{subfigure}{0.45\textwidth}
\begin{centertikz}%
\begin{pgfonlayer}{foreground}
%Uncomment line below for help lines
%\draw[help lines] grid(5,4);


\node(locx)  {$\ke_\vx \mapsto$};

\matrix(versionx) [version list, column 2/.style = {text width=14mm}]
    at ([xshift=\tikzkvspace]locx.east) {
    {a} & $\txid_0$ & {a} & $\txid_1$ & {a} & $\txid_2$\\
    {a} & $\Set{\txid_1, \txid_2}$ & {a} & $\emptyset$ & {a} & $\emptyset$\\
};

\tikzvalue{versionx-1-1}{versionx-2-1}{locx-v0}{0};
\tikzvalue{versionx-1-3}{versionx-2-3}{locx-v1}{1};
\tikzvalue{versionx-1-5}{versionx-2-5}{locx-v2}{1};


\path (locx.south) + (0,\tikzkeyspace) node (locf1) {$\ke_{\pv{f1}} \mapsto$};
\matrix(versionf1) [version list]
    at ([xshift=\tikzkvspace]locf1.east) {
    {a} & $\txid_0$ & {a} & $\txid_1$\\
    {a} & $\Set{\txid_1}$ & {a} & $\emptyset$\\
};
\tikzvalue{versionf1-1-1}{versionf1-2-1}{locf1-v0}{0};
\tikzvalue{versionf1-1-3}{versionf1-2-3}{locf1-v1}{1};

%Location y
\path (locf1.south) + (0,\tikzkeyspace) node (locf2) {$\ke_{\pv{f2}} \mapsto$};
\matrix(versionf2) [version list]
    at ([xshift=\tikzkvspace]locf2.east) {
    {a} & $\txid_0$ & {a} & \(\txid_2\) \\
    {a} & $\emptyset$ & {a} & \(\emptyset\) \\
};
\tikzvalue{versionf2-1-1}{versionf2-2-1}{locf2-v0}{0};
\tikzvalue{versionf2-1-3}{versionf2-2-3}{locf2-v1}{1};


% \draw[-, red, very thick, rounded corners] ([xshift=-5pt, yshift=5pt]locx-v1.north east) |- 
%  ($([xshift=-5pt,yshift=-5pt]locx-v1.south east)!.5!([xshift=-5pt, yshift=5pt]locy-v0.north east)$) -| ([xshift=-5pt, yshift=5pt]locy-v0.south east);

%blue view - I should  check whether I can use pgfkeys to just declare the list of locations, and then add the view automatically.
\draw[-, blue, very thick, rounded corners=10pt]
([xshift=-2pt, yshift=20pt]locx-v0.north east) node (tid1start) {} -- 
([xshift=-2pt, yshift=-5pt]locf2-v0.south east);
 
\path (tid1start) node[anchor=south, rectangle, fill=blue!20, draw=blue, font=\small, inner sep=1pt] {$\thid_3$};

%red view
\draw[-, red, very thick, rounded corners = 10pt]
([xshift=-5pt, yshift=5pt]locx-v2.north east) -- 
([xshift=-5pt, yshift=-5pt]locx-v2.south east) --
([xshift=-5pt, yshift=5pt]locf1-v0.north east) -- 
([xshift=-5pt, yshift=-5pt]locf1-v0.south east) --
([xshift=-5pt, yshift=5pt]locf2-v1.north east) -- 
([xshift=-5pt, yshift=-10pt]locf2-v1.south east) node (tid2start) {};

\path (tid2start) node[anchor=north, rectangle, fill=red!20, draw=red, font=\small, inner sep=1pt] {$\thid_2$};
 
 %green view
\draw[-, DarkGreen, very thick, rounded corners = 10pt]
([xshift=-16pt, yshift=8pt]locx-v1.north east) node (tid3start) {}-- 
([xshift=-16pt, yshift=-5pt]locx-v1.south east) --
([xshift=-16pt, yshift=5pt]locf1-v1.north east) -- 
([xshift=-16pt, yshift=-5pt]locf1-v1.south east) --
([xshift=-16pt, yshift=5pt]locf2-v0.north east) -- 
([xshift=-16pt, yshift=-5pt]locf2-v0.south east);

\path (tid3start) node[anchor=south, rectangle, fill=DarkGreen!20, draw=DarkGreen, font=\small, inner sep=1pt] {$\thid_1$};

\end{pgfonlayer}%
\end{centertikz}%
\caption{After \(\txid_2\)}
\label{fig:ua-after-tx2}
\end{subfigure}
%
&
%
\begin{subfigure}{0.45\textwidth}
\begin{centertikz}
\begin{pgfonlayer}{foreground}
%Uncomment line below for help lines
%\draw[help lines] grid(5,4);

\node(locx)  {$\ke_\vx \mapsto$};

\matrix(versionx) [version list, column 2/.style = {text width=14mm}]
    at ([xshift=\tikzkvspace]locx.east) {
    {a} & $\txid_0$ & {a} & $\txid_1$ & {a} & $\txid_2$\\
    {a} & $\Set{\txid_1, \txid_2}$ & {a} & $\emptyset$ & {a} & $\emptyset$\\
};

\tikzvalue{versionx-1-1}{versionx-2-1}{locx-v0}{0};
\tikzvalue{versionx-1-3}{versionx-2-3}{locx-v1}{1};
\tikzvalue{versionx-1-5}{versionx-2-5}{locx-v2}{1};


\path (locx.south) + (0,\tikzkeyspace) node (locf1) {$\ke_{\pv{f1}} \mapsto$};
\matrix(versionf1) [version list]
    at ([xshift=\tikzkvspace]locf1.east) {
    {a} & $\txid_0$ & {a} & $\txid_1$\\
    {a} & $\Set{\txid_1}$ & {a} & $\emptyset$\\
};
\tikzvalue{versionf1-1-1}{versionf1-2-1}{locf1-v0}{0};
\tikzvalue{versionf1-1-3}{versionf1-2-3}{locf1-v1}{1};

%Location y
\path (locf1.south) + (0,\tikzkeyspace) node (locf2) {$\ke_{\pv{f2}} \mapsto$};
\matrix(versionf2) [version list]
    at ([xshift=\tikzkvspace]locf2.east) {
    {a} & $\txid_0$ & {a} & \(\txid_2\) \\
    {a} & $\emptyset$ & {a} & \(\emptyset\) \\
};
\tikzvalue{versionf2-1-1}{versionf2-2-1}{locf2-v0}{0};
\tikzvalue{versionf2-1-3}{versionf2-2-3}{locf2-v1}{1};


% \draw[-, red, very thick, rounded corners] ([xshift=-5pt, yshift=5pt]locx-v1.north east) |- 
%  ($([xshift=-5pt,yshift=-5pt]locx-v1.south east)!.5!([xshift=-5pt, yshift=5pt]locy-v0.north east)$) -| ([xshift=-5pt, yshift=5pt]locy-v0.south east);

%blue view - I should  check whether I can use pgfkeys to just declare the list of locations, and then add the view automatically.
\draw[-, blue, very thick, rounded corners=10pt]
([xshift=-2pt, yshift=20pt]locx-v2.north east) node (tid1start) {} -- 
([xshift=-2pt, yshift=-7pt]locx-v2.south east) --
([xshift=-2pt, yshift=3pt]locf1-v1.north east) -- 
([xshift=-2pt, yshift=-5pt]locf1-v1.south east) --
([xshift=-2pt, yshift=5pt]locf2-v1.north east) -- 
([xshift=-2pt, yshift=-5pt]locf2-v1.south east);

\path (tid1start) node[anchor=south, rectangle, fill=blue!20, draw=blue, font=\small, inner sep=1pt] {$\thid_3$};

%red view
\draw[-, red, very thick, rounded corners = 10pt]
([xshift=-5pt, yshift=5pt]locx-v2.north east) -- 
([xshift=-5pt, yshift=-5pt]locx-v2.south east) --
([xshift=-5pt, yshift=5pt]locf1-v0.north east) -- 
([xshift=-5pt, yshift=-5pt]locf1-v0.south east) --
([xshift=-5pt, yshift=5pt]locf2-v1.north east) -- 
([xshift=-5pt, yshift=-10pt]locf2-v1.south east) node (tid2start) {};

\path (tid2start) node[anchor=north, rectangle, fill=red!20, draw=red, font=\small, inner sep=1pt] {$\thid_2$};
 
 %green view
\draw[-, DarkGreen, very thick, rounded corners = 10pt]
([xshift=-16pt, yshift=8pt]locx-v1.north east) node (tid3start) {}-- 
([xshift=-16pt, yshift=-5pt]locx-v1.south east) --
([xshift=-16pt, yshift=5pt]locf1-v1.north east) -- 
([xshift=-16pt, yshift=-5pt]locf1-v1.south east) --
([xshift=-16pt, yshift=5pt]locf2-v0.north east) -- 
([xshift=-16pt, yshift=-5pt]locf2-v0.south east);

\path (tid3start) node[anchor=south, rectangle, fill=DarkGreen!20, draw=DarkGreen, font=\small, inner sep=1pt] {$\thid_1$};

\end{pgfonlayer}
\end{centertikz}%
\caption{\(\txid_3\) updates the view}
\label{fig:ua-before-tx2}
\end{subfigure} \\
\end{tabular}
\hrule
\caption{An invalid executions under update atomic for $\prog_3$}
\label{fig:cu.exec}
\label{fig:cu-exec}
\end{figure}




\begin{figure}
\hrule
\begin{tabular}{@{} c c@{}}

\begin{subfigure}{0.45\textwidth}
\begin{centertikz}

\begin{pgfonlayer}{foreground}
%Uncomment line below for help lines
%\draw[help lines] grid(5,4);

\node(locx)  {$\ke_\vx \mapsto$};

\matrix(versionx) [version list]
    at ([xshift=\tikzkvspace]locx.east) {
    {a} & $\txid_0$ & {a} & \(\txid_1\)\\
    {a} & $\Set{\txid_1}$ & {a} & \(\emptyset\)\\
};

\tikzvalue{versionx-1-1}{versionx-2-1}{locx-v0}{0};
\tikzvalue{versionx-1-3}{versionx-2-3}{locx-v1}{1};


\path (locx.south) + (0,\tikzkeyspace) node (locf1) {$\ke_{\pv{f1}} \mapsto$};
\matrix(versionf1) [version list]
    at ([xshift=\tikzkvspace]locf1.east) {
    {a} & $\txid_0$ & {a} & $\txid_1$\\
    {a} & $\Set{\txid_1}$ & {a} & $\emptyset$\\
};
\tikzvalue{versionf1-1-1}{versionf1-2-1}{locf1-v0}{0};
\tikzvalue{versionf1-1-3}{versionf1-2-3}{locf1-v1}{1};

%Location y
\path (locf1.south) + (0,\tikzkeyspace) node (locf2) {$\ke_{\pv{f2}} \mapsto$};
\matrix(versionf2) [version list]
    at ([xshift=\tikzkvspace]locf2.east) {
    {a} & $\txid_0$ \\
    {a} & $\emptyset$ \\
};
\tikzvalue{versionf2-1-1}{versionf2-2-1}{locf2-v0}{0};


% \draw[-, red, very thick, rounded corners] ([xshift=-5pt, yshift=5pt]locx-v1.north east) |- 
%  ($([xshift=-5pt,yshift=-5pt]locx-v1.south east)!.5!([xshift=-5pt, yshift=5pt]locy-v0.north east)$) -| ([xshift=-5pt, yshift=5pt]locy-v0.south east);

%blue view - I should  check whether I can use pgfkeys to just declare the list of locations, and then add the view automatically.
\draw[-, blue, very thick, rounded corners=10pt]
([xshift=-2pt, yshift=20pt]locx-v0.north east) node (tid1start) {} -- 
([xshift=-2pt, yshift=-5pt]locf2-v0.south east);

\path (tid1start) node[anchor=south, rectangle, fill=blue!20, draw=blue, font=\small, inner sep=1pt] {$\thid_3$};

%red view
\draw[-, red, very thick, rounded corners = 10pt]
([xshift=-5pt, yshift=5pt]locx-v1.north east) -- 
([xshift=-5pt, yshift=-5pt]locf1-v1.south east) --
([xshift=-5pt, yshift=5pt]locf2-v0.north east) -- 
([xshift=-5pt, yshift=-10pt]locf2-v0.south east) node (tid2start) {};

\path (tid2start) node[anchor=north, rectangle, fill=red!20, draw=red, font=\small, inner sep=1pt] {$\thid_2$};
 
 %green view
\draw[-, DarkGreen, very thick, rounded corners = 10pt]
([xshift=-16pt, yshift=8pt]locx-v1.north east) node (tid3start) {}-- 
([xshift=-16pt, yshift=-5pt]locf1-v1.south east) --
([xshift=-16pt, yshift=5pt]locf2-v0.north east) -- 
([xshift=-16pt, yshift=-5pt]locf2-v0.south east);

\path (tid3start) node[anchor=south, rectangle, fill=DarkGreen!20, draw=DarkGreen, font=\small, inner sep=1pt] {$\thid_1$};

\end{pgfonlayer}
\end{centertikz}%
\caption{\(\thid_2\) updates the view}
\label{fig:ua-thid-2-update-view}
\end{subfigure} 
&
\begin{subfigure}{0.45\textwidth}
\begin{centertikz}

\begin{pgfonlayer}{foreground}
%Uncomment line below for help lines
%\draw[help lines] grid(5,4);

\node(locx)  {$\ke_\vx \mapsto$};

\matrix(versionx) [version list]
    at ([xshift=\tikzkvspace]locx.east) {
    {a} & $\txid_0$ & {a} & $\txid_1$ & {a} & $\txid_2$\\
    {a} & $\Set{\txid_1}$ & {a} & $\Set{\txid_2}$ & {a} & $\emptyset$\\
};

\tikzvalue{versionx-1-1}{versionx-2-1}{locx-v0}{0};
\tikzvalue{versionx-1-3}{versionx-2-3}{locx-v1}{1};
\tikzvalue{versionx-1-5}{versionx-2-5}{locx-v2}{2};


\path (locx.south) + (0,\tikzkeyspace) node (locf1) {$\ke_{\pv{f1}} \mapsto$};
\matrix(versionf1) [version list]
    at ([xshift=\tikzkvspace]locf1.east) {
    {a} & $\txid_0$ & {a} & $\txid_1$\\
    {a} & $\Set{\txid_1}$ & {a} & $\emptyset$\\
};
\tikzvalue{versionf1-1-1}{versionf1-2-1}{locf1-v0}{0};
\tikzvalue{versionf1-1-3}{versionf1-2-3}{locf1-v1}{1};

%Location y
\path (locf1.south) + (0,\tikzkeyspace) node (locf2) {$\ke_{\pv{f2}} \mapsto$};
\matrix(versionf2) [version list]
    at ([xshift=\tikzkvspace]locf2.east) {
    {a} & $\txid_0$ & {a} & \(\txid_2\) \\
    {a} & $\emptyset$ & {a} & \(\emptyset\) \\
};
\tikzvalue{versionf2-1-1}{versionf2-2-1}{locf2-v0}{0};
\tikzvalue{versionf2-1-3}{versionf2-2-3}{locf2-v1}{1};

% \draw[-, red, very thick, rounded corners] ([xshift=-5pt, yshift=5pt]locx-v1.north east) |- 
%  ($([xshift=-5pt,yshift=-5pt]locx-v1.south east)!.5!([xshift=-5pt, yshift=5pt]locy-v0.north east)$) -| ([xshift=-5pt, yshift=5pt]locy-v0.south east);

%blue view - I should  check whether I can use pgfkeys to just declare the list of locations, and then add the view automatically.
\draw[-, blue, very thick, rounded corners=10pt]
([xshift=-2pt, yshift=20pt]locx-v0.north east) node (tid1start) {} -- 
([xshift=-2pt, yshift=-5pt]locf2-v0.south east);

\path (tid1start) node[anchor=south, rectangle, fill=blue!20, draw=blue, font=\small, inner sep=1pt] {$\thid_3$};

%red view
\draw[-, red, very thick, rounded corners = 10pt]
([xshift=-5pt, yshift=5pt]locx-v2.north east) -- 
([xshift=-5pt, yshift=-5pt]locx-v2.south east) --
([xshift=-5pt, yshift=5pt]locf1-v1.north east) -- 
([xshift=-5pt, yshift=-10pt]locf2-v1.south east) node (tid2start) {};

\path (tid2start) node[anchor=north, rectangle, fill=red!20, draw=red, font=\small, inner sep=1pt] {$\thid_2$};
 
 %green view
\draw[-, DarkGreen, very thick, rounded corners = 10pt]
([xshift=-16pt, yshift=8pt]locx-v1.north east) node (tid3start) {}-- 
([xshift=-16pt, yshift=-5pt]locf1-v1.south east) --
([xshift=-16pt, yshift=5pt]locf2-v0.north east) -- 
([xshift=-16pt, yshift=-5pt]locf2-v0.south east);

\path (tid3start) node[anchor=south, rectangle, fill=DarkGreen!20, draw=DarkGreen, font=\small, inner sep=1pt] {$\thid_1$};

\end{pgfonlayer}
\end{centertikz}%
\caption{After \(\txid_2\)}
\label{fig:ua-correct-after-tx2}
\end{subfigure} 
\\
\end{tabular}
\hrule
\caption{A execution of $\prog_3$ without lost-update}
\label{fig:ua-conf-2}
\end{figure}


\ac{This Consistency Model shows why the notion of consistent views must 
depend on the set of operations that need to be executed.}

The next consistency model that we consider is \emph{update atomic}. 
Although we did not find any implementation of this model, it has been proposed in \cite{framework-concur} as a strengthening to Read Atomic to avoid write-write conflicts.
This model states that: \textbf{(i)} transactions satisfy atomic visibility (\cref{def:readatomic}); and \textbf{(ii)} transactions writing to one same keys cannot be executed concurrently.
\sx{ This appears too earlier:
Update Atomic is also needed to specify more sophisticated consistency models, 
such as \emph{Parallel Snapshot Isolation} and \emph{Snapshot Isolation}.}
\ac{Check: Nobi said he was interested in implementing Update Atomic 
at some point, maybe he ended up doing something.}

Programs under update atomic do not exhibit the \emph{lost update} anomaly: two or more transactions update the same address, for example , both increment its value by $1$, but only one of them will be observed by future transactions, for example, only one of the increments takes effect.
To illustrate the \emph{lost-update anomaly}, consider the following program \( \prog_3 \) where two transactions concurrently increment $\vx$ and the third transaction read the value. 
Note that the \( \pvar{f1} \) and \( \pvar{f2} \) are two flags indicating the corresponding transactions has been committed.
\ac{Intuitive behaviour of the litmus test: two transactions concurrently increment $[\loc_x]$. 
 A third transaction observes that the first two transactions have been executed. 
 However, it only observes one of the two increments taking place.
 }
\[
    \prog_3 \equiv \begin{session}
        \begin{array}{@{}c || c || c@{}}
        \txid_1 : 
        \begin{transaction} 
            \pmutate{\pvar{f1}}{1};\\
            \pderef{\pvar{a}}{\vx};\\
            \pmutate{\vx}{a + 1};\\
        \end{transaction} & 
        \txid_2 : 
        \begin{transaction}
            \pmutate{\pvar{f2}}{1};\\
            \pderef{\pvar{a}}{\vx};\\
            \pmutate{\vx}{a + 1};\\
        \end{transaction} &
        \txid_3 : 
        \begin{transaction}
            \pderef{\pvar{a}}{\vx};\\
            \pderef{\pvar{b}}{\pvar{f1}};\\
            \pderef{\pvar{c}}{\pvar{f2}};\\
            \pifs{\pvar{a}=1 \wedge \pvar{b}=1 \wedge \pvar{c} = 1}\\ 
                \quad \passign{\retvar}{\sadface}
            \pife
        \end{transaction}
        \end{array}
    \end{session}
 \]

We consider an execution in which the transactions contained in the code of threads $\thid_1, \thid_2$ both execute on the same snapshot determined by the initial view. 
The initial configuration of the program coincides with the one given in \cref{fig:ua-init}.
After executing the transaction $\txid_1$, the resulting configuration is the one depicted \ref{fig:ua-after-tx1} and then \( \txid_2 \) shown in in \ref{fig:ua-after-tx2}, where both transactions read the initial version for key $\ke_\vx$. 
The third transaction $\txid_3$ choose to update its view to include the most recent version for all the keys (\ref{fig:ua-before-tx3}), then when executing its code, all the keys will have value $1$, and the return variable will be set to ${\sadface}$.

The program $\prog_3$ might exhibit the lost-update anomaly when the second transaction $\txid_2$ starts, its view did not include the most up-to-date version for key $\ke_\vx$ provided that \( \txid_{2}\) will update the key \( \ke_\vx \).
As consequence, the database \emph{lost the update} of a version of \( \ke_\vx \) installed by the transaction $\txid_1$, in a sense that no transaction will observe such the version.
To forbid this anomaly, the \emph{update atomic} requires that if a transaction writes to a key, the transaction should start with a view including the most recent version for the key.

\begin{definition}
\label{def:update-atomic}
\emph{Update atomic} is stronger than then read atomic (\cref{def:readatomic}) by further requiring for all keys written, it should starts with a view including the most recent version for those key:
\[
\begin{rclarray}
(\hh, \vi) \csat[\mathsf{UA}] \opset: \vi' & \defeq &
\begin{array}[t]{@{}l}
(\hh, \vi) \csat[\mathsf{RA}] \opset: \vi' \land \fora{\addr} 
(\otW, \addr, \stub) \in \opset \implies \vi(\addr)  = \left| \hh(\addr) \right| - 1
\end{array} \\
\end{rclarray}
\]
\end{definition}

\begin{proposition}
The execution test $\comoUA$ does not hinder progress. 
For any $\hh, \vi, \opset$, there exist $\vi' : \vi \leq \vi'$ and $\vi'': \Vupdate(\hh, \vi', \opset) \leq \vi''$ such that $(\hh, \vi') \csatUA \opset, \vi''$.
\end{proposition}

The thread $\thid_2$ from the program \( \prog_3\) , under $\mathsf{UA}$, cannot execute the transaction $\txid_2$ starting from the configuration depicted in \cref{fig:ua-after-tx1}.
Because the view of $\thid_2$ does not include the most recent version for key $\ke_\vx$. 
Instead, before executing, $\thid_2$ must update its view to include the most recent version of $\ke_\vx$ (\cref{fig:ua-thid-2-update-view}).
Then the \( \txid_2\) will install a new version for \( \ke_\vx \) with value 2 instead of 1 as shown in \cref{fig:ua-correct-after-tx2}.
There are now three different possible views in which $\thid_3$ can execute its transaction.
First, executing on the initial view, in which the transaction will observe 0 for the three locations, and the transaction will not return value $\sadface$.
Second, executing on the one in which the view of $\thid_3$ for $\ke_\vx$ points to the version $(1, \tsid_1, \Set{\tsid_2})$. 
Because of atomic visibility, it must also includes the most recent version for key $\ke_\pv{f2}$ since it is installed by \( \txid_2 \).
In this case, it will not return \(\sadface \).
Last, executing on the one in which the view of $\thid_3$ for $\ke_\vx$ points to its most recent version $(2, \txid_2, \emptyset)$.
In this case, it will not return \(\sadface \).

\subsection{Consistency Prefix \( \CP \) }
\label{sec:sound-complete-cp}

Given abstract execution \( \aexec \), we define read-write read-write relation:
\[
    \RW(\aexec,\ke) \defeq \Setcon{(\txid, \txid')}{\txid \toEdge{\AR_\aexec} \txid' \land (\otR,\ke, \stub ) \in \txid \land (\otW,\ke, \stub ) \in \txid'  } 
\]
It is easy to see \( \RW(\aexec,\ke) \)  can be derived from \( \WW(\aexec,\ke) \) and \( \WR(\aexec, \ke ) \) as the following:
\[
    \RW(\aexec,\ke) = \Setcon{(\txid, \txid')}{ \exsts{\txid'' } (\txid'', \txid) \in \WR(\aexec, \ke) \land (\txid'', \txid') \in \WW(\aexec, \ke) }
\]
Then, the notation \( \RW_\aexec \defeq \bigcup\limits_{\ke \in \Keys} \RW(\aexec, \ke) \).
Note that for a key-value store \( \mkvs \) such that \( \mkvs = \mkvs_\aexec \),
by the definition of  \(  \mkvs = \mkvs_\aexec \), 
the following holds:
\[
    \RW_\aexec = \Setcon{(\txid, \txid')}{\exsts{\ke, i,j } \txid \in \RTx(\mkvs(\ke, i)) \land \txid' = \WTx(\mkvs(\ke, j)) \land i < j}
\]
The \( \RW_\aexec \) also coincides with \( \RW_\Gr \) and \( \RW_\mkvs \).


An abstract execution \( \aexec \) satisfies consistency prefix (\(\CP\)), 
if it satisfies \( \AR_\aexec ; \VIS_\aexec \subseteq \VIS_\aexec \) and \( \SO_\aexec \subseteq \VIS_\aexec \).
Given the specification, there is a corresponding specification on dependency graph by solve the following inequalities:
\[
    \begin{array}{@{}l@{}}
        \WR \subseteq \VIS \\
        \WW \subseteq \AR \\
        \VIS \subseteq \AR \\
        \VIS ; \RW \subseteq \AR \\
        \AR ; \AR \subseteq \AR  \\
        \SO \subseteq \VIS \\
        \AR ; \VIS \subseteq \VIS
    \end{array}
\]
By solving the inequalities the visibility and arbitration relations are:
\[
    \begin{rclarray}
        \AR & \defeq & \left( (\SO \cup \WR ) ; \RW? \cup \WW \cup R \right)^+ \\
        \VIS & \defeq & \left( (\SO \cup \WR ) ; \RW? \cup \WW \cup R \right)^* ; (\SO \cup \WR )
    \end{rclarray}
\]
for some relation \( R \subseteq \AR \).
When \( R = \emptyset \), it is the smallest solution therefore the minimum visibility required.

\sx{A bit verbal}
\begin{lemma}
    \label{lem:cp-eauiv-spec}
    For any abstract execution \( \aexec \),
    if it satisfies 
    \[
        \left( (\SO \cup \WR ) ; \RW? \cup \WW \right)^* ; \VIS_\aexec \subseteq \VIS_\aexec 
        \qquad \SO_\aexec \subseteq \VIS_\aexec
    \]
    then there exists a new \( \aexec' \) such that \( \T_\aexec = \T_{\aexec'} \), 
    under last-write-win \( \TtoOp{T}_{\aexec}(\txid) = \TtoOp{T}_{\aexec'}(\txid) \) for all transactions \( \txid \),
    and the relations satisfy the following:
    \[ 
        \AR_{\aexec'} ; \VIS_{\aexec'} \subseteq \VIS_{\aexec'}  \qquad \SO_{\aexec'} \subseteq \VIS_{\aexec'}
    \]
    and vice versa.
\end{lemma}
\begin{proof}
    Assume abstract execution \( \aexec' \) that
    satisfies \( \AR_{\aexec'} ; \VIS_{\aexec'} \subseteq \VIS_{\aexec'} \)
    and  \( \SO_{\aexec'} \subseteq \VIS_{\aexec'} \).
    We already show that:
\[
    \begin{rclarray}
        \AR_{\aexec'} & = & \left( (\SO_\aexec \cup \WR_\aexec ) ; \RW_\aexec? \cup \WW_\aexec \cup R \right)^+ \\
        \VIS_{\aexec'} & = & \left( (\SO_\aexec \cup \WR_\aexec ) ; \RW_\aexec? \cup \WW_\aexec \cup R \right)^* ; (\SO_\aexec \cup \WR_\aexec )
    \end{rclarray}
\]
for some relation \( R \subseteq \AR_{\aexec'} \).
If we take \( R  = \emptyset \), we have the proof for:
\[
        \SO \subseteq \VIS_\aexec \qquad 
        \left( (\SO_\aexec \cup \WR_\aexec ) ; \RW_\aexec? \cup \WW_\aexec \right)^* ; \VIS_\aexec \subseteq \VIS_\aexec
\]
For another way, we pick the \( R \) that extends
\( \left( (\SO_\aexec \cup \WR_\aexec ) ; \RW_\aexec? \cup \WW_\aexec \cup R \right)^+ \) 
to a total order.
\end{proof}

By \cref{lem:cp-eauiv-spec} to prove soundness and completeness of \( \ET_\CP \), it is sufficient to use the specification:
\[
    (\RP_{\LWW}, \Set{\lambda \aexec. \left( (\SO \cup \WR ) ; \RW? \cup \WW \right)^* ; \VIS_\aexec, \lambda \aexec \ldotp \SO_\aexec }) 
\]

For the soundness, we pick the invariant as the following:
\[  
\begin{rclarray}
    I_1(\aexec, \cl) & = & \left( \bigcup\limits_{\{\txid_{\cl}^{i} \in \T_{\aexec} \mid i \in \Nat\}} \VIS_{\aexec}^{-1}(\txid^i_\cl) \right) \setminus \T_\rd \\
    I_2(\aexec, \cl) & = & \left( \bigcup\limits_{\{\txid_{\cl}^{i} \in \T_{\aexec} \mid i \in \Nat\}} (\SO_{\aexec}^{-1})?(\txid^i_\cl) \right) \setminus \T_\rd
\end{rclarray}
\]
where \( \T_\rd \) is all the read-only transactions included in both 
\( \left( \bigcup\limits_{\{\txid_{\cl}^{i} \in \T_{\aexec} \mid i \in \Nat\}} \VIS_{\aexec}^{-1}(\txid^i_\cl) \right)\) 
and \( \left( \bigcup\limits_{\{\txid_{\cl}^{i} \in \T_{\aexec} \mid i \in \Nat\}} (\SO_{\aexec}^{-1})?(\txid^i_\cl) \right) \).
Assume a key-value store $\hh$, an initial and a final view $\vi, \vi'$  a fingerprint $\opset$ 
such that $\ET_{\CP} \vdash (\hh, \vi) \triangleright \opset: \vi'$. 
Also choose an arbitrary $\cl$, a transaction identifier $\txid_\cl^n \in \nextTxId(\hh, \cl)$, 
and an abstract execution $\aexec$ such that $\hh_{\aexec} = \hh$ and 
\( I_1(\aexec, \cl) \cup I_2(\aexec, \cl) \subseteq \Tx(\hh, \vi) \).
Let a new abstract execution \( \aexec' = \extend(\aexec, \txid_\cl^n, \f, \Tx(\mkvs, \vi) \cup \T_\rd) \).
We are about to prove that there exists an extra set of read-only transaction \( \T'_\rd \) such that:
\begin{gather}
    \fora{\txid} (\txid, \txid_\cl^n) \in \SO_{\aexec'} \implies \txid \in \Tx(\mkvs, \vi) \cup \T_\rd \cup \T'_\rd \label{equ:cp-sound-update-so}\\
    \begin{array}{l}
    \fora{\txid} (\txid, \txid_\cl^n) \in \left( (\SO_{\aexec'} \cup \WR_{\aexec'} ) ; \RW_{\aexec'}? \cup \WW_{\aexec'} \right)^* ; \VIS_{\aexec'} \\
    \qqquad \implies \txid \in \Tx(\mkvs, \vi) \cup \T_\rd \cup \T'_\rd 
    \end{array}
    \label{equ:cp-sound-update-arvis}\\
    I_1(\aexec',\cl) \cup I_2(\aexec',\cl) \subseteq \Tx(\mkvs_{\aexec'}, \vi') \label{equ:cp-sound-inv} 
\end{gather}
\begin{itemize}
\item the invariant \( I_2 \) implies the \cref{equ:cp-sound-update-so} where the proof is the same as \( \RYW \) in \cref{sec:sound-complete-ryw}.

\item For \cref{equ:cp-sound-update-arvis}, it is sufficient to prove one step inclusion, \ie
\[
    \begin{array}{l}
    \fora{\txid} (\txid, \txid_\cl^n) \in \left( (\SO_{\aexec'} \cup \WR_{\aexec'} ) ; \RW_{\aexec'}? \cup \WW_{\aexec'} \right) ; \VIS_{\aexec'} \\
    \qqquad \implies \txid \in \Tx(\mkvs, \vi) \cup \T_\rd \cup \T'_\rd 
\end{array}
\]
To prove above, let \( \T'_\rd \) initially be empty set.
We will add more read-only transactions until it satisfies \cref{equ:cp-sound-update-arvis}.
Assume a transaction \( \txid \) such that 
\( (\txid, \txid_\cl^n) \in \left( (\SO_{\aexec'} \cup \WR_{\aexec'} ) ; \RW_{\aexec'}? \cup \WW_{\aexec'} \right) ; \VIS_{\aexec'}\).
There exists a transaction \( \txid' \) such that \( \txid \toEdge{(\SO_{\aexec'} \cup \WR_{\aexec'} ) ; \RW_{\aexec'}? \cup \WW_{\aexec'}} \txid' \toEdge{\VIS_{\aexec'}}  \txid_\cl^n \).
It follows \( \txid'  \in \Tx(\mkvs, \vi) \cup \T_\rd \cup \T'_\rd  \).
Note that \( \txid \) and \( \txid' \) must exist in the abstract execution \( \aexec \) before update.
There are two cases: \( \txid' \) writes to at least a key; or \( \txid' \) is a read-only transaction.
\begin{itemize}
    \item
    If \( \txid' \) writes to at least a key, then \( \txid' \in \Tx(\mkvs, \vi)\).
    %\[
        %\begin{rclarray}
            %\func{RW^{-1}}{\mkvs, \ke, i} & \defeq & \Setcon{\txid}{\exsts{ j \leq i } \txid \in \RTx(\mkvs(\ke,j))} \\
            %\ddagger & \equiv &
            %\begin{array}[t]{@{}l@{}}
                %\fora{\ke, \ke', i, j, m, \txid, \txid', \txid''} \\
                %\left( \begin{array}{@{}l@{}}
                %i \in \vi(\ke) 
                %\land \txid \in \Set{\WTx(\mkvs(\ke,i))} \cup \func{RW^{-1}}{\mkvs, \ke, i} \land {} \\
                %\quad \left(
                    %\begin{array}{@{}l @{}}
                        %\left( \begin{array}{@{}l@{}}
                                %\txid' \in \func{SO^{-1}}{\txid} \land {} \\
                                %\txid' \in \Set{\WTx(\mkvs(\ke',j))} \cup  \RTx(\mkvs(\ke',j))
                        %\end{array} \right)  \lor {} \\
                        %\left( \begin{array}{@{}l@{}}
                                %\txid \in \RTx(\mkvs(\ke',j)) \land \txid' = \WTx(\mkvs(\ke',j))
                        %\end{array} \right)
                        %\end{array} \right) 
                    %\end{array}
                    %\right)  \\
                    %{} \lor \left( \begin{array}{@{}l@{}}
                            %i \in \vi(\ke) \land \ke = \ke' \land j < i
                    %\end{array} \right) \\
                    %\qquad \implies j \in \vi(\ke') 
            %\end{array} \\
        %\end{rclarray}
    %\]
    %We link the conditions in \( \ddagger \) to relation:
    %\begin{itemize}
        %\item \( \RW_\aexec\). Assume a key \( \ke \),  an index \( i \) and the writer \( \txid  = \WTx(\mkvs(\ke,i))\),
    %then \( \txid' \in \RW^{-1}(\mkvs, \ke, i)\) if and only if \( \txid' \toEdge{\RW_\aexec} \txid\).
        %\item \( \SO_\aexec\). The transaction identifiers encode the \( \SO_\aexec \).
        %That is, \( \txid' \in \SO^{-1}(\txid)\) if and only if \(\txid' \toEdge{\SO_\aexec} \txid \).
        %\item  \( \WR_\aexec \). It is easy to see \( \txid \in \RTx(\mkvs(\ke',j)) \land \txid' = \WTx(\mkvs(\ke',j)) \) if and only if \( \txid' \toEdge{\WR_\aexec} \txid \).
        %\item \( \WW_\aexec \). The write-write relation describes the order of write operations for a key which corresponds the version orders in key-value store.
        %That is, \( \txid' = \WTx(\mkvs(\ke,j)) \land \txid = \WTx(\mkvs(\ke,i)) \land j < i\) if and only if
        %\( \txid' \toEdge{\WW_\aexec} \txid\).
    %\end{itemize}
    %Let assume \( \txid' \) writes to i-\emph{th} version a key \( \ke \).
    %Given above and 
    %\[ \txid \toEdge{(\SO_{\aexec'} \cup \WR_{\aexec'} ) ; \RW_{\aexec'}? \cup \WW_{\aexec'}} \txid' \toEdge{\VIS_{\aexec'}}  \txid_\cl^n \] we can substitute and rewrite the \( \ddagger \) as the following:
    %\begin{gather}
        %\begin{array}{@{}l@{}}
            %\fora{\txid'',\ke',j}
            %\WTx(\mkvs(\ke,i)) = \txid' \land {} \\
            %\left( \begin{array}{@{}l@{}}
            %\txid'' \toEdge{\RW_{\aexec'}?} \txid' \land {} \\
            %\quad \left(
                %\begin{array}{@{}l @{}}
                    %\left( \begin{array}{@{}l@{}}
                            %\txid \toEdge{\SO_{\aexec'}} \txid'' \land 
                            %\txid \in \Set{\WTx(\mkvs(\ke',j))} \cup  \RTx(\mkvs(\ke',j))
                    %\end{array} \right)  \\
                    %{} \lor 
                    %\left( \begin{array}{@{}l@{}}
                            %\txid \toEdge{\WR_{\aexec'}} \txid'' \land \txid = \WTx(\mkvs(\ke',j))
                    %\end{array} \right)
                    %\end{array} \right) 
                %\end{array}
                %\right)  \\
                %{} \lor \left( \begin{array}{@{}l@{}}
                        %\txid \toEdge{\WW_{\aexec'}} \txid'' \land \txid = \WTx(\mkvs(\ke',j))
                %\end{array} \right) \\
                %\qquad \implies j \in \vi(\ke') 
        %\end{array} 
        %\label{equ:cp-dagger}
    %\end{gather}
    Now we perform case analysis if \( \txid \) is a read-only transaction.
    \begin{itemize}
        \item if \( \txid \) has write, we prove \( \txid \in \Tx(\mkvs, \vi)\).
        Recall the \( \ddagger \) is defined as the following:
        \begin{equation}
        \label{equ:cp-dagger}
        \ddagger  \equiv 
            \fora{\ke, \ke', i, j}
                i \in \vi(\ke)  \wedge \WTx(\hh(\ke', j)) \toEdge{(((\PO \cup \RF_{\hh}) ; \AD_{\hh}?) \cup \VO_{\hh})^{+}} \WTx(\hh(\ke, i))
            \implies j \in \vi(\ke')  
        \end{equation}
        Since \( \WR_\mkvs \), \( \WW_\mkvs \) and \( \RW_\mkvs \) coincide with
        \( \WR_\aexec \), \( \WW_\aexec \) and \( \RW_\aexec \) respectively.
        Also because \( \txid \) write to at least one key,
        it is easy to see there exists some version \( \ke'',m\) such that 
        \( \txid = \WTx(\mkvs(\ke'',m))\) and \( m \in \vi(\ke'')\).
        By definition of \( \Tx \), it follows \( \txid \in \Tx(\mkvs, \vi) \).
        %Therefore by the definition of \( \Tx \), then \( \txid \in \VIS^{-1}(\txid_\cl^n)\).
        \item if \( \txid \) is a read-only transaction, we add it into \( \T'_\rd \).
    \end{itemize}
    \item 
    if \( \txid' \) is a read-only transaction, then either \( \txid' \in \T_\rd \) or \( \txid' \in \T'_\rd \).
    More specifically we have three cases: \textbf{(i)} \( \txid' \in \bigcup\limits_{\{\txid_{\cl}^{i} \in \T_{\aexec} \mid i \in \Nat\}} \VIS_{\aexec}^{-1}(\txid^i_\cl) \), \textbf{(ii)} \( \txid' \in \bigcup\limits_{\{\txid_{\cl}^{i} \in \T_{\aexec} \mid i \in \Nat\}} (\SO_{\aexec}^{-1})?(\txid^i_\cl) \) or \textbf{(iii)} \( \txid' \in \T'_\rd\).
    \begin{itemize}
        \item
        Assume \( \txid' \in \bigcup\limits_{\{\txid_{\cl}^{i} \in \T_{\aexec} \mid i \in \Nat\}} \VIS_{\aexec}^{-1}(\txid^i_\cl) \).
        It means \( \txid' \) is visible for some previous transaction \( \txid_\cl^m \) (\( m < n \)) from the same client \( cl \), 
        \ie 
        \[ 
            \txid \toEdge{(\SO_{\aexec'} \cup \WR_{\aexec'} ) ; \RW_{\aexec'}? \cup \WW_{\aexec'}} \txid' \toEdge{\VIS_{\aexec'}}  \txid_\cl^m 
        \]
        Note that all the edges before \( \txid_\cl^m \) must exist in \( \aexec \).
        Since \( \aexec \) satisfies the \( \left( (\SO \cup \WR ) ; \RW? \cup \WW \right)^* ; \VIS_\aexec \subseteq \VIS_\aexec \),
        we have \( \txid \toEdge{\VIS_{\aexec'}} \txid_\cl^m \) and then \( \txid \in \bigcup\limits_{\{\txid_{\cl}^{i} \in \T_{\aexec} \mid i \in \Nat\}} \VIS_{\aexec}^{-1}(\txid^i_\cl)\).
        By the invariant \( I_1 \), it means \( \txid \in \Tx(\mkvs, \vi) \cup \T_\rd \).
    \item \( \txid' \in \bigcup\limits_{\{\txid_{\cl}^{i} \in \T_{\aexec} \mid i \in \Nat\}} \SO_{\aexec}^{-1}(\txid^i_\cl) \).
    Since \( \txid' \) is a read-only transaction, 
    the edges can be simplified to \( \txid \toEdge{(\SO_{\aexec'} \cup \WR_{\aexec'} )} \txid' \toEdge{\SO_{\aexec'}}  \txid_\cl^n \).
    Given that \( \SO \) is transitive, then  either \( \txid \toEdge{\SO_{\aexec'}} \txid_\cl^n \) or \( \txid \toEdge{\WR_{\aexec'} } \txid' \toEdge{\SO_{\aexec'}}  \txid_\cl^n \).
    \begin{itemize}
        \item \( \txid \toEdge{\SO_{\aexec'}} \txid_\cl^n \).
            It follows \( \txid \in \bigcup\limits_{\{\txid_{\cl}^{i} \in \T_{\aexec} \mid i \in \Nat\}} \SO_{\aexec}^{-1}(\txid^i_\cl) = \Tx(\mkvs, \vi) \cup \T_\rd \).
        \item \( \txid \toEdge{\WR_{\aexec'} } \txid' \toEdge{\SO_{\aexec'}}  \txid_\cl^n \).
            The \( \WR \) edge must exists in \( \aexec \).
            Because \( \WR_\aexec \subseteq \VIS_\aexec \) then  \( \txid \toEdge{\VIS_{\aexec} } \txid' \toEdge{\SO_{\aexec'}}  \txid_\cl^n  \).
            It means 
            \[ 
                \txid \in \bigcup\limits_{\{\txid_{\cl}^{i} \in \T_{\aexec} \mid i \in \Nat\}} \VIS_{\aexec}^{-1}(\txid^i_\cl) = \Tx(\mkvs, \vi) \cup \T_\rd 
            \]
    \end{itemize}
    \item 
    Last, \( \txid' \in \T'_\rd \).
    Since \( \T'_\rd \) initially is empty set, there exists another write transaction \( \txid'' \) such that:
    \[
        \txid \toEdge{(\SO_{\aexec'} \cup \WR_{\aexec'} ) ; \RW_{\aexec'}? \cup \WW_{\aexec'}} \txid' \toEdge{(\SO_{\aexec'} \cup \WR_{\aexec'} ) ; \RW_{\aexec'}? \cup \WW_{\aexec'}} \txid'' \toEdge{\VIS_{\aexec'}}  \txid_\cl^n
    \]
    %Given that \( \txid' \) is a read-only and \( \txid'' \) has write, the edges can be simplified:
    %\[
        %\txid \toEdge{(\SO_{\aexec'} \cup \WR_{\aexec'} )} \txid' \toEdge{\SO_{\aexec'} ; \RW_{\aexec'}?} \txid'' \toEdge{\VIS_{\aexec'}}  \txid_\cl^n
    %\]
    %Because transitivity of  \( \SO \), we have the following two cases:
    %\[
        %\begin{array}{@{}l@{}}
            %\txid \toEdge{ \WR_{\aexec'} } \txid' \toEdge{\SO_{\aexec'} ; \RW_{\aexec'}?} \txid'' \toEdge{\VIS_{\aexec'}}  \txid_\cl^n \\
            %\txid \toEdge{\SO_{\aexec'} ; \RW_{\aexec'}?} \txid'' \toEdge{\VIS_{\aexec'}}  \txid_\cl^n 
        %\end{array}
    %\]
    %\( \txid \toEdge{ \WR_{\aexec'} } \txid' \toEdge{\SO_{\aexec'} ; \RW_{\aexec'}?} \txid'' \toEdge{\VIS_{\aexec'}}  \txid_\cl^n \).
        If \( \txid \) has write, by \cref{equ:cp-dagger} then \( \txid \in \Tx(\mkvs,\vi) \).
        Otherwise if \( \txid \) is a read only transaction, we add it into \( \T'_\rd \).
            %\( \txid \toEdge{\SO_{\aexec'} ; \RW_{\aexec'}?} \txid'' \toEdge{\VIS_{\aexec'}}  \txid_\cl^n \).
            %Similarly by \cref{equ:cp-dagger}, either \( \txid \in \Tx(\mkvs,\vi) \)  or we add it into \( \T'_\rd \).
    \end{itemize}
\end{itemize}

\item Since \( \CP \) satisfies \( \RYW \) and \( \MRd \), thus invariants \( I_1 \) and  \( I_2 \) are preserved after update.

\end{itemize}

    
For completeness, we prove the three parts of the execution test separately.
\begin{itemize}
\item Since \( \SO_\aexec \subseteq \VIS_\aexec  \), the prove for \( \ET_\RYW \) is the as in \cref{sec:sound-complete-mr}.
\item For any \( \VIS_\aexec \)  satisfies the constraint for \( \CP \), by \cref{lem:cp-eauiv-spec} it satisfies that 
\[
    \VIS \defeq \left( (\SO \cup \WR ) ; \RW? \cup \WW \cup R \right)^* ; (\SO \cup \WR )
\]
for some relation \( R \).
It means \( \VIS_\aexec ; \SO_\aexec \subseteq \VIS_\aexec \).
Therefore it is complete with respect to \( \ET_\MRd \).

\item Let consider the \( \ddagger \).
Assume i-\emph{th} transaction \( \txid_i \) in the arbitrary order,
and let view \( \vi_{i} = \getView(\aexec, \VIS^{-1}_{\aexec}(\txid_{i}) ) \).
We also pick any final view such that \( \vi'_{i} \subseteq \getView(\aexec, (\AR^{-1}_{\aexec})?(\txid_{i}) ) \).
Note that there is nothing to prove for \( \vi'_i \) since the \( \ddagger \) does not constrain the \( \vi'_i \).
Recall the \( \ddagger \):
\[
\ddagger  \equiv 
        \fora{\ke, \ke', m, j}
             m \in \vi(\ke)  \wedge \WTx(\hh(\ke', j)) \toEdge{(((\PO \cup \RF_{\hh}) ; \AD_{\hh}?) \cup \VO_{\hh})^{+}} \WTx(\hh(\ke, m))
         \implies j \in \vi(\ke')  
\]
Assume \( j \in \vi_i(\ke) \) for some key \(\ke \) and index \( i \).
It means the writer of the version is visible by the transaction \( \txid_i\),
\ie \( \WTx(\mkvs(\ke,i)) \in \VIS^{-1}_{\aexec}(\txid_{i}) \).
Let the \( \mkvs = \mkvs_{\cut(\aexec, i-1)} \).
We need to prove the following:
\begin{gather}
    \label{equ:cp-complete-arvis}
    \begin{array}{@{}l@{}}
        \fora{\ke, \ke', m, j, \txid, \txid'} 
        m \in \vi(\ke) 
        \land \WTx(\mkvs(\ke,m)) \in \VIS_\aexec^{-1}(\txid_i) \\
        \quad {} \land \WTx(\hh(\ke', j)) \toEdge{(((\PO \cup \RF_{\hh}) ; \AD_{\hh}?) \cup \VO_{\hh})^{+}} \WTx(\hh(\ke, m)) \\
            \qquad \implies \WTx(\mkvs(\ke',j)) \in \VIS_\aexec^{-1}(\txid_i)
    \end{array}
\end{gather}
%Note that \( \txid \in \Set{\WTx(\mkvs(\ke,i))} \cup \func{RW^{-1}}{\mkvs, \ke, i} \) 
%means \( \txid \toEdge{\RW_{\aexec}?} \WTx(\mkvs(\ke,i)) \),
%the formulae \(\left( \begin{array}{@{}l@{}} \txid \in \RTx(\mkvs(\ke',j)) \land \txid' = \WTx(\mkvs(\ke',j)) \end{array} \right) \) 
%means \( \txid \toEdge{\WR_\aexec} \txid' \),
%and \( \left( \begin{array}{@{}l@{}} \txid = \WTx(\mkvs(\ke',m)) \land \txid' = \WTx(\mkvs(\ke',j)) \land m > j \end{array} \right) \) 
%means \( \txid \toEdge{\WW_\aexec} \txid' \).
%Given all the correspondence, the \cref{equ:cp-complete-arvis} holds if the following holds:
%\[
    %\begin{rclarray}
        %\begin{array}[t]{@{}l@{}}
            %\fora{\ke, \ke', i, j, \txid, \txid'} \\
            %\left( \begin{array}{@{}l@{}}
            %i \in \vi(\ke) 
            %\land \WTx(\mkvs(\ke,i)) \in \VIS_\aexec^{-1}(\txid_i) \\
            %{} \land \txid' = \WTx(\mkvs(\ke',j))
            %\land \txid \toEdge{\RW_{\aexec}?} \WTx(\mkvs(\ke,i)) \land {} \\
            %\left(
                %\begin{array}{@{}l @{}}
                    %\txid' \toEdge{\WR_\aexec ; \SO_\aexec}\txid \lor
                    %\txid' \toEdge{\SO_\aexec}\txid \lor
                    %\txid' \toEdge{\WR_\aexec}\txid
                    %\end{array} \right) 
                %\end{array}
                %\right)  \\
                %{} \lor \txid' \toEdge{\WW_\aexec} \WTx(\mkvs(\ke,i)) \\
                %\qquad \implies \txid' \in \VIS_\aexec^{-1}(\txid_i)
        %\end{array} \\
    %\end{rclarray}
%\]
%Then the above holds, if the following holds:
%\begin{gather}
    %\label{equ:cp-complete-arvis-2}
    %\begin{rclarray}
        %\begin{array}[t]{@{}l@{}}
            %\fora{\ke, i, \txid} \\
            %\left( \begin{array}{@{}l@{}}
            %i \in \vi(\ke) 
            %\land \WTx(\mkvs(\ke,i)) \in \VIS_\aexec^{-1}(\txid_i) \\
            %{} \land \txid \toEdge{( (\WR_\aexec; \SO_\aexec) \cup \SO_\aexec \cup \WR_\aexec) ; \RW_{\aexec}? \cup \WW_\aexec} \WTx(\mkvs(\ke,i)) 
                %\end{array}
                %\right)  \\
                %\qquad \implies \txid \in \VIS_\aexec^{-1}(\txid_i)
        %\end{array} \\
    %\end{rclarray}
%\end{gather}
Since \( \WR_\mkvs \), \( \WW_\mkvs \) and \( \RW_\mkvs \) coincide with
\( \WR_\aexec \), \( \WW_\aexec \) and \( \RW_\aexec \) respectively,
and \( \left( (\SO \cup \WR ) ; \RW? \cup \WW \right)^* ; \VIS_\aexec \subseteq \VIS_\aexec \),
It implies \cref{equ:cp-complete-arvis}.
\end{itemize}

\subsection{Parallel Snapshot Isolation \(\PSI\)}
\label{sec:sound-complete-psi}

The axiomatic definition for \( \PSI \) is 
\[ 
    (\RP_{\LWW}, \Set{\lambda \aexec. \VIS_{\aexec} ; \VIS_{\aexec}, \lambda \aexec \ldotp \SO_\aexec, \lambda \aexec. \WW_\aexec })
\]
Given the definition, there is a corresponding definition on dependency graph by solve the following inequalities:
\[
    \begin{array}{@{}l@{}}
        \WR \subseteq \VIS \\
        \WW \subseteq \VIS \\
        \SO \subseteq \VIS \\
        \VIS ; \VIS \subseteq \VIS 
    \end{array}
\]
We have \( \VIS = (\WR \cup \WW \cup \SO \cup R)^{+} \) for some \( R \subseteq \AR \).
Thus, there exist a minimum visibility such that 
\[ 
    (\RP_{\LWW}, \Set{\lambda \aexec. (\WR_{\aexec} \cup \WW_{\aexec} \cup \SO ) ; \VIS_{\aexec}, \lambda \aexec \ldotp \SO_\aexec, \lambda \aexec. \WW_\aexec })
\]

To prove soundness, we pick an invariant for the \( \ET_\PSI \) as the union of those for \( \MR\) and \( \RYW \) shown in the following:
\begin{align*}
    I_1(\aexec, \cl) & =  \left( \bigcup_{\Set{\txid_{\cl}^{n} \in \txidset_{\aexec} }[ n \in \Nat ]} \VIS_{\aexec}^{-1}(\txid^n_\cl) \right) \setminus \txidset_\rd \\
    I_2(\aexec, \cl) & =  \left( \bigcup_{\Set{\txid_{\cl}^{n} \in \txidset_{\aexec} }[ n \in \Nat ]} (\SO_{\aexec}^{-1})\rflx(\txid^n_\cl) \right) \setminus \txidset_\rd
\end{align*}
where \( \txidset_\rd \) is all the read-only transactions included in both 
\( \left( \bigcup_{\Set{\txid_{\cl}^{n} \in \txidset_{\aexec} }[ n \in \Nat ]} \VIS_{\aexec}^{-1}(\txid^n_\cl) \right)\) 
and \( \left( \bigcup_{\Set{\txid_{\cl}^{n} \in \txidset_{\aexec} }[ n \in \Nat ]} (\SO_{\aexec}^{-1})\rflx(\txid^n_\cl) \right) \).
Assume a kv-store $\mkvs$, an initial and a final view $\vi, \vi'$  a fingerprint $\fp$ 
such that $\ET_{\PSI} \vdash (\mkvs, \vi) \csat \fp: (\mkvs',\vi')$. 
Also choose an arbitrary $\cl$, a transaction identifier $\txid_\cl^n \in \nextTxid(\mkvs, \cl)$, 
and an abstract execution $\aexec$ such that $\mkvs_{\aexec} = \mkvs$ and 
\( I_1(\aexec, \cl) \cup I_2(\aexec, \cl) \subseteq \Tx[\mkvs, \vi] \).
We are about to prove there exists an extra set of read-only transactions \( \txidset'_\rd \) such that
the new abstract execution \( \aexec' = \extend[\aexec, \txid_\cl^n, \fp, \Tx[\mkvs, \vi] \cup \txidset_\rd \cup \txidset'_\rd] \) and:
\begin{gather}
    \fora{\txid} (\txid, \txid_\cl^n) \in \SO_{\aexec'} \implies \txid \in \Tx[\mkvs, \vi] \cup \txidset_\rd \cup \txidset'_\rd \label{equ:psi-sound-update-so}\\
    \fora{\txid} (\txid, \txid_\cl^n) \in \WW_{\aexec'} \implies \txid \in \Tx[\mkvs, \vi] \cup \txidset_\rd \cup \txidset'_\rd \label{equ:psi-sound-update-ua}\\
    \fora{\txid} (\txid, \txid_\cl^n) \in ( \SO_{\aexec'} \cup \WR_{\aexec'} \cup \WW_{\aexec'} )^{+} ; \VIS_{\aexec'} \implies \txid \in \Tx[\mkvs, \vi] \cup \txidset_\rd \cup \txidset'_\rd \label{equ:psi-sound-update-closure}\\
    I_1(\aexec',\cl) \cup I_2(\aexec',\cl) \subseteq \Tx[\mkvs_{\aexec'}, \vi'] \label{equ:psi-sound-inv} 
\end{gather}
\begin{itemize}
\item The invariant \( I_2 \) implies \cref{equ:psi-sound-update-so} as the same as \( \RYW \) in \cref{sec:sound-complete-ryw}.
\item Since \( \PSI \) also satisfies \( \UA \), the \cref{equ:si-sound-update-ww} can be proven as the same as \( \UA \) in \cref{sec:sound-complete-ua}.
\item \cref{equ:psi-sound-update-closure}.
    Note that \( (\txid, \txid_\cl^n) \in ( \SO_{\aexec'} \cup \WR_{\aexec'} \cup \WW_{\aexec'}); \VIS_{\aexec'} \implies (\txid, \txid_\cl^n) \in ( \SO_{\aexec} \cup \WR_{\aexec}  \cup \WW_{\aexec} ) ; \VIS_{\aexec'}\).
    Also, recall that \( \SO_\aexec = \SO_\mkvs \), \( \WR_\aexec = \WR_\mkvs \) and  \( \WW_\aexec = \WW_\mkvs \).
    Let \( \txidset'_\rd = \lfpTx[\mkvs,\vi,\SO_{\mkvs} \cup \WR_{\mkvs} \cup \WW_{\mkvs}] \). 
    This means that \( \aexec' = \extend[\aexec, \txid_\cl^n, \fp, \lfpTx[\mkvs, \vi, \SO_{\mkvs} \cup \WR_{\mkvs}] \cup \txidset_\rd ] \).
    Let assume \( \txid \toEDGE{\SO_{\mkvs} \cup \WR_{\mkvs} \cup \WW_{\mkvs}} \txid' \) and \( \txid' \in \lfpTx[\mkvs, \vi, \SO_{\mkvs} \cup \WR_{\mkvs}] \cup \txidset_\rd \).
    We have two possible cases:
    \begin{itemize}
        \item If \( \txid' \in \lfpTx[\mkvs, \vi, \SO_{\mkvs} \cup \WR_{\mkvs} \cup \WW_{\mkvs}] \), by  \cref{thm:view-vis-relation}, we know \( \txid \in \lfpTx[\mkvs, \vi, \SO_{\mkvs} \cup \WR_{\mkvs} \cup \WW_{\mkvs}] \).
        \item If \( \txid' \in \txidset_\rd \), there are two cases:
        \begin{itemize}
            \item \( \txid' \in  \left( \bigcup_{\Set{\txid_{\cl}^{n} \in \txidset_{\aexec} }[ n \in \Nat ]} \VIS_{\aexec}^{-1}(\txid^n_\cl) \right) \).
                Since \( \txid' \) is a read-only transaction, it means \( \txid \toEDGE{\SO_{\mkvs} \cup \WR_{\mkvs} } \txid' \).
                By the property of \( \aexec \) (before update) that \( \SO \cup \WR_\aexec \in \VIS_\aexec \), it is known that \( \txid \in \left( \bigcup_{\Set{\txid_{\cl}^{n} \in \txidset_{\aexec} }[ n \in \Nat ]} \VIS_{\aexec}^{-1}(\txid^n_\cl) \right) \), that is, \( \txid \in \Tx[\mkvs,\vi] \cup \txidset_\rd\).

            \item \( \txid' \in  \left( \bigcup_{\Set{\txid_{\cl}^{n} \in \txidset_{\aexec} }[ n \in \Nat ]} \SO_{\aexec}^{-1}(\txid^n_\cl) \right) \).
                Given that \( \txid' \) is a read only transaction, we know \( \txid \in (\SO \cup \WR_\aexec)^{-1} \left( \bigcup_{\Set{\txid_{\cl}^{n} \in \txidset_{\aexec} }[ n \in \Nat ]} \SO_{\aexec}^{-1}(\txid^n_\cl) \right) \).
                By the property of \( \aexec \) (before update) that \( \SO \cup \WR_\aexec \in \VIS_\aexec \),
                it follows:
                \begin{align*}
                    \txid & \in VIS_\aexec^{-1} \left( \bigcup_{\Set{\txid_{\cl}^{n} \in \txidset_{\aexec} }[ n \in \Nat ]} \SO_{\aexec}^{-1}(\txid^n_\cl) \right) \\
                          & = \left( \bigcup_{\Set{\txid_{\cl}^{n} \in \txidset_{\aexec} }[ n \in \Nat ]} \VIS_{\aexec}^{-1}(\txid^n_\cl) \right)  \\
                          & = \Tx[\mkvs,\vi] \cup \txidset_\rd
                \end{align*}
                
        \end{itemize}
    \end{itemize}
\item Finally the new abstract execution preserves the invariant \( I_1 \) and \( I_2 \) 
because  \( \CC \) satisfies \( \MW \) and \( \RYW \).
\end{itemize}

Given that \( \VIS_\aexec = (\WR_\aexec \cup \WW_\aexec \cup \SO_\aexec \cup R)^{+} \),
we know \( \VIS_\aexec ; \SO_\aexec \subseteq \VIS_\aexec \).
First the completeness follows \( \MR \) in \cref{sec:sound-complete-mr}, \( \RYW \) in \cref{sec:sound-complete-ryw} and  \( \UA \) in \cref{sec:sound-complete-ua}.
Similarly, by \cref{lem:aexec-spec-cc},

An abstract execution \( \aexec \) satisfies snapshot isolation (\(\SI\)), 
if it satisfies
\( \{\lambda \aexec. \AR[\aexec] ; \VIS[\aexec], \lambda \aexec \ldotp \SO, \allowbreak \lambda \aexec. \WW[\aexec] \}) \),
which is intersection of \( \CP \) and \( \UA \) on abstract executions \citep{SIanalysis}.
\citet{SIanalysis} also proposed the minimum visibility relation that gives rise of the following equivalent definition
\[
    \visaxioms[\SI] \FuncDef
    \Set{\lambda \aexec. \left( (\WR[\aexec]  \cup \SO \cup \WW[\aexec] ) ; \Refl(\RW[\aexec]) \right) ; \VIS[\aexec]
            ,\lambda \aexec \ldotp \SO, \lambda \aexec \ldotp \WW[\aexec] }  .
\]

The execution test \( \et[\CP] \) is sound with respect to the axiomatic definition \( \visaxioms[\SI] \)
We pick the invariant \( \aexecinv[\CP] = \aexecinv[\RYW]\).
\SOUNDLET{\CP}{ \txidsetrd \supseteq
\begin{multlined}[t]
\left( \bigcup_{\Set{\txid[\cl](\idx) | \txid[\cl](\idx) \in \aexec}} 
\VISInv[\aexec](\txid[\cl](\idx)) \cup \Refl((\Inv(\SO)))(\txid[\cl](\idx)) \right) 
\setminus \Set{\txid' | \Forall{l | \key | \val } (l,\key,\val) \in \aexec(\txid') \implies l = \opR } .
\end{multlined} }
Assume 
\[ 
\txidsetrd' = 
\begin{multlined}[t]
\left( \bigcup_{\Set{\txid[\cl](\idx) | \txid[\cl](\idx) \in \aexec}} 
\VISInv[\aexec](\txid[\cl](\idx)) \cup \Refl((\Inv(\SO)))(\txid[\cl](\idx)) \right) 
\setminus \Set{\txid' | \Forall{l | \key | \val } (l,\key,\val) \in \aexec(\txid') \implies l = \opR } .
\end{multlined} 
\]
and \( \txidsetrd'' = \txidsetrd \setminus \txidsetrd' \).
By the definition of soundness, we prove the following result:
\begin{Formulae}
& \begin{Formula}
    \Inv(\SO)(\txid) \subseteq \txidset \cup \txidsetrd' 
    \label{equ:si-sound-update-so}
\end{Formula}
\\ & \begin{Formula}
    \Inv(\WW)(\txid) \subseteq \txidset 
    \label{equ:si-sound-update-ua}
\end{Formula}
\\ & \begin{Formula}
    \Inv(\left( (\WR[\aexec] \cup \SO \cup \WW[\aexec] ) ; \Refl(\RW[\aexec]) \right)) (\txid) 
            \subseteq \txidset \cup \txidsetrd' \cup \txidsetrd''
     \label{equ:si-sound-update-closure}
\end{Formula}
\\ & \begin{Formula}
    \aexecinv[\PSI](\aexec',\cl) \subseteq \VisTrans(\XToK(\aexec'),\vi')
    \label{equ:si-inv-preserve}
\end{Formula}
\end{Formulae}
\Cref{equ:si-sound-update-so,equ:si-sound-update-ua} 
can be proven in the same way as in \cref{sec:sound-complete-mr,sec:sound-complete-ua} respectively.
We now prove \cref{equ:si-sound-update-closure}.
Initially we take \( \txidsetrd'' \) to be an empty set.
Note that \(\VISInv[\aexec'](\txid) = \txidset \cup \txidsetrd' \cup \txidsetrd'' \).
By \cref{thm:view-vis-relation,equ:view-close-to-aexec}, there exists \( \txidsetrd'' \) such that
\( \txidset \cup \txidsetrd'' \) is closed under \( \left( (\WR[\aexec] \cup \SO \cup \WW[\aexec] ) ; \Refl(\RW[\aexec]) \right)\).
Now consider a transaction \( \txidrd \in \txidsetrd' \) and
assume a transaction \( \txid' \) such that \( \ToEdge{ \txid' | (\WR[\aexec] \cup \SO \cup \WW[\aexec] ) ; \Refl(\RW[\aexec]) -> \txidrd } \).
Since \( \txidrd \) is a read-only transaction, thus
\( \ToEdge{ \txid' |  (\SO \cup \WR[\aexec] )  -> \txidrd } \)
and the rest proof is exactly the same as as in \cref{sec:sound-complete-cc}.
Last, \cref{equ:si-inv-preserve} can be proven in the same way as in \cref{sec:sound-complete-mr,sec:sound-complete-ryw}.

The execution test $\et[\SI]$ is complete with respect to the axiomatic definition \( \visaxioms[\SI] \).
By \citet{SIanalysis}, it suffices to prove completeness with respect to the following definition,
\[
\Set{\lambda \aexec \ldotp \AR[\aexec] ; \VIS[\aexec], \lambda \aexec \ldotp \SO , \lambda \aexec \ldotp \WW[\aexec] } .
\]
\COMPLETELET{\SI}
By the definition of \( \et[\SI]\), we prove \( \CanCommit[\SI] \), \( \ViewShift[\MR]\) and \( \ViewShift[\RYW]\) respectively.
Recall that \( \CanCommit[\SI] = \PreClosed(\kvs,\vi,\rel[\UA] \cup \left( (\WR[\aexec] \cup \SO \cup \WW[\aexec] ) ; \Refl(\RW[\aexec]) \right)) \).
It is easy to see that 
\[
\begin{multlined}
\PreClosed(\kvs,\vi,\rel[\UA] \cup \WR[\kvs] \cup \SO \cup \WW[\kvs]) \iff {}
    \\ \PreClosed(\kvs,\vi,\rel[\UA]) 
    \land \PreClosed(\kvs,\vi,\left( (\WR[\aexec] \cup \SO \cup \WW[\aexec] ) ; \Refl(\RW[\aexec]) \right)) . 
\end{multlined}
\]
The predicate \( \PreClosed(\kvs,\vi,\rel[\UA]) \) can be proven in the same way as in \cref{sec:sound-complete-ua}
Because
\begin{align*}
\left( (\WR[\aexec] \cup \SO \cup \WW[\aexec]) ; \Refl(\RW[\aexec])  \right) ; \VIS[\aexec] 
        & \subseteq \left( \VIS[\aexec] ; \Refl(\RW[\aexec]) \right) ;  \VIS[\aexec]
        & \subseteq  \AR[\aexec] ; \VIS[\aexec] \subseteq \VIS[\aexec] .
\end{align*}
Then \( \PreClosed(\kvs,\vi,\left( (\WR[\aexec] \cup \SO \cup \WW[\aexec] ) ; \Refl(\RW[\aexec]) \right)) \)
can be derived from \cref{thm:view-vis-relation,equ:aexec-close-to-view}.
The predicate \( \ViewShift[\RYW] \) can be proven in the same way as in \cref{sec:sound-complete-ryw}.
Since \( \VIS[\aexec] ; \SO \subseteq \AR[\aexec] ; \VIS[\aexec] \subseteq \VIS[\aexec] \)
\( \ViewShift[\MR] \) can be proven in the same way as in \cref{sec:sound-complete-mr}.

\paragraph{Strict serialisability (\(\SER\))}  
This model is the strongest consistency model
in any framework that abstracts from aborted transactions, 
requiring that transactions execute in a total sequential order.
The \(\CanCommit[\SER]\) thus allows a client to commit a transaction only 
when the client view on the kv-store is complete
in that the view is closed with respect to \(\WWInv[\kvs]\). 
This requirement prevents the kv-store in  \cref{fig:ser-disallowed}.
Without loss of generality, suppose that \(\txid\) commits before \(\txid'\),
then the client committing \(\txid'\) must see the version of \(\key_1\) written by \(\txid\), 
and thus cannot read the outdated value \(\val_0\) for \(\key_1\). 
This example, known as \emph{write skew anomaly}, 
is allowed by all other execution tests in \cref{fig:execution-tests}.

\begin{figure}
\centering
\begin{tikzpicture}%
\KVMapping{x}{\key_1}{
    /\val_0/\txid_0/\Set{\boldsymbol{\txid'}}
    , /\val_1/\txid/\emptyset
};
\KVMapping[x]{y}{\key_2}{
    /\val_0/\txid_0/\Set{\txid}
    , /\val_2/\boldsymbol{\txid'}/\emptyset
};
\end{tikzpicture}%

\hrulefill

\caption{Write skew anomaly, disallowed by \(\SER\)}
\label{fig:ser-disallowed}
\end{figure}%

% Program analysis
\section{Program Analysis}

Contents: \\
transactional implementation of a counter. \\
Counter is not robust when a 
weak consistency model (i.e. causal consistency, in practice anything that does not 
guarantee both write-conflict detection and monotonic reads) is assumed. \\
A single counter is robust as long as the consistency model used by the database 
guarantees both write-conflict detection and monotonic reads.\\
Multiple counters are not robust even when monotonic reads and write-conflict 
detection are guaranteed by the database. In particular, multiple counters are 
not robust when PSI is guaranteed by the database.\\
Multiple counters are robust if a stronger consistency models, such as SI, is employed 
by the database.

\paragraph{Counter Code}
\[
\begin{array}{lrlr}
\mathsf{inc}(\ke) = &
\begin{session}
\begin{transaction}
\pderef{\pv{a}}{\ke};\\
\pmutate{\ke}{\pv{a}+1};
\end{transaction}
\end{session}
&
\hspace{10pt}\mathsf{read}(\ke) = &
\begin{session}
\begin{transaction}
\pderef{\pv{a}}{\ke};\\
\end{transaction}
\end{session}
\end{array}
\]

%For the moment, we assume that the key-value store contains a single object $\ke$. 
Also, clients can interact with the key-value store only by invoking the $\mathsf{inc}(\ke)$ and 
$\mathsf{read}(\ke)$ operations.
Given the transactional code 
$\ptrans{\trans}$, we define $\opset(\hh, \vi, \ptrans{\trans})$ 
to be the fingerprint that would be produced by a client that has view $\vi$ 
over the kv-store $\hh$, upon executing $\ptrans{\trans}$.
For example, we have that 
%$\opset(\hh, \vi, \mathsf{inc}(\ke))$ is the fingerprint generated 
%by a client $\cl$ with view $\vi$ after executing the operation $\mathsf{inc}(\ke)$. 
%Specifically, 
$\opset(\hh, \vi, \mathsf{inc}(\ke)) = \{(\otR, \ke, n), (\otW, \ke, n+1) \mid 
n = \snapshot(\hh, \vi)\}$. Similarly, $\opset(\hh, \vi ,\mathsf{read}(\ke)) = 
\{(otR, \ke, n) \mid n = \snapshot(\hh, \vi)\}$.
A transactional module is a set of transaction codes $\{\ptrans{\trans_i}_{i \in I}\}$.
Given an execution test $\ET$, and a transactional module $\{\ptrans{\trans_i}\}_{i \in I}$, 
we define the set of valid $\ET$-traces for the transactional module as the set 
$\mathsf{Traces}(\ET, \{\ptrans{\trans_i}\}_{i \in I})$ 
of $\ET$-traces in which only $\ET$-reductions of the form 
\[
(\hh, \viewFun) \xrightarrowtriangle{\left(\cl, \opset\left(\hh, \viewFun(\cl), \ptrans{\trans_{i}}\right)\right)}_{\ET} (\hh', \viewFun')
\]
are allowed. 
We also let $\mathsf{KVStores}(\ET, \{\ptrans{\trans_i}\}_{i \in I})$ be the set of kv-stores 
that can be obtained when clients can only perform operations from $\{\ptrans{\trans_i}\}_{i \in I}$ 
under the execution test $\ET$. Specifically, 
\[
\mathsf{KVStores}(\ET, \{\ptrans{\trans_i}\}_{i \in I}) = \{ \hh \mid \left( (\hh_0, \viewFun_{0}) \xrightarrowtriangle{\cdot}_{\ET} \cdots 
\xrightarrowtriangle{\cdot}_{\ET} (\hh, \_)\right) \in \mathsf{Traces}(\ET, \{\ptrans{\trans_{i}}\}_{i \in I}) \}.
\]

\paragraph{Anomaly under Causal Consistency.}
It is well known that even when the key-value store consists of a single object $\ke$, which 
can be manipulated via the $\mathsf{inc}(\ke)$ and $\mathsf{read}(\ke)$ transactions, 
then it is possible to obtain non-serialisable executions over the kv-store. Henceforth, 
we write $(\hh, \viewFun) \xrightarrowtriangle{(\cl, \ptrans{\trans})}_{\ET} (\hh', \viewFun')$ 
as a shorthand for $(\hh, \viewFun) \xrightarrowtriangle{\left(\cl, \opset\left(\hh, \viewFun(\cl), \ptrans{\trans}\right)\right)}_{\ET} (\hh', \viewFun')$. 
For simplicity, let us assume that $\Keys = \ke$.
Let $\hh_{0} = [\ke \mapsto (0, \txid_{0}, \emptyset)]$,  
$\hh_1 = [\ke \mapsto (0, \txid_{0}, \{\txid_{\cl_1}^{1}\} \lcat (0, \txid_{\cl_1}^{1}, \emptyset)$, 
$\hh_2 = [\ke \mapsto (0, \txid_{0}, \{\txid_{\cl_1}^{1}, \txid_{\cl_2}^{1}\}) \lcat (0, \txid_{\cl_1}^{1}, \emptyset) 
\lcat (0, \txid_{\cl_2}^{1}, \emptyset)$. Let also
$\vi_{0} = [\ke \mapsto {0}]$. Then we have that 
\[
(\hh_{0}, [\cl_1 \mapsto \vi_0, \cl_2 \mapsto \vi_0]) \xrightarrowtriangle{(\cl_1, \mathsf{inc(\ke)})}_{\ET_{\CC}} 
(\hh_1, [\cl_1 \mapsto \_, \cl_2 \mapsto \vi_0]) \xrightarrowtriangle{(\cl_1, \mathsf{inc(\ke)})}_{\ET_{\CC}} 
(\hh_2, \_).
\]
If we now draw the dependency graph $\graphof(\hh_2)$, we immediately find a cycle of the form 
$\txid_{\cl_1}^{1} \xrightarrow{\AD(\ke)} \txid_{\cl_2}^{1} \xrightarrow{\AD(\ke)} \txid_{\cl_1}^{1}$, 
which proves that $\hh_2$ cannot be obtained in any serialisable execution.

\paragraph{Robustness of a Single counter under Parallel Snapshot Isolation.}
In practice we show that a single counter is robust under any consistency model 
that guarantees both write conflict detection (formalised by the execution test 
$\ET_{\UA}$) and monotonic reads (formalised by the execution test $\ET_{\MRd}$). 
\begin{proposition}
\label{prop:counter_hhshape}
Let $\mathsf{Counter} = \{\mathsf{inc}(\ke), \mathsf{read}(\ke)\}$.
Let $\hh$ be in $\mathsf{KVStores}(\ET_{\UA} \cap \ET_{\MRd} \cap \ET_{\RYW}, \mathsf{Counter})$. Then we have that 
there exist $\{\txid_i\}_{i = 1}^{n}$ and $\{\T_{i}\}_{i = 0}^{n}$ such that 
\begin{align}
\hh(\ke) = \left( (0, \txid_{0}, \T_{0} \uplus \{\txid_1\}) \lcat \cdots \lcat (n-1, \txid_{n-1}, \T_{n-1} \uplus \{\txid_{n}\}) \right) 
\lcat (n, \txid_{n}, \T_{n}) \label{eq:psi_counter_shape}\\
\forall i=0,\cdots,n.\T_{i} \cap \{\txid_{i}\}_{i=0}^{n} = \emptyset \label{eq:psi_counter_rwtxs}\\
\forall \txid, \txid'.\;\forall i,j=0,\cdots, n.\; \txid \xrightarrow{\PO} \txid' 
\wedge \txid \in \{\txid_{i}\} \cup \T_{i} \implies 
\left(\begin{array}{l}
(\txid' = \txid_{j} \implies i < j) \wedge {} \\
(\txid' \in \T_{j} \implies i \leq j) \\
%(\txid \in \T_{i} \wedge \txid' = \txid_{j} \implies i < j) \wedge {} \\
%(\txid \in \T_{i} \wedge \txid' \in \T_{j} \implies i \leq j)
\end{array}\right) \label{eq:psi_counter_so}
\end{align}
%
%
%
%for any $i = 1,\cdots, n$, $\T_{i} \cap \{\txid_i\}_{i=0}^{n} = \emptyset$, and 
%$\hh(\ke) = \left(\prod_{i=0}^{n-1} (i, \txid_{i}, \T_{i} \uplus \{\txid_{i+1}\}) \right) \lcat 
%(n, \txid_{n}, \T_{n})$. Furthermore, if there exist four indexes $i,j, p, q$ such that 
%$\txid_{\cl}^{p} \in \WTx(\hh(\ke, i) \cup \RTx(\hh(\ke, i))$ and 
%$\txid_{\cl}^{q} \in \WTx(\hh(\ke, j) \cup \RTx(\hh(\ke, j))$, then 
%$i < j \implies p < q$.
\end{proposition}

\begin{proof}
It suffices to prove that the properties \eqref{eq:psi_counter_rwtxs}, \eqref{eq:psi_counter_shape} 
\eqref{eq:psi_counter_so} given in \cref{prop:counter_hhshape}, are invariant under 
$\ET$-reductions of the form 
\begin{align}
(\hh, \viewFun) \xrightarrow{(\cl, \mathsf{inc}(\ke))}_{\ET_{\UA} \cap \ET_{\MRd} \cap \ET_{\RYW}} (\hh', \viewFun') \label{eq:psi_counter_inc}\\
(\hh, \viewFun) \xrightarrow{(\cl, \mathsf{read}(\ke))}_{\ET_{\UA} \cap \ET_{\MRd} \cap \ET_{\RYW}} (\hh', \viewFun) \label{eq:psi_counter_read}
\end{align}
to this end, we will need the following auxiliary result which holds for any configuration $(\hh, \viewFun)$ 
that can be obtained under the execution test $\ET_{\RYW} \cap \ET_{\MRd}$:
%\begin{equation}
%\forall i, n.\; \forall \cl.\; \txid_{\cl}^{n} \in \{\WTx(\hh(\ke, i))\} \cup \RTx(\hh(\ke, i)) \implies 
%\exists j \geq i.\;\viewFun(\cl) = [\ke \mapsto \{0, \cdots, j\}] \label{eq:psi_counter_view}
%\end{equation} 

\begin{equation}
\forall i, n.\; \forall \cl.\; \txid_{\cl}^{n} \in \{\WTx(\hh(\ke, i))\} \cup \RTx(\hh(\ke, i)) \implies 
i \in \viewFun(\cl)\label{eq:psi_counter_view}
\end{equation} 
Suppose that there exist two sets $\{\txid_{i}\}_{i=1}^{n}$ and 
$\{\T_{i}\}_{i=0}^{n}$ such that $(\hh, \{\txid_{i}\}_{i=1}^{n}, \{\T_{i}\}_{i=1}^{n})$ 
satisfies the properties \eqref{eq:psi_counter_shape}-\eqref{eq:psi_counter_so}. 
We prove that, for transitions of the form \eqref{eq:psi_counter_inc}-\eqref{eq:psi_counter_read}, 
there exist an index $m$ and two collections $\{\txid_{i}\}_{i=1}^{m}$, $\{\T'_{i}\}_{i=0}^{m}$ 
such that $(\hh', \{\txid_{i}\}_{i=1}^{m}, \{\T'_{i}\}_{i=0}^{m})$ satisfies the properties 
\eqref{eq:psi_counter_shape}-\eqref{eq:psi_counter_so}. We consider the two transitions separately.

\begin{itemize}
\item 
Assume that
\[
(\hh, \viewFun) \xrightarrow{(\cl, \mathsf{inc}(\ke))}_{\ET_{\UA} \cap \ET_{\MRd} \cap \cap \ET_{\RYW}} (\hh', \viewFun')
\]
for some $\cl, \hh', \viewFun'$. Let $n+1 = \lvert \hh(\ke) \rvert$. Because of the definition of 
$\ET_{\UA}$, we must have that $\viewFun(\cl) = [\ke \mapsto \{0, \cdots, n\}]$. Also, 
because $\hh$ satisfies \eqref{eq:psi_counter_shape}, we have that $\snapshot(\hh, \viewFun(\cl))(\ke) = n$. 
In particular, $\opset(\ke, \viewFun(\cl), \mathsf{inc}(\ke)) = \{(\otR, \ke, n), (\otW, \ke, n+1)\}$. 
Thus we have that $\hh' \in \updateKV(\hh, \viewFun(\cl), \cl, \{(\otR, \ke, n), (\otW, \ke, n+1)\})$. 
Let $\txid_{n+1}$ be the transaction identifier 
chosen to update $\hh$, i.e. $\hh' = \updateKV(\hh, \viewFun(\cl), \txid_{n+1}, \{(\otR, \ke, n), (\otW, \ke, n+1)\})$, 
where $\txid_{n+1} \in \nextTxId(\hh, \cl)$; 
let also $\T_{n+1} = \emptyset$. Then we have the following: 
\begin{itemize}
\item  $(\hh', \{\txid_{i}\}_{i=1}^{n+1}, \{\T_{i}\}_{i=0}^{n+1})$ satisfies Property \eqref{eq:psi_counter_shape}. 
%Recall that $\hh' \in \updateKV(\hh, \viewFun(\cl), \cl, \{(\otR, \ke, n), (\otW, \ke, n+1)\}$. 
Recall that $(\hh, \{\txid_{i}\}_{i=1}^{n}, \{\T_{i}\}_{i=0}^{n})$ satisfies \eqref{eq:psi_counter_shape}, 
i.e.
\[\hh(\ke) = \left( (0, \txid_{0}, \T_{0} \uplus \{\txid_1\}) \lcat \cdots \lcat (n-1, \txid_{n-1}, \T_{n-1} \uplus \{\txid_{n}\}) \right) 
\lcat (n, \txid_{n}, \T_{n}).
\]
%for some $\{\txid_{i}\}_{i=1}^{n}$ and $\{\T_{i}\}_{i=0}^{n}$. 
%Let $\txid_{n+1}$ be the transaction identifier 
%chosen to update $\hh$, i.e. $\hh' = \updateKV(\hh, \viewFun(\cl), \txid_{n+1}, \{(\otR, \ke, n), (\otW, \ke, n+1)\})$, 
%where $\txid_{n+1} \in \nextTxId(\hh, \cl)$. 
It follows that $\hh'(\ke) = \left( (0, \txid_{0}, \T_{0} \uplus \{\txid_1\}) \lcat \cdots \lcat (n-1, \txid_{n-1}, \T_{n} \uplus \{\txid_{n+1}\}) \right) 
\lcat (n+1, \txid_{n+1}, \T_{n+1})$, 
where we recall that $\T_{n+1} = \emptyset$.
%By choosing $\T_{n+1} := \emptyset$, then we have that $(\hh', \{\txid_{i}_{i=1}^{n+1}, \{\T_{i}\}_{i=0}^{n+1})$ 
%satisfies Property \eqref{eq:psi_counter_shape}, 

\item $(\hh', \{\txid_{i}\}_{i=1}^{n+1}, \{\T_{i}\}_{i=0}^{n+1})$ 
satisfies Property \eqref{eq:psi_counter_rwtxs}. Let $i =0, \cdots, n+1$. If $i = n+1$, then 
$\T_{i} = \emptyset$, from which $\T_{i} \cap \{\txid_{j}\}_{j=0}^{n+1} = \emptyset$ follows. If $i < n+1$, then 
because $(\hh, \{\txid_{i}\}_{i=1}^{n}, \{\T_{i}\}_{i=0}^{n})$ 
satisfies Property \eqref{eq:psi_counter_rwtxs}, then $\T_{i} \cap \{\txid_{j}\}_{j=0}^{n} = \emptyset$. 
Finally, because $\txid_{n+1}$ was chosen to be fresh with respect to the transaction identifiers appearing in 
$\hh$, and $\T_{i} \subseteq \RTx(\hh(\ke, i))$, then  we also have that $\T_{i} \cap \{\txid_{n+1}\} = \emptyset$. 
%By combining all these facts, we obtain that $\hh'$ satisfies Property \eqref{eq:psi_counter_rwtxs}.
\item $(\hh', \{\txid_{i}\}_{i=1}^{n+1}, \{\T_{i}\}_{i=0}^{n+1})$ satisfies Property \eqref{eq:psi_counter_so}. Let 
$\txid, \txid'$ be such that $\txid \xrightarrow{\PO} \txid'$. Choose two arbitrary indexes $i,j=0,\cdots, n+1$, 
and assume that $\txid \in \{\txid_{i}\} \cup \T_{i}$. Note that if $i \leq n$, $j \leq n$, then 
because $(\hh, \{\txid_{i}\}_{i=1}^{n}, \{\T_{i}\}_{i=0}^{n})$ satisfies Property $\eqref{eq:psi_counter_so}$, then 
if $\txid' = \txid_{j}$ it follows that $i < j$, and if $\txid' \in \T_{j}$ it follows that $i \leq j$, as 
we wanted to prove. 
If $\txid \in \{\txid_{n+1}\} \cup \T_{n+1}$, then it must be $\txid = \txid_{n+1}$ because 
$\T_{n+1} = \emptyset$. Recall that $\txid_{n+1}$ is the transaction identifier that was used 
to update $\hh$ to $\hh'$, i.e. $\hh' = \updateKV(\hh, \viewFun(\cl), \txid_{n+1}, \_)$. By 
definition of $\updateKV$, it follows that $\txid_{n+1} \in \nextTxId(\hh, \cl)$, 
and because $\txid_{n+1} \xrightarrow{\PO} \txid'$, then $\txid'$ cannot appear in $\hh'$. 
In particular, 
$\txid' \notin \{\txid_{j}\}_{j=0}^{n+1} \cup \bigcup \{\T_{j}\}_{j=0}^{n+1}$, hence in this case there is nothing to prove. 
Finally, if $\txid' \in \{\txid_{n+1}\} \cup \T_{n+1}$, then 
it must be the case that $\txid' = \txid_{n+1}$. If $\txid = \txid_{j}$, because 
$\txid \xrightarrow{\SO} \txid'$ and $\txid' = \txid_{n+1}$, it cannot be $\txid = \txid_{n+1}$, 
hence it must be $i \leq n < n+1$. 
%If $\txid \in \T_{i}$, because $\T_{n+1} = \emptyset$ it 
%follows that $j < i$. 
%\item $(\hh', \viewFun')$ satisfies Property \eqref{eq:psi_counter_view}.  
%Consider an arbitrary client $\cl'$. If $\cl' \neq \cl$, $\viewFun'(\cl') = \viewFun(\cl')$: if $\txid_{\cl'}^{m} \in \{\WTx(\hh'(\ke, i))\} 
%\cup \RTx(\hh'(\ke, i))$ for some $i = 0,\cdots, n+1$, then because $\RTx(\hh'(\ke, n+1)) = \emptyset$, $\cl' \neq \cl$ and $\txid_{n+1} = \txid_{\cl}^{\cdot}$, 
%then it must be the case that $i \leq n$. Because $(\hh, \viewFun)$ satisfies Property \eqref{eq:psi_counter_view}, 
%then there exists an index $j \geq i$ such that $\viewFun'(\cl') = \viewFun(\cl') = [\ke \mapsto \{0,\cdots, j\}]$. 
%Finally, suppose that $\cl' = \cl$. 
%By definition of $\ET_{\UA}$, $\viewFun(\cl) = [\ke \mapsto \{0,\cdots, n\}]$, and the definition 
%of $\ET_{\MRd}$ and $\ET_{\RYW}$ imply that $\viewFun'(\cl) = [\ke \mapsto \{0,\cdots, n+1\}]$. 
%Clearly, whenever $\txid_{\cl}^{\cdot} \in  \{\WTx(\hh'(\ke, i))\} 
%\cup \RTx(\hh'(\ke, i))$ for some $i$, then it must be the case that $i \leq n$.
\end{itemize}

\item Suppose that $(\hh, \viewFun)$ satisfies the properties \eqref{eq:psi_counter_rwtxs}-\eqref{eq:psi_counter_view}; 
assume also that 
\[
(\hh, \viewFun) \xrightarrow{(\cl, \mathsf{read}(\ke))}_{\ET_{\UA} \cap \ET_{\MRd} \cap \cap \ET_{\RYW}} (\hh', \viewFun')
\]
As in the previous case, we have that $\hh' = \updateKV(\hh, \viewFun(\cl), \txid, \{(\otR, \ke, m)\})$, where 
$m = \snapshot(\hh, \viewFun(\cl))(\ke)$ and $\txid \in \nextTxId(\hh, \cl)$. 
We also have that $\viewFun'(\cl') = \viewFun(\cl')$ for any $\cl' \neq \cl$, and 
from the definition of $\ET_{\MRd}$ we also have that $\viewFun(\cl) \viewleq \viewFun'(\cl)$. 
Henceforth, we let $i = \max_{<}\viewFun(\cl)(\ke)$.
Putting all these facts together, we obtain the following: 
\begin{itemize}
\item $\hh'$ satisfies Property \eqref{eq:psi_counter_shape}. 
Because $\hh$ satisfies $\eqref{eq:psi_counter_shape}$, then if $i < n$, then $\hh(\ke, i) = 
(i, \txid_{i}, \T_{i} \cup \{\txid_{i+1}\})$;  
otherwise $i = n$ and $\hh(\ke, i) = \hh(\ke, n) = (n, \txid_{n}, \T_{n})$. 
In both cases we have that 
hence $m = \snapshot(\hh(\ke), \viewFun(\cl)) = \valueOf(i, \_, \_) = i$.

Because $\hh' = \updateKV(\hh, \viewFun(\cl), \txid, \{(\otR, \ke, n)\}$, 
then $\lvert \hh'(\ke) \rvert = \lvert \hh(\ke) \rvert = n+1$. 
By definition of $\updateKV$, we have that for any $j=0,\cdots, n$, 
$j \neq i$, then $\hh(\ke, j) = \hh'(\ke, j)$; without loss of generality, 
let us assume that $i \neq n$: we also have that 
\[
\begin{array}{l}
\hh'(\ke, i) = \text{let } (v, \txid', \T) = \hh(\ke, i) \text{ in } (v, \txid', \T \cup \{\txid\}) = \\
\text{let } (v, \txid', \T) = (i, \txid_{i}, \T_{i} \cup \{\txid_{i+1}\}) \text{ in } 
(v, \txid', \T \cup \{\txid\}) = (n ,\txid', \T_{i} \cup \{\txid \} \cup \{\txid_{i+1}\}.
\end{array}
\]
Similarly, in the case that $i = n$, we have that $\hh'(\ke, i) =(i, \txid_{i}, \T_{i} \cup \{\txid\})$.
If we let $\T'_{j} := \T_{j}$ for any $j \neq i$, and $\T'_{i} := \T_{i} \cup \{ \txid \}$, then 
we have that $\hh'$ satisfies Property \ref{eq:psi_counter_shape}, relatively to the sets 
$\{\txid_{i}\}_{i=1}^{n}$ and $\{\T'_{j}\}_{j = 0}^{n}$.

\item $\hh'$ satisfies Property \eqref{eq:psi_counter_rwtxs}. To this end, let $j =0,\cdots, n$, 
and consider the set of transactions $\T'_{i}$. Let again $i = \max_{<}\viewFun(\cl)(\ke)$ . 
If $j \neq i$, then $\T'_{j} = \T_{j}$, and because $\hh$ satisfies Property \eqref{eq:psi_counter_rwtxs} 
we have that $\T'_{j} \cap \{\txid_{i}\}_{i=0}^{n} = \emptyset$. If $j = i$, then 
we have that $\T'_{j} = \T'_{i} = \T_{i} \cup \{\txid\}$, where we recall that $\txid \in \nextTxId(\hh, \cl)$. 
Because $\hh$ satisfies Property \eqref{eq:psi_counter_rwtxs}, we have that $\T_{i} \cap \{\txid_{i}\}_{i=0}^{n} 
= \emptyset$. Finally, because $\txid \in \nextTxId(\hh,\cl)$, then it must be the case that 
for any $h = 0,\cdots, n$, $\txid \notin \{\WTx(\hh(\ke,h))\}$,  from which it follows that $\{\txid\} 
\cap \{\txid_{i}\}_{i=0}^{n} = \emptyset$. By putting these two facts together, we obtain that 
$\T'_{i} \cap \{\txid_{i}\}_{i=0}^{n} = (\T_{i} \cup \{\txid\}) \cap \{\txid_{i}\}_{i=0}^{n} = 
emptyset$. 

\item $\hh'$ satisfies Property \eqref{eq:psi_counter_so}. By assumption, we know 
that $(\hh, \viewFun)$ satisfies properties \eqref{eq:psi_counter_so} and \eqref{eq:psi_counter_view}. 
Let $\txid', \txid''$ be such that $\txid \xrightarrow{\PO} \txid'$. 
Suppose also that $\txid' \in \{\txid_{h}\} \cup \T'_{h}$ for some $h = 0,\cdots, n$. We consider two different cases:
\begin{itemize}
\item $\txid' = \txid_{h}$. Suppose then that $\txid' = \txid_{j}$ for some $j = 0, \cdots, n$. Because 
$\hh$ satisfies Property \eqref{eq:psi_counter_so}, then it must be the case that $h < j$. Otherwise, 
suppose that $\txid'' \in \T'_{j}$ for some $j=0,\cdots,n$. If $j \neq i$, then $\T'_{j} = \T_{j}$, 
and because $\hh$ satisfies Property \eqref{eq:psi_counter_so}, we have that $h \leq j$. 
Otherwise, $\T'_{j} = \T_{j} \cup \{\txid\}$. Without loss of generality, in this case 
we can assume that $\txid'' = \txid$ (we have already shown that if $\txid'' \in \T_{j}$, then 
it must be $h \leq j$. Recall that $j = i = \max(\viewFun(\cl)(\ke))$, and by Definition of 
$\ET_{\UA}$ it must be the case that $\viewFun(\cl) = [\ke \mapsto \{0,\cdots, j\}]$. 
It also follows that $\txid = \txid_{\cl}^{p}$ for some $p \geq 0$, and because $\txid' \xrightarrow{\PO} \txid'' = \txid$, 
then $\txid' = \txid_{\cl}^{q}$ for some $q > 0$. Because $(\hh, \viewFun)$ satisfies Property 
\eqref{eq:psi_counter_view}, and because $\txid' = \txid_{h} = \WTx(\hh(\ke, h))$, then it must be the case that 
$h \leq j$.
\item $\txid' \in \T'_{h}$. We need to distinguish whether $h \neq i$, in which case $\T'_{h} = \T_{h}$, 
or $h = i$, in which case $\T'_{h} = \T_{h} \cup \{\txid\}$. If either $h \neq i$, or $h = i$ and $\txid \in 
\T_{h}$, then we can proceed as in the case $\txid' = \txid_{h}$. Otherwise, suppose that $h = i$ and 
$\txid' = \txid$. Then, because $\txid' \xrightarrow{\PO} \txid''$, and $\txid \in \nextTxId(\hh,\cl)$, 
it must be the case that $\txid = \txid_{\cl}^{p}$ for some $p \geq 0$, and whenever 
$\txid_{\cl}^{q} \in \ke$, then $\txid_{\cl}^{q} \xrightarrow{\PO} \txid$. In particular 
we cannot have that $\txid'' \in \ke$, because $\txid \xrightarrow{\PO} \txid''$, which 
concludes the proof.
\end{itemize}

\item $(\hh', \viewFun')$ satisfies Property \eqref{eq:psi_counter_view}.

\end{itemize}

\end{itemize}
%%It suffices to prove that, given a reduction of the form 
%%\[
%%(\hh, \viewFun) \xrightarrow{(\cl, \mathsf{inc}(\ke))}_{\ET_{\UA} \cap \ET_{\MRd}} (\hh', \viewFun)
%%\]
%%then if $\hh$ has the shape described in \cref{prop:counter_hhshape}, then so has $\hh'$, 
%%and similarly for $\mathsf{read}(\ke)$ operations. In practice, we can show that the result holds for 
%%$\ET$-reductions of the form 
%%\[
%%(\hh, \viewFun) \xrightarrow{(\cl, \mathsf{inc}(\ke))}_{\ET_{\UA}} (\hh', \viewFun)
%%\]
%%and $\ET$-reductions of the form 
%%\[
%%(\hh, \viewFun) \xrightarrow{(\cl, \mathsf{read}(\ke))}_{\ET_{\MRd}} (\hh', \viewFun)
%%\]
%%Proving this second result also requires to place an invariant on $\viewFun(\cl)$, which states 
%%that if $\txid_{\cl}^{n} \in \WTx(\hh) \cup \RTx(\hh)$, then there exists $m \geq n$ such that 
%%$\viewFun(\cl) = \{i \mid i \leq m\}$.
\end{proof}

%\begin{corollary}
%Given $\hh \in \mathsf{KVStores}(\ET_{\UA} \cap \ET_{\MRd} \cap \ET_{\RYW}, \mathsf{Counter})$, 
%then $\graphof(\hh)$ is acyclic.
%\end{corollary}
% verify implementation
\section{Verification of implementations}

We verify two protocols, COPS and Closk-SI, that the former is a full replicated implementation for causal consistency and the latter is a shard implementation for snapshot isolation.

\subsection{COPS}
\label{sec:cops}
\renewcommand{\thelstlisting}{\arabic{lstlisting}}
\lstinputlisting{\RootPath/cops/pseudocode}

\section{semantics\label{sec:semantics}}

Assume that heap and stack are initialised to zero.

\[
    \begin{rclarray}
        \loc \in \Loc & \defeq & \Nat \\
        \val \in \Val & \defeq & \Nat \uplus \Loc \\
        \Var & \defeq & \Set{ \vx, \vy, \dots } \\
        \ts \in \Timestamp & \defeq & \Nat \\
        \hp \in \Heap & \defeq & \Loc \parfun \Val \\
        \stk \in \Stack & \defeq & \Var \to \Val \\
        \rs \in \Readset, \ws \in \Writeset & \defeq & \powerset{\Loc} \\
        \lstt = (\hp, \stk, \rs, \ws ) \in \Localstate & \defeq & \Stack \times \Heap \times \Readset \times \Writeset \\
        \op \in \Operation & \defeq & \Set{\opr, \opw} \\
        \settrans \subseteq \TransID & \defeq & \Set{ \alpha , \beta, \dots } \\
        \tshp \in \Timestampheap & \defeq & \Loc \parfun ( \Timestamp \parfun \Val \times \Operation \times \TransID) \\
        \ThreadID & \defeq & \Set{ i , j, \dots } \\
        (\tshp, \stk, \ts) \in \Threadstate & \defeq & \Timestampheap \times \Stack \times \Timestamp \\
        \tdpl \in \Threadpool & \defeq & \ThreadID \parfun \Stack \times \Timestamp \times \prog \\
        \stt \in \State & \defeq & \Timestampheap \times \Threadpool \\
    \end{rclarray}
\]

No side effect of evaluation of arithmetic expression.

\[
    \begin{syntax}{\texpr}
              \val \quad            |
        \quad \var \quad            |
        \quad \texpr + \texpr \quad |
        \quad \texpr * \texpr \quad |
        \quad \dots 
    \end{syntax}
\]

No side effect of evaluation of boolean expression.

\[
    \begin{rclarray}
        \eval{\val}_{\stk} & \defeq & \val \\
        \eval{\var}_{\stk} & \defeq & \stk(\val) \\
        \eval{\texpr_{1} + \texpr_{2}}_{\stk} & \defeq & \eval{\texpr_{1}}_{\stk} + \eval{\texpr_{2}}_{\stk}   \\
        \eval{\texpr_{1} * \texpr_{2}}_{\stk} & \defeq & \eval{\texpr_{1}}_{\stk} * \eval{\texpr_{2}}_{\stk}  
    \end{rclarray}
\]

\[
    \begin{syntax}{\tbool}
              \true \quad                  |
        \quad \false \quad                 |
        \quad \texpr = \texpr \quad        |
        \quad \texpr < \texpr \quad        |
        \quad \boolnot \tbool \quad        |
        \quad \tbool \booland \tbool \quad |
        \quad \tbool \boolor \tbool \quad  |
        \quad \dots 
    \end{syntax}
\]

\[
    \begin{rclarray}
        \eval{\true}_{\stk}& \defeq & \true \\
        \eval{\false}_{\stk} & \defeq & \false \\
        \eval{\texpr_{1} = \texpr_{2}}_{\stk} & \defeq & \eval{\texpr_{1}}_{\stk} = \eval{\texpr_{2}}_{\stk}   \\
        \eval{\texpr_{1} < \texpr_{2}}_{\stk} & \defeq & \eval{\texpr_{1}}_{\stk} < \eval{\texpr_{2}}_{\stk}   \\
        \eval{\boolnot \tbool}_{\stk} & \defeq & \neg \eval{\tbool}_{\stk} \\
        \eval{\tbool_{1} \booland \tbool_{2}}_{\stk} & \defeq & \eval{\tbool_{1}}_{\stk} \land \eval{\tbool_{2}}_{\stk}  \\
        \eval{\tbool_{1} \boolor \tbool_{2}}_{\stk}& \defeq & \eval{\tbool_{1}}_{\stk} \lor \eval{\tbool_{2}}_{\stk}  
    \end{rclarray}
\]

\[
    \begin{syntax}{\tcmd}
              \tskip \quad                     |
        \quad \tass{\vx}{\texpr} \quad         |
        \quad \tmutate{\texpr}{\texpr} \quad   |
        \quad \tderef{\vx}{\texpr} \quad       |
        \quad \tif{\tbool}{\tcmd}{\tcmd} \quad | \\
              \tloop{\tbool}{\tcmd} \quad      |
        \quad \tcmd \tseq \tcmd
    \end{syntax}
\]

\[
    \begin{rclarray}
        \dontcare, \dontcare, \dontcare, \dontcare, \dontcare \ \localtransfer \ \dontcare, \dontcare, \dontcare, \dontcare, \dontcare & \defeq & \Localstate \times \tcmd \times \Localstate \times \tcmd \\
    \end{rclarray}
\]

\[
    \infer[ass]{%
        \stk, \hp, \rs, \ws, \tass{\var}{\texpr} \ \localtransfer \  \stk \remapsto{\var}{\val}, \hp, \rs, \ws, \tskip
    }{%
    \eval{\texpr}_{\stk} = \val
    }
\]

\[
    \infer[mutate]{%
        \stk, \hp, \rs, \ws, \tmutate{\texpr_{1}}{\texpr_{2}} \ \localtransfer \  \stk, \hp \remapsto{\loc}{\val}, \rs, \ws \cup \Set{\loc}, \tskip
    }{%
        \eval{\texpr_{1}}_{\stk} = \loc \quad 
        \eval{\texpr_{2}}_{\stk} = \val \quad 
        \loc \in \dom(\hp)
    }
\]

\[
    \infer[deref]{%
        \stk, \hp, \rs, \ws, \tderef{\var}{\texpr} \ \localtransfer \  \stk \remapsto{\var}{\val}, \hp, \rs \cup \Set{\loc}, \ws, \tskip
    }{%
        \eval{\texpr}_{\stk} = \loc \quad 
        \val = \hp(\loc) \quad
        \loc \in \dom(\hp)
    }
\]

\[
    \infer[ifelsetrue]{%
        \stk, \hp, \tif{\tbool}{\tcmd_{1}}{\tcmd_{2}} \ \localtransfer \  \stk, \hp, \tcmd_{1}
    }{%
        \eval{\tbool}_{\stk} = \true
    }
\]

\[
    \infer[ifelsefalse]{%
        \stk, \hp, \tif{\tbool}{\tcmd_{1}}{\tcmd_{2}} \ \localtransfer \  \stk, \hp, \tcmd_{2}
    }{%
        \eval{\tbool}_{\stk} = \false
    }
\]

\[
    \infer[whiletrue]{%
        \stk, \hp, \tloop{\tbool} \tcmd \ \localtransfer \  \stk, \hp,  \tcmd \tseq \tloop{\tbool} \tcmd
    }{%
        \eval{\tbool}_{\stk} = \true
    }
\]

\[
    \infer[whilefalse]{%
        \stk, \hp, \tloop{\tbool} \tcmd \ \localtransfer \  \stk, \hp, \tskip
    }{%
        \eval{\tbool}_{\stk} = \false \quad
    }
\]

\[
    \infer[seqskip]{%
        \stk, \hp, \tskip \tseq \tcmd_{2} \ \localtransfer \  \stk, \hp, \tcmd_{2}
    }{%
    }
\]

\[
    \infer[seqnonskip]{%
        \stk, \hp, \tcmd_{1} \tseq \tcmd_{2} \ \localtransfer \  \stk', \hp', \tcmd_{1}' \tseq \tcmd_{2}
    }{%
        \stk, \hp, \tcmd_{1} \ \localtransfer \  \stk', \hp', \tcmd_{1}'
    }
\]

The semantics of transaction are interleaving of start, commit, and restart.

\[
    \begin{syntax}{\prog}
              \pemp \quad               |
        \quad \ptrans{\tcmd} \quad      |
        \quad \prog \pcond \prog \quad  |
        \quad \prept{\prog} \quad       |
        \quad \prog \pseq \prog \quad   |
        \quad \pfork{\var}{\prog} \quad |
        \quad \pjoin{\texpr}   
    \end{syntax}
\]

\[
    \begin{rclarray}
        \prog_{1} \ppar \prog_{2} & \equiv & \pfork{\var}{\prog_{1}} \pseq \prog_{2} \pseq \pjoin{\var} \\
        \tll \in \Translabel & \defeq & 
              \lid \quad                |
              \quad \lfork{\prog} \quad |
        \quad \ljoin{\thid,\ts} \\
        \dontcare, \dontcare, \dontcare, \dontcare \ \threadtransfer{ \dontcare } \ \dontcare, \dontcare, \dontcare, \dontcare & \defeq & \Threadstate \times \prog \times \Translabel \times \Threadstate \times \prog \\
    \end{rclarray}
\]

\[
    \begin{rclarray}
        \func{startstate}(\tshp,\ts) & \defeq & \lambda \loc \ldotp \tshp(\loc)(\max(\Set{\ \ts' \ \middle| \ \ts' \leq \ts \land \tshp(\ts') = (\dontcare,\wop, \dontcare) }))
    \end{rclarray}
\]

\[
    \begin{rclarray}
        \pred{allowcommit}(\tshp,\ws,\rs,\ts_{s},\ts_{e}) & \defeq & 
        \pred{atomicop}(\tshp,\ws,\rs,\ts_{s},\ts_{e}) \land {} \\
        & & \pred{consistent}(\tshp,\ws,\rs,\ts_{s},\ts_{e}) \\
        \pred{atomicop}(\tshp,\ws,\rs,\ts_{s},\ts_{e}) & \defeq  & \forall \loc \in \ws \cup \rs \ldotp \tshp(\loc)(\ts_{s})\undef \land \tshp(\loc)(\ts_{e})\undef \\
        \pred{consistent}(\tshp,\ws,\rs,\ts_{s},\ts_{e}) & \defeq & \forall \ts \in [\ts_{s},\ts_{e}], \loc \in \ws \ldotp \tshp(\loc)(\ts) \neq (\dontcare, \wop, \dontcare) \land {} \\
                                                       & & \exists \ts_{min} = \min(\Set{\ts'' \ \middle| \ \ts'' \geq \ts_{e} \land \tshp(l)(\ts'')\isdef}) \ldotp \\
                                                       & & \ts_{min} \neq \bot \implies \tshp(\loc)(\ts_{min}) = (\dontcare, \wop, \dontcare) \\
        \func{commit}(\tshp,\hp,\ws,\rs,\ts_{s},\ts_{e}) & \defeq &
        \lambda \loc \ldotp
        \begin{funcarray}
            \tshp(\loc) & \loc \notin \ws \cup \rs \\
            \tshp(\loc) \uplus \Set{ \ts_{e} \mapsto (\hp(\loc),\wop,\tsid)} & \loc \in \ws \\
            \tshp(\loc) \uplus \Set{ \ts_{s} \mapsto (\hp(\loc),\rop,\tsid)} & \loc \in \rs \\
        \end{funcarray} \\
        & & \texttt{where} \  \tsid \notin \Set{\tshp(\loc)(\ts)\projection{3} \ \middle| \ \loc \in \dom(\tshp) \land \ts \in \dom(\tshp(\loc))} \\
    \end{rclarray}
\]

\[
    \infer[commit]{%
        \tshp, \stk, \ts, \ptrans{\tcmd} \ \threadtransfer{\lid} \  \tshp', \stk', \ts_{e}, \pemp
    }{%
        \begin{array}{c}
            \ts_{s} \geq \ts \quad \stk, \func{startstate}(\tshp, \ts_{s}), \emptyset, \emptyset \localtransfer^{*} \stk', \hp, \rs, \ws \\
            \pred{allowcommit}(\tshp,\ws,\rs,\ts_{s},\ts_{e}) \quad \ts_{e} > \ts_{s} \quad \tshp' = \func{commit}(\tshp,\hp,\ws,\rs,\ts_{s},\ts_{e})
        \end{array}
    }
\]

\[
    \infer[choiceleft]{%
        \tshp, \stk, \ts, \prog_{1} \pcond \prog_{2} \ \threadtransfer{\lid} \  \tshp, \stk, \ts, \prog_{1}
    }{%
    }
\]

\[
    \infer[choiceright]{%
        \tshp, \stk, \ts, \prog_{1} \pcond \prog_{2} \ \threadtransfer{\lid} \  \tshp, \stk, \ts, \prog_{2}
    }{%
    }
\]

\[
    \infer[norep]{%
        \tshp, \stk, \ts, \prept{\prog} \ \threadtransfer{\lid} \  \tshp, \stk, \ts, \pemp
    }{%
    }
\]

\[
    \infer[rep]{%
        \tshp, \stk, \ts, \prept{\prog} \ \threadtransfer{\lid} \  \tshp, \stk, \ts, \prog \pseq \prept{\prog}
    }{%
    }
\]

\[
    \infer[seqskip]{%
        \tshp, \stk, \ts, \pemp \pseq \prog \ \threadtransfer{\lid} \  \tshp, \stk, \ts, \prog
    }{%
    }
\]

\[
    \infer[seqnoskip]{%
        \tshp, \stk, \ts, \prog_{1} \pseq \prog_{2} \ \threadtransfer{\tll} \  \tshp', \stk', \ts', \prog_{1}' \pseq \prog_{2}
    }{%
        \tshp, \stk, \ts, \prog_{1} \ \threadtransfer{\tll} \  \tshp', \stk', \ts', \prog_{1}' 
    }
\]

\[
    \infer[fork]{%
        \tshp, \stk, \ts, \pfork{\var}{\prog} \ \threadtransfer{\lfork{\thid,\prog}} \  \tshp, \stk\remapsto{\var}{\thid}, \ts, \pemp 
    }{%
    }
\]

\[
    \infer[join]{%
        \tshp, \stk, \ts, \pjoin{\texpr} \ \threadtransfer{\ljoin{\eval{\texpr}_{\stk},\ts'}} \  \tshp, \max\Set{\ts,\ts'}, \pemp 
    }{%
    }
\]

\[
    \begin{rclarray}
        \dontcare, \dontcare \ \globaltransfer{ \dontcare } \ \dontcare, \dontcare & \defeq & \State \times \Translabel \times \State  \\
    \end{rclarray}
\]

\[
    \infer[single]{%
        \tshp, \tdpl \uplus \Set{ \thid \mapsto (\stk, \ts, \prog) } \ \globaltransfer{\tll} \  \tshp', \tdpl \uplus \Set{ \thid \mapsto (\stk', \ts', \prog') }
    }{%
        \tshp, \stk, \ts, \prog \ \threadtransfer{\tll} \  \tshp', \stk', \ts', \prog' 
        \quad \tll \notin \Set{\lfork{\dontcare,\dontcare},\ljoin{\dontcare,\dontcare}}
    }
\]

\[
    \infer[fork]{%
        \tshp, \tdpl \uplus \Set{ \thid \mapsto (\stk, \ts, \prog) } \ \globaltransfer{\lfork{\thid',\prog''}} \  \tshp', \tdpl \uplus \Set{ \thid \mapsto (\stk', \ts', \prog'), \thid' \mapsto (\lambda \var \ldotp 0, \ts', \prog'') }
    }{%
        \tshp, \stk, \ts, \prog \ \threadtransfer{\lfork{\thid',\prog''}} \  \tshp', \stk', \ts', \prog' 
    }
\]

\[
    \infer[join]{%
        \tshp, \tdpl \uplus \Set{ \thid \mapsto (\stk, \ts, \prog), \thid' \mapsto (\stk', \ts'', \pemp) } \ \globaltransfer{\ljoin{\thid',\ts''}} \  \tshp', \tdpl \uplus \Set{ \thid \mapsto (\stk', \ts', \prog')}
    }{%
        \tshp, \stk, \ts, \prog \ \threadtransfer{\ljoin{\thid',\ts''}} \  \tshp', \stk', \ts', \prog' 
    }
\]

\begin{lem}
    A history cannot be overwritten, i.e.\ \( \forall \tshp, \tshp', \loc,\ts \ldotp \tshp, \dontcare \globaltransfer{\dontcare} \tshp', \dontcare \land \tshp(\loc)(\ts)\isdef \implies \tshp(\loc)(\ts) = \tshp'(\loc)(\ts)\)
\end{lem}
\begin{proof}
    From the \( \pred{allowcommit} \).
\end{proof}

\begin{lem}
    \label{lem:read-before-write}
    All the reads of a transaction happen before all the writes. This is 
    \( \forall \tshp, \loc, \loc', \ts, \ts', \tsid \ldotp \tshp(\loc)(\ts) = (\dontcare, \rop, \tsid) \land \tshp(\loc')(\ts') = (\dontcare, \wop, \tsid) \implies \ts < \ts' \).
\end{lem}
\begin{proof}
    From the semantics that \( \ts_{s} < \ts_{e} \).
\end{proof}

\begin{lem}
    \label{lem:atoic-rw}
    All the reads of a transaction happen in the same time, so do all the writes. This is 
    \( \forall \tshp, \loc, \loc', \ts, \ts', \tsid, \op \in \Set{\rop, \wop} \ldotp \tshp(\loc)(\ts) =  \tshp(\loc')(\ts') = (\dontcare, \op, \tsid) \implies \ts = \ts' \).
\end{lem}
\begin{proof}
    From the \( \func{commit} \).
\end{proof}

Now we need to recover \( \vis \) and \( \ar \) from \( \tshp \).
First we need to extend the \( \tshp \) because there are some transactions that only have reads or writes.
We stretch the time by 3, and add extra operation for those transactions.
For a transaction \( \tsid \) that only has reads event, says in time \( \ts \), we add end operations \( (\bot, \tsid, \eop ) \) to the heap cells it reads in time \( (\ts + 1 ) \).
Similarly for a transaction that only has writes event, we add end operations \( (\bot, \tsid, \sop ) \) in time \( (\ts-1) \).

\[
\begin{rclarray}
    \func{stretch}(\tshp) & \defeq & \lambda \loc \ldotp \lambda \ts \ldotp
    \begin{funcarray}
        \tshp(\loc)(\ts') & \ts = 3 * \ts' \\
        \texttt{undef} & o.w. \\
    \end{funcarray} \\
    \func{extend}(\tshp) & \defeq & \lambda \loc \ldotp \tshp(\loc) \uplus \Set{\ts + 1 \mapsto (\bot, \tsid, \eop ) \ \middle| \ \tshp(\loc)(\ts) = (\dontcare, \tsid, \rop) \land \forall \loc', \ts' \ldotp \tshp(\loc')(\ts') = (\dontcare, \tsid, \wop)} \\
                         & & \quad \quad \quad \uplus \Set{\ts - 1 \mapsto (\bot, \tsid, \sop ) \ \middle| \ \tshp(\loc)(\ts) = (\dontcare, \tsid, \wop) \land \forall \loc', \ts' \ldotp \tshp(\loc')(\ts') = (\dontcare, \tsid, \rop)}
\end{rclarray}
\]

\begin{lem}
    After stretching the time by 3, there is no record in time \( 3 * \nat + 1 \) and \( 3 * \nat - 1 \).
    Therefore after extending, there are only \( (\dontcare, \tsid, \eop) \) in time \( 3 * \ts + 1 \) and only \( (\dontcare, \tsid, \sop) \) in time \( 3 * \ts - 1 \).
    This is \( \forall \tshp \ldotp \exists \tshp' = \func{extend} \circ \func{stetch}(\tshp) \ldotp \forall \loc, \ts \ldotp (\tshp'(\loc)(3 * \ts + 1)\isdef \implies \tshp'(\loc)(3 * \ts + 1) = (\dontcare, \dontcare, \eop) ) \land (\tshp'(\loc)(2 * \ts - 1)\isdef \implies \tshp'(\loc)(3 * \ts - 1) = (\dontcare, \dontcare, \sop) ) \).
\end{lem}
\begin{proof}
    trivial.
\end{proof}

\begin{lem}
    \label{lem:start-before-end}
    In the extended heap, all the reads or starts of a transaction happen before all the writes or end. This is 
    \( \forall \tshp, \loc, \loc', \ts, \ts', \op \in \Set{\rop, \sop}, \op' \in \Set{\wop, \eop}, \tsid \ldotp \tshp(\loc)(\ts) = (\dontcare, \op, \tsid) \land \tshp(\loc')(\ts') = (\dontcare, \op', \tsid) \implies \ts < \ts' \).
\end{lem}
\begin{proof}
    From Lemma \ref{lem:read-before-write} and the definition of \func{strech} and \func{extend}.
\end{proof}

\begin{lem}
    \label{lem:happen-in-same-time}
    For an extended heap, all the reads of a transaction happen in the same time, so do all the writes, starts and ends. This is 
    \( \forall \tshp, \loc, \loc', \ts, \ts', \tsid, \op \ldotp \tshp(\loc)(\ts) =  \tshp(\loc')(\ts') = (\dontcare, \op, \tsid) \implies \ts = \ts' \).
\end{lem}
\begin{proof}
    From Lemma \ref{lem:atoic-rw} and the definition of \func{strech} and \func{extend}.
\end{proof}

\begin{lem}
    \label{lem:unique-label}
    A transaction in an extended heap, must have either starts or reads, and either ends or writes.
\end{lem}
\begin{proof}
    From the definition of \func{extend}.
\end{proof}

\[
\begin{rclarray}
    (\settrans, \tvis, \tar) = \func{graph}(\tshp) & \defeq & (\Set{ \tsid \ \middle| \ \forall \loc \ldotp \tshp(\loc) = (\dontcare, \tsid, \dontcare)}, \\
                                                   & & \Set{(\tsid, \tsid') \ \middle| \ 
    \begin{array}{@{}l@{}}
        \exists \loc, \loc', \ts, \ts', \op \in \Set{\wop, \eop}, \op' \in \Set{\rop, \sop} \ldotp \\
        \ts < \ts' \land \tshp(\loc)(\ts) = (\dontcare, \tsid, \op) \land \tshp(\loc')(\ts') = (\dontcare, \tsid', \op')
    \end{array}
}, \\
                                                   & & \Set{(\tsid, \tsid') \ \middle| \ 
    \begin{array}{@{}l@{}}
        \exists \loc, \loc', \ts, \ts', \op, \op' \in \Set{\wop, \eop} \ldotp \\
        \ts < \ts' \land \tshp(\loc)(\ts) = (\dontcare, \tsid, \op) \land \tshp(\loc')(\ts') = (\dontcare, \tsid', \op')
    \end{array}
}, \\
\end{rclarray}
\]

\begin{lem}
    For an extended heap, the corresponding \tvis\ and \tar\ have no circle.
\end{lem}
\begin{proof}
    Assume there is a circle in \(\rvis\), says, \( \tsid_{1} \rvis \tsid_{2} \rvis \dots \rvis \tsid_{n} \rvis \tsid_{n+1} \), where \( \tsid_{1} = \tsid_{n+1} \).
    Therefore, \( \bigwedge\limits_{ 1 \leq i \leq n} \exists \loc, \loc', \ts, \ts', \op \in \Set{\wop, \eop}, \op' \in \Set{\rop, \sop} \ldotp \ts < \ts' \land \tshp(\loc)(\ts) = (\dontcare, \tsid_{i}, \op) \land \tshp(\loc')(\ts') = (\dontcare, \tsid_{i+1}, \op')\).
    By Lemma \ref{lem:unique-label} we can relabel read to start and write to end.
    Then by Lemma \ref{lem:happen-in-same-time}, we can define a list of starts and ends events that is ordered by time: \( \List{ (\tsid_{1},\eop), (\tsid_{2},\sop), (\tsid_{2},\eop), \dots, (\tsid_{n},\eop), (\tsid_{n+1},\sop) } \).
    By the assumption, we have \( \tsid_{1} = \tsid_{n+1} \), thus this contradict Lemma \ref{lem:start-before-end}.

    Similarly for \(\rtar\), the list of write events  \( \List{ (\tsid_{1},\eop), (\tsid_{2},\eop), (\tsid_{2},\eop), \dots, (\tsid_{n},\eop), (\tsid_{n+1},\eop) } \) contradict Lemma \ref{lem:happen-in-same-time}.
\end{proof}

\begin{lem}
    Given a \( \tshp \), the corresponding \((\settrans, \tvis, \tar)\) can be extended to \((\settrans, \vis, \ar)\) so that it is a valid dependency graph of snapshot isolation.
\end{lem}
\begin{proof}
    First, we extend the relations \( \tar \) to a total order \( \ar \).
    Initially, \( \ar \) includes all relations in \( \tar \).
    Given the definition, only if the ends or writes of transactions happen in the same time, those transactions are not ordered by \( \tar \).
    To simplify, we introduce an initial event, i.e.\ \( \forall \tsid \in \settrans \ldotp \tsid_{init} \rar \tsid \).
    From \( \tsid_{init} \), we pick the first two transactions \( \tsid_{1} \) and \( \tsid_{2} \) that are not ordered, this is, \( \forall \tsid, \tsid' \in \Set{\tsid'' \ \middle| \ \tsid'' \rar \tsid_{1} \lor \tsid'' \rar \tsid_{2} } \ldotp \tsid \rar \tsid' \lor \tsid' \rar \tsid \).
    Therefore, there exists an unique \( \tsid_{pre} \) where branching happens, i.e.\ \( \tsid_{pre} \rar \tsid_{1} \land \tsid_{pre} \rar \tsid_{2} \land \nexists \tsid \ldotp \tsid_{pre} \rar \tsid \rar \tsid_{1} \lor \tsid_{pre} \rar \tsid \rar \tsid_{2} \).
\end{proof}

\subsection{Clock-SI}
\label{sec:clock-si}
\renewcommand{\thelstlisting}{\arabic{lstlisting}}
\lstinputlisting{\RootPath/cops/pseudocode}

\section{semantics\label{sec:semantics}}

Assume that heap and stack are initialised to zero.

\[
    \begin{rclarray}
        \loc \in \Loc & \defeq & \Nat \\
        \val \in \Val & \defeq & \Nat \uplus \Loc \\
        \Var & \defeq & \Set{ \vx, \vy, \dots } \\
        \ts \in \Timestamp & \defeq & \Nat \\
        \hp \in \Heap & \defeq & \Loc \parfun \Val \\
        \stk \in \Stack & \defeq & \Var \to \Val \\
        \rs \in \Readset, \ws \in \Writeset & \defeq & \powerset{\Loc} \\
        \lstt = (\hp, \stk, \rs, \ws ) \in \Localstate & \defeq & \Stack \times \Heap \times \Readset \times \Writeset \\
        \op \in \Operation & \defeq & \Set{\opr, \opw} \\
        \settrans \subseteq \TransID & \defeq & \Set{ \alpha , \beta, \dots } \\
        \tshp \in \Timestampheap & \defeq & \Loc \parfun ( \Timestamp \parfun \Val \times \Operation \times \TransID) \\
        \ThreadID & \defeq & \Set{ i , j, \dots } \\
        (\tshp, \stk, \ts) \in \Threadstate & \defeq & \Timestampheap \times \Stack \times \Timestamp \\
        \tdpl \in \Threadpool & \defeq & \ThreadID \parfun \Stack \times \Timestamp \times \prog \\
        \stt \in \State & \defeq & \Timestampheap \times \Threadpool \\
    \end{rclarray}
\]

No side effect of evaluation of arithmetic expression.

\[
    \begin{syntax}{\texpr}
              \val \quad            |
        \quad \var \quad            |
        \quad \texpr + \texpr \quad |
        \quad \texpr * \texpr \quad |
        \quad \dots 
    \end{syntax}
\]

No side effect of evaluation of boolean expression.

\[
    \begin{rclarray}
        \eval{\val}_{\stk} & \defeq & \val \\
        \eval{\var}_{\stk} & \defeq & \stk(\val) \\
        \eval{\texpr_{1} + \texpr_{2}}_{\stk} & \defeq & \eval{\texpr_{1}}_{\stk} + \eval{\texpr_{2}}_{\stk}   \\
        \eval{\texpr_{1} * \texpr_{2}}_{\stk} & \defeq & \eval{\texpr_{1}}_{\stk} * \eval{\texpr_{2}}_{\stk}  
    \end{rclarray}
\]

\[
    \begin{syntax}{\tbool}
              \true \quad                  |
        \quad \false \quad                 |
        \quad \texpr = \texpr \quad        |
        \quad \texpr < \texpr \quad        |
        \quad \boolnot \tbool \quad        |
        \quad \tbool \booland \tbool \quad |
        \quad \tbool \boolor \tbool \quad  |
        \quad \dots 
    \end{syntax}
\]

\[
    \begin{rclarray}
        \eval{\true}_{\stk}& \defeq & \true \\
        \eval{\false}_{\stk} & \defeq & \false \\
        \eval{\texpr_{1} = \texpr_{2}}_{\stk} & \defeq & \eval{\texpr_{1}}_{\stk} = \eval{\texpr_{2}}_{\stk}   \\
        \eval{\texpr_{1} < \texpr_{2}}_{\stk} & \defeq & \eval{\texpr_{1}}_{\stk} < \eval{\texpr_{2}}_{\stk}   \\
        \eval{\boolnot \tbool}_{\stk} & \defeq & \neg \eval{\tbool}_{\stk} \\
        \eval{\tbool_{1} \booland \tbool_{2}}_{\stk} & \defeq & \eval{\tbool_{1}}_{\stk} \land \eval{\tbool_{2}}_{\stk}  \\
        \eval{\tbool_{1} \boolor \tbool_{2}}_{\stk}& \defeq & \eval{\tbool_{1}}_{\stk} \lor \eval{\tbool_{2}}_{\stk}  
    \end{rclarray}
\]

\[
    \begin{syntax}{\tcmd}
              \tskip \quad                     |
        \quad \tass{\vx}{\texpr} \quad         |
        \quad \tmutate{\texpr}{\texpr} \quad   |
        \quad \tderef{\vx}{\texpr} \quad       |
        \quad \tif{\tbool}{\tcmd}{\tcmd} \quad | \\
              \tloop{\tbool}{\tcmd} \quad      |
        \quad \tcmd \tseq \tcmd
    \end{syntax}
\]

\[
    \begin{rclarray}
        \dontcare, \dontcare, \dontcare, \dontcare, \dontcare \ \localtransfer \ \dontcare, \dontcare, \dontcare, \dontcare, \dontcare & \defeq & \Localstate \times \tcmd \times \Localstate \times \tcmd \\
    \end{rclarray}
\]

\[
    \infer[ass]{%
        \stk, \hp, \rs, \ws, \tass{\var}{\texpr} \ \localtransfer \  \stk \remapsto{\var}{\val}, \hp, \rs, \ws, \tskip
    }{%
    \eval{\texpr}_{\stk} = \val
    }
\]

\[
    \infer[mutate]{%
        \stk, \hp, \rs, \ws, \tmutate{\texpr_{1}}{\texpr_{2}} \ \localtransfer \  \stk, \hp \remapsto{\loc}{\val}, \rs, \ws \cup \Set{\loc}, \tskip
    }{%
        \eval{\texpr_{1}}_{\stk} = \loc \quad 
        \eval{\texpr_{2}}_{\stk} = \val \quad 
        \loc \in \dom(\hp)
    }
\]

\[
    \infer[deref]{%
        \stk, \hp, \rs, \ws, \tderef{\var}{\texpr} \ \localtransfer \  \stk \remapsto{\var}{\val}, \hp, \rs \cup \Set{\loc}, \ws, \tskip
    }{%
        \eval{\texpr}_{\stk} = \loc \quad 
        \val = \hp(\loc) \quad
        \loc \in \dom(\hp)
    }
\]

\[
    \infer[ifelsetrue]{%
        \stk, \hp, \tif{\tbool}{\tcmd_{1}}{\tcmd_{2}} \ \localtransfer \  \stk, \hp, \tcmd_{1}
    }{%
        \eval{\tbool}_{\stk} = \true
    }
\]

\[
    \infer[ifelsefalse]{%
        \stk, \hp, \tif{\tbool}{\tcmd_{1}}{\tcmd_{2}} \ \localtransfer \  \stk, \hp, \tcmd_{2}
    }{%
        \eval{\tbool}_{\stk} = \false
    }
\]

\[
    \infer[whiletrue]{%
        \stk, \hp, \tloop{\tbool} \tcmd \ \localtransfer \  \stk, \hp,  \tcmd \tseq \tloop{\tbool} \tcmd
    }{%
        \eval{\tbool}_{\stk} = \true
    }
\]

\[
    \infer[whilefalse]{%
        \stk, \hp, \tloop{\tbool} \tcmd \ \localtransfer \  \stk, \hp, \tskip
    }{%
        \eval{\tbool}_{\stk} = \false \quad
    }
\]

\[
    \infer[seqskip]{%
        \stk, \hp, \tskip \tseq \tcmd_{2} \ \localtransfer \  \stk, \hp, \tcmd_{2}
    }{%
    }
\]

\[
    \infer[seqnonskip]{%
        \stk, \hp, \tcmd_{1} \tseq \tcmd_{2} \ \localtransfer \  \stk', \hp', \tcmd_{1}' \tseq \tcmd_{2}
    }{%
        \stk, \hp, \tcmd_{1} \ \localtransfer \  \stk', \hp', \tcmd_{1}'
    }
\]

The semantics of transaction are interleaving of start, commit, and restart.

\[
    \begin{syntax}{\prog}
              \pemp \quad               |
        \quad \ptrans{\tcmd} \quad      |
        \quad \prog \pcond \prog \quad  |
        \quad \prept{\prog} \quad       |
        \quad \prog \pseq \prog \quad   |
        \quad \pfork{\var}{\prog} \quad |
        \quad \pjoin{\texpr}   
    \end{syntax}
\]

\[
    \begin{rclarray}
        \prog_{1} \ppar \prog_{2} & \equiv & \pfork{\var}{\prog_{1}} \pseq \prog_{2} \pseq \pjoin{\var} \\
        \tll \in \Translabel & \defeq & 
              \lid \quad                |
              \quad \lfork{\prog} \quad |
        \quad \ljoin{\thid,\ts} \\
        \dontcare, \dontcare, \dontcare, \dontcare \ \threadtransfer{ \dontcare } \ \dontcare, \dontcare, \dontcare, \dontcare & \defeq & \Threadstate \times \prog \times \Translabel \times \Threadstate \times \prog \\
    \end{rclarray}
\]

\[
    \begin{rclarray}
        \func{startstate}(\tshp,\ts) & \defeq & \lambda \loc \ldotp \tshp(\loc)(\max(\Set{\ \ts' \ \middle| \ \ts' \leq \ts \land \tshp(\ts') = (\dontcare,\wop, \dontcare) }))
    \end{rclarray}
\]

\[
    \begin{rclarray}
        \pred{allowcommit}(\tshp,\ws,\rs,\ts_{s},\ts_{e}) & \defeq & 
        \pred{atomicop}(\tshp,\ws,\rs,\ts_{s},\ts_{e}) \land {} \\
        & & \pred{consistent}(\tshp,\ws,\rs,\ts_{s},\ts_{e}) \\
        \pred{atomicop}(\tshp,\ws,\rs,\ts_{s},\ts_{e}) & \defeq  & \forall \loc \in \ws \cup \rs \ldotp \tshp(\loc)(\ts_{s})\undef \land \tshp(\loc)(\ts_{e})\undef \\
        \pred{consistent}(\tshp,\ws,\rs,\ts_{s},\ts_{e}) & \defeq & \forall \ts \in [\ts_{s},\ts_{e}], \loc \in \ws \ldotp \tshp(\loc)(\ts) \neq (\dontcare, \wop, \dontcare) \land {} \\
                                                       & & \exists \ts_{min} = \min(\Set{\ts'' \ \middle| \ \ts'' \geq \ts_{e} \land \tshp(l)(\ts'')\isdef}) \ldotp \\
                                                       & & \ts_{min} \neq \bot \implies \tshp(\loc)(\ts_{min}) = (\dontcare, \wop, \dontcare) \\
        \func{commit}(\tshp,\hp,\ws,\rs,\ts_{s},\ts_{e}) & \defeq &
        \lambda \loc \ldotp
        \begin{funcarray}
            \tshp(\loc) & \loc \notin \ws \cup \rs \\
            \tshp(\loc) \uplus \Set{ \ts_{e} \mapsto (\hp(\loc),\wop,\tsid)} & \loc \in \ws \\
            \tshp(\loc) \uplus \Set{ \ts_{s} \mapsto (\hp(\loc),\rop,\tsid)} & \loc \in \rs \\
        \end{funcarray} \\
        & & \texttt{where} \  \tsid \notin \Set{\tshp(\loc)(\ts)\projection{3} \ \middle| \ \loc \in \dom(\tshp) \land \ts \in \dom(\tshp(\loc))} \\
    \end{rclarray}
\]

\[
    \infer[commit]{%
        \tshp, \stk, \ts, \ptrans{\tcmd} \ \threadtransfer{\lid} \  \tshp', \stk', \ts_{e}, \pemp
    }{%
        \begin{array}{c}
            \ts_{s} \geq \ts \quad \stk, \func{startstate}(\tshp, \ts_{s}), \emptyset, \emptyset \localtransfer^{*} \stk', \hp, \rs, \ws \\
            \pred{allowcommit}(\tshp,\ws,\rs,\ts_{s},\ts_{e}) \quad \ts_{e} > \ts_{s} \quad \tshp' = \func{commit}(\tshp,\hp,\ws,\rs,\ts_{s},\ts_{e})
        \end{array}
    }
\]

\[
    \infer[choiceleft]{%
        \tshp, \stk, \ts, \prog_{1} \pcond \prog_{2} \ \threadtransfer{\lid} \  \tshp, \stk, \ts, \prog_{1}
    }{%
    }
\]

\[
    \infer[choiceright]{%
        \tshp, \stk, \ts, \prog_{1} \pcond \prog_{2} \ \threadtransfer{\lid} \  \tshp, \stk, \ts, \prog_{2}
    }{%
    }
\]

\[
    \infer[norep]{%
        \tshp, \stk, \ts, \prept{\prog} \ \threadtransfer{\lid} \  \tshp, \stk, \ts, \pemp
    }{%
    }
\]

\[
    \infer[rep]{%
        \tshp, \stk, \ts, \prept{\prog} \ \threadtransfer{\lid} \  \tshp, \stk, \ts, \prog \pseq \prept{\prog}
    }{%
    }
\]

\[
    \infer[seqskip]{%
        \tshp, \stk, \ts, \pemp \pseq \prog \ \threadtransfer{\lid} \  \tshp, \stk, \ts, \prog
    }{%
    }
\]

\[
    \infer[seqnoskip]{%
        \tshp, \stk, \ts, \prog_{1} \pseq \prog_{2} \ \threadtransfer{\tll} \  \tshp', \stk', \ts', \prog_{1}' \pseq \prog_{2}
    }{%
        \tshp, \stk, \ts, \prog_{1} \ \threadtransfer{\tll} \  \tshp', \stk', \ts', \prog_{1}' 
    }
\]

\[
    \infer[fork]{%
        \tshp, \stk, \ts, \pfork{\var}{\prog} \ \threadtransfer{\lfork{\thid,\prog}} \  \tshp, \stk\remapsto{\var}{\thid}, \ts, \pemp 
    }{%
    }
\]

\[
    \infer[join]{%
        \tshp, \stk, \ts, \pjoin{\texpr} \ \threadtransfer{\ljoin{\eval{\texpr}_{\stk},\ts'}} \  \tshp, \max\Set{\ts,\ts'}, \pemp 
    }{%
    }
\]

\[
    \begin{rclarray}
        \dontcare, \dontcare \ \globaltransfer{ \dontcare } \ \dontcare, \dontcare & \defeq & \State \times \Translabel \times \State  \\
    \end{rclarray}
\]

\[
    \infer[single]{%
        \tshp, \tdpl \uplus \Set{ \thid \mapsto (\stk, \ts, \prog) } \ \globaltransfer{\tll} \  \tshp', \tdpl \uplus \Set{ \thid \mapsto (\stk', \ts', \prog') }
    }{%
        \tshp, \stk, \ts, \prog \ \threadtransfer{\tll} \  \tshp', \stk', \ts', \prog' 
        \quad \tll \notin \Set{\lfork{\dontcare,\dontcare},\ljoin{\dontcare,\dontcare}}
    }
\]

\[
    \infer[fork]{%
        \tshp, \tdpl \uplus \Set{ \thid \mapsto (\stk, \ts, \prog) } \ \globaltransfer{\lfork{\thid',\prog''}} \  \tshp', \tdpl \uplus \Set{ \thid \mapsto (\stk', \ts', \prog'), \thid' \mapsto (\lambda \var \ldotp 0, \ts', \prog'') }
    }{%
        \tshp, \stk, \ts, \prog \ \threadtransfer{\lfork{\thid',\prog''}} \  \tshp', \stk', \ts', \prog' 
    }
\]

\[
    \infer[join]{%
        \tshp, \tdpl \uplus \Set{ \thid \mapsto (\stk, \ts, \prog), \thid' \mapsto (\stk', \ts'', \pemp) } \ \globaltransfer{\ljoin{\thid',\ts''}} \  \tshp', \tdpl \uplus \Set{ \thid \mapsto (\stk', \ts', \prog')}
    }{%
        \tshp, \stk, \ts, \prog \ \threadtransfer{\ljoin{\thid',\ts''}} \  \tshp', \stk', \ts', \prog' 
    }
\]

\begin{lem}
    A history cannot be overwritten, i.e.\ \( \forall \tshp, \tshp', \loc,\ts \ldotp \tshp, \dontcare \globaltransfer{\dontcare} \tshp', \dontcare \land \tshp(\loc)(\ts)\isdef \implies \tshp(\loc)(\ts) = \tshp'(\loc)(\ts)\)
\end{lem}
\begin{proof}
    From the \( \pred{allowcommit} \).
\end{proof}

\begin{lem}
    \label{lem:read-before-write}
    All the reads of a transaction happen before all the writes. This is 
    \( \forall \tshp, \loc, \loc', \ts, \ts', \tsid \ldotp \tshp(\loc)(\ts) = (\dontcare, \rop, \tsid) \land \tshp(\loc')(\ts') = (\dontcare, \wop, \tsid) \implies \ts < \ts' \).
\end{lem}
\begin{proof}
    From the semantics that \( \ts_{s} < \ts_{e} \).
\end{proof}

\begin{lem}
    \label{lem:atoic-rw}
    All the reads of a transaction happen in the same time, so do all the writes. This is 
    \( \forall \tshp, \loc, \loc', \ts, \ts', \tsid, \op \in \Set{\rop, \wop} \ldotp \tshp(\loc)(\ts) =  \tshp(\loc')(\ts') = (\dontcare, \op, \tsid) \implies \ts = \ts' \).
\end{lem}
\begin{proof}
    From the \( \func{commit} \).
\end{proof}

Now we need to recover \( \vis \) and \( \ar \) from \( \tshp \).
First we need to extend the \( \tshp \) because there are some transactions that only have reads or writes.
We stretch the time by 3, and add extra operation for those transactions.
For a transaction \( \tsid \) that only has reads event, says in time \( \ts \), we add end operations \( (\bot, \tsid, \eop ) \) to the heap cells it reads in time \( (\ts + 1 ) \).
Similarly for a transaction that only has writes event, we add end operations \( (\bot, \tsid, \sop ) \) in time \( (\ts-1) \).

\[
\begin{rclarray}
    \func{stretch}(\tshp) & \defeq & \lambda \loc \ldotp \lambda \ts \ldotp
    \begin{funcarray}
        \tshp(\loc)(\ts') & \ts = 3 * \ts' \\
        \texttt{undef} & o.w. \\
    \end{funcarray} \\
    \func{extend}(\tshp) & \defeq & \lambda \loc \ldotp \tshp(\loc) \uplus \Set{\ts + 1 \mapsto (\bot, \tsid, \eop ) \ \middle| \ \tshp(\loc)(\ts) = (\dontcare, \tsid, \rop) \land \forall \loc', \ts' \ldotp \tshp(\loc')(\ts') = (\dontcare, \tsid, \wop)} \\
                         & & \quad \quad \quad \uplus \Set{\ts - 1 \mapsto (\bot, \tsid, \sop ) \ \middle| \ \tshp(\loc)(\ts) = (\dontcare, \tsid, \wop) \land \forall \loc', \ts' \ldotp \tshp(\loc')(\ts') = (\dontcare, \tsid, \rop)}
\end{rclarray}
\]

\begin{lem}
    After stretching the time by 3, there is no record in time \( 3 * \nat + 1 \) and \( 3 * \nat - 1 \).
    Therefore after extending, there are only \( (\dontcare, \tsid, \eop) \) in time \( 3 * \ts + 1 \) and only \( (\dontcare, \tsid, \sop) \) in time \( 3 * \ts - 1 \).
    This is \( \forall \tshp \ldotp \exists \tshp' = \func{extend} \circ \func{stetch}(\tshp) \ldotp \forall \loc, \ts \ldotp (\tshp'(\loc)(3 * \ts + 1)\isdef \implies \tshp'(\loc)(3 * \ts + 1) = (\dontcare, \dontcare, \eop) ) \land (\tshp'(\loc)(2 * \ts - 1)\isdef \implies \tshp'(\loc)(3 * \ts - 1) = (\dontcare, \dontcare, \sop) ) \).
\end{lem}
\begin{proof}
    trivial.
\end{proof}

\begin{lem}
    \label{lem:start-before-end}
    In the extended heap, all the reads or starts of a transaction happen before all the writes or end. This is 
    \( \forall \tshp, \loc, \loc', \ts, \ts', \op \in \Set{\rop, \sop}, \op' \in \Set{\wop, \eop}, \tsid \ldotp \tshp(\loc)(\ts) = (\dontcare, \op, \tsid) \land \tshp(\loc')(\ts') = (\dontcare, \op', \tsid) \implies \ts < \ts' \).
\end{lem}
\begin{proof}
    From Lemma \ref{lem:read-before-write} and the definition of \func{strech} and \func{extend}.
\end{proof}

\begin{lem}
    \label{lem:happen-in-same-time}
    For an extended heap, all the reads of a transaction happen in the same time, so do all the writes, starts and ends. This is 
    \( \forall \tshp, \loc, \loc', \ts, \ts', \tsid, \op \ldotp \tshp(\loc)(\ts) =  \tshp(\loc')(\ts') = (\dontcare, \op, \tsid) \implies \ts = \ts' \).
\end{lem}
\begin{proof}
    From Lemma \ref{lem:atoic-rw} and the definition of \func{strech} and \func{extend}.
\end{proof}

\begin{lem}
    \label{lem:unique-label}
    A transaction in an extended heap, must have either starts or reads, and either ends or writes.
\end{lem}
\begin{proof}
    From the definition of \func{extend}.
\end{proof}

\[
\begin{rclarray}
    (\settrans, \tvis, \tar) = \func{graph}(\tshp) & \defeq & (\Set{ \tsid \ \middle| \ \forall \loc \ldotp \tshp(\loc) = (\dontcare, \tsid, \dontcare)}, \\
                                                   & & \Set{(\tsid, \tsid') \ \middle| \ 
    \begin{array}{@{}l@{}}
        \exists \loc, \loc', \ts, \ts', \op \in \Set{\wop, \eop}, \op' \in \Set{\rop, \sop} \ldotp \\
        \ts < \ts' \land \tshp(\loc)(\ts) = (\dontcare, \tsid, \op) \land \tshp(\loc')(\ts') = (\dontcare, \tsid', \op')
    \end{array}
}, \\
                                                   & & \Set{(\tsid, \tsid') \ \middle| \ 
    \begin{array}{@{}l@{}}
        \exists \loc, \loc', \ts, \ts', \op, \op' \in \Set{\wop, \eop} \ldotp \\
        \ts < \ts' \land \tshp(\loc)(\ts) = (\dontcare, \tsid, \op) \land \tshp(\loc')(\ts') = (\dontcare, \tsid', \op')
    \end{array}
}, \\
\end{rclarray}
\]

\begin{lem}
    For an extended heap, the corresponding \tvis\ and \tar\ have no circle.
\end{lem}
\begin{proof}
    Assume there is a circle in \(\rvis\), says, \( \tsid_{1} \rvis \tsid_{2} \rvis \dots \rvis \tsid_{n} \rvis \tsid_{n+1} \), where \( \tsid_{1} = \tsid_{n+1} \).
    Therefore, \( \bigwedge\limits_{ 1 \leq i \leq n} \exists \loc, \loc', \ts, \ts', \op \in \Set{\wop, \eop}, \op' \in \Set{\rop, \sop} \ldotp \ts < \ts' \land \tshp(\loc)(\ts) = (\dontcare, \tsid_{i}, \op) \land \tshp(\loc')(\ts') = (\dontcare, \tsid_{i+1}, \op')\).
    By Lemma \ref{lem:unique-label} we can relabel read to start and write to end.
    Then by Lemma \ref{lem:happen-in-same-time}, we can define a list of starts and ends events that is ordered by time: \( \List{ (\tsid_{1},\eop), (\tsid_{2},\sop), (\tsid_{2},\eop), \dots, (\tsid_{n},\eop), (\tsid_{n+1},\sop) } \).
    By the assumption, we have \( \tsid_{1} = \tsid_{n+1} \), thus this contradict Lemma \ref{lem:start-before-end}.

    Similarly for \(\rtar\), the list of write events  \( \List{ (\tsid_{1},\eop), (\tsid_{2},\eop), (\tsid_{2},\eop), \dots, (\tsid_{n},\eop), (\tsid_{n+1},\eop) } \) contradict Lemma \ref{lem:happen-in-same-time}.
\end{proof}

\begin{lem}
    Given a \( \tshp \), the corresponding \((\settrans, \tvis, \tar)\) can be extended to \((\settrans, \vis, \ar)\) so that it is a valid dependency graph of snapshot isolation.
\end{lem}
\begin{proof}
    First, we extend the relations \( \tar \) to a total order \( \ar \).
    Initially, \( \ar \) includes all relations in \( \tar \).
    Given the definition, only if the ends or writes of transactions happen in the same time, those transactions are not ordered by \( \tar \).
    To simplify, we introduce an initial event, i.e.\ \( \forall \tsid \in \settrans \ldotp \tsid_{init} \rar \tsid \).
    From \( \tsid_{init} \), we pick the first two transactions \( \tsid_{1} \) and \( \tsid_{2} \) that are not ordered, this is, \( \forall \tsid, \tsid' \in \Set{\tsid'' \ \middle| \ \tsid'' \rar \tsid_{1} \lor \tsid'' \rar \tsid_{2} } \ldotp \tsid \rar \tsid' \lor \tsid' \rar \tsid \).
    Therefore, there exists an unique \( \tsid_{pre} \) where branching happens, i.e.\ \( \tsid_{pre} \rar \tsid_{1} \land \tsid_{pre} \rar \tsid_{2} \land \nexists \tsid \ldotp \tsid_{pre} \rar \tsid \rar \tsid_{1} \lor \tsid_{pre} \rar \tsid \rar \tsid_{2} \).
\end{proof}

%%%%%%%%%%%%%%%%%%%%%%%%%%%%%%%%%%%%%%%%%%%%%%%%%%%%%%%%%%%%%%%%%%%%%%%%
% END OF APPENDIX
%%%%%%%%%%%%%%%%%%%%%%%%%%%%%%%%%%%%%%%%%%%%%%%%%%%%%%%%%%%%%%%%%%%%%%%%
\end{document}
