\subsection{The soundness and completeness of the specification on  key-value semantics}
\label{sec:kv-sound-complete-proof}
\label{sec:spec-proof}

We now show using \cref{def:et_sound,def:et_complete} to prove the soundness and completeness of execution tests with respect to axiomatic specification.
It is sufficient to match these two definition, 
then by \cref{cor:et-soundness,cor:et-completeness} we have \( \CMs(\ET) = \Setcon{\mkvs_\aexec}{\aexec \in \CMa(\RP_\LWW,\Ax)} \).

\subsubsection{Monotonic Read \( \MRd \)}
\label{sec:sound-complete-mr}

The execution test $\ET_\MRd$ is sound with respect to the axiomatic specification $(\RP_{\LWW}, \{\lambda \aexec. \VIS_{\aexec} ; \PO_{\aexec} \})$. 
we choose an invariant as the following,  
\[
    I(\aexec, \cl) = \left( \bigcup_{\{\txid_{\cl}^{n} \in \T_{\aexec} \mid n \in \Nat\}} \VIS_{\aexec}^{-1}(\txid^n_\cl) \right) \setminus \T_\rd
\]
where \( \T_\rd \) is all the read-only transactions in 
\( \bigcup\limits_{\{\txid_{\cl}^{n} \in \T_{\aexec} \mid n \in \Nat\}} \VIS_{\aexec}^{-1}(\txid^n_\cl) \).
Assume a key-value store $\hh$, an initial and a final view $\vi, \vi'$  a fingerprint $\opset$ 
such that $\ET_{\MRd} \vdash (\hh, \vi) \triangleright \opset: \vi'$. 
Also choose an arbitrary $\cl$, a transaction identifier $\txid \in \nextTxId(\hh, \cl)$, 
and an abstract execution $\aexec$ such that $\hh_{\aexec} = \hh$ and 
\begin{equation}
I(\aexec, \cl) \subseteq \Tx(\hh, \vi)
\label{eq:mr_invariant}
\end{equation}
Let \( \aexec' = \extend(\aexec, \txid, \opset, \Tx(\hh, \vi) \cup \T_\rd) \).
We now check if \( \aexec' \) satisfies the axiomatic specification and the invariant is preserved:
\begin{itemize}
\item $\{\txid' \mid (\txid', \txid) \in \VIS_{\aexec'} ; \PO_{\aexec'} \} \subseteq \Tx(\hh, \vi) \cup \T_\rd$. 
Suppose that $\txid' \xrightarrow{\VIS_{\aexec'}} \txid'' \xrightarrow{\PO_{\aexec'}} \txid$ 
for some $\txid', \txid''$. We show that $\txid' \in I(\aexec, \cl)$, and then \cref{eq:mr_invariant} ensures 
that $\txid' \in \Tx(\hh, \vi) \cup \T_{\mathsf{rd}}$. 
Suppose $\txid'' \xrightarrow{\SO_{\aexec'}} \txid$, then $\txid'' = \txid_{\cl}^{n}$ for some $n \in \Nat$.
Because $\txid'' \neq \txid$ and $\T_{\aexec'} \setminus \T_{\aexec} = \{ \txid \}$, we also 
have that $\txid'' \in \aexec$. By the invariant of $I(\aexec, \cl)$, 
we have that $\VIS^{-1}_{\aexec}(\cl) \subseteq I(\aexec, \cl)$:
because $\txid' \xrightarrow{\VIS_{\aexec'}} \txid''$ and $\txid'' \neq \txid$ we have 
that $\txid' \xrightarrow{\VIS_{\aexec}} \txid''$ and therefore $\txid' \in I(\aexec, \cl)$. 

\item $I(\aexec', \cl) \subseteq \Tx(\aexec', \vi') = \Tx(\hh', \vi')$. 
In this case, because $\ET_{\MRd} \vdash (\hh, \vi) \triangleright \opset: \vi'$, 
then it must be the case that $\vi \viewleq \vi'$. 
A trivial consequence of this fact is that $\Tx(\hh, \vi) \subseteq \Tx(\hh, \vi')$.
Also, because $\aexec' = \extend(\aexec, \txid, \Tx(\hh, \vi) \cup \T_{\mathsf{rd}})$, 
we have that $\Tx(\hh_{\aexec}, \vi) = \Tx(\hh_{\aexec'}, \vi)$. 
\ac{to infer this there should be a Lemma that states that if $\vi \in \Views(\hh)$, 
then $\Tx(\updateKV(\hh, \vi', \txid, \opset), \vi) = \Tx(\hh, \vi)$.}
Finally, note that $\{\txid_{\cl}^{n} \in \aexec' \mid n \in \Nat\} = 
\{ \txid_{\cl}^{n} \in \T_{\aexec} \mid n \in \Nat\} \cup \txid$, that for any 
$\txid_{\cl}^{n} \in \T_{\aexec}$ we have that $\VIS^{-1}_{\aexec'}(\txid_{\cl}^{n}) = 
\VIS^{-1}_{\aexec}(\txid_{\cl}^{n})$, and that 
$\VIS_{\aexec'}^{-1}(\txid) = \Tx(\hh, \vi) \cup \T_{\mathsf{rd}}$. 
Using all these facts, we obtain 
\[
\begin{rclarray}
I(\aexec', \cl) &=& \left( \bigcup_{\{\txid_{\cl}^{n} \in \aexec' \mid n \in \Nat\}} \VIS_{\aexec'}^{-1}(\txid_{\cl}^{n}) \right) \setminus \T_\rd \\
                &=& \left( \left( \bigcup_{\{\txid_{\cl}^{n} \in \aexec \mid n \in \Nat\}} \VIS_{\aexec}^{-1}(\txid_{\cl}^{n}) \right) \setminus \T_\rd  \right) \cup \left( \VIS^{-1}_{\aexec'}(\txid) \setminus \T_\rd  \right) \\
&=& I(\aexec, \cl) \cup \Tx(\hh, \vi) \\
&\stackrel{\eqref{eq:mr_invariant}}{\subseteq}& \Tx(\hh, \vi) \\
&=& \Tx(\hh_\aexec, \vi) \\
&=& \Tx(\hh_{\aexec'}, \vi) \\
&\subseteq& \Tx(\hh_{\aexec'}, \vi')
\end{rclarray}
\]
\end{itemize}

We show that the execution test $\ET_{\MRd}$ is complete 
with respect to the axiomatic specification $(\RP_{\LWW}, \{\lambda \aexec.(\VIS_{\aexec};\PO_{\aexec})\})$. 
Let $\aexec$ be an abstract execution that satisfies the specification
$\CMa(\RP_{\LWW}, \{\lambda \aexec.(\VIS_{\aexec};\PO_{\aexec})\})$, 
and consider a transaction $\txid \in \T_{\aexec}$. 
Assume i-\emph{th} transaction \( \txid_i \) in the arbitrary order,
and let $\vi_i = \getView(\aexec, \VIS^{-1}_{\aexec}(\txid_{i}))$.
We have two possible cases: 
\begin{itemize}
\item the transaction $\txid'_{i} = \min_{\PO_{\aexec}}\{\txid' \mid \txid_{i} \xrightarrow{\PO_{\aexec}} \txid'\}$ is 
defined. In this case let $\vi'_{i} =\getView(\aexec, (\AR^{-1}_{\aexec})?(\txid_{i}) \cap \VIS^{-1}_{\aexec}(\txid'_{i}))$. 
Note that $\txid_{i} \xrightarrow{\PO_{\aexec}} \txid'_{i}$, and because $\aexec \models \VIS_{\aexec} ; \PO_{\aexec}$, 
it follows that $\VIS^{-1}_{\aexec}(\txid_{i}) \subseteq \VIS^{-1}_{\aexec}(\txid'_{i})$. 
We also have that $\VIS^{-1}_{\aexec}(\txid_{i}) \subseteq (\AR^{-1}_{\aexec})?(\txid_{i})$ because of 
the definition of abstract execution. It follows that 
\[
\VIS^{-1}_{\aexec}(\txid_{i}) \subseteq (\AR^{-1}_{\aexec})?(\txid_{i}) \cap \VIS^{-1}_{\aexec}(\txid'_{i}),
\]
Recall that  $\vi_i = \getView(\aexec, \VIS^{-1}_{\aexec}(\txid_{i}))$, 
and $\vi'_{i} =\getView(\aexec, (\AR^{-1}_{\aexec})?(\txid_{i}) \cap \VIS^{-1}_{\aexec}(\txid'_{i}))$. 
Thus we have that $\vi_i \viewleq \vi'_{i}$, and therefore $\ET_{\MRd} \vdash (\hh_{\cut(\aexec, i)}, \vi_i) 
\triangleright \TtoOp{T}_{\aexec}(\txid_{i}) : \vi'_{i}$. 
\item the transaction $\txid'_{i} = \min_{\PO_{\aexec}}\{\txid' \mid \txid_{i} \xrightarrow{\PO_{\aexec}} \txid_{i}\}$ 
is not defined. In this case, let $\vi'_{i} = \getView(\aexec, (\AR^{-1}_{\aexec})?(\txid_{i}))$. 
As for the case above, we have that $\vi_i \viewleq \vi'_{i}$, and therefore 
$\ET_{\MRd} \vdash (\hh_{\cut(\aexec, i)}, \vi_i) \triangleright \TtoOp{T}_{\aexec}(\txid_{i}) : \vi'_{i}$. 
\end{itemize}

\subsubsection{Monotonic Write \( \MW \)}

\label{sec:sound-complete-mw}

The execution test $\ET_\MW$ is sound with respect to the axiomatic specification 
$(\RP_{\LWW}, \Set{\lambda \aexec. \PO_{\aexec} ; \VIS_{\aexec} })$.
We pick the invariant as empty set given the fact of no constraint on the view after update:
\[ 
    I( \aexec, \cl ) = \emptyset 
\]
Assume a key-value store $\hh$, an initial and a final view $\vi, \vi'$  a fingerprint $\opset$ 
such that $\ET_{\MW} \vdash (\hh, \vi) \triangleright \opset: \vi'$. 
Also choose an arbitrary $\cl$, a transaction identifier $\txid \in \nextTxId(\hh, \cl)$, 
and an abstract execution $\aexec$ such that $\hh_{\aexec} = \hh$ and 
\( I(\aexec, \cl) =  \emptyset \subseteq \Tx(\hh, \vi) \).
Let \( \aexec' = \extend(\aexec, \txid, \Tx(\mkvs, \vi) \cup \T_\rd, \f ) \).
Note that since the invariant  is empty set, it remains to prove that there exists a set of read-only transactions \( \T_\rd \) such that:
\[
    \begin{array}{@{}l@{}}
        \fora{ \txid' }  (\txid' ,\txid)  \in \PO_{\aexec'} ; \VIS_{\aexec'}
        \implies \txid' \in \Tx(\mkvs, \vi) \cup \T_\rd
    \end{array}
\]
Initially we take \( \T_\rd = \emptyset \), 
and by closing the \( \Tx(\mkvs, \vi) \) with respect to the relation \( \PO_{\aexec'} ; \VIS_{\aexec'} \),
we will add more read-only transactions into the set \( \T_\rd\).
Suppose \( (\txid' ,\txid)  \in \PO_{\aexec'} ; \VIS_{\aexec'} \), 
that is, \( \txid' \toEdge{\SO_{\aexec'}} \txid'' \toEdge{\VIS_{\aexec'}} \txid \).
We perform a case analysis on if \( \txid'' \) has write:
\begin{itemize}
\item If the transaction \( \txid'' \) writes to a key.
For the new abstract execution \( \aexec' \), the visible transactions for \( \txid \) must come from \( \Tx(\mkvs, \vi) \cup \T_\rd \).
It means \( \txid'' \in \Tx(\mkvs, \vi) \cup \T_\rd  \).
Then given that \( \txid'' \) is not a read-only transaction, we have \( \txid'' \in \Tx(\mkvs, \vi) \).
Now there are two cases:
\begin{itemize}
    \item if \( \txid' \) is a read-only transaction, we include \( \txid' \in \T_{\rd} \).
    \item if \( \txid' \) has at least one write, it is easy to see \( \txid' \in \Tx(\mkvs, \vi) \) since \( j \in \vi(\ke) \wedge \WTx(\hh(\ke', i)) \xrightarrow{\PO?} \WTx(\hh(\ke, j)) \implies i \in \vi(\ke') \).
\end{itemize}
\item If the transaction \( \txid'' \in \T_\rd \) is a read-only transaction, 
since \( \T_\rd \) is initial empty, there must exist a later transaction \( \txid''' \) from the same client that writes to a key,
and such transaction \( \txid''' \) is included in \( \Tx(\mkvs, \vi) \):
\[
    \txid' \toEdge{\SO_{\aexec'}} \txid'' 
    \toEdge{\SO_{\aexec'}} \txid''' \toEdge{\VIS_{\aexec'}} \txid 
    \land \txid''' \in \Tx(\mkvs,\vi)
\]
Since \( \SO \) is transitive, 
therefore \( \txid' \toEdge{\SO_{\aexec'}} \txid''' \toEdge{\VIS_{\aexec'}} \txid \),
which we have already proven \( \txid' \in \Tx(\mkvs, \vi) \) or we will include \( \txid' \) in \( \T_\rd \).
Since there are finite transactions from a client in a trace, there must exist a \( \T_\rd \) in the end.
\end{itemize}


The execution test $\ET_{\MW}$ is complete with respect to 
the axiomatic specification $(\RP_{\LWW}, \{\lambda \aexec.(\PO_{\aexec} ; \VIS_{\aexec})\})$. 
Let $\aexec$ be an abstract execution that satisfies the specification
$\CMa(\RP_{\LWW}, \Set{\lambda \aexec.(\PO_{\aexec} ; \VIS_{\aexec})})$, 
and consider a transaction $\txid \in \T_{\aexec}$. 
Assume i-\emph{th} transaction \( \txid_i \) in the arbitrary order,
and let view \( \vi_{i} = \getView(\aexec, \VIS^{-1}_{\aexec}(\txid_{i}) ) \).
We also pick any final view such that \( \vi'_{i} \subseteq \getView(\aexec, (\AR^{-1}_{\aexec})?(\txid_{i}) ) \).
It suffices to prove \( \ET_\MW \vdash (\hh_{\cut(\aexec, i-1)}, \vi_i ) \csat  \TtoOp{T}_{\aexec}(\txid_{i}) : \vi'_{i} \).
It means to prove the follows:
\begin{equation}
\label{equ:mw-complete}
\begin{array}{@{}l@{}}
    \fora{j,m,\ke, \ke' } j \in \vi(\ke)  \\
    \quad {} \wedge \WTx(\hh_{\cut(\aexec, i-1)}(\ke', m)) \xrightarrow{\PO?} \WTx(\hh_{\cut(\aexec, i-1)}(\ke, j))  \\
    \qqquad \implies m \in \vi(\ke')
\end{array}
\end{equation}
Assume \( j \) and \( \ke' \) such that \( j \in \vi(\ke')\), which means \( \WTx(\hh_{\cut(\aexec, i-1)}(\ke', j)) \in \VIS^{-1}_{\aexec}(\txid_{i}) \).
Now let consider transaction \( \txid \) that commits before \( \txid \) from the same session, \ie \( \txid \toEdge{\SO} \WTx(\hh_{\cut(\aexec, i-1)}(\ke, j)) \).
By the constraint \( \lambda \aexec.(\PO_{\aexec} ; \VIS_{\aexec}) \), the transaction \( \txid \in \VIS^{-1}_{\aexec}(\txid_{i}) \).
It means that in the kv-store \(  \hh_{\cut(\aexec, i-1)} \) every version written by \( \txid =  \WTx(\hh_{\cut(\aexec, i-1)}(\ke', m)) \) should be included in the view \( m \in \vi_i(\ke') \).
Thus we have the proof of \cref{equ:mw-complete}.

\subsubsection{Read Your Write \( \RYW \) }

\label{sec:sound-complete-ryw}

The execution test $\ET_\RYW$ is sound with respect to the axiomatic specification 
$(\RP_{\LWW}, \{\lambda \aexec. \PO_{\aexec} \})$.
We pick an invariant for the \( \ET_\RYW \) as the following:
\[
    I(\aexec, \cl) = \left( \bigcup_{\{\txid_{\cl}^{n} \in \T_{\aexec} \mid n \in \Nat\}} (\SO_{\aexec}^{-1})?(\txid^n_\cl) \right) \setminus \T_\rd
\]
where \( \T_\rd \) is all the read-only transactions in \( \bigcup\limits_{\{\txid_{\cl}^{n} \in \T_{\aexec} \mid n \in \Nat\}} (\SO_{\aexec}^{-1})?(\txid^n_\cl) \).
Assume a key-value store $\hh$, an initial and a final view $\vi, \vi'$  a fingerprint $\opset$ 
such that $\ET_{\RYW} \vdash (\hh, \vi) \triangleright \opset: \vi'$. 
Also choose an arbitrary $\cl$, a transaction identifier $\txid_\cl^n \in \nextTxId(\hh, \cl)$, 
and an abstract execution $\aexec$ such that $\hh_{\aexec} = \hh$ and 
\( I(\aexec, \cl) \subseteq \Tx(\hh, \vi) \).
Let a new abstract execution \( \aexec' = \extend(\aexec, \txid_\cl^n, \f, \Tx(\mkvs, \vi) \cup \T_\rd) \).
We need to prove that \( \aexec' \) satisfies the constraint and the invariant is preserved:
\begin{itemize}
    \item \( \txid \in \Tx(\mkvs, \vi) \cup \T_\rd  \) for all \( \txid \) such that \( \txid \toEdge{\SO_{\aexec'}} \txid_\cl^n  \). 
    Assume a transaction \( \txid \) such that \( \txid \toEdge{\SO_{\aexec'}} \txid_\cl^n \).
It immediately implies that \( \txid = \txid_\cl^m\) where \( m < n \) and \( \txid_\cl^m \in \aexec \).
Thus we prove that 
\[ 
    \txid \in \left( \bigcup_{\{\txid_{\cl}^{n} \in \T_{\aexec} \mid n \in \Nat\}} (\SO_{\aexec}^{-1})?(\txid^n_\cl) \right) \subseteq \Tx(\mkvs,\vi) \cup \T_\rd
\]
\item \(I(\aexec',\cl) \subseteq \Tx(\mkvs_{\aexec'}, \vi') \).
Let \( \T'_\rd = \T_\rd \) if the new transaction \( \txid_\cl^n\) has writes, otherwise \( \T'_\rd = \T_\rd \cup \Set{\txid_\cl^n}\).
First we have
\[ I(\aexec', \cl) = \left(\bigcup\limits_{\{\txid_{\cl}^{m} \in \T_{\aexec'} \mid m \in \Nat\}} (\SO_{\aexec'}^{-1})?(\txid^m_\cl) \right) \setminus \T'_{\rd} = \left( (\SO_{\aexec'}^{-1})?(\txid^n_\cl) \right) \setminus \T'_\rd 
\]
Note that \( \txid^n_\cl \) is the latest transaction committed by the client \( \cl \).
For any transaction \( \txid \in (\SO_{\aexec'}^{-1})?(\txid^n_\cl) \setminus \T'_\rd \) that has write,
because execution test requires \( z \in \vi'(\ke) \) for any key \( \ke \) and index \( z \) such that \( \WTx( \mkvs_{\aexec'}(\ke, z) ) \toEdge{\SO_\aexec} \txid \),
then \( \txid \in \Tx(\mkvs_{\aexec'}, \vi') \) as what we wanted.
\end{itemize}

The execution test $\ET_{\RYW}$ is complete with respect to 
the axiomatic specification $(\RP_{\LWW}, \Set{\lambda \aexec.\PO_{\aexec} })$. 
Let $\aexec$ be an abstract execution that satisfies the specification
$\CMa(\RP_{\LWW}, \Set{\lambda \aexec.\PO_{\aexec} })$.
Assume i-\emph{th} transaction \( \txid_i \) in the arbitrary order,
and let view \( \vi_{i} = \getView(\aexec, \VIS^{-1}_{\aexec}(\txid_{i}) ) \).
We construct the final view \( \vi'_i\) depending on whether \( \txid_i \) is the last transaction from the client.
\begin{itemize}
\item If the transaction \( \txid'_i = \min_{\SO_\aexec}\left(\Setcon{\txid'}{\txid_i \toEdge{\SO_\aexec} \txid' } \right) \)  is defined,
then \( \vi'_i = \getView(\aexec, \T_i) \) where \( \T_i \subseteq (\AR_{\aexec}^{-1})?(\txid_i) \cap \VIS_\aexec^{-1}(\txid'_i) \) for some \( \T_i \).
Given the specification \( \lambda \aexec.\PO_{\aexec} \), 
we know \( \SO_\aexec^{-1}(\txid'_i) \subseteq \VIS_\aexec^{-1}(\txid'_i) \).
We pick \( \T_i = (\AR_\aexec^{-1})?(\txid_i) \cap \SO_\aexec^{-1}(\txid'_i) = (\SO_\aexec^{-1})?(\txid_i) \).
To recall \( \vi'_i = \getView(\aexec, \T_i) \), therefore \( \ET_\RYW \vdash (\hh_{\cut(\aexec, i-1)}, \vi_i) \csat \TtoOp{T}_{\aexec}(\txid_{i}) : \vi'_{i} \).
\item If there is no other transaction after \( \txid_i \) from the same client,
we pick \( \vi'_i = \getView(\aexec, \T_i) \) where \( \T_i = (\SO_\aexec^{-1})?(\txid_i) \),
so \( \ET_\RYW \vdash (\hh_{\cut(\aexec, i-1)}, \vi_i) \csat \TtoOp{T}_{\aexec}(\txid_{i}) : \vi'_{i} \).
\end{itemize}

\subsubsection{Write Following Read \( \WFR \) }

\label{sec:sound-complete-wfr}

The write-read relation  on \( \aexec \) is defined as the following:
\[
\WR(\aexec, \ke) \defeq \Setcon{ (\txid, \txid') }{ \exsts{\val} (\otW, \ke, \val) \in_\aexec \txid \land (\otR, \ke, \val) \in_\aexec \txid' \land \txid = \max_\AR(\VIS^{-1}(\txid')) }
\]
The notation \( \WR_\aexec \) is defined as \( \WR_\aexec \defeq \bigcup\limits_{\ke \in \Keys} \WR(\aexec, \ke) \).
Note that for a key-value store \( \mkvs \) such that \( \mkvs = \mkvs_\aexec \),
by the definition of  \(  \mkvs = \mkvs_\aexec \), 
the following holds:
\[
    \WR_\aexec = \Setcon{(\txid, \txid')}{\exsts{\ke, i } \mkvs(\ke, i) = (\stub, \txid, \txid'\cup \stub)}
\]
Note that such \( \WR_\aexec \) coincides with \( \WR_\Gr \).

The execution test $\ET_\WFR$ is sound with respect to the axiomatic specification 
\( (\RP_{\LWW}, \Set{\lambda \aexec. \WR_\aexec ; (\PO_{\aexec})? ; \VIS_{\aexec} })\).
We pick the invariant as \( I( \aexec, \cl ) = \emptyset \), given the fact of no constraint on the view after update.
Assume a key-value store $\hh$, an initial and a final view $\vi, \vi'$  a fingerprint $\opset$ 
such that $\ET_{\WFR} \vdash (\hh, \vi) \triangleright \opset: \vi'$. 
Also choose an arbitrary $\cl$, a transaction identifier $\txid \in \nextTxId(\hh, \cl)$, 
and an abstract execution $\aexec$ such that $\hh_{\aexec} = \hh$ and 
\( I(\aexec, \cl) =  \emptyset \subseteq \Tx(\hh, \vi) \).
Let \( \aexec' = \extend(\aexec, \txid, \Tx(\mkvs, \vi) , \f ) \).
Note that since the invariant is empty set, it remains to prove the following (the read-only transactions set is empty):
\[
    \begin{array}{@{}l@{}}
        \fora{ \txid' } 
        (\txid' ,\txid)  \in \WR(\aexec',\ke) ; (\PO_{\aexec'})? ; \VIS_{\aexec'} 
        \implies \txid' \in \Tx(\mkvs, \vi) 
    \end{array}
\]
Suppose \( (\txid' ,\txid)  \in \WR(\aexec', \ke) ; (\PO_{\aexec'})? ; \VIS_{\aexec'} \) for some key \( \ke \),
that is, \( \txid' \toEdge{\WR(\aexec', \ke)} \txid'' \toEdge{\SO_{\aexec'}?} \txid''' \toEdge{\VIS_{\aexec'}} \txid \) for some transaction \( \txid''' \).
It immediately implies that \( \txid''' \in \Tx(\mkvs, \vi)  \) by \( \aexec' = \extend(\aexec, \txid, \Tx(\mkvs, \vi) , \f ) \).
Because \( \txid' \toEdge{\WR(\aexec', \ke)} \txid'' \), there exists an index \( i \) such that \( \mkvs(\ke, i) = (\stub, \txid', \txid'' \cup \stub) \).
By the execution test \( \ET_\WFR \), we have \( i \in \vi( \ke ) \) then \( \txid' \in \Tx(\mkvs, \vi ) \).


The execution test $\ET_\WFR$ is complete with respect to the axiomatic specification 
\( (\RP_{\LWW}, \{\lambda \aexec. \WR(\aexec', \ke) ; (\PO_{\aexec'})? ; \VIS_{\aexec'} \})\).
Assume i-\emph{th} transaction \( \txid_i \) in the arbitrary order,
and let view \( \vi_{i} = \getView(\aexec, \VIS^{-1}_{\aexec}(\txid_{i}) ) \).
We also pick any final view such that \( \vi'_{i} \subseteq \getView(\aexec, (\AR^{-1}_{\aexec})?(\txid_{i}) ) \).
Note that there is nothing to prove for \( \vi'_i \),
so it is sufficient to prove the following:
\[
    \begin{array}{l}
    \fora{\ke, \ke', m, j, \txid'} j \in \vi(\ke) \\
    \quad {} \land \txid' \in \RTx(\hh_{\cut(\aexec, i-1)}(\ke', m)) \land \txid' {\xrightarrow{\PO?}} \WTx(\hh_{\cut(\aexec, i-1)}(\ke, j)) ) \\
    \qqquad \implies m \in \vi(\ke')
    \end{array}
\]
Given a key \( \ke \) and an index \( j \) such that \( j \in \vi(\ke) \), 
it means that the writer \( \txid \) of the version \( \hh_{\cut(\aexec, i-1)}(\ke, j) \) is visible, \ie \( \txid \in \VIS_{\aexec'}^{-1}(\txid_i) \).
Assume some \( \txid' \) such that \( (\txid', \txid) \in (\PO_{\aexec'})? \) and reads a version of some key \( \mkvs(\ke',m) \).
Therefore, we know the writer of the key \( \txid'' = \WTx(\mkvs(\ke',i)) \) has a write-read edge to \( \txid' \), \ie \( \txid'' \toEdge{\WR_\aexec} \txid'\) 
By the constraint on abstract execution \( \aexec \), we know \( \txid'' \in \VIS^{-1}_{\aexec}(\txid_{i}) \),
which means \( m \in \vi(\ke')\) by the definition of \( \getView \).

\subsubsection{Causal Consistency \( \CC \)}
\label{sec:sound-complete-cc}

The wildly used specification on abstract executions for causal consistency is that \( \VIS \) is transitive.
Yet it is for the sack of elegant specification, while there is a minimum visibility relation given by \( (\WR_\aexec \cup \SO_\aexec)^{+} ; \VIS_\aexec \subseteq \VIS_\aexec \) (\cref{lem:aexec-spec-cc}).

\begin{lemma}
    \label{lem:aexec-spec-cc}
    For any abstract execution \( \aexec \) under last-write-win, if it satisfies the following:
    \[
        (\WR_\aexec \cup \SO_\aexec)^{+} ; \VIS_\aexec \subseteq \VIS_\aexec \quad \SO_\aexec \subseteq \VIS_\aexec
    \]
    There exists a new abstract execution \( \aexec' \) where \( \T_\aexec = \T_{\aexec'} \), \( \AR_\aexec = \AR_{\aexec'} \),
    \( \VIS_{\aexec'} ; \VIS_{\aexec'} \subseteq \VIS_{\aexec'} \), and
    under last-write-win \( \TtoOp{T}_{\aexec}(\txid) = \TtoOp{T}_{\aexec'}(\txid) \) for all transactions \( \txid \).
\end{lemma}
\begin{proof}
    To recall, the write-read relation under a key \( \WR(\aexec, \ke) \) is defined as 
    \( \WR(\aexec, \ke) \defeq \Setcon{ (\txid, \txid') }{ \exsts{\val} (\otW, \ke, \val) \in_\aexec \txid \land (\otR, \ke, \val) \in_\aexec \txid' \land \txid = \max_\AR(\VIS^{-1}(\txid')) }\).
    Given an \( \aexec \) that satisfies the following
    \[
        (\WR_\aexec \cup \SO_\aexec )^{+} ; \VIS_\aexec \subseteq \VIS_\aexec \quad \SO_\aexec \subseteq \VIS_\aexec
    \]
    we erase some visibility relation for each transaction following the order of arbitration \( \AR \) until the visibility is transitive.
    Assume the i-\emph{th} transaction \( \txid_i \)  with respect to the arbitration order.
    Let \( R_i \) denote a new visibility for transaction \( \txid_i \) such that
    \( R_i\projection{2} = \Set{\txid_i}\)
    and the visibility relation before (including) \( \txid_i \) is transitive.
    Let \( \aexec_i = \mkvs_\cut(\aexec, i) \) and \( \VIS_i = \bigcup\limits_{ 0 \leq k \leq i} R_i \).
    For each step, says i-\emph{th} step, we  preserve the following:
    \begin{gather}
        \VIS_i ; \VIS_i \subseteq \VIS_i \label{equ:vis-i-transitive} \\
        \fora{\txid} (\txid,\txid_i) \in R_i \implies (\txid, \txid_i) \in (\WR_i \cup \SO_i)^{+}
        \label{equ:last-read-correct}
    \end{gather}
    
    \begin{itemize}
    \item \caseB{\( i = 1 \) and \( R_1 = \emptyset \)}
    Assume it is from client \( \cl \).
    There is no transaction committed before, so \( \VIS_1 = \emptyset \) and \( \VIS_1 ; \VIS_1 \subseteq \VIS_1 \) as \cref{equ:vis-i-transitive}.

    \item \caseI{i-\emph{th} step}
    Suppose the (i-1)-\emph{th} step satisfies \cref{equ:vis-i-transitive} and \cref{equ:last-read-correct}.
    Let consider i-\emph{th} step and the transaction \( \txid_i \).
    Initially we take \( R_i \) as empty set.
    We first extend \( R_i \) by closing with respect to \( \WR_i \)
    and prove that it does not affect any read from the transaction \( \txid_i \).
    Then we will do the same for \( \SO_i \).
    \begin{itemize}
        \item \( \WR_i\). For any read \( (\otR, \ke, \val ) \in \txid_i \),
        there must be a transaction \( \txid_j \) that \( \txid_j \toEdge{\WR(\aexec_i,\ke), \AR} \txid_i \) and \( j < i \).
        We include \( (\txid_j, \txid_i) \in R_i \).
        Let consider all the visible transactions of \( \txid_j \).
        Assume a transaction \( \txid' \in \VIS_{i-1}^{-1}(\txid_j) \), 
        thus \( \txid' \in \VIS_{j}^{-1}(\txid_j) = R_j^{-1}(\txid_j) \).
        It is safe to include \( (\txid', \txid_i) \in R_i \) without affecting the read result,
        because those transaction \( \txid' \) is already visible for \( \txid_i \) in the abstract execution \( \aexec \):
        by \cref{equ:last-read-correct} we know \( R_j \subseteq (\WR_j \cup \SO_j)^{+} \subseteq (\WR_\aexec \cup \SO_\aexec)^{+}\),
        and by the definition of \( \WR(\aexec_i,\ke) \) we know \( \WR(\aexec_i,\ke) \subseteq \VIS_\aexec\).

        \item Given \( \SO_\aexec \subseteq \VIS_\aexec \), we include \( (\txid_j,\txid_i) \) for some \( \txid_j \)
        such that \( \txid_j \toEdge{\SO_\aexec} \txid_i\).
        For the similar reason as \( \WR \),
        it is safe to includes all the visible transactions \( \txid' \) for \( \txid_j \), \ie \( \txid' \in R_j^{-1}\).
        \end{itemize}
        
    By the construction, both \cref{equ:vis-i-transitive} and \cref{equ:last-read-correct} are preserved. 
    Thus we have the proof.
    \end{itemize}
\end{proof}

By \cref{lem:aexec-spec-cc}, the execution test $\ET_\CC$ is sound with respect to the axiomatic specification 
\( (\RP_{\LWW}, \Set{\lambda \aexec. ( \SO_{\aexec} \cup \WR_{\aexec} )^{+} ; \VIS_{\aexec}, \lambda \aexec \ldotp \SO_\aexec })\).
We pick an invariant for the \( \ET_\CC \) as the union of those for \( \MRd\) and \( \RYW \) shown in the following:
\[  
\begin{rclarray}
    I_1(\aexec, \cl) & = & \left( \bigcup\limits_{\{\txid_{\cl}^{n} \in \T_{\aexec} \mid n \in \Nat\}} \VIS_{\aexec}^{-1}(\txid^n_\cl) \right) \setminus \T_\rd \\
    I_2(\aexec, \cl) & = & \left( \bigcup\limits_{\{\txid_{\cl}^{n} \in \T_{\aexec} \mid n \in \Nat\}} (\SO_{\aexec}^{-1})?(\txid^n_\cl) \right) \setminus \T_\rd
\end{rclarray}
\]
where \( \T_\rd \) is all the read-only transactions included in both 
\( \left( \bigcup\limits_{\{\txid_{\cl}^{n} \in \T_{\aexec} \mid n \in \Nat\}} \VIS_{\aexec}^{-1}(\txid^n_\cl) \right)\) 
and \( \left( \bigcup\limits_{\{\txid_{\cl}^{n} \in \T_{\aexec} \mid n \in \Nat\}} (\SO_{\aexec}^{-1})?(\txid^n_\cl) \right) \).
Assume a key-value store $\hh$, an initial and a final view $\vi, \vi'$  a fingerprint $\opset$ 
such that $\ET_{\CC} \vdash (\hh, \vi) \triangleright \opset: \vi'$. 
Also choose an arbitrary $\cl$, a transaction identifier $\txid_\cl^n \in \nextTxId(\hh, \cl)$, 
and an abstract execution $\aexec$ such that $\hh_{\aexec} = \hh$ and 
\( I_1(\aexec, \cl) \cup I_2(\aexec, \cl) \subseteq \Tx(\hh, \vi) \).
Let a new abstract execution \( \aexec' = \extend(\aexec, \txid_\cl^n, \f, \Tx(\mkvs, \vi) \cup \T_\rd) \).
We are about to prove there exists an extra set of read-only transactions \( \T'_\rd \) such that:
\begin{gather}
    \fora{\txid} (\txid, \txid_\cl^n) \in \SO_{\aexec'} \implies \txid \in \Tx(\mkvs, \vi) \cup \T_\rd \cup \T'_\rd \label{equ:cc-sound-update-so}\\
    \fora{\txid} (\txid, \txid_\cl^n) \in ( \SO_{\aexec'} \cup \WR_{\aexec'} )^{+} ; \VIS_{\aexec'} \implies \txid \in \Tx(\mkvs, \vi) \cup \T_\rd \cup \T'_\rd \label{equ:cc-sound-update-visvis}\\
    I_1(\aexec',\cl) \cup I_2(\aexec',\cl) \subseteq \Tx(\mkvs_{\aexec'}, \vi') \label{equ:cc-sound-inv} 
\end{gather}

\begin{itemize}
\item The invariant \( I_2 \) implies \cref{equ:cc-sound-update-so} as the same as \( \RYW \) in \cref{sec:sound-complete-ryw}.
\item To prove \cref{equ:cc-sound-update-visvis}, let \( \T'_\rd = \emptyset \) initially,
and more read-only transactions will be added in \( \T'_\rd \) until the \cref{equ:cc-sound-update-visvis} holds.
Assume transaction \( \txid \) such that \( ((\txid, \txid_\cl^n) \in ( \SO_{\aexec'} \cup \WR_{\aexec'} )^{+} ; \VIS_{\aexec'}) \).
That is, there exists some transaction \( \txid' \) such that
\( \txid \toEdge{( \SO_{\aexec'} \cup \WR_{\aexec'} )^+}  \txid' \toEdge{\VIS_{\aexec'}} \txid_\cl^n\).
We consider two cases for \( \txid' \): \( \txid' \) is also visible by previous transactions from the same client; 
or \( \txid' \) is a newly visible transaction for the client.
\begin{itemize}
    \item if \( \txid' \) is also visible by previous transactions from the same client, it means \( \txid' \toEdge{\VIS_{\aexec'}} \txid_\cl^m \) for some \( m < n \).
    The edge already exists before, therefore \( \txid' \toEdge{\VIS_\aexec} \txid_\cl^m \).
    Since \( \txid \toEdge{( \SO_{\aexec} \cup \WR_{\aexec} )^{+}} \txid' \) and \( ( \SO_{\aexec} \cup \WR_{\aexec} )^{+} ; \VIS_{\aexec} \subseteq \VIS_{\aexec} \),
    we know \( \txid \toEdge{ \VIS_{\aexec} }  \txid_\cl^m  \).
    Because of \( I_1 \) and \( \txid_\cl^m \toEdge{\SO} \txid_\cl^n \), then \( \txid \in I_1 \cup \T_\rd \subseteq \Tx(\mkvs, \vi) \cup \T_\rd \).
    
    \item if \( \txid' \) is a newly visible transaction for the client \( \cl \),
    it suffices to prove \(\txid \in \Tx(\mkvs, \vi) \cup \T_\rd \cup \T'_\rd \) 
    for some \( \txid \toEdge{ ( \SO_{\aexec'} \cup \WR_{\aexec'} ) ; \VIS_{\aexec'} } \txid_\cl^n  \)%
    \footnote{For two relation \( R_1, R_2\), \( R_1^* ; R_2 \subseteq R_2 \iff R_1 ; R_2 \subseteq R_2 \) }.
    Since \( \txid' \toEdge{\VIS_{\aexec'}} \txid_\cl^n \) so \( \txid' \in Tx(\mkvs, \vi) \cup \T_\rd \cup \T'_\rd\).
    More specifically, \( \txid' \) is not visible for the client before, we know \( \txid' \in Tx(\mkvs, \vi) \cup \T'_\rd\).
    We perform case analysis if \( \txid' \) has write.
    \begin{itemize}

        \item If \( \txid' \) writes to some keys, \ie \( \txid' \in Tx(\mkvs, \vi) \), 
        because  \( \CC \) satisfies \( \MW \) and \( \WFR \),
        and by the execution tests for \( \MW \) and \( \WFR \) (the proofs follows \cref{sec:sound-complete-mw} and \cref{sec:sound-complete-wfr}),
        the \( \txid \) is either already in \( \Tx(\mkvs, \vi) \), 
        or \( \txid \) is a read-only and we include it in \( \T'_\rd \).

        \item If \( \txid' \) is a read-only transaction,
        given that \( \T'_\rd \) initially is empty set,
        we know there exists a third transaction \( \txid'' \) that writes to some keys
        and it satisfies \( \txid \toEdge{( \SO_{\aexec'} \cup \WR_{\aexec'} )}  \txid' \toEdge{( \SO_{\aexec'} \cup \WR_{\aexec'} )}  \txid'' \toEdge{\VIS_{\aexec'}} \txid_\cl^n \).
        Since \( \txid'' \) has a write, it means 
        \( \txid \toEdge{( \SO_{\aexec'} \cup \WR_{\aexec'} )}  \txid' \toEdge{( \SO_{\aexec'} )}  \txid'' \toEdge{\VIS_{\aexec'}} \txid_\cl^n \).
        \begin{itemize}
            \item if \( \txid \toEdge{\WR_{\aexec'} }  \txid' \toEdge{ \SO_{\aexec'} }  \txid'' \toEdge{\VIS_{\aexec'}} \txid_\cl^n \),
            this is exactly \( \WFR \).
            Therefore,the \( \txid \) is either already in \( \Tx(\mkvs, \vi) \), 
            or \( \txid \) is a read-only and we include it in \( \T'_\rd \).

            \item if \( \txid \toEdge{\SO_{\aexec'} }  \txid' \toEdge{ \SO_{\aexec'} }  \txid'' \toEdge{\VIS_{\aexec'}} \txid_\cl^n \),
            because \( \SO \) is transitive, we have \( \txid \toEdge{\SO_{\aexec'}} \txid'' \toEdge{\VIS_{\aexec'}} \txid_\cl^n \).
            By previous case we already know \( \txid \in \Tx(\mkvs, \vi) \cup \T'_\rd \).
        \end{itemize}
    \end{itemize}
    \end{itemize}
    \item Finally the new abstract execution preserves the invariant \( I_1 \) and \( I_2 \) 
    because  \( \CC \) satisfies \( \MW \) and \( \RYW \).
    The proofs are the same as those in \cref{sec:sound-complete-mr} and \cref{sec:sound-complete-ryw}.

\end{itemize}

The execution test $\ET_\CC$ is complete with respect to the axiomatic specification 
\( (\RP_{\LWW}, \Set{\lambda \aexec. \VIS_{\aexec} ; \VIS_{\aexec}, \lambda \aexec \ldotp \SO_\aexec })\).

For \( \MRd \), since \(  \VIS_\aexec ; \SO_\aexec \subseteq  \VIS_\aexec ; \VIS_{\aexec} \subseteq \VIS_\aexec \),
the proof is as the same proof as in \cref{sec:sound-complete-mr}.
For \( \MW \), since \( \SO_\aexec ; \VIS_\aexec \subseteq  \VIS_\aexec ; \VIS_{\aexec} \subseteq \VIS_\aexec \),
the proof is as the same proof as in \cref{sec:sound-complete-mw}.
For \( \RYW \), since \( \SO_\aexec \subseteq \VIS_\aexec \),
the proof is as the same proof as in \cref{sec:sound-complete-ryw}.
For \( \WFR \), since \( \WR_\aexec ; \SO_\aexec? ; \VIS_\aexec \subseteq \VIS_\aexec ; \VIS_{\aexec}? ; \VIS_\aexec \subseteq \VIS_\aexec \),
the proof is as the same proof as in \cref{sec:sound-complete-wfr}.

\subsubsection{Update Atomic \( \UA \)}
\label{sec:sound-complete-ua}

Given abstract execution \( \aexec \), we define write-write relation for a key \( \ke \) as the following:
\[ 
    \WW(\aexec,\ke) \defeq \Setcon{(\txid, \txid')}{\txid \toEdge{\AR_\aexec} \txid' \land (\otW,\ke, \stub ) \in \txid \land (\otW,\ke, \stub ) \in \txid'  } 
\]
Then, the notation \( \WW_\aexec \defeq \bigcup\limits_{\ke \in \Keys} \WW(\aexec, \ke) \).
Note that for a key-value store \( \mkvs \) such that \( \mkvs = \mkvs_\aexec \),
by the definition of  \(  \mkvs = \mkvs_\aexec \), 
the following holds:
\[
    \WW_\aexec = \Setcon{(\txid, \txid')}{\exsts{\ke, i,j } \txid = \WTx(\mkvs(\ke, i)) \land \txid' = \WTx(\mkvs(\ke, j)) \land i < j}
\]
Also the \( \WW_\aexec \) coincide with \( \WW_\Gr \).

The execution test $\ET_\UA$ is sound with respect to the axiomatic specification \( (\RP_{\LWW}, \Set{\lambda \aexec. \WW_\aexec }) \).
We pick the invariant as \( I( \aexec, \cl ) = \emptyset \), given the fact of no constraint on the final view.
Assume a key-value store $\hh$, an initial and a final view $\vi, \vi'$  a fingerprint $\opset$ 
such that $\ET_{\UA} \vdash (\hh, \vi) \triangleright \opset: \vi'$. 
Also choose an arbitrary $\cl$, a transaction identifier $\txid \in \nextTxId(\hh, \cl)$, 
and an abstract execution $\aexec$ such that $\hh_{\aexec} = \hh$ and 
\( I(\aexec, \cl) =  \emptyset \subseteq \Tx(\hh, \vi) \).
Let \( \aexec' = \extend(\aexec, \txid, \Tx(\mkvs, \vi), \f ) \).
Note that since the invariant is empty set, it remains to prove the following:
\[
    \begin{array}{@{}l@{}}
        \fora{ \txid' } \txid' \toEdge{\WW_{\aexec'}} \txid \implies \txid' \in \Tx(\mkvs, \vi)
    \end{array}
\]
Assume a transaction \( \txid' \) that writes to a key \( \ke \) as \( \txid \), \ie \( \txid' \toEdge{\WW_{\aexec'}} \txid \).
Since that \( \txid' \) is a transaction already existing in \( \mkvs\),
we have \( \WTx(\mkvs(\ke, i)) = \txid' \) for some index \( i \).
By the execution test of \( \UA \), we know \( i \in \vi(\ke) \) therefore \( \txid' \in \Tx(\mkvs, \vi) \).

The execution test $\ET_\UA$ is complete with respect to 
the axiomatic specification \( (\RP_{\LWW}, \Set{\lambda \aexec. \WW_\aexec }) \).
Assume i-\emph{th} transaction \( \txid_i \) in the arbitrary order,
and let view \( \vi_{i} = \getView(\aexec, \VIS^{-1}_{\aexec}(\txid_{i}) ) \).
We also pick any final view such that \( \vi'_{i} \subseteq \getView(\aexec, (\AR^{-1}_{\aexec})?(\txid_{i}) ) \).
Note that there is nothing to prove for \( \vi'_i \),
so it is sufficient to prove the following:
\[
    \fora{\ke} (\otW, \ke, \stub) \in \TtoOp{T}_{\aexec}(\txid_{i}) 
    \implies 
    \fora{j : 0 \leq j < \abs{\mkvs_{\cut(\aexec, i-1)}(\ke)}} j \in \vi_i(\ke)
\]
Let consider a key \( \ke \) that have been overwritten by the transaction \( \txid_i \).
By the constraint of \( \aexec \) that \( \WW_\aexec \subseteq \VIS_\aexec \),
for any transaction \( \txid \) that writes to the same key \( \ke \) and committed before \( \txid_i \), 
they are included in the visible set \(\txid \in \VIS^{-1}_{\aexec}(\txid_{i}) \).
Note that \( \txid \toEdge{\WW_\aexec} \txid_i \implies \txid \toEdge{\AR_\aexec} \txid_i \implies \txid \in \mkvs_{\cut(\aexec,i-1)}\).
Since that the transaction \( \txid \) write to the key \( \ke \),
it means \( \WTx(\mkvs_{\cut(\aexec, i-1)}(\ke,j)) = \txid \) for some index \( j \).
Then by the definition of \( \getView \), we have \( j \in \vi_i(\ke)\).

\subsubsection{Consistency Prefix \( \CP \) }

Given abstract execution \( \aexec \), we define read-write read-write relation:
\[
    \RW(\aexec,\ke) \defeq \Setcon{(\txid, \txid')}{\txid \toEdge{\AR_\aexec} \txid' \land (\otR,\ke, \stub ) \in \txid \land (\otW,\ke, \stub ) \in \txid'  } 
\]
It is easy to see \( \RW(\aexec,\ke) \)  can be derived from \( \WW(\aexec,\ke) \) and \( \WR(\aexec, \ke ) \) as the following:
\[
    \RW(\aexec,\ke) = \Setcon{(\txid, \txid')}{ \exsts{\txid'' } (\txid'', \txid) \in \WR(\aexec, \ke) \land (\txid'', \txid') \in \WW(\aexec, \ke) }
\]
Then, the notation \( \RW_\aexec \defeq \bigcup\limits_{\ke \in \Keys} \RW(\aexec, \ke) \).
Note that for a key-value store \( \mkvs \) such that \( \mkvs = \mkvs_\aexec \),
by the definition of  \(  \mkvs = \mkvs_\aexec \), 
the following holds:
\[
    \RW_\aexec = \Setcon{(\txid, \txid')}{\exsts{\ke, i,j } \txid \in \RTx(\mkvs(\ke, i)) \land \txid' = \WTx(\mkvs(\ke, j)) \land i < j}
\]
The \( \RW_\aexec \) also coincides with \( \RW_\Gr \).


An abstract execution \( \aexec \) satisfies consistency prefix (\(\CP\)), 
if it satisfies \( \AR_\aexec ; \VIS_\aexec \subseteq \VIS_\aexec \) and \( \SO_\aexec \subseteq \VIS_\aexec \).
Given the specification, there is a corresponding specification on dependency graph by solve the following inequalities:
\[
    \begin{array}{@{}l@{}}
        \WR \subseteq \VIS \\
        \WW \subseteq \AR \\
        \VIS \subseteq \AR \\
        \VIS ; \RW \subseteq \AR \\
        \AR ; \AR \subseteq \AR  \\
        \SO \subseteq \VIS \\
        \AR ; \VIS \subseteq \VIS
    \end{array}
\]
By solving the inequalities the visibility and arbitration relations are:
\[
    \begin{rclarray}
        \AR & \defeq & \left( (\SO \cup \WR ) ; \RW? \cup \WW \cup R \right)^+ \\
        \VIS & \defeq & \left( (\SO \cup \WR ) ; \RW? \cup \WW \cup R \right)^* ; (\SO \cup \WR )
    \end{rclarray}
\]
for some relation \( R \subseteq \AR \).
When \( R = \emptyset \), it is the smallest solution therefore the minimum visibility required.

\sx{A bit verbal}
\begin{lemma}
    \label{lem:cp-eauiv-spec}
    For any abstract execution \( \aexec \),
    if it satisfies 
    \[
        \left( (\SO \cup \WR ) ; \RW? \cup \WW \right)^* ; \VIS_\aexec \subseteq \VIS_\aexec 
        \qquad \SO_\aexec \subseteq \VIS_\aexec
    \]
    then there exists a new \( \aexec' \) such that \( \T_\aexec = \T_{\aexec'} \), 
    under last-write-win \( \TtoOp{T}_{\aexec}(\txid) = \TtoOp{T}_{\aexec'}(\txid) \) for all transactions \( \txid \),
    and the relations satisfy the following:
    \[ 
        \AR_{\aexec'} ; \VIS_{\aexec'} \subseteq \VIS_{\aexec'}  \qquad \SO_{\aexec'} \subseteq \VIS_{\aexec'}
    \]
    and vice versa.
\end{lemma}
\begin{proof}
    Assume abstract execution \( \aexec' \) that
    satisfies \( \AR_{\aexec'} ; \VIS_{\aexec'} \subseteq \VIS_{\aexec'} \)
    and  \( \SO_{\aexec'} \subseteq \VIS_{\aexec'} \).
    We already show that:
\[
    \begin{rclarray}
        \AR_{\aexec'} & = & \left( (\SO_\aexec \cup \WR_\aexec ) ; \RW_\aexec? \cup \WW_\aexec \cup R \right)^+ \\
        \VIS_{\aexec'} & = & \left( (\SO_\aexec \cup \WR_\aexec ) ; \RW_\aexec? \cup \WW_\aexec \cup R \right)^* ; (\SO_\aexec \cup \WR_\aexec )
    \end{rclarray}
\]
for some relation \( R \subseteq \AR_{\aexec'} \).
If we take \( R  = \emptyset \), we have the proof for:
\[
        \SO \subseteq \VIS_\aexec \qquad 
        \left( (\SO_\aexec \cup \WR_\aexec ) ; \RW_\aexec? \cup \WW_\aexec \right)^* ; \VIS_\aexec \subseteq \VIS_\aexec
\]
For another way, we pick the \( R \) that extends
\( \left( (\SO_\aexec \cup \WR_\aexec ) ; \RW_\aexec? \cup \WW_\aexec \cup R \right)^+ \) 
to a total order.
\end{proof}

By \cref{lem:cp-eauiv-spec} to prove soundness and completeness of \( \ET_\CP \), it is sufficient to use the specification:
\[
    (\RP_{\LWW}, \Set{\lambda \aexec. \left( (\SO \cup \WR ) ; \RW? \cup \WW \right)^* ; \VIS_\aexec, \lambda \aexec \ldotp \SO_\aexec }) 
\]

For the soundness, we pick the invariant as the following:
\[  
\begin{rclarray}
    I_1(\aexec, \cl) & = & \left( \bigcup\limits_{\{\txid_{\cl}^{i} \in \T_{\aexec} \mid i \in \Nat\}} \VIS_{\aexec}^{-1}(\txid^i_\cl) \right) \setminus \T_\rd \\
    I_2(\aexec, \cl) & = & \left( \bigcup\limits_{\{\txid_{\cl}^{i} \in \T_{\aexec} \mid i \in \Nat\}} (\SO_{\aexec}^{-1})?(\txid^i_\cl) \right) \setminus \T_\rd
\end{rclarray}
\]
where \( \T_\rd \) is all the read-only transactions included in both 
\( \left( \bigcup\limits_{\{\txid_{\cl}^{i} \in \T_{\aexec} \mid i \in \Nat\}} \VIS_{\aexec}^{-1}(\txid^i_\cl) \right)\) 
and \( \left( \bigcup\limits_{\{\txid_{\cl}^{i} \in \T_{\aexec} \mid i \in \Nat\}} (\SO_{\aexec}^{-1})?(\txid^i_\cl) \right) \).
Assume a key-value store $\hh$, an initial and a final view $\vi, \vi'$  a fingerprint $\opset$ 
such that $\ET_{\CP} \vdash (\hh, \vi) \triangleright \opset: \vi'$. 
Also choose an arbitrary $\cl$, a transaction identifier $\txid_\cl^n \in \nextTxId(\hh, \cl)$, 
and an abstract execution $\aexec$ such that $\hh_{\aexec} = \hh$ and 
\( I_1(\aexec, \cl) \cup I_2(\aexec, \cl) \subseteq \Tx(\hh, \vi) \).
Let a new abstract execution \( \aexec' = \extend(\aexec, \txid_\cl^n, \f, \Tx(\mkvs, \vi) \cup \T_\rd) \).
We are about to prove that there exists an extra set of read-only transaction \( \T'_\rd \) such that:
\begin{gather}
    \fora{\txid} (\txid, \txid_\cl^n) \in \SO_{\aexec'} \implies \txid \in \Tx(\mkvs, \vi) \cup \T_\rd \cup \T'_\rd \label{equ:cp-sound-update-so}\\
    \begin{array}{l}
    \fora{\txid} (\txid, \txid_\cl^n) \in \left( (\SO_{\aexec'} \cup \WR_{\aexec'} ) ; \RW_{\aexec'}? \cup \WW_{\aexec'} \right)^* ; \VIS_{\aexec'} \\
    \qqquad \implies \txid \in \Tx(\mkvs, \vi) \cup \T_\rd \cup \T'_\rd 
    \end{array}
    \label{equ:cp-sound-update-arvis}\\
    I_1(\aexec',\cl) \cup I_2(\aexec',\cl) \subseteq \Tx(\mkvs_{\aexec'}, \vi') \label{equ:cp-sound-inv} 
\end{gather}
\begin{itemize}
\item the invariant \( I_2 \) implies the \cref{equ:cp-sound-update-so} where the proof is the same as \( \RYW \) in \cref{sec:sound-complete-ryw}.

\item For \cref{equ:cp-sound-update-arvis}, it is sufficient to prove one step inclusion, \ie
\[
    \begin{array}{l}
    \fora{\txid} (\txid, \txid_\cl^n) \in \left( (\SO_{\aexec'} \cup \WR_{\aexec'} ) ; \RW_{\aexec'}? \cup \WW_{\aexec'} \right) ; \VIS_{\aexec'} \\
    \qqquad \implies \txid \in \Tx(\mkvs, \vi) \cup \T_\rd \cup \T'_\rd 
\end{array}
\]
To prove above, let \( \T'_\rd \) initially be empty set.
We will add more read-only transactions until it satisfies \cref{equ:cp-sound-update-arvis}.
Assume a transaction \( \txid \) such that 
\( (\txid, \txid_\cl^n) \in \left( (\SO_{\aexec'} \cup \WR_{\aexec'} ) ; \RW_{\aexec'}? \cup \WW_{\aexec'} \right) ; \VIS_{\aexec'}\).
There exists a transaction \( \txid' \) such that \( \txid \toEdge{(\SO_{\aexec'} \cup \WR_{\aexec'} ) ; \RW_{\aexec'}? \cup \WW_{\aexec'}} \txid' \toEdge{\VIS_{\aexec'}}  \txid_\cl^n \).
It follows \( \txid'  \in \Tx(\mkvs, \vi) \cup \T_\rd \cup \T'_\rd  \).
There are two cases: \( \txid' \) writes to at least a key; or \( \txid' \) is a read-only transaction.
\begin{itemize}
    \item
    If \( \txid' \) writes to at least a key, then \( \txid' \in \Tx(\mkvs, \vi)\).
    Recall the \( \ddagger \) (and auxiliary relation under key-value store \( \mkvs\)) is defined as the following:
    \[
        \begin{rclarray}
            \func{RW^{-1}}{\mkvs, \ke, i} & \defeq & \Setcon{\txid}{\exsts{ j \leq i } \txid \in \RTx(\mkvs(\ke,j))} \\
            \ddagger & \equiv &
            \begin{array}[t]{@{}l@{}}
                \fora{\ke, \ke', i, j, m, \txid, \txid', \txid''} \\
                \left( \begin{array}{@{}l@{}}
                i \in \vi(\ke) 
                \land \txid \in \Set{\WTx(\mkvs(\ke,i))} \cup \func{RW^{-1}}{\mkvs, \ke, i} \land {} \\
                \quad \left(
                    \begin{array}{@{}l @{}}
                        \left( \begin{array}{@{}l@{}}
                                \txid' \in \func{SO^{-1}}{\txid} \land {} \\
                                \txid' \in \Set{\WTx(\mkvs(\ke',j))} \cup  \RTx(\mkvs(\ke',j))
                        \end{array} \right)  \lor {} \\
                        \left( \begin{array}{@{}l@{}}
                                \txid \in \RTx(\mkvs(\ke',j)) \land \txid' = \WTx(\mkvs(\ke',j))
                        \end{array} \right)
                        \end{array} \right) 
                    \end{array}
                    \right)  \\
                    {} \lor \left( \begin{array}{@{}l@{}}
                            i \in \vi(\ke) \land \ke = \ke' \land j < i
                    \end{array} \right) \\
                    \qquad \implies j \in \vi(\ke') 
            \end{array} \\
        \end{rclarray}
    \]
    We link the conditions in \( \ddagger \) to relation:
    \begin{itemize}
        \item \( \RW_\aexec\). Assume a key \( \ke \),  an index \( i \) and the writer \( \txid  = \WTx(\mkvs(\ke,i))\),
    then \( \txid' \in \RW^{-1}(\mkvs, \ke, i)\) if and only if \( \txid' \toEdge{\RW_\aexec} \txid\).
        \item \( \SO_\aexec\). The transaction identifiers encode the \( \SO_\aexec \).
        That is, \( \txid' \in \SO^{-1}(\txid)\) if and only if \(\txid' \toEdge{\SO_\aexec} \txid \).
        \item  \( \WR_\aexec \). It is easy to see \( \txid \in \RTx(\mkvs(\ke',j)) \land \txid' = \WTx(\mkvs(\ke',j)) \) if and only if \( \txid' \toEdge{\WR_\aexec} \txid \).
        \item \( \WW_\aexec \). The write-write relation describes the order of write operations for a key which corresponds the version orders in key-value store.
        That is, \( \txid' = \WTx(\mkvs(\ke,j)) \land \txid = \WTx(\mkvs(\ke,i)) \land j < i\) if and only if
        \( \txid' \toEdge{\WW_\aexec} \txid\).
    \end{itemize}
    Let assume \( \txid' \) writes to i-\emph{th} version a key \( \ke \).
    Given above and 
    \[ \txid \toEdge{(\SO_{\aexec'} \cup \WR_{\aexec'} ) ; \RW_{\aexec'}? \cup \WW_{\aexec'}} \txid' \toEdge{\VIS_{\aexec'}}  \txid_\cl^n \] we can substitute and rewrite the \( \ddagger \) as the following:
    \begin{gather}
        \begin{array}{@{}l@{}}
            \fora{\txid'',\ke',j}
            \WTx(\mkvs(\ke,i)) = \txid' \land {} \\
            \left( \begin{array}{@{}l@{}}
            \txid'' \toEdge{\RW_{\aexec'}?} \txid' \land {} \\
            \quad \left(
                \begin{array}{@{}l @{}}
                    \left( \begin{array}{@{}l@{}}
                            \txid \toEdge{\SO_{\aexec'}} \txid'' \land 
                            \txid \in \Set{\WTx(\mkvs(\ke',j))} \cup  \RTx(\mkvs(\ke',j))
                    \end{array} \right)  \\
                    {} \lor 
                    \left( \begin{array}{@{}l@{}}
                            \txid \toEdge{\WR_{\aexec'}} \txid'' \land \txid = \WTx(\mkvs(\ke',j))
                    \end{array} \right)
                    \end{array} \right) 
                \end{array}
                \right)  \\
                {} \lor \left( \begin{array}{@{}l@{}}
                        \txid \toEdge{\WW_{\aexec'}} \txid'' \land \txid = \WTx(\mkvs(\ke',j))
                \end{array} \right) \\
                \qquad \implies j \in \vi(\ke') 
        \end{array} 
        \label{equ:cp-dagger}
    \end{gather}
    Now we perform case analysis if \( \txid \) is a read-only transaction.
    \begin{itemize}
        \item if \( \txid \) has write, by the \cref{equ:cp-dagger} that \( j \in \vi(\ke')\) for all version \( \mkvs(\ke', j) \) written by the \( \txid\).
            Therefore by the definition of \( \Tx \), then \( \txid \in \VIS^{-1}(\txid_\cl^n)\).
        \item if \( \txid \) is a read-only transaction, we add it into \( \T'_\rd \).
    \end{itemize}
    \item 
    if \( \txid' \) is a read-only transaction, then either \( \txid' \in \T_\rd \) or \( \txid' \in \T'_\rd \).
    More specifically we have three cases: \textbf{(i)} \( \txid' \in \bigcup\limits_{\{\txid_{\cl}^{i} \in \T_{\aexec} \mid i \in \Nat\}} \VIS_{\aexec}^{-1}(\txid^i_\cl) \), \textbf{(ii)} \( \txid' \in \bigcup\limits_{\{\txid_{\cl}^{i} \in \T_{\aexec} \mid i \in \Nat\}} (\SO_{\aexec}^{-1})?(\txid^i_\cl) \) or \textbf{(iii)} \( \txid' \in \T'_\rd\).
    \begin{itemize}
        \item
        Assume \( \txid' \in \bigcup\limits_{\{\txid_{\cl}^{i} \in \T_{\aexec} \mid i \in \Nat\}} \VIS_{\aexec}^{-1}(\txid^i_\cl) \).
        It means \( \txid' \) is visible for some previous transaction \( \txid_\cl^m \) (\( m < n \)) from the same client \( cl \), 
        \ie 
        \[ 
            \txid \toEdge{(\SO_{\aexec'} \cup \WR_{\aexec'} ) ; \RW_{\aexec'}? \cup \WW_{\aexec'}} \txid' \toEdge{\VIS_{\aexec'}}  \txid_\cl^m 
        \]
        Note that all the edges before \( \txid_\cl^m \) must exist in \( \aexec \).
        Since \( \aexec \) satisfies the \( \left( (\SO \cup \WR ) ; \RW? \cup \WW \right)^* ; \VIS_\aexec \subseteq \VIS_\aexec \),
        we have \( \txid \toEdge{\VIS_{\aexec'}} \txid_\cl^m \) and then \( \txid \in \bigcup\limits_{\{\txid_{\cl}^{i} \in \T_{\aexec} \mid i \in \Nat\}} \VIS_{\aexec}^{-1}(\txid^i_\cl)\).
        By the invariant \( I_1 \), it means \( \txid \in \Tx(\mkvs, \vi) \cup \T_\rd \).
    \item \( \txid' \in \bigcup\limits_{\{\txid_{\cl}^{i} \in \T_{\aexec} \mid i \in \Nat\}} \SO_{\aexec}^{-1}(\txid^i_\cl) \).
    Since \( \txid' \) is a read-only transaction, 
    the edges can be simplified to \( \txid \toEdge{(\SO_{\aexec'} \cup \WR_{\aexec'} )} \txid' \toEdge{\SO_{\aexec'}}  \txid_\cl^n \).
    Given that \( \SO \) is transitive, then  either \( \txid \toEdge{\SO_{\aexec'}} \txid_\cl^n \) or \( \txid \toEdge{\WR_{\aexec'} } \txid' \toEdge{\SO_{\aexec'}}  \txid_\cl^n \).
    \begin{itemize}
        \item \( \txid \toEdge{\SO_{\aexec'}} \txid_\cl^n \).
            It follows \( \txid \in \bigcup\limits_{\{\txid_{\cl}^{i} \in \T_{\aexec} \mid i \in \Nat\}} \SO_{\aexec}^{-1}(\txid^i_\cl) = \Tx(\mkvs, \vi) \cup \T_\rd \).
        \item \( \txid \toEdge{\WR_{\aexec'} } \txid' \toEdge{\SO_{\aexec'}}  \txid_\cl^n \).
            The \( \WR \) edge must exists in \( \aexec \).
            Because \( \WR_\aexec \subseteq \VIS_\aexec \) then  \( \txid \toEdge{\VIS_{\aexec} } \txid' \toEdge{\SO_{\aexec'}}  \txid_\cl^n  \).
            It means 
            \[ 
                \txid \in \bigcup\limits_{\{\txid_{\cl}^{i} \in \T_{\aexec} \mid i \in \Nat\}} \VIS_{\aexec}^{-1}(\txid^i_\cl) = \Tx(\mkvs, \vi) \cup \T_\rd 
            \]
    \end{itemize}
    \item 
    Last, \( \txid' \in \T'_\rd \).
    Since \( \T'_\rd \) initially is empty set, there exists another write transaction \( \txid'' \) such that:
    \[
        \txid \toEdge{(\SO_{\aexec'} \cup \WR_{\aexec'} ) ; \RW_{\aexec'}? \cup \WW_{\aexec'}} \txid' \toEdge{(\SO_{\aexec'} \cup \WR_{\aexec'} ) ; \RW_{\aexec'}? \cup \WW_{\aexec'}} \txid'' \toEdge{\VIS_{\aexec'}}  \txid_\cl^n
    \]
    Given that \( \txid' \) is a read-only and \( \txid'' \) has write, the edges can be simplified:
    \[
        \txid \toEdge{(\SO_{\aexec'} \cup \WR_{\aexec'} )} \txid' \toEdge{\SO_{\aexec'} ; \RW_{\aexec'}?} \txid'' \toEdge{\VIS_{\aexec'}}  \txid_\cl^n
    \]
    Because transitivity of  \( \SO \), we have the following two cases:
    \[
        \begin{array}{@{}l@{}}
            \txid \toEdge{ \WR_{\aexec'} } \txid' \toEdge{\SO_{\aexec'} ; \RW_{\aexec'}?} \txid'' \toEdge{\VIS_{\aexec'}}  \txid_\cl^n \\
            \txid \toEdge{\SO_{\aexec'} ; \RW_{\aexec'}?} \txid'' \toEdge{\VIS_{\aexec'}}  \txid_\cl^n 
        \end{array}
    \]
    \begin{itemize}
        \item  \( \txid \toEdge{ \WR_{\aexec'} } \txid' \toEdge{\SO_{\aexec'} ; \RW_{\aexec'}?} \txid'' \toEdge{\VIS_{\aexec'}}  \txid_\cl^n \).
            If \( \txid \) has write, by \cref{equ:cp-dagger} especially \( \txid \in \Set{\WTx(\mkvs(\ke',j))} \cup  \RTx(\mkvs(\ke',j)) \), then \( \txid \Tx(\mkvs,\vi) \).
            Otherwise if \( \txid \) is a read only transaction, we add it into \( \T'_\rd \).
        \item  \( \txid \toEdge{\SO_{\aexec'} ; \RW_{\aexec'}?} \txid'' \toEdge{\VIS_{\aexec'}}  \txid_\cl^n \).
            Similarly by \cref{equ:cp-dagger}, either \( \txid \in \Tx(\mkvs,\vi) \)  or we add it into \( \T'_\rd \).
    \end{itemize}
    \end{itemize}
\end{itemize}

\item Since \( \CP \) satisfies \( \RYW \) and \( \MRd \), thus invariants \( I_1 \) and  \( I_2 \) are preserved after update.

\end{itemize}

    
For completeness, we prove the three parts of the execution test separately.
\begin{itemize}
\item Since \( \SO_\aexec \subseteq \VIS_\aexec  \), the prove for \( \ET_\RYW \) is the as in \cref{sec:sound-complete-mr}.
\item For any \( \VIS_\aexec \)  satisfies the constraint for \( \CP \), by \cref{lem:cp-eauiv-spec} it satisfies that 
\[
    \VIS \defeq \left( (\SO \cup \WR ) ; \RW? \cup \WW \cup R \right)^* ; (\SO \cup \WR )
\]
for some relation \( R \).
It means \( \VIS_\aexec ; \SO_\aexec \subseteq \VIS_\aexec \).
Therefore it is complete with respect to \( \ET_\MRd \).

\item Let consider the \( \ddagger \).
Assume i-\emph{th} transaction \( \txid_i \) in the arbitrary order,
and let view \( \vi_{i} = \getView(\aexec, \VIS^{-1}_{\aexec}(\txid_{i}) ) \).
We also pick any final view such that \( \vi'_{i} \subseteq \getView(\aexec, (\AR^{-1}_{\aexec})?(\txid_{i}) ) \).
Note that there is nothing to prove for \( \vi'_i \) since the \( \ddagger \) does not constrain the \( \vi'_i \).
Recall the \( \ddagger \):
\[
    \begin{rclarray}
        \ddagger & \equiv &
        \begin{array}[t]{@{}l@{}}
            \fora{\ke, \ke', i, j, m, \txid, \txid', \txid''} \\
            \left( \begin{array}{@{}l@{}}
            i \in \vi_i(\ke) 
            \land \txid \in \Set{\WTx(\mkvs(\ke,i))} \cup \func{RW^{-1}}{\mkvs, \ke, i} \land {} \\
            \quad \left(
                \begin{array}{@{}l @{}}
                    \left( \begin{array}{@{}l@{}}
                        \txid' \in \func{SO^{-1}}{\txid}
                        \land \txid' \in \Set{\WTx(\mkvs(\ke',j))} \cup  \RTx(\mkvs(\ke',j))
                    \end{array} \right)  \lor {} \\
                    \left( \begin{array}{@{}l@{}}
                            \txid \in \RTx(\mkvs(\ke',j)) \land \txid' = \WTx(\mkvs(\ke',j))
                    \end{array} \right)
                    \end{array} \right) 
                \end{array}
                \right)  \\
                {} \lor \left( \begin{array}{@{}l@{}}
                        i \in \vi(\ke) \land \ke = \ke' \land j < i
                \end{array} \right) \\
                \qquad \implies j \in \vi_i(\ke') 
        \end{array} \\
    \end{rclarray}
\]
Assume \( j \in \vi_i(\ke) \) for some key \(\ke \) and index \( i \).
It means the writer of the version is visible by the transaction \( \txid_i\),
\ie \( \WTx(\mkvs(\ke,i)) \in \VIS^{-1}_{\aexec}(\txid_{i}) \).
Let the \( \mkvs = \mkvs_{\cut(\aexec, i-1)} \).
We need to prove the following:
\begin{gather}
    \label{equ:cp-complete-arvis}
    \begin{rclarray}
        \begin{array}[t]{@{}l@{}}
            \fora{\ke, \ke', i, j, \txid, \txid'} \\
            \left( \begin{array}{@{}l@{}}
            i \in \vi(\ke) 
            \land \WTx(\mkvs(\ke,i)) \in \VIS_\aexec^{-1}(\txid_i) \\
            \quad {} \land \txid \in \Set{\WTx(\mkvs(\ke,i))} \cup \func{RW^{-1}}{\mkvs, \ke, i} \land {} \\
            \quad \left(
                \begin{array}{@{}l @{}}
                    \left( \begin{array}{@{}l@{}}
                            \txid' \in \func{SO^{-1}}{\txid} \land {} \\
                            \txid' \in \Set{\WTx(\mkvs(\ke',j))} \cup  \RTx(\mkvs(\ke',j))
                    \end{array} \right)  \lor {} \\
                    \left( \begin{array}{@{}l@{}}
                            \txid \in \RTx(\mkvs(\ke',j)) \land \txid' = \WTx(\mkvs(\ke',j))
                    \end{array} \right)
                    \end{array} \right) 
                \end{array}
                \right)  \\
                {} \lor \left( \begin{array}{@{}l@{}}
                        i \in \vi(\ke) \land \ke = \ke' \land j < i
                \end{array} \right) \\
                \qquad \implies \WTx(\mkvs(\ke',j)) \in \VIS_\aexec^{-1}(\txid_i)
        \end{array} \\
    \end{rclarray}
\end{gather}
Note that \( \txid \in \Set{\WTx(\mkvs(\ke,i))} \cup \func{RW^{-1}}{\mkvs, \ke, i} \) 
means \( \txid \toEdge{\RW_{\aexec}?} \WTx(\mkvs(\ke,i)) \),
the formulae \(\left( \begin{array}{@{}l@{}} \txid \in \RTx(\mkvs(\ke',j)) \land \txid' = \WTx(\mkvs(\ke',j)) \end{array} \right) \) 
means \( \txid \toEdge{\WR_\aexec} \txid' \),
and \( \left( \begin{array}{@{}l@{}} \txid = \WTx(\mkvs(\ke',m)) \land \txid' = \WTx(\mkvs(\ke',j)) \land m > j \end{array} \right) \) 
means \( \txid \toEdge{\WW_\aexec} \txid' \).
Given all the correspondence, the \cref{equ:cp-complete-arvis} holds if the following holds:
\[
    \begin{rclarray}
        \begin{array}[t]{@{}l@{}}
            \fora{\ke, \ke', i, j, \txid, \txid'} \\
            \left( \begin{array}{@{}l@{}}
            i \in \vi(\ke) 
            \land \WTx(\mkvs(\ke,i)) \in \VIS_\aexec^{-1}(\txid_i) \\
            {} \land \txid' = \WTx(\mkvs(\ke',j))
            \land \txid \toEdge{\RW_{\aexec}?} \WTx(\mkvs(\ke,i)) \land {} \\
            \left(
                \begin{array}{@{}l @{}}
                    \txid' \toEdge{\WR_\aexec ; \SO_\aexec}\txid \lor
                    \txid' \toEdge{\SO_\aexec}\txid \lor
                    \txid' \toEdge{\WR_\aexec}\txid
                    \end{array} \right) 
                \end{array}
                \right)  \\
                {} \lor \txid' \toEdge{\WW_\aexec} \WTx(\mkvs(\ke,i)) \\
                \qquad \implies \txid' \in \VIS_\aexec^{-1}(\txid_i)
        \end{array} \\
    \end{rclarray}
\]
Then the above holds, if the following holds:
\begin{gather}
    \label{equ:cp-complete-arvis-2}
    \begin{rclarray}
        \begin{array}[t]{@{}l@{}}
            \fora{\ke, i, \txid} \\
            \left( \begin{array}{@{}l@{}}
            i \in \vi(\ke) 
            \land \WTx(\mkvs(\ke,i)) \in \VIS_\aexec^{-1}(\txid_i) \\
            {} \land \txid \toEdge{( (\WR_\aexec; \SO_\aexec) \cup \SO_\aexec \cup \WR_\aexec) ; \RW_{\aexec}? \cup \WW_\aexec} \WTx(\mkvs(\ke,i)) 
                \end{array}
                \right)  \\
                \qquad \implies \txid \in \VIS_\aexec^{-1}(\txid_i)
        \end{array} \\
    \end{rclarray}
\end{gather}
By \( \left( (\SO \cup \WR ) ; \RW? \cup \WW \right)^* ; \VIS_\aexec \subseteq \VIS_\aexec \),
It implies \cref{equ:cp-complete-arvis-2} and then \cref{equ:cp-complete-arvis} holds.
\end{itemize}


\subsubsection{Parallel Snapshot Isolation \(\PSI\)}

The axiomatic specification for \( \PSI \) is 
\[ 
    (\RP_{\LWW}, \Set{\lambda \aexec. \VIS_{\aexec} ; \VIS_{\aexec}, \lambda \aexec \ldotp \SO_\aexec, \lambda \aexec. \WW_\aexec })
\]
First the completeness follows \( \CC \) in \cref{sec:sound-complete-cc} and \( \UA \) in \cref{sec:sound-complete-ua}.
Similarly, by \cref{lem:aexec-spec-cc},
there exist a minimum visibility such that 
\[ 
    (\RP_{\LWW}, \Set{\lambda \aexec. \WR_{\aexec} ; \VIS_{\aexec}, \lambda \aexec \ldotp \SO_\aexec, \lambda \aexec. \WW_\aexec })
\]
Given the minimum visibility, the soundness proof follows \( \CC \) in \cref{sec:sound-complete-cc} and \( \UA \) in \cref{sec:sound-complete-ua}.


\subsubsection{Snapshot Isolation \( \SI \)}

The axiomatic specification for \( \SI \) is 
\[ 
(\RP_{\LWW}, \Set{\lambda \aexec. \AR_\aexec ; \VIS_\aexec, \lambda \aexec \ldotp \SO_\aexec, \lambda \aexec \ldotp \WW_\aexec }) 
\]
By a lemma proven in \cite{cerone:snapshot}, for any \( \aexec \) satisfies the \( \SI \)
there exists an equivalent \( \aexec' \) with minimum visibility \( \VIS_{\aexec'} \subseteq \VIS_\aexec \) satisfying 
\[ 
    \left( \RP_{\LWW}, \Set{\lambda \aexec. \left( (\SO_{\aexec'} \cup \WW_{\aexec'} \cup \WR_{\aexec'} ) ; \RW_{\aexec'}? \right)^+ ; \VIS'_{\aexec'}, \\ 
    \quad \lambda \aexec \ldotp (\WW_{\aexec'} \cup \SO_{\aexec'}) } \right) 
\]
Under the minimum visibility \( \VIS \) all the transactions still have the same behaviour as before,
meaning they do not violate last-write-win.

To prove the soundness, we pick the invariant as the following:
\[  
\begin{rclarray}
    I_1(\aexec, \cl) & = & \left( \bigcup\limits_{\{\txid_{\cl}^{i} \in \T_{\aexec} \mid i \in \Nat\}} \VIS_{\aexec}^{-1}(\txid^i_\cl) \right) \setminus \T_\rd \\
    I_2(\aexec, \cl) & = & \left( \bigcup\limits_{\{\txid_{\cl}^{i} \in \T_{\aexec} \mid i \in \Nat\}} (\SO_{\aexec}^{-1})?(\txid^i_\cl) \right) \setminus \T_\rd
\end{rclarray}
\]
where \( \T_\rd \) is all the read-only transactions included in both 
\( \left( \bigcup\limits_{\{\txid_{\cl}^{i} \in \T_{\aexec} \mid i \in \Nat\}} \VIS_{\aexec}^{-1}(\txid^i_\cl) \right)\) 
and \( \left( \bigcup\limits_{\{\txid_{\cl}^{i} \in \T_{\aexec} \mid i \in \Nat\}} (\SO_{\aexec}^{-1})?(\txid^i_\cl) \right) \).
Assume a key-value store $\hh$, an initial and a final view $\vi, \vi'$  a fingerprint $\opset$ 
such that $\ET_{\SI} \vdash (\hh, \vi) \triangleright \opset: \vi'$. 
Also choose an arbitrary $\cl$, a transaction identifier $\txid_\cl^n \in \nextTxId(\hh, \cl)$, 
and an abstract execution $\aexec$ such that $\hh_{\aexec} = \hh$ and 
\( I_1(\aexec, \cl) \cup I_2(\aexec, \cl) \subseteq \Tx(\hh, \vi) \).
Let a new abstract execution \( \aexec' = \extend(\aexec, \txid_\cl^n, \f, \Tx(\mkvs, \vi) \cup \T_\rd) \).
We are about to prove there exists an extra set of read-only transaction \( \T'_\rd \) such that:
\begin{gather}
    \fora{\txid} (\txid, \txid_\cl^n) \in \SO_{\aexec'} \implies \txid \in \Tx(\mkvs, \vi) \cup \T_\rd \cup \T'_\rd \label{equ:si-sound-update-so}\\
    \fora{\txid} (\txid, \txid_\cl^n) \in \WW_{\aexec'} \implies \txid \in \Tx(\mkvs, \vi) \cup \T_\rd \cup \T'_\rd \label{equ:si-sound-update-ww}\\
    \begin{array}{l}
    \fora{\txid} (\txid, \txid_\cl^n) \in \left( (\SO_{\aexec'} \cup \WW_{\aexec'} \cup \WR_{\aexec'} ) ; \RW_{\aexec'}? \right)^+ ; \VIS_{\aexec'} \\
    \qqquad \implies \txid \in \Tx(\mkvs, \vi) \cup \T_\rd \cup \T'_\rd 
    \end{array}
    \label{equ:si-sound-update-arvis}\\
    I_1(\aexec',\cl) \cup I_2(\aexec',\cl) \subseteq \Tx(\mkvs_{\aexec'}, \vi') \label{equ:si-sound-inv} 
\end{gather}
\begin{itemize}
\item The invariant \( I_2 \) implies \cref{equ:si-sound-update-so} as the same as \( \RYW \) in \cref{sec:sound-complete-ryw}.
\item Since \( \SI \) also satisfies \( \UA \), the \cref{equ:si-sound-update-ww} can be proven as the same as \( \UA \) in \cref{sec:sound-complete-ua}.
\item Note that for \cref{equ:si-sound-update-arvis}, it is sufficient to prove one step instead of transitive.
That is,
\[
    \begin{array}{l}
    \fora{\txid} (\txid, \txid_\cl^n) \in \left( (\SO_{\aexec'} \cup \WW_{\aexec'} \cup \WR_{\aexec'} ) ; \RW_{\aexec'}? \right) ; \VIS_{\aexec'} \\
    \qqquad \implies \txid \in \Tx(\mkvs, \vi) \cup \T_\rd \cup \T'_\rd 
    \end{array}
\]
Let \( \T'_\rd \) initially be empty set.
More read-only transactions will be added in until \cref{equ:si-sound-update-arvis} holds.
Assume a transaction \( \txid \) such that 
\[
    (\txid, \txid_\cl^n) \in \left( (\SO_{\aexec'} \cup \WW_{\aexec'} \cup \WR_{\aexec'} ) ; \RW_{\aexec'}? \right) ; \VIS_{\aexec'} 
\]
It means \( (\txid \toEdge{\left( (\SO_{\aexec'} \cup \WW_{\aexec'} \cup \WR_{\aexec'} ) ; \RW_{\aexec'}? \right) } \txid' \toEdge{\VIS_{\aexec'}}  \txid_\cl^n \) for some transaction \( \txid' \).
There are two cases: \( \txid' \) writes to at least one key; or \( \txid' \) is read-only.
\begin{itemize}
\item \( \txid' \) writes to at least one key.
Assume that the transaction \( \txid' \) writes to a key \( \ke \).
Recall the \( \dagger \) (and auxiliary relation under key-value store \( \mkvs\)) is defined as the following:
\[
    \begin{rclarray}
        \func{RW^{-1}}{\mkvs, \ke, i} & \defeq & \Setcon{\txid}{\exsts{ j \leq i } \txid \in \RTx(\mkvs(\ke,j))} \\
        \dagger & \equiv &
        \begin{array}[t]{@{}l@{}}
            \fora{\ke, \ke', i, j, m, \txid, \txid', \txid''} \\
            \begin{array}{@{}l@{}}
            i \in \vi(\ke) 
            \land \txid \in \Set{\WTx(\mkvs(\ke,i))} \cup \func{RW^{-1}}{\mkvs, \ke, i} \land {} \\
            \quad \left(
                \begin{array}{@{}l @{}}
                    \left( \begin{array}{@{}l@{}}
                            \txid' \in \func{SO^{-1}}{\txid} \land {} \\
                            \txid' \in \Set{\WTx(\mkvs(\ke',j))} \cup  \RTx(\mkvs(\ke',j))
                    \end{array} \right)  \lor {} \\
                    \left( \begin{array}{@{}l@{}}
                        \txid \in \RTx(\mkvs(\ke',j)) 
                        \land \txid' = \WTx(\mkvs(\ke',j))
                    \end{array} \right) \lor {} \\ 
                    \left( \begin{array}{@{}l@{}}
                            \txid = \WTx(\mkvs(\ke',m)) \land {} \\
                            \txid' = \WTx(\mkvs(\ke',j)) \land m > j
                    \end{array} \right) 
                \end{array}
                \right)  \\
            \qquad \implies j \in \vi(\ke') 
            \end{array}
        \end{array} \\
    \end{rclarray}
\]
We link the conditions in \( \ddagger \) to relation:
\begin{itemize}
    \item \( \RW_\aexec\). Assume a key \( \ke \),  an index \( i \) and the writer \( \txid  = \WTx(\mkvs(\ke,i))\),
then \( \txid' \in \RW^{-1}(\mkvs, \ke, i)\) if and only if \( \txid' \toEdge{\RW_\aexec} \txid\).
    \item \( \SO_\aexec\). The transaction identifiers encode the \( \SO_\aexec \).
    That is, \( \txid' \in \SO^{-1}(\txid)\) if and only if \(\txid' \toEdge{\SO_\aexec} \txid \).
    \item  \( \WR_\aexec \). It is easy to see \( \txid \in \RTx(\mkvs(\ke',j)) \land \txid' = \WTx(\mkvs(\ke',j)) \) if and only if \( \txid' \toEdge{\WR_\aexec} \txid \).
    \item \( \WW_\aexec \). The write-write relation describes the order of write operations for a key which corresponds the version orders in key-value store.
    That is, \( \txid' = \WTx(\mkvs(\ke,j)) \land \txid = \WTx(\mkvs(\ke,i)) \land j < i\) if and only if
    \( \txid' \toEdge{\WW_\aexec} \txid\).
\end{itemize}
Let assume \( \txid' \) writes to i-\emph{th} version a key \( \ke \).
Given above and 
\[ (\txid \toEdge{\left( (\SO_{\aexec'} \cup \WW_{\aexec'} \cup \WR_{\aexec'} ) ; \RW_{\aexec'}? \right) } \txid' \toEdge{\VIS_{\aexec'}}  \txid_\cl^n \] we can substitute and rewrite the \( \dagger \) as the following:
\begin{gather}
    \begin{array}{@{}l@{}}
        \fora{\txid'',\ke',j}
        \WTx(\mkvs(\ke,i)) = \txid' \land {} \\
        \left( \begin{array}{@{}l@{}}
        \txid'' \toEdge{\RW_{\aexec'}?} \txid' \land {} \\
        \quad \left(
            \begin{array}{@{}l @{}}
                \left( \begin{array}{@{}l@{}}
                        \txid \toEdge{\SO_{\aexec'}} \txid'' \land 
                        \txid \in \Set{\WTx(\mkvs(\ke',j))} \cup  \RTx(\mkvs(\ke',j))
                \end{array} \right)  \\
                {} \lor 
                \left( \begin{array}{@{}l@{}}
                        \txid \toEdge{\WR_{\aexec'}} \txid'' \land \txid = \WTx(\mkvs(\ke',j))
                \end{array} \right) \\
                {} \lor \left( \begin{array}{@{}l@{}}
                        \txid \toEdge{\WW_{\aexec'}} \txid'' \land \txid = \WTx(\mkvs(\ke',j))
                \end{array} \right) 
                \end{array} \right) 
            \end{array}
            \right)  \\
            \qquad \implies j \in \vi(\ke') 
    \end{array} 
    \label{equ:si-dagger}
\end{gather}
Now we perform case analysis if \( \txid \) is a read-only transaction.
\begin{itemize}
    \item if \( \txid \) has write, by the \cref{equ:si-dagger} that \( j \in \vi(\ke')\) for all version \( \mkvs(\ke', j) \) written by the \( \txid \).
        Therefore by the definition of \( \Tx \), then \( \txid \in \VIS^{-1}(\txid_\cl^n)\).
    \item if \( \txid \) is a read-only transaction, we add it into \( \T'_\rd \).
\end{itemize}

\item If \( \txid' \) is a read-only transaction, we know \( \txid' \in \T_\rd \) or \( \txid' \in \T'_\rd \).
    More specifically we have three cases: \textbf{(i)} \( \txid' \in \bigcup\limits_{\{\txid_{\cl}^{i} \in \T_{\aexec} \mid i \in \Nat\}} \VIS_{\aexec}^{-1}(\txid^i_\cl) \), \textbf{(ii)} \( \txid' \in \bigcup\limits_{\{\txid_{\cl}^{i} \in \T_{\aexec} \mid i \in \Nat\}} (\SO_{\aexec}^{-1})?(\txid^i_\cl) \) or \textbf{(iii)} \( \txid' \in \T'_\rd\).
    \begin{itemize}
    \item Assume \( \txid' \in \bigcup\limits_{\{\txid_{\cl}^{i} \in \T_{\aexec} \mid i \in \Nat\}} \VIS_{\aexec}^{-1}(\txid^i_\cl) \).
        There exist a previous transaction from the same client \( \txid_\cl^m\) such that \( m < n \) and 
        \[ 
            \txid \toEdge{\left( (\SO_{\aexec'} \cup \WW_{\aexec'} \cup \WR_{\aexec'} ) ; \RW_{\aexec'}? \right) } \txid' \toEdge{\VIS_{\aexec'}}  \txid_\cl^m 
        \]
        Since those edges already exists in the abstract execution \( \aexec\), and by the constraints of \( \aexec \) we have \( \txid \toEdge{\VIS_{\aexec'}} \txid_\cl^m\).
        Therefore by the invariant \( I_1 \), either \( \txid \in \Tx(\mkvs, \vi) \cup \T_\rd \).
    \item Assume \( \txid' \in \bigcup\limits_{\{\txid_{\cl}^{i} \in \T_{\aexec} \mid i \in \Nat\}} (\SO_{\aexec}^{-1})?(\txid^i_\cl) \).
        Note that  \( \txid' \) is a read-only transaction so we can simplify the edge,
        \[ 
            \txid \toEdge{\left( \SO_{\aexec'} \cup \WW_{\aexec'} \cup \WR_{\aexec'}  \right)} \txid_\cl^m \toEdge{\SO_{\aexec'}}  \txid_\cl^n 
        \] 
        for some \( m \) such that \( m < n \).
        The path from \( \txid \) to \( \txid_\cl^m \) must exist in the abstract execution before update and they satisfy the constraint, so \( \txid \toEdge{\VIS_\aexec} \txid_\cl^m \).
        Therefore by the invariant \( I_1 \), we have \( \txid \in \Tx(\mkvs, \vi) \cup \T_\rd \).
    \item \( \txid' \in \T'_\rd \). 
        Given that \( \T'_\rd \) initially is empty, there is anther transaction \( \txid'' \) that has at least a write such that:
        \[
        \begin{array}{l}
            \txid \toEdge{\left( (\SO_{\aexec'} \cup \WW_{\aexec'} \cup \WR_{\aexec'} ) ; \RW_{\aexec'}? \right) } \\
            \qquad \txid' \toEdge{\left( (\SO_{\aexec'} \cup \WW_{\aexec'} \cup \WR_{\aexec'} ) ; \RW_{\aexec'}? \right) } \txid'' \toEdge{\VIS_{\aexec'}}  \txid_\cl^n 
        \end{array}
        \]
        By the fact that \( \txid' \) is read-only and \( \txid'' \) has write, it follows:
        \[
            \txid \toEdge{\left( \SO_{\aexec'} \cup \WR_{\aexec'} \right) } \txid' \toEdge{\left( \SO_{\aexec'} ; \RW_{\aexec'}? \right) } \txid'' \toEdge{\VIS_{\aexec'}}  \txid_\cl^n 
        \]
        Since \( \SO \) is transitive, we have the following two cases:
        \[
            \begin{array}{@{}c@{}}
                \txid \toEdge{\WR_{\aexec'} } \txid' \toEdge{\SO_{\aexec'} ; \RW_{\aexec'}? } \txid'' \toEdge{\VIS_{\aexec'}}  \txid_\cl^n  \\
                \txid \toEdge{\SO_{\aexec'} ; \RW_{\aexec'}? } \txid'' \toEdge{\VIS_{\aexec'}}  \txid_\cl^n 
            \end{array}
        \]
        For both cases, by \( \dagger \), if \( \txid \) has write then \( \txid \in \Tx(\mkvs, \vi) \).
        Otherwise we add \( \txid \) into \(\T'_\rd \).
    \end{itemize}
\end{itemize}
\item Since \( \SI \) satisfies \( \RYW \) and \( \MRd \), thus invariants \( I_1 \) and  \( I_2 \) are preserved, that is, \cref{equ:si-sound-inv}.
\end{itemize}

To prove completeness, we prove four parts of the execution test separately.
\begin{itemize}
\item It is easy to see it is complete with respect to \( \UA \) and \( \RYW \) as \( \WW \cup \SO \subseteq \VIS \).
    The details are the same as proofs for \( \UA \) in \cref{sec:sound-complete-ua} and \( \RYW \) in \cref{sec:sound-complete-ryw}.

\item By the \cite{cerone:snapshot}, we know that for any abstract execution that satisfies \( (\WW_\aexec \cup \SO_\aexec) \subseteq \VIS_\aexec \)
and \( \left( (\SO_\aexec \cup \WW_\aexec \cup \WR_\aexec ) ; \RW_{\aexec}? \right) ; \VIS_\aexec \subseteq \VIS_\aexec \),
the visibility relation will have have the following form for some relation \( R \):
\[
    \VIS_\aexec = \left( \left( (\SO_\aexec \cup \WW_\aexec \cup \WR_\aexec ) ; \RW_{\aexec}? \right)  \cup R \right)^* ; (\SO_\aexec \cup \WR_\aexec \cup \WW_\aexec) 
\]
Given above, it is easy to see \( \VIS_\aexec ; \SO_\aexec \subseteq \VIS_\aexec \) since \( \SO \) is transitive, 
thus we have the proof for \( \MRd \) where the detail is the same same \( \MRd \) in \cref{sec:sound-complete-mr}.

\item Let consider \( \dagger \).
Assume i-\emph{th} transaction \( \txid_i \) in the arbitrary order,
and let view \( \vi_{i} = \getView(\aexec, \VIS^{-1}_{\aexec}(\txid_{i}) ) \).
We also pick any final view such that \( \vi'_{i} \subseteq \getView(\aexec, (\AR^{-1}_{\aexec})?(\txid_{i}) ) \).
Note that there is nothing to prove for \( \vi'_i \) since the \( \dagger \) does not constrain the \( \vi'_i \).
Let the \( \mkvs = \mkvs_{\cut(\aexec, i-1)} \).
Now we need to prove the following:
\begin{gather}
    \label{equ:si-complete-arvis}
    \begin{array}{@{}l@{}}
        \fora{\ke, \ke', i, j, m, \txid, \txid', \txid''} \\
        \begin{array}{@{}l@{}}
        i \in \vi(\ke) 
        \land \WTx(\mkvs(\ke,i)) \in \VIS^{-1}_{\aexec}(\txid_{i})  \\
        \quad {} \land \txid \in \Set{\WTx(\mkvs(\ke,i))} \cup \func{RW^{-1}}{\mkvs, \ke, i} \land {} \\
            \quad \left(
                \begin{array}{@{}l @{}}
                    \left( \begin{array}{@{}l@{}}
                        \txid' \in \func{SO^{-1}}{\txid}
                        \land \txid' \in \Set{\WTx(\mkvs(\ke',j))} \cup  \RTx(\mkvs(\ke',j))
                    \end{array} \right)  \lor {} \\
                    \left( \begin{array}{@{}l@{}}
                        \txid \in \RTx(\mkvs(\ke',j)) 
                    \end{array} \right) \lor 
                    \left( \begin{array}{@{}l@{}}
                        \txid = \WTx(\mkvs(\ke',m)) 
                        \land m > j
                    \end{array} \right) 
                \end{array}
                \right)  \\
        \qquad \implies \WTx(\mkvs(\ke',j)) \in \VIS^{-1}_{\aexec}(\txid_{i})
        \end{array} 
    \end{array} 
\end{gather}
Note that \( \txid \in \Set{\WTx(\mkvs(\ke,i))} \cup \func{RW^{-1}}{\mkvs, \ke, i} \) 
means \( \txid \toEdge{\RW_{\aexec}?} \WTx(\mkvs(\ke,i)) \),
the formulae \(\left( \begin{array}{@{}l@{}} \txid \in \RTx(\mkvs(\ke',j)) \land \txid' = \WTx(\mkvs(\ke',j)) \end{array} \right) \) 
means \( \txid \toEdge{\WR_\aexec} \txid' \),
and \( \left( \begin{array}{@{}l@{}} \txid = \WTx(\mkvs(\ke',m)) \land \txid' = \WTx(\mkvs(\ke',j)) \land m > j \end{array} \right) \) 
means \( \txid \toEdge{\WW_\aexec} \txid' \).
So \cref{equ:si-complete-arvis} holds if the following holds:
\[
    \begin{array}{@{}l@{}}
        \fora{\ke, \ke', i, j, \txid, \txid'} \\
        \left( \begin{array}{@{}l@{}}
        i \in \vi(\ke) 
        \land \WTx(\mkvs(\ke,i)) \in \VIS^{-1}_{\aexec}(\txid_{i}) 
        \land \txid' = \WTx(\mkvs(\ke',j)) \\
        \quad {} \land \txid \toEdge{\RW_{\aexec}?} \WTx(\mkvs(\ke,i))  \land {} \\
        \quad \left(
        \begin{array}{@{}l @{}}
            \txid' \toEdge{\SO_\aexec} \txid \lor
            \txid' \toEdge{\WR_\aexec ; \SO_\aexec} \txid \lor
            \txid' \toEdge{\WR_\aexec} \txid \lor
            \txid' \toEdge{\WW_\aexec} \txid 
        \end{array}
        \right)  \\
        \qqquad \implies \txid' \in \VIS^{-1}_{\aexec}(\txid_{i})
        \end{array} \right)
    \end{array} \\
\]
The above holds, if the following holds:
\begin{gather}
    \label{equ:si-complete-arvis-2}
    \begin{array}{@{}l@{}}
        \fora{\ke, i, \txid} \\
        \left( \begin{array}{@{}l@{}}
        i \in \vi(\ke) 
        \land \WTx(\mkvs(\ke,i)) \in \VIS^{-1}_{\aexec}(\txid_{i})  \\
        {} \land \txid \toEdge{(\SO_\aexec \cup (\WR_\aexec ; \SO_\aexec ) \cup \WR_\aexec \cup \WW_\aexec ) ; \RW_{\aexec}?} \WTx(\mkvs(\ke,i))  \\
        \quad \implies \txid \in \VIS^{-1}_{\aexec}(\txid_{i})
        \end{array} \right)
    \end{array} 
\end{gather}
Note that the constraint for the \( \aexec \):
\[ 
    \left( (\SO_{\aexec} \cup \WW_{\aexec} \cup \WR_{\aexec} ) ; \RW_{\aexec}? \right)^* ; \VIS_{\aexec}  \subseteq \VIS_\aexec
\] 
It implies the \cref{equ:si-complete-arvis-2}.
Therefore we have the proof for \cref{equ:si-complete-arvis}.

\end{itemize}


\subsubsection{Serialisability \( \SER \)}

The execution test $\ET_\UA$ is sound with respect to the axiomatic specification 
\[ 
    (\RP_{\LWW}, \Set{\lambda \aexec. \AR })
\]
We pick the invariant as \( I( \aexec, \cl ) = \emptyset \), given the fact of no constraint on the view after update.
Assume a key-value store $\hh$, an initial and a final view $\vi, \vi'$  a fingerprint $\opset$ 
such that $\ET_{\SER} \vdash (\hh, \vi) \triangleright \opset: \vi'$. 
Also choose an arbitrary $\cl$, a transaction identifier $\txid \in \nextTxId(\hh, \cl)$, 
and an abstract execution $\aexec$ such that $\hh_{\aexec} = \hh$ and 
\( I(\aexec, \cl) =  \emptyset \subseteq \Tx(\hh, \vi) \).
Let \( \aexec' = \extend(\aexec, \txid, \Tx(\mkvs, \vi), \f ) \).
Note that since the invariant is empty set, it remains to prove there exists a set of read-only transactions \( \T_\rd \) such that:
\[
    \begin{array}{@{}l@{}}
        \fora{ \txid' } 
        \txid' \toEdge{\AR_{\aexec'}} \txid \implies \txid' \in \Tx(\mkvs, \vi) \cup \T_\rd
    \end{array}
\]
Since the abstract execution satisfies the constraint for \( \SER \), \ie \( \AR \subseteq \VIS \), we know \( \AR = \VIS \).
Since \( \Tx(\mkvs, \vi)  \) contains all transactions that write at least a key, 
we can pick a \( \T_\rd \) such that \( \Tx(\mkvs, \vi) \cup \T_\rd = \T_\aexec\),
which gives us the proof.


The execution test $\ET_\UA$ is complete with respect to the axiomatic specification \( (\RP_{\LWW}, \Set{\lambda \aexec. \AR_\aexec }) \).
Assume i-\emph{th} transaction \( \txid_i \) in the arbitrary order,
and let view \( \vi_{i} = \getView(\aexec, \VIS^{-1}_{\aexec}(\txid_{i}) ) \).
We also pick any final view such that \( \vi'_{i} \subseteq \getView(\aexec, (\AR^{-1}_{\aexec})?(\txid_{i}) ) \).
Note that there is nothing to prove for \( \vi'_i \),
Now we need to prove the following:
\[
    \fora{\ke, j}  0 \leq j < \abs{\mkvs_{\cut(\aexec, i-1)}(\ke)} \implies j \in \vi_i(\ke)
\]
Because \( \VIS^{-1}(\txid_i) = \AR^{-1}(\txid_i) = \Setcon{\txid }{\txid \texttt{ appears in } \mkvs_{\cut(\aexec, i-1)} }\),
so for any key \( \ke \) and index \( j \) such that \( 0 \leq j < \abs{\mkvs_{\cut(\aexec, i-1)}(\ke)} \),
the j-\emph{th} version of the key contains in the view, \ie \( j \in \vi(\ke)\).

