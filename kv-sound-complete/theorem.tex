\subsection{Comparing the Abstract Execution and key-value Store Semantics}

Suppose that a given execution test $\ET$ captures precisely 
a consistency model specified in the axiomatic style, using a set of 
axioms $\Ax$ and a resolution policy $\RP$ over abstract executions.
That is, for any abstract execution $\aexec$ that satisfies 
the axioms $\Ax$ and the resolution policy $\RP$, then $\KVtrace(\ET_{\top}, \aexec) \cap \CMs(\ET) \neq \emptyset$; 
and for any $\tr \in \CMs(\ET)$, there exists an abstract execution 
$\aexec \in \aeset(\tr)$ that satisfies the axioms $\Ax$ and the resolution policy $\RP$. 
\ac{In practice, the functions $\KVtrace(\ET, \aexec) = \KVtrace(\ET_{\top}, \aexec) \cap \CMs(\ET)$, 
and $\aeset(\Ax, \aexec) = \aeset(\ET_{\top}, \aexec) \cap \aeset(\Ax)$, abstracted over the second 
argument and lifted to sets of abstract executions/traces, 
define a Galois Connection between the powerset of abstract executions in the axiomatic specification 
of the consistency model, and the powerset of kv-stores generated by such a consistency model.}
Our main aim in this section consists in proving that, for each program $\prog$, the 
set of kv-stores generated by $\prog$ under $\ET$ corresponds to all the possible kv-stores 
that could be obtained by running $\prog$ on a database that satisfies the axiomatic specification 
$\Ax$. In this sense, we aim to establish that our operational semantics is \emph{adequate}.

To tackle this question, we need to define what is the set of all possible behaviours 
that can be produced by a program $\prog$ under a given consistency model $\CM$, for 
which an axiomatic specification $(\RP, \Ax)$ is known. This in turn requires addressing two orthogonal 
problems: \textbf{(i)} defining the set of all possible behaviours that may be exhibited by a program 
$\prog$, independently of the consistency model; and \textbf{(ii)} defining the set of all possible 
behaviours that are allowed by a given consistency model $\CM$. Then the set of all 
possible behaviours of $\prog$ under $\CM$ is obtained by intersecting the two sets 
above.

The kv-store semantics is intrinsically not expressive enough to tackle problem \textbf{(i)}. 
By \cref{cor:kvtrace2aexec}, only kv-stores arising 
from abstract executions satisfying the last write wins resolution policy can be captured in the kv-store 
framework; instead, we seek to model all the behaviours of a program independently of a consistency 
models, and therefore independently of a resolution policy. 

\begin{proposition}
\label{prop:kv2aexec_transition}
Suppose that $(\hh, \viewFun, \thdenv), \prog \toT{(\cl, \vi, \opset)}_{\ET_{\top}} (\hh', 
\viewFun', \thdenv'), \prog'$. Let $\aexec$ be an abstract execution 
such that $\hh_{\aexec} = \hh$, and let $\T \subseteq \T_{\aexec}$ be a 
set of read-only transactions. Then there exists an abstract execution $\aexec'$ 
such that $\hh_{\aexec'} = \hh'$, and 
\[
(\aexec, \thdenv), \prog \toA{(\cl, \T \cup \Tx(\hh, \vi), \opset)}_{(\RP_{\LWW}, \emptyset)}. 
(\aexec', \thdenv'), \prog'
\]
\end{proposition}
\begin{proof}
Suppose that $(\hh, \viewFun, \thdenv), \prog \toT{(\cl, \vi, \opset)}_{\ET_{\top}} (\hh', \viewFun', \thdenv), \prog'$. 
This transition can only be inferred by applying Rule \rl{PSingleThread}, meaning that 
\begin{itemize}
\item $\prog(\cl) \mapsto \cmd$ for some command $\cmd$, 
\item $\cl \vdash (\hh, \viewFun(\cl), \thdenv(\cl)), \cmd \toT{(\cl, \vi, \opset)}_{\ET_{\top}} (\hh', \vi', \stk'), \cmd'$ 
for some $\vi', \stk'$, and 
\item $\viewFun' = \viewFun\rmto{\cl}{\vi'}$, $\thdenv' = \thdenv\rmto{\cl}{\stk'}$, $\prog' = \prog\rmto{\cl}{\cmd'}$. 
\end{itemize}
Let now $\aexec$ be such that $\hh_{\aexec} = \hh$, and let $\T \subseteq \T_{\aexec}$ be a set of read-only 
transactions in $\aexec$. It suffices to show that there exists an abstract execution $\aexec'$ such that 
$\hh_{\aexec'} = \hh'$, and 
\[
\cl \vdash (\aexec, \thdenv(\cl)), \cmd \toA{(\cl, \T \cup \Tx(\hh, \vi), \opset}_{(\RP_{\LWW}, \emptyset)} (\aexec', \stk'), \cmd'.
\]
Then, by applying Rule \rl{ASingleThread}, we obtain 
\[ 
(\aexec, \thdenv) \prog \toA{(\cl, \T \cup \Tx(\hh,\vi), \opset)}_{\RP_{\LWW}, \emptyset)} (\aexec' ,\thdenv'), \prog'.
\]

We perform a rule induction on the derivation of the transition $\cl \vdash (\hh, \viewFun(\cl), \thdenv(\cl)), \cmd \toT{(\cl, \vi, \opset)}_{\ET_{\top}} (\hh', \vi', \h'), \cmd'$. 
The base case corresponds to such a transition being inferred by applying Rule \rl{PCommit}. 
This implies that 
\begin{itemize}
\item $\cmd = \ptrans{\trans}$ for some $\trans$, and $\cmd' = \pskip$,
\item $\viewFun(\cl) \viewleq \vi$, 
\item let $\h = \snapshot(\hh, \vi)$; then $(\thdenv(\cl), \h, \emptyset) \toL^{\ast} (\stk', \_, \opset)$, 
\item $hh' = \updateKV(\hh, \vi, \txid, \opset)$ for some $\txid \in \nextTxId(\hh, \cl)$, 
\item $\ET_{\top} \vdash \hh, \vi \triangleright \opset: \vi'$.
\end{itemize}

Choose an arbitrary set of of read-only transactions $\T \subseteq \T_{\aexec}$: 
we observe that, since $\hh_{\aexec} = \hh$, then  by \cref{prop:getview.tx} $\getView(\aexec, \T \cup \Tx(\hh, \vi)) = \vi$. 
We can now apply \cref{prop:compatible.aexec2kv} and ensure that $\RP_{\LWW}(\aexec, \T \cup \Tx(\hh, \vi)) = \{\h\}$.
Let $\aexec' = \extend(\aexec, \txid, \T \cup \Tx(\hh, \vi)), \opset)$; because $\getView(\aexec, \T \cup \Tx(\hh, \vi)) = \vi$, 
$\hh_{\aexec} = \hh$,
then by \cref{prop:extend.update.sameop} we have that $\hh_{\aexec'} = \updateKV(\hh, \vi, \txid, \opset) = \hh'$. 
We  have that $\T \cup \Tx(\hh, \vi) \subseteq \T_{\aexec}$, $\h \in \RP_{\LWW}(\aexec, \T \cup \Tx(\hh, \vi))$,
$(\thdenv(\cl), \h, \emptyset) \toL^{\ast} (\stk', \_, \opset)$ and $\txid \in \nextTxId(\T_{\aexec}, \cl)$. 
Now we can apply Rule \rl{ACommit} and infer
\[
\cl \vdash (\aexec, \thdenv(\cl)), \ptrans{\trans} \toA{(\cl, \T \cup \Tx(\hh_{\aexec}, \vi))}_{(\RP_{\LWW}, \emptyset)} 
(\aexec', \stk'), \pskip,
\]
which is exactly what we wanted to prove. 
\end{proof}

\begin{proposition}
\label{prop:aexec2kv_transition}
Suppose that $(\aexec, \thdenv), \prog \toA{(\cl, \T, \opset)}_{(\RP_{\LWW})} (\aexec', \thdenv'), \prog'$. 
Then for any $\viewFun$ and $\vi \in \Views(\hh_{\aexec})$ such that $\vi \viewleq \getView(\aexec, \T)$, 
we have that 
\[
(\hh_{\aexec}, \viewFun\rmto{\cl}{\vi}, \thdenv), \prog \toA{(\cl, \getView(\aexec, \T), \opset)}_{\ET_{\top}} (\hh_{\aexec'}, \viewFun, \thdenv'), \prog'.
\]
\end{proposition}
\begin{proof}
Suppose that $(\aexec, \thdenv), \prog \toA{(\cl, \T, \opset)}_{(\RP_{\LWW}, \emptyset)} (\aexec', \thdenv'), \prog'$. 
Fix a function $\viewFun$ from clients in $\dom(\prog)$ to views in $\Views(\hh)$, and a view $\vi \viewleq \getView(\aexec, \T)$.
We show that 
$(\hh_{\aexec}, \viewFun\rmto{\cl}{\vi}, \thdenv) \toT{(\cl, \getView(\aexec, \T), \opset)}_{\ET_{\top}} (\hh_{\aexec'}, 
\viewFun, \thdenv'), \prog'$. 

Note that the transition $(\aexec, \thdenv, \prog \toA{(\cl, \T, \opset)}_{\RP_{\LWW}, \emptyset)} (\aexec', \thdenv'), \prog'$ 
can only be inferred using Rule \rl{ASingleThread}, from which it follows that $\cl \vdash (\aexec, \thdenv(\cl)), \prog(\cl) 
\toA{(\cl, \T, \opset)}_{\RP_{\LWW}, \emptyset)}, (\aexec' ,\stk') \cmd'$ for some $\stk'$ such that $\thdenv' = 
\thdenv\rmto{\cl}{\stk'}$ and $\cmd'$ such that $\prog' = \prog\rmto{\cl}{\cmd'}$.
It suffices to show that $\cl \vdash (\hh_{\aexec}), \vi, \thdenv(\cl), \prog(\cl) \toT{(\cl, \getView(\hh_{\aexec}, \T), \opset)}_{\ET_{\top}} 
(\hh_{\aexec'}, \viewFun(\cl), \stk'), \cmd'$. Then by applying Rule \rl{PSingleThread} we obtain 
\[
(\hh_{\aexec}, \viewFun\rmto{\cl}{\vi}, \thdenv), \prog \toT{(\cl, \getView(\hh_{\aexec}, \T), \opset)}_{\ET_{\top}} 
(\hh_{\aexec'}, \viewFun, \thdenv'), \prog'.
\]

The rest of the proof is performed by a rule induction on the derivation used to infer 
the transition $\cl \vdash (\aexec, \thdenv(\cl)), \prog(\cl) \toA{(\cl, \T, \opset)}_{(\RP_{\LWW}, \emptyset)} (\aexec', \stk'), \cmd'$. 
We only consider the most important case, namely the one in which the derivation above 
has been inferred using Rule \rl{ACommit}. In this case we have that 
$\prog = \ptrans{\trans}$, $\prog' = \pskip$, there exists an index $\h \in \RP_{\LWW}(\aexec, \T)$ such that 
$(\thdenv(\cl), \h, \emptyset), \trans \toL^{\ast} (\stk', \_, \opset), \pskip$, and $\aexec' = 
\extend(\aexec, \txid, \T, \opset)$ for some $\txid \in \nextTxId(\aexec, \cl)$. 
Furthermore, it is possible to prove, by induction on the length of the derivation $(\thdenv(\cl), \h, \emptyset), \trans \toL^{\ast} (\stk', \_, \opset), \pskip$, 
that whenever $(\otR, \ke, \val) \in \opset$, then $\h(\ke) = \val$.
Note that by \cref{prop:compatible.aexec2kv} we have that $\snapshot(\hh_{\aexec}, \getView(\aexec, \T)) = 
\h$. Also, if $(\otR,\ke, \val) \in \opset$ then $\h(\ke) = \val$, which is possible only if  
$\hh_{\aexec}(\ke, \max_{<}(\getView(\aexec, \T)(\ke))) = (\val, \_, \_)$. This ensures 
that $\ET_{\top} \vdash (\hh_{\aexec}, \getView(\aexec, \T)) \triangleright \opset: \viewFun(\cl)$. 
\ac{There should be a condition here that $\viewFun(\cl)(\ke)$ is the same as $\getView(\aexec, \T)(\ke)$ 
for any $\ke$ that is neither read nor written by $\opset$.} 
We can now combine all the facts above to apply rule \rl{PCommit} and show that 
\[
\cl \vdash (\hh_{\aexec}, \vi, \thdenv(\cl)), \ptrans{\trans} \toT{(\cl, \getView(\hh_{\aexec}, \T), \opset}_{\ET_{\top}} 
(\hh', \viewFun(\cl), \stk', \pskip, 
\] 
where $\hh' = \updateKV(\hh_{\aexec}, \txid, \getView(\aexec, \T), \opset)$. 
Recall that $\aexec' = \extend(\aexec, \T, \txid, \opset)$. 
By \cref{prop:extend.update.sameop} we have that $\hh' = \hh_{\aexec'}$, 
which concludes the proof.
\end{proof}

\begin{corollary}
For any program $\prog$, 
\[
\interpr{\prog}_{\ET_{\top}} = \{\hh_{\aexec} \mid \aexec \in \interpr{\prog}_{(\RP_{\LWW}, \emptyset)}\}
\]
\end{corollary}
\begin{proof}
    It can be derived by \cref{prop:aexec2kv_transition} and \cref{prop:kv2aexec_transition}.
\end{proof}


