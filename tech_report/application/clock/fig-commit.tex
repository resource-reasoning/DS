\begin{figure}

\begin{subfigure}{\textwidth}
\begin{lstlisting}
commitTrans( trans )
    prep = [];
    snapshotTime = trans.snapshotTime;
    for (w,k,v) in trans.read-write-set { 
        send ``prepare (k,v,snapshotTime)'' to shardOf(k); (* \label{line:asyn-send-prepare}*)
        wait ``(k,v,t)''; (* \label{line:asyn-wait-prepare}*)
        prep = prep + (k,v,t); (* \label{line:clock-collect-reply}*)
    }

    commitTime = maxTime(prep) (* \label{line:clock-final-commit-time}*)

    for (k,v,t) in prep {
        send ``commit (k,v,commitTime)'' to shardOf(k);
    }

On receive ``prepare (k,v,snapshotTime)'' 
    wait snapshotTime < getShardClockTime(); 
    if noConcurrentWriteSince(k,snapshotTime) { (* \label{line:clock-no-conflict-write}*)
        preparedTime = getShardClockTime(); (* \label{line:clock-assign-prepare-time}*)
        log ``(k,v,preparedTime,prepared)'' in preparation set; (* \label{line:clock-log-prepare-time}*)
        send ``(k,v,preparedTime)''; (* \label{line:clock-echo-prepare-time}*)
    } else {
        // either abort or restart this transaction entirely.
        ...
    }

On receive ``commit (k,v,commitTime)'' 
    insert(k,v,commitTime,committed); (* \label{line:clock-insert-version}*)
\end{lstlisting}

\caption{The reference implementation of transaction commit}
\label{lst:clock-si-commit}
\end{subfigure}

\hrulefill

\begin{subfigure}{\textwidth}
\begin{mathpar}
\inferrule[\rCLOCKPrepare]{ 
    \opW(\key,\val) \in \fp
    \\
    \clockshard = \ShardOf(\clocksi,\key)
    \\
    (\clockkvs,\clocktime'') = \clocksi(\clockshard)
    \\
    \clocktime' < \clocktime''
    \\
    \clockverlist = \clockkvs(\key)
    \\
    \Forall{\idx} 0 \leq \idx < \Abs{\clockverlist} \implies \TimeOf(\clockverlist(\idx)) < \clocktime'
    \\
    \clocksi' = \clocksi\UpdateFunc{\clockshard -> (\clockverlist \ListConcat \List{(\val,\clocktime'',\clocksiprepared)}, \clocktime'') }
}{%
    \ToClockCmd{
    \clocksi | \clocktime | \stk
            | \pruntrans{\pskip}{\fp,\clockbuffer}{\clocktime',\clockshard} 
            | \lbCLOCKOp(\clockshard'){\opP(\key,\val),\clocktime''} -> \ }
   \ToClockCmd{ \clocksi' | \clocktime |  \stk
            | \pruntrans{\pskip}{
                    \fp \setminus \Set{\opW(\key,\val) }
                    , \clockbuffer \uplus \Set{(\key,\Abs{\clockverlist},\clocktime'') }
                }{\clocktime',\clockshard} } 
}
\and
\inferrule[\rCLOCKCommit]{ 
    \Forall{\key,\val} \opW(\key,\val) \notin \fp
    \\
    \clocktime'' = \CLOCKMaxTime(\clockbuffer)
    \\
    \clocksi'' = \CLOCKUpdate(\clocksi,\clockbuffer,\clocktime'')
    %\\
    %\clockshardset = \Set{\ShardOf(\clocksi,\key') | (\key',\stub,\stub) \in \clockbuffer }
}{%
    \ToClockCmd{
    \clocksi | \clocktime | \stk
            | \pruntrans{\pskip}{\fp,\clockbuffer}{\clocktime',\clockshard} 
            | \lbCLOCKEnd{\clocktime''} ->
    \clocksi'' | \clocktime'' |  \stk
            | \pskip }
}
\end{mathpar}

\caption{The semantics of \( \pclockcommit \)}
\label{fig:clock-commit}
\end{subfigure}

\hrulefill

\caption{Clock-SI: transaction commit}
\label{fig:clock-si-commit}
\end{figure}
