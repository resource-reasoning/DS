In an annotated Clock-SI trace, a transaction might be interfered by others,
but in our kv-store trace, a transaction is executed atomically.
We now introduce annotated normalised Clock-SI traces in which
transactions are executed atomically.
We then show that every annotated Clock-SI trace can be transferred to a annotated normalised trace.

\begin{definition}[Annotated Normalised Clock-SI traces]
Given a shard \( \clockshard \),
a trace segment \( \clockexttrc^*\) only contains time-tick steps in \( \clockshard \)
or snapshot steps in \( \clockshard \),
written \( \CLOCKTimeTickSeg(\clockexttrc^*,\clockshard)\),
if and only if
\begin{align*}
\begin{multlined}[t]
\PredDef \Exists{m} m = \Abs{\clockexttrc^*}
\land \Forall{\clocktrcstate_0,\cdots,\clocktrcstate_{m-1}, \lb_1,\cdots,\lb_{m-1}}
\clocktrc^* = \ToClockProg{\clocktrcstate_0 | \lb_1 -> \clocktrcstate_1 | \lb_2 
    -> \cdots | \lb_{n-1} -> \clocktrcstate_{m-1} }
\\ {} \implies \bigwedge_{0 < \idx < m} 
\Exists{\cl',\clocktime'} 
        \left( \lb_i = \lbCLOCKTick{\clocktime'} 
        \lor \lb_i = \lbCLOCKStart[\cl']{\clocktime'} \right) .
\end{multlined}
\end{align*}
An annotated normalised Clock-SI trace, written \( \CLOCKNormalTrace(\clockexttrc) \), is defined by:
for all trace segments \( \clockexttrc',\clockexttrc'',\clockexttrc'''\), labels \( \lb,\lb'\),
clients \( \cl,\cl' \), shards \( \clockshard,\clockshard' \),
keys \( \key,\key' \), values \( \val,\val' \), times \( \clocktime,\clocktime' \)
\begin{multline}
\clockexttrc = \ToClockProg{\clockexttrc' | \lb 
            -> \clockexttrc''| \lb' 
            -> \clockexttrc'''}
\land 
\\ \left(
\lb = \lbCLOCKOp{\opW(\key,\val),\clocktime,\clocktime''}
\lor \lb = \lbCLOCKOp{\opR(\key,\val),\clocktime,\clocktime''}
\lor \lb = \lbCLOCKOp{\opP(\key,\val),\clocktime,\clocktime''}
\right)
\\ {} \implies
\CLOCKTimeTickSeg(\clockexttrc'',\clockshard) \land {}
\\ \left( \begin{array}{@{}l@{}} \lb' = \lbCLOCKOp{\opW(\key',\val'),\clocktime,\clocktime''}
\lor \lb' = \lbCLOCKOp{\opR(\key',\val'),\clocktime,\clocktime''}
\\ \quad {} \lor \lb' = \lbCLOCKOp{\opP(\key',\val'),\clocktime',\clocktime''}
\lor \lb' = \lbCLOCKEnd{\clocktime''}
\end{array} \right) .
\label{equ:clock-normal-def}
\end{multline}
\end{definition}

In a normalised trace \( \CLOCKNormalTrace(\clockexttrc) \), defined by \cref{equ:clock-normal-def},
any read, write or preparation step \( \lb \) from a transaction from a client \( \cl \) and its coordinator \( \clockshard \)
is followed by arbitrary numbers of time tick steps or snapshot steps in the coordinator \( \clockshard \), 
captured by \( \CLOCKTimeTickSeg \) predicate,
and then followed by a step \( \lb' \) from the same transaction.
Note that \( \lb \) and \( \lb' \) are from the same transaction 
because they are annotated by the same client \( \cl \), coordinator \( \clockshard \) and commit time \( \clocktime'' \).


\begin{theorem}[Clock-SI equivalent normalised traces]
\label{thm:clock-si-normalised-trace}
For any annotated Clock-SI trace \( \clockexttrc\),
there exists an equivalent normalised trace \( \clockexttrc^*\),
that is, \( \clockexttrc \clocktrceq \clockexttrc^*\) and \( \CLOCKNormalTrace(\clockexttrc^*) \).
\end{theorem}
\begin{proof}
Initially let \( \clockexttrc^* = \clockexttrc\).
We then preform the following transformation on the trace \( \clockexttrc^* \)
until it satisfies \( \CLOCKNormalTrace(\clockexttrc^*) \).
\begin{enumerate}
\item \textbf{Left mover: Clock-SI preparation and commit steps.}
\label{item:clock-preparation-step}
    Given the trace \( \clockexttrc^* \), take the transaction \( \txid \) with the earliest commit time such that:
    a preparation step of this transaction, \( \lb^{*} = \lbCLOCKOp{\opP(\key,\val),\clocktime',\clocktime}\),
    and the next preparation step or commit step of this transaction, \( \lb' \) have been interfered by steps from other clients.
    This means that:
    \begin{Formulae}
    \begin{Formula}
    \clockexttrc^* = \ToClockProg{\clockexttrc^{**} | \lb^{*}
        -> \clockexttrc' | \lb''
        ->  \clocksi | \clockclientenv | \clenv | \clockrunprog | \lb
        -> \clockexttrc''' | \lb' -> \clockexttrc''}
    \\ {} \land \CLOCKTimeTickSeg(\clockexttrc''',\clockshard)
    \\ {} \land \left( \lb' = \lbCLOCKOp{\opP(\key,\val),\clocktime'',\clocktime}
    \lor \lb' = \lbCLOCKEnd{\clocktime}
    \right)
    \\ {} \land \Exists{\cl',\clockshard',\clockshard'',l',\key',\val',\clocktime',\clocktime^{*}}
    \cl \neq \cl' 
    \land \clockshard \neq \clockshard''
    \\ \land
    \left(
    \begin{array}{@{}l@{}}
    \lb = \lbCLOCKOp[\cl'](\clockshard'){(l',\key',\val'),\clocktime',\clocktime^{*}}
    \lor \lb = \lbCLOCKStart[\cl'](\clockshard''){\clocktime'} 
    \\ {} \lor \lb = \lbCLOCKEnd[\cl'](\clockshard'){\clocktime'} 
    \lor \lb = \lbCLOCKTick(\clockshard''){\clocktime'} 
    \end{array}
    \right)
    \label{equ:normal-trace-first-preparation-and-commit}
    %\label{equ:normal-trace-interleaving-preparation}
    \end{Formula}
    \end{Formulae}
    where \( \clockexttrc' \) may also contains steps that does not from \( \txid \).
    However, let first focus on the step \( \lb \).
    Note that since we always pick the first transaction with labels \( \lb^{*},\lb'\) 
    of the form in \cref{equ:normal-trace-first-preparation-and-commit}.
    Therefore, if \( \lb \) is a read, write or preparation step, or a commit step from other client, 
    the transaction of \( \lb \) must commit after \( \clocktime' \),
    that is, \({\clocktime}^{*} \geq \clocktime \).
    By \cref{equ:normal-trace-first-preparation-and-commit}, \({\clocktime}^{*} \geq \clocktime \)
    and \cref{prop:clock-si-left-mover},
    it allows use to move \( \lb \) to the left:
    for some trace segment \( \clockexttrc^{***} \) modified from \( \clockexttrc''' \)
    and \( \lb^{**}\) modified from \( \lb \):
    \[
        \left( \ToClockProg{\clockexttrc^{**} | \lb^{*}
        -> \clockexttrc' | \lb''
        -> \clockexttrc^{***} | \lb'
        -> \stub | \lb^{**} -> \clockexttrc''} 
        \right)
        \clocktrceq \clockexttrc^* 
    \]                              
    We assign the new trace to \( \clockexttrc^*\) and go back to the beginning of \cref{item:clock-preparation-step}.
    This process must terminate because each iteration the labels \(\lb \) and \( \lb' \) moves closer to each other and there are only finite transactions in the trace.
    In the end, the trace \( \clockexttrc^*\) satisfies:
    \begin{enumerate*}
        \item the preparation steps and commit step for each transaction 
                    are not interfered by other clients; and
        \item transactions commit in the order of their commit times.
    \end{enumerate*}
    This means that:
    for all trace segments \( \clockexttrc',\clockexttrc'',\clockexttrc'''\), labels \( \lb,\lb'\),
    client \( \cl,\cl' \), shard \( \clockshard,\clockshard' \),
    keys \( \key,\key' \), values \( \val,\val' \), times \( \clocktime,\clocktime',\clocktime'' \)
    \begin{multline}
    \clockexttrc^* = \ToClockProg{\clockexttrc' | \lb 
                -> \clockexttrc'' | \lb' 
                -> \clockexttrc'''}
    \land \lb = \lbCLOCKOp{\opP(\key,\val),\clocktime,\clocktime''}
    \\ {} \implies
    \CLOCKTimeTickSeg(\clockexttrc'',\clockshard)
    \land \left( \lb = \lbCLOCKOp{\opP(\key',\val'),\clocktime'}
    \lor \lb = \lbCLOCKEnd{\clocktime'} \right)
    \\ \land \Forall{\lb'' \in \clockexttrc''' | \clockshard',\cl',\key'',\val'',\clocktime''',\clocktime^*}
    \left( \lb'' = \lbCLOCKOp[\cl'](\clockshard'){\opP(\key'',\val''),\clocktime''',\clocktime^*} \implies \clocktime^* \geq \clocktime'' \right) .
    \label{equ:clock-si-after-left-move}
    \end{multline}
\item \textbf{Right mover: Clock-SI internal read and write steps.}
\label{item:clock-read-write-step}
    Given an annotated trace \( \clockexttrc^* \) that satisfies \cref{equ:clock-si-after-left-move}.
    We now move all the internal read and write of a transaction towards the preparation and commit of the transaction.
    Given the trace \( \clockexttrc^* \), we pick a read or write \( \lb \) such that:
    for some  \( \clocksi, \clockclientenv, \clenv, \clockrunprog\),
    \begin{Formulae}*
    \begin{Formula}
        \clockexttrc^* = \ToClockProg{\clockexttrc' | \lb 
            -> \clocksi | \clockclientenv | \clenv | \clockrunprog | \lb' -> \clockexttrc'' } 
        \land \left(
        \lb = \lbCLOCKOp{\opW(\key,\val),\clocktime,\clocktime'}
        \lor \lb = \lbCLOCKOp{\opR(\key,\val),\clocktime,\clocktime'}
        \right)
        \\ {} \land \Forall{l,\key',\val',\clocktime''}
        \lb' \neq \lbCLOCKOp{(l,\key',\val'),\clocktime''}
        \land \lb' \neq \lbCLOCKEnd{\clocktime'} 
    \end{Formula}
    \end{Formulae}
    By \cref{pro:clock-si-right-move-read-write}, we can move the \( \lb \) to the right,
    which yields
    \[
        \left( \ToCOPSProg{\clockexttrc' | \lb' 
            -> \clocksi' | \clockclientenv' | \clenv' | \clockrunprog' | \lb -> \clockexttrc'' }  \right)
        \clocktrceq \clockexttrc^*
    \]
    for some new \( \clocksi', \clockclientenv', \clenv', \clockrunprog'\).
    We assign the new trace to \( \clockexttrc^*\) and go back to the beginning of \cref{item:clock-read-write-step}.
    This process must terminate because each iteration the labels \(\lb \) of a transaction
    moves closer to the first preparation step or commit step of the same transaction,
    and there are only finite transactions in the trace.
    In the end, the trace \( \clockexttrc^*\) is a normalised trace,
    that is, \( \CLOCKNormalTrace(\clockexttrc^*)\). \qedhere
\end{enumerate}
\end{proof}

\begin{toappendix}
\label{sec:proof-right-move-clock-read-write}
\end{toappendix}
\begin{propositionrep}[Right mover: Clock-SI internal read and write steps]
\label{pro:clock-si-right-move-read-write}
Assume a Clock-SI trace \( \clockexttrc \in \CLOCKExtTraces \), and
two adjacent transitions with labels \( \lb,\lb' \) such that
\begin{multline*}
\clockexttrc = \ToClockProg{\clockexttrc' | \lb''
    -> \clocksi | \clockclientenv | \clenv | \clockrunprog | \lb 
    -> \clocksi' | \clockclientenv' | \clenv' | \clockrunprog' | \lb' -> \clockexttrc''}
\\ {} \land \left(
\lb = \lbCLOCKOp{\opW(\key,\val),\clocktime,\clocktime''}
\lor \lb = \lbCLOCKOp{\opR(\key,\val),\clocktime,\clocktime''}
\right)
\\ {} \land \Forall{l,\key',\val',\clocktime',\clocktime'''}
\lb' \neq \lbCLOCKOp{(l,\key',\val'),\clocktime',\clocktime'''}
\land \lb' \neq \lbCLOCKEnd{\clocktime'''} 
\end{multline*}
where the rest free variables are universally quantified.
The transition labelled by \( \lb \) can be moved to the right,
that is,
there exists a new equivalent trace \( \clockexttrc^*\) such that
\[
\clockexttrc^* = \ToClockProg{\clockexttrc' | \lb'' -> \stub | \lb' -> \stub | \lb -> \clockexttrc''}
\land \clockexttrc \clocktrceq \clockexttrc^* .
\]
\end{propositionrep}
\begin{proof}
Assume a trace \( \clockexttrc\) such that
\(
\clockexttrc = \ToClockProg{\clockexttrc' | \lb''
    -> \clocksi | \clockclientenv | \clenv | \clockrunprog | \lb 
    -> \clocksi' | \clockclientenv' | \clenv' | \clockrunprog' | \lb' -> \clockexttrc''} 
\)
and assume \( \lb \) such that 
\(
\lb = \lbCLOCKOp{\opW(\key,\val),\clocktime}
\lor \lb = \lbCLOCKOp{\opR(\key,\val),\clocktime} 
\)
where all other free variables are universal quantified.
By the \( \rCLOCKWrite \), \( \rCLOCKReadLocal\) and \( \rCLOCKReadShard\) rules,
thus there exists a stack \( \stk \) and a Clock-SI runtime command \(\clockruncmd \),
\[
\clocksi = \clocksi' \land  \clockclientenv = \clockclientenv'
\land \clenv' = \clenv\UpdateFunc{\cl -> \stk} 
\land \prog' = \prog\UpdateFunc{\cl -> \clockruncmd} .
\]
This means that
\(
\clockexttrc = \ToClockProg{\clockexttrc' | \lb''
    -> \clocksi | \clockclientenv | \clenv | \clockrunprog | \lb 
    -> \clocksi | \clockclientenv 
                | \clenv\UpdateFunc{\cl -> \stk} 
                | \clockrunprog\UpdateFunc{\cl -> \clockruncmd} | \lb'
    -> \clockexttrc''} .
\)
We perform case analysis on \( \lb' \).
\begin{enumerate}
\Case{\(\lb' = \lbCLOCKStart[\clockshard'](\cl'){\clocktime'}\)}
    We immediately have \( \cl \neq \cl' \).        
    By \( \rCLOCKStart \) rule,
    there exists a Clock-SI runtime command \( \clockruncmd'\) such that
    \[
    \ToClockProg{\clocksi | \clockclientenv 
                    | \clenv\UpdateFunc{\cl -> \stk} 
                    | \clockrunprog\UpdateFunc{\cl -> \clockruncmd} 
                    | \lbCLOCKStart[\clockshard'](\cl'){\clocktime'}
        -> \clocksi | \clockclientenv 
                    | \clenv\UpdateFunc{\cl -> \stk} 
                    | \clockrunprog\UpdateFunc{\cl -> \clockruncmd | \cl' -> \clockruncmd' }} .
    \]
    Note that \( \clockrunprog\UpdateFunc{\cl -> \clockruncmd | \cl' -> \clockruncmd' } 
                    = \clockrunprog\UpdateFunc{\cl' -> \clockruncmd' | \cl -> \clockruncmd }\).
    This means that 
    \[
    \clockexttrc \clocktrceq \left( 
    \ToClockProg{\clockexttrc' | \lb''
        -> \clocksi | \clockclientenv | \clenv | \clockrunprog | \lb'
        -> \clocksi | \clockclientenv 
                    | \clenv 
                    | \clockrunprog\UpdateFunc{\cl' -> \clockruncmd'} | \lb
        -> \clockexttrc''}
    \right) .
    \]
\Case{\(\lb' = \lbCLOCKOp[\clockshard'](\cl'){(l',\key',\val'),\clocktime'}\)}
    We immediately have \( \cl \neq \cl' \) and \( l \in \Set{\opW,\opR,\opP}\).
    By \( \rCLOCKWrite \) (for \(\opW\)), 
    \( \rCLOCKReadLocal\), \( \rCLOCKReadShard\) (for \( \opR\))
    and \(\rCLOCKPrepare \) (for \( \opP \)) rules,
    there exists a stack \( \stk' \) and a Clock-SI runtime command \( \clockruncmd'\) such that
    \begin{multline*}
    \ToClockProg{\clocksi | \clockclientenv 
                    | \clenv\UpdateFunc{\cl -> \stk} 
                    | \clockrunprog\UpdateFunc{\cl -> \clockruncmd} 
                    | \lbCLOCKOp[\clockshard'](\cl'){(l',\key',\val'),\clocktime'} -> \ }
    \\ \ToClockProg{ \clocksi | \clockclientenv 
                    | \clenv\UpdateFunc{\cl -> \stk | \cl' -> \stk'} 
                    | \clockrunprog\UpdateFunc{\cl -> \clockruncmd | \cl' -> \clockruncmd' }} .
    \end{multline*}
    Note that 
    \[ \clenv\UpdateFunc{\cl -> \stk | \cl' -> \stk'} 
                = \clenv\UpdateFunc{\cl' -> \stk' | \cl -> \stk} 
            \land \clockrunprog\UpdateFunc{\cl -> \clockruncmd | \cl' -> \clockruncmd' } 
                    = \clockrunprog\UpdateFunc{\cl' -> \clockruncmd' | \cl -> \clockruncmd }.
    \]
    This means that 
    \[
    \clockexttrc \clocktrceq \left( 
    \ToClockProg{\clockexttrc' | \lb''
        -> \clocksi | \clockclientenv | \clenv | \clockrunprog | \lb'
        -> \clocksi | \clockclientenv 
                    | \clenv\UpdateFunc{\cl' -> \stk'} 
                    | \clockrunprog\UpdateFunc{\cl' -> \clockruncmd'} | \lb
        -> \clockexttrc''}
    \right) .
    \]
\Case{\(\lb' = \lbCLOCKEnd[\clockshard'](\cl'){\clocktime'}\)}
    We immediately have \( \cl \neq \cl' \).        
    By \( \rCLOCKCommit \) rule
    there exists a new Clock-SI database \( \clocksi' \),
    and a Clock-SI runtime command \( \clockruncmd'\) such that
    \begin{multline*}
    \ToClockProg{\clocksi | \clockclientenv 
                    | \clenv\UpdateFunc{\cl -> \stk} 
                    | \clockrunprog\UpdateFunc{\cl -> \clockruncmd} 
                    | \lbCLOCKOp[\clockshard'](\cl'){(l',\key',\val'),\clocktime'} -> \  }
     \\ \ToClockProg{ \clocksi' | \clockclientenv\UpdateFunc{\cl' -> \clocktime'}
                    | \clenv\UpdateFunc{\cl -> \stk} 
                    | \clockrunprog\UpdateFunc{\cl -> \clockruncmd | \cl' -> \clockruncmd' }} 
    \land \clocksi \clocksileq \clocksi' .
    \end{multline*}
    Note that \( \clockrunprog\UpdateFunc{\cl -> \clockruncmd | \cl' -> \clockruncmd' } 
                    = \clockrunprog\UpdateFunc{\cl' -> \clockruncmd' | \cl -> \clockruncmd }\).
    Then by \( \clocksi \clocksileq \clocksi' \) and \cref{prop:read-write-on-bigger-database},
    the transition labelled by \( \lb \) can be proceeded on \( \clocksi \).
    This means that 
    \[
    \clockexttrc \clocktrceq \left( 
    \ToClockProg{\clockexttrc' | \lb''
        -> \clocksi | \clockclientenv | \clenv | \clockrunprog | \lb'
        -> \clocksi' | \clockclientenv\UpdateFunc{\cl' -> \clocktime'}  
                    | \clenv
                    | \clockrunprog\UpdateFunc{\cl' -> \clockruncmd'} | \lb
        -> \clockexttrc''}
    \right) . \qedhere
    \] 
\end{enumerate}
\end{proof}
\begin{proofsketch}
By \( \rCLOCKWrite\), any write step does not depend on the state of Clock-SI database.
By \( \rCLOCKReadLocal\), any read step that reads from the local state of transaction 
does not depend on the state of Clock-SI database.
Therefore, for the these two cases, it is trivial that we can move these \( \lb \) to the right.
By \( \rCLOCKReadShard\), any read step that reads from a shard, 
must access a shard whose local time is greater than the snapshot time of the transaction
and all versions in the shard with times smaller than the snapshot time must be committed successful.
This means, the read step \( \lb \) can be moved to the right, since any future steps will not affect the read.
The full proof is in \cref{sec:proof-right-move-clock-read-write} on page \pageref{sec:proof-right-move-clock-read-write}.
\end{proofsketch}

\begin{toappendix}
\begin{proposition}[Read and write steps on a larger Clock-SI database]
\label{prop:read-write-on-bigger-database}
Assume a read or write transition
\[
\ToClockCmd{ \clocksi | \clocktime | \stk | \clockruncmd | \lb 
    -> \clocksi | \clocktime' | \stk' | \clockruncmd' } 
\land 
\left( 
\lb = \lbCLOCKOp{\opW(\key,\val),\clocktime''}
\lor \lb = \lbCLOCKOp{\opR(\key,\val),\clocktime''} 
\right) .
\]
Given a Clock-SI database \( \clocksi' \) such that
\( \clocksi \clocksileq \clocksi' \),
the same transition can be proceeded on \( \clocksi' \),
that is,
\[
\ToClockCmd{ \clocksi | \clocktime | \stk | \clockruncmd | \lb 
    -> \clocksi | \clocktime' | \stk' | \clockruncmd' } 
\]
\end{proposition}
\begin{proof}
We consider rules \( \rCLOCKWrite \), \(\rCLOCKReadLocal\) and \( \rCLOCKReadShard\) separately.
\begin{enumerate}
\Cases{{\(\rCLOCKWrite\)} and {\(\rCLOCKReadLocal\)}}
    Given the premise of \( \rCLOCKWrite \) and \( \rCLOCKReadLocal \),
    these two transitions do not depend on the state of \( \clocksi \).
\Case{{\(\rCLOCKReadShard\)}}
    We have \( \lb = \lbCLOCKOp{\opR(\key,\val),\clocktime} \)
    for some \( \clockshard,\cl,\key,\val,\clocktime\).
    Let \( \clockshard' = \ShardOf(\clocksi,\key) \) and \( (\clockkvs,\clocktime^*) = \clocksi(\clockshard') \).
    By the premise of \( \rCLOCKReadShard \),
    the shard local time must be greater than the snapshot time, that is \( \clocktime^* > \clocktime'' \).
    Let \( \clockverlist = \Set{\clockver | \Exists{\idx} 
                \clockkvs(\key,\idx) = \clockver
                \land \TimeOf(\clockver) \leq \clocktime'' }\) be the set of versions of \( \key \) 
    that have time-stamp smaller than the snapshot time \( \clocktime'' \).
    By the premise of \( \rCLOCKReadShard \),
    \( (\key\val,\stub) = \Max(\clockverlist) \).
    Because the hypothesis that \( \clocksi \clocksileq \clocksi' \).
    Let \( Tuple{\clockkvs',{\clocktime}^{**}} = \clocksi'(\clockshard')\).
    We have \( \clockkvs \clockkvsleq \clockkvs'  \) and \( \clocktime^* \leq {\clocktime}^{**} \).
    Therefore \( {\clocktime}^{**} > \clocktime'' \) and
    \( \clockverlist = \Set{\clockver | \Exists{\idx} 
                \clockkvs'(\key,\idx) = \clockver
                \land \TimeOf(\clockver) \leq \clocktime }\),
    which means the same transition can be proceeded on \( \clocksi' \).
    In other words,
    \[
    \ToClockCmd{ \clocksi | \clocktime | \stk | \clockruncmd | \lb 
        -> \clocksi | \clocktime' | \stk' | \clockruncmd' }  \qedhere
    \]
\end{enumerate}
\end{proof}
\end{toappendix}


\begin{toappendix}
\label{sec:proof-left-move-preparation-commit}
\end{toappendix}
\begin{propositionrep}[Left mover: Clock-SI preparation and commit steps]
\label{prop:clock-si-left-mover}
Assume a Clock-SI trace \( \clockexttrc \in \CLOCKExtTraces \),
two transitions with labels \( \lb,\lb' \),
and a time-tick trace segment \( \clockexttrc^* \) such that
\begin{multline*}
\clockexttrc = \ToClockProg{\clockexttrc' | \lb''
    -> \clocksi | \clockclientenv | \clenv | \clockrunprog | \lb 
    ->  \clockexttrc^* | \lb' -> \clockexttrc''}
\\ {} \land \CLOCKTimeTickSeg(\clockexttrc^*,\clockshard)
\land \left( \lb' = \lbCLOCKOp{\opP(\key,\val),\clocktime'',\clocktime}
\lor \lb' = \lbCLOCKEnd{\clocktime}
\right)
\\ {} \land \Exists{\cl',\clockshard',\clockshard'',l',\key',\val',\clocktime',\clocktime^{*}}
\cl \neq \cl' 
\land \clockshard \neq \clockshard''
\land {\clocktime}^{*} \geq \clocktime 
\\ \land
\left(
\lb = \lbCLOCKOp[\cl'](\clockshard'){(l',\key',\val'),\clocktime',\clocktime^{*}}
\lor \lb = \lbCLOCKStart[\cl'](\clockshard''){\clocktime'} 
\lor \lb = \lbCLOCKEnd[\cl'](\clockshard'){\clocktime'} 
\lor \lb = \lbCLOCKTick(\clockshard''){\clocktime'} 
\right)
\end{multline*}
where the rest free variables are universally quantified.
The transition labelled by \( \lb \), together with the time-tick trace segment \( \clockexttrc^*\),
can be moved to the left, that is,
there exists a new time-tick trace segment \( \clockexttrc^{**} \),
a new label \( \lb^* \) and a new equivalent trace such that
\begin{align}
\CLOCKTimeTickSeg(\clockexttrc^{**},\clockshard)
\land \left( \ToClockProg{\clockexttrc' | \lb''
                -> \clockexttrc^{**} | \lb' -> \stub | \lb^* -> \clockexttrc''} \right)
\clocktrceq \clockexttrc .
\label{equ:clock-left-move-goal}
\end{align}
\end{propositionrep}
\begin{proof}
Assume a trace \( 
\clockexttrc = \ToClockProg{\clockexttrc' | \lb''
    -> \clocksi | \clockclientenv | \clenv | \clockrunprog | \lb 
    ->  \clockexttrc^* | \lb' -> \clockexttrc''} \).
Assume \( \lb' = \lbCLOCKOp{\opP(\key,\val),\clocktime'', \clocktime} \) or \( \lb' = \lbCLOCKEnd{\clocktime} \).
Let \( \left( \ToCOPSProg{\clocksi' | \clockclientenv' | \clenv' | \clockrunprog'}\right)  = \clockexttrc''\Proj{0}\) 
be the first state of the trace segment \( \clockexttrc''\).
Let \( (\clockkvs,\clocktime^{**}) = \clocksi(\clockshard) \) 
and \( \Tuple{\clockkvs',{\clocktime}^{***}} = \clocksi(\clockshard') \).
By \( \rCLOCKPrepare\)  and \( \rCLOCKCommit \),
\begin{Formulae}
& \begin{Formula}
\clockkvs \clockkvsleq \clockkvs' \land \clocktime^{**} < {\clocktime}^{***}
\land \clocksi' = \clocksi\UpdateFunc{\clockshard -> (\clockkvs',\clocktime^{***})} 
\land \clockclientenv \clockclientenvleq \clockclientenv'
\label{equ:clocksi-left-move-preparation-step-or-end-step}
\end{Formula}
\end{Formulae}
We perform case analysis on the label \( \lb \).
\begin{enumerate}
\Case{{\(\lb = \lbCLOCKStart[\cl'](\clockshard'){\clocktime'}\)}}
    We immediately have \( \clockshard \neq \clockshard' \) and \( \cl \neq \cl' \).
    By \( \rCLOCKStart \), the transition labelled by \( \lb \) only depends on the state of \( \clockshard \),
    that is, \( \clocksi(\clockshard') = (\stub,\clocktime')\).
    This transition also does not change the state of the local time.
    By \cref{equ:clocksi-left-move-preparation-step-or-end-step}, 
    the state of \( \clockshard \) remain unchanged in \( \clocksi' \),
    that is, \( \clocksi(\clockshard') = \clocksi'(\clockshard')\).
    Therefore, there exists a new time-tick trace segment \( \clockexttrc^{**} \) 
    and a runtime program \( \clockrunprog'' \) such that
    \[
    \left( \ToClockProg{\clockexttrc' | \lb''
                    -> \clockexttrc^{**} | \lb' -> 
                    \clocksi' | \clockclientenv' | \clenv' 
                                | \clockrunprog'' | \lb 
                    -> \clockexttrc''} \right)
    \clocktrceq \clockexttrc 
    \]
    where \( \clockexttrc^{**} \) a trace contains the same steps as \( \clockexttrc^{*} \).
\Case{{\(\lb = \lbCLOCKOp[\cl'](\clockshard'){\opR(\key',\val'), \clocktime'}\)} or {\(\lb = \lbCLOCKOp[\cl'](\clockshard'){\opW(\key',\val'), \clocktime'}\)}}
We immediately have \( \cl \neq \cl' \).
Note that any start label in the trace segment \( \clockexttrc^*\) cannot come from the client \( \cl' \),
because the transaction corresponds to the label \( \lb \) have not finish.
Therefore, by \cref{prop:read-write-on-bigger-database}, we move \( \lb \) to the right until we have the proof.
\Case{\(\lb = \lbCLOCKOp[\cl'](\clockshard'){\opP(\key',\val'),\clocktime',{\clocktime}^{*}}\)}
    We immediately have \( \cl \neq \cl' \) and \( {\clocktime}^{*} \geq \clocktime \).
    There are two cases: \( \clockshard = \clockshard' \) or \( \clockshard \neq \clockshard' \).
    \begin{enumerate}
    \Case{\( \clockshard = \clockshard' \)}
        There are two cases: \( \lb' = \lbCLOCKOp{\opP(\key,\val),\clocktime'', \clocktime} \) or \( \lb' = \lbCLOCKEnd{\clocktime} \).
        \begin{enumerate}
        \Case{\( \lb' = \lbCLOCKOp{\opP(\key,\val),\clocktime'', \clocktime} \)}
            Note that because both \(\lb \) and \( \lb' \) operate on the same shard \( \clockshard \),
            and by \( \rCLOCKPrepare \) that there must be no concurrent writes on the keys \( \key \) and \( \key' \)
            until the preparation time \( \clocktime'' \) and \( \clocktime' \) respectively
            (modelled by the premise 
            \( \Forall{\idx} 0 \leq \idx < n \implies \TimeOf(\clockverlist(\idx)) < \clocktime' \) in \( \rCLOCKCommit\)),
            then \( \key \neq \key'\).
            These two preparation steps, \( \lb \) and \( \lb' \), 
            also stop the preparation step from other transactions that want to write to
            keys \( \key \) and \( \key' \) in between \( \clocktime''\) and \( \clocktime \),
            and between \( \clocktime'\) and \( \clocktime^*\) respectively.
            Now consider the first preparation step \( \lb \).
            It is safe to advanced the preparation time of this transition up to the actual commit time \( \clocktime^* \).
            Recall that: \begin{enumerate*}
            \item by the hypothesis, we have \(\clocktime \leq \clocktime^*\);
            \item any preparation steps must happen before the actual commit, 
            that is, \( \clocktime'' \leq \clocktime\) and  \( \clocktime' \leq \clocktime^*\)
            respectively;
            \item for the same shard, the time must monotonically increase, thus
            \( \clocktime' \leq \clocktime'' \).
            \end{enumerate*} 
            Therefore, for the two preparation steps in the original trace,  
            we have
            \[
                \clocktime' \leq \clocktime'' \leq \clocktime \leq \clocktime^* .
            \]
            This means it is allowed to delay the preparation step \( \lb \) after \( \lb' \),
            even thought the step \( \lb \) might be assigned a larger time as \( \clocktime'' \).
            Let \( \lb^* = \lbCLOCKOp[\cl'](\clockshard'){\opP(\key',\val'),\clocktime'',{\clocktime}^{*}} \) 
            Therefore, there exists a new trace segment \(  \clockexttrc^{**} \) such that:
            \[
            \left( \ToClockProg{\clockexttrc' | \lb''
                            -> \clockexttrc^{**} | \lb' -> 
                            \clocksi' | \clockclientenv' | \clenv' 
                                        | \clockrunprog'' | \lb^*
                            -> \clockexttrc''} \right)
            \clocktrceq \clockexttrc .
            \]
        \Case{\( \lb' = \lbCLOCKEnd{\clocktime} \)}
            Follow the similar argument as the previous case,
            we have 
            \[
                \clocktime' \leq \clocktime \leq \clocktime^* .
            \]
            Let \( \lb^* = \lbCLOCKOp[\cl'](\clockshard'){\opP(\key',\val'),\clocktime'',{\clocktime}^{*}} \) 
            Therefore, there exists a new trace segment \(  \clockexttrc^{**} \) such that:
            \[
            \left( \ToClockProg{\clockexttrc' | \lb''
                            -> \clockexttrc^{**} | \lb' -> 
                            \clocksi' | \clockclientenv' | \clenv' 
                                        | \clockrunprog'' | \lb^*
                            -> \clockexttrc''} \right)
            \clocktrceq \clockexttrc .
            \]
        \end{enumerate}
    \Case{\( \clockshard \neq \clockshard' \)}
        Since steps from \( \clockexttrc^{*} \) and \( \lb \) operates on different shard,
        it is trivial to see that 
        there exists a new time-tick trace segment \( \clockexttrc^{**} \) 
        and a runtime program \( \clockrunprog'' \) such that
        \[
        \left( \ToClockProg{\clockexttrc' | \lb''
                        -> \clockexttrc^{**} | \lb' -> 
                        \clocksi' | \clockclientenv' | \clenv' 
                                    | \clockrunprog'' | \lb 
                        -> \clockexttrc''} \right)
        \clocktrceq \clockexttrc . \qedhere
        \]
    \end{enumerate}
\end{enumerate}
\end{proof}
\begin{proofsketch}
We know that \( \lb \) must come from a client that differs from the client of \( \lb' \).
\begin{enumerate}
\item Consider the cases that \( \lb \) is a read and write step.
Since any future steps will not affect any read and write step, 
thus, it is safe to move \( \lb' \) and \( \clockexttrc^*\) to the left.
In \cref{equ:clock-left-move-goal}, \( \lb^{*} = \lb \).
\item If \( \lb \) is a transaction snapshot step,
then \( \lb \) must starts at a shard that is different from the shard of \( \lb' \).
Therefore, it is safe to move \( \lb' \) and \( \clockexttrc^*\) to the left.
In \cref{equ:clock-left-move-goal}, \( \lb^{*} = \lb \).
\item If \( \lb \) is a time-tick step,
then \( \lb \) must on a shard that is different from the shard of \( \lb' \).
Therefore, it is safe to move \( \lb' \) and \( \clockexttrc^*\) to the left.
In \cref{equ:clock-left-move-goal}, \( \lb^{*} = \lb \).
\item Consider \( \lb \) is a transaction preparation step.
If this is a preparation step on a shard that is different a from from the shard of \( \lb' \),
it is safe to move \( \lb' \) and \( \clockexttrc^*\) to the left. 
For this case, \( \lb^{*} = \lb \) in \cref{equ:clock-left-move-goal}.
Consider this is a preparation step on the same shard of \( \lb' \).
By the hypothesis, the transaction of \( \lb = \lbCLOCKOp[\cl'](\clockshard'){(l,\key',\val'),\clocktime',\clocktime^{*}} \) 
must commit after the transaction of \( \lb' \),
thus it is safe to alter the preparation time in \( \lb \) to a later time as the same as the preparation time in \( \lb' \).
This means change \( \lb \)  to \( \lb^{*} = \lbCLOCKOp[\cl'](\clockshard'){(l,\key',\val'),\clocktime,\clocktime^{*}} \) 
This allows us to move \( \lb^{*} \) and \( \clockexttrc^*\) to the left. 
\item Consider \( \lb \) is a transaction commit step.
This case is similar to the case of preparation step.
In \cref{equ:clock-left-move-goal}, \( \lb^{*} = \lb \).
\end{enumerate}
The full proof is given in \cref{sec:proof-left-move-preparation-commit}
on page \pageref{sec:proof-left-move-preparation-commit}.
\end{proofsketch}
