\begin{figure}

\centering

\begin{subfigure}{\textwidth}
\centering
\begin{tikzpicture}
\OperationsBox[above]{r1}{\clockshard_1,\ \texttt{Local Time:} \ 2}{
    /\texttt{key-value store:},
    /{\Tuple{\key_1 , \valinit, 0} },
    /{},
    /\texttt{Preparation set:},
    /\emptyset
};

\OperationsBox((r1.east) + (2,0.35))[above]{r2}{\clockshard_2,\ \texttt{Local Time:} \ 2}{
    /\texttt{key-value store:},
    /{\Tuple{\key_2 , \valinit, 0} },
    /{\Tuple{\key_2 , \val', 2} },
    /\texttt{Preparation set:},
    /\emptyset
};

\OperationsBox((r2.south) + (-2.8,-1))[above]{t2}{\txid_2,\ \texttt{Snapshot Time:} \ 1}{
    /{\opR(\key_1,\valinit)},
    /{\opR(\key_2,\valinit)},
    /{\opW(\key_1,\val)}
};
\end{tikzpicture}
\caption{Client \( \cl_1 \) commits \( \txid_1 \) to \(\clockshard_1 \), 
while \( \txid_2\) are still running}
\label{fig:clock-si-commit-txid-1}
\end{subfigure}

\hrulefill

\begin{subfigure}{\textwidth}
\centering
\begin{tikzpicture}
\OperationsBox[above]{r1}{\clockshard_1,\ \texttt{Local Time:} \ 2}{
    /\texttt{key-value store:},
    /{\Tuple{\key_1 , \valinit, 0} },
    /{\Tuple{\key_1 , \val, 2} },
    /\texttt{Preparation set:},
    /\emptyset
};

\OperationsBox((r1.east) + (1.5,0.35))[above]{r2}{\clockshard_2,\ \texttt{Local Time:} \ 2}{
    /\texttt{key-value store:},
    /{\Tuple{\key_2 , \valinit, 0} },
    /{\Tuple{\key_2 , \val', 2} },
    /\texttt{Preparation set:},
    /\emptyset
};
\end{tikzpicture}
\caption{Client \( \cl_2 \) commits \( \txid_2 \) to \(\clockshard_2 \)}
\label{fig:clock-si-commit-txid-2}
\end{subfigure}

\hrulefill

\caption{an example result of \( \txid_1 \) and \( \txid_2 \)}
\label{fig:clock-si-result}

\end{figure}
