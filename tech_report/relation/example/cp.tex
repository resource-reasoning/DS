An abstract execution \( \aexec \) satisfies consistency prefix (\(\CP\)), 
if it satisfies \( \AR[\aexec] ; \VIS[\aexec] \subseteq \VIS[\aexec] \) and \( \SO \subseteq \VIS[\aexec] \) \citep{giovanni_concur16}.
\citet{SIanalysis} proposed a method that can compute  and express \( \AR \) and \( \VIS \)
using \( \WR, \WW, \RW \) and \( \SO \).

\begin{theorem}[Minimum visibility relation for (\texorpdfstring{\CP}{\texttt{CP}})]
\label{thm:cp-eauiv-spec}
For any abstract execution \( \aexec \),
if it satisfies 
\begin{Formulae}
\begin{Formula}
    \left( (\SO \cup \WR[\aexec] ) ; \Refl(\RW[\aexec]) \cup \WW[\aexec] \right) ; \VIS[\aexec] \subseteq \VIS[\aexec]
    \qquad \SO \subseteq \VIS[\aexec]
    \label{equ:kvstore-cp-spec}
\end{Formula}
\end{Formulae}
then there exists a new \( \aexec' \) such that \( \aexec(\txid) = \aexec'(\txid) \)
for all \( \txid \in \Dom(\aexec) \), and
\begin{Formulae}
\begin{Formula}
    \AR[\aexec'] ; \VIS[\aexec'] \subseteq \VIS[\aexec']  \qquad \SO \subseteq \VIS[\aexec'] 
    \label{equ:aexec-cp-spec}
\end{Formula}
\end{Formulae}
and vice versa.
\end{theorem}
\begin{proof}
\begin{enumerate}
\Case{\(\aexec\) to \( \aexec' \)}
    Assuming an abstract execution \( \aexec \) that satisfies \cref{equ:kvstore-cp-spec},
    by solving the following inequalities 
    (the first five are universally true for all abstract execution 
    and the last two are for \cref{equ:aexec-cp-spec})
    \[
    \begin{array}{@{}l@{}}
        \WR \subseteq \VIS 
        \qquad \WW \subseteq \AR 
        \qquad \VIS \subseteq \AR \\
        \VIS ; \RW \subseteq \AR 
        \qquad \AR ; \AR \subseteq \AR  \\
        \SO \subseteq \VIS 
        \qquad \AR ; \VIS \subseteq \VIS
    \end{array}
    \]
    the visibility and arbitration relations can be defined by 
    \begin{align}
        \AR  & \FuncDef \left( (\SO \cup \WR ) ; \Refl(\RW) \cup \WW \cup \rel \right)^+ \label{equ:cp-ar}
        \\ \VIS & \FuncDef \left( (\SO \cup \WR ) ; \Refl(\RW) \cup \WW \cup \rel \right)^* ; (\SO \cup \WR ) \label{equ:cp-vis} 
    \end{align}
    for some relation \( \rel \subseteq \AR \).
    We take \( \rel \) such that it extends the relation \cref{equ:cp-ar} to a total order and thus 
    we have a new \( \aexec' \) that satisfies \cref{equ:aexec-cp-spec}.
\Case{\(\aexec'\) to \( \aexec \)}
    Assume an abstract execution \( \aexec' \) that satisfies \cref{equ:aexec-cp-spec}.
    Let \( \rel = \emptyset \) and thus we get an abstract execution \( \aexec \) that satisfies \cref{equ:kvstore-cp-spec}. \qedhere
\end{enumerate}
\end{proof}

The execution test \( \et[\CP] \) is sound with respect to the axiomatic definition 
\[
\visaxioms[\CP] \FuncDef \Set{\lambda \aexec \ldotp \left( (\SO \cup \WR[\aexec] ) ; \Refl(\RW[\aexec]) \cup \WW[\aexec] \right) ; \VIS[\aexec], \lambda \aexec \ldotp \SO } .
\]
We pick the invariant \( \aexecinv[\CP] = \aexecinv[\RYW]\).
\SOUNDLET{\CP}{ \txidsetrd \supseteq
\begin{multlined}[t]
\left( \bigcup_{\Set{\txid[\cl](\idx) | \txid[\cl](\idx) \in \aexec}} 
\VISInv[\aexec](\txid[\cl](\idx)) \cup \Refl((\Inv(\SO)))(\txid[\cl](\idx)) \right) 
\setminus \Set{\txid' | \Forall{l | \key | \val } (l,\key,\val) \in \aexec(\txid') \implies l = \opR } .
\end{multlined} }
Assume 
\[ 
\txidsetrd' = 
\begin{multlined}[t]
\left( \bigcup_{\Set{\txid[\cl](\idx) | \txid[\cl](\idx) \in \aexec}} 
\VISInv[\aexec](\txid[\cl](\idx)) \cup \Refl((\Inv(\SO)))(\txid[\cl](\idx)) \right) 
\setminus \Set{\txid' | \Forall{l | \key | \val } (l,\key,\val) \in \aexec(\txid') \implies l = \opR } .
\end{multlined} 
\]
and \( \txidsetrd'' = \txidsetrd \setminus \txidsetrd' \).
By the definition of soundness, we prove the following result:
\begin{Formulae}
& \begin{Formula}
\Inv(\SO)(\txid) \subseteq \txidset \cup \txidsetrd'
\label{equ:cp-so-vis}
\end{Formula}
\\ & \begin{Formula}
\Inv(\left( (\SO \cup \WR[\aexec] ) ; \Refl(\RW[\aexec]) \cup \WW[\aexec] \right)) (\txid) \subseteq \txidset \cup \txidsetrd' \cup \txidsetrd''
\label{equ:cp-vis-ar}
\end{Formula}
\\ & \begin{Formula}
\aexecinv[\CP](\aexec',\cl) \subseteq \VisTrans(\XToK(\aexec'),\vi')
\label{equ:cp-inv-preserve}
\end{Formula}
\end{Formulae}
\Cref{equ:cp-so-vis} can be proven in the same way as in \cref{sec:sound-complete-mr}.
We now prove \cref{equ:cp-vis-ar}.
Similar to \cref{sec:sound-complete-cc}, initially we take \( \txidsetrd'' \) to be an empty set.
By \cref{thm:view-vis-relation,equ:view-close-to-aexec}, there exists \( \txidsetrd'' \) such that
\( \txidset \cup \txidsetrd'' \) is closed under 
\( \left( (\SO \cup \WR[\aexec] ) ; \Refl(\RW[\aexec]) \cup \WW[\aexec] \right) \).
Now consider a transaction \( \txidrd \in \txidsetrd' \) and
assume a transaction \( \txid' \) such that 
\( \ToEdge{ \txid' | \left( (\SO \cup \WR[\aexec] ) ; \Refl(\RW[\aexec]) \cup \WW[\aexec] \right)  -> \txidrd } \).
Since \( \txidrd \) is a read-only transaction, thus
\( \ToEdge{ \txid' |  (\SO \cup \WR[\aexec] )  -> \txidrd } \)
and the rest proof is exactly the same as as in \cref{sec:sound-complete-cc}.
Last, \cref{equ:cp-inv-preserve} can be proven in the same way as in \cref{sec:sound-complete-mr,sec:sound-complete-ryw}.


The execution test \(\et[\CP]\) is complete with respect to \( \visaxioms[\CP] \).
By \cref{thm:cp-eauiv-spec}, it suffice to prove that it is complete with respect to the following definition,
\[ 
\Set{\lambda \aexec.  \AR[\aexec] ; \VIS[\aexec], \lambda \aexec \ldotp \SO }
\]
\COMPLETELET{\CP}
By the definition of \( \et[\CP]\), we prove \( \CanCommit[\CP]\), \( \ViewShift[\MR]\) and \( \ViewShift[\RYW]\) respectively.
Recall that
\begin{align*}
\left( (\SO \cup \WR[\aexec] ) ; \Refl(\RW[\aexec]) \cup \WW[\aexec] \right) ; \VIS[\aexec] 
        & \subseteq \left( \VIS[\aexec] ; \Refl(\RW[\aexec]) \cup \AR[\aexec] \right) ;  \VIS[\aexec]
        & \subseteq  \AR[\aexec] ; \VIS[\aexec] \subseteq \VIS[\aexec] .
\end{align*}
Then \( \CanCommit[\CP]\) can be derived from \cref{thm:view-vis-relation,equ:aexec-close-to-view}.
The predicate \( \ViewShift[\RYW] \) can be proven in the same way as in \cref{sec:sound-complete-ryw}.
Since \( \VIS[\aexec] ; \SO \subseteq \AR[\aexec] ; \VIS[\aexec] \subseteq \VIS[\aexec] \)
\( \ViewShift[\MR] \) can be proven in the same way as in \cref{sec:sound-complete-mr}.
