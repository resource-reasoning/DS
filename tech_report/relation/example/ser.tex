The execution test $\et[\SER]$ is sound with respect to the axiomatic definition 
\( \visaxioms[\SER] \FuncDef \Set{\lambda \aexec \ldotp \AR } \).
We pick the invariant as \( \aexecinv[\SER]( \aexec, \cl ) = \emptyset \).
\SOUNDLET{\SER}{ \txidsetrd =
\begin{multlined}[t]
\left( \bigcup_{\Set{\txid[\cl](\idx) | \txid[\cl](\idx) \in \aexec}} 
\VISInv[\aexec](\txid[\cl](\idx)) \cup \Refl((\Inv(\SO)))(\txid[\cl](\idx)) \right) 
\setminus \Set{\txid' | \Forall{l | \key | \val } (l,\key,\val) \in \aexec(\txid') \implies l = \opR } .
\end{multlined} }
It is easy to see that \( \ARInv[\aexec'] (\txid) \).

\COMPLETELET{\SER}
Since \( \AR[\aexec] = \VIS[\aexec]\) thus it must be the case that \( \vi \) includes all the versions.
