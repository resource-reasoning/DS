The execution test \(\et[\MR]\) is sound with respect to the axiomatic definition
\(\visaxioms[\MR] \FuncDef \Set{\lambda \aexec. \VIS[\aexec] ; \SO })\)  \cite{repldatatypes}.
By \cref{def:aexec-invariant-for-clients,def:et-sound-constructor},
we choose the invariant as the following,  
\[
    \aexecinv[\MR](\aexec, \cl) \FuncDef
    \begin{multlined}[t]
    \left( \bigcup_{\Set{\txid[\cl](\idx) | \txid[\cl](\idx) \in \aexec}} \VISInv[\aexec](\txid[\cl](\idx)) \right) 
    \setminus \Set{\txid | \Forall{l | \key  | \val } (l,\key,\val) \in \aexec(\txid) \implies l = \opR } .
    \end{multlined}
\]

\SOUNDLET{\MR}{ \txidsetrd = 
\begin{multlined}[t]
\left( \bigcup_{\Set{\txid[\cl](\idx) | \txid[\cl](\idx) \in \aexec}} \VISInv[\aexec](\txid[\cl](\idx)) \right) 
\cap \Set{\txid' | \Forall{l | \key | \val } (l,\key,\val) \in \aexec(\txid') \implies l = \opR } .
\end{multlined} }
\begin{enumerate}
\Case{\(\Forall{\visaxiom \in \visaxioms[\MR] }
            \Inv(\visaxiom(\aexec'))(\txid) \subseteq \txidset \cup \txidsetrd \)}
    By the definition of \( \visaxioms[\MR]\), suppose transactions \( \txid',\txid'' \) such that
    \( \ToEdge{\txid' | \VIS[\aexec'] -> \txid'' | \SO -> \txid} \).
    Note that \( \txid', \txid'' \in \aexec\).
    Consider \( \txid' \).
    If \( \txid' \) is read-only transaction, \( \txid' \in \txidsetrd \).
    If \( \txid' \) has write then \( \txid' \in \aexecinv[\MR](\aexec,\cl) \).
    Since \( \aexecinv[\MR](\aexec,\cl) \subseteq \txidset \), therefore \( \txid' \in \txidset \).
\Case{\(\aexecinv[\MR](\aexec', \cl) \subseteq \VisTrans(\XToK(\aexec'), \vi')\)}
    Because \(\ToET[\MR]{ \kvs |  \vi | \fp | \kvs' | \vi' } \), 
    it must be the case that \(\vi \vileq \vi'\) and thus
    \(\txidset = \VisTrans(\kvs, \vi) = \VisTrans(\kvs', \vi) \subseteq \VisTrans(\kvs', \vi')\).
    Also, because \( \kvs = \XToK(\aexec)\) and \( \kvs' = \XToK(\aexec') \), 
    thus \(\txidset \subseteq \VisTrans(\XToK(\aexec'), \vi') \).
    Note that 
    \(\Set{\txid[\cl](\idx) | \txid[\cl](\idx) \in \aexec'} 
            = \Set{\txid[\cl](\idx) | \txid[\cl](\idx) \in \aexec} \cup \txid\).
    %Therefore, for any \(\txid[\cl](n) \in \aexec\), 
    %we have that \(\VIS^{-1}_{\aexec'}(\txid_{\cl}^{n}) = \VIS^{-1}_{\aexec}(\txid_{\cl}^{n})\), and that 
    %\(\VIS_{\aexec'}^{-1}(\txid) = \Tx[\mkvs, \vi] \cup \txidset_{\mathsf{rd}}\). 
    Last, we obtain 
    \begin{align*}
        \aexecinv[\MR](\aexec', \cl) 
        & = \begin{multlined}[t] 
                \left( \bigcup_{\Set{\txid[\cl](\idx) | \txid[\cl](\idx) \in \aexec'}} 
                            \VISInv[\aexec'](\txid[\cl](\idx)) \right) 
                \setminus \Set{\txid | \Forall{l | \key | \val } 
                                (l,\key,\val) \in \aexec'(\txid) \implies l = \opR } 
        \end{multlined}
        \\ & = \begin{multlined}[t] 
                \left( \bigcup_{\Set{\txid[\cl](\idx) | \txid[\cl](\idx) \in \aexec}} 
                            \VISInv[\aexec](\txid[\cl](\idx)) \right) \cup \txidset
                \\ {} \setminus \Set{\txid | \Forall{l | \key | \val } 
                                (l,\key,\val) \in \aexec(\txid) \implies l = \opR } 
        \end{multlined}
        \\ & = \aexecinv[\MR](\XToK(\aexec),\vi) \cup \txidset 
        \\ & = \txidset \subseteq \VisTrans(\XToK(\aexec'), \vi')
    \end{align*}
\end{enumerate}

\COMPLETELET{\MR}
Consider if there are more transaction from the client \( cl \).
\begin{enumerate}
\Case{\( (\txid[\cl](n), \txid') \in \SO \) for \( \txid' \in \aexec \)}
    Let the transaction 
    \( \txid = \Min[\SO](\Set{ \txid' | (\txid[\cl](n), \txid') \in \SO \land \txid' \in \aexec' }) \).
    For this case, let view 
    \( \vi' = \GetView(\aexec, \Refl((\ARInv[\aexec]))(\txid[\cl](\idx)) \cap \VISInv[\aexec](\txid)) \).
    By \( \visaxioms[\MR] \), it follows that 
    \( \txidset = \VISInv[\aexec](\txid[\cl](n)) \subseteq \VISInv[\aexec](\txid) \),
    and therefore \( \vi \vileq \vi' \) which implies \( \et[\MR] \).
\Case{\( \neg \left((\txid[\cl](n), \txid') \in \SO \right) \)}
For this case, let view 
\( \vi' = \GetView(\aexec, \Refl((\ARInv[\aexec]))(\txid[\cl](\idx))) \);
this trivially implies \( \vi \vileq \vi' \) and thus \( \et[\MR] \). 
\end{enumerate}
