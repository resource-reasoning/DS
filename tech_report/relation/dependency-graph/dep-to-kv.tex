Next, we show how to convert a dependency graph \(\dgraph\) to a kv-store via function \( \DToK\).
%Given a key \(\key\), each write operation, \(\opW(\key,\val) \), of a transaction, \( \txid \),
%corresponds to a version comprising a value \(\val \) and the writer being \( \txid \)
%and the reader set containing all transactions, \( \txid' \), that read the version,
%that is, \( (\txid,\txid') \in \WR(\key) \).
%The write-write relation \(\WW[\dgraph](\key)\) determines the order of versions of \( \key \).

\begin{definition}[Dependency graphs to kv-stores]
\label{def:dependency-graph-kv-store}
Given a dependency graph \(\dgraph\), and a transaction \( \txid \) that writes a key \(\key\),
the \emph{version} for \( \key \) written by \( \txid \) is defined by:
\[ 
\VerOf(\dgraph, \key, \txid) \FuncDef 
    \CodeFont{let} \ \opW(\val,\key) \in \dgraph(\txid) 
        \ \CodeFont{in} \ \Tuple{\val,\txid,\Set{\txid' | (\txid, \txid') \in \WR[\dgraph](\key) }}.
\]
The \emph{version list} for \( \key \), written \VerListOf(\dgraph,\key), is defined by:
\begin{align*}
\VerListOf(\dgraph,\key) & \FuncDef
\begin{multlined}[t]
\Let \txidset = \Set{\txid,\txid' | (\txid,\txid') \in \WW[\dgraph] } \In 
    \\ \List{\VerOf(\dgraph, \key, {\Func{WWnth}(\dgraph,\txidset,0)}), \cdots, \VerOf(\dgraph, \key, {\Func{WWnth}(\dgraph,\txidset,\Abs{\txidset} - 1)})}
\end{multlined}
\\ \Func{WWnth}(\dgraph,\txidset,0) & \FuncDef \txid \ \text{where} \ 
            \Forall{\txid' \in \txidset} \txid = \txid' \lor (\txid,\txid') \in \WW[\dgraph]
\\ \Func{WWnth}(\dgraph,\txidset,n + 1) & \FuncDef \Func{WWnth}(\dgraph,\txidset \setminus \Set{\txid} ,n) \ \text{where} \ 
            \Forall{\txid' \in \txidset} \txid = \txid' \lor (\txid,\txid') \in \WW[\dgraph]
\end{align*}
The kv-store induced by \( \dgraph \), written \( \DToK(\dgraph) \), is defined by:
\(
    \DToK(\dgraph) = \lambda \key \in \Keys \ldotp \VerListOf(\dgraph,\key).
\)
\end{definition}

Given a key \( \key \) and a transaction \( \txid \) that writes \( \key \),
the version \( \VerOf(\dgraph,\key,\txid) \) comprises the value of \( \key \) written by \( \txid \),
the writer being \( \txid \), and the reader set that contains all transactions \( \txid' \)
that read from \( \txid \), that is, \( (\txid,\txid') \in \WR \).
Given all versions of \( \key \), function \( \VerListOf(\dgraph,\key) \)
determines the order over the versions with respect to \( \WW(\key) \) relation.
Recall that if \( (\txid,\txid') \in \WW[\dgraph](\key) \), then \( \txid'\) committed after \( \txid \).
This means that the version written by \( \txid' \) 
precedes the version written by \( \txid \) in the list of \( \VerListOf(\dgraph,\key) \).
It is straightforward that \( \DToK(\dgraph) \) satisfies 
the well-formed conditions for kv-store in \cref{def:well-formed-kv-store}.
More detail is given in \cref{prop:dtok-well-defined} on page \pageref{sec:proof-dep-graph-to-kvs}.

\begin{toappendix}
\label{sec:proof-dep-graph-to-kvs}
\begin{proposition}[Well-defined \(\DToK\)]
\label{prop:dtok-well-defined}
Given a dependency graph \( \dgraph \in \DependencyGraphs \), 
the kv-store induced by dependency graph, \( \DToK(\dgraph) \), is well-formed,
that is, \( \DToK(\dgraph) \in \KVSs\).
\end{proposition}
\begin{proof}
Let \( \kvs = \DToK(\dgraph) \).
Given a key \( \key \), by the well-formedness definition for kv-store,
it is sufficient to prove the following cases.
\begin{enumerate}
\item \label{item:well-form-dtok-read-read} Suppose two indexes \( \idx, \idx' \) 
    such that \( \RsOf(\kvs(\key,\idx)) \cup \RsOf(\kvs(\key,\idx')) \neq \emptyset \).
    Assume \( \txid \in \RsOf(\kvs(\key,\idx)) \cup \RsOf(\kvs(\key,\idx)) \).
    There must exist edges
    \[ (\WtOf(\kvs(\key,\idx)),\txid), (\WtOf(\kvs(\key,\idx')),\txid) \in \WR[\dgraph]. \]
    Because \dgraph is well-formed (\cref{equ:dgraph-wr-unique}), 
    \( \WtOf(\kvs(\key,\idx)) = \WtOf(\kvs(\key,\idx')) \) thus \( \idx = \idx' \).
\item Suppose two indexes \( \idx, \idx' \) 
    such that \( \WtOf(\kvs(\key,\idx)) = \WtOf(\kvs(\key,\idx')) \).
    We prove \( \idx = \idx' \) by contradiction.
    Assume \( \idx \neq \idx' \).
    Without losing generality, this means that \( (\WtOf(\kvs(\key,\idx)),\WtOf(\kvs(\key,\idx'))) \in \WW[\dgraph]\);
    By well-formedness of \dgraph (\cref{equ:dgraph-ww-irreflexive}), it cannot be true.
    Thus we have \( \idx = \idx' \).
    Combine \cref{item:well-form-dtok-read-read}, we prove \cref{equ:kvs-wf-txid-snapshot-property}.
\item Given \cref{equ:dgraph-ww-init} we know that \( (\txidinit, \txid) \in \WW[\dgraph]\) 
    for all \( \txid \in \dgraph \land \txid \neq \txidinit \);
    and \( \opW(\key,\valinit) \in \dgraph(\txidinit) \).
    By the definition of \( \VerListOf \), \( \kvs(\key,0) = (\valinit,\txidinit,\stub) \),
    which implies \cref{equ:kvs-wf-init-version}.
\item Suppose an index \( \txid, \txid' \) such that 
    such that \( \txid = \WtOf(\kvs(\key,\idx)) \) and \( \txid' \in \RsOf(\kvs(\key,\idx)) \)
    for some index \( \idx \).
    This means \( (\txid,\txid') \in \WR[\dgraph] \).
    By \cref{equ:dgraph-wr-so}, it follows \( (\txid,\txid') \notin \Refl(\SO)\) 
    which implies \cref{equ:kvs-wf-so-wr}.
\item Suppose an index \( \txid, \txid' \) such that 
    such that \( \txid = \WtOf(\kvs(\key,\idx)) \) and \( \txid' = \WtOf(\kvs(\key,\idx')) \)
    for some indexes \( \idx, \idx' \) with \( \idx < \idx' \).
    This means \( (\txid,\txid') \in \WW[\dgraph] \).
    By \cref{equ:dgraph-ww-so}, it follows \( (\txid,\txid') \notin \SO\) 
    which implies \cref{equ:kvs-wf-so-ww}. \qedhere
\end{enumerate}
\end{proof}
%\begin{proofsketch}
%It is straightforward that \( \DToK(\dgraph) \) satisfies 
%the well-formed conditions for kv-store in \cref{def:well-formed-kv-store}.
%The full proof is given in \cref{sec:proof-dep-graph-to-kvs} on page \pageref{sec:proof-dep-graph-to-kvs}.
%\end{proofsketch}
\end{toappendix}

We now show the bijection between our kv-stores and dependency graphs.
This is a key intermediate step to link our global kv-stores to abstract executions.

\begin{theorem}[Bijection between kv-stores and dependency graphs]
\label{thm:isomorphism-kvstore-dgraph}
There is a bijection between kv-stores and dependency graphs.
\end{theorem}
\begin{proof}
Because \KToD(\kvs) and \DToK(\dgraph) are well-defined respectively 
(detail is given in \cref{prop:ktod-well-defined,prop:dtok-well-defined}).
It remains to prove \( \kvs = \DToK(\KToD(\kvs)) \) and \( \dgraph = \KToD(\DToK(\dgraph)) \).
\begin{enumerate}
\Case{\kvs = \DToK(\KToD(\kvs))}
    Let \( \dgraph  = \KToD(\kvs) \).
    Fix a key \( \key \) and 
    \[ \kvs(\key) = \List{(\valinit,\txidinit,\txidset_0), \cdots,(\val_n,\txid_n,\txidset_n) }. \]
    By the definition of \(\KToD\), it follows that:
    \begin{enumerate*} \item \( \opW(\key,\val_i) \in \dgraph(\txid_i) \);
    \item for all \( \txid \in \txidset_i \), \( \opR(\key,\val_i) \in \dgraph(\txid) \); and
    \item \( \WW[\dgraph] = \WW[\kvs] \) and \( \WR[\dgraph] = \WR[\kvs] \).
    \end{enumerate*}
    Then by definition of \DToK, specifically \( \VerListOf \), 
    it is easy to see \( \DToK(\dgraph)(\key) = \VerListOf(\dgraph,\key) = \kvs(\key) \).
\Case{\dgraph = \KToD(\DToK(\dgraph))}
    Let \( \kvs  = \DToK(\dgraph) \).
    First, we prove \( \txidop[\dgraph] = \txidop[\KToD(\kvs)] \).
    Consider a transaction \( \txid \) and an operation \( \op \) such that \( \op \in \dgraph(\txid)\).
    \begin{enumerate}
    \Case{\(\op = \opR(\key,\val)\)}
        Because \dgraph is well-formed, there exists a transaction \( \txid' \) such that
        \( (\txid',\txid) \in \WR[\dgraph](\key) \).
        By definition of \DToK,
        there exists an index \( \idx \) and a reader set \( \txidset \) such that 
        \( 
        \op \in \dgraph(\txid) \iff \kvs(\key,\idx) = (\val,\txid',\txidset) \land \txid \in \txidset .
        \)
        Then by definition of \KToD, we have \( \op \in \KToD(\kvs)(\txid)\).
        This means that \( \op \in \dgraph(\txid) \iff \op \in \KToD(\kvs)(\txid)\).
    \Case{\(\op = \opW(\key,\val)\)}
        Because \dgraph is well-formed, that is, \WW[\dgraph](\key) is 
        a total order on all transactions that wrote \( key \);
        by definition of \DToK,
        there exists an index \( \idx \) and a reader set \( \txidset \)  such that 
        \( 
        \op \in \dgraph(\txid) \iff \kvs(\key,\idx) = (\val,\txid',\txidset) .
        \)
        By definition of \KToD, we have \( \op \in \KToD(\kvs)(\txid)\).
        This means that \( \op \in \dgraph(\txid) \iff \op \in \KToD(\kvs)(\txid)\).
    \end{enumerate}
    Second, we prove \( \WR[\dgraph] = \WR[\KToD(\kvs)]\) and  \( \WW[\dgraph] = \WW[\KToD(\kvs)]\).
    consider a key \( \key \) and the write-write relation 
    on all transactions \( \txid_0, \cdots, \txid_n \) that write \( \key \),
    \( \ToEdge{\txid_0 | \WW[\dgraph](\key) -> \cdots | \WW[\dgraph](\key) -> \txid_n } \).
    Consider \( \txidset_i \) for \( 0 \leq i < n\) such that
    \( \Forall{\txid \in \txidset_i} (\txid_i,\txid) \in \WR[\dgraph] \).
    By the definition of \DToK,
    \( \kvs(\key) = \DToK(\dgraph)(\key) = 
        \List{(\valinit,\txidinit,\txidset_0), \cdots,(\val_n,\txid_n,\txidset_n) } \).
    Now by the definition of \( \WW[\kvs] \) and \( \WR[\kvs] \), 
    we have \( \WR[\dgraph](\key) = \in \WR[\kvs](\key) \)
    and \( \WW[\dgraph](\key)  = \WW[\kvs](\key). \)
    %for \( \idx < \idx'\) and \( \txid \in \txidset_i\);
    By definition of \KToD, then we have \( \WW[\kvs] = \WW[\KToD(\kvs)]\) and \(  \WR[\kvs] = \WR[\KToD(\kvs)] \),
    it follows \(  \WW[\dgraph] = \WW[\KToD(\DToK(\dgraph))]\) and \(  \WR[\dgraph] = \WR[\KToD(\DToK(\dgraph))] \).
    Last, because \( \RW[\dgraph] \) relation can be derived from \( \WW[\dgraph] \) and \( \WR[\dgraph] \),
    it is easy to see \( \RW[\dgraph] = \RW[\KToD(\kvs)] \). \qedhere
\end{enumerate}
\end{proof}
