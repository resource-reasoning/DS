The command semantics describes transitions of the form
\(
    \ToCmd{\kvs | \vi | \stk | \cmd | \lbCl{\stub} -> \kvs' | \vi' | \stk' | \cmd' }
\),
stating that given the kv-store \(\kvs\), client view \(\vi\) and client-local stack \(\stk\), 
the client \(\cl\) may execute command \(\cmd\) for one step,
updating the kv-store to \(\kvs'\), the client-local stack to \(\stk'\), and the command to its continuation \(\cmd'\).
The rules for commands are given in \cref{fig:command-semantics}.
The rules for compound commands are standard.
The only non-standard is \( \rCAtomicTrans \). 

Each transition step is labelled with the client \( \cl \) and the information about the transition,
which is either of the form \(\lbPri\) denoting that \(\cl\) executes a primitive command
that requires no access to \(\kvs\), 
or \(\lbTrans{\vi,\fp}\) denoting that \(\cl\) commits an atomic transaction 
with final fingerprint \(\fp\) under the view \(\vi\).

\begin{definition}[Command semantics labels]
The set of kv-store semantics labels, \( \PLabels \ni \lb \),
is defined by:
\[ 
\begin{multlined}
\Labels \TypeDef \Set{\lbPri | \cl \in \Clients } 
    \uplus
    {} \\ \Set{\lbTrans{\vi,\fp} | \Exists{\kvs \in \KVSs} \cl \in \Clients \land \fp \in \Fingerprints \land \vi \in \Views(\kvs)} 
\end{multlined}
\]
\end{definition}

The semantics is parametric in the choice of \emph{execution test} \(\et\), 
which is used to generate the \emph{consistency model} 
on kv-stores under which a transaction can execute.
In \cref{sec:sound-complete}, we give many examples of 
execution tests for well-known consistency models.
In \cref{sec:sound-complete-et}, 
we prove that our definitions of consistency models using execution tests
are equivalent to the existing declarative definitions of consistency models. 

In \cref{fig:command-semantics}, the rule for primitive commands, \(\rCPrimitive\), 
depends on the transition relation, \(\ToCmdPri{\cmdpri} \subseteq \Stacks \times \Stacks\),
that simply describes how the primitive command \(\cmdpri\) affects a client local stack.
As the rule \( \rCPrimitive \) only changes the client local state, 
this rule is labelled with \( \lbPri \).

\begin{figure}
\[
    \ToCmd*{} 
    \IsTyped 
    \begin{gathered}[t]
    (\KVSs \times \Views  \times \Commands)
    \times \PLabels \times
    (\KVSs \times \Views \times \Commands)
     \end{gathered}
\]

\begin{mathpar}
\inferrule[\rCAtomicTrans]{%
    \vi \vileq \vi'' 
    \\ \snap = \Snapshot(\kvs,\vi'')
    \\\\ \ToTrans{\stk | \snap | \emptyset| \trans  | * -> \stk' | \snap' | \fp | \pskip }
    \\ \CanCommit[\et](\kvs,\vi'',\fp)
    \\\\ \txid \in \NextTxid(\cl, \kvs)
    \\ \kvs' = \UpdateKV(\kvs, \vi'', \fp, \txid)
    \\ \ViewShift[\et](\kvs,\vi'',\kvs',\vi')
}{%
    \ToCmd{ \kvs | \vi | \stk | \ptrans{\trans} | \lbTrans{\vi'',\fp} -> 
    \kvs' | \vi' | \stk' | \pskip }
}%
\and
\inferrule[\rCPrimitive]{
	\stk \ToCmdPri{\cmdpri} \stk'
}{
    \ToCmd{ \kvs | \vi | \stk | \cmdpri | \lbPri -> \kvs | \vi | \stk' | \pskip }
}%
\and
\inferrule[\rCChoice]{%
    \idx\in \Set{1,2}
}{%
    \ToCmd{ \kvs | \vi | \stk | \cmd_{1} \pchoice \cmd_{2} | \lbPri -> \kvs | \vi | \stk | \cmd_{\idx} }
}
\and
\inferrule[\rCIter]{ }{%
    \ToCmd{ \kvs | \vi | \stk | \prepeat(\cmd) | \lbPri ->
    \kvs | \vi | \stk | \pskip \pchoice (\cmd \pseq \prepeat(\cmd)) }
}
\and
\inferrule[\rCSeqSkip]{ }{%
    \ToCmd{ \kvs | \vi | \stk | \pskip \pseq \cmd | \lbPri -> \kvs | \vi | \stk | \cmd }
}
\and
\inferrule[\rCSeq]{% 
    \ToCmd{ \kvs | \vi | \stk | \cmd_{1} | \lb -> \kvs' | \vi' | \stk' | \cmd'_{1} }
}{%
    \ToCmd{ \kvs | \vi | \stk | \cmd_{1} \pseq \cmd_{2} | \lb 
            -> \kvs' | \vi' | \stk' | \cmd'_{1} \pseq \cmd_{2} }
}
\end{mathpar}

\hrulefill 

\caption{Operational semantics of sequential client commands parametrised by \( \et \)}
\label{fig:command-semantics}
\end{figure}


\begin{definition}[Primitive command transition relation]
Given a primitive command \( \cmdpri \in \Commands\), 
the \emph{transition relation} over client local stacks,
\( \ToCmdPri{\cmdpri} \IsTyped \Stacks \times \Stacks \),
is defined by:
\begin{align*}
  \stk & \ToCmdPri{\passign{\var}{\expr}} \stk\UpdateFunc{\var -> \EvalE{\expr}} ,
       & \stk & \ToCmdPri{\passume(\bool)} \stk \ \text{if} \ \EvalB{\bool}= \true .
\end{align*}
\end{definition}

\Cref{fig:command-semantics} gives the operational semantics of commands.
The rules for compound commands are standard.
Rules \rCChoice, \rCIter and \rCSeqSkip are associated with primitive command label \lbPri
because they only affect the commands.
The label for \rCSeq is the same as the premise.

The only interesting rule is  the \(\rCAtomicTrans\) rule,
which describes the execution of an atomic transaction under the execution test \(\et\). 
The first premise, \( \vi \vileq \vi'' \), states that the current client view \(\vi\)
of the executing command may be advanced to a newer view \(\vi''\).
This premise captures that a client, prior to committing a transaction, receives synchronisation messages.
Given the new view \(\vi''\), the transaction obtains the snapshot \(\snap\) of the kv-store \(\kvs\), 
and executes \(\trans\) locally to completion (\(\pskip\)), 
updating the stack to \(\stk'\) while accumulating the fingerprint \(\fp\).
These steps are given by the second, \( \snap = \Snapshot(\kvs,\vi'') \),
and the third premise,
\( \ToTrans{\stk | \snap | \emptyset| \trans  | *
        -> \stk' | \snap' | \fp | \pskip }  \), of \( \rCAtomicTrans \).
Note that the resulting snapshot \( \snap' \) is ignored 
as the effect of the transaction is recorded in the fingerprint \(\fp\). 
This is because we focus on consistency models satisfying snapshot property,
hence all the intermediate steps of a transaction are not observable to other transactions.
Prior to commit, the \(\CanCommit[\et](\kvs,\vi'',\fp)\) premise ensures that under the execution test \(\et\), 
the final fingerprint \(\fp\) of the transaction is compatible with the (original) kv-store
\(\kvs\) and the client view \(\vi''\), which is used to take the initial snapshot,
and thus the transaction \emph{can commit}. 
Note that the \(\CanCommit\) check is parametrised by the execution test \(\et\).
This is because the conditions checked upon committing 
depend on the consistency model under which the transaction is to commit. 
In \cref{sec:consistency-model-on-kv}, we define \( \CanCommit \)
for several execution tests associated with well-known consistency models.

Now the client \(\cl\) is ready to actually commit the transaction.
This results in a new kv-store \(\kvs'\) and the client view \(\vi''\) \emph{shifts} to a new view \(\vi'\): 
\begin{enumerate}
    \item pick a fresh transaction identifier \(\txid \in \NextTxid(\cl, \kvs)\) that is greater than any previously used identifiers;
    \item compute the new kv-store via \(\kvs' = \UpdateKV(\kvs, \vi'', \fp, \txid)\); and 
    \item shift the view to \( \vi' \), checking if it is permitted under execution test \(\et\) 
            using \(\ViewShift[\et](\kvs,\vi'',\kvs',\vi')\). 
\end{enumerate}

\begin{definition}[Fresh transaction identifiers]
\label{def:next-txid}
Given a kv-store \( \kvs \in \KVSs \), 
the set of \emph{next available transaction identifiers} for a client \( \cl \),
written \( \NextTxid(\cl,\kvs)\),
is defined by:
\[ 
    \NextTxid(\cl,\kvs) \FuncDef 
    \Set{ \txid[\cl](n) | \txid[\cl](n) \in \TxIDs 
            \land \Forall{m \in \Nat | \txid[\cl](m) \in \TxIDs}
            \txid[\cl](m) \in \kvs 
            \implies m < n }. 
\]
\end{definition}

The fresh transaction identifier for a client \( \cl \) must be strictly greater than 
all existing identifiers in the kv-store for the same client \( \cl \).
This ensures that 
\( \SO \) can be determined by the transaction identifiers.
%the new transaction is \( \SO \)-after all previous transactions for \( \cl \).

\begin{definition}[Kv-store update]
\label{def:update-kvstore}
Given a kv-store \( \kvs \in \KVSs \),
a view \( \vi \in \Views(\kvs) \), a fingerprint \( \fp \in \Fingerprints \) 
and a fresh transaction identifier \( \txid \in \TxIDs\),
the new kv-store, \( \UpdateKV(\kvs, \vi, \fp, \txid ) \), is defined by:
\begin{align*}
\UpdateKV(\kvs, \vi, \emptyset, \txid) & \FuncDef \kvs ,
\\ \UpdateKV(\kvs, \vi, \Set{\opR(\key, \val)} \uplus \fp, \txid)
        & \FuncDef 
        \begin{multlined}[t]
        \CodeFont{let} \ i = \Max[<](\vi(\key)) \ \CodeFont{and} \ (\val,\txid',\txidset) = \kvs(\key,i) \ \CodeFont{in} 
        \\ \CodeFont{let} \ \verlist = \kvs(\key)\UpdateFunc{i -> (\val, \txid', \txidset \cup \Set{\txid})} \ \CodeFont{in}
        \\ \UpdateKV(\kvs\UpdateFunc{\key -> \verlist},\vi, \fp, \txid)  ,
        \end{multlined} 
\\ \UpdateKV(\kvs, \vi, \Set{\opW(\key, \val)} \uplus \fp, \txid)
        & \FuncDef 
        \begin{multlined}[t]
        \CodeFont{let} \; \verlist = \kvs(\key) \ListConcat \List{(\val, \txid, \emptyset)} \; \CodeFont{in} \;
        \UpdateKV(\kvs\UpdateFunc{\key -> \verlist}, \vi, \fp, \txid)
        \end{multlined} .
\end{align*}
\end{definition}

The function \(\UpdateKV(\kvs, \vi, \fp, \txid)\)
describes how the fingerprint \(\fp\) of the transaction \(\txid\) 
is executed under the view \(\vi\) updates the kv-store \(\kvs\):
\begin{enumerate*}
    \item for each read \(\opR(\key, \val) \in \fp\), it adds \(\txid\) 
        to the reader set of the last version of \(\key\) in \(\vi\); and
    \item for each write \(\opW(\key, \val)\), 
        it appends a new version \((\val, \txid, \emptyset)\) to \(\kvs(\key)\). 
\end{enumerate*}

The function \( \UpdateKV \) is well-defined in that
for any \( \kvs,\vi,\fp,\txid,\cl \) such that \( \kvs \) and \( \fp \) are well-formed,
\( \vi \in \Views(\kvs) \) and \( \txid \in \NextTxid(\kvs,\cl) \),
the resulting kv-store \( \UpdateKV(\kvs,\vi,\fp,\txid) \) is a well-formed kv-store.

\subsection{Well-defined \( \updateKV \)}
\label{sec:updatekv-well-defined}
The \cref{lem:updatekv.explicit} and then \cref{cor:updatekv.singlecell} shows that
swapping the operations of one key yields the same result.

\begin{lemma}[Swapping Operation]
\label{lem:updatekv.explicit}
Let $\mkvs$ be a kv-store, $\vi \in \Views(\mkvs)$, $\txid \in \TxID$ and $\fp \in \pset{\Ops}$. 
Let also $\key \in \Keys$. Then
\begin{enumerate}
    \item\label{item:updatekv.explicit.none} 
        $\fora{\val} (\otR, \key, \val) \notin \fp \land (\otW, \key, \val) \notin \fp \implies \updateKV[\mkvs, \vi, \fp, \txid](\key) = \mkvs(\key)$
\item\label{item:updatekv.explicit.rd} 
    $\fora{\val} (\otR, \key, \stub) \in \fp \land (\otW, \key, \val) \notin \fp 
    \implies 
    \begin{array}[t]{@{}l@{}}
    \updateKV[\mkvs, \vi, \fp, \txid](\key) = \\
    \qquad \text{let} \ (\val', \txid', \txidset') = \mkvs(\key, \max<(\vi(\key))) \\
    \qquad \text{in} \ \mkvs(\key)\rmto{\max<(\vi(\key))}{(\val', \txid', \txidset' \cup \Set{\txid})}
    \end{array}
    $
\item\label{item:updatekv.explicit.wr} 
    $
    \begin{array}[t]{@{}l@{}}
    \fora{ \val, \val'}(\otR, \key, \val) \notin \fp \land (\otW, \key, \val) \in \fp 
    \implies \updateKV[\mkvs, \vi, \fp, \txid](\key) = \mkvs(\key) \lcat (\val, \txid, \emptyset)
    \end{array}
    $
\item\label{item:updatekv.explicit.rdwr}
    $
    \fora{\val} (\otR, \key, \stub) \in \fp \land (\otW, \key, \val) \in \fp 
    \implies 
    \begin{array}[t]{@{}l@{}}
    \updateKV[\mkvs,\vi,\fp,\txid](\key) =  \\
    \qquad \text{let} \ (\val', \txid', \txidset') = \mkvs(\key, \max(\vi(\key)))  \\
    \qquad \text{in} \ \mkvs(\key)\rmto{\max(\vi(\key))}{(\val', \txid', \txidset' \cup \Set{\txid}) } \lcat (\val, \txid, \emptyset)
    \end{array}
    $
\end{enumerate}
\end{lemma}

\begin{proof}
All the four statements are proved by induction on $\fp$, by keeping the variable $\mkvs$ universally quantified in the inductive hypothesis. 
Statement \cref{item:updatekv.explicit.rd} and \cref{item:updatekv.explicit.wr} requires 
proving \cref{item:updatekv.explicit.none} first, while Statement \cref{item:updatekv.explicit.rdwr} requires proving all the other statements. 
Fix then an arbitrary $\key \in \Keys$.
\begin{enumerate}
	\item 
	Suppose that for any $\val$, $(\otR, \key, \val) \notin \fp$ and $(\otW, \key, \val)) \notin \fp$. We prove that $\updateKV[\mkvs, \vi, \fp, \txid](\key) = 
	\mkvs(\key)$.
	\begin{itemize}
        \item \caseB{$\fp = \emptyset$} in this case we have that 
		\[
		\updateKV[\mkvs, \vi, \emptyset, \txid](\key) \stackrel{\cref{eq:updatekv}}{=} \mkvs(\key).
		\]
    \item  
        Suppose that $\fp = \fp' \uplus \Set{(\otR, \key', \val')}$ for some $\key', \val'$. Because we are assuming that 
		$(\otR, \key, \val) \notin \fp$ for any $\val \in \Val$, then it must be the case that 
		\begin{equation}
		\label{eq:updatekv.explicit.none.keneqkepRD}
		\key \neq \key'.
		\end{equation}
		Also, we have that $(\otR,\key, \val) \notin \fp'$ and $(\otW, \key, \val) \notin \fp$ for any $\val \in \Val$. 
		By inductive hypothesis we can assume 
		\begin{equation}
            \fora{ \mkvs' }\updateKV[\mkvs', \vi, \fp', \txid](\key) = \mkvs'(\key)
		\label{eq:updatekv.explicit.none.IHrd}
		\end{equation} 
		Therefore we have 
		\begin{align*}
        \updateKV[\mkvs, \vi, \fp, \txid](\key) 
            & =  
            \updateKV[\mkvs, \vi, \fp' \uplus \Set{(\otR, \key', \val')}, \txid](\key) \\
            & \stackrel{\cref{eq:updatekv}}{=}
            \begin{multlined}[t]
                \text{let} \ (\val', \txid', \txidset') = \mkvs(\key', \max_{<}(\vi(\key))) \\
                \text{in} \ \updateKV[\mkvs\rmto{\key'}{\mkvs(\key')\rmto{\max_{<}(\vi(\key'))}{\left(\val', \txid', \txidset' \cup \Set{\txid}\right)}}, \vi, \fp', \txid](\key) 
            \end{multlined} \\
            & \stackrel{\cref{eq:updatekv.explicit.none.IHrd}}{=}
            \begin{multlined}[t]
		    \text{let} \ (\val', \txid', \txidset') = \mkvs(\key', \max_{<}(\vi(\key'))) \\
            \text{in} \ \mkvs\rmto{\key'}{\mkvs(\key')\rmto{\max_{<}(\vi(\key'))}{\left(\val', \txid', \txidset' \Set{\txid}\right)}}(\key) 
            \end{multlined} \\
            &\stackrel{\cref{eq:updatekv.explicit.none.keneqkepRD}}{=} 
		    \text{let} \ (\val', \txid', \txidset') = \mkvs(\key', \max_{<}(\vi(\key'))) \text{ in } \mkvs(\key) \big) \\
            & = \mkvs(\key)
		\end{align*}

		\item Suppose that $\fp = \fp' \uplus \Set{(\otW, \key', \val')}$ for some $\val' \in \Val$. Then it must be the 
		case that 
		\begin{equation}
		\label{eq:updatekv.explicit.none.keneqkepWR}
		\key \neq \key'
		\end{equation}
		Also, we have that $(\otR,\key, \val) \notin \fp'$ and $(\otW, \key, \val) \notin \fp$ for any $\val \in \Val$. 
		By inductive hypothesis we can assume 
		\begin{equation}
            \fora{ \mkvs'}\updateKV[\mkvs', \vi, \fp', \txid](\key) = \mkvs'(\key)
		\label{eq:updatekv.explicit.none.IHwr}
		\end{equation}
		Therefore we have 
        \begin{align*}
            \updateKV[\mkvs, \key, \fp, \txid](\key)
            & =
            \updateKV[\mkvs, \key, \fp \uplus \Set{(\otW, \key', \val')}, \txid](\key) \\
            & \stackrel{\cref{eq:updatekv}}{=} 
            \updateKV[\mkvs\rmto{\key'}{\mkvs(\key')\lcat (\val', \txid, \emptyset)}, \vi, \fp, \txid ](\key)  \\
            &\stackrel{\cref{eq:updatekv.explicit.none.IHwr}}{=}
            \mkvs\rmto{\key'}{\mkvs(\key') \lcat (\val', \txid, \emptyset)}(\key) \\
            & \stackrel{\cref{eq:updatekv.explicit.none.keneqkepWR}}{=} \mkvs(\key)
		\end{align*}
	\end{itemize}

	\item Suppose $(\otR, \key, \stub) \in \fp$, and $(\otW, \key, \val) \notin \fp$ for all $\val \in \Val$. 
        Let $(\val, \txid', \txidset) = \mkvs(\key, \max_{<}(\vi(\key)))$. We prove that 
    \[
        \updateKV[\mkvs, \vi, \fp, \txid](\key) = \mkvs(\key)\rmto{\vi(\key)}{(\val, \txid', \txidset \cup \Set{\txid})}
    \]
		\begin{itemize}
        \item \caseB{$\fp = \emptyset$} this case is vacuous, as $(\otR, \key, \val) \notin \fp$ for all $\val \in \Val$, 
		against the assumption that $(\otR, \key, \stub) \in \fp$. 

		\item Suppose that $\fp = \fp' \cup \Set{(\otR, \key', \stub)}$ for some $\key'$. 
            We have two possible cases: 
			\begin{enumerate}
			\item $\key = \key'$, in which case we know that $(\otR, \key, \val') \notin \fp'$ for all $\val' \in \Val$ because of 
			the assumptions that we make on the structure of $\fp$. 
            By \cref{lem:updatekv.explicit}\cref{item:updatekv.explicit.none} we have that
			\begin{equation}
            \fora{ \mkvs' } \updateKV[\mkvs', \vi, \fp', \txid](\key) = \mkvs'(\key).
			\label{eq:updatekv.explicit.rd.applynone}
			\end{equation}
			In this case we have that 
            \begin{align*}
                \updateKV[\mkvs, \vi, \fp, \txid](\key) 
                & =
                \updateKV[\mkvs,\vi, \fp' \uplus \Set{(\otR, \key, \val')}, \txid](\key) \\
                & \stackrel{\cref{eq:updatekv}}{=} 
                \updateKV[\mkvs\rmto{\key}{\mkvs(\key)\rmto{\max_{<}(\vi(\key))}{(\val, \txid', \txidset \cup \Set{\txid})}}, \vi, \fp', \txid](\key) \\
                & \stackrel{\cref{eq:updatekv.explicit.rd.applynone}}{=}
                \mkvs\rmto{\key}{\mkvs(\key)\rmto{\max_{<}(\vi(\key))}{(\val, \txid', \txidset \cup \Set{\txid})}}(\key) \\
                & = \mkvs(\key)\rmto{\max(\vi(\key))}{(\val, \txid', \txidset \cup \Set{\txid})}
            \end{align*}
            \item \( \key \neq \key' \).
			In this case we know that because $(\otR, \key, \stub) \in \fp$, then 
            it must be $(\otR, \key, \stub) \in \fp'$. We also know that $\fora{ \val } (\otW, \key, \val) \notin \fp$. 
			By the inductive hypothesis, we have that 
			\begin{equation}
			\label{eq:updatekv.explicit.rd.IHrd}
            \begin{array}{l}
			\fora{ \mkvs' } \updateKV[\mkvs', \vi, \fp', \txid](\key) 
            = \mkvs'(\key)\rmto{\max_{<}(\vi(\key))}{(\val, \txid', \txidset \cup \Set{\txid})}
            \end{array}
			\end{equation}
			In this case we have 
			\begin{align*}
                \updateKV[\mkvs, \vi, \fp, \txid](\key) 
                & =
                \updateKV[\mkvs, \vi, \fp' \uplus \Set{(\otR, \key', \stub)}, \txid](\key) \\
                & \stackrel{\cref{eq:updatekv}}{=}
			    \updateKV[\mkvs\rmto{\key'}{\stub}, \vi, \fp, \txid](\key) \\
                & \stackrel{\cref{eq:updatekv.explicit.rd.IHrd}}{=} 
                \mkvs\rmto{\key'}{\stub}(\key)\rmto{\max_{<}(\vi(\key))}{(\val, \txid', \txidset' \cup \Set{\txid})} \\
                &\stackrel{\key \neq \key'}{=}
                \mkvs(\key)\rmto{\max_{<}(\vi(\key))}{(\val, \txid', \txidset' \cup \Set{\txid})}
			\end{align*}
		\end{enumerate}

		\item $\fp = \fp' \uplus \Set{(\otW, \key', \val')}$ for some $\val' \in \Val$. Because $(\otW, \key, \val) \notin \fp$ 
		for any $\val \in \Val$, it must be the case that 
		\begin{equation}
		\key \neq \key'
		\label{eq:updatekv.explicit.rd.keneqkepWR}
		\end{equation}
		Because $(\otR, \key, \stub) \in \fp$, it must also be the case that $(\otR, \key, \stub) \in \fp'$. By the inductive hypothesis, 
		we have that 
		\begin{equation}
		\label{eq:updatekv.explicit.rd.IHwr}
        \begin{array}{l}
        \fora{ \mkvs' }\updateKV[\mkvs', \vi, \fp', \txid](\key) 
        {} = \mkvs(\key)\rmto{\max_{<}(\vi(\key))}{(\val, \txid', \txidset \cup \Set{\txid})}
        \end{array}
		\end{equation}
		It follows that 
        \begin{align*}
		    \updateKV[\mkvs, \vi, \fp, \txid](\key) 
            & =
            \updateKV[\mkvs, \vi, \fp' \uplus \Set{(\otW, \key', \val')}, \txid](\key) \\
            & \stackrel{\cref{eq:updatekv}}{=}
		    \updateKV[\mkvs\rmto{\key'}{\stub}, \vi, \fp', \txid](\key) \\
            & \stackrel{\cref{eq:updatekv.explicit.rd.IHwr}}{=} 
            \mkvs(\rmto{\key'}{\stub}(\key)\rmto{\max_{<}(\vi(\key))}{(\val, \txid', \txidset \cup \Set{\txid})} \\
            & \stackrel{\cref{eq:updatekv.explicit.rd.keneqkepWR}}{=} 
            \mkvs(\key)\rmto{\max_{<}(\vi(\key))}{(\val, \txid', \txidset \cup \Set{\txid})}
        \end{align*}
	\end{itemize}
	
	\item Suppose that $(\otW, \key, \val) \in \fp$ for some $\val \in \Val$, and 
	$(\otR, \key, \val') \notin \fp$ for any $\val' \in \Val$. We prove that 
	$\updateKV[\mkvs, \vi, \fp, \txid](\key) = \mkvs(\key) \lcat (\val, \txid, \emptyset)$. 
		\begin{itemize}
        \item \caseB{$\fp = \emptyset$} This case is vacuous, as $(\otW, \key, \val) \in \fp$.
		\item Suppose that $\fp = \fp' \uplus \Set{(\otR, \key', \stub)}$ for some 
		$\key'$. Note that, because we are assuming that $\Set{(\otR, \key, \val')} \notin \fp$ 
		for all $\val' \in \Val$, then it must be the case that 
		\begin{equation}
		\key \neq \key'
		\label{eq:updatekv.explicit.wr.kenqkepRD}
		\end{equation}	
		We also have that $\Set{(\otR, \key, \val')} \notin \fp'$ for all $\val' \in \Val$, and 
		$(\otW, \key, \val) \in \fp'$. By the inductive hypothesis we have that 
		\begin{equation}
        \fora{\mkvs'} \updateKV[\mkvs', \vi, \fp', \txid](\key) = \mkvs'(\key) \lcat (\val, \txid, \emptyset)
		\label{eq:updatekv.explicit.wr.IHrd}
		\end{equation}
		Therefore, we have that 
        \begin{align*}
		    \updateKV[\mkvs, \vi, \fp, \txid]]\key) 
            & = 
            \updateKV[\mkvs, \vi, \fp' \uplus \Set{(\otR, \key', \stub)}, \txid](\key) \\
            & \stackrel{\cref{eq:updatekv}}{=}
		    \updateKV[\mkvs\rmto{\key'}{\stub}, \vi, \fp',\txid ](\key) \\
            & \stackrel{\cref{eq:updatekv.explicit.wr.IHrd}}{=} 
            \mkvs\rmto{\key'}{\stub}(\key) \lcat (\val, \txid, \emptyset) \\
            & \stackrel{\cref{eq:updatekv.explicit.wr.kenqkepRD}}{=} 
		    \mkvs(\key) \lcat (\val, \txid, \emptyset)
		\end{align*}
		
		\item Suppose that $\fp = \fp' \uplus \Set{(\otW, \key', \val')}$ 
		for some $\key'$. We distinguish two possible cases:
			\begin{enumerate}
			\item $\key = \key'$. In this case the structure of $\fp$ also imposes that $\val = \val'$, 
			and $(\otW, \key, \val'') \notin \fp'$ for any $\val'' \in \Val$. Furthermore, we have 
			that $(\otR, \key, \val'') \notin \fp'$ for any $\val'' \in \Val$. 
			By \cref{lem:updatekv.explicit}\cref{item:updatekv.explicit.none}, we have that 
			\begin{equation}
            \fora{\mkvs'} \updateKV[\mkvs', \vi, \fp', \txid](\key) = \mkvs(\key)
			\label{eq:updatekv.explicit.wr.applynone}
			\end{equation}
			from which it follows 
            \begin{align*}
			    \updateKV[\mkvs, \vi, \fp, \txid](\key) 
                & =
			    \updateKV[\mkvs, \vi, \fp' \uplus \Set{(\otW, \key', \val')}, \txid](\key) \\
                & = 
                \updateKV[\mkvs, \vi, \fp' \uplus \Set{(\otW, \key, \val)}, \txid](\key) \\
                & \stackrel{\cref{eq:updatekv}}{=} 
                \updateKV[\mkvs\rmto{\key}{\mkvs(\key) \lcat (\val, \txid, \emptyset)}, \vi, \fp', \txid](\key) \\
                & \stackrel{\cref{eq:updatekv.explicit.wr.applynone}}{=} 
                \mkvs\rmto{\key}{\mkvs(\key) \lcat (\val, \txid, \emptyset)}(\key) \\
                & = \mkvs(\key) \lcat (\val, \txid, \emptyset)
			\end{align*}
			
            \item \( \key \neq \key'\).
			In this case we have that, because $(\otW, \key, \val) \in \fp$, then it must 
			be $(\otW, \key, \val) \in \fp'$. Furthermore, we also have that $(\otR, \key, \val'') \notin \fp'$ 
			for any $\val'' \in \Val$. By the inductive hypothesis, we have that 
			\begin{equation}
            \fora{\mkvs'}\updateKV[\mkvs', \vi, \fp', \txid](\key) = \mkvs(\key) \lcat (\val, \txid, \emptyset)
			\label{eq:updatekv.explicit.wr.IHwr}
			\end{equation}
			from which it follows 
            \begin{align*}
			    \updateKV[\mkvs, \vi, \fp, \txid](\key) 
                & =
                \updateKV[\mkvs, \vi, \fp' \uplus \Set{(\otW,\key', \val')}, \txid](\key) \\
                & \stackrel{\cref{eq:updatekv}}{=}
			    \updateKV[\mkvs\rmto{\key'}{\stub}, \vi, \fp, \txid](\key) \\
                & \stackrel{\cref{eq:updatekv.explicit.wr.IHwr}}{=} 
                \mkvs\rmto{\key'}{\stub}(\key) \lcat (\val, \txid, \emptyset) \\
                & \stackrel{\key \neq \key'}{=} 
                \mkvs(\key) \lcat (\val, \txid, \emptyset)
			\end{align*}
			\end{enumerate}
		\end{itemize}
		
		\item Suppose that $(\otW, \key, \val) \in \fp$ for some $\val \in \Val$, and $(\otR, \key, \stub) \in \fp$. 
		Let $\mkvs(\key, \vi) = (\val', \txid', \txidset')$. We prove that 
        \[ 
            \updateKV[\mkvs, \vi, \fp, \txid](\key) = 
            \mkvs(\key)\rmto{\vi(\key)}{(\val', \txid', \txidset' \cup \Set{\txid})} \lcat (\val, \txid, \emptyset)
        \]
		by induction on $\fp$:
			\begin{itemize}
			\item $\fp = \emptyset$; this case is vacuous.
			\item $\fp = \fp' \uplus \Set{(\otR, \key', \stub)}$. We distinguish two cases, according to 
			whether $\key = \key'$ or $\key \neq \key'$. If $\key = \key'$, then we know that 
			$(\otW, \key, \val) \in \fp'$ and $(\otR, \key, \val'') \notin \fp$ for any $\val'' \in \Val$. 
			By Lemma \cref{lem:updatekv.explicit}\cref{item:updatekv.explicit.wr} we have that 
			\begin{equation}
            \fora{\mkvs'}\updateKV[\mkvs,\vi,\fp',\txid](\key) = \mkvs(\key) \lcat (\val, \txid, \emptyset)
			\label{eq:updateKV.explicit.rdwr.applyWR}
			\end{equation}
			from which it follows that 
			\begin{align*}
			    \updateKV[\mkvs, \vi, \fp, \txid](\key)
                & =
                \updateKV[\mkvs, \vi, \fp' \uplus \Set{(\otR, \key', \stub)}, \txid](\key) \\
                & = 
			    \updateKV[\mkvs, \vi, \fp' \uplus \Set{(\otR, \key, \stub)}, \txid](\key) \\
                & \stackrel{\cref{eq:updatekv}}{=}
                \updateKV[\mkvs\rmto{\key}{\mkvs(\key)\rmto{\max_{<}(\vi(\key))}{(\val', \txid', \txidset' \cup \Set{\txid})}}, \vi, \fp', \txid](\key) \\
                & \stackrel{\cref{eq:updateKV.explicit.rdwr.applyWR}}{=} 
                \mkvs\rmto{\key}{\mkvs(\key)\rmto{\max_{<}(\vi(\key))}{(\val', \txid', \txidset' \cup \Set{\txid})}}(\key) \lcat (\val, \txid, \emptyset) \\
                & = 
			    \mkvs(\key)\rmto{\max_{<}(\vi(\key))}{(\val' \txid', \txidset' \cup \Set{\txid}} \lcat (\val, \txid, \emptyset)
            \end{align*}
			If $\key \neq \key'$, then we have that both $(\otR, \key, \stub) \in \fp'$ and 
			$(\otW, \key, \val) \in \fp'$. In this case, by the inductive hypothesis we have that 
			\begin{equation}
			\label{eq:updatekv.explicit.rdwr.IHrd}
            \begin{array}{l}
            \fora{\mkvs'}\updateKV[\mkvs,\vi,\fp',\txid](\key) = 
            \mkvs'(\key)\rmto{\max_{<}(\vi(\key))}{(\val', \txid', \txidset' \cup \Set{\txid})} \lcat (\val, \txid, \emptyset)
            \end{array}
			\end{equation}
			from which it follows that 
            \begin{align*}
			    \updateKV[\mkvs, \vi, \fp,\txid](\key)
                & = 
                \updateKV[\mkvs, \vi, \fp' \uplus \Set{(\otR, \key', \stub)}, \txid](\key) \\
                & \stackrel{\cref{eq:updatekv}}{=}
			    \updateKV[\mkvs\rmto{\key'}{\stub}, \vi, \fp', \txid](\key) \\
                & \stackrel{\cref{eq:updatekv.explicit.rdwr.IHrd}}{=} 
                \mkvs\rmto{\key'}{\stub}(\key)\rmto{\max_{<}(\vi(\key))}{(\val', \txid', \txidset' \cup \Set{\txid})} \lcat (\val, \txid, \emptyset) \\
                & = 
                \mkvs(\key)\rmto{\max_{<}(\vi(\key))}{(\val', \txid', \txidset' \cup \Set{\txid})} \lcat (\val, \txid, \emptyset)
            \end{align*}
			
			\item $\fp = \fp' \uplus \Set{(\otW, \key'', \val'')}$ for some $\key'', \val''$. Again, 
			there are two possible cases to consider. If $\key = \key''$, then $\val = \val''$ because of the structure imposed on $\fp$.
			Furthermore, we have that 
			$(\otR, \key, \stub) \in \fp'$ and $(\otW, \key, \val''') \notin \fp$ for all $\val''' \in \Val$.
			By \cref{lem:updatekv.explicit}\cref{item:updatekv.explicit.rd} we have that 
			\begin{equation}
			\label{eq:updatekv.explicit.rdwr.applyRD}
            \begin{array}{l}
            \fora{\mkvs'}\updateKV[\mkvs', \vi, \fp', \txid](\key) =
            \mkvs'(\key)\rmto{\max_{<}(\vi(\key))}{(\val', \txid', \txidset' \cup \Set{\txid})} 
            \end{array}
			\end{equation}
			We have that 
            \begin{align*}
			    \updateKV[\mkvs,\vi,\txid, \fp](\key)
                & =
                \updateKV[\mkvs, \vi, \fp' \cup \Set{(\otW, \key'', \val'')}, \txid](\key) \\
                & =
			    \updateKV[\mkvs,\vi, \fp' \cup \Set{(\otW, \key, \val)}, \txid](\key) \\
                & \stackrel{\cref{eq:updatekv}}{=}
			    \updateKV[\mkvs\rmto{\key}{\mkvs(\key) \lcat (\val, \txid, \emptyset)}, \vi, \fp', \txid](\key) \\
                & \stackrel{\cref{eq:updatekv.explicit.rdwr.applyRD}}{=}
			    \mkvs\rmto{\key}{\mkvs(\key) \lcat (\val, \txid, \emptyset)}(\key)\rmto{\max_{<}(\vi(\key))}{(\val', \txid', \txidset' \cup \Set{\txid})} \\
                & =
			    \left(\mkvs(\key) \lcat (\val, \txid, \emptyset)\right)\rmto{\max_{<}(\vi(\key))}{(\val', \txid', \txidset' \cup \Set{\txid})} \\
                & = 
			    \mkvs(\key)\rmto{\max_{<}(\vi(\key))}{(\val',\txid', \txidset' \cup \Set{\txid})} \lcat (\val, \txid, \emptyset)
            \end{align*}
			Finally, if $\key \neq \key'$, then we have that $(\otR, \key, \stub) \in \fp'$ and $(\otW, \key, \val) \in \fp'$. 
			By the inductive hypothesis, we obtain 
			\begin{equation}
			\label{eq:updatekv.explicit.rdwr.IHwr}
            \begin{array}{l}
            \fora{\mkvs'}\updateKV[\mkvs', \vi, \fp', \txid](\key) = 
            \mkvs'(\key)\rmto{\max_{<}(\vi(\key))}{(\val', \txid', \txidset' \cup \Set{\txid})} \lcat (\val, \txid, \emptyset)
            \end{array}
			\end{equation}
			It follows that 
            \begin{align*}
			    \updateKV[\mkvs, \vi, \fp, \txid](\key)
                & =
                \updateKV[\mkvs, \vi, \fp' \uplus \Set{(\otW, \key', \stub)}, \txid](\key) \\
                & \stackrel{\cref{eq:updatekv}}{=} 
			    \updateKV[\mkvs\rmto{\key'}{\stub}, \vi, \txid, \fp'](\key) \\
                & \stackrel{\cref{eq:updatekv.explicit.rdwr.IHwr}}{=}
                \mkvs\rmto{\key'}{\stub}(\key)\rmto{\max_{<}(\vi(\key))}{(\val', \txid', \txidset' \cup \Set{\txid})} \lcat (\val, \txid, \emptyset) \\
                & =
                \mkvs\rmto{\max_{<}(\vi(\key))}{(\val', \txid', \txidset' \cup \Set{\txid})} \lcat (\val, \txid, \emptyset)
            \end{align*}
			\end{itemize}
\end{enumerate}
\end{proof}

In the following, given a version $\ver = (\val, \txid', \txidset)$ and a set of 
transaction identifiers $\txidset'$, we let $\ver \oplus \txidset' = (\val, \txid', \txidset \cup \txidset')$. 
Clearly the operator $\oplus$ is commutative over sets of transactions: 
$\fora{ \ver, \txidset, \txidset' } (\ver \oplus \txidset) \oplus \txidset' = (\ver \oplus \txidset') \oplus \txidset = 
\ver \oplus (\txidset \cup \txidset')$.

\begin{corollary}
\label{cor:updatekv.singlecell}
Let $\mkvs$ be a kv-store, $\vi \in \Views(\mkvs)$, $\txid \in \TxID$ and $\fp \in \pset{\Ops}$. 
Let also $\key \in \Keys$. Then 
\begin{enumerate}
\item\label{item:updatekv.singlecell.noview} 
    $ 
    \begin{array}[t]{l}
        \fora{ i } 0 \leq i < \abs{\mkvs(\key) } - 1 \land i \neq \max_{<}(\vi(\key)) 
        \implies \updateKV[\mkvs, \vi, \fp, \txid](\key, i) = \mkvs(\key, i)
    \end{array}
    $
\item\label{item:updatekv.singlecell.rd} $\fora{ \val } (\otR, \key, \stub) \in \fp \implies \updateKV[\mkvs,\vi, \fp,\txid](\key, \vi) = \mkvs(\key, \max_{<}(\vi(\key))) \oplus \Set{\txid}$
\item\label{item:updatekv.singlecell.nord} $\fora{ \val } (\otR,\key, \val) \notin \fp \implies \updateKV[\mkvs,\vi, \fp,\txid](\key,\vi) = \mkvs(\key, \max_{<}(\vi(\key)))$
\item\label{item:updatekv.singlecell.wr} 
    $\begin{array}[t]{@{}l@{}}
        \fora{\val} (\otW, \key, \val) \in \fp \\
        \quad {} \implies
        \lvert \updateKV[\mkvs,\vi,\fp,\txid](\key) \rvert = 
        \lvert \mkvs(\key) \rvert + 1 \wedge
        \updateKV[\mkvs,\vi,\fp,\txid](\key, \lvert \mkvs(\key) \rvert) = (\val, \txid, \emptyset)
    \end{array}$
\item\label{item:updatekv.singlecell.nowr} $\fora{ \val } (\otW, \key, \val) \notin \fp \implies \lvert \updateKV[\mkvs,\vi,\fp,\txid](\key) \rvert = \lvert \mkvs(\key) \rvert$
\end{enumerate}
\end{corollary}

\begin{proof}
A simple consequence of \cref{lem:updatekv.explicit}.
\end{proof}


Observe that as with \(\CanCommit\), 
the \(\ViewShift\) predicate is parametrised by the execution test \(\et\). 
Again, this is because the conditions checked 
for shifting the client view depend on the consistency model. 
In \cref{sec:consistency-model-on-kv}, together with  \(\CanCommit\), 
we define \(\ViewShift\) for several execution tests 
associated with well-known consistency models.

Instead of the \(\rCAtomicTrans\) rule,
it would be possible to separate the non-deterministic view  shift, \(  \vi \vileq \vi'' \) into a separate rule: that is
a client can always non-deterministically advance its view.
This choice models client non-deterministically receiving an update from the distributed system.
However, many real-world distributed systems only passively respond to client requests.
We choose the approach in our \(\rCAtomicTrans\) rule where clients update their views only when they commit a new transaction.

We focus on consistency models that satisfies \emph{snapshot property}.
For brevity, our semantics has the \emph{snapshot property} built in:
\begin{enumerate}
    \item a view \( \vi \) must be atomic in that it can see either all or none of 
    the update of any committed transaction (\cref{equ:view-wf-atomic});
    \item the initial snapshot for a transaction, \( \snap = \Snapshot(\kvs,\vi) \), 
        contains the latest observable value for each key, 
        which means that \( \snap \) contains either all or none of the update from other transactions; and
    \item the fingerprint of a transaction contains the first read before any write 
        and the last write for each key, 
        which means that no internal operations are observable by other transactions.
\end{enumerate}

\begin{figure}

\begin{subfigure}{\textwidth}
\centering
\[
\ToCOPSProg*{\;} :  \COPSConfs \times \COPSRunPrograms \times \Labels \times \COPSConfs \times \COPSRunPrograms 
\]

\begin{mathpar}
\inferrule[\rCOPSClient]{
    \cops(\repl) = (\copskvs,\ts)
    \\ \copsctxenv(\cl) = (\copsctx, \repl)
    \\ \copsrunprog(\cl) = \copsruncmd 
    \\ \ToCOPSCmd{ \copskvs | \copsctx | \ts | \copsruncmd | \lb ->
    \copskvs' | \copsctx' | \ts' | \copsruncmd' }
    \\ \cops' = \cops\UpdateFunc{\repl -> (\copskvs, \ts') }
    \\ \copsctxenv' = \copsctxenv\UpdateFunc{\cl -> (\copsctx', \repl) }
    \\ \copsrunprog' = \copsrunprog\UpdateFunc{\cl -> \copsruncmd }
}{%
    \ToCOPSProg{ \cops | \copsctxenv | \copsrunprog  | \lb ->  \cops' | \copsctxenv' | \copsrunprog' }
}
\and
\inferrule[\rCOPSSync]{
    \repl \neq \repl'
    \\ \cops(\repl) = (\copskvs,\ts)
    \\ \cops(\repl') = (\copskvs',\ts')
    \\ \copskvs(\key,\idx) = \copsver
    \\ \Forall{\copstxid} (\stub,\copstxid) \in \DepOf(\copsver) \implies \copstxid \in\copskvs'
    \\ \copskvs'' = \COPSInsert(\copskvs',\key,\copsver)
    \\ \copstxid[\stub][\stub](\ts'') = \WtOf(\copsver)
    \\ \cops\UpdateFunc{\repl' -> (\copskvs'',\Max(\ts',\ts''))}
}{%
    \ToCOPSProg{ \cops | \copsctxenv | \copsrunprog  | \lbCOPSSync(\repl'){\copstxid} ->  \cops' | \copsctxenv | \copsrunprog }
}
\end{mathpar}

\caption{The reference semantics for COPS synchronisation and programs}
\label{fig:cops-semantics-program}

\end{subfigure}

\hrulefill

\begin{subfigure}{\textwidth}

\begin{lstlisting}
on_receive(repl,k,v,id,deps) {
    for (k',id') in deps { wait until (_,id',_) (*$\in$*) repl.kv(k'); }

    list_isnert(repl.kv[k],(v,id,deps));
    (remote_local_time ++ replid) = id;
    repl.local_time = max(remote_local_time, repl.local_time);
}
\end{lstlisting}

\caption{The reference implementation for COPS synchronisation}
\label{lst:cops-receive-msg}

\end{subfigure}

\hrulefill

\caption{COPS synchronisation and programs}

\end{figure}


The operational semantics of programs is given in \cref{fig:program-semantics}, comprising rule \( \rProg\) which
captures the execution of a program step \( \prog \)
given configuration, \( (\kvs, \vienv ) \in \Confs \), and \emph{client environment}, \(\clenv \in \ClientEnvs\).
A client environment \( \clenv \) tracks the local stacks for clients.
%modelled by a partial finite function from client identifiers to client local stacks.

\begin{definition}[Client environments]
\label{def:client-envs}
The set of \emph{client environments}, \(\ClientEnvs \ni \clenv \), 
is defined by: \( \ClientEnvs \TypeDef \Clients \ToPFFunc \Stacks\).
\end{definition}

We assume that the domain of client environments contains 
the domain of the program throughout the execution: \(\Dom(\prog) \subseteq \Dom(\clenv)\).
Program transitions are simply defined in terms of the transitions of their constituent client commands. 
This yields a standard interleaving semantics for transactions of different clients:  
a client executes a transaction in an atomic step without interference from the other clients. 
%The function \( \LastConf \) that is polynomial for any type of trace \( \mathsf{t} \)
%returns the last state in the trace. 
Given an execution test \( \et \),
a valid kv-store program trace \( \progtrc \) of a program \( \prog \)
is a finite trace induced by our operational semantics starting a valid initial state \( (\kvsinit,\vienvinit), \clenv_0 \): 
that is, the domains of \( \vienvinit \) and \( \clenv_0 \) contains the domain of \( \prog \).

\begin{definition}[\( \FirstConf \) and \( \LastConf \) functions]
\label{def:last-conf}
Given any trace \( \mathsf{t} \) of the form
\(
\ToProg[]{\mathsf{s}_0 | \ -> \cdots | \ -> \mathsf{s}_n },
\)
functions \( \FirstConf(\mathsf{t}) \) and \( \LastConf(\mathsf{t}) \) are defined by:
\( \FirstConf(\mathsf{t}) \FuncDef \mathsf{s}_0 \) and \( \LastConf(\mathsf{t}) \FuncDef \mathsf{s}_n \).
\end{definition}

\begin{definition}[Program traces and reachable kv-stores]
\label{def:kv-store-prog-traces}
Given an execution test \( \et \in \ExTests\)  defined in \cref{def:execution-test}, and a program \( \prog \),
the set of \emph{program traces}, \( \ProgTraces(\et,\prog,n) \ni \progtrc \),
is defined by:
\begin{align*}
    \ProgTraces(\et,\prog,0) & \FuncDef 
        \Set{\ToProg{ \kvsinit | \vienvinit | \clenv | \prog} 
                | \vienvinit \in \ViewEnvs(\kvsinit) 
                    \\ {} \land \Dom(\prog) \subseteq \Dom(\vienvinit) 
                    \land \Dom(\prog) \subseteq \Dom(\clenv) }
    \\ \ProgTraces(\et,\prog,n+1) & \FuncDef 
        \Set{ \ToProg{ \progtrc | \stub 
                                -> \kvs | \vienv | \clenv | \prog' } 
                | \progtrc \in \ProgTraces(\et,\prog,n)
                    \\ {} \land \Dom(\prog') \subseteq \Dom(\vienv) 
                    \land \Dom(\prog') \subseteq \Dom(\clenv) }
\end{align*}
and the set of \emph{reachable kv-stores}, written \( \EvalET{\prog} \), is defined by:
\begin{align*}
\EvalET{\prog} & \FuncDef 
        \Set{\kvs | \Exists{ n | \progtrc \in \ProgTraces(\et,\prog,n) } ((\kvs,\stub,\stub), \stub) = \LastConf(\progtrc) }.
\end{align*}
\end{definition}

A kv-store \(\kvs \) is \emph{reachable} with respect to a program \( \prog \),
written \( \kvs \in \EvalET{\prog} \), 
if and only if \( \kvs \) is the kv-store of the final state of a valid kv-store program trace of \( \prog \).

