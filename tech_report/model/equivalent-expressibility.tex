We now prove that our operational semantics and the \(\et \)-traces have the same expressibility,
in the sense that both give rise of the same set of kv-stores.

\begin{toappendix}
\label{sec:proof-equivalent-expressibility}
\end{toappendix}
\begin{theoremrep}[Equivalent expressibility]
\label{thm:ettrc-et-prog}
For any \(\et \in \ExTests\), \(\CMET(\et) = \bigcup_{\prog \in \Programs} \EvalET{\prog}\),
where \(\CMET \) is defined in \cref{def:consistency-model-et} 
and \EvalET{\prog} is defined in \cref{def:kv-store-prog-traces}.
\end{theoremrep}
\begin{proof}
We prove \( \CMET(\et) \supseteq \bigcup_{\prog \in \Programs} \EvalET{\prog} \)
and \( \CMET(\et) \subseteq \bigcup_{\prog \in \Programs} \EvalET{\prog} \) respectively.
\begin{enumerate}
\Case{\(\bigcup_{\prog \in \Programs} \EvalET{\prog} \subseteq \CMET(\et)\)}
    By definition of \( \bigcup_{\prog \in \Programs} \EvalET{\prog} \) It is sufficient to prove that
    for any trace \( \progtrc \), 
    initial configuration \( (\kvsinit,\vienvinit, \clenv_0) \) and 
    final configuration \( (\kvs, \vienv, \clenv) \) with program \( \prog' \),
    \begin{Formulae}
    \begin{Formula}
    \Dom(\prog) \subseteq \Dom(\clenv) 
    \land \progtrc = \ToProg{ \kvsinit | \vienvinit | \clenv_0 | \prog | \stub | * 
                                    -> \kvs | \vienv | \clenv | \prog' }
    \\ \implies \ToRed{\kvsinit | \vienvinit | \stub | * -> \kvs' | \vienv'  }.
    \label{equ:prog-trace-to-et-trace}
    \end{Formula}
    \end{Formulae}
    We prove \cref{equ:prog-trace-to-et-trace} by induction on the length of the trace \( \progtrc \).
    \begin{enumerate}
    \CaseBase{\(\progtrc = \ToProg{ \kvsinit | \vienvinit | \clenv_0 | \prog }\)}
        \Cref{equ:prog-trace-to-et-trace} trivially holds.
    \CaseInd{\(\progtrc = 
             \ToProg{\kvsinit| \vienvinit | \clenv_0 | \prog | \stub | n 
                    ->  \kvs' | \vienv' | \clenv' | \prog'' | \lb 
                    -> \kvs | \vienv | \clenv | \prog' } \)}
        By \ih, there must exist a \et-trace \( \ettrc \) such that 
        \( \ettrc =  \ToRed{\kvsinit | \vienvinit | \stub | * -> \kvs' | \vienv'} \).
        We append steps to \et-trace \( \ettrc \) depending on the step \( \lb \).
        \begin{enumerate*}
        \Case{\( \lb = \lbPri \)}
            This means \( \kvs' = \kvs'' \) and \(\vienv' = \vienv'' \), 
            and we immediately have the proof for \cref{equ:prog-trace-to-et-trace} with \et-trace \ettrc.
        \Case{\( \lb = \lbTrans{\vi,\fp} \)}
            This means \( \kvs' = \UpdateKV(\kvs'',\vi,\fp,\txid) \) and \(\vienv' = \vienv''\UpdateFunc{\cl -> \vi' } \)
            for a fresh \( \txid \) and a view \( \vi' \) such that \( \ToET{\kvs'' | \vi | \fp | \kvs' | \vi'} \).
            Let the new \et-trace \( \ettrc' \) be 
            \(
                \ettrc' =  \ToRed{\ettrc | \lbView{\vi} ->  \kvs'' | \vienv''\UpdateFunc{\cl -> \vi} | \lbFp{\fp} -> \kvs' | \vienv' }
            \)
            which implies \cref{equ:prog-trace-to-et-trace}.
        \end{enumerate*}
    \end{enumerate}
\Case{\(\CMET(\et) \subseteq \bigcup_{\prog \in \Programs} \EvalET{\prog} \)}
    Since there exist equivalent normal traces for all traces in \( \CMET(\et) \) by \cref{thm:normal-trace},
    it is sufficient to prove that for any trace \( \ettrc \in \CMET(\et) \),
    initial configuration \( (\kvsinit,\vienvinit) \) and final configuration \( (\kvs, \vienv) \),
    \begin{Formulae}
    \begin{Formula}
    \NormalTrace(\ettrc)
    \land \ettrc =  \ToRed{ \kvsinit | \vienvinit | \stub | * -> \kvs | \vienv }
    \implies 
    \\ \Exists{ \clenv_0,\clenv \in \ClientEnvs | \prog,\prog' \in \Programs }
    \\ \Dom(\prog) = \Dom(\clenv_0) 
    \land  \ToProg{\kvsinit | \vienvinit | \clenv_0 | \prog | \stub | * 
                            -> \kvs | \vienv | \clenv| \prog' }
    \\ \land \Forall{\cl \in \Dom(\prog')} \prog'(\cl) = \pskip .
    \label{equ:et-trace-to-prog-trace}
    \end{Formula}
    \end{Formulae}
    We prove \cref{equ:et-trace-to-prog-trace} by induction on the length of the trace \( \ettrc \).
    Note that the number of steps in trace \( \ettrc \) must be an \emph{even} number.
    \begin{enumerate}
    \CaseBase{\(\ettrc =  \ToRed{ \kvsinit | \vienvinit }\)}
        We pick \( \prog = \vienv_0 = \emptyset \) and 
        \( \progtrc = \ToProg{ \kvsinit | \vienvinit | \vienv_0 | \prog } \) 
        that implies \cref{equ:et-trace-to-prog-trace}.
    \CaseInd{\( \ettrc = \ToRed{ \kvsinit | \vienvinit | \stub | n -> \kvs' | \vienv' | \lbView{\vi} 
                -> \kvs' | \vienv'\UpdateFunc{\cl -> \vi} | \lbFp{\fp} -> \kvs | \vienv } \)}
        By \ih, there must be a program trace \( \progtrc \) such that 
        \[
            \progtrc =  \ToProg{\kvsinit | \vienvinit | \clenv_0 |\prog | \stub | * 
                -> \kvs' | \vienv' | \clenv' | \prog' }
            \land \Forall{\cl \in \Dom(\prog')} \prog'(\cl) = \pskip.
        \]
        for client environment \(\clenv_0, \clenv' \) and program \( \prog' \).
        We now construct a new initial by extending a new transaction for client \( \cl \);
        this transaction has the fingerprint \( \fp \).
        Recall that a well-formed fingerprint contains at most one read and one write for each key.
        We define the transactional command for fingerprint \( \fp \), written \( \TransFp(\fp)\) as the following:
        \begin{align*}
           \TransFp(\emptyset) & \FuncDef \pskip ,
        \\ \TransFp(\fp \uplus \opR(\key,\val)) & \FuncDef 
            \plookup{\var}{\key} \pseq \TransFp(\fp)  ,
        \\ \TransFp(\fp \uplus \opW(\key,\val)) & \FuncDef
            \begin{multlined}[t]
            \pmutate{\key}{\val} \pseq \TransFp(\fp)  
            \\ \text{where} \ \Forall{\key' \in \Keys | \val' \in \Values} \opR(\key',\val') \notin \fp .
            \end{multlined}
        \end{align*}
        We then define a function that extends the command for a client:
        \[ 
        \ExtendProgram(\prog, \cl, \cmd) \FuncDef 
        \begin{cases}
            \prog\UpdateFunc{ \cl -> ( \cmd' \pseq \cmd)} & \text{if} \ \prog(\cl) = \cmd'
            \\ \prog\uplus \Set{ \cl \mapsto \cmd} & \ow 
        \end{cases}
        \]
        Let transactional command \( \trans = \TransFp(\fp) \),
        the new initial program as \( \ExtendProgram(\prog', \cl, \ptrans{\trans}) \) 
        and the new final program \( \prog'' = \ExtendProgram(\prog', \cl, \ptrans{\trans}) \);
        and consequently apply \( \ExtendProgram \) to all intermediate programs.
        This means, we have a new trace \( \progtrc' \) such that
        \begin{Formulae}*
        \begin{Formula}
            \progtrc' = \ToProg{\kvsinit | \vienvinit | \clenv_0 
                                            | \ExtendProgram(\prog, \cl, \ptrans{\trans}) | \stub | * 
                                -> \kvs' | \vienv' | \clenv' | \prog''  }
            \\ {} \land \Forall{\cl' \in \Dom(\prog'')} 
            \begin{Bracketed} \cl = \cl' \implies \prog''(\cl') = \ptrans{\trans} \end{Bracketed}
            \land \begin{Bracketed} \cl \neq \cl' \implies \prog''(\cl') = \pskip \end{Bracketed} .
        \end{Formula}
        \end{Formulae}
        The only next step for \( \progtrc' \) is to execute the \( \ptrans{\trans} \) for \( \cl \).
        Recall that \[ \ToRed{ \kvs' | \vienv'\UpdateFunc{\cl -> \vi} | \lbFp{\fp} -> \kvs | \vienv }. \]
        Given \rCAtomicTrans shown in \cref{fig:command-semantics}, 
        the client local stack is \( \stk = \clenv'(\cl)\) and the initial snapshot \( \snap \) for 
        \( \trans \) is \( \snap =  \Snapshot(\kvs', \vi) \).
        By the hypothesis, we know that \( \ToET{\kvs' | \vi | \fp | \kvs' | \vienv(\cl) }  \)
        and by well-formed condition for \( \et \), we know
        \[
        \Forall{\key \in \Keys | \val \in \Values } \opR(\key,\val) \in \fp \implies \snap(\key) = \val ,
        \]
        which implies 
        \( \ToTrans{\stk | \snap | \emptyset | \trans  | * -> \stk' | \snap' | \fp | \pskip } \)
        for some stack \( \stk' \) and snapshot \( \snap \).
        By \( \ToET{\kvs' | \vi | \fp | \kvs' | \vienv(\cl) } \),
        \( \CanCommit(\kvs,\vi,\fp)\) and \( \ViewShift(\kvs,\vi,\kvs',\vienv(\cl)) \) must hold.
        Thus we have a trace such that 
        \[
         \ToCmd{\progtrc' | \lbTrans{\vi,\fp} -> \kvs | \vienv | \clenv'\UpdateFunc{\cl -> \stk'} | \prog^* }
        \land \Forall{\cl \in \Dom(\prog^*)} \prog^*(\cl) = \pskip
        \]
        for some program \( \prog^* \), which implies \cref{equ:et-trace-to-prog-trace}. \qedhere
    \end{enumerate}
\end{enumerate}
\end{proof}
\begin{proofsketch}
We prove \( \CMET(\et) \supseteq \bigcup_{\prog \in \Programs} \EvalET{\prog} \)
and \( \CMET(\et) \subseteq \bigcup_{\prog \in \Programs} \EvalET{\prog} \) respectively.
%We prove both directions by inductively constructing traces respectively.
For \( \CMET(\et) \supseteq \bigcup_{\prog \in \Programs} \EvalET{\prog} \),
given any program trace \( \progtrc \), 
by induction on the length of \( \progtrc \), we construct a \( \et \)-trace \( \ettrc \)
and show that the final kv-stores in \( \progtrc \) and \( \ettrc \) are the same.
In each step, if the next transition in \( \progtrc \) is a local computation, we throw it always.
Otherwise, the next transition is a transaction step.
We split the view-shifts from the transaction commits, 
and append these two steps to \( \ettrc \).
It is easy to see that the final kv-stores of these two traces are the same.

Consider \( \CMET(\et) \subseteq \bigcup_{\prog \in \Programs} \EvalET{\prog} \).
By \cref{thm:normal-trace}, assume a normalised \( \et \)-trace \( \ettrc \).
By induction on the length of \( \ettrc \), two steps per iteration,
We construct a program \( \prog \) that matches the fingerprints,
and then enforce the same scheduling, so that we have a program trace \( \progtrc \) with the same final kv-store.
In each step, let \( \cl \) be the next scheduled client and \( \fp \) be the fingerprint of the new transaction.
We now inductively construct a new transaction with the transactional command \( \trans \), 
which is then appended to the end of the command of \( \cl \).
For any read operation in the fingerprint, we append a new look-up command to \( \trans \).
After we convert all read operations,
for any write operation, we append a new mutate command to \( \trans \).
Now we appends a transaction step, combining the view-shift and fingerprint steps, to the trace \( \progtrc \).
It is easy to see that the final kv-stores of these two traces are the same.
The full proof is given in \cref{sec:proof-equivalent-expressibility} on \pageref{sec:proof-equivalent-expressibility}.
\end{proofsketch}
