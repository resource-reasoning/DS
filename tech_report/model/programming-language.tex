We assume standard variable stacks which are total functions from variables to values.
We assume standard arithmetic expressions and boolean expressions
Assume a countably infinite set of variables \( \Vars \ni \var \).
Recall the definition of values in \cref{def:values}.
We assume standard arithmetic expressions and boolean expressions
built from values and program variables.
The evaluation of expressions has no side-effects.

\begin{definition}[Stacks]
The set of \emph{client-local stacks} or just \emph{stacks}, \( \Stacks \ni \stk \), is defined by 
\(\Stacks \TypeDef \Vars \ToTFunc \Values\).
\end{definition}

\begin{definition}[Arithmetic and boolean expressions]
The set of \emph{arithmetic expressions}, \( \Expressions \ni \expr \), is defined by the grammar:
\[
    \expr ::= \val \mid \var \mid \expr + \expr \mid \expr \times \expr \mid \dots
\]
The \emph{evaluation} of an expression \(\expr\) with respect to stack \( \stk \), 
written \(\EvalE{\expr}\), is defined inductively by:
\[
\EvalE{\val} \FuncDef \val
\quad
\EvalE{\var} \FuncDef \stk(\var)
\quad
\EvalE{\expr_{1} + \expr_{2}} \FuncDef
\EvalE{\expr_{1}} + \EvalE{\expr_{2}}
\quad
\EvalE{\expr_{1} \times \expr_{2}} \FuncDef
\EvalE{\expr_{1}} \times \EvalE{\expr_{2}}
\quad
\dots
\]
for \( \var \in \Vars \) and \( \val \in \Values \).
The set of \emph{boolean expressions}, \( \Booleans \ni \bool \), is defined by the grammar:
\[
    \bool ::= \true \mid \false \mid \expr = \expr \mid \expr < \expr \mid 
              \bool \land \bool \mid \bool \lor \bool \mid \neg \bool \mid \dots 
\]
The \emph{evaluation} of an expression \(\bool\) with respect to stack \( \stk \),
written \(\EvalB{\bool}\), is defined inductively by:
\begin{gather*}
\EvalB{\true} \FuncDef \true
\quad
\EvalB{\false} \FuncDef \false
\\
\EvalB{\expr_{1} = \expr_{2}} \FuncDef
\EvalE{\expr_{1}} = \EvalE{\expr_{2}}
\quad
\EvalB{\expr_{1} < \expr_{2}} \FuncDef
\EvalE{\expr_{1}} < \EvalE{\expr_{2}}
\\
\EvalB{\bool_{1} \land \bool_{2}} \FuncDef
\EvalB{\bool_{1}} \land \EvalB{\bool_{2}}
\quad
\EvalB{\bool_{1} \lor \bool_{2}} \FuncDef
\EvalB{\bool_{1}} \lor \EvalB{\bool_{2}}
\quad
\EvalB{\neg \bool } \FuncDef
\neg \EvalB{\bool} 
\dots
\end{gather*}
\end{definition}

Other unary and binary operations, such as subtraction \( - \) for arithmetic expression 
and greater or equal comparison \( \geq \) for boolean expression, are defined analogously.

%Other common boolean expressions like \( \expr \neq \expr' \) and \( \expr \leq \expr' \) 
%can be encoded as \( \neg (\expr = \expr') \) and \( \expr = \expr' \lor \expr < \expr' \).

A \emph{program} \(\prog\) is a finite partial function from client identifiers to sequential client commands.
For clarity, we often write \( \cmd_{1}\ppar \dots \ppar \cmd_{n}\) as syntactic sugar 
for a program \( \prog \) with \(n\) clients associated with identifiers \(\cl_1 \dots \cl_n\), 
where each client \(\cl_i\) executes \(\cmd_i\). 
\emph{Client commands} or just \emph{commands}, are built up 
from atomic transactions of the form of \( \ptrans{\trans} \) and primitive commands for manipulating stacks.
%\emph{Transaction commands} \(\trans\) in a transaction \( \ptrans{\trans} \) are executed in one atomic step.

\begin{definition}[Programs, client commands and transactional commands]
\label{def:program}
\label{def:command}
\label{def:transactional-command}
The set of \emph{programs}, \( \Programs \ni \prog  \), is defined by:
\( \Programs \TypeDef \Clients \ToPFFunc \Commands \), 
where the set of \emph{client commands}, \( \Commands \ni \cmd \), is defined by:
\begin{align*}
\cmd & ::=  
\pskip 
\mid \cmdpri 
\mid \ptrans{\trans} 
\mid \cmd \pseq \cmd 
\mid \cmd \pchoice \cmd 
\mid \prepeat(\cmd)
& \cmdpri & ::=  
\passign{\var}{\expr} 
\mid \passume(\bool)
\end{align*}
for \( \var \in \Vars \), \( \expr \in \Expressions \) and \( \bool \in \Booleans \).
The set of \emph{transactional commands}, \( \Transactions \ni \trans \), is defined 
by the grammar:
\begin{align*}
\trans & ::=
\pskip 
\mid \transpri 
\mid \trans \pseq \trans
\mid \trans \pchoice \trans 
\mid \prepeat(\trans)
& \transpri & ::= 
\passign{\var}{\expr} 
\mid \passume(\bool) 
\mid \plookup{\var}{\expr} 
\mid \pmutate{\expr}{\expr} 
\end{align*}
\end{definition}

Client commands \(\cmd\) comprise \(\pskip\), primitive commands
\(\cmdpri\), atomic transactions \(\ptrans{\trans}\), and standard
compound constructs: sequential composition \( \cmd ; \cmd \) ;
non-deterministic choice \( \cmd + \cmd \); and iteration \( \prepeat(\cmd) \). 
The \( \pif(\bool) \ \cmd_1 \ \pelse \ \cmd_2 \) can be encoded as 
\( (\passume(\bool) \pseq \cmd_1) \pchoice (\passume(\neg\bool) \pseq \cmd_2)  \).
Primitive commands include variable assignment
\(\passign{\var}{\expr}\) and assume statements \(\passume(\bool)\)
which can be used to encode conditionals.
The primitive commands are used for computations based on client-local variables 
and can hence be invoked without restriction. 
Transactional commands \(\trans\) comprise
\(\pskip\), primitive transactional commands \(\transpri\), 
and the standard compound constructs. 
Primitive transactional commands comprise primitive commands, 
lookup \(\plookup{\var}{\expr}\) and mutation \(\pmutate{\expr}{\expr}\) 
used for reading and writing to kv-stores respectively, 
which can only be invoked as part of an atomic transaction.
%and loop \( \pwhile(\bool) \cmd \) as 
%\(\pif(\bool) \ \pwhile(\bool) \cmd  \ \pelse \ \pskip  \).

