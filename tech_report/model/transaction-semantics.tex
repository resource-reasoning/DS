%\begin{sx}
%MOVE to somewhere else: It is still possible for an underlying implementation, 
%such as COPS, to update the distributed kv-store while
%the transaction is in progress. It just means that, given the
%abstractions captured by our global kv-stores and partial views, 
%such an update is modelled as  an instantaneous  atomic
%update.
%\end{sx}

In our operational semantics, transactions are executed \emph{atomically}: that is, reduced in one step.
This choice simplifies our semantics in that there is no fine-grained interleaving between transactions.
It just means that transactions can be treated as executed in one step from the point of individual clients.
However, it is still possible for an underlying implementation such as COPS,
to update the distributed kv-store in a fine-grained manner.

Intuitively, given a configuration \(\conf = (\kvs, \vienv)\), 
when a client \(\cl\) executes a transaction \(\ptrans{\trans}\), 
it performs the following steps: 
\begin{enumerate}
	\item \label{item:enter-transaction} the client constructs an initial \emph{snapshot} \(\snap\) of \(\kvs\) 
        using its view \(\vienv(\cl)\) as defined in \cref{def:view-snapshot};  
	\item \label{item:transacional-local-step} the client executes \(\trans\) in isolation over \(\snap\)
        accumulating the effects (the reads and writes) of the execution of \(\trans\); and
	\item \label{item:exit-transaction} the client commits the transaction \(\ptrans{\trans}\) by incorporating these effects into \(\kvs\).
\end{enumerate}
We explain the semantics for executing \( \trans \) (\cref{item:transacional-local-step}) in this section.
In \cref{sec:program-semantics},
we explain how to construct the initial snapshot (\cref{item:enter-transaction}) 
and commit the transaction (\cref{item:exit-transaction}).
The rules for executing transactional commands is given in \cref{fig:command-semantics}.
The machine state of a transaction consists of a stack \( \stk \),
a transactional snapshot \( \snap \) that tracks the current value for each key,
and a fingerprint \( \fp \) that tracks the effect of a transaction.

\begin{definition}[Transactional snapshots]
\label{def:heaps}
The set of \emph{transactional snapshots}, \( \Snapshots \ni \snap\),
is defined by: \( \Snapshots \TypeDef \Keys \to \Values\).
\end{definition}

Note that the view snapshot \(\Snapshot(\kvs,\vi)\) defined in \cref{def:view-snapshot}, is a transactional snapshot.
When the meaning is clear, we call a \emph{transactional snapshot} as a \emph{snapshot}. 

To capture the effects of executing a transaction \(\trans\) on a snapshot \(\snap\) of kv-store \(\kvs\), 
we identify a \emph{fingerprint}  of \(\trans\) on \(\snap\) which captures
the values that \(\trans\) reads from \(\snap\), and
the values that \(\trans\) writes to \(\snap\) and intends to commit to \(\kvs\). 
Execution of a transaction in a given configuration (\cref{def:configurations}) may result in more than
one fingerprint due to non-determinism, for example non-deterministic choice \(\trans \pchoice \trans\). 

\begin{definition}[Fingerprints]
\label{def:fingerprints}
Let \( \Operations \ni \op \) denote the set of \emph{read operations} (\( \opR\)) and \emph{write operations} (\(\opW\)) defined by:
\(\Operations \TypeDef \Set{(l, \key, \val) | l \in \Set{\opR, \opW} \land \key \in \Keys \land \val \in \Values }\).
A \emph{fingerprint} \(\fp\) is a set of operations, \(\fp \subseteq \Operations\).
The set of \emph{well-formed fingerprints}, \( \Fingerprints \ni \fp \), is defined by:
\[
    \Fingerprints \TypeDef \Set{\fp | \fp \subseteq \Operations 
        \land \Forall{\key \in \Keys |  l \in \Set{\opR, \opW} | \val, \val' \in \Values}  
        \\ (l, \key, \val), (l, \key, \val') \in \fp \implies \val = \val'
    }.
\]
\end{definition}
A well-formed fingerprint, \( \fp \in \Fingerprints \), contains 
at most one read operation and at most one write operation for each key. 
%This reflects our assumption regarding transactions that satisfy the snapshot property: 
%reads are taken from a single snapshot of the kv-store and 
%only the last write of a transaction to each key is committed to the kv-store. 

\begin{figure}
\[
    \ToTrans* \cdot 
    \IsTyped 
    \begin{gathered}[t]
    (\Stacks \times \Snapshots \times \Fingerprints \times \Transactions) \times
    (\Stacks \times \Snapshots  \times \Fingerprints \times \Transactions)
    \end{gathered}
\]

\begin{mathpar}
    \inferrule[\rTPrimitive]{%
        (\stk, \snap) \ToTransPri{\transpri} (\stk', \snap') 
       \\
       \op = \GetOp(\stk, \snap, \transpri)
    }{%
         \ToTrans{\stk | \snap | \fp | \transpri -> \stk' | \snap' | \fp \AddOp \op | \pskip }
    }%

     \inferrule[\rTChoice]{%
		i \in \Set{1,2}
    }{%
        \ToTrans{\stk | \snap | \fp | \trans_{1} \pchoice \trans_{2} -> \stk | \snap | \fp | \trans_{i} }
    }

    \inferrule[\rTIter]{ }{%
        \ToTrans{\stk | \snap | \fp | \prepeat(\trans) -> \stk | \snap | \fp | \pskip \pchoice (\trans \pseq \prepeat(\trans)) }
    }%

    \inferrule[\rTSeqSkip]{ }{%
        \ToTrans{\stk | \snap | \fp | \pskip \pseq \trans -> \stk | \snap | \fp | \trans }
    }%

    \inferrule[\rTSeq]{%
        \ToTrans{\stk | \snap | \fp | \trans_{1}  -> \stk' | \snap' | \fp' | \trans_{1}' }
    }{%
        \ToTrans{\stk | \snap | \fp | \trans_{1} \pseq \trans_{2} -> \stk' | \snap' | \fp' | \trans_{1}' \pseq \trans_{2} }
    }%
\end{mathpar}
\hrulefill
\caption{Operational semantics of transactional commands}
\label{fig:semantics-trans}
\label{fig:transactional-semantics}
\end{figure}


\Cref{fig:semantics-trans} presents the rules for transactional commands. 
The rules for compound transactional commands are standard.
The rule \(\rTChoice\) non-deterministically chooses one side.
The rule \(\rTIter\) either terminates the loop \( \trans^* \) reducing to \( \pskip \),
or unwinds one step reducing to \( \trans \pseq \trans^* \).
A sequence of transactional commands is executed from left to right as expected, 
modelled by the rules \(\rTSeqSkip\) and \(\rTSeq\).

The only non-standard rule is \( \rTPrimitive \), 
which updates the snapshot and the fingerprint of a transaction: 
\begin{enumerate*}
\item the premise \((\stk, \snap) \ToTransPri{\transpri} (\stk', \snap')\) describes how executing
\(\transpri\) affects the local state (the client stack \( \stk \) and the snapshot \( \snap \)) of a transaction; and 
\item the premise \(\op =\GetOp(\stk, \snap, \transpri)\) identifies the operation associated with \(\transpri\),
which can be a read, a write or an empty operation.
\end{enumerate*}
The fingerprint combination operation, \( \fp \AddOp \op \),
adds a read/write operation to a fingerprint and ignores the empty operation \(\opEmp\).

\begin{definition}[Transactional local state transition relation]
Given a primitive transactional command \( \transpri \in \Transactions \),
the transition relation over client local stacks and transactional snapshots,
\(
    \ToTransPri{\transpri} \IsTyped ( \Stacks \times \Snapshots ) \times (\Stacks \times \Snapshots),
\)
is defined by:
\begin{align*}                                                                        
   (\stk, \snap) & \ToTransPri{\passign{\var}{\expr}}    (\stk\UpdateFunc{\var -> \EvalE{\expr}}, \snap ) ,
&  (\stk, \snap) & \ToTransPri{\passume(\bool)}          (\stk, \snap ) \ \text{if \EvalB{\bool} = \true}  ,
\\ (\stk, \snap) & \ToTransPri{\plookup{\var}{\expr}}    (\stk\UpdateFunc{\var -> \snap(\EvalE{\expr})}, \snap )  ,
&  (\stk, \snap) & \ToTransPri{\pmutate{\expr}{\expr'}}  (\stk, \snap\UpdateFunc{\EvalE{\expr} -> \EvalE{\expr'}} )  .
\end{align*}
\end{definition}

The assignment, \( \passign{\var}{\expr} \), evaluates the expression \( \expr \) 
and assigns the value to the local variable \( \var \).
The assume command, \( \passume(\bool) \), does not affect the stack nor snapshot
if the boolean \( \bool \) is evaluated to be \( \true \).
Otherwise the boolean \( \bool \) is \( \false \) and there is no transition.
The loop-up, \( \plookup{\var}{\expr} \), reads from the key \( \EvalE{\expr} \),
and the mutation, \( \pmutate{\expr}{\expr'} \),
updates the snapshot, assigning the key \( \EvalE{\expr} \) with a new value \( \EvalE{\expr'} \).

\begin{definition}[\GetOp function]
Given a stack \( \stk \in \Stacks \) and a transactional snapshot \( \snap \in \Snapshots\),
the operation associated with a primitive transactional command \(\transpri \in \Transactions \)
is defined by:
\begin{align*}
       \GetOp(\stk, \snap, \passign{\var}{\expr}) & \FuncDef \opEmp  ,
    &  \GetOp(\stk, \snap, \passume(\expr)) & \FuncDef \opEmp  ,
       \\ \GetOp(\stk, \snap,  \plookup{\var}{\expr}) & \FuncDef \opR( \EvalE{\expr}, \snap(\EvalE{\expr}))  ,
    &  \GetOp(\stk, \snap, \pmutate{\expr}{\expr'}) & \FuncDef \opW( \EvalE{\expr}, \EvalE{\expr'})  .
\end{align*}
where \(\opEmp\) is the empty operation.
\end{definition}

The function \( \GetOp(\stk, \snap, \transpri) \) 
defines the operation associated with every primitive transactional command \( \transpri \):
\begin{enumerate*}
\item the empty operation (\(\opEmp\)) for the assignment and assume commands,
because these primitive command do not contribute to the fingerprint;
\item a read operation (label \( \opR \)) for look-up; and 
\item a write operation (label \( \opW \)) for mutation.
\end{enumerate*}

\begin{definition}[Fingerprint combination operations]
\label{def:addop}
The fingerprint combination operation,
\(\AddOp \IsTyped \PowerSet(\Operations) \times \Operations \uplus \Set{\opEmp} \ToTFunc \PowerSet(\Operations)\),
is defined by:
\begin{align*}
\fp \AddOp \opR(\key, \val)  & \FuncDef
\begin{cases}
    \fp \uplus \Set{\opR(\key, \val)} & \text{if} 
    \ \Forall{l \in \Set{\opR,\opW} | \val' \in \Values} (l, \key, \val') \notin \fp \\
    \fp &  \ow 
\end{cases}  
\\ \fp \AddOp \opW(\key, \val) & \FuncDef 
\begin{Bracketed} \fp \setminus \Set{\opW(\key, \val') | \val' \in \Values} \end{Bracketed}
\uplus \Set{\opW(\key, \val)} ,
\\ \fp \AddOp \opEmp  & \FuncDef \fp  .
\end{align*}
\end{definition}

The fingerprint combination operation, \( \fp \AddOp \op \), accumulates the effect of the operation \( \op \)
to the fingerprint \( \fp \).
A read operation from \(\key\) is added to a fingerprint \(\fp\) only if \(\fp\) has no entry for \(\key\),
thus only recording the first value read for \(\key\) (before a subsequent write). 
In contrast, a write to \(\key\) is always added to \(\fp\) by removing the existing writes,
thus only recording the last write to \(\key\). 
The \AddOp preserves the well-formedness of fingerprints (\cref{prop:well-formed-fp} on page \pageref{sec:proof-well-form-fingerprint-operations}).

\begin{toappendix}
\label{sec:proof-well-form-fingerprint-operations}
\begin{proposition}[Well-defined fingerprint combination operation]
\label{prop:well-formed-fp}
    Given a well-formed fingerprint \( \fp \) and an operation \( \op \in \Operations\),
    the new fingerprint \( \fp \AddOp \op \) is a well-formed fingerprint.
\end{proposition}
\begin{proof}
The operation \( \op \) may be a read or a write.
\begin{enumerate}
\Case{\(\op = \opR(\key,\val)\)}
    If there is a entry for the key \( \key \), that is,
    \( (l,\key,\val') \in \fp \) for some \( l \in \Set{\opR, \opW} \) and value \( \val' \),
    then the new fingerprint \( \fp \AddOp \op  = \fp \) is trivially well-formed.
    Otherwise, there is no entry for \( \key \) 
    and the new fingerprint \( \fp \AddOp \op = \fp \uplus \opR(\key,\val) \)
    is also well-formed.
\Case{\(\op = \opW(\key,\val)\)}
    Let \( \fp' = \begin{Bracketed} 
        \fp \setminus \Set{\opW(\key, \val') | \val' \in \Values} \end{Bracketed}\).
    By the definition of \AddOp, 
    we have \( \fp \AddOp \op = \fp' \uplus \Set{\opW(\key, \val)} \).
    Since fingerprint \( \fp' \) contains no write operation for key \( \key \),
    \( \opW(\key, \val') \notin \fp' \) for all values \( \val' \),
    the new fingerprint \( \fp' \uplus \Set{\opW(\key, \val)}\) is a well-formed fingerprint. \qedhere
\end{enumerate}
\end{proof}
\end{toappendix}
