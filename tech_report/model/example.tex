We give several examples of execution tests which give rise to consistency models on kv-stores.
Recall that the \emph{snapshot property} and the \emph{last-write-wins policy} are hard-wired into our semantics. 
This means that we can only define  consistency models that satisfy these two constraints. 
Although this prevent us expressing interesting consistency models such as \emph{read committed} \citep{si}, 
we are able to express a large variety of consistency models employed by distributed kv-stores,
from \emph{read atomic} (\(\RA\)) to \emph{serialisability} (\(\SER\)).

\paragraph{Notation.} Given two relations \( \MataTypeFont{r},\MataTypeFont{r}' \subseteq \MataTypeFont{A} \times \MataTypeFont{A} \),
the notation \( \ToEdge{\MataTypeFont{a} | \MataTypeFont{r} -> \MataTypeFont{a}' } \)
denotes \( (\MataTypeFont{a}, \MataTypeFont{a}') \in \MataTypeFont{r}\).
The notation \( \Refl(\MataTypeFont{r}), \Trasi(\MataTypeFont{r}), \TraRe(\MataTypeFont{r}) \)
denotes the reflexive, transitive, and transitive and reflexive closure of \( \MataTypeFont{r} \).
The notation \( \Inv(\MataTypeFont{r}) \) denotes the inverse relation defined by 
\( \Inv(\MataTypeFont{r}) \FuncDef 
    \Set{ (\MataTypeFont{a}', \MataTypeFont{a}) | (\MataTypeFont{a}, \MataTypeFont{a}') \in \MataTypeFont{r} } \).
The notation \( \MataTypeFont{r} \RelConcat \MataTypeFont{r}' \) denotes relation composition
defined by \( \MataTypeFont{r} \RelConcat \MataTypeFont{r}' \FuncDef 
    \Set{(\MataTypeFont{a}, \MataTypeFont{a}') | \Exists{\MataTypeFont{a}'' \in \MataTypeFont{A}}              
        (\MataTypeFont{a}, \MataTypeFont{a}'') \in \MataTypeFont{r} \land (\MataTypeFont{a}'', \MataTypeFont{a}') \in \MataTypeFont{r}' } .
\)


Recall that an execution test \(\et\) (\cref{def:execution-test})
comprises tuples of the form \((\kvs,\vi, \fp, \kvs', \vi')\) 
where  \(\CanCommit[\et]( \kvs, \vi, \fp )\) and \(\ViewShift[\et]( \kvs, \vi, \kvs', \vi' )\). 
We define \(\CanCommit\) and \(\ViewShift\) for several consistency models,
using a couple of auxiliary definitions. 
Predicate \(\CanCommit[\et]( \kvs, \vi, \fp ) \) requires the view \( \vi \) be closed
with respect to a relation \( \rel \) on transactions in \( \kvs \),
in the sense that for any transaction \( \txid \) being the \emph{writer} of a version in \( \vi \),
if \( (\txid',\txid) \in \Trasi(\rel) \), then any version \emph{written} by \( \txid' \) must be
included in \( \vi \).
Note that any read-only transactions do not directly affect the view,
however, these transactions may be intermediate steps to link other transactions that have writes.
This closure property is captured by \( \PreClosed \).

\begin{definition}[Visible transactions and prefix closure]
\label{def:vis-transactions-function}
Given a kv-store \(\kvs \in \KVSs\) and a view on the kv-store \(\vi \in \Views(\kvs)\), 
the \emph{set of visible transactions} is defined by:
\begin{align*}
\VisTrans(\kvs, \vi) & \FuncDef \Set{\WtOf(\kvs(\key, \idx)) | \idx \in \vi(\key)}
\end{align*}
where \( \WtOf \) is defined in \cref{def:versions}.
Given a binary relation on transactions, \(\rel \subseteq \TxIDs \times \TxIDs\), 
a view \(\vi\) is \emph{prefix closed} or \emph{closed} 
with respect to a kv-store \(\kvs\) and the relation \(\rel\), 
written \(\PreClosed(\kvs,\vi,\rel)\), if and only if
\begin{align*}
    \begin{multlined}[t]
    \VisTrans(\kvs, \vi) = 
    \left( \Inv((\TraRe(\rel))) (\VisTrans(\kvs, \vi)) 
            \setminus  \ReadOnlyTrans(\kvs) \right) ,
    \end{multlined}
\end{align*}
where the set of read-only transaction, \( \ReadOnlyTrans(\kvs) \), is defined by:
\begin{align*}
\ReadOnlyTrans(\kvs) & \FuncDef
 \Set{\txid | \txid \in \kvs \land \Forall{\key \in \Keys | \idx \in \Nat} 
    \txid \neq \WtOf(\kvs(\key,\idx))} .
\end{align*}
\end{definition}

The prefix closure, \( \PreClosed \), states that
if transaction \(\txid\) is visible in \(\vi\)
in that a version written by \(\txid\) is included in \vi, that is, \( \txid \in \VisTrans(\kvs, \vi) \)
then all transactions \( \txid' \) that are \(\TraRe(\rel)\)-before \(\txid\) 
(\((\txid',\txid) \in \TraRe(\rel)\)) are also visible in \(\vi\).
%that is, any versions written by \( \txid' \) must be included in \vi.
Note that the set of prefix closed transactions, \( \Inv((\TraRe(\rel))) (\VisTrans(\kvs, \vi)) \),
may contain read-only transactions.
However, the read-only transactions have no effect on the view.

We define \emph{dependency relations for kv-stores}, 
inspired by analogous relations for dependency graphs due to \citet{adya}.
The relations are \emph{write-read} (\(\WR\)), \emph{write-write} (\(\WW\)) and \emph{read-write} (\(\RW\)).
These three dependency relations, and the session order \( \SO \) defined in \cref{def:session-order},
are the basic building blocks for defining consistency models using execution tests:
the view must be closed with respect to certain combination of these four relations.
%Many execution tests for well-known consistency models 
%use these dependency relations to constrain the view.
Note that we specifically use the same names as in dependency graphs (\cref{sec:sound-complete}).
This is to emphasis the similarity of dependency relations in kv-stores and in dependency graphs.

\begin{figure}
\centering

\begin{subfigure}{\textwidth}
\centering
\begin{tikzpicture}

\KVMapping{x}{ \key }{ %
    /\;\;\;\;\;/\;\;\;\;\;\;\txid_0\;\;\;/\;\;\;\;\Set{\txid_1}\;\;\;\; 
    , /\;/\;\;\;\;\txid_2\;\;\;\;/\;\;\;\;\emptyset\;\;\;\;\; };

\path[->, thick] ([xshift=10pt]Writerx1.west) edge[bend right=75] node[left, text opacity=1] {$\WR$} ([xshift=10pt]Readersx1.west);
\path[->, thick] ([xshift=-10pt]Writerx1.east) edge[bend left=30] node[above, text opacity=1] {$\WW$} ([xshift=10pt]Writerx2.west);
\path[->, thick] ([xshift=-10pt]Readersx1.east) edge[bend right=30] node[below, text opacity=1] {$\RW$} ([xshift=10pt]Writerx2.west);
\end{tikzpicture}

\caption{An example of dependency relations on key-value store with values omitted}
\label{fig:dependencies-kvstore}

\end{subfigure}


\hrulefill

\begin{subfigure}{\textwidth}
\centering
\begin{tikzpicture}

\node at (0,0) (a) {\(\txid_0\)};
\node at (2,-2) (b) {\(\txid_2\)};
\node at (4.5,0) (c) {\(\txid_1\)};
%\node at (10,-2) (d) {\(\txid_3\)};

\coordinate (a1) at ($(a) + (3,0)$);
\coordinate (b1) at ($(b) + (7,0)$);
\coordinate (c1) at ($(c) + (3,0)$);
%\coordinate (d1) at ($(d) + (3,0)$);

\draw[|-|] (a) -- (a1);
\draw[|-,densely dashed] (b) -- ($(b)!0.5!(b1)$);
\draw[-|] ($(b)!0.5!(b1)$) -- (b1);
\draw[|-] (c) -- ($(c)!0.5!(c1)$);
\draw[-|,densely dashed] ($(c)!0.5!(c1)$) -- (c1);
%\draw[|-|] (d) -- (d1);

\path[->, thick] (a1) edge[bend left=50] node[above, text opacity=1] {$\WR$} ($(c)+(0.3,0.05)$);
\path[->, thick] (a1) edge node[below, text opacity=1] {$\WW$} (b1);
\path[->, thick] (c) edge node[above, text opacity=1] {$\RW$} (b1);

\end{tikzpicture}

\caption{An example time line contains the starts and commits of transactions,
with dashed line being able to stretched}
\label{fig:dependencies-time-line}

\end{subfigure}


\hrulefill

\caption{Dependency relations on key-value store}
\label{fig:dependencies-kvstore}
\end{figure}


\begin{definition}[Dependency relations on kv-stores]
\label{def:dependency-kv-store}
Given a kv-store \(\kvs \in \KVSs\) and a key \( \key \in \Keys \):
\begin{enumerate}
\item the \emph{write-read dependency} on the key, \( \WR[\kvs](\key) \), is defined by:
    \[ 
    \WR[\kvs](\key) \FuncDef \Set{(\txid,\txid') | 
        \Exists{\idx} \txid = \WtOf(\kvs(\key,\idx)) \land \txid' \in \RsOf(\kvs(\key,\idx))} ;
    \]
\item the \emph{write-write dependency} on the key, \( \WW[\kvs](\key) \), is defined by:
    \[ 
    \WW[\kvs](\key) \FuncDef \Set{(\txid,\txid') | 
        \Exists{\idx,\idx'} \txid = \WtOf(\kvs(\key,\idx)) 
        \land \txid' = \WtOf(\kvs(\key,\idx')) \land \idx < \idx'}  ; \ and
    \]
\item the \emph{read-write anti-dependency} on the key, \( \RW[\kvs](\key) \), is defined by:
    \[ 
    \RW[\kvs](\key) \FuncDef \Set{(\txid,\txid') | 
        \Exists{\idx,\idx'} \txid \in \RsOf(\kvs(\key,\idx)) 
        \land \txid' = \WtOf(\kvs(\key,\idx')) 
        \land \idx < \idx' \land \txid \neq \txid'} .
    \]
\end{enumerate}
The \emph{write-read}, \emph{write-write} and \emph{read-write dependencies}
on the kv-store \( \kvs \) are then defined by:
\(  \rel[\kvs] \FuncDef \bigcup_{\key \in \Keys}\rel[\kvs](\key) \)
for \( \rel \in \Set{\WR, \WW, \RW} \).
\end{definition}

\Cref{fig:dependencies-kvstore} illustrates an example kv-store and its dependency relations,
and \cref{fig:dependencies-time-line} is an example time line, 
if transactions in \cref{fig:dependencies-kvstore} are executed in a centralised database.
The write-read dependency, \( (\txid_0, \txid_1) \in \WR \), states that
transaction \( \txid_2 \) reads a version written by \( \txid_1 \).
This means that \( \txid_2 \) \emph{starts} after the commit of \( \txid_1 \), hence 
\( \txid_2 \) observes effect of \( \txid_1 \), depicted in \cref{fig:dependencies-time-line}.
The write-write dependency, \( (\txid_1, \txid_2) \in \WW \), states that 
transaction \( \txid_2 \) overwrites a version written by \( \txid_1 \).
This means that \( \txid_2 \) \emph{commits} after the commit of \( \txid_1 \).
However, there is no information about the starts of these two transaction,
which means these two transactions may be executed concurrently, 
as shown in \cref{fig:dependencies-time-line}.
Last, the read-write anti-dependency, \( (\txid_1,\txid_2) \in \RW \),
states that \( \txid_1 \) reads a version that has been over-written by \( \txid_2 \).
This means that \( \txid_1 \) starts before the commit of \( \txid_2 \), depicted in \cref{fig:dependencies-time-line},
hence, \( \txid_1 \) does not observe the effect of \( \txid_2 \).
The anti-dependency can be derived from \( \WR \) and \( \WW \)
in that \( (\txid,\txid') \in \RW 
        \PredDef \Exists{\txid''} (\txid'',\txid) \in \WR \land (\txid'',\txid') \in \WW \).


Recall that execution tests are defined using  \(\ViewShift\) and \(\CanCommit\)  predicates.
We now give several definitions of \(\ViewShift\) and \(\CanCommit\) for
well-known consistency models in \cref{fig:execution-tests}. 
In \cref{sec:sound-complete}, we demonstrate that the associated consistency
models on kv-stores correspond to well-known consistency models on abstract executions. 
%We anticipate these results, by labelling the
%execution test with their well-known consistency models.

\begin{figure}[t]
%\renewcommand{\true}{\ensuremath{\mathsf{true}}}
\small
\centering
\begin{tabular}{ @{} l @{\hspace{2pt}} || @{\hspace{2pt}} l @{\hspace{2pt}} | @{\hspace{2pt}}  c @{} }
\hline
	\ET 
	& $\CanCommit( \kvs, \vi, \fp) \PredDef \PreClosed(\kvs, \vi, \rel[\et]) $
    & $\ViewShift( \kvs, \vi, \kvs', \vi' )$ 
	\\
	\hline
	\TOP/\RA
	& \true 
	& \true 
	\\ \hline  
%
%	
	\MR 
	& \true 
	& $\vi \vileq \vi'$
	\\ \hline  
%
	\RYW
	& \true
	& 
	$\begin{multlined}
		\Forall{\txid \in \kvs' \setminus \kvs | \key \in \kvs' | \idx} 
        \\[-10pt] \ToEdge{\WtOf(\kvs'(\key, i)) | \Refl(\SO)  -> \txid }  
        \\ \implies \idx \in \vi'(\key) 
	\end{multlined}$
	\\ \hline  
    \MW 
    & \(\rel[\MW] \FuncDef \SO \cap \WW[\kvs]\)
    & \true  
    \\ \hline
%	
    \WFR
    & $\rel[\WFR] \FuncDef \WR[\kvs] ; \Refl((\SO \cap \RW[\kvs])) $ 
    & \true 
    \\ \hline
	\CC
	& $\rel[\CC] \FuncDef \SO \cup \WR[\kvs]$ 
	& $\ViewShift[\MR \cap \RYW]( \kvs,\vi,\kvs',\vi' )$
	\\ \hline  
%
	\UA 
	& $\rel[\UA] \FuncDef {\textstyle\bigcup_{\opW(\key, \stub) \in \fp}} \WWInv[\kvs](\key) $ 
	& \true  
	\\ \hline  
% 
	\PSI
	& $\rel[\PSI] \FuncDef \rel[\UA] \cup \rel[\CC] \cup \WW[\kvs]$ 
	& $\ViewShift[\MR \cap \RYW]( \kvs,\vi,\kvs',\vi' )$
	\\ \hline   
%
	\CP 
	& $ \begin{multlined}
    \rel[\CP] \FuncDef 
    \\[-10pt] \SO ; \Refl(\RW[\kvs]) \cup \WR[\kvs] ; \Refl(\RW[\kvs]) \cup \WW[\kvs]
    \end{multlined} $ 	
	& $\ViewShift[\MR \cap \RYW]( \kvs,\vi,\kvs',\vi' )$
    \\ \hline 
%	
	\WSI
	& $  \rel[\SI]  \FuncDef \rel[\UA] \cup \rel[\CP]$ 
	& $\ViewShift[\MR \cap \RYW]( \kvs,\vi,\kvs',\vi' )$
	\\ \hline  
%	
	\SI
	& $  \rel[\SI]  \FuncDef \rel[\UA] \cup \rel[\CP] \cup (\WW[\kvs]; \RW[\kvs])$ 
	& $\ViewShift[\MR \cap \RYW]( \kvs,\vi,\kvs',\vi' )$
	\\ \hline  
%% 
	\SER
	&$\rel[\SER] \FuncDef \WWInv[\kvs]$
	& \true	
	\\ \hline
\end{tabular}
%

\caption{Execution tests of well-known consistency models, $\SO$ is given in \pageref{def:session-order}.}
\label{fig:execution-tests}
\end{figure}

