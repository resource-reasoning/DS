We define consistency models for our kv-stores,
by introducing the notion of \emph{execution test} 
that specifies  whether a client is allowed to commit a transaction in a given kv-store. 
Each execution test induces a consistency model as the set of kv-stores obtained 
by having clients non-deterministically commit transactions so long as the constraints 
imposed by the execution test are satisfied.
We explore a range of execution tests  associated with well-known consistency models in the literature. 
In this section we formally define execution tests and traces induced by them.
In \cref{sec:consistency-model-on-kv}, 
we give examples of execution tests for several well-known consistency models
and, in \cref{sec:sound-complete}, we demonstrate that our definitions of consistency models
are equivalent to the established axiomatic definitions over abstract executions 
\citep{ev_transactions,framework-concur} and dependency graphs \citep{adya}.

An execution test \( \et \) is a set of tuples of the form \( (\kvs,\vi,\fp,\kvs',\vi') \),
stating that, under \( \et \), a transaction with the fingerprint \( \fp \) and the given view \( \vi \)
is allowed to commit to the kv-store \( \kvs \),
and resulting in the new kv-store \( \kvs' \) and the new view \( \vi' \).

\begin{definition}[Execution tests]
\label{def:execution-test}
Given a \( \CanCommit \) and a \( \ViewShift \) predicate,
an \emph{execution test} \( \et \) is a subset of 
\(\KVSs \times \Views \times \Fingerprints \times \KVSs \times \Views\),
such that: for all \( (\kvs,\vi,\fp,\kvs',\vi') \in \et \),
\begin{Formulae}
& \begin{Formula}
    \vi \in \Views(\kvs) \land \vi' \in \Views(\kvs') 
\end{Formula}
\\ & \begin{Formula}
    \Forall{\key \in \kvs | \val \in \Values}
    \opR(\key, \val) \in \fp \implies \kvs(\key, \Max[<](\vi(\key))) = \val
\end{Formula}
\\ & \begin{Formula}
    \CanCommit[\et](\kvs,\vi,\fp) \land \ViewShift[\et](\kvs,\vi,\kvs',\vi').
\end{Formula}
\end{Formulae}
Let \( \CanCommit[\et] \) and \( \ViewShift[\et] \) denote 
the \( \CanCommit \) and \( \ViewShift \) predicates for \( \et \) respectively.
Let \( \ExTests \ni \et\) denote the set of all execution tests.
\end{definition}

We use the notation \( \ToET{\kvs | \vi | \fp | \kvs' | \vi' } \) as 
we can think of the \( \et \) tuple as a transition on kv-stores and views. 
Note that the well-formed conditions enforce \emph{last-write-wins policy}
as the value \( \val \) read in the fingerprint for any key \( \key \) 
must match with the value in the last version for the key \( \key \) in the view \( \vi \).
An execution test \( \et \)  is defined using two predicates \( \CanCommit[\et] \) and \( \ViewShift[\et] \)
that used in the rule \( \rCAtomicTrans \) in \cref{fig:command-semantics}.
We give several examples of execution tests which give rise to consistency models on kv-stores in \cref{sec:consistency-model-on-kv}.

%We conceptually separate \( \et \) into two predicates for readability.
%The use of \( \et \) and the two predicates are interchangeable.

%An execution test \(\et\) gives raise of a \(\et\)-trace.
%The trace is useful for connecting our definitions of consistency models 
%to the well-know axiomatic definitions over declarative semantics.
%We also use the trace to verify implementations of specific consistency models.

Given an execution test \( \et \),
an \emph{\(\et\)-reduction} of the form \( \ToRed{\kvs | \vienv | \lb -> \kvs' | \vienv' }\)
is a labelled transition relation over configurations,
where the label describes either an arbitrary view shift for a client or the commitment of a fingerprint.
The set of \emph{\(\et\)-traces} contains all finite sequences of \(\et\)-reductions starting from an
initial configuration.

\begin{definition}[\(\et\)-reduction and \(\et\)-traces]
\label{def:et-reduction}
The set of \emph{\et-reduction labels}, \( \ETLabels \ni \lb\), is defined by:
\[
 \ETLabels \TypeDef \Set{\lbView{\vi} | \cl \in \Clients \land \vi \in \Views } 
    \uplus \Set{\lbFp{\fp} | \cl \in \Clients \land \fp \in \Fingerprints} .
\]
Given an execution test \( \et \),
the \emph{\(\et\)-reduction} is a labelled transition over configurations, \( \ToRed{\kvs | \vienv | \lb -> \kvs' | \vienv' }\),
defined by:
\begin{enumerate}
\item if \(\lb = \lbView{\vi}\) for a client \( \cl \in \Clients \) and a view \( \vi \in \Views(\kvs) \), then
\[
    \ToRed{ \kvs | \vienv | \lbView{\vi} -> \kvs  | \vienv\UpdateFunc{\cl -> \vi} }
\]
and \( \vienv(\cl) \vileq \vi\); or
\item if \(\lb = \lbFp{\fp}\) for a fingerprint \(\fp\), then
\[
    \ToRed{ \kvs | \vienv | \lbFp{\fp} -> \kvs' |\vienv\UpdateFunc{\cl -> \vi} }
\] 
where \( \ToET{\kvs | \vienv(\cl) | \fp | \kvs'| \vi} \)
and \(\kvs' = \UpdateKV(\kvs,\vienv(\cl),\fp,\txid)\) defined in \cref{def:update-kvstore}
for \(\txid \in \NextTxid(\cl, \kvs)\) defined in \cref{def:next-txid}.
\end{enumerate}
The set of \emph{\(\et\)-traces}, \(\ETTraces(\et) \ni \ettrc \), is defined by:
\begin{align*}
    \ETTraces(\et) & \TypeDef \bigcup_{n \in \Nat} \ETTracesN(\et,n)
    \\ \ETTracesN(\et,0) & \TypeDef \Set{ \confinit | \confinit \in \ConfInits} 
    \\ \ETTracesN(\et,n+1) & \TypeDef \Set{ \ToRed { \ettrc | \lb -> \conf } | \lb \in \ETLabels } 
\end{align*}
and the set of all \emph{traces induced by execution tests} is then defined by:
\[ \ETTraces \TypeDef \bigcup_{\et \in \ExTests} \ETTraces(\et). \]
\end{definition}

The \emph{consistency model} induced by an execution test \(\et\), 
written \(\CMET(\et)\), is the set of all kv-stores in the \(\et\)-traces. 

\begin{definition}[Consistency models on kv-stores]
\label{def:consistency-model-et}
The \emph{consistency model} induced by an execution test \(\et\), written \(\CMET(\et)\), is defined by
\[
\CMET(\et) \TypeDef 
    \Set{\kvs | \Exists{\ettrc \in \ETTraces(\et)} (\kvs,\stub) = \LastConf(\ettrc) } .
\]
\end{definition}

In this thesis, we mainly use our operational semantics defined in \cref{sec:operational-semantics} 
and kv-store program traces defined in \cref{def:kv-store-prog-traces}.
However, the definition of \(\et\)-trace is an important intermediate step 
to show that our definition of consistency models using execution tests for kv-stores
are equivalent to the declarative definitions on abstract executions (\cref{sec:sound-complete}).
Note that in the definition of \(\et\)-traces, the view-shifts and transaction commits are decoupled.
In contrast, in our operational semantics (\(\rCAtomicTrans\) in \cref{fig:command-semantics}), 
view-shifts (the first premise in \(\rCAtomicTrans\)) 
and transaction commits are combined in a single step.
Despite this difference, \(\et\)-traces and operational semantics are equally expressive
in the sense that any kv-store \(\kvs \in \CMET(\et)\) can be obtained as a result of 
executing some program \(\prog\) under the execution test \(\et\), and vice versa.
This is because there every \( \et \)-trace \( \ettrc \) has an \emph{equivalent normalised} \( \et \)-trace \( \ettrc' \)
and we can construct a program and a trace of this program from the normalised \( \et \)-trace and vice versa.
Two traces \( \ettrc \) and \( \ettrc' \) are equivalent, written \( \ettrc \ettrceq \ettrc' \)
if the kv-stores in the final states of the two traces are the same.
A normalised \(\et\)-trace \( \ettrc \) is a trace where updates and view-shifts for a client must appear together.

\begin{definition}[\(\et\)-trace equivalence]
\label{def:equiv-et-trace}
Two \et-traces \( \ettrc, \ettrc \in \ETTraces\) are equivalent, written \( \ettrc \ettrceq \ettrc'\), 
if and only if,
\( 
    \Exists{\kvs \in \KVSs} \LastConf(\ettrc) = (\kvs,\stub) \land \LastConf(\ettrc') = (\kvs,\stub).
\)
\end{definition}

\begin{definition}[Normalised \et-trace]
\label{def:et-normal-trace}
A \(\et\)-trace \( \ettrc \in \ETTraces \) is \emph{normalised}, 
written \( \NormalTrace(\ettrc) \), is defined by:
\begin{align*}
\NormalTrace(\confinit) & \PredDef \true
\\ \NormalTrace(\ToRed{ \ettrc | \lb -> \conf | \lb' -> \conf' } ) & \PredDef
\begin{multlined}[t]
    \NormalTrace(\ettrc)  
    \\ {} \land \Exists{\cl \in \Clients | \vi \in \Views | \fp \in \Fingerprints} 
    \\ (\lb = \lbView{\vi} \iff \lb'=\lbFp{\fp}) .
\end{multlined}
\end{align*}
\end{definition}

\begin{toappendix}
\label{sec:proof-equivalent-normal-trace}
\end{toappendix}
\begin{theoremrep}[Equivalent normal \et-traces]
\label{thm:normal-trace}
For any \( \ettrc \in \ETTraces \), there exists an equivalent normalised trace \( \ettrc^* \),
that is,
\( \ettrc \ettrceq \ettrc^* \) and \( \NormalTrace(\ettrc^*) \).
\end{theoremrep}
\begin{proof}
Let initially \( \ettrc^* = \ettrc \);
we alters the trace until \( \ettrc^*\) is a normal trace.
\begin{enumerate}
\item \label{item:delete-last-view} 
    Consider the last step \lb for each client \( \cl \) such that
    \begin{Formulae}*
    \begin{Formula}
    \ettrc^* =  \ToRed{\ettrc' | \lb -> \ettrc'' }  \land \lb = \lbCl{\alpha} 
    \land \Forall{\cl'} \ettrc'' = \ToRed{\cdots | \lbCl[\cl']{\stub} -> \cdots } 
    \implies \cl \neq \cl'
    \end{Formula}
    \end{Formulae}
    for two trace segments \( \ettrc',\ettrc'' \).
    Let view \( \vi = \LastConf(\ettrc')\Proj{2}(\cl)\).
    If \( \lb \) is a view shift step,
    \ie \( \alpha = \vi' \) for some view \( \vi' \) such that \( \vi \vileq \vi' \), 
    we delete this step and re-assign the configurations
    in \( \ettrc'' \) resulting a new segment \( \ettrc^\dagger \) such that
    \[
        \Forall{\idx \in \Indexs | \kvs \in \KVSs | \vienv \in \ViewEnvs(\kvs) }  
        \ettrc''\Proj{\idx} = (\kvs,\vienv) \iff \ettrc^\dagger\Proj{\idx} = (\kvs,\vienv\UpdateFunc{ \cl -> \vi}).
    \]
    We rename the new trace as \( \ettrc^* \) and go back \cref{item:delete-last-view}.
    Because trace \( \ettrc^* \)  has finite steps, \cref{item:delete-last-view} must terminate 
    with trace \( \ettrc^* \) that satisfies:
    \begin{Formulae}
    \begin{Formula}
    \Forall{\lb \in \Labels | \cl, \cl' \in \Clients | \alpha | \ettrc',\ettrc''}
    \\ 
    \begin{Bracketed}
    \ettrc^* =  \ToRed{\ettrc' | \lb -> \ettrc'' }  \land \lb = \lbCl{\alpha} 
    \land \ettrc''  =  \ToRed{\cdots | \lbCl[\cl']{\stub} -> \cdots}
    \implies \cl \neq \cl'
    \end{Bracketed}
    \implies \alpha \in \Fingerprints. \label{equ:last-step-must-be-update}
    \end{Formula}
    \end{Formulae}
\item \label{item:view-shift-move-right}
    Given a trace \( \ettrc^* \) satisfying \cref{equ:last-step-must-be-update},
    consider the first view-shift step \lb for a client \( \cl \)
    that is not followed by a step for the same client \( \cl \):
    \( \ettrc^* =  \ToRed{\ettrc' | \lb -> \conf | \lb' -> \ettrc'' } \) 
    such that \( \lb = \lbView{\vi} \) and \( \lb' = \lbCl[\cl']{\alpha} \)
    for some \( \alpha\), 
    view \( \vi \) and some clients \( \cl, \cl' \) with \( \cl \neq \cl' \).
    By \cref{lem:view-shift-right-move} we can move step \( \lb \) to the right resulting
    \(  \ToRed{\ettrc' | \lb' ->  \conf^* | \lb -> \ettrc''} \) for some \( \conf^* \),
    until the immediate next step is for the same client \( \cl \).
    Note that there must be a step for the same client \( \cl \) in \( \ettrc'' \) 
    by \cref{equ:last-step-must-be-update}.
    We rename the new trace as \( \ettrc^* \) and go back to \cref{item:view-shift-move-right}.
    Because there are only finite steps 
    and the number of out-of-order view-shifts decreases after each iteration,
    \cref{item:view-shift-move-right} must terminate with trace \( \ettrc^* \) such that
    \begin{Formulae}
    \begin{Formula}
    \Forall{\lb,\lb' \in \Labels | \cl \in \Clients | \vi \in \Views | \ettrc',\ettrc''}
    \\ \ettrc^* =  \ToRed{\ettrc' | \lb -> \conf | \lb' -> \ettrc''}
    \land \lb = \lbView{\vi}
    \implies \lb' = \lbCl[\cl]{\stub}.
    \label{equ:view-shift-followed-same-client-step}
    \end{Formula}
    \end{Formulae}
\item \label{item:view-shift-absorb}
    Given a trace \( \ettrc^* \) satisfying 
    \cref{equ:last-step-must-be-update,equ:view-shift-followed-same-client-step},
    consider the two adjacent view-shifts steps \(\lb, \lb'\) such that
    \( \ettrc^* =  \ToRed{\ettrc' | \lb -> \conf | \lb' -> \ettrc''} \) 
    where \( \lb = \lbView{\vi} \) and \( \lb' = \lbCl{\vi'} \) for two views \( \vi,\vi'\).
    By \cref{lem:view-shift-absorb}, we can merge these two steps resulting
    \(  \ToRed{\ettrc' | \lb' -> \ettrc'' }  \) and go back to \cref{item:view-shift-absorb}.
    Because there are only finite steps 
    and the number of two adjacent view-shifts decreases after each iteration,
    \cref{item:view-shift-absorb} must terminate with trace \( \ettrc^* \) such that
    \begin{Formulae}
    \begin{Formula}
    \Forall{\lb,\lb' \in \Labels | \cl \in \Clients | \vi \in \Views | \ettrc',\ettrc''}
    \\ \ettrc^* =  \ToRed{\ettrc' | \lb -> \conf | \lb' -> \ettrc'' }
    \land \lb = \lbView{\vi}
    \implies \Exists{\fp \in \Fingerprints}
    \lb' = \lbFp{\fp}.
    \label{equ:view-shift-followed-update}
    \end{Formula}
    \end{Formulae}
\item \label{item:add-view-shift} 
    Last, given a trace \( \ettrc^* \) satisfying 
    \cref{equ:last-step-must-be-update,equ:view-shift-followed-same-client-step,equ:view-shift-followed-update},
    consider an update step \( \lb' \) without a view-shift predecessor,
    that is, \( \ettrc^* =  \ToRed{\ettrc' | \lb -> \conf | \lb' -> \ettrc''} \) 
    such that \( \lb = \lbView{\fp} \) and \( \lb' = \lbCl[\cl']{\fp'} \)
    for some fingerprints \( \fp,\fp'\) and 
    clients \( \cl, \cl' \) with \( \cl \neq \cl' \).
    We inject an identity view shift in between resulting 
    \(  \ToRed{\ettrc' | \lb -> \conf | \lbView{\conf\Proj{2}(\cl)} -> \conf | \lb' -> \ettrc''} \).
    Rename the new trace as \( \ettrc^* \) and go back to \cref{item:add-view-shift}.
    Because there are only finite steps, \cref{item:add-view-shift} must 
    terminate with trace \( \ettrc^* \) satisfying 
    \begin{Formulae}
    \begin{Formula}
    \Forall{\lb,\lb' \in \Labels | \cl \in \Clients | \fp \in \Fingerprints | \ettrc',\ettrc''}
    \\ \ettrc^* =  \ToRed{\ettrc' | \lb -> \conf | \lb' -> \ettrc''}
    \land \lb' = \lbFp{\fp}
    \implies \Exists{\vi \in \Views}
    \lb' = \lbView{\vi}.
    \label{equ:update-impede-view}
    \end{Formula}
    \end{Formulae}
\end{enumerate}
Now we have an equivalent trace \( \ettrc^* \) such that \( \ettrc \ettrceq \ettrc^* \) 
and \Cref{equ:view-shift-followed-update,equ:update-impede-view} imply \( \NormalTrace(\ettrc^*) \).
\end{proof}
\begin{proofsketch}
We perform the following transformations over the trace \( \ettrc \) until it is normalised:
for any client \( \cl \),
\begin{enumerate}
\item we eliminate the last view-shift(s), if there is no more commit steps from \( \cl \) afterwards;
\item we move any intermediate view-shift to the right until 
    it is immediately followed by a commit or another view-shift from \( \cl \);
\item we combine any adjacent view-shifts from \( \cl \) into one view-shift; and
\item for any commit step that does not follow a view-shift,
    we insert an identical view-shift before.
\end{enumerate}
All steps are guaranteed to terminate since there only are finite steps in the trace, 
and after each transformation the new trace is one step closer to become a normalised trace.
The full proofs is in \cref{sec:proof-equivalent-normal-trace} on page \pageref{sec:proof-equivalent-normal-trace}.
\end{proofsketch}

\begin{toappendix}
\begin{lemma}[View-shift right move]
\label{lem:view-shift-right-move}
Given a trace \( \ettrc \in \ETTraces(\et) \) for some \( \et \),
and a view-shift step \( \lbView{\vi} \) for a client \( \cl \) in the trace \( \ettrc \),
if the view-shift is not followed by a step for the same client \( \cl \),
this view-shift can be moved right without changing the final configuration.
\begin{Formulae}
\begin{Formula}
    \Forall{ \cl' \in \Clients | \lb,\lb' \in \Labels | \alpha, \ettrc',\ettrc'' } 
    \\ \ettrc =  \ToRed{\ettrc' | \lb -> \conf | \lbView{\vi} -> \conf' | \lbCl[\cl']{\alpha} 
                    -> \conf'' | \lb' -> \ettrc''}
    \land \cl \neq \cl'
    \\ \implies \Exists{\conf^* \in \Confs }
    \ToRed{\ettrc' | \lb -> \conf | \lbCl[\cl']{\alpha} 
        -> \conf^* | \lbView{\vi} -> \conf'' | \lb' -> \ettrc''}.
    \label{equ:view-left-move}
\end{Formula}
\end{Formulae}
\end{lemma}
\begin{proof}
Let configuration \( (\kvs, \vienv) = \conf \); we preform case analysis on \( \alpha \).
\begin{enumerate}
    \Case{\(\alpha=\vi'\)}
    Given the two view-shifts for clients \( \cl \) and \( \cl' \) respectively,
    we know the configuration \( \conf'' \)  is given by 
    \( \conf'' = \Tuple{\kvs, \vienv\UpdateFunc{\cl -> \vi | \cl' -> \vi' }} \).
    Since two clients are distinct \( \cl \neq \cl' \), 
    then we have \( \conf'' = \Tuple{\kvs, \vienv\UpdateFunc{ \cl' -> \vi' | \cl -> \vi }} \);
    therefore we prove \cref{equ:view-left-move} by picking \( \conf^* = \Tuple{\kvs, \vienv\UpdateFunc{\cl' -> \vi' }}\).
    \Case{\(\alpha=\fp\)}
    Since \( \cl \neq \cl' \),
    we know \( \conf' = \Tuple{\kvs, \vienv\UpdateFunc{\cl -> \vi }} \) 
    and \( \conf'' = \Tuple{\kvs', \vienv\UpdateFunc{ \cl -> \vi | \cl' -> \vi' }} \)
    for some view \( \vi' \),
    kv-store \( \kvs' = \UpdateKV(\kvs,\vienv(\cl'),\fp,\txid) \) with a fresh transition identifier \( \txid \),
    and \( \ToET{ \kvs | \vienv(\cl') | \fp | \kvs' | \vi' } \).
    Let \( \conf^* = \Tuple{\kvs', \vienv\UpdateFunc{ \cl' -> \vi' }}\) which gives us \cref{equ:view-left-move}. \qedhere
\end{enumerate}
\end{proof}


\begin{lemma}[View-shift absorption]
\label{lem:view-shift-absorb}
Given a trace \( \ettrc \in \ETTraces(\et) \) under \et,
two adjacent view-shifts \( \lbView{\vi},\lbView{\vi'} \) for a client \( \cl \) in the trace can be merged,
\begin{Formulae}
\begin{Formula}
    \Forall{ \ettrc',\ettrc'' | \lb,\lb' \in \Labels} 
    \\ \ettrc = \ToRed{\ettrc' | \lb -> \conf | \lbView{\vi} -> \conf' | \lbView{\vi'} 
        -> \conf'' | \lb' -> \ettrc''}
    \implies  \ToRed{\ettrc' | \lb -> \conf | \lbView{\vi'} -> \conf'' | \lb' ->  \ettrc''}.
    \label{equ:view-absorb}
\end{Formula}
\end{Formulae}
\end{lemma}
\begin{proof}
Let configuration \( (\kvs, \vienv ) = \conf \);
it is easy to see \( \conf = (\kvs, \vienv\UpdateFunc{\cl -> \vi})\), \( \conf = (\kvs, \vienv\UpdateFunc{\cl -> \vi'}) \)
and \( \vienv(\cl) \vileq \vi \vileq \vi' \), which implies \cref{equ:view-absorb}.
\end{proof}
\end{toappendix}

