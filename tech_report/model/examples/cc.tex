We now give the definitions of well-known consistency models in distributed databases,
including \emph{causal consistency} \((\CC)\) \citep{causal-def,cops,ev_transactions}, 
\emph{parallel snapshot isolation} \( (\PSI) \) \citep{PSI,NMSI},
\emph{snapshot isolation} \((\SI)\) \citep{si} 
and \emph{serialisability} \((\SER)\) \citep{Papadimitriou-ser}.
Researchers \citep{cp-def,giovanni_concur16,framework-concur} proposed that 
the definition of \( \SI \) on abstract executions
can be separated into two different consistency models,
\emph{update atomic} (\(\UA\)) and \emph{consistent prefix} \( (\CP) \).
They also realised that \( \PSI \) can be defined 
as the conjunction of \( \UA \) and \( \CC \) on abstract executions.
Note that,
in \cref{fig:execution-tests}, the \( \ViewShift \) for \( \CC \), \(\CP \), \( \PSI \) and \( \SI \)
are defined as the conjunction of the \(\MR\) and \(\RYW\) session guarantees.
This is because \(\MR\) and \( \RYW \) are easy to implement 
in that each client maintains some meta-data about it own history.
As explained in \cref{sec:bg-consis-model},
many consistency models are originally defined using specific implementation strategies
for tackling certain constraints in distributed databases.
 
\paragraph{Causal consistency (\(\CC\))} This models subsumes the four session guarantees discussed above. 
As such, the \(\ViewShift[\CC]\) predicate is defined as \( \ViewShift[\MR \cap \RYW] \).
%the \emph{conjunction} of their associated \ViewShift predicates.
%However, as shown in  \cref{fig:execution-tests}, it is sufficient to define \(\ViewShift[\CC]\)
%as the conjunction of the \(\MR\) and \(\RYW\) session guarantees alone, where for brevity we 
%write \(\ViewShift[\MR \cap \RYW]\) for  \(\ViewShift[\MR] \land \ViewShift[\RYW]\).
%This is because \(\ViewShift[\MW]\) and \(\ViewShift[\WFR]\) are defined simply as \( \true \), 
%allowing us to remove them from \(\ViewShift[\CC]\).
Additionally, \(\CC\) strengthens the session guarantees by 
requiring that if a client observes the effect of a transaction \( \txid \) prior to committing a transaction,
then it must also observes transactions \( \txid' \) that \( \txid \) observes.
%that is, \( \ver \) directly reads from \( \ver' \) or is from the same client as \( \ver \).
If a client observes a version written by \(\txid\), 
then \(\txid\) clearly  observes all transactions \(\txid'\) from which \( \txid \) directly reads. 
Moreover, \( \txid \) must observe previous transactions from the same client.
%Moreover, if \(\ver\) is, or it depends on, a version \(\ver'\) accessed (read or written) by a client \(\cl\), 
%%then it also depends on all versions that were previously read or written by \(\cl\). 
This is captured by the \(\CanCommit[\CC]\) predicate in \cref{fig:execution-tests}, 
defined as \(\PreClosed(\kvs, u, \rel[\CC])\) with \(\rel[\CC] \FuncDef \SO \cup \WR[\kvs]\).

For example, the kv-store in \cref{fig:wr-wfr-allowed-but-cc} is disallowed by \(\CC\): 
the version of key \(\key_3\) carrying value \(\val_3\) depends on 
the version of key \(\key_1\) carrying value \(\val_1\),
since \( \ToEdge{ \txid[\cl](1) | \SO -> \txid[\cl](2) | \WR[\kvs] 
            -> \txid[\cl'](1) | \SO -> \txid[\cl'](2)} \).
However, transaction \(\txid\) must have been committed by a client with 
view included \(\val_3\) but not \(\val_1\).

\begin{figure}
\centering
\begin{tikzpicture}%
\KVMapping{x}{\key_1}{
    /\val_0/\txid_0/\Set{\boldsymbol{\txid}}
    , /\val_1/\txid^{1}_{\cl}/\emptyset
};
\KVMapping[x]{y}{\key_2}{
    /\val_0/\txid_0/\emptyset
    , /\val_2/\txid^{2}_{\cl}/\{\txid^{1}_{\cl'}\}
};
\KVMapping[y]{z}{\key_3}{
    /\val_0/\txid_0/\emptyset
    , /\val_3/\txid^{2}_{\cl'}/\Set{\boldsymbol{\txid}}
};
\end{tikzpicture}%

\hrulefill

\caption{Anomaly disallowed by \( \CC \)}
\label{fig:wr-wfr-allowed-but-cc}
\end{figure}%
