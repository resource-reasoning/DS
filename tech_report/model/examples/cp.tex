\paragraph{Consistent prefix (\( \CP \))}
If the total order in which transactions commit is known, 
\emph{consistent prefix} (\(\CP\))  can be described as a strengthening of \(\CC\): 
if a client sees the versions written by a transaction \(\txid\),
then it must also see all versions written by transactions that \emph{commit} before \(\txid\). 
Although kv-stores only provide \emph{partial} information about 
the transaction commit order via the dependency relations
this is sufficient to formalise \emph{consistent prefix}.

\begin{figure}

\begin{subfigure}{\textwidth}
\centering
\begin{tikzpicture}%
\KVMapping{x}{\key_1}{
    /\val_0/\txid_0/\Set{\boldsymbol{\txid_{\cl'}^2}}
    , /\val_1/\txid/\Set{\txid_{\cl}^1}
};
\KVMapping[x]{y}{\key_2}{
    /\val_0/\txid_0/\Set{\txid_{\cl}^2}
    , /\val_2/\txid'/\Set{\txid_{\cl'}^1}
};
\end{tikzpicture}%

\caption{Long fork anomaly, disallowed by \(\CP\)}
\label{fig:cp-disallowed}

\end{subfigure} 


\begin{tabularx}{\textwidth}{c | X }
\hline
\begin{subfigure} {0.60\textwidth}
\begin{tikzpicture}%

\node at (0,0) (a) {\(\txid\)};
\node at (2,-1.5) (b) {\(\txid''\)};
\node at (5,0) (c) {\(\txid'\)};

\coordinate (a1) at ($(a) + (3,0)$);
\coordinate (b1) at ($(b) + (5,0)$);
\coordinate (c1) at ($(c) + (3,0)$);

\draw[|-|] (a) -- (a1);
\draw[|-|] (c) -- (c1);
\draw[|-,densely dashed] (b) -- ($(b)!0.5!(b1)$);
\draw[-|] ($(b)!0.5!(b1)$) -- (b1);

\path[->, thick] (a1) edge[bend left=50] node[right, text opacity=1] {\(\WR\) or \(\SO\)} ($(c)+(0.3,0.05)$);
\path[->, thick] (a1) edge[bend left=50] node[right, text opacity=1] {\(\texttt{commit before}\)} (c1);
\path[->, thick] ($(c)+(0.3,-0.05)$) edge node[right, text opacity=1] {\(\RW\)} (b1);
\path[->, thick] (a1) edge node[left, text opacity=1] {\(\texttt{commit before}\)} (b1);

\end{tikzpicture}%
\caption{Commit before relation: \( \WR \), \( \SO \), \( \WR ; \RW \) or \( \WR ; \RW \)}
\label{fig:commit-before-wr-rw}
\end{subfigure} 
&
\begin{subfigure} {0.39\textwidth}
\centering
\begin{tikzpicture}%
\node at (0,0) (a) {\(\txid\)};
\node at (1,-2) (b) {\(\txid'\)};

\coordinate (a1) at ($(a) + (3,0)$);
\coordinate (b1) at ($(b) + (3,0)$);

\draw[|-,densely dashed] (a) -- ($(a)!0.5!(a1)$);
\draw[-|] ($(a)!0.5!(a1)$) -- (a1);

\draw[|-,densely dashed] (b) -- ($(b)!0.5!(b1)$);
\draw[-|] ($(b)!0.5!(b1)$) -- (b1);

\path[->, thick] (a1) edge node[text opacity=1] {\(\WW,\texttt{commit before}\)} (b1);
\end{tikzpicture}%
\caption{Commit before: \(\WW\)}
\label{fig:commit-before-ww}
\end{subfigure} 
\\
\hline 
\end{tabularx}


\caption{Long fork anomaly and commit before relation}
\end{figure}%


In practice, we approximate the order in which transactions commit in a trace 
that terminates with a final kv-store \kvs via the \(\WR[\kvs], \WW[\kvs], \RW[\kvs] \) 
and \( \SO\)  relations. 
This approximation defined by 
\[
    \rel[\CP] \FuncDef \WR[\kvs]; \Refl(\RW[\kvs]) \cup \SO; \Refl(\RW[\kvs]) \cup \WW,
\]
is best understood in terms of an idealised implementation of \(\CP\) on a \emph{centralised database},
where the snapshot of a transaction is determined at its \emph{start point}
and its effects are made visible to future transactions at its \emph{commit point}.
With respect to this centralised implementation, first, if \((\txid,\txid') \in \WR[\kvs]\), 
then \(\txid\) must commit before \(\txid'\) starts, and hence before \(\txid'\) commits (\cref{fig:commit-before-wr-rw}).
Similarly, if \((\txid,\txid') \in \SO\), then \(\txid\) commits before \(\txid'\) starts, 
and thus before \(\txid'\) commits (\cref{fig:commit-before-wr-rw}).
Second, recall that \((\txid', \txid'') \in \RW[\kvs]\) denotes that 
\(\txid''\) reads a version that is later overwritten by \(\txid'\);
that is, \(\txid'\) cannot see the write of \(\txid''\), 
and thus \(\txid'\) must starts before \(\txid''\) commits. 
As such, if \(\txid\) commits before \(\txid'\) starts as
\((\txid, \txid') \in \WR[\kvs]\) or \((\txid,\txid') \in \SO\), 
and \(\txid'\) starts before \(\txid''\) commits as \((\txid', \txid'') \in \RW[\kvs]\), 
then \(\txid\) must commit before \(\txid''\) commits (\cref{fig:commit-before-wr-rw}). 
In other words, if \((\txid,\txid'') \in \WR[\kvs];\RW[\kvs]\) or \((\txid,\txid'') \in \SO;\RW[\kvs]\), 
then \(\txid\) commits before \(\txid''\) (\cref{fig:commit-before-wr-rw}).
Last, if \((\txid,\txid') \in\WW[\kvs]\), then \(\txid\) must commit before \(\txid'\) (\cref{fig:commit-before-ww}). 
Therefore the relation, \(\rel[\CP] \), approximates the order in which transactions commit. 
We then define \(\CanCommit[\CP]\) in \cref{fig:execution-tests} by requiring that
the client view be closed with respect to \(\rel[\CP]\).
Our definition for \( \CP \) via approximating the commit order, 
is correct with respect to the declarative definition proposed by \citet{laws}.

Consistent prefix disallows the \emph{long fork anomaly} shown in \cref{fig:cp-disallowed}, 
where clients \(\cl\) and \(\cl'\) observe the updates to \(\key_1\) and \(\key_2\) 
in different orders. 
Assuming without loss of generality that \( \txid_{\cl'}^{2} \) was the last transaction committed, 
then prior to committing its transaction \(\cl'\) must see 
the initial version of \(\key_1\) with value \(\val_0\) and both version of \( \key_2 \);
this is because a view must always include the initial versions and \( \cl' \) read the second version 
of \( \key_2 \) carrying value \( \val_2 \) in transaction \( \txid[\cl'](1)\).
However, since 
\( \ToEdge{ \txid | \WR[\kvs] -> \txid_{\cl}^{1} | \SO
            -> \txid_{\cl}^{2} | \RW[\kvs] -> \txid' } \),
then \(\cl'\) should also see the second version of \(\key_1\) with value \(\val_1\), 
leading to a contradiction.

