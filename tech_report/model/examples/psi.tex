\paragraph{Parallel snapshot isolation (\(\PSI\))} 
This model can be informally described as:
\begin{enumerate}
\item when a transaction \( \txid \) observes the effect of another transaction \( \txid \),
it must also observes the effects of transactions that \( \txid' \) observes;
\item when two transactions, \( \txid \) and \( \txid'\) write to the same key, one must see the effects by another.
\end{enumerate}
Intuitively, this models is the combination of causal consistency and update atomic.
In abstract executions with a total order over transactions,
This model is indeed defined as the conjunction of the definitions of \(\CC\) and \(\UA\) \citep{framework-concur}. 
%As such, the \(\ViewShift[\PSI]\) predicate is defined as the conjunction of the \(\ViewShift\) 
%predicates for \(\CC\) and \(\UA\).
However, we cannot simply define \(\CanCommit[\PSI]\) 
as the conjunction of the \(\CanCommit\) predicates for \(\CC\) and \(\UA\). 
Because in our semantics we do not have the total order.
However, the dependency relations provides enough information.
The challenge here is the meaning of \emph{observation} between transactions.
In \( \CC \), the dependence are straightforwardly defined \( \rel[\CC] = \WR \cup \SO\).
This is not enough here.
Recall that \(\CanCommit[\UA]\) requires that a transaction writing to a key \(\key\) 
must be able to see all previous versions of \(\key\). 
This means that when write-conflict freedom is enforced,
a version \(\ver\) of \(\key\) \emph{observes} on all previous versions of \(\key\),
for example \cref{fig:example-of-concurrent-write}.
This observation leads us to include write-write dependencies (\(\WW[\kvs]\)) in \(\rel[\PSI]\). 
Hence, the closure relation for \( \PSI \) is defined by \( \)
The kv-store in \cref{fig:cc-ua-allowed-but-psi} 
shows an example kv-store that satisfies \(\CanCommit[\CC] \land \CanCommit[\UA]\), 
but not \(\CanCommit[\PSI]\),
since \( \txid \) read the third version of key \( \key_1 \) written by \( \txid[\cl'](1) \)
but not the second version of key \( \key_2 \) written by \( \txid[\cl](1)\), 
where \( (\txid[\cl'](1),\txid[\cl](1)) \in \WW[\kvs]\).

%This is for two reasons. 
%First, their conjunction would only mandate that view \(\vi\) be closed with respect to 
%\(\rel[\CC]\) and \(\rel[\UA]\) \emph{individually}, 
%but \emph{not} with respect to their \emph{union}
%(recall that closure is defined in terms of the transitive closure of 
%a given relation and thus the closure of \(\rel[\CC]\) and \(\rel[\UA]\) 
%is smaller than the closure of \(\rel[\CC] \cup \rel[\UA]\)).
%As such, we define \(\CanCommit[\PSI]\) as closure with respect to \(\rel[\PSI]\)
%which must include \(\rel[\CC] \cup \rel[\UA]\).
