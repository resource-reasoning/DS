\subsection{KV-Store Traces to Abstract Execution Traces}
\label{sec:kvtrace2aexec}

To prove our definitions using execution test on kv-stores 
is sound and complete with respect with the axiomatic definitions on abstract executions (\cref{sec:kv-sound-complete-proof}),
we need to prove trace equivalent between these two models.

In this section, we only consider the trace that does not involve \( \prog \) but only committing fingerprint and view shift.
In \cref{sec:kv-sound-complete-proof}, we will go further and discuss the trace installed with \( \prog \).

Similar to \(\anarchic\), let $\ET_\top$ be the most permissive execution test.
That is $\ET_\top \vdash (\hh, \vi) \csat \opset: (\mkvs',\vi')$ 
such that whenever $\vi(\ke) \neq \vi'(\ke)$ then either $(\otW, \ke, \stub) \in \opset$ or $(\otR, \ke, \stub) \in \opset$.
%{\color{red} I forgot this last constraint in the latest version of the document, definitions 
%and proofs of theorems that follow must be re-factored to take the constraint into account.}
We will relate $\ET_{\top}$-traces to abstract executions that satisfy the last write wins resolution policy, \ie \( (\RP_{\LWW}, \emptyset) \).

To bridge $\ET_{\top}$-traces to abstract executions, 
The \aeset(\tr) function converse the trace of \( \ET_\top \) to set of possible abstract executions (\cref{def:kvtrace2aexec}).
In fact, for any trace \( \tr \) and abstract execution $\aexec \in \aeset(\tr)$, 
the last configuration of $\tr$ is $(\hh_{\aexec}, \stub)$ (\cref{prop:kvtrace2aexec}).
We often use \( \aexec_\tr \) for \( \aexec \in \aeset(\tr) \).

\begin{definition}
\label{def:kvtrace2aexec}
Given a kv-store $\hh$, a view $\vi$, 
an initial abstract execution $\aexec_0 = ( [ ], \emptyset, \emptyset)$, 
an abstract execution $\aexec$, a set of transactions  
$\T \subseteq \T_{\aexec}$, a transaction identifier $\txid$ and a set of operations $\opset$,
the \( \extend \)  function defined as the follows:
\[
\begin{rclarray}
\extend(\aexec, \txid, \T, \opset) & \defeq &
\begin{cases}
\text{undefined} & \text{if }  \txid = \txid_{0}\\
\left(\TtoOp{T}_{\aexec} \uplus \Set{\txid \mapsto \opset}, \VIS', \AR' \right) & \text{if } \dagger \\
\end{cases} \\
\dagger & \equiv &  
\begin{array}[t]{@{}l}
\txid = \txid_{\cl}^{n}
\wedge \VIS' = \VIS_{\aexec} \uplus \Setcon{(\txid', \txid) }{ \txid \in \T}  \\
{} \wedge \AR' = \AR_{\aexec} \uplus \Setcon{(\txid', \txid) }{ \txid' \in \T_{\aexec}}
\end{array}
\end{rclarray}
\]
Given a $\ET_{\top}$ trace $\tr$, let $\lastConf(\tr)$ be the last configuration appearing in $\tr$.
The set of abstract executions $\aeset(\tr)$ is defined as the smallest set such that:
\begin{itemize}
\item $\aexec_{0} \in \aeset((\hh_{0}, \viewFun_{0}))$, 
\item if $\aexec \in \aeset(\tr)$, then $\aexec \in \aeset\left(\tr \xrightarrowtriangle{(\cl, \varepsilon)}_{\ET_{\top}} (\hh, \viewFun) \right)$, 
\item if $\aexec \in \aeset(\tr)$, then $\aexec \in \aeset\left(\tr \xrightarrowtriangle{(\cl, \emptyset)}_{\ET_{\top}} (\hh, \viewFun) \right)$, 
\item 
    let $(\hh', \viewFun') = \lastConf(\tr)$; 
    if $\aexec \in \aeset(\tr)$, $\opset \neq \emptyset$,
    and $\T = \Tx(\hh, \viewFun'(\cl)) \cup \T_\rd$ where \( \T_\rd \) is a set of \emph{read-only transactions}
    such that $(\otW, \ke, \val) \notin_{\aexec} \txid'$ for all keys \( \ke \) and values \( \val \) and transactions \( \txid' \in \T_\rd\),
    and if the transaction $\txid$ is the transaction appearing in $\lastConf(\tr)$ but not in $\hh$, 
    then $\extend(\aexec, \txid, \T, \opset) \in \aeset\left(\tr \xrightarrowtriangle{(\cl, \opset)}_{\ET_{\top}} (\hh, \viewFun) \right)$.
\end{itemize}
\end{definition}

\begin{proposition}[Trace of \( \ET \) to abstract executions]
\label{prop:kvtrace2aexec}
For any $\ET_{\top}$-trace $\tr$, 
the abstract execution $\aexec \in \aeset(\tr)$ satisfies the last write wins policy,
and $(\hh_{\aexec}, \stub) = \lastConf(\tr)$.
\end{proposition}
\begin{proof}
Fix a $\ET_{\top}$-trace $\tau$. 
We prove by induction on the number of transitions $n$ in $\tau$. 
\begin{itemize}
\item \caseB{$n = 0$}
It means $\tr = (\hh_{0}, \stub)$.
It follows from \cref{def:kvtrace2aexec} that $\aexec_{\tau} = ([], \emptyset, \emptyset)$. 
This triple satisfies the constraints of \cref{def:aexec}, as well as the resolution policy $\RP_{\LWW}$. 
It is also immediate to see that $\graphOf(\aexec) = ([], \emptyset, \emptyset, \emptyset)$.
In particular, $\T_{\graphOf(\aexec)} = \emptyset$, 
and the only kv-store $\hh$ such that $\T_{\Gr_{\hh}} = \emptyset$ 
is given by $\hh = \hh_{0}$. 
By definition, $\hh_{\aexec_{\tr}} = \hh_{0}$, as we wanted to prove.

\item \caseI{$n > 0$} In this case, we have that $\tr = \tr' \xrightarrowtriangle{(\cl, \mu)} (\hh, \viewFun)$ 
for some $\cl, \mu, \hh, \viewFun$. The $\ET_{\top}$-trace $\tau'$ contains exactly $n-1$ transitions, 
so that by induction we can assume that $\aexec_{\tau'}$ is a valid abstract execution that satisfies 
$\RP_{\LWW}$. and $\lastConf(\tau') = (\hh_{\aexec_{\tau'}}, \viewFun')$ for some $\viewFun'$. 

We perform a case analysis on $\mu$. 
If $\mu = \varepsilon$, then it follows that $\hh = \hh_{\aexec_{\tau'}}$, 
and $\aexec_{\tau} = \aexec_{\tau'}$ by \cref{def:kvtrace2aexec}. 
Then by the inductive hypothesis $\aexec_{\tau}$ is an abstract execution that satisfies $\RP_{\LWW}$,
$\lastConf(\tau) = (\hh, \stub)$, and $\hh_{\aexec_{\tau}} = \hh_{\aexec_{\tau'}} = \hh$, 
and there is nothing left to prove. 

Suppose now that $\mu = \opset$, for some $\opset$. In this case we have that  
$\hh \in \updateKV(\hh_{\aexec_{\tau'}}, \viewFun'(\cl), \opset, \cl)$. Note that if 
$\opset = \emptyset$, then $\hh = \hh_{\aexec_{\tau'}}$ and $\aexec_{\tau} = \aexec_{\tau'}$. 
By the inductive hypothesis, $\aexec_{\tau}$ is an abstract execution that satisfies 
$\RP_{\LWW}$, and $\hh = \hh_{\aexec_{\tau'}} = \hh_{\aexec_{\tau}}$. 
Assume then that $\opset \neq \emptyset$. By definition, $\hh = \updateKV(\hh_{\aexec_{\tau'}}, 
\viewFun'(\cl), \opset, \txid)$ for some $\txid \in \nextTxId(\cl, \hh_{\aexec_{\tau}})$. It follows that $\txid$ 
is the unique transaction such that $\txid \notin \hh_{\aexec_{\tau'}}$, and $\txid \in \hh$ 
(the fact that $\txid \in \hh$ follows from the assumption that $\opset \neq \emptyset$). Let 
$\T = \Tx(\hh_{\aexec_{\tau'}}, \viewFun'(\cl))$; then $\aexec_{\tau} = \extend(\hh_{\aexec_{\tau'}}, \txid, \T, \opset)$. 
Note that $\aexec_{\tau}$ satisfies the constraints of abstract execution required by \cref{def:aexec}:
\begin{itemize}
\item  Because $\txid \in \nextTxId(\cl, \hh_{\aexec_{\tau}})$, it must be the case that $\txid = \txid_{\cl}^{m}$ for some 
$m \geq 1$; we have that $\TtoOp{T}_{\aexec_{\tau}} = \TtoOp{T}_{\aexec_{\tau'}}\rmto{\txid_{\cl}^{m}}{\opset}$, 
from which it follows that 
\[
\T_{\aexec_{\tau}} = \dom(\TtoOp{T}_{\aexec_{\tau}}) = \dom(\TtoOp{T}_{\aexec_{\tau'}}) \cup 
\Set{\txid_{\cl}^{m} } = \T_{\aexec_{\tau'}} \cup \Set{\txid_{\cl}^{m} }
\]
By inductive hypothesis, $\txid_0 \notin \T_{\aexec_{\tau'}}$, and therefore $\txid_{0} \notin 
\T_{\aexec_{\tau'}} \cup \Set{\txid_{\cl}^{m} } = \T_{\aexec}$.

\item \( \VIS_{\aexec_{\tau}} \subseteq \AR_{\aexec_{\tau}} \).
    Let $(\txid' ,\txid'') \in \VIS_{\aexec_{\tau}}$. Then either $\txid'' = \txid_{\cl}^{m}$ and $\txid' \in \T$, or $(\txid', \txid'') \in 
\VIS_{\aexec_{\tau'}}$. In the former case, we have that $(\txid', \txid_{\cl}^{m}) \in \AR_{\aexec_{\tau}}$ by definition; 
in the latter case, we have that $(\txid', \txid'') \in \AR_{\aexec_{\tau'}}$ because $\aexec_{\tau'}$ is a valid 
abstract execution by inductive hypothesis, and therefore $(\txid', \txid'') \in \AR_{\aexec_{\tau}}$ by definition. 
This concludes the proof that $\VIS_{\aexec_{\tau}} \subseteq \AR_{\aexec_{\tau}}$. 
\item \( \VIS_{\aexec_\tr} \) is irreflexive.
Assume $(\txid', \txid'') \in \VIS_{\aexec_{\tau}}$, then either 
$(\txid' \txid'') \in \VIS_{\aexec_{\tau'}}$, and because $\VIS_{\aexec_{\tau'}}$ is irreflexive by the inductive hypothesis, 
then $\txid' \neq \txid''$; 
or $\txid'' = \txid_{\cl}^{m}$, $\txid' \in \T \subseteq \T_{\aexec_{\tau'}}$, 
and because $\txid_{\cl}^{m} \notin \hh_{\aexec_{\tau'}}$, then $\txid' \neq \txid_{\cl}^{m}$. 

\item $\AR_{\aexec_{\tau}}$ is total. Let $(\txid', \txid'') \in \T_{\aexec_{\tau}}$. 
Suppose that $\txid' \neq \txid''$.
\begin{enumerate}
\item If $\txid' \neq 
\txid_{\cl}^{m}$, $\txid'' \neq \txid_{\cl}^{m}$, then it must be the case that $\txid', \txid'' \in \T_{\aexec_{\tau'}}$; 
this is because we have already argued that $\T_{\aexec_{\tau}} = \T_{\aexec_{\tau'}} \cup \Set{\txid_{\cl}^{m}}$. 
By the inductive hypothesis, we have that either $(\txid', \txid'') \in \AR_{\aexec_{\tau'}}$, or 
$(\txid'', \txid') \in \AR_{\aexec_{\tau'}}$. Because $\AR_{\aexec_{\tau'}} \subseteq \AR_{\aexec_{\tau}}$, 
then either $(\txid', \txid'') \in \AR_{\aexec_{\tau'}}$ or $(\txid'', \txid') \in \AR_{\aexec_{\tau}}$. 
\item if $\txid'' = \txid_{\cl}^{m}$, then it must be $\txid' \in \T_{\aexec_{\tau'}}$. By definition, 
$(\txid', \txid_{\cl}^{m}) \in \AR_{\aexec_{\tau}}$. Similarly, if $\txid' = \txid_{\cl}^{m}$, we 
can prove that $(\txid'', \txid_{\cl}^{m}) \in \AR_{\aexec_{\tau}}$.
\end{enumerate}
\item  $\AR_{\aexec_{\tau}}$ is irreflexive. It follows is the same as the one of $\VIS_{\aexec_{\tau}}$.
\item \( \AR_{\aexec_{\tau}} \) is transitive.
Assume $(\txid', \txid'') \in \AR_{\aexec_{\tau}}$ and $(\txid'', \txid''') \in \AR_{\aexec_{\tau}}$. 
Note that it must be the case that $\txid', \txid'' \in \T_{\aexec_{\tau'}}$ by the definition of 
$\AR_{\aexec}$, and in particular $(\txid', \txid'') \in \AR_{\aexec_{\tau'}}$. 
For $\txid'''$, we have two possible cases. 
\begin{enumerate}
\item Either $\txid''' \in \T_{\aexec_{\tau}}$, from 
which it follows that $(\txid'', \txid''') \in \AR_{\aexec_{\tau'}}$; because
of $\AR_{\aexec_{\tau'}}$ is transitive by the inductive hypothesis, then 
$(\txid', \txid''') \in \AR_{\aexec_{\tau'}}$, and therefore $(\txid' ,\txid''') \in 
\AR_{\aexec_{\tau}}$.
\item Or $\txid''' = \txid_{\cl}^{m}$, and because $\txid' \in \T_{\aexec_{\tau'}}$, then 
$(\txid', \txid_{\cl}^{m}) \in \AR_{\aexec_{\tau}}$ by definition. 
\end{enumerate}
\item \( \SO_{\aexec_{\tau}} \subseteq \AR_{\aexec_{\tau}} \).
Let $\cl'$ be a client such that $(\txid_{\cl'}^{i}, \txid_{\cl'}^{j}) \in \AR_{\aexec_{\tau}}$. 
If $\cl' \neq \cl$, then it must be the case that $\txid_{\cl'}^{i}, \txid_{\cl'}^{j} \in \T_{\aexec_{\tau'}}$, 
and therefore $(\txid_{\cl'}^{i}, \txid_{\cl'}^{j}) \in \AR_{\aexec_{\tau'}}$. By the inductive hypothesis, 
it follows that $i < j$. If $\cl' = \cl$, then by definition of $\AR_{\aexec_{\tau}}$ it must be  $i \neq m$. 
If $j \neq m$ we can proceed as in the previous case to prove that $i < j$. If $j = m$, then 
note that $\txid_{\cl}^{i} \in \T_{\aexec_{\tau}}$ only if $\txid_{\cl}^{i} \in \hh_{\aexec_{\tau'}}$. 
Because $\txid_{\cl}^{m} \in \nextTxId(\hh_{\aexec_{\tau'}}, \cl)$, then we have that $i < m$, 
as we wanted to prove.
\end{itemize}

Next, we prove that $\aexec_{\tau}$ satisfies the last write wins policy. 
Let $\txid' \in \T_{\aexec_{\tau}}$, and suppose that $(\otR, \ke, \val) \in_{\aexec_{\tau}} \txid'$. 
\begin{itemize} 
\item If $\txid' \neq \txid$, then we have that $\txid \in \T_{\aexec_{\tau'}}$. We also have that 
$\VIS^{-1}_{\aexec_{\tau}}(\txid') = \VIS^{-1}_{\aexec_{\tau'}}(\txid')$, $\AR^{-1}_{\aexec_{\tau}}(\txid') 
= \AR^{-1}_{\aexec_{\tau'}}(\txid')$; finally, for any $\txid'' \in \T_{\aexec_{\tau'}}$, 
$(\otW, \ke, \val') \in_{\aexec_{\tau}} \txid''$ if and only if $(\otW, \ke, \val') \in_{\aexec_{\tau'}} 
\txid''$. Therefore, let $\txid_{r} = \max_{\AR_{\aexec_{\tau}}}(\VIS^{-1}_{\aexec_{\tau}}(\txid') \cap 
\Setcon{\txid'' }{ (\otW, \ke, \stub) \in_{\aexec_{\tau}} \txid''})$. We have that $\txid_{r} = \max_{\AR_{\aexec_{\tau'}}}(\VIS^{-1}_{\aexec_{\tau'}}(\txid) 
\cap \Setcon{ \txid'' }{ (\otW, \ke, \stub \in_{\aexec_{\tau'}} \txid''})$, and because $\aexec_{\tau'}$ satisfies the last write 
wins resolution policy,then $(\otW, \ke, \val) \in_{\aexec_{\tau'}} \txid_{r}$. This also implies that 
$(\otW, \ke, \val) \in_{\aexec_{\tau}} \txid_{r}$. 

\item Now, suppose that $\txid' = \txid$. Suppose that $(\otR, \ke, \val) \in_{\aexec_{\tau}} \txid'$. 
By definition, we have that $(\otR, \ke, \val) \in \opset$. Recall that $\tau = \tau' \xrightarrow{(\cl, \opset)}_{\ET_{\top}} (\hh, \viewFun)$, 
and $\lastConf(\tau') = (\hh_{\aexec_{\tau'}}, \viewFun')$ for some $\viewFun'$. 
That is, 
\[
    (\hh_{\aexec_{\tau'}}, \viewFun') \xrightarrowtriangle{(\cl, \opset)}_{\ET_{\top}} (\hh, \viewFun)
\]
which in turn implies that $\ET_{\top} \vdash (\hh_{\aexec_{\tau'}}, \viewFun'(\cl)) \csat \opset : (\mkvs,\viewFun(\cl) )$. 
Let then $r = \max\Setcon{i}{  i \in \viewFun'(\cl)(\ke)}$. 
By definition of execution test, and because $(\otR, \ke, \val) \in \opset$, then it must be the case that 
$\hh_{\aexec_{\tau'}}(\ke, r) = (\val, \txid'', \stub)$ for some $\txid''$. 

We now prove that 
$\txid'' = \max_{\AR_{\aexec_{\tau}}}(\VIS^{-1}_{\aexec_{\tau}}(\txid) \cap \Setcon{ \txid'' }{ (\otW, \ke, \stub) \in_{\aexec_{\tau}} \txid''})$. 
First we have
\[ 
\begin{array}{l}
\VIS^{-1}_{\aexec_{\tau}}(\txid) = 
\Tx(\hh_{\aexec_{\tau'}}, \viewFun'(\cl)) = 
\Setcon{\WTx(\hh_{\aexec_{\tau'}}(\ke',  i)) }{ \ke' \in \Keys \wedge  i \in \viewFun'(\cl)(\ke')}
\end{array}
\]
Note that $r \in \viewFun'(\cl)(\ke)$, and $\txid'' = \WTx(\hh_{\aexec_{\tau'}}(\ke, r))$. 
Therefore, $\txid'' \in \VIS^{-1}_{\aexec_{\tau}}(\txid)$. 
Because $\hh = \updateKV(\hh_{\aexec_{\tau'}}, \viewFun'(\cl), \opset, \txid)$, it 
must be the case that $\WTx(\hh(\ke, r)) = \txid''$. Also, because $\WTx(\hh_{\aexec_{\tau'}}(\ke, r)) = \txid''$, 
then $(\otW, \ke, \stub) \in_{\aexec_{\tau''}} \txid''$, or equivalently $(\otW, \ke, \stub) \in \TtoOp{T}_{\aexec_{\tau'}}(\txid'')$. 
We have already proved that $\VIS_{\aexec_{\tau}}$ is irreflexive, hence it must be the case that $\txid'' \neq \txid$. 
In particular, because $\aexec_{\tau} = \extend(\aexec_{\tau'}, \txid, \stub, \stub)$, then we have that 
$\TtoOp{T}_{\aexec_{\tau}}(\txid'') = \TtoOp{T}_{\aexec_{\tau'}}\rmto{\txid}{\opset}(\txid'') = 
\TtoOp{T}_{\aexec_{\tau'}}(\txid'')$, hence $(\otW, \ke, \stub) \in \TtoOp{T}_{\aexec_{\tau}}(\txid'')$. Equivalently, 
$(\otW, \ke, \stub) \in_{\aexec_{\tau}} \txid''$. We have proved that $\txid'' \in \VIS^{-1}_{\aexec_{\tau}}(\txid)$, 
and $(\otW, \ke, \stub) \in_{\aexec_{\tau}} \txid''$. 

Now let $\txid'''$ be such that $\txid''' \in \VIS^{-1}_{\aexec_{\tau}}(\txid)$, and $(\otW, \ke \stub) \in_{\aexec_{\tau}} \txid'''$. 
Note that $\txid''' \neq \txid$ because $\VIS_{\aexec_{\tau}}$ is irreflexive.
We show that either $\txid''' = \txid''$, or $\txid''' \xrightarrow{\AR_{\aexec_{\tau}}} \txid''$. 
Because $\txid''' \in \VIS^{-1}_{\aexec_{\tau}}(\txid)$, then there exists a key $\ke'$ and an index $i \in \viewFun'(\cl)$ 
such that $\WTx(\hh_{\aexec_{\tau'}}(\ke', i)) = \txid'''$. Because $(\otW, \ke, \stub) \in_{\aexec_{\tau}} \txid'''$, 
and because $\txid''' \neq \txid$, then $(\otW, \ke, \stub) \in_{\aexec_{\tau'}} \txid'''$, and therefore there exists 
an index $j$ such that $\WTx(\hh_{\aexec_{\tau'}}(\ke, j)) = \txid'''$. We have that $\WTx(\hh_{\aexec_{\tau'}}(\ke, j) = 
\WTx(\hh_{\aexec_{\tau'}}(\ke', i))$, and $i \in \viewFun'(\cl)$. By \cref{eq:view.atomic}, it must be $j \in \viewFun'(\cl)$. 
Note that $r = \max\Setcon{i }{ i \in \viewFun'(\cl)}$, hence we have that $j \leq r$. If $j = r$, then $\txid''' = \txid''$ and 
there is nothing left to prove. If $j < r$, then we have that $(\txid''', \txid'') \in \AR_{\aexec_{\tau'}}$, and 
therefore $(\txid''', \txid'') \in \AR_{\aexec_{\tau}}$.
\end{itemize}
Finally, we need to prove that $\hh = \hh_{\aexec_{\tau}}$.
Recall $\hh = \updateKV(\hh_{\aexec_{\tau'}}, \viewFun'(\cl), \opset, \txid)$, 
and $\aexec_{\tau} = \extend(\aexec_{\tau'}, \txid, \Tx(\hh_{\aexec_{\tau'}}, \viewFun'(\cl), \opset)$. 
The result follows then from \cref{prop:extend.update.sameop}. 
\end{itemize}
\end{proof}


\begin{proposition}[\( \extend \) matching \( \updateKV\)]
\label{prop:extend.update.sameop}
Given an abstract execution $\aexec$, a set of transactions $\T \subseteq \T_{\aexec}$,
a transaction $\txid \notin \T_{\aexec}$, and a fingerprint $\opset \subseteq \powerset{\Ops}$,
if the new abstract execution $\aexec' = \extend(\aexec, \T, \txid, \opset)$,
and the view $\vi = \getView(\hh_{\aexec}, \T)$,
then $\updateKV(\hh_{\aexec}, \vi, \opset, \txid) = \hh_{\aexec'}$.
\end{proposition}

\begin{proof}
Let $\Gr = \Gr_{\updateKV(\hh_{\aexec}, \vi, \opset, \txid)}$, $\Gr' = \graphOf(\aexec')$. 
Note that $\hh_{\aexec'}$ is the unique kv-store such that $\Gr_{\hh_{\aexec'}} = \graphOf(\aexec') = \Gr'$. 
It suffices to prove that $\Gr = \Gr'$. Because the function $\Gr_{\cdot}$ is injective, it follows that 
$\updateKV(\hh_{\aexec}, \vi, \opset, \txid) = \hh_{\aexec'}$, as we wanted to prove.  

The proof is a consequence of \cref{lem:graph.extend} and \cref{lem:graph.update}. 
Consider the dependency graph $\Gr_{\hh_{\aexec}}$.
Recall that $\hh_{\aexec}$ is the unique kv-store such that $\Gr_{\hh_{\aexec}} = \graphOf(\aexec)$. 
We prove that $\TtoOp{T}_{\Gr} = \TtoOp{T}_{\Gr'}$, $\RF_{\Gr} = \RF_{\Gr'}$ and 
$\VO_{\Gr} = \VO_{\Gr'}$ (from the last two it follows that $\AD_{\Gr} = \AD_{\Gr'}$). 
\begin{itemize}
\item It is easy to see $\TtoOp{T}_{\Gr} = \TtoOp{T}_{\Gr'}$.

\item $\RF_{\Gr} = \RF_{\Gr'}$.
Let \( \mkvs  = \mkvs_\aexec \).
Suppose that $\txid' \xrightarrow{\RF_{\Gr}(\ke)} \txid''$ for some $\txid', \txid''$. 
By \cref{lem:graph.update} we have that either $\txid' \xrightarrow{\RF_{\Gr_\hh}(\ke)} \txid''$, 
or $\txid'' = \txid$, $(\otR, \ke, \stub) \in \opset$, $\txid' = \max_{\VO_{\Gr_\hh}(\ke)}\Setcon{\WTx(\ke, i) }{ i \in \vi(\ke)}$. 

\begin{itemize}
\item If $\txid' \xrightarrow{\RF_{\Gr_\hh}(\ke)} \txid''$, then because 
$\Gr_\mkvs = \graphOf(\aexec)$, we have that $\txid' \xrightarrow{\RF_{\graphOf(\aexec)}(\ke)} \txid''$. 
Recall that $\Gr' = \graphOf(\extend(\aexec, \T, \txid, \opset))$, hence by \cref{lem:graph.extend} 
we obtain that $\txid' \xrightarrow{\RF_{\Gr'}(\ke)} \txid''$. 

\item If $\txid'' = \txid$, $(\otR, \ke, \stub) \in \opset$, and $\txid' = \max_{\VO_{\Gr_\mkvs}(\ke)} \Setcon{ \WTx(\hh_{\aexec}(\ke, i)) }{ i \in \vi(\ke)}$, 
    then we also have that $\txid' = \max_{\VO_{\graphOf(\aexec)}(\ke)} (\T \cap \Setcon{ \txid''' }{ (\otW, \ke, \stub) \in_{\aexec} \txid'''}) $. 
This is because of the assumption that 
\[
\begin{array}{l}
    \Setcon{\WTx(\hh_{\aexec}(\ke, i)) }{ i \in \vi(\ke)}
    \begin{array}[t]{l}
    = \Setcon{\WTx(\hh_{\aexec}(\ke', i)) }{ \ke' \in \Keys \wedge i \in \vi(\ke')} \cap \Set{\WTx(\hh_{\aexec}(\ke, \stub)} \\
= \Tx(\hh_{\aexec}, \vi) \cap \Set{\WTx(\hh_{\aexec}(\ke, \stub)}  \\
= \T \cap \Setcon{(\txid''' }{ (\otW, \ke, \stub) \in_{\aexec} \txid''')}
\end{array}
\end{array}
\]
Again, it follows from \cref{lem:graph.extend} that $\txid' \xrightarrow{\RF_{\Gr'}(\ke)} \txid''$. 
\end{itemize}
\item \( \VO_{\Gr} = \VO_{\Gr'}\). The \( \VO_{\Gr} = \VO_{\Gr'} \) follows the similar reasons as $\RF_{\Gr} = \RF_{\Gr'}$.
\end{itemize}
\end{proof}

\begin{lemma}[Graph to abstract execution extension]
\label{lem:graph.extend}
Let $\aexec$ be an abstract execution, 
$\txid \notin \T_{\aexec} \cup \Set{\txid_0}$ be a transaction identifier $\T_{\aexec}$, and $\opset \in \Tx$. 
Let $\T \subseteq \T_{\aexec}$ be a set of transaction identifiers.
Let $\Gr = \graphOf(\aexec), \Gr' = \graphOf(\extend(\aexec, \txid, \T, \opset))$. 
We have the following: 
\begin{enumerate}
\item for any $\txid' \in \T_{\Gr'}$, either $\txid' \in \T_{\Gr}$ and $\TtoOp{T}_{\Gr}(\txid') = \TtoOp{T}_{\Gr'}(\txid')$, 
or $\txid' = \txid$ and $\TtoOp{T}_{\Gr'}(\txid) = \opset$.
\item $\txid' \xrightarrow{\RF_{\Gr'}(\ke)} \txid''$ if and only if either 
$\txid' \xrightarrow{\RF_{\Gr}(\ke)_{\Gr}} \txid''$, or $(\otR, \ke, \stub) \in \opset$, $\txid'' = \txid$ and 
$\txid' = \max_{\VO_{\Gr}(\ke)}(\T)$, 
\item $\txid' \xrightarrow{\VO_{\Gr'}(\ke)} \txid''$ if and only if 
either $\txid' \xrightarrow{\VO_{\Gr}(\ke)} \txid''$, or $(\otW, \ke, \stub) \in \opset$, $\txid'' = \txid$, 
and $(\otW, \ke, \stub) \in_{\Gr} \txid'$.
\end{enumerate}
\end{lemma}

\begin{proof}
Fix a key $\ke$. Let $\aexec' = \extend(\aexec, \txid, \T, \opset)$. Recall that $\Gr' = \graphOf(\aexec')$.

\begin{enumerate}
\item By definition of $\extend$, and 
because $\txid \notin \T_{\aexec}$, we have that 
$\T_{\aexec'} = \T_{\aexec} \uplus \Set{\txid}$. Furthermore, $\TtoOp{T}_{\aexec'}(\txid) = \opset$, 
from which it follows that $\TtoOp{T}_{\Gr'}(\txid) = \opset$.
For all $\txid' \in \T_{\aexec}$, we have that $\TtoOp{T}_{\aexec'}(\txid') = 
\TtoOp{T}_{\aexec}(\txid') = \TtoOp{T}_{\Gr}(\txid')$.
\item
There are two cases that either the \( \txid'' \) already exists in the dependency graph before,
or it is the newly committed transaction.
\begin{itemize}
\item Suppose that $\txid' \xrightarrow{\RF(\ke)_{\Gr}} \txid''$ for some $\txid', \txid'' \in \T_{\Gr}$. 
By definition, $(\otR, \ke, \stub) \in_{\aexec} \txid''$,  
and $\txid' = \max_{\AR_{\aexec}}(\VIS_{\aexec}^{-1}(\txid'') \cap \Setcon{\txid''' }{ (\otW, \ke, \stub) \in_{\aexec} \txid'''})$. 
Because $\txid'' \in \T_{\Gr} = \T_{\aexec}$, it follows that $\txid'' \neq \txid$. By definition, 
$\VIS^{-1}_{\aexec'}(\txid'') = \VIS^{-1}_{\aexec}(\txid)$: also, whenever 
$\txid_{a}, \txid_{b} \in \VIS^{-1}_{\aexec'}(\txid)$ we have that $\txid_{a}, \txid_{b} \in \T_{\aexec}$, 
and therefore if $\txid_{a} \xrightarrow{\AR_{\aexec'}} \txid_{b}$, then it must be the case 
that $\txid_{a} \xrightarrow{\AR_{\aexec}} \txid_b$; also, $\TtoOp{T}_{\aexec}(\txid_{a}) = \TtoOp{T}_{\aexec'}(\txid_{a})$. 
As a consequence, we have that 
\[
    \begin{array}{l}
        \max{}_{\AR_{\aexec'}}(\VIS^{-1}_{\aexec'}(\txid) \cap \Setcon{ \txid''' }{ (\otW, \ke, \stub) \in_{\aexec'} \txid'''}) =
        \max{}_{\AR_{\aexec}}(\VIS^{-1}_{\aexec}(\txid) \cap \Setcon{ \txid''' }{ (\otW, \ke, \stub) \in_{\aexec} \txid'''}) = \txid'
    \end{array}
\] 
and therefore $\txid' \xrightarrow{\RF_{\Gr'}} \txid$. 

\item Suppose now that $(\otR,\ke, \stub) \in \opset$, and $\txid' = \max_{\VO(\ke)_{\Gr}}(\T)$. 
    By Definition, $\txid' = \max_{\AR_{\aexec}}(\T) \cap \Setcon{ \txid''' }{ (\otW, \ke, \stub) \in_{\aexec} \txid'''}$, 
and, $\T = \VIS^{-1}_{\aexec'}(\txid)$.
Because $\T \subseteq \T_{\aexec}$, we have 
that for any $\txid_{a}, \txid_{b}$, if $\txid_{a} \xrightarrow{\AR_{\aexec}} \txid_{b}$, 
then $\txid_{a} \xrightarrow{\AR_{\aexec'}} \txid_{b}$; and $\TtoOp{T}_{\aexec'}(\txid_{a}) = 
\TtoOp{T}_{\aexec}(\txid_a)$. Therefore, 
\[
    \txid' = \max{}_{\AR_{\aexec'}}(\VIS^{-1}_{\aexec'}(\txid) \cap \Setcon{ \txid''' }{ (\otW, \ke, \stub) \in_{\aexec'} \txid'''}, 
\] 
from which it follows that $\txid' \xrightarrow{\RF_{\Gr'}(\ke)}\txid$.

Now, suppose that $\txid' \xrightarrow{\RF_{\Gr'}(\ke)} \txid''$ for some $\txid', \txid'' \in \T_{\Gr'} = 
\T_{\aexec'}$. We have that $ (\otR, \ke, \stub) \in_{\aexec'} \txid''$, 
$(\otW, \ke, \stub) \in_{\aexec'} \txid'$, and $\txid'' = \max_{\AR_{\aexec'}}(\VIS_{\aexec'}^{-1}(\txid'') 
\cap \Setcon{ \txid''' }{ (\otW, \ke, \stub) \in_{\aexec'} \txid'''}$. 
We also have that $\T_{\aexec'} = \T_{\aexec} \uplus \Set{\txid}$. We perform a case 
analysis on $\txid''$. 

\begin{itemize}
\item If $\txid'' = \txid$, then by definition of $\extend$ we have that 
$\VIS^{-1}_{\aexec'}(\txid) = \T$. Note that $\T \subseteq \T_{\aexec}$, so that 
for any $\txid_{a}, \txid_{b} \in \T_{\aexec}$, we have that $\txid_{a} \xrightarrow{\AR_{\aexec'}} \txid_{b}$ 
if and only if $\txid_{a} \xrightarrow{\AR_{\aexec}} \txid_{b}$, 
and $(\otW, \ke, \val) \in_{\aexec'} \txid_{a}$ if and only if $(\otW, \ke, \val) \in_{\aexec} \txid_{a}$. 
Thus, $\txid' = \max_{\AR_{\aexec}}(\T 
\cap \Setcon{\txid''' }{ (\otW, \ke, \stub) \in_{\aexec} \txid'''}) = \max_{\VO_{\Gr}(\ke)}(\T)$. 

\item If $\txid'' \in \T_{\aexec}$, then it is the case that 
    $\txid' = \max_{\AR_{\aexec'}}(\VIS^{-1}_{\aexec'}(\txid'') \cap \Setcon{ \txid''' }{ (\otW, \ke, \stub) \in_{\aexec'} \txid'''}$. 
Similarly to the case above, we can prove that $\VIS^{-1}_{\aexec'}(\txid'') = \VIS^{-1}_{\aexec}(\txid)$, 
for any $\txid_{a}, \txid_{b} \in \VIS^{-1}_{\aexec}(\txid)$, $(\otW, \ke, \val) \in_{\aexec'} \txid_{a}$ 
implies $(\otW, \ke, \val) \in_{\aexec} \txid_{a}$, and $\txid_{a} \xrightarrow{\AR_{\aexec'}} \txid_{b}$ 
implies $\txid_{a} \xrightarrow{\AR_{\aexec}} \txid_{b}$, from which it follows that 
$\txid' = \max_{\AR_{\aexec}}(\VIS^{-1}_{\aexec}(\txid'') \cap \Setcon{ \txid''' }{ (\otW, \ke \stub) \in_{\aexec} \txid'''})$, 
and therefore $\txid' \xrightarrow{\RF_{\Gr}(\ke)} \txid''$.
\end{itemize}
\end{itemize}

\item 
Similar to \( \RF(\ke)_{\Gr} \), there are two cases that either the \( \txid'' \) already exists in the dependency graph before,
or it is the newly committed transaction.
\begin{itemize}
\item Suppose that $\txid' \xrightarrow{\VO_{\Gr}(\ke)} \txid''$ for some $\txid', \txid'' \in \T_{\aexec}$. 
Then $(\otW,\ke,\stub) \in_{\aexec} \txid', (\otW, \ke, \stub) \in_{\aexec} \txid''$, and $\txid' \xrightarrow{\AR_{\aexec}} \txid''$. 
By definition of $\extend$, it follows that $\txid' \xrightarrow{\AR_{\aexec'}} \txid''$, and because 
$\txid', \txid'' \in \T_{\aexec}$, hence $\txid', \txid'' \neq \txid$, then 
$(\otW,\ke, \stub) \in_{\aexec'} \txid'$, $(\otW, \ke, \stub) \in_{\aexec'} \txid''$. By definition, 
we have that $\txid' \xrightarrow{\VO_{\aexec'}(\ke)} \txid''$.

\item Suppose that $(\otW, \ke, \stub) \in_{\aexec} \txid'$, $(\otW, \ke, \stub) \in \opset$. Because $\txid' \in \T_{\aexec}$, 
we have that $\txid' \neq \txid$, hence $(\otW, \ke, \stub) \in_{\aexec' }\txid'$. By definition, 
$\TtoOp{T}_{\aexec'}(\txid) = \opset$, hence $(\otW, \ke, \stub) \in_{\aexec'} \txid$. Finally, 
the definition of $\extend$ ensures that $\txid' \xrightarrow{\AR_{\aexec'}} \txid$. Combining 
these three facts together, we obtain that  
$\txid' \xrightarrow{\VO_{\Gr'}(\ke)} \txid$. 

Now, suppose that $\txid' \xrightarrow{\VO_{\Gr'}(\ke)} \txid''$ for some $\txid', \txid'' \in \T_{\aexec}$. 
Then $\txid' \xrightarrow{\AR_{\aexec'}} \txid''$, $(\otW, \ke, \stub) \in_{\aexec'} \txid'$, $(\otW, \ke, \stub) 
\in_{\aexec'} \txid''$. 
Recall that $\T_{\Gr'} = \T_{\aexec'} = \T_{\aexec} \uplus \Set{ \txid }$. We perform a case analysis on $\txid''$. 

\begin{itemize}

\item If $\txid'' = \txid$, then the definition of $\extend$ ensures that $\txid' \xrightarrow{\AR_{\aexec'}} \txid$ 
implies that $\txid \in \T_{\aexec}$, hence $\txid' \neq \txid$. 
Together with $(\otW, \ke, \stub) \in_{\aexec'} 
\txid'$, this leads to $(\otW, \ke, \stub) \in_{\aexec} \txid'$. 

\item If $\txid'' \in \T_{\aexec}$, then $\txid'' \neq \txid$. The definition of $\extend$ ensures that $\txid' \xrightarrow{\AR_{\aexec}} \txid''$. 
This implies that $\txid', \txid'' \in \T_{\aexec}$, hence $\txid', \txid'' \neq \txid$, and $\TtoOp{T}_{\aexec'}(\txid') = \TtoOp{T}_{\aexec}(\txid')$, 
$\TtoOp{T}_{\aexec'}(\txid'') = \TtoOp{T}_{\aexec}(\txid'')$. It follows that $(\otW, \ke, \stub) \in_{\aexec} \txid'$, 
$(\otW, \ke, \stub) \in_{\aexec} \txid''$, and therefore $\txid' \xrightarrow{\VO_{\Gr}}(\ke) \txid''$.

\end{itemize}
\end{itemize}
\end{enumerate}
\end{proof}


\begin{lemma}[Graph to kv-store update]
\label{lem:graph.update}
Let $\hh$ be a kv-store, and $\vi \in \Views(\hh)$. Let $\txid \notin \hh$, and 
$\opset \subseteq \powerset{\Ops}$, and let $\hh' = \updateKV(\hh, \vi, \opset, \txid)$. 
Let $\Gr = \Gr_{\hh}$, $\Gr' = \Gr_{\hh'}$; then for all $\txid', \txid'' \in \T_{\Gr'}$ and keys $\ke$, 
\begin{itemize}
\item $\TtoOp{T}_{\Gr'} = \TtoOp{T}_{\Gr}\rmto{\txid}{\opset}$, 
\item $\txid' \xrightarrow{\RF_{\Gr'}(\ke)} \txid''$ if and only if either 
$\txid' \xrightarrow{\RF_{\Gr}(\ke)} \txid''$, or $(\otR, \ke, \stub) \in \opset$ and 
$\txid' = \max_{\VO_{\Gr}(\ke)}(\Set{\WTx(\hh(\ke, i)) }{ i \in \vi(\ke)})$, 
\item $\txid' \xrightarrow{\VO_{\Gr'}(\ke)} \txid''$ if and only if either 
$\txid' \xrightarrow{\VO_{\Gr}(\ke)} \txid''$, or $(\otW, \ke, \stub) \in \opset$ 
and $\txid' = \WTx(\hh(\ke, \stub))$. 
\end{itemize}
\end{lemma}

\begin{proof}
Fix $\ke \in \Keys$. Because $\txid \notin \hh$, then $\txid \notin \T_{\Gr}$, 
and by definition of $update$ we obtain that $\Setcon{\txid' }{ \txid' \in \hh'} = 
\Setcon{\txid' }{ \txid' \in \hh} \cup \Set{\txid}$. It follows that $\T_{\Gr'} = \T_{\Gr} \uplus \Set{\txid }$.

\begin{enumerate}
\item Suppose that $(\otR, \ke, \val) \in_{\Gr} \txid'$. By definition, 
there exists an index $i$ such 
that $\hh(\ke, i) = (\val, \stub, \Set{\txid'} \cup \stub)$. Because $\hh' = \updateKV(\hh, \vi, \opset, \txid)$, 
it is immediate to observe that $\hh'(\ke, i) = (\val, \stub, \Set{\txid'} \cup \stub)$, and therefore 
$(\otR,\ke, \val) \in_{\Gr'} \txid'$. Conversely, note that if $(\otR, \ke, \val) \in_{\Gr'} \txid$, 
then there exists an index $i = 0,\cdots, \lvert \hh'(\ke) \rvert - 1$ such that 
$\hh'(\ke, i) = (\val, \stub, \Set{\txid'} \cup \stub)$. As a simple consequence of \cref{cor:updatekv.singlecell} 
it follows that it must be the case that $i \leq \lvert \hh(\ke) \rvert - 1$, and because 
$\txid' \neq \txid$, we have that $\hh(\ke, i) = (\val, \stub, \Set{\txid'} \cup \stub)$. Therefore 
$(\otR, \ke, \val) \in_{\Gr} \txid'$. 

Similarly, if $(\otW, \ke, \val) \in_{\Gr} \txid'$, 
then there exists an index $i=0,\cdots, \lvert \hh(\ke) \rvert - 1$ such that 
$\hh(\ke, i) = (\val, \txid', \val)$. It follows that $\hh'(\ke, i) = (\val, \txid', \stub)$, hence 
$(\otW, \ke, \val) \in_{\Gr'} \txid'$. If $(\otW, \ke, \val) \in \opset$, then we 
have from \cref{cor:updatekv.singlecell} that $\hh'(\ke, \lvert \hh'(\ke) \rvert - 1) = (\val, \txid', \stub)$, 
hence $(\otW, \ke, \val) \in_{\Gr'} \txid'$. 
Conversely, if $(\otW, \ke, \val) \in_{\Gr'} \txid'$, then there exists an index 
$i = 0, \cdots, \lvert \hh'(\ke) \rvert - 1$ such that $\hh(\ke, i) = (\val, \txid', \stub)$. 
We have two possible cases: either $i < \lvert \hh'(\ke, i) \rvert - 1$, leading to  
$\txid' \neq \txid$ and $\hh(\ke, i) = (\val, \txid', \stub)$, or equivalently 
$(\otR,\ke, \val) \in_{\Gr} \txid'$; or $i = \lvert \hh'(\ke, i) \rvert - 1$, 
leading to $\txid' = \txid$, and $\hh(\ke, i) = (\val, \txid, \emptyset)$ 
for some $\val$ such that $(\otW, \ke, \val) \in \opset$. 

Putting together the facts above, we obtain that $\TtoOp{T}_{\Gr'} = 
\TtoOp{T}_{\Gr}\rmto{\txid}{\opset}$, as we wanted to prove.

\item There are two cases that either the \( \txid'' \) already exists in the dependency graph before,
or it is the newly committed transaction.
\begin{itemize}
\item Suppose that $\txid' \xrightarrow{\RF_{\Gr}(\ke)} \txid''$. 
By definition, there exists an index $i = 0,\cdots, \lvert \hh(\ke) \rvert - 1$ 
such that $\hh(\ke, i) = (\stub, \txid', \Set{\txid''} \cup \stub)$. It is immediate 
to observer, from the definition of $\updateKV$, that $\hh'(\ke, i) = (\stub, \txid', \Set{\txid''} \cup \stub)$, 
and therefore $\txid' \xrightarrow{\RF_{\Gr'}(\ke)} \txid''$. 

\item Next, suppose that $(\otR, \ke, \stub) \in \opset$, and $\txid' = \max_{\VO_{\Gr}(\ke)}(\Setcon{\WTx(\hh(\ke, i)) }{ i \in \vi(\ke)}$. 
By Definition, $\hh(\ke, i) = (\stub, \txid', \stub)$, where $i = \max(\vi(\ke))$. This is because 
$\txid' \rightarrow{\VO_{\Gr}(\ke)} \txid''$ if and only if $\txid' = \WTx(\hh(\ke, j_1)), \txid'' = 
\WTx(\hh(\ke, j_2))$ for some $j_1, j_2$ such that $j_1 < j_2$. 
The definition of $\updateKV$ now ensures that $\hh'(\ke, i) = (\stub, \txid', \Set{\txid } \cup \stub)$, 
from which it follows that $\txid' \xrightarrow{\RF_{\Gr'}(\ke)} \txid$.

Conversely, suppose that $\txid' \xrightarrow{\RF_{\Gr'}(\ke)} \txid''$. 
Recall that $\T_{\Gr'} = \T_{\Gr} \cup \Set{ \txid }$, hence either 
$\txid'' \in \T_{\Gr}$ or $\txid'' = \txid$. 

\begin{itemize}
\item If $\txid'' = \txid$, then it must be the case that there exists an index $i = 0,\cdots, \lvert \hh'(\ke) \rvert - 1$ 
such that $\hh'(\ke, i) = (\stub, \txid', \Set{\txid } \cup \stub)$. Note that if $\hh'(\ke, \lvert \hh'(\ke) \rvert -1)$ is 
defined, then it must be the case that $\hh'(\ke, \lvert \hh'(\ke) \rvert -1) = (\stub, \txid, \emptyset)$, 
hence it must be the case that $i < \lvert \hh'(\ke) \rvert - 1$. Because $\txid \notin \hh$, 
then by the definition of $\updateKV$ it must be the case that $(\otR, \ke, \stub) \in \opset$, 
$\hh(\ke, i) = (\stub, \txid', \stub)$ and $i = \max(\vi(\ke))$; this also implies that $\txid' = 
\max_{\VO(\ke)}\Setcon{\WTx(\hh(\ke, i)) }{ i \in \vi(\ke)}$. 

\item If $\txid'' \in \T_{\Gr}$, then  it must be the case that $\txid'' \neq \txid$. 
In this case, it also must exist an index $i = 0,\cdots, \lvert \hh'(\ke) \rvert - 1$ 
such that $\hh'(\ke, i) = (\stub, \txid', \Set{\txid''} \cup \stub)$. As in the previous 
case, we note that $i < \lvert \hh'(\ke) \rvert - 1$, which together 
with the fact that $\txid'' \neq \txid$ leads to $\hh(\ke, i) = (\stub, \txid', \Set{\txid''} \cup \stub)$. 
It follows that $\txid' \xrightarrow{\RF_{\Gr}(\ke)} \txid''$.
\end{itemize}
\end{itemize}

\item 
Similar to \( \RF(\ke)_{\Gr} \), there are two cases that either the \( \txid'' \) already exists in the dependency graph before,
or it is the newly committed transaction.
\begin{itemize}
\item Suppose that $\txid' \xrightarrow{\VO_{\Gr}(\ke)} \txid''$. 
By definition, there exist two indexes $i, j$ such that 
$\hh(\ke, i) = (\stub, \txid', \stub)$, $\hh(\ke, j) = (\stub, \txid'', \stub)$ 
and $i < j$. The definition of $\updateKV$ ensures that 
$\hh'(\ke, i) = (\stub, \txid', \stub)$, $\hh'(\ke, j) = (\stub, \txid'', \stub)$, 
and because $i < j$ we obtain that $\txid' \xrightarrow{\VO_{\Gr'}(\ke)} \txid''$. 

\item Suppose that $(\otW, \ke, \stub) \in \opset$. Then $\hh'(\ke, \lvert \hh(\ke) \rvert) = (\stub, \txid, \stub)$.
Let $\txid' \in \T_{\Gr}$; by definition there exists an index $i = 0,\cdots, \lvert \hh(\ke) \rvert$ 
such that $\hh(\ke, i) = (\stub, \txid', \stub)$. It follows that $\hh'(\ke, i) = (\stub, \txid', \stub)$, and 
because $i < \lvert \hh(\ke) \rvert$, then we have that $\txid' \xrightarrow{\VO_{\Gr'}(\ke)} \txid$. 

Conversely, suppose that $\txid' \xrightarrow{\VO_{\Gr'}(\ke)} \txid''$. Because 
$\T_{\Gr'} = \T_{\Gr} \cup \Set{ \txid }$, we have two possibilities. Either $\txid'' = \txid$, 
or $\txid'' \in \T_{\Gr}$. 

\begin{itemize}
\item If $\txid'' = \txid$, then it must be the case that $(\otW, \ke, \stub) \in_{\Gr'} \txid$, 
or equivalently there exists an index $i=0,\cdots, \lvert \hh'(\ke) \rvert -1 $ such that 
$\hh'(\ke, i) = (\stub, \txid, \stub)$. Because $\txid \notin \hh$, and because for any 
$i = 0, \cdots, \lvert \hh(\ke) \rvert - 1$, $\hh'(\ke, i) = (\stub, \txid, \stub) \implies 
\hh(\ke, i) = (\stub, \txid, \stub)$, then it necessarily has to be $i = \hh'(\ke) \rvert - 1$. 
According to the definition of $\updateKV$, this is possible only if $(\otW,\ke, \stub) \in \opset$. 
Finally, note that because $\txid' \xrightarrow{\VO_{\Gr'}(\ke)} \txid$, then 
there exists an index $j < \lvert \hh'(\ke, i) \rvert - 1$ such that 
$\hh'(\ke, j) = (\stub, \txid' ,\stub)$. The fact that $j < \lvert \hh'(\ke, i) \rvert - 1$ 
From \cref{cor:updatekv.singlecell} we obtain that $\hh(\ke, j) = (\stub, \txid', \stub)$, 
or equivalently $\txid' = \WTx(\hh(\ke, \stub))$. 

\item If $\txid'' \in \T_{\Gr}$, then there exist two indexes $i,j$ such that 
$j < \lvert \hh'(\ke, j) \rvert - 1$, $\hh'(\ke, j) = (\stub, \txid'', \stub)$, 
$i < j$, and $\hh'(\ke, i) = (\stub, \txid', \stub)$. It is immediate to observe 
that $\hh(\ke, i) = (\stub, \txid', \stub)$, $\hh(\ke, j) = (\stub, \txid'', \stub)$, 
from which $\txid' \xrightarrow{\VO_{\Gr}(\ke)} \txid''$ follows. 
\end{itemize}
\end{itemize}

\end{enumerate}
\end{proof}
