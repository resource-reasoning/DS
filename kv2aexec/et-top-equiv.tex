\subsection{\( \ET_{\top}\)}
The \( \ET_{\top} \) is the most permissive execution test.
\begin{align*}
    \ET_{\top} & \defeq \Set{(\mkvs,\vi,\cl,\fp)}[%
        \vi \in \Views(\mkvs) 
        \land \fora{\key,\val} (\otR,\key,\val) \in \fp 
        \implies 
        \val = \valueOf[\mkvs(\key,\max(\vi(\key)))]
    ]
\end{align*}

\begin{lemma}
    \CMs[\ET(\emptyset, \emptyset)] = \CMs[\ET_{\top}]
\end{lemma}
\begin{proof}
    It is easy to see \( \ET(\emptyset, \emptyset) \subseteq \ET_{\top}\) and then \( \CMs[\ET(\emptyset, \emptyset)] \subseteq \CMs[\ET_{\top}] \).
    Let consider \( \CMs[\ET_{\top}] \subseteq \CMs[\ET(\emptyset, \emptyset)] \).
    Assume \( \mkvs \in  \CMs[\ET_{\top}] \), which means there is a trace \( \tr \):
    \begin{centermultline}
        \tr = \conf_0 \toET{\alpha_0}[\ET_{\top}] \conf_1 \toET{\alpha_1}[\ET_{\top}] \dots \toET{\alpha_{n-1}}[\ET_{\top}] \conf_n
    \end{centermultline}
    By \cref{prop:et.normalform}, we can assume \( \tr \) is a normal trace.
    We prove that there exists a trace \( \tr' \) with the same length \( n \)
    \begin{centermultline}[equ:et-top-et-empty-sets]
        \tr' = \conf_0 \toET{\alpha_0}[\ET(\emptyset, \emptyset)] \conf'_1 \toET{\alpha_1}[\ET(\emptyset, \emptyset)] \dots \toET{\alpha_{n-1}}[\ET(\emptyset, \emptyset)] \conf'_n \\
        \land \fora{\cl} \conf'_n\projection{2}(\cl) = \vi_0
        \land \conf'_n\projection{1} = \conf_n\projection{1}
    \end{centermultline}
    We prove \cref{equ:et-top-et-empty-sets} by induction on the length \( n \).
    \begin{itemize}
        \item \( n = 0 \). In this case \( \tr = \conf_0 \) and we pick \( \tr'  = \conf_0 \).
        \item \( n + 1 \).
        Suppose \cref{equ:et-top-et-empty-sets} when the length is \( n \), let consider the \( n + 1 \).
        There are two possible cases for the trace \( \tr \): the last step ((n+1)-\emph{th} step) is a view shift or a fingerprint step.
        \begin{itemize}
            \item The last step of \( \tr \) is a view shift, \ie
            \begin{centermultline}
                \tr = \conf_0 \toET{\alpha_0}[\ET_{\top}] \conf_1 \toET{\stub}[\ET_{\top}] \dots \toET{\alpha_1}[\ET_{\top}] \conf_n \toET{\cl_n,\epsilon}[\ET_{\top}] \conf_{n + 1}
            \end{centermultline}
            By \ih there exists a trace \( \tr'' \) with \( n \) steps that satisfies \cref{equ:et-top-et-empty-sets}:
            \begin{centermultline}
                \tr'' = \conf_0 \toET{\alpha_0}[\ET(\emptyset, \emptyset)] \conf'_1 \toET{\alpha_1}[\ET(\emptyset, \emptyset)] \dots \toET{\alpha_{n-1}}[\ET(\emptyset, \emptyset)] \conf'_n \\
                \land \fora{\cl} \conf'_n\projection{2}(\cl) = \vi_0
                \land \conf'_n\projection{1} = \conf_n\projection{1}
            \end{centermultline}
            We can pick the trace \( \tr' = \tr'' \toET{\cl_n,\epsilon} \conf'_{n+1} \), which means appending a view shift step without changing the view of \( \cl_n \).
            Note that \( \fora{\cl} \conf'_{n}(\cl) = \vi_0 \) and then \( \fora{\cl} \conf'_{n+1}(\cl) = \vi_0 \).
            Also we know \( \conf_n\projection{1} = \conf_{n+1}\projection{1} = \conf'_n\projection{1} = \conf'_{n+1}\projection{1} \).
            Thus \( \tr' \) satisfies \cref{equ:et-top-et-empty-sets}.
            \item The last step of \( \tr \) is a fingerprint step, \ie
            \begin{centermultline}
                \tr = \conf_0 \toET{\alpha_0}[\ET_{\top}] \conf_1 \toET{\stub}[\ET_{\top}] \dots \toET{\alpha_1}[\ET_{\top}] \conf_n \toET{\cl_n,\fp_n}[\ET_{\top}] \conf_{n + 1}
            \end{centermultline}
            By \ih there exists a trace \( \tr'' \) with \( n \) steps that satisfies \cref{equ:et-top-et-empty-sets}:
            \begin{centermultline}
                \tr'' = \conf_0 \toET{\alpha_0}[\ET(\emptyset, \emptyset)] \conf'_1 \toET{\alpha_1}[\ET(\emptyset, \emptyset)] \dots \toET{\alpha_{n-1}}[\ET(\emptyset, \emptyset)] \conf'_n \\
                \land \fora{\cl} \conf'_n\projection{2}(\cl) = \vi_0
                \land \conf'_n\projection{1} = \conf_n\projection{1}
            \end{centermultline}
            Let \(\mkvs_n = \conf_n\projection{1} = \conf'_n\projection{1} \), and the views \( \vi_n = \conf_n\projection{2}(\cl_n) \) 
            and \[ \vi'_n = \getView[\mkvs,\lfpTx[\mkvs,\extRead[\mkvs_n,\fp_n,\vi_n], \emptyset] ] \]
            That is, \( (\mkvs_n,\vi'_n,\cl_n,\fp_n) \in \ET(\emptyset,\emptyset) \) and \( \updateKV[\mkvs_n,\vi_n,\fp_n,\cl_n] = \updateKV[\mkvs_n,\vi'_n,\fp_n,\cl_n] \).
            The new trace \( \tr''' \) can be constructed by appending a view shift for client \( \cl_n \) and the fingerprint step to \( \tr'' \):
            \begin{centermultline}
                \exsts{\conf'_{n+1}}
                \tr''' = \tr'' \toET{\cl_n,\epsilon}[\ET(\emptyset,\emptyset)] (\mkvs_n,\conf'_{n}\projection{2}\rmto{\cl_n}{\vi'_n}) 
                \toET{\cl_n,\fp_n}[\ET(\emptyset,\emptyset)] \conf'_{n+1} \\
                {} \land \fora{\cl} \conf'_{n+1}\projection{2}(\cl) = \vi_0
                \land \conf'_{n+1}\projection{1} = \conf_{n+1}\projection{1}
            \end{centermultline}
            Note that the last step of \( \tr'' \) is also a view shift.
            By \cref{lem:et.absorb}, 
            the new view shift in \( \tr''' \) can be absorbed with previous view shift which yields \( \tr' \) that has length \( n + 1 \) and satisfies \cref{equ:et-top-et-empty-sets}. \qedhere
        \end{itemize}
    \end{itemize}
\end{proof}
