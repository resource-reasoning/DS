\section{Logic}


\sx{
    Parametrised by CM.

    Generalise CAP by generalise the meaning of box assertions, stable and atomic update rule.

    Two example write skew and long fork.

    Discussion: our first step. For each consistency models, we want to build specific pattern to use the capabilities
    so to provide a more syntactic rules.

    Cite: FCSL(for the idea that recording some history in the assertions), 
    CAP,
    Iris (they have powerful framework, whether it also fits in the weaker consistency world?)
    TaDA
    Relaxed SL (The first separation logic that deal with weaker behaviour)

    Alexey, Reaon about ..... in Dis. Sys. A proof system to verify if 
    a system (a set of operations that might be installed with tokens to specify synchronisation ) satisfies invariant.
}

We present a logic that is parametrised by consistency models that have strong session guarantees.
The logic has syntax that similar to concurrent abstract predicate (CAP),
but we generalise \emph{assertions} and \emph{atomic update rule} so it is parametrised by consistency models.


We motivate the logic by the \emph{write skew} example under snapshot isolation (\cref{fig:write-skew-si-proof}).
This example distinguishes snapshot isolation (SI) from serialisability.
For serialisability that transactions appear one after another, only one key, \( \vx \) or \( \vy \), will be 1 at the end.
While for SI, transactions take a snapshot when they start, and concurrent transactions can commit as long as they write to different keys.
In \cref{fig:write-skew-si-proof}, both transactions may take snapshots where \( \vx \) and \( \vy \) are 0, and can commit, 
which yields result that \( \vx \) and \( \vy \) are 1.

\begin{figure}[!t]
\hrule
\[
\intass :
\begin{array}[t]{@{} c @{\quad} c @{\quad}  @{} }
\begin{rclarray}[t]
    \CB{L} & : & \vx \fpW 1 \sep \vy \fpR 0 \sep \null \fpA \cass{\CB{L}}{\lrid} \\
\end{rclarray}
&
\begin{rclarray}[t]
    \CB{R} & : & \vx \fpR 0 \sep \vy \fpW 1 \sep \null \fpA \cass{\CB{R}}{\lrid} \\
\end{rclarray}
\\
\begin{rclarray}[t]
    \CB{U} & : & \exsts{\V n} \vx \fpR \V{n} \\
\end{rclarray} 
&
\begin{rclarray}[t]
    \CB{U} & : & \exsts{\V n} \vy \fpR \V{n} \\
\end{rclarray} \\
\end{array}
\]
\[
\CB{L} \composeK \CB{L} \ \text{is undefined} \quad  \CB{R} \composeK \CB{R} \ \text{is undefined} \quad \CB{U} \ \text{is the unit}
\]
\hrule\vspace{5pt}
\[
\begin{session}
{\color{blue}P : } \specline{ \cass{\CB{L}}{\lrid} \sep \cass{\CB{R}}{\lrid} \sep \boxass{\vx \pt 0 \sep \vy \pt 0 }{\lrid}{\intass}  } \\
\begin{parl}
\begin{session}
    {\color{blue}P1 : } \specline{\cass{\CB{L}}{\lrid} \sep 
            \boxass{\vx \pt 0 \sep \vy \pt 0 }{\lrid}{\intass} \\
            {} \lor \boxass{\vx \pt 0 \sep ( \vy \pt 0 \lor \vy \pt 1 ) \sep \cass{\CB{R}}{\lrid} }{\lrid}{\intass} 
    } \\
    \txid_1 : \begin{transaction}
        {\color{blue}p1 : } \specline{\vx \fpI 0 \sep ( \vy \fpI 0 \lor \vy \fpI 1 )} \\
        \plookup{\pvar{b}}{\vy} ; 
        \quad \pifs{\pvar{b} = 0} 
        \pmutate{\vx}{1} ;
        \pife \\
        {\color{blue}q1 : } \specline{\vx \fpW 1 \sep  \vy \fpR 0 \lor {}\\
        \quad \vx \fpI 0 \sep \vy \fpR 1 )} \\
    \end{transaction} \\
    {\color{blue}Q1 : } \specline{ 
            \boxass{ \vx \pt 1 \sep \vy \pt 0 \sep \cass{\CB{L}}{\lrid} }{\lrid}{\intass} \\
            {} \lor \boxass{\vx \pt 1 \sep ( \vy \pt 0 \lor \vy \pt 1 ) \sep {} \\
            \cass{\CB{R}}{\lrid} \sep \cass{\CB{L}}{\lrid} }{\lrid}{\intass}  \\
            {} \lor \cass{\CB{L}}{\lrid} \sep \boxass{\vx \pt 0 \sep \vy \pt 1 \sep \cass{\CB{R}}{\lrid} }{\lrid}{\intass}  \\
    } \\
\end{session}
&
\begin{session}
    {\color{blue}P2 : } \specline{\cass{\CB{R}}{\lrid} \sep 
            \boxass{ ( \vx \pt 0 \sep \vy \pt 0 }{\lrid}{\intass} \\
            {} \lor \boxass{ ( \vx \pt 0 \lor \vx \pt 1 ) \sep \vy \pt 0 \sep \cass{\CB{L}}{\lrid} }{\lrid}{\intass} 
    } \\
    \txid_2 : \begin{transaction}
        {\color{blue}p2 : } \specline{ ( \vx \fpI 0 \lor \vx \fpI 1 ) \sep \vy \fpI 0 )} \\
        \plookup{\pvar{a}}{\vx} ; 
        \quad \pifs{\pvar{a} = 0} 
        \pmutate{\vy}{1} ; 
        \pife \\
        {\color{blue}q2 : } \specline{ \vx \fpR 0 \sep \vy \fpW 1 \lor {} \\
        \quad \vx \fpR 1 \sep \vy \fpI 0 )} \\
    \end{transaction} \\
    {\color{blue}Q2 : } \specline{ 
            \boxass{ \vx \pt 0 \sep \vy \pt 1 \sep \cass{\CB{R}}{\lrid} }{\lrid}{\intass} \\
            {} \lor \boxass{ ( \vx \pt 0 \lor \vx \pt 1 ) \sep \vy \pt 1 \sep {} \\
            \cass{\CB{L}}{\lrid} \sep \cass{\CB{R}}{\lrid} }{\lrid}{\intass}  \\
            {} \lor \cass{\CB{R}}{\lrid} \sep \boxass{\vx \pt 1 \sep \vy \pt 0 \sep \cass{\CB{L}}{\lrid} }{\lrid}{\intass}  \\
    } \\
\end{session}
\end{parl} \\
{\color{blue}Q : } \specline{ 
        \cass{\CB{L}}{\lrid} \sep \boxass{ \vx \pt 0 \sep \vy \pt 1 \sep \cass{\CB{R}}{\lrid} }{\lrid}{\intass} 
        \lor \cass{\CB{R}}{\lrid} \sep \boxass{\vx \pt 1 \sep \vy \pt 0 \sep \cass{\CB{L}}{\lrid} }{\lrid}{\intass}
        \lor \boxass{ \vx \pt 1 \sep \vy \pt 1 \sep \cass{\CB{L}}{\lrid} \sep \cass{\CB{R}}{\lrid} }{\lrid}{\intass}  \\
} \\
\end{session}
\]
\hrule
\caption{Interference (the top) and the sketch proof (the bottom) for write skew under snapshot isolation}
\label{fig:write-skew-si-proof}
\end{figure}

We often use \( \lpre, \lpost \) to denote \emph{transactional assertions} and \( \gpre, \gpost \) for \emph{program assertions}.
The transactional assertions, such as \( p1 \) in \cref{fig:write-skew-si-proof}, 
describe the state of local snapshots for transactions and more importantly the fingerprints.
The fingerprint is the transaction's contribution to the key-value store, that is, \emph{the first read preceding any write} and \emph{last write} of each key.
The transactional assertions for individual keys have the following forms: \( \vx \fpI 0 \), \( \vx \fpR 0\), \( \vx \fpW 0\), and \( \vx \fpRW (0,1) \),
where \( \otR \) and \( \otW \) are read and write labels respectively.
The first three asserts the only key \( \vx \) in the local snapshot has value 0 and it has not been touched (no label), 
has been read (\(\otR\)) and has been written (\(\otW\)) respectively.
The last one asserts that the key \( \vx \) currently has value 1, 
\emph{the first read preceding any write} fetched 0 and the \emph{the last write} updated the key to \( 1 \).
We extend the standard sequential separation logic rules in a ways that the first read to a key adds a read label to the assertion; 
and a write to a key adds a write label to assertion and updates the value.
It is easy to see that transactional assertions are interpreted as pairs of snapshots and fingerprints.
We use \( (\sn,\fp) \in \evalLS{\lpre} \) to denote the snapshot \( \sn \) and fingerprint \( \fp \) satisfy 
the assertion \( \lpre \) under the stack \( \stk \) and logical environment \( \lenv \).
All the detail is in \cref{sec:reasoning-transaction}.

The program assertions, such as \( P1 \) in \cref{fig:write-skew-si-proof}, 
describe clients' views on key-value stores, together with some capabilities for the purpose of reasoning.
The assertion in the form of \( \boxass{\bar{\lpre}}{\lrid}{\intass}\),
for example \( \boxass{\vx \pt 1 \sep \vy \pt 1}{\lrid}{\intass} \) in \( \gpre_1 \),
is called the \emph{shared region assertions} also know as \emph{boxed assertions},
where \( \lrid \) is a unique \emph{region identifier},  \( \intass \) is \emph{interference},
and \( \lpre \) describes the views on key-value stores.
Region are shareable and indivisible, \ie 
\( \boxass{\bar{\lpre}}{\lrid}{\intass} \sep \boxass{\bar{\lpost}}{\lrid}{\intass} \iff \boxass{\bar{\lpre} \land \bar{\lpost}}{\lrid}{\intass}\).

Interference is a set of actions to specify what transactions are allowed by the region.
Each action has the form \( \kap : \assfp \),
where \( \kap \) is the \emph{client-specified capability} and \( \assfp \) is the \emph{fingerprint},
describes that if a client holds the capability \( \kap \), 
it is allowed to execute a transaction with the fingerprint \( \assfp \).
For instance in \cref{fig:write-skew-si-proof}, 
the \( \CB{L}\) allows a client to read \( \vy \) when it is 0 (\(\vy \fpR 0\)) and write 1 to \( \vx \) (\(\vx \fpW 1\)),
and similarly \( \CB{R} \) allows a client to read \( \vx \) when it is 0 and write 1 to \( \vy \).
The fingerprint \( \assfp \) in an action also specifies capabilities transformation.
For example, \( \null \fpA \cass{\CB{L}}{\lrid} \) means that \( \CB{L} \) needs to return to the shared region after the transaction is committed.
Last, the capabilities forms \emph{a partial commutative monoid (PCM)} where \( \composeK \) denotes the composition function.
In the write skew example (\cref{fig:write-skew-si-proof}), 
both \( \CB{L} \) and \( \CB{R} \) are unique as the compositions are undefined and \( \CB{U} \) is the unit.

Interference induces a label transitions system upon states \( (\mkvs, \vi, \ca) \)
where the labels are actions from the interference.
Then the \emph{invariant} of a region \( \func{inv}{\lrid} \) is the set of states reachable from 
the initial states of the regions provided by the function \( \func{init}{\lrid}\).

Given the invariant, a state \( (\mkvs, \vi, \ca) \) satisfies the assertion \( \boxass{\bar{\lpre}}{\lrid}{\intass} \),
if it is in the invariant of the region \( (\mkvs, \vi, \ca) \in \func{inv}{\lrid} \) and the snapshot induced by the view \( \snapshot(\mkvs, \vi) \) and 
the shared capabilities satisfy \( \bar{\lpre} \) (details in \cref{def:prog-assertion}).

Now let discuss the sketch proof in \cref{fig:write-skew-si-proof}, 
especially the \(\rl{PRCommit} \) rule for committing a transaction (other rules are standard and can be found in \cref{sec:reasoning-prog}):
\[
    \inferrule[\rl{PRCommit}]{%
        \tripleL{\lpre}{\trans}{\lpost} 
        \\ \repartition{\gpre}{\gpost}{\lpre}{\lpost}
        \\\\ \stable{\gpre, \ET} 
        \\ \stable{\gpost, \ET} 
    }{%
        \tripleG{\gpre}{ \ptrans{\trans} }{\gpost}
    }
\]
An assertion \( \gpre \) is \emph{stable} under execution test \( \ET \), written \( \stable{\gpre, \ET} \), if it is true against the interference.
Stabilisation needs to takes the following two situations into account: 
\textbf{(i)} the environment might interfere and change the state of key-value stores, 
note that, but not the views of the current client since the views are local; 
\textbf{(ii)} consequentially, the client might advance the views.
For example, \( P1 \) is stable as it describes the states that 
either the environment does nothing \( \boxass{\vx \pt 0 \sep \vy \pt 0 }{\lrid}{\intass} \),
or the environment has performed the action associated with \( \cass{\CB{R}}{\lrid}\), 
after which the environment cannot do any further action since the capabilities \( \cass{\CB{R}}{\lrid}\) is in the shared region,
\ie \( \boxass{\vx \pt 0 \sep ( \vy \pt 0 \lor \vy \pt 1 ) \sep \cass{\CB{R}}{\lrid} }{\lrid}{\intass}  \).
Let discuss further the assertions \( \vx \pt 0 \sep ( \vy \pt 0 \lor \vy \pt 1 ) \sep \cass{\CB{R}}{\lrid} \).
The environment returns the capability \( \cass{\CB{R}}{\lrid} \), 
it means the environment commits a transaction that writes 1 to \( \vy \) and the capability \( \cass{\CB{R}}{\lrid} \), 
yet since the current client still have the \( \vy \pt 0 \) as its view on the \( \vy \).
Afterword, the view might be advanced so we get an more up-to-date view with \( \vy \pt 1 \).

We generalise the \emph{repartitioning} \( \repartition{\gpre}{\gpost}{\lpre}{\lpost} \) from CAP \cite{cap}:
\begin{definition}[Repartitioning]
The repartitioning  \( \repartition{\gpre}{\gpost}{\lpre}{\lpost} \) is defined as that 
given any key-value store, view and capabilities, \( (\mkvs, \vi, \ca) \) that satisfy the precondition \( \gpre \):
\begin{itemize}
    \item the snapshot \( \snapshot(\mkvs,\vi) \) together with empty fingerprint satisfies the transactional precondition \( \lpre \); and
    \item for any snapshot and fingerprint \( \fp \) satisfy the transactional postcondition \( \lpost \), if
        there exists a new key-value store \( \mkvs' \), a new view \( \vi' \) and new capabilities \( \ca' \) such that 
    \begin{itemize}
        \item the new key-value store \( \mkvs' \in \updateKV(\mkvs, \vi, \stub, \fp) \),
        \item the update is allowed by the execution test, \ie \( (\mkvs, \vi) \csat \fp : \vi' \), and
        \item the transition from \( (\mkvs, \vi, \ca) \) to \( (\mkvs', \vi', \ca') \) is allowed by guarantee, \ie some local capabilities,
    \end{itemize} 
\end{itemize}
then the state \( (\mkvs', \vi', \ca') \) satisfies the postcondition \( \gpost \).
\end{definition}

For example,  let consider the transaction \( \txid_1 \) in \cref{fig:write-skew-si-proof}.
The precondition \( P1 \) asserts following views:

%\begin{center}
%\begin{tikzpicture}
%\begin{pgfonlayer}{foreground}
%%\draw[help lines] grid(5,4);

%%Location x
%\node(locx) {$\vx \mapsto$};

%\matrix(versionx) [version list]
    %at ([xshift=\tikzkvspace]locx.east) {
    %{a} \& $\txid_0$ \\
    %{a} \& $\emptyset $ \\
%};
%\tikzvalue{versionx-1-1}{versionx-2-1}{locx-v0}{$0$};

%%Location y
%\path (locx.south) + (0,\tikzkeyspace) node (locy) {$\vy \mapsto$};
%\matrix(versiony) [version list]
   %at ([xshift=\tikzkvspace]locy.east) {
  %{a} \& $\txid_0$  \\
  %{a} \& $\emptyset $ \\
%};

%\tikzvalue{versiony-1-1}{versiony-2-1}{locy-v0}{$0$};

%\end{pgfonlayer}
%\end{tikzpicture}
%%
%\quad
%%
%\begin{tikzpicture}
%\begin{pgfonlayer}{foreground}
%%\draw[help lines] grid(5,4);

%%Location x
%\node(locx) {$\vx \mapsto$};

%\matrix(versionx) [version list]
    %at ([xshift=\tikzkvspace]locx.east) {
    %{a} \& $\txid_0$ \\
    %{a} \& $\Set{\txid}$  \\
%};
%\tikzvalue{versionx-1-1}{versionx-2-1}{locx-v0}{$0$};

%%Location y
%\path (locx.south) + (0,\tikzkeyspace) node (locy) {$\vy \mapsto$};
%\matrix(versiony) [version list, column 3/.style={dotted}, column 4/.style={dotted}]
   %at ([xshift=\tikzkvspace]locy.east) {
       %{a} \& $\txid_0$ \& {a} \& \color{gray}$\txid$ \\
  %{a} \& $\emptyset$ \& {a} \&  \color{gray} $\emptyset$\\
%};

%\tikzvalue{versiony-1-1}{versiony-2-1}{locy-v0}{$0$};
%\node[version node,draw=none,fit=(versiony-1-3) (versiony-2-3),fill=white, inner sep=0pt] (locx-v1) {\color{gray}$1$};

%\end{pgfonlayer}
%\end{tikzpicture}
%%
%\quad
%%
%\begin{tikzpicture}
%\begin{pgfonlayer}{foreground}
%%\draw[help lines] grid(5,4);

%%Location x
%\node(locx) {$\vx \mapsto$};

%\matrix(versionx) [version list]
    %at ([xshift=\tikzkvspace]locx.east) {
    %{a} \& $\txid_0$ \\
    %{a} \& $\Set{\txid}$  \\
%};
%\tikzvalue{versionx-1-1}{versionx-2-1}{locx-v0}{$0$};

%%Location y
%\path (locx.south) + (0,\tikzkeyspace) node (locy) {$\vy \mapsto$};
%\matrix(versiony) [version list]
   %at ([xshift=\tikzkvspace]locy.east) {
 %{a} \& $\txid_0$ \& {a} \& $\txid$ \\
  %{a} \& $\emptyset$ \& {a} \& $\emptyset$\\
%};

%\tikzvalue{versiony-1-1}{versiony-2-1}{locy-v0}{$0$};
%\tikzvalue{versiony-1-3}{versiony-2-3}{locy-v1}{$1$};

%\end{pgfonlayer}
%\end{tikzpicture}
%\end{center}
%Note that the they must satisfy the invariant of the region \( \lrid \).
%The transactional precondition \( p1 \) describes exactly the snapshots given the views above.
%The transaction \( \txid_1 \) has the fingerprints \( \Set{(\otW, \vx, 1), (\otR, \vy, 0)} \) or  
%\( \Set{(\otR, \vx, 1)} \) by the transactional postcondition \( \lpost \).
%The repartitioning asserts that any new key-value stores \( \mkvs' \) and new views \( \vi' \)
%by committing the fingerprint under snapshot isolation \( \SI \):
%\begin{center}
%\begin{tikzpicture}
%\begin{pgfonlayer}{foreground}
%%\draw[help lines] grid(5,4);

%%Location x
%\node(locx) {$\vx \mapsto$};

%\matrix(versionx) [version list, column 2/.style={text width=7mm}]
    %at ([xshift=\tikzkvspace]locx.east) {
        %{a} \& $\txid_0$ \& {a} \& $\txid'$ \\
        %{a} \& $\emptyset$ \& {a} \& $\emptyset$ \\
%};
%\tikzvalue{versionx-1-1}{versionx-2-1}{locx-v0}{$0$};
%\tikzvalue{versionx-1-3}{versionx-2-3}{locx-v1}{$1$};

%%Location y
%\path (locx.south) + (0,\tikzkeyspace) node (locy) {$\vy \mapsto$};
%\matrix(versiony) [version list, column 2/.style={text width=7mm}]
   %at ([xshift=\tikzkvspace]locy.east) {
  %{a} \& $\txid_0$  \\
  %{a} \& $\Set{\txid'}$ \\
%};

%\tikzvalue{versiony-1-1}{versiony-2-1}{locy-v0}{$0$};

%\end{pgfonlayer}
%\end{tikzpicture}
%%
%\quad
%%
%\begin{tikzpicture}
%\begin{pgfonlayer}{foreground}
%%\draw[help lines] grid(5,4);

%%Location x
%\node(locx) {$\vx \mapsto$};

%\matrix(versionx) [version list, column 2/.style={text width=7mm}]
    %at ([xshift=\tikzkvspace]locx.east) {
        %{a} \& $\txid_0$ \& {a} \& $\txid'$ \\
        %{a} \& $\Set{\txid}$ \& {a} \& $\emptyset$ \\
%};
%\tikzvalue{versionx-1-1}{versionx-2-1}{locx-v0}{$0$};
%\tikzvalue{versionx-1-3}{versionx-2-3}{locx-v1}{$1$};

%%Location y
%\path (locx.south) + (0,\tikzkeyspace) node (locy) {$\vy \mapsto$};
%\matrix(versiony) [version list, column 2/.style={text width=7mm}, column 3/.style={dotted}, column 4/.style={dotted}]
   %at ([xshift=\tikzkvspace]locy.east) {
       %{a} \& $\txid_0$ \& {a} \& \color{gray}$\txid$ \\
       %{a} \& $\Set{\txid'}$ \& {a} \&  \color{gray} $\emptyset$\\
%};

%\tikzvalue{versiony-1-1}{versiony-2-1}{locy-v0}{$0$};
%\node[version node,draw=none,fit=(versiony-1-3) (versiony-2-3),fill=white, inner sep=0pt] (locx-v1) {\color{gray}$1$};

%\end{pgfonlayer}
%\end{tikzpicture}
%%
%\quad
%%
%\begin{tikzpicture}
%\begin{pgfonlayer}{foreground}
%%\draw[help lines] grid(5,4);

%%Location x
%\node(locx) {$\vx \mapsto$};

%\matrix(versionx) [version list]
    %at ([xshift=\tikzkvspace]locx.east) {
    %{a} \& $\txid_0$ \\
    %{a} \& $\Set{\txid}$  \\
%};
%\tikzvalue{versionx-1-1}{versionx-2-1}{locx-v0}{$0$};

%%Location y
%\path (locx.south) + (0,\tikzkeyspace) node (locy) {$\vy \mapsto$};
%\matrix(versiony) [version list]
   %at ([xshift=\tikzkvspace]locy.east) {
 %{a} \& $\txid_0$ \& {a} \& $\txid$ \\
  %{a} \& $\emptyset$ \& {a} \& $\emptyset$\\
%};

%\tikzvalue{versiony-1-1}{versiony-2-1}{locy-v0}{$0$};
%\tikzvalue{versiony-1-3}{versiony-2-3}{locy-v1}{$1$};

%\end{pgfonlayer}
%\end{tikzpicture}
%\end{center}
they should satisfy the postcondition \( \gpost \).
It is easy to see that
the fingerprint \( \Set{(\otW, \vx, 1), (\otR, \vy, 0)} \) is allowed to commit 
by \( \cass{\CB{R}}{\lrid} \) after which it returns to the region,
and \( \Set{(\otR, \vy, 1)} \) is allowed by \( \CB{U} \).
This will give us a postcondition \emph{before stabilisation}:
\[
\boxass{ \vx \pt 1 \sep \vy \pt 0 \sep \cass{\CB{L}}{\lrid} }{\lrid}{\intass} 
\lor \boxass{\vx \pt 1 \sep \vy \pt 1  \sep \cass{\CB{R}}{\lrid} \sep \cass{\CB{L}}{\lrid} }{\lrid}{\intass}
\lor \cass{\CB{L}}{\lrid} \sep \boxass{\vx \pt 0 \sep \vy \pt 1 \sep \cass{\CB{R}}{\lrid} }{\lrid}{\intass} 
\]
By stabilisation we will get the final postcondition \( \gpost_1 \) shown in \cref{fig:write-skew-si-proof}.


We can also prove the \emph{long fork} (\cref{fig:long-fork-proof}), 
which distinguish snapshot isolation (SI) from parallel snapshot isolation (PSI).
We use the similar pattern by putting back the capabilities to the region once the capabilities has been used for helping the reasoning.


\subsection{Reasoning inside transactions}
\label{sec:reasoning-transaction}

Recall that a transaction takes a snapshot of the kv-store and commits the fingerprint by the end.
Because of the atomicity, only the \emph{first reads preceding any write} and the \emph{last writes} of keys are contained in the fingerprint.
All the intermediate reads and writes are not observable to other transactions and have no effect on the key-value store.
To capture the state of the local snapshot as well as the fingerprint, 
the \emph{transactional assertions} (\cref{def:local_assertions}) extend normal sequential separate logic assertions with read and/or write labels, 
\eg \( \vx \fpR 0 \), \( \vy \fpW 1 \) and \( \pv{z} \fpRW (0,1) \).


\begin{definition}[Transactional assertions]
\label{def:fingerprint}
\label{def:local_assertions}
\label{def:logical-expr}
Assume a countably infinite set of \emph{logical variables} $\lvar \in \LVar$.
The set of \emph{logical expressions} $\lexpr \in \LExpr$ is defined by the inductive grammar:
\(
\begin{rclarray}
   \lexpr & ::= & \val \mid \lvar \mid \var \mid \lexpr + \lexpr \mid  \dots 
\end{rclarray}
\)
where \(\val \in \Val\)  and \(\var \in \Vars\).
The \emph{logical expression evaluation} function, $\evalLE[(., .)]{.}:\LExpr \times \Stacks \times \LEnv\rightharpoonup \Val$, is defined inductively over the structure of logical expressions,
where the \emph{logical environments} \(\lenv \in \LEnv: \LVar \parfun \Val\) associates logical variables with values:
%
\[
\begin{array}{@{}c@{}}
    \begin{rclarray}
        \evalLE{\val} & \defeq & \val \\
    \end{rclarray}
    \quad
    \begin{rclarray}
        \evalLE{\lvar} & \defeq & \lenv(\lvar) \\
    \end{rclarray}
    \quad
    \begin{rclarray}
        \evalLE{\var} & \defeq & \stk(\var) \\
    \end{rclarray} 
    \quad
    \begin{rclarray}
        \evalLE{\lexpr_1 + \lexpr_2} & \defeq & \evalLE{\lexpr_1} + \evalLE{\lexpr_2} \\
    \end{rclarray}
    \dots
\end{array}
\]
The set of \emph{transactional assertions}, $\lpre,  \lpost \LAst$, is defined by the following grammars:
\[
\begin{rclarray}
	\assfp & ::= & \lexpr \fpI \lexpr \mid \lexpr \fpR \lexpr \mid \lexpr \fpW \lexpr \mid \lexpr \fpRW (\lexpr, \lexpr)  \\
	\lpre, \lpost & ::= & \assfalse \mid \asstrue \mid \lpre \land \lpost \mid \lpre \lor \lpost \mid \exsts{\lvar} \lpre \mid \lpre \implies \lpost
    \mid \assemp \mid \assfp \mid \lpre \sep \lpost  \\
\end{rclarray}	 
\]
The \emph{transactional assertion interpretation function}, $\evalLS[(.,.)]{.}: \LAst \times \LEnv \times \LAst \parfun \pset{\Snapshots \times \Fingerprints} $, is defined over the structure of local assertions, where the composition for snapshots \( \composeH \defeq \uplus \) is standard disjointed union on two functions and the composition for fingerprints \( \fp \composeO \fp' \defeq \fp \uplus \fp'\) when they contain different keys, \ie \( \fp\projection{2} \cap \fp'\projection{2} = \emptyset\):
\[
\begin{rclarray}
	\evalLS{\assfalse} & \defeq & \emptyset \\
	\evalLS{\asstrue} & \defeq & \Snapshots \times \Fingerprints \\
	\evalLS{\lpre \land \lpost} & \defeq & \evalLS{\lpre} \cap \evalLS{\lpost} \\
	\evalLS{\lpre \lor \lpost} & \defeq & \evalLS{\lpre} \cup \evalLS{\lpost} \\
	\evalLS{\exsts{\lvar} \lpre} & \defeq & \bigcup_{\val \in \textnormal{\Val}}\evalLS[\lenv\rmto{\lvar}{\val}, \stk]{\lpre}  \\
	\evalLS{\lpre \implies \lpost} & \defeq & \Setcon{(\sn, \fp)}{(\sn , \fp) \in \evalLS{\lpre} \implies (\sn , \fp) \in \evalLS{\lpost}}\\
	\evalLS{\assemp} & \defeq & \Set{ ( \unitH, \unitE) }  \\
	\evalLS{ \lexpr_1 \fpI \lexpr_2 } & \defeq & \Set{\left(\Set{\evalLE{\lexpr_1} \mapsto \evalLE{\lexpr_2} }, \unitO\right)} \\
	\evalLS{ \lexpr_1 \fpR \lexpr_2 } & \defeq & \Set{\left(\Set{\evalLE{\lexpr_1} \mapsto \evalLE{\lexpr_2} }, \Set{(\otR, \evalLE{\lexpr_1},\evalLE{\lexpr_2})}\right)} \\
	\evalLS{ \lexpr_1 \fpW \lexpr_2 } & \defeq & \Set{\left(\Set{\evalLE{\lexpr_1} \mapsto \evalLE{\lexpr_2} }, \Set{(\otW, \evalLE{\lexpr_1},\evalLE{\lexpr_2})}\right)} \\
	\evalLS{ \lexpr_1 \fpRW (\lexpr_2, \lexpr_3) } & \defeq & \Set{\left(\Set{\evalLE{\lexpr_1} \mapsto \evalLE{\lexpr_3} }, \Set{(\otR, \evalLE{\lexpr_1},\evalLE{\lexpr_2}), \\ \quad (\otW, \evalLE{\lexpr_1},\evalLE{\lexpr_3})}\right)} \\
	\evalLS{\lpre \sep \lpost} & \defeq & 
    \Setcon{
        (\sn_1 \composeH \sn_2, \fp_{1} \composeE \fp_{2})
    }{ 
        (\sn_{1},\fp_{1}) \in \evalLS{\lpre} 
        \land (\sn_{2}, \fp_{2} ) \in \evalLS{\lpost} 
    } 
\end{rclarray}%
\]
\end{definition}

The \emph{transactional assertions} (\cref{def:local_assertions}) have \( \assfalse \), \(\asstrue \), conjunction \( \land \), disjunction \( \lor \), existential quantification \( \exists \), implication \( \implies  \), empty \( \assemp \), fingerprint assertions \( \stub \stackrel{\stub}{\hookrightarrow} \stub \) and separation conjunction \( \sep \).
They describes the state of local snapshot used by a transaction and more importantly the fingerprint of the transaction.
They are interpreted to pairs of snapshots and fingerprints.

A \emph{fingerprint assertion} describes the possible global effect from a transaction.
It includes the default \(\lexpr_{1} \fpI \lexpr_{2} \), the \emph{first read preceding any write} \( \lexpr_{1} \fpR \lexpr_{2} \), \emph{last write} \( \lexpr_{1} \fpW \lexpr_{2} \) for the key \( \lexpr_{1} \) and the combination of them \( \lexpr_{1} \fpRW \lexpr_{2} \).
The \( \lexpr_{1} \fpI \lexpr_{2} \) means the only key \( \lexpr_{1} \) in the local snapshot has value \( \lexpr_{2} \),
and the key has no associated fingerprint.
The \( \lexpr_{1} \fpR \lexpr_{2} \) means the key has been read before any other write carrying value \( \lexpr_{2} \) and the current value for the key is also \( \lexpr_{2} \).
The \( \lexpr_{1} \fpW \lexpr_{2} \) means the key has been written at least once, and the last written value is \( \lexpr_{2} \).
Because read does not change the state of snapshot, and the write fingerprint corresponds to the last write for the key,
so the state of the snapshot matches the fingerprint for cases \( \lexpr_{1} \fpR \lexpr_{2} \) and  \( \lexpr_{1} \fpW \lexpr_{2} \).
Last, The combined fingerprint \( \lexpr_{1} \fpRW (\lexpr_{2}, \lexpr_{3}) \) means the key has been read and then written at least once, the first read fetched value \( \lexpr_{2} \), and the last written value and the current state of the local snapshot for the key are both \( \lexpr_{3} \).

Other transactional assertions have standard interpretations.
Note that the separation conjunction \( \sep \) asserts two local snapshots and fingerprints when the keys are disjointed.
Observe that program expressions $\Expr$ (\cref{fig:semantics}) are contained in logical expressions $\LExpr$ (\cref{def:local_assertions}), \ie $\Expr \subset \LExpr$. 

The proof rules for transactions (\cref{fig:rule-trans}) are standard except \rl{TRLookup} and \rl{TRMutate}.
The \rl{TRLookup} rule adds read label only if there is no write label.
Because once a key has been written, the following reads are local to the transaction.
However, the local read always needs to match the current state of snapshot.
Especially for the case when the precondition is \( \lexpr \fpRW (\lexpr'', \lexpr') \),
the current state of the key is the last written value \( \lexpr' \).
The\rl{TRMutate} rule changes the state of the key and more importantly adds write label.
For the case when the precondition is \( \lexpr \fpRW (\lexpr'', \lexpr') \), 
the rule changes the value \( \lexpr' \) to the new written value \( \lexpr''' \)
but keeps the old read value \( \lexpr'' \) the same.

\begin{figure}[!t]
\sx{Font for E}
\hrule
\begin{mathpar}
    \inferrule[\rl{TRLookup}]{%
        \var \notin \func{fv}{\lexpr}  
        \\ \lpre \toFP{\otR(\expr, \lexpr)} \lpost
    }{%
        \tripleL{ \lpre }{ \plookup{\var}{\expr} }{\var \dot= \lexpr \sep \lpost\sub{\var}{\lexpr} }
    }
    \and
    \inferrule[\rl{TRMutate}]{
        \lpre \toFP{\otW(\expr_{1},\expr_{2})} \lpost
    }{%
        \tripleL{ \lpre }{ \pmutate{\expr_1}{\expr_2} }{ \lpost } 
    }
    \and
    \inferrule[\rl{TRAss}]{
        \var \notin \func{fv}{\lexpr}
    }{%
        \tripleL{\var \dot= \lexpr }{ \passign{\var}{\expr} }{\var \dot= \expr\sub{\var}{\lexpr} }
    }
    \and
    \inferrule[\rl{TRAssume}]{ }{%
        \tripleL{ \expr \dot\neq 0 }{ \passume{\expr} }{ \expr \dot\neq 0 } 
    }
    \and
    \inferrule[\rl{TRChoice}]{%
        \tripleL{ \lpre }{ \trans_{1} }{ \lpost } 
        \\ \tripleL{ \lpre }{ \trans_{2} }{ \lpost } 
    }{%
        \tripleL{ \lpre }{ \trans_{1} \pchoice \trans_{2} }{ \lpost }
    }
    \and
    \inferrule[\rl{TRSeq}]{%
        \tripleL{ \lpre }{ \trans_{1} }{ \lframe }
        \\ \tripleL{ \lframe }{ \trans_{2} }{ \lpost }
    }{%
        \tripleL{ \lpre }{ \trans_{1} \pseq \trans_{2} }{ \lpost }
    }
    \and
    \inferrule[\rl{TRIter}]{%
        \tripleL{ \lpre }{ \trans }{ \lpre } 
    }{%
        \tripleL{ \lpre }{ \trans\prepeat }{ \lpre }
    }
    \and
    \inferrule[\rl{TRFrame}]{%
        \tripleL{ \lpre }{ \trans }{ \lpost } \and \func{fv}{\lframe} \cup \func{modify}{\trans} = \emptyset
    }{% 
        \tripleL{ \lpre \sep \lframe }{ \trans }{ \lpost \sep \lframe }
    }
\end{mathpar}


\hrule
\[
\begin{array}{@{} c @{\qquad} c @{}}
\begin{rclarray}
    \lexpr \fpI \lexpr' & \toFP{\otR(\lexpr,\lexpr')} & \lexpr \fpR \lexpr' \\
    \lexpr \fpR \lexpr' & \toFP{\otR(\lexpr,\lexpr')} & \lexpr \fpR \lexpr' \\
    \lexpr \fpW \lexpr' & \toFP{\otR(\lexpr,\lexpr')} & \lexpr \fpW \lexpr' \\
    \lexpr \fpRW (\lexpr'', \lexpr') & \toFP{\otR(\lexpr,\lexpr')} & \lexpr \fpRW (\lexpr'', \lexpr') \\
\end{rclarray}
&
\begin{rclarray}
    \lexpr \fpI \lexpr' & \toFP{\otW(\lexpr,\lexpr'')} & \lexpr \fpW \lexpr'' \\
    \lexpr \fpR \lexpr' & \toFP{\otW(\lexpr,\lexpr'')} & \lexpr \fpRW (\lexpr',\lexpr'') \\
    \lexpr \fpW \lexpr' & \toFP{\otW(\lexpr,\lexpr'')} & \lexpr \fpW \lexpr'' \\
    \lexpr \fpRW (\lexpr'', \lexpr') & \toFP{\otW(\lexpr,\lexpr''')} & \lexpr \fpRW (\lexpr'', \lexpr''') \\
\end{rclarray}
\end{array}
\]
\hrule
\caption{The rules for transactions}
\label{fig:rule-trans}
 \end{figure}



\subsection{Reasoning programs}

Each region has \emph{client-specified capabilities} \( \kap \in \Kaps \) that forms \emph{a partial commutative monoid (PCM)} (\cref{def:capabilities}).
To recall, \emph{a PCM} is a partially ordered set that is closed under a commutative binary operation \( \compose \) and has a set of identify elements \( \unitelem \).
The client-specified capabilities are lifted to \emph{capability composition function} \( \ca \in \Caps \) with their associated region identifiers (\cref{def:capabilities}).
For brevity, we often say \emph{capabilities} for \emph{capability composition function}.
The composition function for \emph{capabilities} \( \ca_l \composeC \ca_r \) is defined as compositing each region point-wise
and the units \( \unitC \) are functions where regions map to units of client-specified capabilities.

\begin{definition}[Capabilities]
\label{def:capabilities}
Assume a \emph{partial commutative monoid (PCM)} of \emph{client-specified capabilities} \( (\Kaps, \composeK, \unitK) \) with \( \kap \in \Kaps \), 
the composition \( \composeK \) and the units set \( \unitK \).
Given a set of \emph{region identifiers} \( \rid, \lrid \in \RegionID \), 
the \emph{capability composition function} or \emph{capabilities} is \( \ca \in \Caps \defeq \RegionID \parfun \Kaps \),
where the composition \( \composeC \) is defined as the follows:
\[
    \begin{rclarray}
        (\ca_{l} \composeC \ca_{r})(\rid) & \defeq  &
        \begin{cases}
            \ca_{l}(\rid) \composeK \ca_{r}(\rid) & \rid \in \dom(\ca_{l}) \cap \dom(\ca_{l}) \\
            \ca_{l}(\rid)  & \rid \in \dom(\ca_{l}) \setminus \dom(\ca_{l}) \\
            \ca_{r}(\rid) & \rid \in \dom(\ca_{r}) \setminus \dom(\ca_{l}) \\
            \text{undefined} & \text{otherwise} \\
        \end{cases}
    \end{rclarray}
\]
and the units set \( \unitC \defeq \Setcon{\ca}{\fora{\rid} \ca(\rid) \in \unitK } \).
A capability assertion is in the form of \( \cass{\kap(\vec{\lvar})}{\lrid} \in \CAst \).
The assertion is interpreted to a capability:
\(
\begin{rclarray}
    \evalC{\cass{\kap(\vec{\lvar})}{\lrid}} & \defeq & \Set{\lrid \mapsto \kap(\evalLE{\vec{\lvar}})} \\
\end{rclarray}
\).
\end{definition}

The \emph{capability assertions} are in the form of \( \cass{\kap(\vec{\lvar})}{\lrid} \) 
where \( \kap(\vec{\lvar}) \) is a client-specified syntactic capability parametrised by logical variables \( \vec{\lvar} \) and \( \lrid \) is the region identifier.
They are interpreted to some capabilities \( \Caps \) by interpreting the syntactic capabilities \( \kap(\vec{\lvar}) \).

Capabilities are resources that grant abilities to access the region.
That is, the \emph{interference} (\cref{def:intf}) of a region specifies the allowed fingerprints associated with certain capabilities,
and if a client holds those capabilities, it is allowed to perform those fingerprints.
Capabilities also can be used as ghost resources to track extra information and/or build protocols between clients.
For example in \cref{fig:write-skew-si-proof}, the capabilities \( \cass{\CB{L}}{\lrid} \) and \( \cass{\CB{L}}{\lrid} \) returns to shared region once being used,
so that other clients know the environment has executed transactions with fingerprints allowed by the capabilities.

Each shared region assertion is associated with a \emph{interference assertion} \( \intass \) to specify how the region can evolve (\cref{def:intf}).
\emph{A action} in the interference \( \exsts{\vec{\lvar}} \perm{\kap} : \bar{\fp} \) says that 
if a client holds the capability \( \perm{\kap} \), it is allowed to execute a transaction and transfer capabilities with respect to \( \bar{\fp} \).
The \emph{fingerprint and capability transferring assertions} \( \bar{\fp} \) has fingerprint assertion (\cref{def:fingerprint}) and additionally assertions for transferring of capabilities, 
\ie adding capabilities to the shared region \( \null \fpA \cass{\kap}{\lrid} \), 
deleting capabilities from the shared region \( \null \fpD \cass{\kap}{\lrid} \) and updating the shared capabilities \( \cass{\kap(\vec{\lvar})}{\lrid} \fpU \cass{\kap(\vec{\lvar})}{\lrid} \). 
Those assertions \( \bar{\fp} \) are interpreted as triples \( ( \fp, \ca, \ca' ) \) of fingerprints \( \opset \), capabilities that moves into the region \( \ca \) and capabilities that moves out of the region \( \ca' \).
The existential quantification is for binding variables between the capability \( \kap \) and the assertions \( \bar{\fp}\).

\emph{Interference assertions} are interpreted to \emph{interference environment}, \( \intf \in \Interference \) (\cref{def:intf}).
\emph{Interference environment} also \emph{interference} is a function from client-specified capabilities \( \kap \) to sets of possible transitions over shared states for a region, 
\ie tuples consisting of key-value stores, views, and (shared) capabilities (\cref{def:intf}). 
Given a interpretation of an action \( (\fp, \ca_d, \ca_a) \in \evalF[\lenv',\stk]{\bar{\fp}} \) and any initial state \( (\hh, \vi, \ca_r \composeC \ca_f ) \),
the final states \( (\hh',\vi', \ca_f \composeC \ca_a) \) are defined by committing the fingerprint \( \opset \) through \( \HHupdate \) function,
updating the view \( \vi \leq \vi' \), taking out \( \ca_r \) and adding \( \ca_a \).
Note that \( \ca_f \) are frames of capabilities as long as they are compatible before and after the update.
All the read values should match to the versions with respect to the view \( \vi \).
For technical reasons, the view \( \vi' \) after updating is any view greater than before.

\begin{definition}[Interference]
\label{def:intf}
The \emph{fingerprint and capability transferring assertions} is defined as:
\[
\begin{rclarray}    
    \bar{\fp}, \bar{\fp}' & ::= & 
    %\lexpr \fpI \lexpr 
    %\mid 
    \lexpr \fpR \lexpr 
    \mid \lexpr \fpW \lexpr 
    \mid \lexpr \fpRW (\lexpr, \lexpr)
    \mid \null \fpA \cass{\kap(\vec{\lvar})}{\lrid}  
    \mid \null \fpD \cass{\kap(\vec{\lvar})}{\lrid} 
    \mid \cass{\kap(\vec{\lvar})}{\lrid} \fpU \cass{\kap(\vec{\lvar})}{\lrid} 
    \mid \bar{\fp} \sep \bar{\fp}'
\end{rclarray}
\] 
Given a logical environment $\lenv \in \LEnv$ (\cref{def:local_assertions}), a stack $\stk \in \Stacks$ (\cref{def:stacks}) and the  interpretation function for transactional assertions \( \evalLS[(.,.)]{.} \) (\cref{def:fingerprint}), the \emph{fingerprint and capability transferring assertions} is interpreted through function, $\evalF[(., .)]{.}: \FAst \times \LEnv \times \Stacks \parfun \Opsets \times \Caps \times \Caps$:
\[
\begin{rclarray}
    \evalF{ \bar{\fp} } & \defeq &
        \Setcon{(\opset, \ca, \ca')}{
            (\stub,\opset) \in \evalLS{\bar{\fp}} \land \ca, \ca' \in \unitC
        } \quad \text{where} \ \bar{\fp} \in \LAst \\
    \evalF{\null \fpA \cass{\kap(\vec{\lvar})}{\lrid} } & \defeq & 
        \Setcon{(\unitO, \ca, \ca')}{
            \ca = \evalC{\cass{\kap(\vec{\lvar})}{\lrid}} \land \ca' \in \unitC
        } \\
    \evalF{\null \fpD \cass{\kap(\vec{\lvar})}{\lrid} } & \defeq &
        \Setcon{(\unitO, \ca, \ca')}{
            \ca \in \unitC \land \ca'  = \evalC{\cass{\kap(\vec{\lvar})}{\lrid}} 
        } \\
    \evalF{\cass{\kap(\vec{\lvar})}{\lrid} \fpU \cass{\kap'(\vec{\lvar}')}{\lrid} } & \defeq &
        \Setcon{(\unitO, \ca, \ca')}{
            \ca = \evalC{\cass{\kap(\vec{\lvar})}{\lrid}} \land \ca'  = \evalC{\cass{\kap'(\vec{\lvar}')}{\lrid}} 
        } \\
    \evalF{\fp_{1} \sep \fp_{2}} & \defeq & \Setcon{ ( \opset_{1} \composeO \opset_{2}, \ca_{1} \composeC \ca_{2}, \ca'_{1} \composeC \ca'_{2} ) }{(\opset_{1}, \ca_{1}, \ca'_{1}) \in \evalF{\fp_{1}}  \land (\opset_{2}, \ca_{2}, \ca'_{2}) \in \evalF{\fp_{2}}}\\

\end{rclarray}
\]
The grammar of \emph{interference assertions}, \( \intass \in \IAst \), is defined as:
\(
\begin{rclarray}
	\intass & ::=  & \emptyset \mid \Set{ \exsts{\vec{\lvar}} \perm{\kap} : \fp } \cup \intass 
\end{rclarray}
\).
The \emph{interference environments} \( \intf \) is defined as the follows:
\[
\begin{rclarray}
    \inter \in \Interference & \defeq & \Kaps \to ( \HisHeaps \times \Views \times \Caps ) \times  ( \HisHeaps \times \Views \times \Caps )
\end{rclarray}
\]
The interference assertions are interpreted to interference environments via \emph{interference interpretation} function, $\evalI[(., .)]{.}: \IAst \times \LEnv \times \Stacks \to \Interference$:
\[
\begin{array}{@{}l}
	\evalI{\Set{ \exsts{\vec{\lvar}} \perm{\kap} : \fp } \cup \intass }(\kap') \eqdef \\
    	\quad \left\{ 
            \begin{array}{@{}l @{\qqquad} l}
            \multicolumn{2}{@{}l@{}}{
                    \Setcon{
                        \begin{B}
                            (\hh, \vi, \ca_r \composeC \ca_f ), \\ 
                            (\hh',\vi', \ca_f \composeC \ca_a)
                        \end{B}
                    }{ 
                        \pred{atomic}{\hh, \vi} 
                        \land \exsts{\txid, \opset, \cl} 
                        ( \opset, \ca_{a}, \ca_{r} ) \in \evalF[\lenv',\stk]{\fp} \\
                        \quad {} \land \txid \in \func{nextTxid}{\hh, \cl}
                        \land \hh' = \updM{\hh, \vi, \txid, \opset}  \\
                        \quad {} \land \pred{readFrom}{\hh, \vi, \opset} 
                        \land \vi' \geq \vi 
                        \land \pred{atomic}{\hh', \vi'}
                        \\
                    } 
                    \cup \evalI{\intass}(\kap')%
            } \\
            & \text{if there exist a logical environment} \ \lenv' \ \text{by replacing} \ \vec{\lvar} \ \text{with some} \ \vec{\val} \ \text{\ie} \\ 
            & \lenv' = \lenv\rmto{\vec{\lvar}}{\vec{\val}}, \ \text{and under the new logical environment} \ \kap' = \evalI[\lenv', \stk]{\kap} \\
            \evalI{\intass}(\kap') 
            & \text{otherwise} \\
    	    \end{array}
        \right.  \\
\end{array}
\]
%\sx{maybe put readfrom as a side condition to the updateMKVS }
where the \( \predn{readFrom} \) asserts the fingerprint makes sense with respect to the view:
\[
\begin{rclarray}
    \pred{readFrom}{\hh, \vi, \opset} & \defeq & \fora{\ke, \val} (\otR, \ke, \val) \in \opset \implies \valueOf(\hh(\ke,\vi(\ke))) = \val
\end{rclarray}
\]
\end{definition}

A \emph{world} \( \w \in \World \) (\cref{def:world}) is a pair of \emph{local capabilities} \( \ca \) (\cref{def:capabilities}) and \emph{a shared state} \( \gs \).
The shared state is a function from region identifiers to quadruples consisting of key-value stores, views, shared capabilities and interference environments.
The composition functions for shared states \( \composeS \) asserts two shared states are identical.

\begin{definition}[Worlds]
\label{def:invariant-region}
\label{def:world}
Given the set of key-value stores $\HisHeaps$ (\cref{def:his_heap}), views \( \Views \) (\cref{def:views}), capabilities \( \Caps\) (\cref{def:capabilities}) and region identifiers \( \RegionID \), the set of \emph{shared states} is \( \SStates \eqdef \RegionID \parfun \HisHeaps \times \Views \times \Caps \times \Interference \).
%Each region has its current state and the interference.
The \emph{shared state composition function}, $\composeS: \SStates \times \SStates \parfun \SStates$, is defined as $\composeS \eqdef \composeEq$, where for all domains $\sort M$ and all $m, m' \in \sort M$:
%
\[
\begin{rclarray}
	m \composeEq m' &  \eqdef  &
	\begin{cases}
		m & \text{if } m = m'\\
		\text{undefined} & \text{otherwise}
	\end{cases}
\end{rclarray}
\]
The set of \emph{worlds} \( \World \) is defined as the follows:
\[
\begin{rclarray}
	\world \in \World  & \eqdef & 
    \Setcon{
        (\ca, \gs) 
    }{ 
        \ca \in \Caps \land \gs \in \SStates
        \land \exsts{ \ca' } 
        (\stub, \stub, \ca') \in \func{collapse}{\gs} \\
        \quad {} \land \dom(\ca \composeC \ca') \subseteq \dom(\gs) 
        \land \fora{\rid \in \dom(\gs)}
        \exsts{\hh, \vi, \ca'', \intf}  \\
        \qquad \gs(\rid) = (\hh, \vi, \ca'', \intf) 
        \land ( \mkvs, \vi, \ca'' ) \in \func{inv}{\rid, \intf} 
        \land \dom(\hh) = \dom(\vi) \\
        \qquad {} \land \fora{ \addr \in \dom(\vi) }
        0 \leq \cu( \addr ) < \left| \hh(\addr) \right|
    }
\end{rclarray}
\]               
The \( \funcn{collapse}\) function is defined as the follows:
\[
\begin{rclarray}
    \func{collapse}{\emptyset} & \defeq & \Setcon{(\unitHH, \unitVI, \ca )}{ \ca \in \unitC } \\
    \func{collapse}{\Set{\rid \mapsto (\hh, \vi, \ca, \intf)} \uplus \gs } & \defeq & 
        \Setcon{ 
            (\hh \composeHH \hh', \vi \composeCU \vi', \ca \composeC \ca') 
        }{ 
            \land (\hh', \vi', \ca') \in \func{collapse}{\gs} }\\
\end{rclarray}
\] 
where the composition for kv-store are standard disjointed union for functions \( \composeHH \defeq \uplus\).
Assuming a global function \( \funcn{init} : \RegionID \to \HisHeaps \times \Views \times \Caps \) that returns the initial state for a region,
the \( ( \mkvs, \vi, \ca ) \in \func{inv}{\rid, \intf} \) iff:
\[
\begin{array}[t]{@{}l}
    ( \mkvs, \vi, \ca ) = \func{init}{\rid} 
    \lor \exsts{\mkvs', \vi', \ca'}
    ( \mkvs' , \vi', \ca' ) \in \func{inv}{\rid, \intf} \land (( \mkvs' , \vi', \ca' ), ( \mkvs, \vi, \ca )) \in \intf(\kap) 
\end{array}
\]
% 
The \emph{world composition function}, $\composeW: \World \times \World \parfun \World$, is defined component-wise as: $\composeW \eqdef (\composeC, \composeS)$.
The \emph{world unit set} is $\unitW \eqdef \Setcon{(\ca, \gs)}{(\ca, \gs) \in \World \land \ca \in \unitC}$.
The \emph{partial commutative monoid of worlds} is $(\World, \composeW, \unitW)$.
\end{definition}

The well-formed conditions for a world are:
\textbf{(i)} the state of every region is disjointed with each others (induced by the \(\funcn{collapse}\) function);
\textbf{(ii)} capabilities from regions are compatible with local capabilities and there is no capability of which the region identifier never appears in the shared state \ie \( \dom(\ca \composeC \ca') \in \dom(\gs) \); 
\textbf{(iii)} the state of a region \( (\mkvs, \vi, \ca'') \) satisfies the invariant of the region \( (\mkvs, \vi, \ca'') \in \func{inv}{\rid, \intf}\) and the domain of the view is the same as the domain of the key-value store, meaning they have exactly the same keys;
and \textbf{(iv)} the view of each region should be in the range of the key-value store.
In the \cref{def:world}, the \( \funcn{collapse} \) function erases the region identifiers and composites the states point-wise.
Because of the well-formedness, we introduce the \emph{erase function} \( \eraseW{.} : \World \to \HisHeaps \times \Views \) that collapses a world to \emph{a unique pair of a key-value store and a view}:
\(
\begin{rclarray}
    \eraseW{(\ca,\gs)} & \defeq & (\hh, \vi)
\end{rclarray}
\),
where \( (\hh, \vi, \stub) \in \func{collapse}{\gs} \).

The \emph{program assertions} also \emph{assertions} (\cref{def:assertion}) have capability assertions \( \cass{\kap}{\lrid} \) for local capabilities, \emph{shared region assertions} also known as \emph{boxed assertions} \( \boxass{\bar{\lpre}}{\lrid}{\intass} \) for the shared states and standard separation logic assertions.
We assume the entire key-value store are shared, thus we do not have assertions related to key-value stores outside shared regions.

\begin{definition}[Program assertions]
\label{def:assertion}
\label{def:prog-assertion}
Given the set of logical expression \( \lexpr \in \LExpr\), the set of \emph{program assertions}, $\gpre, \gpost \in \Ast$, are defined by the following inductive grammar:
\[
\begin{rclarray}
    \bar{\lpre}, \bar{\lpost} & ::= & \False \mid \True \mid \bar{\lpre} \land \bar{\lpost} \mid \bar{\lpre} \lor \bar{\lpost} \mid \exsts{\lvar} \bar{\lpre} \mid \bar{\lpre} \implies \bar{\lpost} \mid \assemp \mid \cass{\kap}{\lrid} \mid \lexpr \pt \lexpr \mid \bar{\lpre} \sep \bar{\lpost} \\
	\gpre , \gpost & ::= & \False \mid \True \mid \gpre \land \gpost \mid \gpre \lor \gpost \mid \exsts{\lvar}\gpre \mid \gpre \implies \gpost \mid \assemp \mid \cass{\kap}{\lrid} \mid \boxass{\bar{\lpre}}{\lrid}{\intass} \mid \gpre \sep \gpost \\
\end{rclarray}
\]
%
where $\lvar, \lrid \in \LVar$, $\lexpr_1, \lexpr_2 \in \LExpr$ (\cref{def:local_assertions}), $\kap \in \Kaps$ (\cref{def:capabilities}) and $\intass \in \IAst$ (\cref{def:intf}).
Given a logical environment $\lenv \in \LEnv$ and a stack $\stk \in \Stacks$, the \emph{assertion interpretation} function, $\evalW[(., .)]{.}: \Ast \times \LEnv \times \Stacks \to \powerset{\World}$, is defined as follows:
%
\[
\begin{array}{@{} l @{\qquad} l @{}}
\begin{rclarray}
	\evalW{\False} & \defeq & \emptyset \\
	\evalW{\True} & \defeq & \World \\
	\evalW{\gpre \land \gpost} & \defeq & \evalW{\gpre} \cap \evalW{\gpost} \\
	\evalW{\gpre \lor \gpost} & \defeq & \evalW{\gpre} \cup \evalW{\gpost} \\ 
\end{rclarray} 
&
\begin{rclarray}
	\evalW{\exsts{\lvar}  \gpre} & \defeq & \bigcup\limits_{\val \in \textnormal{\Val}} \evalW[\lenv\remapsto{\lvar}{\val}, \stk]{\gpre} \\
	\evalW{\gpre \implies \gpost} & \defeq & \Setcon{\w}{\w \in \evalW{\gpre} \implies \w \in \evalW{\gpost}} \\
	\evalW{\emp} & \defeq & \unitW \\
	\evalW{\cass{\kap}{\lrid}} & \defeq & \Setcon{ (\Set{\lrid \mapsto \evalI{\kap}}, \gs) }{\gs \in \SStates} \\
\end{rclarray}  
\end{array}
\]
\[
\begin{rclarray}
	\evalW{ \boxass{\bar{\lpre}}{\lrid}{\intass} } & \defeq & 
    \Setcon{
        (\ca, \gs)
    }{         
        \exsts{\hh, \vi, \ca'}
        \ca \in \unitC
        \land \gs(\lrid) = (\hh, \vi, \ca', \intf\evalI{\intass}) 
        \land (\hh, \vi, \ca') \in \evalAUX{\bar{\lpre}} 
    } \\
	\evalW{ \gpre \sep \gpost } & \defeq & 
	\Setcon{
	   (\world_1 \composeW \world_2) 
    }{
       \world_1 \in \evalW{\gpre} \land \world_2 \in \evalW{\gpost}
	} \\
\end{rclarray}%
\]
The function \( \evalAUX[(., .)]{.} \) evaluates the assertions \( \bar{\lpre} \):
\[
\begin{array}{@{} l @{\qquad} l @{}}
\begin{rclarray}
    \evalAUX{\assfalse} & \defeq & \emptyset \\
    \evalAUX{\asstrue} & \defeq & \HisHeaps \times \Views \times \Caps \\
    \evalAUX{\bar{\lpre} \land \bar{\lpost}} & \defeq & \evalAUX{\bar{\lpre}} \cap \evalAUX{\bar{\lpost}} \\ 
    \evalAUX{\bar{\lpre} \lor \bar{\lpost}} & \defeq & \evalAUX{\bar{\lpre}} \cup \evalAUX{\bar{\lpost}} \\ 
\end{rclarray} 
&
\begin{rclarray}
    \evalAUX{\exsts{\lvar} \bar{\lpre}} & \defeq & \bigcup\limits_{\val \in \Val} \evalAUX[\lenv\rmto{\lvar}{\val}, \stk]{\bar{\lpre}} \\
    \evalAUX{\bar{\lpre} \implies \bar{\lpost}} & \defeq & \Setcon{ (\hh, \vi, \ca) }{ (\hh, \vi, \ca) \in \evalAUX{\bar{\lpre}} \\ \quad \implies (\hh, \vi, \ca) \in \evalAUX{\bar{\lpost}} }\\
    \evalAUX{\assemp} & \defeq & \Setcon{ (\unitHH, \unitVI, \ca) }{\ca \in \unitC } \\
\end{rclarray}
\end{array}
\]
\[
\begin{rclarray}
    \evalAUX{\cass{\kap}{\lrid}} & \defeq & \Set{ (\unitHH, \unitVI, \Set{\lrid \mapsto \evalC{\kap}}) }\\
    %\evalAUX{\lexpr_{1} \pt \lexpr_2} & \defeq & 
    %\Setcon{ (\hh, \vi, \ca) }{%
        %(\hh, \vi, \ca) \in \evalAUX{\lexpr_{1} \pt \luexpr_2}
        %\land \exsts{\ke = \evalLE{\lexpr_{1}}} 
        %\vi(\ke) = \lvert \mkvs(\ke) \rvert - 1
    %} \\
    \evalAUX{\lexpr_{1} \pt \lexpr_2} & \defeq & \Setcon{ (\hh, \vi, \ca) }{ \Set{ \evalLE{\lexpr_{1}} \mapsto \evalLE{\lexpr_{2}} } = \clpsHH{\hh, \vi} \land \ca \in \unitC } \\
    \evalAUX{\bar{\lpre} \sep \bar{\lpost}} & \defeq & 
    \Setcon{ (\hh \composeHH \hh', \vi \composeVI \vi', \ca \composeC \ca') }{ (\hh, \vi, \ca) \in \evalAUX{\bar{\lpre}} \land (\hh', \vi', \ca') \in \evalAUX{\bar{\lpost}} } \\
\end{rclarray}%
\]
\end{definition}


The assertions inside regions \( \bar{\lpre} \) have capability assertions for shared capabilities, single-key assertions \( \lexpr \pt \lexpr \) and other standard separation logic assertions.
The single-key assertions \( \lexpr_1 \pt \lexpr_2 \) are interpreted to any pairs of kv-stores and views.
The view of each pair points to a version of the key \( \evalLE{\lexpr_1 }\) carrying value \( \evalLE{\lexpr_2}\).
The interpretation \( \evalAUX[(.,.)]{.} \) for the rest of \( \bar{\lpre} \) are standard.

Given \( \bar{\lpre} \) and its interpretation, shared region assertions are in the form of \( \boxass{\bar{\lpre}}{\lrid}{\intass} \).
They are interpreted to any world \( (\ca, \gs) \) where the local capabilities \( \ca \) is of unit elements,
the state of the region \( \lrid \) satisfies the assertion \( \bar{\lpre} \) and the interference environment of the region satisfies \( \intass \).
Because the composition of worlds is defined when the shared states are identical,
it means \( \boxass{\bar{\lpre}}{\lrid}{\intass} \sep \boxass{\bar{\lpost}}{\lrid}{\intass} \implies \boxass{\bar{\lpre} \land \bar{\lpost} }{\lrid}{\intass} \).

The \emph{rely} \( \Rely \) describes the allowed actions for the environment while the \emph{guarantee} \( \Guar \) are for the current client (\cref{def:rely-guarantee}).
They are defined as transitions over worlds.

\begin{definition}[Rely and guarantee]
\label{def:rely-guarantee}
Given the set of worlds $\World$ (\cref{def:world}), the \emph{update rely} relation, $\relyU \subseteq \World \times \World$, is defined as the follows:
\[	
    \begin{rclarray}
	\relyU & \eqdef &
	\Setcon{
		((\ca_l,\gs), (\ca_l, \gs'))	
	}{
        \fora{\rid}
        \gs(\rid) = \gs'(\rid) \lor 
        \exsts{\kap, \hh, \hh', \vi_\rid, \vi_{e}, \vi_{e}', \ca, \ca_\rid, \ca_{\rid}', \intf}   \\
        \quad \gs(\rid) = (\hh, \vi_\rid, \ca_\rid, \intf)
        \land \gs'(\rid) = (\hh', \vi_{\rid}, \ca_{\rid}',\intf) 
        \land ( \ca \composeC \ca_l \composeC \ca_r )\isdef
        \\
        \quad {} \land \kap \sqsubseteq \ca(\rid) 
        \land ( (\hh, \vi_e, \ca_\rid), (\hh', \vi_{e}', \ca_{\rid}') )  \in \intf(\kap)
        \land \vi_{\rid}' \geq \vi_\rid
	} \\
    \end{rclarray}
\]
The \emph{view shift rely} relation $\relyV \subseteq \World \times \World$, is defined as follows:
\[
    \begin{rclarray}
	\relyV & \eqdef &
	\Setcon{
		((\ca_l,\gs), (\ca_l, \gs'))	
	}{
        \fora{\rid}
        \gs(\rid) = \gs'(\rid) \lor 
        \exsts{\hh, \vi, \vi'\ca, \ca, \intf}
        \gs(\rid) = (\hh, \vi, \ca, \intf)
        \land \gs'(\rid) = (\hh, \vi', \ca, \intf) 
	} \\
    \end{rclarray}
\]
The \emph{rely} relation \( \Rely \) is the transitive closure of updates and view shift: \( \Rely \defeq {(\relyU \cup \relyV)}^{*} \).
The \emph{guarantee} relation, $\Guar \subseteq \World \times \World$, is defined as follows:
\[	
    \begin{rclarray}
	\Guar & \eqdef &
	\Setcon{
		((\ca_l,\gs), (\ca_{l}', \gs'))	
	}{
        \fora{\rid}
        \gs(\rid) = \gs'(\rid) \lor {}
        \exsts{\kap, \hh, \hh', \vi_\rid, \vi_{\rid}', \ca_\rid, \ca_{\rid}', \intf}   \\
        \quad \gs(\rid) = (\hh, \vi_\rid, \ca_\rid,\intf)
        \land \gs'(\rid) = (\hh', \vi_{\rid}', \ca_{\rid}',\intf) 
        \land \kap \sqsubseteq \ca_{l}(\rid)  \\
        \quad {} \land ( (\hh, \vi_\rid, \ca_\rid), (\hh', \vi_{\rid}', \ca_{\rid}') )  \in \intf(\kap)
        \land (\ca_{l} \composeC \ca_\rid)^{\perp} = (\ca_{l}' \composeC \ca_{\rid}')^{\perp}
	} \\
    \end{rclarray}
\]
where for any element \( m \) from its domain \( \sort{M} \), the  \emph{orthogonal} is defined as:
\[
\begin{rclarray}
m^{\perp} & \defeq & \Setcon{m'}{(m \compose{} m')\isdef \land m' \in \sort{M}} 
\end{rclarray}
\]
\end{definition}

The \emph{rely} \( \Rely \) describes how the environment can change the state of the shared regions.
Given the local capabilities \( \ca_l \) and shared capabilities \( \ca_r \), the environment might own any capabilities \( \ca\) that are compatible, \ie \( (\ca \composeC \ca_l \composeC \ca_r)\isdef \).
Then the \emph{update rely} \( \relyU \) allows the environment to perform those actions associated with the capabilities \( \ca \) with their own view \( \vi_e \) to update the key-value store and shared capabilities.
Note that the \emph{update rely} does not change the views, since the views are actually local for clients.
Separately, the \emph{view shift rely} \( \relyV \) allows to advance the local view.
%For technical reasons, even though the environment cannot change the view of the current client \( \vi_r\), but it is allowed to arbitrarily shift to the later versions due to the fact that for certain execution tests, the old view might be valid under the new key-value store.

The \emph{guarantee} \( \Guar \) describes the allowed actions for the current client (\cref{def:rely-guarantee}).
The current client can perform actions associated with the local capabilities \( \ca_l \) to update the shared region and consequently the local capabilities.
Yet it should ensure no resource created or deleted through requiring that the \emph{orthogonal} of local capabilities and shared capabilities together remains unchanged.
The orthogonal of a capability \( \ca \) is a set of capabilities that are compatible with the capability \( \ca \).
The guarantee allows to update several regions, but each region can be updated at most once.


\begin{definition}[Stable]
\label{def:stable}
A set of worlds $\setworld \subseteq \World$ is \emph{stable}, written $\stable{\setworld, \como}$, if and only if it is closed under the rely relation: 
\[
    \begin{rclarray}
        \stable{\setworld, \como} & \eqdef & 
        \begin{array}[t]{@{}l}
            \fora{\w, \w', \mkvs, \mkvs', \vi, \vi', \vi''} 
            \w \in \setworld 
            \land (\w, \w') \in \Rely  
            \land (\hh, \vi) = \eraseW{\w} \\
            \quad \land (\hh', \vi') = \eraseW{\w'} 
            \land \hh' \in \mathcal{CM}(\como)
            \land \et \vdash (\mkvs', \vi') \csat \unitO :  \vi'' 
            \implies \w' \in \setworld
        \end{array} \\
    \end{rclarray}
\]
\end{definition}

Assertions are \emph{stable} if they remain true against the rely (\cref{def:stable}).
Formally speaking, a set of worlds \( \setworld \) is stable under certain execution test \( \et \) if the set is closed under rely relation \( (\w, \w') \in \Rely \) with the side conditions:
\textbf{(i)} the new key-value store after a rely step is allowed by the consistency model induced by the execution test \( \mkvs' \in \CMs(\et) \);
and \textbf{(ii)} the new view under the new key-value store is able to progress, \ie \( \et \vdash (\mkvs', \vi') \csat \unitO :  \vi'' \).
Given the progress condition the new view is at least able to execute a \emph{dummy transaction}, \ie a transaction with empty fingerprint.
It ensures that the view make sense with respect to kv-store.
For example under serialisibility the progress condition ensure the view has to be always up-to-date with respect to the kv-store.

The rules for programs are in \cref{fig:rule-prog}.
The triple \( \tripleG{\gpre}{\prog}{\gpost} \) asserts that given any state satisfying the precondition \( \gpre \), if the program \( \prog \) terminates under the execution test \( \et \), the state after satisfies the postcondition \( \gpost \).
The \rl{PRCommit} rule is the key rule.
It commits the local effect of a transaction \( \trans \) via \emph{repartitioning} \( \repartition{\gpre}{\gpost}{\lpre}{\lpost} \).
The repartitioning means:
\begin{itemize}
\item
for any possible world \( \w \) satisfying the precondition \( \gpre \), there exists a snapshot corresponding the world \( \getSN{\mkvs, \vi} \) with empty fingerprint \( \unitO \) that overall satisfies the transactional precondition \( \lpre \);
\item
for any possible fingerprints  \( \opset \) satisfying the transactional postcondition \( \lpost \), there exists a world \( \w' \) satisfying the postcondition \( \gpost \) such that:
\begin{itemize}
\item 
the new key-value store in the new world \( \mkvs' \) is updated via committing the fingerprints \( \opset \) to \( \mkvs \);
\item
the new view \( \vi' \) is picked so that the update passes the execution test \( \et \vdash (\mkvs,\vi) \csat \opset : \vi' \);
\item
and, the update is allowed by the guarantee \( (\w, \w') \in \Guar\).
\end{itemize}
\end{itemize}


\begin{figure}[t!]
\hrule\vspace{5pt}

\begin{mathpar}
    \inferrule[\rl{PRCommit}]{%
        \tripleL{\lpre}{\trans}{\lpost} 
        \\ \repartition{\gpre}{\gpost}{\lpre}{\lpost}
        \\\\ \stable{\gpre, \como} 
        \\ \stable{\gpost, \como} 
    }{%
        \tripleG{\gpre}{ \ptrans{\trans} }{\gpost}
    }
    \and
    \inferrule[\rl{PRPar}]{%
        \tripleG{ \gpre_{1} }{ \cmd_{1} }{ \gpost_{1} }
        \\ \tripleG{ \gpre_{2} }{ \cmd_{2} }{ \gpost_{2} } 
        \\\\ \stable{\gpre_{1}, \como} 
        \\ \stable{\gpre_{2}, \como} 
    }{%
        \tripleG{ \gpre_{1} \sep \gpre_{2} }{ \cmd_{1} \ppar \cmd_{2} }{ \gpost_{1} \sep \gpost_{2} }
    }
    \and
    \inferrule[\rl{PRAss}]{%
        \thvar \notin \func{fv}{\lexpr} 
    }{%
        \tripleG{\thvar \dot= \lexpr }{ \pass{\thvar}{\expr} }{\thvar \dot= \expr\sub{\thvar}{\lexpr} }
    }
    \and
    \inferrule[\rl{PRAssume}]{ }{%
        \tripleG{ \expr \dot\neq 0 }{ \passume{\expr} }{ \expr \dot\neq 0 } 
    }
    \and
    \inferrule[\rl{PRChoice}]{%
        \tripleG{ \gpre }{ \cmd_{1} }{ \gpost } 
        \\ \tripleG{ \gpre }{ \cmd_{2} }{ \gpost } 
    }{%
        \tripleG{ \gpre }{ \cmd_{1} \pchoice \cmd_{2} }{ \gpost }
    }
    \and
    \inferrule[\rl{PRSeq}]{%
        \tripleG{ \gpre }{ \cmd_{1} }{ \gframe }
        \\ \tripleG{ \gframe }{ \cmd_{2} }{ \gpost }
    }{%
        \tripleG{ \gpre }{ \cmd_{1} \pseq \cmd_{2} }{ \gpost }
    }
    \and
    \inferrule[\rl{PRIter}]{%
        \tripleG{ \gpre }{ \cmd }{ \gpre } 
    }{%
        \tripleG{ \gpre }{ \cmd\prepeat }{ \gpre }
    }
    \and
    \inferrule[\rl{PRFrame}]{%
        \tripleG{ \gpre }{ \cmd }{ \gpost } 
        \\ \stable{\gframe, \como}
        \\ \func{fv}{\gframe} \cap \func{modify}{\cmd} = \emptyset 
    }{%
        \tripleG{ \gpre \sep \gframe }{ \cmd }{ \gpost \sep \gframe }
    }
\end{mathpar}


\hrule\vspace{5pt}
\[
\begin{rclarray}
    \repartition{\gpre}{\gpost}{\lpre}{\lpost} & \defeq & 
    \begin{array}[t]{@{}l@{}}
        \fora{ \w, \mkvs, \vi, \lenv, \stk } 
        \w \in \evalW{\gpre} 
        \land (\mkvs, \vi) = \eraseW{\w}
        \implies 
        (\getSN{\mkvs, \vi}, \unitO) \in \evalLS{\lpre}  \\
        \quad {} \land \fora{\w', \mkvs', \vi', \stk', \txid, \opset, \cl} 
        \txid \in \fresh{\mkvs, \cl} 
        \land (\stub, \opset) \in \evalLS[\lenv, \stk']{\lpost} \\
        \qquad {} \land \mkvs' = \updM{\mkvs, \vi, \txid, \opset}
        %\land \eraseW{\w}\projection{2} \leq \eraseW{\w'}\projection{2} 
        \land \et \vdash (\mkvs, \vi) \csat \opset : \vi' \\
        \qquad {} \land (\mkvs', \vi') = \eraseW{\w'} 
        \land (\w, \w') \in \Guar
        \implies \w' \in \evalW[\lenv, \stk']{\gpost}
    \end{array} 
\end{rclarray}                          
\]

\hrule\vspace{5pt}
\caption{The rules for programs}
\label{fig:rule-prog}
\end{figure}

\subsection{Transaction Soundness}


\begin{thm}[Transaction soundness]
\label{thm:transaction-soundness}
The transaction soundness is as follows:
\[
    \begin{array}{@{}l@{}}
        \for{ \lpre, \trans, \lpost } \tripleL{\lpre}{\trans}{\lpost} \implies \ \tripleSemL{\lpre}{\trans}{\lpost} \\
    \end{array}
\]
where,
\[
    \begin{rclarray}
    \tripleSemL{\lpre}{\trans}{\lpost} & \eqdef &
    \begin{array}[t]{@{}l@{}}
        \for{\lenv, \stk, \stk', \h, \h', \opset, \opset' } 
        (\h, \opset) \in \evalLS[\lenv, \stk]{\lpre} \\
        \quad {} \land \vdash (\stk, \h, \opset ), \trans \toL^{*}  (\stk', \h', \opset' ), \pskip 
        \implies (\h', \opset') \in \evalLS[\lenv, \stk']{\lpost}
    \end{array}
    \end{rclarray}
\]
\end{thm}
\begin{proof}
Induction on the derivations.

\caseB{\rl{TRSkip}}

We have  \(\trans \equiv \pskip\), \( \lpre \equiv \lpost \equiv \assemp \), thus \( \h_{p} = \h_{q} = \unitH \), \( \opset = \opset' \) and \( \stk = \stk' \), and then \( (\unitH,\unitO ) \in \evalLS[\lenv, \stk']{\assemp} \) holds.

\caseB{\rl{TRAss}}

We have \(\trans \equiv ( \pass{\var}{\expr} ) \), \( \lpre \equiv ( \var \doteq \lexpr ) \) and \( \lpost \equiv ( \var \doteq \expr\sub{\var}{\lexpr} ) \) for some \( \expr, \lexpr \) and \( \var \) such that \( \var \notin \func{fv}{\lexpr} \land \var \in \Vars\).
Given the transaction semantics (\figref{fig:thread_semantics}), it has \( \stk' = \stk\rmto{\var}{\val} \) where \( \val = \evalLE[\lenv, \stk]{\expr\sub{\var}{\lexpr}} \).
Since \( \var \notin \func{fv}{\lexpr} \), we know \( \evalLE[\lenv, \stk]{\lexpr} = \evalLE[\lenv, \stk']{\lexpr} \), and then \( \evalLE[\lenv, \stk]{\expr\sub{\var}{\lexpr}} = \evalLE[\lenv, \stk']{\expr\sub{\var}{\lexpr}} \).
This means the assertions related to stack holds even thought the stack changes.
Also because the heap and events set remains unchanged, this is \( \h = \h' \) and \( \opset = \opset' \), we have \( (\h', \opset' ) \in \evalLS[\lenv, \stk']{\lpost} \).

\caseB{\rl{TRLookup}}

We have  \(\trans \equiv ( \plookup{\var}{\expr} ) \) and four cases for pre- and post-conditions defined by the relation \( \toFP{\otR(\expr, \lexpr)}\).
In all the four cases, the heap remains the same \( \h = \h' = \Set{\addr \mapsto \val}\) and the stack get updated to \( \stk' = \stk\rmto{\var}{\val} \), yet since \( \var \notin \func{fv}{\lexpr}\), the logic value \( \lexpr \) and new logic address \( \expr\sub{\var}{\lexpr}\) are evaluated to the same value \( \val \) and \( \addr \).
While different cases have different operations.
Also note that the evaluation of the pre- and post-conditions in all four cases are singleton sets.
Now we do case analysis on four case and focus on the operations before and after.

If \( \lpre \equiv \expr \fpI \lexpr \) and \( \lpost \equiv \expr\sub{\var}{\lexpr} \fpR \lexpr \sep \var \dot= \lexpr \), the only interpretation for pre-condition \( \lpre \) is \( (\Set{\addr \mapsto \val}, \emptyset) \) where \( \addr = \evalLE{\expr} \) and \( \val = \evalLE{\lexpr}\).
In this case, a read operation is added \( \opset' = \Set{(\otR, \addr, \val)} \) and it is reflected in the post condition  \( \expr\sub{\var}{\lexpr} \fpR \lexpr  \).

If \( \lpre \equiv \expr \fpR \lexpr \) and \( \lpost \equiv \expr\sub{\var}{\lexpr} \fpR \lexpr \sep \var \dot= \lexpr \), since there is already a read operation, the new operations, adding a new read operation to the same address does not change anything, \ie \( \opset' = \Set{(\otR, \addr, \val)} \addO (\otR, \addr, \val ) = \Set{(\otR, \addr, \val)} \).
This is exactly the post-condition.
For the similar reason that the operations remains the same, it is sound when \( \lpre \equiv \expr \fpW \lexpr \) and \( \lpost \equiv \expr\sub{\var}{\lexpr} \fpW \lexpr \sep \var \dot= \lexpr \) and when \( \lpre \equiv \expr \fpRW (\lexpr,\lexpr') \) and \( \lpost \equiv \expr\sub{\var}{\lexpr} \fpRW (\lexpr,\lexpr') \sep \var \dot= \lexpr' \).

\caseB{ \rl{TRMutate} }

We have  \( \trans \equiv (\pmutate{\expr_{1}}{\expr_{2}}) \) and four cases for pre- and post-conditions defined by the relation \( \toFP{\otW(\expr, \lexpr)}\). 
In all the four cases, the stack remains untouched and the heap is updated to \( \h' = \Set{\addr \mapsto \val' }\) where the logical address \( \evalLE[\lenv,\stk']{\expr_{1}}  = \addr \) and the new values \( \evalLE[\lenv,\stk']{\expr_{2}} = \val'\).
Now we do case analysis on four case and focus on the operations before and after.

If \( \lpre \equiv \expr_{1} \fpI \lexpr \) and \( \lpost \equiv \expr_{1} \fpW \expr_{2} \), a new write operation is added to the initial empty operation set, this is, \( \opset' = \Set{(\otW, \addr, \val)}\) where \( \addr = \evalLE[\lenv,\stk']{\expr_{1}}\) and \( \val = \evalLE[\lenv,\stk']{\expr_{1}}\).
This is exactly the post-condition \( \expr_{1} \fpW \expr_{2} \).
If \( \lpre \equiv \expr_{1} \fpW \lexpr \) and \( \lpost \equiv \expr_{1} \fpW \expr_{2} \), the operations set before execution is \( \opset = \Set{(\otW, \addr, \val)}\) where \( \addr = \evalLE{\expr_{1}}\) and \( \val = \evalLE{\lexpr}\).
Since the set only have the last write, so the set after is \( \opset' = \opset \addO (\otW, \addr, \val') = \Set{(\otW, \addr, \val')}\), where \( \val' = \evalLE[\lenv,\stk']{\expr_{2}}\).
Note that since the stack remains untouched, so we have \( \addr = \evalLE{\expr_{1}} = \evalLE[\lenv,\stk']{\expr_{1}} \).
Thus, we have the proof for this case.
For the remaining two cases, they follows the same argument as the operations set only have the last write.

\caseI{\rl{TRChoice}}

We have  \(\trans \equiv \trans_{1} + \trans_{2} \), where \( \tripleL{\lpre}{\trans_{1}}{\lpost} \) and \( \tripleL{\lpre}{\trans_{2}}{\lpost} \) hold, for some \( \trans_{1}, \trans_{2}, \lpre, \lpost \).
Given the transaction semantics (\figref{fig:thread_semantics}), it either has \( ( \stk, \h, \opset ), \trans_{1} \pchoice \trans_{2} \toL ( \stk, \h, \opset ), \trans_{1} \) or  \( ( \stk, \h, \opset ), \trans_{1} \pchoice \trans_{2} \toL ( \stk, \h, \opset ), \trans_{2} \).
Let us pick \( \trans_{1} \) and  assume it can be reduced to \( \pskip \) from the initial state, \ie \( ( \txstk, \h, \opset ), \trans_{1}  \toL^{*} ( \txstk', \h', \opset' ), \pskip \).
By the premiss of the rule \( \tripleL{\lpre}{\trans_{1}}{\lpost} \) and the \ih, it implies \( \tripleSemL{\lpre}{\trans_{1}}{\lpost} \), so we prove \( (\h', \opset') \in \evalLE[\lenv, \stk']{\lpost} \).
Symmetrically, if we pick \( \trans_{2} \), it gives the same result.

\caseI{\rl{TRSeq}}

We have \( \trans \equiv \trans_{1} \pseq \trans_{2} \) where \( \tripleL{\lpre}{\trans_{1}}{\lframe} \) and \( \tripleL{\lframe}{\trans_{2}}{\lpost} \) hold, for some \( \trans_{1}, \trans_{2}, \lpre, \lpost, \lframe \).
Given the transaction semantics (\figref{fig:thread_semantics}), it has \( \vdash ( \stk, \h, \opset ), \trans_{1} \pseq \trans_{2} \toL^{*} ( \stk'', \h'', \opset'' ), \pskip \pseq \trans_{1} \toL ( \stk'', \h'', \opset'' ), \trans_{1} \toL^{*} ( \stk', \h', \opset' ), \pskip \) for some intermediate state \( (\stk'', \h'', \opset'') \).
By the premiss of the rule and the \ih, we have \( \tripleSemL{\lpre}{\trans_{1}}{\lframe} \) and so \( (\h'', \opset'') \in \evalLE[\lenv, \stk'']{\lframe} \).
The elimination of prefix \( \pskip\) does not change any state, so \( (\h'', \opset'') \in \evalLE[\lenv, \stk'']{\lframe} \) still holds.
Then, by the premiss and the \ih, we know \( \tripleSemL{\lframe}{\trans_{2}}{\lpost} \) and therefore the proof that \( (\h', \opset') \in \evalLE[\lenv, \stk']{\lpost} \).

\caseI{\rl{TRLoop}}

Since the triple is only partial correct, meaning that if the transaction \( \trans \) terminates it will reach a state satisfying the post-condition \( \lpost \), it is sufficient to prove the follows,
\[
    \for{\lpre, \trans, \nat > 0} \tripleL{\lpre}{\trans^{\nat}}{\lpre} \implies \ \tripleSemL{\lpre}{\trans^{\nat}}{\lpre} 
\]
where,
\[
\begin{rclarray}
    \trans^{1} & \defeq  & \trans \\
    \trans^{\nat} & \defeq  & \trans \pseq \trans^{\nat - 1} \\
\end{rclarray}
\]

We prove that by induction on the number \( \nat \).
For \( \nat = 1 \), it is proven directly by the \ih
For \( \nat > 1 \), we have \( \vdash (\stk, \h, \opset), \trans \pseq \trans^{\nat - 1} \toL^{*} (\stk'', \h'', \opset''), \trans^{\nat - 1} \toL^{*} (\stk', \h', \opset'), \pskip \) for some intermediate state \( ( \stk'', \h'', \opset'' ) \).
By the premiss and the \ih, we have \(\tripleSemL{\lpre}{\trans}{\lpre} \) and thus \(  (\h'', \opset'') \in \evalLS[\lenv, \stk'']{\lpre} \).
Then by the \ih that \(\tripleSemL{\lpre}{\trans^{\nat - 1}}{\lpre} \), we prove \(  (\h', \opset') \in \evalLS[\lenv, \stk']{\lpre} \).

\caseI{\rl{TRFrame}}

We need to prove \( \tripleSemL{\lpre \sep \lframe }{\trans}{\lpost \sep \lframe} \) given that \( \tripleSemL{\lpre}{\trans}{\lpost} \).
Assume variables \( \h, \h', \h'', \opset, \opset', \opset'', \stk, \stk' \) such that \( ( \h, \opset ) \in \evalLS[\lenv, \stk]{\lpre} \), \( ( \h', \opset' ) \in \evalLS[\lenv, \stk']{\lpost} \) and \( ( \h'', \opset'' ) \in \evalLS[\lenv, \stk]{\lframe}\).
Since \( \lpre \sep \lframe \), the point-wise composition is defined, \ie \( (\h \composeH \h'', \opset \composeO \opset'') \in \evalLS[\lenv, \stk]{\lpre \sep \lframe} \).
The domain of the heaps and operations sets, therefore, are disjointed.
The domain of a operations set is all the addresses \( \dom(\opset) \defeq \opset\projection{2}\).
We also know the domain of the operation set is a subset of the domain of the heap, \(\dom(\opset) \subseteq \dom(\h) \) which can be proven by induction on the structures of local assertions \( \LAst \).
By the hypothesis \( \tripleSemL{\lpre}{\trans}{\lpost} \), we know \( ( \stk, \h, \opset ), \trans \toL^{*} ( \stk', \h', \opset' ), \pskip \).
The heap after the execution should contain the same resources as before, this is \( \dom(\h) = \dom(\h') \).
Since \( \dom(\opset') \subseteq \dom(\h') \), we know the compositions \( \h' \composeH \h''\) and \( \opset' \composeO \opset''\) exist.
This means the frame does not affect the semantic steps, \ie \( ( \stk, \h \composeH \h'', \opset \composeO \opset''), \trans \toL^{*} ( \stk', \h' \composeH \h'', \opset' \composeO \opset'' ), \pskip \).
Finally, because there is no free variables overlap between \( \lframe \) and \( \lpre, \lpost \), the update of stack does not change the evaluation of the frame, this is, \( \evalLS[\lenv, \stk]{\lframe} = \evalLS[\lenv, \stk']{\lframe} \) which then gives us the result \( (\h' \composeH \h'', \opset' \composeO \opset'') \in \evalLS[\lenv, \stk']{\lpost \sep \lframe} \).


%and similarly \( ( \h_{q} \composeH \h_{r}, \opset_{q} \uplus \opset_{r} ) \in \evalLS[\lenv, \thstk \uplus \txstk_{q}]{\lpost \sep \lframe} \land ( \h_{q} ,\opset_{q} ) \in \evalLS[\lenv, \thstk \uplus \txstk_{q}]{\lpost} \land ( \h_{r}, \opset_{r} ) \in \evalLS[\lenv, \thstk \uplus \txstk_{q}]{\lframe}\).
%By the \ih that \( \tripleSemL{\lpre}{\trans}{\lpost} \), it means .
%Now we need to prove the follows,
%\[
    %\thstk \vdash ( \txstk_{p}, \h_{p}  \composeH \h_{r}, \opset_{p} \composeO \opset_{r}), \trans \toL^{*} ( \txstk_{q}, \h_{q} \composeH \h_{r}, \opset_{q} \composeO \opset_{r}), \pskip 
%\]
%First for the heaps part, since both \( (\h_{p} \composeH \h_{r}) \) and  \( (\h_{q} \composeH \h_{r}) \) are defined, this means the domain of the frame \( \h_{r} \) are separate from the ones of \( \h_{p}\) and \( \h_{q} \).
%Then by the \ih, the transaction \( \trans \) does not need any resource from \( \h_{r} \) to progress.
%Second for the event sets part, since it has \( ( \h_{q} \composeH \h_{r}, \opset_{q} \composeO \opset_{r} ) \in \evalLS[\lenv, \thstk \uplus \txstk_{q}]{\lpost \sep \lframe} \), so \( \opset_{q} = \unitO \lor \opset_{r} = \unitO \).
%if \( \opset_{q} = \unitO \), it must be that \( \opset_{p} = \unitO \), it holds by the \ih
%If \( \opset_{r} = \unitO \), it also holds by the \ih
%Given above we have the prove that \( \tripleSemL{\lpre \sep \lframe }{\trans}{\lpost \sep \lframe} \).


\end{proof}

\subsection{Program Soundness}
\begin{lem}[Locality of cut]
\label{lem:locality-cut}
A thread can only update its own cut when commit a new transaction,
\[
\begin{array}{@{}l}
    \for{\thstk, \thstk', \hh, \hh', \thcu, \thcu', \cu, \cu', \thid, \como} \exsts{\thcu''} \\
    \quad \thid, \como \vdash (\thstk, \hh, \thcu), \ptrans{\trans} \toT{\lbC{\stub}} (\thstk', \hh', \thcu'), \pskip \\
    \qquad \implies \thcu  = \thcu'' \uplus \setminus \Set{\thid \mapsto \cu } \land \thcu' = \thcu'' \uplus \setminus \Set{ \thid \mapsto \cu' }
\end{array}
\]
This means the cut environment can be arbitrary as long as the local update remains the same and the entire environment satisfy the consistency model.
\[
\begin{array}{@{}l}
    \for{\thstk, \thstk', \hh, \hh', \thcu, \thcu', \cu, \cu', \thid, \thid', \como} \\
    \quad \thid, \como \vdash (\thstk, \hh, \thcu \uplus \Set{\thid \mapsto \cu}), \ptrans{\trans} \toT{\lbC{\stub}} (\thstk', \hh', \thcu \uplus \Set{\thid \mapsto \cu'}), \pskip \\
    \quad {} \land ((\hh,\thcu' \uplus \Set{\thid' \mapsto \cu}),(\hh',\thcu' \uplus \Set{\thid' \mapsto \cu'})) \in \como \\
    \qquad \implies \thid', \como \vdash (\thstk, \hh, \thcu' \uplus \Set{\thid' \mapsto \cu}), \ptrans{\trans} \toT{\lbC{\stub}} (\thstk', \hh', \thcu' \uplus \Set{\thid' \mapsto \cu'}), \pskip \\
\end{array}
\]
\end{lem}
\begin{proof}
The first one is trivial as \( \thcu' = \thcu\rmto{\thid}{\cu'}\) in the \rl{PCommit} rule, and no other side condition has side effect on \( \thcu' \).
For the second part, by the hypothesis we have the following which are exactly the side conditions of the \rl{PCommit},
\[
\begin{array}{@{}l}
    \exsts{\h, \h', \txstk_{0}, \txstk, \txid, \opset } \\
    \quad \txid \in \func{fresh}{\hh}  
    \land \h = \clpsHH{\hh,\cu}
    \land \txstk_{0} = \emptyset 
    \land \thstk \vdash (\txstk_{0}, \h, \unitO), \trans \ \toL^{*} \  (\txstk, \h', \opset) , \pskip \\
    \quad {} \land \thstk' = \thstk\rmto{\ret}{\txstk(\ret)} 
    \land \hh' = \func{commit}{\hh, \cu, \txid, \opset}  \\
    \quad {} \land \cu' = \func{update}{\hh', \cu, \opset} 
    \land ((\hh,\thcu \uplus \Set{\thid \mapsto \cu}),(\hh',\thcu \uplus \Set{\thid \mapsto \cu})) \in \como
\end{array}
\]
If we replace \( \thcu \) by \( \thcu' \) and the thread identifier such that \( ((\hh, \thcu' \uplus \Set{\thid' \mapsto \cu}),(\hh',\thcu' \uplus \Set{\thid' \mapsto \cu'})) \in \como \) holds, it is easy to see other side conditions still hold, therefore we have the proof.
\end{proof}
\sx{need change for how we collapse a world}
\begin{lem}
\label{lem:rely-guar-como}
Any transition in the rely or guarantee should satisfies the consistency model,
\[
\begin{array}{@{}l}
    \for{\w, \w', \hh, \hh', \cu, \cu'}
    \exsts{\thcu, \thcu', \thid} \\
    \quad (\w, \w') \in \Rely \cup \Guar
    \land (\hh, \cu) \in \clpsW{\w}
    \land (\hh', \cu') \in \clpsW{\w'} \\
    \qquad \implies 
    \thcu(\thid) = \cu
    \land \thcu'(\thid) = \cu'
    \land ((\hh, \thcu),(\hh', \thcu')) \in \func{como}{\w}
\end{array}
\]
\end{lem}
\begin{proof}
The rely and guarantee are defined using the \( \predn{to}\) predicate by plugging in different capabilities.
It is sufficient to prove that any transition allowed by the \(\predn{to}\) predicate is also allowed by the consistency model,
\[
\begin{array}{@{}l}
    \for{\opset, \ca, \gs, \gs', \hh, \hh', \cu, \cu'}
    \exsts{\thcu, \thcu', \thid} \\
    \quad \pred{to}{\opset, \ca, \gs, \gs'} 
    \land (\hh, \cu) \in \clpsS{\gs}
    \land (\hh', \cu') \in \clpsS{\gs'} \\
    \qquad \implies 
    \thcu(\thid) = \cu
    \land \thcu'(\thid) = \cu'
    \land ((\hh, \thcu),(\hh', \thcu')) \in \como
\end{array}
\]
where the \( \como \) is the consistency model associated to regions.
Since it is a recursive predicate, we are going to prove the property by induction.
\caseB{\(\opset = \unitO\)}
It is trivial because \(\gs = \gs'\), therefore \( \hh = \hh' \) and \( \thcu = \thcu' \) and the consistency model is reflexive.
\caseI{\(\opset \neq \unitO\)}
There exist a region \( \rid \) that has been updated from \( (\hh, \cu) \) to \( (\hh', \cu') \) which is allowed by the invariant, and the update for other regions satisfies the \predn{to} predicate, which is allowed by the consistency model by \ih,
\[
    \begin{array}[t]{@{}l}
    \exsts{\rid, \hh, \hh', \hh'', \hh''', \cu, \cu', \thcu'', \thcu''', \opset', \intf, \gs'',\gs''' }  \\
    \quad \opset' \subseteq \opset
    \land \gs = \gs'' \uplus \Set{\rid \mapsto (\hh, \cu, \intf)} 
    \land \gs' = \gs''' \uplus \Set{\rid \mapsto (\hh', \cu', \intf)}  \\
    \quad {} \land (\hh,\cu) \toLTS{\opset'} (\hh',\cu') \in \func{inv}{\rid, \intf }
    \land ((\hh'', \thcu''),(\hh''',\thcu''')) \in \como
    \end{array}
\]
Given the definition of labelled transition system for a region (\defref{def:labelled-transition-system}), the transition \( (\hh, \cu) \toLTS{\opset} (\hh', \cu') \) should satisfy the consistency model,
\[
\begin{array}{@{}l}
    \exsts{ \thcu, \thcu', \thid } 
    (\hh, \cu) \toLTS{\opset'} (\hh', \cu') \in \func{inv}{\rid, \intf} \\
    \quad \implies
    \begin{B}
    \txid \in \func{fresh}{\hh} 
    \land \hh' = \func{commit}{\hh, \cu, \txid, \opset} 
    \land \cu' = \func{update}{\hh', \cu, \opset} \\
    {} \land ((\hh,\thcu),(\hh',\thcu')) \in \como
    \land \thcu(\thid) = \cu 
    \land \thcu'(\thid) = \cu' 
    \end{B}
\end{array}
\]
Because the well-formed condition for a world (\defref{def:world}), \ie regions must be disjointed, we know the compositions \( \hh \composeHH \hh'' \) and \( \hh' \composeHH \hh''' \) exist.
Also because of the locality of cut (\lemref{lem:locality-cut}), we can pick a minimum cut environment \( \thcu = \Set{\thid \mapsto \cu} \), \( \thcu' = \Set{\thid \mapsto \cu'}\) and a fresh thread identifier, in a way that it satisfies the consistency when combined with the others.
This is,
\[
\begin{array}{@{}l}
    \exsts{\thid} \\
    \begin{B}
    \thid \notin \dom(\thcu'') 
    \land (\hh \composeHH \hh'')\isdef 
    \land (\hh' \composeHH \hh''')\isdef  \\
    \land ((\hh', \Set{\thid \mapsto \cu}),(\hh'',\Set{\thid \mapsto \cu'})) \in \como 
    \land ((\hh'', \thcu''),(\hh''',\thcu''')) \in \como 
    \end{B} \\
    \quad \implies ((\hh' \composeHH \hh'', \Set{\thid \mapsto \cu} \uplus \thcu''), (\hh'' \composeHH \hh''' ,\Set{\thid \mapsto \cu'} \uplus \thcu''')) \in \como
\end{array}
\]
\end{proof}
\begin{thm}[Program soundness]
The program soundness is the follows,
\[
    \for{\gpre, \prog, \gpost}
    \como \tripleG{\gpre}{\prog}{\gpost} 
    \implies 
    \como \tripleSemG{\gpre}{\prog}{\gpost} 
\]
\end{thm}
\begin{proof}
Induction on the derivations.
\caseB{\rl{PRCommit}}
We have \( \prog \equiv \ptrans{\trans} \).
Because a transaction \( \ptrans{\trans} \) is reduced by one step in the semantics, it is sufficient to prove for any state \(\w\) that satisfies pre-condition, if a machine state \((\hh',\vi', \ca') \in \clpsW{\w'}\),  after arbitrary steps of rely, \ie \( (\w, \w') \in \Rely^{*} \) (\equref{equ:stable-pre-condition}), can transfers to a new state \((\hh'',\vi'', \ca'')\) (\equref{equ:commit-transaction}) then followed by arbitrary steps of rely \((\w'',\w''') \in \Rely^{*} \), the final state \( \w''' \) should satisfy the post-condition \(\gpost\) (\equref{equ:stable-post-condition}).
\sx{typesetting is a bit strange}
\begin{gather}
    \for{\w, \w',\lenv, \stk} 
    \stable{\gpre, \como} 
    \land \w \in \evalW{\gpre} 
    \land (\w, \w') \in \Rely
    \land (\w, \w') \in \como
    \implies \w' \in \evalW{\gpre} \tag{Stable Pre} \label{equ:stable-pre-condition} \\
    \begin{array}{@{}l}
    \begin{B}
        \tripleL{\lpre}{\trans}{\lpost}
        \land \repartition{\gpre}{\gpost}{\lpre}{\lpost}
    \end{B} \\
    \implies 
    \for{\w, \w', \hh, \hh', \vi, \vi', \vienv, \vienv', \ca, \ca', \thid, \lenv, \stk, \stk'} \\
    \quad \begin{B}
        \w \in \evalW{\gpre}
        \land (\hh, \vi, \ca) \in \clpsW{\w}
        \land \vienv(\thid) = \vi \\
        {} \land \thid, \como \vdash (\stk, \hh, \vienv), \ptrans{\trans} 
        \toT{\lbC{\txid}} (\stk', \hh', \vienv'), \pskip  \\
        {} \land \vienv'(\thid) = \vi'
        \land (\hh', \vi', \ca') \in \clpsW{\w'} 
    \end{B} 
    \implies  \w' \in \evalW[\lenv, \stk']{\gpost} 
    \end{array} \label{equ:commit-transaction} \tag{Commit} \\
    \for{\w, \w',\lenv, \stk}  
    \stable{\gpost, \como} 
    \land \w \in \evalW{\gpost} 
    \land (\w, \w') \in \Rely
    \land (\w, \w') \in \como
    \implies \w' \in \evalW{\gpost} \tag{Stable Post} \label{equ:stable-post-condition} 
\end{gather}
\sx{make sure the stack is correct}
\textbf{Stable pre-condition.} 
The \( \stable{\gpre, \como} \) predicate asserts any world \( \w \) that satisfies the pre-condition \( \gpre \), if the world can transfer to another world \( \w' \) through rely \( \Rely \), and if the transfer also satisfies the consistency model \( \como \), the new world \( \w' \) satisfies the pre-condition, which implies \equref{equ:stable-pre-condition}. 
\\
\textbf{Commit.}
For any \( \w, \hh, \vi, \lenv, \stk \) such that \( \w \in \evalW{\gpre} \) and \( (\hh, \vi, \ca) \in \clpsW{\w} \), by the predicate \( \pred{unbox}{\gpre, \lpre} \) (wrapped inside the repartitioning) we know \( (\clpsHH{\hh, \vi}, \unitO) \in \evalLS{\lpre} \), this is,
\begin{equation}
\label{equ:local-pre-condition}
\for{\w, \hh, \vi, \lenv, \stk} \w \in \evalW{P} \land (\hh, \vi, \stub) \in \clpsW{\w} \implies (\clpsHH{\hh, \vi}, \unitO) \in \evalLS{\lpre}
\end{equation}
Because of the soundness of transaction (\thmref{thm:transaction-soundness}), given a stack \( \stk \) and a logical environment \( \lenv \), if a initial configuration \( (\stk, \h, \unitO), \trans \) satisfies the pre-condition, \ie \( (\h, \unitO) \in \evalLS[\lenv,\stk]{\lpre} \), and if it can transfer to a final configuration \( (\stk', \h', \opset), \pskip \), the final configuration should satisfy the post-condition \( \lpost \).
This is, for any \( \thstk, \txstk, \txstk', \h, \h', \opset \), they satisfy the follows,
\begin{equation}
\label{equ:local-transaction-sound}
\begin{array}{@{}l}
    \for{\stk, \stk', \hh, \vi, \h, \h', \opset} 
    \h = \func{clps}{\hh,\vi} \\
    \quad (\h, \unitO) \in \evalLS[\lenv,\stk]{\lpre}
    \land \vdash (\stk, \h, \unitO), \trans \toL (\stk', \h', \opset), \pskip
    \implies (\h', \opset) \in \evalLS[\lenv,\stk']{\lpost}
\end{array}
\end{equation}
The repartition \( \repartition{\gpre}{\gpost}{\lpre}{\lpost} \) also asserts that any world \( \w \) satisfying the pre-condition \( \gpre \), if the corresponding machine of the world, \ie \( (\hh, \vi) \), can transfer to a new state \( (\hh',\vi') \), by committing the operations \( \opset \), then if a world \( \w' \) can collapses to the new machine state \( (\hh',\vi') \) and the transition \( (\w, \w') \) is allowed by both the guarantee and the consistency model, the new world \( \w' \) should satisfy the post-condition.
\begin{equation}
\label{equ:repartition}
\begin{array}{@{}l}
    \for{\w, \w', \hh, \hh', \vi, \vi', \stk, \stk', \opset, \lenv, \txid} \\
    \begin{B}
        \w \in \evalW{\gpre}
        \land (\hh, \vi, \stub) \in \eraseW{\w}
        \land \txid \in \func{fresh}{\hh} 
        \land \opset \in \evalLS[\lenv, \stk']{\lpost} \\
        {} \land \hh' = \func{updHisHp}{\hh, \vi, \txid, \opset}  
        \land \vi' = \func{updView}{\hh, \vi, \opset} \\
        {} \land (\hh',\vi', \stub) \in \eraseW{\w'}
        \land (\w, \w') \in \Guar 
        \land (\w, \w') \in \como 
    \end{B}
    \implies \w' \in \evalW[\lenv, \stk']{\gpost}
\end{array}
\end{equation}
First by \equref{equ:local-transaction-sound}, we know that for any world that satisfies \( \gpre \), there exist sets of operations corresponding to the transaction code \( \trans \) which then update the history heap and local view.
If we combine  \equref{equ:local-transaction-sound} and \equref{equ:repartition}, we have the follows,
\begin{equation}
\label{equ:combined-transaction-sound}
\begin{array}{@{}l}
    \for{\w, \w', \hh, \hh', \vi, \vi', \stk, \stk', \h, \h', \opset, \lenv, \txid} \\
    \begin{B}
        \w \in \evalW{\gpre}
        \land (\hh, \vi, \stub) \in \eraseW{\w}
        \land \txid \in \func{fresh}{\hh} 
        \land \h = \clpsHH{\hh, \cu}  \\
        {} \land \vdash (\stk, \h, \unitO), \trans \toL (\stk', \h', \opset), \pskip  \\
        {} \land \hh' = \func{updHisHp}{\hh, \vi, \txid, \opset}  
        \land \vi' = \func{updView}{\hh, \vi, \opset} \\
        {} \land (\hh',\vi', \stub) \in \eraseW{\w'}
        \land (\w, \w') \in \Guar 
        \land (\w, \w') \in \como 
    \end{B}
    \implies \w' \in \evalW[\lenv, \stk']{\gpost}
\end{array}
\end{equation}
\sx{
    Assume by the way \( (\w, \w') \in \como \) is constructed, it implies side condition for the consistency model in the semantics.
    Since how to specify consistency model does not settle down yet.
    Leave the soundness prove like this for now.
}
%Then, by the \lemref{lem:rely-guar-como} that the guarantee \( \Guar \) ensures the transition from \( \w \) to \( \w' \) satisfies the consistent model and picking the empty transaction stack as the initial \( \txstk = \txstk_{0} = \emptyset \) transaction stack for the transaction code \( \trans \), we have the follows,
%\[
%\begin{array}{@{}l}
    %\for{\w, \w', \hh, \hh', \cu, \cu', \lenv, \stk, \txid, \h, \h', \thstk} 
    %\exsts{\thcu, \thcu', \thid } \\
    %\begin{B}
        %\exsts{\h, \h', \txstk_{0}, \txstk', \txid, \opset } \\
        %\quad \w \in \evalW{\gpre} 
        %\land (\hh, \cu) \in \eraseW{\w}
        %\land \txid \in \func{fresh}{\hh} 
        %\land \h = \clpsHH{\hh, \cu}
        %\land \txstk_{0} = \emptyset \\
        %\quad {} \land \thstk \vdash (\txstk, \h, \unitO), \trans \toL (\txstk', \h', \opset), \pskip 
        %\land \hh' = \func{commit}{\hh, \cu, \txid, \opset} \\
        %\quad {} \land \cu' = \func{update}{\hh, \cu, \opset} 
        %\land (\hh',\cu') \in \eraseW{\w'}  \\
        %\quad {} \land ((\hh,\thcu),(\hh',\thcu')) \in \func{como}{\w}
        %\land \thcu(\thid) = \cu 
        %\land \thcu'(\thid) = \cu' \\
    %\end{B}
    %\implies \w' \in \evalW{\gpost}
%\end{array}
%\]
%In the equation above, we have all the side conditions of the \rl{PCommit} rule except the return value, \ie \( \thstk' = \thstk\rmto{\ret}{\txstk(\ret)} \).
%Since the return value does not affect how the post condition \( \gpost \) is interpreted by the repartition, we can fold all the side conditions to the follows,
%\[
%\begin{array}{@{}l}
    %\for{\w, \w', \hh, \hh', \cu, \cu', \lenv, \stk, \txid, \h, \h', \thstk, \thstk', \txid} 
    %\exsts{\thcu, \thcu', \thid } \\
    %\begin{B}
        %\quad \w \in \evalW{\gpre} 
        %\land (\hh, \cu) \in \eraseW{\w}
        %\land \thid, \func{como}{\w} \vdash (\thstk, \hh, \thcu), \ptrans{\trans} 
        %\toT{\lbC{\txid}} (\thstk', \hh', \thcu'), \pskip  \\
        %\quad {} \land (\hh',\cu') \in \eraseW{\w'}  
        %{} \land ((\hh,\thcu),(\hh',\thcu')) \in \func{como}{\w}
        %\land \thcu(\thid) = \cu 
        %\land \thcu'(\thid) = \cu' \\
    %\end{B}
    %\implies \w' \in \evalW{\gpost}
%\end{array}
%\]
%Now we can apply the \lemref{lem:locality-cut} which allows us to convert the existential quantification for \( \thcu, \thcu', \thid\) to global quantification, thus we have the proof for committing a new transaction, \ie \equref{equ:commit-transaction}. 
\textbf{Stable post-condition.} 
It can be proven for the similar reason as the proof for stable pre-condition.
\end{proof}

