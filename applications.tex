\section{Applications}
\label{sec:applications}

\subsection{Program Analysis}

\subsection{Verifying Implementations}
\sx{
    Short intro why it is good to use our models

    What are cops and then clock-si.
    how we verify
}
Compared with graph-based models including dependency graphs and abstract executions,
our state-based model are useful for verifying implementations,
because many implementations embed notions of global kv-stores and local views.
For centralise databases, the state of the centralised server corresponds the kv-stores,
and the views reflects the initial local states that transactions work on.
For distributed databases, since most distributed databases satisfy eventual consistency, 
\ie servers among the system eventually agree on their states,
%and use last-write-win to resolve conflicts,
the final states of all servers collectively correspond to the kv-stores,
while views reflect the local states of servers.
We verify two implementations in the literature, COPS~\cite{Lloyd:2011:DSE:2043556.2043593} and Clock-SI~\cite{Du:2013:CSI:2553409.2553434}.

\paragraph{\bf COPS}
COPS is a fully replicated database, that is, each server also known as replica contains all keys yet the associated values may be out of date.
It implements causal consistency by storing all versions and their dependencies.
Versions among all keys are totally ordered given that COPS restraints the API that only allows single-write or multiple reads transactions.
Each version has \emph{dependency}, that are other versions it depends on.
%A read-only transaction fetches versions that are self-included or after 
A version is accepted by a replica
only until all versions from the dependency have been aware to the same replica,
otherwise it is put in the pending list.

To verify COPS, we construct a view from the state of a replica.
The dependencies of versions shape the view so 
when a new transaction comes, 
the view satisfies the execution tests for causal consistency in \cref{fig:execution.tests}.
The full detail is in \cref{.....}.

\paragraph{\bf Clock-SI}
Clock-SI is sharded database that overall implements snapshot isolation.
A server also called a shard hosts a disjointed fragment of keys.
To implements snapshot isolation, Clock-SI uses physical times of shards.
Each key keeps all the versions and the physical times when they committed.
Obviously, times on shards will not always match, but the difference is bounded.
When a transaction starts, it gets a snapshot time by a shard and from now on the shard is the coordinator for the transaction.
Such transaction can read other shard, delegated by the coordinator, only if the shard's time is greater than the snapshot time.
At the end, a transaction can commits if there is no other commits to the same keys since the snapshot time.

To verify the Clock-SI, we show the view constructed from the snapshot time, if the transaction successfully commits later on, satisfies the execution tests for snapshot isolation in \cref{fig:execution.tests}.
The full detail is in \cref{.....}.
