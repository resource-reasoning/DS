\section{Applications}
\label{sec:applications}
\label{sec:program-analysis}

To show the applications of our operational semantics, 
we use it to verify several distributed protocols (\cref{sec:verify-impl}) and
to prove the invariants of several transactional libraries (\cref{sec:robustness}).

\subsection{Application: Verifying Database Protocols}
\label{sec:verify-impl}
Kv-stores and views faithfully abstract the state of geo-replicated and partitioned
databases, and execution tests provide a powerful abstraction of the synchronisation mechanisms 
enforced by these databases when committing a transaction. 
This makes it possible to use our 
semantics to verify the correctness of distributed database protocols. 
We demonstrate this by showing that the replicated database,
COPS \cite{cops}, satisfies causal consistency and
the partitioned database, Clock-SI \cite{clocksi}, satisfies snapshot isolation.
We present an intuitive account of how we verify the COPS protocol using trace refinement.
We refer the reader to \cref{sec:cops} for the full details.
In \cref{sec:clock-si}, we apply the same method to verify Clock-SI.

\begin{figure*}[!t]
\captionsetup[subfigure]{aboveskip=-5pt, belowskip=0pt}

\begin{subfigure}{\textwidth}
\begin{centertikz}

\OperationsBox[above][2]{r1}{r1}{
        nonvisible/{(\key_1,\val_0,{(t_0,\repl_0),\emptyset})}%
        , nonvisible/{(\key_2,\val_0,{(t_0,\repl_0),\emptyset})}%
        , /{(\key_1,\val_1,{(t_1,\repl_1),\emptyset})}%
};
\OperationsBox((r1.east) + (1.5,0.3))[above][2]{r2}{r2}{
          nonvisible/{(\key_1,\val_0,{(t_0,\repl_0),\emptyset})}%
        , nonvisible/{(\key_2,\val_0,{(t_0,\repl_0),\emptyset})}%
        , nonvisible/{(\key_1,\val'_1,{(t_1,\repl_2),\emptyset})}%
        , nonvisible/{(\key_2,\val_2,{(t_2,\repl_2),\Set{(\key_1,t_1,\repl_2)}})}%
};

\end{centertikz}
\caption{Client \( \cl_1 \) commits a new version of \( \key_1 \) carrying value \( \val_1 \) to replica \( \repl_1 \);
other clients commit versions to $\repl_2$. The new versions in $\repl_1$ and $\repl_2$ have not yet been propagated.}
\label{fig:initial-cops}
\label{fig:cops-after-write-transaction}
\end{subfigure}

\begin{tabularx}{\textwidth}{@{} X | c @{}}
\hline\\[-5pt]
\begin{subfigure}{0.55\textwidth}
\begin{centertikz}
\OperationsBox[above][2]{r1}{r1}{
          nonvisible/{(\key_1,\val_0,{(t_0,\repl_0),\emptyset})}%
        , nonvisible/{(\key_2,\val_0,{(t_0,\repl_0),\emptyset})}%
        , fillshade/{(\key_1,\val_1,{(t_1,\repl_1),\emptyset})}%
        , /{(\key_1,\val'_1,{(t_1,\repl_2),\emptyset})}%
        , fillshade/{(\key_2,\val'_2,{(t_2,\repl_2),\Set{(\key_1,t_1,\repl_2)}})}%
};
\end{centertikz}
\caption{Replica $\repl_1$ optimistically fetches the newest versions for \( \key_1,\key_2 \) one by one, during which it receives synchronisation messages from \( \repl_2 \).}
\vspace{-10pt}%
\label{fig:cops-request-values}
\end{subfigure}

& 

\begin{subfigure}{0.43\textwidth}
\begin{centertikz}
\OperationsBox[above][2]{r1}{r1}{
          nonvisible/{(\key_1,\val_0,{(t_0,\repl_0),\emptyset})}%
        , nonvisible/{(\key_2,\val_0,{(t_0,\repl_0),\emptyset})}%
        , nonvisible/{(\key_1,\val_1,{(t_1,\repl_1),\emptyset})}%
        , /{(\key_1,\val'_1,{(t_1,\repl_2),\emptyset})}%
        , /{(\key_2,\val'_2,{(t_2,\repl_2),\Set{(\key_1,t_1,\repl_2)}})}%
};
\end{centertikz}%
\caption{Replica \( \repl_1 \) re-fetches a causally consistent snapshot for \( \key_1,\key_2 \) using the dependency sets.}
\vspace{-10pt}%
\label{fig:cops-re-read-values}
\end{subfigure}%
\\ \hline
\end{tabularx}

\caption{COPS protocol}
\label{fig:cops-digraph}
\end{figure*}


\subsubsection{COPS Protocol} 
COPS is a fully replicated database, with each replica storing multiple versions of each key as shown in \cref{fig:cops-after-write-transaction}. 
Each COPS version \( \ver \)
such as \( (\key_1,\val_1,(\txid_1,\repl_1), \emptyset) \) in \cref{fig:cops-after-write-transaction},
contains a key ($\key_1$), a value ($\val_1$), a \emph{unique} time-stamp ($\txid_1, \repl_1$) denoting
when a client first wrote the version to the replica, 
and a set of dependencies ($\emptyset$), written $\depOf[\ver]$. 
The time-stamp associated with a version $\ver$ has the form $(\txid, \repl)$, where $\repl$ identifies the replica that committed $\ver$, 
and $\txid$ denotes the local time when $\repl$ committed $\ver$. 
Each dependency in $\depOf[\ver]$ comprises a key and the time-stamp of the versions on which $\ver$ directly depends.  
We define the \( \DEP \) relation, \( (\txid,\repl) {\xrightarrow{\DEP}} (\txid',\repl') \),
to denote that version \( (\txid,\repl) \) is included in the dependency set of version \( (\txid',\repl')\).
COPS assumes a total order over replica identifiers. 
As such, versions can be totally ordered lexicographically. 

The COPS API provides two operations: one for writing to a \emph{single}
key; and another for atomically reading from \emph{multiple} keys. 
Each call to a COPS operation is processed by a single replica. 
Each client maintains a \emph{context}, which is a set of dependencies
tracking the versions the client observes.  
We demonstrate how a client \( \cl \) interacts with a replica through the program \(\prog_{\CodeFont{COPS}} \):

\spaceshrink{-16pt}
{\displaymathfont
\begin{align*}
    \prog_{\CodeFont{COPS}} & \defeq
    ( \cl: \begin{transaction} \pmutate{\key_1}{\val_1} \end{transaction} ; \ 
    \begin{transaction} \pderef{\vx}{\key_1}; \ \pderef{\vy}{\key_2} \end{transaction} )
    %\tag{\textsc{cops-cl}}
    %\label{equ:client-repl-1}
\end{align*}
\normalsize}
\spaceshrink{-16pt}

For brevity, we assume that there are two keys, $\key_1$ and $\key_2$, 
and two replicas, $\repl_1$ and $\repl_2$, where $\repl_1 < \repl_2$ (\cref{fig:initial-cops}).
Initially, client \( \cl \) connects to replica \( \repl_1 \) and initialises its local context as $ctx {=} \emptyset$.
To execute its first single-write transaction, $\cl$ requests to write $\val_1$ to $\key_1$
by sending the message $(\key_1, \val_1, ctx)$ to its associated replica $\repl_1$
and awaits a reply.
Upon receiving the message, $\repl_1$ produces a monotonically increasing local time $\txid_1$, 
and uses it to install  a new version $\ver {=} (\key_1,\val_1, (\txid_1,\repl_1), ctx)$, as shown in \cref{fig:cops-after-write-transaction}.
Note that the dependency set of $\ver$ is the $\cl$ context ($ctx {=} \emptyset$).
Replica $\repl_1$ then sends the time-stamp $(\txid_1,\repl_1)$ back to $\cl_1$, and $\cl_1$ in turn  incorporates $(\key_1, \txid_1,\repl_1)$ in its local context,
\ie $\cl$ observes its own write. 
Finally, $\repl_1$ propagates the written version to other replicas asynchronously by sending a \emph{synchronisation message} 
using \emph{causal delivery}:
when a replica $\repl'$ receives a version $\ver'$ from another replica $\repl$, 
it waits for all $\ver'$ dependencies to arrive at $\repl'$, and then accepts $\ver'$.
As such, the set of versions contained in each replica is closed with respect to the \( \DEP \) relation.
In the example above, when other replicas receive $\ver$ from $\repl_1$, they can immediately accept $\ver$ as \( \depOf[\ver]{=}\emptyset\). 
Note that replicas may accept new versions from different clients in parallel.

To execute its second multi-read transaction,
client  \( cl \) requests to read from the $\key_1, \key_2$ keys by sending the message 
\( \Set{\key_1, \key_2} \) to replica $\repl_1$ and awaits a reply.
Upon receiving this message, $\repl_1$ builds a \emph{\( \DEP \)-closed snapshot} (a mapping from $\Set{\key_1, \key_2}$ to values) in two phases as follows. 
First, $\repl_1$ \emph{optimistically reads} the most recent versions for $\key_1$ and $\key_2$,
\emph{one at a time}. 
This process may be interleaved with other writes and synchronisation messages. 
For instance, \cref{fig:cops-request-values} depicts a scenario where \( \repl_1 \):
\begin{enumerate*}
	\item first reads \( (\key_1,\val_1,(\txid_1,\repl_1), \emptyset) \) for $\key_1$ (highlighted); %(highlighted in \cref{fig:cops-request-values})
	\item then receives two synchronisation messages from \( \repl_2 \), 
containing versions \( ( \key_1, \val'_1, (\txid_1,\repl_2),\emptyset ) \) and \( ( \key_2, \val'_2, (t_2,\repl_2),\Set{(\key_1,\txid_1,\repl_2)} ) \); and
	\item finally reads \( ( \key_2, \val'_2, (t_2,\repl_2),\Set{(\key_1,\txid_1,\repl_2)} ) \) for $\key_2$ (highlighted).
\end{enumerate*}
As such, the current snapshot for \( \Set{\key_1,\key_2}\) are not \( \DEP \)-closed: 
\( ( \key_2, \val'_2, (t_2,\repl_2),\Set{(\key_1,\txid_1,\repl_2)} ) \) depends on 
a $\key_1$ version with time-stamp $(\txid_1, \repl_2)$ which is bigger than $(\txid_1, \repl_1)$ for $\key_1$.
To remedy this, after the first phase of optimistic reads,
$\repl_1$ combines (unions) all dependency sets of the versions from the first phase as a \emph{re-fetch set},
and uses it to \emph{re-fetch}
the most recent version of each key with the biggest time-stamp 
from the union of the re-fetch set and the versions from first phase.
For instance, in \cref{fig:cops-re-read-values}, replica $\repl_1$ re-fetches 
the newer version \( ( \key_1, \val'_1, (\txid_1,\repl_2),\emptyset ) \) for \( \key_1 \).
Finally, the snapshot obtained after the second phase 
are sent to the client, and then added to the client context.
For their specific setting, \citet{cops} \emph{informally} argue that the snapshot sent to the client is causally consistent.
By contrast, we \emph{verify} the COPS protocol with our general definition of causal consistency.

%The versions obtained after the second phase is thus 

\subsubsection{COPS Verification} To prove that COPS satisfies causal consistency,
we give an operational semantics for COPS that is faithful to the protocol, which allows fine-graind reads and writes
and show that COPS traces can be refined to traces in our semantics using \( \ET[\CC] \) in three steps:
\begin{enumerate*}
\item every COPS trace can be transferred to a normalised COPS trace, 
in which multiple reads of a transaction are not interleaved by other transactions; and
\item the normalised COPS trace can be refined to traces in our semantics, in which \item each step satisfies \( \ET[\CC] \).
\end{enumerate*}

The COPS Operational semantics
describes transitions over abstract states \( \copsconf \) comprising a set of replicas,
a set of client contexts and a program.
For instance, the COPS trace that produces \cref{fig:cops-request-values,fig:cops-re-read-values} is depicted in \cref{fig:cops-trace} (top), stating that
%A transition in the semantic describes that:
given client \(\cl\) and replica \( \repl_1 \),
\begin{enumerate*}
\item \( \cl \) writes version \( (\opW,\key_1, (\txid_1,\repl_1)) \) to $\repl_1$;
\item \( \cl \) starts a multi-read transaction (\( \mathtt{s} \));
\item \( \cl \) reads  \( (\opR,\key_1, (\txid_1,\repl_1)) \) from $\repl_1$;
\item \( \repl_1 \) receives synchronisation messages (\(\mathtt{sync}\));
\item \( \cl \) enters the second phase of the multi-read transaction (\(\mathtt{p}\));
\item \( \cl \) receives retuned versions of the transaction (\(\mathtt{e}\)).
\end{enumerate*}

\begin{figure*}[!t]
\captionsetup[subfigure]{aboveskip=0pt, belowskip=5pt}

\[
\displaymathfont
\begin{multlined}
\ToCOPSProg{\copsconf_0 |  \cl,\repl_1:(\opW,\key_1,(\txid_1,\repl_1)) 
            -> \copsconf_1 | \cl,\repl_1:\mathtt{s}
            -> \copsconf_2 | \cl,\repl_1:(\opR,\key_1,(\txid_1,\repl_1))
            -> \copsconf_3 | \repl_1:\mathtt{sync}
            -> \copsconf_4 | \cl,\repl_1:(\opR,\key_2,(\txid_2,\repl_2)) 
            -> \!
}
\\[-3pt] \ToCOPSProg{ \copsconf_5 | \cl,\repl_1:\mathtt{p}
            -> \copsconf_6 | \lb
            -> \copsconf_7 | \cl,\repl_1:(\opR,\key_1,(\txid_1,\repl_2))
            -> \copsconf_8 | \lb'
            -> \copsconf_9 | \cl,\repl_1:(\opR,\key_2,(\txid_2,\repl_2))
            -> \copsconf_{10} | \cl,\repl_1:\mathtt{e}
            -> \cdots
            }
\end{multlined}
\normalsize
\]

\vspace*{-10pt}

\hrulefill

\vspace*{-20pt}

\[
\displaymathfont
\begin{multlined}
\ToCOPSProg{ \copsconf'_5 | \lb 
            -> \copsconf'_6 | \lb'  
            -> \copsconf'_7 | \cl,\repl_1:\mathtt{p}
            -> \copsconf'_8 | \cl,\repl_1:(\opR,\key_1,(\txid_1,\repl_2))
            -> \copsconf'_9 | \cl,\repl_1:(\opR,\key_2,(\txid_2,\repl_2))
            -> \copsconf'_{10} | \cl,\repl_1:\mathtt{e}
            -> \cdots
            }
\end{multlined}
\normalsize
\]

\vspace*{-10pt}

\hrulefill

\vspace*{-10pt}

\noindent%
\begin{multline*}
\hspace*{-10pt}%
\scalebox{.65}{%
\begin{tikzpicture}[font=\large]%
%Location x
\node(locx) {$\key_1 \mapsto$};
\draw pic at ([xshift=\tikzkvspace]locx.east) {vlist={versionx}{%
    /\val_0/{(\txid_0,\repl_0)}/\stub
    , /\val_1/{(\txid_1,\repl_1)}/\stub
    , nonvisible/\val'_1/{(\txid_1,\repl_2)}/\stub
}};
%Location y
\path (versionx.east) + (0.55,0) node (locy) {$\key_2 \mapsto$};
\draw pic at ([xshift=\tikzkvspace]locy.east) {vlist={versiony}{%
    /\val_0/{(\txid_0,\repl_0)}/\stub
    , nonvisible/\val_2/{(\txid_2,\repl_2)}/\stub
}};
\end{tikzpicture}%
}
\xrightarrow[\ ]{\scriptsize\begin{array}{@{}l@{}}\cl,\vi:\{ \key_1 \mapsto \{ 0,1,2 \},\key_2 \mapsto \{ 0,1 \} \}, \\ \fp: \Set{\opR(\key_1,\val'_1),\opR(\key_2,\val_2)} \end{array} }
\\[-3pt]
\scalebox{.65}{%
\begin{tikzpicture}[font=\large]%
%Location x
\node(locx) {$\key_1 \mapsto$};
\draw pic at ([xshift=\tikzkvspace]locx.east) {vlist={versionx}{%
    /\val_0/{(\txid_0,\repl_0)}/\stub
    , /\val_1/{(\txid_1,\repl_1)}/\stub
    , /\val'_1/{(\txid_1,\repl_2)}/\stub \cup \{ \txidrd \}
}};
%Location y
\path (versionx.east) + (0.55,0) node (locy) {$\key_2 \mapsto$};
\draw pic at ([xshift=\tikzkvspace]locy.east) {vlist={versiony}{%
    /\val_0/{(\txid_0,\repl_0)}/\stub
    , /\val_2/{(\txid_2,\repl_2)}/\stub \cup \{ \txidrd \}
}};
\end{tikzpicture}%
}
\end{multline*}

\vspace*{-15pt}

\hrulefill

\caption{Top: the COPS trace that preduces \cref{fig:cops-request-values,fig:cops-re-read-values}. 
Middle: the normalised COPS trace.
Bottom: the step encoding the multi-read transaction depicted above, the the kv-store encoding of \cref{fig:initial-cops} (left), and the views (highlighted) encoding of the client contexts before and after the transaction.}
\label{fig:cops-trace}
\label{fig:cops-encode}
\label{fig:encode-mkvs}
\label{fig:encode-view}
\spaceshrink{-5pt}
\end{figure*}



Recall that a multi-read transaction does not executed atomically in the replica,
which is captured by multiple read transitions in the trace.
For example, the \( \lb\) and \( \lb' \) steps in \cref{fig:cops-trace} (top)
interleave the multi-read transaction from \( \cl \).
%Read transitions may be interleaved by other transitions. 
Note that the optimistic reads are not observable to the client and
thus it suffices to show that the reads from second re-fetch phase are atomic.
To show this, we \emph{normalise} the trace as follows. 
For any multi-read transaction, 
we move all reads from the re-fetch phase to the right towards the return step \( \mathtt{e}\),
so that these reads are no longer interleaved by others.
An example of a normalised trace is given in \cref{fig:cops-trace} (middle).
For any multi-read transaction,
the re-fetch phase can only read a version committed before the \( \mathtt{p}\) step.
For example, in \cref{fig:cops-trace} (top)
the multi-read transaction from \( \cl \) can only read versions in \( \copsconf \) and before.
As such, normalising traces does not alter the returned versions of transactions.
After normalisation, transactions in the resulting trace can be seen as if executed atomically. 

We next show that normalised COPS traces can be refined to traces in our semantics.
To do this, we encode the abstract COPS states \(\copsconf \) as configurations
in our operational semantics (\cref{fig:encode-mkvs}). 
%As each replica stores several COPS versions, 
%we can project COPS replicas into kv-stores
We map all COPS replicas to a single kv-store.
The writer of a mapped version is uniquely
determined by the time-stamp of the corresponding COPS version, 
while its reader set 
can be recovered by annotating read-only transactions in the traces.
For example, the COPS state in \cref{fig:cops-after-write-transaction}
can be encoded as the kv-store depicted in \cref{fig:encode-mkvs} (bottom left).
Similarly, as the context of a client $\cl$ identifies the set of COPS versions that $\cl$ sees, 
we can project COPS client contexts to our client views over kv-stores. 
For example, the contexts of \( \cl \) 
before and after committing its second multi-read transaction in \( \prog_{\CodeFont{COPS}} \) 
is encoded as the client view depicted in \cref{fig:encode-view}.

\begin{figure*}[!t]
\captionsetup[subfigure]{aboveskip=0pt, belowskip=5pt}

\begin{tabularx}{\textwidth}{@{} X | c @{}}
\hline\\[-5pt]
\begin{subfigure}{0.53\textwidth}
\begin{centertikz}
%Location x
\node(locx) {$\key_1 \mapsto$};
\draw pic at ([xshift=\tikzkvspace]locx.east) {vlist={versionx}{%
    /$\val_0$/${(\txid_0,\repl_0)}$/$\stub$
    , /$\val_1$/${(\txid_1,\repl_1)}$/$\stub$
    , /$\val'_1$/${(\txid_1,\repl_2)}$/$\stub$
}};

%Location y
\path (versionx.east) + (0.75,0) node (locy) {$\key_2 \mapsto$};
\draw pic at ([xshift=\tikzkvspace]locy.east) {vlist={versiony}{%
    /$\val_0$/${(\txid_0,\repl_0)}$/$\stub$
    , /$\val_2$/${(\txid_2,\repl_2)}$/$\stub$
}};

\end{centertikz}
%\caption{The kv-store encoding of the COPS state in \cref{fig:cops-after-write-transaction}}
\caption{A kv-store encoding of a COPS state}
\vspace{-15pt}%
\label{fig:encode-mkvs}
\end{subfigure}

& 

\begin{subfigure}{0.42\textwidth}
\begin{centertikz}
%Location x
\node(locx) {$\key_1 \mapsto$};
\draw pic at ([xshift=\tikzkvspace]locx.east) {vlist={versionx}{%
    /$\val_0$/${(\txid_0,\repl_0)}$/$\stub$
    , /$\val_1$/${(\txid_1,\repl_1)}$/$\stub$
}};

%Location y
\path (versionx.east) + (0.75,0) node (locy) {$\key_2 \mapsto$};
\draw pic at ([xshift=\tikzkvspace]locy.east) {vlist={versiony}{%
    /$\val_0$/${(\txid_0,\repl_0)}$/$\stub$
}};

\end{centertikz}
\caption{A view encoding of a COPS client context}
\vspace{-15pt}%
\label{fig:encode-view}
\end{subfigure}%
\\ \hline
\end{tabularx}

\caption{COPS encoding}
\label{fig:cops-encode}
\end{figure*}


We finially show that every step in the kv-store trace satisfies \( \et[\CC] \).
Note that existing verification techniques \cite{framework-concur,seebelieve} require examining 
the \emph{entire} sequence of operations of a protocol to show that it implements a consistency model.
By contrast, we only need to look at how the state evolves after a \emph{single} transaction is executed.
In particular, we check the client views over the kv-store.
Intuitively, we observe that when a COPS client $\cl$ executes a transaction then:
\begin{enumerate*} 
\item the $\cl$ context grows, and thus we obtain a more up-to-date view of the associated kv-store; \ie $\vshiftname_{\MR}$ holds;
\item the $\cl$ context always includes the time-stamp of the versions written by itself, and thus the 
corresponding client view always includes the versions $\cl$ has written; \ie $\vshiftname_{\RYW}$ holds; and
\item the $\cl$ context is always closed to the relation \( \DEP \), 
which contains the relation $\SO \cup \WR_{\mkvs}$; \ie $\closed[\mkvs, \vi, \rel[\CC]])$ holds.
\end{enumerate*}
We have thus demonstrated that COPS satisfies causal consistency; 
we refer the reader to \cref{sec:cops} for the full details.

\subsection{Application: Invariant Properties of Transactional Libraries}
\label{sec:robustness}
\label{sec:invariant-client-programs}

A transactional library provides 
a set of transactional operations which can be used by its clients to access the underlying kv-store.
For instance, the counter library on key $\key$ in \cref{sec:overview} is
$\Counter(\key) \defeq \Set{\ctrinc(\key), \ctrread(\key)}$.
Our operational semantics enables us to prove invariant properties of kv-stores, such as:
\begin{enumerate*}
\item the robustness of the single
counter library discussed above against \(\PSI\);
\item the robustness of a multi-counter library and the banking library of \citet{bank-example-wsi}
against our new proposed model \(\WSI\) and any stronger model such as \(\SI\); and
\item the correctness of a lock paradigm under \( \UA \), even though it is not robust.
\end{enumerate*}

%A program $\prog$ is a \emph{client program} of $L$ if the only transactional calls in $\prog$ are to operations of $L$.  
%Let $\CMs(\ET, L)$ denote the set of kv-stores obtained by running $L$ clients under $\ET$.
%A library $L$ is \emph{robust} against an execution test
%$\ET$ if for all clients of $L$, the kv-stores obtained under $\ET$ can also be obtained under $\SER$; 
%\ie $\CMs(\ET, L) \subseteq \CMs(\SER)$.


\subsubsection{Robustness of a Single Counter against $\PSI$}

A library $L$ is \emph{robust} against an execution test $\ET$,
if for all client programs \( \prog \) of $L$, 
the kv-stores \( \kvs \) obtained under $\ET$ can also be obtained under $\SER$;
that is, if \( \kvs {\in} \CMs(\ET) \) then \( \kvs {\in} \CMs(\SER) \).

\spaceshrink{-3pt}
\begin{theorem}
\label{thm:serialisable_nocycle}
A kv-store $\mkvs {\in} \CMs(\SER)$ iff $(\SO {\cup} \WR_{\mkvs} 
{\cup} \WW_{\mkvs} {\cup} \RW_{\mkvs})^{+} \cap \CodeFont{Id} = \emptyset$.% is irreflexive.
\end{theorem}
\spaceshrink{-3pt}

\Cref{thm:serialisable_nocycle} states that any kv-stores in trace under serialisability must contain no cycle, and vice versa.
While previous static-analysis techniques for checking robustness \citep{giovanni_concur16,SIanalysis,laws,sureshConcur}
cannot be extended to support client sessions, 
we give the first robustness proofs that take client sessions into account.
%This theorem is an adaptation of a well-known result on
%dependency graphs~\cite{adya} stating that $\mkvs \in \CMs(\SER)$ iff its associated dependency graph is acyclic. 
Using this theorem, we prove the robustness of a single counter against $\PSI$.
%As discussed in \cref{sec:overview}, the multi-counter library is not robust against $\PSI$. 
%We thus prove the robustness of the multi-counter library and the banking library of \citet{bank-example-wsi} against $\WSI$ instead. 

In the single-counter library, $\Counter(\key)$, 
a client reads from $\key$ by calling $\ctrread(\key)$, and writes to $\key$ by calling $\ctrinc(\key)$ 
which first reads the value of $k$ and subsequently increments it by one.
%Pick an arbitrary key $\key$ and a kv-store $\mkvs \in \CMs(\PSI, \CodeFont{Counter}(\key))$.
As $\PSI$ enforces write conflict freedom (\(\UA\)),
we know that if a transaction $\txid$ updates $\key$ (by calling $\ctrinc(\key)$) and writes version $\ver$ to $\key$, 
then it must have read the version of $\key$ immediately preceding $\ver$:
$\fora{\txid, i > 0}\!\! \txid {=} \wtOf(\mkvs(\key, i)) \implies \txid \in \rsOf(\mkvs(\key, i{-}1))$. 
Moreover, as $\PSI$ enforces monotonic reads ($\MR$),
the order in which clients observe the versions of $\key$ (by calling $\ctrread(\key)$)
is consistent with the order of versions in $\mkvs(\key)$. 
As such, the invariant depicted as below always holds, where  
$\{\txid^{\CodeFont{}}_{i}\}_{i=1}^{n}$ and $\bigcup_{i=0}^{n} \txidset^{\CodeFont{}}_{i}$ 
denote disjoint sets of transactions calling $\ctrinc(\key)$ and $\ctrread(\key)$, respectively:\\[3pt]
%
\noindent
\ 
\begin{tabular}{@{} c @{\quad}| c @{} }
{%
\displaymathfont
\(%
\begin{multlined}
\\[-25pt]
	(0, \txid_{0}, \txidset_{0} \cup \Set{\txid_1}) 
	\lcat (1, \txid_{1}, \txidset_{1} \cup \Set{\txid_2}) \lcat \cdots 
    \\ \lcat (n{-}1, \txid_{n{-}1}, \txidset_{n{-}1} \cup \Set{\txid_n})
	\lcat (n, \txid_n, \txidset_{n}) 
\end{multlined}
\)%
\normalsize
}%
%
& 
%
\scalebox{.6}{%
\begin{tikzpicture}%
%Location x
\node(locx) {};
\draw pic at ([xshift=\tikzkvspace]locx.east) {vlist={versionx}{%
    /0/\txid_0/\Set{\txid_1} \cup \T_0
    , /1/\txid_1/\Set{\txid_2} \cup \T_1
}};
\path(versionx.east)+(0.3,0) node (dots) {$\cdots$};
\draw pic at ([xshift=0.2]dots.east) {vlist= {versionxlast}{
	/n{-}1/\txid_{n{-}1}/\Set{\txid_{n}} \cup \T_{n{-}1}
	, /n/\txid_{n}/\T_{n}
}};
\coordinate  (A) at ([xshift=30,yshift=-14] versionx.north west);
\coordinate  (B) at ([xshift=17,yshift=-9] A.center);
\coordinate  (C) at ([xshift=26,yshift=10] B.center);
\coordinate  (D) at ([xshift=16,yshift=0] C.center);
\coordinate  (E) at ([xshift=21,yshift=-9] D.center);
\coordinate  (F) at ([xshift=79,yshift=9] E.center);
\coordinate  (G) at ([xshift=31,yshift=-9] F.center);
\coordinate  (H) at ([xshift=31,yshift=9] G.center);
\coordinate  (I) at ([xshift=12,yshift=0] H.center);
\coordinate  (J) at ([xshift=0,yshift=-17] I.center);

\path[->,dashed,thick] ([xshift=0,yshift=6]A) edge[bend left=30] ([xshift=-3,yshift=4]B)
([xshift=3,yshift=4]B) edge[bend left=30] ([xshift=2,yshift=6]C)
 ([xshift=0,yshift=6]D) edge[bend left=30] ([xshift=-3,yshift=4]E)
 ([xshift=0,yshift=6]F) edge[bend left=30] ([xshift=-9,yshift=3]G)
([xshift=3,yshift=4]G) edge[bend left=30] ([xshift=2,yshift=6]H)
([xshift=0,yshift=6]I) edge[bend left=60] ([xshift=-2,yshift=5]J);
\end{tikzpicture}%
}

\end{tabular}\\[3pt]
%
%
We define the $\dashrightarrow$ relation depicted above by extending 
$\SO \cup 
\{
	(\txid_i, \txid_j) 
	\mid 
	\txid_j \in \T_i \lor 	
	(\txid_i \in \T_i \land j {=} i {+} 1)
\}$
to a strict total order (\ie an irreflexive and transitive relation). 
Note that $\dashrightarrow$ contains $\SO \cup \WR_{\mkvs} \cup \WW_{\mkvs} \cup \RW_{\mkvs}$ and thus
$(\SO \cup \WR_{\mkvs} \cup \WW_{\mkvs} \cup \RW_{\mkvs})^+$ is irreflexive; 
\ie $\Counter(\key)$ is robust against $\PSI$.

A multi-counter library on a set of keys \( \keyset \) is
\( \mathsf{Counters}(\keyset) \defeq \bigcup_{\key \in \keyset } \Counter(\key) \).
Recall from \cref{sec:overview} that unlike in $\SER$ and $\SI$, under $\PSI$ clients can observe 
the increments on different keys in different orders (see \cref{fig:cp-disallowed}).
As such, the multi-counter library is not robust against $\PSI$. 
The following result shows that 
the multi-counter library is robust against \( \WSI \) and any stronger model such as \( \SI \).

\subsubsection{Robustness Conditions against $\WSI$}
Many libraries \citep{snapshot-isolation-robust-tool,giovanni_concur16,bank-example-wsi} 
yield kv-stores that have a particular pattern that guarantees robustness against \( \WSI \):
%Several libraries in the literature that are robust against $\SI$ 
%\citep{giovanni_concur16,bank-example-wsi} are also robust against $\WSI$.
%The operations of these libraries all yield kv-stores that adhere to a particular pattern:% captured by the following definition.

\spaceshrink{-3pt}
\begin{definition}[\(\WSI\)-safe]
\label{def:main-body-wsi-safe}
    A kv-store \( \mkvs \) is \emph{\(\WSI\)-safe}, if it is 
    reachable by executing a program \( \prog \) % from an initial configuration \( \conf_0 \) 
    under $\WSI$,
    \ie \( \conf_0, \prog \toPROG{}_{\ET_\WSI} (\mkvs, \stub), \stub\), and for all $\txid, \key, i$:

   \spaceshrink{-15pt}
   {\displaymathfont
    \begin{align}
         & \txid \in \rsOf[\mkvs(\key,i)] \land \txid \neq \wtOf[\mkvs(\key,i)]  \implies \fora{\key',j} \txid \neq \wtOf[\mkvs(\key',j)] , \label{equ:main-wsi-safe-read-only} \\
         & \txid \neq \txid_0 \land \txid = \wtOf[\mkvs(\key,i)] \implies \exsts{j} \txid \in \rsOf[\mkvs(\key,j)] , \label{equ:main-wsi-safe-write-must-read} \\
         & \txid \neq \txid_0 \land \txid = \wtOf[\mkvs(\key,i)] \land \exsts{k', j} \txid \in \rsOf[\mkvs(\key',j)] \implies \txid = \wtOf[\mkvs(\key',j)] . \label{equ:main-wsi-safe-all-write}
    \end{align}
    \normalsize
    }
   \spaceshrink{-15pt}

\end{definition}
\spaceshrink{-9pt}

This definition states that a kv-store $\mkvs$ is \(\WSI\)-safe if for each transaction $\txid$: 
\begin{enumerate*} 
    \item if $\txid$ reads from $\key$ without writing to it then $\txid$ must be a read-only transaction \eqref{equ:main-wsi-safe-read-only}; 
    \item if \( \txid \) writes to $\key$, then it must also read from it \eqref{equ:main-wsi-safe-write-must-read}, a property known as \emph{no-blind writes}; and
	\item if \( \txid \) writes to $\key$, then it must also write to all keys it reads \eqref{equ:main-wsi-safe-all-write}.
\end{enumerate*}
It is straightforward to see that the version $j$ read by \( \txid \) in \eqref{equ:main-wsi-safe-write-must-read} must be written immediately before the version $i$ written by \( \txid \), \ie \( i {=} j + 1 \).

\spaceshrink{-5pt}
\begin{theorem}[\( \WSI \) robustness]
 \label{thm:main-wsi-robust}
    Any \(\WSI\)-safe kv-store \( \mkvs \) is robust against \(\WSI\).   
\end{theorem}
\spaceshrink{-5pt}

\noindent From \cref{thm:serialisable_nocycle} it suffices to prove that $(\SO \cup \WR_{\mkvs} \cup \WW_{\mkvs} \cup \RW_{\mkvs})^{+}$ is irreflexive.
We proceed by contradiction and assume that there exists $\txid_1$ such that $\txid_1 \toEDGE{(\SO \cup \WR_{\mkvs} \cup \WW_{\mkvs} \cup \RW_{\mkvs})^{+}} \txid_1$. 
As \( \mkvs \) is reachable under \( \WSI \) and thus \( \CC \), we have:% also reachable under \( \CC \),
%this cycle is of the form:

\spaceshrink{-5pt}
{\displaymathfont
\[
    \txid_1 \toEDGE{\rel^*} \txid_2 \toEDGE{\RW} \txid_3 \toEDGE{\rel^*} \cdots \toEDGE{\rel^*} \txid_{n-2} \toEDGE{\RW} \txid_{n-1} \toEDGE{\rel^*} \txid_n = \txid_1
\]
\normalsize}%
\spaceshrink{-10pt}
 
\noindent{}where \( \rel \defeq \WR \cup \SO \cup \WW \).
From \ref{equ:main-wsi-safe-write-must-read,equ:main-wsi-safe-all-write} we know that 
an \( \RW \) edge starting from a writing transaction can be replaced by a \( \WW \) edge.
Moreover, \( \WW \) edges can be replaced by \( \WR^* \) since \( \kvs \) is reachable under \( \UA \).
We thus have:

\spaceshrink{-5pt}
{\displaymathfont
\[
    \txid_1 \toEDGE{{\rel'}^*} \txid'_2 \toEDGE{\RW} \txid'_3 \toEDGE{{\rel'}^+} \cdots \toEDGE{{\rel'}^+} \txid'_{m-2} \toEDGE{\RW} \txid'_{m-1} \toEDGE{{\rel'}^*} \txid'_m = \txid_n = \txid_1
\]
\normalsize}
\spaceshrink{-10pt}

%
\noindent where \( \rel' \defeq \WR \cup \SO \).
That is, \( \txid_1 \toEDGE{( \rel'; \RW^? )^* } \txid_1 \).
This leads to a contradiction as \( \rel'; \RW^? \subseteq \rel_\CP \) and 
\( \WSI \) requires views to be closed under \( \rel[\CP] \) (\cref{fig:execution_tests}). 

%\subsubsection{Robustness of Multiple Counters against $\WSI$} 
%\label{sec:multi-counter-robust}
%A multi-counter library on a set of keys \( \codeFont{\keyset} \) is: 
%\( \mathsf{Counters(\keyset)} \defeq \bigcup_{\codeFont{\key} \in \codeFont{\keyset} } \Counter(\codeFont{\key}) \).
%We next show that a multi-counter library is \( \WSI \)-safe, and is therefore robust against \( \WSI \) and all stronger models such as \( \SI \).

%\begin{theorem}
  %A multi-counter library \( \mathsf{Counters(\keyset)}  \) is robust against \( \WSI \).
%\end{theorem}

%It is sufficient to show that a kv-store obtained by executing arbitrary transactional calls from the library are \( \WSI \)-safe.
%We proceed by induction on the length of traces.
%Let \( \conf_0 = (\mkvs_0, \vienv_0) \) be an initial configuration and \( \prog_0 \) be a program such that \( \dom(\prog_0) \subseteq \dom(\vienv_0) \).
%The initial kv-store trivially satisfies \eqref{equ:main-wsi-safe-read-only}, \eqref{equ:main-wsi-safe-write-must-read} and \eqref{equ:main-wsi-safe-all-write} above. 
%Let \( \mkvs_i \) be the resulting kv-store after \( i \) steps of execution under \( \WSI \).
%The next transaction \( \txid_{i+1} \) may then be a call to either \( \ctrinc(\pv{\key}) \) or \( \ctrread(\pv{\key}) \).
%If \( \txid_{i+1} \) is a \( \ctrread(\pv{\key}) \) call,
%then the resulting kv-store is: 
%{\displaymathfont
%\[ 
	%\mkvs_{i+1} = \mkvs_{i}\rmto{\pv{\key}}{\mkvs_i(\pv{\key})\rmto{j}{(\val, \txid, \txidset \uplus \Set{\txid_{i+1}})}} 
%\]
%\normalsize}
%%
%for some \( j \) and \( \mkvs_i(\pv{\key},j) {=} (\val,\txid,\txidset)\).
%Since \( \txid_{i+1} \) is a read-only transaction,
%then  \eqref{equ:main-wsi-safe-read-only}, \eqref{equ:main-wsi-safe-write-must-read} and \eqref{equ:main-wsi-safe-all-write} immediately hold.
%On the other hand, if \( \txid_{i+1} \) is an \( \ctrinc(\codeFont{\key}) \) call, 
%then the resulting kv-store is: 
%{\displaymathfont
%\[
    %\mkvs_{i+1} = \mkvs_{i}\rmto{\pv{\key}}{( \mkvs_i(\pv{\key})\rmto{j}{(\val, \txid, \txidset \uplus \Set{\txid_{i+1}})} ) \lcat (\val+1, \txid_{i+1}, \emptyset) } 
%\]
%\normalsize}
%where \( j  {=} \abs{\mkvs_i(\pv{\key})} \) and \( \mkvs_i(\pv{\key},j) {=} (\val,\txid,\txidset)\).
%As \( \txid_{i+1} \) reads the latest version of \( \pv{\key} \) and writes a new version to \( \pv{\key} \),
%the new kv-store \( \mkvs_{i+1} \) satisfies \eqref{equ:main-wsi-safe-read-only}, \eqref{equ:main-wsi-safe-write-must-read} and \eqref{equ:main-wsi-safe-all-write}.

\subsubsection{Robustness of a Banking Library against $\WSI$}

Using \cref{thm:main-wsi-robust},
we are able to prove the robustness of the banking library in \citet{bank-example-wsi} against \( \WSI \).
The banking example is based on relational databases and has three tables: \emph{account}, \emph{saving} and \emph{checking}.
The account table maps customer names to customer IDs (\( \codeFont{Account(\underline{Name}, CID )} \)).
The saving table maps customer IDs to their saving balances (\( \codeFont{Saving(\underline{CID}, Balance )} \)), and
the checking table maps customer IDs to their checking balances (\( \codeFont{Checking(\underline{CID}, Balance )} \)).

For simplicity, we encode the saving and checking tables as a kv-store,
and forgo the account table as it is an immutable lookup table.
We model a customer ID as an integer \( n \in \mathbb{N}\), and assume that balances are integer values. 
We then define the key associated with customer $n$ in the checking table as 
$n_c \defeq 2 n$; 
and define the key associated with $n$ in the saving table as 
$n_s \defeq 2n {+} 1$. 
That is, \( \Keys \defeq \bigcup_{n \in \Nat} \Set{n_c, n_s} \).
Moreover, if \( n \) identifies a customer, \ie $(\stub, n) \in \codeFont{Account(\underline{Name}, CID )}$,
then
\( (n, \valueOf[\mkvs(n_s, \abs{\mkvs(n_s)} - 1)]) \in \codeFont{Saving(\underline{CID}, Balance )} \)
and \( (n, \valueOf[\mkvs(n_s, \abs{\mkvs(n_c)} - 1)]) \in \codeFont{Checking(\underline{CID}, Balance )} \).

The banking library provides five transactional operations.% for accessing the database.

\spaceshrink{-15pt}
{
\displaymathfont
\begin{align*}
    \codeFont{balance(n)} & \defeq
    \begin{transaction}
    \plookup{\pv{x}}{\pv{n}_c}; \ 
    \plookup{\pv{y}}{\pv{n}_s}; \ 
    \passign{\ret}{\pv{x}+\pv{y}}
    \end{transaction} \\
    \codeFont{depositChecking(n,v)} & \defeq
    \begin{transaction}
    \pifs{\pv{v} \geq 0} \ 
    \plookup{\pv{x}}{\pv{n}_c}; \ 
    \pmutate{\pv{n}_c}{\pv{x} + \pv{v}}; \ 
    \pife
    \end{transaction} \\
    \codeFont{transactSaving(n,v)} & \defeq
    \begin{transaction}
    \plookup{\pv{x}}{\pv{n}_s}; \ 
    \pifs{\pv{v} + \pv{x} \geq 0} \ 
    \pmutate{\pv{n}_s}{\pv{x} + \pv{v}}; \ 
    \pife
    \end{transaction} \\
%
	 \codeFont{amalgamate(n,n')} & \defeq
    \begin{transaction}
    \plookup{\pv{x}}{\pv{n}_s}; \ 
    \plookup{\pv{y}}{\pv{n}_c}; \ 
    \plookup{\pv{z}}{\pv{n'}_c}; \\
    \pmutate{\pv{n}_s}{0}; \ 
    \pmutate{\pv{n}_c}{0}; \ 
    \pmutate{\pv{n'}_c}{\pv{x} + \pv{y} + \pv{z}}; 
    \end{transaction} \\
    \codeFont{writeCheck(n,v)} & \defeq
    \begin{transaction}
    \plookup{\pv{x}}{\pv{n}_s}; \ 
    \plookup{\pv{y}}{\pv{n}_c}; \\
    \pifs{\pv{x} + \pv{y} < \pv{v} } \
        \pmutate{\pv{n}_c}{\pv{y} - \pv{v} - 1 }; \ \} \\
    \CodeFont{else} \{ \ 
        \pmutate{\pv{n}_c}{\pv{y} - \pv{v} };  \
    \pife \ 
    \pmutate{\pv{n}_s}{\pv{x}}; 
    \end{transaction}     
\end{align*}
%
\normalsize
}
\spaceshrink{-15pt}

\noindent
The \( \codeFont{balance(n)} \) operation returns the total balance of customer \codeFont{n} in  \( \ret \).
The \( \codeFont{depositChecking(n,v)} \) operation deposits \codeFont{v} to the checking account of customer \codeFont{n} when \codeFont{v} is non-negative,
otherwise the checking account remains unchanged.
%

%\spaceshrink{-15pt}
%{
%\displaymathfont
%\begin{align*}
%    \codeFont{transactSaving(n,v)} & \defeq
%    \begin{transaction}
%    \plookup{\pv{x}}{\pv{n}_s}; \ 
%    \pifs{\pv{v} + \pv{x} \geq 0} \ 
%    \pmutate{\pv{n}_s}{\pv{x} + \pv{v}}; \ 
%    \pife
%    \end{transaction} \\
%%
%	 \codeFont{amalgamate(n,n')} & \defeq
%    \begin{transaction}
%    \plookup{\pv{x}}{\pv{n}_s}; \ 
%    \plookup{\pv{y}}{\pv{n}_c}; \ 
%    \plookup{\pv{z}}{\pv{n'}_c}; \\
%    \pmutate{\pv{n}_s}{0}; \ 
%    \pmutate{\pv{n}_c}{0}; \ 
%    \pmutate{\pv{n'}_c}{\pv{x} + \pv{y} + \pv{z}}; 
%    \end{transaction} \\
%    \codeFont{writeCheck(n,v)} & \defeq
%    \begin{transaction}
%    \plookup{\pv{x}}{\pv{n}_s}; \ 
%    \plookup{\pv{y}}{\pv{n}_c}; \\
%    \pifs{\pv{x} + \pv{y} < \pv{v} } \
%        \pmutate{\pv{n}_c}{\pv{y} - \pv{v} - 1 }; \ \} \\
%    \CodeFont{else} \{ \ 
%        \pmutate{\pv{n}_c}{\pv{y} - \pv{v} };  \
%    \pife \ 
%    \pmutate{\pv{n}_s}{\pv{x}}; 
%    \end{transaction}     
%\end{align*}
%%
%\normalsize
%}
%\spaceshrink{-8pt}

%
\noindent
When $\codeFont{v} \geq 0$,  \( \codeFont{transactSaving(n,v)} \) deposits \codeFont{v} to the saving account of \codeFont{n}.
When $\codeFont{v} < 0$, \( \codeFont{transactSaving(n,v)} \) withdraws \codeFont{v} from the saving account of \codeFont{n} only if the resulting balance is non-negative,
otherwise the saving account remains unchanged.
The \( \codeFont{amalgamate(n,n')} \) operation moves the combined checking and saving funds of costumer \codeFont{n} to the checking account of customer $\codeFont{n}'$.
Lastly, \( \codeFont{writeCheck(n,v)} \) cashes a cheque of customer \codeFont{n} in the amount  \codeFont{v} by deducting \codeFont{v} from its checking account.
If \codeFont{n} does not hold sufficient funds (\ie the combined checking and saving balance is less than \codeFont{v}), customer \codeFont{n} is penalised by deducting one additional pound. 
%
\citet{bank-example-wsi} argue that to make the banking library robust against \( \SI \),
the \( \codeFont{writeCheck(n,v)} \) operation must be strengthened by writing back the balance to the saving account 
(via \(\pmutate{\pv{n}_s}{\pv{x}} \)),
even though the saving balance is unchanged.
The banking library is more complex than the multi-counter library.
Nevertheless, all banking transactions are either read-only or
satisfy the strictly-no-blind-writes property; \ie the banking library is \(\WSI\)-safe.
As such, we can prove its robustness against $\WSI$ and hence \( \SI \).% in a similar fashion to that of the multi-counter library.
%More concretely, given a \( \WSI \)-safe kv-store \( \mkvs\),
%we show that the kv-store resulting from executing a banking operation on \( \mkvs \) remains \WSI-safe. 

%As \(\codeFont{balance(n)} \) is read-only, it immediately satisfies \eqref{equ:main-wsi-safe-read-only}--\eqref{equ:main-wsi-safe-all-write}.
%When $\codeFont{v} \geq 0$, then \(\codeFont{depositChecking(n,v)} \) both reads and writes \( n_c \), and thus preserves  
%\eqref{equ:main-wsi-safe-read-only}--\eqref{equ:main-wsi-safe-all-write}.
%When $\codeFont{v} < 0$, then  \(\codeFont{depositChecking(n,v)} \) leaves the kv-store unchanged and thus \eqref{equ:main-wsi-safe-read-only}--\eqref{equ:main-wsi-safe-all-write} are trivially preserved.
%Lastly, the
%\( \codeFont{transactSaving(n,v)}, \allowbreak \codeFont{amalgamate(n,n')} \) and \( \codeFont{writeCheck(n,v)}\) operations
%always read and write the keys they access, thus satisfying \eqref{equ:main-wsi-safe-read-only}--\eqref{equ:main-wsi-safe-all-write}.


\subsubsection{Invariant of Lock}

The distributed lock library provides \(\plock(k)\), \(\ptrylock(k)\) and \(\punlock(k)\) 
operations on the key \( \key \):

\spaceshrink{-15pt}
{\displaymathfont
\begin{align*}
    \ptrylock(k) & \defeq \begin{transaction}
    \plookup{\pv{x}}{k}; \ \pifs{\pv{x} {=} 0} \;
    \pmutate{k}{ \var(ClientID) }; \; \passign{\var(m)}{\true} \; \} 
    \CodeFont{else} \{ \; \passign{\var(m)}{\false} \; \pife 
    \end{transaction}     
    \\ \plock(k) & \defeq \passign{\var(m)}{\false} ;  \;
    \codeFont{while}(\var(m) {=} \false)  \{\; \ptrylock(\key) \; \}
    \qquad \punlock(k) \defeq \begin{transaction}
        \pmutate{k}{ 0 }
    \end{transaction}     
\end{align*}
\normalsize}
\spaceshrink{-15pt}

\noindent 
The \( \ptrylock \) operation reads the \( \key \) vale.
If it is zero, \ie the lock is available, 
the operation sets it to the client ID and returns \( \true \);
otherwise it leaves it unchanged and returns \( \false \).
The \( \plock \) operation calls \( \ptrylock \) until it successfully acquires the lock.
The \( \punlock \) operation simply set the value to zero.

Consider the program \( \prog_{\CodeFont{LK}} \) where clients \( \cl\) and \( \cl'\) compete the lock \( \key \):

\spaceshrink{-17pt}
{\displaymathfont
\begin{align*}
    \prog_{\CodeFont{LK}} & \defeq ( \cl: (\plock(\key); \ \codeFont{...} ; \ \punlock(\key))^*
    \ \| \ 
    \cl': (\plock(\key); \ \codeFont{...} ; \ \punlock(\key))^* )
    %\tag{\textsc{Lk}}
    %\label{equ:lock-program}
\end{align*}
\normalsize}
\spaceshrink{-17pt}

\noindent
The locking paradigm in \( \prog_{\CodeFont{LK}} \) is correct, in that
only one client can hold the lock at a time,
when executed under serialisability.
Since all the operations are trivially \( \WSI \)-safe,
\( \prog_{\CodeFont{LK}} \) is robust and hence correct under \( \WSI \) 
and any stronger model such as \( \SI \).
However, \( \prog_{\CodeFont{LK}} \) is not robust under \( \UA \), 
because the \( \plock \) operation might read any old value of key \( \key \)
until it reads the up-to-date value of \( \key \) and acquires the lock.
Nevertheless, we show that \( \prog_{\CodeFont{LK}} \) is still correct under \( \UA \).
We capture thus by the following invariant:
for all \( i \), if  \( i > 0\), then:

\spaceshrink{-15pt}
{\displaymathfont
\begin{align}
& \valueOf(\kvs(\key,i)) \neq 0 \iff \valueOf(\kvs(\key,i-1)) = 0
\label{equ:lock-unique-hold}
\\ & \valueOf(\kvs(\key,i)) = 0 \implies \wtOf(\kvs(\key,i)) = \wtOf(\kvs(\key,i-1))
\label{equ:lock-release}
\end{align}
\normalsize}
\spaceshrink{-15pt}

\noindent
It is straightforward to show that under \( \UA \), 
only one client can hold the key \eqref{equ:lock-unique-hold},
and the same client releases the lock \eqref{equ:lock-release}.

%\subsubsection{Clock-SI}
%Clock-SI is a distributed protocol for partitioned key-value
%stores. The set of keys  and 
%their respective versions are partitioned across different machines; versions of keys are time-stamped. 
%Clock-SI supports transactions with arbitrary reads and writes. 
%The protocol assigns a \emph{coordinator} partition to each transaction, which    
%is responsible for contacting other partitions when reading keys remotely or 
%when committing the transaction. Upon starting, a transaction records 
%the current time of the coordinator, which it uses as an upper bound for the time-stamp 
%of versions that the transaction can read. To ensure that the snapshot built by a transaction is consistent, 
%remote reads are delayed until the clock of the contacted partition catches up with the time recorded by 
%the coordinator. To ensure write-conflict-freedom, Clock-SI employs a two-phase-commit protocol, a specialised 
%form of consensus where all the partitions check locally for write-conflict freedom, and the transaction 
%is committed only after all partitions agree to commit.
%The proof that Clock-SI is sound with respect to $\ET_\SI$ is given in
%\cref{sec:clock-si}.
