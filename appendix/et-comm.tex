\ac{This is a different topic than separation logic. These 
proofs should be in their own individual appendix.}

\begin{theorem}
\label{thm:appendix-et-composition-1}
Let $\ET_1, \ET_2$ be two execution tests. If $\ET_1$ is commutative, 
then $\CMs(\ET_1 \cap \ET_2) = \CMs(\ET_1) \cap \CMs(\ET_2)$. 
\end{theorem}
\begin{proof}
Let \( \ET_{12} = \ET_1 \cap \ET_2 \).
\ac{$\ET_{12}$ is awkward. Just go for $\ET$}
Given the definition of the \( \CMs(.) \) function (\cref{def:cm}), we need to prove \( \Confs(\ET_{12}) \subseteq \Confs(\ET_1) \cap \Confs(\ET_2) \) and \( \CMs(\ET_1) \cap \CMs(\ET_2) \subseteq \CMs(\ET_{12}) \).
\ac{Why do you use $\Confs$ for one inclusion and $\CMs$ for the other? It's better to split these two implications in separate lemmas. 
The inclusion $\CMs(\ET_1 \cap \ET_2) \subseteq \CMs(\ET_1) \cap \CMs(\ET_2)$ does not have any commutativity requirement 
on the execution test, so the statement can be more generic in this case.}
\ac{We do not \textbf{need} to prove that \( \Confs(\ET_{12} \subseteq \Confs(\ET_1) \cap \Confs(\ET_2) \). Rather, 
it \textbf{suffices} to prove that \( \Confs(\ET_{12}) \subseteq \Confs(\ET_1) \cap \Confs(\ET_2) \). Then, by \cref{def:cm} 
it follows that \( \CMs(\ET_{12}) \subseteq \CMs(\ET_1) \cap \CMs(\ET_2) \).}
To prove the former, it suffices to prove the follows:
\ac{follows --> following}
\[
\begin{array}{@{}l}
    \fora{n,\conf_0, \conf_1, \dots, \conf_n} \conf_0 \text{ is initial } 
    \land \conf_0 \xrightarrowtriangle{\stub}_{\ET_{12}} \cdots \xrightarrowtriangle{\stub}_{\ET_{12}} \conf_n \implies {} \\
    \quad \conf_0 \xrightarrowtriangle{\stub}_{\ET_{1}} \cdots \xrightarrowtriangle{\stub}_{\ET_{1}} \conf_n \land \conf_0 \xrightarrowtriangle{\stub}_{\ET_{2}} \cdots \xrightarrowtriangle{\stub}_{\ET_{2}} \conf_n 
\end{array}
\]
We prove it by induction on the number \( n \).
\ac{Later you keep saying \emph{it holds}, but it is not clear anymore to what you are referring. Instead, 
label the equation and refer to its tag. E.g. Equation (Eq) holds when $n = 0$. Suppose Equation (Eq) 
holds for some $n = i$, etc.}
It holds when \( n = 0 \). 
\ac{You must always say why something holds. If it is obvious or trivial, then you must say that it is obvious or trivial. 
The sentence \textbf{It holds when \( n = 0 \)} is not a proof by itself.}
Assume that it holds when \( n = i \), let consider \( n = i + 1 \).
For any \( \conf_{i+1} \) induced by \( \ET_{12} \), there exist some client, view and fingerprint such that \( \conf_i \xrightarrowtriangle{\cl, \vi, \opset}_{\ET_{12}} \conf_{i+1} \).
Since \( \ET_{12} = \ET_1 \cap \ET_2 \), it is easy to see that \( \conf_i \xrightarrowtriangle{\cl, \vi, \opset}_{\ET_{1}} \conf_{i+1} \) and \( \conf_i \xrightarrowtriangle{\cl, \vi, \opset}_{\ET_{2}} \conf_{i+1} \).
\ac{Add a separate lemma: If $\conf \xrightarrowtriangle{\cl, \vi, \opset}_{\ET} \conf'$ and $\ET \subseteq \ET'$, 
then $\conf \xrightarrowtriangle{\cl, \vi, \opset}_{\ET'} \conf'$. It should simplify this proof quite a bit. With \emph{it is easy to see} you are actually using this 
lemma.}

To prove the latter, assuming there is no transaction with empty fingerprint, we need to prove the follows:
\ac{What does this mean that there is no transaction with empty fingerprint? 
We agreed that we must be formal, so here you must say something on these lines: 
Suppose that 
\[
\conf_0 \text{ is initial } \land  \land \conf_0 \xrightarrowtriangle{(\cl_0, \vi_0, \opset_0)}_{\ET_{1}} \cdots \xrightarrowtriangle{(\cl_{n-1}, \vi_{n-1}, 
\opset_{n-1})}_{\ET_{1}} \cdots \xrightarrowtriangle{\stub}_{\ET_{1}} \conf_n
    \]
for some $\{\cl_{i}\}_{i=0}^{n-1}$, $\{\vi_{i}\}_{i=0}^{n-1}$ and $\{\opset_{i}\}_{i=0}^{n-1}$ such that, 
for any $i=0,\cdots, n-1$, $\opset_{i} \neq \emptyset$.\\
This statement is much longer than \textbf{Assuming that there is no transaction with empty fingerprint}, but it 
does not leave space to free interpretation: it has a precise, well-defined meaning. From now on, all the sentences 
in the proof should have such level of precision.}
\ac{You cannot just assume that none of the $\opset$ in actions is equivalent to the emptyset. 
Rather, you first need to prove that if $\conf \in \CMs(\ET_1)$, then there exists a sequence of transitions 
\[
\conf_0 \text{ is initial } \land  \land \conf_0 \xrightarrowtriangle{(\cl_0, \vi_0, \opset_0)}_{\ET_{1}} \cdots \xrightarrowtriangle{(\cl_{n-1}, \vi_{n-1}, 
\opset_{n-1})}_{\ET_{1}} \conf_n
    \]
such that $\conf_{n} = \conf$, and for any $i=0,\cdots, n-1$, $\opset_{i} \neq \emptyset$.
To prove this statement, you may want to define the following auxiliary lemma: 
Whenever $\conf \xrightarrowtriangle{(\cl, \vi, \emptyset)} \conf'$, then $\conf = \conf'$. }
\ac{follows --> following}
\[
\begin{array}{@{}l}
    \fora{n,\conf_0, \conf_1,\conf_1',\dots,\conf_{n-1}, \conf_{n-1}', \conf_n} \conf_0 \text{ is initial } \\
    \quad {} \land \conf_0 \xrightarrowtriangle{\stub}_{\ET_{1}} \conf_1 \xrightarrowtriangle{\stub}_{\ET_{1}} \cdots \xrightarrowtriangle{\stub}_{\ET_{1}} \conf_n \land \conf_0 \xrightarrowtriangle{\stub}_{\ET_{2}} \conf_1' \xrightarrowtriangle{\stub}_{\ET_{2}} \cdots \xrightarrowtriangle{\stub}_{\ET_{2}} \conf_n \\
    \qquad {} \implies \conf_0, \conf'_1, \dots, \conf'_{n} \in \Confs(\ET_{1})
\end{array}
\]
It means that for two traces by \( \ET_1 \) and \( \ET_2 \) respectively, 
if the final configurations \( \conf_n \) are the same, 
then all the configurations from trace of \( \ET_2 \) satisfy \( \ET_{1} \).
We prove it by induction on the number \( n \).
It holds when \( n = 0 \) because any initial configuration is included in \( \Confs(\ET_1) \) by the definition of the function (\cref{def:cm}).
\ac{What function?}
Assume that it holds when \( n = i \), let consider \( n = i + 1 \).
Assume a client \( \cl'_{i+1} \), a view \( \vi'_{i+1} \) and a fingerprint \( \opset'_{i+1} \) that transfer the second last configuration \( \conf'_i \) in the trace of \( \ET_2 \) to the final configuration \( \conf_{i+1} \):
\[
    \conf'_i \xrightarrowtriangle{\cl'_{i+1}, \vi'_{i+1}, \opset'_{i+1}}_{\ET_2} \conf_{i+1}
\]
\ac{Again, too much English. Consider writing the exact trace of $\ET_2$ in the text.}
Because the final configuration is the same, 
\ac{The final configuration is the same with respect to who?}
both traces must contain exactly the same transactions where the client, the fingerprint, and the view for all keys appearing in the fingerprint are the same.
\ac{How do you know that? This is the main point of the proof. There is going to be a separate proposition to prove that, and it's not going to be 
easy.}
Given that, there exist a client \( \cl_j \), a view \( \vi_j \) and a fingerprint \( \opset_j \) from the trace of \( \ET_1 \) such that \( \cl'_{i+1} = \cl_{j} \), \( \vi'_{i+1} \approx \vi_j \) and \( \opset'_{i+1} = \opset_{j}\), where the \( \vi_{i+1} \approx \vi_j \) is defined as the follows:
\[
    \fora{\ke} (\stub, \ke, \stub) \in \opset_j \implies \vi_{i+1}(\ke) = \vi_j(\ke)
\]
For brevity we write \( (\cl, \vi, \opset) = (\cl', \vi', \opset') \) for \( \cl = \cl' \), \( \vi \approx \vi' \) and \( \opset = \opset'\).
\ac{This equality does not need to be explained, the fact that $(x,y) = (x', y')$ when $x = x'$, $y = y'$ follows from standard 
set theory.}
Since a client always picks a greater transaction identifier, the session order from one trace is the same as the one of another trace.
\ac{Again, lots of details are missing here. Also, what do you mean here by \textbf{the session order is the same?} The session order 
is fixed once among all the kv-stores, and its embedded inside the structure of transaction identifiers.}
Similarly, because a write fingerprint for a key installs a new version for that key, the order in which transactions write to keys are the same for both traces.
\ac{Again, lots of details missing here: you need a precise statement of what you are proving. And a proof of what you are proving.}
Thus, all transactions after the j-\emph{th} transaction from the trace of \( \ET_1 \) do not conflict with the j-\emph{th} transaction:
\[
    \fora{k, \cl_k, \vi_k, \opset_k} j < k \leq i + 1 \land \conf_{k-1} \xrightarrowtriangle{\cl_k, \vi_k, \opset_k}_{\ET_{1}} \conf_k \implies \cl_k \neq \cl_j \land \nexists{\ke} \ldotp (\otW, \ke, \stub) \in \opset_k \cap \opset_j
\]
By the commutative of \( \ET_1 \), we can move the j-\emph{th} transaction to the end of the trace without changing the final configuration:
\[
\begin{array}{@{}l}
    \exsts{\conf''_{j}, \dots \conf''_i}  \\
    \quad \conf_{j-1} \xrightarrowtriangle{\cl_{j+1}, \vi_{j+1}, \opset_{j+1}}_{\ET_{1}} \conf''_{j} \xrightarrowtriangle{\cl_{j+2}, \vi_{j+2}, \opset_{j+2}}_{\ET_{1}} \\
    \qqquad \cdots \xrightarrowtriangle{\cl_{i+1}, \vi_{i+1}, \opset_{i+1}}_{\ET_{1}} \conf''_i \xrightarrowtriangle{\cl_{j}, \vi_{j}, \opset_{j}}_{\ET_{1}} \conf_{i+1}
\end{array}
\]
Because \( (\cl_{j}, \vi_{j}, \opset_{j}) = (\cl'_{i+1}, \vi'_{i+1}, \opset'_{i+1}) \), 
we know that the second last (the i-\emph{th}) configuration from the new trace of \( \ET_1 \) equals to the second last from the trace of \( \ET_2 \), \ie \( \conf''_i = \conf'_i \).
By the \ih, we know \( \conf_0, \conf'_1, \dots, \conf'_i \in \Confs(\ET_{1}) \).
Last, we know \( \conf''_i \xrightarrowtriangle{\cl_{j}, \vi_{j}, \opset_{j}}_{\ET_1} \conf_{i+1} \), \( (\cl_{j}, \vi_{j}, \opset_{j}) = (\cl'_{i+1}, \vi'_{i+1}, \opset'_{i+1}) \) and \( \conf''_i = \conf'_i \), so we know \( \conf_{i+1} \in \Confs(\ET_1)\).
\end{proof}

We define function that returns the fingerprint associate with a transaction identifier:
\[
    \begin{rclarray}
        \mkvs(\txid) & \defeq & \Setcon{(\otW, \ke, \val)}{\exsts{i} \mkvs(\ke)(i) = (\val, \txid, \stub)} \cup  \Setcon{(\otR, \ke, \val)}{\exsts{i,\txidset} \mkvs(\ke)(i) = (\val, \stub, \txidset) \land \txid \in \txidset}
    \end{rclarray}
\]

\begin{lemma}
    \label{lem:mono-fingerprint}
    \[
        \fora{\ET,\conf,\conf',\txid,\f} \conf\projection{1}(\txid) = \f \land \conf \toET{\stub}{\ET} \conf' \implies \conf'\projection{1}(\txid) = \f
    \]
\end{lemma}

\begin{lemma}
\[
\begin{array}{@{}l}
    \fora{n,\conf_0, \conf_1, \dots, \conf_n, \cl_1, \dots, \cl_n, \vi_1, \dots, \vi_n, \f_1, \dots, \f_n } \\
    \fora{m,\conf'_1, \dots, \conf'_m, \cl'_1, \dots, \cl'_m, \vi'_1, \dots, \vi'_m, \f'_1 \dots, \f'_m } \\
    \quad \conf_0 \xrightarrowtriangle{\cl_1, \vi_1, \f_1}_{\ET_{1}} \cdots \xrightarrowtriangle{\cl_n, \vi_n, \f_n}_{\ET_{1}} \conf_n \land \conf_0 \xrightarrowtriangle{\cl'_1, \vi'_1, \f'_1}_{\ET_{2}} \cdots \xrightarrowtriangle{\cl'_m, \vi'_m, \f'_m}_{\ET_{2}} \conf'_m \\
    \quad \land \bigwedge\limits_{ 0 < k \leq n } \f_k \neq \unitO \land \bigwedge\limits_{ 0 < l \leq m } \f'_l \neq \unitO \land \conf_n = \conf'_m \implies \\
    \qquad \fora{i: 0 < i \leq n} \exsts{j: 0 < j \leq m} \cl_i = \cl'_j \land \f_i = \f'_j \land ( \fora{\ke} (\stub, \ke, \stub) \in \f_i \implies \vi_i(\ke) = \vi'_j(\ke) )
\end{array}
\]
\end{lemma} 
\begin{proof}
    Let \(\conf_n = \mkvs_n,\viewFun_n \) and \(\conf'_m = \mkvs'_m,\viewFun'_m \).
    Assume a step \( \conf_i \toET{\cl, \vi ,\f }{\ET_1} \conf_{i+1} \) where the transaction identifier is \( \txid_\cl^{n}\).
    It means \( \mkvs_n(\txid_\cl^{n}) = \f \) by \cref{lem:mono-fingerprint}.
    By \( \mkvs_n = \mkvs'_m \), we also know that:
    \[
        \fora{\txid, \f} \mkvs_n(\txid) = \f \iff \mkvs'_m(\txid) = \f
    \]
    so it must exists a step from the trace of \( \ET_2 \) such that \( \conf'_j \toET{\cl, \vi ,\f }{\ET_1} \conf'_{j+1} \).
\end{proof}

\begin{lemma}
    \label{lem:kv-same-number}
    \[
    \fora{\conf, \conf' ,\cl, \vi, \f, \ET}
    \conf \xrightarrowtriangle{\cl, \vi, \f}_{\ET}  \conf' \land \f \neq \unitO \implies \max{}_\cl(\conf) < \max{}_\cl(\conf')
    \]
    where
    \[
        \begin{rclarray}
            \max_\cl((\mkvs, \viewFun)) & \defeq & \max\Setcon{\txid^{n}_\cl}{\txid^{n}_\cl \text{ appear in } \mkvs} \\
        \end{rclarray}
    \]
\end{lemma}
\begin{proof}
    Assume  \( (\mkvs, \viewFun) \xrightarrowtriangle{\cl, \vi, \f}_{\ET} (\mkvs', \viewFun') \).
    By the definition of \( \toET{\stub}{\ET}\), we know \( \mkvs' \in \updKV{\mkvs, \vi, \cl, \f} \).
    The \( \updKV{\mkvs, \vi, \cl, \f} \) picks a fresh transaction identifier \( \txid_\cl^{m} \) that is greater than any transaction identifiers \( \txid_\cl^{n} \) in \( \mkvs \) via \( \nextTxId \) function, \ie \( m > n \).
    Since the fingerprint is not empty, the new identifier appear in \( \mkvs' \), therefore we have the proof.
\end{proof}

\begin{proposition}
\label{thm:appendix-et-composition-2}
\label{prop:appendix-et-composition-2}
if $\ET_1, \ET_2$ are commutative, then $\ET_1 \cap \ET_2$ is commutative.
\end{proposition}
\begin{proof}
Let \( \ET_{12} = \ET_1 \cap \ET_2 \).
Assume \(\conf_1, \conf_2, \conf_3, \cl, \cl', \vi, \vi', \opset, \opset' \) such that:
\[
    \conf_1 \xrightarrowtriangle{\cl, \vi, \opset}_{\ET_{12}} \conf_2 \xrightarrowtriangle{\cl', \vi', \opset'}_{\ET_{12}} \conf_3
\]
Therefore, we have:
\[
    \conf_1 \xrightarrowtriangle{\cl, \vi, \opset}_{\ET_{1}} \conf_2 \xrightarrowtriangle{\cl', \vi', \opset'}_{\ET_{1}} \conf_3 \land 
    \conf_1 \xrightarrowtriangle{\cl, \vi, \opset}_{\ET_{2}} \conf_2 \xrightarrowtriangle{\cl', \vi', \opset'}_{\ET_{2}} \conf_3
\]
Because \( ET_1 \)  and \( \ET_2 \) are commutative, there exists a configuration \( \conf'_2 \) such that:
\[
    \conf_1 \xrightarrowtriangle{\cl', \vi', \opset'}_{\ET_{1}} \conf'_2 \xrightarrowtriangle{\cl, \vi, \opset}_{\ET_{1}} \conf_3 \land 
    \conf_1 \xrightarrowtriangle{\cl', \vi', \opset'}_{\ET_{2}} \conf'_2 \xrightarrowtriangle{\cl, \vi, \opset}_{\ET_{2}} \conf_3
\]
so we have the proof that: 
\[
    \conf_1 \xrightarrowtriangle{\cl', \vi', \opset'}_{\ET_{12}} \conf'_2 \xrightarrowtriangle{\cl, \vi, \opset}_{\ET_{12}} \conf_3
\]
\end{proof}

\begin{lemma}
    \label{lem:mr-comm}
    \(\MRd\) is commutative.
\end{lemma}
\begin{proof}
    Let assume:
    \[
        \mkvs_0, \viewFun_0 \toET{\cl, \vi, \opset}{\MRd} \mkvs_1, \viewFun_1 \toET{\cl', \vi', \opset'}{\MRd} \mkvs_2, \viewFun_2 
    \]
    where \( \cl \neq \cl' \), \( \fora{ \ke } (\otW, \ke, \stub) \notin \opset \cap \opset' \) and \( \vi, \vi' \in \Views(\mkvs_0)\).
    Given the specification of \(\MRd\) (\cref{fig:execution-tests}), we know \( \viewFun_1 = \viewFun_0\rmto{\cl}{\vi''}\) for some \( \vi'' \) such that \( \vi \sqsubseteq \vi'' \).
    Similarly \( \viewFun_2 = \viewFun_1\rmto{\cl'}{\vi'''} \) for some \( \vi''' \) such that \(  \vi' \sqsubseteq \vi''' \). 
    Given that \( \cl \neq \cl' \), we can pick \( \mkvs'_1 \) and \( \viewFun'_1 \) such that \( \mkvs'_1 \in \updKV{\mkvs_0,\vi',\cl',\f'} \) and \( \viewFun'_1 = \viewFun_0\rmto{\cl'}{\vi'''} \), so we have:
    \[
        \mkvs_0, \viewFun_0 \toET{\cl', \vi', \opset'}{\MRd} \mkvs'_1, \viewFun'_1 \toET{\cl, \vi, \opset}{\MRd} \mkvs_2, \viewFun_2 
    \]
\end{proof}                                                                                    

\begin{lemma}
    \label{lem:mw-comm}
    \(\MW\) is commutative.
\end{lemma}
\begin{proof}
    Let assume:
    \[
        \mkvs_0, \viewFun_0 \toET{\cl, \vi, \opset}{\MW} \mkvs_1, \viewFun_1 \toET{\cl', \vi', \opset'}{\MW} \mkvs_2, \viewFun_2 
    \]
    where \( \cl \neq \cl' \), \( \fora{ \ke } (\otW, \ke, \stub) \notin \opset \cap \opset' \) and \( \vi, \vi' \in \Views(\mkvs_0)\).
    Because the client \( \cl' \) can commit the fingerprint \( \f' \) under \( \vi' \) to the kv-store \( \mkvs_1 \), it can also commit to \( \mkvs_0 \):
    \[
        \begin{array}{@{}l}
            \vi' \in \Views(\mkvs_0) \land \MW \vdash (\mkvs_1, \vi') \csat \f', \viewFun_2(\cl') \implies  \MW \vdash (\mkvs_0, \vi') \csat \f', \viewFun_2(\cl')
        \end{array}
    \]
    Assuming \( \mkvs'_1 \) and \( \viewFun'_1 \) such that \( \mkvs'_1 \in \updKV{\mkvs_0,\vi',\cl',\f'} \) and \( \viewFun'_1 = \viewFun_0\rmto{\cl'}{\viewFun_2(\vi')} \), we need to prove the specification for \( \MW \) (\cref{fig:execution-tests}):
    \[
        \fora{j,i, \ke, \ke'} j \leq \vi(\ke) \wedge \WTx(\hh'_1(\ke', i)) \xrightarrow{\PO ?} \WTx(\hh'_1(\ke, j)) \implies i \leq \vi'(\ke')  \\
    \]
    For any key written by \( \cl' \), there is a new version of the key in \( \mkvs'_1 \).
    Given \( \vi \in \Views(\mkvs_0)\), those new versions installed by \( \cl' \) do not affect \( \vi \):
    \[
        \fora{j,i, \ke} j \leq \vi(\ke) \land \hh'_1(\ke, j)\isdef \implies \hh_0(\ke, j)\isdef
    \]
    so we have:
    \[
        \begin{array}{@{}l}
            \MW \vdash (\mkvs_0, \vi) \csat \f, \viewFun_2(\cl) \implies  \MW \vdash (\mkvs'_1, \vi) \csat \f, \viewFun_2(\cl)
        \end{array}
    \]
\end{proof}


\begin{lemma}
    \label{lem:ryw-comm}
    \(\RYW\) is commutative.
\end{lemma}
\begin{proof}
    Let assume:
    \[
        \mkvs_0, \viewFun_0 \toET{\cl, \vi, \opset}{\RYW} \mkvs_1, \viewFun_1 \toET{\cl', \vi', \opset'}{\RYW} \mkvs_2, \viewFun_2 
    \]
    where \( \cl \neq \cl' \), \( \fora{ \ke } (\otW, \ke, \stub) \notin \opset \cap \opset' \) and \( \vi, \vi' \in \Views(\mkvs_0)\).
    %Assume \( \mkvs'_1 \) and \( \viewFun'_1 \) such that \( \mkvs'_1 \in \updKV{\mkvs_0,\vi',\cl',\f'} \) and \( \viewFun'_1 = \viewFun_0\rmto{\cl'}{\viewFun_2(\vi')} \).
    Given \( \fora{ \ke } (\otW, \ke, \stub) \notin \opset \cap \opset' \) and the specification for \( \RYW \) (\cref{fig:execution-tests}), we have:
    \[
        \fora{\ke} (\otW, \ke, \stub) \in \f \implies \viewFun_1(\cl)(\ke) = \viewFun_2(\cl)(\ke) = \lvert \mkvs_1(\ke) \rvert - 1 = \lvert \mkvs_2(\ke) \rvert - 1
    \]
    We can pick \( \mkvs'_1 \) and \( \viewFun'_1 \) such that \( \mkvs'_1 \in \updKV{\mkvs_0,\vi',\cl',\f'} \) and \( \viewFun'_1 = \viewFun_0\rmto{\cl'}{\viewFun_2(\cl')} \), so we have:
    \[
        \mkvs_0, \viewFun_0 \toET{\cl', \vi', \opset'}{\RYW} \mkvs'_1, \viewFun'_1 \toET{\cl, \vi, \opset}{\RYW} \mkvs_2, \viewFun_2 
    \]
\end{proof}

\begin{lemma}
    \label{lem:wfr-comm}
    \(\WFR\) is commutative.
\end{lemma}
\begin{proof}
    Let assume:
    \[
        \mkvs_0, \viewFun_0 \toET{\cl, \vi, \opset}{\WFR} \mkvs_1, \viewFun_1 \toET{\cl', \vi', \opset'}{\WFR} \mkvs_2, \viewFun_2 
    \]
    where \( \cl \neq \cl' \), \( \fora{ \ke } (\otW, \ke, \stub) \notin \opset \cap \opset' \) and \( \vi, \vi' \in \Views(\mkvs_0)\).
    Because the client \( \cl' \) can commit the fingerprint \( \f' \) under \( \vi' \) to the kv-store \( \mkvs_1 \), it can also commit to \( \mkvs_0 \):
    \[
        \begin{array}{@{}l}
            \vi' \in \Views(\mkvs_0) \land \WFR \vdash (\mkvs_1, \vi') \csat \f', \viewFun_2(\cl') \implies  \WFR \vdash (\mkvs_0, \vi') \csat \f', \viewFun_2(\cl')
        \end{array}
    \]
    Assume \( \mkvs'_1 \) and \( \viewFun'_1 \) such that \( \mkvs'_1 \in \updKV{\mkvs_0,\vi',\cl',\f'} \) and \( \viewFun'_1 = \viewFun_0\rmto{\cl'}{\viewFun_2(\vi')} \).
    For any key written by \( \cl' \), there is a new version of the key in \( \mkvs'_1 \).
    Given \( \vi \in \Views(\mkvs_0)\), those new versions installed by \( \cl' \) do not affect \( \vi \):
    \[
        \fora{j,i, \ke} j \leq \vi(\ke) \land \hh'_1(\ke, j)\isdef \implies \hh_0(\ke, j)\isdef
    \]
    so we have:
    \[
        \begin{array}{@{}l}
            \WFR \vdash (\mkvs_0, \vi) \csat \f, \viewFun_2(\cl) \implies \WFR \vdash (\mkvs'_1, \vi) \csat \f, \viewFun_2(\cl)
        \end{array}
    \]
\end{proof}

\begin{lemma}
    \label{lem:ua-comm}
    \(\UA\) is commutative.
\end{lemma}
\begin{proof}
    Let assume:
    \[
        \mkvs_0, \viewFun_0 \toET{\cl, \vi, \opset}{\WFR} \mkvs_1, \viewFun_1 \toET{\cl', \vi', \opset'}{\WFR} \mkvs_2, \viewFun_2 
    \]
    where \( \cl \neq \cl' \), \( \fora{ \ke } (\otW, \ke, \stub) \notin \opset \cap \opset' \) and \( \vi, \vi' \in \Views(\mkvs_0)\).
    Given the specification for \( \UA \) (\cref{fig:execution-tests}) we have:
    \[
        \begin{array}{@{}l}
            \vi' \in \Views(\mkvs_0) \land \fora{ \ke } (\otW, \ke, \stub) \notin \opset \cap \opset' \land \\ 
            \quad \UA \vdash (\mkvs_1, \vi') \csat \f', \viewFun_2(\cl') \implies  \UA \vdash (\mkvs_0, \vi') \csat \f', \viewFun_2(\cl')
        \end{array}
    \]
    Assume \( \mkvs'_1 \) and \( \viewFun'_1 \) such that \( \mkvs'_1 \in \updKV{\mkvs_0,\vi',\cl',\f'} \) and \( \viewFun'_1 = \viewFun_0\rmto{\cl'}{\viewFun_2(\vi')} \).
    Because transactions write different keys, It is easy to see that \( \UA \vdash (\mkvs'_1, \vi) \csat \f, \viewFun_2(\cl) \).
\end{proof}

\begin{proposition}
    \( \CC \) and \( \PSI \) are commutative.
\end{proposition}
\begin{proof}
    Because \( \CC = \MR \cap \MW \cap \RYW \cap \WFR \) and all four are commutative (\cref{lem:mr-comm,lem:mw-comm,lem:ryw-comm,lem:wfr-comm}), so \( \CC \) is commutative by the \cref{prop:appendix-et-composition-2}.
    Similarly, since \( \PSI = \CC \cap \UA \) and \( \UA \) is commutative (\cref{lem:ua-comm}), so \( \PSI \) is commutative.
\end{proof}

\( \CP \) is not commutative because the view after update must be up-to-date.

\begin{lemma}
    \label{lem:ser-comm}
    \( \SER \) is commutative.
\end{lemma}
\begin{proof}
    Given the specification for \( \SER \) (\cref{fig:execution-tests}), the hypothesis that \( \vi, \vi' \in \Views(\mkvs_0) \) never holds.
\end{proof}

