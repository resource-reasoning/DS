\section{Formal Model}
\label{sec:model}
\label{sec:semantics}
\ac{I am going to split this section into two, for the sake of clarity. 
As of now, the structure that I have in mind for the paper is as follows: 
Section 2 (i.e. this section) contains the notion of key-value store, view, snapshot, 
execution tests, and consistency model  -i.e. sets of key-value stores- 
induced by an execution test). Then Section 3 discusses 
the semantics of programs, and possibly we state the result 
of adequateness. In Section 4 we discuss the equivalence of consistency 
models specifications with respect to axiomatic ones. In Section 5 
we present the logic and litmus tests examples.}

\begin{wrapfigure}[7]{r}{0.33\textwidth}
\vspace{-10pt}
\begin{verbatim}
interface Transaction {
    Start(); 
    Read(Key k);
    Write(Key k, Value v); 
    Commit();    }
\end{verbatim}
\vspace{-10pt}
\caption{Example of Transaction API.}
\label{fig:api}
\end{wrapfigure}
We focus on an abstract computational model where multiple client programs can access and update keys in a key-value store using atomic transactions. 
In general, clients are provided with a simple \textit{API} such as the one depicted in \cref{fig:api} \cite{gdur,physicsnmsi,clockSI}\footnote{It is 
often the case that key-value stores provide a mechanism to wrap more transactions inside a session, and give 
provide appropriate APIs to handle sessions. For the sake of simplicity, in this paper we assume that each client executes transactions 
within a single session.}, while both the implementation details and the system architecture are hidden from clients. 
Because (distributed) key-value stores only give weak consistency guarantees of the data to their clients, the latter are 
not ensured to read the most up-to-date version of a key.
%In an ideal world, when executing a transaction clients would read the most up-to-date version of a key. In a distributed setting 
%this approach, known as (strict) serialisability, would require a continuous synchronisation between the different components of 
%the system, which impacts performance and limits scalability. To this end, the database only provides weak consistency model
%\ac{This sentence should probably be in the introduction.}

Following these intuitions, we model a key-value stores, or \emph{kv-store}, as a centralised unit where multiple versions 
are stored for each key (\cref{sec:mkvs-view}). Versions consist of a value and the meta-data of the transactions that wrote and 
read such a version. In practical, distributed systems, the meta-data is usually encoded using either timestamps 
\cite{physicsnmsi,clockSI} or vector clocks \cite{gdur}. We focus on key-value stores whose transactions 
enjoy \emph{atomic visibility}, meaning that \textbf{(i)} transactions read their data from an atomic 
snapshot of the key-value store, and \textbf{(ii)} a transaction can observe either none or all 
of the updates performed by another transaction. In other words, a transaction only reads (writes) at most 
one version for each key.
Because clients may observe potentially out-of-date versions of the system, we introduce the notion of \emph{views}. 
Intuitively, a view records the version of each key that a client observes at a given time. We use views 
to determine the snapshot taken by transactions executed by clients.

A consistency model is a contract between the key-value store and its clients. We distinguish 
between \emph{client-centric} consistency models \cite{terry1994session}, 
which impose constraints on the observations and updates made by a single client, 
and \emph{data-centric} consistency models \cite{framework-concur}, which impose constraints 
on the structure of the key value store.
To specify weak consistency models, 
we introduce the notion of \emph{execution tests} (\cref{sec:execution.tests}). An execution test
 specifies when a client is allowed to execute a transaction carrying a given 
set of read and write operations, or \emph{fingerprint}. Therefore, an execution 
test constrains how the state of the key-value store may evolve;
by considering all the possible evolutions of the key-value store under said execution test, 
we determine a consistency model. For example, an execution test for (strict) serialisability 
requires that a transaction can be executed by a client only if it observes the most up-to-date 
version for each key. 
We give several examples of execution tests that capture both client-centric and 
data-centric consistency models. 

The idea of specifying consistency models using execution tests has been 
already proposed in \cite{seebelieve}; however, their notion of execution 
test is intrinsically more complex than ours: to determine 
whether a transaction can commit, the total order in which all past transactions 
have committed must be known. This knowledge is not needed in our setting.
\ac{Some line about how for this reason, our notion of execution test is closer 
to the implementation of commit tests in real database implementations.}
%the author require 
%the knowledge of the total order in which all past transactions have been 
%executed, to determine whether a new transaction can commit.

%%\ac{Can't cite documentations of real databases, as they usually have a much more complicated API.}
%
%%Transactions in our model execute atomically, though they have different effects on the key-value stores depending on their associated \emph{consistency model}.
%%A consistency model controls how the key-value store evolves.
%%A common model is \emph{serialisability}, where transactions appear to execute one after another in a sequential order.
%%This notion of sequential execution is however not necessary for many weaker models. As such, upon commencing execution, a transaction may not observe the most up-to-date values for keys. 
%
%To address this, we 
%first model 
%the state of the system using \emph{multi-version key-value stores (MKVSs)} (\cref{sec:mkvs-view}). 
%An MKVS keeps track of all versions (values) written for keys, as well as the information about the transactions that read and wrote such versions. 
%To model the potential out-of-date observation, we introduce \emph{views}.
%A view decides the observable versions of keys for a client.
%Therefore, in order to execute a transaction, the client first takes \emph{a snapshot} of the system with the view, executes the transaction locally with respect to its snapshot (\cref{sec:trans-semantics}), and afterwards commits the effect of the transaction if the change is allowed by the underlying consistency model (\cref{sec:prog-semantics}).


%We first introduce MKVSs and views (\secref{sec:mkvs-view}) and.
%We starts with the syntax of programs followed by the semantics of transaction.
%Finally, we will give the semantics for the entire programs.


\subsection{Multi-version Key-value Stores and Views}
\label{sec:mkvs-view}

\subsubsection{Multi-version Key-value Stores} 
We assume a countably infinite set of \emph{client identifiers} $\Clients \defeq \Set{\cl, \cl',\cdots}$. 
We define the set of \emph{transaction identifiers} 
$\TxID \defeq  \Set{\txid_{0}} \uplus \Set{ \txid_{\cl}^{n} \mid \cl \in \Clients \wedge n \geq 0 }$, where 
 $\txid_0$ denotes a designated transaction used for initialisation, 
 and for each $n \in \mathbb{N}$, $\txid_{\cl}^{n}$ identifies a transaction 
 committed by client $\cl$.
% the $n$\textsuperscript{th} transaction of client $\cl$. 
Elements of $\TxID$ are ranged over by $\txid, \txid', \cdots$, 
while subsets of $\TxID$ are ranged over by $\txidset, \txidset', \cdots$. 
We define $\TxID_{0} \defeq \TxID \setminus \{ \txid_0\}$.
As we will see, we assume that each client is bound to a single session, and 
we use the superscript $n$, in a transaction identifier $\txid_{\cl}^{n}$, 
to embed the information about the session order $\PO$ in which clients execute 
transactions:  
%
%The structure of $\TxID$  
%embeds the transaction execution order for each client, or the \emph{session order} $\PO$. 
%More concretely, 
$\PO \defeq \Set{ (\txid, \txid') \mid \exsts{ \cl, n,m } \txid = \txid_{\cl}^{n} \wedge \txid' = \txid_{\cl}^{m} \wedge n < m}$.
%As such, $(\txid, \txid') \in \PO$ denotes that 
%client $\cl$ executes $\txid$ before $\txid'$.
For readability, we often write  $\txid \xrightarrow{\PO} \txid'$ for $(\txid, \txid') \in \PO$.

%Given a set $X$, 
%%then $\powerset{X}$ denotes 
%%the powerset of $X$,
%we write $X^{\ast}$ for the free monoid induced by $X$.
%We next define the notion of \emph{multi-version key-value stores}.


\begin{definition}[Multi-version Key-value Stores]
\label{def:his_heap}
\label{def:mkvs}
Assume a countably infinite set of \emph{keys} $\Keys = \Set{\ke, \ke', \cdots}$, 
and a set of \emph{values} $\Val = \{\val, \val', \cdots\}$.

A \emph{version} is a triple $\ver \in \Versions \defeq \Val \times \TxID \times \powerset{\TxID_{0}}$. 
%The set of versions is denoted by $$.
A \emph{key-value store} is a mapping $\hh: \Keys \rightarrow \Versions^{\ast}$, 
where we recall that $\Versions^{\ast}$ is the free monoid generated by $\Versions$.
%\ac{ The superscript fin over the $\rightharpoonup$ needs to be fixed. You may want to look at the package extpfeil.}
\end{definition}

%For simplicity, we instantiate the set of values as  $\Val \eqdef \Nat \uplus \Keys$,
Among the elements of $\Val$, we distinguish a default value $\val_0 \in \Val$. 
A \emph{version} $\ver = (\val, \txid, \txidset)$ comprises a value $\val$,
and the meta-data of the transactions that accessed it.
Specifically, the \emph{writer} $\txid$ identifies the transaction that wrote version $\ver$, 
and the \emph{readers} $\txidset$ denote the set of transactions that read from $\ver$.
Given a version $\ver = (\val, \txid, \txidset)$, we define $\valueOf(\ver) \defeq \val$,
$\WTx(\ver) \defeq \txid$ and $\RTx(\ver) \defeq \txidset$.
Lists of versions (\ie elements of $\Versions^{\ast}$) are ranged over by $\vilist, \vilist',\cdots$.

%A \emph{multi-version key-value store}, or \emph{kv-store}, 
%is a mapping from keys to lists of versions. 
Given kv-store $\hh$, key $\ke$ and index $i \geq 0$, 
we write $\hh(\ke, i)$ for the $i$-th version (starting from $0$) of $\ke$.
That is, if $\hh(\ke) = \ver_0 \cdots\ver_{n}$, then $\hh(\ke, i) \defeq \ver_{i}$ for $i \leq n$; 
and it is undefined otherwise. 
We write $\lvert \hh(\ke) \rvert$ for the length of $\hh(\ke)$. 

We focus on key-value stores whose consistency model enforces the \emph{atomic visibility} of transactions~\cite{framework-concur}. 
This amounts to requiring that transaction reads at most one version of each key, and similarly 
it writes at most one version for each key. From the point of view of key-value stores, 
these conditions amount to require that \textbf{(i)}
$\fora{\ke i, j} (o \leq i, j \abs{ \hh(\ke) } \land \RTx(\hh(\ke, i)) \cap \RTx(\hh(\ke, j)) \neq \emptyset ) \implies i = j$, 
\textbf{(ii)}
$\fora{\ke, i, j} (0 \leq i,j < \abs{ \hh(\ke) } \wedge \WTx(\hh(\ke, i)) = \WTx(\hh(\ke, j)) ) \implies i = j$. 
We also assume that the list of version for each key has an initial version carrying a default value $\val_0$, 
written by the designated initialisation transaction $\txid_0$: \textbf{(iii)} $\fora{\ke} \hh(\ke, 0) = (\val_0, \txid_0, \stub)$.
Finally, we assume that the state of a key-value store is consistent with 
the session order of clients; a client cannot read a version of a key that has 
been installed by a future transactions within the same session, and 
the order in which versions are installed by a single client must agree 
with its session order: \textbf{(iv)}
$\fora{ \ke, \cl, i,j, n, m} 0 \leq i < j < \abs{\hh(\ke)} 
    \land \txid_{\cl}^{n} = \WTx(\hh(\ke,i)) {} \wedge \txid_{\cl}^{m} \in \Set{\WTx(\hh(\ke,j))} \cup \RTx(\hh(\ke, i))
    \implies n < m $.
%
%Formally, we require the following well-formedness requirement from key-value stores: 
%
%\begin{enumerate}%[label=(\roman*)]
%\item\label{kv:wf.init} 
%    $\hh(\ke, 0) = (\val_0, \txid_0, \stub)$ for $\ke \in \dom(\hh)$, where $\val_0$ is the default value in $\Val$;
%\item\label{kv:wf.onewrite} 
%    transactions write at most one version for each key:
%\[
%\fora{\ke, i,j }
%0 \leq i, j < \abs{ \hh(\ke) }
%\land \WTx(\hh(\ke, i)) = \WTx(\hh(\ke, j))
%\implies i = j 
%\]
%\item\label{kv:wf.oneread} 
%    transactions read at most one version for each key:
%\[
%\fora{\ke, i,j } 
%0 \leq i, j < \abs{ \hh(\ke) }
%\land \RTx(\hh(\ke, i)) \cap \RTx(\hh(\ke, j)) \neq \emptyset 
%\implies i = j
%\]
%\item\label{kv:wf.so} 
%	transactions (of the same client) install different versions of a key in the session order; 
%%    the order in which transactions issued by the same client install different versions for a key $\ke$, 
%%    is consistent with the session order;
%    a client $\cl$ can read versions written by $\cl$ itself only after they have been installed:
%\begin{multline*}
%    \fora{ \ke, \cl, i,j, n, m} 
%    0 \leq i < j < \abs{\hh(\ke)} 
%    \land \txid_{\cl}^{n} = \WTx(\hh(\ke,i)) \\
%    {} \wedge \txid_{\cl}^{m} \in \Set{\WTx(\hh(\ke,j))} \cup \RTx(\hh(\ke, i))
%    \implies n < m
%\end{multline*}
%\end{enumerate}
%
We say that kv-stores that satisfy the conditions \textbf{(i)}-\textbf{(iv)} above are 
\emph{well-formed}.
Henceforth, we will always assume kv-stores tp be well-formed, and we use $\HisHeaps$ to denote 
the set of well-formed kv-stores.

\subsubsection{Views and Configurations}

Key-value stores track the global state of a database. 
However, clients do not need to agree on the portion of 
the state of the database that they observe. Different clients 
may observe different different subsets of versions of the same key.
when executing transactions,
%different \emph{clients} may observe different versions of the same key. 
To keep track of the versions observed by clients,
we introduce the notion of \emph{views} (\cref{def:view}). 

\begin{definition}[Views, configurations]
\label{def:view}
\label{def:cuts}
\label{def:views}
\label{def:configuration}
A \emph{view} of a kv-store $\hh$ is a mapping  
$\vi \in \Views(\hh) \defeq \Keys \to\powerset{\Nat}$ such that:
\begin{align}
    & \fora{ \ke } 
    0 \in \vi(\ke) 
    \wedge \fora{ i \in \vi(\ke) } 
    i < \abs{ \hh(\ke) } 
    \tag{WF}
    \label{eq:view.wf}\\
    %\Set{0} \subseteq \vi(\ke) \subseteq \Setcon{i}{ 0 \leq i < \abs{\mkvs(\ke)}}
    & 
    \fora{ \ke_1,\ke_2, i_1, i_2} 
	i_1 \in \vi(\ke_1) 
	\land \WTx(\hh(\ke_1, i_1)) = \WTx(\hh(\ke_2, i_2)) 
	\implies i_2 \in \vi(\ke_2)
	\tag{Atomic}
	\label{eq:view.atomic}
\end{align}
A \emph{configuration} $\conf \in \Confs$, is a pair $(\hh, \viewFun)$, 
where $\hh \in \HisHeaps$ and
$\viewFun : \Clients \parfinfun \Views(\hh)$. 
\end{definition}
Configurations extend key-value stores with the information of 
the views of each client. In a configuration $\conf = (\hh, \viewFun)$, the view of client 
$\cl$, $\vi = \viewFun(\cl)$ (if defined) determines for each key $\ke \in \Keys$ the sub-list of versions in $\hh$ 
that the client is aware of, or equivalently that it can observe. If $i,j \in \vi(\ke)$ and $i < j$, then the client is 
aware of the fact that $\hh$ contains the versions $\hh(\ke, i)$, $\hh(\ke, j)$, and that $\hh(\ke, j)$ is more 
up-to-date than $\hh(\ke, i)$. The client also observes the information relative to the versions $\hh(\ke, i)$ and 
$\hh(\ke, j)$, i.e. the value they carry and the meta-data relative to writing and reading transactions of such 
versions. 
Equation \eqref{eq:view.wf} in \cref{def:view} is a natural requirement, while \eqref{eq:view.atomic} 
models the atomic visibility of transactions: if a client observes the updates of a transaction $\txid$, then 
it must observe all the updates from $\txid$. 
We let $\Views = \bigcup_{\hh \in \HisHeaps} \Views(\hh)$ be the set of all view. 
Given a kv-store $\hh$ and two views $\vi, \vi' \in \Views(\hh)$, 
we write $\vi \viewleq \vi'$ when $\vi(k) \subseteq \vi'(\ke)$ for all $\ke \in \dom(\hh)$. 
The initial view $\vi_{0}$ is defined as $\vi_{0}(\ke) = \{0\}$ for each $\ke \in \Keys$.
A configuration $\conf_{0} = (\hh_{0}, \viewFun_{0})$ is
\emph{initial} if $\hh_{0}(\ke) = (\val_0, \txid_0, \emptyset)$ for all $\ke \in \Keys$.

Given a configuration $\conf = (\hh, \viewFun)$ and a client $\cl$ for which 
$\viewFun(\cl)$ is defined, the view $\vi(\cl)$ is used to determine a \emph{snapshot}, i.e. 
a mapping from each key to a unique value that the client observes when executing a transaction. 
In general, the snapshot of a transaction is also determined by a \emph{resolution policy} 
cite{}. Throughout this paper, we assume that the database employ the \emph{Last Writer 
Wins} \cite{} resolution policy to determine the snapshot of clients, although generalisation 
to different resolution policies is straightforward.

\begin{definition}[Snapshots]
\label{def:heaps}
\label{def:snapshot}
A snapshot is a mapping from keys to values \( \ss \in \Snapshots  \defeq \Keys \to \Val\).
Given $\hh \in \HisHeaps$ and $\vi \in \Views(\hh)$, the \emph{snapshot} of $\vi$ in 
$\hh$ is defined as $\snapshot(\hh, \vi) \defeq \lambda \ke \ldotp \valueOf(\hh(\ke, \max_{<}(\vi(\ke)))$, 
where we recall that $\max_{<}(\vi(\ke))$ is the maximum element in $\vi(\ke)$ with respect to the natural 
order $<$ over $\mathbb{N}$.
\end{definition}
Given a kv-store $\hh$, a key $\ke$ and a view $\vi$, we often commit an abuse of notation and write 
$\hh(\ke, vi)$ as a shorthand for 
$\hh(\ke, \max_{<}(\vi(\ke))$. Thus, $\snapshot(\hh, \vi) = \lambda \ke \ldotp \valueOf(\hh(\ke, \vi))$. 

\begin{remark}
Because the function $\snapshot(\hh, \vi)$ only selects the last version of $\hh$ comprised 
in $\vi$, one may wonder about the necessity of including multiple versions in the view of a 
key $\ke$.  Here we only point that requiring a view to contain a single version for each key 
would impair the expressiveness of our framework in terms of consistency models that it captures; 
unfortunately, we have to wait until \cref{} before giving more details on this issue.
\end{remark}

%Let $\vi$ be on a key-value store $\hh$; the view $\vi$ determines, 
%for each key $\ke \in \Keys$, the sub-list of versions in $\hh(\ke)$ 
%$\vi(\ke)$ determines the subset of versions that 
%  
%\emph{The set of views} 
%$
%\Views \defeq \bigcup_{\hh \in \HisHeaps} \Views(\hh)
%$.
%A \emph{configuration}, $\conf \in \Confs$, is a pair $(\hh, \viewFun)$, 
%where $\hh \in \HisHeaps$ and
%$\viewFun : \Clients \parfinfun \Views(\hh)$. 
%The $\conf_{0} = (\hh_{0}, \viewFun_{0})$ is an
%\emph{initial configuration} if, $\hh_{0}(\ke) = (\val_0, \txid_0, \emptyset)$ for all $\ke \in \Keys$.
%
%A view of a kv-store $\hh$ is a mapping from the keys in $\hh$ to a non-empty set of natural numbers. 
%For each $\ke \in \dom(\hh)$, $\vi(\ke)$ denotes the indices of versions in $\hh(\ke)$ recorded in $\vi$. 
%As such, when $i \in \vi(\ke)$ then $ i < \abs{ \hh(\ke) }$. 
%Moreover, the initialisation version (at index $0$) must be included in all views. 
%These two properties are captured by \eqref{eq:view.wf} in \cref{def:view} below. 
%Lastly, views cannot observe \emph{partial} effects of a given transaction. 
%That is, if a view includes a version written by a transaction $\txid$, it must include \emph{all} versions written by $\txid$. 
%This is formalised by \eqref{eq:view.atomic} in \cref{def:view} below, and captures the \emph{atomic visibility} of transactions. 


%At any point during execution, the overall state is captured by \emph{a configuration}. 
%A configuration includes a kv-store and a partial mapping from clients to views.
%The view of the client $\cl$ in $\hh$ reflects the set of versions for each key 
%that the client \(\cl \) observes upon executing a transaction. 
%The constraint of \cref{eq:view.atomic} establishes that if a client observes 
%a version of some key written by a transaction $\txid$, then it must observe all the versions of 
%all keys that $\txid$ wrote. 


%\begin{definition}[Views, configurations]
%\label{def:view}
%\label{def:cuts}
%\label{def:views}
%\label{def:configuration}
%A \emph{view} of a kv-store $\hh$ is a mapping  
%$\vi \in \Views(\mkvs) \defeq \dom(\hh) \to\powerset{\Nat}$ such that:
%\begin{align}
%    & \fora{ \ke } 
%    0 \in \vi(\ke) 
%    \wedge \fora{ i \in \vi(\ke) } 
%    i < \abs{ \hh(\ke) } 
%    \tag{WF}
%    \label{eq:view.wf}\\
%    %\Set{0} \subseteq \vi(\ke) \subseteq \Setcon{i}{ 0 \leq i < \abs{\mkvs(\ke)}}
%    & 
%    \fora{ \ke,\ke', i,j} 
%	j \in \vi(\ke) 
%	\land \WTx(\hh(\ke, j)) = \WTx(\hh(\ke', i) 
%	\implies i \in \vi(\ke')
%	\tag{Atomic}
%	\label{eq:view.atomic}
%\end{align}
%
%\emph{The set of views} is
%$
%\Views \defeq \bigcup_{\hh \in \HisHeaps} \Views(\hh)
%$.
%A \emph{configuration}, $\conf \in \Confs$, is a pair $(\hh, \viewFun)$, 
%where $\hh \in \HisHeaps$ and
%$\viewFun : \Clients \parfinfun \Views(\hh)$. 
%The $\conf_{0} = (\hh_{0}, \viewFun_{0})$ is an
%\emph{initial configuration} if, $\hh_{0}(\ke) = (\val_0, \txid_0, \emptyset)$ for all $\ke \in \Keys$.
%\end{definition}



%Given a kv-store $\hh$ and two views $\vi, \vi' \in \Views(\hh)$, 
%we write $\vi \viewleq \vi'$ when $\vi(k) \subseteq \vi'(\ke)$ for all $\ke \in \dom(\hh)$. 
%Also, we commit an abuse of notation and write $\hh(\ke, \vi)$ as a shorthand 
%for $\hh(\ke, \max_{<}(\vi(\ke)))$. 
%Note that such a version always exists as
%$\vi(\ke) \neq \emptyset$ (see \eqref{eq:view.wf} above).

%\subsubsection{Snapshots}
%Transactions are executed with respect to a \emph{snapshot} of a kv-store.
%A snapshot $\h$ is a mapping from keys to values (\cref{def:snapshot}). 
%Given a view $\vi$ of a transaction, a snapshot can be induced 
%by extracting the value of the latest observable version for each key $\ke \in \dom(\hh)$. 
%%Views are used to determine the snapshot in which a transaction 
%%is executed, according to the following definition.


%\begin{definition}[Snapshots]
%\label{def:heaps}
%\label{def:snapshot}
%A snapshot is a mapping from keys to values \( \ss \in \Snapshots  \defeq \Keys \to \Val\).
%Given $\hh \in \HisHeaps$ and $\vi \in \Views(\hh)$, the \emph{snapshot} of $\vi$ in 
%$\hh$ is defined as $\snapshot(\hh, \vi) \defeq \lambda \ke \ldotp \valueOf(\hh(\ke, \max_{<}(\vi(\ke)))$.
%\end{definition}

\paragraph{Fingerprints}
Once the execution of a transaction is completed, its effects are committed to the kv-store. 
The effects of transactions are modelled as a \emph{fingerprint} $\opset$. 
A finger print comprises a set of \emph{operations} $\Ops$: $\opset \subseteq \Ops$. 
An operation is a triple of the form $(l, \ke, \val)$ with $l \in \{\otR, \otW\}$.    
Intuitively, given the fingerprint $\opset$ of a transaction $\txid$, 
$(\otR, \ke, \val) \in \opset$ denotes that 
$\txid$ requested to read key $\ke$ from the kv-store, 
and it fetched a version carrying value $\val$.
Similarly, $(\otW, \ke, \val) \in \opset$ denotes that 
$\txid$ attempted to write value $v$ for key $\ke$. 
A fingerprint includes at most one read operation per key;
this formalises the intuition that, in our setting, 
transactions always read from an atomic snapshot of the kv-store. 
Analogously, a fingerprint includes at most one write operation per key.
%because a client either observes none or all the updates of a transaction.

\begin{definition}[Fingerprints]
The set of \emph{operations} is
$\Ops \defeq \{(l, \ke, \val) \mid$ $ l \in \Set{\otW, \otR} \land \ke \in \Keys \wedge \val \in \Val \}$.
A \emph{fingerprint} $\opset$ is a subset of operations, $\opset \subseteq \Ops$,
such that for all $\ke \in \Keys$ and \( l \in \Set{\otW, \otR} \),
if $(l, \ke, \val_1), (l, \ke, \val_2) \in \opset$, then $\val_1 = \val_2$.
\end{definition}



%\ac{Note that I now require fingerprints to be non-empty sets of transactions. This simplifies a lot the development of 
%the theory of kv-stores, and it fixes a problem that was spotted by Shale, that breaks the compositionality of 
%execution tests (see later). The main reason why we allowed empty fingerprints is that in the semantics, a client can 
%execute a transaction with no access to the memory. In practice, in the semantics we can require that at least 
%one access to the database must be performed in transactions. This can be checked syntactically, and nobody 
%should complain about that. I can put a remark about how this is a natural requirement that, if violated, 
%breaks the compositionality of consistency models.\\ 
%\textbf{Update 02/08/2018}: empty fingerprints are now allowed again. We still had some problems with compositionality, 
%one of which has to do with the fact that we allow the view of a client over some key to move freely after executing a transaction, 
%even if such a key was not accessed by the transaction. Later, I forbid this behaviour by requiring in execution tests that the 
%view of an client for a given key cannot be shifted if the transaction executed by the client did not access such a key.}
%$(\otW, \ke, \val) \in \opset$ means that the transaction writes a new version, carrying value $\val$, for key $\ke$. 




%of $\vi$ by accessing the value of 
%A view $\vi$ in $\hh$ naturally defines a snapshot $\snapshot(\hh, \nu)$
%A MKVS tracks the global state of the system; however, different \emph{clients} may observe different versions of the same key. 
%To model this, we introduce the notion of \emph{views} (\cref{def:views}). 
%A view $V$ reflects the particular version for each key that a client observes upon executing a transaction. 
%%We present an example of views in \cref{fig:hheap-a} with two views: $\client_1$ in red and $\client_2$ in blue.
%More concretely, the view for \( \client_1 \) is given formally as $\vi_1 = \Set{\key{k}_1 \mapsto 1, \key{k}_2 \mapsto 0}$.
%That is, the client with view $\vi_1$ observes the second version (at index 1) of key \( \ke_{1} \) with value $v_1$, and the first version (at index 0) of key \( \ke_2 \) with value $v'_0$.
%%, and 
%%the first version of $\key{k}_2$, carrying value $0$. Similarly, according to its view 
%%$V_2 = [\key{k}_1 \mapsto 2, \key{k}_2 \mapsto 2]$, the client $\txid_2$ observes 
%%in $\hh$the second and most up-to-date version for both $\key{k}_1$ and $\key{k}_2$.
%
%\begin{definition}[Views]
%\label{def:view}
%\label{def:cuts}
%\label{def:views}
%\emph{A view} is a partial finite function from keys to indexes:
%$
%\vi \in \Views \defeq \Addr \parfinfun \Nat 
%%\begin{rclarray}
%%    \vi \in \Views & \defeq & \Addr \parfinfun \Nat 
%%\end{rclarray}
%$.                                                                 
%The \emph{view composition}, $\composeVI: \Views \times \Views \rightharpoonup \Views$ is defined as the standard disjoint function union: $\composeVI \eqdef \uplus$. 
%% \( \vi \composeVI \vi' \defeq \vi \uplus \vi'\) 
%The \emph{unit view}, $\unitVI \in \Views$, is a function with an empty domain: $\unitVI \eqdef \emptyset$. 
%% and the unit is \( \unitVI \defeq \emptyset\).
%The \emph{order relation} on views, $\orderVI: \Views \times \Views$, is defined between two views with the same domain as the point-wise comparison of their indexes for each entry: 
%\[
%\begin{rclarray}
%    \vi \orderVI \vi' & \defiff & \dom(\vi) = \dom(\vi') \land \fora{\ke} \cu(\ke) \leq \cu'(\ke) \\
%\end{rclarray}
%\]
%\end{definition}
%%
%We say view $\vi$ is \emph{older} than view $\vi'$ (or $\vi'$ is \emph{newer} than $\vi$) whenever $\vi \orderVI \vi'$ holds.
%
%
%\mypar{Configurations} A \emph{configuration} comprises an MKVS, and the views associated with clients.
%In \cref{fig:hheap-a} we present an example of a configuration comprising an MKVS and the two views associated with clients $\client_1$ and $\client_2$. 
%We write $\version(\hh, \ke, \vi)$ for $\hh(\ke, \vi(\ke))$; 
%and write $\valueOf(\hh, \ke, \vi)$ as a shorthand for $ \valueOf(\version(\hh, \key{k}, V))$; similarly for $\WTx, \RTx$.
%%we commit an abuse of notation and often write $\valueOf(\hh, \ke, \vi)$ in lieu of $ \valueOf(\version(\hh, \key{k}, V))$, and similarly for $\WTx, \RTx$.
%When $\ver = \version(\hh, \ke, \vi)$, we say that \emph{$\vi$ $\ke$-points to $\ver$ in $\hh$}. 
%When $\ver = \hh(\ke, i)$ for some $0 \leq i \le \vi(\ke)$, we say that \emph{$\vi$ $\ke$-includes $\ver$ in $\hh$}.
%Lastly, we always assume that MKVSs, views, and configurations are well-formed, unless otherwise stated.
%
%
%
%\begin{definition}[Configurations]
%A view $\vi$ is \emph{well-formed with respect to an MKVS} $\mkvs$, written \( \wfV{\mkvs, \vi} \),  iff they have the same domain and every index from $\vi$ is within the range of the corresponding entry in $\mkvs$ and the view is \emph{atomic} with  respect to the key-value store: 
%\[
%\begin{rclarray}
%    \wfV{\mkvs, \vi} & \defeq & \dom(\mkvs) = \dom(\vi) \land \fora{\ke \in \dom(\vi)} 0 \leq \vi(\ke) < \lvert \mkvs(\ke) \rvert \\
%    \pred{atomic}{\vi ,\hh} & \eqdef & \fora{\txid } \exsts{\ke, i} i \leq \vi(\ke) \land \hh(\ke,i) = (\stub, \txid, \stub) \implies \pred{visible}{\txid, \vi, \hh} \\ 
%    \pred{visible}{\txid, \vi, \hh} & \eqdef & \fora{\ke, i} \hh(\ke,i) = (\stub, \txid, \stub) \implies i \leq \vi(\ke) 
%\end{rclarray}
%\]
%%
%\azalea{We need a symbol for this to fill the ???? above. Also ???? below. \sx{Done}}
%A \emph{configuration} $\conf$ is a pair of the form $(\hh, \viewFun)$, where $\hh$ denotes an MKVS, and $\viewFun: \Clients \parfinfun \Views$ is a partial finite function from clients to views. 
%A configuration $\conf = (\hh, \viewFun)$ is \emph{well-formed}, written \( \wfC{\conf}\), iff for all clients $\cl \in \dom(\viewFun)$, the view $\viewFun(\txid)$ is well-formed with respect to $\hh$. 
%%We say that a view $V$ is well-defined with respect to the 
%%MKVS $\hh$ if, $\forall \key{k} \in \ke. 0 < V(\key{k}) \leq 
%%\lvert \hh(\key{k}) \rvert$. 
%%Given a view $V$ that is well-defined 
%%with respect to a 
%
%\end{definition}
%
%\mypar{Snapshots} When a client executes a transaction on the $\mkvs$ MKVS, it extracts a \emph{snapshot} of it via the \( \func{snapshot}{\mkvs, \vi} \) function, extracting the values corresponding to the versions indexed by its view \( \vi \) (\cref{def:snapshot}).
%For instance, for client \( \client_1 \) in \cref{fig:hheap-a}, the $\func{snapshot}{\cdots}$ functions yields a state where key $\ke_1$ carries value $v_1$ and second key \( \ke_2 \) carries value $v'_0$.
%%The concrete state extracted in this way takes the name of the \emph{snapshot} of the transaction.
%%In general, the process of determining the view of a client, hence the snapshot in which such a client executes transactions, is non-deterministic.
%
%\azalea{Before in MKVSs we had values drawn from $\Nat$ in \cref{def:mkvs}. Now we use $\Val$. I think you mean to use $\Val$ in both places? \sx{I would say so} }
%\begin{definition}[Snapshots]
%\label{def:heaps}
%\label{def:snapshot}
%Given the sets of values $\Val$  and keys \( \Addr\)  (\cref{def:mkvs}), the set of \emph{snapshots} is:
%$
%    \h \in \Heaps \eqdef \Addr \parfinfun \Val
%$. 
%%\[
%%\begin{rclarray}
%%    \h \in \Heaps & \eqdef & \Addr \parfinfun \Val
%%\end{rclarray}
%%\]
%The \emph{snapshot composition function}, $\composeH: \Heaps \times \Heaps \parfun \Heaps$, is defined as $\composeH \eqdef \uplus$, where $\uplus$ denotes the standard disjoint function union. The \emph{ snapshot unit element} is $\unitH \eqdef \emptyset$, denoting a function with an empty domain.
%The \emph{partial commutative monoid of snapshots} is $(\Heaps, \composeH, \{\unitH\})$.
%Given an MKVS $\hh$ and a view $\vi$, the snapshot of $\vi$ in $\hh$, written $\snapshot(\hh, \vi) $, is defined as:
%$
%    \snapshot(\hh, \vi) \defeq \lambda \ke \ldotp \valueOf(\hh, \ke, \vi)
%$.
%%\[
%%\begin{rclarray}
%%    \snapshot(\hh, \vi) & \defeq & \lambda \ke \ldotp \valueOf(\hh, \ke, \vi).
%%\end{rclarray}
%%\]
%\end{definition}
%
%\sx{Need some explanation}
%\ac{General Comment on this Section: it is too abstract. We 
%should give either here or in the introduction an example of computation - 
%the write skew program should be okay that helps the reader understanding 
%what's going on. Also, it could be also good to illustrate the notions 
%of execution tests and consistency models.}
%
%\sx{From Andrea: introduce the execution test here with a table, also introduce fingerprint here}


\subsection{Consistency Models and Execution Tests}
Formally, a \emph{consistency model} $\CMs$ is a set of key-value stores. 
Each $\hh \in \CMs$ represents a possible scenario that 
can be obtained as a result of multiple clients committing transactions. 
To specify consistency models we introduce the notion of \emph{execution tests}. 
\ac{
For example, \emph{serialisability} can be described as the set 
of key-value stores for which it is possible to recover a total schedule of transactions, 
such that each read operation on key $\ke$ fetches its value from the 
most recent write on the same key \cite{??????}.
In this sense, the kv-store $\hh$ from \cref{fig:hheap-a} is not serialisable: 
transaction $\txid_1$ reads the initial version carrying value $\val'_0$ for key $\ke_{2}$, 
and installs a new version of $\ke_{2}$ carrying value $\val_1$. The transaction $\txid_2$ 
reads the initial version carrying value $\val'_0$, and therefore, 
cannot be scheduled after $\txid_1$. Similarly, $\txid_2$ cannot be scheduled after $\txid_1$.
}
\begin{definition}
\label{def:execution.test}
An \emph{execution test} is a set of tuples $\ET \subseteq \HisHeaps \times \Views \times \powerset{\Ops} \times \Views$ 
such that for every element $(\hh, \vi, \opset, \vi') \in \ET$,
\textbf{(i)} for any read operations $(\otR, \ke, \val) \in \opset$ then $\hh(\ke, \max_{<}(\vi(\ke))) = \val$, 
and \textbf{(ii)}  for any key \( \ke \) such that $\vi(\ke) \neq \vi'(\ke)$, 
then $( (\otR, \ke, \_) \in \opset \vee (\otW, \ke, \_)) \in \opset)$.
%\textbf{(iii)} $\forall \opset' \subseteq \opset.\; (\hh, \vi, \opset', \vi') \in \ET$
\end{definition}
%\sx{The definition has a problem that for the subset \( \f'\) the post-view \( \vi' \) might point to an undefined version, I also thing it should satisfy \( \fora{\vi''} \vi \sqsubseteq \vi'' \implies (\hh, \vi, \opset', \vi'') \)}.
%\ac{I removed this condition, as I do not think that it was used anywhere. The new definition requires that 
%you cannot change the view for keys that you do not read nor write.}
%\sx{The current \CP will not satisfy the \textbf{(ii)}. }
Given an execution test $\ET$, 
then $(\hh, \vi, \opset, \vi') \in \ET$ means that 
a client whose view over the key-value store $\hh$ is $\vi$, 
can commit a transaction whose fingerprint is $\opset$;
as a result of this operation, the view of the client must be updated to $\vi'$.
Henceforth, we adopt the more suggestive notation $\ET \vdash (\hh, \vi) \triangleright \opset: \vi'$ 
in lieu of $(\hh, \vi, \opset, \vi') \in \ET$.
Execution tests induce \emph{consistency models} \( \CMs(\ET) \) as defined in \cref{def:reduction,def:cm}.
\begin{definition}[$\ET$-reductions]
\label{def:reduction}
Let $\cl$ be a client and $\opset$ be a fingerprint. 
An \emph{action} $\alpha \in \Act$ has either the form $(\cl, \varepsilon)$, 
or $(\cl, \opset)$. 
Given an execution test $\ET$ the action-labelled relation 
$\xrightarrowtriangle{}_{\ET} \subseteq \Confs \times \Act \times \Confs$ 
is defined as the smallest relation such that:
\begin{itemize}
\item 
    $\forall \vi, \vi', \cl, \hh, \viewFun.\; 
    \viewFun(\cl) = \vi 
    \wedge \vi \sqsubseteq \vi' 
    \implies (\hh, \viewFun) \xrightarrowtriangle{(\cl, \varepsilon)}_{\ET} 
    (\hh, \viewFun\rmto{\cl}{\vi'})$
\item 
    $\begin{array}[t]{@{}l@{}}
        \forall \vi, \vi', \cl, \opset, \hh, \hh', \viewFun.\; 
        \viewFun(\cl) = \vi
        \wedge (\ET \vdash (\hh, \vi) \triangleright \opset: \vi')  \\
        \quad {} \wedge \hh' \in \updateKV(\hh, \vi, \cl, \opset) 
        \implies (\hh, \viewFun) \xrightarrowtriangle{(\cl, \opset)}_{\ET} (\hh', \viewFun\rmto{\cl}{\vi'})
    \end{array}$
\end{itemize}
Such relations take the name of $\ET$-reductions, or simply reductions.
\end{definition}
Given an execution test $\ET$, sequences of $\ET$-reductions of the form $\conf_{0} \xrightarrowtriangle{\alpha_{0}}_{\ET} \cdots 
\xrightarrow{\alpha_{n-1}} \conf_{n}$ take the name of \emph{$\ET$-traces}.
\begin{definition}[Consistency Models]
\label{def:cm}
Given an execution test $\ET$, the set of configurations induced by $\ET$ is given by:
\[
\Confs(\ET) \defeq \Setcon{ \conf}{ \exsts{\conf_0} \conf_0 \text{ is initial } \wedge \conf_0 \xrightarrowtriangle{\stub}_{\ET} \cdots \xrightarrowtriangle{\stub}_{\ET} \conf }
\]
The \emph{consistency model} induced by $\ET$ is:
\( 
\CMs(\ET) \defeq \Setcon{ \hh }{ (\hh, \stub) \in \Confs(\ET) }
\)
\end{definition}
Thus, consistency models are computed from execution tests by closing the set of initial key-value stores with respect to two operations: 
\textbf{(i)} advancing the view of a client, 
and \textbf{(ii)} committing a fingerprint of a transaction. 

Last, for sanity check, consistency models induced by execution tests are monotonic in the following sense.
\begin{proposition}
\label{prop:mono-et}
Let $\ET_1 \subseteq \ET_2$. Then $\CMs(\ET_1) \subseteq \CMs(\ET_2)$.
\end{proposition}
\begin{proof}
    \ifTechReport
    It is sufficient to prove that \(\ET_1 \subseteq \ET_2 \implies \Confs(\ET_1) \subseteq\ Confs(\ET_2) \).
We prove it by induction on the length of the traces, \( n \).

\caseB{n = 0}
We have \( \conf_0 \in \Confs(\ET_1) \) and \( \conf_0 \in \Confs(\ET_2)\).
\caseI(n = i + 1)
Suppose identical traces of \( \ET_1 \) and \( \ET_2 \) respectively with length \( i \).
Let the final configuration be \( \conf_i = ( \mkvs_i, \viewFun_i ) \).
If the next step is a view shift or a step with empty fingerprint, it trivially holds.
If the next step is a step by a client \( \cl \) with fingerprint \( \f \),
we have \( \ET_1 \vdash \mkvs_i, \viewFun_i(\cl) \csat \f : vi' \).
The next configuration from \( \ET_1 \) is \( \conf_{i+1} = (\updateKV{ \mkvs_i, \viewFun_i(\cl), \f, \txid_\cl}) \).
Since \( \ET_1 \subseteq \ET_2 \), so \( \ET_1 \vdash \mkvs_i, \viewFun_i(\cl) \csat \f : vi' \) holds.
It is possible for \( \ET_2 \) to have the exactly same next configuration \( \conf_{n+1}\).

    \else
    See \cref{sec:mono-et}.
    \fi
\end{proof}

\subsection{Specifications of Consistency Models}
\begin{figure}
\begin{tabular}{ l @{} r }
\hline
\textbf{Consistency Model} & \textbf{Execution Test}\\
\hline
\MRd & $\vi \viewleq \vi'$\\
\MW & 
$j \in \vi(\ke) \wedge \WTx(\hh(\ke', i)) \xrightarrow{\PO?} \WTx(\hh(\ke, j)) 
\implies i \in \vi(\ke')$
\\
\RYW & $ \mkvs' = \updateKV(\hh, \vi, \txid, \opset) \implies \WTx( \mkvs'(\ke, i) ) \leq \txid \implies i \in \vi'(\ke) $\\
\WFR & $j \in \vi(\ke) \wedge \txid \in \RTx(\hh(\ke', i)) \wedge \txid {\xrightarrow{\PO?}}
\WTx(\ke, j) ) \implies i \in \vi(\ke')$\\
\CC & $\ET_{\CC} = \ET_{\MRd} \cap \ET_{\MW} \cap \ET_{\RYW} \cap \ET_{\WFR}$\\
\hline
\hline
\UA & $(\otW, \ke,  \stub) \in \opset \implies \fora{ i : 0 \leq i < \lvert \hh(\ke) \rvert } i \in \vi(\ke) $\\
\CP & \( \Setcon{(\mkvs, \vi, \f, \vi')}{\ddagger} \cap \ET_\MRd \cap \ET_\RYW \) \\
\PSI & $\ET_{\PSI} = \ET_{\CC} \cap \ET_{\UA}$\\
$\SI$ & $\Setcon{(\mkvs, \vi, \f, \vi')}{\dagger} \cap \ET_\MRd \cap \ET_\RYW  \cap \ET_\UA $\\
\SER & $\fora{ i : 0 \leq i < \lvert \hh(\ke) \rvert } i \in \vi(\ke) $\\
\hline
\end{tabular}

Given \( \txid \) that is the writer of a version \((k,i)\) observed by the view \( \vi \) or has a \( \AD \) edge to a version \( (\ke,i) \), 
the version \( (\ke',j) \) written by \( \txid' \) must be observed by the view \( \vi \), if \( (\txid', \txid) \in ((\PO \cup \VO \cup \RF) ; \AD ?)^{+}\).

\[
    \begin{rclarray}
        \func{RW^{-1}}{\mkvs, \ke, i} & \defeq & \Setcon{\txid}{\exsts{ j < i } \txid \in \RTx(\mkvs(\ke,j))} \\
        \func{SO^{-1}}{\txid} & \defeq & \Setcon{\txid'}{ \exsts{ \cl, m, n } \txid = \txid_{\cl}^{n} \land \txid' = \txid_{\cl}^{m} \land m < n } \\
        \dagger & \equiv &
        \begin{array}[t]{@{}l@{}}
            \fora{\ke, \ke', i, j, m, \txid, \txid', \txid''} \\
            \begin{array}{@{}l@{}}
            i \in \vi(\ke) 
            \land \txid \in \Set{\WTx(\mkvs(\ke,i))} \cup \func{RW^{-1}}{\mkvs, \ke, i} \land {} \\
            \quad \left(
                \begin{array}{@{}l @{}}
                    \left( \begin{array}{@{}l@{}}
                        \txid' \in \func{SO^{-1}}{\txid}
                        \land \txid' \in \Set{\WTx(\mkvs(\ke',j))} \cup  \RTx(\mkvs(\ke',j))
                    \end{array} \right)  \lor {} \\
                    \left( \begin{array}{@{}l@{}}
                        \txid \in \RTx(\mkvs(\ke',j)) 
                        \land \txid' = \WTx(\mkvs(\ke',j))
                    \end{array} \right) \lor {} \\ 
                    \left( \begin{array}{@{}l@{}}
                        \txid = \WTx(\mkvs(\ke',m)) 
                        \land \txid' = \WTx(\mkvs(\ke',j)) \land m > j
                    \end{array} \right) 
                \end{array}
                \right)  \\
            \qquad \implies j \in \vi(\ke') 
            \end{array}
        \end{array} \\

        \ddagger & \equiv &
        \begin{array}[t]{@{}l@{}}
            \fora{\ke, \ke', i, j, m, \txid, \txid', \txid''} \\
            \left( \begin{array}{@{}l@{}}
            i \in \vi(\ke) 
            \land \txid \in \Set{\WTx(\mkvs(\ke,i))} \cup \func{RW^{-1}}{\mkvs, \ke, i} \land {} \\
            \quad \left(
                \begin{array}{@{}l @{}}
                    \left( \begin{array}{@{}l@{}}
                        \txid' \in \func{SO^{-1}}{\txid}
                        \land \txid' \in \Set{\WTx(\mkvs(\ke',j))} \cup  \RTx(\mkvs(\ke',j))
                    \end{array} \right)  \lor {} \\
                    \left( \begin{array}{@{}l@{}}
                            \txid \in \RTx(\mkvs(\ke',j)) \land \txid' = \WTx(\mkvs(\ke',j))
                    \end{array} \right)
                    \end{array} \right) 
                \end{array}
                \right)  \\
                {} \lor \left( \begin{array}{@{}l@{}}
                        i \in \vi(\ke) \land \ke = \ke' \land j < i
                \end{array} \right) \\
                \qquad \implies j \in \vi(\ke') 
        \end{array} \\
    \end{rclarray}
\]
\caption{Execution tests for both client-centric (top) and data-centric consistency (bottom) models. 
The condition column define a necessary and sufficient condition for inferring $\ET_{\CM} \vdash \hh, \vi \triangleright \opset : \vi'$,  
where $\CM$ is the consistency model from the left column.
All the free variables are universally quantified.
}
\label{fig:execution.tests}
\label{fig:execution-tests}
\end{figure}

\begin{figure*}[t]
\captionsetup[subfigure]{aboveskip=-5pt, belowskip=5pt}
\begin{tabular}{@{} c | c | c @{}}
\hline
\phantom{-}& \phantom{-}& \phantom{-}\\
\begin{subfigure}{0.2\textwidth}
\begin{centertikz}

%Location x
\node(locx) {$\ke_1 \mapsto$};
\draw pic at ([xshift=\tikzkvspace]locx.east) {vlist={versionx}{%
    /$\val_0$/$\txid_0$/$\Set{\txid_\cl^2}$
    , /$\val_1$/$\txid_1$/$\Set{\txid_\cl^1}$
}};

\end{centertikz}\vspace{5pt}%
\caption{Disallowed by \(\MRd\)}
\label{fig:mr-disallowed}
\end{subfigure}
%\quad
&
\begin{subfigure}{0.36\textwidth}
\begin{centertikz}

%Location x

\node(locx) {$\ke_1 \mapsto$};
\draw pic at ([xshift=\tikzkvspace]locx.east) {vlist={versionx}{%
    /$\val_0$/$\txid_0$/$\Set{\txid'}$
    , /$\val_1$/$\txid_\cl^1$/$\emptyset$
}};

%Location y
\path (versionx.east) + (1,0) node (locy) {$\ke_2 \mapsto$};
\draw pic at ([xshift=\tikzkvspace]locy.east) {vlist={versiony}{%
    /$\val_0$/$\txid_0$/$\emptyset$
    , /$\val_2$/$\txid_\cl^2$/$\Set{\txid'}$
}};

\end{centertikz}\vspace{5pt}
\caption{Disallowed by \(\MW\)}
\label{fig:mw-disallowed}
\end{subfigure}
%\quad
&
\begin{subfigure}{0.39\textwidth}
\begin{centertikz}

%Location x
\node(locx) {$\ke_1 \mapsto$};
\draw pic at ([xshift=\tikzkvspace]locx.east) {vlist={versionx}{%
    /$\val_0$/$\txid_0$/$\Set{\txid}$
    , /$\val_1$/$\txid'$/$\Set{\txid_\cl^1}$
}};

%Location y
\path (versionx.east) + (1,0) node (locy) {$\ke_2 \mapsto$};
\draw pic at ([xshift=\tikzkvspace]locy.east) {vlist={versiony}{%
    /$\val_0$/$\txid_0$/$\emptyset$
    , /$\val_2$/$\txid_\cl^2$/$\Set{\txid}$
}};

\end{centertikz}

\vspace{5pt}
\caption{Disallowed by \(\WFR\)}
\label{fig:wfr-disallowed}
\end{subfigure}\\
\hline
\end{tabular}
%
%
%
%
\begin{tabular}{@{} c | c | c @{}}
\phantom{-}& \phantom{-}& \phantom{-}\\
\begin{subfigure}{0.25\textwidth}
\begin{centertikz}%

%Location x
\node(locx) {$\ke_1 \mapsto$};
\draw pic at ([xshift=\tikzkvspace]locx.east) {vlist={versionx}{%
    /$0$/$\txid_0$/$\Set{\txid_\cl^1,\txid_\cl^2}$
    , /$1$/$\txid_\cl^1$/$\emptyset$
    , /$1$/$\txid_\cl^2$/$\emptyset$
}};

\end{centertikz}%
\vspace{5pt}
\caption{Disallowed by \(\RYW\)}
\label{fig:ryw-disallowed}
\end{subfigure}
& 

\begin{subfigure}{0.40\textwidth}
\begin{centertikz}%

%Location x
\node(locx) {$\ke_1 \mapsto$};
\draw pic at ([xshift=\tikzkvspace]locx.east) {vlist={versionx}{%
    /$\val_0$/$\txid_0$/$\Set{\txid_2}$
    , /$\val_1$/$\txid_1$/$\emptyset$
}};

%Location y
\path (versionx.east) + (1,0) node (locy) {$\ke_2 \mapsto$};
\draw pic at ([xshift=\tikzkvspace]locy.east) {vlist={versiony}{%
    /$\val_0$/$\txid_0$/$\Set{\txid_1}$
    , /$\val_2$/$\txid_2$/$\emptyset$
}};

\end{centertikz}%
\vspace{5pt}
\caption{Write skew, disallowed by \(\SER\)}
\label{fig:ser-disallowed}
\end{subfigure}%
&
\begin{subfigure}{0.30\textwidth}
\begin{centertikz}

\node(locx) {$\ke_1 \mapsto$};
\draw pic at ([xshift=\tikzkvspace]locx.east) {vlist={versionx}{%
    /$0$/$\txid_0$/$\Set{\txid,\txid'}$
    , /$1$/$\txid$/$\emptyset$
    , /$1$/$\txid'$/$\emptyset$
}};

\end{centertikz}
\vspace{5pt}
\caption{Lost update, disallowed by \(\UA\)}
\end{subfigure}
\\
\hline
\end{tabular}
%
%
%
%
\phantom{x}\vspace{7pt}
\begin{tabular}{@{} c | c @{}}
\phantom{-}& \phantom{-} \\
\begin{subfigure}{0.42\textwidth}
\begin{centertikz}%
%Location x
\node(locx) {$\ke_1 \mapsto$};
\draw pic at ([xshift=\tikzkvspace]locx.east) {vlist={versionx}{%
    /$\val_0$/$\txid_0$/$\Set{\txid_{\cl_2}^1}$
    , /$\val_1$/$\txid$/$\Set{\txid_{\cl_1}^1}$
}};

%Location y
\path (versionx.east) + (1,0) node (locy) {$\ke_2 \mapsto$};
\draw pic at ([xshift=\tikzkvspace]locy.east) {vlist={versiony}{%
    /$\val_0$/$\txid_0$/$\Set{\txid_{\cl_1}^2}$
    , /$\val_1$/$\txid$/$\Set{\txid_{\cl_2}^2}$
}};

\end{centertikz}%
\vspace{5pt}
\caption{Long fork, disallowed by \(\CP\)}
\label{fig:cp-disallowed-2}
\label{fig:cp-disallowed}
\end{subfigure}
&
\begin{subfigure}{0.542\textwidth}%
\begin{centertikz}%
%Location x
\node(locx) {$\ke_1 \mapsto$};
\draw pic at ([xshift=\tikzkvspace]locx.east) {vlist={versionx}{%
    /$\val_0$/$\txid_0$/$\Set{\txid_4}$
    , /$\val_1$/$\txid_1$/$\emptyset$
    , /$\val_2$/$\txid_2$/$\emptyset$
}};

%Location y
\path (versionx.east) + (1,0) node (locy) {$\ke_2 \mapsto$};
\draw pic at ([xshift=\tikzkvspace]locy.east) {vlist={versiony}{%
    /$\val_0$/$\txid_0$/$\Set{\txid_2}$
    , /$\val_3$/$\txid_3$/$\Set{\txid_4}$
    , /$\val_4$/$\txid_4$/$\emptyset$
}};

%%Location z
%\path (versiony.east) + (1,0) node (locz) {$\ke_3 \mapsto$};
%\matrix(versionz) [version list,column 2/.style={text width=7mm}]
   %at ([xshift=\tikzkvspace]locz.east) {
 %{a} & $\txid_0$ & {a} & $\txid_3$ & {a} & $\txid_4$ \\
  %{a} & $\emptyset$ & {a} & $\emptyset$ & {a} & $\emptyset$\\
%};

%\tikzvalue{versionz-1-1}{versionz-2-1}{locz-v0}{$\val_0$};
%\tikzvalue{versionz-1-3}{versionz-2-3}{locz-v1}{$\val_1$};
%\tikzvalue{versionz-1-5}{versionz-2-5}{locz-v2}{$\val_2$};

%%Location w
%\path (versionz.east) + (1,0) node (locw) {$\ke_4 \mapsto$};
%\matrix(versionw) [version list,column 2/.style={text width=7mm}]
    %at ([xshift=\tikzkvspace]locw.east) {
    %{a} & $\txid_0$ & {a} & $\txid_1$ \\
    %{a} & $\{\txid_4\}$ & {a} & $\emptyset$ \\
%};
%\tikzvalue{versionw-1-1}{versionw-2-1}{locw-v0}{$\val_0$};
%\tikzvalue{versionw-1-3}{versionw-2-3}{locw-v1}{$\val_1$};
\end{centertikz}
\vspace{5pt}
\caption{Allowed by \( \UA \) and \( \CP \) but disallowed by \(\SI\)}%
\label{fig:si-disallowed}%
\end{subfigure} \\
\hline
\end{tabular}
%\begin{tabular}{@{}c @{} c @{}}
%\begin{minipage}{0.4\textwidth}
%\begin{subfigure}{\textwidth}
%\begin{centertikz}
%%Location x
%\node(locx) {$\ke_1 \mapsto$};
%
%\matrix(versionx) [version list,,column 2/.style={text width=8mm},column 4/.style={text width=7mm}]
%    at ([xshift=\tikzkvspace]locx.east) {
%    {a} & $\txid_0$ & {a} & $\txid_{\cl}^{1}$\\
%    {a} & $\left\{\txid_{\cl'}^{2}\right\}$ & {a} & $\emptyset$ \\
%};
%\tikzvalue{versionx-1-1}{versionx-2-1}{locx-v0}{$\val_0$};
%\tikzvalue{versionx-1-3}{versionx-2-3}{locx-v1}{$\val_1$};
%
%%Location y
%\path (locx.south) + (0,\tikzkeyspace) node (locy) {$\ke_2 \mapsto$};
%\matrix(versiony) [version list,column 2/.style={text width=8mm},column 4/.style={text width=7mm}]
%   at ([xshift=\tikzkvspace]locy.east) {
% {a} & $\txid_0$ & {a} & $\txid_{\cl'}^1$ \\
%  {a} & $\left\{\txid_{cl}^{2}\right\}$ & {a} & $\emptyset$\\
%};
%
%\tikzvalue{versiony-1-1}{versiony-2-1}{locy-v0}{$\val_0$};
%\tikzvalue{versiony-1-3}{versiony-2-3}{locy-v1}{$\val_2$};
%
%\end{centertikz}
%\caption{Disallowed by \(\CP\)}
%\label{fig:cp-disallowed-2}
%\end{subfigure}
%
%\begin{subfigure}{\textwidth}
%\begin{centertikz}
%%Location x
%\node(locx) {$\ke_1 \mapsto$};
%
%\matrix(versionx) [version list,column 2/.style={text width=7mm},column 4/.style={text width=7mm}]
%    at ([xshift=\tikzkvspace]locx.east) {
%    {a} & $\txid_0$ & {a} & $\txid_1$\\
%    {a} & $\left\{\txid_2\right\}$ & {a} & $\emptyset$ \\
%};
%\tikzvalue{versionx-1-1}{versionx-2-1}{locx-v0}{$\val_0$};
%\tikzvalue{versionx-1-3}{versionx-2-3}{locx-v1}{$\val_1$};
%
%%Location y
%\path (locx.south) + (0,\tikzkeyspace) node (locy) {$\ke_2 \mapsto$};
%\matrix(versiony) [version list,column 2/.style={text width=7mm},column 4/.style={text width=7mm}]
%   at ([xshift=\tikzkvspace]locy.east) {
% {a} & $\txid_0$ & {a} & $\txid_2$ \\
%  {a} & $\left\{\txid_1\right\}$ & {a} & $\emptyset$\\
%};
%
%\tikzvalue{versiony-1-1}{versiony-2-1}{locy-v0}{$\val_0$};
%\tikzvalue{versiony-1-3}{versiony-2-3}{locy-v1}{$\val_2$};
%\end{centertikz}
%\caption{Disallowed by \(\SER\)}
%\label{fig:ser-disallowed}
%\end{subfigure}
%
%\end{minipage}
%
%&
%\begin{subfigure}{0.55\textwidth}%
%\begin{centertikz}%
%%Location x
%\node(locx) {$\ke_1 \mapsto$};
%
%\matrix(versionx) [version list,column 2/.style={text width=7mm}]
%    at ([xshift=\tikzkvspace]locx.east) {
%    {a} & $\txid_0$ & {a} & $\txid_1$ & {a} & $\txid_2$\\
%    {a} & $\emptyset$ & {a} & $\emptyset$ & {a} & $\emptyset$\\
%};
%\tikzvalue{versionx-1-1}{versionx-2-1}{locx-v0}{$\val_0$};
%\tikzvalue{versionx-1-3}{versionx-2-3}{locx-v1}{$\val_1$};
%\tikzvalue{versionx-1-5}{versionx-2-5}{locx-v1}{$\val_2$};
%%Location y
%\path (locx.south) + (0,\tikzkeyspace) node (locy) {$\ke_2 \mapsto$};
%\matrix(versiony) [version list,column 2/.style={text width=7mm}]
%   at ([xshift=\tikzkvspace]locy.east) {
% {a} & $\txid_0$ & {a} & $\txid_3$ \\
%  {a} & $\left\{\txid_2\right\}$ & {a} & $\emptyset$\\
%};
%
%\tikzvalue{versiony-1-1}{versiony-2-1}{locy-v0}{$\val_0$};
%\tikzvalue{versiony-1-3}{versiony-2-3}{locy-v1}{$\val_1$};
%
%%Location z
%\path (locy.south) + (0,\tikzkeyspace) node (locz) {$\ke_3 \mapsto$};
%\matrix(versionz) [version list,column 2/.style={text width=7mm}]
%   at ([xshift=\tikzkvspace]locz.east) {
% {a} & $\txid_0$ & {a} & $\txid_3$ & {a} & $\txid_4$ \\
%  {a} & $\emptyset$ & {a} & $\emptyset$ & {a} & $\emptyset$\\
%};
%
%\tikzvalue{versionz-1-1}{versionz-2-1}{locz-v0}{$\val_0$};
%\tikzvalue{versionz-1-3}{versionz-2-3}{locz-v1}{$\val_1$};
%\tikzvalue{versionz-1-5}{versionz-2-5}{locz-v2}{$\val_2$};
%
%%Location w
%\path (locz.south) + (0,\tikzkeyspace) node (locw) {$\ke_4 \mapsto$};
%\matrix(versionw) [version list,column 2/.style={text width=7mm}]
%    at ([xshift=\tikzkvspace]locw.east) {
%    {a} & $\txid_0$ & {a} & $\txid_1$ \\
%    {a} & $\{\txid_4\}$ & {a} & $\emptyset$ \\
%};
%\tikzvalue{versionw-1-1}{versionw-2-1}{locw-v0}{$\val_0$};
%\tikzvalue{versionw-1-3}{versionw-2-3}{locw-v1}{$\val_1$};
%\end{centertikz}%
%\caption{Disallowed by \(\SI\)}%
%\label{fig:si-disallowed}%
%\end{subfigure} \\
%\end{tabular}
\hrulefill
\vspace*{-5pt}
\caption{Behaviours disallowed under different consistency models}
\label{fig:anomalies}
\vspace*{-10pt}
\end{figure*}


We now give the execution tests for widely adopted consistency models of distributed and replicated databases. 
These are summarised in \cref{fig:execution.tests}.
Following \cite{distrprinciples}, we distinguish between client-centric and data-centric consistency models. 
The former constrain the versions of keys that individual clients can observe. 
%such consistency models  
%include the session guarantees from \cite{terry1994sessions}, namely \emph{monotonic reads} (\MRd), \emph{monotonic writes} (\MW), \emph{read your writes} (\RYW) and \emph{write follows reads} (\WFR).
%The client-centric consistency model is also known as \emph{session guarantees} \cite{terry1994sessions}.
The latter impose conditions on the shape of the state of the database, in our case the structure of the kv-store.
%The data-centric consistency models include \emph{update atomic} (\UA), \emph{consistent prefix} (\CP) and \emph{serialisibility} (\SER).
%The remained models are combinations of both types, including \emph{causal consistency} (\CC), \emph{parallel snapshot Isolation} (\PSI) and \emph{snapshot Isolation} (\(\SI\)).
%Both kinds of models can be induced by execution tests. 
In \cref{sec:spec-proof} we prove that specification of consistency models using execution tests
are both sound and complete with respect to alternative specifications from the literature.
Due to space constraints, we only give examples of allowed and disallowed key-value stores for relevant consistency models. 

\paragraph{Monotonic Reads ($\MRd$).}
It ensures that read operations from subsequent transactions always return a more up-to-date versions.
For example, the key-value store of \cref{fig:mr-disallowed} is disallowed by $\MRd$.
Because the client $\cl$ first observes the latest version of $\ke$ in $\txid_{\cl}^{1}$,
then it observes the initial version of $\ke$ in $\txid_{\cl}^{2}$.
The execution test $\ET_{\MRd}$ prevents this scenario by forcing clients to always update their views to newer ones. 
%Because the versions observed by a client of a kv-store 
%are determined by the view of the former, monotonic reads can be enforced in our framework by ensuring that 
%a client can never replace its view with an older one. According to the definition of $\CMs(\_)$, 
%a client can only update its view to an older one upon committing a transaction. 
%$\ET_{\MRd}$ in \cref{fig:execution.tests} forces clients to always update their views to newer ones.

\paragraph{Monotonic Writes ($\MW$).}
It states that whenever a transaction observes the effects of a version installed by some client $\cl$,
then the transaction observes all the transactions executed by the client. 
It prevents the scenario of \cref{fig:mw-disallowed}, 
where transaction $\txid'$ observes the second version of $\ke_2$ carrying value $\val_2$, written by client $\cl$;
but it does not observe the second version of $\ke_1$ carrying value $\val_1$, previously written by the same client.
The execution test $\ET_{\MW}$ (\cref{fig:execution.tests}) ensures that, prior to executing a transaction,
the set of versions included in the view of the client must be prefix-closed with respect to the relation $\xrightarrow{\PO}$.
%The order of updates of transaction identifiers is embedded in the set of transaction identifiers, 
%and it is given by $\txid \xrightarrow{\PO} \txid' \iff \exists \cl, n,m. n < m \wedge \txid = \txid_{\cl}^{n} \wedge 
%\txid' = \txid_{\cl}^{m}$. 

\paragraph{Read Your Writes (\RYW).}
It states that a client must always be able to read any version of a key that was previously written by the same client.
This prevents the key-value store of \cref{fig:ryw-disallowed}. 
In the \cref{fig:ryw-disallowed}, the initial version of $\ke$ carries value $0$, 
and the client $\cl$ tries to increment the value of $\ke$ by $1$ twice.
For the first time, it reads the initial value $0$ and then installing a new version carrying  value $1$ within a single transaction.
However, since the client does not need to read its own writes, 
the client might read the initial value $0$ again in the second increment transaction \( \txid_\cl^2 \),
and install a new version carrying value $1$.
The Read Your Writes ($\RYW$) (\cref{fig:execution.tests}) enforces that after committing a transaction, 
a client includes all the versions it wrote.  
%A client always appends the version of a key written by  
%a transaction at the tail of the version list for such a key. Therefore, to enforce the 

\paragraph{Write Follows Reads (\WFR).}
It states that if a client \( \cl \) writes some version $\ver$ in a transaction,
following  another transaction (or in the same transaction of) who reads of some version $\ver'$, 
then a transaction may observe version $\ver$ only if it also observes $\ver'$. 
The Write Follow Reads ($\WFR$) disallows the scenario of \cref{fig:wfr-disallowed} 
where a transaction $\txid$ observes the version $\ver_2$ of $\ke_2$ carrying value $\val_2$ written by client $\cl$,
but the same transaction $\txid$ does not observe the version of $\ke_1$ carrying value $\val_1$, read by $\cl$ prior to writing $\ver$. 
The execution test $\ET_{\WFR}$ (\cref{fig:execution.tests}) prevents this scenarios 
by enforcing a view includes all the versions previous read by some client \( \cl \), 
if the view already include a write from that client \( \cl \).

\paragraph{Causal Consistency (\CC).}
Causal Consistency requires that if a client observes a version $\ver$, 
then it must also observe any version $\ver'$ from which $\ver$ potentially depends \cite{cops}. 
The dependency here means session order and write-read relation.
For session order, it means when a view includes some effect from a client, 
it must include previous effect from the same client.
For write-read relation, it means when a view includes a transaction (the versions it write),
it must include all the writes that the transaction read from.
A necessary and sufficient condition is to enforce the four session guarantees $\MRd, \MW, \RYW$ and $\WFR$ \cite{session2causal}.
Therefore, we let $\ET_{\CC} = \ET_{\MRd} \cap \ET_{\MW} \cap \ET_{\RYW} \cap \ET_{\WFR}$. 
%By \cref{prop:et.compositional}, 
%kv-stores disallowed by causal consistency are exactly the kv-stores disallowed by at least one of the 
%four session guarantees.
%However, for the sake of completeness we prefer to 
%give a definition of execution test for causal consistency after the standard definition. 
%The notion of \emph{potential dependency} $\xrightrrow{\pdep}$ between versions is defined by 
%letting $\ver \xrightarrow{\pdep} \ver'$ if  
%$\exists \txid, \txid'. \txid \in \{\WTx(\ver)\} \cup \RTx(\ver) \wedge 
%\txid' \in \{\WTx(\ver')\} \cup \RTx(\ver') \wedge \txid \xrightarrow{\SO} \txid'$: 
%this corresponds to the intuition that operations within a session potentially depends from previous operations in the same session.
%The notion of \emph{potential data dependency} between versions is given 
%by $\ver \xrightarrow{\ddep} \ver'$ if $\WTx(\ver') \in  \RTx(\ver)$.

\paragraph{Update Atomic ($\UA$).}
This consistency model has been proposed in \cite{framework-concur}, 
though we are not aware of any implementation. 
However, many implemented consistency models can be obtained by strengthening Update Atomic.
Update Atomic disallows concurrent transactions writing to the same key. 
This property is known as \emph{write conflict detection}.
For example, $\UA$ prevents the key-value store of \cref{fig:ua-disallowed},
where two transactions $\txid, \txid'$ concurrently increment the initial version of $\ke$ by $1$.
Note that this scenario generalises the one exhibited by $\RYW$, 
since we do not require $\txid, \txid'$ to be executed by the same client.
%(in fact, the kv-store to the right is allowed by causal consistency and monotonic writes).
To prevent this scenario, the execution test $\ET_\UA$ requires that 
a client $\cl$ can write to key $\ke$ in a transaction,
only if its view prior to the execution includes the last version of $\ke$.
%framework, we can compute the set of transactions that are concurrent by transaction $\txid$ 
%immediately before executing such a transaction. At the moment a client $\cl$ tries to 
%commit the effects of transaction $\txid$, then any transaction $\txid'$ that read or wrote versions not included 
%in the view of $\cl$ is concurrent to $\txid$. Following this intuition, we can enforce write 
%conflict detection by requiring that whenever a client $\cl$ wants to commit the effect of 
%transactions writing key $\ke$ in the kv-store $\hh$, then the view of $\cl$ must include all the versions of $\ke$. 
%Formally, this leads to the execution test $\ET_{\UA}$ defined in \cref{fig:execution.tests}

\paragraph{Consistent Prefix ($\CP$).}
\label{para:cp}
In centralised databases, where there is a total order in which transactions commit, 
Consistent Prefix is described by the following property: 
if a client observes the effect of a transaction $\txid$,
then it also observe the effect of any transaction $\txid'$ that commits before $\txid$.
It is difficult to formulate in key-value store,
because key-value stores do not contain the full information about the total order in which transactions committed. 
Inspired by dependency graph \cite{.....},
there are minimum observable transactions for each transaction derived from the following:
\[
    \SO  \subseteq  \VIS \qquad
    ( ( ( \SO \cup \WR ) ; \RW? )^* \cup \WW ) ; \VIS \subseteq \VIS
\]
where the \( R? \) is the reflexive closure of the relation \( R \) 
and \( R_1 ; R_2 \defeq \Setcon{(a,b)}{\exsts{c} (a,c) \in R_1 \land (c,b) \in R_2 } \) is the composition of the two relation.
The session order relation \( (\txid, \txid') \in \SO \) means the session order;
write-read relation \( (\txid, \txid') \in \WR \) means \( \txid' \) reads the write of \( \txid \);
read-write relation \( (\txid, \txid') \in \RW \) means \( \txid' \) read a old version of a key 
and \( \txid' \) installs a new version for the same key;
and \( (\txid, \txid') \in \VIS \) means the view exactly before \( \txid' \) should include all effect \( \txid \).
Given the minimum observable transactions, we can specify $\CP$. 
First, \( \SO \subseteq \VIS \) means a transaction observes all previous transactions from the same client,
and it is enforced by \( \ET_\RYW \).
Then the combination of \( \ddagger\) (\cref{fig:execution-tests}) and \( \ET_\MRd \) gives us \( ( ( ( \SO \cup \WR ) ; \RW? )^* \cup \WW ) ; \VIS \subseteq \VIS \).
Let consider a client \( \cl \) and the view \( \vi \).
Assume two transactions \( \txid, \txid' \)  such that \( \txid' \) is in the view \( \vi \) and \( \txid \toEdge{( ( ( \SO \cup \WR ) ; \RW? )^* \cup \WW )} \txid' \).
If \( \txid' \) is a transaction already observable by some previous transaction from the client \( \cl \), 
the transaction \( \txid \) must be observable by that time,
therefore by the \( \ET_\MRd \), the transaction \( \txid \) is in the current view \( \vi \).
Otherwise, if \( \txid' \) is a transaction that is first time observed by the client \( \cl \),
the \( \dagger \) predicate enforces \( \txid \) is also in the view \( \vi \).
Intuitively, the \( \CP \) disallows that a transaction observes updates in different order (\cref{fig:cp-disallowed-1}).
In \cref{fig:cp-disallowed-1}, transactions $\txid_{3}$ and \( \txid_4 \) observes updates in different order.
That is, \( \txid_3 \) observes that the update of $\ke_2$ carrying value $\val_2$ happens before the update of $\ke_1$ carrying value $\val_2$,
yet $\txid_{4}$ observes that the update of $\ke_1$ carrying value $\val_1$ happens before the update of $\ke_2$ carrying value $\val_2$. 
\ac{
    \sx{ Not sure I understand the words but the counterexample has  been ruled out by \( \ddagger \). 
         It looks like the same as observe update in different order without having a transaction actually reading. }
The second property required by $\CP$ is that at any given time, 
a client observes the effects of all the transactions that  
executed before the last transaction that such a client executed. 
This property prevents a scenario like the one depicted to the 
right: client $\cl$ does not observe the update to $\ke_2$ performed 
by $\cl'$, and $\cl'$ does not observe the update to $\ke_1$ performed 
by $\cl$. This property can be formalised by requiring that, after 
a client executes a transaction, its view is shifted to the most recent 
view of the data. The execution test $\ET_{\CP}$ is defined formally 
in \cref{fig:execution.tests}; in \cref{sec:?} we prove that our specification  
of $\CP$ using execution tests is precise. 
}

\sx{introduce transaction in a view before}
%First, clients agree on the order in which transactions install versions in a kv-store; 
%and a client $\cl$ always observes all the transactions that executed before its last transaction. 


%Recall that in our setting clients shift their view upon executing the 
%transaction: the initial view abstracts the starting point of the 
%transaction, while the final view abstracts its commit point.
%Following this intuition, we compute an 
%under-approximation $\CBef_{\CP}(\cl, \vi, \hh, \opset)$ 
%of the set of transactions that a client $\cl$ with view $u$ 
%must observe to have committed, when executing a transaction with 
%fingerprint $\opset$. The definition of $\CBef_{\CP}$ is recursive, 
%and follows an approach similar to the one proposed in \cite{laws}.
%Formally, we let $\CBef_{\CP}(\cl, \vi, \hh, \opset)$ 
%be the smallest set such that for all $i, j, j', n, m \in \Nat$, $\ke, \ke' \in \Keys$, 
%$\val \in \Val$ and $\cl' \in \Clients$:
%\[
%\begin{array}{l}
%%Base Cases
%(\otR, \ke, \val) \in  \opset \implies   \WTx(\hh(\ke, \vi)) \in \CBef_{\CP}(\cl, \vi, \hh, \opset)\\
%%(\oTW, \ke, \val) \in \opset \implies  \WTx(\hh(\ke, i)) \in \CBef_{\CP}(\cl, \vi, \hh, \opset)\\
%\txid_{\cl}^{n} \text{ appears in } \hh \implies \txid_{\cl}^{n} \in \CBef_{\CP}(\cl, \vi, \hh \opset)\\
%%Inductive Cases
%% \WR \subseteq \AR
%(\RTx(\hh(\ke, i)) \cap \CBef_{\CP}(\cl, \vi, \hh, \opset) \neq \emptyset \implies  \WTx(\hh(\ke,i)) \in \CBef_{\CP}(\cl, \vi, \hh, \opset)\\
%%\VO \subseteq \AR
%i < j \wedge \WTx(\hh(\ke,j)) \in \CBef_{\CP}(\cl, \vi, \hh, \opset) \implies  \WTx(\hh(\ke, i)) \in \CBef_{\CP}(\cl, \vi, \hh, \opset)\\
%% PO \subseteq \AR 
%m \leq n \wedge \txid_{\cl}^{n} \in \CBef_{\CP}(\cl, \vi, \hh, \opset) \implies  \txid_{\cl}^{m} \in \CBef_{\CP}(\cl, \vi, \hh, \opset)\\
%% RF;RW \subseteq \AR
%\txid \in \RTx(\hh(\ke, i)) \cap \RTx(\hh(\ke', j)) \wedge j < j' \wedge \WTx(\hh(\ke', j')) \in \CBef_{\CP}(\cl, \vi, \hh, \opset) 
%\implies \\ \hspace{20pt} \WTx(\hh(\ke, i)) \in \CBef_{\CP}(\cl, \vi, \hh, \opset)\\
%%\PO;RW \subseteq \AR 
%m < n \wedge \txid_{\cl}^{n} \in \RTx(\hh(\ke,i)) \wedge i < j \wedge \WTx(\hh(\ke, j)) \in \CBef_{\CP}(\cl, \vi, \hh, \opset) 
%\implies \\ \hspace{20pt} \txid_{\cl}^{m} \in \CBef_{\CP}(\cl, \vi, \hh, \opset)
%\end{array}
%\]
%\begin{itemize}
%\item if $(\oTR, \ke, \_) \in \opset$, then $\WTx(\hh(\ke, u)) \in \CB(\hh, \vi, \opset)$: 
%because the $\txid_{\mathsf{now}}$ reads the version of 
%$\ke$ at $\hh(\ke, \vi)$, then the transaction $\txid$ that wrote such 
%a version must commit before $\txid_{\mathsf{now}}$,
%\item for any $\txid_{\cl}^{n}$ appearing in the kv-store, $\txid_{\cl}^{n} 
%\in \CB(\cl, \vi, \hh, \opset)$: any previous transaction executed by $\cl$ 
%must commit before $\txid_{\mathsf{now}}$, 
%\item 
%\end{itemize}
%However, suppose that in the kv-store $\hh$, a client $\cl$ with view $\vi$
%wants to execute a transaction with fingerprint $\opset$. In the case 
%that $(\otR, \ke, ) \in \opset$, then we can observe the following: 
%\begin{itemize}
%\item{\color{red} note to self: in CP $\AR_{\CP} = (\PO \cup \RF \cup \VO \cup \PO;\AD \cup \RF;\AD)^{+}$}
%\item the client will read the value of $\ke$ from the version $\hh(\ke,u)$. 
%For this to be possible, the transaction $\WTx(\hh(\ke,u))$ must have committed 
%before the transaction to be executed by $\cl$, {\color{red} Note to self: base case}, 
%\item for any client $\cl'$ and index $n$, $\txid_{\cl'}^{m}$ commits before $\txid_{\cl}^{n}$ 
%for all $m < n$, {\color{red} - case $\PO \subseteq \AR_{\CP}$}
%\item for any key $\ke'$ and index $i$, the transaction $\WTx(\hh(\ke', i))$ commits 
%before all of the transactions in $\RTx(\hh(\ke', i))$, {\color{red} - case $\RF \subseteq \AR_{\CP}$}, 
%\item for any $\ke'$ and index $i$, then $\WTx(\hh(ke', j))$ commits before $\WTx(\hh(\ke', i))$ for 
%any $j < i$,  {\color{red} - case $\VO \subseteq \AR_{\CP}$}
%\item for any $\ke'$ and integers $i,n$, if $\txid_{\cl'}^{n} \in \RTx(\hh(\ke', i))$, 
%then for any $j: i < j \leq \lvert \hh(\ke') \rvert - 1$, and index $m < n$, then 
%$\txid_{\cl'}^{m}$ commits before $\WTx(\hh(\ke,j))$ {\color{red} - to be explained 
%why: intuitively if $\txid_{\cl'}^{m}$ committed after $\WTx(\hh(\ke, j))$, 
%then $\txid_{\cl}^{m}$ would start after $\WTx(\hh(\ke, j))$ committed, 
%hence it would not be able to read a former version of $\ke$. This case corresponds 
%to $\PO; \AD \subseteq \AR_{\CP}$},
%\item for any $\ke',\ke''$ and indexes $i,j$, if $\txid \in \RTx(\hh(\ke', i)) \cap 
%\RTx(\hh(\ke'', j))$, then for any $j': j < j' \leq \lvert \hh(\ke'') \rvert - 1$, 
%$\WTx(\hh(\ke',i))$ comitted before $\WTx(\hh(\ke'', j'))$ {\color{red} 
%explanation similar to the case above - this is the case $\RF ; \AD \subseteq \AR_{\CP}$}, 
%\item if $\txid$ committed before $\txid'$, and $\txid'$ committed before $\txid''$, 
%then $\txid$ committed before $\txid''$.
%\end{itemize}
%We can define a relation $\mathsf{CommitBefore}_{\CP}(\hh, \cl, \vi, \ke)$ that 
%includes all the transactions that we know must have committed prior to the 
%execution of a transaction from client $\cl$, whose view on $\hh$ is $\vi$, 
%assuming that said transaction will read value $\ke$.
%Using a technique similar to the one proposed in \cite{laws}, it is possible to prove 
%that for $\cl$ to execute safely a transaction with fingerprint $\opset$, 
%then for each $\ke$ that is read in $\opset$, the view of $\cl$ must include 
%at least the transactions in $\mathsf{CommitsBefore}_{\CP}$. Following this intuition, 
%we let 
%\[ 
%\ET_{\CP} \vdash \hh, \vi \triangleright \opset: \vi' \iff 
%\forall (\otR, \ke, \_) \in \opset.\; \forall \ke', j. \WTx(\hh(\ke', j)) \in \mathsf{CommitBefore}_{\CP}(\hh, \cl, \vi, \ke) 
%\implies j \leq \vi(\ke)
%\]

%{\color{red} the execution test as it is right now does not enforce consistent 
%prefix. An alternative would be to encode kv-stores into dependency graphs, 
%define the relation $\mathsf{CB} = 
%((\PO \cup \RF) ; \AD?) \cup \VO)^{+}$ - I know, it does not make any sense, 
%will try to explain at the meeting - and require that if $\txid$ is included 
%in a view $\vi$, then all the transactions $\txid'$ such that $\txid' \xrightarrow{\mathsf{CB}} 
%\txid$ must also be included in $\txid$. This works, but the problem is going to be how 
%to explain it to people.}
%An alternative 
%formulation is that concurrent transactions never observe updates on the kv-store 
%in different order. Consistent Prefix prevents the scenario depicted to the right: 
%transactions $\txid_4$ reads the up-to-date version of $\ke_1$ and a stale version 
%of $\ke_2$; in contrast, transaction $\txid_3$ reads a stale version of $\ke_2$ and 
%an up-to-date version of $\ke_1$.
%The execution test $\ET_{\CP}$ prevents this scenario by requiring that, immediately 
%after committing a transaction $\txid$, a client $\cl$ brings its view to point to the 
%most recent version of each key: this amount to require that the next time that $\cl$ 
%executes a transaction, it will observe at least the effects of all the transactions that 
%committed before $\txid$.
%By Looking at the structure 
%of a kv-store, it is not immediate to infer a total order in which transactions have been 
%executed (this problem has been analysed in a slightly different setting, see \cite{SIanalysis,laws} 
%for details). An equivalent definition of consistent prefix requires that different clients 
%never see updates to the kv-stores performed in different order. We can enforce this property 
%by strengthening causal consistency with the requirement that clients, 
%after committing the effects of a transaction, always shift their view to 
%the most recent version of each key.  
%The execution $\ET_{\CP}$ for consistent 
%prefix is defined in \cref{fig:execution.tests}.

\paragraph{Parallel Snapshot Isolation (\PSI)} 
Parallel Snapshot Isolation (\PSI) can be obtained by combining causal consistency with update atomic, 
$\ET_{\PSI} = \ET_{\CC} \cap \ET_{\UA}$, and $\ET_{\SI} = \ET_{\CC} \cap \ET_{\UA}$.

\paragraph{Snapshot Isolation (\SI)}
When the total order in which transactions commit is known,
SI can be specified as the weakest consistency model that guarantees both 
Consistency Prefix and Update Atomic \cite{gsi,framework-concur}. 
Yet, it is not true in our framework, since transactions are not totally ordered.
For example, the kv-store of \cref{fig:si-disallowed} is included in both $\CMs(\ET_{\CP})$ and $\CMs(\ET_{\UA})$, 
but it is forbidden by snapshot isolation in general.
The reason is \( \CMs(\ET_1 \cap \ET_2) = \CMs(\ET_1) \cap \CMs(\ET_2) \) holds 
only under some conditions of the \( \ET_1 \) and \( \ET_2 \) (more detail in \cref{thm:et-comm} \cref{sec:et-comp}).
\footnote{%
    This problem is not limited to our setting: 
    because kv-stores are isomorphic to Adya's dependency 
    graph, the same problem arises there.%
} 
We are inspired by the following constraint that has been proven satisfying \( \SI \) \cite{cerone:snapshot}:
\[
    (\SO \cup \WW) \subseteq \VIS \quad  ( (\SO \cup \WW \cup \WR) ; \RW? ) ; \VIS \subseteq \VIS
\]
where write-write relation \( (\txid, \txid') \in \WW \) means the transaction \( \txid \) installs a version for a key \( \ke \) following by \( \txid' \) installing a new version for the key \( \ke \).
The constraint \( \SO \subseteq \VIS \) coincides with \( \ET_\RYW \).
The \( \WW \subseteq \VIS \) means two transactions cannot concurrently write to the same key,
which is enforced by \( \ET_\UA \).
Let consider \( ( (\SO \cup \WW \cup \WR) ; \RW? ) ; \VIS \subseteq \VIS \).
Similar to the argument we made in Consistent Prefix (\pageref{para:cp}), 
let assume a client \( \cl \), its view \( \vi \) and two transactions \( \txid, \txid' \) such that 
\( \txid' \) is in the view \( \vi \)
and \( \txid \toEdge{(\SO \cup \WW \cup \WR) ; \RW?} \txid' \).
If \( \txid' \) is observable by any previous transaction of the client \( \cl \),
then \( \txid \) is also observable before.
By \( \ET_\MRd\), it is the case \( \txid \) is in the view \( \vi \).
If \( \txid' \) is a new transaction observed by the client \( \cl \),
the \( \dagger \) enforces that \( \txid \) should be included \( \vi \).

\ac{\color{red} This is going to be difficult to be put in words: 
both SI and CP require that the snapshot taken by clients are monotonically 
increasing (at least for transactions that write to at least one key). 
In both SI and CP, we enforce this property by computing, at the moment 
of trying to commit a transaction, a view of all the transactions that executed before 
(and hence appear in the kv-store); then we require that such views do not cross 
with the one that is being used to commit the current transaction. Here is where 
the twist happens: the way in which the relevant fragment of a view of 
a transaction is obtained is different for $\CP$ and $\SI$. 
For $\CP$, this relevant fragment is obtained 
by looking at the version reads for a transaction. If transaction $\txid$ read the 
$i$-th version of $\ke$, then we can be sure that the view $\vi_{\txid}$ 
that was used to execute $\txid$ was such that $\vi_{\txid}(\ke) = i$. 
For $\SI$ we also know that if transaction $\txid$ wrote the $i$-th version 
of key $\ke$, then because of write conflict detection, $\vi_{\txid}$ pointed 
to the previous version of $\ke$ i.e. $\vi_{\txid}(\ke) = i - 1$. 
In \cref{fig:execution.tests}, the execution test $\ET_{\SI}$ enforces three properties: 
the check on the first line mandates that  if a transaction wants to write key $\ke$, then the view of the client wishing 
to execute such a transaction must be up-to-date for that client; the check 
on the second line mandates that upon committing 
a transaction, a client shifts its view to the most up-to-date version of each 
key (this is done to ensure both $\RYW$ and $\MRd$); 
the check on the last two lines ensures that, in order to commit a transaction, the 
view of a client must not be crossing the view that was used to commit a previous 
transactions $\txid'$, at least for the objects that were accessed by $\txid'$.\\
Following a chat with Shale: it looks that there is a check on the program order missing 
here. I need to correct this.}
%We basically compute the relevant fragment of the view that we
%To overcome this problem, we place in 
%$\ET_{\SI}$ one more constraint in addition to the ones that 
%define $\ET_{\CP}$ and $\ET_{\UA}$: if a version read by transaction $\txid$ 
%is to the left of the view $\vi$ of the client wishing to update a transaction, then 
%all the versions written by $\txid$ must be to the left of $\vi$. In \cref{sec:?} 
%we prove that the execution test $\ET_{\SI}$, defined in \cref{fig:execution.tests}, 
%precisely capture SI.

\sx{put back long fork graph}
\paragraph{(Strict) Serialisibility (\SER)}
Serialisability is the strongest consistency model, 
which requires that there exists a serial or sequential schedule of transaction. 
This prevents scenarios of \cref{fig:hheap-a},
which is instead allowed by all the other execution tests that we have presented.
The execution test $\ET_{\SER}$ requires 
clients to execute transactions only when their view of the key-value store is up-to-date.

\subsection{Encoding into Abstract Executions}
It remains to prove that our specifications given using execution tests 
capture the intended consistency models. 
In this section we relate the consistency models induced by execution tests with 
the axiomatic specifications of consistency models given in terms of abstract executions 
\cite{framework-concur,laws}. 

Abstract executions are a framework originally introduced in \cite{ev-transactions} 
to capture the run-time behaviour of clients interacting with a database. In this 
formalism, two relations between transactions are introduced: the first one, \emph{visibility}, 
establishes when a transaction observes the effects of another transaction; the 
second one,  \emph{arbitration}, is used to determine the value of a key $\ke$ read by 
a transaction, in the case that it observes multiple updates to $\ke$ performed by different 
transactions. 
\begin{definition}
\label{def:absexec}
\label{def:aexec}
An abstract execution is a triple $\aexec = (\TtoOp{T}, \VIS, \AR)$, where 
\begin{itemize}
    \item $\TtoOp{T}: \TxID_{0} \parfinfun \powerset{\Ops}$ is a partial, 
finite function mapping transaction identifiers to the set of operations that they perform,
\item $\VIS \subseteq \dom(\TtoOp{T}) \times \dom(\TtoOp{T})$ is an irreflexive relation, 
called \emph{visibility}, 
\item $\AR \subseteq \dom(\TtoOp{T}) \times \dom(\TtoOp{T})$ is a strict, total order 
such that $\VIS \subseteq \AR$, and whenever $\txid_{\cl}^{n} \xrightarrow{\AR} 
\txid_{\cl}^{m}$, then $n < m$.
\end{itemize} 
\end{definition}
The set of abstract executions is denoted by $\aeset$.
Given an abstract execution $\aexec = (\TtoOp{T}, \VIS, \AR)$, we let 
$\TtoOp{T}_{\aexec} = \TtoOp{T}$, $\T_{\aexec} = \dom(\TtoOp{T})$, $\VIS_{\aexec} = \VIS$ 
and $\AR_{\aexec} = \AR$. We also let $\PO_{\aexec}(\cl) = \Setcon{(\txid_{\cl}^{n}, \txid_{\cl}^{m})}{ \cl \in \Clients 
\wedge \txid_{\cl}^{n} \in \T_{\aexec} \wedge \txid_{\cl}^{m} \in \T_{\aexec} \wedge n < m}$, and 
$\PO_{\aexec} = \bigcup\limits_{\cl \in \Clients} \PO_{\aexec}(\cl)$.
We also use the notation $(\otR, \ke, \val) \in_{\aexec} \txid$ for $(\otR, \ke, \val) \in \TtoOp{T}_{\aexec}(\txid)$, 
and similarly for write operations. 
Given an abstract execution $\aexec$, a transaction $\txid \in \T_{\aexec}$, and a key $\ke$, 
we let $\visibleWrites_{\aexec}(\ke, \txid) = \{ \txid' \mid \txid' \xrightarrow{\VIS_{\aexec}} \txid \wedge 
(\otW, \ke, \_) \in_{\aexec} \txid'\}$.

Let $\aexec_1, \aexec_2 \in \aeset$, let $\T \subseteq \T_{\aexec_1} \cap \T_{\aexec_2}$: 
$\aexec_1$ and $\aexec_2$ \emph{agree} on $\T$ if, and only if 
\[
\forall \txid, \txid' \in \T.\;\TtoOp{T}_{\aexec_1}(\txid) = \TtoOp{T}_{\aexec_2}(\txid) \wedge 
((\txid \xrightarrow{\VIS_{\aexec_1}} \txid') \iff (\txid \xrightarrow{\VIS_{\aexec_2}} \txid'))
\wedge ((\txid \xrightarrow{\AR_{\aexec_1}} \txid') \iff (\txid \xrightarrow{\AR_{\aexec_2}} \txid')).
\]
funcn
\begin{definition}
\label{def:rp}
A resolution policy $\RP$ is a function $\RP: \aeset \times \powerset{\TxID} \rightarrow \powerset{\Snapshots}$ 
such that, for any $\aexec_1, \aexec_2$ that agree on a subset of transactions $\T$, then 
$\RP(\aexec_1, \T) = \RP(\aexec_2, \T)$.

An abstract execution $\aexec$ satisfies the execution policy $\RP$ if, 
\[
\forall \txid \in \T_{\aexec}. \exists \h \in \RP(\aexec, \VIS_{\aexec}^{-1}(\txid)).\;\forall \ke \in \Keys.\forall \val \in \Val.\; (\otR, \ke, \val) \in_{\aexec} \txid 
\implies \h(\ke) = \val.
\]
\end{definition}

Throughout this report we will work mainly with the \emph{Last Write Wins} resolution policy, 
defined below. When discussing the operational semantics of transactional programs, we will also 
introduce the \emph{Anarchic} resolution policy.

\begin{definition}
\label{def:lww}
The Last Write Wins resolution policy $\RP_{\LWW}$ is defined by letting 
$\RP_{\LWW}(\aexec, \T) \defeq \{\h\}$, where
\[
\h = \lambda \ke. \text{let } \T_{\ke} = ( \T \cap \{\txid \mid (\otW,\ke, \_) \in_{\aexec} \txid\})  \text{ in }
\begin{cases}
\val_{0} &\impliedby \T_{\ke} =  \emptyset\\
\val &\impliedby (\otW, \ke, \val) \in_{\aexec} \max_{\AR_{\aexec}}(\T_{\ke})
\end{cases}
\]
\end{definition}
%
%An abstract execution satisfies the \emph{last write win} policy if, whenever $(\otR, \ke, \val) \in_{\aexec} \txid$, 
%then either $\val = \val_0$ or $(\otW, \ke, \val) \in_{\aexec} \txid'$, where $\txid' = 
%\max_{\AR_{\aexec}}(\VIS_{\aexec}^{-1}(\txid))$.

%\begin{definition}
%\label{def:anarchic}
%The Anarchic resolution policy $\RP_{\anarchic}$ is defined by letting $\RP_{\anarchic}(\aexec, \T) = \Snapshots$.
%\end{definition}
%\ac{I know that the symbol $\anarchic$ will change in the future. This is just a comment to let you know that, when it 
%will happen, I will be very sad.}

\mypar{Specification of Consistency Models via Abstract Executions.}
A consistency model can be thought as a set of abstract executions. 
We specify consistency models axiomatically: an axiomatic specification 
comprises a resolution policy (such as, for example, $\RP_{\LWW}$), 
and a set of axioms, which constrain the visibility and arbitration relations 
of abstract executions. 

\begin{definition}
An axiom $\A$ is a function from abstract executions to relations between 
transactions, $\A: \aeset \rightarrow \powerset{\TxID \times \TxID}$, 
such that whenever $\aexec_1, \aexec_2$ agree on a subset of 
transactions $\T$, then $\A(\aexec_1) \cap (\T \times \T) \subseteq \A(\aexec_2)$.
\end{definition}
\ac{Axioms of a consistency model are constraints of the form $\A(\aexec) \subseteq \VIS_{\aexec}$. 
For example, if we require $A(\aexec) = \AR_{\aexec}$, then the corresponding axiom is given 
by $\AR_{\aexec} \subseteq \VIS_{\aexec}$, thus capturing the serialisability of transactions 
(this axiom is equivalent to require that $\VIS_{\aexec}$ is a total order). The requirement on 
subsets of transactions on which abstract executions agree will be needed later, when 
we define an operational semantics of transactions where clients can append a new transaction $\txid$ 
at the tail of an abstract execution $\aexec$, which satisfies an axiom $\A$. This requirement ensures that 
we only need to check that the axiom is $\A$ is satisfied by the pre-visibility and pre-arbitration relation 
of the transaction $\txid$ in $\aexec'$. In fact, 
the resulting abstract execution $\aexec'$ agrees with $\aexec$ 
on the set $\T_{\aexec}$: in this case we'll note that we can rewrite 
$\A(\aexec') = \A(\aexec') \cap ((\T_{\aexec} \times \T_{\aexec}) ) \cup (\T_{\aexec} \times \{\txid\}))$. Then
 $\A(\aexec') \cap ((\T_{\aexec} \times \T_{\aexec})) \subseteq \A(\aexec) \cap (\T_{\aexec} \times \T_{\aexec}) 
 \subseteq \VIS_{\aexec} \cap (\T_{\aexec} \times \T_{\aexec}) \subseteq \VIS_{\aexec'}$, 
hence we only need to check that $\A(\aexec') \cap (\T_{\aexec} \times \{\txid\}) \subseteq \VIS_{\aexec'}$.}
We say that an abstract execution $\aexec$ satisfies an axiom $\A$, written 
$\aexec \models \A$, if 
$\A(\aexec) \subseteq \VIS_{\aexec}$. An abstract execution $\aexec$ 
satisfies a set of axioms $\Ax$, written $\aexec \models \Ax$,  if $\aexec \models 
\A$ for all $\A \in \Ax$. 

An axiomatic specification of a consistency model is given by a pair $(\RP, \Ax)$, 
where $\RP$ is a resolution policy and $\Ax$ is a set of axioms. An 
abstract execution $\aexec$ satisfies the consistency model, 
written $\aexec \models (\RP, \Ax)$ if it satisfies its individual components. 
The set of abstract executions induced by an axiomatic specification is given 
by $\CMa(\RP, \Ax) = \{ \aexec \mid \aexec \models (\RP, \Ax)\}$.

\mypar{Compatibility of kv-Stores and Abstract Executions.}
Throughout this section we focus on abstract executions that satisfy the 
resolution policy $\RP_{\LWW}$.
Such abstract executions can be converted into dependency graphs \cite{SIanaysis,laws}, 
which in turn are isomorphic to kv-stores (\cref{thm:kv2graph}). For the sake of completeness, 
we include the definition of the dependency graph and kv-store induced by an abstract execution below: 

\begin{definition}
\label{def:aexec2graph}
Given an abstract execution $\aexec$ that satisfies the last write wins policy,
the dependency graph $\graphof(\aexec) \defeq (\TtoOp{T}_{\aexec}, \RF_{\aexec}, 
\VO_{\aexec}, \AD_{\aexec})$ is defined by letting
\begin{itemize}
\item $\txid \xrightarrow{\RF_{\aexec}(\ke)} \txid'$ if and only if 
$\txid = \max_{\AR_{\aexec}}(\visibleWrites_{\aexec}(\ke, \txid'))$, 
\item $\txid \xrightarrow{\VO_{\aexec}(\ke)} \txid'$ if and only 
$\txid, \txid' \in_{\aexec} (\otW, \;\ke: \_)$ 
and $\txid \xrightarrow{\AR_{\aexec}} \txid'$,
\item $\txid \xrightarrow{\AD_{\aexec}(\ke)} \txid'$ if and only if either 
$(\otR, \ke, \_) \in_{\aexec} \txid, (\otW, \ke, \_) \in_{\aexec} \txid'$ and 
whenever $\txid'' \xrightarrow{\RF_{\aexec}(\ke)} \txid$, 
then $\txid'' \xrightarrow{\VO_{\aexec}(\ke)} \txid'$.
\end{itemize}
\end{definition}
Note that each abstract execution $\aexec$ determines a kv-store 
$\hh_{\aexec}$, as per \cref{def:aexec2graph} and \cref{thm:kv2graph}. 
Let $\hh$ be the unique kv-store such that $\Gr_{\hh} = \graphof(\aexec)$: then 
$\hh_{\aexec} = \hh$. As we discuss later in this Section, this mapping 
$\hh_{(\stub)}$ is not a bijection, in that several abstract executions may be 
encoded in the same kv-store. This is because kv-stores abstract the total 
arbitration order of transactions away.

%On the other hand, 
%abstract executions are endowed with more information than kv-stores, namely the total 
%arbitration order of transactions: for our purposes, this may be considered to be the order in which transactions 
%are executed by all clients.

%Although abstract executions contain more information than kv-stores, 
%we can 
Next, we define a notion of \emph{compatibility} between abstract executions and 
kv-stores. Roughly speaking, 
this notion is based on the intuition that clients can make observations over kv-stores and abstract 
executions, in terms of snapshots. In kv-stores, 
observations are snapshots induced by views of clients over the kv-store. 
In abstract executions, observations correspond to the snapshots induced by the subsets of
of transactions of the abstract executions, according to some resolution policy (in this 
case, $\RP_{\LWW}$): this approach is analogous to the one used by 
opetation contexts in \cite{repldatatypes}. Thus a kv-store $\hh$ is 
compatible with an abstract execution $\aexec$ if any observation 
made on $\hh$ can be replicated by an observation made on 
$\aexec$, and vice-versa. 

\begin{definition}
\label{def:compatible}
Given a key-value store $\hh$,
an abstract execution $\aexec$ is compatible with $\hh$, written 
$\aexec \compatible \hh$, if and only if there exists a  mapping 
$f: \powerset{\T_{\aexec}} \rightarrow \Views(\hh)$
such that  
\begin{itemize}
\item for any subset $\T \subseteq \T_{\aexec}$, then $\RP_{\LWW}(\aexec, \T) = \{\snapshot(\hh, f(\T))\}$; 
\item for any view $\vi \in \Views(\hh)$, there exists a subset $\T \subseteq \T_{\aexec}$ 
such that $f(\T) = \vi$, and $\RP_{\LWW}(\aexec, \T) = \{\snapshot(\hh_{\aexec}, \vi)\}$.
\end{itemize}
\end{definition}

The compatibility relation $\simeq$ includes the encoding from abstract executions to kv-stores, 
as formalised by the next result.
\begin{theorem}
\label{thm:aexec2kv.compatible}
For any abstract execution $\aexec$ that satisfies the last write wins policy, $\aexec \compatible \hh_{\aexec}$.
\end{theorem}

The proof of Theorem \ref{thm:aexec2kv.compatible} requires defining a mapping 
$\getView(\aexec, \stub)$ from $\powerset{\T_{\aexec}}$ to $\Views(\hh_{\aexec})$ 
that satisfies the constraint of \cref{def:compatible}.
The function $\getView(\aexec, \stub)$ is defined by letting 
$\getView(\aexec, \T) = \lambda \ke. \{0\} \cup \{ i \mid i \in [\lvert \hh_{\aexec} \rvert - 1] \wedge 
\WTx(\hh_{\aexec}(\ke, i)) \in \T\}$.
We give details of the proof of the Theorem in \cref{app:aexec2kv}.
\begin{proposition}
\label{prop:getview.valid}
For any abstract execution $\aexec$, and $\T \subseteq \T_{\aexec}$, 
$\getView(\aexec, \T) \in \Views(\hh_{\aexec})$.
\end{proposition}

\mypar{Relating $\ET$-traces to Abstract Executions.}
Let $,\ET_\top$ be the most permissive execution test, that is 
$\ET_\top \vdash (\hh, \vi) \triangleright \opset: \vi'$ for any $\hh, \vi, \opset, \vi'$ 
such that whenever $\vi(\ke) \neq \vi'(\ke)$ then either $(\otW, \ke, \_) \in \opset$ 
or $(\otR, \ke, \_) \in \opset$.
{\color{red} I forgot this last constraint in the latest version of the document, definitions 
and proofs of theorems that follow must be re-factored to take the constraint into account.}
In this Section, we relate $\ET_{\top}$-traces to abstract executions that satisfy the last write wins 
resolution policy. Specifically, we prove the following: 
\begin{theorem}
\label{thm:kvtrace2aexec}
Let $\tr$ be a $\ET_{\top}$-trace; there exists a set of abstract executions $\aeset({\tr})$ 
such that, for any $\hh \in \aeset(\tr)$, $\lastConf(\tr) = (\hh, \_)$.

Let $\aexec$ be an abstract execution that satisfies the last write wins resolution policy. 
There exists a set of $\ET_{\top}$-traces $\KVtrace({\aexec})$ in normal form, 
such that for any $\tr \in \KVtrace({\aexec})$, $\lastConf(\tr) = (\hh_{\aexec}, \_)$. 
%Furthermore, for any $\aexec_{\tr_{\aexec}} = \aexec$.

{\color{red} Given a $\ET_{\top}$-trace $\tr$ and a kv-store $\aexec \in \aeset(\tr)$, then 
$\tr \in \KVtrace(\aexec)$. Given an abstract execution $\aexec$ that satisfies 
the last write wins policy and a $\ET_{\top}$-trace $\tr \in \KVtrace(\hh)$, 
then $\aexec \in \aeset(\tr)$. }
\end{theorem}
\begin{proof}
    See \cref{sec:kvtrace2aexec}.
\end{proof}

\begin{corollary} 
\label{cor:kvtrace2aexec}
$\CMs(\ET_{\top}) = \{\hh_{\aexec} \mid \aexec \text{ satisfies } \RP_{\LWW}\}$.
\end{corollary}

The last statement in \cref{thm:kvtrace2aexec} implies that there is a \emph{Galois connection}
between the set of $\ET_{\top}$-traces, and the set of abstract executions that satisfy the 
last write wins policy. The lower and upper adjoints of this connection are the 
lifting of the functions $\aeset(\cdot)$ and $\KVtrace(\cdot)$ to sets of $\ET_{\top}$-traces 
and abstract executions, respectively. Note however that these two sets are not isomorphic: 
when converting a set of abstract executions into kv-traces, we abstract away the 
pairs $\txid \xrightarrow{\VIS} \txid'$ in the visibility relation where $\txid$ is a read-only 
transaction. When converting a $\ET_{\top}$-trace into a set of abstract executions, 
we (partially) lose the information about the view that a clients has, immediately after it executed a transaction.

%\ac{Actually it is the case that each abstract execution defines several kv-traces in normal form. 
%In particular, the view of a client after executing a transaction should be irrelevant, as long as it's 
%included in the view associated with the set of transactions that are seen by the next transaction 
%executed by the same client, as per the definition of $\getView$.}
%\ac{Random though of 10:30pm - so maybe not so random... the comment above is because in the construction of 
%$\aexec_{\tr}$ I define, for a given transaction $\txid$, $\VIS_{\aexec_{\tr}}^{-1}(\txid)$ to contain 
%exactly the transactions that wrote a version included in the view of the client executing $\txid$, at the 
%moment it executed $\txid$. However, here I could be loose and allow also read-only transactions 
%to be included in $\VIS_{\aexec_{\tr}}^{-1}(\txid)$ - thus a trace would induce a set of 
%abstract executions, rather than a single abstract execution. I need to think about it more carefully. }


%A consequence of Theorem \ref{thm:kvtrace2aexec} is that $\ET_{\top}$-traces in normal form are isomorphic 
%to abstract executions that satisfy the last write wins policy.
%\ac{Not true. I thought that $\aexec_{\tr_{\aexec}} = \aexec$ for any $\aexec$, but this is not 
%the case. If in $\aexec$ there is a read-only transaction $\txid$ and $\txid \xrightarrow{\VIS_{\aexec}} \txid'$, 
%then it won't be the case that $\txid \xrightarrow{\VIS_{\aexec_{\tr_{\aexec}}}} \txid'$. This will have some 
%consequences when proving the correctness of axiomatic specifications of consistency models.}
%\ac{Random though of 10:45pm - following the previous random thought: given an abstract execution $\aexec$, 
%I define a set of traces $\KVtrace(\aexec)$. Given a trace $\tr$, I define a set of abstract executions $\aeset(\tr)$. 
%For any given $\aexec$ and $\tr \in \KVtrace(\aexec)$, $\aexec \in \aeset(\tr)$. More specifically, I can lift 
%$\aeset$ and $\traces$ to operate on sets of traces and abstract executions $\mathbb{X}, \mathbb{T}$, respectively. 
%For any $\mathbb{X}$ and $\mathbb{T}$ such that $\aeset(\mathbb{T}) \subseteq \mathbb{X}$, then 
%$\mathbb{T} \subseteq \traces(\mathbb{X}).$ It follows that $(\aeset, \traces)$ form a Galois Connection.}

Note that our definition of views (\cref{def:views}) is critical to the validity of Theorem \ref{thm:kvtrace2aexec}, 
as the following example shows. 

\begin{figure}
\begin{center}
\hrule
\begin{tabular}{@{}c @{\qquad}|@{\qquad} c@{}}
\begin{tikzpicture}[scale=0.85, every node/.style={transform shape}]
%\draw[help lines] grid(6,4);

\node (t1wx) {$(\otW, \ke_1, \val_1)$}; 
\path (t1wx.east) + (0.2,0) node[anchor = west] (t1wy) {$(\otW, \ke_2, \val_1)$};
\path (t1wx.south east) + (0.1, -1) node[anchor = north] (t2wx) {$(\otW, \ke_1, \val_2)$};
\path (t1wy.south east) + (1,-1) node[anchor = north west] (t3rx) {$(\otR, \ke_1, \val_2$)};
\path (t3rx.east) + (0,0) node[anchor=west] (t3ry) {$(\otR, \ke_2, \val_0)$};

\begin{pgfonlayer}{background}
\node[background, fit=(t1wx) (t1wy)] (t1) {};
\node[background, fit= (t2wx)] (t2) {};
\node[background, fit= (t3rx) (t3ry)] (t3) {};

\path(t1.north) node[anchor=south] (t1lbl) {$\txid_{\cl_1}^{1}$};
\path(t2.south) node[anchor=north] (t2lbl) {$\txid_{\cl_2}^{1}$};
\path(t3.east) node[anchor=west] (t3lbl) {$\txid_{\cl_3}^{1}$};

\path[->]
(t1.south) edge node[right] {$\AR$} (t2.north)
(t2.east) edge node[above] {$\AR, \VIS$} (t3.west);
\end{pgfonlayer}
\end{tikzpicture}
&
\begin{tikzpicture}
\begin{pgfonlayer}{foreground}
%Uncomment line below for help lines
%\draw[help lines] grid(5,4);

%Location x
\node(locx) {$\ke_1 \mapsto$};

\matrix(versionx) [version list]
    at ([xshift=\tikzkvspace]locx.east) {
    {a} & $\txid_0$ & {a} & $\txid_{\cl_1}^{1}$ & {a} & $\txid_{\cl_2}^{1}$\\
    {a} & $\emptyset$ & {a} & $\emptyset$ & {a} & $\left\{\txid_{\cl_3}^{1}\right\}$\\
};
\tikzvalue{versionx-1-1}{versionx-2-1}{locx-v0}{$\val_0$};
\tikzvalue{versionx-1-3}{versionx-2-3}{locx-v1}{$\val_1$};
\tikzvalue{versionx-1-5}{versionx-2-5}{locx-v2}{$\val_2$};

%Location y
\path (locx.south) + (0,\tikzkeyspace) node (locy) {$\ke_2 \mapsto$};
\matrix(versiony) [version list]
   at ([xshift=\tikzkvspace]locy.east) {
 {a} & $\txid_0$ & {a} & $\txid_{\cl_1}^{1}$ \\
  {a} & $\left\{\txid_{\cl_3}^1\right\}$ & {a} & $\emptyset$\\
};

\tikzvalue{versiony-1-1}{versiony-2-1}{locy-v0}{$\val_0$};
\tikzvalue{versiony-1-3}{versiony-2-3}{locy-v1}{$\val_1$};

\end{pgfonlayer}
\end{tikzpicture}
\end{tabular}
\end{center}
\hrule
\caption{An example of abstract execution and associated kv-store.}
\label{fig:aexec.noncontiguous}
\end{figure}

\begin{example}
\label{ex:badviews}
At an earlier stage of this research, views of kv-stores were defined to include only contiguous sets of versions 
for any key $\ke$: that is, for given a kv-store $\hh$, we required from a view $\vi \in \Views(\hh)$ that 
$\vi(\ke) = [i]$ for some $i: 0 < \lvert \hh(\ke) \rvert$, in addition to the constraints posed by \cref{def:views}. 
In this setting, the most permissive execution test $\ET$ is the following: 
\[
\left(\ET \vdash (\hh, \vi) \triangleright \opset : \vi' \right) \iff \forall \ke.\;\exists i_{\ke}, j_{\ke}.\; \vi(\ke) = [ i ] \wedge \vi'(\ke) = [ j ].
\]

Consider the abstract execution $\aexec$ depicted in Figure \ref{fig:aexec.noncontiguous}, 
and the associated kv-store $\hh_{\aexec}$, depicted to the right in the same Figure.
We argue that $\hh_{\aexec} \notin \CMs(\ET)$. According to \cref{thm:et.normalform} it suffices to prove 
that there exists no $\ET$-trace that is in normal form, and whose last configuration is $(\hh_{\aexec} ,\_)$. 
Suppose that such an $\ET$-trace $\tr$ existed: because we require that the last configuration in $\tr$ is 
$(\hh_{\aexec}, \_)$, then it must be the case that in $\tr$ transaction $\txid_{\cl_1}^{1}$ is executed 
before $\txid_{\cl_2}^{1}$, and the latter is executed before $\txid_{\cl_3}^1$. This is because 
$\txid_{\cl_2}^{1}$ installs a newer version of $\ke_1$ than the one read by $\txid_{\cl_2}^1$, 
and $\txid_{\cl_3}^{1}$ reads the version of $\ke_1$ installed by $\txid_{\cl_2}^{1}$. Because, 
$\tr$ is in normal form, we have the following: 
\[
\begin{array}{ll}
\tr = & (\hh_{0}, \viewFun_{0}) \xrightarrowtriangle{(\cl_1, \varepsilon)}_{\ET} (\hh_0, \viewFun_0') 
\xrightarrowtriangle{(\cl_1, \opset_1)}_{\ET} (\hh_1, \viewFun_1) \xrightarrowtriangle{(\cl_2, \varepsilon)}_{\ET} \\
& (\hh_1, \viewFun_1') \xrightarrowtriangle{(\cl_2, \opset_2)}_{\ET} (\hh_2, \viewFun_2) \xrightarrowtriangle{(\cl_3, \varepsilon)}_{\ET} 
(\hh_2, \viewFun_2')\xrightarrowtriangle{(\cl_3, \opset_3)}_{\ET} (\hh_3, \viewFun_3),
\end{array}
\]
where $\opset_1 = \{(\otW, \ke_1, \val_1), (\otW, \ke_2, \val_1)\}$, $\opset_2 = \{(\otW, \ke_1, \val_2)\}$, 
$\opset_3 = \{(\otR, \ke_1, \val_2), (\otR, \ke_2, \val_0)\}$,\\ $\hh_1 = \updateKV(\hh_0, \cl_1, \_, \opset_1)$, 
$\hh_2 = \updateKV(\hh_1, \cl_2, \_, \opset_2)$, and $\hh_3 = \updateKV(\hh_2, \cl_3, \viewFun_2'(\cl_3), \opset_3)$ 
for a suitable $\viewFun_2'$.
Furthermore, we require that $\hh_3 = \hh_{\aexec}$, which is possible only if $\viewFun_2'(\cl_3) = [\ke_1 \mapsto [2], \ke_2 \mapsto [0] ]$ 
(recall that by definition of $\ET$, it must be that for any key $\ke$, $\viewFun_2'(\cl)(\ke) = [i_{\ke}]$ for some $i_{\ke} \geq 0$); 
this is because, in $\hh_3 = \hh_{\aexec}$, transaction $\txid_{\cl}^3$ reads the $2$-nd version of $\ke_1$ and the $0$-th version of 
$\ke_2$. It remains to note that the view $\viewFun_{2}'(\cl_3)$ is not atomic w.r.t. $\hh_2$, causing a contradiction: 
in fact, we have that $\WTx(\hh_{2}(\ke_1, 1) = \txid_{\cl_1}^{1} = \WTx(\hh_2(\ke_2, 1))$. However, 
$1 \in \viewFun_{2}'(\cl_3)(\ke_1)$, but $1 \notin \viewFun_{2}'(\cl_3)(\ke_2)$: the view $\viewFun_{2}'(\cl_3)$ 
contains the update of $\ke_1$ performed by $\txid_{\cl_1}^{1}$, but not the update of $\ke_2$ performed by the 
same transaction.
\end{example}

We now turn our attention to the proof of \cref{thm:kvtrace2aexec}. 
The first step is that of constructing, given a $\ET_{\top}$-trace $\tr$, 
a set of abstract execution $\aeset(\tr)$ such that, for any $\aexec \in \aeset(\tr)$, 
the last configuration of $\tr$ is $(\hh_{\aexec}, \_)$. 
%In practice, we map each $\ET_{\top}$-trace 
%$\tr$ into an abstract execution $\aexec_{\tr}$, and a function $\visibleTx: \Clients \parfun \T_{\aexec}$. 
%The latter are to abstract execution what view functions are to kv-stores in configurations: they keep 
%track of what transactions would be observed by a client, if it were to execute a transaction. 
The definition of $\aeset(\aexec)$ is natural, and is given inductively below: 

\begin{definition}
\label{def:kvtrace2aexec}
Let $\hh$ be a kv-store, and $\vi$ be a view. 
Define $\Tx(\hh, \vi) \defeq \Setcon{ \WTx(\hh(\ke, i)) }{ \ke \in \Keys \wedge i \in \vi(\ke) }$. 
Let $\aexec_0 = ( [ ], \emptyset, \emptyset)$. Given an abstract execution $\aexec$, a set of transactions  
$\T \subseteq \T_{\aexec}$, a transaction identifier $\txid$ and a set of operations $\opset$, we let 
\[
\extend(\aexec, \txid, \T, \opset) \defeq
\begin{cases}
\textit{undefined} &\impliedby \txid = \txid_{0}\\
(\TtoOp{T}_{\aexec} \uplus \Set{\txid_{\cl}^{n} \mapsto \opset}, \VIS', \AR') & \impliedby \txid = \txid_{\cl}^{n} \wedge {}\\
& \VIS' = \VIS_{\aexec} \uplus \{(\txid', \txid) \mid \txid \in \T)\} \wedge {}\\
& \AR' = \AR_{\aexec} \uplus \{(\txid', \txid) \mid \txid' \in \T_{\aexec})\}\\
\end{cases}
\]

Given a $\ET_{\top}$ trace $\tr$, we let $\lastConf(\tr)$ to be the last configuration appearing in 
$\tr$. The set of abstract executions $\aeset(\tr)$ is defined as the smallest set such that:
\begin{itemize}
\item $\aexec_{0} \in \aeset((\hh_{0}, \viewFun_{0}))$, 
\item if $\aexec \in \aeset(\tr)$, then $\aexec \in \aeset\left(\tr \xrightarrowtriangle{(\cl, \varepsilon)}_{\ET_{\top}} (\hh, \viewFun) \right)$, 
\item if $\aexec \in \aeset(\tr)$, then $\aexec \in \aeset\left(\tr \xrightarrowtriangle{(\cl, \emptyset)}_{\ET_{\top}} (\hh, \viewFun) \right)$, 
\item let $(\hh', \viewFun') = \lastConf(\tr)$; 
    if $\aexec \in \aeset(\tr)$, $\opset \neq \emptyset$, and $\T = \Tx(\hh, \viewFun'(\cl)) \cup \T_{rd}$ where \( \T_{rd} \) 
    is a set of \emph{read-only transactions} such that $\forall \txid' \in \T_{rd}, \ke, \val.\; (\otW, \ke, \val) \notin_{\aexec} \txid'$.
    Then for the transaction $\txid$ that is a transaction appearing in $\lastConf(\tr)$ but not in $\hh$, then  
$\extend(\aexec, \txid, \T, \opset) \in \aeset\left(\tr \xrightarrowtriangle{(\cl, \opset)}_{\ET_{\top}} (\hh, \viewFun) \right)$.
\end{itemize}
\end{definition}

\begin{proposition}
\label{prop:kvtrace2aexec}
For any $\ET_{\top}$-trace $\tau$, and $\aexec \in \aeset(\tau)$ is an abstract execution that 
satisfies the last write wins policy.
Furthermore, $\hh_{\aexec} = \hh$ where $(\hh, \stub) = \lastConf(\tau)$.
\end{proposition}
\begin{proof}
    See \cref{sec:kvtrace2aexec-under-lww}.
\end{proof}

Next, we show to construct, given an abstract execution $\aexec$, 
a set of $\ET_{\top}$-traces $\KVtrace(\ET_{\top}, \aexec)$ in normal form such that for any 
$\tr \in \KVtrace(\ET_{\top}, \aexec)$, $\lastConf(\tr) = (\hh_{\aexec}, \_)$. 
To define the function $\KVtrace(\ET_{\top}, \_)$ formally, we first provide a principle 
to reason about abstract executions inductively. 
%\begin{proposition}
%Let $\aexec$ be an abstract execution. Then either $\T_{\aexec} = \emptyset$, 
%or $\aexec = \extend(\aexec', \VIS^{-1}_{\aexec}(\txid), \txid, \TtoOp{T}_{\aexec}(\txid))$ 
%for some $\aexec'$ such that $\T_{\aexec'} = \T_{\aexec} \setminus \txid$, and $\txid = \max_{\AR}(\T_{\aexec})$.
%\end{proposition}
\begin{definition}
\label{def:aexec.inductive}
Let $\aexec$ be an abstract execution, let $n = \lvert \T_{\aexec} \rvert$, and let 
$\{\txid_{i}\}_{i=1}^{n} \subseteq \T_{\aexec}$ be such that for any $i=1,\cdots,n-1$, 
$\txid_{i} \xrightarrow{\AR_{\aexec}} \txid_{i+1}$. 
For $i = 0,\cdots, n-1$, define 
\[
\begin{array}{lll}
\cut(\aexec, 0) & \defeq & ([], \emptyset, \emptyset)\\
\cut(\aexec , i+1) & \defeq & \extend(\cut(\aexec, i), \txid_{i+1}, \VIS^{-1}_{\aexec}(\txid_{i+1}), \TtoOp{T}_{\aexec}(\txid_{i+1}))
\end{array}
\]
\end{definition}

\begin{proposition}
\label{prop:aexec.inductive}
For any abstract execution $\aexec$, $\aexec = \cut(\aexec, \lvert \T_{\aexec} \rvert)$.
\end{proposition}
\begin{proof}
    See \cref{sec:aexec-inductive}.
\end{proof}

\begin{definition}
\label{def:aexec2kvtrace}
Given an abstract execution $\aexec$, a client $\cl$ and an integer $i=0,\cdots, \lvert \aexec \rvert$, 
define $\nextTx(\aexec, \cl, i) \defeq \min_{\AR_{\aexec}}\{\txid_{\cl}^{j} \mid \txid_{\cl}^{n} \notin \T_{\cut(\aexec, i)}\}$. 
Note that $\nextTx(\aexec, \cl, i)$ could be undefined. 
Let $\Clients(\aexec) \defeq \{\cl \mid \exists n.\;\txid_{\cl}^{n} \in \T_{\aexec}\}$.

Given an abstract execution $\aexec$ and an integer $i = 0,\cdots, \lvert \T_{\aexec} \rvert$, let 
$\KVtrace(\ET_{\top}, \aexec, i)$ be the smallest set such that 
\begin{itemize}
\item 
$(\hh_{0}, \lambda \cl \in \Clients(\aexec). \lambda \ke.\{0\}) \in \KVtrace(\ET_{\top}, \aexec, 0)$, 
\item suppose that $\tr \in \KVtrace(\ET_{\top}, \aexec, i)$ for some $i = 0,\cdots, \lvert \T_{\aexec} \rvert - 1$.  
Let
\begin{itemize} 
\item $\txid = \min_{\AR_{\aexec}}(\T_{\aexec} \setminus T_{\cut(\aexec, i)})$, 
\item  $\cl, n$ be such that $\txid = \txid_{\cl}^{n}$, 
\item  $\vi = \getView(\aexec, \VIS^{-1}_{\aexec}(\txid_{\cl}^{n}))$, 
\item $\vi' = \getView(\aexec, \T)$, where $\T$ is an arbitrary subset of $\T_{\aexec}$ if 
$\nextTx(\aexec, \cl, i+1)$ is undefined, or is such that 
$\T \subseteq (\AR_{\aexec}^{-1})?(\txid) \cap \VIS^{-1}_{\aexec}(\nextTx(\cl, i+1))$, 
\item $\opset = \TtoOp{T}_{\aexec}(\txid)$, 
\item $(\hh_{\tr}, \viewFun_{\tr}) = \lastConf(\tr)$, 
\item $\hh = \updateKV(\hh_{\tr}, \vi, \txid, \opset)$.
\end{itemize}
Then
\[
\big( \tr \xrightarrowtriangle{(\cl, \varepsilon)}_{\ET_{\top}} (\hh_{\tr}, \viewFun_{\tr}\rmto{\cl}{\vi}) 
\xrightarrowtriangle{(\cl, \opset)}_{\ET_{\top}} (\hh, \viewFun_{\tr}\rmto{\cl}{\vi'}) \big) \in \KVtrace(\ET_{\top}, \aexec, i+1)
\]
\end{itemize}

Finally, we let $\KVtrace(\ET_{\top}, \aexec) \defeq \KVtrace(\ET_{\top}, \aexec, \lvert \T_\aexec \rvert)$.
\end{definition}

\begin{proposition}
\label{prop:aexec2kvtrace}
Let $\aexec$ be an abstract execution that satisfies $\RP_{\LWW}$, 
and let $\tr \in \KVtrace(\ET_{\top}, \aexec)$. Then $\lastConf(\tr) = (\hh_{\aexec}, \_)$ and $\hh_{\aexec} \in \CMs(\ET_{\top})$. 
\end{proposition}
\begin{proof}
    See \cref{sec:aexec2kvtrace}.
\end{proof}

%\mypar{Specification of Consistency Models via Abstract Executions.}
%In the abstract execution framework, specifications of consistency models are given by placing \emph{axioms} that the 
%arbitration and visibility relations of abstract executions must satisfy. Such axioms usually 
%have the form $\mathscr{F}(\aexec) \subseteq \VIS_{\aexec}$, 
%where $\mathscr{F}$ maps abstract executions into relations between transactions \cite{laws}. 
%For example, let $[(\otW, \ke, \_)]_{\aexec} = \{(\txid, \txid) \mid (\otW, \ke, \_) \in \TtoOp{T}(\T_{\aexec}(\txid) \}$. 
%Then (strong session) snapshot isolation can be described as the set of abstract executions $\aexec$ that 
%satisfy the axioms $\PO_{\aexec} \subseteq \VIS_{\aexec}$, $\AR_{\aexec} ; \VIS_{\aexec} \subseteq \VIS_{\aexec}$, 
%and for each $\ke$, $[(\otW, \ke, \_)]_{\aexec} ; \AR_{\aexec} ; [(\otW, \ke, \_)]_{\aexec} \subseteq \VIS_{\aexec}$. 
%Actual specifications of consistency models using this formalism are outside the scope of this article, and we 
%refer the reader to \cite{laws} for a comprehensive list of axioms for specifying different consistency models.
%
%We show that there is a Galois connection between the set of  abstract executions that satisfy 
%the last write win resolution policy, and the set of \emph{kv-traces}. 
%These are sequences of the form 
%\[ 
%\conf_0 \xrightarrowtriangle{\_}_{\ET_{\bot}} \cdots \xrightarrowtriangle{\_}_{\ET_{\bot}} \conf_n.
%\]
%Let then $\aexec$ be an abstract execution that satisfies the last write win policy. We now show 
%how to construct the kv-trace $\KVtrace(\aexec)$.
%\begin{definition}
%Let $\aexec$ be an abstract execution, and let $\txid^{0} \xrightarrowtriangle{\AR_{\aexec}} \txid^{n-1} \xrightarrow{\AR_{\aexec}} \txid^{n}$. 
%We recursively define $\KVtrace(\aexec, m)$ for all $m \leq n$, and we let $\KVtrace(\aexec) = \KVtrace(\aexec, n)$. 
%For $m = 0$, we let $\KVtrace(\aexec, m) = \hh_0$, where $\hh_0 = \lambda \ke. (\val_0, \txid_0, \emptyset)$. 
%Assuming that $\KVtrace(\aexec, m) = \conf_0 \xrightarrowtriangle{(\cl^{0}, \vi^{0},\opset^{0})}_{\ET_{\bot}} \cdots \xrightarrowtriangle{
%(\cl^{m-1}, \vi^{m-1}, \opset^{m-1})}_{\ET_{\bot}} \conf_{m}$ 
%has been defined for some $m < n$, consider the set of transactions $\VIS^{-1}(\txid_{m})$. Let 
%$\vi^{m} = \lambda \ke. \lvert \VIS^{-1}(\ke) \cap \{ \txid \mid (\otW, \ke, \_) \in_{\aexec} \txid \} \rvert$ 
%{\color{red} basically I am counting the number of versions of that $\txid_{m}$ sees, for each $\ke$.}, 
%and define $\hh_{m+1} = \updateKV(\hh_{m}, \vi, \txid, \opset)$. Finally, suppose that 
%$\txid^{m} = \txid_{\cl}^{\_}$ for some $\cl$, and let $m: m < m' \leq n$ be the smallest 
%index (if any) such that $\txid_{m'} = \txid_{\cl}^{\_}$. Define $\vi'_{m} = \lambda \ke. \lvert 
%\VIS^{-1}(\txid_{m'}) \cap (\AR^{-1}(\txid_{m}) \cup \{\txid_m\}) \cap \{ \txid \mid (\otW, \ke, \_) \in_{\aexec} \txid \} \rvert$. 
%If such an $m'$ does not exist, then let $\vi'_{m} = \lambda \ke. \lvert \hh_{m+1}(\ke) \rvert -1$. 
%{\color{red} basically I select the final view after committing $\txid_{m}$ by looking at 
%the versions of each $\ke$ that are already present in the kv-store at the moment 
%$\txid_{m}$ committed, and which are going to be observed by the next transaction in the same session. If there 
%are no future tranactions for the current session, just shift the view to the end, it's never going to be used again.} 
%Then we let $\conf_{m+1} = \hh_{m+1}, \viewFun_{m}\rmto{\cl}{\vi'_{m}}$, and define 
%\[
%\KVtrace(\aexec, m) = \conf_0 \xrightarrowtriangle{(\cl^{0}, \vi^{0},\opset^{0})}_{\ET_{\bot}} \cdots \xrightarrowtriangle{
%(\cl^{m-1}, \vi^{m-1}, \opset^{m-1})}_{\ET_{\bot}} \conf_{m} \xrightarrowtriangle{(\cl, \vi^{m}, \TtoOp(\txid_{m})} 
%\conf^{m+1}.
%\]
%\end{definition}

\section{Operational Semantics of Key-Value Stores}

We define a simple programming language for client programs interacting with a kv-store.
Clients may only interact with a kv-store by issuing read and write requests via transactions. 
For simplicity, we abstract away from aborting transactions:
rather than assuming a transaction may abort due to a violation of the consistency guarantees given by the kv-store,
we only allow a transaction to execute if its effects are guaranteed to not violate the consistency model. 
This emulates the setting in which clients always restart a transaction if it aborts.


\subsubsection{Programming Language}

\emph{A program} \( \prog \) comprises a finite number of clients,
where each client is associated with a unique identifier \( \cl \in \Clients \), 
and executes a sequential \emph{command} $\cmd$, given by the following grammar:
\[
\begin{array}{@{} r @{\hspace{2pt}} l @{\hspace{30pt}} r @{\hspace{2pt}} l@{} }
	\cmd ::=  &
        \pskip \mid 
        \ptrans{\trans} \mid 
	    \cmdpri \mid  
        \cmd \pseq \cmd \mid 
        \cmd \pchoice \cmd \mid 
        \cmd \prepeat 
        
   & \cmdpri ::=  &
   		\passign{\txvar}{\expr} \mid 
   		\passume{\expr} \\
%   
	\trans ::= &
        \pskip \mid
        \transpri \mid 
        \trans \pseq \trans \mid
        \trans \pchoice \trans \mid
        \trans\prepeat    
	& \transpri ::= &
   		\cmdpri \mid
        \plookup{\txvar}{\expr} \mid
        \pmutate{\expr}{\expr} 
 \end{array} 
\]
%
Sequential commands include the standard constructs of $\pskip$, sequential composition ($\cmd \pseq \cmd$), non-deterministic choice ($\cmd \pchoice \cmd$), loops $\cmd\prepeat$, 
as well as \emph{transactions} ($\ptrans{\trans}$) and primitive commands. 
Primitive commands include variable assignments ($\passign{\txvar}{\expr}$) and assume statements ($\passume{\expr}$) used to encode conditionals,
and are split into transactional ones (the $\transpri$ clause) 
and non-transactional ones (the $\cmdpri$ clause).
Transactional primitive commands are used for reading and writing to kv-stores and 
can be invoked only within the boundaries of transactions (the $\ptrans{\trans}$ clause).
Non-transactional primitive commands are used for computations over client-local data
and can be invoked without restrictions.

For clarity, we often write \( \cmd_{1}\ppar \dots \ppar \cmd_{n}\) as syntactic sugar 
for a program \( \prog \) with $n$ implicit clients associated with identifiers, 
$\cl_1 \dots \cl_n$, where each client $\cl_i$ executes $\cmd_i$: 
\( \prog = \Set{\cl_{1} \mapsto \cmd_{1}, \dots, \cl_{n} \mapsto \cmd_{n}  }\).

As is standard, we assume a language of expressions built from values ($\val \in \Val$), 
and (client-local) \emph{program variables} $\Vars$, ranged over by $\pvar{x}, \pvar{y}, \cdots$. 
The evaluation $\evalE{\expr}$ of  expression $\expr$ is parametric in the (client-local) \emph{stack} 
$\stk \in \Stacks \defeq \Vars \to \Val$, mapping program variables to values. 
\[
\expr  ::= 
        \val \mid
        \var \mid
        \expr + \expr \mid
        \dots  
\qquad   
\evalE{\val}  =  \val \quad 
\evalE{\var} = \stk(\var)  \quad  
\evalE{\expr_{1} + \expr_{2}}  =  \evalE{\expr_{1}} + \evalE{\expr_{2}} \quad
\dots
\]


\subsubsection{Operational Semantics of Transactions}
Transactional commands in $\transpri$ are associated with a transition system: 
$\toLTS{\transpri} \subseteq (\Stacks \times \Snapshots) \times (\Stacks \times \Snapshots)$, 
describing how the snapshot and stack evolve upon executing $\transpri$:
% \cref{fig:semantics}.
\[
\begin{array}{rcl @{\qquad} rcl}
(\stk, \sn)  & \toLTS{\passign{\var}{\expr}}          & (\stk\rmto{\var}{\evalE{\expr}}, \sn)                  &
(\stk, \sn)  & \toLTS{\passume{\expr}}                & (\stk, \sn) \text{ where } \evalE{\expr} \neq 0        \\
(\stk, \sn)  & \toLTS{\plookup{\var}{\expr}}           & (\stk\rmto{\var}{\sn(\evalE{\expr})}, \sn)              &
(\stk, \sn)  & \toLTS{\pmutate{\expr_{1}}{\expr_{2}}} & (\stk, \sn\rmto{\evalE{\expr_{1}}}{\evalE{\expr_{2}}}) \\
\end{array}                                                                                               
\]

To compute the overall effect of a transaction,
we define a \emph{fingerprint function}, 
$\func{fp}{\stub} : \Stacks \times \Snapshots \times \transpri \rightarrow \Ops \cup \{\varepsilon\}$, extracting the operation of a primitive transactional command: 
%
\[
\begin{array}{rcl @{\quad} rcl}
\func{fp}{\stk, \sn, \passign{\var}{\expr}}          & \defeq & \emptyop                                     &
\func{fp}{\stk, \sn, \passume{\expr}}                & \defeq & \emptyop                                     \\
\func{fp}{\stk, \sn, \plookup{\var}{\expr}}           & \defeq & (\otR, \evalE{\expr}, \sn(\evalE{\expr}))     &
\func{fp}{\stk, \sn, \pmutate{\expr_{1}}{\expr_{2}}} & \defeq & (\otW, \evalE{\expr_{1}}, \evalE{\expr_{2}}) \\
\end{array}
\]
%and a \emph{fingerprint composition operator} \( \addO \).
Note that non-transactional primitive commands are associated with the empty operation $\varepsilon$,
as they only access the local stack and do not access the kv-store.

We further define a \emph{fingerprint composition operator}, \( \addO \), 
incrementally computing the fingerprint of a transaction, after executing each constituent primitive command. 
For instance, when executing $ \ptrans{\trans}$ with $\trans \defeq \transpri^1; \cdots ; \transpri^n$,
the effect of each $\transpri^i$ is calculated via the $\op_i = \func{fp}{-, -, \transpri^i}$ function, 
with the overall fingerprint given as the $\addO$-composition of the constituent effects: $\op_1 \addO \cdots \addO \op_n$. 

For each key $\key$, the composition operator \( \addO \) records
the first value a transaction reads (before a subsequent write) for $\key$, 
and the last value the transaction writes for $\key$.
This is in line with our assumption that transactions read from an atomic snapshot of the kv-store.
In particular, for each key $\key$, 
only the first read from $\key$ fetches its value from the kv-store,
and since clients observe either none or all effects of a transaction, 
only the last write to $\key$ is committed.
%
\begin{align*}
    \fp \addO (\otR, \key, \val)  & \defeq
    \begin{cases}
        \fp \cup \{(\otR, \key, \val)\} & \text{if} \ \fora{l, v'} (l, \key, v') \notin \fp \\
        \fp &  \text{otherwise} \\
    \end{cases} 
    \fp \addO \emptyop  \defeq  \fp \\
    \fp \addO (\otW, \key, \val) & \defeq 
    \left( \fp \setminus \Set{(\otW, \key, \val')}[\val' \in \Val] \right) \cup \Set{(\otW, \key, \val)} 
\end{align*}
\]
%
\ifTechReport
    
\begin{figure*}[!t]
\[
\begin{rclarray}
\toL & : & ((\Stacks \times \Snapshots \times \Fingerprints) \times \Transactions) \times ((\Stacks \times \Snapshots \times \Fingerprints) \times \Transactions)
\end{rclarray}
\]
\begin{mathpar}
    \infer[\rl{TPrimitive}]{%
        (\stk, \h, \fp) , \transpri \ \toL \  (\stk', \h', \fp \addO \op) , \pskip 
    }{%
        (\stk, \h) \toLTS{\transpri} (\stk', \h')
        \\ \op = \func{op}{\stk, \h, \transpri}
    }
    \\\
    \infer[\rl{TChoice}]{%
        (\stk, \h, \fp) , \trans_{1} \pchoice \trans_{2} \ \toL \  (\stk, \h, \fp) , \trans_{i}
    }{
        i \in \Set{1,2}
    }
    \and
    \infer[\rl{TIter}]{%
        (\stk, \h, \fp),  \trans\prepeat \ \toL \  (\stk, \h, \fp), \pskip \pchoice (\trans \pseq \trans\prepeat)
    }{ } 
    \and
    \infer[\rl{TSeqSkip}]{%
        (\stk, \h, \fp), \pskip \pseq \trans \ \toL \  (\stk, \h, \fp), \trans
    }{ }
    \and
    \infer[\rl{TSeq}]{%
        (\stk, \h, \fp), \trans_{1} \pseq \trans_{2} \ \toL \  (\stk', \h', \fp'), \trans_{1}' \pseq \trans_{2}
    }{%
        (\stk, \h, \fp), \trans_{1} \ \toL \  (\stk', \h', \fp'), \trans_{1}'
    }
\end{mathpar}
\hrulefill
\[
\begin{rclarray}
	\toT{}  & : &
    \begin{array}[t]{@{}c@{}}
    \Clients \; \times \;
	\left( ( \HisHeaps \times \Views \times \Stacks ) \times \Commands \right)  
    \; \times\; \ETs \;\times \sort{Labels} \times \;
	\left( ( \HisHeaps \times \Views \times \Stacks ) \times \Commands \right) 
    \end{array}
\end{rclarray}
\]
\begin{mathpar}
    \infer[\rl{CAtomicTrans}]{%
        \cl \vdash 
        ( \mkvs, \vi, \stk ), \ptrans{\trans} \ 
        \toT{(\cl, \vi'', \fp)}_{\ET} \ 
        (\mkvs',\vi', \stk' ) , \pskip
    }{%
		\begin{array}{@{} c @{}}
			\vi \orderVI  \vi''
			\quad \h = \clpsHH{\mkvs,\vi''}
			\quad \txid \in \nextTxId(\cl, \mkvs) \\
			(\stk, \h, \emptyset), \trans \toL^{*}   (\stk', \stub,  \fp) , \pskip
            \quad \mkvs' = \updateKV(\mkvs, \vi'', \fp, \txid) 
            \quad \ET \vdash (\mkvs, \vi'') \csat \f : (\mkvs',\vi')
			%\qquad \ET \vdash (\mkvs, \vi'') \triangleright \fp : \vi'
		\end{array}
    }
    \and
    \infer[\rl{CPrimitive}]{%
    \cl \vdash ( \mkvs, \vi, \stk ) , \cmdpri \ \toT{(\cl,\iota)}_{\ET} \  ( \mkvs, \vi, \stk' ) , \pskip
    }{
        \stk \toLTS{\cmdpri} \stk'
    }
    \and
    \infer[\rl{CChoice}]{%
        \cl \vdash ( \mkvs, \vi, \stk ) , \cmd_{1} \pchoice \cmd_{2} \ \toT{(\cl,\iota)}_{\ET} \  ( \mkvs, \vi, \stk ) , \cmd_{i}
    }{
        i \in \Set{1,2}
    }
    \quad
    \infer[\rl{CIter}]{%
        \cl \vdash ( \mkvs, \vi, \stk ) , \cmd\prepeat \ \toT{(\cl,\iota)}_{\ET} \  ( \mkvs, \vi, \stk ) , \pskip \pchoice (\cmd \pseq \cmd\prepeat)
    }{ }
    \and
    \infer[\rl{CSeqSkip}]{%
        \cl \vdash ( \mkvs, \vi, \stk ) , \pskip \pseq \cmd \ \toT{(\cl,\iota)}_{\ET} \  ( \mkvs, \vi, \stk ) , \cmd
    }{ }
    \quad
    \infer[\rl{CSeq}]{%
        \cl \vdash ( \mkvs, \vi, \stk ) , \cmd_{1} \pseq \cmd_{2} \ \toT{(\cl,\iota)}_{\ET} \ ( \mkvs, \vi', \stk' ) , {\cmd_{1}}' \pseq \cmd_{2}
    }{% 
        \cl \vdash ( \mkvs, \vi, \stk ) , \cmd_{1} \ \toT{(\cl,\iota)}_{\ET} \  ( \mkvs, \vi', \stk' ) , {\cmd_{1}}' 
    }
\end{mathpar}
\vspace*{5pt}

\hrulefill

\[
	\toG{} : 
    ( \Confs \times \ThdEnv \times \Programs) 
    \;\times\; \ETs \;\times \sort{Label} \times \;
    ( \Confs \times \ThdEnv \times \Programs) 
\]
\begin{mathpar}
    \infer[\rl{PProg}]{%
    ( \mkvs, \viewFun, \thdenv ) , \prog \ \toG{\lambda}_{\ET} \  ( \mkvs', \viewFun\rmto{\cl}{\vi'}, \thdenv\rmto{\cl}{\stk'}) , \prog\rmto{\cl}{\cmd'} 
    }{%
        \cl \vdash ( \mkvs, \viewFun(\cl), \thdenv(\cl) ), \prog(\cl), \ \toT{\lambda}_{\ET} \  ( \mkvs', \vi', \stk' ) , \cmd'  
    }
\end{mathpar}
%
\hrulefill
\caption{Operational Semantics on Key-value Store}
\label{fig:transaction_semantics}
\label{def:thread_semantics}
\label{fig:thread_semantics}
\label{def:thread_pool_semantics}
\label{fig:thread_pool_semantics}
\label{def:program_semantics}
\label{fig:program_semantics}
\label{fig:full-semantics}
\end{figure*}

\else
\begin{figure}[t]
\hrulefill
\begin{mathpar}
    \inferrule[\rl{TPrimitive}]{%
        (\stk, \sn) \toLTS{\transpri} (\stk', \sn')
        \\ \op = \func{fp}{\stk, \sn, \transpri}
    }{%
        (\stk, \sn, \fp) , \transpri \ \toTRANS \  (\stk', \sn', \fp \addO \op) , \pskip 
    }
    \\
    \inferrule[\rl{TChoice}]{
        i \in \Set{1,2}
    }{%
        (\stk, \sn, \fp) , \trans_{1} \pchoice \trans_{2} \ \toTRANS \  (\stk, \sn, \fp) , \trans_{i}
    }
    \and
    \inferrule[\rl{TIter}]{ }{%
        (\stk, \sn, \fp),  \trans\prepeat \ \toTRANS \  (\stk, \sn, \fp), \pskip \pchoice (\trans \pseq \trans\prepeat)
    } 
    \and
    \inferrule[\rl{TSeqSkip}]{ }{%
        (\stk, \sn, \fp), \pskip \pseq \trans \ \toTRANS \  (\stk, \sn, \fp), \trans
    }
    \and
    \inferrule[\rl{TSeq}]{%
        (\stk, \sn, \fp), \trans_{1} \ \toTRANS \  (\stk', \sn', \fp'), \trans_{1}'
    }{%
        (\stk, \sn, \fp), \trans_{1} \pseq \trans_{2} \ \toTRANS \  (\stk', \sn', \fp'), \trans_{1}' \pseq \trans_{2}
    }
\end{mathpar}
\hrulefill
\[
    \inferrule[\rl{PCommit}]{%
        \vi \viewleq  \vi''
        \qquad \sn = \snapshot[\mkvs,\vi'']
        \qquad \txid \in \nextTxid(\cl, \mkvs)
        \\\\ (\stk, \sn, \emptyset), \trans \ \toTRANS^{*} \  (\stk', \stub,  \fp) , \pskip
        \\ \ET \vdash (\mkvs, \vi'') \triangleright \fp : \vi'
    }{%
        \cl \vdash ( \mkvs, \vi, \stk ), \ptrans{\trans} \ \toCMD{(\cl, \vi'', \fp)}_{\ET} \ ( \updateKV(\mkvs, \vi, \txid, \fp),\vi', \stk' ) , \pskip
    }
\]
\[
    \inferrule[\rl{PPrimitive}]{
        \stk \toLTS{\cmdpri} \stk'
    }{%
    \cl \vdash ( \mkvs, \vi, \stk ) , \cmdpri \ \toCMD{(\cl,\iota)}_{\ET} \  ( \mkvs, \vi, \stk' ) , \pskip
    }
\]
\hrulefill
\[
    \inferrule[\rl{PProg}]{%
         \cl \vdash ( \mkvs, \vi, \thdenv(\cl) ) , \prog(\cl), \ \toCMD{\lambda}_{\ET} \  ( \mkvs', \vi', \stk' ) , \cmd'  
    }{%
         (\mkvs, \vienv\rmto{\cl}{\vi}, \thdenv ), \prog  \ \toCMD{\lambda}_{\ET} \  ( \mkvs', \vienv\rmto{\cl}{\vi'}, \thdenv\rmto{\cl}{\stk'} ) , \prog\rmto{\cl}{\cmd'} ) 
    }
\]
\hrulefill
\caption{Selected operational semantics rules for transactions (above); sequential commands (middle); and programs (below)}
\label{fig:semantics}
\end{figure}
\fi

In the top part of \cref{fig:semantics} we present a few selected rules of the operational semantics of transactional code (within the transaction boundaries in $\ptrans{}$). 
We refer the reader to \cref{sec:full-semantics} for the full operational semantics.
Transactional rules are of the form $(\stk, \sn, \fp), \toTRANS (\stk', \sn', \fp'), \trans'$, 
stating that when the fingerprint accumulated so far is recorded in $\fp$, 
executing $\trans$ for one step transforms the client-local stack $\stk$ 
and snapshot $\sn$ %(used to fetch the values of read operations),
to $\stk'$ and $\sn'$, respectively, with the extended fingerprint $\fp'$, and continuation $\trans'$.
The \rl{TSeq} rule is in a standard format; 
the \rl{TPrimitive}  models the execution of primitive transactional commands.
When $(\stk, \sn)$ is transformed to $(\stk', \sn')$ via the transition system $\toLTS{\transpri}$,
$\func{fp}{\stk, \sn, \transpri} = \op$, and the fingerprint recorded so far is given by $\fp$, 
then executing $\transpri$ transforms $(\stk, \sn)$ to $(\stk', \sn')$, with the fingerprint updated to $\fp \addO \op$ (to incorporate the effect of $\transpri$), with continuation $\pskip$. 

\subsubsection{Operational Semantics of Commands}
In \cref{fig:semantics} we present a select number of \emph{(sequential) command operational semantics}.
We refer the reader to \cref{sec:full-semantics} for the full operational semantics.
Command transitions are of the form $\cl \vdash (\mkvs, \vi, \stk), \cmd \ \toCMD{\lambda}_{\ET} \ (\mkvs', \vi', \stk') , \cmd'$, 
stating that given the kv-store $\mkvs$, view $\vi$ and stack $\stk$, client $\cl$ may execute command $\cmd$ for one step under $\ET$, update the kv-store to $\mkvs'$, the stack to $\stk'$, and the command to its continuation $\cmd'$, with label $\lambda$.
A transition label $\lambda$ is either
\begin{enumerate*}
	\item of the form $(\cl, \iota)$, denoting that the transition involved 
client-local computation that did not require access to the kv-store (\eg primitive commands $\cmdpri$; or
	\item of the form $(\cl, \vi, \fp)$, denoting that client $\cl$ commits a transaction with fingerprint $\fp$ under the view $\vi$.
\end{enumerate*}
Note that the operational semantics of commands are parametric in the choice of execution test $\ET$, 
and thus the choice of the underlying consistency model.


With the exception of the \rl{PCommit} rule, the remaining command transitions are standard and behave as expected. 
For instance, the \rl{PPrimitive} rule models the execution of a non-transactional primitive command $\cmdpri$, where the $\stk \toLTS{\cmdpri} \stk'$ transition describes how the stack of a client 
evolve upon executing $\cmdpri$:
\[
\begin{array}{rcl @{\qquad} rcl}
\stk  & \toLTS{\passign{\var}{\expr}} & \stk\rmto{\var}{\evalE{\expr}} &
\stk  & \toLTS{\passume{\expr}} & \stk \text{ where } \evalE{\expr} \neq 0 \\
\end{array}                                                                                               
\]
%
%
The \rl{PCommit} rule models the execution of a transaction $\ptrans{\trans}$, under the execution test $\ET$. 
The first premise of \rl{PCommit} states that the current view $\vi$ of the executing command maybe advanced to a newer atomic view $\vi''$ (see \cref{def:views}). 
The semantics only allows to advance the view to latest versions, which corresponds to \emph{monotonic read} \cite{.......}.
Given the new view $\vi''$, the transaction proceeds by obtaining a snapshot $\sn$ of the kv-store $\mkvs$, and executing $\trans$ locally to completion ($\pskip$), updating the stack to $\stk'$, while accumulating the fingerprint $\fp$. Note that the resulting snapshot is ignored (denoted by $\stub$) as the effect of the transaction is recorded in the fingerprint $\fp$. 
%

The transaction is now ready to commit and may propagate its changes to $\mkvs$.
To this end, a \emph{fresh} transaction identifier $\txid \in \nextTxid(\cl, \mkvs)$ (defined shortly) is picked
to identify the completed transaction, and the changes performed by $\txid$ are propagated to $\mkvs$. 
This is done via the $\func{updateMKVS}{\mkvs, \vi, \txid, \fp}$ function (defined shortly) to update $\mkvs$. 
%The fresh identifier $\txid$ is given by the $t \in \nextTxid(\cl, \mkvs)$ function defined shortly.
Once the kv-store is updated, the client subsequently updates its view to $\vi'$ with respect to its fingerprint. 
Lastly, to ensure that the effect of the transaction (its fingerprint  $\fp$) is permitted by the underlying consistency model, 
the last premise requires that the updates be permitted by the execution test $\ET$, \ie \( (\mkvs, \vi'') \csat \fp : \vi'\).

The $\nextTxid(\cl, \mkvs)$ returns the set of transactions identifiers associated with $\cl$ that are fresh with respect to $\mkvs$: 
$\nextTxid(\cl, \mkvs) \defeq \Setcon{\txid_{\cl}^{n}}{\fora{m} \txid_{\cl}^{m} \in \mkvs \Rightarrow m < n }$.
Note that when $\txid_\cl^n = \nextTxid(\cl, \mkvs)$ in the premise of \rl{PCommit}, then $\txid_\cl^n$ is greater than any transaction identifier 
(with respect to session order $\toEDGE{\SO}$) 
of the form $\txid_{\cl}^{i}$ appearing in $\mkvs$,
as to reflect the fact that $\txid_\cl^n$ is the most recent transaction executed by $\cl$.

%
The \( \updateKV \) function is defined below, where $\lcat$ denotes list concatenation; 
and when $\vilist = \ver_0, \cdots, \ver_n$ and $i=0,\cdots,n$, 
$\vilist\rmto{i}{\ver}$ denotes the updated list 
$\vilist' = \ver_0, \cdots, \ver_{i-1}, \ver, \ver_{i+1}, \cdots, \ver_{n}$. 
%
%
%
%
\begin{align*}         
    \updateKV[\mkvs, \vi, \txid, \emptyset] &\defeq \mkvs \\
    \updateKV[\mkvs, \vi, \txid, \fp \uplus \Set{(\otR, \key, \stub)}] & \defeq 
    \begin{multlined}[t]
        \texttt{let} \ (\val, \txid', \txidset) = \mkvs(\key, \max_{<}(\vi(\key))), \\
        \vilist = \mkvs(\key)\rmto{\max_{<}(\vi(\key))}{(\val, \txid', \txidset \uplus \Set{ \txid })}\\
        \texttt{in} \ \updateKV(\mkvs\rmto{\key}{\vilist}, \vi, \txid, \fp)
    \end{multlined} \\
    \updateKV(\mkvs, \vi, \txid, \fp \uplus \Set{(\otW, \key, \val)} )& \defeq &  
    \begin{multlined}[t]
        \texttt{let } \mkvs' = \mkvs\rmto{\key}{ \mkvs(\key) \lcat (\val, \txid, \emptyset) } \\
        \quad \texttt{in } \updateKV(\mkvs', \vi, \txid, \fp)
    \end{multlined} \\
\end{align*}
%
%
For every read operation $(\otR, \key, \stub)$ in fingerprint $\fp$,
since a transaction reads the values of keys from 
the atomic snapshot determined by the view of the client, 
the version read by $\txid$ for key $\key$ corresponds to $\mkvs(\key, \max_{<}(\vi(\key)))$.
As such, to commit the transaction, 
the reader set of the version $\mkvs(\key, \max_{<}(\vi(\key)))$ is extended with a new reader $\txid_\cl^n$.
Similarly, for every write operation $(\otW, \key, \val)$ in fingerprint $\fp$, 
the list of versions $\mkvs(\key)$ is extended with a new version $(\val, \txid, \emptyset)$, 
denoting that $\txid_\cl^n$ is the writer of this version which has no readers as of yet. 

Note that the assumption that 
versions of a given key are totally ordered and consistent with the order in which 
transactions commit is standard, 
which corresponds to last-write-win resolution policy \cite{adya,framework-concur,seebelieve}. 
Moreover, under the assumption that fingerprints contain at most one read and one write 
operation per key, and the identifier is fresh $\txid \notin \mkvs$%
\footnote{%
We write $\txid \in \mkvs$ when there exists a key 
$\key$ and an index $i=0,\cdots, \lvert \mkvs(\key) \rvert -1$ such that $\txid \in \{\wtOf(\mkvs(\key, i)\} \cup \rsOf(\mkvs(\key, i))$.}, 
the $\updateKV$ is well-defined.

\subsubsection{Operational Semantics of Programs}

The \emph{operational semantics of programs} are given at the bottom of \cref{fig:semantics}. 
Programs transitions are of the form $(\conf,  \thdenv, \prog) \ \toPROG{\lambda}_{\ET} (\conf',  \thdenv', \prog')$,
stating that given the configuration $\conf$ and the \emph{client environment} $\thdenv$, executing program $\prog$ for one step under $\ET$, updates the configuration to $\conf'$, the client environment to $\thdenv'$, and the program to its continuation $\prog'$. 
A \emph{client environment}, $\thdenv \in \ThdEnv$, is a mapping from client identifiers to pairs of stacks and views. 
We assume that the clients in the domain of client environments are are those in the domain of the program throughout the execution: 
$\dom(\thdenv) = \dom(\prog)$.
Program transitions are simply defined in terms of the transitions of their constituent client commands.
This in turn yields the standard interleaving semantics for concurrent programs. 
That is, a client performs a reduction in an atomic step, without affecting other clients.


\subsection{Traces of Programs}
In this Section we define the set of $\ET$-traces generated by a program 
$\prog$. Our main goal is that of proving that our semantics is correct, 
meaning that if a program $\prog$ executing under the execution 
test $\ET$ terminates in a state $(\hh, \_)$, then $\hh \in \CMs(\ET)$. 

\begin{definition}
Let $\ET$ be an execution test. For each program $\prog$ and state 
$(\hh, \viewFun, \thdenv)$, we define $\OPtraces(\ET, \prog, \hh, \viewFun, \thdenv)$ 
as the smallest set such that 
\begin{itemize}
\item $(\hh, \viewFun) \in \OPtraces(\ET, \prog, \hh, \viewFun, \thdenv)$, 
\item if $\tr \in \OPtraces(\ET, \prog', \hh', \viewFun',\thdenv')$, 
and $((\hh, \viewFun, \thdenv) , \prog) \toT{(\cl, \iota)}_{\ET} (\hh', \viewFun', \thdenv')$, 
then $\tr \in \OPtraces(\ET, \prog, \hh, \viewFun, \thdenv')$, 
\item if $\tr$ in $\OPtraces(\ET, \prog', \hh', \viewFun', \thdenv')$ and 
\newline $(\hh, \viewFun, \thdenv), \prog) \toT{(\cl, \vi, \opset)} ((\hh', \viewFun', \thdenv'), \prog')$,  
then $(\hh, \viewFun) \xrightarrowtriangle{(\cl, \varepsilon)}_{\ET} (\hh, \viewFun\rmto{\cl}{\vi}) 
\xrightarrowtriangle{(\cl, \opset)}_{\ET} \tr$ in $\OPtraces(\ET, \prog, \hh, \viewFun, \thdenv)$. 
\end{itemize}

The set of traces generated by a program $\prog$ under the execution test $\ET$ is 
then defined as $\OPtraces(\ET, \prog) \triangleq (\OPtraces(\ET, \prog, \hh_{0}, \viewFun_{0}, 
\thdenv_{0})$, where $\viewFun_{0} = \lambda \cl \in \dom(\prog).\lambda \ke.\{0\}$, and 
$\thdenv_{0} = \lambda \cl \in \dom(\prog).\lambda a.0$.

\end{definition}

\begin{proposition}
    \label{prop:program-trace-in-et-trace}
For each program $\prog$ and execution test $\ET$, 
$\OPtraces(\ET, \prog) \subseteq \Confs(\ET)$. 
Furthermore, each $\tr \in \OPtraces(\ET, \prog)$ is in normal form. 
\end{proposition}
\begin{proof}
    First, by the definition of \( \OPtraces \), 
    it only constructs trace in normal form.
    It is easy to prove that for any trace \( \tau \) in \( \OPtraces(\ET, \prog) \), by induction on the trace length,
    the trace is also in \( \Confs(\ET) \).
\end{proof}

\begin{corollary}
If $(\hh_{0}, \viewFun_{0}, \thdenv_{0}), \prog) \toL_{\ET} \cdots \toL_{\ET} 
(\hh, \viewFun, \thdenv, \lambda \cl \in \dom(\prog). \pskip)$, then $\hh \in \CMs(\ET)$.
\end{corollary}
\begin{proof}
    By the definition of \( \OPtraces \), 
    there exists a corresponding trace \( \tau \in \OPtraces(\ET, \prog) \).
    By \cref{prop:program-trace-in-et-trace}, such trace \( \tau \in \Confs(\ET) \),
    therefore \( \mkvs \in \CMs(\ET)\) by definition of \( \CMs(\ET) \).
\end{proof}

\begin{proposition}
For any program $\prog$ and execution test $\ET$, $\OPtraces(\ET, \prog) = \OPtraces(\ET_{\top}, \prog) 
\cap \Confs(\ET)$.
\end{proposition}
\begin{proof}
    It is easy to see \(\OPtraces(\ET, \prog) \subseteq \OPtraces(\ET_\top, \prog) \).
    By \cref{prop:program-trace-in-et-trace}, we know \( \OPtraces(\ET, \prog) \subseteq \Confs(\ET)\).
    Therefore \(  \OPtraces(\ET, \prog) \subseteq \OPtraces(\ET_\top, \prog) \cap \Confs(\ET) \).

    Let consider a trace \( \tau \) in \( \OPtraces(\ET_\top, \prog) \cap \Confs(\ET) \).
    By inductions on the length of trace, 
    every step that commits a new transaction  must satisfy \( \ET \) as \( \tau \in \Confs(\ET) \).
    It also reduce the program \( \prog \) as \( \tau \in \OPtraces(\ET_\top, \prog) \).
    By the definition \( \OPtraces(\ET, \prog) \),
    we can construct the same trace \( \tau \),
    so \( \tau \in \OPtraces(\ET, \prog) \).
\end{proof}

\ac{These results should be easy to prove, either I or somebody else will type up a proof 
at some point.}

\subsection{Adequacy of the Semantics.}
\begin{figure}
\[
    \inferrule[\rl{ACommit}]{
        \T \subseteq \T_{\aexec} \qquad \h \in \RP(\aexec, \T) \qquad
		(\stk, \h, \emptyset), \trans \ \toL^{*} \  (\stk', \stub,  \opset) , \pskip \\\\
		\txid \in \nextTxId(\T_{\aexec}, \cl) \qquad \aexec' = \extend(\aexec, \txid, \T, \opset) \qquad 
		\forall A \in \Ax.\;\{\txid' \mid (\txid', \txid) \in \A(\aexec') \} \subseteq \T
    }{
        \cl \vdash ( \aexec, \stk ), \ptrans{\trans} \ \toA{(\cl, \T, \opset)}_{(\RP, \Ax)} \ ( \aexec', \stk' ) , \pskip
    }
\]
\[
    \inferrule[\rl{ASingleThread}]{%
         \cl \vdash ( \aexec, \thdenv(\thid) ) , \prog(\thid), \ \toA{\lambda}_{(\RP, \Ax)} \  ( \aexec', \stk' ) , \cmd'  
    }{%
         (\aexec, \thdenv ), \prog  \ \toA{\lambda}_{(\RP, \Ax)} \  ( \aexec', \thdenv\rmto{\thid}{\stk'} ) , \prog\rmto{\thid}{\cmd'} ) 
    }
\]
\hrulefill
\caption{Selected Rule of the Semantics for Programs using Abstract Executions.}
\label{fig:aexec.semantics}
\end{figure}
Suppose that a given execution test $\ET$ captures precisely 
a consistency model specified in the axiomatic style, using a set of 
axioms $\Ax$ and a resolution policy $\RP$ over abstract executions.
That is, for any abstract execution $\aexec$ that satisfies 
the axioms $\Ax$ and the resolution policy $\RP$, then $\KVtrace(\ET_{\top}, \aexec) \cap \CMs(\ET) \neq \emptyset$; 
and for any $\tr \in \CMs(\ET)$, there exists an abstract execution 
$\aexec \in \aeset(\tr)$ that satisfies the axioms $\Ax$ and the resolution policy $\RP$. 
\ac{In practice, the functions $\KVtrace(\ET, \aexec) = \KVtrace(\ET_{\top}, \aexec) \cap \CMs(\ET)$, 
and $\aeset(\Ax, \aexec) = \aeset(\ET_{\top}, \aexec) \cap \aeset(\Ax)$, abstracted over the second 
argument and lifted to sets of abstract executions/traces, 
define a Galois Connection between the powerset of abstract executions in the axiomatic specification 
of the consistency model, and the powerset of kv-stores generated by such a consistency model.}
Our main aim in this section consists in proving that, for each program $\prog$, the 
set of kv-stores generated by $\prog$ under $\ET$ corresponds to all the possible kv-stores 
that could be obtained by running $\prog$ on a database that satisfies the axiomatic specification 
$\Ax$. In this sense, we aim to establish that our operational semantics is \emph{adequate}.

To tackle this question, we need to define what is the set of all possible behaviours 
that can be produced by a program $\prog$ under a given consistency model $\CM$, for 
which an axiomatic specification $(\RP, \Ax)$ is known. This in turn requires addressing two orthogonal 
problems: \textbf{(i)} defining the set of all possible behaviours that may be exhibited by a program 
$\prog$, independently of the consistency model; and \textbf{(ii)} defining the set of all possible 
behaviours that are allowed by a given consistency model $\CM$. Then the set of all 
possible behaviours of $\prog$ under $\CM$ is obtained by intersecting the two sets 
above.

The kv-store semantics is intrinsically not expressive enough to tackle problem \textbf{(i)}. 
By \cref{cor:kvtrace2aexec}, only kv-stores arising 
from abstract executions satisfying the last write wins resolution policy can be captured in the kv-store 
framework; instead, we seek to model all the behaviours of a program independently of a consistency 
models, and therefore independently of a resolution policy. 

Our solution requires defining an alternative semantics of programs under weak consistency models, 
based on abstract executions. The operational semantics we propose is parameterised in 
the axiomatic specification $(\RP, \Ax)$ of a consistency model: transitions take the form 
$(\aexec, \Env, \prog) \toA{\_}_{(\RP, \Ax)} (\aexec', \Env', \prog')$. 

In \cref{fig:aexec.semantics} we illustrate two rules of the operational semantics of programs 
based on abstract executions. The complete operational semantics is given in \ref{app:aexec.semantics}. 
\ac{Todo: type down the semantics. This must be done in order to prove the Theorem that 
establishes the correspondence between kv-store semantics and abstract execution semantics.} 
Rule \rl{ACommit} is the abstact execution counterpart of rule \rl{PCommit} for kv-stores, in that 
it models how an abstract execution $\aexec$ evolves when a client wants to execute a transaction whose 
code is $\ptrans{\trans}$. In this rule, $\T$ is the set of transactions of $\aexec$ that are visible to the client 
$\cl$ that wishes to execute $\ptrans{\trans}$. Such a set of transactions is used to determine a snapshot 
$\h \in \RP(\aexec, \T)$ that the client $\cl$ uses to execute the code $\ptrans{\trans}$, and obtain a 
fingerprint $\opset$. This fingerprint is then used to extend abstract execution $\aexec$ with 
a transaction from the set $\nextTxId(\T_{\aexec}, \cl)$. Another rule in \cref{fig:aexec.semantics} 
is Rule \rl{ASinglethread}; the structure of this rule is analogous to \rl{PSingleThread}, and it models 
multi-thread concurrency in an interleaving fashion. All the rules of the abstract operational semantics 
that are not illustrated in \cref{fig:aexec.semantics} have a similar counterpart in the kv-store semantics.

In some sense that is going to be made mathematically precise later, Rule \rl{ACommit} is more general 
than Rule \rl{Pcommit} in the kv-store semantics. In the latter, the snapshot of a transaction is uniquely 
determined from a view of the client, in a way that roughly corresponds to the last write wins policy 
in the abstract execution framework. In contrast, in Rule \rl{ACommit} the snapshot of a transaction 
is chosen non-deterministically from those made available to the client by the resolution policy 
$\RP$ adopted by a weak consistency model, which may not necessarily be $\RP_{\LWW}$. 
One example of resolution policy that we will use in this Section is given by the anarchic resolution policy. 

\begin{definition}
The anarchic resolution policy $\RP_{\anarchic}$ is defined by letting, 
$\RP_{\anarchic}(\_, \_) = \Snapshots$. The \emph{anarchic consistency model} is 
specified axiomatically by the pair $\anarchicCM = (\RP_{\anarchic}, \emptyset)$.
\end{definition}

\begin{example}
Suppose that we want to execute the single-threaded program $\prog$ that maps client 
$\cl$ to the transactional code below:
\[
\begin{session}
%\ptrans{\pmutate{\ke}{\val_2}}; \\
\ptrans{\pderef{\pvar{a}}{\ke}; \\
\pifs{\pvar{a} = \val_{1}} \pmutate{\ke'}{\val_1} \pife}
\end{session}
\]
Suppose that the program is executed under a consistency model that adopts the last write 
wins resolution policy $\RP_{\LWW}$, and with no additional axioms. Then the behaviour of $\prog$ is 
completely deterministic (up-to the choice of transaction identifiers), and the execution of $\prog$ terminates in a 
state corresponding to the abstract execution below: 

\begin{center}
\begin{tikzpicture}[scale=0.85, every node/.style={transform shape}]

\node(t0rx) at (-1,2) {$(\otR, \ke, \val_0)$}; 
%\path (t0wx.south) + (0,-0.2) node[anchor=north] (t0wy) {$(\otW, \ke_2, \val'_0)$};

\begin{pgfonlayer}{background}
\node[background, fit=(t0rx)]  {};

\path(t0.west) node[anchor=east] (t0lbl) {$\txid_{\cl}^{\_}$};
%\path(t1.north) node[anchor=south] (t1lbl) {$\txid_1$};
%\path(t2.south) node[anchor=north] (t2lbl) {$\txid_2$};

%\path[->]
%(t0.north) edge[bend left=70] node[above, yshift=7pt, xshift=-1pt, pos=0.3] {$\RF(\ke_2), \VO(\ke_1)$} (t1.west)
%(t0.south) edge[bend right=70] node[below, yshift=-8pt, xshift=-1pt, pos=0.3] {$\RF(\ke_1), \VO(\ke_2)$} (t2.west)
%([xshift=-8pt]t2.north) edge[bend left=40] node[left] {$\AD(\ke_1)$} ([xshift=-8pt]t1.south) 
%([xshift=8pt]t1.south) edge[bend left=40] node[right] {$\AD(\ke_2)$} ([xshift=8pt]t2.north);
\end{pgfonlayer}
\end{tikzpicture}
\end{center}

However, if we replace the consistency model specification $(\RP_{\LWW}, \emptyset)$ with the 
anarchic one $\anarchicCM$. Because the snapshot in which client $\cl$ executes the 
transactional code above is chosen non-deterministically, 
the program $\prog$ exhibits infinitely many additional behaviours. 
In particular, the program may now terminate in a state corresponding 
to the abstract execution below: 

\begin{center}
\begin{tikzpicture}[scale=0.85, every node/.style={transform shape}]

\node(t0rx) at (-1,2) {$(\otR, \ke, \val_1)$}; 
\path (t0wx.east) + (0,0.2) node[anchor=west] (t0wy) {$(\otW, \ke', \val_1)$};

\begin{pgfonlayer}{background}
\node[background, fit=(t0rx) (t0wy)]  {};

\path(t0.west) node[anchor=east] (t0lbl) {$\txid_{\cl}^{\_}$};
%\path(t1.north) node[anchor=south] (t1lbl) {$\txid_1$};
%\path(t2.south) node[anchor=north] (t2lbl) {$\txid_2$};

%\path[->]
%(t0.north) edge[bend left=70] node[above, yshift=7pt, xshift=-1pt, pos=0.3] {$\RF(\ke_2), \VO(\ke_1)$} (t1.west)
%(t0.south) edge[bend right=70] node[below, yshift=-8pt, xshift=-1pt, pos=0.3] {$\RF(\ke_1), \VO(\ke_2)$} (t2.west)
%([xshift=-8pt]t2.north) edge[bend left=40] node[left] {$\AD(\ke_1)$} ([xshift=-8pt]t1.south) 
%([xshift=8pt]t1.south) edge[bend left=40] node[right] {$\AD(\ke_2)$} ([xshift=8pt]t2.north);
\end{pgfonlayer}
\end{tikzpicture}
\end{center}

It is important to note, however, that the set of abstract executions generated by $\prog$ is still bound 
to the structure of the program itself. For example, executing $\prog$ under the anarchic execution model 
will never lead to an abstract execution with multiple transactions, or to an abstract execution where a transaction 
writes a key other than $\ke'$ is written.
\end{example}

\begin{definition}
The semantics of a program $\prog$ under a consistency model with axiomatic specification 
$(\RP, \Ax)$ is given by 
\[
\interpr{\prog}_{(\RP, \Ax)} = \{ \aexec \mid (\aexec_{0}, \Env_{0}, \prog) \toA{\_}_{(\RP, \Ax)}^{\ast} (\aexec, \_, \prog_{f}) \}, 
\]
where $\Env_{0} = \lambda \cl \in \dom(\prog).\lambda \pvar{x}.0$ and $\prog_{f} = \lambda \cl \in \dom(\prog).\pskip$.
\end{definition}

We define the set of all the possible behaviours of a program $\prog$ to be $\interpr{\prog}_{\anarchic}$. 
The following result supports our claim that this definition is indeed accurate: 

\begin{proposition}
Any abstract execution $\aexec$ satisfies $\anarchicCM$.
\end{proposition}
\begin{proof}
    (I did not find a definition of satisfies, but I guess is first visibility relation satisfies the constrain, 
    and second, there exists a snapshot decided by the visibility relation and resolution policy is consistent with the operations inside transactions.).
    It is trivial since $\anarchicCM$ does not have any constraint for visibility relation,
    and the snapshot for each transaction can be any possible one.
\end{proof}

\begin{example}
One may argue that the axiomatic specification $\anarchicCM$ does not 
truly represent an anarchic consistency model. Consider for example the single-threaded 
program $\prog'$ that associates to a client $\cl$ the following code:
\[
\begin{session}
\ptrans{
\pderef{\pvar{a}}{\ke}; \\
\pderef{\pvar{b}}{\ke};\\
\pifs{\pvar{a} != \pvar{b}} \pmutate{\ke'}{\val_1} \pife}
\end{session}
\]
One would expect that, under a truly anarchic consistency model, it would be possible 
for program $\prog'$ to write the value $\val_1$ for key $\ke'$. However, 
this never happens if $\prog'$ is executed under $\anarchicCM$. This is because 
we embedded into abstract execution the assumption that transactions only read 
at most one value for each key. 

In theory, we could lift this limitation and still retain 
the validity of all the results contained in this report; however, doing so would 
require to work with mathematical structures that are far more complex than 
abstract executions, and we preferred to avoid this issue. 
Furthermore, the constraint that an object is never read twice in transactions is enforced 
at client side in virtually all the implementations {\color{red} to be checked} 
of libraries for accessing kv-stores. When a client first requests to fetch 
the value of some key $\ke$ within a transaction, a local copy of the value fetched is 
saved on the client (typically in an object containing the meta-data of the transaction); 
if a request to read the same key is performed again within the same client, the local 
copy of the value previously fetched for that key is returned, instead of issuing a second 
read request to the kv-store.
\end{example}

As explained above, the set of all possible behaviours exhibited by a program $\prog$ under a 
consistency model $(\RP, \Ax)$ can be defined by intersecting the set of executions 
that $\prog$ exhibits under the anarchic consistency model, with the set of all executions 
allowed by the axiomatic specification $(\RP, \Ax)$. As the next theorem shows, 
this is exactly the set of abstract executions in which $\prog$ terminates, 
when executed under the axiomatic specification $(\RP, \Ax)$.

\begin{theorem}
For any program $\prog$ and axiomatic specification $(\RP, \Ax)$:
\[
\interpr{\prog}_{(\RP, \Ax)} = \interpr{\prog}_{\anarchic} \cap \CMa(\RP, \Ax). 
\]
\end{theorem}
\begin{proof}
AC: The proof was already typed up in the other set of notes, will need to do 
cut, paste and adapt in the appendix.
\end{proof}


\mypar{Comparing the abstract execution and kv-store semantics.}

\begin{proposition}
\label{prop:kv2aexec_transition}
Suppose that $(\hh, \viewFun, \thdenv), \prog \toT{(\cl, \vi, \opset)}_{\ET_{\top}} (\hh', 
\viewFun', \thdenv'), \prog'$. Let $\aexec$ be an abstract execution 
such that $\hh_{\aexec} = \hh$, and let $\T \subseteq \T_{\aexec}$ be a 
set of read-only transactions. Then there exists an abstract execution $\aexec'$ 
such that $\hh_{\aexec'} = \hh'$, and 
\[
(\aexec, \thdenv), \prog \toA{(\cl, \T \cup \Tx(\hh, \vi), \opset)}_{(\RP_{\LWW}, \emptyset)}. 
(\aexec', \thdenv'), \prog'
\]
\end{proposition}
\begin{proof}
See \cref{sec:kv2aexec-transaction}.
\end{proof}

\begin{proposition}
\label{prop:aexec2kv_transition}
Suppose that $(\aexec, \thdenv), \prog \toA{(\cl, \T, \opset)}_{(\RP_{\LWW})} (\aexec', \thdenv'), \prog'$. 
Then for any $\viewFun$ and $\vi \in \Views(\hh_{\aexec})$ such that $\vi \viewleq \getView(\aexec, \T)$, 
we have that 
\[
(\hh_{\aexec}, \viewFun\rmto{\cl}{\vi}, \thdenv), \prog \toA{(\cl, \getView(\aexec, \T), \opset)}_{\ET_{\top}} (\hh_{\aexec'}, \viewFun, \thdenv'), \prog'.
\]
\end{proposition}
\begin{proof}
See \cref{sec:aexec2kv-transaction}.
\end{proof}

\begin{corollary}
For any program $\prog$, 
\[
\interpr{\prog}_{\ET_{\top}} = \{\hh_{\aexec} \mid \aexec \in \interpr{\prog}_{(\RP_{\LWW}, \emptyset)}\}
\]
\end{corollary}
\begin{proof}
    It can be derived by \cref{prop:aexec2kv_transition} and \cref{prop:kv2aexec_transition}.
\end{proof}


\mypar{Putting it all together}
To conclude this Section, we show how all the results illustrated so far 
can be put together to show that the kv-store operational semantics is adequate. 
\begin{definition}
A \emph{client-based invariant condition}, or simply \emph{invariant}, is a 
function $I : \aeset \times \Clients \rightarrow \powerset{\TxID}$ 
such that for any $\cl$ we have that $I(\aeset, \cl) \subseteq \T_{\aexec}$, and 
for any  $\cl'$ such that $\cl' \neq \cl$ we have that 
$I(\extend(\aexec, \txid_{\cl'}^{\cdot}, \_, \_), \cl) = I(\aexec, \cl)$.
%of triples of the form $(\aexec, \cl, \T)$ such that whenever 
%$(\aexec, \cl, \T) \in I$ then $\T \subseteq \aexec$, 
%for any $\cl' \neq \cl$ and transaction $\txid = \txid_{\cl'}^{\cdot}$, 
%then $(\extend(\aexec, \txid, \_, \_)), \cl, \T) \in I$, 
%and for any $\T' : \T \subseteq \T' \subseteq \T_{\aexec}$, 
%$(\aexec, \cl ,\T') \in I$.
\end{definition}
\ac{The idea behind client-based invariant being that $I(\aexec, \cl)$ represents 
the minimal set of transactions that $\cl$ must see in $\aexec$, before 
updating the view and performing a transaction. Such a set of transaction 
roughly correspond to the view of the client before performing a 
sequence of \emph{update view+execute transaction} operations, 
or equivalently from the view obtained after the execution of the 
last transaction from that client.}

%\begin{definition}
%\label{def:et_sound}
%An execution test $\ET$ is sound with respect to an axiomatic 
%specification $(\RP_{\LWW}, \Ax)$ if and only if there exists an 
%invariant condition $I$ such that whenever $\ET \vdash (\hh, \vi) 
%\triangleright \opset: \vi'$, for any $\aexec$ such that 
%$\hh_{\aexec} = \hh$, and for any client $\cl$, and for any 
%transaction identifier $\txid \in \nextTxId(\hh, \cl)$, 
%%and for any $\T \subseteq \T_{\aexec}$ such that 
%%$\getView(\aexec, \T) = \vi$ 
%and $I(\aexec, \cl) \subseteq \Tx(\hh, \vi)$, then  
%there exist a set of read-only transactions $\T_{\mathsf{rd}}$ and 
%a set of transactions 
%%$\T \subseteq \T_{\aexec}$ and 
%$\T' \subseteq \T_{\aexec} \cup \{\txid\}$ 
%\begin{itemize}
%%\item $\getView(\aexec, \T) = \vi$, $I(\aexec, \cl) \subseteq \T$, 
%\item $\forall \A \in \Ax. \{\txid' \mid (\txid', \txid) \in \A(\aexec')\} \subseteq \Tx(\hh, \vi) \cup \T_{\mathsf{rd}}$, 
%where $\aexec' = \extend(\aexec, \txid, \Tx(\hh, \vi), \opset)$ for $\txid \in \nextTxId(\hh, \cl)$, 
%\item $I(\aexec', \cl) \subseteq \T'$, where $\T' = \Tx(\aexec', \vi')$. 
%\end{itemize}
%\end{definition}

\begin{definition}
\label{def:et_sound}
An execution test $\ET$ is sound with respect to an axiomatic 
specification $(\RP_{\LWW}, \Ax)$ if and only if there exists an 
invariant condition $I$ such that whenever $\ET \vdash (\hh, \vi) 
\triangleright \opset: \vi'$, for any $\aexec$ such that 
$\hh_{\aexec} = \hh$, for any client $\cl$, for any 
transaction identifier $\txid \in \nextTxId(\hh, \cl)$, 
%and for any set of transactions $\T_{\rd}$ that are read-only in $\aexec$ 
such that  $I(\aexec, \cl) \subseteq \Tx(\hh, \vi)$, then  
there exist two other sets of transactions $\T_{\rd}$, 
%which are read-only in $\aexec$ and $\aexec' := \extend(\aexec, \txid, 
%\Tx(\hh,\vi) \cup \T_{\mathsf{rd}} \cup \T'_{\mathsf{rd}}, \opset)$, respectively, 
such that 
\begin{itemize}
\item $\forall \A \in \Ax. \Setcon{\txid' }{ (\txid', \txid) \in \A(\aexec')} \subseteq \Tx(\hh, \vi) \cup \T_{\rd}$, 
\item $I(\aexec', \cl) \subseteq \Tx(\mkvs_{\aexec'}, \vi')$
\end{itemize}
\end{definition}

\begin{theorem}
\label{thm:et_soundness}
If $\ET$ is sound with respect to $(\RP_{\LWW}, \Ax)$, then 
\[
    \CMs(\ET) \subseteq \{ \hh \mid \exists \aexec \in \CMa(\RP_{\LWW}, \Ax)).\;\hh_{\aexec} = \hh\}
\].
\end{theorem}
\begin{proof}
    See \cref{sec:thm-et-soundness}.
\end{proof}

\begin{corollary}
\label{cor:et-soundness}
If $\ET$ is sound with respect to $(\RP_{\LWW}, \Ax)$, then 
for any program $\prog$, $\interpr{\prog}_{\ET} \subseteq \{ \hh_{\aexec} \mid \aexec \in \interpr{P}_{(\RP_{\LWW}, \Ax)} \}$.
\end{corollary}
\begin{proof}
See \cref{sec:cor-et-soundness}.
\end{proof}

\begin{definition}
\label{def:et_complete}
An execution test $\ET$ is \emph{complete} with respect 
to an axiomatic specification $(\RP_{\LWW}, \Ax)$ if, for any 
abstract execution $\aexec \in \CMa(\RP_{\LWW}, \Ax)$ 
such that $\AR_{\aexec} = \{(\txid_{i}, \txid_{i+1})\}_{i=1}^{\lvert \T_{\aexec} \rvert - 1}$, 
and for any $i=1,\cdots, n$, there exist two views $\vi_{i}, \vi_{i}'$ such that 
\begin{itemize}
\item $\vi_{i} = \getView(\aexec, \VIS_{\aexec}^{-1}(\txid_{i}))$, 
\item let $\txid_{i} = \txid_{\cl}^{n}$ for some $\cl, n$; if the
transaction $\txid_{i}' = \min_{\PO_{\aexec}}\{\txid' \mid \txid_i \xrightarrow{\PO_{\aexec}} \txid'\}$  
is defined, then $\vi' = \getView(\aexec, \T_{i})$, where $\T_{i} \subseteq (\AR_{\aexec}^{-1})?(\txid_{i}) \cap \VIS_{\aexec}^{-1}(\txid_{i}'))$; 
otherwise $\vi' = \getView(\aexec, \T_{i})$, where $\T_{i} \subseteq (\AR_{\aexec}^{-1})?(\txid_{i})$, 
\item $\ET \vdash (\hh_{\cut(\aexec, i-1)}, \vi_{i}) \triangleright \TtoOp{T}_{\aexec}(\txid_{i}) : \vi_{i}'$.
\end{itemize}
\end{definition}

\begin{theorem}
\label{thm:et_complete}
Let $\ET$ be an execution test that is complete with respect to 
an axiomatic specification $(\RP_{\LWW}, \Ax)$. Then 
$\CMa(\RP_{\LWW}, \Ax) \subseteq \CMs(\ET)$.
\end{theorem}
\begin{proof}
See \cref{sec:et-completeness}.
\end{proof}

%
%\begin{definition}
%An execution test $\ET$ precisely captures an axiomatic specification 
%$(\RP_{\LWW}, \Ax)$ if and only if
%\begin{itemize}
%\item given a kv-store $\hh$ and a view $\vi \in \Views(\hh)$ and a set of operations 
%$\opset$ such that $\ET \vdash \hh, \vi \triangleright \opset : \_$, then for any 
%abstract execution $\aexec$ such that $\hh_{\aexec} = \hh$, there exists a set 
%of transactions $\T = \T' \cup \Tx(\hh, \vi)$, where $\T' \subseteq \T_{\aexec}$ 
%only contains read-only transactions, such that for any $\A \in \Ax.\;\{ \txid' \mid (\txid', \txid) \in \A(\extend(\aexec, \_, \T, \opset)) \times \subseteq \T$, 
%\item given an abstract execution $\aexec$, a transaction identifier $\txid \notin \T_{\aexec}$, a 
%set of operations $\opset$ and a set of transactions $\T$ such that 
%$\forall \A \in \Ax.\{\txid' \mid (\txid', \txid) \in \A(\extend(\aexec, \txid, \T, \opset))\} \subseteq \T$, 
%then $\ET \vdash \hh_{\aexec}, \getView(\aexec, \T) \triangleright \opset : \_$. 
%\end{itemize}
%\end{definition}
%
%\begin{theorem}
%If $\ET$ precisely captures the axiomatic specification $(\RP_{\LWW}, \Ax)$, then 
%\label{thm:prooftechnique_sound}
%\begin{itemize}
%\item $\CMs((\RP_{\LWW}, \Ax)) = \{ \hh_{\aexec} \mid \aexec \in \CMa(\RP_{\LWW}, \Ax)\}$, 
%\item $\forall \prog. \interpr{\prog}_{(\RP_{\LWW}, \Ax)} = \{ \hh_{\aexec} \mid \aexec \in \interpr{\prog}_{\ET} \}$.
%\end{itemize}
%\end{theorem}

%\subsection{Operational Semantics of Programs}
\label{sec:prog-semantics}
\azalea{I rewrote most of this section.}
%\sx{
%For weak consistency models, or weak isolation levels as a common term used in database community, a thread is not necessary to work on the up-to-date version of a database in exchange for better performance. 
%Even in a single machine database, a thread running under weak consistency model can make less synchronisation with the hard drivers and other running threads, which means the thread could observe out-of-date state.
%Therefore, we introduce \emph{views} to model threads of a database.
%A \emph{view} is a cut in a history heap that corresponds the indexes of versions that a thread work with.
%We also define a order between two views, if they contain the same addresses and the indexes are ordered point-wise.
%This is to model the synchronisation between threads.
%For example \( \vi \orderVI \vi' \) could mean that a thread updates it view from \( \vi \) to \( \vi' \) by synchronisation with others.
%}

Before proceeding with the program operational semantics, we formalise the notion of \emph{execution tests}.
Execution tests are used to determine whether a transaction may commit its effects (fingerprint) to the MKVS by ensuring that its  effects comply with the underlying consistency model.

\mypar{Execution Tests}
An execution tests of a transaction is a quadruples of the form \( (\mkvs, \vi, \opset, \vi') \), where $\mkvs$ denotes the MKVS;
the $\vi$ denotes the \emph{initial} view, recorded at the beginning of the transactions; 
the $\opset$ denotes the fingerprint of the transaction; and 
$\vi'$ demotes the \emph{final} view of the transaction, obtained after committing the transaction. 
An execution test $(\mkvs, \vi, \opset, \vi')$ states that when the MKVS is described by $\mkvs$, a client with view $\vi$ is allowed to execute a single transaction with fingerprints $\opset$, commit the transaction and obtain an updated view $\vi'$. 
%
%
\azalea{What do we mean by at least $\vi'$? \sx{ meaning the lower bound of the view. The semantics  model this by shift the view at the beginning, so I think we should say update to the exact view.} }
%
%
\azalea{Why do we call these execution tests and not the consistency model? \sx{Andrea thinks it is more precise than consistency model, if I remember.}}
\begin{definition}[Execution Tests]
\label{def:consistency-models}
\label{def:executiontests}
Given the set of $\HisHeaps$ (\cref{def:his_heap}), fingerprints $\Opsets$ (\cref{def:ops}), and views $\Views$ (\cref{def:views}), the set of \emph{execution tests}, \( \como \in \Como \), is:
\[
        \ETS \eqdef  
		\setcomp{
			(\mkvs, \vi, \opset, \vi') \in \HisHeaps \times \Views \times \Opsets \times \Views
		} 
		{
		\opset\projection{2} \subseteq \dom(\vi) = \dom(\vi') = \dom(\hh)
		}       
\]
\end{definition}
%
We often write $(\mkvs, \vi) \etto \opset : \vi'$ for  $(\mkvs, \vi, \opset, \vi') \in \et$.
\sx{
    Maybe also some version of composition requirement.
    For example, the composition of two should be also included in the consistency model.
    \[
        \fora{m,m'} m \in \como \land m' \in \como \implies m \compose{} m' \in \como
    \]
    where \( \compose{} \defeq (\composeHH, \composeVI,\composeO, \composeVI)\).
}


%\begin{definition}[Execution tests]
%\label{def:consistency-models}
%\label{def:executiontests}
%Given the set of key-value stores \( \hh \in \HisHeaps \) (\cref{def:his_heap}), fingerprints \( \opset \in \Opsets \) (\cref{def:ops}) and views \( \vi, \vi' \in \Views \) (\cref{def:views}), \emph{execution tests} \( \como \in \Como \) is a set of quadruples in the form of \( ( \hh, \vi, \opset, \vi' ) \):
%\[
%    \begin{rclarray}
%        \et \in \ETS & \defeq & \powerset{\HisHeaps \times \Views \times \Opsets \times \Views}
%    \end{rclarray}
%\]
%Well-formed  execution tests \( \et \) require the domain of views and the key-value store are the same, and the fingerprints only have keys included in the previous domain:
%\[
%    \begin{rclarray}
%         &&\fora{\hh, \vi, \vi', \opset } (\hh, \vi, \opset, \vi') \in \como \implies \opset\projection{2} \subseteq \dom(\vi) = \dom(\vi') = \dom(\hh)
%    \end{rclarray}
%\]
%\sx{
%    Maybe also some version of composition requirement.
%    For example, the composition of two should be also included in the consistency model.
%    \[
%        \fora{m,m'} m \in \como \land m' \in \como \implies m \compose{} m' \in \como
%    \]
%    where \( \compose{} \defeq (\composeHH, \composeVI,\composeO, \composeVI)\).
%}
%\end{definition}
%
%

%Execution tests is a set of quadruples \( (\mkvs, \vi, \opset, \vi') \) consisting of a key-value store, a view before execution, a operation set and a view after committing of the operation set. 
%We often write \( (\mkvs, \vi) \etto \opset : \vi'\) in lieu of \( (\mkvs, \vi, \opset, \vi') \in \et\).
%The quadruple describes that when the state of the key-value store is \( \hh \), a client who has view \( \vi \) is allowed to execute a single transaction that has  the fingerprints \( \opset \), and then after the commit the thread view must be updated to at least \( \vi' \).

\ac{
There we also note that by tweaking the execution test used by the 
semantics, we capture different consistency models of 
key-value stores.
}


\begin{figure}[!t]
\sx{drop \( \func{updateView}{\hh', \vi'', \opset} \orderVI \vi'\) and use \( \vi'' \leq \vi' \) which corresponds to monotonic read. }
\hrule
%
\[
\begin{rclarray}
	\toT{}  & : &
    \ClientID \; \times \;
	\left( ( \HisHeaps \times \Views \times \Stacks ) \times \Commands \right) 
	\; \times\; \Como \;\times \;
	\left( ( \HisHeaps \times \Views \times \Stacks ) \times \Commands \right) 
	\vspace{5pt}
\end{rclarray}
\]
\begin{mathpar}
    \inferrule[\rl{PCommit}]{%
        \vi \orderVI  \vi''
        \\ \h = \clpsHH{\hh,\vi''}
        \\ (\stk, \h, \unitO), \trans \ \toL^{*} \  (\stk', \stub,  \opset) , \pskip
        \\ \txid \in \func{nextTxid}{\hh,\cl}  
        \\\\ \hh' = \func{updateMKVS}{\hh, \vi'', \txid, \opset}  
        \\ \func{updateView}{\hh', \vi'', \opset} \orderVI \vi'
        \\ (\hh, \vi) \csat \opset : \vi'
    }{%
        \cl \vdash ( \hh, \vi, \stk ), \ptrans{\trans} \ \toT{\como} \ ( \hh', \vi', \stk' ) , \pskip
    }
    \and
    \inferrule[\rl{PAssign}]{
        \val = \evalE{\expr}
    }{%
        \cl \vdash ( \hh, \vi, \stk ) , \passign{\var}{\expr} \ \toT{\como} \  ( \hh, \vi, \stk\rmto{\var}{\val} ) , \pskip
    }
    \and
    \inferrule[\rl{PAssume}]{%
        \evalE[\thstk]{\expr} \neq 0
    }{%
        \cl \vdash ( \hh, \vi, \stk ) , \passume{\expr} \ \toT{\como} \  ( \hh, \vi, \stk ) , \pskip
    }
    \and
    \inferrule[\rl{PChoice}]{
        i \in \Set{1,2}
    }{%
        \cl \vdash ( \hh, \vi, \stk ) , \cmd_{1} \pchoice \cmd_{2} \ \toT{\como} \  ( \hh, \vi, \stk ) , \cmd_{i}
    }
    \and
    \inferrule[\rl{PIter}]{ }{%
        \cl \vdash ( \hh, \vi, \stk ) , \cmd\prepeat \ \toT{\como} \  ( \hh, \vi, \stk ) , \pskip \pchoice (\cmd \pseq \cmd\prepeat)
    }
    \and
    \inferrule[\rl{PSeqSkip}]{ }{%
        \cl \vdash ( \hh, \vi, \stk ) , \pskip \pseq \cmd \ \toT{\como} \  ( \hh, \vi, \stk ) , \cmd
    }
    \and
    \inferrule[\rl{PSeq}]{% 
        \cl \vdash ( \hh, \vi, \stk ) , \cmd_{1} \ \toT{\como} \  ( \hh, \vi', \stk' ) , {\cmd_{1}}' 
    }{%
        \cl \vdash ( \hh, \vi, \stk ) , \cmd_{1} \pseq \cmd_{2} \ \toT{\como} \ ( \hh, \vi', \stk' ) , {\cmd_{1}}' \pseq \cmd_{2}
    }\vspace{5pt}
\end{mathpar}
\begin{flushleft} 
with
\quad
$
\func{nextTxid}{\hh,\cl}  \eqdef
\Setcon{ \txid^{\cl} }{ 
	\txid^{\cl} \in \TxID \land \fora{\addr, i, \txid} \txid^{\cl} > \WTx(\hh(\addr,i)) 
	\land \txid^{\cl} > \txid \land \txid \in \RTx(\hhR(\addr,i))
} 
$, and
\vspace{5pt}
 \end{flushleft}
%
\[
\begin{rclarray}         
%	 \func{updateMKVS}{., ., ., .} & : & \MKVSs \times \Views \times  \\                        
    \func{updateMKVS}{\hh, \vi, \txid, \unitO} & \defeq & \hh \\
    \func{updateMKVS}{\hh, \vi, \txid, \opset \uplus \Set{(\otR, \ke, \stub)}} & \defeq &  
    \begin{array}[t]{@{}l}
        \texttt{let } (\nat, \txid', \txidset) = \hh(\ke, \vi(\ke)) \\
        \texttt{and } \hh' = \hh\rmto{\ke}{%
            \hh(\ke)\rmto{\vi(\addr)}{%
                (\nat, \txid', \txidset \uplus \Set{\txid}) } } \\
        \texttt{ in } \func{updateMKVS}{\hh', \vi, \txid, \opset}
    \end{array} \\
    \func{updateMKVS}{\hh, \vi, \txid, \opset \uplus \Set{(\otW, \ke, \nat)}} & \defeq &  
    \begin{array}[t]{@{}l}
        \texttt{let } \hh' = \hh\rmto{\ke}{ ( \hh(\ke) \lcat \List{(\nat, \txid, \emptyset)} ) } \\
        \texttt{ in } \func{updateMKVS}{\hh', \vi, \txid, \opset}
    \end{array} 
%
\end{rclarray}
\]
\begin{flushleft} and \end{flushleft}
%
\[
\begin{rclarray}
    \func{updateView}{\hh, \vi, \unitO} & \defeq & \vi \\
    \func{updateView}{\hh, \vi, \opset \uplus \Set{(\otR, \ke, \stub)}} & \defeq & \func{updateView}{\hh, \vi, \opset}\\
    \func{updateView}{\hh, \vi, \opset \uplus \Set{(\otW, \ke, \stub)}} & \defeq & \func{updateView}{\hh, \vi\rmto{\addr}{(\left| \hh(\addr) \right| - 1)}, \opset}\\
%
%%              
%	\func{fresh}{\hh}  & \defeq & 
%	\Setcon{ \txid }{ 
%		\txid \in \TxID \land \fora{\addr, i} \txid \neq \WTx(\hh(\addr,i)) \\
%		\land\ \txid \notin \RTx(\hhR(\addr,i))
%	} 
\end{rclarray}
\]
\vspace{5pt}
\hrule%\vspace{5pt}
%\begin{flushleft}
%The thread environment is a partial function from thread identifiers to pairs of stacks and views \( \thdenv \in \ThdEnv \defeq \ThreadID \parfinfun \Stacks \times \Views \).
%Given the set of execution tests \( \ConsisModels \) (\cref{def:consistency-models}) and key-value stores \(\HisHeaps\) (\cref{def:mkvs}), the \emph{semantics for programs}:
%\end{flushleft}
\[
	\toG{} : 
    ( \Confs \times \ThdEnv \times \Programs) 
    \;\times\; \Como \;\times\;
    ( \Confs \times \ThdEnv \times \Programs) 
\]
\begin{mathpar}
    \inferrule[\rl{PSingleThread}]{%
        \cl \vdash ( \mkvs, \viewFun(\thid), \thdenv(\thid) ), \prog(\thid), \ \toT{\como} \  ( \mkvs', \vi', \stk' ) , \cmd'  
    }{%
        ( (\mkvs, \viewFun), \thdenv, \prog ) \ \toG{\como} \  ( \mkvs', \viewFun\rmto{\thid}{\vi'}) \thdenv\rmto{\thid}{\stk'} , \prog\rmto{\thid}{\cmd'} ) 
    }
\end{mathpar}
%
\hrule
\caption{Operational semantics of commands (above) and programs (below)}
\label{def:thread_semantics}
\label{fig:thread_semantics}
\label{def:thread_pool_semantics}
\label{fig:thread_pool_semantics}
\label{def:program_semantics}
\label{fig:program_semantics}
\end{figure}


\mypar{Operational Semantics of Commands}
The \emph{operational semantics of commands} is given at the top of \cref{fig:thread_semantics}. 
Command transitions are of the form $(\mkvs, \vi, \stk), \cmd \ \toT{\et} \ (\mkvs', \vi', \stk') , \cmd'$, stating that given the MKVS $\mkvs$, view $\vi$ and stack $\stk$, executing command $\cmd$ for one step under $\como$, updates the MKVS to $\mkvs'$, the stack to $\stk'$, and the command to its continuation $\cmd'$. 
Note that $\ET$ denotes the set of execution tests, prescribing the permissible executions of transactions. 
We keep the command operational semantics in \cref{fig:thread_semantics} parametric in the choice of execution tests. 
Later in \cref{sec:cmexamples} we present several examples of execution tests of well-known consistency models in the literature. 

With the exception of the \rl{PCommit} transition, the remaining command transitions are standard and behave as expected. We write 
In \cref{fig:thread_semantics} we write $\lcat$ for denotes list concatenation.
We write $f \rmto{a}{b}$ for function update: $f \rmto{a}{b}(a) = b$, and for all $c \ne a$, $f \rmto{a}{b}(c) = f(c)$.

%The notation \( l\rmto{i}{k} \) on a list \( l \) means the result by replacing the \emph{i-th} element to \( k \).
%%To record the index of list starts from 0.
%The \( \lcat \) denotes list concatenation.

The premise of \rl{PCommit} state that the current view $\vi$ of the executing command maybe advanced to a newer view $\vi''$ (see \cref{def:views}). 
Given the new view $\vi''$, the transaction proceeds by obtaining a snapshot $\sn$ of the MKVS $\mkvs$, and executing $\trans$ locally to completion ($\pskip$), updating the stack to $\stack'$, while accumulating the fingerprint $\opset$. Note that the resulting snapshot is ignored (denoted by $\stub$) as the effect of the transaction is recorded in the fingerprint $\opset$. 
%

The transaction is now ready to commit and may propagate its changes to $\mkvs$.
To this end, a \emph{fresh} transaction identifier $\txid$ is picked (\ie one that does not appear in $\mkvs$ as defined in \cref{fig:thread_semantics}) to identify the completed transaction, and the changes performed by $\txid$ are propagated to $\mkvs$. 
This is done via the $\func{updateMKVS}{\mkvs, \vi'', \txid, \opset}$ function (defined in \cref{fig:thread_semantics}) to update $\mkvs$ to  $\mkvs'$. 
As expected, for every read operation $(\otR, \ke, -)$ in fingerprint $\opset$, the readers of $\ke$ at index $\vi(\ke)$ are extended with $\txid$.
For every write operation $(\otW, \ke, \val)$ in fingerprint $\opset$, the $\mkvs(\ke)$ entry is extended with a new version $(\val, \txid, \emptyset)$, denoting that $\txid$ is responsible for creating this version which has no readers as of yet. 

Once the MKVS is updated to $\mkvs'$, the client subsequently updates its view $\vi''$ with respect to its fingerprint via the $\func{updateView}{\mkvs', \vi'', \opset}$ function, defined in \cref{fig:thread_semantics}.
Whilst the client does not need to update its view for those keys it has read from, 
%for each key $\ke$ that the client has written to, 
it must update its view to the \emph{latest} version available for those keys it has written to. %$\ke$. 
This definition of $\funcFont{updateView}$ imposes a lower bound on the updated view by ensuring that the view of the client is up-to-date for all those keys it has written to. 
This in turn guarantees a strong program order, meaning that the following transactions of the same client will at least read the previous writes of the client  itself.
Assuming that client commands are wrapped within a single session, this lower bound of the view corresponds to the strong session guarantees introduced by \cite{.........}.
%\azalea{I have rewritten this whole section. However, I don't quite understand the motivation behind $\func{updateView}{\mkvs', \vi'', \opset}$. Please elaborate.}

The updated view of the client may then be advanced to a newer view $\vi'$.
Lastly, to ensure that the effect of the transaction (its fingerprint  $\opset$) is permitted by the underlying consistency model, 
the final premise of the transition requires that the updates be permitted by the execution test $\ET$, \ie \( (\mkvs, \vi'') \etto \opset : \vi'\).





%First, the view can shift to later versions before executing the transaction to model the client might gain more information about the key-value store since its last commit.
%To recall The order between two views with the same domain, for example \( \vi'' \geq \vi \) in \rl{PCommit}, is defined by the order of the indexes (\cref{def:views}).
%This new local view \( \vi'' \) should also be consistent with the key-value stores, \ie it leads to a situation where the current client is allowed to execute the transaction.
%The transaction code \( \trans \) is executed locally given an initial snapshot \( \sn = \clpsHH{\hh, \vi''}\) (\cref{def:snapshot}) decided by the current state of key-value store \( \mkvs \) and the local view \( \vi'' \).
%The \( \funcn{localHeap} \) function uniquely determined a (local) heap from a history heap and a view by picking the versions of addresses indexed by the view.
%After local execution via the semantics for transactions (\cref{fig:transaction_semantics}), we propagate the stack \( \stk' \) and more importantly obtain the fingerprints \( \opset \), while the snapshot \( \sn' \) will be throw away.
%Then the transaction picks a fresh identifier \( \txid \), \ie one that does not appear in the key-value store, and commits the fingerprint \( \opset \), which will update the key-value store and local view.
%%The operation set includes the first read and last write of each key, which are the operations might affect the key-value store, because of the atomicity of transactions.
%The \funcn{updateMKVS} function updates the history heap using the fingerprint, \( \mkvs' = \func{updateMKVS}{\mkvs,\vi'',\txid, \opset}\).
%For read operations, it includes the new identifier to the read set of the version the is pointed by the local view \( \vi''\).
%For write operations of the form \( (\otW, \ke, \nat) \), it extends a new version written by the new transaction \( (\nat, \txid, \emptyset) \) to the tail of \( \mkvs(\ke) \).
%For updating the view, we set a lower bound for the new local view by \funcn{updateView} function.
%Assuming the commands executed by clients are wrapped with in a single session, the lower bound of the view corresponds to the strong session guarantees introduced by \cite{.........}.
%This function shifts the view to the up-to-date version in the new key-value store if the version is installed by the current transaction.
%This guarantees strong program order, meaning the following transaction will at least read its own write.
%Finally, the actual new local view \( \vi' \) is any view greater than the lower bound \( \vi' \geq \func{updateView}{\mkvs', \vi'', \opset}\).
%The overall execution satisfied the execution tests, \ie \( (\mkvs, \vi'') \etto \opset : \vi'\).

\ac{The paragraph below should probably go when discussing the rules of the semantics:

Note that the way in which MKVSs and views are updated ensure the following: 
$\bullet$ a client always reads its own preceding writes; 
$\bullet$ clients always read from an increasingly up-to-date state of the database; 
$\bullet$ the order in which clients update a key $\key{k}$ is consistent with the 
order of the versions for such keys in the MKVS; 
$\bullet$ writes take place after reads on which they depend. 
}



\mypar{Program Operational Semantics}
The \emph{operational semantics of programs} are given at the bottom of \cref{fig:program_semantics}. 
Programs transitions are of the form $(\conf,  \thdenv, \prog) \ \toG{\como} (\conf',  \thdenv', \prog')$,
stating that given the configuration $\mkvs$ and the \emph{client environment} $\thdenv$, executing program $\prog$ for one step under $\como$, updates the configuration to $\conf'$, the client environment to $\thdenv'$, and the program to its continuation $\prog'$. 
A \emph{client environment}, $\thdenv \in \ThdEnv$, tracks the local variable stack ($\stack$). 
That is, a client environment is a mapping from client identifiers to pairs of stacks and views. 
We assume that the client identifiers in the domain of client environments are are those in the domain of the program throughout the execution. 
%$\prog$: $\dom(\thdenv) = \dom(\prog)$; and that 
Program transitions are simply defined in terms of the transitions of their constituent client commands, defined by $\toT{\como}$. 
This in turn yields the standard interleaving semantics for concurrent programs. 

%Last, the program has standard interleaving semantics by picking a client and then progressing one step (\cref{fig:thread_pool_semantics}).
%To achieve that a thread environment holds the stacks and views associated with all active clients \( Env \in \sort{ThdEnv} \).
%We assume the client identifiers from client environment match with those in the program \( \prog \).
%We also assume all the stacks and views initially are the same respectively.



\begin{lemma}
\label{lem:hhupdate.welldefined}
%The function $\HHupdate$ is well-defined over well-formed MKVS, fingerprints and views. 
Given a well-formed MKVS $\hh$, a view $\vi$ that is well-formed with respect to \( \mkvs \), a fingerprint \( \opset \), and a $\txid$ that does not appear in $\hh$, then $\HHupdate(\hh, \opset, \tsid, \vi)$ is uniquely determined and yields a well-formed MKVS.
\end{lemma}

\begin{lemma}
%The function $\Vupdate$ is well-defined.
Given a well-formed MKVS $\hh$, a view $\vi$ that is well-formed with respect to \( \mkvs \), and a fingerprint \( \opset \), the $\Vupdate(\hh, \opset, \vi)$ uniquely determines a view which is well-formed with respect to $\mkvs$.
\end{lemma}

%\begin{lem}[Confluence of \funcn{updateMKVS} and \funcn{updateView}]
%Given a valid operation set \( \opset \), the order of applying \( \funcn{updateMKVS} \) function (\( \funcn{updateView} \) function) to the elements from operation set does not affect the final result.
%\end{lem}
                                                                                                         

