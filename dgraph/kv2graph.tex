\subsection{Relationship between key-value stores and dependency graphs}
\ac{It could be that this subsection gets moved to a different section, where 
we also relate specifications of consistency models using dependency graphs
with execution tests.}
\emph{Dependency graphs} are introduced by Adya to specify consistency models of transactional databases \cite{adya}. 
They are directed graphs consisting of transactions as nodes, 
each of which is labelled with transaction identifier and a set of read and write operations,
and labelled edges between transactions for specifying how information flows between nodes. 
Specifically, a transaction $\txid$ reads a version for a key $\ke$ that has been written by another transaction $\txid'$ 
(\emph{write-read dependency} \( \WR\)), overwrites a version of $\ke$ written by $\txid'$ (\emph{write-write dependency} \( \WW \)),
or reads a version of $\ke$ that is later overwritten by $\txid'$ (\emph{read-write anti-dependency} \( \RW \)).
Note that here we purposely use the same names \( \WR, \WW, \RW \) as those used in key-value store,
since there is one-to-one map between key-value stores and dependency graphs.

\begin{figure}[t]
\begin{center}

\hrulefill

\begin{halfsubfig} 
\begin{center}
\begin{tikzpicture}[scale=0.85, every node/.style={transform shape}]
%\draw[help lines] grid(6,4);

\node(t0wx) at (-1,2) {$(\otW, \ke_1, \val_1)$}; 
\path (t0wx.south) + (0,-0.2) node[anchor=north] (t0wy) {$(\otW, \ke_2, \val'_1)$};
\path (t0wx.north east) + (1,0.5) node[anchor = west] (t1ry) {$(\otR, \ke_2, \val'_1)$}; 
\path (t1ry.east) + (0.2,0) node[anchor = west] (t1wx) {$(\otW, \ke_1, \val_2)$};
\path (t0wy.south east) + (1,-0.5) node[anchor = west] (t2rx) {$(\otR, \ke_1, \val_1)$};
\path (t2rx.east) + (0.2,0) node[anchor = west] (t2wy) {$(\otW, \ke_2, \val'_2$)};

\begin{pgfonlayer}{background}
\node[background, fit=(t0wx) (t0wy)] (t0) {};
\node[background, fit= (t1ry) (t1wx)] (t1) {};
\node[background, fit= (t2rx) (t2wy)] (t2) {};

\path(t0.west) node[anchor=east] (t0lbl) {$\txid_1$};
\path(t1.north) node[anchor=south] (t1lbl) {$\txid_2$};
\path(t2.south) node[anchor=north] (t2lbl) {$\txid_3$};

\path[->]
(t0.north) edge[bend left=70] node[above, yshift=7pt, xshift=-1pt, pos=0.3] {$\RF(\ke_2), \VO(\ke_1)$} (t1.west)
(t0.south) edge[bend right=70] node[below, yshift=-8pt, xshift=-1pt, pos=0.3] {$\RF(\ke_1), \VO(\ke_2)$} (t2.west)
([xshift=-8pt]t2.north) edge[bend left=40] node[left] {$\AD(\ke_1)$} ([xshift=-8pt]t1.south) 
([xshift=8pt]t1.south) edge[bend left=40] node[right] {$\AD(\ke_2)$} ([xshift=8pt]t2.north);
\end{pgfonlayer}

\end{tikzpicture}
\end{center}
\caption{Dependency graph}
\label{fig:dependency-graph}
\end{halfsubfig} 
%
\begin{halfsubfig} 
\begin{center}
\begin{tikzpicture}[scale=0.85, every node/.style={transform shape}]
%\draw[help lines] grid(6,4);

\node(t0wx) at (-1,2) {$(\otW, \ke_1, \val_1)$}; 
\path (t0wx.south) + (0,-0.2) node[anchor=north] (t0wy) {$(\otW, \ke_2, \val'_1)$};
\path (t0wx.north east) + (1,0.5) node[anchor = west] (t1ry) {$(\otR, \ke_2, \val'_1)$}; 
\path (t1ry.east) + (0.2,0) node[anchor = west] (t1wx) {$(\otW, \ke_1, \val_2)$};
\path (t0wy.south east) + (1,-0.5) node[anchor = west] (t2rx) {$(\otR, \ke_1, \val_1)$};
\path (t2rx.east) + (0.2,0) node[anchor = west] (t2wy) {$(\otW, \ke_2, \val'_2$)};

\begin{pgfonlayer}{background}
\node[background, fit=(t0wx) (t0wy)] (t0) {};
\node[background, fit= (t1ry) (t1wx)] (t1) {};
\node[background, fit= (t2rx) (t2wy)] (t2) {};

\path(t0.west) node[anchor=east] (t0lbl) {$\txid_1$};
\path(t1.north) node[anchor=south] (t1lbl) {$\txid_2$};
\path(t2.south) node[anchor=north] (t2lbl) {$\txid_3$};

\path[->]
(t0.north) edge[bend left=70] node[above, yshift=7pt, xshift=-1pt, pos=0.3] {$\VIS, \AR$} (t1.west)
(t0.south) edge[bend right=70] node[below, yshift=-8pt, xshift=-1pt, pos=0.3] {$\VIS, \AR$} (t2.west)
%([xshift=-8pt]t2.north) edge[bend left=40] node[left] {$\AD(\ke_1)$} ([xshift=-8pt]t1.south) 
([xshift=8pt]t1.south) edge[bend left=40] node[right] {$\AR$} ([xshift=8pt]t2.north);
\end{pgfonlayer}

\end{tikzpicture}
\end{center}
\caption{Abstract execution}
\label{fig:abstract-execution}
\end{halfsubfig} 
\end{center}
\hrulefill

\caption{Dependency graph and abstract execution}

\end{figure}


\begin{definition}
A \emph{dependency graph} is a quadruple $\Gr = (\TtoOp{T}, \RF, \VO, \AD)$, where
\begin{itemize}
\item 
    $\TtoOp{T}_0: \TxID \parfinfun \powerset{\Ops}$ is a partial finite mapping from transaction identifiers 
    to the set of operations, where there are at most one read operation and one write operation per key;
\item 
    $\RF : \Keys \to \pset{\dom(\TtoOp{T}) \times \dom(\TtoOp{T})}$ is a function that 
maps each key $\ke$ into a relation between transactions, such that for any $\txid, \txid_1, \txid_2, 
\ke, \cl, m, n$: 
\begin{itemize}
\item if $(\otR, \ke, \val) \in \TtoOp{T}(\txid)$, either $\val = \val_0$ 
and there exists no $\txid'$ such that $\txid' \xrightarrow{\RF(\ke)} \txid$,  
or there exists $\txid'$ such that $(\otW, \ke, \val) \in \TtoOp{T}(\txid')$, and $\txid' \xrightarrow{\RF(\ke)} \txid$, 
\item if $\txid_1 \xrightarrow{\RF(\ke)} \txid$ and $\txid_2 \xrightarrow{\RF(\ke)} \txid$, then 
$\txid_2 = \txid_1$.
\item if $\txid_{\cl}^{m} \xrightarrow{\RF(\ke)} \txid_{\cl}^{n}$, then $m < n$.
\end{itemize}
\item $\VO: \Keys \to \pset{\dom(\TtoOp{T}) \times \dom(\TtoOp{T})}$ is a function 
that maps each key into an irreflexive relation between transactions, such that for any $\txid, \txid', \ke, \cl, m, n$, 
\begin{itemize}
\item if $\txid \xrightarrow{\VO(\ke)} \txid'$, then $(\otW, \ke, \_) \in \TtoOp{T}(\txid), (\otW, \ke, \_) \in \TtoOp{T}(\txid')$, 
\item if $(\otW, \ke, \_) \in \TtoOp{T}(\txid), (\otW, \ke, \_) \in \TtoOp{T}(\txid')$, then either $\txid = \txid'$, 
$\txid \xrightarrow{\VO(\ke)} \txid'$, or $\txid' \xrightarrow{\VO(\ke)} \txid$.
\item if $\txid_{\cl}^{m} \xrightarrow{\WW(\ke)} \txid_{\cl}^{n}$, then $m < n$.
\end{itemize}
\item $\AD: \Keys \to \pset{\dom(\TtoOp{T}) \times \dom(\TtoOp{T})}$ is defined 
by letting $\txid \xrightarrow{\AD(\ke)} \txid'$ if and only if $(\otR, \ke, \_) \in \TtoOp{T}(\txid)$, 
$(\otW, \ke, \_) \in \TtoOp{T}(\txid')$ and 
either there exists no $\txid''$ such that $\txid'' \xrightarrow{\RF(\ke)} \txid$, or 
$\txid'' \xrightarrow{\RF(\ke)} \txid$, $\txid'' \xrightarrow{\VO(\ke)} \txid'$ for 
some $\txid''$.
\end{itemize}
\end{definition}
Given a dependency graph $\Gr = (\TtoOp{T}, \RF, \VO, \AD)$, we often 
commit an abuse of notation and use $\RF$ to denote the relation 
$\bigcup\limits_{\ke \in \Keys} \RF(\ke)$; a similar notation is adopted for $\VO, \AD$. 
It will always be clear from the context whether the symbol $\RF$ refers to a function 
from keys to relations, or to a relation between transactions. 

%A dependency graph $\Gr = (\TtoOp{T}, \RF, \VO, \AD)$ is well-formed if 
%$(\PO \cup \RF \cup \VO)$ is acyclic, i.e. its transitive closure is irreflexive. 
%Henceforth, we always assume that dependency graphs are well-formed, 
%and we let 
We let $\Dgraphs$ be the set of all dependency graphs.
It is always possible to convert a kv-store $\hh$ into a well-formed dependency 
graph. For example, \cref{fig:hheap-b} illustrates the dependency graph constructed 
from the kv-store depicted in \cref{fig:hheap-a}.

\begin{definition}
\label{def:kv2graph}
Given a kv-store $\hh$, the \emph{dependency graph} $\Gr_{\hh} = (\TtoOp{T}_{\hh}, \RF_{\hh}, 
\VO_{\hh}, \AD_{\hh})$ is defined as follows: 
\begin{itemize}
\item for any $\txid \neq \txid_0$, $\TtoOp{T}_{\hh}(\txid)$ is defined if and only if there exists an index $i$ and a key 
$\ke$ such that either $\txid = \WTx(\hh(\ke, i))$, or $\txid \in \RTx(\hh(\ke,i))$; furthermore, 
$(\otW, \ke, \val) \in \TtoOp{T}(\txid)$ if and only 
if $\txid = \WTx(\hh(\ke, i))$ for some $i$, and 
$(\otR, \ke, \val) \in \TtoOp{T}(\txid)$ if and only if $\txid \in \RTx(\hh(\ke, i))$ for some $i$, 
\item $\txid \xrightarrow{\RF(\ke)} \txid'$ if and only if there exists an index $i: 0 < i < \lvert \hh(\ke) \rvert$ 
such that $\txid = \WTx(\hh(\ke, i))$, and $\txid' \in \RTx(\hh(\ke, i))$, 
\item $\txid \xrightarrow{\VO(\ke)} \txid'$ if and only if there exist two indexes $i,j$: $0 < i < j < \lvert \hh(\ke) \rvert$ 
such that $\txid = \WTx(\ke, i)$, $\txid' = \WTx(\ke, j)$, 
\item $\txid \xrightarrow{\AD(\ke)} \txid'$ if and only if there exist two indexes $i,j$: $0 < i < j < \lvert \hh(\ke) \rvert$ 
such that $\txid \in \RTx(\ke, i)$ and $\txid' = \WTx(\ke, j)$.
\end{itemize}
\end{definition}

\begin{theorem}
\label{thm:kv2graph}
There is a one-to-one map between key-value stores and dependency graphs.
\end{theorem}
\begin{proof}
See \cref{sec:kv2graph-proof}.
\end{proof}
