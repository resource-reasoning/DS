\subsubsection{Code}
\paragraph{\bf Structure}
COPS is a fully replicated database implementation for causal consistency.
There are two versions, 
that the simplified version only supports either a single read or a single write per transaction, 
and the full version supports either multiple reads or a single write pe transactions.
Here we verify the full version.

The overall database is modelled as a key-value store.
Each key, instead of a single value, is associated with a list of versions,
consisting of value, version number \verb|VersionNo| and dependencies \verb|Dep|.
COPS relies on the version number to resolve conflict,
that is, the write with greater version number wins.
Version number composite by time (higher bits) and replica identifier (lower bits).
Since the replica identifiers are full ordered, therefore version numbers are full ordered.
The dependencies is a set of versions (pairs of keys and versions numbers)
that the current version depends on.

\begin{lstlisting}[caption={COPS Structure},label={lst:cops-structure}]
VersionNo :: LocalTime + ID
Dep :: Set(Key,VersionNo)
KV :: Key -> List(Val,VersionNo,Dep)
\end{lstlisting}

Under the hood, there are many replicas,
where each replica has a unique identifier
and contains the full key-value store yet might be out of data.
Replica also tracks its local time.

\begin{lstlisting}[caption={COPS Replicas},label={lst:cops-replica}]
Replicas :: ID -> (KV,LocalTime)
\end{lstlisting}

To track session order, Client interact with a replica via certain API.
To call those API,
client has the responsibility to provide context \verb|Ctx| 
which contains versions that has been read from or written to the replica from the same client.

\begin{lstlisting}[caption={Client context},label={lst:cops-client-ctx}]
Ctx :: {
    readers : List(Key,VersionNo,Dep)
    writers : List(Key,VersionNo,Dep)
}
\end{lstlisting}

\paragraph{\bf Write}
The client call \verb|put| to commit a new write to a key \verb|k| with value \verb|v| with the context \verb|ctx|.
It extracts the dependencies from the context, by unioning all the versions inside the context,
then calls the replica API \verb|ver = put_after(k,v,deps,dep_to_nearest(deps),dep)|,
which require to provide the dependencies so that the replica can check whether they are exists.
Note \verb|dep_to_nearest(deps)| is for performance by providing the latest versions.
The replica returns a version number allocated for the new version for key \verb|k|,
and then the version is added into the context.

\begin{lstlisting}[caption={Client API for write},label={lst:cops-client-api-write}]
Ctx put(k,v,ctx) {

    // add up all the read and write dependency
    deps = ctx_to_dep(ctx);
    // call replica API
    ver = put_after(k,v,deps,dep_to_nearest(deps));
    // update context
    ctx.writers += (k,ver,deps);
    return ctx;
}

Dep ctx_to_dep(ctx) {
    return { (k,ver) | (k,ver,_) (*$\in$*) ctx.readers (*$\lor$*)
                (k,ver,_) (*$\in$*) ctx.writers }
}

Dep dep_to_nearest(deps) {
    return { (k,ver) | (*$\forall$*)k',ver',deps'.
            (k',ver',deps') (*$\in$*) deps (*$\implies$*) (k,ver) (*$\notin$*) deps' };
}
\end{lstlisting}

The \verb|put_after| waits until all the versions contained in \verb|nearest| exists,
consequently all the versions contained in \verb|deps| exists.
The replica increments the local time and insert the new version with version numbers \verb|time ++ id| (local time concatenating replica identifier) to the key \verb|k|.
At this point, replica returns the new version number to client and 
later on it will broad case to other replica\footnote{It uses message queue for broad-casting}.

\begin{lstlisting}[caption={Replica API for write},label={lst:cops-replica-api-write}]
VersionNo put_after(k,v,deps,nearest,vers){
    for (k,ver) in nearest
        wait until (_,ver,_) (*$\in$*) kv(k);

    time = inc(local_time);

    // appending local kv with a new version.
    list_isnert(kv[k], (v, (time ++ id), deps) );

    asyn_brordcase(k, v, (time ++ id), deps);
    return (local_time + id);
}
\end{lstlisting}

When a replica receives a update message, it checks the existence of versions included in the dependencies and then adds the new version to the replica.
Last, the replica updates the local time if the new version's time is greater than the local time.


\begin{lstlisting}[caption={Receive update message},label={lst:cops-replica-receive-msg}]
on_receive(k,v,ver,deps) {
    for (k',ver') in deps {
        wait until (_,ver',_) (*$\in$*) kv(k');
    }

    list_isnert(kv[k],(v,ver,deps));
    (remote_local_time + id) = ver;
    local_time = max(remote_local_time, local_time);
}
\end{lstlisting}


\paragraph{\bf Read}
To read multiple keys \verb|ks| in a transaction, client calls the \verb|get_trans(ks,ctx)|.
Note that between two reads for different keys, 
the replica might be interleave and schedule other transactions.
The challenge here is to ensure all the values are consistent, \ie
overall they satisfy the causal consistency.
That is, if the transaction fetches a version \( \ver \) for key \( \ke \),
and this version \( \ver \) depends on anther version \( \ver' \) for another key \( \ke' \), 
then the transaction should at least fetches \( \ver' \) for the key \( \ke' \),
or any later version  for the key \( \ke' \).

The algorithm, in the first phase, optimistically reads the current latest version for each key from the replica via the replica API \verb|rst[k]=get_by_version(k,LATEST)|.
In the second phase, it computes the maximum version \verb|ccv[k]| from any dependencies \verb|rst[k].deps| read in the first phase.
Such \verb|ccv[k]| is the minimum version that should be fetched.
Therefore at the end, it only needs to re-fetched the specifically version \verb|ccv[k]|,
if the version fetched in the first phase is older than \verb|ccv[k]|.

\begin{lstlisting}[caption={Reads},label={lst:cops-client-read}]
List(Val) get_trans(ks,ctx) {
    for k in ks {
        // only guarantee to read up-to-date value 
        // the moment reading the individual key
        rst[k] = get_by_version(k,LATEST);
    }

    for k in ks {
        ccv[k] = max (ccv[k],rst[k].ver);
        for dep in rst[k].deps
            if ( dep.key (*$\in$*) ks )
                ccv[k] = max (ccv[dep.key],dep.vers);
    }

    for k in ks 
        if ccv[k] > rst[k].vers
            rst[k] = get_by_version(k,ccv[k]);

    // update the ctx
    for (k,ver,deps) in rst
        ctx.readers += (k,ver,deps);

    return to_vals(rst);
}                                   
\end{lstlisting}

The client API \verb|get_by_version(k,ver)| returns the version \verb|ver| for key \verb|k|.

\begin{lstlisting}[caption={Replica API for read},label={lst:cops-replica-read}]
(Val,Version,Dep) get_by_version(k,ver) {

    if (ver  = LATEST){
        ver = max(kv[k].vers);
    }

    wait until (_,ver,_) (*$\in$*) kv(k);
    
    let (val,ver,deps) from kv[k];
    return (val,ver,deps);
}
\end{lstlisting}
