A COPS client interacts with the replica via calling \( \pcopsput \) and \( \pcopsread \):
\begin{enumerate*}
\item function \( \pcopsput(\key,\val) \) commits a new value \( \val \) for  key \( \key \),
and the replica will accept this version immediately and then broadcast to other replicas via synchronised messages;
\item function \( \pcopsread(\keyset) \) requests a causally dependent snapshot for the key set \( \keyset \),
and for better performance, the replica will prepare this snapshot in a fine-grained manner, reading keys one by one.
\end{enumerate*}
The API is modelled in \cref{def:cops-commands}.
To models the fine-grained read in the replica side, 
we introduce \emph{runtime commands} \( \COPSRunCommands \) to track intermediate states of \(\pcopsread\).

\begin{definition}[COPS Commands and Programs]
\label{def:cops-commands}
A \emph{COPS command} \( \copscmd \) is defined by the following grammar:
for any key \( \key \in \Keys\), key set \( \keyset \subseteq \Keys\) and \( \val \in \Values\),
\[
\copscmd ::= \pcopsput(\key,\val) | \pcopsread(\keyset) | \copscmd \pseq \copscmd  .
\]
Let \( \COPSCommands \ni \copscmd \) denote the set of \emph{COPS commands}.
A \emph{COPS runtime command} \( \copsruncmd \) is defined by the following grammar:
for any \( \copsverlist,\copsverlist' \in \List{\COPSVersions}\) and \( \copsdepset \in \List{\COPSDependencies} \),
\[
\copsruncmd ::= \pcopsput(\key,\val) | \pcopsread(\keyset) 
        | \pcopsread(\keyset) : \copsverlist 
        | \pcopsread(\keyset) : (\copsverlist,\copsdepset) 
        | \pcopsread(\keyset) : (\copsverlist,\copsdepset,\copsverlist') 
        | \copsruncmd \pseq \copsruncmd .
\]
Let \( \COPSRunCommands \ni \copsruncmd \) denote the set of \emph{COPS runtime commands}.
The sets of \emph{COPS programs}, \( \COPSPrograms \ni \copsprog \),
and \emph{COPS runtime programs}, \( \COPSRunPrograms \ni \copsrunprog \),
are defined by \( \COPSPrograms \TypeDef \Clients \ToPFFunc \COPSCommands\)
and \( \COPSRunPrograms \TypeDef \Clients \ToPFFunc \COPSRunCommands\) respectively.
\end{definition}

A replica prepares the causally dependent snapshot for a key set \( \keyset \) in three phases,
\( \pcopsread(\keyset) : \copsverlist \),
\( \pcopsread(\keyset) : (\copsverlist,\copsdepset) \) and 
\( \pcopsread(\keyset) : (\copsverlist,\copsdepset,\copsverlist') \).
We will give the semantics later.

\begin{proposition}[COPS Runtime Command Inclusion]
\(\COPSCommands \subseteq \COPSRunCommands\).
\end{proposition}

We verify COPS via trace refinement.
To help the proofs, we label each step in the COPS trace:
each label corresponds to a rule in the semantics
(\cref{fig:cops-semantics-write,fig:cops-semantics-read,fig:cops-semantics-program}).
Any local computation is labelled with \( \lbCOPSPri \).
The reduction step for a single-write transaction is labelled with \( \lbCOPSWrite{ \opW(\key,\val), \copstxid,\copsdep} \).
As explained in \cref{sec:overall-cops},
multiple-read requests are processed in two phase internally in the replica.
The labels \( \lbCOPSOptRead{\opR(\key,\val), \copstxid,\copsdep}, \lbCOPSBound,
    \lbCOPSRefetch{\opR(\key,\val), \copstxid,\copsdep}, \lbCOPSFinishRead{\copsctx} \)
means optimistic read of \( \key \), phase change, re-fetch of \( \key \)
and return the set of versions \( \copsctx \) respectively.
Last, \( \lbCOPSSync{\copstxid} \) labels the step where replica \( \repl \) receives new version \( \copsver \).
                                                                                                          
The reference semantics for COPS commands is defined in \cref{fig:cops-semantics-write,fig:cops-semantics-read}.
Rules are labelled with information that are usefully for proving properties.

\begin{definition}[COPS Labels]
The set of \emph{COPS labels} \( \Labels \ni \lb \) is defined by 
\[
    \Labels \supseteq 
    \bigcup_{\substack{\cl \in \Clients, \repl \in \COPSReplicas
                    \\ \key \in \Keys, \val \in \Values
                    \\ \copstxid \in \COPSTxIDs, \copsdep \in \COPSDependencies } }
    \Set{\lbCOPSPri, \lbCOPSWrite{ \opW(\key,\val), \copstxid,\copsdep},
              \\ \lbCOPSOptRead{\opR(\key,\val), \copstxid,\copsdep}, \lbCOPSBound,
              \\ \lbCOPSRefetch{\opR(\key,\val), \copstxid,\copsdep}, \lbCOPSFinishRead{\copsctx}
              , \lbCOPSSync{\copstxid}}.
\]
\end{definition}

\paragraph{Rule for \pcopsput.} 
When a client calls \pcopsput function (\cref{lst:cops-put}),
it sends a new value \( \var(v) \) for key \( \pk \) 
with the context \( \copsctx \) to a replica  \( \repl \).
Once the replica receives the update request, 
it creates a new version for \( \pk \) with the new value \( \var(v) \), 
allocates this version with a new version identifier consisting of 
the next local time-stamp and the replica identifier
and assigns the client context \( \pctx \) as the dependency set 
(\cref{line:cops-put-update-kv} in \cref{lst:cops-put}).
This new version will be added into the context in \cref{line:cops-put-update-ctx} in \cref{lst:cops-put}.
Now the replica is ready to send back the acknowledgement to the client.
Last, the new version is broadcasted to all other replicas.

\begin{figure}
\begin{subfigure}{\textwidth}
\begin{mathpar}
\inferrule[\rCOPSWrite]{%
    \copstxid = \copstxid[\cl](\ts+1)
    \\ \copsver = (\val,\copstxid, \copsctx)
    \\ \copskvs' = \COPSInsert(\copskvs, \key, \copsver)
    \\ \copsctx' = \copsctx \uplus \Set{(\key,\copstxid)}
}{%
    \ToCOPSCmd{ 
    \copskvs | \copsctx | \ts | \pcopsput(\key,\val) |
    \lbCOPSWrite{\opW(\key,\val), \copstxid,\copsctx } ->
    \copskvs' | \copsctx' | \ts + 1 | \pskip }
}
\end{mathpar}

\caption{The reference semantics for \pcopsput}
\label{fig:cops-semantics-write}
\end{subfigure}

\hrulefill

\begin{subfigure}{\textwidth}
\begin{lstlisting}
// mixing the client API and system API
put(repl,k,v,ctx) {
    deps = ctx; // Dependency set as previous reads and writes(*\label{line:cops-ctx-to-deps}*) 
    // increase local time and appending local kv with a new version.
    ltime = inc(repl.local_time);(*\label{line:cops-put-inc-local}*) 
    txid = (ltime ++ repl.id);
    list_isnert(repl.kv[k],(v, txid, deps));(*\label{line:cops-put-update-kv}*)
    ctx += (k, txid); // the new id put into client context (*\label{line:cops-put-update-ctx}*) 
    asyn_brordcase(k, v, txid, deps); // broad case
}
\end{lstlisting}
\caption{The reference implementation for \pcopsput}
\label{lst:cops-put}
\end{subfigure}

\hrulefill

\caption{COPS API: \pcopsput}

\end{figure}


In \rCOPSWrite rule in \cref{fig:cops-semantics-write},
the second and third premises model \cref{line:cops-put-inc-local,line:cops-put-update-kv} 
in \cref{lst:cops-put} respectively.
Function \( \COPSInsert(\copskvs, \key, \copsver) \) inserts the version 
in a position that preserves the order.

\begin{definition}[List Insertion]
Given a COPS store \( \copskvs \in \COPSKVSs \) and 
a version \( \copsver \in \COPSVersions \) for a key \( \key \in \Keys \).
Function \( \COPSInsert(\copskvs, \key, \copsver) \) is defined by
\[
    \COPSInsert(\copskvs, \key, \copsver) \FuncDef 
    \begin{multlined}[t]
    \CodeFont{let} \ \copsverlist = \copskvs(\key)
    \ \CodeFont{and} \ \List{\copsver_0,\cdots, \copsver_i, \copsver_{i+1}, \cdots, \copsver_n} = \copsverlist
    \\ \CodeFont{where} \ \WtOf(\copsver_i) \copstxidleq  \WtOf(\copsver) \copstxidleq  \WtOf(\copsver_{i+1})
    \\ \CodeFont{in} \ \copskvs\UpdateFunc{
        \key -> \List{\copsver_0,\cdots, \copsver_i, \copsver, \copsver_{i+1}, \cdots, \copsver_n}} .
    \end{multlined}
\]
\end{definition}

\begin{proposition}[Well-defined \COPSInsert]
Given a COPS store \( \copskvs \in \COPSKVSs \) and 
a version \( \copsver \in \COPSVersions \) for a key \( \key \in \Keys \)
such that \( \WtOf(\copsver) \notin \copskvs \),
the new store \( \COPSInsert(\copskvs, \key, \copsver) \) 
is a well-formed COPS store.
\end{proposition}


\paragraph{Rules for \pcopsread.} \Cref{lst:cops-read} presents the reference implementation for \pcopsread that
a client requests a snapshot for keys \( \pks \) from a replica \( \repl \).
The semantics for \pcopsread is given in \cref{fig:cops-semantics-read}.
As explained in \cref{sec:overall-cops}, 
the replica constructs this snapshot in two phases.

\begin{figure}[!hp]
\vspace{-10pt}
\begin{subfigure}{\textwidth}
\begin{mathpar}
\inferrule[\rCOPSStartRead]{ }{%
    \ToCOPSCmd{
    \copskvs | \copsctx | \ts | \pcopsread( \keyset ) | \lbCOPSPri ->
    \copskvs |  \copsctx |  \ts | 
    \pcopsread( \keyset ) : \emptyset }
}
\and
\inferrule[\rCOPSOptRead]{%
    \Unique(\keyset)
    \\ \keyset = \List{\key_0, \cdots, \key_i, \cdots, \key_m }
    \\ \copsverlist = \List{\copsver_0, \cdots, \copsver_{i-1}}
    \\ \copskvs(\key, \Abs{\copskvs(\key)} - 1) = (\val_i, \copstxid[\cl_i][\repl_i](\ts_i), \copsdep_i) = \copsver_i
    \\ \lb =  \lbCOPSOptRead{\opR(\key_i, \val_i), \copstxid[\cl_i][\repl_i](\ts_i),\copsdep_i} 
}{%
    \ToCOPSCmd{
    \copskvs | \copsctx | \ts | \pcopsread( \keyset ) : \copsverlist | \lb ->
    \copskvs |  \copsctx |  \ts | 
    \pcopsread( \keyset ) : (\copsverlist \ListConcat \List{\copsver_i}) }
}
\and
\inferrule[\rCOPSLowerBound]{%
    \Abs{\keyset} = \Abs{\copsverlist} 
    \\ \copsdepset = \VerLower(\keyset,\copsverlist)
}{%
    \ToCOPSCmd{ \copskvs | \copsctx | \ts | \pcopsread( \keyset ) : \copsverlist |
    \lbCOPSBound ->
    \copskvs | \copsctx | \ts | \pcopsread( \keyset ) : (\copsverlist, \copsdepset)  }
}
\and
\inferrule[\rCOPSRefetch]{%
    \copsverlist' = \List{\copsver_0, \cdots, \copsver_{\idx-1}}
    \\ \copsdepset\Proj{\idx} = (\key_i, \copstxid)
    \\ {(\val_i, \copstxid_i,\copsdep_i) = \begin{cases}
        \copskvs(\key_i,m) 
                & \If  \WtOf(\copsverlist\Proj{\idx}) \copstxidleq \copstxid
                    \land \Exists{m} \copstxid = \WtOf(\copskvs(\key_i,m))
        \\ \copsverlist\Proj{\idx} & \ow
    \end{cases}}
    \\ \copsverlist'' = \copsverlist' \ListConcat \List{(\val_i, \copstxid_i,\copsdep_i)}
    \\ \lb' = \lbCOPSRefetch{\opR(\key_i, \val_i), \copstxid_i, \copsdep_i}
}{%
    \ToCOPSCmd{ \copskvs | \copsctx | \ts 
        | \pcopsread( \keyset ) : (\copsverlist, \copsdepset, \copsverlist') | \lb ->
    \copskvs | \copsctx | \ts | \pcopsread( \keyset ) : (\copsverlist, \copsdepset, \copsverlist'')  }
}
\and
\inferrule[\rCOPSFinishRead]{%
    \Abs{\keyset} = \Abs{\copsverlist'}
    \\ \copsctx' = \Set{ (\key,\copstxid) | (\key,\copstxid,\stub) \in \copsverlist'}
}{%
    \ToCOPSCmd{ \copskvs | \copsctx | \ts 
        | \pcopsread( \keyset ) : (\copsverlist, \copsdepset, \copsverlist')
        | \lbCOPSFinishRead{\copsctx'} ->
    \copskvs | \copsctx \cup \copsctx' | \ts | \pskip  }
}
\end{mathpar}


\caption{The reference semantics for \pcopsread}
\label{fig:cops-semantics-read}

\end{subfigure}

\hrulefill

\begin{subfigure}{\textwidth}

\begin{lstlisting}[caption={},label={}]
List(Val) read(repl,ks,ctx) {
    // Only guarantee to read up-to-date value at the moment
    for k in ks { rst[k] = get_by_version(repl,k,LATEST); } (*\label{line:cops-optimistic-read}*)
    // initially the list of lower bound for keys
    for k in ks { ccv[k] = rst[k].id; (*\label{line:cops-init-lower-bound}*) }
    // compute the lower bound
    for k in ks { 
        for dep in rst[k].deps 
            if ( dep.key (*$\in$*) ks ) ccv[k] = max (ccv[dep.key].id, dep.id); (*\label{line:cops-compute-lower-bound}*) 
    } 
    // re-fetch a new version for some key
    for k in ks 
        if ( ccv[k] > rst[k].vers ) rst[k] = get_by_version(k,ccv[k]); (*\label{line:cops-final-read}*)
    // update the ctx
    for (k,ver,deps) in rst { ctx.readers += (k,ver,deps); } (*\label{line:cops-read-update-ctx}*)
    return to_vals(rst); 
}                                   
\end{lstlisting}

\caption{The reference implementation for \pcopsread}
\label{lst:cops-read}

\end{subfigure}

\hrulefill

\caption{COPS API: \pcopsread}
\label{fig:cops-api-read}

\end{figure}


In the first phase (\cref{line:cops-optimistic-read} in \cref{lst:cops-read}),
replica reads the latest version for each key 
(using \( \CodeFont{get\_by\_version}\), with \( \CodeFont{LATEST} \) label).
However, interleaving may happen between two reads.
\rCOPSOptRead rule models this phase: 
the latest version \( \copsver_i \) for key \( \key_i \) 
in the replica kv-store \( \copskvs \) is added to the runtime version lists \( \copsverlist \),
that is, \( \copsverlist \ListConcat \List{\copsver_i} \).
The label for this rule tracks the client, the replica, the version and a \( \dagger \)
to indicate the first phase.

At the end of the first phase, 
replica collects all dependency sets and uses them as an lower-bound in the second phase.
This lower bound contains minimum versions in term of the version identifiers that can form
a causal dependent snapshot for the key set \( \keyset \).
In \(\rCOPSLowerBound\) rule, we use \(\VerLower( \keyset,\copsverlist ) \) 
function to model the computation shown in \cref{line:cops-compute-lower-bound}.

\begin{definition}[Version Lower-bound]
Given a key set \( \keyset \) and list of versions for the key set \( \copsverlist \) 
the function \( \VerLower( \keyset,\copsverlist ) \) is defined by
\begin{align*}
    \VerLower( \keyset,\copsverlist ) & \FuncDef 
    \Let n = \Abs{\keyset} - 1 \
    \In \List{\VerLowerN(\keyset\Proj{0},\copsverlist, 0), \cdots, \VerLowerN(\keyset\Proj{n},\copsverlist, n)} 
    \\ \VerLowerN(\key,\copsverlist, \idx) & \FuncDef
    \Tuple{ \key, \Max[\copstxidleq](\Set{\copstxid | 
            \copstxid = \WtOf(\copsverlist\Proj{\idx}) \lor 
            \Exists{\idx'} (\key, \copstxid) \in \DepOf(\copsverlist\Proj{\idx'}) 
        }) } .
\end{align*}
\end{definition}

The second phase re-fetches the version \( \copsver \) for each key \( \key \) 
contained in the lower-bound, if the version that was read in the first phase is older than \( \copsver \),
which is captured by \cref{line:cops-final-read} in \cref{lst:cops-read}.
We model this second phase by \rCOPSRefetch rule:
it re-fetches the version contained in the lower-bound \( \copsdepset \)
if the version read in the first phase, that is, \( \copsverlist\Proj{\idx} \),
is older than the version contained in the lower-bound, that is, \( copsdepset\Proj{\idx} \).

Last, the replica returns the list of versions for the key set \( \keyset \)
to the client and the client must include the versions in its context, 
which is captured by \cref{line:cops-read-update-ctx}
and modelled by \rCOPSFinishRead rule.

All the rules for read and write are lift to a program level 
standard interleaving semantics presented in \cref{fig:cops-semantics-program}:
\rCOPSClient rule models that a client takes steps in turn.

\begin{figure}

\begin{subfigure}{\textwidth}
\centering
\[
\ToCOPSProg*{\;} :  \COPSConfs \times \COPSRunPrograms \times \Labels \times \COPSConfs \times \COPSRunPrograms 
\]

\begin{mathpar}
\inferrule[\rCOPSClient]{
    \cops(\repl) = (\copskvs,\ts)
    \\ \copsctxenv(\cl) = (\copsctx, \repl)
    \\ \copsrunprog(\cl) = \copsruncmd 
    \\ \ToCOPSCmd{ \copskvs | \copsctx | \ts | \copsruncmd | \lb ->
    \copskvs' | \copsctx' | \ts' | \copsruncmd' }
    \\ \cops' = \cops\UpdateFunc{\repl -> (\copskvs, \ts') }
    \\ \copsctxenv' = \copsctxenv\UpdateFunc{\cl -> (\copsctx', \repl) }
    \\ \copsrunprog' = \copsrunprog\UpdateFunc{\cl -> \copsruncmd }
}{%
    \ToCOPSProg{ \cops | \copsctxenv | \copsrunprog  | \lb ->  \cops' | \copsctxenv' | \copsrunprog' }
}
\and
\inferrule[\rCOPSSync]{
    \repl \neq \repl'
    \\ \cops(\repl) = (\copskvs,\ts)
    \\ \cops(\repl') = (\copskvs',\ts')
    \\ \copskvs(\key,\idx) = \copsver
    \\ \Forall{\copstxid} (\stub,\copstxid) \in \DepOf(\copsver) \implies \copstxid \in\copskvs'
    \\ \copskvs'' = \COPSInsert(\copskvs',\key,\copsver)
    \\ \copstxid[\stub][\stub](\ts'') = \WtOf(\copsver)
    \\ \cops\UpdateFunc{\repl' -> (\copskvs'',\Max(\ts',\ts''))}
}{%
    \ToCOPSProg{ \cops | \copsctxenv | \copsrunprog  | \lbCOPSSync(\repl'){\copstxid} ->  \cops' | \copsctxenv | \copsrunprog }
}
\end{mathpar}

\caption{The reference semantics for COPS synchronisation and programs}
\label{fig:cops-semantics-program}

\end{subfigure}

\hrulefill

\begin{subfigure}{\textwidth}

\begin{lstlisting}
on_receive(repl,k,v,id,deps) {
    for (k',id') in deps { wait until (_,id',_) (*$\in$*) repl.kv(k'); }

    list_isnert(repl.kv[k],(v,id,deps));
    (remote_local_time ++ replid) = id;
    repl.local_time = max(remote_local_time, repl.local_time);
}
\end{lstlisting}

\caption{The reference implementation for COPS synchronisation}
\label{lst:cops-receive-msg}

\end{subfigure}

\hrulefill

\caption{COPS synchronisation and programs}

\end{figure}


\paragraph{Synchronisation Between Replicas}
Replicas broadcast synchronisation messages for every new version that directly comes from client.
Since versions are ordered and messages are assumed to be eventually delivered,
this guarantees COPS satisfies eventual consistency, that is,
all replicas eventually have the same state.
Replica maintains queue for all new versions to be broadcasted.
Once a replica receives a message with a new version \( \copsver \),
it accepts the version only if all the versions that \(\copsver \) depends on exist in the replica.
Otherwise, the replica waits the condition to be satisfied.
This is captured by \cref{lst:cops-receive-msg}
and \rCOPSSync rule in \cref{fig:cops-semantics-program} models the synchronisation.

\begin{definition}[COPS Traces]
The set of \emph{COPS traces} \( \COPSTraces \ni \copstrc \) is defined by 
\( \COPSTraces \TypeDef \Set{\copsconfinit,\copsprog \ToCOPSProg*{\;}^* \copsconf,\copsprog' | 
        \copsconfinit \in \COPSConfInits \land \copsprog, \copsprog' \in \COPSPrograms } \).
Two traces \( \copstrc, \copstrc' \) are equivalent, written \( \copstrc \copstrceq \copstrc' \),
if an only if, the last configurations are equal,
\( \copstrc \copstrceq \copstrc' \PredDef \LastConf(\copstrc) = \LastConf(\copstrc') \).
\end{definition}

Note that a COPS trace has no unfinished read operation.
Two COPS traces are equivalent if and only if the last configurations match.

