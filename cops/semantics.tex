A COPS client interacts with a replica via two fixed APIs: \( \pcopsput \) and \( \pcopsread \).
The client commits a new value \( \val \) for a key \( \key \) to a replica \( \repl \) by calling  \( \pcopsput(\key,\val) \).
Upon receiving the new value  \( \val \) for \( \key \),
the replica assigns a new version identifier to the value, sends back the acknowledgement to the client
and then broadcasts the new version to other replicas via synchronisation messages (\cref{fig:app-cops-after-write-transaction}).
COPS only allows writing of one key per transaction.
By contrast, the client can read a list of keys \( \keyset \) by calling \( \pcopsread(\keyset) \).
Upon receiving the request,
the replica prepares a snapshot for \( \keyset \) in a fine-grained way, reading one key at a time.
To track the intermediate states of \(\pcopsread(\keyset)\), 
we introduce \emph{fine-grained commands} \( \COPSRunCommands \).
As explained in \cref{sec:overall-cops},
a replica prepares the causally dependent snapshot for \( \keyset \) in two phases.
The syntax \( \pcopsread(\keyset) : \copsverlist \) corresponds to the optimistic-read phase.
The version buffer \(  \copsverlist \) initially is empty.
For each key in \( keyset \), 
the client reads the current latest version for next key in \( \keyset \) 
and put the version in \( \copsverlist \).
The syntax \( \pcopsread(\keyset) : (\copsverlist,\copsdepset) \) corresponds to the phase change.
The re-fetch set \( \copsdepset \) contains, for each key in \( \keyset \), 
the version with the maximum version identifier over versions \( \copsver \) in \( \copsverlist \) 
or any versions on which \( \copsver \) depend.
The syntax \( \pcopsread(\keyset) : (\copsverlist,\copsdepset,\copsverlist') \) corresponds to the re-fetch phase.
For each key in \( \keyset \), 
the client reads the version \( \ver \) contained in \( \copsdepset \) and put it in \( \copsverlist' \),
if \( \ver \) has greater version identifier than the version in \( \copsverlist \).
Otherwise, the client copies the version in \( \copsverlist \) to \( \copsverlist' \).


\begin{definition}[COPS commands and programs]
\label{def:cops-commands}
The set of \emph{COPS commands}, \( \COPSCommands \ni \copscmd \), is defined by:
\[
\copscmd ::= \pcopsput(\key,\val) | \pcopsread(\keyset) | \copscmd \pseq \copscmd  
\]
for any key \( \key \in \Keys\), key set \( \keyset \subseteq \Keys\) and \( \val \in \Values\).
The set of \emph{COPS fine-grained commands}, \( \COPSRunCommands \ni \copsruncmd \), is defined by:
\[
\copsruncmd ::= \pcopsput(\key,\val) | \pcopsread(\keyset) 
        | \pcopsread(\keyset) : \copsverlist 
        | \pcopsread(\keyset) : (\copsverlist,\copsdepset) 
        | \pcopsread(\keyset) : (\copsverlist,\copsdepset,\copsverlist') 
        | \copsruncmd \pseq \copsruncmd 
\]
for any \( \copsverlist,\copsverlist' \in \List{\COPSVersions}\) and \( \copsdepset \in \List{\COPSDependencies} \).
The sets of \emph{COPS programs}, \( \COPSPrograms \ni \copsprog \),
and \emph{COPS fine-grained programs}, \( \COPSRunPrograms \ni \copsrunprog \),
are defined by: \( \COPSPrograms \TypeDef \Clients \ToPFFunc \COPSCommands\)
and \( \COPSRunPrograms \TypeDef \Clients \ToPFFunc \COPSRunCommands\) respectively.
\end{definition}

We now explain the reference implementation of the COPS protocol, and the semantics of the protocols.
For manipulating COPS program traces, we label each transition in the traces.
Given a client \( \cl \) and replica \( \repl \), the labels can be introduced intuitively as following:
\begin{enumerate}
%\item \( \cl \) conducts local computation without interacting with any replica, \( \lbCOPSPri \);
\item \( \cl \) commits a new version of a key \( \key \) to a replica \( \repl \)
                with a value \( \val \), a version identifier  \( \copstxid = (\repl,\ts) \)
                and a dependency set \( \copsdep \),
                \( \lbCOPSWrite{ \opW(\key,\val), \copstxid,\copsdep} \);
\item \( \cl \) starts of a multi-read transaction, \( \lbCOPSPri \), where \( \opS \) means the start of a multiple read transaction;
\item \( \cl \) reads a version of key \( \key \) from \( \repl \) indexed by a version identifier \( \copstxid \) 
                with a value \( \val \) and a dependency set \( \copsdep \) in the optimistic-read phase, 
                \( \lbCOPSOptRead{\opR(\key,\val), \copstxid,\copsdep} \), where \( \CodeFont{Opt}\) means a optimistic read;
\item \( \cl \) starts re-fetch phase, \( \lbCOPSBound \), where \( \opP \) means phase change;
\item \( \cl \) reads a version of key \( \key \) from \( \repl \) indexed by a version identifier \( \copstxid \)
                with a value \( \val \) and a dependency set \( \copsdep \) in the re-fetch phase,
                \( \lbCOPSRefetch{\opR(\key,\val), \copstxid,\copsdep} \), where \( \CodeFont{Ref} \) means a re-fetch read;
\item \( \repl \) returns the set of versions \( \copsctx \) to \( \cl \), \( \lbCOPSFinishRead{\copsctx} \), where \( \CodeFont{E} \) means the end of a multiple read transaction; 
and 
\item \( \repl \) receives a synchronisation message containing a version with a identifier \( \copstxid \), \( \lbCOPSSync{\copstxid} \).
\end{enumerate}

\begin{definition}[COPS labels]
The set of \emph{COPS labels}, \( \COPSLabels \ni \lb \), is defined by:
\[
    \COPSLabels \TypeDef 
    \bigcup_{\substack{\cl \in \Clients, \repl \in \COPSReplicas
                    \\ \key \in \Keys, \val \in \Values
                    \\ \copstxid \in \COPSVerIDs, \copsdep \in \COPSDependencies } }
    \Set{\lbCOPSPri, \lbCOPSWrite{ \opW(\key,\val), \copstxid,\copsdep},
              \\ \lbCOPSOptRead{\opR(\key,\val), \copstxid,\copsdep}, \lbCOPSBound,
              \\ \lbCOPSRefetch{\opR(\key,\val), \copstxid,\copsdep}, \lbCOPSFinishRead{\copsctx}
              , \lbCOPSSync{\copstxid}}.
\]
\end{definition}

COPS provides two APIs: \( \pcopsput\) and \( \pcopsread\).
A client will send the request to a replica and awaiting a reply.
For these two APIs, we explain the reference implementation first and give the semantics.
We argue that the semantics captures the implementation.

\paragraph{Reference implementation and rule for \(\pcopsput\).} 
When a client calls \pcopsput function depicted in \cref{lst:cops-put},
it sends a new value \( \pv \) for key \( \pk \) 
with the context \( \var(ctx) \) to a replica  \( \var(repl) \).
Upon receiving the new value,  
replica \( \repl \) creates a new version for \( \pk \) where:
\begin{enumerate}
\item the value is \( \pv \);
\item the identifier consists of the next available local time-stamp \( \var(ltime) \) (\cref{line:cops-put-inc-local}) 
and the replica identifier \( \var(repl.id)\) (\cref{line:cops-put-id}); and
\item the dependency set is the client context \( \var(deps) = \pctx \) (\cref{line:cops-ctx-to-deps}).
\end{enumerate}
Replica \( \var(repl) \) then commits the new version to its local store via \( \verb|list_insert| \) (\cref{line:cops-put-update-kv}),
which inserts the new version \verb|(v, txid, deps)| to the list \verb|repl.kv[k]| 
in the sense that the new list preserves the order over version identifiers.
Now the replica is ready to send back the acknowledgement message to the client comprising the new version identifier.
Upon receiving the acknowledgement message, the client updates the client context,
adding this new version in its context (\cref{line:cops-put-update-ctx}).
Last, replica \( \var(repl) \) broadcasts this new version by calling \( \verb|asyn_broadcast| \).

\begin{figure}
\begin{subfigure}{\textwidth}
\begin{mathpar}
\inferrule[\rCOPSWrite]{%
    \copstxid = \copstxid[\cl](\ts+1)
    \\ \copsver = (\val,\copstxid, \copsctx)
    \\ \copskvs' = \COPSInsert(\copskvs, \key, \copsver)
    \\ \copsctx' = \copsctx \uplus \Set{(\key,\copstxid)}
}{%
    \ToCOPSCmd{ 
    \copskvs | \copsctx | \ts | \pcopsput(\key,\val) |
    \lbCOPSWrite{\opW(\key,\val), \copstxid,\copsctx } ->
    \copskvs' | \copsctx' | \ts + 1 | \pskip }
}
\end{mathpar}

\caption{The reference semantics for \pcopsput}
\label{fig:cops-semantics-write}
\end{subfigure}

\hrulefill

\begin{subfigure}{\textwidth}
\begin{lstlisting}
// mixing the client API and system API
put(repl,k,v,ctx) {
    deps = ctx; // Dependency set as previous reads and writes(*\label{line:cops-ctx-to-deps}*) 
    // increase local time and appending local kv with a new version.
    ltime = inc(repl.local_time);(*\label{line:cops-put-inc-local}*) 
    txid = (ltime ++ repl.id);
    list_isnert(repl.kv[k],(v, txid, deps));(*\label{line:cops-put-update-kv}*)
    ctx += (k, txid); // the new id put into client context (*\label{line:cops-put-update-ctx}*) 
    asyn_brordcase(k, v, txid, deps); // broad case
}
\end{lstlisting}
\caption{The reference implementation for \pcopsput}
\label{lst:cops-put}
\end{subfigure}

\hrulefill

\caption{COPS API: \pcopsput}

\end{figure}


The semantics of \(\pcopsput\) is captured by rule \(\rCOPSWrite\) in \cref{fig:cops-semantics-write}.
The configuration comprises a COPS store \( \copskvs \), a client context \( \copsctx \) and a replica local time  \( \ts \).
The first and second premises construct the new identifier \( \clocktxid = (\repl,\ts+1) \)
and the new version \( \copsver \) as expected.
This new version is inserted into \( \copskvs \) via \( \COPSInsert \),
updating the COPS store to \( \copskvs' = \COPSInsert(\copskvs, \key, \copsver)\).

\begin{definition}[List insertion]
Given a COPS store \( \copskvs \in \COPSKVSs \) and 
a version \( \copsver \in \COPSVersions \) for a key \( \key \in \Keys \),
the function \( \COPSInsert(\copskvs, \key, \copsver) \) is defined by:
\[
    \COPSInsert(\copskvs, \key, \copsver) \FuncDef 
    \begin{multlined}[t]
    \CodeFont{let} \ \copsverlist = \copskvs(\key)
    \ \CodeFont{and} \ \List{\copsver_0,\cdots, \copsver_i, \copsver_{i+1}, \cdots, \copsver_n} = \copsverlist
    \\ \CodeFont{where} \ \IdOf(\copsver_i) \copstxidleq  \IdOf(\copsver) \copstxidleq  \IdOf(\copsver_{i+1})
    \\ \CodeFont{in} \ \copskvs\UpdateFunc{
        \key -> \List{\copsver_0,\cdots, \copsver_i, \copsver, \copsver_{i+1}, \cdots, \copsver_n}} 
    \end{multlined}
\]
where the version identifier order \( \copstxidleq \) is defined in \cref{fig:cops-ver-id}.
\end{definition}

\begin{proposition}[Well-defined \COPSInsert]
Given a COPS store \( \copskvs \in \COPSKVSs \),
a key \( \key \in \Keys \) and a version \( \copsver \in \COPSVersions \)
such that \( \IdOf(\copsver) \notin \copskvs \),
then the new COPS store \( \COPSInsert(\copskvs, \key, \copsver) \) 
is well-formed, where the conditions are defined in \cref{def:cops-kv-store}.
\end{proposition}

The function \( \clockkvs' = \COPSInsert(\copskvs, \key, \copsver) \) inserts the version \( \copsver \)
in a position that preserves the order, hence \( \clockkvs' \) is well-formed.
The last premise in rule \(\rCOPSWrite\) in \cref{fig:cops-semantics-write}, \( \copsctx' = \copsctx \uplus \Set{(\key,\copstxid)} \),
updates the client context, incorporating the new version in the new context \( \copsctx' \).

\paragraph{Reference implementation and rules for \pcopsread.} 
\Cref{lst:cops-read} presents the reference implementation for \pcopsread where
a client requests a snapshot for a list of distinct keys \( \pks \) from a replica \( \repl \).
As explained in \cref{sec:overall-cops}, 
the replica constructs this snapshot in two phases.
In the first optimistic-read phase (\cref{line:cops-optimistic-read} in \cref{lst:cops-read}),
client reads the latest version 
for each key using \( \verb|get_by_version| \) with \( \CodeFont{LATEST} \) label, one key at a time.
However, interleaving may happen between two reads.

At the end of the first phase, 
the client collects the versions (\cref{line:cops-init-lower-bound}) taken in the optimistic phase
and their dependency sets (\cref{line:cops-compute-lower-bound}),
and computes a \emph{re-fetch set}, which determines versions that will be fetched in the second phase.
Specifically, the re-fetch set, \( \verb|ccv| \) in \cref{line:cops-compute-lower-bound}, contains the version with the maximum time-stamp for each key.
The client initialises \verb|ccv| with versions fetched in the optimistic phase,
and traverses all dependency entry \verb|dep| in all dependency sets \verb|deps|.
If \verb|dep| contains a version of a key \verb|dep.key| in \verb|ks|,
and if the version identifier \verb|dep.id| is greater than the current version identifier stored in the \verb|ccv[dep.key]|,
then \verb|ccv[dep.key]| is updated to \verb|dep.id|.

In the re-fetch phase, for each key \( \var(k) \) in \( \var(ks) \), 
if the version fetched in the first phase, \verb|rst[k].vers|,
has smaller time-stamp than that of the version in the re-fetch set, \( \var(ccv[k]) \),
then client should re-fetch the version of \( \var(ccv[k]) \) (\cref{line:cops-final-read}).
At the end of this re-fetch phase, \( \CodeFont{rst} \) contains the snapshot, which
\citet{cops} informally argued that it should guarantee causal consistency.
Last, this snapshot, \( \CodeFont{rst} \), is sent back to the client, and 
the client adds all the versions in \( \CodeFont{rst} \) to the context (\cref{line:cops-read-update-ctx}).

\begin{figure}[!hp]
\vspace{-10pt}
\begin{subfigure}{\textwidth}
\begin{mathpar}
\inferrule[\rCOPSStartRead]{ }{%
    \ToCOPSCmd{
    \copskvs | \copsctx | \ts | \pcopsread( \keyset ) | \lbCOPSPri ->
    \copskvs |  \copsctx |  \ts | 
    \pcopsread( \keyset ) : \emptyset }
}
\and
\inferrule[\rCOPSOptRead]{%
    \Unique(\keyset)
    \\ \keyset = \List{\key_0, \cdots, \key_i, \cdots, \key_m }
    \\ \copsverlist = \List{\copsver_0, \cdots, \copsver_{i-1}}
    \\ \copskvs(\key, \Abs{\copskvs(\key)} - 1) = (\val_i, \copstxid[\cl_i][\repl_i](\ts_i), \copsdep_i) = \copsver_i
    \\ \lb =  \lbCOPSOptRead{\opR(\key_i, \val_i), \copstxid[\cl_i][\repl_i](\ts_i),\copsdep_i} 
}{%
    \ToCOPSCmd{
    \copskvs | \copsctx | \ts | \pcopsread( \keyset ) : \copsverlist | \lb ->
    \copskvs |  \copsctx |  \ts | 
    \pcopsread( \keyset ) : (\copsverlist \ListConcat \List{\copsver_i}) }
}
\and
\inferrule[\rCOPSLowerBound]{%
    \Abs{\keyset} = \Abs{\copsverlist} 
    \\ \copsdepset = \VerLower(\keyset,\copsverlist)
}{%
    \ToCOPSCmd{ \copskvs | \copsctx | \ts | \pcopsread( \keyset ) : \copsverlist |
    \lbCOPSBound ->
    \copskvs | \copsctx | \ts | \pcopsread( \keyset ) : (\copsverlist, \copsdepset)  }
}
\and
\inferrule[\rCOPSRefetch]{%
    \copsverlist' = \List{\copsver_0, \cdots, \copsver_{\idx-1}}
    \\ \copsdepset\Proj{\idx} = (\key_i, \copstxid)
    \\ {(\val_i, \copstxid_i,\copsdep_i) = \begin{cases}
        \copskvs(\key_i,m) 
                & \If  \WtOf(\copsverlist\Proj{\idx}) \copstxidleq \copstxid
                    \land \Exists{m} \copstxid = \WtOf(\copskvs(\key_i,m))
        \\ \copsverlist\Proj{\idx} & \ow
    \end{cases}}
    \\ \copsverlist'' = \copsverlist' \ListConcat \List{(\val_i, \copstxid_i,\copsdep_i)}
    \\ \lb' = \lbCOPSRefetch{\opR(\key_i, \val_i), \copstxid_i, \copsdep_i}
}{%
    \ToCOPSCmd{ \copskvs | \copsctx | \ts 
        | \pcopsread( \keyset ) : (\copsverlist, \copsdepset, \copsverlist') | \lb ->
    \copskvs | \copsctx | \ts | \pcopsread( \keyset ) : (\copsverlist, \copsdepset, \copsverlist'')  }
}
\and
\inferrule[\rCOPSFinishRead]{%
    \Abs{\keyset} = \Abs{\copsverlist'}
    \\ \copsctx' = \Set{ (\key,\copstxid) | (\key,\copstxid,\stub) \in \copsverlist'}
}{%
    \ToCOPSCmd{ \copskvs | \copsctx | \ts 
        | \pcopsread( \keyset ) : (\copsverlist, \copsdepset, \copsverlist')
        | \lbCOPSFinishRead{\copsctx'} ->
    \copskvs | \copsctx \cup \copsctx' | \ts | \pskip  }
}
\end{mathpar}


\caption{The reference semantics for \pcopsread}
\label{fig:cops-semantics-read}

\end{subfigure}

\hrulefill

\begin{subfigure}{\textwidth}

\begin{lstlisting}[caption={},label={}]
List(Val) read(repl,ks,ctx) {
    // Only guarantee to read up-to-date value at the moment
    for k in ks { rst[k] = get_by_version(repl,k,LATEST); } (*\label{line:cops-optimistic-read}*)
    // initially the list of lower bound for keys
    for k in ks { ccv[k] = rst[k].id; (*\label{line:cops-init-lower-bound}*) }
    // compute the lower bound
    for k in ks { 
        for dep in rst[k].deps 
            if ( dep.key (*$\in$*) ks ) ccv[k] = max (ccv[dep.key].id, dep.id); (*\label{line:cops-compute-lower-bound}*) 
    } 
    // re-fetch a new version for some key
    for k in ks 
        if ( ccv[k] > rst[k].vers ) rst[k] = get_by_version(k,ccv[k]); (*\label{line:cops-final-read}*)
    // update the ctx
    for (k,ver,deps) in rst { ctx.readers += (k,ver,deps); } (*\label{line:cops-read-update-ctx}*)
    return to_vals(rst); 
}                                   
\end{lstlisting}

\caption{The reference implementation for \pcopsread}
\label{lst:cops-read}

\end{subfigure}

\hrulefill

\caption{COPS API: \pcopsread}
\label{fig:cops-api-read}

\end{figure}


The rules for \( \pcopsread \) are given in \cref{fig:cops-semantics-read}.
The rule \( \rCOPSStartRead \) ( \cref{fig:cops-semantics-read} ) 
checks the uniqueness of \( \keyset \) by predicate \( \Unique(\keyset) \) and
initialises the first phase, updating the command to 
\( \pcopsread( \keyset ) : \emptyset \) with an empty list \( \emptyset \).
This transition does not access the store, hence is labelled with \( \lbCOPSPri \).
This optimistic-read phase is captured by rule \( \rCOPSOptRead \) in \cref{fig:cops-semantics-read}.
Client \( \cl \) reads the latest version \( \copsver_i \) in \( \copskvs \) for next key \( \key_i \):
that is, \( \copsver_i = \copskvs(\key, \Abs{\copskvs(\key)} - 1) \).
The resulting fine-grained command tracks version \( \copsver_i \) in the version list \( (\copsverlist \ListConcat \List{\copsver_i}) \),
analogous to \( \CodeFont{rst} \) in \cref{line:cops-optimistic-read}.
This transition is labelled with the information including client \( \cl \),
replica \( \repl \), read operation \( \opR(\key_i,\val_i) \), the version identifier \(\copstxid_i\) and the dependency set \(\copsdep_i\).
The \( \CodeFont{Opt} \) in the label indicates the first phase.

The phase change is captured by rules \( \rCOPSLowerBound \),
which computes the re-fetch set, analogous to \cref{line:cops-init-lower-bound,line:cops-compute-lower-bound}.
The function \(\copsdepset = \VerLower(\keyset,\copsverlist ) \) 
is the re-fetch set of \( \keyset \),
in the sense that \( \copsdepset\Proj{i} \) is the version of \( \keyset\Proj{i} \),
which has the maximum time-stamp over versions of \( \keyset\Proj{i} \) 
that is either \( \copsverlist\Proj{i} \) or any versions on which \( \copsverlist \) depends.

\begin{definition}[Re-fetch set]
Given a key set \( \keyset \) and list of versions for the key set \( \copsverlist \) 
the function \( \VerLower(\keyset,\copsverlist ) \) is defined by:
\begin{align*}
    \VerLower( \keyset,\copsverlist ) & \FuncDef 
    \begin{multlined}[t]
    \Let n = \Abs{\keyset} - 1 \ \AND \ \List{\key_0, \cdots, \key_n} = \keyset 
    \\ \In \List{\VerLowerN(\key_0,\copsverlist,0), \cdots, \VerLowerN(\key_n,\copsverlist,n)} 
    \end{multlined}
    \\ \VerLowerN(\key,\copsverlist,\idx) & \FuncDef
    \Max[\copstxidleq](\Set{\copstxid | 
            \copstxid = \IdOf(\copsverlist\Proj{\idx}) \lor 
            \Exists{\idx'} (\key, \copstxid) \in \DepOf(\copsverlist\Proj{\idx'}) 
        }) .
\end{align*}
\end{definition}

The rule \( \rCOPSLowerBound \) then update the fine-grained command to
\(\pcopsread( \keyset ) : (\copsverlist, \copsdepset) \), tracking of \( \copsdepset \).
This rule is labelled with \( \lbCOPSBound \), which contains the client \( \cl\),
the relica \( \repl \) and the phase change label \( \opP \).

The rule \(\rCOPSRefetch\) in \cref{fig:cops-semantics-read} captures the semantics of the re-fetch phase.
In this phase, the client re-fetches a newer version once at a time.
If the version identifier contained in the re-fetch set, \( \copstxid = \copsdepset\Proj{\idx} \),
for the key ,\( \key_\idx \), has greater time-stamp than the version of \( \key_i \) read in the first phase,
that is, \( \copsver_\idx \),
then it re-fetches the version identified by \( \copstxid \), that is, \(\copskvs(\key_i,z)\).
Otherwise, there is no need to re-fetch for the key \( \key_i \).
In the resulting buffer \( \verlist'' \),
we track either the re-fetched version of \( \key_i \)
or copy the same version, \( \copsver_i \), from the first phase.
Note that versions in \( \verlist'' \) matches those in \( \CodeFont{rst} \) in \cref{line:cops-final-read}.
The label of this transition is tracks the client \( \cl \), the replica \( \repl \), 
the key \( \key \), the value \( \val \), the version identifier \( \copstxid \) and the dependency set \( \copsdep \).
The label \( \CodeFont{Ref} \) denotes re-fetch phase.
At the end of the re-fetch phase,
the final snapshot is returned to the client, which subsequently is added in the context.
The rule \(\rCOPSFinishRead\) captures \cref{line:cops-read-update-ctx} in the reference implementation.
This rule update the client context to \( \copsctx \cup \copsctx'\),
and the command to \( \pskip \).

The rules for read and write are lifted to a program-level 
standard interleaving semantics presented in \cref{fig:cops-semantics-program}.
The rule \( \rCOPSClient \) captures that a client takes a fine-grained step.
%We have standard interleaving semantics: a client takes one step at a time.
%This is captured by \( \rCOPSClient \) in \cref{fig:cops-semantics-program}.

\begin{figure}

\begin{subfigure}{\textwidth}
\centering
\[
\ToCOPSProg*{\;} :  \COPSConfs \times \COPSRunPrograms \times \Labels \times \COPSConfs \times \COPSRunPrograms 
\]

\begin{mathpar}
\inferrule[\rCOPSClient]{
    \cops(\repl) = (\copskvs,\ts)
    \\ \copsctxenv(\cl) = (\copsctx, \repl)
    \\ \copsrunprog(\cl) = \copsruncmd 
    \\ \ToCOPSCmd{ \copskvs | \copsctx | \ts | \copsruncmd | \lb ->
    \copskvs' | \copsctx' | \ts' | \copsruncmd' }
    \\ \cops' = \cops\UpdateFunc{\repl -> (\copskvs, \ts') }
    \\ \copsctxenv' = \copsctxenv\UpdateFunc{\cl -> (\copsctx', \repl) }
    \\ \copsrunprog' = \copsrunprog\UpdateFunc{\cl -> \copsruncmd }
}{%
    \ToCOPSProg{ \cops | \copsctxenv | \copsrunprog  | \lb ->  \cops' | \copsctxenv' | \copsrunprog' }
}
\and
\inferrule[\rCOPSSync]{
    \repl \neq \repl'
    \\ \cops(\repl) = (\copskvs,\ts)
    \\ \cops(\repl') = (\copskvs',\ts')
    \\ \copskvs(\key,\idx) = \copsver
    \\ \Forall{\copstxid} (\stub,\copstxid) \in \DepOf(\copsver) \implies \copstxid \in\copskvs'
    \\ \copskvs'' = \COPSInsert(\copskvs',\key,\copsver)
    \\ \copstxid[\stub][\stub](\ts'') = \WtOf(\copsver)
    \\ \cops\UpdateFunc{\repl' -> (\copskvs'',\Max(\ts',\ts''))}
}{%
    \ToCOPSProg{ \cops | \copsctxenv | \copsrunprog  | \lbCOPSSync(\repl'){\copstxid} ->  \cops' | \copsctxenv | \copsrunprog }
}
\end{mathpar}

\caption{The reference semantics for COPS synchronisation and programs}
\label{fig:cops-semantics-program}

\end{subfigure}

\hrulefill

\begin{subfigure}{\textwidth}

\begin{lstlisting}
on_receive(repl,k,v,id,deps) {
    for (k',id') in deps { wait until (_,id',_) (*$\in$*) repl.kv(k'); }

    list_isnert(repl.kv[k],(v,id,deps));
    (remote_local_time ++ replid) = id;
    repl.local_time = max(remote_local_time, repl.local_time);
}
\end{lstlisting}

\caption{The reference implementation for COPS synchronisation}
\label{lst:cops-receive-msg}

\end{subfigure}

\hrulefill

\caption{COPS synchronisation and programs}

\end{figure}


\paragraph{Synchronisation Between Replicas}
Replicas broadcast synchronisation messages for every new version.
Since versions are ordered and messages are assumed to be eventually delivered,
this guarantees COPS satisfies eventual consistency: that is,
all replicas eventually reach the same state.
Once a replica receives a message with a new version \( \copsver \),
it accepts the version only if all the versions that \(\copsver \) depends on exist in the replica (\cref{lst:cops-receive-msg}).
Otherwise, the replica waits this condition to be satisfied (\cref{line:cops-wait-deliver}).
Rule \( \rCOPSSync \) in \cref{fig:cops-semantics-program} models the synchronisation.

\begin{definition}[COPS traces]
The set of \emph{COPS traces}, \( \COPSTraces \ni \copstrc \), is defined by:
\( \COPSTraces \TypeDef \Set{ \copstrc | \Exists{n,\copsconf,\copsprog} \copstrc \in \COPSTraceN(n) \land \Last(\copstrc) = (\copsconf,\copsprog) \land \copsprog \in \COPSPrograms } \),
where \( \COPSPrograms \) is defined in \cref{def:cops-commands}.
The function \( \COPSTraceN \) is defined by:
\begin{align*}
\COPSTraceN(0) & \FuncDef \Set{\ToCOPSProg{\copsconfinit | \copsprog_0} | \copsconfinit \in \COPSConfInits \land \copsprog_0 \in \COPSPrograms }
\\  \COPSTraceN(n+1) &  \FuncDef \Set{ \ToCOPSProg{\copstrc | \lb -> \copsconf | \copsrunprog} | \copstrc \in \COPSTraceN(n) \land \copsconf \in \COPSConfs \\ {} \land \copsrunprog \in \COPSRunPrograms }
\end{align*}
where \( \COPSConfInits \) and \( \COPSConfs \) are defined in \cref{def:cops-conf},
and \( \COPSRunPrograms \) is defined in \cref{def:cops-commands}.
Two traces \( \copstrc, \copstrc' \) are equivalent, written \( \copstrc \copstrceq \copstrc' \),
if and only if, the last configurations are equal,
\( \copstrc \copstrceq \copstrc' \PredDef \LastConf(\copstrc) = \LastConf(\copstrc') \).
\end{definition}

Note that a COPS trace has no unfinished read operation.
Two COPS traces are equivalent if and only if the last configurations match.

