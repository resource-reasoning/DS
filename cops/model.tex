In COPS protocol, there is no explicit transactional identifier,
but the version identifier,
which consist of the replica identifier who initially accepted the version 
and the local time when the version was accepted,
can be seen as the transactional identifier who wrote the version.

\begin{definition}[COPS Replica and Transaction Identifiers]
The set of \emph{COPS replicas}, \( \COPSReplicas \ni \repl \), 
is a countably infinite set that is totally ordered on relation \( \replleq \).
The set of \emph{COPS transaction identifiers}, \( \COPSTxIDs \ni \copstxid \),
is defined by
\[ 
    \COPSTxIDs \TypeDef \Set{\copstxid[\cl][\repl](n,m) | 
    \cl \in \Clients \land \repl \in \COPSReplicas \land n,m \in \Nat } \cup \Set{\txidinit}.
\]
The order over transactions is defined by
\[ 
\copstxid[\cl][\repl](n,m) \copstxidleq \copstxid[\cl'][\repl'](n',m')
\PredDef n < n' \lor ( n = n' \land ( \repl < \repl' \lor ( \repl = \repl' \land m < m' ) ) .
\]
\end{definition}

We annotate the COPS transaction identifier \(\copstxid[\cl][\repl](n,m)  \)
with the client identifier \( \cl \), the local time \(n\),
the replica identifier \( \repl \)
and the extra counter \( m \) for read-only transactions.
In the model and semantics of COPS protocol (\cref{sec:cops-model,sec:cops-semantics}), 
the extra counter \(m\) is always ZERO.
Later in the verification of COPS protocol (\cref{sec:cops-extended-trace}),
we use the extra counter \( m \) to identify read-only transactions.
Note that, all transactions, including read-only transactions,
are totally order by \( (n,\repl,m) \) lexicographically.

As explained in \cref{sec:overall-cops},
a COPS version \( \copsver \) consists of the value, the identifier 
and the dependency set that is other versions \( \copsver \) depends on.

\begin{definition}[COPS versions and dependency sets]
A \emph{dependency set} \( \copsdep \) 
is defined by \( \copsdep \subseteq \Keys \times \COPSTxIDs \).
Let \( \COPSDependencies \ni \copsdep \) denote the set of all dependency sets.
The set of \emph{COPS versions} \( \COPSVersions \ni \copsver \), is defined by 
\( \COPSVersions \TypeDef \Values \times \COPSTxIDs  \times \COPSDependencies \).
Let \(\ValOf(\copsver)\), \(\WtOf(\copsver)\)  
and \( \DepOf(\copsver)\) return the first to third projections of \( \copsver \) respectively.
\end{definition}

Each replica \( \repl \) consists of a local multi-versioning key-value store and a local time.
A COPS database contains finite number of replicas that are opaque to the clients.

\begin{definition}[COPS Key-value Stores and Databases]
the set of \emph{COPS local key-value stores} or \emph{COPS local stores}
is defined by
\( \COPSKVSs \FuncDef \Set{\copskvs \in \Keys \ToTFunc \List{\COPSVersions} | \WfCOPSKvs(\copskvs)}. \)
where \( \WfCOPSKvs \) is defined by: for any keys \( \key, \key' \in \Keys \),
indexes \( \idx, \idx' \in \Indexs \) and the initial value \( \valinit \),
\begin{Formulae}
    & \begin{Formula}
        \copskvs(\key,0) = (\valinit,\txidinit,\stub),
        \label{equ:cops-kvs-init}
    \end{Formula}
    \\ & \begin{Formula}
        \WtOf(\copskvs(\key,\idx)) = \WtOf(\copskvs(\key',\idx')) 
            \implies \key = \key' \land \idx = \idx',
        \label{equ:cops-version-unique}
    \end{Formula}
    \\ & \begin{Formula}
        \WtOf(\copskvs(\key,\idx)) \copstxidleq \WtOf(\copskvs(\key,\idx')) 
            \iff \idx < \idx'.
        \label{equ:cops-version-order}
    \end{Formula}
\end{Formulae}
Given two COPS local stores \(\copskvs, \copskvs' \), 
the order \( \copskvs \copskvsleq \copskvs' \) is defined by
\[
    \copskvs \copskvsleq \copskvs' \PredDef \Forall{\key \in \Keys | \idx \in \Indexs }
    \copsver = \copskvs(\key, \idx) \implies \Exists{\idx \in \Indexs} \copsver = \copskvs'(\key, \idx').
\]
The set of \emph{COPS databases} is defined by
\[
    \COPSs \FuncDef \Set{\cops \in \COPSReplicas \ToPFFunc 
        \\  ( \COPSKVSs \times \Nat ) | 
        \Forall{\repl, \repl' \in \Dom(\cops) | \key, \key' \in \Keys | \idx, \idx' \in \Indexs }
            \\ \WtOf(\cops(\repl)(\key,\idx)) = \WtOf(\cops(\repl')(\key',\idx')) \iff
            \\ \key = \key' \land \cops(\repl)(\key,\idx) = \cops(\repl')(\key',\idx') }.
\]
\end{definition}

A COPS replica tracks all the history versions that are uniquely identified and ordered by their writers.
Note that, without losing generality, versions for a key are organised into a list;
the relative position of versions does not matter since all versions are ordered by their writers.
Since versions will be eventually delivered to all sites,
this means that versions with the same writer from different replicas must be the same version,
which is captured by the well-formed condition \( \WfCOPSKvs \).

Each COPS client maintain a local context to trace versions that the client read or wrote before.
The client contexts is used to build the correct dependency sets for versions.
Each client always interacts with a assigned replica (identifier).

\begin{definition}[COPS Client Contexts and Environments]
A \emph{COPS context} \( \copsctx \) is defined by 
\(
    \copsctx \subseteq \Keys \times \COPSTxIDs
\).
Let \( \COPSContexts \in \copsctx \) denotes the set of \emph{COPS contexts}.
The set of \emph{COPS client contexts} \( \COPSContextEnvs \ni \copsctxenv \) is defined by
\( \COPSContextEnvs \TypeDef \Clients \ToPFFunc (\COPSContexts \times \COPSReplicas ) \).
\end{definition}

When a client commits a new version, 
the client context becomes the dependency set of the new version.
Thus both dependency sets of versions and client contexts 
are modelled as sets of tuples with keys and identifiers.

Last, a COPS configuration consists of a COPS database  \( \cops \) and a COPS client environment \( \copsclenv \).
Any client from the environment \( \copsclenv \) must be served by a replica that exists in \( \cops \).

\begin{definition}[COPS Configurations]
A COPS configuration \( \copsconf \in \COPSConfs \) comprises 
a COPS database \( \cops \) and a client context \( \copsctxenv \), 
that is, \[ 
    \COPSConfs \TypeDef \Set{ (\cops, \copsctxenv) \in \\ \COPSs \times \COPSContextEnvs |
        \Forall{\cl \in \Dom(\copsctxenv) | \repl \in \COPSReplicas} 
        \\ \copsctxenv(\cl) = (\stub, \repl) \implies \repl \in \Dom(\cops) }.
\]
The set of \emph{initial COPS configurations}, \( \COPSConfInits \ni \copsconfinit\),
is defined by 
\[ \COPSConfInits \TypeDef \Set{ (\cops, \copsctxenv) \in \\ \COPSConfs | 
        \Forall{\repl \in \COPSReplicas | \cl \in \Clients } 
        \\ \cops(\repl) = \lambda \key \in \Keys \ldotp ( \List{(\valinit, \txidinit, \emptyset)}, 0 )
        \land \copsctxenv(\cl) = (\emptyset, \stub) }. 
\]
\end{definition}
