We introduce \emph{normalised COPS traces}, where the re-fetch phase of any multiple read transaction is executed atomically,
and \emph{annotated normalised COPS traces}, where transactions are assigned with unique \emph{COPS transaction identifiers}.

A \emph{normalised COPS trace} is a trace where: 
\begin{enumerate*}
\item the order of single-write transactions in the trace agrees on the order over the transaction identifiers (\cref{equ:cops-normal-trace-write-order});
\item the re-fecth phase of multiple-read transactions cannot be interleaved (\cref{equ:cops-normal-trace-second-read}).
\end{enumerate*}
The first property ensures that any version of a key be always the latest of version of the key.
The second property ensures that multiple-read transactions appear to be atomic with respect to clients.
In COPS, there is no directly information about read-only transactions.
We give explicit identifiers for read-only transactions in a normalised trace.
We call the resulting trace as an \emph{annotated normalised COPS trace}.

\begin{definition}[Normalised COPS traces]
\label{def:cops-normal-trace}
A \emph{normalised COPS trace} \( \copstrc \in \COPSTraces\), 
written \( \COPSNormalTrace(\copstrc) \), if and only if:
for the length of the trace \( n = \Abs{\copstrc} \),
states \( \copsconf_0, \cdots, \copsconf_{n-1} \),
labels \( \lb_1, \cdots, \lb_{n-1} \),
clients \( \cl, \cl'\), replicas \( \repl, \repl'\), keys \( \key, \key' \)
values \( \val,\val' \), version identifiers \( \copsverid, \copsverid' \),
contexts \( \copsctx, \copsctx' \) and indexes \( \idx, \idx' \),
\begin{Formulae}
& \begin{Formula}
    \copstrc = \ToCOPSProg{\copsconf_0 | \lb_1 -> \cdots | \lb_{n-1} -> \copsconf_{n-1} }
    \\ {} \land \lb_\idx = \lbCOPSWrite[\cl](\repl){\opW(\key,\val), \copsverid, \copsdep }
    \land \lb_{\idx'} = \lbCOPSWrite[\cl'](\repl'){\opW(\key',\val'), \copsverid', \copsdep' }
    \land \idx < \idx'
    \implies \copsverid \copsveridleq  \copsverid',
\label{equ:cops-normal-trace-write-order}
\end{Formula}
\\ & \begin{Formula}
    \lb_\idx = \lbCOPSRefetch[\cl](\repl){\opR(\key,\val), \copsverid, \copsdep }
    \land \idx' = \idx + 1
    \\ {} \implies \lb_{\idx'} = \lbCOPSRefetch{\opR(\key',\val'), \copsverid', \copsdep' }
    \lor \lb_{\idx'} = \lbCOPSFinishRead{\copsctx}.
\label{equ:cops-normal-trace-second-read}
\end{Formula}
\end{Formulae}
\end{definition}

The normalised COPS trace is the key for proving that COPS satisfies causal consistency against our semantics.
In \(\rCOPSWrite\) rule, the new version is inserted into 
the list via \( \COPSInsert \) that preserves the order of the new list.
However, by \cref{equ:cops-normal-trace-write-order}, it is enough to append the new version and 
it guarantees to preserve the order (\cref{prop:cops-append-write}).
The second property, \cref{equ:cops-normal-trace-second-read}, ensures that 
any multiple-read transaction can be treated as an atomic step.
Note that, in \(\rCOPSRefetch\) rule,
only the versions that are put in the buffer \( \verlist''\)
will be included in the snapshot to the client.
This means that the atomicity of the re-fetch phase is enough for showing atomicity of the entire transaction.
Every COPS trace can be transferred to its equivalent normalised COPS trace by swapping steps.

\begin{theorem}[Equivalent normalised COPS traces]
Given a COPS trace \( \copstrc \in \COPSTraces \),
there is an equivalent normalised COPS trace \( \copstrc^* \)
such that \( \copstrc \copstrceq \copstrc^* \land \COPSNormalTrace(\copstrc^*) \).
\end{theorem}
\begin{proof}
let \( \copstrc^* = \copstrc \) initially.
We perform the following equivalent transformation until \( \copstrc^* \) is a normalised COPS trace.
First, we move every re-fetch operation (in the second phase) to the right, delaying the re-fetch,
until the operation is directly followed by the end of the transaction or other re-fetch operation from the same transaction.
Second, we move the out-of-order write operation to the left.
\begin{enumerate}
\item{\textbf{Right mover: re-fetch operation}.} \label{item:cops-second-read}
    Take the first end of transaction step, \( \lbCOPSFinishRead{\copsctx} \), where
    one of its re-fetch operations, \(\lbCOPSRefetch{\opR(\key,\val), \copsverid, \copsdep' }\), 
    is interleaved by other client: that is,
    for three trace segments \( \copstrc', \copstrc'', \copstrc''' \) and a step \( \lb \),
    \begin{Formulae}*
    \begin{Formula}
        \copstrc^* = \ToCOPSProg{\copstrc' | \lbCOPSRefetch{\opR(\key,\val), \copsverid, \copsdep' }
            -> \stub | \lb -> \copstrc'' | \lbCOPSFinishRead{\copsctx} -> \copstrc''' }
        \land \lb \neq \lbCOPSRefetch{\opR(\key',\val'), \copsverid', \copsdep'' } .
    \end{Formula}
    \end{Formulae}
    The \( \lb \) can be moved left:
    \begin{Formulae}*
    \begin{Formula}
        \begin{Bracketed} \ToCOPSProg{\copstrc' | \lb
            -> \stub | \lbCOPSRefetch{\opR(\key,\val), \copsverid, \copsdep' } 
            -> \copstrc'' | \lbCOPSFinishRead{\copsctx} -> \copstrc''' } 
        \end{Bracketed} \copstrceq \copstrc^* .
    \end{Formula}
    \end{Formulae}
    If the label \( \lb \) is not a write operation, 
    it is easy to see that \( \lb \) and the re-fetch operation can be swapped.
    If the label \( \lb \) is a write operation, it must not interfere the re-fetch operation.
    Because any re-fetch operations can only read versions written before the phase change step.
    The full proof of the right mover is given in \cref{lem:cops-no-interleave-second-read}
    on \pageref{lem:cops-no-interleave-second-read}.

    Assign the new trace to \( \copstrc^* \) and go back to \cref{item:cops-second-read}.
    There are finite number of transitions in a trace.
    After each iteration of \cref{item:cops-second-read}, a re-fetch transition of a transaction moves closer 
    to the end label of the transaction.
    Therefore, \cref{item:cops-second-read} must terminate.
    The final trace \( \copstrc^* \) satisfies \cref{equ:cops-normal-trace-second-read}.

\item{\textbf{Left mover: out-of-order writes}.} \label{item:cops-write}
    Given a trace \( \copstrc^* \) satisfying \cref{equ:cops-normal-trace-second-read},
    take two out-of-order write operations with the shortest distance in between,
    \( \lbCOPSWrite[\cl](\repl){\opW(\key,\val), \copsverid, \copsdep } \) and
    \( \lbCOPSWrite[\cl'](\repl'){\opW(\key',\val'), \copsverid', \copsdep' } \),
    and the operation immediately before the latter write, \( \lb \):
    \begin{Formulae}*
    \begin{Formula}
        \copstrc^* = \ToCOPSProg{\copstrc' | \lbCOPSWrite[\cl](\repl){\opW(\key,\val), \copsverid, \copsdep }
            -> \copstrc'' | \lb
            -> \stub | \lbCOPSWrite[\cl'](\repl'){\opW(\key',\val'), \copsverid', \copsdep' }
            -> \copstrc''' }
        \land \copsverid' \copsveridleq \copsverid ,
    \end{Formula}
    \end{Formulae}
    where all the free variables are universally quantified.
    Note that there is no other out-of-order in the trace segment \( \copstrc'' \).

    Let first consider that the step \( \lb \) is from \( \cl' \).
    Take the possible steps \( \lb' \) from \( \cl' \) in the trace segment \( \clocktrc'' \). 
    Both \( \lb \) and \( \lb' \) cannot be a write. 
    This is because a replica local time is monotonic increases (detail is given in \cref{prop:cops-inv}).
    If \( \lb \) is a write operation is from client \( \cl' \),
    it must commit before \( \copsverid' \).
    However we have picked the two out-of-order write with the shortest distance.
    Hence, \( \lb \) and \( \lb' \) can be snapshot, read, phase change and commit steps for multi-reads.
    All these \( \lb \) and \( \lb' \) cannot depends on \( \lbCOPSWrite[\cl](\repl){\opW(\key,\val), \copsverid, \copsdep }  \),
    otherwise we will have \( \copsverid \copsveridleq \copsverid' \), which contradict to our hypothesis.
    Now let consider the dependency of \( \lb \) and \( \lb' \), which can be write or synchronisation steps.
    These steps might depend on synchronisation steps \( \lb'' \) in \( \clocktrc'' \),
    but cannot depend on any write in in \( \clocktrc'' \) hence any read in \( \clocktrc'' \).
    This is because \( \lb \) and \( \lb' \) depend on a write in \( \clocktrc'' \),
    this write must have time-stamp smaller than \( \copsverid' \),
    which contradict our hypothesis that we have pick the out-of-order writes with shortest distance.
    Given above, it allows us to move the following steps to 
    the left of \( \lbCOPSWrite[\cl](\repl){\opW(\key,\val), \copsverid, \copsdep } \): 
    \begin{enumerate*}
    \item the step \( \lb' \) and any steps \( \lb' \) in the trace segment \( \clocktrc'' \); and 
    \item any synchronisations steps \( \lb'' \)  in the trace segment \( \clocktrc'' \) that  \( \lb' \) and \( \lb' \) depend on.
    \end{enumerate*}
    This yields a new trace:
    \begin{Formulae}*
    \begin{Formula}
    \left(
        \ToCOPSProg{\copstrc' | \stub
            -> \copstrc^{***} | \lbCOPSWrite[\cl](\repl){\opW(\key,\val), \copsverid, \copsdep }
            -> \copstrc^{**} | \lb^{*} 
            -> \stub |  \lbCOPSWrite[\cl'](\repl'){\opW(\key',\val'), \copsverid', \copsdep' }     
            -> \copstrc''' }
            \right) \copstrceq \copstrc^* 
        \land \copsverid' \copsveridleq \copsverid 
    \end{Formula}
    \end{Formulae}
    where \( \copstrc^{***} \) contains any steps in \( \clocktrc'' \) that are from \( \cl' \),
    while \( \copstrc^{**} \) are the rest of \( \clocktrc'' \).
    All steps in \( \copstrc^{**} \) and \( \lb^{*} \) are from a client different from \( \cl' \)


    Now consider \( \lb^{*} \) being a step from a client different from \( cl' \).
    For this case, 
    because clients does not interfere with each others,
    it is easy to see that \( \lb^{*} \) can be move to right, delaying this step.
    In other words, the out-of-order write operation can be moved to the left:
    \begin{Formulae}*
    \begin{Formula}
    \left(
        \ToCOPSProg{\copstrc' | \stub
            -> \copstrc^{***} | \lbCOPSWrite[\cl](\repl){\opW(\key,\val), \copsverid, \copsdep }
            -> \copstrc^{**} | \lbCOPSWrite[\cl'](\repl'){\opW(\key',\val'), \copsverid', \copsdep' }      
            -> \stub |  \lb^{*}  
            -> \copstrc''' }
            \right) \copstrceq \copstrc^* 
        \land \copsverid' \copsveridleq \copsverid .
    \end{Formula}
    \end{Formulae}
    The full proof of this left mover is given in \cref{lem:cops-write-left-move} on \pageref{lem:cops-write-left-move}.
    Continue to apply the left mover until the two write operations are in order:
    for some new trace segment \( \copstrc^{****} \) that contains the same steps as in \( \copstrc^{**}\),
    \begin{Formulae}*
    \begin{Formula}
    \left(
        \ToCOPSProg{\copstrc' | \stub
            -> \copstrc^{***} | \lbCOPSWrite[\cl'](\repl'){\opW(\key',\val'), \copsverid', \copsdep' }       
            -> \stub |  \lbCOPSWrite[\cl](\repl){\opW(\key,\val), \copsverid, \copsdep } 
            -> \copstrc^{****} }
            \right) \copstrceq \copstrc^* 
        \land \copsverid' \copsveridleq \copsverid .
    \end{Formula}
    \end{Formulae}
    Assign the new trace to \( \copstrc^* \) and go back to \cref{item:cops-write}.
    For each iteration of \cref{item:cops-write},
    two out-of-order write operations move closer to each other and eventually swap position.
    Because there are finite number of transitions in a trace and thus finite number of out-of-order write operations,
    then \cref{item:cops-write} must terminate.
    The final trace \( \copstrc^* \) satisfies \cref{equ:cops-normal-trace-write-order}.
\end{enumerate}
After \cref{item:cops-second-read,item:cops-write}, 
the trace \( \copstrc^*\) is a normalised COPS trace as \( \COPSNormalTrace(\copstrc^*) \).
\end{proof} 


\begin{toappendix}
\begin{theorem}[Right mover: re-fetch operations]
\label{lem:cops-no-interleave-second-read}
Assume a COPS trace \( \copstrc \in \COPSTraces \) that
contains a re-fetch operation, \lbCOPSRefetch{\opR(\key,\val), \copsverid,\copsctx }, 
that is interleaved by other operation, \( \lb \):
\[
\copstrc = \ToCOPSProg{ \copstrc' | \stub -> \cops | \copsctxenv | \copsrunprog | \lbCOPSRefetch{\opR(\key,\val), \copsverid,\copsctx } 
    ->  \cops' | \copsctxenv' | \copsrunprog'
    | \lb -> \cops'' | \copsctxenv'' | \copsrunprog'' | \lb' -> \copstrc'' } ,
\]
and
\[
    \lb \neq \lbCOPSRefetch{\opR(\key',\val'), \copsverid',\copsctx' }  
    \land \lb \neq \lbCOPSFinishRead{\copsctx}
\]
where all other free variables are universally quantified.
Then the re-fetch operation can be moved to the right, that is,
\begin{Formulae}
\begin{Formula}
\begin{Bracketed} 
\ToCOPSProg{ \copstrc' | \stub -> \cops | \copsctxenv | \copsrunprog | \lb  
    ->  \stub | \lbCOPSRefetch{\opR(\key,\val), \copsverid,\copsctx }
    -> \cops'' | \copsctxenv'' | \copsrunprog'' | \lb' -> \copstrc'' } 
\end{Bracketed} \copstrceq \copstrc .
\label{equ:cops-interleaving-left-move}
\end{Formula}
\end{Formulae}
\end{theorem}
\begin{proof}
By \(\rCOPSRefetch\) in \cref{fig:cops-semantics-read}, any read from the second phase
does not change the store nor the context, thus
\[
\copstrc = \ToCOPSProg{ \copstrc' | \stub -> \cops | \copsctxenv | \copsrunprog | \lbCOPSRefetch{\opR(\key,\val), \copsverid,\copsctx } 
    ->  \cops | \copsctxenv' | \copsrunprog'
    | \lb -> \cops'' | \copsctxenv'' | \copsrunprog'' | \lb' -> \copstrc'' }.
\]
We perform case analysis on the label \( \lb \).
\begin{enumerate}
\Case{\( \lbCOPSWrite[\cl'](\repl'){\opW(\key',\val'), \copsverid', \copsctx' } \)}
    We immediately know \( \cl \neq \cl' \).
    If \( \repl \neq \repl' \), \cref{equ:cops-interleaving-left-move} trivially holds.
    Consider \( \repl = \repl' \).
    Let \( \copskvs = \cops(\repl) \) and \( \copskvs'' = \cops''(\repl )\).
    It is easy to see that \( \copskvs \copskvsleq \copskvs'' \),
    thus by \cref{lem:cops-kvs-leq}, we have 
    \[
    \ToCOPSProg{ \copstrc' | \stub -> \cops | \copsctxenv | \copsrunprog |  \lb
        ->  \cops'' | \copsctxenv'' | \copsrunprog^*
        | \lbCOPSRefetch{\opR(\key,\val), \copsverid,\copsctx }
        -> \cops'' | \copsctxenv'' | \copsrunprog'' | \lb' -> \copstrc'' }.
    \]
    for some program \( \copsrunprog^*\); this implies \cref{equ:cops-interleaving-left-move}.
\Cases{\( \lbCOPSOptRead[\cl'](\repl'){\opR(\key',\val'), \copsverid', \copsctx'} \)
        , \( \lbCOPSOptRead[\cl'](\repl'){\opR(\key',\val'), \copsverid', \copsctx'} \)
        , \( \lbCOPSRefetch[\cl'](\repl'){\opR(\key',\val'), \copsverid', \copsctx' }\)
        , \( \lbCOPSPri[\cl'](\repl') \)
        and \( \lbCOPSBound[\cl'](\repl') \) }
    First it is easy to see that \( \cl \neq \cl'\);
    and these steps do not change the states of the database nor the contexts.
    Therefore \cref{equ:cops-interleaving-left-move} holds for these steps.
\Case{\(\lbCOPSFinishRead[\cl'](\repl'){\copsctx} \) }
    By \rCOPSFinishRead in \cref{fig:cops-semantics-read}, it is easy to see that \( \cl \neq \cl'\);
    and this step does not change the states of the database.
    Because \( \cl \neq \cl' \) the new context for \( \cl' \)
    will not affect the \( \cl \), thus 
    \[
    \ToCOPSProg{ \copstrc' | \stub -> \cops | \copsctxenv | \copsrunprog |  \lb
        ->  \cops | \copsctxenv'' | \copsrunprog^* | \lbCOPSFinishRead[\cl'](\repl'){\copsctx}
        -> \cops'' | \copsctxenv'' | \copsrunprog'' | \lb' -> \copstrc'' },
    \]
    which implies \cref{equ:cops-interleaving-left-move}.
\Case{\(\lbCOPSSync(\repl'){\copsverid} \) }
    Let \( \copskvs = \cops(\repl) \) and \( \copskvs'' = \cops''(\repl )\).
    By \rCOPSSync in \cref{fig:cops-semantics-program},
    It is easy to see that \( \copskvs \copskvsleq \copskvs'' \),
    \( \copsctxenv' = \copsctxenv'' \) and \( \copsrunprog' = \copsrunprog'' \);
    thus by \cref{lem:cops-kvs-leq} and the fact that \rCOPSSync does not rely on any context nor program,
    we have 
    \[
    \ToCOPSProg{ \copstrc' | \stub -> \cops | \copsctxenv | \copsrunprog |  \lb
        ->  \cops'' | \copsctxenv | \copsrunprog
        | \lbCOPSRefetch{\opR(\key,\val), \copsverid,\copsctx }
        -> \cops'' | \copsctxenv'' | \copsrunprog'' | \lb' -> \copstrc'' },
    \]
    which implies \cref{equ:cops-interleaving-left-move}. \qedhere
\end{enumerate}
\end{proof}
%\begin{proofsketch}
%We perform case analysis on the label \( \lb \).
%If the label \( \lb \) is not a write operation, 
%it is easy to see that \( \lb \) and the re-fetch operation can be swapped.
%If the label \( \lb \) is a write operation, it must not interfere the re-fetch operation.
%This is because any re-fetch operations can only read versions written before the lower-bound computation step.
%The full proof is given in \cref{sec:proof-right-mover-re-fetch} on \pageref{sec:proof-right-mover-re-fetch}.
%\end{proofsketch}
\end{toappendix}

%\begin{lemma}[COPS session order]
%\label{lem:cops-session-order}
%Steps in a COPS trace \( \copstrc \in \COPSTraces \) agree with the session order,
%that is,
%\[
%\copstrc = \ToCOPSProg{ \cdots | \lbCOPSWrite{\opW(\key,\val), \copsverid, \copsctx }
    %->  \copstrc' | \lb -> \cdots }
    %\land \lb = \lbCOPSOptRead{\opR(\key,\val'), \copsverid', \copsctx'}
    %\implies \copsverid \copsveridleq \copsverid'
%\]
%\end{lemma}
%\begin{proof}
%Let \( (\cops',\copsctxenv')\) be the first configuration of \( \copstrc\),
%that is \( (\cops',\copsctxenv') = \copstrc\Proj{0}\), and let \( \copskvs' = \cops'(\repl)\).
%By \rCOPSWrite, there exists an index \( \idx \) such that 
%\( \copskvs(\key,\idx) = (\val,\copsverid, \copsctx )\);
%this means \( \copsverid \copsveridleq \IdOf(\copskvs(\key,\Abs{\copskvs(\key)}-1)) \) 
%since \( \copskvs \) is a well-formed COPS store.
%By \cref{prop:cops-inv}, 
%\[ \copsverid \copsveridleq \IdOf(\copskvs(\key,\Abs{\copskvs(\key)}-1)) \copskvsleq \copsverid' . \]
%\end{proof}

%\begin{lemma}[COPS replica order]
%\label{lem:cops-replica-order}
%Steps in a COPS trace \( \copstrc \in \COPSTraces \) agree with the replica local times,
%that is,
%\[
%\copstrc = \ToCOPSProg{ \cdots | \lbCOPSWrite{\opW(\key,\val), \copsverid, \copsctx }
    %->  \copstrc' | \lb -> \cdots }
    %\land \lb = \lbCOPSWrite[\cl']{\opW(\key',\val'), \copsverid', \copsctx'}
    %\implies \copsverid \copsveridleq \copsverid' .
%\]
%\end{lemma}
%\begin{proof}
%It can be derived from \cref{prop:cops-inv}.
%\end{proof}

%In \cref{def:cops-normal-trace} we rely on that
%both replicas and clients are monotonic in that:
%\begin{enumerate*}
%\item every replica contains increasingly more versions and the local time ticks forward;
%and \item every client context contains increasingly more versions.
%\end{enumerate*}
%This is a useful property that allows us to swap out-of-order write operations.

\begin{toappendix}
\label{sec:proof-monotonic-cops-replca-client}
\begin{proposition}[Monotonicity of COPS replica and client]
\label{prop:cops-inv}
For any COPS databases \( \cops, \cops'\), client context environments \( \copsctxenv, \copsctxenv' \),
programs \( \copsrunprog, \copsrunprog' \) and label \( \lb \),
\begin{Formulae}
\begin{Formula}
    \ToCOPSProg{ \cops | \copsctxenv | \copsrunprog | \stub | * -> \cops' | \copsctxenv' | \copsrunprog' }
    \\ \implies \Forall{\repl \in \Dom(\cops) | \copskvs, \copskvs' \in \COPSKVSs | \ts, \ts' \in \Nat} 
    \\ \left( \cops(\repl) = (\copskvs,\ts) \land
    \cops'(\repl) = (\copskvs',\ts') 
    \implies \copskvs \copskvsleq \copskvs'
    \land \ts \leq \ts'  \right)
    \\ {} \land \Forall{\cl \in \Dom(\copsctxenv) | \copsctx, \copsctx' \in \COPSContexts} 
    \\ \left( \copsctxenv(\cl) = (\copsctx,\stub) 
    \land  \copsctxenv'(\cl) = (\copsctx',\stub) 
    \implies \copsctx \subseteq  \copsctx' \right) .
\label{equ:cops-replica-inv}
\end{Formula}
\end{Formulae}
\end{proposition}
\begin{proof}
We prove by induction on the length of the trace \( n \).
\CaseBase{\(n = 0\)}
    \Cref{equ:cops-replica-inv} is trivially true as \( \cops = \cops' \) and \( \copsctxenv = copsctxenv' \).
\CaseBase{\(n > 0\)}
    Assume
    \(
        \ToCOPSProg{ \cops | \copsctxenv | \copsrunprog | \stub | (n-1)
        -> \cops'' | \copsctxenv'' | \copsrunprog'' | \lb
        -> \cops' | \copsctxenv' | \copsrunprog' }.
    \)
    By \ih, we know \( \cops(\repl)\Proj{0} \copskvsleq \cops'(\repl)\Proj{0} \) 
    and \( \cops(\repl)\Proj{1} \leq \cops'(\repl)\Proj{1} \)  for \( \repl \in \Dom(\cops) \).
    Consider the label \( \lb \).
    If \( \lb = \lbCOPSWrite[cl](\repl){\opW(\key,\val), \copsverid, \copsctx } \),
    by \rCOPSWrite, we have \( \cops' = \cops''\UpdateFunc{{\repl} -> \Tuple{\copskvs^*, \cops''(\repl)\Proj{1} + 1}}\),
    for \( \copskvs^* = \COPSInsert(\cops''(\repl)\Proj{0},\key,\copsver) \) for a version \( \copsver \)
    and, thus \cref{equ:cops-replica-inv}.
    If \( \lb = \lbCOPSSync(\repl){\copsverid} \),
    by \rCOPSSync we have \( \cops' = \cops''\UpdateFunc{\repl -> \Tuple{\copskvs^*, m } }\) such that
    \( m =\Max(\cops''(\repl)\Proj{1},m^*) \) where \( \copsverid = \copsverid[\cl^*][\repl^*](m^*)\), 
    and \( \copskvs' = \COPSInsert(\cops''(\repl)\Proj{0},\key,\copsver) \) for a version \( \copsver \),
    and, thus \cref{equ:cops-replica-inv}.
    For the rest cases, they do not change any COPS store and local time;
    by \ih, \cref{equ:cops-replica-inv} holds. \qedhere
\end{proof}
%\begin{proofsketch}
%It is easy to prove by induction on the length of the trace.
%The full proof is given in \cref{sec:proof-monotonic-cops-replca-client}
%on \pageref{sec:proof-monotonic-cops-replca-client}.
%\end{proofsketch}
\end{toappendix}

%A write operation is out-of-order, 
%if in COPS trace there is any write operation (that may from different a client on different a replica),

\begin{toappendix}
\begin{theorem}[Left mover: out-of-order write]
\label{lem:cops-write-left-move}
Assume a COPS trace \( \copstrc \in \COPSTraces \) such that
\[
\copstrc = \ToCOPSProg{ \copstrc' | \stub -> \cops | \copsctxenv | \copsrunprog | \lb ->  \cops' | \copsctxenv' | \copsrunprog'
   | \lbCOPSWrite{\opW(\key,\val), \copsverid,\copsctx } -> \cops'' | \copsctxenv'' | \copsrunprog'' | \lb' -> \copstrc'' }
\]
for a write operation \( \opW(\key,\val) \) from a client \( \cl \) with context \( \copsctx \)
and a replica \( \repl \).
If the step, \( \lb \), from a different client \( \cl' \) before the write operation satisfies:
\begin{Formulae}
\begin{Formula}
\begin{array}{@{}l@{}}
    \left(
    \lb = \lbCOPSWrite[\cl'](\repl'){\opW(\key',\val'), \copsverid', \copsctx' } \land (\repl \neq \repl')
    \right)
    \\ {} \lor
    \left(
    \lb = \lbCOPSOptRead[\cl'](\repl'){\opR(\key',\val'), \copsverid', \copsctx'} 
    \land ( \repl = \repl' 
            \land \key' = \key 
            \implies \copsverid \copsveridleq \copsverid' )
    \right) 
    \\ {} \lor \lb = \lbCOPSRefetch[\cl'](\repl'){\opR(\key',\val'), \copsverid', \copsctx' }
    \lor \lb = \lbCOPSPri[\cl'](\repl') 
    \\ {} \lor \lb = \lbCOPSBound[\cl'](\repl') 
    \lor \lb = \lbCOPSFinishRead[\cl'](\repl'){\copsctx'}
    \lor \lb = \lbCOPSSync(\repl'){\copsverid'},
\end{array}
\label{equ:cops-label-property-for-write-left-move}
\end{Formula}
\end{Formulae}
then the write operation can be moved left, that is,
\begin{Formulae}
\begin{Formula}
\begin{Bracketed} 
 \ToCOPSProg{ \copstrc' | \stub -> \cops | \copsctxenv | \copsrunprog | \lbCOPSWrite{\opW(\key,\val), \copsverid,\copsctx } -> \stub
    | \lb -> \cops'' | \copsctxenv'' | \copsrunprog'' | \lb' -> \copstrc'' }
\end{Bracketed} \copstrceq \copstrc .
\label{equ:cops-write-left-move}
\end{Formula}
\end{Formulae}
\end{theorem}
\begin{proof}
Note that \( \lb \) are all annotated with replica \( \repl' \).
If \(\repl \neq \repl' \) 
by \rCOPSClient in \cref{fig:cops-semantics-program} \cref{equ:cops-write-left-move} trivially holds.
Now consider \( \repl = \repl' \).
We prove \cref{equ:cops-write-left-move} by case analysis on label \( \lb \).
\begin{enumerate}
%\Case{\( \lbCOPSWrite[\cl'](\repl){\opW(\key',\val'), \copsverid', \copsctx' } \)}
    %Given \cref{prop:cops-inv}, there is nothing to prove since the hypothesis is false,
    %since \( \repl = \repl \implies \copsverid' \copsveridleq \copsverid \).
\Case{\( \lbCOPSOptRead[\cl'](\repl){\opR(\key',\val'), \copsverid', \copsctx'} \)}
    For this case, \( \cl \) and \( \cl' \) must be two distinct clients.
    If \( \key \neq \key' \), \cref{equ:cops-write-left-move} trivially holds.
    Consider \( \key = \key' \) and \( \copsverid \copsveridleq \copsverid' \) 
    Let \( \copsver = (\val,\copsverid, \copsctx ) \) and \( \copsver' = (\val',\copsverid', \copsctx' ) \).
    Let \( \copskvs = \cops(\repl) \) and \( \copskvs'' = \cops''(\repl) \).
    By \rCOPSOptRead in \cref{fig:cops-semantics-read}, \( \copskvs = \cops(\repl) = \cops'(\repl) \),
    and \( \copsver' = \copskvs(\key, \Abs{\copskvs(\key)} - 1 )\).
    By \rCOPSWrite in \cref{fig:cops-semantics-write}, we know \( \copskvs'' = \COPSInsert(\copskvs,\key,\copsver' ) \);
    since \( \copsverid \copsveridleq \copsverid' \), the version \( \copsver \)
    is inserted in the middle of \( \copskvs(\key) \).
    Therefore we know that the last version of \( \copskvs''(\key) \) is still \( \copsver' \),
    that is, \( \copsver' = \copskvs''(\key, \Abs{\copskvs''(\key)} - 1 ) \).
    This means that the write operation does not interfere with the previous optimistic read,
    thus \cref{equ:cops-write-left-move} holds.
\Case{\(\lbCOPSOptRead[\cl'](\repl){\opR(\key',\val'), \copsverid', \copsctx' } \)}
    For this case, \( \cl \) and \( \cl' \) must be two distinct clients.
    Let \( \copsver = (\val,\copsverid, \copsctx ) \) and \( \copsver' = (\val',\copsverid', \copsctx' ) \).
    Let \( \copskvs = \cops(\repl) \) and \( \copskvs'' = \cops''(\repl) \).
    By \rCOPSRefetch in \cref{fig:cops-semantics-read}, \( \copskvs = \cops(\repl) = \cops'(\repl) \); 
    and by \rCOPSWrite in \cref{fig:cops-semantics-write}, \( \copskvs'' = \COPSInsert(\copskvs,\key',\copsver' ) \).
    It is easy to see that \( \copskvs \copskvsleq \copskvs'' \) defined in \cref{lem:cops-kvs-leq}.
    By \cref{lem:cops-refetch-independence} we know that \cref{equ:cops-write-left-move} holds.
\Cases{\(\lbCOPSPri[\cl'](\repl)\), \(\lbCOPSBound[\cl'](\repl)\) and  \(\lbCOPSFinishRead[\cl'](\repl){\copsctx'}\)}
    Given the rules presented in \cref{fig:cops-semantics-read}, these steps are local to the client \( \cl' \)
    without any interaction to the key-value store; thus \cref{equ:cops-write-left-move} holds. 
\Case{\(\lbCOPSSync{\copsverid'}\)}
    A new version \( \copsver' \) indexed by \( \copsverid' \) arrives to the replica \( \repl \).
    Let \( \copsver = (\val,\copsverid, \copsctx ) \).
    Let \( \copskvs = \cops(\repl) \), \( \copskvs' = \cops'(\repl) \) and \( \copskvs'' = \cops''(\repl) \).
    By \rCOPSSync in \cref{fig:cops-semantics-program}, we know \( \copskvs' = \COPSInsert(\copskvs,\key',\copsver' ) \),
    \( \copskvs'' = \COPSInsert(\copskvs',\key,\copsver ) \) and \( \copsverid \neq \copsverid'\).
    By the definition  of \( \COPSInsert\), there exists \( \copskvs^*\) such that
    \begin{Formulae}*
    \begin{Formula}
    \copskvs^* = \COPSInsert(\copskvs,\key,\copsver ) \land \copskvs'' = \COPSInsert(\copskvs^*,\key,\copsver' ),
    \end{Formula}
    \end{Formulae}
    and therefore, a new cops database \( \cops^* \) such that
    \[
    \cops^* = \cops\UpdateFunc{\repl -> \copskvs^*}  \land \cops'' = \cops^*\UpdateFunc{\repl -> \copskvs''}.
    \]
    The rule \rCOPSSync does not depend nor change the context environment, thus we have the proof for
    \cref{equ:cops-write-left-move}. \qedhere
\end{enumerate}
\end{proof}
\end{toappendix}

\begin{toappendix}
\begin{lemma}[Re-fetching version on a larger COPS store]
\label{lem:cops-kvs-leq}
\label{lem:cops-refetch-independence}
Given two COPS stores \( \copskvs, \copskvs' \) such \( \copskvs \copskvsleq \copskvs' \), then 
\[
\begin{multlined}
    \ToCOPSCmd{ \copskvs | \copsctx | \ts 
        | \pcopsread( \keyset ) : (\copsverlist, \copsdepset, \copsverlist') | \lbCOPSRefetch{\opR(\key_i, \val_i), \copsverid_i, \copsdep_i} ->
    \copskvs | \copsctx | \ts | \pcopsread( \keyset ) : (\copsverlist, \copsdepset, \copsverlist'')  }
    \\ \implies
    \ToCOPSCmd{ \copskvs' | \copsctx | \ts 
        | \pcopsread( \keyset ) : (\copsverlist, \copsdepset, \copsverlist') | \lbCOPSRefetch{\opR(\key_i, \val_i), \copsverid_i, \copsdep_i} ->
    \copskvs' | \copsctx | \ts | \pcopsread( \keyset ) : (\copsverlist, \copsdepset, \copsverlist'')  }
\end{multlined}
\]
\end{lemma}
\begin{proof}
Depending on \( \copsverlist\Proj{\idx} \), we have two possible cases.
\begin{enumerate}
\Case{\( \IdOf(\copsverlist\Proj{\idx}) \copsveridleq \copsverid_\idx \)}
    For this case, there exists an version \( \copskvs(\key_\idx,m) = (\val_\idx, \copsverid_\idx, \copsdep_\idx) \) for some index \( m \).
    Given a new store \( \copskvs' \) such that \( \copskvs \copskvsleq \copskvs' \), this version must be also included in  \( \copskvs' \),
    thus \( \copskvs'(\key_\idx,m') = (\val_\idx, \copsverid_\idx, \copsdep_\idx ) \) for some \( m' \); therefore we have the proof.
\Case{\( \IdOf(\copsverlist\Proj{\idx}) = \copsverid_\idx \)}
    For this case, the state of \( \copskvs \) is irrelevant. \qedhere
\end{enumerate}
\end{proof}
\end{toappendix}
