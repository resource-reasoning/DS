A normal COPS trace is a trace that: \begin{enumerate*}
\item single-write transactions agree on the order over their the transaction identifiers (\cref{equ:cops-normal-trace-write-order});
\item the second phase of multiple-read transactions cannot be interleaved (\cref{equ:cops-normal-trace-second-read}).
\end{enumerate*}

\begin{definition}[Normal COPS traces]
\label{def:cops-normal-trace}
A COPS trace \( \copstrc \in \COPSTraces \) is in the \emph{normal form} or a \emph{normal COPS trace}, 
written \( \COPSNormalTrace(\copstrc) \), if and only if,
for any clients \( cl, cl'\), replicas \( \repl, \repl'\), keys \( \key, \key' \)
values \( \val,\val' \), transaction identifiers \( \copstxid, \copstxid' \),
contexts \( \copsctx, \copsctx' \) and labels \( \lb,\lb'\),
\begin{Formulae}
& \begin{Formula}
    \copstrc = \ToCOPSProg{\cdots | \lb -> \cdots | \lb' -> \cdots }
    \land \lb = \lbCOPSWrite[\cl](\repl){\opW(\key,\val), \copstxid, \copsctx }
    \\ {} \land \lb' = \lbCOPSWrite[\cl'](\repl'){\opW(\key',\val'), \copstxid', \copsctx' }
    \implies \copstxid \copstxidleq  \copstxid',
\label{equ:cops-normal-trace-write-order}
\end{Formula}
\\ & \begin{Formula}
    \copstrc = \ToCOPSProg{\cdots | \lb -> \stub | \lb' -> \cdots }
    \land \lb = \lbCOPSRefetch[\cl](\repl){\opW(\key,\val), \copstxid, \copsctx }
    \\ \implies \lb' = \lbCOPSRefetch{\opR(\key',\val'), \copstxid', \copsctx' }
    \lor \lb' = \lbCOPSFinishRead{\copsctx}.
\label{equ:cops-normal-trace-second-read}
\end{Formula}
\end{Formulae}
\end{definition}

The normal COPS trace is the key for proving that COPS satisfies causal consistency against our semantics.
In \(\rCOPSWrite\) rule, the new version is inserted into 
the list of the key via \( \COPSInsert \) function that ensures the order of the new list.
However, by \cref{equ:cops-normal-trace-write-order}, it is enough to append the new version and 
it guarantees to preserve the order (\cref{prop:cops-append-write}).
The second property, \cref{equ:cops-normal-trace-second-read}, ensures that 
any multiple-read transaction can be treated as an atomic step:
only the versions that are staged in the second phase (\(\rCOPSRefetch\))
will be returned to the client, 
thus it is enough to ensure no interleaving of the second phase.
Every COPS trace can be transfer by swapping steps into an equivalent normal COPS trace.

\begin{theorem}[Equivalent normal COPS traces]
Given a COPS trace \( \copstrc \in \COPSTraces \),
there is an equivalent normal COPS trace \( \copstrc^* \),
that is, \( \copstrc \copstrceq \copstrc^* \land \COPSNormalTrace(\copstrc^*) \).
\end{theorem}
\begin{proof}
let \( \copstrc^* = \copstrc \) initially.
We perform the following equivalent transformation until \( \copstrc^* \) is a normal COPS trace.
First, we move every re-fetch operation (in the second phase) to the right, that is, delaying the re-fetch,
until the operation is followed by the return of the transaction or other re-fetch operation from the same transaction.
Second, we move the out-of-order write operation to the left.
\begin{enumerate}
\item{\textbf{Right mover: second phase of reads}.} \label{item:cops-second-read}
    Take the first read return step, \( \lbCOPSFinishRead{\copsctx} \),
    with a re-fetch operation, \(\lbCOPSRefetch{\opR(\key,\val), \copstxid, \copsctx' }\),  that 
    is interleaved by other client, that is,
    \begin{Formulae}*
    \begin{Formula}
        \copstrc^* = \ToCOPSProg{\copstrc' | \stub 
            -> \stub | \lbCOPSRefetch{\opR(\key,\val), \copstxid, \copsctx' }
            -> \stub | \lb -> \copstrc'' | \lbCOPSFinishRead{\copsctx} -> \copstrc''' }
        \land \lb \neq \lbCOPSRefetch{\opR(\key',\val'), \copstxid', \copsctx'' }
    \end{Formula}
    \end{Formulae}
    for three trace segments \( \copstrc', \copstrc'', \copstrc''' \).
    The \( \lb \) can be moved left:
    \begin{Formulae}*
    \begin{Formula}
        \begin{Bracketed} \ToCOPSProg{\copstrc' | \stub 
            -> \stub | \lb
            -> \stub | \lbCOPSRefetch{\opR(\key,\val), \copstxid, \copsctx' } 
            -> \copstrc'' | \lbCOPSFinishRead{\copsctx} -> \copstrc''' } .
        \end{Bracketed} \copstrceq \copstrc^*
    \end{Formula}
    \end{Formulae}
    If the label \( \lb \) is not a write operation, 
    it is easy to see that \( \lb \) and the re-fetch operation can be swapped.
    If the label \( \lb \) is a write operation, it must not interfere the re-fetch operation.
    This is because any re-fetch operations can only read versions written before the lower-bound computation step.
    The full proof of the right mover is given in \cref{lem:cops-no-interleave-second-read}
    on \pageref{lem:cops-no-interleave-second-read}.

    Assign the new trace to \( \copstrc^* \) and go back to \cref{item:cops-second-read}.
    There are finite number of steps in a trace 
    and for each iteration, a re-fetch label moves closer to its read return label,
    thus \cref{item:cops-second-read} must terminate.
    The final trace \( \copstrc^* \) satisfies \cref{equ:cops-normal-trace-second-read}.

\item{\textbf{Left mover: out-of-order writes}.} \label{item:cops-write}
    Given a trace \( \copstrc^* \) satisfying \cref{equ:cops-normal-trace-second-read},
    take two out-of-order write operations with the shortest distance in between,
    \( \lbCOPSWrite[\cl](\repl){\opW(\key,\val), \copstxid, \copsctx } \) and
    \( \lbCOPSWrite[\cl'](\repl'){\opW(\key',\val'), \copstxid', \copsctx' } \),
    and the operation immediately before the latter write, \( \lb \),
    that is,
    \begin{Formulae}*
    \begin{Formula}
        \copstrc^* = \ToCOPSProg{\copstrc' | \stub 
            -> \stub | \lbCOPSWrite[\cl](\repl){\opW(\key,\val), \copstxid, \copsctx }
            -> \copstrc'' | \lb
            -> \stub | \lbCOPSWrite[\cl'](\repl'){\opW(\key',\val'), \copstxid', \copsctx' }
            -> \copstrc''' }
        \land \copstxid' \copstxidleq \copstxid ,
    \end{Formula}
    \end{Formulae}
    where all the free variables are universally quantified.
    Note that there is no other out-of-order in the trace segment \( \copstrc'' \).
    Let consider the step \( \lb \):
    \begin{enumerate}
    \item if the step is not a read or write operation, 
    the step can be delayed, that is, the out-of-order write operation can be moved to the left;
    \item if the step is a write operation,
    by \cref{prop:cops-inv} it must come from a client that differs from \( \cl' \),
    thus it is safe to move the out-of-order write operation to the left; and 
    \item  if the step is a read operation, 
    it is trivial that 
    the read cannot depend on the out-of-order write operation,
    thus it is safe to move the out-of-order write operation to the left.
    \end{enumerate}
    The full proof of the left mover is given in \cref{lem:cops-write-left-move} on \pageref{lem:cops-write-left-move}.

    Continue to apply the left mover until the two write operations are in order:
    \[
        \begin{Bracketed} \ToCOPSProg{\copstrc' | \stub 
            -> \stub | \lbCOPSWrite[\cl'](\repl'){\opW(\key',\val'), \copstxid', \copsctx' }
            -> \stub | \lbCOPSWrite[\cl](\repl){\opW(\key,\val), \copstxid, \copsctx } 
            -> {\copstrc''}^* | \lb
            -> \copstrc''' } \end{Bracketed} \copstrceq \copstrc^*
    \]
    for some new trace segment \( {\copstrc''}^* \) that
    contains the same operations as in \( \copstrc''\),
    however, the states must incorporate the effect of 
    \( \lbCOPSWrite[\cl'](\repl'){\opW(\key',\val'), \copstxid', \copsctx' } \).
    Assign the new trace to \( \copstrc^* \) and go back to \cref{item:cops-write}.
    For each iteration, two out-of-order write operations move closer to each other 
    and eventually swap position;
    there are finite number of steps in a trace, thus finite number of out-of-order write operations.
    This means \cref{item:cops-write} must terminate.
    The final trace \( \copstrc^* \) satisfies \cref{equ:cops-normal-trace-write-order}.
\end{enumerate}
After \cref{item:cops-second-read,item:cops-write}, 
the trace \( \copstrc^*\) is in the normal form as \( \COPSNormalTrace(\copstrc^*) \).
\end{proof} 

%%%%%%%%%%%%%%%%%%%%%%%%%%%%%%%%%%%%%%%%%%%%%%%%%%%%%%

\begin{theorem}[Right mover: re-fetch operations]
\label{lem:cops-no-interleave-second-read}
Assume a COPS trace \( \copstrc \in \COPSTraces \) that
contains a re-fetch operation, \lbCOPSRefetch{\opR(\key,\val), \copstxid,\copsctx }, 
that is interleaved by other operation, \( \lb \):
\[
\copstrc = \ToCOPSProg{ \copstrc' | \stub -> \cops | \copsctxenv | \copsrunprog | \lbCOPSRefetch{\opR(\key,\val), \copstxid,\copsctx } 
    ->  \cops' | \copsctxenv' | \copsrunprog'
    | \lb -> \cops'' | \copsctxenv'' | \copsrunprog'' | \lb' -> \copstrc'' } ,
\]
and
\[
    \lb \neq \lbCOPSRefetch{\opR(\key',\val'), \copstxid',\copsctx' }  
    \land \lb \neq \lbCOPSFinishRead{\copsctx}
\]
where all other free variables are universally quantified.
Then the re-fetch operation can be moved to the right, that is,
\begin{Formulae}
\begin{Formula}
\begin{Bracketed} 
\ToCOPSProg{ \copstrc' | \stub -> \cops | \copsctxenv | \copsrunprog | \lb  
    ->  \stub | \lbCOPSRefetch{\opR(\key,\val), \copstxid,\copsctx }
    -> \cops'' | \copsctxenv'' | \copsrunprog'' | \lb' -> \copstrc'' } 
\end{Bracketed} \copstrceq \copstrc .
\label{equ:cops-interleaving-left-move}
\end{Formula}
\end{Formulae}
\end{theorem}
\begin{proof}
By \(\rCOPSRefetch\) in \cref{fig:cops-semantics-read}, any read from the second phase
does not change the store nor the context, thus
\[
\copstrc = \ToCOPSProg{ \copstrc' | \stub -> \cops | \copsctxenv | \copsrunprog | \lbCOPSRefetch{\opR(\key,\val), \copstxid,\copsctx } 
    ->  \cops | \copsctxenv' | \copsrunprog'
    | \lb -> \cops'' | \copsctxenv'' | \copsrunprog'' | \lb' -> \copstrc'' }.
\]
We perform case analysis on the label \( \lb \).
\begin{enumerate}
\Case{\( \lbCOPSWrite[\cl'](\repl'){\opW(\key',\val'), \copstxid', \copsctx' } \)}
    We immediately know \( \cl \neq \cl' \).
    If \( \repl \neq \repl' \), \cref{equ:cops-interleaving-left-move} trivially holds.
    Consider \( \repl = \repl' \).
    Let \( \copskvs = \cops(\repl) \) and \( \copskvs'' = \cops''(\repl )\).
    It is easy to see that \( \copskvs \copskvsleq \copskvs'' \),
    thus by \cref{lem:cops-kvs-leq}, we have 
    \[
    \ToCOPSProg{ \copstrc' | \stub -> \cops | \copsctxenv | \copsrunprog |  \lb
        ->  \cops'' | \copsctxenv'' | \copsrunprog^*
        | \lbCOPSRefetch{\opR(\key,\val), \copstxid,\copsctx }
        -> \cops'' | \copsctxenv'' | \copsrunprog'' | \lb' -> \copstrc'' }.
    \]
    for some program \( \copsrunprog^*\); this implies \cref{equ:cops-interleaving-left-move}.
\Cases{\( \lbCOPSOptRead[\cl'](\repl'){\opR(\key',\val'), \copstxid', \copsctx'} \)
        , \( \lbCOPSOptRead[\cl'](\repl'){\opR(\key',\val'), \copstxid', \copsctx'} \)
        , \( \lbCOPSRefetch[\cl'](\repl'){\opR(\key',\val'), \copstxid', \copsctx' }\)
        , \( \lbCOPSPri[\cl'](\repl') \)
        and \( \lbCOPSBound[\cl'](\repl') \) }
    First it is easy to see that \( \cl \neq \cl'\);
    and these steps do not change the states of the database nor the contexts.
    Therefore \cref{equ:cops-interleaving-left-move} holds for these steps.
\Case{\(\lbCOPSFinishRead[\cl'](\repl'){\copsctx} \) }
    By \rCOPSFinishRead in \cref{fig:cops-semantics-read}, it is easy to see that \( \cl \neq \cl'\);
    and this step does not change the states of the database.
    Because \( \cl \neq \cl' \) the new context for \( \cl' \)
    will not affect the \( \cl \), thus 
    \[
    \ToCOPSProg{ \copstrc' | \stub -> \cops | \copsctxenv | \copsrunprog |  \lb
        ->  \cops | \copsctxenv'' | \copsrunprog^* | \lbCOPSFinishRead[\cl'](\repl'){\copsctx}
        -> \cops'' | \copsctxenv'' | \copsrunprog'' | \lb' -> \copstrc'' },
    \]
    which implies \cref{equ:cops-interleaving-left-move}.
\Case{\(\lbCOPSSync(\repl'){\copstxid} \) }
    Let \( \copskvs = \cops(\repl) \) and \( \copskvs'' = \cops''(\repl )\).
    By \rCOPSSync in \cref{fig:cops-semantics-program},
    It is easy to see that \( \copskvs \copskvsleq \copskvs'' \),
    \( \copsctxenv' = \copsctxenv'' \) and \( \copsrunprog' = \copsrunprog'' \);
    thus by \cref{lem:cops-kvs-leq} and the fact that \rCOPSSync does not rely on any context nor program,
    we have 
    \[
    \ToCOPSProg{ \copstrc' | \stub -> \cops | \copsctxenv | \copsrunprog |  \lb
        ->  \cops'' | \copsctxenv | \copsrunprog
        | \lbCOPSRefetch{\opR(\key,\val), \copstxid,\copsctx }
        -> \cops'' | \copsctxenv'' | \copsrunprog'' | \lb' -> \copstrc'' },
    \]
    which implies \cref{equ:cops-interleaving-left-move}. 
\end{enumerate}
\end{proof}

Both replicas and clients are monotonic in that:
\begin{enumerate*}
\item every replica contains increasingly more versions and the local time must tick forward;
and \item similarly, every client context contains increasingly more versions.
\end{enumerate*}
This is a useful property that allows us to swap out-of-order write operations.

%%%%%%%%%%%%%%%%%%%%%%%%%%%%%%

\begin{proposition}[Monotonicity of COPS replica and client]
\label{prop:cops-inv}
For any COPS databases \( \cops, \cops'\), client context environments \( \copsctxenv, \copsctxenv' \),
programs \( \copsrunprog, \copsrunprog' \) and label \( \lb \),
\begin{Formulae}
\begin{Formula}
    \ToCOPSProg{ \cops | \copsctxenv | \copsrunprog | \stub | * -> \cops' | \copsctxenv' | \copsrunprog' }
    \\ \implies \Forall{\repl \in \Dom(\cops)} \cops(\repl)\Proj{0} \copskvsleq \cops'(\repl)\Proj{0}
    \land \cops(\repl)\Proj{1} \leq \cops'(\repl)\Proj{1} 
    \\ {} \land \Forall{\cl \in \Dom(\copsctxenv)} \copsctxenv(\cl)\Proj{0} \subseteq  \copsctxenv'(\cl)\Proj{0} .
\label{equ:cops-replica-inv}
\end{Formula}
\end{Formulae}
\end{proposition}
\begin{proof}
We prove by induction on the length of the trace \( n \).
\CaseBase{\(n = 0\)}
    \Cref{equ:cops-replica-inv} is trivially true as \( \cops = \cops' \) and \( \copsctxenv = copsctxenv' \).
\CaseBase{\(n > 0\)}
    Assume
    \(
        \ToCOPSProg{ \cops | \copsctxenv | \copsrunprog | \stub | (n-1)
        -> \cops'' | \copsctxenv'' | \copsrunprog'' | \lb
        -> \cops' | \copsctxenv' | \copsrunprog' }.
    \)
    By \ih, we know \( \cops(\repl)\Proj{0} \copskvsleq \cops'(\repl)\Proj{0} \) 
    and \( \cops(\repl)\Proj{1} \leq \cops'(\repl)\Proj{1} \)  for \( \repl \in \Dom(\cops) \).
    Consider the label \( \lb \).
    If \( \lb = \lbCOPSWrite[cl](\repl){\opW(\key,\val), \copstxid, \copsctx } \),
    by \rCOPSWrite, we have \( \cops' = \cops''\UpdateFunc{{\repl} -> \Tuple{\copskvs^*, \cops''(\repl)\Proj{1} + 1}}\),
    for \( \copskvs^* = \COPSInsert(\cops''(\repl)\Proj{0},\key,\copsver) \) for a version \( \copsver \)
    and, thus \cref{equ:cops-replica-inv}.
    If \( \lb = \lbCOPSSync(\repl){\copstxid} \),
    by \rCOPSSync we have \( \cops' = \cops''\UpdateFunc{\repl -> \Tuple{\copskvs^*, m } }\) such that
    \( m =\Max(\cops''(\repl)\Proj{1},m^*) \) where \( \copstxid = \copstxid[\cl^*][\repl^*](m^*)\), 
    and \( \copskvs' = \COPSInsert(\cops''(\repl)\Proj{0},\key,\copsver) \) for a version \( \copsver \),
    and, thus \cref{equ:cops-replica-inv}.
    For the rest cases, they do not change any COPS store and local time;
    by \ih, \cref{equ:cops-replica-inv} holds. 
\end{proof}

%A write operation is out-of-order, 
%if in COPS trace there is any write operation (that may from different a client on different a replica),

\begin{theorem}[Left mover: out-of-order write]
\label{lem:cops-write-left-move}
Assume a COPS trace \( \copstrc \in \COPSTraces \) such that
\[
\copstrc = \ToCOPSProg{ \copstrc' | \stub -> \cops | \copsctxenv | \copsrunprog | \lb ->  \cops' | \copsctxenv' | \copsrunprog'
   | \lbCOPSWrite{\opW(\key,\val), \copstxid,\copsctx } -> \cops'' | \copsctxenv'' | \copsrunprog'' | \lb' -> \copstrc'' }
\]
for a write operation \( \opW(\key,\val) \) from a client \( \cl \) with context \( \copsctx \)
and a replica \( \repl \).
If the step, \( \lb \), before the write operation satisfies:
\begin{Formulae}
\begin{Formula}
\begin{array}{@{}l@{}}
    \left(
    \lb = \lbCOPSWrite[\cl'](\repl'){\opW(\key',\val'), \copstxid', \copsctx' } \land (\repl \neq \repl')
    \right)
    \\ {} \lor
    \left(
    \lb = \lbCOPSOptRead[\cl'](\repl'){\opR(\key',\val'), \copstxid', \copsctx'} 
    \land ( \repl = \repl' 
            \land \key' = \key 
            \implies \copstxid \copstxidleq \copstxid' )
    \right) 
    \\ {} \lor \lb = \lbCOPSRefetch[\cl'](\repl'){\opR(\key',\val'), \copstxid', \copsctx' }
    \lor \lb = \lbCOPSPri[\cl'](\repl') 
    \\ {} \lor \lb = \lbCOPSBound[\cl'](\repl') 
    \lor \lb = \lbCOPSFinishRead[\cl'](\repl'){\copsctx'}
    \lor \lb = \lbCOPSSync(\repl'){\copstxid'},
\end{array}
\label{equ:cops-label-property-for-write-left-move}
\end{Formula}
\end{Formulae}
then the write operation can be moved left, that is,
\begin{Formulae}
\begin{Formula}
\begin{Bracketed} 
 \ToCOPSProg{ \copstrc' | \stub -> \cops | \copsctxenv | \copsrunprog | \lbCOPSWrite{\opW(\key,\val), \copstxid,\copsctx } -> \stub
    | \lb -> \cops'' | \copsctxenv'' | \copsrunprog'' | \lb' -> \copstrc'' }
\end{Bracketed} \copstrceq \copstrc .
\label{equ:cops-write-left-move}
\end{Formula}
\end{Formulae}
\end{theorem}
\begin{proof}
Note that \( \lb \) are all annotated with replica \( \repl' \).
If \(\repl \neq \repl' \) 
by \rCOPSClient in \cref{fig:cops-semantics-program} \cref{equ:cops-write-left-move} trivially holds.
Now consider \( \repl = \repl' \).
We prove \cref{equ:cops-write-left-move} by case analysis on label \( \lb \).
\begin{enumerate}
%\Case{\( \lbCOPSWrite[\cl'](\repl){\opW(\key',\val'), \copstxid', \copsctx' } \)}
    %Given \cref{prop:cops-inv}, there is nothing to prove since the hypothesis is false,
    %since \( \repl = \repl \implies \copstxid' \copstxidleq \copstxid \).
\Case{\( \lbCOPSOptRead[\cl'](\repl){\opR(\key',\val'), \copstxid', \copsctx'} \)}
    For this case, \( \cl \) and \( \cl' \) must be two distinct clients.
    If \( \key \neq \key' \), \cref{equ:cops-write-left-move} trivially holds.
    Consider \( \key = \key' \) and \( \copstxid \copstxidleq \copstxid' \) 
    Let \( \copsver = (\val,\copstxid, \copsctx ) \) and \( \copsver' = (\val',\copstxid', \copsctx' ) \).
    Let \( \copskvs = \cops(\repl) \) and \( \copskvs'' = \cops''(\repl) \).
    By \rCOPSOptRead in \cref{fig:cops-semantics-read}, \( \copskvs = \cops(\repl) = \cops'(\repl) \),
    and \( \copsver' = \copskvs(\key, \Abs{\copskvs(\key)} - 1 )\).
    By \rCOPSWrite in \cref{fig:cops-semantics-write}, we know \( \copskvs'' = \COPSInsert(\copskvs,\key,\copsver' ) \);
    since \( \copstxid \copstxidleq \copstxid' \), the version \( \copsver \)
    is inserted in the middle of \( \copskvs(\key) \).
    Therefore we know that the last version of \( \copskvs''(\key) \) is still \( \copsver' \),
    that is, \( \copsver' = \copskvs''(\key, \Abs{\copskvs''(\key)} - 1 ) \).
    This means that the write operation does not interfere with the previous optimistic read,
    thus \cref{equ:cops-write-left-move} holds.
\Case{\(\lbCOPSOptRead[\cl'](\repl){\opR(\key',\val'), \copstxid', \copsctx' } \)}
    For this case, \( \cl \) and \( \cl' \) must be two distinct clients.
    Let \( \copsver = (\val,\copstxid, \copsctx ) \) and \( \copsver' = (\val',\copstxid', \copsctx' ) \).
    Let \( \copskvs = \cops(\repl) \) and \( \copskvs'' = \cops''(\repl) \).
    By \rCOPSRefetch in \cref{fig:cops-semantics-read}, \( \copskvs = \cops(\repl) = \cops'(\repl) \); 
    and by \rCOPSWrite in \cref{fig:cops-semantics-write}, \( \copskvs'' = \COPSInsert(\copskvs,\key',\copsver' ) \).
    It is easy to see that \( \copskvs \copskvsleq \copskvs'' \) defined in \cref{lem:cops-kvs-leq}.
    By \cref{lem:cops-refetch-independence} we know that \cref{equ:cops-write-left-move} holds.
\Cases{\(\lbCOPSPri[\cl'](\repl)\), \(\lbCOPSBound[\cl'](\repl)\) and  \(\lbCOPSFinishRead[\cl'](\repl){\copsctx'}\)}
    Given the rules presented in \cref{fig:cops-semantics-read}, these steps are local to the client \( \cl' \)
    without any interaction to the key-value store; thus \cref{equ:cops-write-left-move} holds. 
\Case{\(\lbCOPSSync{\copstxid'}\)}
    A new version \( \copsver' \) indexed by \( \copstxid' \) arrives to the replica \( \repl \).
    Let \( \copsver = (\val,\copstxid, \copsctx ) \).
    Let \( \copskvs = \cops(\repl) \), \( \copskvs' = \cops'(\repl) \) and \( \copskvs'' = \cops''(\repl) \).
    By \rCOPSSync in \cref{fig:cops-semantics-program}, we know \( \copskvs' = \COPSInsert(\copskvs,\key',\copsver' ) \),
    \( \copskvs'' = \COPSInsert(\copskvs',\key,\copsver ) \) and \( \copstxid \neq \copstxid'\).
    By the definition  of \( \COPSInsert\), there exists \( \copskvs^*\) such that
    \begin{Formulae}*
    \begin{Formula}
    \copskvs^* = \COPSInsert(\copskvs,\key,\copsver ) \land \copskvs'' = \COPSInsert(\copskvs^*,\key,\copsver' ),
    \end{Formula}
    \end{Formulae}
    and therefore, a new cops database \( \cops^* \) such that
    \[
    \cops^* = \cops\UpdateFunc{\repl -> \copskvs^*}  \land \cops'' = \cops^*\UpdateFunc{\repl -> \copskvs''}.
    \]
    The rule \rCOPSSync does not depend nor change the context environment, thus we have the proof for
    \cref{equ:cops-write-left-move}. 
\end{enumerate}
\end{proof}

\begin{lemma}[Re-fetching version on a larger COPS store]
\label{lem:cops-kvs-leq}
\label{lem:cops-refetch-independence}
Given two COPS stores \( \copskvs, \copskvs' \) such \( \copskvs \copskvsleq \copskvs' \), then 
\[
\begin{multlined}
    \ToCOPSCmd{ \copskvs | \copsctx | \ts 
        | \pcopsread( \keyset ) : (\copsverlist, \copsdepset, \copsverlist') | \lbCOPSRefetch{\opR(\key_i, \val_i), \copstxid_i, \copsdep_i} ->
    \copskvs | \copsctx | \ts | \pcopsread( \keyset ) : (\copsverlist, \copsdepset, \copsverlist'')  }
    \\ \implies
    \ToCOPSCmd{ \copskvs' | \copsctx | \ts 
        | \pcopsread( \keyset ) : (\copsverlist, \copsdepset, \copsverlist') | \lbCOPSRefetch{\opR(\key_i, \val_i), \copstxid_i, \copsdep_i} ->
    \copskvs' | \copsctx | \ts | \pcopsread( \keyset ) : (\copsverlist, \copsdepset, \copsverlist'')  }
\end{multlined}
\]
\end{lemma}
\begin{proof}
Depending on \( \copsverlist\Proj{\idx} \), we have two possible cases.
\begin{enumerate}
\Case{\( \WtOf(\copsverlist\Proj{\idx}) \copstxidleq \copstxid_\idx \)}
    For this case, there exists an version \( \copskvs(\key_\idx,m) = (\val_\idx, \copstxid_\idx, \copsdep_\idx) \) for some index \( m \).
    Given a new store \( \copskvs' \) such that \( \copskvs \copskvsleq \copskvs' \), this version must be also included in  \( \copskvs' \),
    thus \( \copskvs'(\key_\idx,m') = (\val_\idx, \copstxid_\idx, \copsdep_\idx ) \) for some \( m' \); therefore we have the proof.
\Case{\( \WtOf(\copsverlist\Proj{\idx}) = \copstxid_\idx \)}
    For this case, the state of \( \copskvs \) is irrelevant. 
\end{enumerate}
\end{proof}
