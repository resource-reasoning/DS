In our kv-store semantics (\cref{sec:model}), 
we assign a unique identifier for every transition (\(\rCAtomicTrans\)),
but in COPS there is explicit transaction identifier.
We encode the COPS version identifier as the transaction identifier
for the single-write transaction who committed the version.
We then annotate multiple-read transactions with transaction identifiers that preserves the session order.

\begin{definition}[Extended COPS traces]
\emph{A client local time environments} is defined by \(\copsclenv \IsTyped \cl \ToTFunc \COPSTxIDs \).
Given a normal COPS trace \( \copstrc \in \COPSTraces \),
the extended COPS trace for \( \copstrc \) is defined by
\begin{align*}
   \COPSToExt(\copsconfinit,\copsclenv) & \FuncDef \copsconfinit ,
\\ \multicolumn{2}{l}{
        \COPSToExt( \begin{Bracketed} \ToCOPSProg{ \cops | \copsctxenv 
            | \copsrunprog | \lb -> \copstrc } \end{Bracketed} , \copsclenv ) }
\\ \multicolumn{2}{l}{
        \( \quad \FuncDef 
        \begin{cases}
            \ToCOPSProg{ \cops | \copsctxenv | \copsrunprog | \lb
                -> \COPSToExt(\copstrc, \copsclenv\UpdateFunc{\cl -> \copstxid[\cl][\repl](n,1) }) }    
                & \If \lb =  \lbCOPSWrite{\opW(\key,\val), \copstxid[\cl][\repl](n,0), \copsctx } ,
         \\ \ToCOPSProg{ \cops | \copsctxenv | \copsrunprog | \lb, \copsclenv(\cl)
                -> \COPSToExt(\copstrc, \copsclenv) }
                & \If \lb =  \lbCOPSRefetch{\opR(\key,\val), \copstxid, \copsctx }  ,
         \\ \ToCOPSProg{ \cops | \copsctxenv | \copsrunprog | \lb, \copsclenv(\cl)
                -> \COPSToExt(\copstrc, \copsclenv\UpdateFunc{\cl -> \copstxid[\cl][\repl](n,m+1) }) }
                & \If \lb =  \lbCOPSFinishRead{ \copsctx } \land \copsclenv(\repl) = \copstxid[\cl][\repl](n,m) ,
         \\ \ToCOPSProg{ \cops | \copsctxenv | \copsrunprog | \lb
                -> \COPSToExt(\copstrc, \copsclenv) }
                & \ow .
        \end{cases} \)
    }
\end{align*}
Then the set of \emph{extended COPS traces}, \( \ExtCOPSTraces \ni \copsexttrc \), is defined by
defined by 
\[ 
    \ExtCOPSTraces \TypeDef \Set{\COPSToExt(\copstrc,\copsclenv) | \copstrc \in \COPSTraces 
        \\ {} \land \Exists{\copsctxenv \in \COPSContextEnvs} 
        (\stub,\copsctxenv) = \copstrc\Proj{0}
        \\ \implies 
        \Forall{\cl \in \Dom(\copsctxenv) }
        \Forall{ \repl \in \COPSReplicas } 
        \\ \copsctxenv(\cl) = (\stub,\repl) 
        \implies \copsclenv(\cl) = \copstxid[\cl][\repl](0,1) 
    } .
\]
\end{definition}

We track the next available multiple-read transaction identifier for every client,
that is, \( \copsclenv \).
Recall that each COPS transaction identifier \( \copstxid[\cl][\repl](n,m) \) is annotated with
a extra counter \( m \), namely the read counter.
In the COPS semantics, the read counter is always set to be ZERO.
Here, we assign transaction identifiers with non-zero read counters to multiple-read transactions:
for each multiple-read transaction, 
we annotate the re-fetch operations and the read return operation 
with the next available identifier in \( \copsclenv \).
To preserve the session order,
we update \( \copsclenv \) for the client \( \cl \),
if \( \cl \) commits a new transaction.
By construction, the transaction identifiers that are assigned to multiple-read transactions must be unique.

\begin{proposition}[Fresh multiple-read transaction identifiers]
\label{prop:cops-read-id-unique}
Given an extended COPS trace \( \copsexttrc \in \ExtCOPSTraces \),
the annotated multiple-read transaction identifiers must be fresh with respect the key being read,
that is,
\begin{Formulae}*
\begin{Formula}
\copsexttrc = \ToCOPSProg{ \copsexttrc' | \stub 
                -> \cops |  \copsctxenv | \copsrunprog | \lbCOPSRefetch{\opR(\key,\val), \copstxid, \copsdep }, \copstxid'
                -> \copsexttrc'' }
        \\ \land
        \Forall{\repl^* \in \COPSReplicas | \cl^* \in \Clients | \val^* \in \Values | \copsdep^* \in \COPSDependencies }
        \\ \copsexttrc'' = \ToCOPSProg{\cdots | \lbCOPSRefetch[\cl^*](\repl^*){\opR(\key,\val^*), \copstxid^*, \copsdep^* }, \copstxid'' -> \cdots }
        \implies  \copstxid' \neq \copstxid''
\end{Formula}
\end{Formulae}
\end{proposition}
\begin{proof}
Suppose a replica \( \repl^* \), a client \( \cl^* \), a value \( \val^* \) and dependency set \( \copsdep^* \)
such that
\[
\copsexttrc'' = \ToCOPSProg{\cdots | \lbCOPSRefetch[\cl^*](\repl^*){\opR(\key,\val^*), \copstxid^*, \copsdep^* }, \copstxid'' -> \cdots }.
\]
If \( \repl \neq \repl^*\) or \( \cl \neq \cl^* \), 
by the definition of \(\ExtCOPSTraces\) (especially \(\COPSToExt\)), it is trivial that \(\copstxid \neq \copstxid''\).
Consider \( \repl = \repl^*\) or \( \cl = \cl^* \).
Because a client must provide a unique set of key \( \keyset \) when calling \( \pcopsread \)
by the rules in \cref{fig:cops-semantics-read},
it means that
\[
\copsexttrc'' = \ToCOPSProg{\cdots | \lbCOPSRefetch{ \copsdep' }, \copstxid'
    -> \cdots | \lbCOPSRefetch[\cl^*](\repl^*){\opR(\key,\val^*), \copstxid^*, \copsdep^* }, \copstxid'' -> \cdots }.
\]
For this case, by the definition of \(\ExtCOPSTraces\) (especially \(\COPSToExt\)), 
it must be the case that \(\copstxid \neq \copstxid''\).
\end{proof}


In our kv-store semantics, new versions are appended to the end,
however, in COPS semantics, new versions are inserted into the lists to preserve
the version order within the lists.
As explained before, write operations in a normal COPS trace,
henceforth in an extended COPS trace, must be in order.
Thus, it is safe to \emph{append} any write operation to the end of the list of the key,
instead of inserting into the list.

\begin{proposition}[Appending write operation]
\label{prop:cops-append-write}
Given an extended COPS trace \( \copsexttrc \),
for every write operation, \( \lb = \lbCOPSWrite{\opW(\key,\val), \copstxid, \copsctx } \),
the version identifier, \(\copstxid \), is strictly greater than all version identifiers included in all replicas,
which means that
\begin{Formulae}*
\begin{Formula}
\copsexttrc = \ToCOPSProg{ \copsexttrc' | \stub 
                -> \cops |  \copsctxenv | \copsrunprog | \lbCOPSWrite{\opW(\key,\val), \copstxid, \copsctx }
                -> \cops' | \copsctxenv' | \copsrunprog' | \stub
                -> \cdots }
        \\ \land 
        \Forall{\repl' \in \COPSReplicas | \copskvs' \in \COPSKVSs | \key' \in \Keys | \idx' \in \Indexs}
        \\ \cops(\repl') = (\copskvs',\stub)
        \land 0 \leq \idx < \copskvs(\key')
        \implies  \WtOf(\copskvs'(\key',\idx')) \copstxidle \copstxid
\end{Formula}
\end{Formulae}
\end{proposition}
\begin{proof}
Suppose a replica \( \repl' \) with key-value store \( \copskvs' \) (\( \cops(\repl') = (\copskvs',\stub)\)), a key \( \key' \)
and an index \( \idx' \) such that \( 0 \leq \idx < \copskvs(\key') \).
Let the \( \copstxid' = \WtOf(\copskvs(\key',\idx'))\).
There must exist a step with label \( \lb' \) with some client \( \cl' \)
in the trace \( \copsexttrc' \) that committed this version:
\[ 
\copsexttrc' = \ToCOPSProg{ \cdots | \lb' -> \cdots }
\land \lb' = \lbCOPSWrite{\opW(\key',\val'), \copstxid', \copsdep } 
\]
for some value \( \val ' \) and dependent set \(\copsdep \).
Since an extended trace \( \copsexttrc \) is also in its normal form, that is, \( \COPSNormalTrace(\copsexttrc) \),
and the new kv-store \( \cops'(\repl')\Proj{0} \) must be a well-formed COPS key-value store,
therefore \( \copstxid' \copstxidle \copstxid \).
\end{proof}
