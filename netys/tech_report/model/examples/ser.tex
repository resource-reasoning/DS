\paragraph{Strict serialisability (\(\SER\))}  
This model is the strongest consistency model
in any framework that abstracts from aborted transactions, 
requiring that transactions execute in a total sequential order.
The \(\CanCommit[\SER]\) thus allows a client to commit a transaction only 
when the client view on the kv-store is complete
in that the view is closed with respect to \(\WWInv[\kvs]\). 
This requirement prevents the kv-store in  \cref{fig:ser-disallowed}.
Without loss of generality, suppose that \(\txid\) commits before \(\txid'\),
then the client committing \(\txid'\) must see the version of \(\key_1\) written by \(\txid\), 
and thus cannot read the outdated value \(\val_0\) for \(\key_1\). 
This example, known as \emph{write skew anomaly}, 
is allowed by all other execution tests in \cref{fig:execution-tests}.

\begin{figure}
\centering
\begin{tikzpicture}%
\KVMapping{x}{\key_1}{
    /\val_0/\txid_0/\Set{\boldsymbol{\txid'}}
    , /\val_1/\txid/\emptyset
};
\KVMapping[x]{y}{\key_2}{
    /\val_0/\txid_0/\Set{\txid}
    , /\val_2/\boldsymbol{\txid'}/\emptyset
};
\end{tikzpicture}%

\hrulefill

\caption{Write skew anomaly, disallowed by \(\SER\)}
\label{fig:ser-disallowed}
\end{figure}%
