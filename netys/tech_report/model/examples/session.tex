\citet{session-guarantee} proposed four \emph{session guarantees},
including \emph{monotonic read} \((\MR)\), \emph{read your write} \((\RYW)\), 
\emph{monotonic write} \((\MW)\) and \emph{write follows read} \((\WFR)\) models  
However, they 
only gave informal description of these models in replicated databases.
\citet{repldatatypes} gave formal definitions on abstract executions
using axioms on the visibility relations.
Our definitions of these models are correct with respect to theirs.
\citet{repldatatypes} also proposed two extra session guarantees,
called \emph{write follow read in arbitration} and \emph{monotonic write in arbitration}.
These two models restrict the arbitration relations on abstract executions.
We may not be able to capture these two models, 
because we do not have the total order information over transactions in our kv-store.

\paragraph{Monotonic read (\( \MR \)) } 
\label{sec:et-mr} 
This model states that after committing a transaction, 
a client cannot lose information in that 
it can only see increasingly more versions from a kv-store.
This prevents, for example, the kv-store in \cref{fig:mr-disallowed},
since client \(\cl\) first reads the latest version of \(\key\) in \(\txid_{\cl}^{1}\), 
and then reads the older, initial version of \(\key\) in \(\txid_{\cl}^{2}\).  
As such, the \(\ViewShift[\MR]\) predicate in \cref{fig:execution-tests} 
ensures that clients can only extend their views,
that is, \( \vi \vileq \vi' \) for views \( \vi, \vi'\) before and after committing.
When this is the case, clients can then \emph{always} commit their transactions,
and thus \(\CanCommit[\MR]\) is simply defined as \(\true\). 

\paragraph{Read your write (\(\RYW\))} 
\label{sec:et-ryw}
This model states that 
a client must always see all the versions written by the client itself. 
Under \(\RYW\) the kv-store in \cref{fig:ryw-disallowed} 
is prohibited as the initial version of \(\key\) holds value \(\val_0\) 
and client \(\cl\) tries to \emph{update} the value of \(\key\) twice.  
For its first transaction \( \txid_{\cl}^1\), 
it reads the initial value \(\val_0\) and then writes a new version with value \(\val_1\). 
For its second transaction \( \txid_{\cl}^2\), 
it reads the initial value \(\val_0\) again and write a new version with value \(\val_2\).
The \(\ViewShift[\RYW]\) predicate defined by:
\[
\ViewShift[\RYW](\kvs,\vi,\kvs',\vi') \PredDef
	\begin{multlined}[t]
		\Forall{\txid \in \kvs' \setminus \kvs | \key \in \kvs' | \idx} 
        \\ \ToEdge{\WtOf(\kvs'(\key, i)) | \Refl(\SO)  -> \txid }  
        \implies \idx \in \vi'(\key) ,
	\end{multlined}
\]
rules out this kv-store in \cref{fig:ryw-disallowed}
by requiring that the client view \( \vi' \), 
after the commits the transaction \(\txid_{\cl}^{1}\) with \( \txid_{\cl}^{1} \in \kvs' \setminus \kvs \),
must include \( \txid_{\cl}^{1} \).  
When this is the case, clients can always commit their transactions, 
and thus \(\CanCommit[\RYW]\) is simply \(\true\).

%The \(\ViewShift[\RYW]\) predicate thus states that 
%after executing a transaction, the resulting view of the client
%contains all the versions the client wrote in its view. 
%This ensures that such versions will be included in the view of the client 
%when committing future transactions.

\begin{figure}

\begin{tabularx}{\textwidth}{@{}c@{}|@{}X@{}}

\begin{subfigure}{0.394\textwidth}
\centering
\begin{tikzpicture}%
\KVMapping{x}{\key}{
    /\val_{0}/\txid_0/\Set{\txid_\cl^1,\boldsymbol{\txid_\cl^2}}
    , /\val_{1}/\txid_\cl^1/\emptyset
    , /\val_{2}/\boldsymbol{\txid_\cl^2}/\emptyset
};
\end{tikzpicture}%
\caption{Disallowed by \(\RYW\).}
\label{fig:ryw-disallowed}
\end{subfigure}

&

\begin{subfigure}{0.594\textwidth}
\centering
\begin{tikzpicture}
\KVMapping{x}{\key_1}{
    /\val_0/\txid_0/\Set{\boldsymbol{\txid}}
    , /\val_1/\txid_{\cl}^1/\emptyset
};

%Location y
\KVMapping[x]{y}{\key_2}{
    /\val_0/\txid_0/\emptyset
    , /\val_2/\txid_\cl^1/\emptyset
    , /\val_3/\txid_\cl^2/\Set{\boldsymbol{\txid}}
};
\end{tikzpicture}
\caption{Disallowed by \(\MW\)}
\label{fig:mw-disallowed}
\end{subfigure}

\\ \hline

\end{tabularx}

\vspace*{1em}

\begin{subfigure}{\textwidth} 
\centering
\begin{tikzpicture}
\KVMapping{x}{\key_1}{
    /\val_0/\txid_0/\Set{\boldsymbol{\txid}}
    , /\val_1/\txid'/\emptyset
};
\KVMapping[x]{y}{\key_2}{
    /\val_0/\txid_0/\emptyset
    , /\val_2/\txid'/\Set{\txid_\cl^1}
    , /\val_3/\txid_\cl^2/\Set{\boldsymbol{\txid}}
};
\end{tikzpicture}

\caption{Disallowed by \(\WFR\)}
\label{fig:wfr-disallowed}

\end{subfigure}

\hrulefill

\caption{Anomalies for \( \RYW\), \( \MW \) and \( \WFR \)}

\end{figure}


\paragraph{Monotonic write (\(\MW\))} 
\label{sec:et-mw} 
This model states that if a client \(\cl\) sees 
a version of a key \( \key \) that was written by another client \(\cl'\), 
then it must see all versions of \( \key \) that were 
previously written by \( \cl' \).
In other words, the view \( \vi \) of the client \(\cl\) over a kv-store \(\kvs\) 
must be closed with respect to the relation \(\SO \cap \WW[\kvs]\), 
before \(\cl\) can commit a transaction;
this is modelled by \( \PreClosed(\kvs, \vi, \SO \cap \WW[\kvs]) \).
The resulting view can be any view.
Thus \(\ViewShift[\MW]\) is simply \(\true\).
Monotonic write prohibits the kv-store in \cref{fig:mw-disallowed},
since transaction \( \txid \) reads the third version of \( \key_2 \) written by \( \txid[\cl](2) \),
but not version written by \( \txid[\cl](1) \) in key \( \key_1 \), 
where \( (\txid[\cl](1), \txid[\cl](2)) \in \SO \cap \WW[\kvs](\key_2) \).

\paragraph{Write follow read (\(\WFR \))} 
\label{sec:et-wfr} 
This model states that, prior to committing a transaction, 
if a client \(\cl\) sees a version on a key \(\key\) written by some client \(\cl'\) (possibly equal to \(\cl\)), 
then it must also see the versions of the same key \(\key\) previously read by \(\cl'\). 
This condition is modelled by the predicate \( \PreClosed(\kvs,\vi,\rel[\WFR]) \)
with \( \rel[\WFR] = \WR[\kvs] ; \Refl((\SO \cap \RW[\kvs]))\).
Note that if \( \ToEdge{\txid | \WR[\kvs] -> \txid' | \Refl((\SO \cap \RW[\kvs])) -> \txid'' } \),
then:
\begin{enumerate*}
\item transactions \( \txid',\txid'' \) are from the same client;
\item \( \txid' \) reads a version of some \( \key \) written by \( \txid \); and 
\item later \( \txid'' \) writes a newer version of the same key \( \key \).
\end{enumerate*}
The kv-store in \cref{fig:wfr-disallowed} is disallowed by \(\WFR\),
since transaction \(\txid\) reads the third version of key \( \key_2 \) written by \(\cl\),
who previous read the second version of key \(\key_2 \) written by \( \txid' \).
However, transaction \(\txid\) did not read the second version of \( \key_1 \) also written by \( \txid' \).

