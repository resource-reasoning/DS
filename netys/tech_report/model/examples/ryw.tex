\emph{Read your write} (\(\RYW\)) \citep{session-guarantee,repldatatypes} states that 
a client must always see all the versions written by the client itself. 
Under \(\RYW\) the kv-store in \cref{fig:ryw-disallowed} 
is prohibited as the initial version of \(\key\) holds value \(\val_0\) 
and client \(\cl\) tries to \emph{update} the value of \(\key\) twice.  
For its first transaction \( \txid_{\cl}^1\), 
it reads the initial value \(\val_0\) and then writes a new version with value \(\val_1\). 
For its second transaction \( \txid_{\cl}^2\), 
it reads the initial value \(\val_0\) again and write a new version with value \(\val_2\).
The \(\ViewShift[\RYW]\) predicate defined by:
\[
\ViewShift[\RYW](\kvs,\vi,\kvs',\vi') \PredDef
	\begin{multlined}[t]
		\Forall{\txid \in \kvs' \setminus \kvs | \key \in \kvs' | \idx} 
        \\ \ToEdge{\WtOf(\kvs'(\key, i)) | \Refl(\SO)  -> \txid }  
        \implies \idx \in \vi'(\key) ,
	\end{multlined}
\]
rules out this kv-store in \cref{fig:ryw-disallowed}
by requiring that the client view \( \vi' \), 
after the commits the transaction \(\txid_{\cl}^{1}\) with \( \txid_{\cl}^{1} \in \kvs' \setminus \kvs \),
must include \( \txid_{\cl}^{1} \).  
When this is the case, clients can always commit their transactions, 
and thus \(\CanCommit[\RYW]\) is simply \(\true\).

%The \(\ViewShift[\RYW]\) predicate thus states that 
%after executing a transaction, the resulting view of the client
%contains all the versions the client wrote in its view. 
%This ensures that such versions will be included in the view of the client 
%when committing future transactions.

\begin{figure}

\begin{tabularx}{\textwidth}{@{}c@{}|@{}X@{}}

\begin{subfigure}{0.394\textwidth}
\centering
\begin{tikzpicture}%
\KVMapping{x}{\key}{
    /\val_{0}/\txid_0/\Set{\txid_\cl^1,\boldsymbol{\txid_\cl^2}}
    , /\val_{1}/\txid_\cl^1/\emptyset
    , /\val_{2}/\boldsymbol{\txid_\cl^2}/\emptyset
};
\end{tikzpicture}%
\caption{Disallowed by \(\RYW\).}
\label{fig:ryw-disallowed}
\end{subfigure}

&

\begin{subfigure}{0.594\textwidth}
\centering
\begin{tikzpicture}
\KVMapping{x}{\key_1}{
    /\val_0/\txid_0/\Set{\boldsymbol{\txid}}
    , /\val_1/\txid_{\cl}^1/\emptyset
};

%Location y
\KVMapping[x]{y}{\key_2}{
    /\val_0/\txid_0/\emptyset
    , /\val_2/\txid_\cl^1/\emptyset
    , /\val_3/\txid_\cl^2/\Set{\boldsymbol{\txid}}
};
\end{tikzpicture}
\caption{Disallowed by \(\MW\)}
\label{fig:mw-disallowed}
\end{subfigure}

\\ \hline

\end{tabularx}

\vspace*{1em}

\begin{subfigure}{\textwidth} 
\centering
\begin{tikzpicture}
\KVMapping{x}{\key_1}{
    /\val_0/\txid_0/\Set{\boldsymbol{\txid}}
    , /\val_1/\txid'/\emptyset
};
\KVMapping[x]{y}{\key_2}{
    /\val_0/\txid_0/\emptyset
    , /\val_2/\txid'/\Set{\txid_\cl^1}
    , /\val_3/\txid_\cl^2/\Set{\boldsymbol{\txid}}
};
\end{tikzpicture}

\caption{Disallowed by \(\WFR\)}
\label{fig:wfr-disallowed}

\end{subfigure}

\hrulefill

\caption{Anomalies for \( \RYW\), \( \MW \) and \( \WFR \)}

\end{figure}

