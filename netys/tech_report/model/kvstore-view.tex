We formally define the abstract states of our semantics in this section,
which consists of a global, centralised key-value store (kv-store) and several independent partial client views.
A kv-store comprises key-indexed lists of versions which record
the history of the key with values and meta-data of the
transactions that have accessed the version including the writer and readers.

\begin{definition}[Client and transactional identifiers]
    \label{def:client-id}
    \label{def:transaction-id}
    The set of \emph{client identifiers}, \( \Clients \ni \cl \), is a countably infinite set.
    The set of \emph{transaction identifiers}, \(\TxIDs \ni \txid\), is defined by:
    \( \TxIDs \TypeDef \Set{\txid_{0}} 
    \uplus \Set{ \txid_{\cl}^{n} | \cl \in \Clients \land n \in \Nat \land n \geq 0 } \).
    Let \(\TxIDZs \TypeDef \TxIDs \setminus \Set{\txidinit}\).
\end{definition}

Subsets of \(\TxIDs\)  are ranged over by \(\txidset, \txidset', \txidset_1, \cdots\). 
The transaction identifier \(\txidinit\) denotes the \emph{initialisation transaction}
and \(\txid_{\cl}^{n}\) identifies a transaction committed by client
\(\cl\) with \(n\) being used to determine the client session order.

\begin{definition}[Session order]
\label{def:session-order}
The session order, \( \SO \), is defined by:
\[
    \SO \TypeDef \Set{ (\txid[\cl](n), \txid[\cl](n')) | \cl \in \Clients 
        \land \txid[\cl](n), \txid[\cl](n') \in \TxIDs \land n < n'}. 
\]
\end{definition}

Each version has the form \(\ver = (\val, \txid, \txidset)$,
where \(\val\) is a value, 
the \emph{writer} \(\txid\) identifies the transaction that wrote \(\val\), 
and the \emph{reader set} \(\txidset\) identifies the transactions that read \(\val\). 

\begin{definition}[Keys, values and versions]
    \label{def:keys}
    \label{def:values}
    \label{def:versions}
    Assume a countably infinite set of \emph{keys}, \(\Keys \ni \key\), 
    and a countably infinite set of \emph{values}, \(\Values \ni \val\),
    such that 
    \( \begin{Bracketed} \Keys \cup \valinit \end{Bracketed} \subseteq \Values, \)
    where \(\valinit\) is the \emph{initialisation value}.
    The set of \emph{versions}, \(\Versions \ni \ver\), is defined by:
    \(\Versions \TypeDef \Values \times \TxIDs \times \PowerSet(\TxIDZs)\). 
    Let \(\ValOf(\ver)\), \(\WtOf(\ver)\) and \(\RsOf(\ver)\) 
    return the first, second and third components of \(\ver\).
    Given a transaction identifier \( \txid \in \TxIDs \), 
    the notation \( \txid \in \ver \) is defined by: \( \txid \in \Set{\WtOf(\ver)} \cup \RsOf(\ver)\).
\end{definition}

Our global centralised key-value stores (kv-stores) keep track of all the history versions for each key.
These stores model the overall global state of a system.
They provide an abstraction to real world distributed systems where versions are stored in distributed sites.
For example, the COPS replicated database  can be abstracted to centralised kv-stores.

\begin{definition}[Kv-stores]
\label{def:kv-store}
A \emph{kv-store} is a function \(\kvs \in \Keys \ToTFunc \List{\Versions}\), 
where \(\List{\Versions} \ni \verlist\) denotes the set of lists of versions.
Given an index \( \idx \in \Indexs \), 
let \(\kvs(\key, i)\) denote \Th{i} element of the list of versions of \(\key\), defined by:
\( \kvs(\key,\idx) \FuncDef \kvs(\key)\Proj{\idx}.  \)
Given a transaction \(\txid\) and a kv-store \( \kvs \),
the transaction is included in \( \kvs \), written \(\txid \in \kvs\),
if and only if \(\txid\) is either the writer 
or one of the readers of a version in \(\kvs\):
\( 
    \txid \in \kvs \PredDef 
    \Exists{\key,\idx} \txid \in \kvs(\key,\idx).
\)
The \emph{initial kv-store}, \(\kvsinit\), is defined by:
\(\kvsinit(\key) \TypeDef \List{(\valinit, \txidinit, \emptyset)}\) for
all \(\key \in \Keys\). 
\end{definition}



\begin{definition}[Well-formed kv-store]
\label{def:well-formed-kv-store}
A kv-store \( \kvs \) is well-formed, written \( \WfKvs(\kvs)\),
if and only if, for any \( \key \in \Keys\) and \( \idx,\idx' \in \Indexs \),
\begin{Formulae}
    & \begin{Formula}
        \RsOf(\kvs(\key,\idx)) \cap \RsOf(\kvs(\key,\idx')) \neq \emptyset
        \\ {} \lor \WtOf({\kvs(\key,\idx)}) = \WtOf({\kvs(\key,\idx')}) 
        \implies \idx = \idx'
        \label{equ:kvs-wf-txid-snapshot-property} 
    \end{Formula}  
    \\ & \begin{Formula}
        \ValOf(\kvs(\key,0)) = \valinit
        \label{equ:kvs-wf-init-version}
    \end{Formula} 
    \\ & \begin{Formula}
        \Forall{\txid,\txid' \in \TxIDs}
        \txid = \WtOf(\kvs(\key,\idx))
        \land \txid' \in \RsOf(\kvs(\key,\idx))
        \implies (\txid', \txid) \notin \Refl(\SO)
        \label{equ:kvs-wf-so-wr}
    \end{Formula} 
    \\ & \begin{Formula}
        \Forall{\txid,\txid' \in \TxIDs}
        \txid = \WtOf(\kvs(\key,\idx))
        \land \txid' = \WtOf(\kvs(\key,\idx'))
        \land \idx < \idx'
        \implies (\txid', \txid) \notin \Refl(\SO)
        \label{equ:kvs-wf-so-ww}
    \end{Formula} 
\end{Formulae}
where \( \Refl(\rel) \) denotes the reflexive closure of \( \rel \).
Let \(\KVSs \ni \kvs\) denotes the set of \emph{well-formed kv-stores}.
\end{definition}

We focus on kv-stores whose consistency model satisfies the \emph{snapshot property}, 
ensuring that a transaction reads and writes at most one version for each key (\cref{equ:kvs-wf-txid-snapshot-property}).
Snapshot property is a common assumption for distributed databases.
We also assume that the version list for each key \(\key\) has the initial version (the left-most version)
carrying the initialisation value \(\valinit\), written by the initialisation transaction \txidinit (\cref{equ:kvs-wf-init-version}).
Finally, we assume that the kv-store agrees with the session order of clients: 
\begin{enumerate*}
\item a client cannot read a version of a key that 
has been written by a future transaction within the same session (\cref{equ:kvs-wf-so-wr}); and 
\item the order in which versions are written by a client must agree with its session order (\cref{equ:kvs-wf-so-ww}).
\end{enumerate*}


A global kv-store provides an abstract centralised description of
updates associated with distributed kv-stores that is \emph{complete} in 
that no update has been lost in the description.
By contrast, in both replicated and partitioned distributed databases, 
a client may have incomplete information about updates distributed between machines. 
For example, a COPS client \( \cl \) 
only tracks versions that \( \cl \) has accessed in its client context.
We model this \emph{incomplete} information using
a \emph{view} of the kv-store which provides 
a \emph{partial} record of the updates observed by a client. 

\begin{definition}[Views]
\label{def:views}
Given a well-formed kv-store \(\kvs \in \KVSs\), 
the set of \emph{views on a kv-store}, \( \Views(\kvs) \ni \vi\), is defined by:
\[
    \Views(\kvs) \TypeDef \Set{ \vi \in \Keys \ToTFunc \PowerSet(\Nat) | \WfView(\kvs,\vi) },
\]
where \( \WfView(\kvs,\vi) \) is defined by:
for any \( \key, \key' \in \Keys \) and \( \idx, \idx' \in \Indexs\),
\begin{Formulae}
    & \begin{Formula}
    0 \in \vi(\key),
    \label{equ:view-wf-initial}
    \end{Formula} 
    \\ & \begin{Formula}
    i \in \vi(\key) \implies 0 \leq i < \Abs{ \kvs(\key) },
    \label{equ:view-wf-with-in-range}
    \end{Formula} 
    \\ & \begin{Formula}
	i \in \vi(\key)  
  	\land \WtOf(\kvs(\key, i)) = \WtOf(\kvs(\key', i'))  
  	\implies i' \in \vi(\key').
	\label{equ:view-wf-atomic}
    \end{Formula}
\end{Formulae}
Given two views \(\vi, \vi' \in \Views(\kvs)\), 
the order between them is defined by: \(\vi \vileq \vi' \PredDef \Forall{\key \in \Dom(\kvs)} \vi(k) \subseteq \vi'(\key)\).
The set of \emph{views} is \(\Views \TypeDef \bigcup_{\kvs \in \KVSs} \Views(\kvs)\).
The \emph{initial view}, \(\viinit\), is defined by:
\(\viinit = \lambda \key \in \Keys \ldotp \Set{0}\). 
\end{definition}

A well-formed view \( \vi \) on a kv-store \( \kvs \)
must contain the initial version of each key (\cref{equ:view-wf-initial})
and the indexes  in \( \vi \) must be in range (\cref{equ:view-wf-with-in-range}).
We require that a client view be \emph{atomic} in that it can
see either all or none of the updates of a transaction (\cref{equ:view-wf-atomic}).

A view provides an abstraction to client local information.
In COPS protocol, each client maintains a context that contains all the version it has accessed.
This information in the context can be captured by our view.

\begin{definition}[Configurations]
\label{def:configurations}
A \emph{configuration}, \(\conf \in \Confs\),  is a pair \((\kvs, \vienv)\)
with a kv-store \(\kvs \in \KVSs\),
and a \emph{view environment} on the kv-store \( \kvs \),
defined by: \(\vienv \in \ViewEnvs(\kvs) \TypeDef \Clients \ToPFFunc \Views(\kvs)\). 
Let \( \ViewEnvs \TypeDef \bigcup_{\kvs \in \KVSs} \ViewEnvs(\kvs)\).
The set of \emph{initial configurations}, \(\ConfInits \subseteq \Confs\),
contains configurations of the form \( (\kvsinit, \vienvinit)\),
where \( \kvsinit \) is the initial kv-store 
and \( \vienvinit \) is an initial view environment 
defined by: for any client \( \cl \in \Dom(\vienv)\), \( \vienv(\cl) = \viinit \).
\end{definition}

An \emph{configuration} is a pair comprising a kv-store and a function describing the
views of a finite set of clients. 
Given a configuration \((\kvs, \vienv)\) and a client \(\cl\), 
if \(\vi = \vienv(\cl)\) is defined then, for each \(\key\), 
the configuration determines the sublist of versions in \(\kvs\) that \(\cl\) observes.
If \(\idx, \idx' \in \vi(\key)\) and \(\idx < \idx'\), then \(\cl\) observes the values 
carried by versions \(\kvs(\key, \idx)\) and  \(\kvs(\key, \idx')\), 
and it also sees that the version \(\kvs(\key, \idx')\) is more up-to-date than \(\kvs(\key, \idx)\). 
It is therefore possible to associate a \emph{snapshot} with the view \(\vi\), 
which identifies, for each key \(\key\), the last version included in the view. 
This definition assumes that the database satisfies the \emph{last-write-wins}
resolution policy, employed by many distributed key-value stores.
However, our formalism can be adapted to capture other resolution policies. 

\begin{definition}[View snapshots]
\label{def:view-snapshot}
Given \(\kvs \in \KVSs\) and \(\vi \in \Views(\kvs)\),
the \emph{snapshot} of \(\vi\) on \(\kvs\) is a function, 
\(\Snapshot(\kvs, \vi) : \Keys \to \Values\),
defined by:
\[
    \Snapshot(\kvs, \vi) \FuncDef \lambda \key \ldotp \ValOf(\kvs(\key, \Max[<](\vi(\key)))),
\]
where \(\Max[<](\vi(\key))\) is the maximum element in \(\vi(\key)\) with respect to the natural 
order \(<\) over \Nat.
\end{definition}

In the COPS protocol, a multi-read transaction constructs a snapshot step by step.
In our semantics, the construction steps are captured by \( \Snapshot \).
