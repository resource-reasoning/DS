A client program might still have the desirable behaviour even if it is not robust.
We now show a lock paradigm is correct under \( \UA \).
The distributed lock library providing \(\plock(k)\), \(\ptrylock(k)\) and \(\punlock(k)\) 
operations on the key \( \key \), is defined by:
\begin{align*}
    \ptrylock(k) & \FuncDef 
    \begin{Transaction}
    \plookup{\var}{k}; \ \pif(\var = 0) \ 
    \{ \pmutate{k}{ \var(ClientID) }; \  \passign{\var(m)}{\true} \  \} 
    \CodeFont{else} \{ \  \passign{\var(m)}{\false} \  \} 
    \end{Transaction}     
    \\ \plock(k) & \FuncDef 
     \CodeFont{do} \{\  \ptrylock(\key) \  \} \CodeFont{until}(\var(m) =  \true) 
    \\ \punlock(k) & \FuncDef 
    \begin{Transaction}
        \pmutate{k}{ 0 };
        \ \passign{m}{\false};
    \end{Transaction}     
\end{align*}
The \( \ptrylock \) operation reads the value of the key \( \key \).
If the value is zero (the lock is available), 
the operation sets it to the client ID, \( \CodeFont{ClientID} \), 
and returns \( \true \) via the variable \( \var(m) \).
Otherwise it leaves it unchanged and returns \( \false \).
The \( \plock \) operation calls \( \ptrylock \) until it successfully acquires the lock.
The \( \punlock \) operation simply sets the key \( \key \) to zero.

Consider the program \( \prog_{\CodeFont{LK}} \) where clients \( \cl\) and \( \cl'\) compete the lock \( \key \):
\begin{align*}
    \prog_{\CodeFont{LK}} & \FuncDef ( \cl: (\plock(\key); \ \CodeFont{...} ; \ \punlock(\key))^*
    \ \| \ 
    \cl': (\plock(\key); \ \CodeFont{...} ; \ \punlock(\key))^* )
\end{align*}
The locking paradigm in \( \prog_{\CodeFont{LK}} \) is correct, in that
only one client can hold the lock at a time,
when executed under serialisability.
Since all the operations are trivially \( \WSI \)-safe,
\( \prog_{\CodeFont{LK}} \) is robust and hence correct under \( \WSI \) 
and any stronger model such as \( \SI \).
However, \( \prog_{\CodeFont{LK}} \) is not robust under \( \UA \), 
because the \( \plock \) operation might read any old value of key \( \key \)
until it reads the up-to-date value of \( \key \) and acquires the lock.
Nevertheless, we show that \( \prog_{\CodeFont{LK}} \) is still correct under \( \UA \).
We capture the correctness by the following invariant:
for all \( i \), if  \( i > 0\), then:

\begin{Formulae}
& \begin{Formula}
\ValOf(\kvs(\key,i)) \neq 0 \iff \ValOf(\kvs(\key,i-1)) = 0
\label{equ:lock-unique-hold}
\end{Formula}
\\ & \begin{Formula}
\ValOf(\kvs(\key,i)) = \CodeFont{ClientID}_\cl \implies \Exists{n} \WtOf(\kvs(\key,i)) = \WtOf(\kvs(\key,i-1)) = \txid[\cl](n).
\label{equ:lock-release}
\end{Formula}
\end{Formulae}

It is straightforward to show that under \( \UA \), 
only one client can hold the key \cref{equ:lock-unique-hold},
and the same client releases the lock \cref{equ:lock-release}.

\begin{theorem}[Correctness of a lock paradigm under \texorpdfstring{\( \UA \)}{\texttt{UA}}]
A lock paradigm, \( \prog_{\CodeFont{LockPara}}\), is define by:
for a set of clients \( \clset \), a key \( \key \) and a critical section \( \cmd \),
\begin{align*}
\prog_{\CodeFont{LockPara}} & \FuncDef  \lambda \cl \in \clset \ldotp
\left( \plock(\key); \ \cmd ; \ \punlock(\key) \right)^*
\end{align*}
where \( \key \notin \Func{fv}(\cmd) \).
All reachable kv-stores \( \kvs \in \EvalET{\prog_{\CodeFont{LockPara}}}[\UA] \)
satisfy \cref{equ:lock-unique-hold,equ:lock-release}.
\end{theorem}
\begin{proof}
By the definition of  \( \EvalET{\prog_{\CodeFont{LockPara}}}[\UA] \) defined in \cref{def:kv-store-prog-traces},
if \( \kvs \in \EvalET{\prog_{\CodeFont{LockPara}}}[\UA] \),
then there exists a trace \( \progtrc \) with the final state begin \( (\kvs,\vienv,\clenv,\prog) \):
\[
    \progtrc \in \ProgTraces(\prog_\CodeFont{LockPara},\UA) \land (\kvs,\vienv,\clenv,\prog) = \Last(\progtrc) .
\]
Let \( \FirstTrans(\cmd) \) returns the first transaction of the command \( \cmd \).
We prove that that \( \kvs \) satisfies \cref{equ:lock-unique-hold,equ:lock-release} and the following result:
\begin{Formulae}
\begin{Formula}
    \left( \begin{array}{@{}l@{}} 
    \ValOf(\kvs(\key,\Abs{\kvs(\key)} - 1)) = 0 
        \implies \Forall{\cl \in \Dom(\prog)} 
        \\ \FirstTrans(\prog(\cl)) = \ptrylock(\key) \land \clenv(\cl)(\var(m)) = \false
    \end{array} \right) 
    \\ {} \land \left( \begin{array}{@{}l@{}} 
                \Exists{\cl,n} \kvs(\key,\Abs{\kvs(\key)} - 1) = (\CodeFont{ClientID}_{\cl},\txid[\cl](n), \stub)
                    \implies 
                    \Forall{\cl' \in \Dom(\prog)}
                    \\ \quad ( \cl = \cl' \land \FirstTrans(\prog(\cl)) = \punlock(\key)  \land \clenv(\cl)(\var(m)) = \true )
                    \\{} \quad \lor ( \cl \neq \cl' \land \FirstTrans(\prog(\cl)) = \ptrylock(\key) \land \land \clenv(\cl)(\var(m)) = \false)
    \end{array} \right) 
    \label{equ:lock-program-invariant}
\end{Formula}
\end{Formulae}
by induction on the trace \( \progtrc \).
\begin{enumerate}
\CaseBase{\( \progtrc = \ToProg{\kvsinit | \vienvinit | \clenv_0 | \prog }\)}
    it is trivial that the configuration satisfies \cref{equ:lock-unique-hold,equ:lock-release,equ:lock-program-invariant}.
\CaseInd{\( \progtrc = \ToProg{\progtrc' | \lb -> \kvs'' | \vi'' | \clenv'' | \prog'' }\)}
    Let \( (\kvs',\vienv',\clenv',\prog') = \Last(\progtrc') \).
    By \ih, \( \kvs',\clenv',\prog' \) satisfies \cref{equ:lock-unique-hold,equ:lock-release,equ:lock-program-invariant}.
    If \( \lb \) is \( \lbPri \), we have \( \kvs'' = \kvs' \) and \( \vienv''  = \vienv' \).
    Therefore, the final configuration trivially satisfies \cref{equ:lock-unique-hold,equ:lock-release,equ:lock-program-invariant}.
    Consider that \( \lb \) is a transaction step.
    Assume \( \lb = \lbTrans[\cl]{\vi,\fp} \),
    for a scheduled client \( \cl \) with view \( \vi \) and fingerprint \( \fp \).
    Given the value of the last version of \( \key \), we consider the following two possible cases.
    \begin{enumerate}
    \Case{\( \kvs'(\key,\Abs{\kvs'(\key)} - 1) = (0,\txid[\cl'](n'),\txidset') \)}
        By \cref{equ:lock-program-invariant}, we have 
        \[
            \Forall{\cl'' \in \Dom(\prog')} \FirstTrans(\prog'(\cl'')) = \ptrylock(\key)  
        \]
        and the next transaction for \( \cl \) is \(\ptrylock(\key) \) transaction.
        \begin{enumerate}
        \Case{\( \vi(\key) = \Set{\idx | 0 \leq \idx < \Abs{\kvs'(\key)} - 1}\)}
            The value for \( \key \) in the snapshot \( \snap = \Snapshot(\kvs',\vi) \) must be zero,
            that is, \( \snap(\key) = 0 \).
            By the definition of \( \ptrylock \),
            the final fingerprint will be \( \fp' =  \Set{\opR(\key,0), \opW(\key,\CodeFont{ClientID}_{\cl}) }\)
            where \( \CodeFont{ClientID}_{\cl} \) is the unique client ID  of \( \cl \).
            By the execution test for \( \UA \), this fingerprint \( \fp \) is allowed to commit 
            with the given view \( \vi' \).
            Let \( \kvs'' = \UpdateKV(\kvs',\vi,\fp,\txid) \) 
            for a fresh \( \txid \in \NextTxid(\kvs',\cl) \).
            By the definition of \( \ptrylock \),
            we also have \( \clenv''(\cl)(\var(m)) = \true\).
            It is easy to see that the final configuration satisfies 
            \cref{equ:lock-unique-hold,equ:lock-release,equ:lock-program-invariant}.
        \Case{\( \vi(\key) \subset \Set{\idx | 0 \leq \idx < \Abs{\kvs'(\key)} - 1}\)}
            Note that the view \( \vi \) cannot be a view that leads to a snapshot
            in which the value of key \( key \) is zero.
            Because in this case,
            the fingerprint will be  \( \fp =  \Set{\opR(\key,0), \opW(\key,\CodeFont{ClientID}) }\).
            However it is disallowed to commit under \( \UA \).
            Thus, \( \vi \) must be a view such that \( \Snapshot(\kvs',\vi)(\key) \neq 0 \).
            The final fingerprint will be \( \Set{\opR(\key,\Snapshot(\kvs',\vi)(\key))} \).
            Because it is a read-only transaction,
            it is trivial that this fingerprint is allowed to commit under \( \UA \) and 
            the new configuration satisfies 
            \cref{equ:lock-unique-hold,equ:lock-release,equ:lock-program-invariant}.
        \end{enumerate}
    \Case{\( \kvs'(\key,\Abs{\kvs'(\key)} - 1) = (\CodeFont{ClientID}_{\cl'},\txid[\cl'](n'),\txidset') \)}
        By \cref{equ:lock-program-invariant}, we have two possible cases, depending on if \( \cl = \cl' \).
        \begin{enumerate}
        \Case{\( \cl = \cl' \)}
            The next transaction for \( \cl \)  is \( \punlock \).
            The final fingerprint \( \fp \) must be \( \fp = \Set{\opW(\key,0)} \).
            In this case, the view must be \( \vi(\key) = \Set{\idx | 0 \leq \idx < \Abs{\kvs'(\key)} - 1}\),
            otherwise it is disallowed to commit.
            By the definition of \( \punlock \)
            the new configuration satisfies 
            \cref{equ:lock-unique-hold,equ:lock-release,equ:lock-program-invariant}.
        \Case{\( \cl \neq \cl' \)}
            The next transaction for \( \cl \) is \(\ptrylock(\key) \) transaction,
            and the view \( \vi \) cannot be a view such that \( \Snapshot(\kvs',\vi)(\key) \neq 0 \).
            Because it is a read-only transaction,
            it is trivial that this fingerprint is allowed to commit under \( \UA \) and 
            the new configuration satisfies 
            \cref{equ:lock-unique-hold,equ:lock-release,equ:lock-program-invariant}. \qedhere
        \end{enumerate}
    \end{enumerate}
\end{enumerate}
\end{proof}
