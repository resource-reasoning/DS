By the result of \cref{thm:well-formed-clock-trace},
kv-store program traces induced by annotated normalised Clock-SI traces are valid traces:
that is, there traces can be obtained under \( \et[\top] \).
Now we prove that each transition in these kv-store traces satisfies \( \et[\SI]\).

\begin{toappendix}
\label{sec:proof-clock-si-sat-si}
\end{toappendix}
\begin{theoremrep}[Clock-SI traces satisfying snapshot isolation]
Given an annotated normalise Clock-SI trace \( \clockexttrc \),
the kv-store trace induced by the Clock-SI trace \( \progtrc = \ClockToKVTrace(\clockexttrc)\)
is a trace that can be obtained with the execution test for \( \SI \).
\end{theoremrep}
\begin{proof}
Given a normalised Clock-SI trace \( \clockexttrc \),
by \cref{thm:well-formed-clock-trace}, 
the induced kv-store trace \( \progtrc = \ClockToKVTrace(\clockexttrc) \) 
is a trace that can be obtained under \( \et[\top]\).
We prove that every transition in the trace \( \progtrc \) satisfies the constrains of \( \et[\PSI] \) by induction on \( \clockexttrc \).
\begin{enumerate}
\CaseBase{\clockexttrc = \ToClockProg{\clocksiinit | \clockclientenvinit | \clenv_0 | \prog_0 }}
    By the definition of \( \ClockToKVTrace \),
    \( \progtrc = \ToProg{ \kvsinit | \vienvinit | \clenv_0 | \prog_0 } \),
    and there is nothing to prove since no transition has been made.
\CaseInd{ \clockexttrc = \ToClockProg{\clockexttrc' | \lb -> \clockexttrc'' | \lb' 
        -> \clocksi'' | \clockclientenv'' | \clenv'' | \clockrunprog'' } }
    If \( \lb  = \lbCLOCKTick{\clocktime} \) or \(  \lb = \lbCLOCKStart{\clocktime} \),
    by the definition of \( \ClockToKVTrace \),
    we have \( \ClockToKVTrace(\clockexttrc) = \ClockToKVTrace(\clockexttrc') \),
    and thus by \ih, we have the proof.
    Now consider 
    \begin{Formulae}*
    \begin{Formula}
    \clockexttrc = \ToClockProg{\clockexttrc' | \lb -> \clockexttrc'' | \lb'
                -> \clocksi'' | \clockclientenv'' | \clenv'' | \clockrunprog'' }
    \\ {} 
    \land \lb = \lbCLOCKOp{(l,\key,\val),m,n} \land l \in \Set{\opR,\opW}
    \land \lb' = \lbCLOCKEnd{n} 
    \end{Formula}
    \end{Formulae}
    Let \( \clockexttrc^* = 
            \ToClockProg{\LastConf(\clockexttrc) | \lb 
                        -> \clockexttrc' | \lb' 
                        -> \clocksi'' | \clockclientenv'' | \clenv'' | \clockrunprog'' } \).
    %In this case, we have \( \CLOCKAtomic(\clockexttrc^*) \).
    Let \( \progtrc' = \ClockToKVTrace(\clockexttrc') \)
    and \( (\kvs',\vienv', \allowbreak \stub,\stub)  = \LastConf(\progtrc') \).
    By the definition of \( \ClockToKVTrace \), we have
    \[
    \ClockToKVTrace(\clockexttrc) = 
        \ToProg[\top]{\progtrc' | \lbTrans{\vi,\fp} 
                -> \kvs'' | \vienv'\UpdateFunc{\cl -> \vi'} | \clenv'' | \prog''  } ,
    \]
    where
    \begin{Formulae}*
    \begin{Formula}
        \fp = \ClockFp(\emptyset, \clockexttrc^*)
        \land \vi = \ClockView(\kvs',\clocktime)  
        \\ {} \land \kvs'' =  \UpdateKV(\kvs',\vi,\fp,\clocktxid[\cl](n,m) )
        \land \vi' = \ClockView(\kvs'',n) 
        \land \prog' = \ClockStaticProg(\clockrunprog) .
    \end{Formula}
    \end{Formulae}
    Note that in the proof we call \( \clocktime \) as the snapshot time and 
    \( n \) as the commit time.
    By the definition of \( \PSI \),
    we need to prove the following results:
    \begin{Formulae}
    & \begin{Formula}
    \vi \vileq \vi'
    \label{equ:clock-vi-increase}
    \end{Formula}
    \\ & \begin{Formula}
    \Forall{ \txid \in \kvs'' \setminus \kvs' | \key \in \Keys | \idx \in \Indexs } 
    \ToEdge{ \WtOf(\kvs''(\key,\idx)) | \Refl(\SO) -> \txid }
    \implies \idx \in \vi'(\key)
    \label{equ:clock-vi-write}
    \end{Formula}
    \\ & \begin{Formula}
    \PreClosed(\kvs',\vi,\TraRe(\left({\Trasi(( \WR[\kvs'] \cup \SO \cup \WW[\kvs']))} ; \Refl(\RW[\kvs']) \right)))
    \label{equ:clock-vi-cloce-cycle}
    \end{Formula}
    \\ & \begin{Formula}
    \PreClosed(\kvs,\vi,\bigcup_{\opW(\key,\stub) \in \fp}\WWInv[\kvs'](\key))
    \label{equ:clock-vi-ua}
    \end{Formula}
    \end{Formulae}
    We prove \cref{equ:clock-vi-increase,equ:clock-vi-write,equ:clock-vi-cloce-cycle,equ:clock-vi-ua} separately.
    \begin{enumerate}
    \Case{{\cref{equ:clock-vi-increase}}}
        By \cref{lem:clock-si-local-time-mono}, we have the commit time \( n \) is greater or equal to the snapshot time \( m \).
        This means that \( m \leq n \) and thus 
                \( \vi = \ClockView(\kvs',m) \vileq \ClockView(\kvs'',n) = \vi' \).
    \Case{{\cref{equ:clock-vi-write}}}
        Consider any write operation \( \opW(\key,\val) \in \fp \).
        By \( \rCLOCKPrepare \), \( \rCLOCKCommit\) and the definition of annotated Clock-SI traces, \(\CLOCKExtTraces\), 
        there must exist a write step in the Clock-SI trace segment \( \clockexttrc' \)
        of the form \( \lbCLOCKOp{\opW(\key, \allowbreak \val),m,n}\) such that \( m \leq n \).
        Recall that \( \kvs'' =  \UpdateKV(\kvs',\vi,\fp,\txid[\cl](n,m) ) \),
        thus we know that 
        \[
            \kvs''(\key,\Abs{\kvs'(\key)} - 1) = (\val,\txid[\cl](n,m),\emptyset) .
        \]
        By the definition of \( \vi' = \ClockView(\kvs'',n) \),
        we have \( \left( \Abs{\kvs''(\key)} - 1 \right) \in \vi'(\key) \).
    \Case{\cref{equ:clock-vi-cloce-cycle}}
        We prove the correspondence between the four relations, \( \WR, \allowbreak \WW, \allowbreak \RW \) and \( \SO\),
        with respect to the snapshot and commit times.
        Let consider two transactions \( \txid[\cl'](n',m'),\txid[\cl''](n'',m'') \) that accessed key \(key\) in \( \kvs' \).
        Note that \( \txid[\cl'](n',m'),\txid[\cl''](n'',m'') \) might be the writers or the readers of some versions.
        By \cref{thm:well-formed-clock-trace},
        there must exist read or write steps 
        that corresponds to \( \txid[\cl'](n',m'),\txid[\cl''](n'',m'') \) in the trace \( \clockexttrc' \).  
        Without losing generality, we assume the steps for \( \txid[\cl'](n',m') \) 
        appear before the steps for \( \txid[\cl''](n'',m'') \):
        \[
            \clockexttrc' = \ToClockProg{\cdots | \lbCLOCKOp{(l',\key,\val'),m',n'} 
                        -> \cdots | \lbCLOCKOp{(l'',\key,\val''),m'',n''} -> \cdots }
            \land l',l'' \in \Set{\opW,\opR}
        \]
        Given that the snapshot time \( m\) of the new transaction \( \txid[\cl](n,m) \),
        assume that \( n'' < m \).
        This means that \( \txid[\cl](n,m) \) observes \( \txid[\cl''](n'',m'') \).
        Now we prove that \( \txid[\cl](n,m) \) also observes \( \txid[\cl'](n',m') \)
        for the following cases.
        \begin{enumerate}
        \Case{\( (\txid[\cl'](n',m'),\txid[\cl''](n'',m'')) \in \WR[\kvs'](\key) \)}
            This means \( \cl'' \) read a version written by \( \cl' \):
            there exists an index \( \idx \) such that
            \[
                \txid[\cl'](n',m')  = \WtOf(\kvs'(\key,\idx))
                \land \txid[\cl''](n'',m'')  \in \RsOf(\kvs'(\key,\idx)) .
            \]
            We then have \( l' = \opW \) and \( \l'' = \opR \).
            By \cref{lem:clock-si-wr}
            \( \cl'' \) must have a snapshot time \( m'' \) greater or equal to the commit time \( n' \),
            and then by \cref{lem:clock-si-local-time-mono}, we have \( n' < n'' \).
            In other words,
            \begin{Formulae}
            \begin{Formula}
                (\txid[\cl'](n',m'),\txid[\cl''](n'',m'')) \in \WR[\kvs'] \implies n' < m'' \leq n'' .
                \label{equ:clock-wr-commit-time}
            \end{Formula}
            \end{Formulae}
        \Case{\( (\txid[\cl'](n',m'),\txid[\cl''](n'',m'')) \in \WW[\kvs'](\key) \)}
            This means \( \cl'' \) write a new version for \( \key \) after the version written by \( \cl' \):
            \[
                \txid[\cl'](n',m')  = \WtOf(\kvs'(\key,\idx))
                \land \txid[\cl''](n'',m'')  \in \WtOf(\kvs'(\key,\idx')) 
                \idx < \idx'
            \]
            The transaction \( \txid[\cl''](n'',m'') \) commits after  \( \txid[\cl'](n',m') \) 
            in the kv-store trace \( \progtrc' \),
            thus in the Clock-SI trace \( \clockexttrc' \).
            By \( \rCLOCKPrepare \) rule, when \( \cl'' \) prepares the new version, 
            there must be no version that commit after the snapshot time of \( \cl'' \),
            which means
            \begin{Formulae}
            \begin{Formula}
                (\txid[\cl'](n',m'),\txid[\cl''](n'',m'')) \in \WW[\kvs] \implies n' < m'' \leq n''.
                \label{equ:clock-ww-commit-time}
            \end{Formula}
            \end{Formulae}
        \Case{\( (\txid[\cl'](n',m'),\txid[\cl'](n'',m'')) \in \SO \)}
            Because the local times for Clock-SI clients monotonically increase,
            we know that
            \begin{Formulae}
            \begin{Formula}
                (\txid[\cl'](n',m'),\txid[\cl'](n'',m'')) \in \SO \implies n' < m'' \leq n'' .
                \label{equ:clock-so-commit-time}
            \end{Formula}
            \end{Formulae}
        \Case{\( \ToEdge{\txid[\cl'](n',m') | \rel
                        -> \txid[\cl^*](n^*,m^*) | \RW[\kvs'](\key) -> \txid[\cl''](n'',m'') } \) with \( \rel \in \Set{\WR[\kvs'],\WW[\kvs'],\SO}\)}
            This means there exist two indexes \( \idx, \idx' \) such that
            \[
                \idx < \idx'
                \land \Exists{\txidset}
                \kvs(\key,\idx) = (\stub, \stub, \txidset)
                \land \txid[\cl^*](n^*) \in \txidset 
                \land \kvs(\key,\idx') = (\stub, \txid[\cl''](n'',m''), \stub)
            \]
            Consider the transaction \( \txid[\cl^*](n^*,m^*) \) in the Clock-SI trace.
            By \cref{lem:clock-si-wr},
            %the snapshot time \( m^* \) of this transaction must be strictly smaller than the commit time \( n^* \).
            The snapshot time \( m^* \)  must be strictly smaller than than \( n'' \) (otherwise it contradict to \cref{lem:clock-si-wr}),
            that is, \(m^{*} < n'' \).
            By \ToEdge{\txid[\cl'](n',m') | \rel -> \txid[\cl^*](n^*,m^*) } for \( \rel \in \Set{\WR[\kvs'],\WW[\kvs'],\SO} \)
            and \cref{equ:clock-wr-commit-time,equ:clock-ww-commit-time,equ:clock-so-commit-time},
            we then have \(  n' < m^* < n^* \), that is,
            \begin{Formulae}
            \begin{Formula}
                \ToEdge{\txid[\cl'](n',m') | \rel -> \txid[\cl^*](n^*,m^*)| \RW[\kvs'](\key) -> \txid[\cl''](n'',m'') }
                \implies n' < n''
                \label{equ:clock-rw-commit-time}
            \end{Formula}
            \end{Formulae}
        \end{enumerate}
        Combine \cref{equ:clock-wr-commit-time,equ:clock-ww-commit-time,equ:clock-so-commit-time,equ:clock-rw-commit-time},
        we have
            \begin{Formulae}
            \begin{Formula}
                (\txid[\cl'](n',m'),\txid[\cl''](n'',m'')) \in 
                    \TraRe(\left({\Trasi(( \WR[\kvs] \cup \SO \cup \WW[\kvs]))} ; (\RWRelf[\kvs']) \right))
                \implies n' < n'' .
                \label{equ:clock-si-commit-time}
            \end{Formula}
            \end{Formulae}
        This means, by the definition of \( \vi = \ClockView(\kvs,\clocktime) \),
        if the view \( \vi \) includes version written by  \( \txid[\cl''](n'',m'') \),
        then the view also includes version written by \( \txid[\cl'](n',m') \),
        thus we prove \cref{equ:clock-vi-cloce-cycle}.
    \Case{\cref{equ:clock-vi-ua}}
        Recall that the new transaction \( \txid[\cl](n,m) \).
        Consider any write operation \( \opW(\key,\val) \in \fp \).
        Let \( \clockshard = \ShardOf(\clocksi,\key) \) be the shard that contains \( \key \),
        and \( (\clockkvs,\stub) = \clocksi(\clockshard) \) be the local key-value store.
        By \( \rCLOCKPrepare \), 
        there must be no version for the key \( \key \) 
        with the committed or preparation time greater than \( m \),
        that is,
        \[
            \Forall{\idx,n'} \clockkvs(\key,\idx) = (\stub,n',\stub) \implies n' < m .
        \] 
        By the definition of \( \vi = \ClockView(\kvs,\clocktime) \),
        we prove \cref{equ:clock-vi-ua}. \qedhere
    \end{enumerate}
\end{enumerate}
\end{proof}
\begin{proofsketch}
    Consider a step in the trace \( \progtrc \):
    \[
        \ToProg{\kvs | \vienv | \clenv | \prog | \lbTrans{\vi,\fp}
                -> \kvs' | \vienv\UpdateFunc{\cl -> \vi} | \clenv' | \prog' } .
    \]
    By the definition of \( \et[\SI] \),
    we need to prove \( \et[\MR]\), \( \et[\RYW] \) and
    \begin{Formulae}
    & \begin{Formula}
    \PreClosed(\kvs',\vi,\TraRe(\left({\Trasi(( \WR[\kvs'] \cup \SO \cup \WW[\kvs']))} ; \RWInv[\kvs'] \right)))
    \label{equ:proof-sketch-clock-vi-cloce-cycle}
    \end{Formula}
    \\ & \begin{Formula}
    \PreClosed(\kvs,\vi,\bigcup_{\opW(\key,\stub) \in \fp}\WWInv[\kvs'](\key))
    \label{equ:proof-sketch-clock-vi-ua}
    \end{Formula}
    \end{Formulae}
    \( \et[\MR]\) can be derived by \cref{lem:clock-si-local-time-mono} 
    that client local time monotonically increases,
    and the definition of \( \ClockView \).
    \( \et[\RYW]\) can be derived by \( \rCLOCKCommit \), 
    where the client local time is set to be the commit time,
    and  then by the definition of \( \ClockView \):
    all the versions committed by the client must be included in the new view.
    Given the property of normalised traces:
    \begin{enumerate}
    \item if two transaction \( \Tuple{\clocktxid[\cl](n,m), \clocktxid[\cl'](n',m')} \in \rel \)
    for \( \rel \in \Set{\WR,\WW,\SO} \),
    we have the commit time of the first transaction \( n \) is smaller than the snapshot time of the second transaction \( m' \),
    thus \( n < m' < n'\); and
    \item if two transaction \( Tuple{\clocktxid[\cl](n,m), \clocktxid[\cl'](n',m')} \in \RW \),
    we have the snapshot time of the first transaction \( m \) is smaller than the commit time of the second transaction \( n' \),
    that is \( m < n'\).
    \end{enumerate}
    Recall that the snapshot time is always smaller than commit time.
    If \( \Tuple{\clocktxid[\cl](n,m), \clocktxid[\cl'](n',m')} \in \TraRe(\left({\Trasi(( \WR[\kvs'] \cup \SO \cup \WW[\kvs']))} ; \RWInv[\kvs'] \right)) \)
    then \( n < n' \).
    Then by the definition of \( \ClockView \), we prove \cref{equ:proof-sketch-clock-vi-cloce-cycle}.
    Last, by the \( \rCLOCKPrepare \), when a version is set to preparation stage, there is no conflict write,
    then \cref{equ:proof-sketch-clock-vi-ua} holds.
    The full detail is given in \cref{sec:proof-clock-si-sat-si}
    on page \pageref{sec:proof-clock-si-sat-si}.
\end{proofsketch}

\begin{toappendix}
\begin{lemma}[Clock-SI \texorpdfstring{\(\WR\)}{\texttt{WR}}]
\label{lem:clock-si-wr}
Assume an annotated normalised  Clock-SI trace \( \clockexttrc \),
and the kv-store trace induced by the Clock-SI trace \( \progtrc = \ClockToKVTrace(\clockexttrc)\).
Given final kv-store \( \kvs \) in \( \progtrc \), 
that is, \( (\kvs,\stub,\stub,\stub) = \LastConf(\progtrc) \),
and a write-read edge, \( \Tuple{ \clocktxid[\cl](n,m) , \clocktxid[\cl'](n',m') } \in \WR[\kvs](\key) \),
then \( \clocktxid[\cl](n,m) \) is the latest transaction that
write to key \( \key \) and commits before \( \txid[\cl'](n',m') \) starts,
that is, 
\begin{Formulae}*
& \begin{Formula}
    n = \Max(\Set{n'' | \Exists{\idx,\cl'',m''} \clocktxid[\cl''](n'',m'') = \WtOf(\kvs(\key,\idx))} )
    \land n < m' .
\end{Formula}
\end{Formulae}
\end{lemma}
\begin{proof}
By the definition of \( \ClockToKVTrace \)
and \( \Tuple{ \clocktxid[\cl](n,m) , \clocktxid[\cl'](n',m') } \in \WR[\kvs](\key) \),
the transaction  \( \clocktxid[\cl'](n',m')  \) 
commits after the commit of \( \clocktxid[\cl](n,m) \), that is, \( m < m'\).
This means that there exists a prefix of \( \clockexttrc \) and a corresponding prefix of \( \progtrc \),
in which the last transaction is \(  \clocktxid[\cl'](n',m') \).
Let \( \clockexttrc' \) denote the prefix of \( \clockexttrc \),
and \( \progtrc' \) denote the prefix of \( \progtrc \).
We have \( \progtrc' = \ClockToKVTrace(\clockexttrc') \).
Since \( \clockexttrc' \) is a normalised trace, then
\[
    \clockexttrc' = \ToClockProg {\clockexttrc'' | \lb -> \clockexttrc^* } 
    \land \CLOCKAtomic(\clockexttrc^*,\cl',\clockshard,n') .
\]
Let \( (\clocksi,\clockclientenv,\clenv,\clockrunprog) = \clockexttrc^*\Proj{0} \).
Let \( \progtrc'' = \ClockToKVTrace(\ToClockProg{\clockexttrc'' | \lb -> \clocksi | \clockclientenv  | \clenv | \clockrunprog } ) \)
and \( (\kvs'',\vienv'',\clenv'',\prog'') = \Last(\progtrc'') \).
By the definition of \( \ClockToKVTrace \), we have
\[
 \progtrc' = \ToProg{ \progtrc'' | \lbTrans{\vi,\fp} -> \kvs^* | \vienv''\UpdateFunc{\cl' -> \vi'} | \clenv^* | \prog^*}
\]
where 
\begin{Formulae}*
\begin{Formula}
\vi = \ClockView(\kvs'',m') \land \fp = \ClockFp(\emptyset,\clockexttrc^*) 
\\ {} \land \kvs^* = \UpdateKV(\kvs'',\vi,\fp,\clocktxid[\cl'](n',m')) \land \vi' = \ClockView(\kvs^*,m')
\end{Formula}
\end{Formulae}
Because \( \clocktxid[\cl'](n',m') \) read the key \( \key \) in \( \kvs^* \),
then \( \opR(\key,\val) \in \fp \).
By the definition of  \( \fp = \ClockFp(\emptyset,\clockexttrc^*) \) and \cref{prop:well-formed-clock-fp},
there exists a read step that read from the shard \( \clockshard \) in  \( \clockexttrc^* \):
\begin{Formulae}*
\begin{Formula}
\clockexttrc^* = \ToClockProg{\cdots | \stub 
                            -> \clocksi | \stub | \stub | \clockrunprog\UpdateFunc{\cl' -> 
                                                        \pruntrans{\plookup{\var}{\expr}; \trans}%
                                                        {\fp',\emptyset}%
                                                        {m',\clockshard} } 
                                        | \lbCLOCKOp[\cl'](\clockshard){\opR(\key,\val),m',n'} -> \ }
                \\ \ToClockProg{ \clocksi | \stub | \stub | \clockrunprog\UpdateFunc{\cl' -> 
                                                        \pruntrans{\trans}%
                                                        {\fp' \AddOp \opR(\key,\val),\emptyset}%
                                                        {m',\clockshard} } 
                            | \stub -> \cdots }
\end{Formula}
\end{Formulae}
where the free variables are existentially quantified.
By the \( \rCLOCKReadShard \) rule, 
step \( \lbCLOCKOp[\cl'](\clockshard){\opR(\key,\val),m',n'}  \) read the version of \( \key \)
with time \( \clocktime \)
such that
\begin{Formulae}
\begin{Formula}
    \Exists{\clockkvs} (\clockkvs,\stub) = \clocksi(\clockshard)
    \land \clocktime = \Max(\Set{\clocktime' | \Exists{\idx} \clockkvs(\key,\idx) = (\stub,\clocktime') \land \clocktime' < m'}) .
    \label{equ:read-max-in-kvs-match-in-clock-si}
\end{Formula}
\end{Formulae}
Note that the state of \( \clocksi \) has not changed until the read step, 
because of \( \CLOCKAtomic(\clockexttrc^*,\cl',\clockshard,n') \).
Thus, by the definition of \( \progtrc'' = \ClockToKVTrace(\ToClockProg{\clockexttrc'' | \lb
            -> \clocksi | \clockclientenv  | \clenv | \clockrunprog } ) \)
and \( (\kvs'',\vienv'',\clenv'',\prog'') = \Last(\progtrc'') \),
we have \( \kvs''(\key,\idx) = (\val,\clocktxid[\cl](n,m),\stub)\) for some \( \idx \),
By the definition of \( \vi = \ClockView(\kvs'',m') \), we have \( \idx \in \vi(\key) \).
Then by \cref{equ:read-max-in-kvs-match-in-clock-si},
and the definition of \( \kvs^* = \UpdateKV(\kvs'',\vi,\fp,\clocktxid[\cl'](n',m'))  \),
we have the proof.
\end{proof}
\end{toappendix}
