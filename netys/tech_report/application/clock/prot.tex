\begin{figure}

\centering

\begin{tikzpicture}
\OperationsBox[above]{r1}{\clockshard_1,\ \texttt{Local Time:} \  0}{
    /\texttt{key-value store:},
    /{\Tuple{\key_1 , \valinit, 0} },
    /{},
    /\texttt{Preparation set:},
    /\emptyset
};

\OperationsBox((r1.east) + (2,0.35))[above]{r2}{\clockshard_2,\ \texttt{Local Time:} \  0}{
    /\texttt{key-value store:},
    /{\Tuple{\key_2 , \valinit, 0} },
    /{},
    /\texttt{Preparation set:},
    /\emptyset
};
\end{tikzpicture}

\hrulefill

\caption{An initial Clock-SI state with two shards, $\clockshard_1$ and $\clockshard_2$,
which contains keys $\key_1$ and $\key_2$ respectively}
\label{fig:initial-clock-si}

\end{figure}


Clock-SI is a fully partitioned distributed key-value database,
with each shard storing all history version of a distinct part of keys
as shown in \cref{fig:initial-clock-si,fig:clock-si-start-exec,fig:clock-si-example,fig:clock-si-result}.
Each version in a shard, for example \( \Tuple{{\key_1}, \valinit, 0} \)  in \cref{fig:initial-clock-si},
consists of a key, \( \key_1 \), a value, \( \valinit \), and the time this version has been committed, \(\clocktime = 0\).
A transaction in Clock-SI contains arbitrary read and write operations, and internal computation.
For example, the following program \( \prog_{\CodeFont{Clock}} \), contains two clients, 
each of which has one transactions.
Both the transactions read the value of keys \( \key_1 \) and \( \key_2 \),
and if the values are the initial value \( \valinit \), they updates \( key_1 \) or \( key_2 \) respectively.
\[          
\begin{array}{c || c }
    \prog_{\CodeFont{Clock}} \FuncDef \cl_1 : \ptrans{ \plookup{\var(x)}{\key_1} ; \plookup{\var(y)}{\key_2} ;
                        \\ \pif(\var = \valinit \ \&\& \ \var(y) = \valinit) \ \pmutate{\key_2}{\val'}} 
    & \cl_2: \ptrans{ \plookup{\var(z)}{\key_1} ; \plookup{\var(m)}{\key_2} ;
                        \\ \pif(\var(z) = \valinit \ \&\& \ \var(m) = \valinit) \ \pmutate{\key_1}{\val}} .
\end{array}
\]
The idea of Clock-SI is that:
\begin{enumerate}
\item a transaction tracks a snapshot time \( \clocktime \), and when reading a key \( \key \), 
        the transaction reads the latest value for \( \key \) before \( \clocktime \); and
\item when committing, the transaction uses a two-phase commit protocol
        in that new versions first commits to \emph{preparation sets} of shards,
        as shown in \cref{fig:clock-si-example}, and then commits to key-value stores.
\end{enumerate}
Each shard \( \clockshard \) tracks a local time, which is the actual physical time of the shard.
In an ideal situation, local times in different shards should be the same.
However, this is impossible in practice.
This means the protocol must cope with the time difference between shards.
%In practice, Clock-SI assume that there is a upper bound of the difference between shard local times.
A client also carries a local time that tracks the commit time of its last transaction.
This is used to maintain the session order over transactions from the client.

Let us describe the protocol using the program \( \prog_{\CodeFont{Clock}} \).
A client commits a transaction and a minimum snapshot time being the current client local time
to an \emph{arbitrary} shard, and awaiting the confirmation.
%For example, in \( \prog_{\CodeFont{Clock}} \)  \( \cl_1 \) commits 
%its transaction \( \txid_1\) to shard \( \clockshard_1 \) with local time \( 0 \),
%and \( \cl_2 \) commits \( \txid_2 \) to shard \( \clockshard_2 \) with local time \( 0 \).
Upon receiving the transaction \( \txid \), the shard \( \clockshard \) acts as the coordinator of the transaction:
\begin{enumerate}
\item \( \clockshard \) decides the snapshot time of \( \txid \);
\item \( \clockshard \) executes the commands of \( \txid \) and tracks the effect of these commands in a \emph{read-write set}; and 
\item \( \clockshard \) commits the effect of \( \txid \), any new values,
    to the appropriate shards using a two-phase commits protocol.
\end{enumerate}

Assume that \( \cl_1 \) commits 
its transaction \( \txid_1\) to shard \( \clockshard_1 \) with client local time \( 0 \).
This local time is the minimum snapshot time for the new transaction.
Once receiving the new transaction,
coordinator \( \clockshard_1 \) assigns the coordinator local time \( \clocktime \) 
as the snapshot time of \( \txid_1 \),
only if \( \clocktime \) is greater than the minimum snapshot time.
For example, in \cref{fig:clock-si-start-exec},
\( \clockshard_1 \) waits until the local time becomes 1,
and then assigns it as the snapshot time of transaction \( \txid_1\).
Similarly, \( \clockshard_2 \) assigns 1 as the snapshot time of transaction \( \txid_2 \) in parallel.
The coordinator also initialises the read-write set being an empty set.
The read-write set is the same as the fingerprint in our kv-store semantics,
which tracks the observable effect of the transaction,
containing at most one read and one write operations of each key.

\begin{figure}
\centering
\begin{tikzpicture}
\OperationsBox[above]{r1}{\clockshard_1,\ \texttt{Local Time:} \ 1}{
    /\texttt{key-value store:},
    /{\Tuple{\key_1 , \valinit, 0} },
    /{},
    /\texttt{Preparation set:},
    /\emptyset
};

\OperationsBox((r1.east) + (2,0.35))[above]{r2}{\clockshard_2,\ \texttt{Local Time:} \ 1}{
    /\texttt{key-value store:},
    /{\Tuple{\key_2 , \valinit, 0} },
    /{},
    /\texttt{Preparation set:},
    /\emptyset
};

\OperationsBox((r1.south) + (-0.4,-1))[above]{t1}{\txid_1,\ \texttt{Snapshot Time:} \ 1}{
    /\emptyset
};

\OperationsBox((r2.south) + (-0.4,-1))[above]{t2}{\txid_2,\ \texttt{Snapshot Time:} \ 1}{
    /\emptyset
};
 
\end{tikzpicture}

\hrulefill

\caption{Shard \( \clockshard_1 \) assigns the snapshot time to transaction \( \txid_1\) and 
shard \( \clockshard_2 \) assigns the snapshot time to transaction \( \txid_2 \) in parallel}
\label{fig:clock-si-start-exec}
\end{figure}



After the initialisation, it is ready to execute the transaction.
Let us take \( \txid_1 \) as an example.
For any lookup command, for example \( \plookup{\var(x)}{\key_1}  \),
the coordinator first checks if there is an entry of \( \key_1 \) in the read-write set of \( \txid_1 \).
If there is a write operation of \( \key_1 \), for example \( \opW(\key_1,\val) \),
it means that the transaction has updated the key internally.
If there is a read operation of \( \key_1 \), for example \( \opR(\key_1,\val) \),
it means that the transaction has read the key before.
In both cases, because snapshot isolation satisfies snapshot property,
hence the transaction reads value \( \val \) 
in the write operation or in the read operation respectively.
If there is no entry of \( \key_1 \),
the coordinator sends a request to the shard \( \clockshard_1 \) containing key \( \key_1 \),
and the snapshot time of the transaction, which is used to fetch to correct value.
Upon receiving the request, shard \( \clockshard_1 \) waits until: 
\begin{enumerate}
\item the shard local time is greater than the snapshot time; and 
\item there is no version of \( \key_1 \) in the preparation set that has time-stamp smaller than the snapshot time.
\end{enumerate}
If both conditions are satisfied,
then there is no new version of \( \key_1 \) that might commit before the snapshot time.
Shard \( \clockshard_1 \) then replies the latest value of \( \key_1 \) before the snapshot time.
For any mutation command, for example \( \plookup{\key_2}{\val'}  \),
the coordinator erases any previous write operations of \( \key_2 \) and 
adds a new write operation \( \opW(\key_2, \val') \).
Note that different transactions might execute in parallel as shown in \cref{fig:clock-si-parallel-exec}.

\begin{figure}

\centering

\begin{subfigure}{\textwidth}
\centering
\begin{tikzpicture}
\OperationsBox[above]{r1}{\clockshard_1,\ \texttt{Local Time:} \ 2}{
    /\texttt{key-value store:},
    /{\Tuple{\key_1 , \valinit, 0} },
    /{},
    /\texttt{Preparation set:},
    /\emptyset
};

\OperationsBox((r1.east) + (2,0.35))[above]{r2}{\clockshard_2,\ \texttt{Local Time:} \ 2}{
    /\texttt{key-value store:},
    /{\Tuple{\key_2 , \valinit, 0} },
    /{},
    /\texttt{Preparation set:},
    /\emptyset
};

\OperationsBox((r1.south) + (-2.8,-1))[above]{t1}{\txid_1,\ \texttt{Snapshot Time:} \ 1}{
    /{\opR(\key_1,\valinit)},
    /{\opR(\key_2,\valinit)},
    /{\opW(\key_2,\val')}
};

\OperationsBox((r2.south) + (-1.7,-1))[above]{t2}{\txid_2,\ \texttt{Snapshot Time:} \ 1}{
    /{\opR(\key_1,\valinit)},
    /{\opR(\key_2,\valinit)}
};
 
\end{tikzpicture}
\caption{The final fingerprint of \( \txid_1 \)}
\label{fig:clock-si-parallel-exec}
\end{subfigure}

\hrulefill

\begin{subfigure}{\textwidth}
\centering
\begin{tikzpicture}
\OperationsBox[above]{r1}{\clockshard_1,\ \texttt{Local Time:} \ 2}{
    /\texttt{key-value store:},
    /{\Tuple{\key_1 , \valinit, 0} },
    /{},
    /\texttt{Preparation set:},
    /\emptyset
};

\OperationsBox((r1.east) + (2,0.35))[above]{r2}{\clockshard_2,\ \texttt{Local Time:} \ 2}{
    /\texttt{key-value store:},
    /{\Tuple{\key_2 , \valinit, 0} },
    /{},
    /\texttt{Preparation set:},
    /{\Tuple{\key_2 , \val', 2} }
};

\OperationsBox((r1.south) + (-2.8,-1))[above]{t1}{\txid_1,\ \texttt{Snapshot Time:} \ 1}{
    /{\opR(\key_1,\valinit)},
    /{\opR(\key_2,\valinit)},
    /{\opW(\key_2,\val')}
};

\OperationsBox((r2.south) + (-1.7,-1))[above]{t2}{\txid_2,\ \texttt{Snapshot Time:} \ 1}{
    /{\opR(\key_1,\valinit)},
    /{\opR(\key_2,\valinit)}
};
 
\end{tikzpicture}
\caption{The preparation phase of \( \txid_1 \)}
\label{fig:clock-si-preparation}
\end{subfigure}

\hrulefill

\begin{subfigure}{\textwidth}
\centering
\begin{tikzpicture}
\OperationsBox[above]{r1}{\clockshard_1,\ \texttt{Local Time:} \ 2}{
    /\texttt{key-value store:},
    /{\Tuple{\key_1 , \valinit, 0} },
    /{},
    /\texttt{Preparation set:},
    /\emptyset
};

\OperationsBox((r1.east) + (2,0.35))[above]{r2}{\clockshard_2,\ \texttt{Local Time:} \ 2}{
    /\texttt{key-value store:},
    /{\Tuple{\key_2 , \valinit, 0} },
    /{},
    /\texttt{Preparation set:},
    /{\Tuple{\key_2 , \val', 2} }
};

\OperationsBox((r1.south) + (-2.8,-1))[above]{t1}{\txid_1,\ \texttt{Commit Time:} \ 2}{
    /{\opR(\key_1,\valinit)},
    /{\opR(\key_2,\valinit)},
    /{\opW(\key_2,\val')}
};

\OperationsBox((r2.south) + (-1.7,-1))[above]{t2}{\txid_2,\ \texttt{Snapshot Time:} \ 1}{
    /{\opR(\key_1,\valinit)},
    /{\opR(\key_2,\valinit)}
};
 
\end{tikzpicture}
\caption{The commit phase of \( \txid_1 \)}
\label{fig:clock-si-preparation}
\end{subfigure}

\hrulefill

\caption{Clock-SI two-phase commit protocol}
\label{fig:clock-si-example}

\end{figure}


When a transaction reaches the end, \( \pskip \),
the final read-write set contains the effect of this transaction,
for example the read-write set for \( \txid \) depicted in \cref{fig:clock-si-parallel-exec}.
The coordinator then commits any updates in the read-write set to appropriate shards,
using a two-phase commits protocol.
In the first commit phase, for any write operation such as \( \opW(\key_2,\val')\) in \cref{fig:clock-si-preparation},
the coordinator sends a commit request to the shard that contains the key.
Upon receiving the commit request,
the shard  will accept the new version if there is no conflict write:
that is, any versions of the key that has been committed or prepared in the shard
since the snapshot time of this transaction.
For example in \cref{fig:clock-si-parallel-exec},
the new value \( \val' \) for \( \key_2 \) can commit in shard \( \clockshard_2 \),
since there is no other versions of \( \key_2 \) with time-stamp greater than the snapshot time \( 1 \).
If there is no conflict write, 
the shard assigns the shard local time as the preparation time of the new version,
and puts the version in the preparation set as shown in \cref{fig:clock-si-preparation}.
In the successful case, 
the shard replies with the preparation time of the new versions.
Otherwise, the shard rejects the new version due to conflict writes; and
the coordination might abort or restart the transaction.

In the second phase,
the coordinator picks the maximum time over all preparation times as the commit time of this transaction.
For example in \cref{fig:clock-si-preparation}, the commit time of \( \txid_1 \) is 2.
The coordinator finally sends the commit time to all shards 
that contain prepared versions for this transaction.
Upon receiving the commit time \( \clocktime \), 
the shard first updates its local time to \( \clocktime \), if the local time is behind \( \clocktime \).
The shard updates the time of prepared versions for this transaction to be \( \clocktime \) 
and commit them to the local key-value store.
For example, \cref{fig:clock-si-commit-txid-1} shows the resulting state after \( \txid_1 \).
Note that, despite there is a new version for \( \key_2 \), this new version does not affect \( \txid_2 \) 
because:
\begin{enumerate*}
\item the new version commits after the snapshot time of \( \txid_2 \),
hence it does not affect any read of \( \txid_2 \); and 
\item \( \txid_2 \) does not update \( \key_2 \), hence \( \txid_2 \) can commit successful.
\end{enumerate*}
An example result of\( \prog_{\CodeFont{Clock}} \) is given in \cref{fig:clock-si-commit-txid-2},
which is the write skew anomaly.

\begin{figure}

\centering

\begin{subfigure}{\textwidth}
\centering
\begin{tikzpicture}
\OperationsBox[above]{r1}{\clockshard_1,\ \texttt{Local Time:} \ 2}{
    /\texttt{key-value store:},
    /{\Tuple{\key_1 , \valinit, 0} },
    /{},
    /\texttt{Preparation set:},
    /\emptyset
};

\OperationsBox((r1.east) + (2,0.35))[above]{r2}{\clockshard_2,\ \texttt{Local Time:} \ 2}{
    /\texttt{key-value store:},
    /{\Tuple{\key_2 , \valinit, 0} },
    /{\Tuple{\key_2 , \val', 2} },
    /\texttt{Preparation set:},
    /\emptyset
};

\OperationsBox((r2.south) + (-2.8,-1))[above]{t2}{\txid_2,\ \texttt{Snapshot Time:} \ 1}{
    /{\opR(\key_1,\valinit)},
    /{\opR(\key_2,\valinit)},
    /{\opW(\key_1,\val)}
};
\end{tikzpicture}
\caption{Client \( \cl_1 \) commits \( \txid_1 \) to \(\clockshard_1 \), 
while \( \txid_2\) are still running}
\label{fig:clock-si-commit-txid-1}
\end{subfigure}

\hrulefill

\begin{subfigure}{\textwidth}
\centering
\begin{tikzpicture}
\OperationsBox[above]{r1}{\clockshard_1,\ \texttt{Local Time:} \ 2}{
    /\texttt{key-value store:},
    /{\Tuple{\key_1 , \valinit, 0} },
    /{\Tuple{\key_1 , \val, 2} },
    /\texttt{Preparation set:},
    /\emptyset
};

\OperationsBox((r1.east) + (1.5,0.35))[above]{r2}{\clockshard_2,\ \texttt{Local Time:} \ 2}{
    /\texttt{key-value store:},
    /{\Tuple{\key_2 , \valinit, 0} },
    /{\Tuple{\key_2 , \val', 2} },
    /\texttt{Preparation set:},
    /\emptyset
};
\end{tikzpicture}
\caption{Client \( \cl_2 \) commits \( \txid_2 \) to \(\clockshard_2 \)}
\label{fig:clock-si-commit-txid-2}
\end{subfigure}

\hrulefill

\caption{an example result of \( \txid_1 \) and \( \txid_2 \)}
\label{fig:clock-si-result}

\end{figure}


\citet{clocksi} informally argued that Clock-SI implements snapshot isolation.
First, a transaction carrying a snapshot time \( \clocktime \) read the latest values of keys.
Note that the two-phase commits protocol guarantees, when the transaction reads from a shard, 
there must be no version that can commit before the snapshot time.
Second, the transaction commits its final effect via the two-phase commits protocol,
which guarantee all the versions commit at the same time to appropriate shards.
Recall that the commit time is the maximum time over all preparation times.
This means if there is no conflict write between the snapshot and the preparation time in a shard,
then it must be no conflict write until the commit time,
because any prepared version blocks further update to the key.
