In this section, we introduce annotated Clock-SI traces and annotated normalised Clock-SI traces.
For the purpose of manipulating a Clock-SI trace,
we annotate the commit time to the read, write and preparation transitions for each transition.
An annotate normalised trace is a trace where 
transitions of each transaction cannot be interfered by transitions from other transactions.
Annotate normalised traces are the key for verifying Clock-SI protocol,
because transaction in the traces are executed atomically.

Given a Clock-SI trace, for each transaction, 
we annotate read, write and preparation steps with the commit time of the transaction.
This extra information is useful for swapping these steps.
We call the new traces as \emph{annotated Clock-SI traces}.

\begin{definition}[Annotated Clock-SI traces]
Given a Clock-SI trace,
an extended Clock-SI trace is defined inductively by:
\begin{align*}
\ClockToExd(\ToClockProg{\clocksi | \clockclientenv | \clenv | \prog})
& \FuncDef \ToClockProg{\clocksi | \clockclientenv | \clenv | \prog}
\\ \ClockToExd(\ToClockProg{\clocksi | \clockclientenv | \clenv | \prog | \lb  -> \clocktrc})
& \FuncDef \begin{cases}
 \ToClockProg{\clocksi | \clockclientenv | \clenv | \prog | \lb  -> \ClockToExd(\clocktrc)} 
 & \lb \neq \lbCLOCKOp{(l,\key,\val),\clocktime}
 \\ \ToClockProg{\clocksi | \clockclientenv | \clenv | \prog | \lb,\clocktime'  -> \ClockToExd(\clocktrc)}
 & 
 \\ \multicolumn{2}{l}{\quad \lb = \lbCLOCKOp{(l,\key,\val),\clocktime} \land \lbCLOCKEnd{\clocktime'} = \ClockFirstCommit(\clocktrc,\cl)}
 \end{cases}
\end{align*}
where \( \ClockFirstCommit(\clocktrc,\cl) \) returns the first commit for client \( cl \) in the trace \( \clocktrc \).
Let, \( \CLOCKExtTraces \ni \clockexttrc \), denotes the set of the extended traces.
\end{definition}

