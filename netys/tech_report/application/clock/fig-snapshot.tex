\begin{figure}
\centering
\begin{tikzpicture}
\OperationsBox[above]{r1}{\clockshard_1,\ \texttt{Local Time:} \ 1}{
    /\texttt{key-value store:},
    /{\Tuple{\key_1 , \valinit, 0} },
    /{},
    /\texttt{Preparation set:},
    /\emptyset
};

\OperationsBox((r1.east) + (2,0.35))[above]{r2}{\clockshard_2,\ \texttt{Local Time:} \ 1}{
    /\texttt{key-value store:},
    /{\Tuple{\key_2 , \valinit, 0} },
    /{},
    /\texttt{Preparation set:},
    /\emptyset
};

\OperationsBox((r1.south) + (-0.4,-1))[above]{t1}{\txid_1,\ \texttt{Snapshot Time:} \ 1}{
    /\emptyset
};

\OperationsBox((r2.south) + (-0.4,-1))[above]{t2}{\txid_2,\ \texttt{Snapshot Time:} \ 1}{
    /\emptyset
};
 
\end{tikzpicture}

\hrulefill

\caption{Shard \( \clockshard_1 \) assigns the snapshot time to transaction \( \txid_1\) and 
shard \( \clockshard_2 \) assigns the snapshot time to transaction \( \txid_2 \) in parallel}
\label{fig:clock-si-start-exec}
\end{figure}

