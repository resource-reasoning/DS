In the COPS protocol, 
each replica tracks all the history versions, depicted in \cref{fig:app-cops-digraph}.
There is no explicit transactional identifier.
Instead, there is the \emph{COPS version identifier},
which consists of the replica identifier which initially accepted the version 
and the local time when the version was accepted.
%can be seen as the transactional identifier who wrote the version.

\begin{figure}

\centering
\begin{tikzpicture}
\OperationsBox[above][2]{r1}{\repl_1}{
    /{\Tuple{\key_1 , \valinit , (\txid_0,\repl_0), \emptyset} }%
    , /{\Tuple{\key_2 , \valinit , (\txid_0,\repl_0), \emptyset} }%
    , /{\Tuple{\key_1 , \val_1 , (\txid_1,\repl_1), \emptyset} }%
};

\OperationsBox((r1.east) + (1,0.37))[above][2]{r2}{\repl_2}{
    /{\Tuple{\key_1 , \valinit , (\txid_0,\repl_0), \emptyset} }%
    , /{\Tuple{\key_2 , \valinit , (\txid_0,\repl_0), \emptyset} }%
    , /{\Tuple{\key_1 , \val'_1 , (\txid_1,\repl_2), \emptyset} }%
    , /{\Tuple{\key_2 , \val_2 , (\txid_2,\repl_2), \Set{(\key_1,\txid_1,\repl_2)}} }%
};
\end{tikzpicture}

\hrulefill

\caption{Examples of COPS states}
\label{fig:app-cops-digraph}
\label{fig:app-cops-after-write-transaction}
\label{fig:app-initial-cops}
\end{figure}


\begin{definition}[COPS replica and version identifiers]
\label{fig:cops-ver-id}
The set of \emph{COPS replica identifiers}, \( \COPSReplicas \ni \repl \), 
is a countably infinite set that is totally ordered by a relation \( < \).
The set of local time, \( \Times \ni \ts\), is defined by \( \Times \TypeDef \Nat \).
The set of \emph{COPS version identifiers}, \( \COPSVerIDs \ni \copsverid \),
is defined by: \( \COPSVerIDs \TypeDef \Times \times \COPSReplicas .  \)
The order over versions identifiers is defined by:
\[ 
\copsverid \copsveridleq \copsverid'
\PredDef 
\copsverid = (n,\repl)
\land \copsverid' = (n',\repl')
\land n < n' \lor ( n = n' \land ( \repl < \repl' ) .
\]
\end{definition}

As explained in \cref{sec:overall-cops},
a COPS version \( \copsver \) consists of a value, a identifier 
and a dependency set that contains the versions on which \( \copsver \) depends.

\begin{definition}[COPS versions and dependency sets]
The set of \emph{dependency sets}, \( \COPSDependencies \ni \copsdep \) 
is defined by: \( \COPSDependencies \TypeDef \PowerSet(\copsdep \subseteq \Keys \times \COPSVerIDs) \).
The set of \emph{COPS versions}, \( \COPSVersions \ni \copsver \), is defined by: 
\( \COPSVersions \TypeDef \Values \times \COPSVerIDs  \times \COPSDependencies \).
Let \(\ValOf(\copsver)\), \(\IdOf(\copsver)\)  
and \( \DepOf(\copsver)\) denote the first, second and third projections of \( \copsver \) respectively.
\end{definition}

A COPS database comprises a finite number of replicas.
Each replica \( \repl \) consists of a local multi-version key-value store and a local time.
A COPS replica tracks all the history versions that are uniquely identified and ordered by their writers.
Without losing generality, versions for a key are organised as a list in the key-value store.
However, the relative position of versions does not affect the semantics, since all versions are ordered by their writers.

\begin{definition}[COPS key-value stores and databases]
\label{def:cops-kv-store}
The set of \emph{COPS local key-value stores} or just \emph{COPS local stores},
is defined by
\( \COPSKVSs \FuncDef \Set{\copskvs \in \Keys \ToTFunc \List{\COPSVersions} | \WfCOPSKvs(\copskvs)} \)
where \( \WfCOPSKvs \) is defined by: for any keys \( \key, \key' \in \Keys \),
indexes \( \idx, \idx' \in \Indexs \) and the initial value \( \valinit \),
\begin{Formulae}
    & \begin{Formula}
        \copskvs(\key,0) = (\valinit,(\repl_0,0),\emptyset),
        \label{equ:cops-kvs-init}
    \end{Formula}
    \\ & \begin{Formula}
        \IdOf(\copskvs(\key,\idx)) = \IdOf(\copskvs(\key',\idx')) 
            \implies \key = \key' \land \idx = \idx',
        \label{equ:cops-version-unique}
    \end{Formula}
    \\ & \begin{Formula}
        \IdOf(\copskvs(\key,\idx)) \copsveridleq \IdOf(\copskvs(\key,\idx')) 
            \iff \idx < \idx'.
        \label{equ:cops-version-order}
    \end{Formula}
\end{Formulae}
Given two COPS local stores, 
the order, written \( \copskvs \copskvsleq \copskvs' \), is defined by:
\[
    \copskvs \copskvsleq \copskvs' \PredDef \Forall{\key \in \Keys | \idx \in \Indexs }
    \copsver = \copskvs(\key, \idx) \implies \Exists{\idx' \in \Indexs} \copsver = \copskvs'(\key, \idx').
\]
The set of \emph{COPS databases}, \( \COPSs \ni \cops \), is defined by:
\[
    \COPSs \FuncDef \Set{\cops \in \COPSReplicas \ToPFFunc 
        \\  ( \COPSKVSs \times \Times ) | 
        \Forall{\repl, \repl' \in \Dom(\cops) | \copskvs, \copskvs' \in \COPSKVSs }
        \\ \Forall{ \key, \key' \in \Keys | \idx, \idx' \in \Indexs }
            \\ \cops(\repl) = (\copskvs,\stub) \land \cops(\repl') = (\copskvs',\stub) 
            \\ \left( \begin{array}{@{}r@{}}\IdOf(\copskvs(\key,\idx)) = \IdOf(\copskvs'(\key',\idx')) \iff
            \\ \qquad \key = \key' \land \copskvs(\key,\idx) = \copskvs'(\key',\idx') \end{array} \right)  }.
\]
\end{definition}

%Because versions will be eventually delivered to all sites,
A COPS version will be eventually replicated to all replica.
This means that versions with the same identifier in different replicas must be the same version.
This is captured by the well-formed condition for \( \COPSs \).

Each COPS client \( \cl \) maintains a local context that tracks versions \( \cl \) read or wrote before.
When a client commits a new version, 
the client context becomes the dependency set of the new version.
Hence, we model both dependency sets and client contexts as sets of pairs comprising keys and replica identifiers.
Each client \( \cl \) also tracks the replica identifier with which \( \cl \) interacts.

\begin{definition}[COPS client contexts and environments]
The set of \emph{COPS client contexts}, \( \COPSContexts \in \copsctx \), is defined by
\(
    \COPSContexts \TypeDef \PowerSet(\Keys \times \COPSVerIDs)
\).
The set of \emph{COPS client context environments}, \( \COPSContextEnvs \ni \copsctxenv \), is defined by:
\( \COPSContextEnvs \TypeDef \Clients \ToPFFunc (\COPSContexts \times \COPSReplicas ) \).
\end{definition}

A COPS configuration \( \copsconf \) comprises a COPS database  \( \cops \) and a COPS client environment \( \copsctxenv \).
Any client from the environment \( \copsctxenv \) must be served by a replica in \( \cops \).
%
\begin{definition}[COPS configurations]
\label{def:cops-conf}
The set of \emph{COPS configurations}, \( \COPSConfs \ni \copsconf \), is defined by:
\[ 
    \COPSConfs \TypeDef \Set{ (\cops, \copsctxenv) \in \\ \COPSs \times \COPSContextEnvs |
        \Forall{\cl \in \Dom(\copsctxenv) | \repl \in \COPSReplicas} 
        \\ \copsctxenv(\cl) = (\stub, \repl) \implies \repl \in \Dom(\cops) }.
\]
The set of \emph{initial COPS configurations}, \( \COPSConfInits \ni \copsconfinit\),
is defined by: 
\[ \COPSConfInits \TypeDef \Set{ (\cops, \copsctxenv) \in \\ \COPSConfs | 
        \Forall{\repl \in \COPSReplicas | \cl \in \Clients } 
        \\ \cops(\repl) = ((\lambda \key \in \Keys \ldotp  \List{(\valinit, \txidinit, \emptyset)}), 0 )
        \land \copsctxenv(\cl) = (\emptyset, \stub) }. 
\]
\end{definition}
