In our kv-store semantics (\cref{sec:model}), 
we assign a unique identifier for every transaction in \(\rCAtomicTrans\) in \cref{fig:command-semantics}.
In COPS semantics, there is no explicit transaction identifier.
We encode the COPS version identifier as \emph{COPS transaction identifier}
for the single-write transaction who committed the version.
We then annotate multiple-read transactions with transaction identifiers that preserves the session order
using the read-only transaction counter \( m \) in \(\copstxid[\cl][\repl](n,m)  \).

\begin{definition}[COPS transaction identifiers]
The set of \emph{COPS transaction identifiers}, \( \COPSTxIDs \ni \copstxid \),
is defined by:
\[ 
    \COPSTxIDs \TypeDef \Set{\copstxid[\cl][\repl](n,m) | 
    \cl \in \Clients \land \repl \in \COPSReplicas \land n,m \in \Nat } \cup \Set{\txidinit}.
\]
The order over transactions is defined by:
\[ 
\copstxid[\cl][\repl](n,m) \copstxidleq \copstxid[\cl'][\repl'](n',m')
\PredDef n < n' \lor ( n = n' \land ( \repl < \repl' \lor ( \repl = \repl' \land m < m' ) ) .
\]
\end{definition}

The COPS transaction identifier \(\copstxid[\cl][\repl](n,m)  \) contains
a client identifier \( \cl \), a local time \(n\),
a replica identifier \( \repl \)
and an extra counter \( m \) for read-only transactions.
For the single write transaction, the transaction identifier is the version identifier \( (n,\repl) \) and \( m = 0 \).
%In the operational semantics of COPS protocol (\cref{sec:cops-model,sec:cops-semantics}), 
%the extra counter \(m\) is always ZERO.
%Because COPS does not directly tracks read-only transactions.
%Later, in the verification of COPS protocol (\cref{sec:cops-extended-trace}),
we then use the extra counter \( m \) to annotate read-only transactions.
Note that, transaction identifiers are order lexicographically.
If we encode all tuples, \( (\repl,n,m) \), as individual numbers, 
all these encodes numbers must be totally ordered.
Therefore, there is a bijection between \( \COPSTxIDs \) 
and kv-store transaction identifiers \( \TxIDs\) defined in \cref{def:transaction-id}.

We first assign COPS transaction identifiers to single-write transactions:
that is, we extend the version identifiers with the client identifiers and the reader counters \( m \) being zero.
We then assign COPS transaction identifiers with non-zero read-only counters to multiple-read transactions:
that is, for each multiple-read transaction, 
we annotate the re-fetch operations and the end operation 
with the next available identifier in \( \copsclenv \in \Clients \ToTFunc \COPSTxIDs  \).
To preserve the session order, we update \( \copsclenv \) for the client \( \cl \),
if \( \cl \) commits a new transaction.
By construction, the transaction identifiers that are assigned to multiple-read transactions must be unique.
The detail is given in \cref{prop:cops-read-id-unique} on page \pageref{sec:proof-multiple-read-trans-id-unique}.

%We track the next available multiple-read transaction identifier for every client,
%that is, \(\copsclenv \in \Clients \ToTFunc \COPSTxIDs \).
%Recall that each COPS transaction identifier \( \copstxid[\cl][\repl](n,m) \) is annotated with
%a read-only transaction counter \( m \).
%In the COPS semantics, this read-only transaction counter is always set to be ZERO.

\begin{definition}[Annotated normalised COPS traces]

A \emph{client local time environment} is defined by: \(\copsclenv \in \Clients \ToTFunc \COPSTxIDs \).
Given a normalised COPS trace \( \copstrc \in \COPSTraces \),
the annotated normalised COPS trace is defined by:
\begin{align*}
   \COPSToExt((\copsconfinit,\copsprog_0),\copsclenv) & \FuncDef (\copsconfinit,\copsprog_0) ,
\\ \multicolumn{2}{l}{
        \COPSToExt( \begin{Bracketed} \ToCOPSProg{ \cops | \copsctxenv 
            | \copsrunprog | \lb -> \copstrc } \end{Bracketed} , \copsclenv ) }
\\ \multicolumn{2}{l}{
        \( \quad \FuncDef 
        \begin{cases}
            \ToCOPSProg{ \cops | \copsctxenv | \copsrunprog | \lb, \copstxid[\cl][\repl](n,0)
                -> \COPSToExt(\copstrc, \copsclenv\UpdateFunc{\cl -> \copstxid[\cl][\repl](n,1) }) }    
                & \If \lb =  \lbCOPSWrite{\opW(\key,\val), (n,0), \copsdep } ,
         \\ \ToCOPSProg{ \cops | \copsctxenv | \copsrunprog | \lb, \copsclenv(\cl)
                -> \COPSToExt(\copstrc, \copsclenv) }
                & \If \lb =  \lbCOPSRefetch{\opR(\key,\val), \copstxid, \copsdep }  ,
         \\ \ToCOPSProg{ \cops | \copsctxenv | \copsrunprog | \lb, \copsclenv(\cl)
                -> \COPSToExt(\copstrc, \copsclenv\UpdateFunc{\cl -> \copstxid[\cl][\repl](n,m+1) }) }
                & \If \lb =  \lbCOPSFinishRead{ \copsctx } \land \copsclenv(\cl) = \copstxid[\cl][\repl](n,m) ,
         \\ \ToCOPSProg{ \cops | \copsctxenv | \copsrunprog | \lb
                -> \COPSToExt(\copstrc, \copsclenv) }
                & \ow .
        \end{cases} \)
    }
\end{align*}
%where \( \COPSToExtW \) is defined by:
%\begin{align*}
%\COPSToExtW((\copsconfinit,\copsprog_0),(\ts,\repl),\cl) & \FuncDef (\copsconfinit,\copsprog_0)
%\\ \COPSToExtW( \begin{Bracketed} \ToCOPSProg{ \cops | \copsctxenv 
            %| \copsrunprog | \lb -> \copstrc } \end{Bracketed} , (\ts,\repl),\cl) & 
%\\ \multicolumn{2}{l}{
        %\( \quad \FuncDef 
        %\begin{cases}
        %a
        %\end{cases} \)
    %}
%\end{align*}
%Then the set of \emph{annotated normalised COPS traces}, \( \ExtCOPSTraces \ni \copsexttrc \), is defined by
%defined by 
%\[ 
    %\ExtCOPSTraces \TypeDef \Set{\COPSToExt(\copstrc,\copsclenv) | \copstrc \in \COPSTraces 
        %\\ {} \land \Exists{\copsctxenv \in \COPSContextEnvs} 
        %(\stub,\copsctxenv, \stub) = \First(\copstrc)
        %\\ \implies 
        %\Forall{\cl \in \Dom(\copsctxenv) }
        %\Forall{ \repl \in \COPSReplicas } 
        %\\ \copsctxenv(\cl) = (\stub,\repl) 
        %\implies \copsclenv(\cl) = \copstxid[\cl][\repl](0,1) 
    %} .
%\]
\end{definition}

\begin{toappendix}
\label{sec:proof-multiple-read-trans-id-unique}
\begin{proposition}[Fresh multiple-read transaction identifiers]
\label{prop:cops-read-id-unique}
Given an annotated normalised COPS trace \( \copsexttrc \in \ExtCOPSTraces \),
the annotated multiple-read transaction identifiers must be fresh with respect the key being read,
that is,
\begin{Formulae}*
\begin{Formula}
\copsexttrc = \ToCOPSProg{ \copsexttrc' | \stub 
                -> \cops |  \copsctxenv | \copsrunprog | \lbCOPSRefetch{\opR(\key,\val), \copstxid, \copsdep }, \copstxid'
                -> \copsexttrc'' }
        \\ \land
        \Forall{\repl^* \in \COPSReplicas | \cl^* \in \Clients | \val^* \in \Values | \copsdep^* \in \COPSDependencies }
        \\ \copsexttrc'' = \ToCOPSProg{\cdots | \lbCOPSRefetch[\cl^*](\repl^*){\opR(\key,\val^*), \copstxid^*, \copsdep^* }, \copstxid'' -> \cdots }
        \implies  \copstxid' \neq \copstxid''
\end{Formula}
\end{Formulae}
\end{proposition}
\begin{proof}
Suppose a replica \( \repl^* \), a client \( \cl^* \), a value \( \val^* \) and dependency set \( \copsdep^* \)
such that
\[
\copsexttrc'' = \ToCOPSProg{\cdots | \lbCOPSRefetch[\cl^*](\repl^*){\opR(\key,\val^*), \copstxid^*, \copsdep^* }, \copstxid'' -> \cdots }.
\]
If \( \repl \neq \repl^*\) or \( \cl \neq \cl^* \), 
by the definition of \(\ExtCOPSTraces\) (especially \(\COPSToExt\)), it is trivial that \(\copstxid \neq \copstxid''\).
Consider \( \repl = \repl^*\) or \( \cl = \cl^* \).
Because a client must provide a unique set of key \( \keyset \) when calling \( \pcopsread \)
by the rules in \cref{fig:cops-semantics-read},
it means that
\[
\copsexttrc'' = \ToCOPSProg{\cdots | \lbCOPSRefetch{ \copsdep' }, \copstxid'
    -> \cdots | \lbCOPSRefetch[\cl^*](\repl^*){\opR(\key,\val^*), \copstxid^*, \copsdep^* }, \copstxid'' -> \cdots }.
\]
For this case, by the definition of \(\ExtCOPSTraces\) (especially \(\COPSToExt\)), 
it must be the case that \(\copstxid \neq \copstxid''\).
\end{proof}
%\begin{proofsketch}
%It is trivial by the construction of annotated normalised traces.
%The full proof is given in \cref{sec:proof-multiple-read-trans-id-unique}
%on \pageref{sec:proof-multiple-read-trans-id-unique}.
%\end{proofsketch}
\end{toappendix}


%In our kv-store semantics, any new version is appended to the end of a list.
%However, in COPS semantics, any version is inserted into the list to preserve the version order of the new list.
%As explained before, write operations in a normalised COPS trace,
%henceforth in an annotated normalised COPS trace, must be in order.
%Thus, in a normalised COPS trace, 
%it is safe to \emph{append} any write operation to the end of the list 
%instead of inserting into the list.
%The detail is given in \cref{prop:cops-append-write} on page \pageref{sec:proof-append-write-op}.

\begin{toappendix}
\label{sec:proof-append-write-op}
\begin{proposition}[Appending write operations]
\label{prop:cops-append-write}
Given an annotated normalised COPS trace \( \copsexttrc \),
for every write operation, \( \lb = \lbCOPSWrite{\opW(\key,\val), \copstxid, \copsdep } \),
the version identifier, \(\copstxid \), is strictly greater than all version identifiers included in all replicas:
that is,
\begin{Formulae}*
\begin{Formula}
\copsexttrc = \ToCOPSProg{ \copsexttrc' | \stub 
                -> \cops |  \copsctxenv | \copsrunprog | \lbCOPSWrite{\opW(\key,\val), \copstxid, \copsdep }
                -> \cops' | \copsctxenv' | \copsrunprog' | \stub
                -> \cdots }
        \\ \land 
        \Forall{\repl' \in \COPSReplicas | \copskvs' \in \COPSKVSs | \key' \in \Keys | \idx' \in \Indexs}
        \\ \cops(\repl') = (\copskvs',\stub)
        \land 0 \leq \idx < \copskvs(\key')
        \implies  \WtOf(\copskvs'(\key',\idx')) \copstxidle \copstxid
\end{Formula}
\end{Formulae}
\end{proposition}
\begin{proof}
Suppose a replica \( \repl' \) with key-value store \( \copskvs' \) (\( \cops(\repl') = (\copskvs',\stub)\)), a key \( \key' \)
and an index \( \idx' \) such that \( 0 \leq \idx < \copskvs(\key') \).
Let the \( \copstxid' = \WtOf(\copskvs(\key',\idx'))\).
There must exist a step with label \( \lb' \) with some client \( \cl' \)
in the trace \( \copsexttrc' \) that committed this version:
\[ 
\copsexttrc' = \ToCOPSProg{ \cdots | \lb' -> \cdots }
\land \lb' = \lbCOPSWrite{\opW(\key',\val'), \copstxid', \copsdep } 
\]
for some value \( \val ' \) and dependent set \(\copsdep \).
Since an annotated normalised trace \( \copsexttrc \) is also in its normalised form, that is, \( \COPSNormalTrace(\copsexttrc) \),
and the new kv-store \( \cops'(\repl')\Proj{0} \) must be a well-formed COPS key-value store,
therefore \( \copstxid' \copstxidle \copstxid \).
\end{proof}
%\begin{proofsketch}
%It is trivial by the definition of normalised COPS traces (\cref{def:cops-normal-trace}),
%therefore it also holds for annotated normalised COPS traces.
%The full proof is given in \cref{sec:proof-append-write-op}
%on \pageref{sec:proof-append-write-op}.
%\end{proofsketch}
\end{toappendix}
