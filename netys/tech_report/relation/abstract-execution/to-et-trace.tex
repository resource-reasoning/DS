Inversely, we show to construct, given an abstract execution \(\aexec\), 
a set of \(\et[\TOP]\)-traces in its normal form such that any such trace
\(\ettrc\) satisfies \(\LastConf(\ettrc)\Proj{0} = \XToK(\aexec)\). 
%We first define the \( \AexecCut(\aexec,n) \) that
%gives the prefix of the first \( n \) transactions of the abstract execution \( \aexec \).

\begin{definition}[Abstract Executions to {\( \et[\TOP]\)}-traces]
\label{def:aexec-to-et-top-trace}
Given an abstract execution \( \aexec \in \AbstractExecutions \) 
and an index \( \idx \in \Indexs \),
the set of \( \et[\TOP] \)-traces induced by the abstract execution, \( \XToTrace(\aexec) \),
is defined by:
\[
    \XToTrace(\aexec) = \XToTraceN(\aexec,\Abs{\aexec} - 1) ,
\]
where, given definition of \( \AexecCut \) in \cref{def:aexec-cut},
\( \XToTraceN(\aexec,\Abs{\aexec})  \) is defined by:
\begin{align*}
    \XToTraceN(\aexec, 0) 
            & \FuncDef \Set{ \ToRed{\kvsinit | \vienvinit} |
                            \Forall{\cl \in \Dom(\vienvinit) } \vienvinit(\cl) = \Set{0} }
    \\  \XToTraceN(\aexec, \idx + 1) & 
            \FuncDef \begin{array}[t]{@{}l@{}}
                \Let 
                    \aexec' = \AexecCut(\aexec,\idx), \aexec'' = \AexecCut(\aexec,\idx + 1) ,
                \\ \LetQuad 
                    \txid[\cl](n) = \aexec'' \setminus \aexec' ,
                    \ettrc \in \XToTraceN(\aexec, \idx) ,
                    (\kvs,\vienv) = \LastConf(\ettrc) ,
                \\ \LetQuad \fp = \aexec(\txid[\cl](n)) ,
                   \kvs' = \UpdateKV(\kvs,\vi,\fp,\txid[\cl](n)) ,
                \\ \AND \vi = \GetView(\aexec,\VISInv(\txid[\cl](n))) \ \In
                \\  {\Set{ \ToRed{\ettrc | \lbView{\vi} 
                            -> \kvs | \vienv\UpdateFunc{\cl -> \vi}  } 
                          \\[-15pt] \qquad \ToRed{ | \lbFp{\fp} -> \kvs' | \vienv\UpdateFunc{\cl -> \vi'} } 
                                    | \vi' \in \ApproxView(\aexec,\idx+1,\txid[\cl](n)) }}
            \end{array}
    \\ \ApproxView(\aexec,\idx,\txid[\cl](n)) 
    & \FuncDef 
    \begin{multlined}[t]
    \Let \aexec' = \AexecCut(\aexec,\idx) \In \
    \\ \begin{cases}
        \Set{\GetView(\aexec', \txidset) | 
                \txidset \subseteq \VISInv[\aexec](\txid[\cl](m)) \cap \Refl((\ARInv[\aexec]))(\txid[\cl](n))} \\
                        \multicolumn{2}{r}{\If \txid[\cl](m) \in \aexec \land m > n \land 
                        \Forall{\txid} \neg\left(\ToEdge{\txid[\cl](n) | \SO 
                                -> \txid | \SO -> \txid[\cl](m) }\right)}
        \\ \Views(\XToK(\aexec')) & \ow
    \end{cases}
    \end{multlined}
\end{align*}
\end{definition}

Given an abstract execution \( \aexec \), we can construct an \( \ET[\top] \)-trace inductively in the order of \( \AR \).
For each transaction \( \txid = \txid[\cl](n) \) from a client \( \cl \) in the abstract execution \( \aexec \),
there is an view-shift transition and a fingerprint transition in the trace \( \ettrc \).
Note that in the inductively case \Th{(i+1)}, 
the next transaction is given by the \( \aexec'' \setminus \aexec' \), 
where \( \aexec'' \) and \( \aexec' \) are the \Th{(i+1)} and \Th{i} cuts respectively.
The corresponding view-shift transition advances the view of \( \cl \) to 
a new view \( \vi \) that extracts from the set of visible transactions of \( \txid \), 
that is, \( \vi = \GetView(\aexec,\VISInv(\txid[\cl](n))) \).
The fingerprint transition simply commits the fingerprint \( \aexec(\txid[\cl](n)) \) in the sense that
it updates the kv-store to \( \kvs' \) and shift the view to a new view \( \vi' \) afterwards.
The abstract execution does not contains the precise information about the view \( \vi' \) after commit, 
but we can approximate this view using \( \ApproxView \) function,
that is:
\begin{enumerate*}
\item if there is more transactions from the same client \( \cl \),
the view after commit can be extracted from the intersection of 
the visible set of the immediate next transaction of \( \cl \), that is, \( \VISInv[\aexec](\txid[\cl](m)) \),
and the set of transactions committed before \( \txid[\cl](n) \), 
that is, \( \Refl((\ARInv[\aexec]))(\txid[\cl](n)) \); and
\item otherwise, the view after commit can be any valid view on \( \kvs' \).
\end{enumerate*}
It is easy to see that \( \ApproxView \) always returns a set of valid view.
The detail is given in  \cref{prop:well-defined-approx-view}  on page \pageref{sec:proof-approx-view}.
Given an abstract execution, any trace \(\ettrc \) in \( \XToTrace(\aexec) \) is a valid \( \et[\TOP] \)-trace,
and the kv-store \(\kvs \)  in the final configuration can be directly extracted from \( \aexec \) 
in the sense that \( \kvs = \XToK(\aexec) \).

\begin{toappendix}
\label{sec:proof-approx-view}
\begin{proposition}[Well-defined {\(\ApproxView\)}]
\label{prop:well-defined-approx-view}
Given an abstract execution \( \aexec \),
let \( \aexec' = \AexecCut(\aexec,\idx) \) and 
\( \vi = \ApproxView(\aexec,\idx,\txid) \) for some number \( \idx \) and transaction \( \txid \).
If \( \txid \) is the last transaction with respect to arbitration order, \( \txid = \Max[\AR[\aexec']](\aexec')\),
then the view \( \vi \in \Views(\XToK(\aexec')) \).
\end{proposition}
\begin{proof}
It is trivial from the defintion of \(\ApproxView\).
\end{proof}
\end{toappendix}

\begin{toappendix}
\label{sec:proof-well-formed-aexec-to-et-trace}
\end{toappendix}
\begin{theoremrep}[Abstract executions to well-formed {\( \et[\TOP]\)}-traces]
\label{thm:aexec-to-et-top-trace}
Given an abstract execution \( \aexec \),
any trace \( \ettrc \in \XToTrace(\aexec) \) is a valid {\( \et[\TOP]\)}-trace
and the finial kv-store \( \kvs \) such that \( (\kvs,\stub) = \LastConf(\ettrc) \) satisfies
that \( \kvs = \XToK(\aexec) \).
\end{theoremrep}
\begin{proof}
Given the defintion of \( \XToTrace \), we prove a stronger result that,
for any number \( \idx \), kv-store \( \kvs \), view environment \( \vienv \) and trace \( \ettrc \),
\begin{Formulae}
\begin{Formula}
\ettrc = \XToTraceN(\aexec,\idx) \land (\kvs,\vienv) = \LastConf(\ettrc) 
        \land \kvs = \XToK(\AexecCut(\aexec,\idx))
\label{equ:inv-aexec-to-et-trace}
\end{Formula}
\end{Formulae}
We prove by induction on the number \( \idx \).
\begin{enumerate}
\CaseBase{\( \idx = 0 \)}
    Any trace \( \ettrc \in \XToTraceN(\aexec,0) \) is of the form \( (\kvsinit, \vienvinit) \).
    By defintion of \( \AexecCut \), it follows \( \kvsinit = \XToK(\AexecCut(\aexec,0)) \).
\CaseBase{\( \idx > 0 \)}
    Suppose that \cref{equ:inv-aexec-to-et-trace} holds for \( \idx - 1 \) and consider \( \idx \).
    Let \( \aexec' = \AexecCut(\aexec,\idx-1) \) and \( \aexec'' = \AexecCut(\aexec,\idx) \).
    Assume the new transaction \( \txid[\cl](n) = \aexec'' \setminus \aexec' \) 
    and its fingerprint \( \fp = \aexec(\txid[\cl](n)) \).
    By \ih, assume a valid \(\et[\TOP] \)-trace \( \ettrc \in \XToTraceN(\aexec,\idx) \)
    and its last configuration \( (\kvs,\vienv) = \LastConf(\ettrc)  \).
    Let the view \( \vi  = \GetView(\aexec,\VISInv(\txid[\cl](n))) \),
    the new kv-store \( \kvs' = \UpdateKV(\kvs,\vi,\fp,\txid[\cl](n)) \),
    and the new view \( \vi' = \ApproxView(\aexec,\idx, \txid[\cl](n)) \).
    Therefore the new trace \( \ettrc' \) is of the form
    \[
        \ettrc' = \ToRed{\ettrc | \lbView{\vi} 
                -> \kvs | \vienv\UpdateFunc{\cl -> \vi} | \lbFp{\fp}
                -> \kvs' | \vienv\UpdateFunc{\cl -> \vi'} }
    \]
    We now prove the two new steps, \( \lbView{\vi}\) and \( \lbFp{\fp} \), satisfy
    the definition of \( \et[\TOP]\)-trace in \cref{def:et-reduction}.
    \begin{enumerate}
    \Case{\( \lbView{\vi} \)}
        By the definition of \( \XToTraceN\), there must exist a transaction \( \txid[\cl](l)\)
        for the client \( \cl \) such that
        \[
            \begin{multlined}[t]
            \ettrc = \ToRed{\cdots | \stub
                            -> \kvs^* | \vienv^* }
                \\ \ToRed{ {} |  \lbFp{\fp^*} -> \UpdateKV(\kvs^*, \vienv^*(\cl), \fp^*, \txid[\cl](l))
                                        | \vienv^*\UpdateFunc{\cl -> \vi^*} | \stub 
                            -> \ettrc^{**} }
            \\ {} \land \Forall{ j } 0 \leq j < \Abs{\ettrc^{**}} \land (\ettrc'\Proj{j})\Proj{1}(\cl) = \vi^* ,
            \end{multlined}
        \]
        which means that \( \txid[\cl](m)\) is the last transaction for \( \cl \) in the trace \( \ettrc \).
        Again, by the definitions of \( \XToTraceN \) and \( \ApproxView \), 
        for the view \( \vi \), there exists a set of transactions \( \txidset \) such that,
        \[
            \vi^* = \GetView(\aexec,\txidset) 
            \land \txidset \subseteq \VISInv[\aexec](\txid[\cl](n)) \cap \Refl((\ARInv[\aexec]))(\txid[\cl](j)) .
        \]
        Note that \(\vi^*\) must be a valid view because of \cref{prop:well-defined-approx-view}.
        Last, by definition of \( \GetView \),
        \[
            \vi^* = \GetView(\aexec,\txidset) \subseteq \GetView(\aexec,\VISInv[\aexec](\txid[\cl](n))) = \vi
        \]
    \Case{\( \lbFp{\fp} \) and \( \kvs = \XToK(\AexecCut(\aexec,\idx)) \)}
        We prove that the new kv-store \( \kvs' \) is well-formed,
        \( \kvs = \XToK(\aexec'') \) and 
        \( \ToET[\TOP]{\kvs | \vi | \fp | \kvs' | \vi' } \).
        The first and third sub-goals imply that \( \lbFp{\fp} \) is a valid \( \et[\TOP]\) step.
        \begin{enumerate}
        \Case{Well-formed \( \kvs' \)} 
            Consider view \( \vi \) defined by \( \vi = \GetView(\aexec,\VISInv[\aexec](\txid[\cl](n)))\).
            Since \( \txid[\cl](n) = \Max[\AR[\aexec'']](\aexec'')\) and \( \aexec'' \) is well-formed,
            then \( \VISInv[\aexec](\txid[\cl](n)) \subseteq \aexec' \),
            which means that 
            all visible transactions must exist in \( \aexec' = \AexecCut(\aexec, \idx - 1) \).
            By \ih, \( \kvs =  \XToK(\aexec')  \) and therefore \( \vi \in \Views(\vi) \).
            It is easy to see that \( \txid[\cl](n) \notin \kvs \) and \( \fp \in \Fingerprints \);
            thus by \cref{thm:well-deifned-updatekv}, the new kv-store \( \kvs' \) is well-formed.
        \Case{\( \kvs' = \XToK(\aexec'') \)}
            Let \( \txidset =  \VISInv[\aexec](\txid[\cl](n)) \).
            Previously, 
            we known that \( \vi = \GetView(\aexec,\txidset) \) and \( \txidset \subseteq \aexec' \);
            therefore \( \vi = \GetView(\aexec',\txidset) \).
            By \cref{prop:aexec-cut-to-update}, \( \aexec'' = \UpdateAExec(\aexec',\txidset, \fp,\txid[\cl](n))\).
            By \ih, \( \kvs = \XToK(\aexec') \).
            Then by \cref{prop:update-aexec-to-udpate-kv}, \( \kvs' = \XToK(\aexec'') \).
        \Case{\( \ToET[\TOP]{\kvs | \vi | \fp | \kvs' | \vi' } \)}
            Previously, we proved \( \vi \in \Views(\kvs)\) and \( \kvs' = \XToK(\aexec'')\).
            Then by \cref{prop:well-defined-approx-view}, we know that \( \vi' \in \Views(\kvs')\).
            Because \( \CanCommit[\TOP] \) and \( \ViewShift[\TOP]\) are simply true,
            it remains to prove that every read operation \( \opR(\key,\val) \) agrees with the view \( \vi \).
            Note that by \cref{prop:aexec-cut-to-update}, 
            \( \aexec'' = \UpdateAExec(\aexec',\VISInv[\aexec](\txid[\cl](n)),\fp,\txid[\cl](n))\).
            Given the definition of \(\UpdateAExec \),
            \[
                \begin{multlined}[t]
                    \opR(\key,\val) \in \aexec''(\txid[\cl](n)) \iff 
                    \\ \Exists{\txid \in \aexec'}
                    \txid = \MaxVisTrans(\aexec',\VISInv[\aexec](\txid[\cl](n)),\key)
                    \land \opW(\key,\val) \in \aexec'(\txid) .
                \end{multlined}
            \]
            Because \( \vi = \GetView(\aexec,\VISInv[\aexec](\txid[\cl](n))) \), then
            \[
                \begin{multlined}[t]
                    \opR(\key,\val) \in \aexec''(\txid[\cl](n)) \iff 
                    \ValOf(\kvs(\key,\Max[<](\vi(\key)))) = \val .
                \end{multlined}
            \] 
            Given above we proved \( \ToET[\TOP]{\kvs | \vi | \fp | \kvs' | \vi' } \). \qedhere
        \end{enumerate}
    \end{enumerate}
\end{enumerate}
\end{proof}
\begin{proofsketch}
Given the definition of \( \XToTrace \), we prove a stronger result that,
for any number \( \idx \), kv-store \( \kvs \), view environment \( \vienv \) and trace \( \ettrc \),
\begin{Formulae}*
\begin{Formula}
\ettrc = \XToTraceN(\aexec,\idx) \land (\kvs,\vienv) = \LastConf(\ettrc) 
        \land \kvs = \XToK(\AexecCut(\aexec,\idx)) .
\end{Formula}
\end{Formulae}
This result can be proved by induction on the number \( \idx \).
Note that in each inductive step,
there are two more transitions, a view-shift transition and a fingerprint transition.
Assume the next transaction is identified by \( \txid \), from a client \( \cl \).
For the view-shift transition, we prove that the new view,
which is the view induced by the visible transactions of \( \txid \),
contains more versions than the view of \( \cl \).
This can be derived from the definition of \( \ApproxView \).
For the fingerprint transition,
we prove the new kv-store \( \kvs \) by committing \( \txid \) 
is equivalent to the \Th{\idx} cut of \( \aexec \), 
that is,   \( \kvs = \XToK(\AexecCut(\aexec,\idx)) \).
This can be proven by the inductive case of the definition of  \( \XToTrace \).
The full proof is given in \cref{sec:proof-well-formed-aexec-to-et-trace} 
on page \pageref{sec:proof-well-formed-aexec-to-et-trace}.
\end{proofsketch}
