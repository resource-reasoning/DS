We prove all the execution tests in \cref{sec:consistency-model-on-kv}
are sound and complete with the corresponding set of visibility axioms.
We first prove \cref{thm:view-vis-relation} stating that 
the prefix the closure property on views on kv-stores 
matches the closure property of visibility relation on abstract executions.

\begin{theorem}[View closure equal to visibility closure]
\label{thm:view-vis-relation}
Assume \( \kvs \) and \( \aexec \) such that \( \kvs = \XToK(\aexec) \).
Assume relations \( \rel[\kvs] \) and \( \rel[\aexec] \) such that \( \rel[\kvs] = \rel[\aexec] \).
For any transaction \(\txid \) and fingerprint \( \fp \), view \( \vi \in \Views(\kvs) \)
and set of transaction \( \txidset \subseteq \aexec\),
\begin{Formulae}
    & \begin{Formula} 
        \PreClosed(\kvs,\vi,\rel[\kvs]) \land \txidset = \VisTrans(\kvs,\vi) 
            \\ \implies \Exists{\txidsetrd \subseteq 
                    \Set{\txid | \txid \in \aexec \land \Forall{l, \key, \val } 
                                        (l,\key,\val) \in \aexec(\txid) \implies l = \opR } }
            ( \txidset \cup \txidsetrd ) \subseteq \rel[\aexec]^{-1}(\txidset \cup \txidsetrd) ,
        \label{equ:view-close-to-aexec}
    \end{Formula}
    \\ & \begin{Formula} 
        \txidset \subseteq \rel[\aexec]^{-1}(\txidset) \land \vi = \GetView(\aexec,\txidset)
            \implies \PreClosed(\kvs,\vi,\rel[\kvs]) .
        \label{equ:aexec-close-to-view}
    \end{Formula}
\end{Formulae}
%%%%%%%%%%%%%%% maybe useful to push further
%if there is a view \( \vi = \getView[\mkvs,\left(\rel_\mkvs^{-1}\right)^{*}(\Tx[\mkvs,\vi])] \),
%then the new abstract execution \( \aexec' = \extend[\aexec, \txid, \fp, \left(\rel_\mkvs^{-1}\right)^{*}(\Tx[\mkvs,\vi])] \)
%satisfies \( \rel_\aexec^{-1}(\VIS_{\aexec'}^{-1}(\txid)) \subseteq \VIS_{\aexec'}^{-1}(\txid) \).
%Conversely,
%If there a new abstract execution \( \aexec' = \extend[\aexec, \txid, \fp, \txidset] \) for some \( \txidset \)
%that satisfies \( \rel_\aexec^{-1}(\txidset) \subseteq \txidset \),
%and if a view \( \vi = \getView[\mkvs,\txidset] \),
%then the view \( \vi = \getView[\mkvs,\left(\rel_\mkvs^{-1}\right)^{*}(\Tx[\mkvs,\vi])] \).
%%%%%%%%%%%%%%% maybe useful to push further
\end{theorem}
\begin{proof}
We prove \cref{equ:view-close-to-aexec,equ:aexec-close-to-view} respectively.
\begin{enumerate}
\Case{\cref{equ:view-close-to-aexec}}
    Recall \( \txidset = \VisTrans(\kvs,\vi) \).
    Let the set of read-only transactions 
    \( \txidsetrd = \Inv((\rel[\kvs]^*)) (\VisTrans(\kvs, \vi)) \setminus \txidset \).
    Recall the definition of \( \PreClosed \) that
    \[
        \begin{multlined}[t]
        \VisTrans(\kvs, \vi) = 
        \\ \left( \Inv((\rel[\kvs]^*)) (\VisTrans(\kvs, \vi)) 
                \setminus \Set{\txid | \txid \in \kvs \land \Forall{\key | \idx} \txid \neq \WtOf(\kvs(\key,\idx))}  \right) .
        \end{multlined}
    \]
    It follows that
    \( \txidset = \left( \Inv((\rel[\kvs]^*))(\txidset) \setminus \txidsetrd \right) \) 
    and by \( \rel[\kvs] = \rel[\aexec] \), then
    \( \txidset \cup \txidsetrd = \Inv((\rel[\kvs]^*))(\txidset) \).
    For any transaction \( \txid \in \txidsetrd \), it must be the case that 
    \( (\txid, \txid') \in (\rel[\kvs]^*) \) and \( \txid' \in \txidset \);
    therefore \( \txidset \cup \txidsetrd = \Inv((\rel[\kvs]^*))(\txidset \cup \txidsetrd) \).
    %%%%%%%%%%%%%%% maybe useful to push further
    %By the definition of \( \extend \),  the visible transactions 
    %\( \VIS_{\aexec'}^{-1}(\txid) = \txidset \).
    %Let consider transactions \( \txid', \txid'' \) such that \( \txid' \toEDGE{\rel_{\aexec}} \txid'' \toEDGE{\VIS_{\aexec'}} \txid \).
    %This means there exists a natural number \( n \) such that 
    %\( \txid'' \in \left(\rel_\mkvs^{-1}\right)^{n}(\Tx[\mkvs,\vi])\).
    %Given that \( \rel_\mkvs = \rel_\aexec \), it follows \( \txid' \in \left(\rel_\mkvs^{-1}\right)^{n + 1}(\Tx[\mkvs,\vi])\), 
    %then \( \txid' \in \txidset \) and so \( \txid' \toEDGE{\VIS_{\aexec'}} \txid \).
    %%%%%%%%%%%%%%% maybe useful to push further

\Case{\cref{equ:aexec-close-to-view}}
    Let the set of visible transactions \( \txidset' = \VisTrans(\kvs,\vi) \)
    and the set of all read-only transactions 
    \( \txidsetrd = \Set{\txid | \txid \in \kvs \land \Forall{\key | \idx} \txid \neq \WtOf(\kvs(\key,\idx))} \).
    Since \( \vi = \GetView(\aexec,\txidset) \) then \( \txidset' \subseteq \txidset \setminus \txidsetrd \).
    By the definition \( \PreClosed \), it suffices to prove the following result,
    \[
        \txidset' =  \Inv((\rel[\kvs]^*))(\txidset') \setminus \txidsetrd 
    \]
    It is trivially \( \txidset' \subseteq \left( \Inv((\rel[\kvs]^{*}))(\txidset') \setminus \txidsetrd \right) \)
    then it remains to prove that 
    \( \left( \Inv((\rel[\kvs]^{*}))(\txidset') \setminus \txidsetrd \right) \subseteq \txidset' \).
    Since \( \rel[\kvs]^{*} = \bigcup_{\idx \in \Nat} \rel[\kvs]^{\idx} \),
    Now we prove a stronger result as the following
    \[
        \left( \Inv((\rel[\kvs]^*))(\txidset') \setminus \txidsetrd \right) \subseteq \txidset' 
        \land \Inv((\rel[\kvs]^*))(\txidset') \in \txidset ,
    \]
    by induction on \( \idx \).
    \begin{enumerate}
    \CaseBase{\( \idx = 0 \)}
    Since \( \txidset' = \txidset \setminus \txidsetrd \),
    it is trivial that 
    \( \left( \Inv((\rel[\kvs]^{0}))(\txidset') \setminus \txidsetrd \right) \subseteq \txidset' \)
    and \( \Inv((\rel[\kvs]^{0}))(\txidset') \subseteq \txidset \).
    \CaseInd{\( \idx > 0 \)}
    Suppose \( \left( \Inv((\rel[\kvs]^{\idx - 1}))(\txidset') \setminus \txidsetrd \right) \subseteq \txidset' \)
    and consider \( \left( \Inv((\rel[\kvs]^{\idx}))(\txidset') \setminus \txidsetrd \right) \subseteq \txidset' \).
    Assume two transactions \( \txid, \txid' \)  such that \( \txid \in \txidset' \)
    and \( (\txid',\txid) \in \rel[\kvs]^{\idx} \).
    Note that \( \txid \) must have write since \( \txid \in \txidset' \).
    There must exist \( \txid'' \) such that \( \ToEdge{\txid' | \rel[\kvs] 
                    -> \txid'' | \rel[\kvs]^{\idx - 1} -> \txid } \).
    By \ih, \( \txid'' \in \txidset \).
    Since \( \rel[\kvs] = \rel[\aexec] \) and \( \txidset \subseteq \rel[\aexec]^{-1}(\txidset) \),
    then \( \txid' \in \txidset \).
    If \( \txid' \) wrote to a key, then \( \txid' \in \txidset' \).
    Otherwise \( \txid' \) is a read only transaction and \( \txid' \in \txidsetrd \).
    Thus \( \left( \Inv((\rel[\kvs]^{\idx}))(\txidset') \setminus \txidsetrd \right) \subseteq \txidset' \)
    and \( \Inv((\rel[\kvs]^{\idx}))(\txidset') \subseteq \txidset \). \qedhere
    \end{enumerate}

    %%%%%%%%%%%%%%% maybe useful to push further
    %Assume there a new abstract execution \( \aexec' = \extend[\aexec, \txid, \fp, \txidset] \),
    %that satisfies \( \rel_\aexec^{-1}(\txidset) \subseteq \txidset \).
    %Assume \( \vi = \getView[\mkvs,\txidset] \).
    %%By the definition of \( \getView \), we know \( \txidset = \Tx[\mkvs,\getView[\mkvs,\txidset]] \cup \txidset_\rd \) 
    %%for some read-only transactions  \( \txidset_\rd \).
    %%This means \( \Tx[\mkvs, \vi] \cup \txidset_\rd = \mu X. \Tx[\mkvs, \vi] \cup \txidset_\rd \cup \rel_\mkvs^{-1}(X)\).
    %Note that \( \rel_\aexec = \rel_\mkvs \).
    %It suffices to prove
    %\( \Set{\txid' \in \txidset }[\txid' \ \text{has writes}] = \Set{ \txid' \in \left(\rel_\mkvs^{-1}\right)^{*}(\Tx[\mkvs,\vi]) }[\txid' \ \text{has writes}]\).

    %\begin{itemize}
    %\item Assume a transaction \( \txid' \in \txidset \) that has writes.
    %It is easy to see there are \( \key,i \) such that \( i \in \vi(\key) \) and
    %\( \wtOf[\mkvs(\key,i)] \in \txidset \).
    %This means \( \txid' \in \Tx[\mkvs,\vi] \).
    %\item Assume a transaction \( \txid' \in \Tx[\mkvs,\vi] \),
    %we now prove \( \left(\rel_\mkvs^{-1}\right)^{n}(\txid') \subseteq \txidset \) for all \( n \).
    %\begin{itemize}
        %\item \caseB{n = 0} It trivially holds that  \( \txid' \in \Tx[\mkvs,\vi]  \subseteq \txidset \).
        %\item \caseI{n + 1} 
            %Assume a transaction \( \txid''' \in \left(\rel_\mkvs^{-1}\right)^{n + 1}(\txid') \).
            %It means there is a \( \txid'' \in \left(\rel_\mkvs^{-1}\right)^{n}(\txid') \) such that \( \txid''' \toEDGE{\rel_\mkvs} \txid'' \).
            %By \ih, \( \txid'' \in \txidset \).
            %Given \( \rel_\mkvs = \rel_\aexec \) and \( \rel_\aexec^{-1}(\txidset) \subseteq \txidset\),
            %it is known that \( \txid''' \in \txidset \).
    %\end{itemize}
    %\end{itemize}
    %%%%%%%%%%%%%%% maybe useful to push further
\end{enumerate}
\end{proof}

We prove all the execution tests in \cref{sec:consistency-model-on-kv}
are sound and complete with the corresponding set of visibility axioms:\\
\begin{center}
\begin{tabular}{l c || l c || l c || l c || l c }
\hline
   \MR  & page \pageref{sec:sound-complete-mr}
&  \RYW & page \pageref{sec:sound-complete-ryw} 
&  \MW  & page \pageref{sec:sound-complete-mw} 
&  \WFR & page \pageref{sec:sound-complete-wfr} 
&  \CC  & page \pageref{sec:sound-complete-cc} 
\\ \UA  & page \pageref{sec:sound-complete-ua} 
&  \CP  & page \pageref{sec:sound-complete-cp} 
&  \PSI & page \pageref{sec:sound-complete-psi} 
&  \SI  & page \pageref{sec:sound-complete-si} 
&  \SER & page \pageref{sec:sound-complete-ser} 
\\ \hline
\end{tabular}
\end{center}
