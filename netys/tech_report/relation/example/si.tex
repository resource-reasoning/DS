An abstract execution \( \aexec \) satisfies snapshot isolation (\(\SI\)), 
if it satisfies
\( \{\lambda \aexec. \AR[\aexec] ; \VIS[\aexec], \lambda \aexec \ldotp \SO, \allowbreak \lambda \aexec. \WW[\aexec] \}) \),
which is intersection of \( \CP \) and \( \UA \) on abstract executions \citep{SIanalysis}.
\citet{SIanalysis} also proposed the minimum visibility relation that gives rise of the following equivalent definition
\[
    \visaxioms[\SI] \FuncDef
    \Set{\lambda \aexec. \left( (\WR[\aexec]  \cup \SO \cup \WW[\aexec] ) ; \Refl(\RW[\aexec]) \right) ; \VIS[\aexec]
            ,\lambda \aexec \ldotp \SO, \lambda \aexec \ldotp \WW[\aexec] }  .
\]

The execution test \( \et[\CP] \) is sound with respect to the axiomatic definition \( \visaxioms[\SI] \)
We pick the invariant \( \aexecinv[\CP] = \aexecinv[\RYW]\).
\SOUNDLET{\CP}{ \txidsetrd \supseteq
\begin{multlined}[t]
\left( \bigcup_{\Set{\txid[\cl](\idx) | \txid[\cl](\idx) \in \aexec}} 
\VISInv[\aexec](\txid[\cl](\idx)) \cup \Refl((\Inv(\SO)))(\txid[\cl](\idx)) \right) 
\setminus \Set{\txid' | \Forall{l | \key | \val } (l,\key,\val) \in \aexec(\txid') \implies l = \opR } .
\end{multlined} }
Assume 
\[ 
\txidsetrd' = 
\begin{multlined}[t]
\left( \bigcup_{\Set{\txid[\cl](\idx) | \txid[\cl](\idx) \in \aexec}} 
\VISInv[\aexec](\txid[\cl](\idx)) \cup \Refl((\Inv(\SO)))(\txid[\cl](\idx)) \right) 
\setminus \Set{\txid' | \Forall{l | \key | \val } (l,\key,\val) \in \aexec(\txid') \implies l = \opR } .
\end{multlined} 
\]
and \( \txidsetrd'' = \txidsetrd \setminus \txidsetrd' \).
By the definition of soundness, we prove the following result:
\begin{Formulae}
& \begin{Formula}
    \Inv(\SO)(\txid) \subseteq \txidset \cup \txidsetrd' 
    \label{equ:si-sound-update-so}
\end{Formula}
\\ & \begin{Formula}
    \Inv(\WW)(\txid) \subseteq \txidset 
    \label{equ:si-sound-update-ua}
\end{Formula}
\\ & \begin{Formula}
    \Inv(\left( (\WR[\aexec] \cup \SO \cup \WW[\aexec] ) ; \Refl(\RW[\aexec]) \right)) (\txid) 
            \subseteq \txidset \cup \txidsetrd' \cup \txidsetrd''
     \label{equ:si-sound-update-closure}
\end{Formula}
\\ & \begin{Formula}
    \aexecinv[\PSI](\aexec',\cl) \subseteq \VisTrans(\XToK(\aexec'),\vi')
    \label{equ:si-inv-preserve}
\end{Formula}
\end{Formulae}
\Cref{equ:si-sound-update-so,equ:si-sound-update-ua} 
can be proven in the same way as in \cref{sec:sound-complete-mr,sec:sound-complete-ua} respectively.
We now prove \cref{equ:si-sound-update-closure}.
Initially we take \( \txidsetrd'' \) to be an empty set.
Note that \(\VISInv[\aexec'](\txid) = \txidset \cup \txidsetrd' \cup \txidsetrd'' \).
By \cref{thm:view-vis-relation,equ:view-close-to-aexec}, there exists \( \txidsetrd'' \) such that
\( \txidset \cup \txidsetrd'' \) is closed under \( \left( (\WR[\aexec] \cup \SO \cup \WW[\aexec] ) ; \Refl(\RW[\aexec]) \right)\).
Now consider a transaction \( \txidrd \in \txidsetrd' \) and
assume a transaction \( \txid' \) such that \( \ToEdge{ \txid' | (\WR[\aexec] \cup \SO \cup \WW[\aexec] ) ; \Refl(\RW[\aexec]) -> \txidrd } \).
Since \( \txidrd \) is a read-only transaction, thus
\( \ToEdge{ \txid' |  (\SO \cup \WR[\aexec] )  -> \txidrd } \)
and the rest proof is exactly the same as as in \cref{sec:sound-complete-cc}.
Last, \cref{equ:si-inv-preserve} can be proven in the same way as in \cref{sec:sound-complete-mr,sec:sound-complete-ryw}.

The execution test $\et[\SI]$ is complete with respect to the axiomatic definition \( \visaxioms[\SI] \).
By \citet{SIanalysis}, it suffices to prove completeness with respect to the following definition,
\[
\Set{\lambda \aexec \ldotp \AR[\aexec] ; \VIS[\aexec], \lambda \aexec \ldotp \SO , \lambda \aexec \ldotp \WW[\aexec] } .
\]
\COMPLETELET{\SI}
By the definition of \( \et[\SI]\), we prove \( \CanCommit[\SI] \), \( \ViewShift[\MR]\) and \( \ViewShift[\RYW]\) respectively.
Recall that \( \CanCommit[\SI] = \PreClosed(\kvs,\vi,\rel[\UA] \cup \left( (\WR[\aexec] \cup \SO \cup \WW[\aexec] ) ; \Refl(\RW[\aexec]) \right)) \).
It is easy to see that 
\[
\begin{multlined}
\PreClosed(\kvs,\vi,\rel[\UA] \cup \WR[\kvs] \cup \SO \cup \WW[\kvs]) \iff {}
    \\ \PreClosed(\kvs,\vi,\rel[\UA]) 
    \land \PreClosed(\kvs,\vi,\left( (\WR[\aexec] \cup \SO \cup \WW[\aexec] ) ; \Refl(\RW[\aexec]) \right)) . 
\end{multlined}
\]
The predicate \( \PreClosed(\kvs,\vi,\rel[\UA]) \) can be proven in the same way as in \cref{sec:sound-complete-ua}
Because
\begin{align*}
\left( (\WR[\aexec] \cup \SO \cup \WW[\aexec]) ; \Refl(\RW[\aexec])  \right) ; \VIS[\aexec] 
        & \subseteq \left( \VIS[\aexec] ; \Refl(\RW[\aexec]) \right) ;  \VIS[\aexec]
        & \subseteq  \AR[\aexec] ; \VIS[\aexec] \subseteq \VIS[\aexec] .
\end{align*}
Then \( \PreClosed(\kvs,\vi,\left( (\WR[\aexec] \cup \SO \cup \WW[\aexec] ) ; \Refl(\RW[\aexec]) \right)) \)
can be derived from \cref{thm:view-vis-relation,equ:aexec-close-to-view}.
The predicate \( \ViewShift[\RYW] \) can be proven in the same way as in \cref{sec:sound-complete-ryw}.
Since \( \VIS[\aexec] ; \SO \subseteq \AR[\aexec] ; \VIS[\aexec] \subseteq \VIS[\aexec] \)
\( \ViewShift[\MR] \) can be proven in the same way as in \cref{sec:sound-complete-mr}.
