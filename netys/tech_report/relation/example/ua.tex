The execution test \(\et[\UA]\) is sound with respect to 
the axiomatic definition \( \visaxioms[\UA] = \Set{\lambda \aexec. \WW[\aexec] } \).
We pick the invariant as \( \aexecinv[\UA]( \aexec, \cl ) = \emptyset \), 
given the fact of no constraint on the final view.
\SOUNDLET{\UA}{ \txidsetrd \subseteq
\begin{multlined}[t]
\Set{\txid' | \Forall{l | \key | \val } (l,\key,\val) \in \aexec(\txid') \implies l = \opR } .
\end{multlined} } 
Note that since the invariant is empty set, it remains to prove the following result:
\[
   \WWInv[\aexec'](\txid) \subseteq \txidset \cup \txidsetrd .
\]
Assume a transaction \( \txid' \) such that \( (\txid', \txid \in \WW[\aexec'](\key) \) for some key \( \key \).
Since \( \txid' \) is a transaction already existing in \( \kvs \),
we have \( \WtOf(\kvs(\key, \idx)) = \txid' \) for some index \( \idx \).
Since \( \txid \) also wrote to the key \( \key \), that is,
\( \opW(\key, \val) \in \fp \) for some value \( \val \).
By the definition of \( \et[\UA] \), 
we know that \( \idx \in \vi(\key) \) and therefore \( \txid' \in \txidset \).

\COMPLETELET{\UA}
Let consider a key \( \key \) that have been overwritten by the transaction \( \txid[\cl](\idx) \).
By the visibility axiom \( \WW[\aexec] \subseteq \VIS[\aexec] \),
for any transaction \( \txid \) that writes to the same key \( \key \) 
and was committed before \( \txid[\cl](\idx) \), 
they must be included in the visible set, that is, \(\txid \in \VISInv[\aexec](\txid[\cl](\idx)) \).
Note that \( \ToEdge{\txid | \WW[\aexec]  -> \txid[\cl](\idx) } 
                \implies \ToEdge{ \txid | \AR[\aexec]  -> \txid[\cl](\idx) } 
                    \implies \txid \in \kvs''\).
Since the transaction \( \txid \) wrote to the key \( \key \),
it means that \( \WtOf(\XToK(\aexec')(\key,j)) = \txid \) for some index \( j \) and
then by the definition of \( \GetView \), we have \( j \in \vi(\key)\),
which implies \( \CanCommit[\UA](\XToK(\aexec'),\vi,\fp) \).
