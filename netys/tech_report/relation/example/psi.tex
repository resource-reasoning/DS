An abstract execution \( \aexec \) satisfies parallel snapshot isolation (\(\PSI\)), 
if it satisfies
\( \{\lambda \aexec. \VIS[\aexec] ; \VIS[\aexec], \allowbreak \lambda \aexec \ldotp \SO, \lambda \aexec. \WW[\aexec] \}) \),
which is intersection of \( \CC \) and \( \UA \) on abstract executions \citep{psi-chopping}.
Similar to \cref{sec:sound-complete-cc}, there exist a minimum visibility.

\begin{theorem}[Minimum visibility relation for (\texorpdfstring{\PSI}{\texttt{PSI}})] 
\label{thm:aexec-spec-psi}
For two abstract executions \( \aexec,\aexec' \),
the following constrain on visibility,
\begin{Formulae}
\begin{Formula}
    (\WR[\aexec] \cup \WW[\aexec] \cup \SO ) ; \VIS[\aexec] \subseteq \VIS[\aexec] \qquad \SO \subseteq \VIS[\aexec] \qquad \WW[\aexec] \subseteq \VIS[\aexec]
    \label{equ:kvstore-psi-spec}
\end{Formula}
\end{Formulae}
is equivalent to
\begin{Formulae}
\begin{Formula}
    \VIS[\aexec'] ; \VIS[\aexec'] \subseteq \VIS[\aexec']  \qquad \SO \subseteq \VIS[\aexec'] \qquad \WW[\aexec] \subseteq \VIS[\aexec] 
    \label{equ:aexec-psi-spec}
\end{Formula}
\end{Formulae}
in that 
\(
    \Forall{\txid \in \TxIDs | \fp } \left( \fp = \aexec(\txid) \iff \fp = \aexec'(\txid) \right)
    \land \AR[\aexec] = \AR[\aexec'] .
\)
\end{theorem}
\begin{proof}
The proof is similar to \cref{thm:cc-visaxioms}.
For \( \aexec \) that satisfies \cref{equ:kvstore-psi-spec}, 
we construct a new abstract execution \( \aexec' \) 
with a new visibility relation \( \VIS[\aexec'] \) such that 
\( \VIS[\aexec'] = (\WR[\aexec] \cup \SO \cup \WW[\aexec])^{+} \)
and for the similar reason as \cref{lem:aexec-spec-cc}, the new abstract execution \( \aexec' \)
satisfies \cref{equ:aexec-psi-spec}.
Assume an abstract execution \( \aexec' \) that satisfies \cref{equ:aexec-psi-spec}.
Since \( \WR[\aexec'] \subseteq \VIS[\aexec']\) by the definition of \( \WR[\aexec']\),
thus \( \aexec' \) satisfies \cref{equ:kvstore-psi-spec}.
\end{proof}


%%%%%%%%%%%%%%%%%%%%% archive proof
%we erase some visibility relation for each transaction following the order of arbitration \( \AR \) until the visibility is transitive.
%Assume the i-\emph{th} transaction \( \txid_i \)  with respect to the arbitration order.
%Let \( R_i \) denote a new visibility for transaction \( \txid_i \) such that
%\( R_i\projection{2} = \Set{\txid_i}\)
%and the visibility relation before (including) \( \txid_i \) is transitive.
%Let \( \aexec_i = \mkvs_{\cut[\aexec, i]} \) and \( \VIS_i = \bigcup_{0 \leq k \leq i} R_i \).
%For each step, says i-\emph{th} step, we  preserve the following:
%\begin{gather}
    %\VIS_i ; \VIS_i \subseteq \VIS_i \label{equ:vis-i-transitive-psi} \\
    %\fora{\txid} (\txid,\txid_i) \in R_i \implies (\txid, \txid_i) \in (\WR_i \cup \WW_i \cup \SO_i)
    %\label{equ:last-read-correct-psi}
%\end{gather}
%\begin{itemize}
%\item \caseB{\( i = 1 \) and \( R_1 = \emptyset \)}
%Assume it is from client \( \cl \).
%There is no transaction committed before, so \( \VIS_1 = \emptyset \) and \( \VIS_1 ; \VIS_1 \subseteq \VIS_1 \) as \cref{equ:vis-i-transitive-psi}.

%\item \caseI{i-\emph{th} step}
%Suppose the (i-1)-\emph{th} step satisfies \cref{equ:vis-i-transitive-psi} and \cref{equ:last-read-correct-psi}.
%Let consider i-\emph{th} step and the transaction \( \txid_i \).
%Initially we take \( R_i \) as empty set.
%We first extend \( R_i \) by closing with respect to \( \WR_i \)
%and prove that it does not affect any read from the transaction \( \txid_i \).
%Then we will do the same for \( \SO_i \) and \( \WW_i \).
%\begin{itemize}
    %\item \( \WR_i\). For any read \( (\otR, \key, \val ) \in \txid_i \),
    %there must be a transaction \( \txid_j \) that \( \txid_j \toEDGE{\WR(\aexec_i,\key), \AR} \txid_i \) and \( j < i \).
    %We include \( (\txid_j, \txid_i) \in R_i \).
    %Let consider all the visible transactions of \( \txid_j \).
    %Assume a transaction \( \txid' \in \VIS_{i-1}^{-1}(\txid_j) \), 
    %thus \( \txid' \in \VIS_{j}^{-1}(\txid_j) = R_j^{-1}(\txid_j) \).
    %It is safe to include \( (\txid', \txid_i) \in R_i \) without affecting the read result,
    %because those transaction \( \txid' \) is already visible for \( \txid_i \) in the abstract execution \( \aexec \):
    %by \cref{equ:last-read-correct-psi} we know \( R_j \subseteq (\WR_j \cup \SO_j \cup \WW_j)^{+} \subseteq (\WR_\aexec \cup \SO_\aexec \cup \WW_\aexec)^{+}\),
    %and by the definition of \( \WR(\aexec_i,\key) \) we know \( \WR(\aexec_i,\key) \subseteq \VIS_\aexec\).

%\item Given \( \SO_\aexec \subseteq \VIS_\aexec \) (and \( \WW_\aexec \subseteq \VIS_\aexec \) respectively) we include \( (\txid_j,\txid_i) \) for some \( \txid_j \)
    %such that \( \txid_j \toEDGE{\SO_\aexec} \txid_i\) (and \( \txid_j \toEDGE{\WW_\aexec} \txid_i \) respectively).
    %For the similar reason as \( \WR \),
    %it is safe to includes all the visible transactions \( \txid' \) for \( \txid_j \), \ie \( \txid' \in R_j^{-1}\).
    %\end{itemize}
    
%By the construction, both \cref{equ:vis-i-transitive} and \cref{equ:last-read-correct} are preserved. 
%Thus we have the proof.
%\end{itemize}
%%%%%%%%%%%%%%%%%%%%% archive proof


The execution test \( \et[\PSI] \) is sound with respect to the axiomatic definition 
\[
\visaxioms[\PSI] \FuncDef \Set{\lambda \aexec \ldotp \left( \WR[\aexec] \cup \WW[\aexec] \cup \SO \right) ; \VIS[\aexec], \lambda \aexec \ldotp \SO, \lambda \aexec \ldotp \WW[\aexec] } .
\]
We pick the invariant as \( \aexecinv[\PSI] = \aexecinv[\MR] \cup \aexecinv[\RYW]  \).
\SOUNDLET{\CC}{ \txidsetrd \supseteq
\begin{multlined}[t]
\left( \bigcup_{\Set{\txid[\cl](\idx) | \txid[\cl](\idx) \in \aexec}} 
\VISInv[\aexec](\txid[\cl](\idx)) \cup \Refl((\Inv(\SO)))(\txid[\cl](\idx)) \right) 
\setminus \Set{\txid' | \Forall{l | \key | \val } (l,\key,\val) \in \aexec(\txid') \implies l = \opR } .
\end{multlined} }
Assume 
\[ 
\txidsetrd' = 
\begin{multlined}[t]
\left( \bigcup_{\Set{\txid[\cl](\idx) | \txid[\cl](\idx) \in \aexec}} 
\VISInv[\aexec](\txid[\cl](\idx)) \cup \Refl((\Inv(\SO)))(\txid[\cl](\idx)) \right) 
\setminus \Set{\txid' | \Forall{l | \key | \val } (l,\key,\val) \in \aexec(\txid') \implies l = \opR } .
\end{multlined} 
\]
and \( \txidsetrd'' = \txidsetrd \setminus \txidsetrd' \).
By the definition of soundness, we prove the following result
\begin{Formulae}
& \begin{Formula}
    \Inv(\SO)(\txid) \subseteq \txidset \cup \txidsetrd' 
    \label{equ:psi-sound-update-so}
\end{Formula}
\\ & \begin{Formula}
    \Inv(\WW)(\txid) \subseteq \txidset 
    \label{equ:psi-sound-update-ua}
\end{Formula}
\\ & \begin{Formula}
    \Inv(( \WR[\aexec'] \cup \SO \cup \WW[\aexec'] )) (\txid) \subseteq \txidset \cup \txidsetrd' \cup \txidsetrd''
     \label{equ:psi-sound-update-closure}
\end{Formula}
\\ & \begin{Formula}
    \aexecinv[\PSI](\aexec',\cl) \subseteq \VisTrans(\XToK(\aexec'),\vi')
    \label{equ:psi-inv-preserve}
\end{Formula}
\end{Formulae}
\Cref{equ:psi-sound-update-so,equ:psi-sound-update-ua} 
can be proven in the same way as in \cref{sec:sound-complete-mr,sec:sound-complete-ua} respectively.
We now prove \cref{equ:psi-sound-update-closure}.
Initially we take \( \txidsetrd'' \) to be an empty set.
Note that \(\VISInv[\aexec'](\txid) = \txidset \cup \txidsetrd' \cup \txidsetrd'' \).
By \cref{thm:view-vis-relation,equ:view-close-to-aexec}, there exists \( \txidsetrd'' \) such that
\( \txidset \cup \txidsetrd'' \) is closed under \( \WR[\aexec'] \cup \SO \cup \WW[\aexec']\).
Now consider a transaction \( \txidrd \in \txidsetrd' \) and
assume a transaction \( \txid' \) such that \( \ToEdge{ \txid' | \WR[\aexec'] \cup \SO \cup \WW[\aexec'] -> \txidrd } \).
Since \( \txidrd \) is a read-only transaction, thus
\( \ToEdge{ \txid' |  (\SO \cup \WR[\aexec] )  -> \txidrd } \)
and the rest proof is exactly the same as as in \cref{sec:sound-complete-cc}.
Last, \cref{equ:psi-inv-preserve} can be proven in the same way as in \cref{sec:sound-complete-mr,sec:sound-complete-ryw}.

The execution test $\et[\PSI]$ is complete with respect to the axiomatic definition \( \visaxioms[\PSI] \).
By \cref{thm:aexec-spec-psi}, it suffices to prove completeness with respect to the following definition,
\[
\Set{\lambda \aexec \ldotp \VIS[\aexec] ; \VIS[\aexec], \lambda \aexec \ldotp \SO , \lambda \aexec \ldotp \WW[\aexec] }
\]
\COMPLETELET{\PSI}
By the definition of \( \et[\PSI]\), we prove \( \CanCommit[\PSI] \), \( \ViewShift[\MR]\) and \( \ViewShift[\RYW]\) respectively.
Recall that \( \CanCommit[\PSI] = \PreClosed(\kvs,\vi,\rel[\UA] \cup \WR[\kvs] \cup \SO \cup \WW[\kvs]) \).
It is easy to see that 
\( \PreClosed(\kvs,\vi,\rel[\UA] \cup \WR[\kvs] \cup \SO \cup \WW[\kvs]) \iff \PreClosed(\kvs,\vi,\rel[\UA]) 
            \land \PreClosed(\kvs,\vi,\WR[\kvs] \cup \SO \cup \WW[\kvs]) \).
The predicate \( \PreClosed(\kvs,\vi,\rel[\UA]) \) can be proven in the same way as in \cref{sec:sound-complete-ua}
since \( \WW[\aexec] \subseteq \VIS[\aexec]\).
Because 
\begin{align*}
\left( \WR[\aexec] \cup \SO \cup \WW[\aexec] \right) ; \VIS[\aexec] 
        & \subseteq \left( \VIS[\aexec] \right) ;  \VIS[\aexec] \subseteq \VIS[\aexec] .
\end{align*}
Then \( \CanCommit[\CP]\) can be derived from \cref{thm:view-vis-relation,equ:aexec-close-to-view}.
The predicate \( \ViewShift[\RYW] \) can be proven in the same way as in \cref{sec:sound-complete-ryw}.
Since \( \VIS[\aexec] ; \SO \subseteq \AR[\aexec] ; \VIS[\aexec] \subseteq \VIS[\aexec] \)
\( \ViewShift[\MR] \) can be proven in the same way as in \cref{sec:sound-complete-mr}.
