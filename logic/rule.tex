\subsection{Rules for Local}

The proof rules (\figref{fig:rule-trans}) are standard except \rl{TRLookup} and \rl{TRMutate}.
The \rl{TRLookup} rule adds read fingerprint only if there is no write fingerprint.
This is because once an address has been re-written, the rest reads are local.
The\rl{TRMutate} rule adds write fingerprint if there no write fingerprint and keeps the last written value.

\sx{Font for E}

\begin{figure}[!t]
\hrule\vspace{5pt}
\[
    \infer[\rl{TRSkip}]{%
        \tripleL{\assemp }{ \pskip }{\assemp }
    }{}
\]

\[
    \infer[\rl{TRAss}]{%
        \tripleL{\var \dot= \lexpr }{ \pass{\var}{\expr} }{\var \dot= \expr\sub{\var}{\lexpr} }
    }{
        \var \notin \func{fv}{\lexpr}
    }
\]

\[
    \infer[\rl{TRLookup}]{%
        \tripleL{ \lpre }{ \plookup{\var}{\expr} }{\var \dot= \lexpr \sep \lpost\sub{\var}{\lexpr} }
    }{%
        \var \notin \func{fv}{\lexpr}  
        && \lpre \toFP{\otR(\expr, \lexpr)} \lpost
    }
\]

\[
    \infer[\rl{TRMutate}]{%
        \tripleL{ \lpre }{ \pmutate{\expr_1}{\expr_2} }{ \lpost } 
    }{
        \lpre \toFP{\otW(\expr_{1},\expr_{2})} \lpost
    }
\]

\[
    \infer[\rl{TRAssume}]{%
        \tripleL{ \expr \doteq 0 }{ \passume{\expr} }{ \expr \doteq 0 } 
    }{}
\]

%\[
    %\infer[\rl{TRReturn}]{%
        %\tripleL{ \assemp }{ \preturn{\expr} }{ \ret \doteq \expr } 
    %}{}
%\]

\[
    \infer[\rl{TRChoice}]{%
        \tripleL{ \lpre }{ \trans_{1} \pchoice \trans_{2} }{ \lpost }
    }{%
        \tripleL{ \lpre }{ \trans_{1} }{ \lpost } && 
        \tripleL{ \lpre }{ \trans_{2} }{ \lpost } 
    }
\]

\[
    \infer[\rl{TRSeq}]{%
        \tripleL{ \lpre }{ \trans_{1} \pseq \trans_{2} }{ \lpost }
    }{%
        \tripleL{ \lpre }{ \trans_{1} }{ \lframe }  && 
        \tripleL{ \lframe }{ \trans_{2} }{ \lpost }
    }
\]


\[
    \infer[\rl{TRIter}]{%
        \tripleL{ \lpre }{ \trans\prepeat }{ \lpre }
    }{%
        \tripleL{ \lpre }{ \trans }{ \lpre } 
    }
\]
 
\[
   \infer[\rl{TRFrame}]{%
       \tripleL{ \lpre \sep \lframe }{ \trans }{ \lpost \sep \lframe }
   }{%
       \tripleL{ \lpre }{ \trans }{ \lpost } 
        && \func{fv}{\lpre} \cap \func{fv}{\lframe} = \emptyset
        && \func{fv}{\lpost} \cap \func{fv}{\lframe} = \emptyset
   }
\]
\sx{ \( \lexpr \fpRW (\lexpr'', \lexpr') \toFP{\otR(\lexpr,\lexpr')} \lexpr \fpRW (\lexpr'', \lexpr') \) means it read locally from the last written value.  }


\hrule\vspace{5pt}
%\sx{The% font is a bit overwhelming}
\[
\begin{rclarray}
    \lexpr \fpI \lexpr' & \toFP{\otR(\lexpr,\lexpr')} & \lexpr \fpR \lexpr' \\
    \lexpr \fpR \lexpr' & \toFP{\otR(\lexpr,\lexpr')} & \lexpr \fpR \lexpr' \\
    \lexpr \fpW \lexpr' & \toFP{\otR(\lexpr,\lexpr')} & \lexpr \fpW \lexpr' \\
    \lexpr \fpRW (\lexpr'', \lexpr') & \toFP{\otR(\lexpr,\lexpr')} & \lexpr \fpRW (\lexpr'', \lexpr') \\
    \lexpr \fpI \lexpr' & \toFP{\otW(\lexpr,\lexpr'')} & \lexpr \fpW \lexpr'' \\
    \lexpr \fpR \lexpr' & \toFP{\otW(\lexpr,\lexpr'')} & \lexpr \fpRW (\lexpr',\lexpr'') \\
    \lexpr \fpW \lexpr' & \toFP{\otW(\lexpr,\lexpr'')} & \lexpr \fpW \lexpr'' \\
    \lexpr \fpRW (\lexpr'', \lexpr') & \toFP{\otW(\lexpr,\lexpr''')} & \lexpr \fpRW (\lexpr'', \lexpr''') \\
\end{rclarray}
\]
\hrule\vspace{5pt}
\caption{The rules for transactions}
\label{fig:rule-trans}
 \end{figure}

\subsection{Rely, Guarantee and Consistency Model}

%\sx{To allow a transaction update multiple regions but a region multiple times.}

We allow a transaction to update several regions together, but each region can be updated at most once.
Given that, the guarantee is a set of pairs of worlds that are allowed by the local capabilities.
Each pairs assertions how a world can evolve.
The rely is a set of pairs of worlds asserting how the history heaps are changed with respect to capabilities the current thread does not own.
Note that the rely does not change the view, because this corresponds a thread from the environment that change the history heap with respect to its own view.

\sx{ IN CASE TO REVERSE
\begin{defn}[Invariant of the world]
The \emph{invariant of a world} \( \func{inv}{\w}\) is the compositions of all the invariants of regions included.
Given two regions \( \rid, \rid' \) with disjointed history heaps and views, the composition of their invariants \( \func{inv}{\rid,\intf} \compose \func{inv}{\rid', \intf'} \) are defined by the following two rules.
If it is between pairs, for example \( ( m, n ) \compose ( m', n' ) \), the compositions is defined as the follows,
\[
\begin{rclarray}
    ( m, n ) \compose ( m', n' ) & \defeq & ( m \compose  m',  n  \compose n' )
\end{rclarray}
\]
where \( m\) and \( m' \) is from a domains that composition is defined and similar to \( n\) and \( n' \).
This can be generalised for tuples with arbitrary numbers of elements.
If it is between two sets, it is defined as the compositions between all elements,
\[
\begin{rclarray}
\sort{M} \compose \sort{M}' & \defeq & \Setcon{m \compose m'}{m \in \sort{M} \land m' \in \sort{M}'} \\
\end{rclarray}
\]
where \( \sort{M}, \sort{M}'\) are subsets of a domain that composition is defined.
\end{defn}
}


\begin{defn}[Rely and guarantee]
\label{def:rely-guarantee}
%The \( \predn{updWorlds} \) predicate asserts that the world transfers from \( \w \) to \( \w' \) which is allowed by the capabilities \(\ca\).
%\[
%\begin{rclarray}
    %\pred{updWorlds}{\ca, \w, \w'} & \defeq & \w = \w' \\
    %\pred{updWorlds}{\ca, (\ca',\gs) ,(\ca'',\gs')} & \defeq & 
    %\begin{array}[t]{@{}l}
    %\exsts{\kap, \hh, \hh', \vi, \vi', \ca''', \ca''''}
    %\kap \sqsubseteq \ca(\rid) 
    %\land \pred{updW}{\kap, (\ca',\gs),(\ca'',\gs'')} \\
    %\quad {} \land \pred{updWorlds}{\ca, (\ca''',\gs),(\ca'',\gs'')}

    %\end{array} \\
    %\pred{updW}{\kap, (\ca, \gs), (\ca' ,\gs')} & \defeq &  
    %\begin{array}[t]{@{}l}
        %\exsts{\rid, \hh, \hh', \vi, \vi', \ca'', \ca''', \intf } \\
        %\quad \gs = \Set{\rid \mapsto (\hh, \vi, \ca'', \intf)} 
        %\land \gs' = \Set{\rid \mapsto (\hh', \vi', \ca''', \intf)}  \\
        %\quad {} \land (\hh, \vi, \ca'') \toLTS{\kap} (\hh', \vi', \ca''') \in \func{inv}{\rid, \intf} 
        %\land \ca \composeC \ca'' = \ca' \composeC \ca'''
    %\end{array}
%\end{rclarray}
%\]
Given the set of worlds $\World$ (\defref{def:world}), the \emph{rely} relation, $\Rely \subseteq \World \times \World$, is defined as follows,
\sx{In case I get confused again, it is world to world so the shared and local capabilities should always make sense.}
\[	
    \begin{rclarray}
	\Rely & \eqdef &
	\Setcon{
		((\ca,\gs), (\ca, \gs'))	
	}{
        \fora{\rid}
        \gs(\rid) = \gs'(\rid) \lor {}
        \exsts{\kap, \hh, \hh', \vi, \vi', \vi'', \vi''', \ca', \ca'', \ca''', \intf}   \\
        \quad \gs(\rid) = (\hh, \vi, \ca'',\intf)
        \land \gs'(\rid) = (\hh', \vi', \ca''',\intf) \\
        \quad {} \land \vi' \geq \vi
        \land (\ca' \composeC \ca) \isdef
        \land \kap \sqsubseteq \ca'(\rid) \\
        \quad {} \land ( (\hh, \vi'', \ca''), (\hh', \vi''', \ca''') )  \in \intf(\kap)
	} \\
    \end{rclarray}
\]
%The \emph{shift rely} relation \( \relyV \subseteq \World \times \World\) allows to shift the view as long as the new view is within the range of history heap.
%\[
    %\begin{rclarray}
	%\relyV & \eqdef &
	%\Setcon{
		%((\ca, \gs), (\ca, \gs'))	
	%}{
        %\exsts{\ca', \hh, \vi, \ca'}  
        %(\hh, \vi, \ca') \in \func{collapse}{\gs}
        %\land (\hh, \vi', \ca') \in \func{collapse}{\gs} 
	%} \\
    %\end{rclarray}
%\]
\azalea{\sx{The view cannot be out of the range of the history heap. Put the constraint in the type of world.}What is $\pred{wfView}{\vi',\hh}$?}
%Thus the \emph{rely} relation is the union of updates and shifts \( \Rely = \relyU \cup \relyV \).
%The invariant of a shared state is a lift of the invariants of interferences of regions.
%The \emph{rely} relation, $\RelyI \eqdef \World \times \World$, is defined as follows:
%\[
    %\begin{rclarray}
         %\RelyI &\eqdef & \closure{\left(\relyU\right)} \\
    %\end{rclarray}
%\]
\sx{ALREADY: \( \ca \composeC \ca'' = \ca' \composeC \ca''' \) might strengthen to the residual remains the same, to cope with updating ghost resources (capabilities) within the regions \eg \( \ca = \ca' = \assemp \land \ca'' = \cass{\kap(1)}{\rid} \land \ca''' = \cass{\kap(2)}{\rid}\). }
The \emph{guarantee} relation, $\Guar: \World \times \World$, is defined as follows,
\[	
    \begin{rclarray}
	\Guar & \eqdef &
	\Setcon{
		((\ca,\gs), (\ca', \gs'))	
	}{
        \fora{\rid}
        \gs(\rid) = \gs'(\rid) \lor {}
        \exsts{\kap, \hh, \hh', \vi, \vi', \ca', \ca'', \ca''', \intf}   \\
        \quad \gs(\rid) = (\hh, \vi, \ca'',\intf)
        \land \gs'(\rid) = (\hh', \vi', \ca''',\intf) \\
        \quad {} \land \kap \sqsubseteq \ca(\rid) 
        \land (\ca \composeC \ca'')^{\perp} = (\ca' \composeC \ca''')^{\perp}  \\
        \quad {} \land ( (\hh, \vi, \ca''), (\hh', \vi', \ca''') )  \in \intf(\kap)
	} \\
    \end{rclarray}
\]
where for any element \( m \) from its domain \( \sort{M} \), the  \emph{orthogonal} is defined as, \( m^{\perp} \defeq \Setcon{m'}{(m \compose{} m')\isdef \land m' \in \sort{M}} \).
%The \emph{guarantee} relation, $\GuarI \subseteq \World \times \World$, is defined as follows:
%\sx{take away of the closure}
%\[
	%\GuarI \eqdef \guarU
%\]
\end{defn}

%\begin{defn}
%\label{def:world-consistency-model}
%A world transition is allowed by a consistency model \( (\w, \w') \in \como \), if there exist a operation set \( \opset \) that update the corresponding machine states from \( (\hh, \vi) \)  to \( (\hh', \vi') \), and such update is allowed by the consistency model.
%\[
%\begin{rclarray}
    %(\w, \w') \in \como & \defeq & 
    %\begin{array}[t]{@{}l}
    %\fora{\hh, \vi} \exsts{\txid, \hh', \vi', \opset} \\
    %\quad (\hh,\vi,\stub) \in \clpsW{\w} \land (\hh',\vi',\stub) \in \clpsW{\w'}
    %\land \txid \in \func{fresh}{\hh} \\
    %\quad {} \land \hh' = \func{updHisHp}{\hh, \vi, \txid, \opset}
    %\land \vi' \geq \func{updView}{\hh, \vi, \opset}
    %\land (\hh,\vi) \csat \opset : \vi'
    %\end{array}
%\end{rclarray}
%\]
%\end{defn}
%\sx{Make the consistency model relation reflexive for manipulating capabilities. Might not need reflexivity to achieve that but keep this in mind in case.}

\begin{defn}[Stable]
A set of worlds $\setworld \subseteq \World$ is \emph{stable}, written $\stable{\setworld, \como}$, if and only if it is closed under the rely relation: 
\[
    \begin{rclarray}
        \stable{\setworld, \como} & \eqdef & 
        \begin{array}[t]{@{}l}
            \fora{\w, \w'} \\
            \begin{B}
            \w \in \setworld 
            \land (\w, \w') \in \Rely  
            \land \exsts{\hh, \hh', \vi, \vi', \opset} \\
            \quad (\hh, \stub, \stub) \in \clpsW{\w}
            \land (\hh', \vi, \stub) \in \clpsW{\w'} \\
            \quad {} \land \opset \neq \unitO 
            \land (\hh', \vi) \csat \opset : \vi'
            \land (\hh, \hh') \in \como
            \end{B}
            \implies \w' \in \setworld
        \end{array}
    \end{rclarray}
\]
If a update history heap update is allowed by consistency model, \ie \( (\hh, \hh') \in \como \), it means there exist some view \( \vi \) and operation set \( \opset \) allowed by the consistency model and the history is updated to \( \hh' \) via them.
\[
    \begin{rclarray}
        (\hh, \hh') \in \como & \eqdef & 
        \begin{array}[t]{@{}l}
            \hh = \hh' \lor 
            \exsts{ \vi, \vi', \opset, \txid}  \\
            \quad (\hh, \vi) \csat \opset : \vi' 
            \land \txid \in \func{fresh}{\hh} 
            \land \hh'  = \func{updHisHp}{\hh,\vi, \txid, \opset}
        \end{array}
    \end{rclarray}
\]
\end{defn}

\subsection{Rules for Global}

The \rl{PRCommit} rule lifts the local effect of transaction \( \trans \) to global level by first converting global state to (local) observable state and then propagating the local fingerprint to the global state.
%The \( \predn{down} \) predicate asserts that the local predicate \( \lpre \) is a over-approximation of the valid observation that is given by the interference.
%The \( \predn{up} \) predicate says the post-condition \( \gpost \) is the result by lifting the local fingerprints \( \fp \) to pre-condition \( \gpre \).


\begin{figure}[t!]
\hrule\vspace{5pt}

%\[
    %\infer[\rl{PRCommit}]{%
        %\tripleG{\gpre}{ \ptrans{\trans} }{\gpost}
    %}{%
        %\begin{array}{c}
        %\gpre \snap \lpre
        %\quad \tripleL{\lpre \sep \fpE}{\trans}{\lpost \sep \fp} \\
        %\pred{noFingerprint}{\lpre} 
        %\quad \pred{noFingerprint}{\lpost} \\
        %\rpt{\gpre}{\gpost}{\fp}
        %\quad \stable{\gpre} 
        %\quad \stable{\gpost} 
        %\end{array}
    %}
%\]

\[
    \infer[\rl{PRCommit}]{%
        \tripleG{\gpre}{ \ptrans{\trans} }{\gpost}
    }{%
        \begin{array}{c}
        %\gpre \snap \lpre
        \tripleL{\lpre}{\trans}{\lpost} 
        \quad \repartition{\gpre}{\gpost}{\lpre}{\lpost} \\
        %\pred{noFingerprint}{\lpre} 
        %\quad \pred{noFingerprint}{\lpost} \\
        \stable{\gpre, \como} 
        \quad \stable{\gpost, \como} 
        \end{array}
    }
\]


\[
    \infer[\rl{PRAss}]{%
        \tripleG{\thvar \dot= \lexpr }{ \pass{\thvar}{\expr} }{\thvar \dot= \expr\sub{\thvar}{\lexpr} }
    }{%
        \thvar \notin \func{fv}{\lexpr} 
    }
\]

\[
    \infer[\rl{PRAssume}]{%
        \tripleG{ \expr \doteq 0 }{ \passume{\expr} }{ \expr \doteq 0 } 
    }{}
\]

\[
    \infer[\rl{PRChoice}]{%
        \tripleG{ \gpre }{ \cmd_{1} \pchoice \cmd_{2} }{ \gpost }
    }{%
        \tripleG{ \gpre }{ \cmd_{1} }{ \gpost } && 
        \tripleG{ \gpre }{ \cmd_{2} }{ \gpost } 
    }
\]

\[
    \infer[\rl{TRSeq}]{%
        \tripleG{ \gpre }{ \cmd_{1} \pseq \cmd_{2} }{ \gpost }
    }{%
        \tripleG{ \gpre }{ \cmd_{1} }{ \gframe }  && 
        \tripleG{ \gframe }{ \cmd_{2} }{ \gpost }
    }
\]

\[
    \infer[\rl{TRIter}]{%
        \tripleG{ \gpre }{ \cmd\prepeat }{ \gpre }
    }{%
        \tripleG{ \gpre }{ \cmd }{ \gpre } 
    }
\]
 
\[
   \infer[\rl{TRFrame}]{%
       \tripleG{ \gpre \sep \gframe }{ \cmd }{ \gpost \sep \gframe }
   }{%
       \tripleG{ \gpre }{ \cmd }{ \gpost } 
       && \stable{\gframe, \como}
   }
\]

\sx{The stable pre-conditions here are for the soundness when \( \cmd_{1} \equiv \passign{\vx}{1}; \). }
 
\[
   \infer[\rl{TRPar}]{%
       \tripleG{ \gpre_{1} \sep \gpre_{2} }{ \cmd_{1} \ppar \cmd_{2} }{ \gpost_{1} \sep \gpost_{2} }
   }{%
   \begin{array}{@{}c@{}}
       \tripleG{ \gpre_{1} }{ \cmd_{1} }{ \gpost_{1} }
       \quad \tripleG{ \gpre_{2} }{ \cmd_{2} }{ \gpost_{2} } \\
        \stable{\gpre_{1}, \como} 
        \quad \stable{\gpre_{2}, \como} 
    \end{array}
   }
\]

%\sx{type mismatch for interpretation of fingerprint in  repartition.}

\hrule\vspace{5pt}
\[
\begin{rclarray}
    \repartition{\gpre}{\gpost}{\lpre}{\lpost} & \defeq & 
    \begin{array}[t]{@{}l@{}}
        \fora{ \w, \hh, \vi, \lenv, \stk } 
        \w \in \evalW{\gpre} 
        \land (\hh, \cu, \stub) \in \clpsW{\w} \implies \\
        \quad \exsts{\h}
        \h = \func{clps}{\hh, \cu} 
        \land (\h, \unitO) \in \evalLS{\lpre}\\
        \qquad {} \land
        \fora{\stk', \txid, \opset, \hh', \vi'} 
        \exsts{ \w'} 
        \txid \in \func{fresh}{\hh} 
        \land (\stub, \opset) \in \evalF[\lenv, \stk']{\lpost} \\
        \qqquad {} \land \hh' = \func{updHisHp}{\hh, \vi, \txid, \opset} 
        \land \vi' \geq \func{updView}{\hh, \vi, \opset} \\
        \qqquad {} \land (\w, \w') \in \Guar  
        \land (\hh,\vi) \csat \opset : \vi'
        \land (\hh',\vi', \stub) \in \clpsW{\w'} \land \w' \in \evalW[\lenv, \stk']{\gpost}
    \end{array} 
\end{rclarray}                          
\]

\hrule\vspace{5pt}
\caption{The rules for programs}
\label{fig:rule-prog}
\end{figure}

\azalea{
    \sx{How to deal with the stack here? As the stack for P and Q are different, just for all quantify two stacks??}
The quantification seems wrong. Especially, the $\extopset$ needs to be for all quantified, $\h$ needs to be there exist quantified.
\[
 	\repartition{\gpre}{\gpost}{\lpre}{\lpost} \defeq 
 	\begin{array}[t]{@{}l@{}}
	 	\fora{\w, \hh, \vi, \ca, \lenv, \stk, \txid } 
    		\w \in \evalW{\gpre} 
        	\land (\hh, \cu, \ca) \in \eraseW{\w}
        	\Rightarrow\\
        	\quad \exsts{\h}
        	\begin{array}[t]{@{} l @{}}
			\h = \func{clps}{\hh, \cu}  
        		\land (\h, \unitO) \in \evalLS{\lpre} \\
        		\land\, \fora{\extopset \in \evalF{\lpost}} 
        			\exsts{\w', \hh', \vi', \ca'} \\
        				\quad \hh' = \func{updHisHp}{\hh, \vi, \txid, \extopset} 
           			\land \cu' = \func{updView}{\hh, \vi, \extopset} 
           			\land (\hh',\vi', \ca') \in \eraseW{\w'} \\
            			\quad \land (\w, \w') \in \Guar  
            			\land (\w, \w') \in \como
            			\land \w' \in \evalW{\gpost}			
		\end{array}	
%		\right)			        	
    \end{array} 
\]
}
The \( \funcn{updHisHp}\) and \( \funcn{updView} \) in the repartition can be replaced by syntactic rules.

\begin{figure}
\hrule\vspace{5pt}

\[
   \infer[\rl{FInit}]{%
       \tripleF{ \lexpr \pt \lexpr }{ \lexpr \fpI \lexpr }{ \lexpr \pt \lexpr }
   }{}
\]

\[
   \infer[\rl{FRead}]{%
       \tripleF{ \lexpr \pt \lexpr }{ \lexpr \fpR \lexpr }{ \lexpr \pt \lexpr }
   }{}
\]

\[
   \infer[\rl{FWrite}]{%
       \tripleF{ \lexpr \pt \lexpr  }{  \lexpr \fpW \lexpr' }{ \lexpr \pt \lexpr' }
   }{}
\]

\[
   \infer[\rl{FReWrt}]{%
       \tripleF{ \lexpr \pt \lexpr  }{  \lexpr \fpRW (\lexpr,\lexpr') }{ \lexpr \pt \lexpr' }
   }{}
\]

\[
   \infer[\rl{FFrame}]{%
       \tripleF{ \bar{\lpre}_{1} \sep \bar{\lpre}_{2}  }{  \bar{\fp}_{1} \sep \bar{\fp}_{2} }{ \bar{\lpost}_{1} \sep \bar{\lpost}_{2} }
   }{
       \tripleF{ \bar{\lpre}_{1} }{ \bar{\fp}_{1} }{ \bar{\lpost}_{1} }
       && \tripleF{ \bar{\lpre}_{2}  }{ \bar{\fp}_{2} }{ \bar{\lpost}_{2} }
    }
\]

%\[
   %\infer[\rl{FFrame}]{%
       %\tripleF{ \lpre \sep \lframe  \mid \lpre' \sep \lframe' }{ \fp }{ \lpost \sep \lframe \mid \lpost' \sep \lframe' }
   %}{%
       %\tripleF{ \lpre \mid \lpre' }{ \fp }{ \lpost \mid \lpost' }
   %}
%\]

%\[
   %\infer[\rl{FContinue}]{%
       %\tripleF{ \lpre \sep \lframe  \mid \lpre' \sep \lframe' }{ \fp  \uplus \fp' }{ \lpost \sep \lframe \mid \lpost' \sep \lframe' }
   %}{%
       %\tripleF{ \lpre \sep \lframe  \mid \lpre' \sep \lframe' }{ \fp }{ \lpost \sep \lframe \mid \lpost' \sep \lframe' }
   %}
%\]

%\[
%\begin{rclarray}
    %\rpt{\gpre}{\gpost}{\fp} & \defeq & 
    %\begin{array}[t]{@{}l}
    %\fora{\w, \w', \lpre, \lpre', \lpost, \lpost', \lenv, \stk} \\
    %\quad 
    %\begin{B}
        %\w \in \evalW{\gpre}
        %\land \pred{unboxleft}{\w, \lpre}
        %\land \gpre \snap \lpre' 
        %\land {} \tripleF{ \lpre \mid \lpre' }{ \fp }{ \lpost \mid \lpost'} \\
        %{} \land (\w, \w') \in \Guar
        %\land \pred{unboxleft}{\w',\lpost}
        %\land \gpost \snap \lpost'
    %\end{B}
    %\implies \w' \in \evalW{\gpost}
    %\end{array} \\
    %\pred{unboxleft}{\w, \lpre} & \defeq & \fora{\hh} (\hh, \stub) \in \eraseW{\w} \implies \func{clps}{\hh} \in \evalLS{\lpre}
%\end{rclarray}
%\]

\hrule\vspace{5pt}
\caption{Syntactic rule for \funcn{updHisHp} and \funcn{updView}}
\label{fig:rule-prog}
\end{figure}

