\newcommand{\TtoOp}[1]{\ensuremath{\mathscr{#1}}}
\newcommand{\ClSet}{\ensuremath{\Gamma}}
\newcommand{\progOrd}{\ensuremath{\pi}}
\newcommand{\RP}{\ensuremath{\mathsf{RP}}}
\newcommand{\LWW}{\ensuremath{\mathsf{LWW}}}
\newcommand{\F}{\ensuremath{\mathcal{F}}}
\newcommand{\calA}{\ensuremath{\mathcal{A}}}
\newcommand{\appendTx}{\ensuremath{\mathsf{appendTx}}}
\newcommand{\thdstackFun}{\ensuremath{\mathsf{\Sigma}}}
\newcommand{\absrightarrow}{\ensuremath{\rightarrowtail}}
\newcommand{\allExec}{\ensuremath{\mathsf{allExecs}}}
\newcommand{\anarchic}{\raisebox{-1.5pt}{{\Large \text{\CircledA}}}}
\newcommand{\anarchicsmall}{\raisebox{-1.5pt}{{\small \text{\CircledA}}}}
\newcommand{\Ax}{\ensuremath{\mathsf{Ax}}}
\newcommand{\compatible}{\ensuremath{\mathop{\simeq}}}

\newcommand{\ruleATX}{\ensuremath{\mathit(A-Tx)}}
%\newcommand{\Clients}{\ensuremath{\mathsf{Clients}}}

In this Section we propose an alternative 
semantics of programs. In this semantics, states 
correspond to \emph{abstract executions} \cite{framework-concur}, 
overloaded with information regarding the transactions that are made 
visible to individual clients. 
The abstract execution semantics provides a bridge between specification 
of consistency models given in terms of MKVSs and views, and the already 
existing axiomatic ones. This section makes the following contributions: 
\begin{itemize}
\item \textbf{Done!} a definition of abstract executions, 
\item \textbf{Done!} the axiomatic specification of consistency models borrowed from cite{framework-concur},
\item \textbf{Done!} an encoding of abstract executions into MKVSs, 
\item an encoding of executions in the MKVS semantics into abstract 
executions, 
\item \textbf{Done!} an operational semantics based on abstract executions, 
which is parametric in the axiomatic specification of a consistency model given using 
the declarative style of \cite{framework-concur,SIanalysis,laws}, 
\item \textbf{Done!} a proof that, the semantics is \emph{faithful} with respect to any consistency model $\CM$: 
for any program $\prog$ and consistency model specification $\CM$, 
the abstract execution semantics captures all the potential behaviours that $\prog$ can 
exhibit under $\CM$. Faithfulness is necessary to validate the soundness of 
program analysis techniques, such as program logics, with respect to arbitrary consistency model 
specifications,
\ac{Before I was referring to what I call faithfulness as completeness. I am now changing this, because 
it looks like it only created confusion. In short words, this result requires defining the anarchic model 
of transactions - one in which the database actually reads and writes non-deterministic values 
despite the client requests, and prove that any abstract execution which is 
valid w.r.t. $\CM$ and can be obtained in the anarchic semantics, can also be obtained 
in the $\CM$ semantics. More details on this later.} 
\item a proof that each of the execution tests of Section \ref{sec:cmexamples}
captures precisely the corresponding consistency model. Specifically,
\begin{description}
\item[Soundness - ] for any program $\prog$, any execution of 
$\prog$ under the MKVSs semantics using the execution 
test $\aexec_{\CM}$ $\CM$, is encoded into an abstract execution 
that satisfies the axioms of $\CM$,
\item[Completeness - ] for any program $\prog$, any 
execution of $\prog$ under the abstract execution semantics of 
$\CM$, can be projected to an execution of $\prog$ under the history 
heap semantics using the execution test $\ET_{\CM}$. An immediate 
consequence of this completeness result, and of the faithfulness 
of the abstract execution semantic with respect to $\CM$, is that the MKVS semantics 
of $\CM$ is also faithful.
\end{description}
\end{itemize}

\paragraph{Abstract Executions}

\begin{definition}
A runtime abstract execution is a tuple $\aexec = (\TtoOp{T}, \PO, \VIS, \AR)$ where: 
\begin{enumerate}
\item $\TtoOp{T} : \TransID \rightharpoonup \powerset{\Op}$ is a partial function mapping transaction 
identifiers into sets of operations; if $\tsid \in \dom(\TtoOp{T})$, then the transaction $\tsid$ happened 
at some point in the abstract execution ${\aexec}$, by performing operations $\TtoOp{T}(\T)$, 
%\item $\ClSet \subseteq \tidset$ is a set of client identifiers, 
\item $\PO: \tidset \rightharpoonup (\dom(\TtoOp{T}) \times \dom(\TtoOp{T})$ is a function modelling the 
program order of clients: that is, there exists a $\dom(\PO)$-indexed partition $\{\T_{\tid}\}_{\tid \in \dom(\PO)}$ of 
$\dom(\TtoOp{\T})$ such that, for any $\tid \in \ClSet$ $\PO(\tid)$ is a strict, total order over $\T_{\tid}$.
\item $\VIS \subseteq \dom(\TtoOp{T}) \times (\dom(\TtoOp{T}))$ is a strict, irreflexive relation, 
\item $\AR \subseteq \dom(\TtoOp{T}) \times \dom(\TtoOp{T})$ is a strict, total order such that $\PO \cup \VIS \subseteq \AR$.
\end{enumerate}

The set of all abstract executions is denoted by $\aeset$.
\end{definition}
In general, given an abstract execution ${\aexec} = (\TtoOp{T}, \ClSet, \PO, \VIS, \AR)$, we let ${\TtoOp{T}}_{\aexec} = \TtoOp{T}, 
 \PO_{\aexec} = \PO, {\VIS}_{\aexec} = \VIS, \AR_{\aexec} = \AR$. We also let $\T_{\aexec} = \dom(\TtoOp{T}_{\aexec})$, 
 $\ClSet_{\aexec} = \dom(\PO_{\aexec})$, and $\T_{\aexec}(\tid) = \{\tsid \in \T_{\aexec} \mid (\tsid \xrightarrow{\PO_{\aexec}(\tid)} \_) 
 \vee (\_ \xrightarrow{\PO_{\aexec}} \tsid \}$. Without loss of generality, we always assume that $\T_{\aexec}(\tid) = \{ \tsid_{\tid}^{n} \mid 
 \exists m \geq 0 \wedge n \leq m \}$, and that $\tsid_{\tid}^{n} \xrightarrow{\PO_{\aexec}(\tid)} \tsid_{\tid}^{m}$ if and 
 only if $n < m$.
 For $o \in \Op$, we also write $o \in_{{\aexec}} \tsid$ in lieu of $o \in \TtoOp{T}_{\aexec}(\tsid)$.
We often write $\tsid \xrightarrow{\VIS_{{\aexec}}} \tsid'$ in lieu of $(\tsid, \tsid') \in \xrightarrow{\VIS_{{\aexec}}}$, 
and similarly for other relations. We also commit an abuse of notation, and let $\PO_{{\aexec}} = 
\bigcup_{\tid \in \ClSet_{\aexec}} \PO(\tid)$. Henceforth, it will always be clear from the context 
whether the symbol $\PO_{\aexec}$ refers to a function from client identifiers to a relation between transaction, or to a relation
between transactions. 
\ac{This is way too much notation. Move into a figure.}
Given two abstract executions $\aexec_1, \aexec_2$ and a subset $\T \subseteq \T_{\aexec_1} \cap \T_{\aexec_2}$, 
we say that $\aexec_1$ and $\aexec_2$ agree on $\T$ if, and only if, $\forall \tsid,\tsid' \in \T. \TtoOp{T}_{\aexec_1}(\tsid) = 
\TtoOp{T}_{\aexec_2}(\tsid)$, $\tsid \xrightarrow{\PO_{\aexec_1(\tid)}} \tsid'$ if and only if $\tsid \xrightarrow{\PO_{\aexec_2}(\tid)} 
\tsid'$, and similarly for $\VIS_{\_}$ and $\AR_{\_}$.

Given an abstract execution $\aexec$, a key $\key{k}$ and a transaction identifier $\tsid \in \dom(\T_{\aexec})$, we 
define $\visibleWrites_{{\aexec}}(\key{k}, \tsid) = \{ \tsid' \mid \tsid' \xrightarrow{\VIS_{{\aexec}}} \tsid \wedge (\WR\;\key{k}:\_) 
\in_{\aexec} \tsid' \}$.

\begin{definition}
A \emph{resolution policy} \RP is a function $\RP: \aexec \times \powerset{\TxID} \rightarrow \powerset{\Snapshots}$ 
such that, for any $\aexec_1, \aexec_2$ that agree on a subset of transactions $\T$, then $\RP(\aexec_1, \T) = 
\RP(\aexec_2, \T)$.
\end{definition}
Intuitively, $\RP(\aexec, \T)$ denotes the set of possible snapshot that a client would observe when executing a transaction 
in a state of the system corresponding to the abstract execution $\aexec$, when observing only the subset 
of transactions in $\T$, and when using the resolution policy $\RP$. The first condition on resolution policies 
states that the possible snapshots observed do not depend on the behaviour of any other transaction other 
than $\T$.
We will mainly work with two different 
resolution policies, which are illustrated below. 

\begin{example}
The \emph{Last Write Wins} execution policy $\RP_{\LWW}$ is defined by letting, for a given abstract execution 
$\aexec$ and a set $\T \subseteq \T_{\aexec}$, $\RP_{\LWW}(\aexec, \T) = \h_{\aexec, \T}$, where
\[
\h_{\aexec, \T} = \lambda \key{k}.
\begin{cases}
n & \text{if } \T_{\key{k}} \neq \emptyset \wedge (\WR\; \key{k}: n) \in_{\aexec} \max_{\AR_{\aexec}}(\T_{\key{k}}), \text{where } 
\T_{\key{k}} (\T \cap \{\tsid \mid \tsid \in_{\aexec} \WR k: \_ \})\\
0 & \text{otherwise} 
\end{cases}
\]

The \emph{anarchic} execution policy $\RP_{\bot}$ is defined by letting $\RP_{\bot}(\aexec, \T) = \Snapshots$ for any $\aexec, \T$.
\end{example}

\begin{definition}
An abstract execution $\aexec$ satisfies an execution policy $\RP$, written $\aexec \models \RP$, if and only if, 
\[
\forall \tsid \in \T_{\aexec}. \exists \h \in \RP(\aexec, \VIS_{\aexec}^{-1}(\tsid)).\;
\forall \key{k} \in \Addr, n \in \nat.\; (\RD\;\key{k}: n) \in_{\aexec} \tsid \implies 
h(\key{k}) = n.
\]
\end{definition}

\paragraph{Axiomatic Specification of Consistency Models}
\ac{This will go in a Figure. We never encoded session guarantees into abstract 
executions, sot that they will need to be checked. Guarantees that have never been 
proved to be sound are marked with (?)}

A consistency model consist of a set of abstract executions. 
To specify consistency models, we fix a resolution policy and 
a set of axioms $\calA$ that pose conditions on the relations $\PO_{\aexec}, \VIS_{\aexec}, \AR_{\aexec}$. 
\begin{definition}
Formally speaking, we define an axiom $\Ax$ is a function from abstract executions to relations between transactions: 
$\Ax: \aeset \rightarrow \powerset{\TxID \times \TxID}$, such that whenever 
$\aexec_1, \aexec_2$ agree on a set of transactions $\T$, then whenever $(\tsid, \tsid') \in \Ax(\aexec_1)$ 
for some $\tsid, \tsid' \in \T$, then $(\tsid, \tsid') \in \Ax(\aexec_2)$.

For a given axiom $\Ax$ and abstract execution $\aexec$, 
we say that $\aexec$ satisfies $\Ax$, written $\aexec \models \Ax$ if $\Ax(\aexec) \subseteq \VIS_{\aexec}$. 
For a set of axioms $\calA$ and an abstract execution $\aexec$, we write $\aexec \models \calA$ if 
and only if $\aexec \models \Ax$ for any $\Ax \in \calA$. 
\end{definition}

For a consistency model specification 
$\CM = (\RP, \calA)$, we write $\aexec \models \CM$ if and only if $\aexec \models \RP$, 
and $\aexec \models \calA$. We also let $\aeset(\CM) = \{ \aexec \mid \aexec \models \CM \}$.

\begin{proposition}
\label{prop:ax2ar}
Let be an $\Ax$ axiom, and let $\aexec$ be an abstract execution $\aexec \models \Ax$. 
Whenever $(\tsid, \tsid') \in \Ax(\aexec)$, then $\tsid \xrightarrow{\AR_{\aexec}} \tsid'$.
\end{proposition}

\begin{proof}
Because $\aexec \models \Ax$, $(\tsid, \tsid') \in \Ax(\aexec)$ implies that 
$\tsid \xrightarrow{\VIS_{\aexec}} \tsid'$, and therefore $\tsid \xrightarrow{\AR_{\aexec}} \tsid'$.
\end{proof}

We leave the language used for specifying the function $\F$ in axioms of consistency models unspecified, though 
we remark that all our examples can be captured using the style of specification previously 
proposed in \cite{laws}. In the examples below, the notation  
$[\WR(\key{k})]_{\aexec} = \{ (\tsid, \tsid) \mid (\RD\; \key{k}: n) \in_{\aexec} \tsid \}$, and 
similarly for other operations. 

\begin{itemize}
\item Read Your Writes (?): read operations reflect previous writes
\[
[\WR(\key{k})] ; \PO ; [\RD(\key{k})] \subseteq \VIS
\]
\item Monotonic Reads (?): successive reads reflect a non-decreasing set of writes
\[
\VIS; [\RD] \PO ; [\RD] \subseteq \VIS
\]
\item Write Follows Reads (?): writes are propagated after reads on which they depend 
\[
\VIS; [\RD]; \PO; [\WR] \subseteq \VIS 
\]
\item Monotonic Writes (?): writes are propagated after writes that logically precede them 
\[ 
[\WR]; \PO; [\WR]; \VIS \subseteq \VIS 
\]
\item Strong sessions: sessions appear to be sequentially consistent: 
\[ 
\PO \subseteq \VIS 
\]
\item Causal Consistency: strong sessions + visibility is transitive, 
\[ 
\VIS ; \VIS \subseteq \VIS
\]
\item Consistent Prefix: strong sessions + if transaction $\tsid_1$ observes transaction $\tsid_2$, it also observes 
anything that precedes $\tsid_2$ in the arbitration order: 
\[
\AR ; \VIS \subseteq \VIS
\]
\item Update Atomic: transactions that update one same object, do not execute concurrently: 
\item PSI: update atomic + causal consistency, SI: update atomic + prefix consistency, 
\item Serialisability: visibility is a total order 
\[
\AR \subseteq \VIS 
\]
\end{itemize}


\subsection{Operational Semantics using Abstract Executions}
We give an alternative operational semantics of transactions, that uses abstract executions as states. 
In this semantics judgements take the form 
\[
\langle \aexec, \thdstackFun, \prog \rangle \absrightarrow_{\CM} \langle \aexec', \thdstackFun', \prog' \rangle
\]
\ac{Want to change the symbol for the arrow, but if I try to change to \emph{rightarrowtriangle} then 
I need to load the package \emph{stmaryrd}, which in turn causes latex to load too many math alphabets. 
Need to find a solution for this.}
where $\CM = (\RP, \calA)$ is an axiomatic specification of a consistency model
consisting of a resolution policy and a set of axioms, and $\thdstackFun: \tidset \rightharpoonup \ThdStacks$ 
is a partial mapping from client identifiers to stacks.

\begin{figure}
\begin{center}
\begin{tabular}{|@{}cc@{}|}
\hline
\multicolumn{2}{|@{}c@{}|}{
$
\infer[{\scriptstyle \ruleATX}]{\langle \aexec, \thdstackFun, \tid: [\trans] \rangle \absrightarrow_{\CM} \langle \aexec', \thdstackFun', \tid: \nil \rangle}
{\begin{array}{c}
\CM = (\RP, \calA) \qquad \T \subseteq \T_{\aexec} \qquad
\h \in \RP(\aexec, \T) \\
\langle \h, \thdstack, \emptyset, \trans \rangle \rightarrow^{\ast} \langle \_, \thdstack', \mathcal{O}, \nil \rangle \qquad \thdstackFun(\tid) = \sigma 
\qquad \thdstackFun' = \thdstackFun[\tid \mapsto \thdstack']\\
%\aexec' = \appendTx_{\tsid}(\aexec, \tid, \T, \mathcal{O}) \qquad \aexec' \models \calA \qquad \thdstackFun' = 
%\thdstackFun[\tid \mapsto \thdstack'] 
\aexec' = \appendTx_{\tsid}(\aexec, \tid, \T, \mathcal{O}) \qquad 
\forall \Ax \in \calA. \; \{\tsid' \mid (\tsid', \tsid) \in \Ax(\aexec') \} \subseteq \T 
\end{array}
}
$
}\\
\hline
\end{tabular}
\end{center}
\caption{Rules of the Operational Semantics of Abstract Executions.}
\end{figure}

The rules of the semantics for executing transactional code are the same of Figure \ref{fig:opsem.tx}.
Selected rules of the semantics of commands and programs are given in Figure \ref{fig:opsem.absexec}. 
\ac{Going to type down only the rule for tranactions, as all the other rules are analogous 
to the rules of Figure \ref{fig:opsem.cmd} and \ref{fig:opsem.prog}.}

As for the MKVS semantics,  the only non-trivial rule in the semantics of commands and programs 
is the one modelling a client executing a transaction, Rule \ruleATX.
The function $\appendTx_{\tsid}$ in the premiss of this rule takes in input an abstract execution 
$\aexec$, a client $\tid$, a subset $\T \subseteq \T_{\aexec}$, and a set of operations $\mathcal{O}$. 
It then returns a new abstract execution, corresponding to the result of client $\tid$, observing the
transactions included in $\T$, executing a transaction with set of operations 
are $\mathcal{O}$. The set of transactions $\T$ is used, together with the resolution policy 
$\RP$ of the consistency model, to determine one snapshot under which the transactional 
code $\ptrans{\trans}$ is executed. Upon executing this code, we obtain a set of operations 
$\mathcal{O}$, which is then associated to the transaction identifier $\tsid$
The transaction is obtained from $\aexec$ by appending the novel transaction $\tsid$
in $\PO_{\aexec}(\tid)$; the arbitration order of $\aexec$ is also obtained by extending 
$\AR_{\aexec}$ so that $\tsid$ is the last transaction in such an order; finally, the visibility 
relation is obtained by extending $\VIS_{\aexec}$ with the set $\{(\tsid', \tsid) \mid \tsid' \in \T\}$, 
as to reflect the fact that the transactions in $\T$ have been observed by $\tsid'$. 
\begin{definition}
Let $\aexec$ be an abstract execution, $\tsid$ be a transaction identifier such 
that $\tsid \notin \T_{\aexec}$,  $\tid$ be a client in $\I_{\aexec}$, $\T$ be a set of transactions, 
and $\mathcal{O}$ be a set of operations. Then we let 
$\appendTx_{\tsid}(\aexec, \tid, \T, \mathcal{O})$ be the abstract execution $\aexec'$ defined as 
follows: 
\[
\begin{array}{lcl}
\TtoOp{T}_{\aexec'} &=& \TtoOp{T}_{\aexec}[\tid \mapsto \mathcal{O}]\\
\PO_{\aexec'} &=& \PO_{\aexec}\left[\tid \mapsto \left( \PO_{\aexec}(\tid) \cup \{(\tsid', \tsid) \mid \tsid' \in \T_{\aexec}(\tid)\} \right) \right]\\
\VIS_{\aexec'} &=& \VIS_{\aexec} \cup \left(\{(\tsid', \tsid) \mid \tsid' \in \T\}\right)\\
\AR_{\aexec'} &=& \AR_{\aexec} \cup \left(\{(\tsid', \tsid) \mid \tsid' \in \T_{\aexec}\}\right)
\end{array}
\]
\end{definition}

\begin{proposition}
\label{prop:txappend.wellformed}
Let $\aexec$ be an abstract execution, $\tid$ be a client, $\tsid \notin \T_{\aexec}$, $\T \subseteq \T_{\aexec}$ and 
$\mathcal{O}$ be a set of operations. Then $\appendTx_{\tsid}(\aexec, \tid, \T, \mathcal{O})$ is an abstract 
execution.
\end{proposition}

\begin{proposition}
\label{prop:appendTx.RPpreserved}
Let $\aexec$ be an abstract execution, $\tsid \notin \T_{\aexec}$, $\T \subseteq \T_{\aexec}$ 
and $\mathcal{O}$ be a set of operations. Let also $\aexec' = \appendTx_{\tsid}(\aexec, \tid, \T, \mathcal{O})$. 
Then $\aexec$ and $\aexec'$ agree on $\T$. 
\end{proposition}

\paragraph{Faithfulness of the Semantics.}
Next, we prove that the operational semantics of weak consistency models 
is \emph{faithful} with respect to the axiomatic specification of any consistency 
model. 
Roughly speaking, an operational semantics is faithful with respect to a consistency model 
$\CM$.
if it captures, for any program $\prog$, all the abstract executions that $\prog$ can exhibit 
under $\CM$. By definition, consistency models are sets of abstract executions, 
each corresponding to the behaviour that could be obtained by the MKVS when interacting 
with one or more clients. To formalise the notion of all possible behaviours 
allowed by a client, it suffices then to define the set $\interpr{\prog}_{\anarchicsmall}$ of abstract executions 
produced by program $\prog$ under the most permissive (or anarchic) 
consistency model, that is one in which 
the results of reads issued  by clients to the data-store are non-deterministic. 
\ac{I know that the $\anarchic$ symbol will be replaced by something else, this 
is just a note to say that when it will happen, I will be very sad.}
Then the set of all possible executions of $\prog$ under $\CM$ is defined as
\[
\interpr{\prog}_{\CM} = 
\interpr{\prog}_{\CM_{\anarchicsmall}} \cap \aeset(\CM).
\]

Given a program $\prog$, the set of abstract executions $\interpr{\prog}_{\anarchicsmall}$
is defined via the consistency model specification $\CM_{\anarchicsmall} = (\RP_{\bot}, \emptyset)$, 
where we recall that $\RP_{\bot}$ is the demonic resolution policy. 
\begin{definition}
Let $\prog$ be a program, and $\CM$ be a consistency model specification. Then 
\[ 
\interpr{\prog}_{\CM} = \{ \aexec \mid \langle \aexec_0, \thdstackFun_0 \rangle \absrightarrow_{\CM} 
\cdots \absrightarrow_{\CM} 
\langle \aexec, \thdstackFun, \prod_{i=1}^{n} \nil \wedge n \geq 0 \rangle \}
\]
where $\aexec_0$ is the only abstract execution for which $\T_{\aexec_{0}} = \emptyset$, 
and $\thdstackFun_0$ is the function that associates to each client in $\prog$, the function 
that maps local variables to $0$.
\end{definition}

The following result states that this is really the most permissive consistency model specification which is 
possible in our setting. 
\begin{proposition}
Let $\aexec$ be an abstract execution. Then $\aexec \models \CM_{\anarchicsmall}$.
\end{proposition}

Note that abstract executions capture interactions of clients with a database where 
transactions satisfy the
internal consistency guarantee: a client never observes different values for subsequent 
reads of the same key in the same transaction. In theory, transactions in an anarchic database 
would not enjoy internal consistency. On the other hand, internal consistency is often implemented 
on the client side of transactional libraries, by letting a transaction read each key from the database 
at most once, storing the value read locally and re-using it as a return value for subsequent reads 
\cite{jessy,physicsNMSI}. Therefore we can safely assume that transactions provide at least internal 
consistency, despite the database being anarchic. \ac{and the results would still apply even if 
internal consistency were dropped from the anarchic model, but we would have so many more 
complications.}

The next two lemmas are crucial to proving both correctness and faithfulness of the operational 
semantics based on abstract executions.

\begin{lemma}[Subject Reduction]
\label{lem:subj.red}
Suppose that $\langle \aexec, \thdstackFun, \prog \rangle \absrightarrow_{\CM} \langle \aexec', \thdstackFun', \prog' \rangle$. 
If $\aexec \in \aeset(\CM)$, then $\aexec' \in \aeset(\CM)$.
\end{lemma}

\begin{proof}
Let $\CM = (\RP, \calA)$ for some $\RP, \calA$.
If the transition 
\[ 
\langle \aexec, \thdstackFun, \prog \rangle \absrightarrow_{\CM} 
\langle \aexec', \thdstackFun', \prog' \rangle
\] 
is derived by any rule other than \ruleATX, then $\aexec = \aexec'$, and there is nothing 
left to prove. 
Otherwise, if the transition above has been derived using Rule \ruleATX, 
then $\aexec' = \appendTx_{\tsid}(\aexec, \tid, \T)$ for some $\tid \notin \T_{\aexec}$, 
and $\T$ such that, for any axiom $\Ax \in calA$, $\Ax(\aexec') \cap \{(\tsid', \tsid) \mid \tsid' \in \T\} 
\subseteq \VIS_{\aexec'}$. 
Suppose that $\aexec \in \aeset(\CM)$. We prove that $\aexec' \in \CM$. 
By definition, we need to prove that $\aexec' \models \RP$ and $\aexec' \models \calA$.
	\begin{itemize}
		\item  $\aexec' \models \RP$: we need to show that, 
		\begin{equation}
		\label{eq:correctness.RP1}
		\forall \tsid' \in \T_{\aexec'}.\; \exists h_{\tsid'} \in \RP(\aexec, \VIS^{-1}_{\aexec'}(\tsid')).
		\forall \key{k},n.\; (\RD\;\key{k}:n) \in_{\aexec'} \tsid' \implies h_{\tsid'}(\key{k}) = n.
		\end{equation}
		Let then $\tsid' \in \T_{\aexec'}$. 
		By definition, $\T_{\aexec'} = \T_{\aexec} \cup \{ \tsid \}$, hence either 
		$\tsid' \in \T_{\aexec}$ or $\tsid' = \tsid$. 
		\begin{itemize}
			\item Suppose first that $\tsid' \in \T_{\aexec}$, 
			and observe that $\aexec \models \RP$: that is, there exists a snapshot 
			$\h_{\tsid'} \in \RP(\aexec, \VIS^{-1}(\tsid'))$ such that whenever $(\RD\;\key{k}:n) \in_{\aexec} \tsid'$, then 
			$h_{\tsid'}(\key{k}) = n$. By definition of $\appendTx_{\tsid}$, because of the fact that 
			$\tsid' \neq \tsid$, and because $\aexec, \aexec'$ agree on $\T_{\aexec}$ 
			(Proposition \ref{prop:appendTx.RPpreserved}, 
			then  $\TtoOp{T}_{\aexec}(\tsid') = \TtoOp{T}_{\aexec'}(\tsid')$ and 
			$\VIS_{\aexec}^{-1}(\tsid') = \VIS_{\aexec'}^{-1}(\tsid)$. 
			Furthermore, $\aexec$ and $\aexec'$ agree on $\VIS^{-1}_{\aexec}(\tsid')$, 
			hence the snapshot $h_{\tsid'}$ is also included in $\RP(\aexec', \VIS_{\aexec}^{-1}(\tsid))$. 
			Combining all the facts above we get Equation \eqref{eq:correctness.RP1}, which is what we 
			wanted to prove.
%			\[
%			\exists h_{\tsid'} \in \RP(\aexec', \VIS_{\aexec'}^{-1}(\tsid)).\; \forall \key{k}. (\RD\;\key{k}: n) \in_{\aexec'} \tsid' 
%			\implies h_{\tsid}(\key{k}) = n.
%			\]
			\item Next, suppose that $\tsid' = \tsid$.
			By the premises of the Rule \ruleATX and the definition of $\appendTx_{\tsid}(\aexec, \tid, \T, \mathcal{O})$, we 
			have that $\VIS_{\aexec'}^{-1}(\tsid) = \T$, and $\TtoOp{T}_{\aexec'}(\tsid) = \mathcal{O}$. 
			Also, there exists a snapshot $\h_{\tsid} \in \RP(\aexec, \T)$ such that $(\h,\thdstackFun(\tid), \emptyset, \_) \rightarrow^{\ast}  
			(\_, \_, \mathcal{O}, \nil)$. It is easy to prove by induction on the number of transitions in such a derivation, that 
			whenever $(\RD\;\key{k} : n) \in \mathcal{O}$, then $h(\key{k}) = n$. Combining all these facts, 
			we obtain that 
			\[
			\exists \h_{\tsid} \in \RP(\aexec, \VIS^{-1}_{\aexec'}(\tsid)).\; \forall \key{k},n.\RD\;\key{k}:n \in_{\aexec'} \tsid 
			\implies h_{\tsid}(\key{k}) = n.
			\]
		\end{itemize}
		\item $\aexec' \models \calA$. We need to show, that for any $\Ax \in \calA$, then 
		$\Ax(\aexec') \subseteq \VIS_{\aexec'}$. 
		First, note that by Proposition \ref{prop:appendTx.RPpreserved}, $\aexec$ and $\aexec'$ 
		agree on $\T_{\aexec}$. Also, $\T_{\aexec'} = (\T_{\aexec} \cup \{\tsid\})$. 
		By definition of $\appendTx_{\tsid}(\aexec, \tid, \T, \mathcal{O})$, 
		Suppose then that  $(\tsid', \tsid'') \in \Ax(\aexec)$ for some $\tsid', \tsid''$; we have four possible cases:
		\begin{itemize}
			\item $\tsid' \in \T_{\aexec}$, $\tsid'' \in \T_{\aexec}$. Because $\aexec, \aexec'$ agree on $\T_{\aexec}$ 
			(Proposition \ref{prop:appendTx.RPpreserved}), 
			it follows that $(\tsid', \tsid'') \in \Ax(\aexec') \implies (\tsid', \tsid'') \in \Ax(\aexec)$. Because 
			Because $\aexec \models \Ax$, we have that $\tsid' \xrightarrow{\VIS_{\aexec}} \tsid''$. Finally, 
			because $\aexec$ and $\aexec'$ agree on $\T_{\aexec}$, we obtain that $\tsid' \xrightarrow{\VIS_{\aexec'}} \tsid''$. 
			\item $\tsid' \in \T_{\aexec}$, $\tsid'' = \tsid$. The premise of Rule \ruleATX explicitly requires that 
			because $(\tsid', \tsid) \in \Ax(\aexec')$, then $\tsid' \in \T$, and by definition $\tsid' \xrightarrow{\VIS_{\aexec'}} \tsid$.
			\item $\tsid' = \tsid, \tsid'' \in \T_{\aexec}$. This case cannot happen: because $(\tsid, \tsid') \in \Ax(\aexec')$, 
			then by Proposition \ref{prop:ax2ar} it follows that $\tsid \xrightarrow{\AR_{\aexec'}} \tsid'$. By definition of 
			$\appendTx_{\aexec}$, it follows that $\tsid' \xrightarrow{\AR_{\aexec'}} \tsid$; this contradicts the 
			fact that $\aexec'$ is a well-defined abstract execution (Proposition \ref{prop:txappend.wellformed},
			\item $\tsid' = \tsid'' = \tsid$; this case is also not possible, because it would violate the fact 
			that $\aexec'$ is a well-defined abstract execution.
		\end{itemize}
	\end{itemize}
\end{proof}

\begin{lemma}[Subject Expansion]
\label{lem:subj.exp}
\ac{Thanks Wikipedia for telling me what's the opposite of subject reduction.}
Suppose that $\langle \aexec, \thdstackFun, \prog \rangle \absrightarrow_{\CM_{\anarchicsmall}} \langle \aexec', \thdstackFun', \prog' \rangle$. 
If $\aexec' \in \aeset(\CM)$, then $\aexec \in \aexec(\CM)$.
\end{lemma}

%\begin{lemma}
%\label{lem:transition.cm}
%Suppose that $\langle \aexec, \thdstackFun, \prog \rangle \absrightarrow_{\CM_{\anarchicsmall}}\langle \aexec', \thdstackFun', 
%\prog' \rangle$, $\aexec' \in \aeset(\CM)$. Then $\aexec \in \aeset(\CM)$ if and only if $\aexec' \in \aeset(\CM)$.
%\ac{Wrong Statement. Split into two lemmas: subject reduction - $\absrightarrow_{\CM}$ preserves abstract executions in 
%$\CM$, and subject regression $\absrightarrow_{\CM_{\anarchic}}$ leads to executions in $\CM$ only if it started from executions 
%in $\CM$.}
%\end{lemma}

\begin{proof}
Suppose that $\aexec' \in \aeset(\CM)$. We need to prove that $\aexec \in \aeset(\CM)$. 
	This amounts to proving that $\aexec' \models \RP$, and $\aexec' \models \calA$. 
	\begin{itemize}
		\item $\aexec \models \RP$. That is, 
		\begin{equation}
		\label{eq:aexec1.rp}
		\forall \tsid' \in \aexec. \exists \h_{\tsid'} \in \RP(\aexec, \VIS^{-1}_{\aexec}(\tsid')). \forall \key{k},n,\; (\RD\;\key{k}:n) \in_{\aexec} \tsid' \implies \h(\key{k}) = n.
		\end{equation}
		Because $\aexec' \models \RP$ by hypothesis, we have that 
		\begin{equation}
		\label{eq:exec2.rp}
		\forall \tsid'' \in \aexec'. \exists \h_{\tsid} \in \RP(\aexec, \VIS^{-1}_{\aexec}(\tsid'')).\forall \key{k},n\; (\RD\;\key{k}: n) \in_{\aexec'} \tsid'' \implies \h(\key{k}) = n.
		\end{equation}
		Now note that:
		\begin{enumerate}
			\item\label{item:rp1} the definition of $\appendTx$ ensures that $\T_{\aexec} \subseteq \T_{\aexec'}$
			\item\label{item:rp2} By Proposition \ref{prop:appendTx.RPpreserved}, $\aexec'$ and $\aexec$ agree on all subsets of $\T_{\aexec}$. 
			\item\label{item:rp3} In particular, a consequence of \eqref{item:rp2} is that $\TtoOp{T}_{\aexec}(\tsid') = \TtoOp{T}_{\aexec'}(\tsid')$. 
			\item\label{item:rp4} Another consequence of \eqref{item:rp2} and the definition of resolution policy, is that 
			$\RP(\aexec, \T) = \RP(\aexec', \T)$ for any $\T \subseteq \T_{\aexec}$, 
			\item\label{item:rp5} Because $\tsid' \in \T_{\aexec}$ and $\aexec', \aexec$ agree on $\T_{\aexec}$, we have that 
		$\VIS^{-1}_{\aexec}(\tsid') = \VIS^{-1}_{\aexec'}(\tsid')$; together with \eqref{item:rp4}, and the fact that clearly 
		$\VIS^{-1}_{\aexec}(\tsid) \subseteq \T_{\aexec}$, is that $\RP(\aexec, \VIS_{\aexec}^{-1}(\tsid')) = \RP(\aexec', \VIS_{\aexec'}^{-1}(\tsid')$.
		\end{enumerate}
		Combining Equation \eqref{eq:exec2.rp} with items \eqref{item:rp1}, \eqref{item:rp3} and \eqref{item:rp5}, 
		we obtain exactly Equation \eqref{eq:aexec1.rp}.
		\item $\aexec \models \calA$. Fix an axiom $\Ax \in calA$. We need to prove that, whenever $(\tsid', \tsid'') \in Ax(\aexec)$ for 
		some $\tsid, \tsid' \in \T_{\aexec}$, then $\tsid' \xrightarrow{\VIS_{\aexec}} \tsid''$. Let then $(\tsid', \tsid'') \in \Ax(\aexec)$. 
		Because $\aexec$ and $\aexec'$ agree on $\T_{\aexec}$, and $\tsid', \tsid'' \in \T_{\aexec}$, 
		then $(\tsid', \tsid'') \in \Ax(\aexec')$, and therefore $\tsid' \xrightarrow{\VIS_{\aexec'}} \tsid''$ because of the assumption 
		that $\aexec' \models \Ax$. It remains to note that $\aexec, \aexec'$ agree on $\T_{\aexec}$ to infer 
		that $\tsid' \xrightarrow{\VIS_{\aexec}} \tsid''$.
\end{itemize}
\end{proof}

\begin{lemma}[Predicate Conversion (?)]
\label{lem:pred.conv}
If $\langle \aexec, \thdstackFun, \prog \rangle \absrightarrow_{\CM_{\anarchicsmall}} \langle \aexec', \thdstackFun, \prog' \rangle$ 
and $\aexec' \in \aeset(\CM)$, then $\langle \aexec, \thdstackFun, \prog \rangle \absrightarrow_{\CM} \langle \aexec', \thdstackFun', \prog' \rangle$. 
\end{lemma}

\begin{proof}
We only consider the case where the transition $\langle \aexec, \thdstackFun, \prog \rangle \absrightarrow_{\CM_{\anarchicsmall}} 
\langle \aexec', \thdstackFun', \prog' \rangle$ is inferred using Rule \ruleATX.
In this case, the premiss of the rule ensures that there exist $\h, \tsid \notin \T_{\aexec}, \tid, \T \subseteq \T_{\aexec}, \mathcal{O}$ such that:
\begin{enumerate}
\item \label{item:pconv.1} $\aexec' = \appendTx_{\tsid}(\aexec, \tid, \T, \mathcal{O})$, 
\item \label{item:pconv.2} $\T = \VIS_{\aexec'}^{-1}(\tsid)$,
\item \label{item:pconv.4} $\TtoOp{T}_{\aexec'}(\tsid) = \mathcal{O}$,
\item \label{item:pconv.3} $\prog = \tid: \ptrans{\trans}$ for some $\trans$, and $\langle \h, \thdstackFun(\tid), \emptyset, \trans \rangle 
\rightarrow^{\ast} \langle \_, \thdstack', \mathcal{O}, \nil \rangle$, and $\thdstackFun' = \thdstackFun[\tid \mapsto \thdstack']$, 
\end{enumerate}
We prove that, under the hypothesis that $\aexec' \in \aeset(\CM)$, all the premiss required to infer 
the transition $\langle \aexec, \thdstackFun, \prog \rangle \absrightarrow_{\CM} \langle \aexec, thdstackFun', \prog' \rangle$ 
are also satisfied. To this end, let $\CM = (\RP, \aexec)$. Recall that $\prog = \ptrans{\trans}$ for some $\trans$.
\begin{itemize}
\item there exists a snapshot $\h' \in \RP(\aexec, \T)$ such that $\langle \h', \thdstackFun(\tid), \emptyset, \trans 
\rightarrow^{\ast} \langle \_, \thdstack', \mathcal{O}, \nil \rangle$. 
Because $\T = \VIS^{-1}_{\aexec'}(\tsid)$, and $\aexec' \models \RP$,  we know that there exists a 
snapshot $\h' \in \RP(\aexec, \T)$ such that whenever $(\RD\;\key{k}: n) \in \mathcal{O}$, then 
$\h(\key{k}) = n$.

By Item \eqref{item:pconv.3}, 
we also have that $\langle \h, \thdstackFun(\tid), \emptyset, \trans \rangle 
\rightarrow^{\ast} \langle \_, \thdstack', \mathcal{O}, \nil \rangle$, so that it suffices 
to show that $\h \in \RP(\aexec, \T)$;  Items \eqref{item:pconv.4} and \eqref{item:pconv.3} imply that whenever $(\RD\; \key{k}: n) \in_{\aexec'} \tsid'$, 
then  $\h(\key{k}) = n$. In particular, we have that whenever $(\RD\;\key{k}: n) \in_{\aexec'} \tsid'$, then
$h'(\key{k}) = h(\key{k}) = n$. We can perform a simple induction on the length of the derivation 
\[
\langle \h, \thdstackFun(\tid), \emptyset, \trans 
\rightarrow^{\ast} \langle \_, \thdstack', \mathcal{O}, \nil \rangle
\]
using the assumption that $\h(\key{k}) = h'(\key{k})$ for any $(\RD\;\key{k}: n) \in \mathcal{O}$, to 
prove that
\[
\langle \h', \thdstackFun(\tid), \emptyset, \trans 
\rightarrow^{\ast} \langle \_, \thdstack', \mathcal{O}, \nil \rangle.
\]

\item let $\Ax \in \calA$. Let also $\tsid'$ be such that $(\tsid', \tsid) \in \Ax(\aexec')$. Because 
we are assuming that $\aexec' \models \calA$, then it must be $\tsid' \xrightarrow{\VIS_{\aexec'}} \tsid$. 
By Item \eqref{item:pconv.2}, this implies that $\tsid' \in \T$.
\end{itemize}
\end{proof}

\begin{theorem}
For any program $\prog$ and consistency model specification 
$\CM$: 
\begin{description}
\item[Correctness: ] $\interpr{\prog}_{\CM} \subseteq \aeset(\CM)$, 
\item[Faithfulness: ] $\interpr{\prog}_{\CM_{\anarchicsmall}} \cap \aeset(\CM) \subseteq \interpr{\prog}_{\CM}$. 
\end{description}
\end{theorem}

\begin{proof}
Let $\prog$ be a program, and $\CM$ be a consistency model specification.
\begin{description}
\item[Correctness:] Let $\aexec \in \interpr{\prog}_{\CM}$. By definition, we have that 
\[ 
\langle \aexec_0, \thdstackFun_{0}, \prog \rangle \absrightarrow_{\CM} \cdots \absrightarrow_{\CM} \langle \aexec_{n}, \thdstackFun_{n}, 
\prog_{n} \rangle 
\]
for some $n \geq 0$, where $\T_{\aexec_0} = \emptyset$, $\thdstackFun_{0}$ maps every client of $\prog$ into the initial stack, 
$\aexec_n = \aexec$
and $\prog_{n} = \prod_{\tid \in \Clients(\prog)} \tid: \nil$. It is immediate to observe that 
$\aexec_0 \in \aeset(\CM)$. Now a simple induction on $n$, using Lemma \ref{lem:subj.red} 
in the inductive step, shows that $\aexec_n \in \aeset(\CM)$.
\item[Faithfulness:] let $\aexec \in \interpr{\prog}_{\CM_{\anarchicsmall}} \cap \aeset(\CM)$. 
That is, there exists a sequence of transitions 
\[
\langle \aexec_0, \thdstackFun_{0}, \prog \rangle \absrightarrow_{\CM_{\anarchicsmall}} \cdots \absrightarrow_{\CM_{\anarchicsmall}} \langle \aexec_{n}, \prog_n, 
\]
where $\prod_{\tid \in \Clients(\prog)} \tid: \nil \rangle$.
Using lemmas \ref{lem:subj.exp} and \ref{lem:pred.conv}, we can infer from the above a sequence of transitions
\[
\langle \aexec_0, \thdstackFun_{0}, \prog \rangle \absrightarrow_{\CM} \cdots \absrightarrow_{\CM} \langle \aexec_{n}, \thdstackFun_{n}, 
\prog_{n} \rangle,
\]
from which it follows that $\aexec \in \interpr{\prog}_{\CM}$.
\end{description}
\end{proof}

\subsection{Encoding into Key-value store semantics}
Let $\CM$ be an axiomatic specification, with $\CM = (\RP_{\LWW}, \calA)$. 
The aim of this section is that the abstract execution semantics of $\CP$ can be 
simulated by the key-value store semantics, for an appropriate execution test $\ET$. 
That is, given an execution of the form 
\[
\langle \aexec_0, \thdstackFun_{0}, \prog_0 \rangle \absrightarrow_{\CM} \cdots \absrightarrow_{\CM} \langle \aexec_{n}, \thdstackFun_{n}, \prog_{n} \rangle
\]
we want to find an equivalent execution 
\[
\langle \mathcal{C}_{0}, \thdstackFun_{0}, \prog_0 \rangle \rightarrow_{\ET} \cdots \langle \rightarrow_{\ET} \rangle \mathcal{C}_n, \thdstackFun_{n}, \prog_{n} 
\rangle.
\]

Formally speaking, we need to define what it means for two executions, defined in the abstract operational semantics and key-value semantics, respectively, 
to be equivalent. We start by defining a notion of \emph{compatibility} between abstract executions and key-value stores. 
This definition follows the intuition that an abstract execution $\aexec$ is compatible with a key-value store $s$ if 
any observation made by a client in the former can be simulated by an observation made by a client in the latter, and vice-versa. 
For abstract executions, observations of clients amount to select a subset of visible transaction and extract 
a snapshot from there, according to the last write wins resolution policy. For key-value stores, 
observations of clients corresponds to selecting a view over the key-value store, and extracting the snapshot 
associated with such a view.

\begin{definition}
Let $\aexec$ be an abstract execution, $\hh$ be a key-value store. 
Let $\Views_{\hh}$ be the set of atomic views that are well-formed with 
respect to $\hh$.
\ac{The constraint that views are atomic is necessary for later results.}
We say that $\aexec$ is compatible with $\hh$, written 
$\aexec \compatible \hh$, if and only if there exists a  mapping 
$f: \powerset(\T_{\aexec}) \rightarrow \Views_{\hh}$
such that  
\begin{itemize}
\item for any subset $\T \subseteq \T_{\aexec}$, let$\RP_{\LWW}(\aexec, \T) = \{\h\}$; 
then $\h = \snapshot(\hh, f(\T))$, 
\item for any view $V$, there exists a subset $\T \subseteq \T_{\aexec}$ 
such that $f(\T) = V$, and $\snapshot(\aexec, V) = \h$, where 
$\RP_{\LWW}(\aexec, \T) = \{h\}$.
\end{itemize}
\end{definition}

It is always possible to convert an abstract execution that satisfies the $\RP_{\LWW}$ 
policy into a compatible 
key-value store, as the following definitions and result show.
\begin{definition}
Let $\aexec$ be an abstract execution such that $\aexec \models \RP_{\LWW}$. 
Let also $\tsid_0$ be a transaction identifier not included in $\aexec$. 
\end{definition}
\ac{Following this definition, it may be the case that we could define 
$\TxID = \{ \tsid_0 \} \cup \{ \tsid_{\tid}^{n} \mid \tid \in \tidset wedge n \in \mathbb{N}\}$.
}

\paragraph{Recovering Compatible Key-value Stores}
\ac{IMPORTANT: This definition assumes that the definition of configurations in the MKVS framework has changed 
to that the set of transaction identifiers is given by $\{\tsid_0\} \cup \{ \tsid_{\tid}^{n} \mid \tid \in \tidset \wedge n \geq 0 \}$, 
and that $\tid_0$ does not play occur into $\aexec$.}
%
%In the following, we assume a special transaction identifier $\tsid_0$ that is not used 
%in abstract executions. This can be thought as an initial transaction that initialises the value of all 
%keys in the key-value store to the default value $0$.
\begin{definition}[Transaction Dependencies]
\ac{This is basically the definition of Adya's/Jade's dependencies. The encoding 
of abstract executions into key-value-stores is basically done in two steps: first 
convert the abstract execution into a dependency graph, then convert the dependency graph 
into a key-value store. Recall that, now that we have the information about the program order, 
there is a isomorphism between dependency graphs and key-value stores.}
Let $\aexec$ be an abstract execution. For each $\key{k} \in \Addr$, 
we define the relations $\RF(\key{k}), \VO(\key{k}) \subseteq (\T_{\aexec} \uplus \{\tsid_0\}) \times 
\T_{\aexec}$ as follows: 
\begin{itemize}
\item $\tsid \xrightarrow{\RF_{\aexec}(\key{k})} \tsid'$ if and only if either $\tsid = \tsid_0$ and 
$\visibleWrites_{\aexec}(\key{k}, \tsid) = \emptyset$, or 
$\tsid = \max_{\AR_{\aexec}}(\visibleWrites_{\aexec}(\key{k}, \tsid))$, 
\item $\tsid \xrightarrow{\VO_{\aexec}(\key{k})} \tsid'$ if and only if either $\tsid = \tsid_0$ 
and $(\WR\;\key{k}: \_) \in_{\aexec} \tsid'$, or $\tsid, \tsid' \in_{\aexec} (\WR\;\key{k}: \_)$ 
and $\tsid \xrightarrow{\AR_{\aexec}} \tsid'$.
\end{itemize}
\end{definition}

Given a set $X$, an element $x \in X$ and a list $l \in X^{\ast}$, we write 
$x \in l$ as a shorthand for $\exists l_1, l_2. \; l = l_1 \cdot x \cdot \l_2$. 
Here $\cdot$ is the list concatenation operator. Given $x_1, x_2 \in X$ and 
$l \in L$, we let $x_1 <_{l} x_2$ if $l = \_ \cdot x_1 \cdot \_ \cdot x_2 \cdot \_$.
\ac{The sentence above should appear much before.}

\begin{definition}
Let $\aexec$ be an abstract execution. 
%Let $\tsid_0$ be a transaction identifier 
%that is not included in $\dom(\TtoOp)$. 
We define the key-value store $\hh_{\aexec}$ as follows: 
\begin{itemize}
\item For each transaction $\tsid \in \T_{\aexec}$ and key $\key{k}$, 
we define the set $\Versions_{\aexec}(\key{k}) \subseteq \Versions$ to be the smallest set such that 
\begin{enumerate}
\item for any $\tsid \in \T_{\aexec}$ such that $(\WR\;\key{k}: n) \in_{\aexec} \tsid$,
then
$(n, \tsid, \mathcal{R}(\tsid, \key{k})) \in \Versions_{aexec}(\key{k})$, 
where $\mathcal{R}(\tsid, \key{k}) = \{\tsid' \mid \tsid \xrightarrow{\RF_{\aexec}(\key{k})} \tsid'\}$ 
\item $(0, \tsid_0, \mathcal{R}(\tsid_{0}, \key{k}) \in \Versions_{\aexec}(\key{k})$, 
where $\mathcal{R}(\tsid_{0}, \key{k}) = \{\tsid' \mid \tsid_0 \xrightarrow{\RF_{\aexec}(\key{k})} \tsid'\}$.
\end{enumerate}
\item The key-value store $\hh_{\aexec}$ is defined to be the unique one such that 
for any key $\key{k} \in \Addr$, $\nu \in \hh_{\hat{\aexec}}(\key{k})$ if and only if 
$\nu \in \Versions_{\hat{\aexec}}(\key{k})$; furthermore, for any $\nu, \nu' \in \hh_{\hat{\aexec}}(\key{k})$, 
we have that $\nu <_{\hh_{\aexec}(\key{k})} \nu'$ if and only if $\nu = (\_, \tsid, \_)$, 
$\nu' = (\_, \tsid', \_)$, and $\tsid \xrightarrow{\VO_{\aexec}(\key{k})} \tsid'$.
%\item $\progOrd_{\hat{\aexec}}$ is the function that maps each $\tid \in \dom(\I_{\aexec})$ into a list of transaction 
%identifiers which is consistent with the program order of $\tid$ in $\hat{\aexec}$, and which 
%contains $\tid_0$ as its initial element : $\tsid \in \progOrd_{\hat{\aexec}}(\tid)$ 
%if and only if either $\tsid = \tsid_0$, or $\tsid \xrightarrow{\PO_{\hat{\aexec}}} \tid$. 
%Furthermore, whenever $\tsid <_{\progOrd_{\hat{\aexec}(\tid)}} \tsid'$, then either 
%$tsid = \tsid_0$, or $\tsid \xrightarrow{\PO_{\hat{\aexec}}} \tsid'$. 
%\ac{This automatically implies that $\tsid_0$ is the first element in the program order 
%of each client.}
\end{itemize}
\end{definition}

\begin{proposition}
Let $\aexec$ be an abstract execution such that $\aexec \models \RP_{\LWW}$. 
%Then $(\hh_{\hat{\aexec}}, \pi_{\hat{\aexec}})$ is well-formed.
Then $\aexec \compatible \hh_{\aexec}$. 
\end{proposition}

\begin{proof}
For each subset $\T \subseteq \T_{\aexec}$ and key $\key{k}$, let 
$\tsid_{\T}^{\key{k}} = \tsid_0$ if 
$(\T \cap \{\tsid \mid (\WR\; \key{k}: \_) \in_{\aexec} \tsid\}) = \emptyset$, 
$\tsid_{\T}^{\key{k}} = \max_{\AR_{\aexec}}(\T \cap \{\tsid \mid (\WR\; \key{k}: \_) \in_{\aexec} \tsid\})$ otherwise.
Let also $\nu_{\T}^{\key{k}}$ be the 
 version in $\hh_{\aexec}(\key{k})$ such that $\WTx(\nu_{\T}^{\key{k}}) = \tsid_{\T}^{\key{k}}$. 
Note that the version $\nu_{\T}^{\key{k}}$ exists and is uniquely determined because of the 
definition of $\Versions_{\aexec}(\key{k})$.
Define $f = \lambda \T. V_{\T}$, where $V_{\T} = (\lambda \key{k}. \nu_{\T}^{\key{k}}$. 
We prove the following facts: 
\begin{itemize}
\item let $\h = \RP_{\LWW}(\aexec, \T)$. Then we have that, for each key $\key{k}$, 
$\h(\key{k}) = n$, where either $n = 0$ and $\tsid_{\T}^{\key{k}} = \tsid_0$, or 
$(\WR\;\key{k}: n) \in_{\aexec} \tsid_{\T}^{\key{k}}$. 
If $\tsid_{\T}^{\key{k}} = \tsid_0$, then it must be the case that 
$\nu_{\T}^{\key{k}} = (0, \tsid_0, \_)$, and therefore $\snapshot(\hh_{\aexec},V_{\T})(\key{k}) = 0$; 
otherwise, $\nu_{\T}^{\key{k}} = (n, \tsid_{\T}^{\key{k}}, \_)$, and 
$\snapshot(\hh_{\aexec}, V_{\T})(\key{k}) = n$.

\item let now $V$ be an atomic view over $\hh_{\aexec}$. For each key $\key{k}$, $\hh(\key{k}, V(\key{k}))$ 
is included in $\Versions_{\aexec}(\key{k})$. Therefore, either $\hh_{\aexec}(\key{k}, V(\key{k})) = 
(0, \tsid_0, \_)$, or there exists a transaction $\tsid_{V}^{\key{k}} \in \T_{\aexec}$ such that 
$(\WR\; \key{k}: n) \in_{\aexec} \tsid_{V}^{\key{k}}$ and $\hh_{\aexec}(\key{k}, V(\key{k})) = 
(n, \tsid_{V}^{\key{k}}, \_)$. Let $\T_{V} = \{ \tsid_{V}^{\key{k}} \mid k \in \Addr\}$. 

Note that, if $(\WR\;\key{k}: \_ ) \in_{\aexec} \tsid$ for some $\tsid \in \T_{V}$, 
then we must have that either $\tsid = \tsid_{V}^{\key{k}}$, or $\tsid \xrightarrow{\AR_{\aexec}} 
\tsid_{V}^{\key{k}}$. This is because, if it were $\tsid_{V}^{\key{k}} \xrightarrow{\AR_{\aexec}} 
\tsid$, then there would exists an index $i: V(\key{k}) < i$ such that  $\hh(\key{k}, i) = (\_, \tsid, \_)$;  
that is, the view $V$ includes the update to some $\key{k}' \neq \key{k}$ performed by $\tsid$ 
(such an update must exist because we are assuming that $\tsid \in \T_{V}$, but not the update 
to key $\key{k}$ performed by the same transaction $\tsid$. This violates the assumption that $V$ is 
atomic. 

We have proved that either $\hh(\key{k}, V(\key{k}) = (0, \tsid_0, \_)$, or $\hh_{\aexec}(\key{k}, V(\key{k})) = 
(n, \tsid_{V}^{\key{k}}, \_)$, where $(\WR\;\key{k}:n) \in_{\aexec} \tsid_{V}^{\key{k}}$.
Furthermore, whenever $\tsid \in \T_{V}$ and $(\WR\;\key{k}: \_) \in_{\aexec} \tsid$ 
then either $\tsid = \tsid_{V}^{\key{k}}$, or $\tsid \xrightarrow{\AR_{\aexec}} \tsid_{V}^{\key{k}}$. 
This implies that, for any key $\key{k}$, $f(\T_{V}) = V$; let also $\{ \h \} = \RP_{\LWW}(\aexec, \T_{V})$. 
For each key $\key{k}$, the value of $\h(\key{k})$ is the value of the write of the last transaction 
(in the arbitration order) that wrote to key $\key{k}$, or $0$ if there is no such transaction in $\T$. 
This is exactly the value of the version of $\key{k}$ to which view $V$ points in $\hh_{\aexec}$, 
that is $\h(\key{k}) = \snapshot(\hh_{\aexec}, V)(\key{k})$.

\end{itemize}
\end{proof}
