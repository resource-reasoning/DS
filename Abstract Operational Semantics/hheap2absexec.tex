\newcommand{\TtoOp}[1]{\ensuremath{\mathscr{#1}}}
\newcommand{\ClSet}{\ensuremath{\Gamma}}
\newcommand{\progOrd}{\ensuremath{\pi}}
\newcommand{\RP}{\ensuremath{\mathsf{RP}}}
\newcommand{\LWW}{\ensuremath{\mathsf{LWW}}}
\newcommand{\F}{\ensuremath{\mathcal{F}}}
\newcommand{\calA}{\ensuremath{\mathcal{A}}}
\newcommand{\appendTx}{\ensuremath{\mathsf{appendTx}}}
\newcommand{\thdstackFun}{\ensuremath{\mathsf{\Sigma}}}
\newcommand{\absrightarrow}{\ensuremath{\rightarrowtail}}
\newcommand{\allExec}{\ensuremath{\mathsf{allExecs}}}
\newcommand{\anarchic}{\raisebox{-1.5pt}{{\Large \text{\CircledA}}}}
\newcommand{\Ax}{\ensuremath{\mathsf{Ax}}}

In this Section we propose an alternative 
semantics of programs. In this semantics, states 
correspond to \emph{abstract executions} \cite{framework-concur}, 
overloaded with information regarding the transactions that are made 
visible to individual clients. 
The abstract execution semantics provides a bridge between specification 
of consistency models given in terms of MKVSs and views, and the already 
existing axiomatic ones. This section makes the following contributions: 
\begin{itemize}
\item \textbf{Done!} a definition of abstract executions, 
\item \textbf{Done!} the axiomatic specification of consistency models borrowed from cite{framework-concur},
\item \textbf{Done!} an encoding of abstract executions into MKVSs, 
\item an encoding of executions in the MKVS semantics into abstract 
executions, 
\item \textbf{Done!} an operational semantics based on abstract executions, 
which is parametric in the axiomatic specification of a consistency model given using 
the declarative style of \cite{framework-concur,SIanalysis,laws}, 
\item a proof that, the semantics is \emph{faithful} with respect to any consistency model $\CM$: 
for any program $\prog$ and consistency model specification $\CM$, 
the abstract execution semantics captures all the potential behaviours that $\prog$ can 
exhibit under $\CM$. Faithfulness is necessary to validate the soundness of 
program analysis techniques, such as program logics, with respect to arbitrary consistency model 
specifications,
\ac{Before I was referring to what I call faithfulness as completeness. I am now changing this, because 
it looks like it only created confusion. In short words, this result requires defining the anarchic model 
of transactions - one in which the database actually reads and writes non-deterministic values 
despite the client requests, and prove that any abstract execution which is 
valid w.r.t. $\CM$ and can be obtained in the anarchic semantics, can also be obtained 
in the $\CM$ semantics. More details on this later.} 
\item a proof that each of the execution tests of Section \ref{sec:cmexamples}
captures precisely the corresponding consistency model. Specifically,
\begin{description}
\item[Soundness - ] for any program $\prog$, any execution of 
$\prog$ under the MKVSs semantics using the execution 
test $\aexec_{\CM}$ $\CM$, is encoded into an abstract execution 
that satisfies the axioms of $\CM$,
\item[Completeness - ] for any program $\prog$, any 
execution of $\prog$ under the abstract execution semantics of 
$\CM$, can be projected to an execution of $\prog$ under the history 
heap semantics using the execution test $\ET_{\CM}$. An immediate 
consequence of this completeness result, and of the faithfulness 
of the abstract execution semantic with respect to $\CM$, is that the MKVS semantics 
of $\CM$ is also faithful.
\end{description}
\end{itemize}

\paragraph{Abstract Executions}

\begin{definition}
A runtime abstract execution is a tuple $\aexec = (\TtoOp{T}, \PO, \VIS, \AR)$ where: 
\begin{enumerate}
\item $\TtoOp{T} : \TransID \rightharpoonup \powerset{\Op}$ is a partial function mapping transaction 
identifiers into sets of operations; if $\tsid \in \dom(\TtoOp{T})$, then the transaction $\tsid$ happened 
at some point in the abstract execution ${\aexec}$, by performing operations $\TtoOp{T}(\T)$, 
%\item $\ClSet \subseteq \tidset$ is a set of client identifiers, 
\item $\PO: \tidset \rightharpoonup (\dom(\TtoOp{T}) \times \dom(\TtoOp{T})$ is a function modelling the 
program order of clients: that is, there exists a $\dom(\PO)$-indexed partition $\{\T_{\tid}\}_{\tid \in \dom(\PO)}$ of 
$\dom(\TtoOp{\T})$ such that, for any $\tid \in \ClSet$ $\PO(\tid)$ is a strict, total order over $\T_{\tid}$.
\item $\VIS \subseteq \dom(\TtoOp{T}) \times (\dom(\TtoOp{T}))$ is a strict, irreflexive relation, 
\item $\AR \subseteq \dom(\TtoOp{T}) \times \dom(\TtoOp{T})$ is a strict, total order such that $\VIS \subseteq \AR$.
\end{enumerate}
\end{definition}
In general, given an abstract execution ${\aexec} = (\TtoOp{T}, \ClSet, \PO, \VIS, \AR)$, we let ${\TtoOp{T}}_{\aexec} = \TtoOp{T}, 
 \PO_{\aexec} = \PO, {\VIS}_{\aexec} = \VIS, \AR_{\aexec} = \AR$. We also let $\T_{\aexec} = \dom(\TtoOp{T}_{\aexec})$, 
 $\ClSet_{\aexec} = \dom(\PO_{\aexec})$, and $\T_{\aexec}(\tid) = \{\tsid \in \T_{\aexec} \mid (\tsid \xrightarrow{\PO_{\aexec}(\tid)} \_) 
 \vee (\_ \xrightarrow{\PO_{\aexec}} \tsid \}$. For $o \in \Op$, we also write $o \in_{{\aexec}} \tsid$ in lieu of $o \in \TtoOp{T}_{\aexec}(\tsid)$.
We often write $\tsid \xrightarrow{\VIS_{{\aexec}}} \tsid'$ in lieu of $(\tsid, \tsid') \in \xrightarrow{\VIS_{{\aexec}}}$, 
and similarly for other relations. We also commit an abuse of notation, and let $\PO_{{\aexec}} = 
\bigcup_{\tid \in \ClSet_{\aexec}} \PO(\tid)$. Henceforth, it will always be clear from the context 
whether the symbol $\PO_{\aexec}$ refers to a function from client identifiers to a relation between transaction, or to a relation
between transactions.

Given an abstract execution $\aexec$, a key $\key{k}$ and a transaction identifier $\tsid \in \dom(\T_{\aexec})$, we 
define $\visibleWrites_{{\aexec}}(\key{k}, \tsid) = \{ \tsid' \mid \tsid' \xrightarrow{\VIS_{{\aexec}}} \tsid \wedge (\WR\;\key{k}:\_) 
\in_{\aexec} \tsid' \}$.

\begin{definition}
A \emph{resolution policy} \RP is a function $\RP: \aexec \times \powerset{\TxID} \rightarrow \powerset{\Snapshots}$. 
\end{definition}

intuitively, $\RP(\aexec, \T)$ denotes the set of possible snapshot that a client would observe when executing a transaction 
in a state of the system corresponding to the abstract execution $\aexec$, when observing only the subset 
of transactions in $\T$, and when using the resolution policy $\RP$. We will mainly work with two different 
resolution policies, which are illustrated below. 

\begin{example}
The \emph{Last Write Wins} execution policy $\RP_{\LWW}$ is defined by letting, for a given abstract execution 
$\aexec$ and a set $\T \subseteq \T_{\aexec}$, 
\[
\RP_{\LWW}(\aexec, \T) = \left \{ \lambda \key{k}.
\begin{cases}
n & \text{if } \T_{\key{k}} \neq \emptyset \wedge (\WR\; \key{k}: n) \in_{\aexec} \max_{\AR_{\aexec}}(\T_{\key{k}}), \text{where } 
\T_{\key{k}} (\T \cap \{\tsid \mid \tsid \in_{\aexec} \WR k: \_ \})\\
0 & \text{otherwise} 
\end{cases}
\right\}
\]

The \emph{demonic} execution policy $\RP_{\bot}$ is defined by letting $\RP_{\bot}(\aexec, \T) = \Snapshots$ for any $\aexec, \T$.
\end{example}

\begin{definition}
An abstract execution $\aexec$ satisfies an execution policy $\RP$, written $\aexec \models \RP$, if and only if, 
\[
\forall \tsid \in \T_{\aexec}. \exists \h \in \RP(\aexec, \VIS^{-1}(\tsid).\;
\forall \key{k} \in \Addr, n \in \nat.\; (\RD\;\key{k}: n) \in_{\aexec} \tsid \implies 
h(\key{k}) = n.
\]
\end{definition}

\paragraph{Axiomatic Specification of Consistency Models}
\ac{This will go in a Figure. We never encoded session guarantees into abstract 
executions, sot that they will need to be checked. Guarantees that have never been 
proved to be sound are marked with (?)}

A consistency model consist of a set of abstract executions. 
To specify consistency models, we fix a resolution policy and 
a set of axioms $\calA$ that pose conditions on the relations $\PO_{\aexec}, \VIS_{\aexec}, \AR_{\aexec}$. 
Specifically, each axiom $\Ax \in \calA$  have either the form $\F(\PO_{\aexec}, \VIS_{\aexec}, \AR_{\aexec}) \subseteq \VIS_{\aexec}$, 
or $\F(\aexec) \subseteq \AR_{\aexec}$. For a given axiom $\Ax$ and abstract execution $\aexec$, 
we write $\aexec \models \Ax$ if $\aexec$ satisfies the constraint specified by $\Ax$. 
For a set of axioms $\calA$ and an abstract execution $\aexec$, we write $\aexec \models \calA$ if 
and only if $\aexec \models \Ax$ for any $\Ax \in \calA$. For a consistency model specification 
$\CM = (\RP, \calA)$, we write $\aexec \models \CM$ if and only if $\aexec \models \RP$, 
and $\aexec \models \calA$.

We leave the language used for specifying the function $\F$ in axioms of consistency models unspecified, though 
we remark that all our examples can be captured using the style of specification previously 
proposed in \cite{laws}. In the examples below, the notation  
$[\WR(\key{k})]_{\aexec} = \{ (\tsid, \tsid) \mid (\RD\; \key{k}: n) \in_{\aexec} \tsid \}$, and 
similarly for other operations. 

\begin{itemize}
\item Read Your Writes (?): read operations reflect previous writes
\[
[\WR(\key{k})] ; \PO ; [\RD(\key{k})] \subseteq \VIS
\]
\item Monotonic Reads (?): successive reads reflect a non-decreasing set of writes
\[
\VIS; [\RD] \PO ; [\RD] \subseteq \VIS
\]
\item Write Follows Reads (?): writes are propagated after reads on which they depend 
\[
\VIS; [\RD]; \PO; [\WR] \subseteq \VIS 
\]
\item Monotonic Writes (?): writes are propagated after writes that logically precede them 
\[ 
[\WR]; \PO; [\WR]; \VIS \subseteq \VIS 
\]
\item Strong sessions: sessions appear to be sequentially consistent: 
\[ 
\PO \subseteq \VIS 
\]
\item Causal Consistency: strong sessions + visibility is transitive, 
\[ 
\VIS ; \VIS \subseteq \VIS
\]
\item Consistent Prefix: strong sessions + if transaction $\tsid_1$ observes transaction $\tsid_2$, it also observes 
anything that precedes $\tsid_2$ in the arbitration order: 
\[
\AR ; \VIS \subseteq \VIS
\]
\item Update Atomic: transactions that update one same object, do not execute concurrently: 
\item PSI: update atomic + causal consistency, SI: update atomic + prefix consistency, 
\item Serialisability: visibility is a total order 
\[
\AR \subseteq \VIS 
\]
\end{itemize}


\subsection{Operational Semantics using Abstract Executions}
We give an alternative operational semantics of transactions, that uses abstract executions as states. 
In this semantics judgements take the form 
\[
\langle \aexec, \thdstackFun, \prog \rangle \absrightarrow_{\CM} \langle \aexec', \thdstackFun', \prog' \rangle
\]
\ac{Want to change the symbol for the arrow, but if I try to change to \emph{rightarrowtriangle} then 
I need to load the package \emph{stmaryrd}, which in turn causes latex to load too many math alphabets. 
Need to find a solution for this.}
where $\CM = (\RP, \calA)$ is an axiomatic specification of a consistency model
consisting of a resolution policy and a set of axioms, and $\thdstackFun: \tidset \rightharpoonup \ThdStacks$ 
is a partial mapping from client identifiers to stacks.

\begin{figure}
\begin{center}
\begin{tabular}{|@{}cc@{}|}
\hline
\multicolumn{2}{|@{}c@{}|}{
$
\infer[{\scriptstyle(A-Tx)}]{\langle \aexec, \thdstackFun, \tid: [\trans] \rangle \absrightarrow_{\CM} \langle \aexec', \thdstackFun', \nil \rangle}
{\begin{array}{c}
\CM = (\RP, \calA) \qquad \T \subseteq \T_{\aexec} \qquad
\h \in \RP(\aexec, \T) \\
\langle \h, \thdstack, \emptyset, \trans \rangle \rightarrow^{\ast} \langle \_, \thdstack', \mathcal{O}, \nil \rangle\\
\aexec' = \appendTx_{\tsid}(\aexec, \tid, \T, \mathcal{O}) \qquad \aexec' \models \calA \qquad \thdstackFun' = 
\thdstackFun[\tid \mapsto \thdstack']
\end{array}
}
$
}\\
\hline
\end{tabular}
\end{center}
\ac{It is useless to check the constraints on the consistency models for all transactions in $\T_{\aexec}'$, 
only the edges of the form $\_\xrightarrow{\VIS_{\aexec}'} \tsid$ and $\_ \xrightarrow{\AR_{\aexec'}} \tsid$ that end in $\tsid$ should be checked.}
\caption{Rules of the Operational Semantics of Abstract Executions.}
\end{figure}

The rules of the semantics for executing transactional code are the same of Figure \ref{fig:opsem.tx}.
Selected rules of the semantics of commands and programs are given in Figure \ref{fig:opsem.absexec}. 
\ac{Going to type down only the rule for tranactions, as all the other rules are analogous 
to the rules of Figure \ref{fig:opsem.cmd} and \ref{fig:opsem.prog}.}

As for the MKVS semantics,  the only non-trivial rule in the semantics of commands and programs 
is the one modelling a client executing a transaction, Rule $(A-Tx)$.
The function $\appendTx_{\tsid}$ in the premiss of this rule takes in input an abstract execution 
$\aexec$, a client $\tid$, a subset $\T \subseteq \T_{\aexec}$, and a set of operations $\mathcal{O}$. 
It then returns a new abstract execution, corresponding to the result of client $\tid$, observing the
transactions included in $\T$, executing a transaction with set of operations 
are $\mathcal{O}$. The set of transactions $\T$ is used, together with the resolution policy 
$\RP$ of the consistency model, to determine one snapshot under which the transactional 
code $\ptrans{\trans}$ is executed. Upon executing this code, we obtain a set of operations 
$\mathcal{O}$, which is then associated to the transaction identifier $\tsid$
The transaction is obtained from $\aexec$ by appending the novel transaction $\tsid$
in $\PO_{\aexec}(\tid)$; the arbitration order of $\aexec$ is also obtained by extending 
$\AR_{\aexec}$ so that $\tsid$ is the last transaction in such an order; finally, the visibility 
relation is obtained by extending $\VIS_{\aexec}$ with the set $\{(\tsid', \tsid) \mid \tsid' \in \T\}$, 
as to reflect the fact that the transactions in $\T$ have been observed by $\tsid'$. 
\begin{definition}
Let $\aexec$ be an abstract execution, $\tsid$ be a transaction identifier such 
that $\tsid \notin \T_{\aexec}$,  $\tid$ be a client $\in \I_{\aexec}$, $\T$ be a set of transactions, 
and $\mathcal{O}$ be a set of operations. Then we let 
$\appendTx_{\tsid}(\aexec, \tid, \T, \mathcal{O})$ be the abstract execution $\aexec'$ defined as 
follows: 
\[
\begin{array}{lcl}
\TtoOp{T}_{\aexec'} &=& \TtoOp{T}_{\aexec}[\tid \mapsto \mathcal{O}]\\
\PO_{\aexec'} &=& \PO_{\aexec}\left[\tid \mapsto \left( \PO_{\aexec}(\tid) \cup \{(\tsid', \tsid) \mid \tsid' \in \T_{\aexec}(\tid)\} \right) \right]\\
\VIS_{\aexec'} &=& \VIS_{\aexec} \cup \left(\{(\tsid', \tsid) \mid \tsid' \in \T\}\right)\\
\AR_{\aexec'} &=& \AR_{\aexec} \cup \left(\{(\tsid', \tsid) \mid \tsid' \in \T_{\aexec}\}\right)
\end{array}
\]
\end{definition}

\paragraph{Faithfulness of the Semantics.}
Next, we prove that the operational semantics of weak consistency models 
is \emph{faithful} with respect to the axiomatic specification of any consistency 
model. 
Roughly speaking, an operational semantics is faithful with respect to a consistency model 
$\CM$.
if it captures, for any program $\prog$, all the abstract executions that $\prog$ can exhibit 
under $\CM$. By definition, consistency models are sets of abstract executions, 
each corresponding to the behaviour that could be obtained by the MKVS when interacting 
with one or more clients. To formalise the notion of all possible behaviours 
allowed by a client, it suffices then to define the set $\anarchic(\prog)$ of abstract executions 
produced by program $\prog$ under the most permissive (or anarchic) 
consistency model, that is one in which 
the results of reads issued  by clients to the data-store are non-deterministic. Then 
the set of all executions of $\prog$ under $\CM$ is defined as
\[
\allExec(\prog, CM) = \anarchic (\prog) \cap \CM.
\]

Given a program $\prog$, the set of abstract executions $\anarchic(\prog)$
is defined via the consistency model specification $\CM_{\anarchic} = (\RP_{\bot}, \emptyset)$, 
where we recall that $\RP_{\bot}$ is the demonic resolution policy. The following result 
states that this is really the most permissive consistency model specification which is 
possible in our setting. 
\begin{proposition}
Let $\aexec$ be an abstract execution. Then $\aexec \models \CM_{\anarchic}$.
\end{proposition}

\subsection{Encoding of Abstract Executions into configurations}
\ac{IMPORTANT: This definition assumes that the definition of configurations in the MKVS framework has changed as to include, besides 
the mapping $\kappa: \Keys \rightarrow \Versions^{\ast}$ and the mapping from clients to views, 
also a partial function $\pi: \tidset \rightharpoonup \TransID^{\ast}$ corresponding to the order 
in which transactions have been executed by clients.}

In the following, we assume a special transaction identifier $\tsid_0$ that is not used 
in abstract executions. This can be thought as an initial transaction that initialises the value of all 
keys in the key-value store to the default value $0$.
\begin{definition}[Transaction Dependencies]
\ac{This is basically the definition of Adya's/Jade's dependencies.}
Let $\hat{\aexec}$ be an abstract execution. For each $\key{k} \in \Addr$, 
we define the relations $\RF(\key{k}), \VO(\key{k}) \subseteq (\T_{\hat{\aexec}} \uplus \{\tsid_0\}) \times 
\T_{\hat{\aexec}}$ as follows: 
\begin{itemize}
\item $\tsid \xrightarrow{\RF(\key{k})} \tsid'$ if and only if either $\tsid = \tsid_0$ and 
$\visibleWrites_{\hat{\aexec}}(\key{k}, \tsid) = \emptyset$, or 
$\tsid = \max_{\AR_{\hat{\aexec}}}(\visibleWrites_{\hat{\aexec}}(\key{k}, \tsid))$, 
\item $\tsid \xrightarrow{\VO(\key{k}} \tsid'$ if and only if either $\tsid = \tsid_0$ 
and $\WR\;\key{k}: \_ \in_{\hat{\aexec}} \tsid'$, or $\tsid, \tsid' \in_{\hat{\aexec}} (\WR\;\key{k}: \_)$ 
and $\tsid \xrightarrow{\AR_{\aexec}} \tsid'$.
\end{itemize}
\end{definition}

Given a set $X$, an element $x \in X$ and a list $l \in X^{\ast}$, we write 
$x \in l$ as a shorthand for $\exists l_1, l_2. \; l = l_1 \cdot x \cdot \l_2$. 
Here $\cdot$ is the list concatenation operator. Given $x_1, x_2 \in X$ and 
$l \in L$, we let $x_1 <_{l} x_2$ if $l = \_ \cdot x_1 \cdot \_ \cdot x_2 \cdot \_$.
\ac{The sentence above should appear much before.}

\begin{definition}
Let $\hat{\aexec}$ be an abstract execution. 
%Let $\tsid_0$ be a transaction identifier 
%that is not included in $\dom(\TtoOp)$. 
We define the MKVS $(\hh_{\hat{\aexec}}, \progOrd_{\hat{\aexec}})$ as follows: 
\begin{itemize}
\item For each transaction $\tsid \in \T_{\hat{\aexec}}$ and key $\key{k}$, 
we define the set $\Versions_{\hat{\aexec}}(\key{k}) \subseteq \Versions$ to be the smallest set such that 
\begin{enumerate}
\item for any $\tsid \in \T_{\hat{\aexec}}$ such that $\WR\;\key{k}: n \in_{\aexec} \TtoOp(\tsid)$,
then
$(n, \tsid, \mathcal{R}(\tsid, \key{k})) \in \Versions_{\hat{\aexec}}(\key{k})$, 
where $\mathcal{R}(\tsid, \key{k}) = \{\tsid' \mid \tsid \xrightarrow{\RF(\key{k})} \tsid'\}$ 
\item $(0, \tsid_0, \mathcal{R}(\tsid_{0}, \key{k}) \in \Versions_{\hat{\aexec}}(\key{k})$, 
where $\mathcal{R}(\tsid_{0}, \key{k}) = \{\tsid' \mid \tsid_0 \xrightarrow{\RF(\key{k})} \tsid'\}$.
\end{enumerate}
\item The MKVS $\hh_{\hat{\aexec}}$ is defined to be the unique MKVS such that 
for any key $\key{k} \in \Addr$, $\nu \in \hh_{\hat{\aexec}}(\key{k})$ if and only if 
$\nu \in \Versions_{\hat{\aexec}}(\key{k})$; furthermore, for any $\nu, \nu' \in \hh_{\hat{\aexec}}(\key{k})$, 
we have that $\nu <_{\hh_{\hat{\aexec}(\key{k})}} \nu'$ if and only if $\nu = (\_, \tsid, \_)$, 
$\nu' = (\_, \tsid', \_)$, and $\tsid \xrightarrow{\VO_{\hat{\aexec}}(\key{k})} \tsid'$.
\item $\progOrd_{\hat{\aexec}}$ is the function that maps each $\tid \in \dom(\I_{\aexec})$ into a list of transaction 
identifiers which is consistent with the program order of $\tid$ in $\hat{\aexec}$, and which 
contains $\tid_0$ as its initial element : $\tsid \in \progOrd_{\hat{\aexec}}(\tid)$ 
if and only if either $\tsid = \tsid_0$, or $\tsid \xrightarrow{\PO_{\hat{\aexec}}} \tid$. 
Furthermore, whenever $\tsid <_{\progOrd_{\hat{\aexec}(\tid)}} \tsid'$, then either 
$tsid = \tsid_0$, or $\tsid \xrightarrow{\PO_{\hat{\aexec}}} \tsid'$. 
\ac{This automatically implies that $\tsid_0$ is the first element in the program order 
of each client.}
\end{itemize}
\end{definition}

\begin{proposition}
Let $\hat{\aexec}$ be an abstract execution that satisfies the last write wins policy. 
Then $(\hh_{\hat{\aexec}}, \pi_{\hat{\aexec}})$ is well-formed.
\end{proposition}


