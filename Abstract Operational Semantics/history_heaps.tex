%% For double-blind review submission, w/o CCS and ACM Reference (max submission space)
\documentclass[acmsmall,review,anonymous]{acmart}\settopmatter{printfolios=true,printccs=false,printacmref=false}
%% For double-blind review submission, w/ CCS and ACM Reference
%\documentclass[acmsmall,review,anonymous]{acmart}\settopmatter{printfolios=true}
%% For single-blind review submission, w/o CCS and ACM Reference (max submission space)
%\documentclass[acmsmall,review]{acmart}\settopmatter{printfolios=true,printccs=false,printacmref=false}
%% For single-blind review submission, w/ CCS and ACM Reference
%\documentclass[acmsmall,review]{acmart}\settopmatter{printfolios=true}
%% For final camera-ready submission, w/ required CCS and ACM Reference
%\documentclass[acmsmall]{acmart}\settopmatter{}


%% Journal information
%% Supplied to authors by publisher for camera-ready submission;
%% use defaults for review submission.
\acmJournal{PACMPL}
\acmVolume{1}
\acmNumber{CONF} % CONF = POPL or ICFP or OOPSLA
\acmArticle{1}
\acmYear{2018}
\acmMonth{1}
\acmDOI{} % \acmDOI{10.1145/nnnnnnn.nnnnnnn}
\startPage{1}

%% Copyright information
%% Supplied to authors (based on authors' rights management selection;
%% see authors.acm.org) by publisher for camera-ready submission;
%% use 'none' for review submission.
\setcopyright{none}
%\setcopyright{acmcopyright}
%\setcopyright{acmlicensed}
%\setcopyright{rightsretained}
%\copyrightyear{2018}           %% If different from \acmYear

%% Bibliography style
\bibliographystyle{ACM-Reference-Format}
%% Citation style
%% Note: author/year citations are required for papers published as an
%% issue of PACMPL.
\citestyle{acmauthoryear}   %% For author/year citations


%%%%%%%%%%%%%%%%%%%%%%%%%%%%%%%%%%%%%%%%%%%%%%%%%%%%%%%%%%%%%%%%%%%%%%
%% Note: Authors migrating a paper from PACMPL format to traditional
%% SIGPLAN proceedings format must update the '\documentclass' and
%% topmatter commands above; see 'acmart-sigplanproc-template.tex'.
%%%%%%%%%%%%%%%%%%%%%%%%%%%%%%%%%%%%%%%%%%%%%%%%%%%%%%%%%%%%%%%%%%%%%%
\PassOptionsToPackage{svgnames}{xcolor}
%\usepackage[usenames,dvipsnames,svgnames,table]{xcolor}

\definecolor{DarkGreen}{rgb}{0, 0.5, 0}


%% Some recommended packages.
\usepackage{booktabs}   %% For formal tables:
                        %% http://ctan.org/pkg/booktabs
\usepackage{subcaption} %% For complex figures with subfigures/subcaptions
                        %% http://ctan.org/pkg/subcaption

%\documentclass[a4paper,UKenglish]{article}%This is a template for producing LIPIcs articles. 

%%**********************************************************************************************************************************
% Note: this file is shared between several documents, please do not delete any macros.
% Maintained by Shale Xiong <sx14@ic.ac.uk>
%**********************************************************************************************************************************

\ifNonESOPMode
%
% for theorem, proof, etc.
%\usepackage{amsthm} 
%
\theoremstyle{definition}
%\newtheorem{definition}[thm]{Definition}
%\newtheorem{lemma}[thm]{Lemma}
%\newtheorem{proposition}[thm]{Proposition}
%\newtheorem{example}[thm]{Example}
\newtheorem{definition}{Definition}[section]
\newtheorem{theorem}{Theorem}[section]
\newtheorem{lemma}{Lemma}[section]
\newtheorem{proposition}{Proposition}[section]
\newtheorem{example}{Example}[section]
%
%
\usepackage{titlesec}
%
\usepackage[titletoc]{appendix}
%
%caption margin
\usepackage[margin=2cm]{caption}
%
\usepackage{bold-extra}
%
\else
%
% to fix the proof without QED from llncs
\let\proof\relax\let\endproof\relax
\usepackage{amsthm}
\fi

% typesetting enum, etc.
\usepackage{enumerate}
\usepackage[inline]{enumitem}

% For url 
\usepackage{url}

% for color 
\usepackage[usenames,dvipsnames,svgnames,table]{xcolor}

\usepackage{hyperref}
\hypersetup{
    colorlinks,
    citecolor=black,
    filecolor=black,
    linkcolor=black,
    urlcolor=black
}


% clearly for the colour needed everywhere
\usepackage{color}

% for bibliography
\usepackage[numbers,sort]{natbib}

% for the line number at the edge
\usepackage{lineno}

% for equation number
\makeatletter
\@addtoreset{equation}{section}
\makeatother
\renewcommand{\theequation}{\arabic{section}.\arabic{equation}}

% For icons
\usepackage{fontawesome}

\usepackage{dsfont}
\usepackage{amsmath}

% the inter command for operational semantices
\usepackage{proof}
%%%%%%%%%%%%%%%%% submission version start from the follows %%%%%%%%%%%%%%%%%%%%%%%


% for xspace comment used in macro
\usepackage{xspace}


\usepackage{centernot}

% sub-figure
\usepackage{subcaption}
\captionsetup{compatibility=false}

% For math font and some commands
\usepackage{amssymb,stmaryrd}
\expandafter\def\csname opt@stmaryrd.sty\endcsname
{only,shortleftarrow,shortrightarrow}
\usepackage{extpfeil}

% For the box assertion
\usepackage{varwidth}

% tikz
\usepackage{tikz}
\usetikzlibrary{positioning, shapes, decorations.pathmorphing, arrows, calc,fit,matrix}
\usepackage{tikz-cd}

% for code 
\usepackage{listings}
\lstset{%
    basicstyle=\footnotesize\ttfamily,
    breaklines=true,
    numberstyle=\scriptsize,
    numbers=left,
    breakatwhitespace=false,
    escapeinside={(*}{*)},
    captionpos=b
}
\renewcommand{\lstlistingname}{Code}

% for the substitute E[e/x]
\usepackage{xfrac}

% better frac
\usepackage{nicefrac}

% For long table, tabularx
\usepackage{ltablex}

% For show the bib entry in the main body.
\usepackage{bibentry}

% box for math env
\usepackage{empheq}

% for better math typesetting 
\usepackage{mathtools}

% for resize the font 
\usepackage{relsize}

% For multirow and multicolumn in math and table
\usepackage{multirow}

% for the smile and sad face 
\usepackage{wasysym}

% typeset rules 
\usepackage{mathpartir}

% for scalebox
\usepackage{graphicx}

%For fig floating next to text
\usepackage{wrapfig}

% For font mathsfs
\usepackage{mathrsfs}  

% for anarchy symbol circledA
\usepackage{marvosym}


% xparse for more powerful macro definition
\usepackage{xparse}

%For reference 
\usepackage{cleveref}


\usepackage{../hhmacros}
\usepackage{wasysym}
\usepackage{mathrsfs}  
%\usepackage{ stmaryrd }
\usepackage{marvosym}
%\usepackage{hheapdraw}




 
%\usepackage{microtype}%if unwanted, comment out or use option "draft"
%\graphicspath{{./graphics/}}%helpful if your graphic files are in another directory

%\bibliographystyle{plainurl}% the recommended bibstyle

%\usepackage[a4paper]{geometry}
%\usepackage[letterpaper,left=2.4cm,right=2.4cm,top=2.4cm,bottom=2.4cm]{geometry}

\usepackage{opsem}

\usepackage{etex}

%\usepackage{stackengine}

\usepackage{authblk}



%%%%%%%%%%%%%%%%%%%%%% edit mode
\newif\iflong
\longfalse  % uncomment for short version
%\longtrue  % uncomment for long version

\newif\ifdraft
%\draftfalse %uncomment for deleting notes
\drafttrue %uncomment for showing notes

%\newif\ifEditing
%\Editingtrue
%\ifEditing
%    \linenumbers
%\fi

\newif\ifComments
\Commentstrue
\newcommand{\pg}[1]{%
\ifCommentEdits
    \begin{center}
    \fbox{%
    \begin{minipage}{6.5in} \color{red}
    {\bf PG:} {\rm #1}
    \end{minipage}
    }
    \end{center}
\fi
}

\newcommand{\sx}[1]{%
\ifCommentEdits
    \begin{center}
    \fbox{%
    \begin{minipage}{0.9\textwidth} \color{blue}
    {\bf SX:} {\rm #1}
    \end{minipage}
    }
    \end{center}
\fi
}

\definecolor{darkred}{rgb}{0.5, 0, 0}
\newcommand{\azalea}[1]{%
\ifCommentEdits
    \begin{center}
    \fbox{%
    \begin{minipage}{0.9\textwidth} \color{darkred}
    {\bf AR:} {\rm #1}
    \end{minipage}
    }
    \end{center}
\fi
}

\definecolor{darkblue}{rgb}{0.3,0.0,0.7}
\newcommand{\ac}[1]{%
\ifCommentEdits
    \begin{center}
    \fbox{%
    \begin{minipage}{0.9\textwidth} \color{darkblue}
    {\bf AC:} {\rm #1}
    \end{minipage}
    }
    \end{center}
\fi
}

%%%%%%%%%%%%%%%%%%%%%% end edit mode

%\newcommand{\tr}[2]{\iflong{}\S#1\else{}\cite[\S{}#2]{ext}\fi}
%\newcommand{\tra}[2]{\iflong{}(\S#1)\else{}\cite[\S{}#2]{ext}\fi}
%
%\newcommand{\nanomalies}{A}
%\newcommand{\ngeneral}{B}
%\newcommand{\nproofs}{C}
%\newcommand{\ncompleteness}{D}
%
%%\renewcommand{\ttdefault}{cmtt}
%%\renewcommand{\sfdefault}{cmss}
%
%\newcommand{\cross}{\ding{56}}


%\renewcommand{\rmdefault}{ptm}      %Space Hacks

\renewcommand{\O}{\mathcal{O}}

%\newcommand{\theproof}{{\noindent\hskip\labelsep
%        \color{darkgray}\sffamily\bfseries \proofname.}}
        
\newcommand{\myparagraph}[1]{\textbf{\color{darkgray}\sffamily#1.}}
\newcommand{\newtext}[1]{{\color{red}{\bf #1}}}
\newcommand{\dfont}[1]{{\bf\em #1}}
% comments


\usepackage{pifont}
\usepackage[frame,all]{xy}
\usepackage{nicefrac}
\usepackage{executions}
\usepackage{execgraphs}

%\usepackage{graphicx}
%\usepackage{wrapfig}
\usepackage{proof}

\DeclareMathAlphabet{\mathpzc}{OT1}{pzc}{m}{it}

%\usepackage{mathtools}


%%%%%%%%%%%%%%%%%%%%%%%%%%%%%%%%%%% Packages above %%%%%%%%%%%%%%%%%%%%%%%


%\theoremstyle{plain}

%\newcommand{\qed}{$\Box$}
%\newenvironment{proof}{\begin{trivlist}
%\item[\hskip\labelsep{\sc Proof.}]}
%{$\hfill\Box$\end{trivlist}}
%\newenvironment{sketch}{\begin{trivlist}
%\item[\hskip\labelsep{\sc Proof sketch.}]}
%{$\hfill\Box$\end{trivlist}}
%\newenvironment{scheme}{\begin{trivlist}
%\item[\hskip\labelsep{\sc Proof scheme.}]}
%{$\hfill\Box$\end{trivlist}}


%\newcommand{\account}{\ensuremath{\mathsf{acct}}}
%\newcommand{\emptytxt}{{\ensuremath{\mathsf{empty}}}}
%\newcommand{\post}{\ensuremath{\mathsf{post}}}
%\newcommand{\comment}{\ensuremath{\mathsf{comment}}}

%\def\qed{\ifmmode\squareforqed\else{\unskip\nobreak\hfil
%\penalty50\hskip1em\null\nobreak\hfil\squareforqed
%\parfillskip=0pt\finalhyphendemerits=0\endgraf}\fi}


\newcommand{\leaveout}[1]{}

\newcommand\Tstrut{\rule{0pt}{3.5ex}}         % = `top' strut
\newcommand\Bstrut{\\[3pt]}   % = `bottom' strut

%usepackage{paralist}
%\setdefaultleftmargin{10pt}{}{}{}{}{}
%\setlength{\topsep}{3pt}
%\setlength{\itemsep}{2pt}
% \setlength{\partopsep}{0pt}
% \setlength{\parsep}{0pt}
%\setlength{\parskip}{0pt}

%\newenvironment{subitemize}{
%\begin{itemize}
% \setlength{\topsep}{0pt}
%  \setlength{\itemsep}{0pt}
%  \setlength{\parskip}{0pt}
%}{\end{itemize}}
%
%\multlinegap=3pt
%\setlength{\jot}{2pt}
%\newcommand{\ms}{\\[2pt]}

%
%\setlength{\floatsep}{10pt plus 6pt minus 2pt}
%\setlength{\textfloatsep}{10pt plus 6pt minus 3pt}
%\setlength{\intextsep}{10pt plus 6pt minus 3pt}
%\setlength{\dblfloatsep}{18pt plus 4pt minus 2pt}
%\setlength{\dbltextfloatsep}{20pt plus 4pt minus 3pt}

%\newenvironment{mywrapfigure}[3][]{
%  \floatstyle{boxed}
%  \restylefloat{figure}
%  \wrapfigure[#1]{#2}{#3}}
%  {\endwrapfigure
%    \floatstyle{ruled}
%    \restylefloat{figure}
%    }
    

% \setlength{\floatsep}{18pt plus 4pt minus 2pt}
% \setlength{\textfloatsep}{18pt plus 4pt minus 3pt}
% \setlength{\intextsep}{10pt plus 4pt minus 3pt}
% \setlength{\dblfloatsep}{18pt plus 4pt minus 2pt}
% \setlength{\dbltextfloatsep}{20pt plus 4pt minus 3pt}
%
%\makeatletter
%\renewcommand{\section}{\@startsection{section}{1}{0pt}%
%{-3ex plus -1ex minus -.2ex}{1.5ex plus.2ex}%
%{\normalfont\large\bfseries}}
%\renewcommand{\subsection}{\@startsection{subsection}{1}{0pt}%
%{-2ex plus -1ex minus -.2ex}{1ex plus.2ex}%
%{\bfseries}}
%\def \@begintheorem #1#2{%                      {name}{number}
%  \trivlist
%  \item[\hskip \labelsep \textbf{#1 #2.}]%
%  \itshape\selectfont
%  \ignorespaces}
%\newcommand{\nut}{\hspace{.35em}}
%\def \@opargbegintheorem #1#2#3{%               {name}{number}{title}
%  \trivlist
%  \item[%
%    \hskip\labelsep \textsc{#1\ #2}\nut (#3).]%
%  \itshape\selectfont
%  \ignorespaces}
%
%\def\@listI{\leftmargin\leftmargini
%            \parsep 0\p@ \@plus1\p@ \@minus\p@
%            \topsep 6\p@ \@plus2\p@ \@minus0\p@
%            \itemsep 0\p@}
%\let\@listi\@listI
%\@listi
%
%\makeatother
%
%\sloppy

% Author macros:begin %%%%%%%%%%%%%%%%%%%%%%%%%%%%%%%%%%%%%%%%%%%%%%%%
%\iflong
%\title{Specification, Verification and Program Correctness of Transactional Consistency Models using History Heaps}
%%\titlerunning{History Heaps} %optional, in case that the title is too long; the running title should fit into the top page column
%\else
%\title{Specification, Verification and Program Correctness of Transactional Consistency Models using History Heaps}
%\fi
%
%%% Please provide for each author the \author and \affil macro, even when authors have the same affiliation, i.e. for each author there needs to be the  \author and \affil macros
%\iflong
%
%\author[1]{Shale Xiong, Andrea Cerone, Philippa Gardner}
%\author[2]{Azalea Raad}
%\affil[1]{Imperial College London, UK}
%\affil[2]{MPI KaiserSlautern}
%%\authorrunning{A. Cerone} %mandatory. First: Use abbreviated first/middle names. Second (only in severe cases): Use first author plus 'et. al.'
%\else
%
%\author[1]{Shale Xiong}
%\author[1]{Andrea Cerone}
%\author[2]{Azalea Raad}
%\author[1]{Philippa Gardner}
%\affil[1]{Imperial College London, UK}
%\affil[2]{MPI Kaiserslautern}
%\fi

\begin{document}

%% Title information
\title{Weak Consistency in Transactional Systems: a Multi-version Based Operational Approach}         %% [Short Title] is optional;
                                        %% when present, will be used in
                                        %% header instead of Full Title.
%\titlenote{with title note}             %% \titlenote is optional;
                                        %% can be repeated if necessary;
                                        %% contents suppressed with 'anonymous'
%\subtitle{Subtitle}                     %% \subtitle is optional
%\subtitlenote{with subtitle note}       %% \subtitlenote is optional;
                                        %% can be repeated if necessary;
                                        %% contents suppressed with 'anonymous'


%% Author information
%% Contents and number of authors suppressed with 'anonymous'.
%% Each author should be introduced by \author, followed by
%% \authornote (optional), \orcid (optional), \affiliation, and
%% \email.
%% An author may have multiple affiliations and/or emails; repeat the
%% appropriate command.
%% Many elements are not rendered, but should be provided for metadata
%% extraction tools.

%% Author with single affiliation.
\author{Shale Xiong}
%\authornote{with author1 note}          %% \authornote is optional;
                                        %% can be repeated if necessary
%\orcid{nnnn-nnnn-nnnn-nnnn}             %% \orcid is optional
\affiliation{
  \position{Ph.D. Student}
  \department{Department of Computing}              %% \department is recommended
  \institution{Imperial College London}            %% \institution is required
  \streetaddress{Huxley Building}
  \city{London}
  \state{}
  \postcode{SW7 2AZ}
  \country{United Kingdom}                    %% \country is recommended
}
\email{shale.xiong14@imperial.ac.uk}          %% \email is recommended

%% Author with single affiliation.
\author{Andrea Cerone}
%\authornote{with author1 note}          %% \authornote is optional;
                                        %% can be repeated if necessary
%\orcid{nnnn-nnnn-nnnn-nnnn}             %% \orcid is optional
\affiliation{
  \position{Research Associate}
  \department{Department of Computing}              %% \department is recommended
  \institution{Imperial College London}            %% \institution is required
  \streetaddress{Huxley Building}
  \city{London}
  \state{}
  \postcode{SW7 2AZ}
  \country{United Kingdom}                    %% \country is recommended
}
\email{a.cerone@imperial.ac.uk}          %% \email is recommended

%% Author with single affiliation.
\author{Philippa Gardner}
%\authornote{with author1 note}          %% \authornote is optional;
                                        %% can be repeated if necessary
%\orcid{nnnn-nnnn-nnnn-nnnn}             %% \orcid is optional
\affiliation{
  \position{Professor}
  \department{Department of Computing}              %% \department is recommended
  \institution{Imperial College London}            %% \institution is required
  \streetaddress{Huxley Building}
  \city{London}
  \state{-}
  \postcode{SW7 2AZ}
  \country{United Kingdom}                    %% \country is recommended
}
\email{p.gardner@imperial.ac.uk}          %% \email is recommended

\author{Azalea Raad}
%\authornote{with author1 note}          %% \authornote is optional;
                                        %% can be repeated if necessary
%\orcid{nnnn-nnnn-nnnn-nnnn}             %% \orcid is optional
\affiliation{
  \position{Post-doctoral Researcher}
  \department{}              %% \department is recommended
  \institution{Max Planck Institute}            %% \institution is required
  \streetaddress{no idea}
  \city{Kaiserslautern}
  \state{-}
  \postcode{-}
  \country{Germany}                    %% \country is recommended
}
\email{a.raad@mpi-sws.org}          %% \email is recommended

%%%% Author with two affiliations and emails.
%\author{First2 Last2}
%\authornote{with author2 note}          %% \authornote is optional;
%                                        %% can be repeated if necessary
%\orcid{nnnn-nnnn-nnnn-nnnn}             %% \orcid is optional
%\affiliation{
%  \position{Position2a}
%  \department{Department2a}             %% \department is recommended
%  \institution{Institution2a}           %% \institution is required
%  \streetaddress{Street2a Address2a}
%  \city{City2a}
%  \state{State2a}
%  \postcode{Post-Code2a}
%  \country{Country2a}                   %% \country is recommended
%}
%\email{first2.last2@inst2a.com}         %% \email is recommended
%\affiliation{
%  \position{Position2b}
%  \department{Department2b}             %% \department is recommended
%  \institution{Institution2b}           %% \institution is required
%  \streetaddress{Street3b Address2b}
%  \city{City2b}
%  \state{State2b}
%  \postcode{Post-Code2b}
%  \country{Country2b}                   %% \country is recommended
%}
%\email{first2.last2@inst2b.org}         %% \email is recommended


%% Abstract
%% Note: \begin{abstract}...\end{abstract} environment must come
%% before \maketitle command
\begin{abstract}
Contents of this set of notes: 
History heaps. Semantics of Programs 
running under weak consistency models using history heaps as states. 
Simulation technique for comparing weak consistency models defined using 
history heaps. Verification of implementations.
\textbf{Points following Dagstuhl: Viktor seemed positive about the 
history heap work. His question was whether the framework is generic 
enough to capture the protocols that they are developing with Azalea. 
Alexey's opinion is that the framework may have some use if we 
manage to prove implementations of protocols correct. 
I would also like to have Azalea's opinion on a semantics based 
on history heaps.}
\end{abstract}


%% 2012 ACM Computing Classification System (CSS) concepts
%% Generate at 'http://dl.acm.org/ccs/ccs.cfm'.
\begin{CCSXML}
<ccs2012>
<concept>
<concept_id>10011007.10011006.10011008</concept_id>
<concept_desc>Software and its engineering~General programming languages</concept_desc>
<concept_significance>500</concept_significance>
</concept>
<concept>
<concept_id>10003456.10003457.10003521.10003525</concept_id>
<concept_desc>Social and professional topics~History of programming languages</concept_desc>
<concept_significance>300</concept_significance>
</concept>
</ccs2012>
\end{CCSXML}

\ccsdesc[500]{Software and its engineering~General programming languages}
\ccsdesc[300]{Social and professional topics~History of programming languages}
%% End of generated code


%% Keywords
%% comma separated list
\keywords{keyword1, keyword2, keyword3}  %% \keywords are mandatory in final camera-ready submission


%% \maketitle
%% Note: \maketitle command must come after title commands, author
%% commands, abstract environment, Computing Classification System
%% environment and commands, and keywords command.
\maketitle

\newcommand{\RootPath}{..}
\section{Introduction}
Transactions are the \emph{de facto} synchronisation mechanism in modern distributed databases.
To achieve scalability and performance, distributed databases  
often use weak transactional consistency guarantees. 
Much work has been done to formalise the semantics of such consistency guarantees, both
declaratively and operationally.
On the declarative side, several \emph{general} formalisms have been proposed, 
such as dependency graphs~\cite{adya} and abstract executions~\cite{ev_transactions}, to provide a unified
semantics for formulating different consistency models.  
On the operational side, the semantics of \emph{specific} consistency models have
been captured using reference implementations~\cite{si,PSI,PSI-RA}. 
However, unlike declarative approaches, there has been
little work on \emph{general} operational semantics for describing a range
of consistency models, and no work on a general operational semantics
in which to both verify protocols of distributed databases and 
enable the program analysis of clients.

As discussed in \cref{sec:conclusions}, there are several formalisms for a general operational semantics.
%~\cite{sureshConcur,alonetogether,seebelieve}. 
%\citeauthor{alonetogether} 
\cite{alonetogether} propose an operational
semantics over a global, centralised store for reasoning about clients using a program logic; 
they can model several isolation levels, but they cannot
capture consistency models of distributed data-stores, e.g.  
parallel snapshot isolation (\PSI). 
%Moreover, they do not establish the equivalence of their definitions
%of consistency model
%with existing declarative definitions in the literature. 
%\citeauthor{sureshConcur} 
\cite{sureshConcur} propose an operational semantics over abstract executions, 
rather than a concrete centralised store. This semantics captures weaker consistency models
such as \(\PSI\) and has been used to prove the robustness of applications against
a given consistency model. However, although they focus on consistency models with atomic 
visibility (a transaction observes either all or none the updates of another transaction), 
their semantics allows the interleaving of operations in different transactions, resulting in an unnecessarily complicated model.
%However, this semantics cannot model client sessions.
\cite{seebelieve} provide a trace semantics over a global
centralised store, where the behaviour of clients is formalised by the   
observations they can make on the totally ordered history of states of the system, prior to executing a transaction. 
Their framework is tailored at proving the equivalence different specifications of consistency models, 
%but it does little towards proving the correctness of protocols employed by distributed databases. 
but its usefulness  for analysing client programs is not clear, 
%program analysis for clients; in particular, we believe that in their framework the latter task would be 
%difficult,
 given that observations made by clients involve information that is not generally 
available to real world client programs, such as the total order in which transactions commit.
%or to p
%They focus on proving implementations correct. They do not consider program analysis for clients;
%indeed we believe it would be difficult  given their choice to
%keep track of the entire system.

In this paper, we introduce a general operational semantics for describing the
client-observable behaviour of distributed {atomic} transactions
(\cref{sec:overview}, \cref{sec:model}), while successfully abstracting from the 
internal details of protocols of geo-replicated and partitioned databases. In our semantics 
transactions execute atomically, preventing the interleaving of the  
operations they perform: this makes it feasible both to prove interesting properties of client applications 
of the database, and to verify that a distributed protocol correctly implements a consistency model.
Our model comprises a global, centralised
key-value store (kv-store) with {\em multi-versioning},  and
{\em client views}
inspired by the C11 operational semantics
in~\cite{promises}; 
using these mechanisms,
we record all the versions written for each key, and  let clients see only a subset 
of such versions.
Similarly to \cite{seebelieve}, our  operational semantics  is parametric in the notion of {\em
  execution test},  determining if a client with a given view is
allowed to commit a transaction; however, our notion of execution test does not rely on the 
knowledge of the whole history of states preceding a transaction. Using this approach, 
%Different execution tests give rise
%to different consistency models in our semantics. In contrast with \cite{seebelieve}, 
%we restrict observations of clients  to the current state of the system: this leads to 
%execution tests being a more faithful abstraction of the behaviour of real distributed databases protocols.
  we  are able to capture  most of the well-known consistency models in a uniform way (\cref{sec:cm}) using kv-stores and views: e.g.,  causal consistency (\CC), \PSI, snapshot isolation (\SI) and serialisability (\SER); one of our contributions is the development of a general proof technique for proving the correspondence between 
  execution tests and axiomatic specifications of consistency models using abstract executions (\cref{sec:other_formalisms}), 
  which we successfully applied to all the consistency models we consider.
  
  Our framework is thought with \emph{Atomic visibility} 
%  (a transaction 
%  sees either none or all the effects of another transaction) 
  in mind, an in fact we cannot capture popular 
  consistency models such as \emph{Read Committed}. However, because our focus is on protocols and applications employed by distributed databases, 
  most of whose guarantee atomic visibility, we do not find this constraint to not be a severe limitation.
%We define these models using kv-stores and views, 
%provide a correspondence between our kv-stores and dependency graphs, 
%We introduce novel proof techniques for demonstrating that our
%definitions of consistency models 
%are equivalent to existing declarative definitions (\cref{sec:other_formalisms}).
%

We showcase  our semantics by verifying the correctness of two database protocols, 
COPS \cite{cops} and Clock-SI \cite{clocksi}, and by analysing the robustness of simple client applications: in 
particular, we prove the robustness of a single counter against \PSI, and the robustness of multiple counters in \SI  (\cref{sec:applications}). 
%
%For the former, we show that the COPS protocol of a 
%replicated database satisfies our definition of $\CC$ and that the Clock-SI protocol of a partitioned database satisfies our definition of $\SI$.  
%For the latter, we show the robustness of applications against our consistency models: 
%we prove that a transactional library comprising a single counter is robust against $\PSI$; 
%and  that the library with multiple counters is robust against $\SI$, but not $\PSI$.  
%To our knowledge, our robustness results are the first to take into account client sessions.
%Without sessions, multiple counters can be proved to be robust against \(\PSI\) using 
%the static analysis check from \cite{giovanni_concur16}. 
We remark here that we verify protocols and analyse clients in the \emph{same} operational
semantics. By contrast, in existing literature these two tasks are carried out in \emph{different} semantics: for example, protocols are verified using abstract executions;
clients are analysed using dependency graphs; and equivalence results are used to move between the two.

%\mypar{Outline}
%The remainder of this article is organised as follows.
%In \cref{sec:overview} we give an intuitive overview of our ideas. 
%In \cref{sec:model} we present our general operational semantics. 
%In \cref{sec:cm} we show how we encode consistency models in our semantics.
%In \cref{sec:other_formalisms} we relate our formalism to existing declarative formalisms.
%In \cref{sec:applications} we showcase the applications of our semantics.  
%In \cref{sec:conclusions} we discuss related work and conclude. 



\section{semantics\label{sec:semantics}}

Assume that heap and stack are initialised to zero.

\[
    \begin{rclarray}
        \loc \in \Loc & \defeq & \Nat \\
        \val \in \Val & \defeq & \Nat \uplus \Loc \\
        \Var & \defeq & \Set{ \vx, \vy, \dots } \\
        \ts \in \Timestamp & \defeq & \Nat \\
        \hp \in \Heap & \defeq & \Loc \parfun \Val \\
        \stk \in \Stack & \defeq & \Var \to \Val \\
        \rs \in \Readset, \ws \in \Writeset & \defeq & \powerset{\Loc} \\
        \lstt = (\hp, \stk, \rs, \ws ) \in \Localstate & \defeq & \Stack \times \Heap \times \Readset \times \Writeset \\
        \op \in \Operation & \defeq & \Set{\opr, \opw} \\
        \settrans \subseteq \TransID & \defeq & \Set{ \alpha , \beta, \dots } \\
        \tshp \in \Timestampheap & \defeq & \Loc \parfun ( \Timestamp \parfun \Val \times \Operation \times \TransID) \\
        \ThreadID & \defeq & \Set{ i , j, \dots } \\
        (\tshp, \stk, \ts) \in \Threadstate & \defeq & \Timestampheap \times \Stack \times \Timestamp \\
        \tdpl \in \Threadpool & \defeq & \ThreadID \parfun \Stack \times \Timestamp \times \prog \\
        \stt \in \State & \defeq & \Timestampheap \times \Threadpool \\
    \end{rclarray}
\]

No side effect of evaluation of arithmetic expression.

\[
    \begin{syntax}{\texpr}
              \val \quad            |
        \quad \var \quad            |
        \quad \texpr + \texpr \quad |
        \quad \texpr * \texpr \quad |
        \quad \dots 
    \end{syntax}
\]

No side effect of evaluation of boolean expression.

\[
    \begin{rclarray}
        \eval{\val}_{\stk} & \defeq & \val \\
        \eval{\var}_{\stk} & \defeq & \stk(\val) \\
        \eval{\texpr_{1} + \texpr_{2}}_{\stk} & \defeq & \eval{\texpr_{1}}_{\stk} + \eval{\texpr_{2}}_{\stk}   \\
        \eval{\texpr_{1} * \texpr_{2}}_{\stk} & \defeq & \eval{\texpr_{1}}_{\stk} * \eval{\texpr_{2}}_{\stk}  
    \end{rclarray}
\]

\[
    \begin{syntax}{\tbool}
              \true \quad                  |
        \quad \false \quad                 |
        \quad \texpr = \texpr \quad        |
        \quad \texpr < \texpr \quad        |
        \quad \boolnot \tbool \quad        |
        \quad \tbool \booland \tbool \quad |
        \quad \tbool \boolor \tbool \quad  |
        \quad \dots 
    \end{syntax}
\]

\[
    \begin{rclarray}
        \eval{\true}_{\stk}& \defeq & \true \\
        \eval{\false}_{\stk} & \defeq & \false \\
        \eval{\texpr_{1} = \texpr_{2}}_{\stk} & \defeq & \eval{\texpr_{1}}_{\stk} = \eval{\texpr_{2}}_{\stk}   \\
        \eval{\texpr_{1} < \texpr_{2}}_{\stk} & \defeq & \eval{\texpr_{1}}_{\stk} < \eval{\texpr_{2}}_{\stk}   \\
        \eval{\boolnot \tbool}_{\stk} & \defeq & \neg \eval{\tbool}_{\stk} \\
        \eval{\tbool_{1} \booland \tbool_{2}}_{\stk} & \defeq & \eval{\tbool_{1}}_{\stk} \land \eval{\tbool_{2}}_{\stk}  \\
        \eval{\tbool_{1} \boolor \tbool_{2}}_{\stk}& \defeq & \eval{\tbool_{1}}_{\stk} \lor \eval{\tbool_{2}}_{\stk}  
    \end{rclarray}
\]

\[
    \begin{syntax}{\tcmd}
              \tskip \quad                     |
        \quad \tass{\vx}{\texpr} \quad         |
        \quad \tmutate{\texpr}{\texpr} \quad   |
        \quad \tderef{\vx}{\texpr} \quad       |
        \quad \tif{\tbool}{\tcmd}{\tcmd} \quad | \\
              \tloop{\tbool}{\tcmd} \quad      |
        \quad \tcmd \tseq \tcmd
    \end{syntax}
\]

\[
    \begin{rclarray}
        \dontcare, \dontcare, \dontcare, \dontcare, \dontcare \ \localtransfer \ \dontcare, \dontcare, \dontcare, \dontcare, \dontcare & \defeq & \Localstate \times \tcmd \times \Localstate \times \tcmd \\
    \end{rclarray}
\]

\[
    \infer[ass]{%
        \stk, \hp, \rs, \ws, \tass{\var}{\texpr} \ \localtransfer \  \stk \remapsto{\var}{\val}, \hp, \rs, \ws, \tskip
    }{%
    \eval{\texpr}_{\stk} = \val
    }
\]

\[
    \infer[mutate]{%
        \stk, \hp, \rs, \ws, \tmutate{\texpr_{1}}{\texpr_{2}} \ \localtransfer \  \stk, \hp \remapsto{\loc}{\val}, \rs, \ws \cup \Set{\loc}, \tskip
    }{%
        \eval{\texpr_{1}}_{\stk} = \loc \quad 
        \eval{\texpr_{2}}_{\stk} = \val \quad 
        \loc \in \dom(\hp)
    }
\]

\[
    \infer[deref]{%
        \stk, \hp, \rs, \ws, \tderef{\var}{\texpr} \ \localtransfer \  \stk \remapsto{\var}{\val}, \hp, \rs \cup \Set{\loc}, \ws, \tskip
    }{%
        \eval{\texpr}_{\stk} = \loc \quad 
        \val = \hp(\loc) \quad
        \loc \in \dom(\hp)
    }
\]

\[
    \infer[ifelsetrue]{%
        \stk, \hp, \tif{\tbool}{\tcmd_{1}}{\tcmd_{2}} \ \localtransfer \  \stk, \hp, \tcmd_{1}
    }{%
        \eval{\tbool}_{\stk} = \true
    }
\]

\[
    \infer[ifelsefalse]{%
        \stk, \hp, \tif{\tbool}{\tcmd_{1}}{\tcmd_{2}} \ \localtransfer \  \stk, \hp, \tcmd_{2}
    }{%
        \eval{\tbool}_{\stk} = \false
    }
\]

\[
    \infer[whiletrue]{%
        \stk, \hp, \tloop{\tbool} \tcmd \ \localtransfer \  \stk, \hp,  \tcmd \tseq \tloop{\tbool} \tcmd
    }{%
        \eval{\tbool}_{\stk} = \true
    }
\]

\[
    \infer[whilefalse]{%
        \stk, \hp, \tloop{\tbool} \tcmd \ \localtransfer \  \stk, \hp, \tskip
    }{%
        \eval{\tbool}_{\stk} = \false \quad
    }
\]

\[
    \infer[seqskip]{%
        \stk, \hp, \tskip \tseq \tcmd_{2} \ \localtransfer \  \stk, \hp, \tcmd_{2}
    }{%
    }
\]

\[
    \infer[seqnonskip]{%
        \stk, \hp, \tcmd_{1} \tseq \tcmd_{2} \ \localtransfer \  \stk', \hp', \tcmd_{1}' \tseq \tcmd_{2}
    }{%
        \stk, \hp, \tcmd_{1} \ \localtransfer \  \stk', \hp', \tcmd_{1}'
    }
\]

The semantics of transaction are interleaving of start, commit, and restart.

\[
    \begin{syntax}{\prog}
              \pemp \quad               |
        \quad \ptrans{\tcmd} \quad      |
        \quad \prog \pcond \prog \quad  |
        \quad \prept{\prog} \quad       |
        \quad \prog \pseq \prog \quad   |
        \quad \pfork{\var}{\prog} \quad |
        \quad \pjoin{\texpr}   
    \end{syntax}
\]

\[
    \begin{rclarray}
        \prog_{1} \ppar \prog_{2} & \equiv & \pfork{\var}{\prog_{1}} \pseq \prog_{2} \pseq \pjoin{\var} \\
        \tll \in \Translabel & \defeq & 
              \lid \quad                |
              \quad \lfork{\prog} \quad |
        \quad \ljoin{\thid,\ts} \\
        \dontcare, \dontcare, \dontcare, \dontcare \ \threadtransfer{ \dontcare } \ \dontcare, \dontcare, \dontcare, \dontcare & \defeq & \Threadstate \times \prog \times \Translabel \times \Threadstate \times \prog \\
    \end{rclarray}
\]

\[
    \begin{rclarray}
        \func{startstate}(\tshp,\ts) & \defeq & \lambda \loc \ldotp \tshp(\loc)(\max(\Set{\ \ts' \ \middle| \ \ts' \leq \ts \land \tshp(\ts') = (\dontcare,\wop, \dontcare) }))
    \end{rclarray}
\]

\[
    \begin{rclarray}
        \pred{allowcommit}(\tshp,\ws,\rs,\ts_{s},\ts_{e}) & \defeq & 
        \pred{atomicop}(\tshp,\ws,\rs,\ts_{s},\ts_{e}) \land {} \\
        & & \pred{consistent}(\tshp,\ws,\rs,\ts_{s},\ts_{e}) \\
        \pred{atomicop}(\tshp,\ws,\rs,\ts_{s},\ts_{e}) & \defeq  & \forall \loc \in \ws \cup \rs \ldotp \tshp(\loc)(\ts_{s})\undef \land \tshp(\loc)(\ts_{e})\undef \\
        \pred{consistent}(\tshp,\ws,\rs,\ts_{s},\ts_{e}) & \defeq & \forall \ts \in [\ts_{s},\ts_{e}], \loc \in \ws \ldotp \tshp(\loc)(\ts) \neq (\dontcare, \wop, \dontcare) \land {} \\
                                                       & & \exists \ts_{min} = \min(\Set{\ts'' \ \middle| \ \ts'' \geq \ts_{e} \land \tshp(l)(\ts'')\isdef}) \ldotp \\
                                                       & & \ts_{min} \neq \bot \implies \tshp(\loc)(\ts_{min}) = (\dontcare, \wop, \dontcare) \\
        \func{commit}(\tshp,\hp,\ws,\rs,\ts_{s},\ts_{e}) & \defeq &
        \lambda \loc \ldotp
        \begin{funcarray}
            \tshp(\loc) & \loc \notin \ws \cup \rs \\
            \tshp(\loc) \uplus \Set{ \ts_{e} \mapsto (\hp(\loc),\wop,\tsid)} & \loc \in \ws \\
            \tshp(\loc) \uplus \Set{ \ts_{s} \mapsto (\hp(\loc),\rop,\tsid)} & \loc \in \rs \\
        \end{funcarray} \\
        & & \texttt{where} \  \tsid \notin \Set{\tshp(\loc)(\ts)\projection{3} \ \middle| \ \loc \in \dom(\tshp) \land \ts \in \dom(\tshp(\loc))} \\
    \end{rclarray}
\]

\[
    \infer[commit]{%
        \tshp, \stk, \ts, \ptrans{\tcmd} \ \threadtransfer{\lid} \  \tshp', \stk', \ts_{e}, \pemp
    }{%
        \begin{array}{c}
            \ts_{s} \geq \ts \quad \stk, \func{startstate}(\tshp, \ts_{s}), \emptyset, \emptyset \localtransfer^{*} \stk', \hp, \rs, \ws \\
            \pred{allowcommit}(\tshp,\ws,\rs,\ts_{s},\ts_{e}) \quad \ts_{e} > \ts_{s} \quad \tshp' = \func{commit}(\tshp,\hp,\ws,\rs,\ts_{s},\ts_{e})
        \end{array}
    }
\]

\[
    \infer[choiceleft]{%
        \tshp, \stk, \ts, \prog_{1} \pcond \prog_{2} \ \threadtransfer{\lid} \  \tshp, \stk, \ts, \prog_{1}
    }{%
    }
\]

\[
    \infer[choiceright]{%
        \tshp, \stk, \ts, \prog_{1} \pcond \prog_{2} \ \threadtransfer{\lid} \  \tshp, \stk, \ts, \prog_{2}
    }{%
    }
\]

\[
    \infer[norep]{%
        \tshp, \stk, \ts, \prept{\prog} \ \threadtransfer{\lid} \  \tshp, \stk, \ts, \pemp
    }{%
    }
\]

\[
    \infer[rep]{%
        \tshp, \stk, \ts, \prept{\prog} \ \threadtransfer{\lid} \  \tshp, \stk, \ts, \prog \pseq \prept{\prog}
    }{%
    }
\]

\[
    \infer[seqskip]{%
        \tshp, \stk, \ts, \pemp \pseq \prog \ \threadtransfer{\lid} \  \tshp, \stk, \ts, \prog
    }{%
    }
\]

\[
    \infer[seqnoskip]{%
        \tshp, \stk, \ts, \prog_{1} \pseq \prog_{2} \ \threadtransfer{\tll} \  \tshp', \stk', \ts', \prog_{1}' \pseq \prog_{2}
    }{%
        \tshp, \stk, \ts, \prog_{1} \ \threadtransfer{\tll} \  \tshp', \stk', \ts', \prog_{1}' 
    }
\]

\[
    \infer[fork]{%
        \tshp, \stk, \ts, \pfork{\var}{\prog} \ \threadtransfer{\lfork{\thid,\prog}} \  \tshp, \stk\remapsto{\var}{\thid}, \ts, \pemp 
    }{%
    }
\]

\[
    \infer[join]{%
        \tshp, \stk, \ts, \pjoin{\texpr} \ \threadtransfer{\ljoin{\eval{\texpr}_{\stk},\ts'}} \  \tshp, \max\Set{\ts,\ts'}, \pemp 
    }{%
    }
\]

\[
    \begin{rclarray}
        \dontcare, \dontcare \ \globaltransfer{ \dontcare } \ \dontcare, \dontcare & \defeq & \State \times \Translabel \times \State  \\
    \end{rclarray}
\]

\[
    \infer[single]{%
        \tshp, \tdpl \uplus \Set{ \thid \mapsto (\stk, \ts, \prog) } \ \globaltransfer{\tll} \  \tshp', \tdpl \uplus \Set{ \thid \mapsto (\stk', \ts', \prog') }
    }{%
        \tshp, \stk, \ts, \prog \ \threadtransfer{\tll} \  \tshp', \stk', \ts', \prog' 
        \quad \tll \notin \Set{\lfork{\dontcare,\dontcare},\ljoin{\dontcare,\dontcare}}
    }
\]

\[
    \infer[fork]{%
        \tshp, \tdpl \uplus \Set{ \thid \mapsto (\stk, \ts, \prog) } \ \globaltransfer{\lfork{\thid',\prog''}} \  \tshp', \tdpl \uplus \Set{ \thid \mapsto (\stk', \ts', \prog'), \thid' \mapsto (\lambda \var \ldotp 0, \ts', \prog'') }
    }{%
        \tshp, \stk, \ts, \prog \ \threadtransfer{\lfork{\thid',\prog''}} \  \tshp', \stk', \ts', \prog' 
    }
\]

\[
    \infer[join]{%
        \tshp, \tdpl \uplus \Set{ \thid \mapsto (\stk, \ts, \prog), \thid' \mapsto (\stk', \ts'', \pemp) } \ \globaltransfer{\ljoin{\thid',\ts''}} \  \tshp', \tdpl \uplus \Set{ \thid \mapsto (\stk', \ts', \prog')}
    }{%
        \tshp, \stk, \ts, \prog \ \threadtransfer{\ljoin{\thid',\ts''}} \  \tshp', \stk', \ts', \prog' 
    }
\]

\begin{lem}
    A history cannot be overwritten, i.e.\ \( \forall \tshp, \tshp', \loc,\ts \ldotp \tshp, \dontcare \globaltransfer{\dontcare} \tshp', \dontcare \land \tshp(\loc)(\ts)\isdef \implies \tshp(\loc)(\ts) = \tshp'(\loc)(\ts)\)
\end{lem}
\begin{proof}
    From the \( \pred{allowcommit} \).
\end{proof}

\begin{lem}
    \label{lem:read-before-write}
    All the reads of a transaction happen before all the writes. This is 
    \( \forall \tshp, \loc, \loc', \ts, \ts', \tsid \ldotp \tshp(\loc)(\ts) = (\dontcare, \rop, \tsid) \land \tshp(\loc')(\ts') = (\dontcare, \wop, \tsid) \implies \ts < \ts' \).
\end{lem}
\begin{proof}
    From the semantics that \( \ts_{s} < \ts_{e} \).
\end{proof}

\begin{lem}
    \label{lem:atoic-rw}
    All the reads of a transaction happen in the same time, so do all the writes. This is 
    \( \forall \tshp, \loc, \loc', \ts, \ts', \tsid, \op \in \Set{\rop, \wop} \ldotp \tshp(\loc)(\ts) =  \tshp(\loc')(\ts') = (\dontcare, \op, \tsid) \implies \ts = \ts' \).
\end{lem}
\begin{proof}
    From the \( \func{commit} \).
\end{proof}

Now we need to recover \( \vis \) and \( \ar \) from \( \tshp \).
First we need to extend the \( \tshp \) because there are some transactions that only have reads or writes.
We stretch the time by 3, and add extra operation for those transactions.
For a transaction \( \tsid \) that only has reads event, says in time \( \ts \), we add end operations \( (\bot, \tsid, \eop ) \) to the heap cells it reads in time \( (\ts + 1 ) \).
Similarly for a transaction that only has writes event, we add end operations \( (\bot, \tsid, \sop ) \) in time \( (\ts-1) \).

\[
\begin{rclarray}
    \func{stretch}(\tshp) & \defeq & \lambda \loc \ldotp \lambda \ts \ldotp
    \begin{funcarray}
        \tshp(\loc)(\ts') & \ts = 3 * \ts' \\
        \texttt{undef} & o.w. \\
    \end{funcarray} \\
    \func{extend}(\tshp) & \defeq & \lambda \loc \ldotp \tshp(\loc) \uplus \Set{\ts + 1 \mapsto (\bot, \tsid, \eop ) \ \middle| \ \tshp(\loc)(\ts) = (\dontcare, \tsid, \rop) \land \forall \loc', \ts' \ldotp \tshp(\loc')(\ts') = (\dontcare, \tsid, \wop)} \\
                         & & \quad \quad \quad \uplus \Set{\ts - 1 \mapsto (\bot, \tsid, \sop ) \ \middle| \ \tshp(\loc)(\ts) = (\dontcare, \tsid, \wop) \land \forall \loc', \ts' \ldotp \tshp(\loc')(\ts') = (\dontcare, \tsid, \rop)}
\end{rclarray}
\]

\begin{lem}
    After stretching the time by 3, there is no record in time \( 3 * \nat + 1 \) and \( 3 * \nat - 1 \).
    Therefore after extending, there are only \( (\dontcare, \tsid, \eop) \) in time \( 3 * \ts + 1 \) and only \( (\dontcare, \tsid, \sop) \) in time \( 3 * \ts - 1 \).
    This is \( \forall \tshp \ldotp \exists \tshp' = \func{extend} \circ \func{stetch}(\tshp) \ldotp \forall \loc, \ts \ldotp (\tshp'(\loc)(3 * \ts + 1)\isdef \implies \tshp'(\loc)(3 * \ts + 1) = (\dontcare, \dontcare, \eop) ) \land (\tshp'(\loc)(2 * \ts - 1)\isdef \implies \tshp'(\loc)(3 * \ts - 1) = (\dontcare, \dontcare, \sop) ) \).
\end{lem}
\begin{proof}
    trivial.
\end{proof}

\begin{lem}
    \label{lem:start-before-end}
    In the extended heap, all the reads or starts of a transaction happen before all the writes or end. This is 
    \( \forall \tshp, \loc, \loc', \ts, \ts', \op \in \Set{\rop, \sop}, \op' \in \Set{\wop, \eop}, \tsid \ldotp \tshp(\loc)(\ts) = (\dontcare, \op, \tsid) \land \tshp(\loc')(\ts') = (\dontcare, \op', \tsid) \implies \ts < \ts' \).
\end{lem}
\begin{proof}
    From Lemma \ref{lem:read-before-write} and the definition of \func{strech} and \func{extend}.
\end{proof}

\begin{lem}
    \label{lem:happen-in-same-time}
    For an extended heap, all the reads of a transaction happen in the same time, so do all the writes, starts and ends. This is 
    \( \forall \tshp, \loc, \loc', \ts, \ts', \tsid, \op \ldotp \tshp(\loc)(\ts) =  \tshp(\loc')(\ts') = (\dontcare, \op, \tsid) \implies \ts = \ts' \).
\end{lem}
\begin{proof}
    From Lemma \ref{lem:atoic-rw} and the definition of \func{strech} and \func{extend}.
\end{proof}

\begin{lem}
    \label{lem:unique-label}
    A transaction in an extended heap, must have either starts or reads, and either ends or writes.
\end{lem}
\begin{proof}
    From the definition of \func{extend}.
\end{proof}

\[
\begin{rclarray}
    (\settrans, \tvis, \tar) = \func{graph}(\tshp) & \defeq & (\Set{ \tsid \ \middle| \ \forall \loc \ldotp \tshp(\loc) = (\dontcare, \tsid, \dontcare)}, \\
                                                   & & \Set{(\tsid, \tsid') \ \middle| \ 
    \begin{array}{@{}l@{}}
        \exists \loc, \loc', \ts, \ts', \op \in \Set{\wop, \eop}, \op' \in \Set{\rop, \sop} \ldotp \\
        \ts < \ts' \land \tshp(\loc)(\ts) = (\dontcare, \tsid, \op) \land \tshp(\loc')(\ts') = (\dontcare, \tsid', \op')
    \end{array}
}, \\
                                                   & & \Set{(\tsid, \tsid') \ \middle| \ 
    \begin{array}{@{}l@{}}
        \exists \loc, \loc', \ts, \ts', \op, \op' \in \Set{\wop, \eop} \ldotp \\
        \ts < \ts' \land \tshp(\loc)(\ts) = (\dontcare, \tsid, \op) \land \tshp(\loc')(\ts') = (\dontcare, \tsid', \op')
    \end{array}
}, \\
\end{rclarray}
\]

\begin{lem}
    For an extended heap, the corresponding \tvis\ and \tar\ have no circle.
\end{lem}
\begin{proof}
    Assume there is a circle in \(\rvis\), says, \( \tsid_{1} \rvis \tsid_{2} \rvis \dots \rvis \tsid_{n} \rvis \tsid_{n+1} \), where \( \tsid_{1} = \tsid_{n+1} \).
    Therefore, \( \bigwedge\limits_{ 1 \leq i \leq n} \exists \loc, \loc', \ts, \ts', \op \in \Set{\wop, \eop}, \op' \in \Set{\rop, \sop} \ldotp \ts < \ts' \land \tshp(\loc)(\ts) = (\dontcare, \tsid_{i}, \op) \land \tshp(\loc')(\ts') = (\dontcare, \tsid_{i+1}, \op')\).
    By Lemma \ref{lem:unique-label} we can relabel read to start and write to end.
    Then by Lemma \ref{lem:happen-in-same-time}, we can define a list of starts and ends events that is ordered by time: \( \List{ (\tsid_{1},\eop), (\tsid_{2},\sop), (\tsid_{2},\eop), \dots, (\tsid_{n},\eop), (\tsid_{n+1},\sop) } \).
    By the assumption, we have \( \tsid_{1} = \tsid_{n+1} \), thus this contradict Lemma \ref{lem:start-before-end}.

    Similarly for \(\rtar\), the list of write events  \( \List{ (\tsid_{1},\eop), (\tsid_{2},\eop), (\tsid_{2},\eop), \dots, (\tsid_{n},\eop), (\tsid_{n+1},\eop) } \) contradict Lemma \ref{lem:happen-in-same-time}.
\end{proof}

\begin{lem}
    Given a \( \tshp \), the corresponding \((\settrans, \tvis, \tar)\) can be extended to \((\settrans, \vis, \ar)\) so that it is a valid dependency graph of snapshot isolation.
\end{lem}
\begin{proof}
    First, we extend the relations \( \tar \) to a total order \( \ar \).
    Initially, \( \ar \) includes all relations in \( \tar \).
    Given the definition, only if the ends or writes of transactions happen in the same time, those transactions are not ordered by \( \tar \).
    To simplify, we introduce an initial event, i.e.\ \( \forall \tsid \in \settrans \ldotp \tsid_{init} \rar \tsid \).
    From \( \tsid_{init} \), we pick the first two transactions \( \tsid_{1} \) and \( \tsid_{2} \) that are not ordered, this is, \( \forall \tsid, \tsid' \in \Set{\tsid'' \ \middle| \ \tsid'' \rar \tsid_{1} \lor \tsid'' \rar \tsid_{2} } \ldotp \tsid \rar \tsid' \lor \tsid' \rar \tsid \).
    Therefore, there exists an unique \( \tsid_{pre} \) where branching happens, i.e.\ \( \tsid_{pre} \rar \tsid_{1} \land \tsid_{pre} \rar \tsid_{2} \land \nexists \tsid \ldotp \tsid_{pre} \rar \tsid \rar \tsid_{1} \lor \tsid_{pre} \rar \tsid \rar \tsid_{2} \).
\end{proof}

%%%%%%%%%%%%%%%%%%%%%%%%%%%%%%%%%%%%%%%%%%%%%%%%%%%%%
%
% the small intro for semantics is in semantics.tex
% section 2.1  in semantics/database-view.tex
% section 2.2.1 programming lang. in semantics/tarns.tex
% section 2.2.2 transactional semantics in semantics/tarns.tex
% section 2.2.3 prog semantics in semantics/prog.tex
% section 2.2.4 example  in semantics/example.tex
%
%%%%%%%%%%%%%%%%%%%%%%%%%%%%%%%%%%%%%%%%%%%%%%%%%%%%%
%
% This is your ordinal version. 
%
\ac{ORDINAL VERNON starts from here.}
\input{andrea.tex}
%%%%%%%%%%%%%%%%%%%%%%%%%%%%%%%%%%%%%%%%%%%%%%%%%%%%%

\section{Examples of Consistency Models}
\label{sec:cmexamples}
%\[
%    \begin{session}
%        \begin{array}{@{}c || c ||  c || c@{}}
%            \begin{transaction}      		
%                \pderef{\vx}{\loc_x}; \\
%                \pifs{\vx=0} \\
%                \quad \pmutate{\loc_y}{1}
%            \end{transaction} & 
%            \begin{transaction}
%                \pderef{\vy}{\loc_y}; \\
%                \pifs{\vx=0} \\
%                \quad \pmutate{\loc_x}{1}
%            \end{transaction} & 
%            \begin{transaction}
%                \pmutate{\loc_x}{2}
%            \end{transaction} & 
%            \begin{transaction}
%                \pmutate{\loc_y}{2}
%            \end{transaction} \\
%        \end{array}
%    \end{session}
%\]
\ac{This Section is going to become heavy in pictures, which should be 
organised into figures.}
In this Section we present different consistency models specifications. 
For each of them, we give: 
\begin{itemize}
\item an informal definition, describing the consistency guarantees that 
schedules of the database should have in plain English, 
\item examples of litmus tests that, when executed,  give rise to the anomalies that should be forbidden 
from the consistency model, 
\item a formal consistency model specification, in the style described in \S \ref{sec:semantics.programs},
\item an explanation of why the consistency model forbids the litmus tests to exhibit the anomaly that 
should be forbidden. 
\end{itemize}
Later on in the paper, we will show how to compare our consistency 
models specifications with those already existing in the 
literature.
\ac{There is still a long-way to go before proving correspondence with dependency graph specifications, 
but this should be mentioned here.}

\subsection{Read Atomic} 
Read atomic \cite{ramp} is the weakest consistency model among 
those that enjoy \emph{atomic visibility} \cite{framework-concur}. 
It requires transactions to read an atomic snapshot of the database. 
It also requires transactions to never observe the partial effects of other transactions. 
This is also known as the \emph{all-or-nothing} property: A transaction 
observes either none or all the updates performed by another transaction. 

One litmus test that should \textbf{not} be failed in RAMP consists of the program 
$\prog_1$ from \S \ref{sec:semantics.example}, which we already observed 
to produce a violation of atomic visibility if no constraints on the consistency 
model are placed.
\ac{not be failed. Double negation. Bad English.}

 Intuitively, in such a program, the violation of atomic 
visibility happened because we allowed to execute the transaction 
\[
\trans_1^2 = \begin{array}{c} 
            \begin{transaction}
            		\pderef{\pvar{a}}{\loc_x};\\
            		\pderef{\pvar{b}}{\loc_y};\\
            		\pifs{\pvar{a}=1 \wedge \pvar{b}=0}\\
            			\;\;\;\;\passign{\retvar}{\Large \frownie{}}
             \end{transaction}
     \end{array}
\]
in the thread-local configuration of $\mathcal{C}'$ relative to $\tid_2$, which is obtained 
by removing all the information about $\tid_1$ (view and stack) in Figure \ref{fig:opsem.example}(c).
%
%\begin{center}
%\begin{tikzpicture}[font=\large]
%
%\begin{pgfonlayer}{foreground}
%%Uncomment line below for help lines
%%\draw[help lines] grid(5,4);
%
%%Location x
%\node(locx) at (1,3) {$[\loc_x] \mapsto$};
%
%\matrix(locxcells) [version list, text width=7mm, anchor=west]
%   at ([xshift=10pt]locx.east) {
% {a} & $T_0$ & {a} &  $T_1$ \\
%  {a} & $\emptyset$ & {a} & $\emptyset$ \\
%};
%\node[version node, fit=(locxcells-1-1) (locxcells-2-1), fill=white, inner sep= 0cm, font=\Large] (locx-v0) {$0$};
%\node[version node, fit=(locxcells-1-3) (locxcells-2-3), fill=white, inner sep=0cm, font=\Large] (locx-v1) {$1$};
%
%%Location y
%\path (locx.south) + (0,-1.5) node (locy) {$[\loc_y] \mapsto$};
%\matrix(locycells) [version list, text width=7mm, anchor=west]
%   at ([xshift=10pt]locy.east) {
% {a} & $T_0$ & {a} &  $T_1$ \\
%  {a} & $\emptyset$ & {a} & $\emptyset$ \\
%};
%\node[version node, fit=(locycells-1-1) (locycells-2-1), fill=white, inner sep= 0cm, font=\Large] (locy-v0) {$0$};
%\node[version node, fit=(locycells-1-3) (locycells-2-3), fill=white, inner sep=0cm, font=\Large] (locy-v1) {$1$};
%
%% \draw[-, red, very thick, rounded corners] ([xshift=-5pt, yshift=5pt]locx-v1.north east) |- 
%%  ($([xshift=-5pt,yshift=-5pt]locx-v1.south east)!.5!([xshift=-5pt, yshift=5pt]locy-v0.north east)$) -| ([xshift=-5pt, yshift=5pt]locy-v0.south east);
%
%%blue view - I should  check whether I can use pgfkeys to just declare the list of locations, and then add the view automatically.
%\draw[-, , blue, very thick, rounded corners=10pt]
% ([xshift=-5pt, yshift=5pt]locx-v1.north east) -- 
% ([xshift=-5pt, yshift=-5pt]locx-v1.south east) --
% ([xshift=-5pt, yshift=5pt]locy-v0.north east) -- 
% ([xshift=-5pt, yshift=-5pt]locy-v0.south east);
%
%\end{pgfonlayer}
%\end{tikzpicture}
%\end{center}

To avoid transactions to only observe the partial effects of other transactions, we 
must ensure that transactional code cannot be executed by a thread whose 
views is up-to-date with respect to some transaction $\tsid$ for some location $[\loc_x]$, 
but not for some other location $[\loc_y]$. This leads to the following definition.
\begin{definition}
\label{def:readatomic}
%Let $\hh$ be a history heap,$V$ be a view, $[\loc_x]$ be 
%a location and $\nu$ be a version. We say that $V$ $[\loc_x]$-\emph{sees} version 
%$\nu$ if there exists an index $i \leq V([\loc_x])$ such that $V(i) = \nu$. 
%We say that $V$ $[\loc_x]$-\emph{sees} transaction $\tsid$ if 
%$V$ $[\loc_x]$-sees a version $\nu = (\_, \tsid, \_)$. 
Let $V$ be  view, $\hh$ be a history heap, $\tsid$ be a transaction identifier. 
We say that $V$ \emph{sees} transaction $\tsid$ in $\hh$, written 
$\mathsf{Visible}(\tsid, V, \hh)$, iff 
\[
\forall [\loc_x].\;\forall i=0,\cdots, \lvert \hh([\loc_x]) \rvert -1.\;\hh([\loc_x])(i) = 
(\_, \tsid, \_) \implies i \leq V([\loc_x]).
\]
\ac{In English: the view is up-to-date with respect to all the updates 
performed by transaction $\tsid$.}

We say that the view $V$ is \emph{consistent} with respect to atomic 
visibility and the history heap $\hh$,  written $\mathsf{Atomic}(V, \hh)$, 
if
\[
\forall \tsid. \left( \exists\;[\loc_x]. \exists i \leq V([\loc_x]).\; \hh([\loc_x])(i) = (\_, \tsid, \_) 
\right) \implies \mathsf{Visible}(\tsid, V, \hh).
\]
\ac{In English: if the view $V$ is up-to-date with some of the updates performed 
by $\tsid$, then it must be up-to-date with all the updates performed by $\tsid$. 
This is the all-or-nothing property.}
%for all location 
%$[\loc_x]$, if there exists an index $i = 0,\cdots, \lvert \hh([\loc_x]) \rvert - 1$, 
%such that $\hh([\loc_x])(i) = (\_, \tsid, \_)$, then $i \leq V([\loc_x])$.

The consistency model specification $\mathsf{RA}$ is defined as the smallest set such that  
\[
\mathsf{atomic}(\hh, V) \implies (\hh, V) \triangleright_{\mathsf{RA}} \_: \_
\]
\ac{In English: Before executing a transaction, either you observe all or none the 
updates of all other transactions. We may strengthen the consistency model and 
require that the same property must be satisfied at the end as well, though 
this is not strictly necessary. In this case the check becomes: 
\[
\mathsf{atomic}(\hh, V) \wedge \mathsf{atomic}(\hh, V') \wedge \mathsf{UpdateView}(\hh, V, \mathcal{O}) 
\sqsubseteq V' \implies (\hh, V) \triangleright_{\mathsf{RA}} \mathcal{O}: V'.
\]
}
%written $\mathsf{up-to-date}(\hh, V, \tsid, [\loc_x])$, 
%if either 
%
%\begin{itemize}
%\item for all indexes $i = 0,\cdots, \lvert \hh([\loc_x]) - 1 \rvert$, 
%$\WS(\hh([\loc_x])(i)) \neq \tsid)$, or 
%\item if $\WS(\hh([\loc_x])(i)) = \tsid$ for some $i = 0,\cdots, \lvert \hh([\loc_n]) -1 \rvert$, 
%then $i \leq V([\loc_n])$.
%\end{itemize}
\end{definition}

Suppose that we execute the program $\prog_1$ under the consistency model specification $\mathsf{RA}$.
We can proceed as in Section \ref{sec:semantics.example} to infer the transition 
$\langle \mathcal{C_0}, \prog_1 \rangle \xrightarrow{\mathsf{RA}} \langle \mathcal{C}_1, \prog_1' \rangle$, 
where we recall that $\mathcal{C}_0$, $\mathcal{C}_1$ are depicted in Figure \ref{fig:opsem.exampe}(a), 
\ref{fig:opsem.example}(b), respectively. 

It is immediate to observe that the only way in which the execution of transaction 
$\ptrans{\trans}$ from $\tid_2$ in $\prog_1'$ can return value ${\Large \frownie}$ 
is the following: 
\begin{itemize}
\item first, push the view $V$ of thread $\tid_2$ in the configuration 
$\mathcal{C}_1$ of Figure \ref{fig:opsem.example}(b) to observe the update of location $[\loc_x]$, but not the update of 
$[\loc_y]$. This view is the one labelled with $\tid_2$ in Figure \ref{fig:opsem.example}(c), and we refer 
to it as $V'$;
\item then, execute the transaction $\ptrans{\trans}$ in $\tid_2$. 
\end{itemize}
%However, this second step cannot be performed under the consistency model specification $\mathsf{RA}$. 
%In fact, the view $V'$ is not atomic, according to Definition \ref{def:readatomic}, and therefore, 
%thread $\tid_2$ cannot execute transaction $\ptrans{\trans}$ using the heap extracted from view 
%$V'$. A consequence of this fact is that, under $\mathsf{RA}$ there exists no execution 
%of $\prog_1$ in which the transaction $\ptrans{\trans}$ returns value ${\Large \frownie}$.
%
%Consider the  history heap $\hh_1'$ and view $V_1'$ of thread $\tid_2$, taken 
%from the configuration $\mathcal{C}'$ of Figure \ref{fig:opsem.example}(c).  
%In this case, we have that $\neg(\mathsf{atomic}(\hh, V)$ (indicated by the blue line in the figure), 
%is up-to-date with respect to the update of location $[\loc_x]$ performed by $\tsid_1$: 
%formally, there exists an index $i \leq V_2([n])$ such that $\hh([\loc_{x}])(i) = (\_, \tsid_1, \_)$. 
%On the other hand, the view $V_2$ is not up-to-date with respect to the updated of 
%location $[\loc_y]$ performed by $\tsid_1$:  $\hh([\loc_y])(1) = (\_, \tsid_1, \_)$, 
%but $V_2([\loc_y]) = 0$. 
%
%
%We say that the view $V$ is $RA$-consistent with respect to $\hh$, written $\mathsf{RA-cons}(V, \hh)$, if
%\[
%\forall \tsid, [\loc_x], [\loc_y].\; \mathsf{up-to-date}(\hh, V, \tsid, [\loc_x]) \implies \mathsf{up-to-date}(\hh, V, \tsid, [\loc_y]).
%\]
%
%As we have already discussed, the thread-local configuration $(\hh_2, V_2)$ is not $\mathsf{RA}$-consistent, 
%and therefore it is not possible to infer a transition of the form $(\hh_2, [\tid_2 \mapsto \_, V_2], \trans.\nil ) \xrightarrow{\_}_{\mathsf{RA}} 
%\_$ using Rule $(P-thd-exec)$. In contranst, we have shown in \S \ref{sec:semantics.example} that, if no 
%constraints are placed in the consistency model, then it is possible to derive a transition of the form 
%$(\hh_2, [\tid_2 \mapsto \_, V_2], \trans.\nil) \xrightarrow{\_}_{\mathsf{CM}_{\top}}$ 
%$(\hh_2', [\tid_2 \mapsto \left( [\retvar \mapsto {\Large \frownie{}}], V_2' \right) ], \nil)$.

%The first condition is built-in in the rules of the operational semantics: transactions 
%are always executed by taking a snapshot of the database as the initial state; such a 
%snapshot is determined by the history heap, and the view of the thread that executes 
%the transaction. However, the all-or-nothing property is not enforced by the operational 
%semantics, as the following example shows: 
%\begin{example}
%Consider the following program $\prog_1$:
%\[
%    \begin{session}
%        \begin{array}{@{}c || c@{}}
%            \begin{transaction}
%            		\pmutate{\loc_x}{1};
%            		\pmutate{\loc_y}{1};
%              \end{transaction} &
%              \begin{transaction}
%            		\pderef{\vx}{\loc_x};
%            		\pderef{\vy}{\loc_y};
%            		\pifs{\vx=1 \wedge \vy=0}\\
%            			\passign{\retvar}{\Large \frownie{}}
%             \end{transaction}
%        \end{array}
%    \end{session}
% \]
% \ac{Note that in this example I don't differentiate between thread stack and transaction stack}
%In the program above we committed an abuse of notation and did not insert thread identifiers. In general, 
%we assume that thread identifiers are fixed by the order in which commands appear in a parallel 
%composition. For exampe, we write $\cmd_1 \Par \cdots \Par \cmd_n$ as a shorthand for 
%$\tid_1: \cmd_1 \Par \cdots \Par \tid_{n}: \cmd_n$. 
%Also, note that the transaction for thread $\tid_2$ may return a special value ${\Large \frownie{}}$, which 
%we did not introduce before. Although in practice we could have chosen any natural number 
%to assign to $\retvar$, we use the special symbol ${\Large \frownie{}}$ to emphasise 
%the fact that the transaction gives rise to an anomaly that should not be allowed by the 
%consistency model that we are considering. In this case, this happens if the transaction 
%executed by thread $\tid_2$ observes the update of $\loc_x$ performed by 
%the transaction executed by $\tid_1$, but not the update of $\loc_y$ performed by 
%the same transaction: in this scenario, atomic visibility is violated.
%
%We illustrate the semantics of this program under the 
%most general consistency model $CM_{\mathsf{all}}$ such that 
%$(\hh, V, \mathcal{V}) \triangleright_{CM_{\mathsf{all}}} \mathcal{O}$ for all 
%$\hh, V, \mathcal{V}, \mathcal{O}$. 
%For this example, we assume that the database consists of only two keys $\loc_x, \loc_y$.
%The program is executed starting from the initial configuration 
%$(\hh_{0}, [\tsid_1 \mapsto (\thdstack_0, V_{0}), \tsid_2 \mapsto (\thdstack_0, V_0)]$, 
%where: 
%\begin{enumerate}
%\item $\hh_{0}([\loc_x]) = \hh_{0}([\loc_y]) = (0, \tsid_0, \emptyset)$, and $\tsid_0$ is a random transaction identifier, 
%\item $\thdstack_0 = \lambda \vx.0$ is the initial thread stack.
%\item $V_{0}([n]) = [ [\loc_x] \mapsto 0, \loc_y \mapsto 0] ]$ is the view that maps every 
%key to its initial version.
%\end{enumerate}
%Here and in the rest of the report we will give a graphical representation of views. For example, 
%below we depict the inital configuration in which $\prog_1$ is executed:
%\begin{center}
%\begin{tikzpicture}[font=\large]
%\begin{pgfonlayer}{foreground}
%\node (locx) {$[\loc_x] \mapsto$};
%\path (locx.south) + (0,-1.5) node (locy) {$[\loc_y] \mapsto$};
%\path (locx.east) + (0.5,0)  node (locx0) {$0$};
%\path (locx0.north east) + (0.3,0) node[font=\small] (locx0ws) {$\tsid_0$};
%\path (locx0.south east) + (0.3,0) node[font=\small] (locx0rs) {$\emptyset$};
%\path (locy.east) + (0.5,0)  node (locy0) {$0$};
%\path (locy0.north east) + (0.3,0) node[font=\small] (locy0ws) {$\tsid_0$};
%\path (locy0.south east) + (0.3,0) node[font=\small] (locy0rs) {$\emptyset$};
%
%\path(locy0.south) + (0,-1) node[font = \small, text=red] (tid1ret) {$\tid_1: \retvar = 0$};
%\path(tid1ret.south) + (0,-0.3) node[font= \small, text=blue] (tid2ret) {$\tid_2: \retvar = 0$};
%\end{pgfonlayer}
%
%\begin{pgfonlayer}{main}
%\node[hheapcell, fit=(locx0) (locx0ws) (locx0rs)] (locx0cell) {};
%\node[hheapcell, fit=(locy0) (locy0ws) (locy0rs)] (locy0cell) {};
%
%\path[-]
%(locx0cell.north) edge (locx0cell.south)
%(locx0cell.center) edge (locx0cell.east)
%(locy0cell.north) edge (locy0cell.south) 
%(locy0cell.center) edge (locy0cell.east);
%
%\path(locx0cell.north) + (-0.5, 0.2) node(v1beforex) {};
%\path(locx0cell.south) + (-0.5, -0.1) node(v1afterx) {};
%\path(locy0cell.north) + (-0.5, 0.1) node(v1beforey) {};
%\path(locy0cell.south) + (-0.5, -0.2) node(v1aftery) {};
%\path(v1beforex.center) + (-0.25,0.15) node[font=\normalsize, text=red] (t1) {$\tid_1$};
%
%\path[-, draw=red!80, very thick]
% (v1beforex.center) edge (v1afterx.center)
% (v1afterx.center) edge (v1beforey.center)
% (v1beforey.center) edge (v1aftery.center);
% 
% \path(locx0cell.north) + (-0.1, 0.2) node(v2beforex) {};
%\path(locx0cell.south) + (-0.1, -0.1) node(v2afterx) {};
%\path(locy0cell.north) + (-0.1, 0.1) node(v2beforey) {};
%\path(locy0cell.south) + (-0.1, -0.2) node(v2aftery) {};
%\path(v1beforex.center) + (0.6,0.15) node[font=\normalsize, text=blue] (t2) {$\tid_2$};
%
%\path[-, draw=blue!80, very thick]
% (v2beforex.center) edge (v2afterx.center)
% (v2afterx.center) edge (v2beforey.center)
% (v2beforey.center) edge (v2aftery.center);
% 
%\end{pgfonlayer}
%
%\end{tikzpicture}
%\end{center}
%In the figure above, each location is mapped to a list of version (one in this case), depicted as a cell 
%consisting of three components: the component at  the left (which is $0$ for both cells in the figure 
%above) consists of the value stored in the version of the location. The component at the top 
%left identifies the transaction that wrote such a version, while the component at the bottom left 
%includes the set of transactions that read such a version. Views of threads are represented each by a vertical  
%line that crosses a location exactly at the cell corresponding to the version to which the view points. 
%At the bottom of the picture, we also included information about the thread stack of individual threads.
%\ac{The discussion about graphical representation of configurations should be before.
%In fact, $\prog_1$ should be introduced before the section on consistency models, as 
%an example of the semantics.} In practice, we should 
%We are now ready to start analysing the behaviour 
%
%\end{example}
%which a transaction executes is fixed at the moment it starts. However, the 
%The weakest consistency model specification that we consider is given by \emph{Read Atomic}. 
%\ac{Following a discussion with Azalea and Shale, it looks like history heaps alone do not enforce 
%atomic visibility. I find this to be actually good, because that means that there may be some way to 
%reduce transactions in one step, while modelling non-atomic behaviours}.
%This consistency models require that transactions observe either none, or all the updates performed by 
%another transaction (i.e. they enjoy atomic visibility). 
%In terms of history heaps and views, we must ensure that views do not observe only the partial 
%effects of transactions. Given a transaction $\tsid$, if a view of a history heap points to a version of location $[n]$ that 
%is more up-to-date than the one written by some transaction $\tsid$, then we require that the 
%same property holds also for any location $[m]$ that has been written by $\tsid$.
%
%
%\begin{definition}
%We say that a view $V$ \emph{respects atomic visibility} w.r.t a history heap $\hh$ if, 
%whenever $\hh([n])(i) = (\_, \tsid, \_)$, $\hh([m])(j) = (\_, \tsid, \_)$ for some addresses 
%$[n], [m]$ and indexes $i, j$,  then $i \leq V([n])$ implies $j \leq V([n])$.
%The consistency model specification $\mathsf{RA}_{\mathcal{H}}$ is defined to be the smallest set 
%such that 
%\begin{itemize}
%\item for any history-heaps $\hh$, $\hh'$ and  $V$, $V'$, if $V$ respects atomic visibility w.r.t. 
%$\hh$, then $(\hh, \{V\}) \leadsto_{\mathsf{RA}_{\mathcal{H}}} (\hh', \{V'\})$.
%\item for any history-heaps $\hh,\hh'$ and  multi-set of views $\mathcal{V}, \mathcal{V}', 
%\mathcal{V}''$, $\mathcal{V}'' \Vdash (\hh, \mathcal{V}') \leadsto_{\mathsf{RA}_{\mathcal{H}}} 
%(\hh', \mathcal{V}')$.  
%\end{itemize}
%\ac{Hold on! I'm rethinking consistency models specifications completely. It seems that 
%the best way to specify a consistency model is given by triples of the form 
%$(\hh, V, \mathcal{V}, \mathcal{O})$, which may be interpreted as: 
%given a history heap $\hh$, a thread with view $V$ can safely execute a transaction 
%with fingerprint $\mathcal{O}$, when running in an environment where the views 
%of other threads are given by $\mathcal{V}$.) This seems to be closer in spirit 
%to what Crooks and Alvisi do as well in their PODC'17 paper.}
%
%\end{definition}

\subsection{Causal Consistency}
\begin{figure}
\begin{tabular}{|c|c|}
\hline
\begin{tikzpicture}[font=\large]

\begin{pgfonlayer}{foreground}
%Uncomment line below for help lines
%\draw[help lines] grid(5,4);

%Location x
\node(locx) at (1,3) {$[\loc_x] \mapsto$};

\matrix(locxcells) [version list, text width=7mm, anchor=west]
   at ([xshift=10pt]locx.east) {
 {a} & $T_0$ \\
  {a} & $\emptyset$ \\
};
\node[version node, fit=(locxcells-1-1) (locxcells-2-1), fill=white, inner sep= 0cm, font=\Large] (locx-v0) {$0$};

%Location y
\path (locx.south) + (0,-1.5) node (locy) {$[\loc_y] \mapsto$};
\matrix(locycells) [version list, text width=7mm, anchor=west]
   at ([xshift=10pt]locy.east) {
 {a} & $T_0$ \\
  {a} & $\emptyset$ \\
};
\node[version node, fit=(locycells-1-1) (locycells-2-1), fill=white, inner sep= 0cm, font=\Large] (locy-v0) {$0$};

% \draw[-, red, very thick, rounded corners] ([xshift=-5pt, yshift=5pt]locx-v1.north east) |- 
%  ($([xshift=-5pt,yshift=-5pt]locx-v1.south east)!.5!([xshift=-5pt, yshift=5pt]locy-v0.north east)$) -| ([xshift=-5pt, yshift=5pt]locy-v0.south east);

%blue view - I should  check whether I can use pgfkeys to just declare the list of locations, and then add the view automatically.
\draw[-, blue, very thick, rounded corners=10pt]
 ([xshift=-2pt, yshift=20pt]locx-v0.north east) node (tid1start) {} -- 
% ([xshift=-2pt, yshift=-5pt]locx-v0.south east) --
% ([xshift=-2pt, yshift=5pt]locy-v0.north east) -- 
 ([xshift=-2pt, yshift=-5pt]locy-v0.south east);
 
 \path (tid1start) node[anchor=south, rectangle, fill=blue!20, draw=blue, font=\small, inner sep=1pt] {$\tid_3$};

%red view
\draw[-, red, very thick, rounded corners = 10pt]
 ([xshift=-5pt, yshift=5pt]locx-v0.north east) -- 
% ([xshift=-8pt, yshift=-5pt]locx-v0.south east) --
% ([xshift=-8pt, yshift=5pt]locy-v0.north east) -- 
 ([xshift=-5pt, yshift=-10pt]locy-v0.south east) node (tid2start) {};
 
\path (tid2start) node[anchor=north, rectangle, fill=red!20, draw=red, font=\small, inner sep=1pt] {$\tid_2$};
 
 %green view
\draw[-, draw=DarkGreen, very thick, rounded corners = 10pt]
 ([xshift=-16pt, yshift=8pt]locx-v0.north east) node (tid3start) {}-- 
% ([xshift=-15pt, yshift=-5pt]locx-v0.south east) --
% ([xshift=-15pt, yshift=5pt]locy-v0.north east) -- 
 ([xshift=-16pt, yshift=-5pt]locy-v0.south east);
 
 \path (tid3start) node[anchor=south, rectangle, fill=DarkGreen!20, draw=DarkGreen, font=\small, inner sep=1pt] {$\tid_1$};

\end{pgfonlayer}
\end{tikzpicture}
&
\begin{tikzpicture}[font=\large]

\begin{pgfonlayer}{foreground}
%Uncomment line below for help lines
%\draw[help lines] grid(5,4);

%Location x
\node(locx) at (1,3) {$[\loc_x] \mapsto$};

\matrix(locxcells) [version list, text width=7mm, anchor=west]
   at ([xshift=10pt]locx.east) {
 {a} & $\tsid_0$ & {a} & $\tsid_1$\\
  {a} & $\emptyset$ & {a} & $\emptyset$ \\
};
\node[version node, fit=(locxcells-1-1) (locxcells-2-1), fill=white, inner sep= 0cm, font=\Large] (locx-v0) {$0$};
\node[version node, fit=(locxcells-1-3) (locxcells-2-3), fill=white, inner sep=0cm, font=\Large] (locx-v-1) {$1$};
%Location y
\path (locx.south) + (0,-1.5) node (locy) {$[\loc_y] \mapsto$};
\matrix(locycells) [version list, text width=7mm, anchor=west]
   at ([xshift=10pt]locy.east) {
 {a} & $\tsid_0$ \\
   {a} & $\emptyset$ \\
};
\node[version node, fit=(locycells-1-1) (locycells-2-1), fill=white, inner sep= 0cm, font=\Large] (locy-v0) {$0$};
% \draw[-, red, very thick, rounded corners] ([xshift=-5pt, yshift=5pt]locx-v1.north east) |- 
%  ($([xshift=-5pt,yshift=-5pt]locx-v1.south east)!.5!([xshift=-5pt, yshift=5pt]locy-v0.north east)$) -| ([xshift=-5pt, yshift=5pt]locy-v0.south east);

%blue view - I should  check whether I can use pgfkeys to just declare the list of locations, and then add the view automatically.
\draw[-, blue, very thick, rounded corners=10pt]
 ([xshift=-2pt, yshift=20pt]locx-v0.north east) node (tid1start) {} -- 
% ([xshift=-2pt, yshift=-5pt]locx-v0.south east) --
% ([xshift=-2pt, yshift=5pt]locy-v0.north east) -- 
 ([xshift=-2pt, yshift=-5pt]locy-v0.south east);
 
 \path (tid1start) node[anchor=south, rectangle, fill=blue!20, draw=blue, font=\small, inner sep=1pt] {$\tid_3$};

%red view
\draw[-, red, very thick, rounded corners = 10pt]
 ([xshift=-5pt, yshift=5pt]locx-v0.north east) -- 
% ([xshift=-8pt, yshift=-5pt]locx-v0.south east) --
% ([xshift=-8pt, yshift=5pt]locy-v0.north east) -- 
 ([xshift=-5pt, yshift=-10pt]locy-v0.south east) node (tid2start) {};
 
\path (tid2start) node[anchor=north, rectangle, fill=red!20, draw=red, font=\small, inner sep=1pt] {$\tid_2$};
 
 %green view
\draw[-, DarkGreen, very thick, rounded corners = 10pt]
 ([xshift=-16pt, yshift=8pt]locx-v1.north east) node (tid3start) {}-- 
 ([xshift=-16pt, yshift=-5pt]locx-v1.south east) --
 ([xshift=-16pt, yshift=5pt]locy-v0.north east) -- 
 ([xshift=-16pt, yshift=-5pt]locy-v0.south east);
 
 \path (tid3start) node[anchor=south, rectangle, fill=DarkGreen!20, draw=DarkGreen, font=\small, inner sep=1pt] {$\tid_1$};

\end{pgfonlayer}
\end{tikzpicture}
\\
{\small (a)} & {\small (b)}\\
\hline
\begin{tikzpicture}[font=\large]

\begin{pgfonlayer}{foreground}
%Uncomment line below for help lines
%\draw[help lines] grid(5,4);

%Location x
\node(locx) at (1,3) {$[\loc_x] \mapsto$};

\matrix(locxcells) [version list, text width=7mm, anchor=west]
   at ([xshift=10pt]locx.east) {
 {a} & $\tsid_0$ & {a} & $\tsid_1$\\
  {a} & $\emptyset$ & {a} & $\emptyset$ \\
};
\node[version node, fit=(locxcells-1-1) (locxcells-2-1), fill=white, inner sep= 0cm, font=\Large] (locx-v0) {$0$};
\node[version node, fit=(locxcells-1-3) (locxcells-2-3), fill=white, inner sep=0cm, font=\Large] (locx-v-1) {$1$};
%Location y
\path (locx.south) + (0,-1.5) node (locy) {$[\loc_y] \mapsto$};
\matrix(locycells) [version list, text width=7mm, anchor=west]
   at ([xshift=10pt]locy.east) {
 {a} & $\tsid_0$ \\
   {a} & $\emptyset$ \\
};
\node[version node, fit=(locycells-1-1) (locycells-2-1), fill=white, inner sep= 0cm, font=\Large] (locy-v0) {$0$};
% \draw[-, red, very thick, rounded corners] ([xshift=-5pt, yshift=5pt]locx-v1.north east) |- 
%  ($([xshift=-5pt,yshift=-5pt]locx-v1.south east)!.5!([xshift=-5pt, yshift=5pt]locy-v0.north east)$) -| ([xshift=-5pt, yshift=5pt]locy-v0.south east);

%blue view - I should  check whether I can use pgfkeys to just declare the list of locations, and then add the view automatically.
\draw[-, blue, very thick, rounded corners=10pt]
 ([xshift=-2pt, yshift=20pt]locx-v0.north east) node (tid1start) {} -- 
% ([xshift=-2pt, yshift=-5pt]locx-v0.south east) --
% ([xshift=-2pt, yshift=5pt]locy-v0.north east) -- 
 ([xshift=-2pt, yshift=-5pt]locy-v0.south east);
 
 \path (tid1start) node[anchor=south, rectangle, fill=blue!20, draw=blue, font=\small, inner sep=1pt] {$\tid_3$};

%red view
\draw[-, red, very thick, rounded corners = 10pt]
 ([xshift=-5pt, yshift=5pt]locx-v1.north east) -- 
 ([xshift=-5pt, yshift=-5pt]locx-v1.south east) --
 ([xshift=-5pt, yshift=3pt]locy-v0.north east) -- 
 ([xshift=-5pt, yshift=-10pt]locy-v0.south east) node (tid2start) {};
 
\path (tid2start) node[anchor=north, rectangle, fill=red!20, draw=red, font=\small, inner sep=1pt] {$\tid_2$};
 
 %green view
\draw[-, DarkGreen, very thick, rounded corners = 10pt]
 ([xshift=-16pt, yshift=8pt]locx-v1.north east) node (tid3start) {}-- 
 ([xshift=-16pt, yshift=-5pt]locx-v1.south east) --
 ([xshift=-16pt, yshift=5pt]locy-v0.north east) -- 
 ([xshift=-16pt, yshift=-5pt]locy-v0.south east);
 
 \path (tid3start) node[anchor=south, rectangle, fill=DarkGreen!20, draw=DarkGreen, font=\small, inner sep=1pt] {$\tid_1$};

\end{pgfonlayer}
\end{tikzpicture}
&
\begin{tikzpicture}[font=\large]

\begin{pgfonlayer}{foreground}
%Uncomment line below for help lines
%\draw[help lines] grid(5,4);

%Location x
\node(locx) at (1,3) {$[\loc_x] \mapsto$};

\matrix(locxcells) [version list, text width=7mm, anchor=west]
   at ([xshift=10pt]locx.east) {
 {a} & $\tsid_0$ & {a} & $\tsid_1$\\
  {a} & $\emptyset$ & {a} & $\{\tsid_2\}$ \\
};
\node[version node, fit=(locxcells-1-1) (locxcells-2-1), fill=white, inner sep= 0cm, font=\Large] (locx-v0) {$0$};
\node[version node, fit=(locxcells-1-3) (locxcells-2-3), fill=white, inner sep=0cm, font=\Large] (locx-v-1) {$1$};
%Location y
\path (locx.south) + (0,-1.5) node (locy) {$[\loc_y] \mapsto$};
\matrix(locycells) [version list, text width=7mm, anchor=west]
   at ([xshift=10pt]locy.east) {
 {a} & $\tsid_0$ & {a} & $\tsid_2$ \\
   {a} & $\emptyset$ & {a} & $\emptyset$\\
};
\node[version node, fit=(locycells-1-1) (locycells-2-1), fill=white, inner sep= 0cm, font=\Large] (locy-v0) {$0$};
\node[version node, fit=(locycells-1-3) (locycells-2-3), fill=white, inner sep=0cm, font=\Large] (locy-v-1) {$1$};
% \draw[-, red, very thick, rounded corners] ([xshift=-5pt, yshift=5pt]locx-v1.north east) |- 
%  ($([xshift=-5pt,yshift=-5pt]locx-v1.south east)!.5!([xshift=-5pt, yshift=5pt]locy-v0.north east)$) -| ([xshift=-5pt, yshift=5pt]locy-v0.south east);

%blue view - I should  check whether I can use pgfkeys to just declare the list of locations, and then add the view automatically.
\draw[-, blue, very thick, rounded corners=10pt]
 ([xshift=-2pt, yshift=20pt]locx-v0.north east) node (tid1start) {} -- 
% ([xshift=-2pt, yshift=-5pt]locx-v0.south east) --
% ([xshift=-2pt, yshift=5pt]locy-v0.north east) -- 
 ([xshift=-2pt, yshift=-5pt]locy-v0.south east);
 
 \path (tid1start) node[anchor=south, rectangle, fill=blue!20, draw=blue, font=\small, inner sep=1pt] {$\tid_3$};

%red view
\draw[-, red, very thick, rounded corners = 10pt]
 ([xshift=-5pt, yshift=5pt]locx-v1.north east) -- 
% ([xshift=-5pt, yshift=-5pt]locx-v1.south east) --
% ([xshift=-5pt, yshift=3pt]locy-v0.north east) -- 
 ([xshift=-5pt, yshift=-10pt]locy-v1.south east) node (tid2start) {};
 
\path (tid2start) node[anchor=north, rectangle, fill=red!20, draw=red, font=\small, inner sep=1pt] {$\tid_2$};
 
 %green view
\draw[-, DarkGreen, very thick, rounded corners = 10pt]
 ([xshift=-16pt, yshift=8pt]locx-v1.north east) node (tid3start) {}-- 
 ([xshift=-16pt, yshift=-5pt]locx-v1.south east) --
 ([xshift=-16pt, yshift=5pt]locy-v0.north east) -- 
 ([xshift=-16pt, yshift=-5pt]locy-v0.south east);
 
 \path (tid3start) node[anchor=south, rectangle, fill=DarkGreen!20, draw=DarkGreen, font=\small, inner sep=1pt] {$\tid_1$};

\end{pgfonlayer}
\end{tikzpicture}\\
{\small (c)} & {\small (d)} \\
\hline
\begin{tikzpicture}[font=\large]

\begin{pgfonlayer}{foreground}
%Uncomment line below for help lines
%\draw[help lines] grid(5,4);

%Location x
\node(locx) at (1,3) {$[\loc_x] \mapsto$};

\matrix(locxcells) [version list, text width=7mm, anchor=west]
   at ([xshift=10pt]locx.east) {
 {a} & $\tsid_0$ & {a} & $\tsid_1$\\
  {a} & $\emptyset$ & {a} & $\{\tsid_2\}$ \\
};
\node[version node, fit=(locxcells-1-1) (locxcells-2-1), fill=white, inner sep= 0cm, font=\Large] (locx-v0) {$0$};
\node[version node, fit=(locxcells-1-3) (locxcells-2-3), fill=white, inner sep=0cm, font=\Large] (locx-v-1) {$1$};
%Location y
\path (locx.south) + (0,-1.5) node (locy) {$[\loc_y] \mapsto$};
\matrix(locycells) [version list, text width=7mm, anchor=west]
   at ([xshift=10pt]locy.east) {
 {a} & $\tsid_0$ & {a} & $\tsid_2$ \\
   {a} & $\emptyset$ & {a} & $\emptyset$\\
};
\node[version node, fit=(locycells-1-1) (locycells-2-1), fill=white, inner sep= 0cm, font=\Large] (locy-v0) {$0$};
\node[version node, fit=(locycells-1-3) (locycells-2-3), fill=white, inner sep=0cm, font=\Large] (locy-v-1) {$1$};
% \draw[-, red, very thick, rounded corners] ([xshift=-5pt, yshift=5pt]locx-v1.north east) |- 
%  ($([xshift=-5pt,yshift=-5pt]locx-v1.south east)!.5!([xshift=-5pt, yshift=5pt]locy-v0.north east)$) -| ([xshift=-5pt, yshift=5pt]locy-v0.south east);

%blue view - I should  check whether I can use pgfkeys to just declare the list of locations, and then add the view automatically.
\draw[-, blue, very thick, rounded corners=10pt]
 ([xshift=-2pt, yshift=20pt]locx-v0.north east) node (tid1start) {} -- 
 ([xshift=-2pt, yshift=-5pt]locx-v0.south east) --
 ([xshift=-2pt, yshift=5pt]locy-v1.north east) -- 
 ([xshift=-2pt, yshift=-5pt]locy-v1.south east);
 
 \path (tid1start) node[anchor=south, rectangle, fill=blue!20, draw=blue, font=\small, inner sep=1pt] {$\tid_3$};

%red view
\draw[-, red, very thick, rounded corners = 10pt]
 ([xshift=-5pt, yshift=5pt]locx-v1.north east) -- 
% ([xshift=-5pt, yshift=-5pt]locx-v1.south east) --
% ([xshift=-5pt, yshift=3pt]locy-v0.north east) -- 
 ([xshift=-5pt, yshift=-10pt]locy-v1.south east) node (tid2start) {};
 
\path (tid2start) node[anchor=north, rectangle, fill=red!20, draw=red, font=\small, inner sep=1pt] {$\tid_2$};
 
 %green view
\draw[-, DarkGreen, very thick, rounded corners = 10pt]
 ([xshift=-16pt, yshift=8pt]locx-v1.north east) node (tid3start) {}-- 
 ([xshift=-16pt, yshift=-5pt]locx-v1.south east) --
 ([xshift=-16pt, yshift=5pt]locy-v0.north east) -- 
 ([xshift=-16pt, yshift=-5pt]locy-v0.south east);
 
 \path (tid3start) node[anchor=south, rectangle, fill=DarkGreen!20, draw=DarkGreen, font=\small, inner sep=1pt] {$\tid_1$};

\end{pgfonlayer}
\end{tikzpicture}
&
\begin{tikzpicture}[font=\large]

\begin{pgfonlayer}{foreground}
%Uncomment line below for help lines
%\draw[help lines] grid(5,4);

%Location x
\node(locx) at (1,3) {$[\loc_x] \mapsto$};

\matrix(locxcells) [version list, text width=7mm, anchor=west]
   at ([xshift=10pt]locx.east) {
 {a} & $\tsid_0$ & {a} & $\tsid_1$\\
  {a} & $\{\tsid_3\}$ & {a} & $\{\tsid_2\}$ \\
};
\node[version node, fit=(locxcells-1-1) (locxcells-2-1), fill=white, inner sep= 0cm, font=\Large] (locx-v0) {$0$};
\node[version node, fit=(locxcells-1-3) (locxcells-2-3), fill=white, inner sep=0cm, font=\Large] (locx-v1) {$1$};
%Location y
\path (locx.south) + (0,-1.5) node (locy) {$[\loc_y] \mapsto$};
\matrix(locycells) [version list, text width=7mm, anchor=west]
   at ([xshift=10pt]locy.east) {
 {a} & $\tsid_0$ & {a} & $\tsid_2$ \\
   {a} & $\emptyset$ & {a} & $\{\tsid_3\}$\\
};
\node[version node, fit=(locycells-1-1) (locycells-2-1), fill=white, inner sep= 0cm, font=\Large] (locy-v0) {$0$};
\node[version node, fit=(locycells-1-3) (locycells-2-3), fill=white, inner sep=0cm, font=\Large] (locy-v1) {$1$};
% \draw[-, red, very thick, rounded corners] ([xshift=-5pt, yshift=5pt]locx-v1.north east) |- 
%  ($([xshift=-5pt,yshift=-5pt]locx-v1.south east)!.5!([xshift=-5pt, yshift=5pt]locy-v0.north east)$) -| ([xshift=-5pt, yshift=5pt]locy-v0.south east);

%blue view - I should  check whether I can use pgfkeys to just declare the list of locations, and then add the view automatically.
\draw[-, blue, very thick, rounded corners=10pt]
 ([xshift=-2pt, yshift=20pt]locx-v0.north east) node (tid1start) {} -- 
 ([xshift=-2pt, yshift=-5pt]locx-v0.south east) --
 ([xshift=-2pt, yshift=5pt]locy-v1.north east) -- 
 ([xshift=-2pt, yshift=-5pt]locy-v1.south east);
 
 \path (tid1start) node[anchor=south, rectangle, fill=blue!20, draw=blue, font=\small, inner sep=1pt] {$\tid_3$};

%red view
\draw[-, red, very thick, rounded corners = 10pt]
 ([xshift=-5pt, yshift=5pt]locx-v1.north east) -- 
% ([xshift=-5pt, yshift=-5pt]locx-v1.south east) --
% ([xshift=-5pt, yshift=3pt]locy-v0.north east) -- 
 ([xshift=-5pt, yshift=-10pt]locy-v1.south east) node (tid2start) {};
 
\path (tid2start) node[anchor=north, rectangle, fill=red!20, draw=red, font=\small, inner sep=1pt] {$\tid_2$};
 
 %green view
\draw[-, DarkGreen, very thick, rounded corners = 10pt]
 ([xshift=-16pt, yshift=8pt]locx-v1.north east) node (tid3start) {}-- 
 ([xshift=-16pt, yshift=-5pt]locx-v1.south east) --
 ([xshift=-16pt, yshift=5pt]locy-v0.north east) -- 
 ([xshift=-16pt, yshift=-5pt]locy-v0.south east);
 
 \path (tid3start) node[anchor=south, rectangle, fill=DarkGreen!20, draw=DarkGreen, font=\small, inner sep=1pt] {$\tid_1$};

\end{pgfonlayer}
\end{tikzpicture}
\\
{\small (e)} & {\small (f)} \\
\hline
\end{tabular}
\caption{History heaps obtained in a execution of $\prog_2$.}
\label{fig:cc.exec}
\end{figure}
The next consistency model that we consider is \emph{transactional causal consistency} 
\cite{cops}. Intuitively, in this consistency model transactions must be ensured 
that versions read by transactions are closed with respect to causal dependencies. 
Consider for example the following program: 
\[
    \prog_2 := \begin{session}
        \begin{array}{@{}c || c || c@{}}
            \begin{transaction}
            		 \pmutate{\loc_x}{1};\\
              \end{transaction} &
              \begin{transaction}
            		  \pderef{\pvar{a}}{\loc_x};\\
            		  \pmutate{\loc_y}{\pvar{a}};\\
            	  \end{transaction} &
              \begin{transaction}
            		   \pderef{\pvar{a}}{\loc_x};\\
            		   \pderef{\pvar{b}}{\loc_y};\\
            		   \pifs{\pvar{a}=0 \wedge \pvar{b}=1}\\
            				\;\;\;\;\passign{\retvar}{\Large \frownie{}}
            			\}
             \end{transaction}
        \end{array}
    \end{session}
 \]
For the sake of simplicity, we label the code of the three transactions above 
ast $\trans_1, \trans_2, \trans_3$, from left to right.
It is easy to see that, if no constraints are placed on the  consistency model, 
then we can have $\ptrans{\trans_3}$ return ${\Large \frownie{}}$. 
The same is true even if the consistency model specification $\mathsf{RA}$ is assumed. 
Informally, the return of value ${\Large \frownie{}}$ by $\ptrans{\trans_3}$ can be 
obtained from the execution outlined below. 
\begin{itemize}
\item 
%In the initial history heap of the execution, 
%$\hh_0$,  both locations $\loc_x, \loc_y$ contain a single version with value $0$, written by transaction $\tsid_0$. 
The initial configuration of this execution is depicted in Figure \ref{fig:cc.exec}(a).
%\begin{center}
%\begin{tikzpicture}[font=\large]
%
%\begin{pgfonlayer}{foreground}
%%Uncomment line below for help lines
%%\draw[help lines] grid(5,4);
%
%%Location x
%\node(locx) at (1,3) {$[\loc_x] \mapsto$};
%
%\matrix(locxcells) [version list, text width=7mm, anchor=west]
%   at ([xshift=10pt]locx.east) {
% {a} & $T_0$ \\
%  {a} & $\emptyset$ \\
%};
%\node[version node, fit=(locxcells-1-1) (locxcells-2-1), fill=white, inner sep= 0cm, font=\Large] (locx-v0) {$0$};
%
%%Location y
%\path (locx.south) + (0,-1.5) node (locy) {$[\loc_y] \mapsto$};
%\matrix(locycells) [version list, text width=7mm, anchor=west]
%   at ([xshift=10pt]locy.east) {
% {a} & $T_0$ \\
%  {a} & $\emptyset$ \\
%};
%\node[version node, fit=(locycells-1-1) (locycells-2-1), fill=white, inner sep= 0cm, font=\Large] (locy-v0) {$0$};
%
%% \draw[-, red, very thick, rounded corners] ([xshift=-5pt, yshift=5pt]locx-v1.north east) |- 
%%  ($([xshift=-5pt,yshift=-5pt]locx-v1.south east)!.5!([xshift=-5pt, yshift=5pt]locy-v0.north east)$) -| ([xshift=-5pt, yshift=5pt]locy-v0.south east);
%
%%blue view - I should  check whether I can use pgfkeys to just declare the list of locations, and then add the view automatically.
%\draw[-, blue, very thick, rounded corners=10pt]
% ([xshift=-2pt, yshift=20pt]locx-v0.north east) node (tid1start) {} -- 
%% ([xshift=-2pt, yshift=-5pt]locx-v0.south east) --
%% ([xshift=-2pt, yshift=5pt]locy-v0.north east) -- 
% ([xshift=-2pt, yshift=-5pt]locy-v0.south east);
% 
% \path (tid1start) node[anchor=south, rectangle, fill=blue!20, draw=blue, font=\small, inner sep=1pt] {$\tid_3$};
%
%%red view
%\draw[-, red, very thick, rounded corners = 10pt]
% ([xshift=-5pt, yshift=5pt]locx-v0.north east) -- 
%% ([xshift=-8pt, yshift=-5pt]locx-v0.south east) --
%% ([xshift=-8pt, yshift=5pt]locy-v0.north east) -- 
% ([xshift=-5pt, yshift=-10pt]locy-v0.south east) node (tid2start) {};
% 
%\path (tid2start) node[anchor=north, rectangle, fill=red!20, draw=red, font=\small, inner sep=1pt] {$\tid_2$};
% 
% %green view
%\draw[-, DarkGreen, very thick, rounded corners = 10pt]
% ([xshift=-16pt, yshift=8pt]locx-v0.north east) node (tid3start) {}-- 
%% ([xshift=-15pt, yshift=-5pt]locx-v0.south east) --
%% ([xshift=-15pt, yshift=5pt]locy-v0.north east) -- 
% ([xshift=-16pt, yshift=-5pt]locy-v0.south east);
% 
% \path (tid3start) node[anchor=south, rectangle, fill=DarkGreen!20, draw=DarkGreen, font=\small, inner sep=1pt] {$\tid_1$};
%
%\end{pgfonlayer}
%\end{tikzpicture}
%\end{center}
\item $\ptrans{\trans_1}$ executes with the initial view, which points to the 
initial (and only) version for each location; after this transaction is 
executed, a new version $\langle 1, T_1, \emptyset \rangle$ is appended 
at the end of $\hh(\loc_{x})$. The resulting history heap is depicted in Figure \ref{fig:cc.exec}(b).
%\begin{center}
%\begin{tikzpicture}[font=\large]
%
%\begin{pgfonlayer}{foreground}
%%Uncomment line below for help lines
%%\draw[help lines] grid(5,4);
%
%%Location x
%\node(locx) at (1,3) {$[\loc_x] \mapsto$};
%
%\matrix(locxcells) [version list, text width=7mm, anchor=west]
%   at ([xshift=10pt]locx.east) {
% {a} & $\tsid_0$ & {a} & $\tsid_1$\\
%  {a} & $\emptyset$ & {a} & $\emptyset$ \\
%};
%\node[version node, fit=(locxcells-1-1) (locxcells-2-1), fill=white, inner sep= 0cm, font=\Large] (locx-v0) {$0$};
%\node[version node, fit=(locxcells-1-3) (locxcells-2-3), fill=white, inner sep=0cm, font=\Large] (locx-v-1) {$1$};
%%Location y
%\path (locx.south) + (0,-1.5) node (locy) {$[\loc_y] \mapsto$};
%\matrix(locycells) [version list, text width=7mm, anchor=west]
%   at ([xshift=10pt]locy.east) {
% {a} & $\tsid_0$ \\
%   {a} & $\emptyset$ \\
%};
%\node[version node, fit=(locycells-1-1) (locycells-2-1), fill=white, inner sep= 0cm, font=\Large] (locy-v0) {$0$};
%% \draw[-, red, very thick, rounded corners] ([xshift=-5pt, yshift=5pt]locx-v1.north east) |- 
%%  ($([xshift=-5pt,yshift=-5pt]locx-v1.south east)!.5!([xshift=-5pt, yshift=5pt]locy-v0.north east)$) -| ([xshift=-5pt, yshift=5pt]locy-v0.south east);
%
%%blue view - I should  check whether I can use pgfkeys to just declare the list of locations, and then add the view automatically.
%\draw[-, blue, very thick, rounded corners=10pt]
% ([xshift=-2pt, yshift=20pt]locx-v0.north east) node (tid1start) {} -- 
%% ([xshift=-2pt, yshift=-5pt]locx-v0.south east) --
%% ([xshift=-2pt, yshift=5pt]locy-v0.north east) -- 
% ([xshift=-2pt, yshift=-5pt]locy-v0.south east);
% 
% \path (tid1start) node[anchor=south, rectangle, fill=blue!20, draw=blue, font=\small, inner sep=1pt] {$\tid_3$};
%
%%red view
%\draw[-, red, very thick, rounded corners = 10pt]
% ([xshift=-5pt, yshift=5pt]locx-v0.north east) -- 
%% ([xshift=-8pt, yshift=-5pt]locx-v0.south east) --
%% ([xshift=-8pt, yshift=5pt]locy-v0.north east) -- 
% ([xshift=-5pt, yshift=-10pt]locy-v0.south east) node (tid2start) {};
% 
%\path (tid2start) node[anchor=north, rectangle, fill=red!20, draw=red, font=\small, inner sep=1pt] {$\tid_2$};
% 
% %green view
%\draw[-, DarkGreen, very thick, rounded corners = 10pt]
% ([xshift=-16pt, yshift=8pt]locx-v1.north east) node (tid3start) {}-- 
% ([xshift=-16pt, yshift=-5pt]locx-v1.south east) --
% ([xshift=-16pt, yshift=5pt]locy-v0.north east) -- 
% ([xshift=-16pt, yshift=-5pt]locy-v0.south east);
% 
% \path (tid3start) node[anchor=south, rectangle, fill=DarkGreen!20, draw=DarkGreen, font=\small, inner sep=1pt] {$\tid_1$};
%
%\end{pgfonlayer}
%\end{tikzpicture}
%\end{center}
\item next, $\tid_2$ updates its view as to see the version of $\loc_x$ installed by $\tid_1$, after 
which it proceeds to execute $\ptrans{\trans_2}$. This results in a new version with value $1$ 
to be installed for $\loc_y$. The configurations before, and after the execution of $\ptrans{\trans_2}$, 
are depicted in figures \ref{fig:cc.exec}(c) and \ref{fig:cc.exec}(d), respectively.
%\begin{center}
%\begin{tikzpicture}[font=\large]
%
%\begin{pgfonlayer}{foreground}
%%Uncomment line below for help lines
%%\draw[help lines] grid(5,4);
%
%%Location x
%\node(locx) at (1,3) {$[\loc_x] \mapsto$};
%
%\matrix(locxcells) [version list, text width=7mm, anchor=west]
%   at ([xshift=10pt]locx.east) {
% {a} & $\tsid_0$ & {a} & $\tsid_1$\\
%  {a} & $\emptyset$ & {a} & $\emptyset$ \\
%};
%\node[version node, fit=(locxcells-1-1) (locxcells-2-1), fill=white, inner sep= 0cm, font=\Large] (locx-v0) {$0$};
%\node[version node, fit=(locxcells-1-3) (locxcells-2-3), fill=white, inner sep=0cm, font=\Large] (locx-v-1) {$1$};
%%Location y
%\path (locx.south) + (0,-1.5) node (locy) {$[\loc_y] \mapsto$};
%\matrix(locycells) [version list, text width=7mm, anchor=west]
%   at ([xshift=10pt]locy.east) {
% {a} & $\tsid_0$ \\
%   {a} & $\emptyset$ \\
%};
%\node[version node, fit=(locycells-1-1) (locycells-2-1), fill=white, inner sep= 0cm, font=\Large] (locy-v0) {$0$};
%% \draw[-, red, very thick, rounded corners] ([xshift=-5pt, yshift=5pt]locx-v1.north east) |- 
%%  ($([xshift=-5pt,yshift=-5pt]locx-v1.south east)!.5!([xshift=-5pt, yshift=5pt]locy-v0.north east)$) -| ([xshift=-5pt, yshift=5pt]locy-v0.south east);
%
%%blue view - I should  check whether I can use pgfkeys to just declare the list of locations, and then add the view automatically.
%\draw[-, blue, very thick, rounded corners=10pt]
% ([xshift=-2pt, yshift=20pt]locx-v0.north east) node (tid1start) {} -- 
%% ([xshift=-2pt, yshift=-5pt]locx-v0.south east) --
%% ([xshift=-2pt, yshift=5pt]locy-v0.north east) -- 
% ([xshift=-2pt, yshift=-5pt]locy-v0.south east);
% 
% \path (tid1start) node[anchor=south, rectangle, fill=blue!20, draw=blue, font=\small, inner sep=1pt] {$\tid_3$};
%
%%red view
%\draw[-, red, very thick, rounded corners = 10pt]
% ([xshift=-5pt, yshift=5pt]locx-v1.north east) -- 
% ([xshift=-5pt, yshift=-5pt]locx-v1.south east) --
% ([xshift=-5pt, yshift=3pt]locy-v0.north east) -- 
% ([xshift=-5pt, yshift=-10pt]locy-v0.south east) node (tid2start) {};
% 
%\path (tid2start) node[anchor=north, rectangle, fill=red!20, draw=red, font=\small, inner sep=1pt] {$\tid_2$};
% 
% %green view
%\draw[-, DarkGreen, very thick, rounded corners = 10pt]
% ([xshift=-16pt, yshift=8pt]locx-v1.north east) node (tid3start) {}-- 
% ([xshift=-16pt, yshift=-5pt]locx-v1.south east) --
% ([xshift=-16pt, yshift=5pt]locy-v0.north east) -- 
% ([xshift=-16pt, yshift=-5pt]locy-v0.south east);
% 
% \path (tid3start) node[anchor=south, rectangle, fill=DarkGreen!20, draw=DarkGreen, font=\small, inner sep=1pt] {$\tid_1$};
%
%\end{pgfonlayer}
%\end{tikzpicture}
%\hfill
%\begin{tikzpicture}[font=\large]
%
%\begin{pgfonlayer}{foreground}
%%Uncomment line below for help lines
%%\draw[help lines] grid(5,4);
%
%%Location x
%\node(locx) at (1,3) {$[\loc_x] \mapsto$};
%
%\matrix(locxcells) [version list, text width=7mm, anchor=west]
%   at ([xshift=10pt]locx.east) {
% {a} & $\tsid_0$ & {a} & $\tsid_1$\\
%  {a} & $\emptyset$ & {a} & $\{\tsid_2\}$ \\
%};
%\node[version node, fit=(locxcells-1-1) (locxcells-2-1), fill=white, inner sep= 0cm, font=\Large] (locx-v0) {$0$};
%\node[version node, fit=(locxcells-1-3) (locxcells-2-3), fill=white, inner sep=0cm, font=\Large] (locx-v-1) {$1$};
%%Location y
%\path (locx.south) + (0,-1.5) node (locy) {$[\loc_y] \mapsto$};
%\matrix(locycells) [version list, text width=7mm, anchor=west]
%   at ([xshift=10pt]locy.east) {
% {a} & $\tsid_0$ & {a} & $\tsid_2$ \\
%   {a} & $\emptyset$ & {a} & $\emptyset$\\
%};
%\node[version node, fit=(locycells-1-1) (locycells-2-1), fill=white, inner sep= 0cm, font=\Large] (locy-v0) {$0$};
%\node[version node, fit=(locycells-1-3) (locycells-2-3), fill=white, inner sep=0cm, font=\Large] (locy-v-1) {$1$};
%% \draw[-, red, very thick, rounded corners] ([xshift=-5pt, yshift=5pt]locx-v1.north east) |- 
%%  ($([xshift=-5pt,yshift=-5pt]locx-v1.south east)!.5!([xshift=-5pt, yshift=5pt]locy-v0.north east)$) -| ([xshift=-5pt, yshift=5pt]locy-v0.south east);
%
%%blue view - I should  check whether I can use pgfkeys to just declare the list of locations, and then add the view automatically.
%\draw[-, blue, very thick, rounded corners=10pt]
% ([xshift=-2pt, yshift=20pt]locx-v0.north east) node (tid1start) {} -- 
%% ([xshift=-2pt, yshift=-5pt]locx-v0.south east) --
%% ([xshift=-2pt, yshift=5pt]locy-v0.north east) -- 
% ([xshift=-2pt, yshift=-5pt]locy-v0.south east);
% 
% \path (tid1start) node[anchor=south, rectangle, fill=blue!20, draw=blue, font=\small, inner sep=1pt] {$\tid_3$};
%
%%red view
%\draw[-, red, very thick, rounded corners = 10pt]
% ([xshift=-5pt, yshift=5pt]locx-v1.north east) -- 
%% ([xshift=-5pt, yshift=-5pt]locx-v1.south east) --
%% ([xshift=-5pt, yshift=3pt]locy-v0.north east) -- 
% ([xshift=-5pt, yshift=-10pt]locy-v1.south east) node (tid2start) {};
% 
%\path (tid2start) node[anchor=north, rectangle, fill=red!20, draw=red, font=\small, inner sep=1pt] {$\tid_2$};
% 
% %green view
%\draw[-, DarkGreen, very thick, rounded corners = 10pt]
% ([xshift=-16pt, yshift=8pt]locx-v1.north east) node (tid3start) {}-- 
% ([xshift=-16pt, yshift=-5pt]locx-v1.south east) --
% ([xshift=-16pt, yshift=5pt]locy-v0.north east) -- 
% ([xshift=-16pt, yshift=-5pt]locy-v0.south east);
% 
% \path (tid3start) node[anchor=south, rectangle, fill=DarkGreen!20, draw=DarkGreen, font=\small, inner sep=1pt] {$\tid_1$};
%
%\end{pgfonlayer}
%\end{tikzpicture}
%\end{center}
\item Finally, thread $\tid_3$ updates its view to observe the update of location $[\loc_y]$, but not the update of 
location $[\loc_y]$, before executing transaction $\ptrans{\trans_3}$. The execution of $\ptrans{\trans_3}$ will 
return the value ${\Large \frownie{}}$. The history heaps immediately before and after 
the execution of $\tid_3$, are depicted in figures \ref{fig:cc.exec}(e) and \ref{fig:cc.exec}(f), respectively. 
%\begin{center}
%\begin{tikzpicture}[font=\large]
%
%\begin{pgfonlayer}{foreground}
%%Uncomment line below for help lines
%%\draw[help lines] grid(5,4);
%
%%Location x
%\node(locx) at (1,3) {$[\loc_x] \mapsto$};
%
%\matrix(locxcells) [version list, text width=7mm, anchor=west]
%   at ([xshift=10pt]locx.east) {
% {a} & $\tsid_0$ & {a} & $\tsid_1$\\
%  {a} & $\emptyset$ & {a} & $\{\tsid_2\}$ \\
%};
%\node[version node, fit=(locxcells-1-1) (locxcells-2-1), fill=white, inner sep= 0cm, font=\Large] (locx-v0) {$0$};
%\node[version node, fit=(locxcells-1-3) (locxcells-2-3), fill=white, inner sep=0cm, font=\Large] (locx-v-1) {$1$};
%%Location y
%\path (locx.south) + (0,-1.5) node (locy) {$[\loc_y] \mapsto$};
%\matrix(locycells) [version list, text width=7mm, anchor=west]
%   at ([xshift=10pt]locy.east) {
% {a} & $\tsid_0$ & {a} & $\tsid_2$ \\
%   {a} & $\emptyset$ & {a} & $\emptyset$\\
%};
%\node[version node, fit=(locycells-1-1) (locycells-2-1), fill=white, inner sep= 0cm, font=\Large] (locy-v0) {$0$};
%\node[version node, fit=(locycells-1-3) (locycells-2-3), fill=white, inner sep=0cm, font=\Large] (locy-v-1) {$1$};
%% \draw[-, red, very thick, rounded corners] ([xshift=-5pt, yshift=5pt]locx-v1.north east) |- 
%%  ($([xshift=-5pt,yshift=-5pt]locx-v1.south east)!.5!([xshift=-5pt, yshift=5pt]locy-v0.north east)$) -| ([xshift=-5pt, yshift=5pt]locy-v0.south east);
%
%%blue view - I should  check whether I can use pgfkeys to just declare the list of locations, and then add the view automatically.
%\draw[-, blue, very thick, rounded corners=10pt]
% ([xshift=-2pt, yshift=20pt]locx-v0.north east) node (tid1start) {} -- 
% ([xshift=-2pt, yshift=-5pt]locx-v0.south east) --
% ([xshift=-2pt, yshift=5pt]locy-v1.north east) -- 
% ([xshift=-2pt, yshift=-5pt]locy-v1.south east);
% 
% \path (tid1start) node[anchor=south, rectangle, fill=blue!20, draw=blue, font=\small, inner sep=1pt] {$\tid_3$};
%
%%red view
%\draw[-, red, very thick, rounded corners = 10pt]
% ([xshift=-5pt, yshift=5pt]locx-v1.north east) -- 
%% ([xshift=-5pt, yshift=-5pt]locx-v1.south east) --
%% ([xshift=-5pt, yshift=3pt]locy-v0.north east) -- 
% ([xshift=-5pt, yshift=-10pt]locy-v1.south east) node (tid2start) {};
% 
%\path (tid2start) node[anchor=north, rectangle, fill=red!20, draw=red, font=\small, inner sep=1pt] {$\tid_2$};
% 
% %green view
%\draw[-, DarkGreen, very thick, rounded corners = 10pt]
% ([xshift=-16pt, yshift=8pt]locx-v1.north east) node (tid3start) {}-- 
% ([xshift=-16pt, yshift=-5pt]locx-v1.south east) --
% ([xshift=-16pt, yshift=5pt]locy-v0.north east) -- 
% ([xshift=-16pt, yshift=-5pt]locy-v0.south east);
% 
% \path (tid3start) node[anchor=south, rectangle, fill=DarkGreen!20, draw=DarkGreen, font=\small, inner sep=1pt] {$\tid_1$};
%
%\end{pgfonlayer}
%\end{tikzpicture}
%\hfill
%\begin{tikzpicture}[font=\large]
%
%\begin{pgfonlayer}{foreground}
%%Uncomment line below for help lines
%%\draw[help lines] grid(5,4);
%
%%Location x
%\node(locx) at (1,3) {$[\loc_x] \mapsto$};
%
%\matrix(locxcells) [version list, text width=7mm, anchor=west]
%   at ([xshift=10pt]locx.east) {
% {a} & $\tsid_0$ & {a} & $\tsid_1$\\
%  {a} & $\{\tsid_3\}$ & {a} & $\{\tsid_2\}$ \\
%};
%\node[version node, fit=(locxcells-1-1) (locxcells-2-1), fill=white, inner sep= 0cm, font=\Large] (locx-v0) {$0$};
%\node[version node, fit=(locxcells-1-3) (locxcells-2-3), fill=white, inner sep=0cm, font=\Large] (locx-v1) {$1$};
%%Location y
%\path (locx.south) + (0,-1.5) node (locy) {$[\loc_y] \mapsto$};
%\matrix(locycells) [version list, text width=7mm, anchor=west]
%   at ([xshift=10pt]locy.east) {
% {a} & $\tsid_0$ & {a} & $\tsid_2$ \\
%   {a} & $\emptyset$ & {a} & $\{\tsid_3\}$\\
%};
%\node[version node, fit=(locycells-1-1) (locycells-2-1), fill=white, inner sep= 0cm, font=\Large] (locy-v0) {$0$};
%\node[version node, fit=(locycells-1-3) (locycells-2-3), fill=white, inner sep=0cm, font=\Large] (locy-v1) {$1$};
%% \draw[-, red, very thick, rounded corners] ([xshift=-5pt, yshift=5pt]locx-v1.north east) |- 
%%  ($([xshift=-5pt,yshift=-5pt]locx-v1.south east)!.5!([xshift=-5pt, yshift=5pt]locy-v0.north east)$) -| ([xshift=-5pt, yshift=5pt]locy-v0.south east);
%
%%blue view - I should  check whether I can use pgfkeys to just declare the list of locations, and then add the view automatically.
%\draw[-, blue, very thick, rounded corners=10pt]
% ([xshift=-2pt, yshift=20pt]locx-v0.north east) node (tid1start) {} -- 
% ([xshift=-2pt, yshift=-5pt]locx-v0.south east) --
% ([xshift=-2pt, yshift=5pt]locy-v1.north east) -- 
% ([xshift=-2pt, yshift=-5pt]locy-v1.south east);
% 
% \path (tid1start) node[anchor=south, rectangle, fill=blue!20, draw=blue, font=\small, inner sep=1pt] {$\tid_3$};
%
%%red view
%\draw[-, red, very thick, rounded corners = 10pt]
% ([xshift=-5pt, yshift=5pt]locx-v1.north east) -- 
%% ([xshift=-5pt, yshift=-5pt]locx-v1.south east) --
%% ([xshift=-5pt, yshift=3pt]locy-v0.north east) -- 
% ([xshift=-5pt, yshift=-10pt]locy-v1.south east) node (tid2start) {};
% 
%\path (tid2start) node[anchor=north, rectangle, fill=red!20, draw=red, font=\small, inner sep=1pt] {$\tid_2$};
% 
% %green view
%\draw[-, DarkGreen, very thick, rounded corners = 10pt]
% ([xshift=-16pt, yshift=8pt]locx-v1.north east) node (tid3start) {}-- 
% ([xshift=-16pt, yshift=-5pt]locx-v1.south east) --
% ([xshift=-16pt, yshift=5pt]locy-v0.north east) -- 
% ([xshift=-16pt, yshift=-5pt]locy-v0.south east);
% 
% \path (tid3start) node[anchor=south, rectangle, fill=DarkGreen!20, draw=DarkGreen, font=\small, inner sep=1pt] {$\tid_1$};
%
%\end{pgfonlayer}
%\end{tikzpicture}
%\end{center}
\end{itemize}

%\ac{The only way I can think to define causal consistency right now, is by using 
%write-read dependencies. Hopefully there is a more operational specification. 
%
%I take it back, I can give an inductive definition of causally consistent views, 
%without using dependencies explicitly - though in practice I am still tracking down 
%write-read dependencies.}
The reason why the execution outlined above is because, in the last step, 
thread $\tid_3$ executed transaction $\ptrans{\trans_3}$ in a 
state where its view observes the update to location $[\loc_y]$, but 
not the update to location $\loc_2$. However, the update of $[\loc_y]$ 
performed in  $\ptrans{\trans_2}$, consisted in copying the value of the update 
of $[\loc_y]$: that is, the update of $[\loc_y]$ \textbf{depends} from the update of $[\loc_x]$. 
Summarising, the execution of transaction $\ptrans{\trans_3}$ resulted in a violation of 
causality: the update of $[\loc_y]$ is observed, but not the update of $[\loc_x]$ on which 
it depends.

To specify formally transactional causal consistency, we define inductively the 
set of views that are consistent with respect to a history heap $\hh$. 
Intuitively, the definition below models the fact that, if we start from a 
causally consistent view, and we wish to update the view for some location $\tid_2$, 


\begin{definition}
Let $\nu_1 = (v_1, \tsid_1, \mathcal{T}_1)$, $\nu_2 = (v_2, \tsid_2, \mathcal{T}_2)$. 
We say that $\nu_1$ directly depends on $\nu_2$, written $\nu_2 \xrightarrow{\ddep} \nu_1$, 
 if $\tsid_1 \in \mathcal{T}_2$. 
\ac{Note to self: the notion of directly depends here has little to do with dependencies 
between transactions. $\nu_1 \xrightarrow{\mathsf{ddep}} \nu_2$ means that 
some transaction $\tsid$ touches both versions. However, it reads $\nu_2$ and 
writes $\nu_1$.}

Let $\hh$ be a history heap. The set of views that are \emph{causally consistent} 
with respect to $\hh$, $\mathsf{CCViews}(\hh)$, is defined as the smallest set such that: 
\begin{enumerate} 
\item $V_0 := \lambda [\loc_x].0 \in \mathsf{CCViews}(\hh)$, 
\item let $V \in \mathsf{CCViews}(\hh)$, $V' = V[[\loc_x] \mapsto (V(x) + 1)]$, 
%for some $i: V([\loc_x]) < i \leq \lvert \hh \rvert -1$
for some $[\loc_x]$ such that $V([\loc_x]) < \lvert \hh([\loc_x]) \rvert - 1$.
Suppose that 
\[
\forall [\loc_y].\;\forall j = 0,\cdots, \lvert \hh([\loc_y]) \rvert -1.\; 
\hh([\loc_y])(j) \xrightarrow{\mathsf{ddep}} \hh([\loc_{x})(V'([\loc_x])) 
\implies j \leq V([\loc_y]).
\]
Then $V' \in \mathsf{CCViews}(\hh)$.
%for any location $[\loc_y]$ and 
%index $j = 0, \cdots, \lvert \hh([\loc_y]) \rvert -1$ such that $\hh([\loc_{x})(V'([\loc_x]))$ 
%directly depends on $\hh([\loc_y])(j)$, then $j \leq V([\loc_y])$. Then 
%$V' \in \mathsf{CCViews}(\hh)$.
% and suppose that $\hh([\loc_{x}])(V'([\loc_x])) = 
%(\_, \tsid, \_ )$ for some $\tsid$. If for all locations $\loc_{y}$ and 
%indexes $j$ such that $\hh([\loc_y])(j) = (\_, \_, \_ \cup \{\tsid\})$, 
%then $j \leq V'([\loc_y])$, then $V' \in \mathsf{CCViews}(\hh)$.
\end{enumerate}
\end{definition}

\begin{figure}
\begin{tabular}{|c|c|} 
\hline
\begin{tikzpicture}[font=\large]

\begin{pgfonlayer}{foreground}
%Uncomment line below for help lines
%\draw[help lines] grid(5,4);

%Location x
\node(locx) at (1,3) {$[\loc_x] \mapsto$};

\matrix(locxcells) [version list, text width=7mm, anchor=west]
   at ([xshift=10pt]locx.east) {
 {a} & $\tsid_0$ & {a} & $\tsid_1$\\
  {a} & $\emptyset$ & {a} & $\{\tsid_2\}$ \\
};
\node[version node, fit=(locxcells-1-1) (locxcells-2-1), fill=white, inner sep= 0cm, font=\Large] (locx-v0) {$0$};
\node[version node, fit=(locxcells-1-3) (locxcells-2-3), fill=white, inner sep=0cm, font=\Large] (locx-v-1) {$1$};
%Location y
\path (locx.south) + (0,-1.5) node (locy) {$[\loc_y] \mapsto$};
\matrix(locycells) [version list, text width=7mm, anchor=west]
   at ([xshift=10pt]locy.east) {
 {a} & $\tsid_0$ & {a} & $\tsid_2$ \\
   {a} & $\emptyset$ & {a} & $\{\tsid_3\}$\\
};
\node[version node, fit=(locycells-1-1) (locycells-2-1), fill=white, inner sep= 0cm, font=\Large] (locy-v0) {$0$};
\node[version node, fit=(locycells-1-3) (locycells-2-3), fill=white, inner sep=0cm, font=\Large] (locy-v1) {$1$};
% \draw[-, red, very thick, rounded corners] ([xshift=-5pt, yshift=5pt]locx-v1.north east) |- 
%  ($([xshift=-5pt,yshift=-5pt]locx-v1.south east)!.5!([xshift=-5pt, yshift=5pt]locy-v0.north east)$) -| ([xshift=-5pt, yshift=5pt]locy-v0.south east);

\path (locy.south) + (0,-1.5) node (locy) {$[\loc_z] \mapsto$};
\matrix(loczcells) [version list, text width=7mm, anchor=west]
   at ([xshift=10pt]locy.east) {
 {a} & $\tsid_0$ & {a} & $\tsid_3$ \\
   {a} & $\emptyset$ & {a} & $\emptyset$\\
};
\node[version node, fit=(loczcells-1-1) (loczcells-2-1), fill=white, inner sep= 0cm, font=\Large] (locz-v0) {$0$};
\node[version node, fit=(loczcells-1-3) (loczcells-2-3), fill=white, inner sep=0cm, font=\Large] (locz-v1) {$1$};

%blue view - I should  check whether I can use pgfkeys to just declare the list of locations, and then add the view automatically.
\draw[-, blue, very thick, rounded corners=10pt]
 ([xshift=-5pt, yshift=5pt]locx-v0.north east) node (tid1start) {} -- 
% ([xshift=-5pt, yshift=-5pt]locx-v0.south east) --
% ([xshift=-5pt, yshift=5pt]locy-v1.north east) -- 
 ([xshift=-5pt, yshift=-5pt]locz-v0.south east);
 
% \path (tid1start) node[anchor=south, rectangle, fill=blue!20, draw=blue, font=\small, inner sep=1pt] {$\tid_3$};
\end{pgfonlayer}
\end{tikzpicture}
&
\begin{tikzpicture}[font=\large]

\begin{pgfonlayer}{foreground}
%Uncomment line below for help lines
%\draw[help lines] grid(5,4);

%Location x
\node(locx) at (1,3) {$[\loc_x] \mapsto$};

\matrix(locxcells) [version list, text width=7mm, anchor=west]
   at ([xshift=10pt]locx.east) {
 {a} & $\tsid_0$ & {a} & $\tsid_1$\\
  {a} & $\emptyset$ & {a} & $\{\tsid_2\}$ \\
};
\node[version node, fit=(locxcells-1-1) (locxcells-2-1), fill=white, inner sep= 0cm, font=\Large] (locx-v0) {$0$};
\node[version node, fit=(locxcells-1-3) (locxcells-2-3), fill=white, inner sep=0cm, font=\Large] (locx-v1) {$1$};

%Location y
\path (locx.south) + (0,-1.5) node (locy) {$[\loc_y] \mapsto$};
\matrix(locycells) [version list, text width=7mm, anchor=west]
   at ([xshift=10pt]locy.east) {
 {a} & $\tsid_0$ & {a} & $\tsid_2$ \\
   {a} & $\emptyset$ & {a} & $\{\tsid_3\}$\\
};
\node[version node, fit=(locycells-1-1) (locycells-2-1), fill=white, inner sep= 0cm, font=\Large] (locy-v0) {$0$};
\node[version node, fit=(locycells-1-3) (locycells-2-3), fill=white, inner sep=0cm, font=\Large] (locy-v1) {$1$};



\path (locy.south) + (0,-1.5) node (locz) {$[\loc_z] \mapsto$};
\matrix(loczcells) [version list, text width=7mm, anchor=west]
   at ([xshift=10pt]locz.east) {
 {a} & $\tsid_0$ & {a} & $\tsid_3$ \\
   {a} & $\emptyset$ & {a} & $\emptyset$\\
};

\node[version node, fit=(loczcells-1-1) (loczcells-2-1), fill=white, inner sep= 0cm, font=\Large] (locz-v0) {$0$};
\node[version node, fit=(loczcells-1-3) (loczcells-2-3), fill=white, inner sep=0cm, font=\Large] (locz-v1) {$1$};

%blue view - I should  check whether I can use pgfkeys to just declare the list of locations, and then add the view automatically.
\draw[-, blue, very thick, rounded corners=10pt]
 ([xshift=-5pt, yshift=5pt]locx-v1.north east) node (tid1start) {} -- 
 ([xshift=-5pt, yshift=-5pt]locx-v1.south east) --
 ([xshift=-5pt, yshift=5pt]locy-v0.north east) -- 
% ([xshift=-5pt, yshift=-5pt]locy-v0.south east) --
% ([xshift=-5pt, yshift=5pt]locz-v0.north east) -- 
 ([xshift=-5pt, yshift=-5pt]locz-v0.south east);
 
% \path (tid1start) node[anchor=south, rectangle, fill=blue!20, draw=blue, font=\small, inner sep=1pt] {$\tid_3$};
\end{pgfonlayer}
\end{tikzpicture}
\\
{\small(a)} & {\small(b)}\\
\hline
\begin{tikzpicture}[font=\large]

\begin{pgfonlayer}{foreground}
%Uncomment line below for help lines
%\draw[help lines] grid(5,4);

%Location x
\node(locx) at (1,3) {$[\loc_x] \mapsto$};

\matrix(locxcells) [version list, text width=7mm, anchor=west]
   at ([xshift=10pt]locx.east) {
 {a} & $\tsid_0$ & {a} & $\tsid_1$\\
  {a} & $\emptyset$ & {a} & $\{\tsid_2\}$ \\
};
\node[version node, fit=(locxcells-1-1) (locxcells-2-1), fill=white, inner sep= 0cm, font=\Large] (locx-v0) {$0$};
\node[version node, fit=(locxcells-1-3) (locxcells-2-3), fill=white, inner sep=0cm, font=\Large] (locx-v1) {$1$};

%Location y
\path (locx.south) + (0,-1.5) node (locy) {$[\loc_y] \mapsto$};
\matrix(locycells) [version list, text width=7mm, anchor=west]
   at ([xshift=10pt]locy.east) {
 {a} & $\tsid_0$ & {a} & $\tsid_2$ \\
   {a} & $\emptyset$ & {a} & $\{\tsid_3\}$\\
};
\node[version node, fit=(locycells-1-1) (locycells-2-1), fill=white, inner sep= 0cm, font=\Large] (locy-v0) {$0$};
\node[version node, fit=(locycells-1-3) (locycells-2-3), fill=white, inner sep=0cm, font=\Large] (locy-v1) {$1$};



\path (locy.south) + (0,-1.5) node (locz) {$[\loc_z] \mapsto$};
\matrix(loczcells) [version list, text width=7mm, anchor=west]
   at ([xshift=10pt]locz.east) {
 {a} & $\tsid_0$ & {a} & $\tsid_3$ \\
   {a} & $\emptyset$ & {a} & $\emptyset$\\
};

\node[version node, fit=(loczcells-1-1) (loczcells-2-1), fill=white, inner sep= 0cm, font=\Large] (locz-v0) {$0$};
\node[version node, fit=(loczcells-1-3) (loczcells-2-3), fill=white, inner sep=0cm, font=\Large] (locz-v1) {$1$};

%blue view - I should  check whether I can use pgfkeys to just declare the list of locations, and then add the view automatically.
\draw[-, blue, very thick, rounded corners=10pt]
 ([xshift=-5pt, yshift=5pt]locx-v1.north east) node (tid1start) {} -- 
% ([xshift=-5pt, yshift=-5pt]locx-v1.south east) --
% ([xshift=-5pt, yshift=5pt]locy-v0.north east) -- 
 ([xshift=-5pt, yshift=-5pt]locy-v1.south east) --
 ([xshift=-5pt, yshift=5pt]locz-v0.north east) -- 
 ([xshift=-5pt, yshift=-5pt]locz-v0.south east);
 
% \path (tid1start) node[anchor=south, rectangle, fill=blue!20, draw=blue, font=\small, inner sep=1pt] {$\tid_3$};
\end{pgfonlayer}
\end{tikzpicture}
&
\begin{tikzpicture}[font=\large]

\begin{pgfonlayer}{foreground}
%Uncomment line below for help lines
%\draw[help lines] grid(5,4);

%Location x
\node(locx) at (1,3) {$[\loc_x] \mapsto$};

\matrix(locxcells) [version list, text width=7mm, anchor=west]
   at ([xshift=10pt]locx.east) {
 {a} & $\tsid_0$ & {a} & $\tsid_1$\\
  {a} & $\emptyset$ & {a} & $\{\tsid_2\}$ \\
};
\node[version node, fit=(locxcells-1-1) (locxcells-2-1), fill=white, inner sep= 0cm, font=\Large] (locx-v0) {$0$};
\node[version node, fit=(locxcells-1-3) (locxcells-2-3), fill=white, inner sep=0cm, font=\Large] (locx-v1) {$1$};

%Location y
\path (locx.south) + (0,-1.5) node (locy) {$[\loc_y] \mapsto$};
\matrix(locycells) [version list, text width=7mm, anchor=west]
   at ([xshift=10pt]locy.east) {
 {a} & $\tsid_0$ & {a} & $\tsid_2$ \\
   {a} & $\emptyset$ & {a} & $\{\tsid_3\}$\\
};
\node[version node, fit=(locycells-1-1) (locycells-2-1), fill=white, inner sep= 0cm, font=\Large] (locy-v0) {$0$};
\node[version node, fit=(locycells-1-3) (locycells-2-3), fill=white, inner sep=0cm, font=\Large] (locy-v1) {$1$};



\path (locy.south) + (0,-1.5) node (locz) {$[\loc_z] \mapsto$};
\matrix(loczcells) [version list, text width=7mm, anchor=west]
   at ([xshift=10pt]locz.east) {
 {a} & $\tsid_0$ & {a} & $\tsid_3$ \\
   {a} & $\emptyset$ & {a} & $\emptyset$\\
};

\node[version node, fit=(loczcells-1-1) (loczcells-2-1), fill=white, inner sep= 0cm, font=\Large] (locz-v0) {$0$};
\node[version node, fit=(loczcells-1-3) (loczcells-2-3), fill=white, inner sep=0cm, font=\Large] (locz-v1) {$1$};

%blue view - I should  check whether I can use pgfkeys to just declare the list of locations, and then add the view automatically.
\draw[-, blue, very thick, rounded corners=10pt]
 ([xshift=-5pt, yshift=5pt]locx-v1.north east) node (tid1start) {} -- 
% ([xshift=-5pt, yshift=-5pt]locx-v1.south east) --
% ([xshift=-5pt, yshift=5pt]locy-v0.north east) -- 
% ([xshift=-5pt, yshift=-5pt]locy-v1.south east) --
% ([xshift=-5pt, yshift=5pt]locz-v0.north east) -- 
 ([xshift=-5pt, yshift=-5pt]locz-v1.south east);
 
% \path (tid1start) node[anchor=south, rectangle, fill=blue!20, draw=blue, font=\small, inner sep=1pt] {$\tid_3$};
\end{pgfonlayer}
\end{tikzpicture}\\
{\small(c)} & {\small(d)}\\
\hline
\end{tabular}
\caption{Building a causally consistent view.}
\label{fig:cc.view}
\end{figure}

\begin{example}
\ac{Note to self: I got this example and the definition wrong several times before getting them
right. Which means that inductive definition of causally dependent views is not really that 
intuitive after all...}
Consider the history heap $\hh$, depicted in Figure \ref{fig:cc.view}(a) 
- ignore for the moment the view in the Figure.
%\begin{center}
%\begin{tikzpicture}[font=\large]
%
%\begin{pgfonlayer}{foreground}
%%Uncomment line below for help lines
%%\draw[help lines] grid(5,4);
%
%%Location x
%\node(locx) at (1,3) {$[\loc_x] \mapsto$};
%
%\matrix(locxcells) [version list, text width=7mm, anchor=west]
%   at ([xshift=10pt]locx.east) {
% {a} & $\tsid_0$ & {a} & $\tsid_1$\\
%  {a} & $\emptyset$ & {a} & $\{\tsid_2\}$ \\
%};
%\node[version node, fit=(locxcells-1-1) (locxcells-2-1), fill=white, inner sep= 0cm, font=\Large] (locx-v0) {$0$};
%\node[version node, fit=(locxcells-1-3) (locxcells-2-3), fill=white, inner sep=0cm, font=\Large] (locx-v-1) {$1$};
%%Location y
%\path (locx.south) + (0,-1.5) node (locy) {$[\loc_y] \mapsto$};
%\matrix(locycells) [version list, text width=7mm, anchor=west]
%   at ([xshift=10pt]locy.east) {
% {a} & $\tsid_0$ & {a} & $\tsid_2$ \\
%   {a} & $\emptyset$ & {a} & $\{\tsid_3\}$\\
%};
%\node[version node, fit=(locycells-1-1) (locycells-2-1), fill=white, inner sep= 0cm, font=\Large] (locy-v0) {$0$};
%\node[version node, fit=(locycells-1-3) (locycells-2-3), fill=white, inner sep=0cm, font=\Large] (locy-v1) {$1$};
%% \draw[-, red, very thick, rounded corners] ([xshift=-5pt, yshift=5pt]locx-v1.north east) |- 
%%  ($([xshift=-5pt,yshift=-5pt]locx-v1.south east)!.5!([xshift=-5pt, yshift=5pt]locy-v0.north east)$) -| ([xshift=-5pt, yshift=5pt]locy-v0.south east);
%
%\path (locy.south) + (0,-1.5) node (locy) {$[\loc_z] \mapsto$};
%\matrix(loczcells) [version list, text width=7mm, anchor=west]
%   at ([xshift=10pt]locy.east) {
% {a} & $\tsid_0$ & {a} & $\tsid_3$ \\
%   {a} & $\emptyset$ & {a} & $\emptyset$\\
%};
%\node[version node, fit=(loczcells-1-1) (loczcells-2-1), fill=white, inner sep= 0cm, font=\Large] (locz-v0) {$0$};
%\node[version node, fit=(loczcells-1-3) (loczcells-2-3), fill=white, inner sep=0cm, font=\Large] (locz-v1) {$1$};
%
%%blue view - I should  check whether I can use pgfkeys to just declare the list of locations, and then add the view automatically.
%%\draw[-, blue, very thick, rounded corners=10pt]
%% ([xshift=-5pt, yshift=5pt]locx-v0.north east) node (tid1start) {} -- 
%% ([xshift=-5pt, yshift=-5pt]locx-v0.south east) --
%% ([xshift=-5pt, yshift=5pt]locy-v1.north east) -- 
%% ([xshift=-5pt, yshift=-5pt]locy-v1.south east);
% 
%% \path (tid1start) node[anchor=south, rectangle, fill=blue!20, draw=blue, font=\small, inner sep=1pt] {$\tid_3$};
%\end{pgfonlayer}
%\end{tikzpicture}
%\end{center}
We want to find a view $V$ that is up-to-date with the update of location $[\loc_z]$; 
that is, $V[\loc_z] = 1$, and $\hh([\loc_z])(V[\loc_z]) = (1, \tsid_3, \emptyset)$.
Furthermore, we want the view $V$ to be causally consistent. 

We construct such a view incrementally, as outlined below. 
Note that for the history heapp $\hh$, we have that 
$\hh([\loc_z])(1) \xrightarrow{\mathsf{ddep}} \hh([\loc_x])(1)$, 
and $\hh([\loc_y])(1) \xrightarrow{\mathsf{ddep}} \hh([\loc_x])(1)$.


We start from the initial view $V_0$, $V_0([\loc_{\_}]) = 0$. This view is 
depicted in Figure \ref{fig:cc.view}(a), and it is 
causally consistent by definition. However, it does not include the update to location $[\loc_z]$: in
 fact, $V_0([\loc_z]) = 0$. As a first try, one could immediately consider a view $V'$ 
where the value for location $[\loc_z]$ is updated to $1$, 
that is $V' = V_0[ [\loc_z] \mapsto 1]$. However, in this case 
the view $V'$ is not causally consistent. In fact, 
we have that $\hh([\loc_z])(V'([\loc_z])) = (1, \tsid_3, \emptyset) \xrightarrow{\mathsf{ddep}} 
(1, \tsid_2, \{\tsid_3\}) = \hh([\loc_y])(1)$, but $V'([\loc_y]) = 0 < 1$. 
That is, the version of $[\loc_z]$ observed by $V'$ directly depends on a version 
of $\loc_y$ that is not observed by $V'$.
%
%the version observed 
%by $V'$ for location $[\loc_z]$ is $\hh([\loc_z])(V'([\loc_z]) = 
%\{1, \tsid_3, \emptyset\}$, meaning that the version has been written 
%by transaction $\tsid_3$. On the other hand, we also have 
%that $V'([\loc_y]) = 0$, while $\hh([\loc_{y}])(1) = (1, \tsid_2, 
%\{\tsid_3\}$. That is, the version observed by $V'$ for $[\loc_z]$ 
%directly depends from the version $\hh([\loc_y])(1)$, but the latter 
%is not observed by $V'$; in fact, $V'([\loc_y]) = 1$. 

As a second attempt, one could then try to update the the view of location 
$[\loc_y]$ to value $1$, resulting in the value $V'' = V_0[ [\loc_y] \mapsto 1]$, 
but similarly we would find that the version 
$\hh([\loc_y])(1) = (\_, \tsid_2, \_)$ directly depends from the version 
$\hh([\loc_x])(1) = (\_, \_, \{\tsid_2\})$, and $V''([\loc_x]) = 0 < 1$. 

Finally, we can update the view $V_0$ to include the update of $[\loc_x]$: 
this results in the view $V_1 = V_0[ [\loc_x] \mapsto 1$. Because $\hh([\loc_x])(1) 
= (\_, \tsid_1, \_)$, and there is no version $\nu$ in $\hh$ on which 
$\hh([\loc_x])(1)$ directly depends - i.e. such that  
$\nu = (\_, \_, \_ \cup \{\tsid_1\})$ - we can conclude that $V_1$ is causally 
consistent. The view $V_1$ is depicted in Figure \ref{fig:cc.view}(b).
%We can now proceed to update the view of location $[\loc_y]$ in $T_1$. 
%Because 

%\begin{center}
%\begin{tikzpicture}[font=\large]
%
%\begin{pgfonlayer}{foreground}
%%Uncomment line below for help lines
%%\draw[help lines] grid(5,4);
%
%%Location x
%\node(locx) at (1,3) {$[\loc_x] \mapsto$};
%
%\matrix(locxcells) [version list, text width=7mm, anchor=west]
%   at ([xshift=10pt]locx.east) {
% {a} & $\tsid_0$ & {a} & $\tsid_1$\\
%  {a} & $\emptyset$ & {a} & $\{\tsid_2\}$ \\
%};
%\node[version node, fit=(locxcells-1-1) (locxcells-2-1), fill=white, inner sep= 0cm, font=\Large] (locx-v0) {$0$};
%\node[version node, fit=(locxcells-1-3) (locxcells-2-3), fill=white, inner sep=0cm, font=\Large] (locx-v1) {$1$};
%
%%Location y
%\path (locx.south) + (0,-1.5) node (locy) {$[\loc_y] \mapsto$};
%\matrix(locycells) [version list, text width=7mm, anchor=west]
%   at ([xshift=10pt]locy.east) {
% {a} & $\tsid_0$ & {a} & $\tsid_2$ \\
%   {a} & $\emptyset$ & {a} & $\{\tsid_3\}$\\
%};
%\node[version node, fit=(locycells-1-1) (locycells-2-1), fill=white, inner sep= 0cm, font=\Large] (locy-v0) {$0$};
%\node[version node, fit=(locycells-1-3) (locycells-2-3), fill=white, inner sep=0cm, font=\Large] (locy-v1) {$1$};
%
%
%
%\path (locy.south) + (0,-1.5) node (locz) {$[\loc_z] \mapsto$};
%\matrix(loczcells) [version list, text width=7mm, anchor=west]
%   at ([xshift=10pt]locz.east) {
% {a} & $\tsid_0$ & {a} & $\tsid_3$ \\
%   {a} & $\emptyset$ & {a} & $\emptyset$\\
%};
%
%\node[version node, fit=(loczcells-1-1) (loczcells-2-1), fill=white, inner sep= 0cm, font=\Large] (locz-v0) {$0$};
%\node[version node, fit=(loczcells-1-3) (loczcells-2-3), fill=white, inner sep=0cm, font=\Large] (locz-v1) {$1$};
%
%%blue view - I should  check whether I can use pgfkeys to just declare the list of locations, and then add the view automatically.
%\draw[-, blue, very thick, rounded corners=10pt]
% ([xshift=-5pt, yshift=5pt]locx-v1.north east) node (tid1start) {} -- 
% ([xshift=-5pt, yshift=-5pt]locx-v1.south east) --
% ([xshift=-5pt, yshift=5pt]locy-v0.north east) -- 
%% ([xshift=-5pt, yshift=-5pt]locy-v0.south east) --
%% ([xshift=-5pt, yshift=5pt]locz-v0.north east) -- 
% ([xshift=-5pt, yshift=-5pt]locz-v0.south east);
% 
%% \path (tid1start) node[anchor=south, rectangle, fill=blue!20, draw=blue, font=\small, inner sep=1pt] {$\tid_3$};
%\end{pgfonlayer}
%\end{tikzpicture}
%\end{center}
We can now update the view $V_1$ to include the version $\hh([\loc_y])(1)$, 
resulting in the view $V_2 := V_1[ [\loc_y] \mapsto 1]$, depicted in Figure \ref{fig:cc.view}(c). 
Because the view $V_1$ is 
causally consistent, and because all the versions on which $\hh([\loc_y])(V_2([\loc_y]))$ directly depends
are observed by $V_1$, then $V_2$ is also causally consistent. Similarly, we can define 
the view $V_3 := V_2[[\loc_z] \mapsto 1]$, depicted in Figure \ref{fig:cc.view}(d), 
and prove that it casually consistent. 

%\begin{center}
%\begin{tikzpicture}[font=\large]
%
%\begin{pgfonlayer}{foreground}
%%Uncomment line below for help lines
%%\draw[help lines] grid(5,4);
%
%%Location x
%\node(locx) at (1,3) {$[\loc_x] \mapsto$};
%
%\matrix(locxcells) [version list, text width=7mm, anchor=west]
%   at ([xshift=10pt]locx.east) {
% {a} & $\tsid_0$ & {a} & $\tsid_1$\\
%  {a} & $\emptyset$ & {a} & $\{\tsid_2\}$ \\
%};
%\node[version node, fit=(locxcells-1-1) (locxcells-2-1), fill=white, inner sep= 0cm, font=\Large] (locx-v0) {$0$};
%\node[version node, fit=(locxcells-1-3) (locxcells-2-3), fill=white, inner sep=0cm, font=\Large] (locx-v1) {$1$};
%
%%Location y
%\path (locx.south) + (0,-1.5) node (locy) {$[\loc_y] \mapsto$};
%\matrix(locycells) [version list, text width=7mm, anchor=west]
%   at ([xshift=10pt]locy.east) {
% {a} & $\tsid_0$ & {a} & $\tsid_2$ \\
%   {a} & $\emptyset$ & {a} & $\{\tsid_3\}$\\
%};
%\node[version node, fit=(locycells-1-1) (locycells-2-1), fill=white, inner sep= 0cm, font=\Large] (locy-v0) {$0$};
%\node[version node, fit=(locycells-1-3) (locycells-2-3), fill=white, inner sep=0cm, font=\Large] (locy-v1) {$1$};
%
%
%
%\path (locy.south) + (0,-1.5) node (locz) {$[\loc_z] \mapsto$};
%\matrix(loczcells) [version list, text width=7mm, anchor=west]
%   at ([xshift=10pt]locz.east) {
% {a} & $\tsid_0$ & {a} & $\tsid_3$ \\
%   {a} & $\emptyset$ & {a} & $\emptyset$\\
%};
%
%\node[version node, fit=(loczcells-1-1) (loczcells-2-1), fill=white, inner sep= 0cm, font=\Large] (locz-v0) {$0$};
%\node[version node, fit=(loczcells-1-3) (loczcells-2-3), fill=white, inner sep=0cm, font=\Large] (locz-v1) {$1$};
%
%%blue view - I should  check whether I can use pgfkeys to just declare the list of locations, and then add the view automatically.
%\draw[-, blue, very thick, rounded corners=10pt]
% ([xshift=-5pt, yshift=5pt]locx-v1.north east) node (tid1start) {} -- 
%% ([xshift=-5pt, yshift=-5pt]locx-v1.south east) --
%% ([xshift=-5pt, yshift=5pt]locy-v0.north east) -- 
% ([xshift=-5pt, yshift=-5pt]locy-v1.south east) --
% ([xshift=-5pt, yshift=5pt]locz-v0.north east) -- 
% ([xshift=-5pt, yshift=-5pt]locz-v0.south east);
% 
%% \path (tid1start) node[anchor=south, rectangle, fill=blue!20, draw=blue, font=\small, inner sep=1pt] {$\tid_3$};
%\end{pgfonlayer}
%\end{tikzpicture}
%\hfill
%\begin{tikzpicture}[font=\large]
%
%\begin{pgfonlayer}{foreground}
%%Uncomment line below for help lines
%%\draw[help lines] grid(5,4);
%
%%Location x
%\node(locx) at (1,3) {$[\loc_x] \mapsto$};
%
%\matrix(locxcells) [version list, text width=7mm, anchor=west]
%   at ([xshift=10pt]locx.east) {
% {a} & $\tsid_0$ & {a} & $\tsid_1$\\
%  {a} & $\emptyset$ & {a} & $\{\tsid_2\}$ \\
%};
%\node[version node, fit=(locxcells-1-1) (locxcells-2-1), fill=white, inner sep= 0cm, font=\Large] (locx-v0) {$0$};
%\node[version node, fit=(locxcells-1-3) (locxcells-2-3), fill=white, inner sep=0cm, font=\Large] (locx-v1) {$1$};
%
%%Location y
%\path (locx.south) + (0,-1.5) node (locy) {$[\loc_y] \mapsto$};
%\matrix(locycells) [version list, text width=7mm, anchor=west]
%   at ([xshift=10pt]locy.east) {
% {a} & $\tsid_0$ & {a} & $\tsid_2$ \\
%   {a} & $\emptyset$ & {a} & $\{\tsid_3\}$\\
%};
%\node[version node, fit=(locycells-1-1) (locycells-2-1), fill=white, inner sep= 0cm, font=\Large] (locy-v0) {$0$};
%\node[version node, fit=(locycells-1-3) (locycells-2-3), fill=white, inner sep=0cm, font=\Large] (locy-v1) {$1$};
%
%
%
%\path (locy.south) + (0,-1.5) node (locz) {$[\loc_z] \mapsto$};
%\matrix(loczcells) [version list, text width=7mm, anchor=west]
%   at ([xshift=10pt]locz.east) {
% {a} & $\tsid_0$ & {a} & $\tsid_3$ \\
%   {a} & $\emptyset$ & {a} & $\emptyset$\\
%};
%
%\node[version node, fit=(loczcells-1-1) (loczcells-2-1), fill=white, inner sep= 0cm, font=\Large] (locz-v0) {$0$};
%\node[version node, fit=(loczcells-1-3) (loczcells-2-3), fill=white, inner sep=0cm, font=\Large] (locz-v1) {$1$};
%
%%blue view - I should  check whether I can use pgfkeys to just declare the list of locations, and then add the view automatically.
%\draw[-, blue, very thick, rounded corners=10pt]
% ([xshift=-5pt, yshift=5pt]locx-v1.north east) node (tid1start) {} -- 
%% ([xshift=-5pt, yshift=-5pt]locx-v1.south east) --
%% ([xshift=-5pt, yshift=5pt]locy-v0.north east) -- 
%% ([xshift=-5pt, yshift=-5pt]locy-v1.south east) --
%% ([xshift=-5pt, yshift=5pt]locz-v0.north east) -- 
% ([xshift=-5pt, yshift=-5pt]locz-v1.south east);
% 
%% \path (tid1start) node[anchor=south, rectangle, fill=blue!20, draw=blue, font=\small, inner sep=1pt] {$\tid_3$};
%\end{pgfonlayer}
%\end{tikzpicture}
%\end{center}
\end{example}

Causal consistency ensures that transactions can only observe 
a causally consistent state of the database, i.e. they can only run 
only by threads with a causally consistent view of the history heap.

\begin{definition}
$(\hh, V) \triangleright_{\mathsf{CC}} \mathcal{O} : V'$ if and only if 
$(\hh, V) \triangleright_{\mathsf{RA}} \mathcal{O}: V'$, 
and $V \in \mathsf{CCViews}(\hh)$.
\end{definition}
\ac{Note to self: here there is something subtle going on. We also need to ensure that dependencies 
caused by the information flow of the program are tracked down. For example, we could have a 
transaction returning the value of a location $[\loc_x]$, and then another transaction copy such a value 
into another location $[\loc_y]$. There is a notion of dependency between $[\loc_x], [\loc_y]$ that 
is not captured by the notion of \emph{directly depends}. On the other hand, the fact that 
stacks are thread local, and we have per-thread view monotonicity, should ensure that also program 
dependencies are preserved. A definitive proof that $\mathsf{CC}$ is equivalent to caual consistency 
specified either in terms of abstract executions or dependency graphs, would settle the argument.}

Note that, the view $V$ of $\tid_3$ in Figure \ref{fig:cc.view}(e) is not causally consistent. This is 
because $V([\loc_y]) = 1$, and $V([\loc_x]) = 0$. However, $\hh([\loc_y])(1)$ directly depends 
in $\hh([\loc_x])(1)$, which is not included in the view. 
In general, the only case in which an execution of $\prog_2$ causes transaction $\ptrans{\trans_3}$ 
to return value ${\Large \frownie{}}$ is when such a transaction is executed using a snapshot determined 
by a non-causally consistent view. There exists no execution of $\prog_2$ under $\mathsf{CC}$ that 
causes $\ptrans{\trans_3}$ to return value ${\Large \frownie{}}$.
%if we execute the program $\prog_2$, illustrated previously in this section, under $\mathsf{CCViews}$,
%it is not possible any more to have the transaction $\ptrans{\trans_3}$ return value ${\Large \frownie}$. 
%This is because, in order for $\ptrans{\trans_3}$ to return such a value, it must be executed in a 
%configuration such as the one of Figure \ref{fig:cc.view}(e) (only the view of $\tid_3$ is relevant here), 
%%state where the view thread of $\tid_3$ observes the update to location $[\loc_y]$, but not the 
%%update of $[\loc_x]$ from which the latter causally depends. 

\ac{In theory, I can do better, and require that I see only the causal dependencies 
of what I read. But at the end of the day, who cares?}

%\begin{definition}
%Let $\hh, V$ and $[\loc_x]$ be a history heap, a view, and a location, respectively. 
%Given two transactions $\tsid_1, \tsid_2$, we say that $\tsid_2$ write-read depends 
%on $\tsid_1$ via $[\loc_x]$, written $\tsid_1 \xrightarrow{\RF([\loc_x])_{\hh}} \tsid_2$, 
%if there exists an index $i = 0,\cdots, \llvert \hh, \rvert -1$ such that $\hh([\loc_n])(x) = 
%(\_, \tsid_1, \mathcal{T})$, and $\tsid_2 \in \mathcal{T}$.
%\end{definition}
%
%\begin{definition}
%Let $\hh$ be a history heap, and $V$ be a view.
%Let $[\loc_x]$ be a location, and let 
%$(\_, \tsid, \_) := \hh([\loc_{n}])(V([\loc_{x}]))$. 
%We say that $V$ respects causality for $[\loc_{x}]$ if, 
%whenever $\tsid' \xrightarrow{\RF([\loc_{x}])}_{\tsid}'$, 
%$\hh([\loc_x])(i)$ as follows: 
%\begin{enumerate}
%\end{definition}

\subsection{Update Atomic}
\begin{figure}
\begin{tabular}{|c|c|}
\hline
\begin{tikzpicture}[font=\large]

\begin{pgfonlayer}{foreground}
%Uncomment line below for help lines
%\draw[help lines] grid(5,4);

%Location x
\node(locx) at (1,3) {$[\loc_x] \mapsto$};

\matrix(locxcells) [version list, text width=7mm, anchor=west]
   at ([xshift=10pt]locx.east) {
 {a} & $T_0$ \\
  {a} & $\emptyset$ \\
};
\node[version node, fit=(locxcells-1-1) (locxcells-2-1), fill=white, inner sep= 0cm, font=\Large] (locx-v0) {$0$};

%Location y
\path (locx.south) + (0,-1.5) node (locf1) {$[\loc_{f_1}] \mapsto$};
\matrix(locf1cells) [version list, text width=7mm, anchor=west]
   at ([xshift=10pt]locf1.east) {
 {a} & $T_0$ \\
  {a} & $\emptyset$ \\
};
\node[version node, fit=(locf1cells-1-1) (locf1cells-2-1), fill=white, inner sep= 0cm, font=\Large] (locf1-v0) {$0$};

%Location y
\path (locf1.south) + (0,-1.5) node (locf2) {$[\loc_{f_2}] \mapsto$};
\matrix(locf2cells) [version list, text width=7mm, anchor=west]
   at ([xshift=10pt]locf2.east) {
 {a} & $T_0$ \\
  {a} & $\emptyset$ \\
};
\node[version node, fit=(locf2cells-1-1) (locf2cells-2-1), fill=white, inner sep= 0cm, font=\Large] (locf2-v0) {$0$};

% \draw[-, red, very thick, rounded corners] ([xshift=-5pt, yshift=5pt]locx-v1.north east) |- 
%  ($([xshift=-5pt,yshift=-5pt]locx-v1.south east)!.5!([xshift=-5pt, yshift=5pt]locy-v0.north east)$) -| ([xshift=-5pt, yshift=5pt]locy-v0.south east);

%blue view - I should  check whether I can use pgfkeys to just declare the list of locations, and then add the view automatically.
\draw[-, blue, very thick, rounded corners=10pt]
 ([xshift=-2pt, yshift=20pt]locx-v0.north east) node (tid1start) {} -- 
% ([xshift=-2pt, yshift=-5pt]locx-v0.south east) --
% ([xshift=-2pt, yshift=5pt]locf1-v0.north east) -- 
% ([xshift=-2pt, yshift=-5pt]locf1-v0.south east) --
% ([xshift=-2pt, yshift=5pt]locf2-v0.north east) -- 
 ([xshift=-2pt, yshift=-5pt]locf2-v0.south east);
 
 \path (tid1start) node[anchor=south, rectangle, fill=blue!20, draw=blue, font=\small, inner sep=1pt] {$\tid_3$};

%red view
\draw[-, red, very thick, rounded corners = 10pt]
 ([xshift=-5pt, yshift=5pt]locx-v0.north east) -- 
% ([xshift=-5pt, yshift=-5pt]locx-v0.south east) --
% ([xshift=-5pt, yshift=5pt]locf1-v0.north east) -- 
% ([xshift=-5pt, yshift=-5pt]locf1-v0.south east) --
% ([xshift=-5pt, yshift=5pt]locf2-v0.north east) -- 
 ([xshift=-5pt, yshift=-10pt]locf2-v0.south east) node (tid2start) {};
 
\path (tid2start) node[anchor=north, rectangle, fill=red!20, draw=red, font=\small, inner sep=1pt] {$\tid_2$};
 
 %green view
\draw[-, DarkGreen, very thick, rounded corners = 10pt]
 ([xshift=-16pt, yshift=8pt]locx-v0.north east) node (tid3start) {}-- 
% ([xshift=-16pt, yshift=-5pt]locx-v0.south east) --
% ([xshift=-16pt, yshift=5pt]locf1-v0.north east) -- 
% ([xshift=-16pt, yshift=-5pt]locf1-v0.south east) --
% ([xshift=-16pt, yshift=5pt]locf2-v0.north east) -- 
 ([xshift=-16pt, yshift=-5pt]locf2-v0.south east);
 
 \path (tid3start) node[anchor=south, rectangle, fill=DarkGreen!20, draw=DarkGreen, font=\small, inner sep=1pt] {$\tid_1$};

\end{pgfonlayer}
\end{tikzpicture}
%
&
%
\begin{tikzpicture}[font=\large]

\begin{pgfonlayer}{foreground}
%Uncomment line below for help lines
%\draw[help lines] grid(5,4);

%Location x
\node(locx) at (1,3) {$[\loc_x] \mapsto$};

\matrix(locxcells) [version list, text width=7mm, anchor=west]
   at ([xshift=10pt]locx.east) {
 {a} & $\tsid_0$ & {a} & $\tsid_1$\\
  {a} & $\{\tsid_1\}$ & {a} & $\emptyset$\\
};
\node[version node, fit=(locxcells-1-1) (locxcells-2-1), fill=white, inner sep= 0cm, font=\Large] (locx-v0) {$0$};
\node[version node, fit=(locxcells-1-3) (locxcells-2-3), fill=white, inner sep= 0cm, font=\Large] (locx-v1) {$1$};

%Location y
\path (locx.south) + (0,-1.5) node (locf1) {$[\loc_{f_1}] \mapsto$};
\matrix(locf1cells) [version list, text width=7mm, anchor=west]
   at ([xshift=10pt]locf1.east) {
 {a} & $\tsid_0$ & {a} & $\tsid_1$\\
  {a} & $\{\tsid_1\}$ & {a} & $\emptyset$\\
};
\node[version node, fit=(locf1cells-1-1) (locf1cells-2-1), fill=white, inner sep= 0cm, font=\Large] (locf1-v0) {$0$};
\node[version node, fit=(locf1cells-1-3) (locf1cells-2-3), fill=white, inner sep= 0cm, font=\Large] (locf1-v1) {$1$};

%Location y
\path (locf1.south) + (0,-1.5) node (locf2) {$[\loc_{f_2}] \mapsto$};
\matrix(locf2cells) [version list, text width=7mm, anchor=west]
   at ([xshift=10pt]locf2.east) {
 {a} & $T_0$ \\
  {a} & $\emptyset$ \\
};
\node[version node, fit=(locf2cells-1-1) (locf2cells-2-1), fill=white, inner sep= 0cm, font=\Large] (locf2-v0) {$0$};

% \draw[-, red, very thick, rounded corners] ([xshift=-5pt, yshift=5pt]locx-v1.north east) |- 
%  ($([xshift=-5pt,yshift=-5pt]locx-v1.south east)!.5!([xshift=-5pt, yshift=5pt]locy-v0.north east)$) -| ([xshift=-5pt, yshift=5pt]locy-v0.south east);

%blue view - I should  check whether I can use pgfkeys to just declare the list of locations, and then add the view automatically.
\draw[-, blue, very thick, rounded corners=10pt]
 ([xshift=-2pt, yshift=20pt]locx-v0.north east) node (tid1start) {} -- 
% ([xshift=-2pt, yshift=-5pt]locx-v0.south east) --
% ([xshift=-2pt, yshift=5pt]locf1-v0.north east) -- 
% ([xshift=-2pt, yshift=-5pt]locf1-v0.south east) --
% ([xshift=-2pt, yshift=5pt]locf2-v0.north east) -- 
 ([xshift=-2pt, yshift=-5pt]locf2-v0.south east);
 
 \path (tid1start) node[anchor=south, rectangle, fill=blue!20, draw=blue, font=\small, inner sep=1pt] {$\tid_3$};

%red view
\draw[-, red, very thick, rounded corners = 10pt]
 ([xshift=-5pt, yshift=5pt]locx-v0.north east) -- 
% ([xshift=-5pt, yshift=-5pt]locx-v0.south east) --
% ([xshift=-5pt, yshift=5pt]locf1-v0.north east) -- 
% ([xshift=-5pt, yshift=-5pt]locf1-v0.south east) --
% ([xshift=-5pt, yshift=5pt]locf2-v0.north east) -- 
 ([xshift=-5pt, yshift=-10pt]locf2-v0.south east) node (tid2start) {};
 
\path (tid2start) node[anchor=north, rectangle, fill=red!20, draw=red, font=\small, inner sep=1pt] {$\tid_2$};
 
 %green view
\draw[-, DarkGreen, very thick, rounded corners = 10pt]
 ([xshift=-16pt, yshift=8pt]locx-v1.north east) node (tid3start) {}-- 
% ([xshift=-16pt, yshift=-5pt]locx-v0.south east) --
% ([xshift=-16pt, yshift=5pt]locf1-v0.north east) -- 
 ([xshift=-16pt, yshift=-5pt]locf1-v1.south east) --
 ([xshift=-16pt, yshift=5pt]locf2-v0.north east) -- 
 ([xshift=-16pt, yshift=-5pt]locf2-v0.south east);
 
 \path (tid3start) node[anchor=south, rectangle, fill=DarkGreen!20, draw=DarkGreen, font=\small, inner sep=1pt] {$\tid_1$};

\end{pgfonlayer}
\end{tikzpicture}
\\
{\small (a)} & {\small (b)}\\
\hline
\begin{tikzpicture}[font=\large]

\begin{pgfonlayer}{foreground}
%Uncomment line below for help lines
%\draw[help lines] grid(5,4);

%Location x
\node(locx) at (1,3) {$[\loc_x] \mapsto$};

\matrix(locxcells) [version list, column 2/.style = {text width=14mm}, text width=7mm, anchor=west]
   at ([xshift=10pt]locx.east) {
 {a} & $\tsid_0$ & {a} & $\tsid_1$ & {a} & $\tsid_2$\\
  {a} & $\{\tsid_1, \tsid_2\}$ & {a} & $\emptyset$ & {a} & $\emptyset$\\
};
\node[version node, fit=(locxcells-1-1) (locxcells-2-1), fill=white, inner sep= 0cm, font=\Large] (locx-v0) {$0$};
\node[version node, fit=(locxcells-1-3) (locxcells-2-3), fill=white, inner sep= 0cm, font=\Large] (locx-v1) {$1$};
\node[version node, fit=(locxcells-1-5) (locxcells-2-5), fill=white, inner sep= 0cm, font=\Large] (locx-v2) {$1$};

%Location y
\path (locx.south) + (0,-1.5) node (locf1) {$[\loc_{f_1}] \mapsto$};
\matrix(locf1cells) [version list, text width=7mm, anchor=west]
   at ([xshift=10pt]locf1.east) {
 {a} & $\tsid_0$ & {a} & $\tsid_1$\\
  {a} & $\{\tsid_1\}$ & {a} & $\emptyset$\\
};
\node[version node, fit=(locf1cells-1-1) (locf1cells-2-1), fill=white, inner sep= 0cm, font=\Large] (locf1-v0) {$0$};
\node[version node, fit=(locf1cells-1-3) (locf1cells-2-3), fill=white, inner sep= 0cm, font=\Large] (locf1-v1) {$1$};

%Location y
\path (locf1.south) + (0,-1.5) node (locf2) {$[\loc_{f_2}] \mapsto$};
\matrix(locf2cells) [version list, text width=7mm, anchor=west]
   at ([xshift=10pt]locf2.east) {
 {a} & $\tsid_0$ & {a} & $\tsid_2$\\
  {a} & $\{\tsid_2\}$ & {a} & $\emptyset$ \\
};
\node[version node, fit=(locf2cells-1-1) (locf2cells-2-1), fill=white, inner sep= 0cm, font=\Large] (locf2-v0) {$0$};
\node[version node, fit=(locf2cells-1-3) (locf2cells-2-3), fill=white, inner sep= 0cm, font=\Large] (locf2-v1) {$1$};

% \draw[-, red, very thick, rounded corners] ([xshift=-5pt, yshift=5pt]locx-v1.north east) |- 
%  ($([xshift=-5pt,yshift=-5pt]locx-v1.south east)!.5!([xshift=-5pt, yshift=5pt]locy-v0.north east)$) -| ([xshift=-5pt, yshift=5pt]locy-v0.south east);

%blue view - I should  check whether I can use pgfkeys to just declare the list of locations, and then add the view automatically.
\draw[-, blue, very thick, rounded corners=10pt]
 ([xshift=-2pt, yshift=20pt]locx-v0.north east) node (tid1start) {} -- 
% ([xshift=-2pt, yshift=-5pt]locx-v0.south east) --
% ([xshift=-2pt, yshift=5pt]locf1-v0.north east) -- 
% ([xshift=-2pt, yshift=-5pt]locf1-v0.south east) --
% ([xshift=-2pt, yshift=5pt]locf2-v0.north east) -- 
 ([xshift=-2pt, yshift=-5pt]locf2-v0.south east);
 
 \path (tid1start) node[anchor=south, rectangle, fill=blue!20, draw=blue, font=\small, inner sep=1pt] {$\tid_3$};

%red view
\draw[-, red, very thick, rounded corners = 10pt]
 ([xshift=-5pt, yshift=5pt]locx-v2.north east) -- 
 ([xshift=-5pt, yshift=-5pt]locx-v2.south east) --
 ([xshift=-5pt, yshift=5pt]locf1-v0.north east) -- 
 ([xshift=-5pt, yshift=-5pt]locf1-v0.south east) --
 ([xshift=-5pt, yshift=5pt]locf2-v1.north east) -- 
 ([xshift=-5pt, yshift=-10pt]locf2-v1.south east) node (tid2start) {};
 
\path (tid2start) node[anchor=north, rectangle, fill=red!20, draw=red, font=\small, inner sep=1pt] {$\tid_2$};
 
 %green view
\draw[-, DarkGreen, very thick, rounded corners = 10pt]
 ([xshift=-16pt, yshift=8pt]locx-v1.north east) node (tid3start) {}-- 
 ([xshift=-16pt, yshift=-5pt]locx-v1.south east) --
 ([xshift=-16pt, yshift=5pt]locf1-v1.north east) -- 
 ([xshift=-16pt, yshift=-5pt]locf1-v1.south east) --
 ([xshift=-16pt, yshift=5pt]locf2-v0.north east) -- 
 ([xshift=-16pt, yshift=-5pt]locf2-v0.south east);
 
 \path (tid3start) node[anchor=south, rectangle, fill=DarkGreen!20, draw=DarkGreen, font=\small, inner sep=1pt] {$\tid_1$};

\end{pgfonlayer}
\end{tikzpicture}
%
&
%
\begin{tikzpicture}[font=\large]

\begin{pgfonlayer}{foreground}
%Uncomment line below for help lines
%\draw[help lines] grid(5,4);

%Location x
\node(locx) at (1,3) {$[\loc_x] \mapsto$};

\matrix(locxcells) [version list, column 2/.style = {text width=14mm}, text width=7mm, anchor=west]
   at ([xshift=10pt]locx.east) {
 {a} & $\tsid_0$ & {a} & $\tsid_1$ & {a} & $\tsid_2$\\
  {a} & $\{\tsid_1, \tsid_2\}$ & {a} & $\emptyset$ & {a} & $\emptyset$\\
};
\node[version node, fit=(locxcells-1-1) (locxcells-2-1), fill=white, inner sep= 0cm, font=\Large] (locx-v0) {$0$};
\node[version node, fit=(locxcells-1-3) (locxcells-2-3), fill=white, inner sep= 0cm, font=\Large] (locx-v1) {$1$};
\node[version node, fit=(locxcells-1-5) (locxcells-2-5), fill=white, inner sep= 0cm, font=\Large] (locx-v2) {$1$};

%Location y
\path (locx.south) + (0,-1.5) node (locf1) {$[\loc_{f_1}] \mapsto$};
\matrix(locf1cells) [version list, text width=7mm, anchor=west]
   at ([xshift=10pt]locf1.east) {
 {a} & $\tsid_0$ & {a} & $\tsid_1$\\
  {a} & $\{\tsid_1\}$ & {a} & $\emptyset$\\
};
\node[version node, fit=(locf1cells-1-1) (locf1cells-2-1), fill=white, inner sep= 0cm, font=\Large] (locf1-v0) {$0$};
\node[version node, fit=(locf1cells-1-3) (locf1cells-2-3), fill=white, inner sep= 0cm, font=\Large] (locf1-v1) {$1$};

%Location y
\path (locf1.south) + (0,-1.5) node (locf2) {$[\loc_{f_2}] \mapsto$};
\matrix(locf2cells) [version list, text width=7mm, anchor=west]
   at ([xshift=10pt]locf2.east) {
 {a} & $\tsid_0$ & {a} & $\tsid_2$\\
  {a} & $\{\tsid_2\}$ & {a} & $\emptyset$ \\
};
\node[version node, fit=(locf2cells-1-1) (locf2cells-2-1), fill=white, inner sep= 0cm, font=\Large] (locf2-v0) {$0$};
\node[version node, fit=(locf2cells-1-3) (locf2cells-2-3), fill=white, inner sep= 0cm, font=\Large] (locf2-v1) {$1$};

% \draw[-, red, very thick, rounded corners] ([xshift=-5pt, yshift=5pt]locx-v1.north east) |- 
%  ($([xshift=-5pt,yshift=-5pt]locx-v1.south east)!.5!([xshift=-5pt, yshift=5pt]locy-v0.north east)$) -| ([xshift=-5pt, yshift=5pt]locy-v0.south east);

%blue view - I should  check whether I can use pgfkeys to just declare the list of locations, and then add the view automatically.
\draw[-, blue, very thick, rounded corners=10pt]
 ([xshift=-2pt, yshift=20pt]locx-v2.north east) node (tid1start) {} -- 
 ([xshift=-2pt, yshift=-7pt]locx-v2.south east) --
 ([xshift=-2pt, yshift=3pt]locf1-v1.north east) -- 
 ([xshift=-2pt, yshift=-5pt]locf1-v1.south east) --
 ([xshift=-2pt, yshift=5pt]locf2-v1.north east) -- 
 ([xshift=-2pt, yshift=-5pt]locf2-v1.south east);
 
 \path (tid1start) node[anchor=south, rectangle, fill=blue!20, draw=blue, font=\small, inner sep=1pt] {$\tid_3$};

%red view
\draw[-, red, very thick, rounded corners = 10pt]
 ([xshift=-5pt, yshift=5pt]locx-v2.north east) -- 
 ([xshift=-5pt, yshift=-5pt]locx-v2.south east) --
 ([xshift=-5pt, yshift=5pt]locf1-v0.north east) -- 
 ([xshift=-5pt, yshift=-5pt]locf1-v0.south east) --
 ([xshift=-5pt, yshift=5pt]locf2-v1.north east) -- 
 ([xshift=-5pt, yshift=-10pt]locf2-v1.south east) node (tid2start) {};
 
\path (tid2start) node[anchor=north, rectangle, fill=red!20, draw=red, font=\small, inner sep=1pt] {$\tid_2$};
 
 %green view
\draw[-, DarkGreen, very thick, rounded corners = 10pt]
 ([xshift=-16pt, yshift=8pt]locx-v1.north east) node (tid3start) {}-- 
 ([xshift=-16pt, yshift=-5pt]locx-v1.south east) --
 ([xshift=-16pt, yshift=5pt]locf1-v1.north east) -- 
 ([xshift=-16pt, yshift=-5pt]locf1-v1.south east) --
 ([xshift=-16pt, yshift=5pt]locf2-v0.north east) -- 
 ([xshift=-16pt, yshift=-5pt]locf2-v0.south east);
 
 \path (tid3start) node[anchor=south, rectangle, fill=DarkGreen!20, draw=DarkGreen, font=\small, inner sep=1pt] {$\tid_1$};

\end{pgfonlayer}
\end{tikzpicture}\\
{\small (c)} & {\small (d)} \\
\hline
\begin{tikzpicture}[font=\large]

\begin{pgfonlayer}{foreground}
%Uncomment line below for help lines
%\draw[help lines] grid(5,4);

%Location x
\node(locx) at (1,3) {$[\loc_x] \mapsto$};

\matrix(locxcells) [version list, text width=7mm, anchor=west]
   at ([xshift=10pt]locx.east) {
 {a} & $\tsid_0$ & {a} & $\tsid_1$\\
  {a} & $\{\tsid_1\}$ & {a} & $\emptyset$\\
};
\node[version node, fit=(locxcells-1-1) (locxcells-2-1), fill=white, inner sep= 0cm, font=\Large] (locx-v0) {$0$};
\node[version node, fit=(locxcells-1-3) (locxcells-2-3), fill=white, inner sep= 0cm, font=\Large] (locx-v1) {$1$};

%Location y
\path (locx.south) + (0,-1.5) node (locf1) {$[\loc_{f_1}] \mapsto$};
\matrix(locf1cells) [version list, text width=7mm, anchor=west]
   at ([xshift=10pt]locf1.east) {
 {a} & $\tsid_0$ & {a} & $\tsid_1$\\
  {a} & $\{\tsid_1\}$ & {a} & $\emptyset$\\
};
\node[version node, fit=(locf1cells-1-1) (locf1cells-2-1), fill=white, inner sep= 0cm, font=\Large] (locf1-v0) {$0$};
\node[version node, fit=(locf1cells-1-3) (locf1cells-2-3), fill=white, inner sep= 0cm, font=\Large] (locf1-v1) {$1$};

%Location y
\path (locf1.south) + (0,-1.5) node (locf2) {$[\loc_{f_2}] \mapsto$};
\matrix(locf2cells) [version list, text width=7mm, anchor=west]
   at ([xshift=10pt]locf2.east) {
 {a} & $T_0$ \\
  {a} & $\emptyset$ \\
};
\node[version node, fit=(locf2cells-1-1) (locf2cells-2-1), fill=white, inner sep= 0cm, font=\Large] (locf2-v0) {$0$};

% \draw[-, red, very thick, rounded corners] ([xshift=-5pt, yshift=5pt]locx-v1.north east) |- 
%  ($([xshift=-5pt,yshift=-5pt]locx-v1.south east)!.5!([xshift=-5pt, yshift=5pt]locy-v0.north east)$) -| ([xshift=-5pt, yshift=5pt]locy-v0.south east);

%blue view - I should  check whether I can use pgfkeys to just declare the list of locations, and then add the view automatically.
\draw[-, blue, very thick, rounded corners=10pt]
 ([xshift=-2pt, yshift=20pt]locx-v0.north east) node (tid1start) {} -- 
% ([xshift=-2pt, yshift=-5pt]locx-v0.south east) --
% ([xshift=-2pt, yshift=5pt]locf1-v0.north east) -- 
% ([xshift=-2pt, yshift=-5pt]locf1-v0.south east) --
% ([xshift=-2pt, yshift=5pt]locf2-v0.north east) -- 
 ([xshift=-2pt, yshift=-5pt]locf2-v0.south east);
 
 \path (tid1start) node[anchor=south, rectangle, fill=blue!20, draw=blue, font=\small, inner sep=1pt] {$\tid_3$};

%red view
\draw[-, red, very thick, rounded corners = 10pt]
 ([xshift=-5pt, yshift=5pt]locx-v1.north east) -- 
% ([xshift=-5pt, yshift=-5pt]locx-v0.south east) --
% ([xshift=-5pt, yshift=5pt]locf1-v0.north east) -- 
 ([xshift=-5pt, yshift=-5pt]locf1-v1.south east) --
 ([xshift=-5pt, yshift=5pt]locf2-v0.north east) -- 
 ([xshift=-5pt, yshift=-10pt]locf2-v0.south east) node (tid2start) {};
 
\path (tid2start) node[anchor=north, rectangle, fill=red!20, draw=red, font=\small, inner sep=1pt] {$\tid_2$};
 
 %green view
\draw[-, DarkGreen, very thick, rounded corners = 10pt]
 ([xshift=-16pt, yshift=8pt]locx-v1.north east) node (tid3start) {}-- 
% ([xshift=-16pt, yshift=-5pt]locx-v0.south east) --
% ([xshift=-16pt, yshift=5pt]locf1-v0.north east) -- 
 ([xshift=-16pt, yshift=-5pt]locf1-v1.south east) --
 ([xshift=-16pt, yshift=5pt]locf2-v0.north east) -- 
 ([xshift=-16pt, yshift=-5pt]locf2-v0.south east);
 
 \path (tid3start) node[anchor=south, rectangle, fill=DarkGreen!20, draw=DarkGreen, font=\small, inner sep=1pt] {$\tid_1$};

\end{pgfonlayer}
\end{tikzpicture}
&
\begin{tikzpicture}[font=\large]

\begin{pgfonlayer}{foreground}
%Uncomment line below for help lines
%\draw[help lines] grid(5,4);

%Location x
\node(locx) at (1,3) {$[\loc_x] \mapsto$};

\matrix(locxcells) [version list, text width=7mm, anchor=west]
   at ([xshift=10pt]locx.east) {
 {a} & $\tsid_0$ & {a} & $\tsid_1$ & {a} & $\tsid_2$\\
  {a} & $\{\tsid_1\}$ & {a} & $\{\tsid_2\}$ & {a} & $\emptyset$\\
};
\node[version node, fit=(locxcells-1-1) (locxcells-2-1), fill=white, inner sep= 0cm, font=\Large] (locx-v0) {$0$};
\node[version node, fit=(locxcells-1-3) (locxcells-2-3), fill=white, inner sep= 0cm, font=\Large] (locx-v1) {$1$};
\node[version node, fit=(locxcells-1-5) (locxcells-2-5), fill=white, inner sep= 0cm, font=\Large] (locx-v2) {$2$};

%Location y
\path (locx.south) + (0,-1.5) node (locf1) {$[\loc_{f_1}] \mapsto$};
\matrix(locf1cells) [version list, text width=7mm, anchor=west]
   at ([xshift=10pt]locf1.east) {
 {a} & $\tsid_0$ & {a} & $\tsid_1$\\
  {a} & $\{\tsid_1\}$ & {a} & $\emptyset$\\
};
\node[version node, fit=(locf1cells-1-1) (locf1cells-2-1), fill=white, inner sep= 0cm, font=\Large] (locf1-v0) {$0$};
\node[version node, fit=(locf1cells-1-3) (locf1cells-2-3), fill=white, inner sep= 0cm, font=\Large] (locf1-v1) {$1$};

%Location y
\path (locf1.south) + (0,-1.5) node (locf2) {$[\loc_{f_2}] \mapsto$};
\matrix(locf2cells) [version list, text width=7mm, anchor=west]
   at ([xshift=10pt]locf2.east) {
 {a} & $T_0$ & {a} & $\tsid_2$\\
  {a} & $\emptyset$ & {a} & $\emptyset$ \\
};
\node[version node, fit=(locf2cells-1-1) (locf2cells-2-1), fill=white, inner sep= 0cm, font=\Large] (locf2-v0) {$0$};
\node[version node, fit=(locf2cells-1-3) (locf2cells-2-3), fill=white, inner sep= 0cm, font=\Large] (locf2-v1) {$1$};

% \draw[-, red, very thick, rounded corners] ([xshift=-5pt, yshift=5pt]locx-v1.north east) |- 
%  ($([xshift=-5pt,yshift=-5pt]locx-v1.south east)!.5!([xshift=-5pt, yshift=5pt]locy-v0.north east)$) -| ([xshift=-5pt, yshift=5pt]locy-v0.south east);

%blue view - I should  check whether I can use pgfkeys to just declare the list of locations, and then add the view automatically.
\draw[-, blue, very thick, rounded corners=10pt]
 ([xshift=-2pt, yshift=20pt]locx-v0.north east) node (tid1start) {} -- 
% ([xshift=-2pt, yshift=-5pt]locx-v0.south east) --
% ([xshift=-2pt, yshift=5pt]locf1-v0.north east) -- 
% ([xshift=-2pt, yshift=-5pt]locf1-v0.south east) --
% ([xshift=-2pt, yshift=5pt]locf2-v0.north east) -- 
 ([xshift=-2pt, yshift=-5pt]locf2-v0.south east);
 
 \path (tid1start) node[anchor=south, rectangle, fill=blue!20, draw=blue, font=\small, inner sep=1pt] {$\tid_3$};

%red view
\draw[-, red, very thick, rounded corners = 10pt]
 ([xshift=-5pt, yshift=5pt]locx-v2.north east) -- 
 ([xshift=-5pt, yshift=-5pt]locx-v2.south east) --
 ([xshift=-5pt, yshift=5pt]locf1-v1.north east) -- 
% ([xshift=-5pt, yshift=-5pt]locf1-v0.south east) --
% ([xshift=-5pt, yshift=5pt]locf2-v0.north east) -- 
 ([xshift=-5pt, yshift=-10pt]locf2-v1.south east) node (tid2start) {};
 
\path (tid2start) node[anchor=north, rectangle, fill=red!20, draw=red, font=\small, inner sep=1pt] {$\tid_2$};
 
 %green view
\draw[-, DarkGreen, very thick, rounded corners = 10pt]
 ([xshift=-16pt, yshift=8pt]locx-v1.north east) node (tid3start) {}-- 
% ([xshift=-16pt, yshift=-5pt]locx-v0.south east) --
% ([xshift=-16pt, yshift=5pt]locf1-v0.north east) -- 
 ([xshift=-16pt, yshift=-5pt]locf1-v1.south east) --
 ([xshift=-16pt, yshift=5pt]locf2-v0.north east) -- 
 ([xshift=-16pt, yshift=-5pt]locf2-v0.south east);
 
 \path (tid3start) node[anchor=south, rectangle, fill=DarkGreen!20, draw=DarkGreen, font=\small, inner sep=1pt] {$\tid_1$};

\end{pgfonlayer}
\end{tikzpicture}\\
{\small(e)} & {\small(f)}\\
\hline
\end{tabular}
\caption{History heaps obtained in a execution of $\prog_3$.}
\label{fig:cu.exec}
\end{figure}
\ac{This Consistency Model shows why the notion of consistent views must 
depend on the set of operations that need to be executed.}

The next consistency model that we consider is \emph{Update Atomic}. 
Although this model has not been implemented, it has been proposed in 
\cite{framework-concur} as a strengthening to Read Atomic to avoid 
conflicts. The informal specification of Consistent Update states that 
\textbf{(i)} transactions enjoy atomic visibility, and \textbf{(ii)} 
transactions writing to one same location cannot execute concurrently.
Update Atomic is also needed to specify more sophisticated consistency models, 
such as \emph{Parallel Snapshot Isolation} and \emph{Snapshot Isolation}.
\ac{Check: Nobi said he was interested in implementing Update Atomic 
at some point, maybe he ended up doing something.}

Programs executing under Update Atomic do not exhibit the \emph{lost update} 
anomaly: two or more transactions update one location (for example, by incrementing its value by $1$), 
but only the effects of one of them (for example, only one of the increments) will 
be observed by future transactions.

To illustrate how the lost update anomaly can arise in practice, 
consider the following program:
\[
    \prog_3 := \begin{session}
        \begin{array}{@{}c || c || c@{}}
            \begin{transaction}
               \pmutate{\loc_{f_1}}{1};\\
                  \pderef{\pvar{a}}{\loc_x};\\
            		  \pmutate{\loc_x}{a + 1};\\
              \end{transaction} & 
              \begin{transaction}
                  \pmutate{\loc_{f_2}}{1};\\
            		  \pderef{\pvar{a}}{\loc_x};\\
            		  \pmutate{\loc_x}{a + 1};\\
            	  \end{transaction} &
              \begin{transaction}
            		   \pderef{\pvar{a}}{\loc_{f_1}};\\
            		   \pderef{\pvar{b}}{\loc_{f_2}};\\
            		   \pderef{\pvar{c}}{\loc_{x}};\\
            		   \pifs{\pvar{a}=1 \wedge \pvar{b}=1 \wedge \pvar{c} = 1}\\
            				\;\;\;\;\passign{\retvar}{\Large \frownie{}}
            			\}
             \end{transaction}
        \end{array}
    \end{session}
 \]
\ac{Intuitive behaviour of the litmus test: two transactions concurrently increment $[\loc_x]$. 
 A third transaction observes that the first two transactions have been executed. 
 However, it only observes one of the two increments taking place.
 }
We consider an execution in which the transactions contained in the code of threads 
$\tid_1, \tid_2$ both execute w.r.t. the snapshot determined by the initial view. 
Initially, the configuration of the program coincides with the one given in 
Figure \ref{fig:cu.exec}(a). After executing the transaction of $\tid_1$, the resulting 
configuration is the one depicted in Figure \ref{fig:cu.exec}(b). Next, after $\tid_2$ executed 
its own transaction, we obtain the configuration in \ref{fig:cu.exec}(c). Note that 
in this configuration, both transactions $\tsid_1, \tsid_2$ read the initial version 
for location $[\loc_x]$. Finally, thread $\tid_3$ updates its view to include the most recent 
version for all the locations $[\loc_x], [\loc_{f_1}], [\loc_{f_2}]$. When executing the 
transaction in its code, all the locations will be found to have value $1$, and 
the return variable will be set to ${\Large \frownie{}}$.

The reason why the program $\prog_3$ exhibited the lost-update anomaly is 
that the transaction of $\tid_2$ executed using a snapshot obtained from 
a view which did not include the most up-to-date version for location $[\loc_x]$. 
However, the same transaction installed a new version for $[\loc_x]$. That is, 
it \emph{lost the update} of the version installed by the transaction executed 
by $\tid_1$. To forbid this anomaly, we require that if a transaction 
writes to some location $[\loc_x]$, then it must have been executed 
in a snapshot obtained from a view including the most recent version of 
$[\loc_x]$.

\begin{definition}
$(\hh, V) \triangleright_{\mathsf{UA}} \mathcal{O}: V'$ iff 
$(\hh, V) \triangleright_{\mathsf{RA}} \mathcal{O}: V'$, and 
for all locations $[\loc_x]$ such that $(\WR\; [\loc_x] : \_ \in \mathcal{O})$ 
then $V([\loc_{x}]) = \lvert \hh([\loc_x]) \rvert - 1$. 
\end{definition}

Note that in $prog_3$, under $\mathsf{UA}$, 
we cannot execute the transaction of $\tid_2$ starting from 
the configuration depicted in Figure \ref{fig:cu.exec}(b). This 
is because the view of $\tid_2$, in such a configuration, does 
not include the most recent version for location $[\loc_x]$. 
Instead, before executing its transaction $\tid_2$ must update 
its view to include the most recent version of $[\loc_x]$. The 
resulting configuration is depicted in Figure \ref{fig:cu.exec}(e). 
After executing its transaction using the snapshot obtained 
from the view of $\tid_2$ in this configuration, the resulting 
configuration is the one obtained in Figure \ref{fig:cu.exec}(f). 
There are now three different possible configurations in which 
$\tid_3$ can execute its transaction: 
\begin{itemize}
\item the initial one, in which case the value $0$ will be observed 
for the three locations $[\loc_x], [\loc_{f1}]$ and $[\loc_{f_2}]$. 
In this case the transaction will not return value ${\Large \frownie{}}$, 
\item one in which the view of $\tid_3$ for $[\loc_x]$ points to the 
version $(1, \tsid_1, \{\tsid_2\})$. Because of atomic visibility, 
it must also be the case that the same view for location $[\loc_{f_2}]$ 
does not include the update performed by $\tsid_2$ (otherwise, 
the view for $[\loc_x]$ should point to its most recent version, which 
was also installed by $\tsid_2$. Thus in this case the transaction executes 
w.r.t a snapshot where $\loc_{f_2} = 0$, and the value ${\Large \frownie{}}$ 
will not be returned, 
\item one in which the view of $\tid_3$ for $[\loc_x]$ points to 
its most recent version $(2, \tsid_2, \emptyset)$; also in this case, 
the value ${\Large \frownie{}}$ will not be returned from the transaction.
\end{itemize}


\subsection{Consistent Prefix} 
\begin{figure}
\begin{center}
\begin{tabular}{|@{}c|c@{}|}
\hline
\begin{tikzpicture}[font=\large]

\begin{pgfonlayer}{foreground}
%Uncomment line below for help lines
%\draw[help lines] grid(5,4);

%Location x
\node(locx) at (1,3) {$[\loc_x] \mapsto$};

\matrix(locxcells) [version list, text width=7mm, anchor=west]
   at ([xshift=10pt]locx.east) {
 {a} & $T_0$ \\
  {a} & $\emptyset$ \\
};
\node[version node, fit=(locxcells-1-1) (locxcells-2-1), fill=white, inner sep= 0cm, font=\Large] (locx-v0) {$0$};
%\node[version node, fit=(locxcells-1-3) (locxcells-2-3), fill=white, inner sep=0cm, font=\Large] (locx-v1) {$1$};

%Location y
\path (locx.south) + (0,-1.5) node (locy) {$[\loc_y] \mapsto$};
\matrix(locycells) [version list, text width=7mm, anchor=west]
   at ([xshift=10pt]locy.east) {
 {a} & $T_0$ \\
  {a} & $\emptyset$ \\
};
\node[version node, fit=(locycells-1-1) (locycells-2-1), fill=white, inner sep= 0cm, font=\Large] (locy-v0) {$0$};
%\node[version node, fit=(locycells-1-3) (locycells-2-3), fill=white, inner sep=0cm, font=\Large] (locy-v1) {$1$};

% \draw[-, red, very thick, rounded corners] ([xshift=-5pt, yshift=5pt]locx-v1.north east) |- 
%  ($([xshift=-5pt,yshift=-5pt]locx-v1.south east)!.5!([xshift=-5pt, yshift=5pt]locy-v0.north east)$) -| ([xshift=-5pt, yshift=5pt]locy-v0.south east);

%blue view - I should  check whether I can use pgfkeys to just declare the list of locations, and then add the view automatically.
\draw[-, blue, very thick, rounded corners=10pt]
 ([xshift=-3pt, yshift=20pt]locx-v0.north east) node (tid1start) {} -- 
% ([xshift=-2pt, yshift=-5pt]locx-v0.south east) --
% ([xshift=-2pt, yshift=5pt]locy-v0.north east) -- 
 ([xshift=-3pt, yshift=-5pt]locy-v0.south east);
 
 \path (tid1start) node[anchor=south, rectangle, fill=blue!20, draw=blue, font=\small, inner sep=1pt] {$\tid_1$};

%red view
\draw[-, red, very thick, rounded corners = 10pt]
 ([xshift=-16pt, yshift=5pt]locx-v0.north east) node (tid2start) {}-- 
% ([xshift=-8pt, yshift=-5pt]locx-v0.south east) --
% ([xshift=-8pt, yshift=5pt]locy-v0.north east) -- 
 ([xshift=-16pt, yshift=-5pt]locy-v0.south east) node {};
 
\path (tid2start) node[anchor=south, rectangle, fill=red!20, draw=red, font=\small, inner sep=1pt] {$\tid_2$};

%%Stack for threads tid_1 and tid_2
%
%\draw[-, dashed] let 
%   \p1 = ([xshift=0pt]locy.west),
%   \p2 = ([yshift=-5pt]locycells.south),
%   \p3 = ([xshift=10pt]locycells.east) in
%   (\x1, \y2) -- (\x3, \y2);
%   
%\matrix(stacks) [
%   matrix of nodes,
%   anchor=north, 
%   text=blue, 
%   font=\normalsize, 
%   row 1/.style = {text = blue}, 
%   row 2/.style = {text = red}, 
%   text width= 13mm ] 
%   at ([xshift=-10pt,yshift=-8pt]locycells.south) {
%   $\tid_1:$ & $\retvar = 0$\\
%   $\tid_2:$ & $\retvar = 0$\\
%   };
\end{pgfonlayer}
\end{tikzpicture} 
%
&
%
\begin{tikzpicture}[font=\large]

\begin{pgfonlayer}{foreground}
%Uncomment line below for help lines
%\draw[help lines] grid(5,4);

%Location x
\node(locx) at (1,3) {$[\loc_x] \mapsto$};

\matrix(locxcells) [version list, text width=7mm, anchor=west]
   at ([xshift=10pt]locx.east) {
 {a} & $T_0$ & {a} & $\tsid_1$\\
  {a} & $\emptyset$ & {a} & $\emptyset$ \\
};
\node[version node, fit=(locxcells-1-1) (locxcells-2-1), fill=white, inner sep= 0cm, font=\Large] (locx-v0) {$0$};
\node[version node, fit=(locxcells-1-3) (locxcells-2-3), fill=white, inner sep=0cm, font=\Large] (locx-v1) {$1$};

%Location y
\path (locx.south) + (0,-1.5) node (locy) {$[\loc_y] \mapsto$};
\matrix(locycells) [version list, text width=7mm, anchor=west]
   at ([xshift=10pt]locy.east) {
 {a} & $T_0$ \\
  {a} & $\emptyset$ \\
};
\node[version node, fit=(locycells-1-1) (locycells-2-1), fill=white, inner sep= 0cm, font=\Large] (locy-v0) {$0$};
%\node[version node, fit=(locycells-1-3) (locycells-2-3), fill=white, inner sep=0cm, font=\Large] (locy-v1) {$1$};

% \draw[-, red, very thick, rounded corners] ([xshift=-5pt, yshift=5pt]locx-v1.north east) |- 
%  ($([xshift=-5pt,yshift=-5pt]locx-v1.south east)!.5!([xshift=-5pt, yshift=5pt]locy-v0.north east)$) -| ([xshift=-5pt, yshift=5pt]locy-v0.south east);

%blue view - I should  check whether I can use pgfkeys to just declare the list of locations, and then add the view automatically.
\draw[-, blue, very thick, rounded corners=10pt]
 ([xshift=-3pt, yshift=20pt]locx-v1.north east) node (tid1start) {} -- 
 ([xshift=-3pt, yshift=-5pt]locx-v1.south east) --
 ([xshift=-3pt, yshift=5pt]locy-v0.north east) -- 
 ([xshift=-3pt, yshift=-5pt]locy-v0.south east);
 
 \path (tid1start) node[anchor=south, rectangle, fill=blue!20, draw=blue, font=\small, inner sep=1pt] {$\tid_1$};

%red view
\draw[-, red, very thick, rounded corners = 10pt]
 ([xshift=-16pt, yshift=5pt]locx-v0.north east) node (tid2start) {}-- 
% ([xshift=-8pt, yshift=-5pt]locx-v0.south east) --
% ([xshift=-8pt, yshift=5pt]locy-v0.north east) -- 
 ([xshift=-16pt, yshift=-5pt]locy-v0.south east) node {};
 
\path (tid2start) node[anchor=south, rectangle, fill=red!20, draw=red, font=\small, inner sep=1pt] {$\tid_2$};

%%Stack for threads tid_1 and tid_2
%
%\draw[-, dashed] let 
%   \p1 = ([xshift=0pt]locy.west),
%   \p2 = ([yshift=-5pt]locycells.south),
%   \p3 = ([xshift=10pt]locycells.east) in
%   (\x1, \y2) -- (\x3, \y2);
%   
%\matrix(stacks) [
%   matrix of nodes,
%   anchor=north, 
%   text=blue, 
%   font=\normalsize, 
%   row 1/.style = {text = blue}, 
%   row 2/.style = {text = red}, 
%   text width= 13mm ] 
%   at ([xshift=-10pt,yshift=-8pt]locycells.south) {
%   $\tid_1:$ & $\retvar = 0$\\
%   $\tid_2:$ & $\retvar = 0$\\
%   };
\end{pgfonlayer}
\end{tikzpicture}
\\
{\small(a)} & {\small(b)}\\
\hline
\begin{tikzpicture}[font=\large]

\begin{pgfonlayer}{foreground}
%Uncomment line below for help lines
%\draw[help lines] grid(5,4);

%Location x
\node(locx) at (1,3) {$[\loc_x] \mapsto$};

\matrix(locxcells) [version list, text width=7mm, anchor=west]
   at ([xshift=10pt]locx.east) {
 {a} & $T_0$ & {a} & $\tsid_1$\\
  {a} & $\emptyset$ & {a} & $\emptyset$ \\
};
\node[version node, fit=(locxcells-1-1) (locxcells-2-1), fill=white, inner sep= 0cm, font=\Large] (locx-v0) {$0$};
\node[version node, fit=(locxcells-1-3) (locxcells-2-3), fill=white, inner sep=0cm, font=\Large] (locx-v1) {$1$};

%Location y
\path (locx.south) + (0,-1.5) node (locy) {$[\loc_y] \mapsto$};
\matrix(locycells) [version list, text width=7mm, anchor=west]
   at ([xshift=10pt]locy.east) {
 {a} & $T_0$ & {a} & $\tsid_2$ \\
  {a} & $\emptyset$ & {a} & $\emptyset$\\
};
\node[version node, fit=(locycells-1-1) (locycells-2-1), fill=white, inner sep= 0cm, font=\Large] (locy-v0) {$0$};
\node[version node, fit=(locycells-1-3) (locycells-2-3), fill=white, inner sep=0cm, font=\Large] (locy-v1) {$1$};

% \draw[-, red, very thick, rounded corners] ([xshift=-5pt, yshift=5pt]locx-v1.north east) |- 
%  ($([xshift=-5pt,yshift=-5pt]locx-v1.south east)!.5!([xshift=-5pt, yshift=5pt]locy-v0.north east)$) -| ([xshift=-5pt, yshift=5pt]locy-v0.south east);

%blue view - I should  check whether I can use pgfkeys to just declare the list of locations, and then add the view automatically.
\draw[-, blue, very thick, rounded corners=10pt]
 ([xshift=-3pt, yshift=20pt]locx-v1.north east) node (tid1start) {} -- 
 ([xshift=-3pt, yshift=-5pt]locx-v1.south east) --
 ([xshift=-3pt, yshift=5pt]locy-v0.north east) -- 
 ([xshift=-3pt, yshift=-5pt]locy-v0.south east);
 
 \path (tid1start) node[anchor=south, rectangle, fill=blue!20, draw=blue, font=\small, inner sep=1pt] {$\tid_1$};

%red view
\draw[-, red, very thick, rounded corners = 10pt]
 ([xshift=-16pt, yshift=5pt]locx-v0.north east) node (tid2start) {}-- 
 ([xshift=-16pt, yshift=-5pt]locx-v0.south east) --
 ([xshift=-16pt, yshift=5pt]locy-v1.north east) -- 
 ([xshift=-16pt, yshift=-5pt]locy-v1.south east) node {};
 
\path (tid2start) node[anchor=south, rectangle, fill=red!20, draw=red, font=\small, inner sep=1pt] {$\tid_2$};

%%Stack for threads tid_1 and tid_2
%
%\draw[-, dashed] let 
%   \p1 = ([xshift=0pt]locy.west),
%   \p2 = ([yshift=-5pt]locycells.south),
%   \p3 = ([xshift=10pt]locycells.east) in
%   (\x1, \y2) -- (\x3, \y2);
%   
%\matrix(stacks) [
%   matrix of nodes,
%   anchor=north, 
%   text=blue, 
%   font=\normalsize, 
%   row 1/.style = {text = blue}, 
%   row 2/.style = {text = red}, 
%   text width= 13mm ] 
%   at ([xshift=-10pt,yshift=-8pt]locycells.south) {
%   $\tid_1:$ & $\retvar = 0$\\
%   $\tid_2:$ & $\retvar = 0$\\
%   };
\end{pgfonlayer}
\end{tikzpicture}
%
&
%
\begin{tikzpicture}[font=\large]

\begin{pgfonlayer}{foreground}
%Uncomment line below for help lines
%\draw[help lines] grid(5,4);

%Location x
\node(locx) at (1,3) {$[\loc_x] \mapsto$};

\matrix(locxcells) [version list, text width=7mm, anchor=west]
   at ([xshift=10pt]locx.east) {
 {a} & $T_0$ & {a} & $\tsid_1$\\
  {a} & $\{\tsid_4\}$ & {a} & $\emptyset$ \\
};
\node[version node, fit=(locxcells-1-1) (locxcells-2-1), fill=white, inner sep= 0cm, font=\Large] (locx-v0) {$0$};
\node[version node, fit=(locxcells-1-3) (locxcells-2-3), fill=white, inner sep=0cm, font=\Large] (locx-v1) {$1$};

%Location y
\path (locx.south) + (0,-1.5) node (locy) {$[\loc_y] \mapsto$};
\matrix(locycells) [version list, text width=7mm, anchor=west]
   at ([xshift=10pt]locy.east) {
 {a} & $T_0$ & {a} & $\tsid_2$ \\
  {a} & $\{\tsid_3\}$ & {a} & $\emptyset$\\
};
\node[version node, fit=(locycells-1-1) (locycells-2-1), fill=white, inner sep= 0cm, font=\Large] (locy-v0) {$0$};
\node[version node, fit=(locycells-1-3) (locycells-2-3), fill=white, inner sep=0cm, font=\Large] (locy-v1) {$1$};

% \draw[-, red, very thick, rounded corners] ([xshift=-5pt, yshift=5pt]locx-v1.north east) |- 
%  ($([xshift=-5pt,yshift=-5pt]locx-v1.south east)!.5!([xshift=-5pt, yshift=5pt]locy-v0.north east)$) -| ([xshift=-5pt, yshift=5pt]locy-v0.south east);

%blue view - I should  check whether I can use pgfkeys to just declare the list of locations, and then add the view automatically.
\draw[-, blue, very thick, rounded corners=10pt]
 ([xshift=-3pt, yshift=20pt]locx-v1.north east) node (tid1start) {} -- 
% ([xshift=-3pt, yshift=-5pt]locx-v1.south east) --
% ([xshift=-3pt, yshift=5pt]locy-v0.north east) -- 
 ([xshift=-3pt, yshift=-5pt]locy-v1.south east);
 
 \path (tid1start) node[anchor=south, rectangle, fill=blue!20, draw=blue, font=\small, inner sep=1pt] {$\tid_1$};

%red view
\draw[-, red, very thick, rounded corners = 10pt]
 ([xshift=-16pt, yshift=5pt]locx-v1.north east) node (tid2start) {}-- 
% ([xshift=-16pt, yshift=-5pt]locx-v0.south east) --
% ([xshift=-16pt, yshift=5pt]locy-v1.north east) -- 
 ([xshift=-16pt, yshift=-5pt]locy-v1.south east) node {};
 
\path (tid2start) node[anchor=south, rectangle, fill=red!20, draw=red, font=\small, inner sep=1pt] {$\tid_2$};

%%Stack for threads tid_1 and tid_2
%
%\draw[-, dashed] let 
%   \p1 = ([xshift=0pt]locy.west),
%   \p2 = ([yshift=-5pt]locycells.south),
%   \p3 = ([xshift=10pt]locycells.east) in
%   (\x1, \y2) -- (\x3, \y2);
%   
%\matrix(stacks) [
%   matrix of nodes,
%   anchor=north, 
%   text=blue, 
%   font=\normalsize, 
%   row 1/.style = {text = blue}, 
%   row 2/.style = {text = red}, 
%   text width= 13mm ] 
%   at ([xshift=-10pt,yshift=-8pt]locycells.south) {
%   $\tid_1:$ & $\retvar = 0$\\
%   $\tid_2:$ & $\retvar = 0$\\
%   };
\end{pgfonlayer}
\end{tikzpicture}
\\
{\small(c)} & {\small(d)}\\
\hline
\begin{tikzpicture}[font=\large]

\begin{pgfonlayer}{foreground}
%Uncomment line below for help lines
%\draw[help lines] grid(5,4);

%Location x
\node(locx) at (1,3) {$[\loc_x] \mapsto$};

\matrix(locxcells) [version list, text width=7mm, anchor=west]
   at ([xshift=10pt]locx.east) {
 {a} & $T_0$ & {a} & $\tsid_1$\\
  {a} & $\emptyset$ & {a} & $\emptyset$ \\
};
\node[version node, fit=(locxcells-1-1) (locxcells-2-1), fill=white, inner sep= 0cm, font=\Large] (locx-v0) {$0$};
\node[version node, fit=(locxcells-1-3) (locxcells-2-3), fill=white, inner sep=0cm, font=\Large] (locx-v1) {$1$};

%Location y
\path (locx.south) + (0,-1.5) node (locy) {$[\loc_y] \mapsto$};
\matrix(locycells) [version list, text width=7mm, anchor=west]
   at ([xshift=10pt]locy.east) {
 {a} & $T_0$ & {a} & $\tsid_2$ \\
  {a} & $\emptyset$ & {a} & $\emptyset$\\
};
\node[version node, fit=(locycells-1-1) (locycells-2-1), fill=white, inner sep= 0cm, font=\Large] (locy-v0) {$0$};
\node[version node, fit=(locycells-1-3) (locycells-2-3), fill=white, inner sep=0cm, font=\Large] (locy-v1) {$1$};

% \draw[-, red, very thick, rounded corners] ([xshift=-5pt, yshift=5pt]locx-v1.north east) |- 
%  ($([xshift=-5pt,yshift=-5pt]locx-v1.south east)!.5!([xshift=-5pt, yshift=5pt]locy-v0.north east)$) -| ([xshift=-5pt, yshift=5pt]locy-v0.south east);

%blue view - I should  check whether I can use pgfkeys to just declare the list of locations, and then add the view automatically.
\draw[-, blue, very thick, rounded corners=10pt]
 ([xshift=-3pt, yshift=20pt]locx-v1.north east) node (tid1start) {} -- 
 ([xshift=-3pt, yshift=-5pt]locx-v1.south east) --
 ([xshift=-3pt, yshift=5pt]locy-v0.north east) -- 
 ([xshift=-3pt, yshift=-5pt]locy-v0.south east);
 
 \path (tid1start) node[anchor=south, rectangle, fill=blue!20, draw=blue, font=\small, inner sep=1pt] {$\tid_1$};

%red view
\draw[-, red, very thick, rounded corners = 10pt]
 ([xshift=-16pt, yshift=5pt]locx-v1.north east) node (tid2start) {}-- 
% ([xshift=-16pt, yshift=-5pt]locx-v0.south east) --
% ([xshift=-16pt, yshift=5pt]locy-v1.north east) -- 
 ([xshift=-16pt, yshift=-5pt]locy-v1.south east) node {};
 
\path (tid2start) node[anchor=south, rectangle, fill=red!20, draw=red, font=\small, inner sep=1pt] {$\tid_2$};

%%Stack for threads tid_1 and tid_2
%
%\draw[-, dashed] let 
%   \p1 = ([xshift=0pt]locy.west),
%   \p2 = ([yshift=-5pt]locycells.south),
%   \p3 = ([xshift=10pt]locycells.east) in
%   (\x1, \y2) -- (\x3, \y2);
%   
%\matrix(stacks) [
%   matrix of nodes,
%   anchor=north, 
%   text=blue, 
%   font=\normalsize, 
%   row 1/.style = {text = blue}, 
%   row 2/.style = {text = red}, 
%   text width= 13mm ] 
%   at ([xshift=-10pt,yshift=-8pt]locycells.south) {
%   $\tid_1:$ & $\retvar = 0$\\
%   $\tid_2:$ & $\retvar = 0$\\
%   };
\end{pgfonlayer}
\end{tikzpicture}
&
\begin{tikzpicture}[font=\large]

\begin{pgfonlayer}{foreground}
%Uncomment line below for help lines
%\draw[help lines] grid(5,4);

%Location x
\node(locx) at (1,3) {$[\loc_x] \mapsto$};

\matrix(locxcells) [version list, text width=7mm, anchor=west]
   at ([xshift=10pt]locx.east) {
 {a} & $T_0$ & {a} & $\tsid_1$\\
  {a} & $\emptyset$ & {a} & $\{\tsid_4\}$ \\
};
\node[version node, fit=(locxcells-1-1) (locxcells-2-1), fill=white, inner sep= 0cm, font=\Large] (locx-v0) {$0$};
\node[version node, fit=(locxcells-1-3) (locxcells-2-3), fill=white, inner sep=0cm, font=\Large] (locx-v1) {$1$};

%Location y
\path (locx.south) + (0,-1.5) node (locy) {$[\loc_y] \mapsto$};
\matrix(locycells) [version list, text width=7mm, anchor=west]
   at ([xshift=10pt]locy.east) {
 {a} & $T_0$ & {a} & $\tsid_2$ \\
  {a} & $\{\tsid_3\}$ & {a} & $\emptyset$\\
};
\node[version node, fit=(locycells-1-1) (locycells-2-1), fill=white, inner sep= 0cm, font=\Large] (locy-v0) {$0$};
\node[version node, fit=(locycells-1-3) (locycells-2-3), fill=white, inner sep=0cm, font=\Large] (locy-v1) {$1$};

% \draw[-, red, very thick, rounded corners] ([xshift=-5pt, yshift=5pt]locx-v1.north east) |- 
%  ($([xshift=-5pt,yshift=-5pt]locx-v1.south east)!.5!([xshift=-5pt, yshift=5pt]locy-v0.north east)$) -| ([xshift=-5pt, yshift=5pt]locy-v0.south east);

%blue view - I should  check whether I can use pgfkeys to just declare the list of locations, and then add the view automatically.
\draw[-, blue, very thick, rounded corners=10pt]
 ([xshift=-3pt, yshift=20pt]locx-v1.north east) node (tid1start) {} -- 
% ([xshift=-3pt, yshift=-5pt]locx-v1.south east) --
% ([xshift=-3pt, yshift=5pt]locy-v0.north east) -- 
 ([xshift=-3pt, yshift=-5pt]locy-v1.south east);
 
 \path (tid1start) node[anchor=south, rectangle, fill=blue!20, draw=blue, font=\small, inner sep=1pt] {$\tid_1$};

%red view
\draw[-, red, very thick, rounded corners = 10pt]
 ([xshift=-16pt, yshift=5pt]locx-v1.north east) node (tid2start) {}-- 
% ([xshift=-16pt, yshift=-5pt]locx-v0.south east) --
% ([xshift=-16pt, yshift=5pt]locy-v1.north east) -- 
 ([xshift=-16pt, yshift=-5pt]locy-v1.south east) node {};
 
\path (tid2start) node[anchor=south, rectangle, fill=red!20, draw=red, font=\small, inner sep=1pt] {$\tid_2$};

%%Stack for threads tid_1 and tid_2
%
%\draw[-, dashed] let 
%   \p1 = ([xshift=0pt]locy.west),
%   \p2 = ([yshift=-5pt]locycells.south),
%   \p3 = ([xshift=10pt]locycells.east) in
%   (\x1, \y2) -- (\x3, \y2);
%   
%\matrix(stacks) [
%   matrix of nodes,
%   anchor=north, 
%   text=blue, 
%   font=\normalsize, 
%   row 1/.style = {text = blue}, 
%   row 2/.style = {text = red}, 
%   text width= 13mm ] 
%   at ([xshift=-10pt,yshift=-8pt]locycells.south) {
%   $\tid_1:$ & $\retvar = 0$\\
%   $\tid_2:$ & $\retvar = 0$\\
%   };
\end{pgfonlayer}
\end{tikzpicture}
\\
{\small(e)} & {\small(f)}\\
\hline
\end{tabular}
\end{center}
\caption{Configurations obtained throughout an execution of 
$\prog_4$.}
\label{fig:cp.exec}
\end{figure}
The next consistency model that we illustrate is given 
by consistent prefix. In this consistency model, we must 
ensure that once a thread observes the effects of some 
transaction $\tsid$, it also observes all the 
transactions that were executed before $\tsid$. 
Another way to express this property is that two different 
transactions never observe the updates to different locations 
in different order.

Consider, for example, the program $\prog_4$ below.
 \[
    \prog_4 :=  \left( \begin{session}
        \begin{array}{@{}c || c @{}}
            \begin{session}
            \begin{transaction}
                \pmutate{\loc_{x}}{1};\\
            \end{transaction}\\ \; \\
            
            \begin{transaction}
              	\pderef{\pvar{a}}{\loc_{y}};\\
              	\pifs{\pvar{a}=0}\\
              		\;\;\;\;\passign{\retvar}{\Large \frownie{}} \};
            \end{transaction}
            \end{session}
              &
              \begin{session}
           \begin{transaction}
                \pmutate{\loc_{y}}{1};\\
            \end{transaction} \\ \;\\
            
            \begin{transaction}
              	\pderef{\pvar{a}}{\loc_{x}};\\
              	\pifs{\pvar{a}=0}\\
              		\;\;\;\;\passign{\retvar}{\Large \frownie{}} \};
            \end{transaction}
            \end{session}
        \end{array}
    \end{session}
    \right)
 \]
In this program, we let 
\[
\begin{array}{cc}
\ptrans{\trans_1} = \ptrans{\pmutate{\loc_x}{1}} & 
\ptrans{\trans_2} = \begin{transaction} 
                                    \pderef{\pvar{a}}{\loc_y};\\
                                    \pifs{\pvar{a}=0}\\
                                    			\;\;\;\;\passign{\retvar}{\Large \frownie{}} \};
                                \end{transaction}
                             \\
\ptrans{\trans_3} = \ptrans{\pmutate{\loc_y}{1}} & 
\ptrans{\trans_4} = \begin{transaction} 
                                    \pderef{\pvar{a}}{\loc_x};\\
                                    \pifs{\pvar{a}=0}\\
                                    			\;\;\;\;\passign{\retvar}{\Large \frownie{}} \};
                                \end{transaction}
\end{array}                    
\]

We argue that, under $\mathsf{RA}$, it is possible to obtain an execution 
of program $\prog_4$ where both $\ptrans{\trans_2}$ and $\ptrans{\trans_4}$ 
return value ${\Large \frownie{}}$.
Such an execution can be summarised as follows: 

\begin{itemize}
\item initially, $\tid_1$ executes transaction $\ptrans{\trans_1}$, 
leading to the configuration of Figure \ref{fig:cp.exec}(b), and the program 
$\ptrans{\trans_2} \Par \ptrans{\trans_3} ; \ptrans{\trans_4}$ to be 
executed, 
\item then $\tid_2$ executes transaction $\ptrans{\trans_3}$, leading 
to the configuration of Figure \ref{fig:cp.exec}(c); the remaining 
program to be executed is $\ptrans{\trans_3} \Par \ptrans{\trans_4}$, 
\item without changing its view, $\tid_1$ executes transaction $\ptrans{\trans_2}$. 
The code $\trans_2$ is executed using the heap $[ [\loc_x] \mapsto 1, [\loc_y] \mapsto 0]$ 
as a snapshot; this means that, by executing $\trans_2$, the variable $\retvar$ of the thread-local 
stack of $\tid_1$ is set to ${\Large \frownie{}}$. Next, the thread $\tid_2$ executes $\ptrans{\trans_4}$ 
without altering its view. Similarly to the execution of $\ptrans{\trans_2}$ in $\tid_1$, this causes the 
$\retvar$ variable of the thread-local stack of $\tid_2$ to be set to ${\Large \frownie{}}$. At this point, 
the final configuration of the program is the one given in Figure \ref{fig:cp.exec}(d).
\end{itemize}

To avoid different threads to observe different updates in different orders, 
we impose a constraint on the consistency model that is known as 
\emph{consistent prefix}: in a centralised database, where 
transactions have a start and a commit point, 
this constraint can be formalised as follows: 
\emph{If a transaction $\tsid_1$ observes the effects of another transaction $\tsid_2$, then it must 
observe the effects of any transaction that committed before $\tsid_2$.}
In the history heaps framework, transactions are executed in a single step; 
however, one may think of the order in which transactions execute in our 
operational semantics to be consistent with the order in which 
transactions commit (this correspondence will be made precise later in 
the document, when we will relate executions in our operational semantics 
to abstract executions used in the declarative style for specifications of 
consistency models). By requiring that, after a thread $\tid$ executes 
a transaction $\tsid$, it pushes its own view to be up-to-date with the state of 
the system, we model the fact that any future transaction executed 
by $\tid$ will observe the effects of anything that committed before 
$\tsid$ (included).

\begin{definition}
We say that $(hh, V) \triangleright_{\mathsf{CP}} \mathcal{O}: V'$ 
if and only if $(\hh, V) \triangleright_{\mathsf{RA}} \mathcal{O}: V'$, 
and for any location $[\loc_x]$, $V'([\loc_x]) = \lvert \mathsf{HHeapUpdate}(\hh, V, \mathcal{O}) \rvert -1$. 
\end{definition}

Consider again th program $\prog_4$, ths time to be executed 
under $\mathsf{CP}$. We argue that in this case it is not possible to have both 
threads $\tid_1$ and $\tid_2$ to set the value of $\retvar$ to ${\Large \frownie{}}$. 
The initial configuration in which the program $\prog_4$ is executed is 
the one depicted in Figure \ref{fig:cp.exec}(a). Initially, either thread $\tid_1$ executes 
the code $\ptrans{\trans_1}$, or thread $\tid_2$ executes transaction $\ptrans{\trans_3}$; without 
loss of generality, we consider the former option (the case in which $\tid_2$ executes 
first is symmetric). Upon executing the code $\ptrans{\trans_1}$, we obtain the 
configuration of Figure \ref{fig:cp.exec}(b), with the program $\ptrans{\trans_2} \Par (\ptrans{\trans_3} ; \ptrans{\trans_4})$ 
to be executed next.
At this point, note that under $\mathsf{CP}$ it is not possible for $\tid_2$ to execute $\ptrans{\trans_3}$ and obtain 
the configuration of Figure \ref{fig:cp.exec}(c) as a result. This is because, in $\mathsf{CP}$, we require the view of $\tid_2$ 
\textbf{after} executing $\ptrans{\trans_3}$ to be up-to-date, i.e. to point to the last location of each version. That is, 
assuming that $\tid_2$ executes $\ptrans{\trans_3}$ next, we obtain the configuration of Figure \ref{fig:cp.exec}(e). 
From this point on, whenever $\tid_2$ will execute transaction $\ptrans{\trans_4}$, it will read value $1$ for 
location $[\loc_x]$, hence it won't be able to set the value of $\retvar$ to ${\Large \frownie{}}$. One possible 
final configuration for the program is given in Figure \ref{fig:cp.exec}(f).

%\begin{definition}
%$(\hh, V, \mathcal{V}) \triangleright_{\mathsf{CP}} \mathcal{O}$ iff 
%$(\hh, V, \mathcal{V}) \triangleright_{\mathsf{RA}} \mathcal{O}$, 
%and for any $V' \in \mathcal{V}$, either $V' \sqsubseteq V$ or 
%$V \sqsubseteq V'$.
%\end{definition}
%\ac{I found that this is a very easy way to encode consistent 
%prefix. In English, a thread can execute a transaction if its view does 
%not cross with the views of any other thread.}

%\ac{Very Important - note to self: It seems that the condition of 
%requiring that views do not cross before executing a transaction 
%does not suffice to model snapshot isolation. In fact, it seems that 
%consistent prefix (when transaction are limited to either one read 
%or one write) coincides with TSO.
%Update, it seems that the condition that I need for Snapshot Isolation 
%(besides write confict detection) is that, after you execute a transaction, 
%you bring your view up-to-date. (So here I have to concede that I was wrong, 
%and the state of the view after you execute a transaction is actually important).\\
%
%}


\subsection{Parallel Snapshot Isolation and Snapshot Isolation}
\emph{Snapshot Isolation} (SI) is a consistency model that has been 
widely employed in both centralised and distributed databases. 
Because snapshot isolation does not scale well to 
geo-replicated and distributed systems, a weakening of this model 
called \emph{Parallel Snapshot Isolation} (PSI) has been recently proposed. 

Both SI and PSI can be specified in the history heap framework by combining 
the specification of other consistency models that we have already introduced. 
In short, SI combines atomic visibility, the snapshot monotonicity property from 
consistent prefix property (if a transaction $\tsid_1$ 
observes the effects of another transaction $\tsid_2$, then it also observes 
the effects of any transaction that committed before), and the write-conflict 
detection from Update Atomic (two committing transactions do not write 
concurrently to the same location). In contrast, PSI only requires atomic 
visibility, causal consistency and write-conflict detection. Formally, we have 
the following:
\begin{definition}
$\mathsf{PSI} = \mathsf{CC} \cap \mathsf{UA}$. \\
$\mathsf{SI} = \mathsf{CP} \cap \mathsf{UA}$.
\end{definition}

\subsection{Serialisability}
Serialisability is the last and strongest consistency 
model that we consider. Informally, under serialisability 
transactions appear to be executed in a sequential order. 
Consider for example the program 
\[
\begin{session}
\begin{array}{@{}c || c@{}}
\begin{transaction}
\pderef{\pvar{a}}{\loc_y};\\
\pifs{\pvar{a} = 0};\\
\;\;\; \pmutate{\loc_x}{1};\\
\;\;\; \passign{\retvar}{\Large \frownie{}};
\}
\end{transaction}
&
\begin{transaction}
\pderef{\pvar{a}}{\loc_x};\\
\pifs{\pvar{a} = 0};\\
\;\;\; \pmutate{\loc_y}{1};\\
\;\;\; \passign{\retvar}{\Large \frownie{}};
\}
\end{transaction}
\end{array}
\end{session}
\]
\begin{figure}
\begin{center}
\begin{tabular}{|@{}c|c@{}|}
\hline
\begin{tikzpicture}[font=\large]

\begin{pgfonlayer}{foreground}
%Uncomment line below for help lines
%\draw[help lines] grid(5,4);

%Location x
\node(locx) at (1,3) {$[\loc_x] \mapsto$};

\matrix(locxcells) [version list, text width=7mm, anchor=west]
   at ([xshift=10pt]locx.east) {
 {a} & $T_0$ \\
  {a} & $\emptyset$ \\
};
\node[version node, fit=(locxcells-1-1) (locxcells-2-1), fill=white, inner sep= 0cm, font=\Large] (locx-v0) {$0$};
%\node[version node, fit=(locxcells-1-3) (locxcells-2-3), fill=white, inner sep=0cm, font=\Large] (locx-v1) {$1$};

%Location y
\path (locx.south) + (0,-1.5) node (locy) {$[\loc_y] \mapsto$};
\matrix(locycells) [version list, text width=7mm, anchor=west]
   at ([xshift=10pt]locy.east) {
 {a} & $T_0$ \\
  {a} & $\emptyset$ \\
};
\node[version node, fit=(locycells-1-1) (locycells-2-1), fill=white, inner sep= 0cm, font=\Large] (locy-v0) {$0$};
%\node[version node, fit=(locycells-1-3) (locycells-2-3), fill=white, inner sep=0cm, font=\Large] (locy-v1) {$1$};

% \draw[-, red, very thick, rounded corners] ([xshift=-5pt, yshift=5pt]locx-v1.north east) |- 
%  ($([xshift=-5pt,yshift=-5pt]locx-v1.south east)!.5!([xshift=-5pt, yshift=5pt]locy-v0.north east)$) -| ([xshift=-5pt, yshift=5pt]locy-v0.south east);

%blue view - I should  check whether I can use pgfkeys to just declare the list of locations, and then add the view automatically.
\draw[-, blue, very thick, rounded corners=10pt]
 ([xshift=-3pt, yshift=20pt]locx-v0.north east) node (tid1start) {} -- 
% ([xshift=-2pt, yshift=-5pt]locx-v0.south east) --
% ([xshift=-2pt, yshift=5pt]locy-v0.north east) -- 
 ([xshift=-3pt, yshift=-5pt]locy-v0.south east);
 
 \path (tid1start) node[anchor=south, rectangle, fill=blue!20, draw=blue, font=\small, inner sep=1pt] {$\tid_1$};

%red view
\draw[-, red, very thick, rounded corners = 10pt]
 ([xshift=-16pt, yshift=5pt]locx-v0.north east) node (tid2start) {}-- 
% ([xshift=-8pt, yshift=-5pt]locx-v0.south east) --
% ([xshift=-8pt, yshift=5pt]locy-v0.north east) -- 
 ([xshift=-16pt, yshift=-5pt]locy-v0.south east) node {};
 
\path (tid2start) node[anchor=south, rectangle, fill=red!20, draw=red, font=\small, inner sep=1pt] {$\tid_2$};

%%Stack for threads tid_1 and tid_2
%
%\draw[-, dashed] let 
%   \p1 = ([xshift=0pt]locy.west),
%   \p2 = ([yshift=-5pt]locycells.south),
%   \p3 = ([xshift=10pt]locycells.east) in
%   (\x1, \y2) -- (\x3, \y2);
%   
%\matrix(stacks) [
%   matrix of nodes,
%   anchor=north, 
%   text=blue, 
%   font=\normalsize, 
%   row 1/.style = {text = blue}, 
%   row 2/.style = {text = red}, 
%   text width= 13mm ] 
%   at ([xshift=-10pt,yshift=-8pt]locycells.south) {
%   $\tid_1:$ & $\retvar = 0$\\
%   $\tid_2:$ & $\retvar = 0$\\
%   };
\end{pgfonlayer}
\end{tikzpicture} 
%
&
%
\begin{tikzpicture}[font=\large]

\begin{pgfonlayer}{foreground}
%Uncomment line below for help lines
%\draw[help lines] grid(5,4);

%Location x
\node(locx) at (1,3) {$[\loc_x] \mapsto$};

\matrix(locxcells) [version list, text width=7mm, anchor=west]
   at ([xshift=10pt]locx.east) {
 {a} & $\tsid_0$ & {a} & $\tsid_1$\\
  {a} & $\emptyset$ & {a} & $\emptyset$ \\
};
\node[version node, fit=(locxcells-1-1) (locxcells-2-1), fill=white, inner sep= 0cm, font=\Large] (locx-v0) {$0$};
\node[version node, fit=(locxcells-1-3) (locxcells-2-3), fill=white, inner sep=0cm, font=\Large] (locx-v1) {$1$};

%Location y
\path (locx.south) + (0,-1.5) node (locy) {$[\loc_y] \mapsto$};
\matrix(locycells) [version list, text width=7mm, anchor=west]
   at ([xshift=10pt]locy.east) {
 {a} & $T_0$ \\
  {a} & $\{\tsid_1\}$ \\
};
\node[version node, fit=(locycells-1-1) (locycells-2-1), fill=white, inner sep= 0cm, font=\Large] (locy-v0) {$0$};
%\node[version node, fit=(locycells-1-3) (locycells-2-3), fill=white, inner sep=0cm, font=\Large] (locy-v1) {$1$};

% \draw[-, red, very thick, rounded corners] ([xshift=-5pt, yshift=5pt]locx-v1.north east) |- 
%  ($([xshift=-5pt,yshift=-5pt]locx-v1.south east)!.5!([xshift=-5pt, yshift=5pt]locy-v0.north east)$) -| ([xshift=-5pt, yshift=5pt]locy-v0.south east);

%blue view - I should  check whether I can use pgfkeys to just declare the list of locations, and then add the view automatically.
\draw[-, blue, very thick, rounded corners=10pt]
 ([xshift=-3pt, yshift=20pt]locx-v1.north east) node (tid1start) {} -- 
 ([xshift=-3pt, yshift=-5pt]locx-v1.south east) --
 ([xshift=-3pt, yshift=5pt]locy-v0.north east) -- 
 ([xshift=-3pt, yshift=-5pt]locy-v0.south east);
 
 \path (tid1start) node[anchor=south, rectangle, fill=blue!20, draw=blue, font=\small, inner sep=1pt] {$\tid_1$};

%red view
\draw[-, red, very thick, rounded corners = 10pt]
 ([xshift=-16pt, yshift=5pt]locx-v0.north east) node (tid2start) {}-- 
% ([xshift=-8pt, yshift=-5pt]locx-v0.south east) --
% ([xshift=-8pt, yshift=5pt]locy-v0.north east) -- 
 ([xshift=-16pt, yshift=-5pt]locy-v0.south east) node {};
 
\path (tid2start) node[anchor=south, rectangle, fill=red!20, draw=red, font=\small, inner sep=1pt] {$\tid_2$};

%%Stack for threads tid_1 and tid_2
%
%\draw[-, dashed] let 
%   \p1 = ([xshift=0pt]locy.west),
%   \p2 = ([yshift=-5pt]locycells.south),
%   \p3 = ([xshift=10pt]locycells.east) in
%   (\x1, \y2) -- (\x3, \y2);
%   
%\matrix(stacks) [
%   matrix of nodes,
%   anchor=north, 
%   text=blue, 
%   font=\normalsize, 
%   row 1/.style = {text = blue}, 
%   row 2/.style = {text = red}, 
%   text width= 13mm ] 
%   at ([xshift=-10pt,yshift=-8pt]locycells.south) {
%   $\tid_1:$ & $\retvar = 0$\\
%   $\tid_2:$ & $\retvar = 0$\\
%   };
\end{pgfonlayer}
\end{tikzpicture}
\\
{\small(a)} & {\small(b)}\\
\hline
\begin{tikzpicture}[font=\large]

\begin{pgfonlayer}{foreground}
%Uncomment line below for help lines
%\draw[help lines] grid(5,4);

%Location x
\node(locx) at (1,3) {$[\loc_x] \mapsto$};

\matrix(locxcells) [version list, text width=7mm, anchor=west]
   at ([xshift=10pt]locx.east) {
 {a} & $T_0$ & {a} & $\tsid_1$\\
  {a} & $\{\tsid_2\}$ & {a} & $\emptyset$ \\
};
\node[version node, fit=(locxcells-1-1) (locxcells-2-1), fill=white, inner sep= 0cm, font=\Large] (locx-v0) {$0$};
\node[version node, fit=(locxcells-1-3) (locxcells-2-3), fill=white, inner sep=0cm, font=\Large] (locx-v1) {$1$};

%Location y
\path (locx.south) + (0,-1.5) node (locy) {$[\loc_y] \mapsto$};
\matrix(locycells) [version list, text width=7mm, anchor=west]
   at ([xshift=10pt]locy.east) {
 {a} & $T_0$ & {a} & $\tsid_2$ \\
  {a} & $\{\tsid_1\}$ & {a} & $\emptyset$\\
};
\node[version node, fit=(locycells-1-1) (locycells-2-1), fill=white, inner sep= 0cm, font=\Large] (locy-v0) {$0$};
\node[version node, fit=(locycells-1-3) (locycells-2-3), fill=white, inner sep=0cm, font=\Large] (locy-v1) {$1$};

% \draw[-, red, very thick, rounded corners] ([xshift=-5pt, yshift=5pt]locx-v1.north east) |- 
%  ($([xshift=-5pt,yshift=-5pt]locx-v1.south east)!.5!([xshift=-5pt, yshift=5pt]locy-v0.north east)$) -| ([xshift=-5pt, yshift=5pt]locy-v0.south east);

%blue view - I should  check whether I can use pgfkeys to just declare the list of locations, and then add the view automatically.
\draw[-, blue, very thick, rounded corners=10pt]
 ([xshift=-3pt, yshift=20pt]locx-v1.north east) node (tid1start) {} -- 
 ([xshift=-3pt, yshift=-5pt]locx-v1.south east) --
 ([xshift=-3pt, yshift=5pt]locy-v0.north east) -- 
 ([xshift=-3pt, yshift=-5pt]locy-v0.south east);
 
 \path (tid1start) node[anchor=south, rectangle, fill=blue!20, draw=blue, font=\small, inner sep=1pt] {$\tid_1$};

%red view
\draw[-, red, very thick, rounded corners = 10pt]
 ([xshift=-16pt, yshift=5pt]locx-v1.north east) node (tid2start) {}-- 
% ([xshift=-16pt, yshift=-5pt]locx-v0.south east) --
% ([xshift=-16pt, yshift=5pt]locy-v1.north east) -- 
 ([xshift=-16pt, yshift=-5pt]locy-v1.south east) node {};
 
\path (tid2start) node[anchor=south, rectangle, fill=red!20, draw=red, font=\small, inner sep=1pt] {$\tid_2$};

%%Stack for threads tid_1 and tid_2
%
%\draw[-, dashed] let 
%   \p1 = ([xshift=0pt]locy.west),
%   \p2 = ([yshift=-5pt]locycells.south),
%   \p3 = ([xshift=10pt]locycells.east) in
%   (\x1, \y2) -- (\x3, \y2);
%   
%\matrix(stacks) [
%   matrix of nodes,
%   anchor=north, 
%   text=blue, 
%   font=\normalsize, 
%   row 1/.style = {text = blue}, 
%   row 2/.style = {text = red}, 
%   text width= 13mm ] 
%   at ([xshift=-10pt,yshift=-8pt]locycells.south) {
%   $\tid_1:$ & $\retvar = 0$\\
%   $\tid_2:$ & $\retvar = 0$\\
%   };
\end{pgfonlayer}
\end{tikzpicture}
%
&
%
\begin{tikzpicture}[font=\large]

\begin{pgfonlayer}{foreground}
%Uncomment line below for help lines
%\draw[help lines] grid(5,4);

%Location x
\node(locx) at (1,3) {$[\loc_x] \mapsto$};

\matrix(locxcells) [version list, text width=7mm, anchor=west]
   at ([xshift=10pt]locx.east) {
 {a} & $\tsid_0$ & {a} & $\tsid_1$\\
  {a} & $\emptyset$ & {a} & $\emptyset$ \\
};
\node[version node, fit=(locxcells-1-1) (locxcells-2-1), fill=white, inner sep= 0cm, font=\Large] (locx-v0) {$0$};
\node[version node, fit=(locxcells-1-3) (locxcells-2-3), fill=white, inner sep=0cm, font=\Large] (locx-v1) {$1$};

%Location y
\path (locx.south) + (0,-1.5) node (locy) {$[\loc_y] \mapsto$};
\matrix(locycells) [version list, text width=7mm, anchor=west]
   at ([xshift=10pt]locy.east) {
 {a} & $T_0$ \\
  {a} & $\{\tsid_1\}$ \\
};
\node[version node, fit=(locycells-1-1) (locycells-2-1), fill=white, inner sep= 0cm, font=\Large] (locy-v0) {$0$};
%\node[version node, fit=(locycells-1-3) (locycells-2-3), fill=white, inner sep=0cm, font=\Large] (locy-v1) {$1$};

% \draw[-, red, very thick, rounded corners] ([xshift=-5pt, yshift=5pt]locx-v1.north east) |- 
%  ($([xshift=-5pt,yshift=-5pt]locx-v1.south east)!.5!([xshift=-5pt, yshift=5pt]locy-v0.north east)$) -| ([xshift=-5pt, yshift=5pt]locy-v0.south east);

%blue view - I should  check whether I can use pgfkeys to just declare the list of locations, and then add the view automatically.
\draw[-, blue, very thick, rounded corners=10pt]
 ([xshift=-3pt, yshift=20pt]locx-v1.north east) node (tid1start) {} -- 
 ([xshift=-3pt, yshift=-5pt]locx-v1.south east) --
 ([xshift=-3pt, yshift=5pt]locy-v0.north east) -- 
 ([xshift=-3pt, yshift=-5pt]locy-v0.south east);
 
 \path (tid1start) node[anchor=south, rectangle, fill=blue!20, draw=blue, font=\small, inner sep=1pt] {$\tid_1$};

%red view
\draw[-, red, very thick, rounded corners = 10pt]
 ([xshift=-16pt, yshift=5pt]locx-v1.north east) node (tid2start) {}-- 
 ([xshift=-16pt, yshift=-3pt]locx-v1.south east) --
 ([xshift=-16pt, yshift=7pt]locy-v0.north east) -- 
 ([xshift=-16pt, yshift=-5pt]locy-v0.south east) node {};
 
\path (tid2start) node[anchor=south, rectangle, fill=red!20, draw=red, font=\small, inner sep=1pt] {$\tid_2$};

%%Stack for threads tid_1 and tid_2
%
%\draw[-, dashed] let 
%   \p1 = ([xshift=0pt]locy.west),
%   \p2 = ([yshift=-5pt]locycells.south),
%   \p3 = ([xshift=10pt]locycells.east) in
%   (\x1, \y2) -- (\x3, \y2);
%   
%\matrix(stacks) [
%   matrix of nodes,
%   anchor=north, 
%   text=blue, 
%   font=\normalsize, 
%   row 1/.style = {text = blue}, 
%   row 2/.style = {text = red}, 
%   text width= 13mm ] 
%   at ([xshift=-10pt,yshift=-8pt]locycells.south) {
%   $\tid_1:$ & $\retvar = 0$\\
%   $\tid_2:$ & $\retvar = 0$\\
%   };
\end{pgfonlayer}
\end{tikzpicture}
\\
{\small(c)} & {\small(d)}\\
\hline
\multicolumn{2}{|c|}{
\begin{tikzpicture}[font=\large]

\begin{pgfonlayer}{foreground}
%Uncomment line below for help lines
%\draw[help lines] grid(5,4);

%Location x
\node(locx) at (1,3) {$[\loc_x] \mapsto$};

\matrix(locxcells) [version list, text width=7mm, anchor=west]
   at ([xshift=10pt]locx.east) {
 {a} \pgfmatrixnextcell $\tsid_0$ \pgfmatrixnextcell {a} \pgfmatrixnextcell $\tsid_1$\\
  {a} \pgfmatrixnextcell $\emptyset$ \pgfmatrixnextcell {a} \pgfmatrixnextcell $\{\tsid_2\}$ \\
};
\node[version node, fit=(locxcells-1-1) (locxcells-2-1), fill=white, inner sep= 0cm, font=\Large] (locx-v0) {$0$};
\node[version node, fit=(locxcells-1-3) (locxcells-2-3), fill=white, inner sep=0cm, font=\Large] (locx-v1) {$1$};

%Location y
\path (locx.south) + (0,-1.5) node (locy) {$[\loc_y] \mapsto$};
\matrix(locycells) [version list, text width=7mm, anchor=west]
   at ([xshift=10pt]locy.east) {
 {a} \pgfmatrixnextcell $\tsid_0$ \\
  {a} \pgfmatrixnextcell $\{\tsid_1\}$ \\
};
\node[version node, fit=(locycells-1-1) (locycells-2-1), fill=white, inner sep= 0cm, font=\Large] (locy-v0) {$0$};
%\node[version node, fit=(locycells-1-2) (locycells-1-3), fill=white, inner sep= 0cm, font= \Large] (locy-v0-ws) {$\tsid_0$};
%\node[version node, fit=(locycells-2-2) (locycells-2-3), fill=white, inner sep=0cm, font=\Large] (locy-v1-rs) {$\{\tsid_1, \tsid_2\}$};

% \draw[-, red, very thick, rounded corners] ([xshift=-5pt, yshift=5pt]locx-v1.north east) |- 
%  ($([xshift=-5pt,yshift=-5pt]locx-v1.south east)!.5!([xshift=-5pt, yshift=5pt]locy-v0.north east)$) -| ([xshift=-5pt, yshift=5pt]locy-v0.south east);

%blue view - I should  check whether I can use pgfkeys to just declare the list of locations, and then add the view automatically.
\draw[-, blue, very thick, rounded corners=10pt]
 ([xshift=-3pt, yshift=20pt]locx-v1.north east) node (tid1start) {} -- 
 ([xshift=-3pt, yshift=-5pt]locx-v1.south east) --
 ([xshift=-3pt, yshift=5pt]locy-v0.north east) -- 
 ([xshift=-3pt, yshift=-5pt]locy-v0.south east);
 
 \path (tid1start) node[anchor=south, rectangle, fill=blue!20, draw=blue, font=\small, inner sep=1pt] {$\tid_1$};

%red view
\draw[-, red, very thick, rounded corners = 10pt]
 ([xshift=-16pt, yshift=5pt]locx-v1.north east) node (tid2start) {}-- 
 ([xshift=-16pt, yshift=-3pt]locx-v1.south east) --
 ([xshift=-16pt, yshift=7pt]locy-v0.north east) -- 
 ([xshift=-16pt, yshift=-5pt]locy-v0.south east) node {};
 
\path (tid2start) node[anchor=south, rectangle, fill=red!20, draw=red, font=\small, inner sep=1pt] {$\tid_2$};

%%Stack for threads tid_1 and tid_2
%
%\draw[-, dashed] let 
%   \p1 = ([xshift=0pt]locy.west),
%   \p2 = ([yshift=-5pt]locycells.south),
%   \p3 = ([xshift=10pt]locycells.east) in
%   (\x1, \y2) -- (\x3, \y2);
%   
%\matrix(stacks) [
%   matrix of nodes,
%   anchor=north, 
%   text=blue, 
%   font=\normalsize, 
%   row 1/.style = {text = blue}, 
%   row 2/.style = {text = red}, 
%   text width= 13mm ] 
%   at ([xshift=-10pt,yshift=-8pt]locycells.south) {
%   $\tid_1:$ & $\retvar = 0$\\
%   $\tid_2:$ & $\retvar = 0$\\
%   };
\end{pgfonlayer}
\end{tikzpicture}
}\\
\multicolumn{2}{|c|}{{\small(e)}}\\
\hline
\end{tabular}
\end{center}
\caption{Graphical representation of configurations obtained by 
executing $\prog_5$.}
\label{fig:ser.exec}
\end{figure}

If we execute the program $\prog_5$ under any of the consistency model 
specifications presented so far, we find out that it is possible to infer 
an execution in which both threads $\tid_1$ and $\tid_2$ set the 
thread-local variable $\retvar$ to value ${\Large \frownie{}}$.
For example, under $\mathsf{SI}$, this can happen as follows: 
\begin{itemize}
\item first, thread $\tid_1$ executes its transactions in the 
configuration of Figure \ref{fig:ser.exec}(a). The snapshot 
under which the transaction is executed is given by $[[\loc_x] \mapsto 0, [\loc_y] \mapsto 0]$, 
hence the execution of the transaction results in the thread-local variable $\retvar$ of $\tid_1$ 
to be set to ${\Large \frownie{}}$, and in the configuration of Figure \ref{fig:ser.exec}(b).
\item Next, thread $\tid_2$ executes its transaction in the initial 
view $[ [\loc_x] \mapsto 0, [\loc_y] \mapsto 0]$. The snapshot under which 
the transaction is executed is again $[[\loc_x \mapsto 0, [\loc_y] \mapsto 0]$. After 
thetransaction has been executed, the thread-local variable $\retvar$ of $\tid_2$ is 
set to ${\Large \frownie{}}$, and the final configuration is the one of Figure \ref{fig:ser.exec}(c). 
Note that, because we are assuming SI as our consistency model, in this configuration 
we require that the view of thread $\tid_2$ points to the last version of each location.
\end{itemize}

To avoid the scenario above, known as the \emph{write skew} anomaly, it suffices to 
ensure that, prior to executing a transaction, the view of a thread is always up-to-date for each location. 


\begin{definition}
$(\hh, V) \triangleright_{\mathsf{SER}} \mathcal{O} : V'$ iff, for any location $[\loc_x]$, 
$V([\loc_x]) = \lvert \hh([\loc_x]) \rvert -1$. 
\end{definition}

Consider again the program $\prog_5$, this time to be executed under $\mathsf{SER}$. 
Similar as for $\mathsf{SI}$, after thread $\tid_1$ has executed its transaction, 
we end up with the variable $\retvar$ of such a thread to be set to value ${\Large \frownie{}}$, 
and with the configuration of Figure \ref{fig:ser.exec}(b). However, at this point we can 
not execute the transaction of thread $\tid_2$ without updating his view beforehand. 
This is because such a view does not point to the most up-to-date version for location 
$[\loc_x]$. Instead, before executing its transaction $\tid_2$, updates its view to 
point to the most up-to-date version of each location, Figure \ref{fig:ser.exec}(d). 
Thus, thread $\tid_2$ will execute its transaction using the snapshot $[[\loc_x] \mapsto 1, [\loc_y] \mapsto 0]$. 
A consequence of this fact is that the thread-local variable $\retvar$ of $\tid_2$ will not be 
set to ${\Large \frownie{}}$, and no new version for location $[\loc_y]$ will be created 
by the execution of the transaction. The final configuration is given in Figure \ref{fig:ser.exec}(e).

\ac{Contents: Read Atomic, Causal Consistency, Update Atomic, Consistent Prefix, Parallel Snapshot Isolation, Snapshot Isolation, 
Serializability. Well-formedness constraint to be placed on consistency models: progress must always be possible - i.e. it is 
always possible to execute a transaction if the view of all threads are up-to-date.}  
%
%
%\subsection{Encoding Dependency Graphs' Specifications}
%\ac{Contents: Isomorphism between history heaps and dependency graphs, 
%converting dependency graphs specifications, well-formedness of dependency graphs' specifications.}

\section{Relationship to Abstract Executions}
\newcommand{\TtoOp}[1]{\ensuremath{\mathscr{#1}}}
\newcommand{\ClSet}{\ensuremath{\Gamma}}
\newcommand{\progOrd}{\ensuremath{\pi}}
\newcommand{\RP}{\ensuremath{\mathsf{RP}}}
\newcommand{\LWW}{\ensuremath{\mathsf{LWW}}}
\newcommand{\F}{\ensuremath{\mathcal{F}}}
\newcommand{\calA}{\ensuremath{\mathcal{A}}}
\newcommand{\appendTx}{\ensuremath{\mathsf{appendTx}}}
\newcommand{\thdstackFun}{\ensuremath{\mathsf{\Sigma}}}
\newcommand{\absrightarrow}{\ensuremath{\rightarrowtail}}

In this Section we propose an alternative 
semantics of programs. In this semantics, states 
correspond to \emph{abstract executions} \cite{framework-concur}, 
overloaded with information regarding the transactions that are made 
visible to individual clients. 
The abstract execution semantics provides a bridge between specification 
of consistency models given in terms of MKVSs and views, and the already 
existing axiomatic ones. This section makes the following contributions: 
\begin{itemize}
\item \textbf{Done!} a definition of abstract executions, 
\item \textbf{Done!} the axiomatic specification of consistency models borrowed from cite{framework-concur},
\item \textbf{Done!} an encoding of abstract executions into MKVSs, 
\item an encoding of executions in the MKVS semantics into abstract 
executions, 
\item an operational semantics based on abstract executions, 
which is parametric in the axiomatic specification of a consistency model given using 
the declarative style of \cite{framework-concur,SIanalysis,laws}, 
\item a proof that, the semantics is \emph{faithful} with respect to any consistency model $\CM$: 
for any program $\prog$ and consistency model specification $\CM$, 
the abstract execution semantics captures all the potential behaviours that $\prog$ can 
exhibit under $\CM$. Faithfulness is necessary to validate the soundness of 
program analysis techniques, such as program logics, with respect to arbitrary consistency model 
specifications,
\ac{Before I was referring to what I call faithfulness as completeness. I am now changing this, because 
it looks like it only created confusion. In short words, this result requires defining the anarchic model 
of transactions - one in which the database actually reads and writes non-deterministic values 
despite the client requests, and prove that any abstract execution which is 
valid w.r.t. $\CM$ and can be obtained in the anarchic semantics, can also be obtained 
in the $\CM$ semantics. More details on this later.} 
\item a proof that each of the execution tests of Section \ref{sec:cmexamples}
captures precisely the corresponding consistency model. Specifically,
\begin{description}
\item[Soundness - ] for any program $\prog$, any execution of 
$\prog$ under the MKVSs semantics using the execution 
test $\aexec_{\CM}$ $\CM$, is encoded into an abstract execution 
that satisfies the axioms of $\CM$,
\item[Completeness - ] for any program $\prog$, any 
execution of $\prog$ under the abstract execution semantics of 
$\CM$, can be projected to an execution of $\prog$ under the history 
heap semantics using the execution test $\ET_{\CM}$. An immediate 
consequence of this completeness result, and of the faithfulness 
of the abstract execution semantic with respect to $\CM$, is that the MKVS semantics 
of $\CM$ is also faithful.
\end{description}
\end{itemize}

\paragraph{Abstract Executions}

\begin{definition}
A runtime abstract execution is a tuple $\aexec = (\TtoOp{T}, \PO, \VIS, \AR)$ where: 
\begin{enumerate}
\item $\TtoOp{T} : \TransID \rightharpoonup \powerset{\Op}$ is a partial function mapping transaction 
identifiers into sets of operations; if $\tsid \in \dom(\TtoOp{T})$, then the transaction $\tsid$ happened 
at some point in the abstract execution ${\aexec}$, by performing operations $\TtoOp{T}(\T)$, 
%\item $\ClSet \subseteq \tidset$ is a set of client identifiers, 
\item $\PO: \tidset \rightharpoonup (\dom(\TtoOp{T}) \times \dom(\TtoOp{T})$ is a function modelling the 
program order of clients: that is, there exists a $\dom(\PO)$-indexed partition $\{\T_{\tid}\}_{\tid \in \dom(\PO)}$ of 
$\dom(\TtoOp{\T})$ such that, for any $\tid \in \ClSet$ $\PO(\tid)$ is a strict, total order over $\T_{\tid}$.
\item $\VIS \subseteq \dom(\TtoOp{T}) \times (\dom(\TtoOp{T}))$ is a strict, irreflexive relation, 
\item $\AR \subseteq \dom(\TtoOp{T}) \times \dom(\TtoOp{T})$ is a strict, total order such that $\VIS \subseteq \AR$.
\end{enumerate}
\end{definition}
In general, given an abstract execution ${\aexec} = (\TtoOp{T}, \ClSet, \PO, \VIS, \AR)$, we let ${\TtoOp{T}}_{\aexec} = \TtoOp{T}, 
 \PO_{\aexec} = \PO, {\VIS}_{\aexec} = \VIS, \AR_{\aexec} = \AR$. We also let $\T_{\aexec} = \dom(\TtoOp{T}_{\aexec})$, 
 $\ClSet_{\aexec} = \dom(\PO_{\aexec})$, and $\T_{\aexec}(\tid) = \{\tsid \in \T_{\aexec} \mid (\tsid \xrightarrow{\PO_{\aexec}(\tid)} \_) 
 \vee (\_ \xrightarrow{\PO_{\aexec}} \tsid \}$. For $o \in \Op$, we also write $o \in_{{\aexec}} \tsid$ in lieu of $o \in \TtoOp{T}_{\aexec}(\tsid)$.
We often write $\tsid \xrightarrow{\VIS_{{\aexec}}} \tsid'$ in lieu of $(\tsid, \tsid') \in \xrightarrow{\VIS_{{\aexec}}}$, 
and similarly for other relations. We also commit an abuse of notation, and let $\PO_{{\aexec}} = 
\bigcup_{\tid \in \ClSet_{\aexec}} \PO(\tid)$. Henceforth, it will always be clear from the context 
whether the symbol $\PO_{\aexec}$ refers to a function from client identifiers to a relation between transaction, or to a relation
between transactions.

Given an abstract execution $\aexec$, a key $\key{k}$ and a transaction identifier $\tsid \in \dom(\T_{\aexec})$, we 
define $\visibleWrites_{{\aexec}}(\key{k}, \tsid) = \{ \tsid' \mid \tsid' \xrightarrow{\VIS_{{\aexec}}} \tsid \wedge (\WR\;\key{k}:\_) 
\in_{\aexec} \tsid' \}$.

\begin{definition}
A \emph{resolution policy} \RP is a function $\RP: \aexec \times \powerset{\TxID} \rightarrow \powerset{\Snapshots}$. 
\end{definition}

intuitively, $\RP(\aexec, \T)$ denotes the set of possible snapshot that a client would observe when executing a transaction 
in a state of the system corresponding to the abstract execution $\aexec$, when observing only the subset 
of transactions in $\T$, and when using the resolution policy $\RP$. We will mainly work with two different 
resolution policies, which are illustrated below. 

\begin{example}
The \emph{Last Write Wins} execution policy $\RP_{\LWW}$ is defined by letting, for a given abstract execution 
$\aexec$ and a set $\T \subseteq \T_{\aexec}$, 
\[
\RP_{\LWW}(\aexec, \T) = \left \{ \lambda \key{k}.
\begin{cases}
n & \text{if } \T_{\key{k}} \neq \emptyset \wedge (\WR\; \key{k}: n) \in_{\aexec} \max_{\AR_{\aexec}}(\T_{\key{k}}), \text{where } 
\T_{\key{k}} (\T \cap \{\tsid \mid \tsid \in_{\aexec} \WR k: \_ \})\\
0 & \text{otherwise} 
\end{cases}
\right\}
\]

The \emph{demonic} execution policy $\RP_{\bot}$ is defined by letting $\RP_{\bot}(\aexec, \T) = \Snapshots$ for any $\aexec, \T$.
\end{example}

\begin{definition}
An abstract execution $\aexec$ satisfies an execution policy $\RP$ if and only if, 
\[
\forall \tsid \in \T_{\aexec}. \exists \h \in \RP(\aexec, \VIS^{-1}(\tsid).\;
\forall \key{k} \in \Addr, n \in \nat.\; (\RD\;\key{k}: n) \in_{\aexec} \tsid \implies 
h(\key{k}) = n.
\]
\end{definition}

\paragraph{Axiomatic Specification of Consistency Models}
\ac{This will go in a Figure. We never encoded session guarantees into abstract 
executions, sot that they will need to be checked. Guarantees that have never been 
proved to be sound are marked with (?)}

A consistency model consist of a set of abstract executions. 
To specify consistency models, we fix a resolution policy and 
a set of axioms that pose conditions on the relations $\PO_{\aexec}, \VIS_{\aexec}, \AR_{\aexec}$. 
Axioms in this framework take either the form $\F(\PO_{\aexec}, \VIS_{\aexec}, \AR_{\aexec}) \subseteq \VIS_{\aexec}$, 
or $\F(\aexec) \subseteq \AR_{\aexec}$. In the following, we illustrate well 
known axiomatic specifications of consistency models, each of which uses the 
last write wins resolution policy. We use the notation 
$[\WR(\key{k})]_{\aexec} = \{ (\tsid, \tsid) \mid (\RD\; \key{k}: n) \in_{\aexec} \tsid \}$, and 
similarly for other operations. 
We leave the language used for specifying axioms of consistency models unspecified, though 
we remark that all our examples can be captured using the style of specification previously 
proposed in \cite{laws}.

\begin{itemize}
\item Read Your Writes (?): read operations reflect previous writes. $[\WR(\key{k})] ; \PO ; [\RD(\key{k})] \subseteq \VIS$, 
\item Monotonic Reads (?): successive reads reflect a non-decreasing set of writes. $\VIS; [\RD] \PO ; [\RD] \subseteq \VIS$, 
\item Write Follows Reads (?): writes are propagated after reads on which they depend. $\VIS; [\RD]; \PO; [\WR] \subseteq \VIS$, 
\item Monotonic Writes (?): writes are propagated after writes that logically precede them. $[\WR]; \PO; [\WR]; \VIS \subseteq \VIS$, 
\item Strong sessions: sessions appear to be sequentially consistent: $\PO \subseteq \VIS$, 
\item Causal Consistency: strong sessions + visibility is transitive, $\VIS ; \VIS \subseteq \VIS$, 
\item Consistent Prefix: strong sessions + if transaction $\tsid_1$ observes transaction $\tsid_2$, it also observes 
anything that precedes $\tsid_2$ in the arbitration order: $\AR ; \VIS \subseteq \VIS$, 
\item Update Atomic: transactions that update one same object, do not execute concurrently: 
\item PSI: update atomic + causal consistency, SI: update atomic + prefix consistency, 
\item Serialisability: visibility is a total order, $\AR \subseteq \VIS$.
\end{itemize}


\subsection{Operational Semantics using Abstract Executions}
We give an alternative operational semantics of transactions, that uses abstract executions as states. 
In this semantics judgements take the form 
\[
\langle \aexec, \thdstackFun, \prog \rangle \absrightarrow_{\CM} \langle \aexec', \thdstackFun', \prog' \rangle
\]
\ac{Want to change the symbol for the arrow, but if I try to change to \emph{rightarrowtriangle} then 
I need to load the package \emph{stmaryrd}, which in turn causes latex to load too many math alphabets. 
Need to find a solution for this.}
where $\CM = (\RP, \calA)$ is an axiomatic specification of a consistency model
consisting of a resolution policy and a set of axioms, and $\thdstackFun: \tidset \rightharpoonup \ThdStacks$ 
is a partial mapping from client identifiers to stacks.

\begin{figure}
\begin{center}
\begin{tabular}{|@{}cc@{}|}
\multicolumn{2}{|@{}c@{}|}{
$
\infer[{\scriptstyle(A-Tx)}]{\langle \aexec, \thdstackFun, \tid: [\trans] \rangle \absrightarrow_{\CM} \langle \aexec', \thdstackFun', \nil \rangle}
{\begin{array}{c}
\CM = (\RP, \calA) \qquad \T \subseteq \T_{\aexec} \qquad
\h \in \RP(\aexec, \T) \\
\langle \h, \thdstack, \emptyset, \trans \rangle \rightarrow^{\ast} \langle \_, \thdstack', \mathcal{O}, \nil \rangle\\
\aexec' = \appendTx_{\tsid}(\aexec, \tid, \T, \mathcal{O}) \qquad \aexec' \models \calA
\end{array}
}
$
}
\end{tabular}
\end{center}
\caption{Rules of the Operational Semantics of Abstract Executions.}
\end{figure}

The rules of the semantics for executing transactional code are the same of Figure \ref{fig:opsem.tx}.
Selected rules of the semantics of commands and programs are given in Figure \ref{fig:opsem.absexec}. 
\ac{Going to type down only the rule for tranactions, as all the other rules are analogous 
to the rules of Figure \ref{fig:opsem.cmd} and \ref{fig:opsem.prog}.}

As for the MKVS semantics,  the only non-trivial rule in the semantics of commands and programs 
is the one modelling a client executing a transaction, Rule $(A-Tx)$.
The function $\appendTx_{\tsid}$ in the premiss of this rule takes in input an abstract execution 
$\aexec$, a client $\tid$, a subset $\T \subseteq \T_{\aexec}$, and a set of operations $\mathcal{O}$. 
It then returns a new abstract execution, corresponding to the result of client $\tid$, observing the
transactions included in $\T$, executing a transaction with set of operations 
are $\mathcal{O}$. The set of transactions $\T$ is used, together with the resolution policy 
$\RP$ of the consistency model, to determine one snapshot under which the transactional 
code $\ptrans{\trans}$ is executed. Upon executing this code, we obtain a set of operations 
$\mathcal{O}$, which is then associated to the transaction identifier $\tsid$
The transaction is obtained from $\aexec$ by appending the novel transaction $\tsid$
in $\PO_{\aexec}(\tid)$; the arbitration order of $\aexec$ is also obtained by extending 
$\AR_{\aexec}$ so that $\tsid$ is the last transaction in such an order; finally, the visibility 
relation is obtained by extending $\VIS_{\aexec}$ with the set $\{(\tsid', \tsid) \mid \tsid' \in \T\}$, 
as to reflect the fact that the transactions in $\T$ have been observed by $\tsid'$. 
\begin{definition}
Let $\aexec$ be an abstract execution, $\tsid$ be a transaction identifier such 
that $\tsid \notin \T_{\aexec}$,  $\tid$ be a client $\in \I_{\aexec}$, $\T$ be a set of transactions, 
and $\mathcal{O}$ be a set of operations. Then we let 
$\appendTx_{\tsid}(\aexec, \tid, \T, \mathcal{O})$ be the abstract execution $\aexec'$ defined as 
follows: 
\[
\begin{array}{lcl}
\TtoOp{T}_{\aexec'} &=& \TtoOp{T}_{\aexec}[\tid \mapsto \mathcal{O}]\\
\PO_{\aexec'} &=& \PO_{\aexec}\left[\tid \mapsto \left( \PO_{\aexec}(\tid) \cup \{(\tsid', \tsid) \mid \tsid' \in \T_{\aexec}(\tid)\} \right) \right]\\
\VIS_{\aexec'} &=& \VIS_{\aexec} \cup \left(\{(\tsid', \tsid) \mid \tsid' \in \T\}\right)\\
\AR_{\aexec'} &=& \AR_{\aexec}' \cup \left(\{(\tsid', \tsid) \mid \tsid' \in \T_{\aexec}\}\right)
\end{array}
\]
\end{definition}

\subsection{Encoding of Abstract Executions into configurations}
\ac{IMPORTANT: This definition assumes that the definition of configurations in the MKVS framework has changed as to include, besides 
the mapping $\kappa: \Keys \rightarrow \Versions^{\ast}$ and the mapping from clients to views, 
also a partial function $\pi: \tidset \rightharpoonup \TransID^{\ast}$ corresponding to the order 
in which transactions have been executed by clients.}

In the following, we assume a special transaction identifier $\tsid_0$ that is not used 
in abstract executions. This can be thought as an initial transaction that initialises the value of all 
keys in the key-value store to the default value $0$.
\begin{definition}[Transaction Dependencies]
\ac{This is basically the definition of Adya's/Jade's dependencies.}
Let $\hat{\aexec}$ be an abstract execution. For each $\key{k} \in \Addr$, 
we define the relations $\RF(\key{k}), \VO(\key{k}) \subseteq (\T_{\hat{\aexec}} \uplus \{\tsid_0\}) \times 
\T_{\hat{\aexec}}$ as follows: 
\begin{itemize}
\item $\tsid \xrightarrow{\RF(\key{k})} \tsid'$ if and only if either $\tsid = \tsid_0$ and 
$\visibleWrites_{\hat{\aexec}}(\key{k}, \tsid) = \emptyset$, or 
$\tsid = \max_{\AR_{\hat{\aexec}}}(\visibleWrites_{\hat{\aexec}}(\key{k}, \tsid))$, 
\item $\tsid \xrightarrow{\VO(\key{k}} \tsid'$ if and only if either $\tsid = \tsid_0$ 
and $\WR\;\key{k}: \_ \in_{\hat{\aexec}} \tsid'$, or $\tsid, \tsid' \in_{\hat{\aexec}} (\WR\;\key{k}: \_)$ 
and $\tsid \xrightarrow{\AR_{\aexec}} \tsid'$.
\end{itemize}
\end{definition}

Given a set $X$, an element $x \in X$ and a list $l \in X^{\ast}$, we write 
$x \in l$ as a shorthand for $\exists l_1, l_2. \; l = l_1 \cdot x \cdot \l_2$. 
Here $\cdot$ is the list concatenation operator. Given $x_1, x_2 \in X$ and 
$l \in L$, we let $x_1 <_{l} x_2$ if $l = \_ \cdot x_1 \cdot \_ \cdot x_2 \cdot \_$.
\ac{The sentence above should appear much before.}

\begin{definition}
Let $\hat{\aexec}$ be an abstract execution. 
%Let $\tsid_0$ be a transaction identifier 
%that is not included in $\dom(\TtoOp)$. 
We define the MKVS $(\hh_{\hat{\aexec}}, \progOrd_{\hat{\aexec}})$ as follows: 
\begin{itemize}
\item For each transaction $\tsid \in \T_{\hat{\aexec}}$ and key $\key{k}$, 
we define the set $\Versions_{\hat{\aexec}}(\key{k}) \subseteq \Versions$ to be the smallest set such that 
\begin{enumerate}
\item for any $\tsid \in \T_{\hat{\aexec}}$ such that $\WR\;\key{k}: n \in_{\aexec} \TtoOp(\tsid)$,
then
$(n, \tsid, \mathcal{R}(\tsid, \key{k})) \in \Versions_{\hat{\aexec}}(\key{k})$, 
where $\mathcal{R}(\tsid, \key{k}) = \{\tsid' \mid \tsid \xrightarrow{\RF(\key{k})} \tsid'\}$ 
\item $(0, \tsid_0, \mathcal{R}(\tsid_{0}, \key{k}) \in \Versions_{\hat{\aexec}}(\key{k})$, 
where $\mathcal{R}(\tsid_{0}, \key{k}) = \{\tsid' \mid \tsid_0 \xrightarrow{\RF(\key{k})} \tsid'\}$.
\end{enumerate}
\item The MKVS $\hh_{\hat{\aexec}}$ is defined to be the unique MKVS such that 
for any key $\key{k} \in \Addr$, $\nu \in \hh_{\hat{\aexec}}(\key{k})$ if and only if 
$\nu \in \Versions_{\hat{\aexec}}(\key{k})$; furthermore, for any $\nu, \nu' \in \hh_{\hat{\aexec}}(\key{k})$, 
we have that $\nu <_{\hh_{\hat{\aexec}(\key{k})}} \nu'$ if and only if $\nu = (\_, \tsid, \_)$, 
$\nu' = (\_, \tsid', \_)$, and $\tsid \xrightarrow{\VO_{\hat{\aexec}}(\key{k})} \tsid'$.
\item $\progOrd_{\hat{\aexec}}$ is the function that maps each $\tid \in \dom(\I_{\aexec})$ into a list of transaction 
identifiers which is consistent with the program order of $\tid$ in $\hat{\aexec}$, and which 
contains $\tid_0$ as its initial element : $\tsid \in \progOrd_{\hat{\aexec}}(\tid)$ 
if and only if either $\tsid = \tsid_0$, or $\tsid \xrightarrow{\PO_{\hat{\aexec}}} \tid$. 
Furthermore, whenever $\tsid <_{\progOrd_{\hat{\aexec}(\tid)}} \tsid'$, then either 
$tsid = \tsid_0$, or $\tsid \xrightarrow{\PO_{\hat{\aexec}}} \tsid'$. 
\ac{This automatically implies that $\tsid_0$ is the first element in the program order 
of each client.}
\end{itemize}
\end{definition}

\begin{proposition}
Let $\hat{\aexec}$ be an abstract execution that satisfies the last write wins policy. 
Then $(\hh_{\hat{\aexec}}, \pi_{\hat{\aexec}})$ is well-formed.
\end{proposition}




\subsection{Completeness of the Semantics} 
\ac{Contents: Abstract execution Semantics and the anarchic model. 
Encoding of abstract executions into history heaps. Also, 
traces in the history heaps semantics can be used to recover 
an abstract execution $\aexec$. Hence it is possible to convert history heaps 
specifications into sets of abstract executions.
I think this should be the Theorem: for every possible trace of a program $\prog$ that is allowed by the anarchic 
model, and that results into an abstract execution $\chi$ that is allowed by $CM$, there 
exists a trace of the same program under the history heaps $CM$-semantics, 
and whose encoding into an abstract execution is exactly $\aexec$.}
\subsection{Remarks on the operational semantics}
\ac{There was a discussion on whether views of threads should always be consistent w.r.t a 
consistency model specification, or whether they should be consistent only w.r.t. to 
a consistency model specification only prior to executing a transaction.\\ 
In this section I should argue that the first option leads to losing the completeness of the 
semantics.}


%\section{Testing Consistency Models Specifications}
%\ac{Definition of may-testing preorder, which is standard and at some point 
%was also in the notes - where did it go? Most General Client Semantics of consistency 
%models under history heaps. Conjecture: two consistency models are equivalent if they 
%produce the same set of histories (transactions labelled with the session order) under 
%the most general client semantics. This is consistent with Hongseok's result that contextual 
%refinement for concurrent libraries coincide with 
%the sequential consistency preorder, when threads can only communicate via the memory regions 
%shared withing the library.}

%\subsection{Applications}
%\ac{Proof of equivalence of consistency models specifications given in terms of history heaps, 
%with respect to the same specifications given in terms of dependnecy graphs. In this section 
%I must be careful to address the work by Crook and Alvisi, as they also have an operational model 
%of transactions (without a programming language) and a proof of equivalence with respect to 
%dependency graphs' specifications.}

%\subsection{Verification of Algorithms} 
%Candidates: 
%\begin{itemize}
%\item Walter (\textbf{https://dl.acm.org/citation.cfm?id=2043592}) - PSI, should be easy, 
%people may find it boring, the Concur'15 model was inspired by Walter and people may argue we 
%already proved its correctness in there.
%\item Wren (\textbf{https://infoscience.epfl.ch/record/254970}) - Causal Consistency - very recent and seems to have 
%a nice pseudo-code.\\
%\item COPS-gt (\textbf{https://www.cs.cmu.edu/~dga/papers/cops-sosp2011.pdf}) - Causal Consistency -  mentioned a lot at 
%talks, may be a viable option.
%\item Distributed SI (\textbf{https://dl.acm.org/citation.cfm?id=2691546}) - Snapshot Isolation - 5 different protocols given in the paper, 
%not sure if any of those is easily verifiable.\\ 
%\item Clock-SI (\textbf{https://dl.acm.org/citation.cfm?id=2553434}) - Using physical clocks as time-stamps, may have a direct encoding 
%into history heaps - has some chances of being verifiable.
%\end{itemize}

\bibliographystyle{abbrv}
\bibliography{bibliography2}

\end{document}

