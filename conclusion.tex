

\section{Conclusions and Future Work}
\label{sec:conclusions}

We have introduced a simple interleaving semantics for atomic
transactions, based on a global, centralised kv-stores and partial
client views. It is expressive enough to capture the anomalous
behaviour of many weak consistency models.  We have demonstrated that
our semantics can be used to both verify protocols of distributed databases
and prove the invariant properties of client programs. 


We have defined a large variety of consistency models for kv-stores
based on execution tests.
In the technical report, we have shown these definitions to be equivalent to
well-known declarative definitions on abstract executions.
We do not know of an appropriate consistency
model that we cannot express with our semantics, bearing in mind the
constraints that our transactions satisfy snapshot property and the last-write-wins policy. 
We have proved the correctness of two real-world protocols employed by distributed 
databases: COPS \cite{cops}, a replicated database that satisfies causal consistency;
and Clock-SI \cite{clocksi}, a partitioned database that satisfies snapshot isolation.
We have also demonstrated the usefulness of our framework
for proving invariant properties: the robustness properties of the counters and banking libraries
against different consistency models.
Particular, we have identified a new consistency model, \emph{weak snapshot isolation},
which lies between $\PSI$ and $\SI$ and inherits many of the good properties of $\SI$.
This new model is used as a proof method to prove the robustness of 
multi-counter and banking libraries against \( \SI \).
We further use our operational semantics to prove \emph{library-specific} properties:
the correctness (despite not robust) of the lock library under \( \UA \),
in that it satisfies the \emph{mutual exclusion guarantee}.

In future, we aim to extend our framework to handle other 
weak consistency models. For example, we believe that, by introducing promises 
in the style of \cite{promises}, we can capture  consistency models such as \emph{Read Committed}. 
We are also interesting in C11 memory model.
For example, 
Doherty et al. \citet{op-semantics-c11-rar} develop an operational semantics 
for release-acquire fragment of C11 memory model, a variant of causal consistency,
on dependency graphs.
We believe that we are also able to model release-acquire fragment of C11.
We plan to validate further the usefulness of our framework by: 
verifying other well-known protocols of distributed databases \cite{ramp,redblue,eiger,wren};
exploring robustness results for OLTP workloads such as TPC-C \cite{tpcc} and RUBiS \cite{rubis}; 
and exploring other program analysis techniques such as transaction chopping \cite{psi-chopping,chopping}, 
invariant checking \cite{cise,repliss} and program logics \cite{alonetogether}. 
We plan to develop tools to generate litmus tests for implementations and to analyse client programs.
