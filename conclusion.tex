\section{Conclusions and Related Work}
\label{sec:conclusions}

We have introduced a simple interleaving semantics for atomic
transactions, based on a global, centralised kv-stores and partial
client views. It is expressive enough to capture the anomalous
behaviour of many weak consistency models.  We have demonstrated that
our semantics can be used to both verify protocols of distributed databases
and prove the invariant properties of client programs. 



We have defined a large variety of consistency models for kv-stores
based on execution tests, and have shown these models  to be equivalent to
well-known declarative consistency models for dependency graphs and
abstract executions. We do not know of an appropriate consistency
model that we cannot express with our semantics, bearing in mind the
constraints that our transactions satisfy snapshot property and the last-write-wins policy. We have
identified a new consistency model, \emph{weak snapshot isolation},
which lies between $\PSI$ and $\SI$ and inherits many of the good properties of $\SI$. 
We have shown that examples are robust against \( \WSI \).
We would need to provide an implementation of 
this model to justify it in full in the future. 
We have proved the correctness of two real-world protocols employed by distributed 
databases: COPS \cite{cops}, a replicated database that satisfies causal consistency;
and Clock-SI \cite{clocksi}, a partitioned database that satisfies snapshot isolation.
We have also demonstrated the usefulness of our framework
for proving invariant properties: the robustness properties of the counters and banking libraries
against different consistency models; 
and the correctness (despite not robust) of the lock paradigm under \( \UA \).

In future, we aim to extend our framework to handle other 
weak consistency models. For example, we believe that, by introducing promises 
in the style of \cite{promises}, we can capture  consistency models such as \emph{Read Committed}. 
We also plan to validate further the usefulness of our framework by: 
verifying other well-known protocols of distributed databases \cite{eiger,wren,redblue,ramp};
exploring robustness results for OLTP workloads such as TPC-C~\cite{tpcc} and RUBiS \cite{rubis}; 
and exploring other program analysis techniques such as transaction chopping \cite{chopping,psi-chopping}, 
invariant checking \cite{cise,repliss} and program logics \cite{alonetogether}. 
We plan to develop tools to generate litmus tests for implementations and to analyse the invariant of client programs.

\mypar{Related Work} 
In the introduction, we highlight some general operational semantics for 
distributed transactional databases.
We discuss these semantics in more detail here.
There are graph-based, state-based and log-based operational semantics.
We then give some additional related work on program analysis. 

\citet{sureshConcur} propose an operational semantics for weak consistency based on abstract executions. 
Their semantics is parametric in the declarative definition of a consistency model. 
They introduce a tool for checking the robustness of transactional  libraries.
They focus on consistency models with snapshot property, but confusingly allow 
the interleaving of fine-grained operations between transactions. 
This results in an unnecessary explosion of the space of traces obtained by 
the program. In our semantics, the interleaving is between transactions.
\citet{op-semantics-c11-rar} develop an operational semantics for release-acquire fragment of
C11 memory model, a variant of causal consistency.
Their semantics is based on a variant of dependency graph where nodes and edges 
are tailored for C11 operations.
They introduce per-thread observations that are compatible for executing next operations;
this is similar to our views and execution tests.
We believe we can model release-acquire fragment of C11.

\citet{seebelieve} propose a state-based formal framework for weak consistency models 
that employs concepts  similar to execution tests and views, called commit tests and read states respectively.
They prove that consistency models previously thought to be different are in fact equivalent in their semantics. 
They capture a wide range of consistency models including read committed which we cannot do. 
In their semantics, one-step trace reduction is determined by the whole previous history of the trace. 
In contrast, our reduction step only depends on the current configuration (kv-store and view).
They do not consider program analysis. Their notion of commit tests and read states requires 
the knowledge of information that is not known to clients of the system, i.e. the total order of system changes that happened in the database 
prior to committing a transaction. For this reason, we believe that
their framework is not suitable for the development of techniques for analysing client programs. 
\citet{alonetogether} propose an operational semantics of SQL transactional programs 
under the consistency models given by the standard ANSI/SQL isolation levels \cite{si}.
In their  framework, transactions work on a local copy of the global state 
of the system, and the local effects of a transaction are committed to the  
system state when it terminates. Because state changes 
are made immediately available to all clients of a system, this model 
is not suitable to capture weak consistency models such as \(\PSI\) or \(\CC\). 
They introduce a program logic and prototype verification tool for reasoning 
about client programs. However, their definitions of consistency models 
are not validated against well-known formal definitions.

\citet{log-based-op} propose two log-based abstract operational semantics,
pessimistic one and optimistic one,
and \citet{push-pull} propose the push/pull semantics,
both of which are used to verify implementations of serilisability.
The machine state of the push/pull semantics consists of a centralised global log 
and partial client local logs.
This is similar to our kv-stores and views.
Both semantics can model key-value stores and also other ADTs.
However, their model is thought with serialisability in mind.
There is no evidence that they could be easily extended to tackle weaker consistency models.

Several other works have focused on program analysis for transactional systems. 
\citet{dias-tm} developed a separation logic for
the robustness of applications against \(\SI\). \citet{fekete-tods} derived 
a static analysis check for \(\SI\) based on dependency graph. 
\citet{giovanni_concur16}
developed a static analysis check for several consistency models with snapshot property. 
\citet{snapshot-isolation-robust-tool} developed a tool based on Lipton's reduction \cite{Lipton-reduction} 
for checking robustness against \( \SI \).
\citet{laws} investigated the relationship between abstract 
executions and dependency graphs from an algebraic perspective, and applied it to infer 
robustness checks for several consistency models. 

