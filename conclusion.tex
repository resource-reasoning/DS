\section{Conclusions and Related Work}
\label{sec:conclusions}
\sx{A cite here \cite{surech-session-guarantee} who mentions composition of 4 session guarantees. 
    \cite{principle-eventual-consistency} mentions causal consistency is combination of session guarantees,
    yet they only define 3 types of session guarantees.
}
We have introduced  an  operational semantics for 
transactional distributed databases, based on a global, centralised 
kv-store and client views. 
In our semantics is simple: transactions are executed atomically in an interleaving; on the other hand, it 
is  expressive enough to capture anomalous behaviours that are proper of weak consistency models. 
The semantics is parametric in the definition of an execution test, and we capture different consistency 
models by tweaking the execution test. We have captured a large variety of well-known consistency 
models, 
%we have captured a large variety of well-known consistency models 
%for distributed databases by tweaking the execution tests, 
and we have identified interesting consistency models that have not been 
implemented yet.
We have proved the correctness of two real-world protocols employed by distributed 
databases: COPS~\cite{cops}, a 
protocol for replicated databases that satisfies our definition of causal consistency;
and Clock-SI~\cite{clocksi}, a protocol for partitioned databases that satisfies our
definition of snapshot isolation. We have also demonstrated the usefulness of our framework
for program analysis, by proving the robustness of simple transactional 
libraries against different consistency models. 
%We believe that we are
%the first to show such  results for  databases which  allow
%the grouping of  transactions into sessions. 

In the future, we plan to extend our framework to handle a larger variety 
of weak consistency model: we believe that, by introducing promises 
in the style of \cite{promises}, we can model consistency models such 
as \emph{Read Committed}. We also plan to validate further the usefulness of our framework
by verifying the correctness of other well-known protocols employed by
distributed databases, such as Eiger~\cite{eiger}, Wren~\cite{wren} and
Red-Blue~\cite{redblue}; by exploring robustness results for OLTP
workloads  such as TPC-C~\cite{tpcc} and RUBiS~\cite{rubis}. 
and by exploring other program-analysis techniques such as
transactional chopping \cite{chopping,psi-chopping}, invariant checking 
\cite{cise,repliss} and program logics \cite{alonetogether}. 

\mypar{Related Work} 

\citeN{alonetogether} proposed an operational semantics of programs 
under different transactional consistency models \cite{alonetogether}, 
corresponding to the ANSI/SQL isolation levels \cite{si}.
In their  framework, transactions work on a local copy of the global state 
of the system, and the local effects of a transaction are pushed as 
system state changes upon commit. Because state changes 
are made immediately available to all clients of a system, this model 
is not suitable to capture weak consistency models such as \(\PSI\) or \(\CC\). 
Also, the definitions given are not validated against previously known 
formal definitions.

\citeN{sureshConcur} proposed an operational semantics for weak consistency 
based on abstract executions. Their semantics 
is parametric in the declarative definition of a consistency model. The authors 
focus on consistency models with atomic visibility, but the semantics allow 
the interleaving of fine-grained operations between different transactions: 
this results in an unnecessary explosion of the space of traces generated by 
the program. 
Because 
%in their model abstract executions are not equipped with sessions, they cannot 
%specify a large variety of consistency models as we do, such as the session guarantees. 
The authors also present a static analysis tool for determining the robustness of transactional 
libraries; however, their robustness results rely on the assumption that the underlying database 
is not equipped with sessions. Because the session order cannot be determined at compile time, 
their tool cannot be easily extended to a setting where sessions are provided by the database. 

\citeN{seebelieve} proposed a state-based formal framework for weak consistency models, 
that also employs notions similar to execution tests and views: commit tests and read states.
On one side, the authors capture 
a wide range of consistency models (in particular, the Read Committed isolation level), 
and they also exploit their formalism to prove that consistency models previously thought to be different 
are indeed equivalent.
However, they do not consider program analysis. Their notion of commit tests and read states requires 
the knowledge of information that is not known to clients of the system, i.e. the total order of system changes that happened in the database 
prior to committing a transaction; for this reason, we believe that their framework is not suitable for the development of program analysis techniques.

Several other works have focused on program analysis for transactional systems. 
%Concerning robustness, 
\citeN{dias-tm} developed a separation logic for
the robustness of applications under SI. \citeN{fekete-tods} derived 
a static analysis check for SI based on dependency graph. \citeN{giovanni_concur16}
developed a static analysis check for several consistency models with atomic visibility. 
\citeN{laws} investigated the relationship between abstract 
executions and dependency graphs from an algebraic perspective, and applied it to infer 
robustness checks for several consistency models. 

