\section{Operational Model}
\label{sec:model}

\pg{Remember: section 3 is all about technical definitions; section 2
 is  about definitions. This saves space and clarifies the purpose of
 each section.}

We give the technical definitions of our operational model: 
the global, centralised key-value stores; the partial client views;  and the
operational semantics. 

\mypar{Notation}
Given a set $\sort A$, we write $\sort A \ni a$ to denote that elements of $\sort A$ are ranged over by $a$ and its variants (\eg $a', a_1, \cdots$). 
Given a relation $\mathsf r \subseteq \sort A \times \sort A$,
we write $\mathsf r^?$, $\mathsf r^+$ and $\mathsf r^*$ for its reflexive, transitive and reflexive-transitive closures, respectively;
and write $a_1 \xrightarrow{\mathsf r} a_2$ for $(a_1, a_2) \in \mathsf r$.
\subsection{Key-value Stores}

Key-value stores are defined using client and transaction identifiers.
We assume a countably infinite set of \emph{client identifiers} $\Clients \ni \cl$. 
%$\Clients \defeq \Set{\cl, \cl',\cdots}$.
The set of \emph{transaction identifiers} is  
$\TxID \defeq  \Set{\txid_{0}} \uplus \Set{ \txid_{\cl}^{n} \mid \cl
  \in \Clients \wedge n \geq 0 }$, 
where  $\txid_0$ denotes  an \emph{initialisation transaction}, 
%and for each $n \in \mathbb{N}$, the $\txid_{\cl}^{n}$ identifies a transaction  committed by client $\cl$.
and $\txid_{\cl}^{n}$ identifies a transaction committed by client $\cl$. 
Elements of $\TxID$ are ranged over by
$\txid, \txid', \cdots$, and subsets by $\txidset, \txidset', \cdots$. 
Let $\TxID_{0} \defeq \TxID \setminus \{ \txid_0\}$. 
For each $n \in \mathbb{N}$, the $\txid_{\cl}^{n}$ identifies the $n$\textsuperscript{th} transaction  committed by $\cl$.
That is, the $n$ records  the \emph{session order}, $\PO$, associated with
the client: 
$\PO \defeq \Set{ (\txid, \txid') \mid \exsts{ \cl, n,m } \txid =
  \txid_{\cl}^{n} \wedge \txid' = \txid_{\cl}^{m} \wedge n < m}$.
%We write $\txid \xrightarrow{\PO} \txid'$ for
%$(\txid, \txid') \in \PO$, and $\txid \xrightarrow{\PO ?} \txid'$ for
%for its reflexive closure. 


\begin{definition}[Key-value stores]
\label{def:his_heap}
\label{def:mkvs}
Assume a countably infinite set of \emph{keys} $\Keys \ni \ke$, 
%$\Keys = \Set{\ke, \ke', \cdots}$
and a countably infinite set of  \emph{values} $\Val \ni \val$, 
% $\Val = \{\val, \val', \cdots\}$, 
including an \emph{initialisation value} $\val_0 $.
The set of \emph{versions}, $\Versions \ni \ver$, is: $\Versions \defeq \Val \times \TxID \times \powerset{\TxID_{0}}$. 
A \emph{key-value store}, abbreviated to kv-store,  is a function $\hh: \Keys \rightarrow \Versions^{\ast}$, 
where $\Versions^{\ast}$ is the free monoid on $\Versions$. 
The \emph{initial key-value store} is given by $\hh_0$, where 
$\hh_{0}(\ke)\defeq  (\val_0, \txid_0, \emptyset)$ for
all $\ke \in \Keys$.



\end{definition}


A \emph{version} $\ver {=} (\val, \txid, \txidset)$ comprises  a value $\val$
and meta-data about the transactions that accessed it: the \emph{writer} $\txid$ identifies the transaction that wrote $\ver$;
and the set of \emph{readers} $\txidset$ identifies  the  transactions
that read from  $\ver$.
We define $\valueOf(\ver) \defeq \val$,
$\WTx(\ver) \defeq \txid$ and $\RTx(\ver) \defeq \txidset$. 
%Elements of $\Versions$ are ranged over by
%$\ver, \ver',  \cdots$, 
Lists of versions (elements of $\Versions^{\ast}$) are ranged over by $\vilist, \vilist',\cdots$.
Given a kv-store $\hh$ and a transaction $\txid$, we write 
$\txid \in \hh$ when $\txid$ appears as either the writer or amongst the readers of a version in the range of $\hh$.
%%%%
%%%%I don't think this technical detail is needed, it's clear
%%:  $\txid \in \hh \defeq 
%\exists \ke, i. 0 \leq i < \lvert \hh(\ke) \rvert \wedge (\txid \in
%\RTx(\hh(\ke, i)) \cup \{\WTx(\hh(\ke, i))\})$.
We write  $\lvert \hh(\ke) \rvert$ for the length of $\hh(\ke)$, 
and write $\hh(\ke, i)$ for the $i$\textsuperscript{th} version (indexed from 0) of $\ke$ when defined, with $i \geq 0$.




We focus on key-value stores whose consistency model enforces the \emph{atomic visibility} of transactions~\cite{framework-concur}. 
This ensures that 
\begin{enumerate*}
	\item a transaction reads and writes at most one version for each key.
We also assume that 
	\item the list of versions for each key has an initial version 
carrying the initialisation value $\val_0$,  written by the designated initialisation transaction $\txid_0$ 
with an initial empty set of readers.
Finally, we assume that 
	\item the state of a kv-store is consistent with 
the session order of clients: a client cannot read a version of a key that has 
been installed by a future transaction within the same session;  and 
the order in which versions are installed by a single client must agree 
with its session order. 
\end{enumerate*}
When a kv-store satisfies these three conditions, we say that it is \emph{well-formed} (defined formally in \todo). 
Henceforth, we assume kv-stores are well-formed, and write  $\HisHeaps$ to denote 
the set of well-formed kv-stores.
%The formal definition of well-formedness can be found in \todo.
\azalea{
Give a formal definition of well-formedness in the appendix and move the below commented out bit to there. Reference it above.
%We focus on key-value stores whose consistency model enforces the
%\emph{atomic visibility} of transactions~\cite{framework-concur}.
%This amounts to requiring that a transaction reads and writes at most
%one version of each key: \textbf{(i)}
%$\fora{\ke, i, j} (o \leq i, j \leq \abs{ \hh(\ke) } \land
%\RTx(\hh(\ke, i)) \cap \RTx(\hh(\ke, j)) \neq \emptyset ) \implies i =
%j$, \textbf{(ii)}
%$\fora{\ke, i, j} (0 \leq i,j < \abs{ \hh(\ke) } \wedge \WTx(\hh(\ke,
%i)) = \WTx(\hh(\ke, j)) ) \implies i = j$.  We also assume that the
%list of versions for each key has an initial version carrying a
%default value $\val_0$, written by the designated initialisation
%transaction $\txid_0$: \textbf{(iii)}
%$\fora{\ke} \hh(\ke, 0) = (\val_0, \txid_0, \stub)$.  Finally, we
%assume that the state of a key-value store is consistent with the
%session order of clients: \textbf{(iv)}
%$\fora{ \ke, \cl, i,j, n, m} 0 \leq i < j < \abs{\hh(\ke)} \land
%\txid_{\cl}^{n} = \WTx(\hh(\ke,i)) {} \wedge \txid_{\cl}^{m} \in
%\Set{\WTx(\hh(\ke,j))} \cup \RTx(\hh(\ke, i)) \implies n < m $.  We
%say that kv-stores that satisfy the conditions
%\textbf{(i)}-\textbf{(iv)} above are \emph{well-formed}.  Henceforth,
%we always assume kv-stores are  well-formed, and  write 
%$\HisHeaps$ to denote the set of well-formed kv-stores.
}

%\pg{Above, I've got rid of the technical details to make it more
%  readable. I've also done a technical version, commented out, which
%  should probably go in the appendix.}




\subsection{Client Views}

Clients often have partial views of kv-stores, 
with different clients observing 
different versions of the same key.
%To keep track of the versions observed by clients,
%we introduce the notion of \emph{client views} (\cref{def:view}). 

\begin{definition}[Views]
\label{def:view}
\label{def:cuts}
\label{def:views}
\label{def:configuration}
A \emph{view} of a kv-store $\hh$ is a function
$\vi \in \Views(\hh) \defeq \Keys \to\powerset{\Nat}$ such that for all $i, i', \ke, \ke'$:
\begin{align}
    & \hspace*{-8pt}
%    \fora{ \ke } 
    0 \in \vi(\ke) 
    \wedge (i \in \vi(\ke) \Rightarrow i < \abs{ \hh(\ke) }) 
    \tag{wf}
    \label{eq:view.wf}\\
    & \hspace*{-8pt}
    \begin{array}{@{}l@{}}
%    \fora{ \ke_1,\ke_2, i_1, i_2} 
	i \in \vi(\ke)  
  	\land \WTx(\hh(\ke, i)) {=} \WTx(\hh(\ke', i'))  
  	\Rightarrow i' \in \vi(\ke')
    \end{array}
	\tag{atomic}
	\label{eq:view.atomic}
\end{align}
The \emph{initial view} is denoted by $\vi_{0}$, where $\vi_{0}(\ke) = \{0\}$ for all $\ke \in \Keys$. 
A \emph{configuration}, $\conf \in \Confs$,  is a pair $ (\hh, \viewFun)$
with $\hh \in \HisHeaps$ and
$\viewFun : \Clients \parfinfun \Views(\hh)$. A configuration is an 
\emph{initial} configuration when $\hh$ is the initial kv-store
$\hh_{0}$. 
\end{definition}
%


Given a configuration $(\hh, \viewFun)$, when the view of client 
$\cl$ is defined, \ie there exists $\vi = \viewFun(\cl)$, then for each key $\ke \in \Keys$, 
$\vi$ determines the sublist of versions in $\hh$ that client $\cl$ can observe. 
If $i,j \in \vi(\ke)$ and $i < j$, then $\cl$ knows about the values and 
transaction meta-data associated with versions $\hh(\ke, i)$ and  $\hh(\ke, j)$, 
and  knows that these versions are contained in $\hh$ with  $\hh(\ke, j)$ more 
up-to-date than $\hh(\ke, i)$. 
Let $\Views \eqdef \bigcup_{\hh \in \HisHeaps} \Views(\hh)$ be the set of all views. 
Given a kv-store $\hh$ and two views $\vi, \vi' \in \Views(\hh)$, 
we define: $\vi \viewleq \vi' \iffdef \for{\ke \in \dom(\hh)} \vi(k) \subseteq \vi'(\ke)$.

Given a configuration $(\hh, \viewFun)$ and a client $\cl$ such that $\viewFun(\cl)$ is
defined, it is possible to define a \emph{snapshot} of the
view $\viewFun  (\cl)$, which identifies the last write of a client
view. This definition assumes that the database satisfies the \emph{last write wins}
resolution policy~\cite{}. It  is straightforward to generalise our work
to other resolution policies~\cite{.}.

%\azalea{$\ss$ has negative (Nazi) connotations. I would rename it if possible.}
\begin{definition}[Snapshots]
\label{def:heaps}
\label{def:snapshot}
A \emph{snapshot}, \( \ss \in \Snapshots  \defeq \Keys \to
\Val\),  is a function  from keys to values.
Given $\hh \in \HisHeaps$ and $\vi \in \Views(\hh)$, the \emph{snapshot} of $\vi$ in 
$\hh$ is defined by  $\snapshot(\hh, \vi) \defeq \lambda \ke \ldotp \valueOf(\hh(\ke, \max_{<}(\vi(\ke)))$, 
where $\max_{<}(\vi(\ke))$ is the maximum element in $\vi(\ke)$ with respect to the natural 
order $<$ over $\mathbb{N}$.
\end{definition}
Given a kv-store $\hh$, a key $\ke$ and a view $\vi$, 
%we abuse
%notation, writing 
we write 
$\hh(\ke, \vi)$ as a shorthand for 
$\hh(\ke, \max_{<}(\vi(\ke))$. Thus, $\snapshot(\hh, \vi) = \lambda \ke \ldotp \valueOf(\hh(\ke, \vi))$. 

\begin{remark}
The client view describes the partial history of the database observed by the client. 
The client snapshot describes the most up-to-date state of the database known to the client (based on its view). 
It is not possible to work with the snapshot alone. 
The client view is essential to express several standard consistency models (see~\cref{subsec:cm_examples}). 
\end{remark}


\subsection{Operational Semantics}

\noindent {\bf Programming Language} A 
\emph{program} \( \prog \) comprises a finite number of clients,
where each client is associated with a unique identifier \( \thid \in \ThreadID \), 
and executes a sequential \emph{command} $\cmd$, given by the following grammar:
\begin{align*}
\cmd & ::=  
\pskip \mid 
\cmdpri \mid  
\ptrans{\trans} \mid 
\cmd \pseq \cmd \mid 
\cmd \pchoice \cmd \mid 
\cmd \prepeat  
\\
\cmdpri & ::=  
\pass{\txvar}{\expr} \mid 
\passume{\expr} 
\\
\trans & ::=
\pskip \mid
\transpri \mid 
\trans \pseq \trans \mid
\trans \pchoice \trans \mid
\trans\prepeat    
\\
\transpri & ::= 
\cmdpri \mid
\pderef{\txvar}{\expr} \mid
\pmutate{\expr}{\expr} 
\end{align*} 
%

\pg{Why are you using $a$ for variable. Later you use $x$. Just be
  consistent. I'm happy with $x$.} 
Sequential commands  comprise $\pskip$,  primitive commands $\cmdpri
$, atomic transactions
$\ptrans{\trans}$,  and standard
compound commands. 
Primitive commands--the variable assignment
$\pass{\txvar}{\expr}$ and the assume statement $\passume{\expr}$
used to encode conditionals--are used for computations based on 
client-local variables 
and can hence be invoked without restriction. The 
transaction commands, $T$, 
comprise primitive commands, 
primitive transactional commands $\transpri$,  and standard compound commands. 
Primitive transactional commands---the lookup $\pderef{\txvar}{\expr}$ and the mutation 
$\pmutate{\expr}{\expr}$---are used for reading and writing to kv-stores respectively and 
can be invoked  as part of an atomic transaction $\ptrans{\trans}$.

A {\em program} is a finite partial function from client identifiers to sequential
commands.
%%%%I don't think this is needed. 
%$\prog = \Set{\thid_{1} \mapsto \cmd_{1}, \dots, \thid_{n} \mapsto \cmd_{n} $.
For clarity, we often write \( \cmd_{1}\ppar \dots \ppar \cmd_{n}\) as syntactic sugar 
for a program \( \prog \) with $n$ clients associated with identifiers
$\thid_1 \dots \thid_n$, where each client $\thid_i$ executes
$\cmd_i$. Each client is associated with its own client-local stack,  \emph{stack} 
$\stk \in \Stacks \defeq \Vars \to \Val$,  mapping program variables
ranged over by $\pvar{x}, \pvar{y}, \cdots$
to values. 
We assume a language of expressions built from values
and program variables, 
\( \expr ::= \val \mid \var \mid \expr + \expr \mid \dots  \).
The evaluation $\evalE{\expr}$ of  expression $\expr$ is parametric in
the client-local stack:
\begin{gather*}
\evalE{\val} \defeq
\val
\quad
\evalE{\var} \defeq
\stk(\var)
\quad
\evalE{\expr_{1} + \expr_{2}} \defeq
\evalE{\expr_{1}} + \evalE{\expr_{2}}
\quad
\dots
\end{gather*}
\pg{In above, you could define stack  $s_i$ for client $cl_i$.}

\noindent {\bf Transaction Semantics}  
In our framework, transactions are executed atomically. 
Roughly speaking, given a configuration $\conf = (\hh, \viewFun)$, 
when a client $\cl$ executes some transactional code $\ptrans{\trans}$, 
 it performs the following steps: 
it constructs a snapshot of $\hh$ from the view $\viewFun(\cl)$ that 
the client has over $\hh$;  it executes the code $\trans$ in isolation, using the 
snapshot constructed in the previous step as the initial, local state
of the client, and it determines the observable 
effects that such an execution has on the key-value store; and  it incorporates 
the effects of executing the code $\trans$ in the initial state determined by the snapshot into 
the key-value store.

We capture  the behaviour of a  transaction,
$\trans$,
by describing how it updates the client stack and snapshot of 
the kv-store and by  identifying   its {\em fingerprint set}.
A fingerprint of a transaction
is a set of read and write operations that describes  the 
reads of a snapshot of the kv-store taken  at the beginning of
the invocation of the transaction, and the writes to be
commited to the kv-store  as long as certain
consistency conditions are met. A transaction can have more than one fingerprint due to
non-determinism. 




\pg{Above  might need to dovetail with what we say in section 2. 
Maybe give an example that relates to
section 2 that shows intuitively how fingerprint works, maybnot not
and instead jsut refer to section 2}




\begin{definition}[Read-Write Set]
\label{beebop}
Let 
%The set of \emph{operations} is
$\Ops \defeq \{(l, \ke, \val) \mid$ $ l \in \Set{\otW, \otR} \land \ke \in \Keys \wedge \val \in \Val \}$ 
be a set of operations. 
A \emph{Read-Write Set} $\opset$ is a subset of operations, $\opset \subseteq \Ops$,
such that, for all $\ke \in \Keys$ and \( l  \in \Set{\otW, \otR} \),
if $(l, \ke, \val_1), (l, \ke, \val_2) \in \opset$ then $\val_1 = \val_2$.
\end{definition}
Note that we have placed a constraint that, per key, a read-write set
contains at most one read operation and at most one write operation.
This reflects the fact that we work with transactions that are
atomically visible~\cite{laws} meaning here that the reads are taken
from a snapshot of the kv-store and, since clients observe either none
or all effects of a transaction, only the last writes are 
committed.

We provide an operational description of the behaviour of a transaction command, $T$,
starting 
from an intial client stack
$s$, a snapshot $ss$  and the empty read-write set. 
%Intuitively,  a fingerprint of a transaction  records, for each key $\ke$,
%the first value a transaction reads (before a subsequent write) for $\ke$, 
%and the last value the transaction writes for $\ke$.
First, we define a transition system which describes how the stack and snapshot are updated  using the
primitive transaction commands.


\begin{definition}
\label{foo}
The transition system, $\toLTS{\transpri}\; \subseteq (\Stacks \times \Heaps) \times (\Stacks \times \Heaps)$, 
is defined by:
\[
\begin{rclarray}
(\stk, \h)  & \toLTS{\passign{\var}{\expr}}          & (\stk\rmto{\var}{\evalE{\expr}}, \h)                  \\
(\stk, \h)  & \toLTS{\passume{\expr}}                & (\stk, \h) \text{ where } \evalE{\expr} \neq 0        \\
(\stk, \h)  
& \toLTS{ \pderef{\var}{\expr} } 
& (\stk\rmto{\var}{\h(\evalE{\expr})}, \h) 
\\
(\stk, \h)
& \toLTS{\pmutate{\expr_{1}}{\expr_{2}}  }
& (\stk, \h\rmto{\evalE{\expr_{1}}}{\evalE{\expr_{2}}}) \\
\end{rclarray}                                                                                               
\]
\end{definition}
%Second, we define the basic actions of the primitive commands and
%primitive transactional commands using  function
% I don't like this name fingerprint function, it's not, the action
% might not be in the fingerprint. we only begin to get the
% fingerprint intruition with the composition operator. 
%we define a \emph{fingerprint function}, 
%$\func{fp}{\stub} \; : \Stacks \times \Heaps \times \transpri \rightarrow \Ops \cup \{\varepsilon\}$:
%

%Note that  the primitive commands are associated with the empty operation $\varepsilon$,
%as they only access the local stack and do not access the kv-store.

\noindent Second, we provide a 
function, $\mathsf{op}$,  connecting the  primitive transaction commands to the
basic read and write operations given in Definition~\ref{beebop}: 
\[
\begin{array}{rcl @{\quad} rcl}
\func{op}{\stk, \h, \passign{\var}{\expr}}          & \defeq & \emptyop                                     \\
\func{op}{\stk, \h, \passume{\expr}}                & \defeq & \emptyop                                     \\
\func{op}{\stk, \h,  \pderef{\var}{\expr}}           & \defeq & (\etR, \evalE{\expr}, \h(\evalE{\expr}))    \\
\func{op}{\stk,  \h, \pmutate{\expr_{1}}{\expr_{2}}} & \defeq & (\etW, \evalE{\expr_{1}}, \evalE{\expr_{2}})\\
\end{array}
\]
The  empty operation $\emptyop$ is used for the primitive commands which do not
contribute to the fingerprint.

Third, we define an operator,
$\addO  : {\cal P}(\Ops)\times \Ops\rightarrow {\cal P}(\Ops)$,  which
adds a  basic
operation to  a read-write set and ignores the empty operation: 
%For instance, when executing $ \ptrans{\trans}$ with $\trans \eqdef \transpri^1; \cdots ; \transpri^n$,
%the effect of each $\transpri^i$ is calculated via the $\op_i = \func{fp}{-, -, \transpri^i}$ function, 
%with the overall fingerprint given as the $\addO$-composition of the constituent effects: $\op_1 \addO \cdots \addO \op_n$. 
\begin{align*}
    \opset \addO (\etR, \addr, \val)  
    & \defeq
    \begin{cases}
        \opset \cup \{(\etR, \addr, \val)\} & \text{if } \for{l, v'} (l, \addr, v') \notin \opset \\
        \opset &  \text{otherwise} \\
    \end{cases}  \\
    \opset \addO (\etW, \addr, \val) 
    & \defeq 
    \left( \opset \setminus \setcomp{(\etW, \addr, v')}{v' \in \Val} \right) 
    \cup \Set{(\etW, \addr, \val)}  \\
    \opset \addO \emptyop  & \defeq  \opset  \\
\end{align*}
A read gets added to the read-write set if there is no read or
write there already, thus only  recording the reads of the
original snapshot of the kv-store at invocation. 
A write always updates the read-write set, corresponding to a
transaction only commiting the final writes. 


%
\begin{figure*}[!t]
\hrulefill
\begin{mathpar}
    \inferrule[\rl{TPrimitive}]{%
        (\stk, \h) \toLTS{\transpri} (\stk', \h')
        \\ \op = \func{op}{\stk, \h, \transpri}
    }{%
        (\stk, \h, \opset) , \transpri \ \toL \  (\stk', \h', \opset \addO \op) , \pskip 
    }
    \\
    \inferrule[\rl{TChoice}]{
        i \in \Set{1,2}
    }{%
        (\stk, \h, \opset) , \trans_{1} \pchoice \trans_{2} \ \toL \  (\stk, \h, \opset) , \trans_{i}
    }
    \and
    \inferrule[\rl{TIter}]{ }{%
        (\stk, \h, \opset),  \trans\prepeat \ \toL \  (\stk, \h, \opset), \pskip \pchoice (\trans \pseq \trans\prepeat)
    } 
    \and
    \inferrule[\rl{TSeqSkip}]{ }{%
        (\stk, \h, \opset), \pskip \pseq \trans \ \toL \  (\stk, \h, \opset), \trans
    }
    \and
    \inferrule[\rl{TSeq}]{%
        (\stk, \h, \opset), \trans_{1} \ \toL \  (\stk', \h', \opset'), \trans_{1}'
    }{%
        (\stk, \h, \opset), \trans_{1} \pseq \trans_{2} \ \toL \  (\stk', \h', \opset'), \trans_{1}' \pseq \trans_{2}
    }
\end{mathpar}
\hrulefill
\caption{Transaction semantics.}
\label{fig:semantics-trans}
\end{figure*}

Finally, we have all the ingredients to describe the behaviour of  a
transaction command.  \cref{fig:semantics-trans} provides the
one-step transaction semantics: the  interesting rule is the \rl{TPrimitive}
rule which describes how a primitive transactional command updates
the client stack, the snapshot and the read-write set; the other
rules are standard rules for compound commands. 



\begin{definition}[Fingerprint Set]
Given client stack $s$
and snapshot $ss$, the \emph{fingerprint set } of $T$ is given by 
\[F \defeq
\{\opset : (\stk, \h, \emptyset), T \toL^* (\stk', \h', \opset),
\pskip \}
\]
 where $\toL^*$ is the transitive closure of $\toL$ given
 in~\cref{fig:semantics-trans}.  A set $\opset$ in $F$ is called a
 \emph{fingerprint} of $T$. 
\end{definition}
\noindent It is straightforward to prove that the  fingerprints of $T$ contain at most one read operation per key and
one write operation per key. \\

\pg{Below, there were some minor inconsistences. Needs checking.I've
  made lots of changes. It's a bit difficult to assess by laptop,
  really needs a paper copy. Hope ok.}

\noindent {\bf Operational Semantics.} We give the operational
semantics of commands and programs. The command semantics describes
transitions of the form
$(\hh, \vi, \stk), \cmd \ \toT{\lambda}_{\ET} \ (\hh', \vi', \stk') ,
\cmd'$, stating that, given the kv-store $\hh$, view $\vi$ and stack
$\stk$, a client $\cl$ may execute command $\cmd$ for one step, updating the
kv-store to $\hh'$, the stack to $\stk'$, and the command to its
continuation $\cmd'$.  The label $\lambda$ is either of the form
$(\cl, \iota)$ denoting that client $\cl$ used a primitive command
that did not require access to the kv-store, or
$(\cl, \vi'', \opset)$ denoting that client $\cl$ commited an atomic
transaction with fingerprint $\opset$ under the view $\vi''$.
Transitions are parametric in the choice of \emph{execution test},
$\ET$,  used to capture well-known consisitency models studied in the literature
(Sections~\ref{4}
and~\ref{5}). 

\pg{Sorry, I think it's better to revert back to the judgement $cl
  \vdash (\hh, \vi, \stk), \cmd \ \toT{\lambda}_{\ET} \ (\hh', \vi', \stk') ,
\cmd'$, please change everywhere. Why is the label needed? It's not clear.}



\begin{figure*}[t]
\hrulefill
\[
    \inferrule[\rl{CPrimitive}]{
        (\stk, \h)  \toLTS{\cmdpri} (\stk', \h)
        \qquad \h = \clpsHH{\hh,\vi}
    }{%
        ( \hh, \vi, \stk ) , \cmdpri \ \toT{(\cl,\iota)}_{\como} \  ( \hh, \vi, \stk' ) , \pskip
    }
\]
\[
    \inferrule[\rl{CAtomicTrans}]{%
        \vi \orderVI  \vi''
        \qquad \h = \clpsHH{\hh,\vi''}
        \qquad \txid \in \nextTxId(\cl, \hh)
        \\\\ (\stk, \h, \emptyset), \trans \ \toL^{*} \  (\stk', \stub,  \opset) , \pskip
        \\ \ET \vdash (\hh, \vi'') \triangleright \opset : \vi'
    }{%
        ( \hh, \vi, \stk ), \ptrans{\trans} \ \toT{(\cl, \vi'', \opset)}_{\ET} \ (\updateKV(\hh, \vi, \txid, \opset),\vi', \stk' ) , \pskip
    }
\]
%\hrulefill
\[
    \inferrule[\rl{PProg}]{%
        ( \mkvs, \vi, \thdenv(\thid) ) , \prog(\thid), \
        \toT{\lambda}_{\ET} \  ( \mkvs', \vi', \stk' ) , \cmd'  \qquad 
\viewFun (\thid) = \vi
    }{%
        (\mkvs, \viewFun, \thdenv ), \prog  \ \toT{\lambda}_{\ET} \  ( \mkvs', \viewFun, \thdenv\rmto{\thid}{\stk'} ) , \prog\rmto{\thid}{\cmd'} ) 
    }
\]
\hrulefill
\caption{Rules for primitive  commands, atomic
  transactions and programs.}
\label{fig:semantics}
\end{figure*}



Figure~\ref{.} contains the rules for primitive commands and atomic
transactions.  The rules for the compound commands are straightforward
and given in the appendix.
The rule for primtive commands, $\rl{CPrimitive}$,  uses the transition
system
 $\toLTS{\transpri}$, given in  Definition~\ref{foo}, applied to the primitive
commands which just affect the client stack. The rule for atomic
transactions, \rl{CAtomicTrans}, describes the execution of an atomic 
transaction under execution test $\ET$.  The first premise
states that the current view $\vi$ of the executing command maybe advanced to a newer atomic view $\vi''$ (see \cref{def:views}). 
The semantics only allows to advance the view to later versions, which corresponds to \emph{monotonic read} \cite{.......}.
Given the new view $\vi''$, the transaction proceeds by obtaining a snapshot $\sn$ of the kv-store $\hh$, and executing $\trans$ locally to completion ($\pskip$), updating the stack to $\stack'$, while accumulating the fingerprint $\opset$. Note that the resulting snapshot is ignored (denoted by $\stub$) as the effect of the transaction is recorded in the fingerprint $\opset$. 
%

The transaction is now ready to commit. The rule picks a fresh
transaction identifier using $\txid \in \nextTxId(\cl, \hh)$, judges
whether the commit is permitted by $\ET$ using the relation
$\ET \vdash (\hh, \vi'') \triangleright \opset : \vi'$, and updates
the kv-store using $\updKV{\hh, \vi, \txid, \opset}$.  The
set $\nextTxId(\cl, \hh)$ provides the  transactions identifiers
associated with $\cl$ that are fresh for  $\hh$:
$
\nextTxId(\cl, \hh) \defeq \Setcon{\txid_{\cl}^{n}}{\fora{m}
  \txid_{\cl}^{m} \in \hh \Rightarrow m < n }.
$
 By construction, all
elements of $\nextTxId(\cl, \hh)$ are greater, with respect to session
order $\xrightarrow{\PO}$,  than any transaction identifier previously
used by $\cl$. The judgement $\ET \vdash (\hh, \vi'') \triangleright
\opset : \vi'$
states that the fingerprint $\opset$ is compatible with kv-store $\hh$
and client view $\vi''$, and the resulting view  is $\vi'$. In
Section~\ref{4}, 
we give many examples of such relations.
Finally, we need to address how a fingerprint  $\opset$ of a
transaction executed
by client $\cl$  with view $\vi$  update a  key-value store $\hh$. 




Having selected a suitable transaction identifier $t$, we push the
fingerprint $\opset$ 
into $\hh$ as follows: for each read operation $(\otR, \ke, \_) \in \opset$, we add $t$ 
to the set of readers of the last version of $\ke$ that is included in the view $\vi$ of the client; 
for each write operation $(\otW, \ke, \val)$, we append a new version $(\val, t, \emptyset)$ 
to the tail of $\hh(\ke)$.
In the definition below, we use $\lcat$ to denote the concatenation of two lists; 
if $\vilist = \ver_0, \cdots, \ver_n$ and $i=0,\cdots,n$, 
$\vilist\rmto{i}{\ver}$ denotes the updated list 
$\vilist' = \ver_0, \cdots, \ver_{i-1}, \ver, \ver_{i+1}, \cdots,
\ver_{n}$. 

\begin{definition}[Transaction Update]
\label{eq:updatekv}
\label{def:updatekv}
Let $\hh \in \HisHeaps, \vi \in \Views(\hh)$, $t \in
\TxID_{0}$ and 
$\opset \subseteq \powerset{\Ops} $.  
The transaction update function,  $\updateKV(\hh, \vi, \txid, \opset) $,  is
defined by:
\begin{align*}         
    & \updateKV(\hh, \vi, \txid, \emptyset) \defeq  \hh \\
    & \updateKV(\hh, \vi, \txid, \opset \uplus \Set{(\otR, \ke, \val)}) \\
    & \quad \defeq 
    \begin{array}[t]{@{}l}
        \texttt{let} \ (\val, \txid', \txidset) = \hh(\ke, \max_{<}(\vi(\ke))), \\
        \vilist = \hh(\ke)\rmto{\max_{<}(\vi(\ke))}{(\val, \txid', \txidset \uplus \{ t \})}\\
        \quad \texttt{in} \ \updateKV(\hh\rmto{\ke}{\vilist}, \vi, \txid, \opset)
    \end{array} \\
    & \updateKV(\hh, \vi, \txid, \opset \uplus \Set{(\otW, \ke, \val)} ) \\
    & \quad \defeq 
    \begin{array}[t]{@{}l}
        \texttt{let } \hh' = \hh\rmto{\ke}{ ( \hh(\ke) \lcat (\val, \txid, \emptyset) ) } \\
        \quad \texttt{in } \updateKV(\hh', \vi, \txid, \opset)
    \end{array}  
\end{align*}
For client $\cl$, given $\hh \in \HisHeaps$, $\vi \in \Views(\hh)$ and 
$\opset \subseteq \powerset{\Ops} $, the transaction update set 
for $\cl$ is 
\begin{align*}
&\updateKV(\hh, \vi, \cl, \opset)  \\
 & \quad \defeq \Setcon{\updateKV(\hh, \vi, \txid, \opset)}{\txid \in
    \nextTxId(\hh, \cl)}
\end{align*}
\end{definition}

\pg{In above, can the layout be better? I would like to put the transaction identifier $t$ at the
  end of the arguments, since this is what's happening in the
  premises. I also don't like $\cl$ where it is.}

\pg{There is a disconnect. $\ET \vdash (\hh, \vi'') \triangleright
  \opset : \vi'$ says $\vi'$ makes sense with respect to $\hh, \vi'',
  \opset$. Nothing says that $\vi'$ makes sense with $\updateKV(\hh,
  \vi, \txid, \opset)$. I always find myself concerned about $\ET \vdash (\hh, \vi'') \triangleright
  \opset : \vi'$ because $\vi'$ is a view of an updated store which is
not part of the $\ET$ relation. I would add it.}


Note that,  under the assumption that read-write set  $\opset$ contains at most one read and one write 
operation per key and the identifier is fresh for $\hh$, 
the transaction update function and the transaction update set for
$\cl$ are well-defined. 


\pg{This is out of place here. It could go after ET-traces.Two read-write sets $\opset_1, \opset_2$ are \emph{conflict-free} if 
$\forall \ke \in \Keys.\; (\otW,\ke, \_) \in \opset_1 \implies \forall
\val \in \Val.\; (\otW, \ke, \val) \notin \opset_2$, and vice versa.
The commitment of two  conflict-free read-write sets 
$\opset_1, \opset_2$ into a key-value store $\hh$ does not depend on the order in which the commits 
are performed. }

Figure~\ref{.} also contains the rule for programs. 
A \emph{client environment}, $\thdenv \in \ThdEnv$, is a function  from client identifiers to stacks. 
We assume that the domain of client environments  is the same as the
the domain of the program throughout the execution: 
that is,
$\dom(\thdenv) = \dom(\prog)$.
Program transitions are simply defined in terms of the transitions of
their constituent client commands. 
This  yields a  standard interleaving semantics for concurrent
programs:
that is, 
a client performs a reduction in an atomic step without
affecting other clients. 
\pg{ I
  don't like the notation $\thdenv $, nothing else looks like this and
it's not as important as the notation suggests.}