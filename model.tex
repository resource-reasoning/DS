\section{Operational Model}
\label{sec:model}

We define an interleaving operational semantics for atomic transactions over
global, centralised kv-stores and partial client views. 

\subsection{Key-Value Stores and Client Views}
\label{subsec:kvstores}
\label{sec:mkvs-view}
Our global, centralised key-value stores (kv-store) and partial client views
provide the abstract machine states for our operational semantics. A
kv-store comprises key-indexed lists of versions which record
the history of the key with values and meta-data of the
transactions that accessed it: the writer and readers.

We assume a countably infinite set of \emph{client identifiers}\footnote{ We use the notation
 $\sort A \ni a$ to denote that elements of $\sort A$ are ranged over
  by $a$ and its variants such as $a', a_1, \cdots$.},
$\Clients \ni \cl$.
The set of \emph{transaction identifiers}, $\TxID \ni t$, 
 is defined by
$\TxID \defeq  \Set{\txid_{0}} \uplus \Set{ \txid_{\cl}^{n} \mid \cl
  \in \Clients \land n \geq 0 }$, 
where  $\txid_0$ denotes  the  \emph{initialisation transaction}
and $\txid_{\cl}^{n}$ identifies a transaction committed by client
$\cl$ with $n$  determining  the client session order: that is, $\SO \defeq \Set{ (\txid, \txid') \mid \exsts{ \cl, n,m } \txid =
\txid_{\cl}^{n} \land \txid' = \txid_{\cl}^{m} \land n < m}$.
Subsets of $\TxID$  are ranged over by $\txidset, \txidset', \cdots$. 
We let $\TxID_{0} \defeq \TxID \setminus \Set{\txid_0}$. 

\spaceshrink{-3pt}
\begin{definition}[Kv-stores]
\label{def:his_heap}
\label{def:mkvs}
Assume a countably infinite set of \emph{keys}, $\Keys \ni \key$, 
and a countably infinite set of  \emph{values}, $\Val \ni \val$, 
which includes the keys and an \emph{initialisation value} $\val_0$.
The set of \emph{versions}, $\Versions \ni \ver$, is defined by $\Versions \defeq \Val \times \TxID \times \pset{\TxID_{0}}$. 
A \emph{kv-store} 
is a function $\mkvs: \Keys \to \func{List}[\Versions]$, 
where $\func{List}[\Versions] \ni \vilist$ is the set of lists of versions. %$\Versions$. 
\end{definition}
\spaceshrink{-3pt}

Each version has the form 
$\ver {=} (\val, \txid, \txidset)$, where $\val$ is
a value, the \emph{writer} $\txid$ identifies the transaction that
wrote $\val$,  and the \emph{reader set} $\txidset$ identifies the
transactions that read $\val$. We use the notation 
$\valueOf(\ver)$,
$\wtOf(\ver)$ and $\rsOf(\ver)$ to project
the individual components of $\ver$.
Given a kv-store $\mkvs$ and a transaction $\txid$, we write 
$\txid \in \mkvs$ if $\txid$ is either the writer or 
one of the readers of a version included in $\mkvs$, 
write $\lvert \mkvs(\key) \rvert$ for the length of the version
list $\mkvs(\key)$,
and write $\mkvs(\key, i)$ for the $i$\textsuperscript{th} version of $\key$.
%with $0 \leq i < \lvert \mkvs(\key) \rvert$.

We assume that the version list for each key has an initialisation version 
carrying the initialisation value $\val_0$,  written by the 
initialisation transaction $\txid_0$ with an initial empty reader set.
We focus on kv-stores whose consistency model satisfies the
\emph{snapshot property}, ensuring that
a transaction reads and writes at most one version for each key.

\spaceshrink{-15pt}
{\displaymathfont
\[
\fora{\key, i, j} 
\rsOf(\kvs(\key, i)) \cap \rsOf(\kvs(\key, j)) \neq \emptyset \lor
\wtOf(\kvs(\key, i)) = \wtOf(\kvs(\key, j))
\implies i = j  
\]
\normalsize}
\spaceshrink{-15pt}

\noindent 
This is a normal assumption for distributed databases, \eg in \cite{ramp,rola,cops,wren,redblue,PSI,NMSI,gdur,clocksi,distrsi}.
Finally, we assume that the kv-store agrees with the session order of clients: 
a client cannot read a
version of a key that has been written by a future transaction within
the same session, and the order in which versions are written by a
client must agree with its session order. 

\spaceshrink{-15pt}
{\displaymathfont
\[
\fora{ \key, \cl, i,j, n, m} 
i < j
\land \txid_{\cl}^{n} = \wtOf(\mkvs(\key,i)) \land \txid_{\cl}^{m} \in
\Set{\wtOf(\mkvs(\key,j))} \cup \rsOf(\mkvs(\key, i)) \implies n < m
\]
\normalsize}
\spaceshrink{-15pt}

\noindent 
A kv-store is
\emph{well-formed} if it satisfies these three assumptions.
Henceforth, we assume kv-stores are well-formed,
and let $\MKVSs$ denote the set of well-formed kv-stores.

A global kv-store provides an abstract centralised description of
updates associated with distributed kv-stores that is \emph{complete} in 
that no update has been lost in the description. By contrast, in
both replicated and partitioned distributed databases, a client may
have incomplete information about updates distributed between
machines.  We model this incomplete information by
defining a {\em view} of the kv-store which provides a {\em
  partial} record of the updates observed by a client. We require that a client view be {\em atomic} in that it can
see either all or none of the updates of a transaction.

\spaceshrink{-3pt}
\begin{definition}[Views]
\label{def:view}
\label{def:cuts}
\label{def:views}
A \emph{view} of a kv-store $\mkvs \in \MKVSs$ is a function
$\vi \in \Views(\mkvs) \defeq \Keys \to\pset{\Nat}$ such that, for all $i, i', \key, \key'$:

\spaceshrink{-12pt}
{%
\displaymathfont
\begin{align*}
    & 
    0 \in \vi(\key) 
    \land (i \in \vi(\key) \implies 0 \leq i < \abs{ \mkvs(\key) }) 
    \tag{well-formed}
    \label{eq:view.wf}\\
    & 
    \begin{array}{@{}l@{}}
	i \in \vi(\key)  
  	\land \wtOf(\mkvs(\key, i)) {=} \wtOf(\mkvs(\key', i'))  
  	\implies i' \in \vi(\key')
    \end{array}
	\tag{atomic}
	\label{eq:view.atomic}
\end{align*}
}%
\spaceshrink{-12pt}

Given two views $\vi, \vi' \in \Views(\mkvs)$, 
the order between them is defined by $\vi \viewleq \vi' \defiff \fora{\key \in \dom(\mkvs)} \vi(k) \subseteq \vi'(\key)$.
The set of views is $\Views \defeq \bigcup_{\mkvs \in \MKVSs} \Views(\mkvs)$.
The \emph{initial view}, $\vi_{0}$,  is defined by
$\vi_{0}(\key) = \Set{0}$ for every $\key \in \Keys$. 
\end{definition}
\spaceshrink{-3pt}

Our operational semantics updates {\em configurations},  which are pairs
comprising a kv-store and a function describing the
views of a finite set of clients. 

\spaceshrink{-3pt}
\begin{definition}[Configurations]
\label{def:configuration}
A \emph{configuration}, $\conf \in \Confs $,  is a pair $ (\mkvs, \vienv)$
with $\mkvs \in \MKVSs$ and
$\vienv : \Clients \parfinfun \Views(\mkvs)$. 
The set of \emph{initial configurations}, $\Confs_0 \subseteq \Confs$,
contains configurations of the form $ (\mkvs_0, \vienv_0)$, where
$\mkvs_0$ is the initial kv-store defined by
$\mkvs_{0}(\key)\defeq  (\val_0, \txid_0, \emptyset)$ for
all $\key \in \Keys$. 
\end{definition}
\spaceshrink{-3pt}


Given a configuration $(\mkvs, \vienv)$ and a client $\cl$, 
if $\vi = \vienv(\cl)$ is defined then, for each $k$,  the
configuration determines the sub-list of versions in $\mkvs$ that $\cl$ sees.
If $i,j \in \vi(\key)$ and $i < j$, then $\cl$ sees the values 
carried by versions $\mkvs(\key, i)$ and  $\mkvs(\key, j)$, 
and it also sees that the version $\mkvs(\key, j)$ is more up-to-date than $\mkvs(\key, i)$. 
It is therefore possible to associate a \emph{snapshot} with the view $\vi$, 
which identifies, for each key $\key$, the last version included in the view. 
This definition assumes that the database satisfies the \emph{last-write-wins}
resolution policy, employed by many distributed key-value stores.
However, our formalism can be adapted straightforwardly  to capture other resolution policies. 

\spaceshrink{-3pt}
\begin{definition}[View Snapshots]
\label{def:snapshot}
Given $\mkvs \in \MKVSs$ and $\vi \in \Views(\mkvs)$, the \emph{snapshot} of $\vi$ in 
$\mkvs$ is a function, $\snapshot[\mkvs, \vi] : \Keys \to
\Val$,   defined by $\snapshot[\mkvs, \vi] \defeq \lambda \key \ldotp \valueOf(\mkvs(\key, \max_{<}(\vi(\key))))$, 
where $\max_{<}(\vi(\key))$ is the maximum element in $\vi(\key)$ with respect to the natural 
order $<$ over $\mathbb{N}$.
\end{definition}
\spaceshrink{-3pt}

\subsection{Operational Semantics}

\mypar{Programming Language}
A \emph{program} \( \prog \) comprises a finite number of clients,
where each client is associated with a unique identifier \( \cl \in \Clients \), 
and executes a sequential \emph{command} $\cmd$, given by the following grammar:

\spaceshrink{-5pt}
{%
\[
\displaymathfont
\begin{aligned}
\cmd & ::=  
\pskip \!\mid\!
\cmdpri \!\mid\!  
\ptrans{\trans} \!\mid\! 
\cmd \pseq \cmd \!\mid\! 
\cmd \pchoice \cmd \!\mid\! 
\cmd \prepeat
&
 \cmdpri & ::=  
\passign{\var}{\expr} \!\mid\! 
\passume(\expr)
\\
\trans & ::=
\pskip \!\mid\!
\transpri \!\mid\! 
\trans \pseq \trans \!\mid\!
\trans \pchoice \trans \!\mid\!
\trans\prepeat    
&
\transpri & ::= 
\cmdpri \!\mid\!
\plookup{\var}{\expr} \!\mid\!
\pmutate{\expr}{\expr} 
\end{aligned}%
\]
}%
%
%
Sequential commands $\cmd$ comprise $\pskip$, primitive commands
$\cmdpri$, atomic transactions $\ptrans{\trans}$, and standard
compound constructs: sequential composition (\( ; \)), non-deterministic
choice (\( + \)) and iteration (\( * \)). 
Primitive commands include variable assignment
$\passign{\var}{\expr}$ and assume statements $\passume(\expr)$
which can be used to encode conditionals. They are used for computations based on client-local variables and can hence be invoked
without restriction.  Transactional commands $\trans$ comprise
$\pskip$, primitive transactional commands $\transpri$, and the
standard compound constructs.  Primitive transactional commands comprise
primitive commands $\cmdpri$, lookup $\plookup{\var}{\expr}$ and mutation
$\pmutate{\expr}{\expr}$ used for reading and writing to kv-stores
respectively, which can only be invoked as part of an atomic
transaction.



A {\em program} is a finite partial function from client identifiers to sequential
commands.
For clarity, we often write \( \cmd_{1}\ppar \dots \ppar \cmd_{n}\) as syntactic sugar 
for a program \( \prog \) with $n$ clients associated with identifiers
$\cl_1 \dots \cl_n$, where each client $\cl_i$ executes
$\cmd_i$. 
Each client $\cl_i$ is associated with its own client-local \emph{stack}, 
$\stk_i \in \Stacks \defeq \Vars \to \Val$,  mapping program variables
(ranged over by $\pvar{x}, \pvar{y}, \cdots$)
to values. 
We assume a language of expressions built from values \( \val \)
and program variables \( \var \), defined by:
$\expr ::= \val \mid \var \mid \expr + \expr \mid \cdots$.
The \emph{evaluation} $\evalE{\expr}$ of expression $\expr$ is parametric in
the client-local stack \( \stk \):%

\spaceshrink{-5pt}
{%
\[
\displaymathfont
\begin{gathered}
\evalE{\val} \defeq
\val
\quad
\evalE{\var} \defeq
\stk(\var)
\quad
\evalE{\expr_{1} + \expr_{2}} \defeq
\evalE{\expr_{1}} + \evalE{\expr_{2}}
\quad
\dots
\end{gathered}%
\]
}%

\mypar{Transactional Semantics} 
In our operational semantics, transactions are executed
\emph{atomically}. It is still possible for an
implementation, such as COPS \cite{cops}, 
to update the underlying distributed key-value stores while
the transaction is in progress. It just means that, given the
abstractions captured by our global kv-stores and partial views, 
such an update is modelled as  an instantaneous  atomic
update.
Intuitively, given a configuration $\conf = (\mkvs, \vienv)$, 
when a client $\cl$ executes a transaction $\ptrans{\trans}$, 
it performs the following steps: 
\begin{enumerate*}
	\item it constructs an initial \emph{snapshot} $\sn$ of $\mkvs$ using its view $\vienv(\cl)$ as defined in \cref{def:snapshot};  
	\item it executes $\trans$ in isolation over $\sn$
        accumulating the effects (the reads and writes) of executing $\trans$; and
	\item it commits $\trans$ by incorporating these effects into $\mkvs$.
\end{enumerate*}

\spaceshrink{-3pt}
\begin{definition}[Transactional snapshots]
\label{def:heaps}
A \emph{transactional snapshot}, \( \sn \in \Snapshots \defeq \Keys \to
\Val\),  is a function from keys to values. When the meaning is clear,
it is just called a {\em snapshot}. 
\end{definition}
\spaceshrink{-3pt}

The rules for transactional commands (\cref{fig:semantics-trans}) will be defined  using an arbitrary 
transactional snapshot. The rules for sequential
commands and programs (\cref{fig:semantics-commands}) will be defined  using a transactional
snapshot given by a view snapshot. 
To capture the effects of executing a transaction $\trans$ on a snapshot $\sn$ of kv-store $\mkvs$, 
we identify a \emph{fingerprint} of $\trans$ on $\sn$ which captures
 the values $\trans$ reads from $\sn$, and
the values $\trans$ writes to $\sn$ and intends to commit to $\mkvs$. 
Execution of a transaction in a given configuration may result in more than one fingerprint due to non-determinism (non-deterministic choice). 

\spaceshrink{-3pt}
\begin{definition}[Fingerprints]
\label{beebop}
\label{def:fingerprint}
Let \( \Ops\) denote the set of \emph{read (\( \otR\)) and write (\(\otW\)) operations} defined by 
$\Ops \defeq \Set{(l, \key, \val) }[ l \in \Set{\otR, \otW} \land \key \in \Keys \land \val \in \Val ]$.
A \emph{fingerprint} $\fp$ is a set of operations, $\fp \subseteq \Ops$,
such that: 
$\fora{\key \in \Keys, l  \in \Set{\otR, \otW}}
	(l, \key, \val_1), (l, \key, \val_2) \in \fp \implies \val_1 = \val_2$.
\end{definition}
\spaceshrink{-3pt}

\noindent 
A fingerprint contains at most one read operation and at most one write operation. 
This reflects our assumption regarding transactions that satisfy the snapshot property: reads are taken from a single snapshot of the kv-store;
and only the last write of a transaction to each key is committed to the kv-store.
\begin{figure*}[!t]
\begin{center}
\scalebox{.9}{%
\begin{tabular}{@{}c@{\hspace{10pt}}c@{}}
    $
    \inferrule[\rl{TPrimitive}]{%
        (\stk, \sn) \toLTS{\transpri} (\stk', \sn') 
       \\
        \op = \func{op}[\stk, \sn, \transpri]
    }{%
        (\stk, \sn, \fp) , \transpri  \toTRANS   (\stk', \sn', \fp \addO \op) , \pskip 
    }%
    $
    &
    $
 \begin{array}{ll}
    \fp \addO (\otR, \key, \val)  
    & \defeq \
    \begin{cases}
        \fp \cup \Set{(\otR, \key, \val)} & \text{if } \fora{l, v'} (l, \key, v') \! \notin \! \fp \\
        \fp &  \text{otherwise} \\
    \end{cases}  \\
    \fp \addO (\otW, \key, \val) 
    & \defeq \
    \left( \fp \! \setminus \! \Set{(\otW, \key, v')}[v' \in \Val] \right)  \!
    \cup \! \Set{(\otW, \key, \val)} \\
    \fp \addO \emptyop  & \defeq \ \fp 
\end{array}
$
\end{tabular}
}
\end{center}

\spaceshrink{-10pt}

\hrulefill

\caption{Rules for transactional commands}
\spaceshrink{-3pt}
\label{fig:semantics-trans}
\end{figure*}

\Cref{fig:semantics-trans} presents the rules for primitive transactional commands, \( \rl{TPrimitive} \) rule. 
The rules for the compound constructs are straightforward and given in \cref{sec:full-semantics}.
The \( \rl{TPrimitive} \) rule updates 
the snapshot and the fingerprint of a transaction: the premise 
$(\stk, \sn) \toLTS{\transpri} (\stk', \sn')$ describes how executing
$\transpri$ affects the local state (the client stack and the snapshot)
of a transaction; and the premise $o =\func{op}(\stk, \sn, \transpri)$ identifies the operation on the 
kv-store associated with $\transpri$. 

\spaceshrink{-5pt}
\begin{definition}
\label{def:primitive_semantics}
The relation $\toLTS{\transpri}\; \subseteq (\Stacks \times \Snapshots) \times (\Stacks \times \Snapshots)$ % \times (\Stacks \times \Snapshots)
is defined by\footnote{ 
For any function \( \mathsf f \), the new function \( \mathsf f
\rmto{\mathsf d}{\mathsf r}\) 
is defined by 
\( \mathsf f
\rmto{\mathsf d}{\mathsf r}
( {\mathsf d}) = {\mathsf r}$, and \( 
\mathsf f
\rmto{\mathsf d}{\mathsf r}
({\mathsf d}') = {\mathsf f}({\mathsf d}')$ if \( 
{\mathsf d}' \neq {\mathsf d}\). } :
%

\spaceshrink{-17pt}
{%
\[
\displaymathfont
    \begin{array}{@{} r @{\ } c @{\ } l @{\quad} r @{\ } c @{\ } l@{}}
(\stk, \sn)  & \toLTS{\passign{\var}{\expr}}
             & (\stk\rmto{\var}{\evalE{\expr}}, \sn) 
&
(\stk, \sn)  & \toLTS{\passume(\expr)}  
             & (\stk, \sn) \text{ where } \evalE{\expr} \neq 0
\\
(\stk, \sn)  & \toLTS{ \plookup{\var}{\expr} } 
             & (\stk\rmto{\var}{\sn(\evalE{\expr})}, \sn) 
&
(\stk, \sn) & \! \toLTS{\pmutate{\expr_{1}}{\expr_{2}}} \!
            & (\stk, \sn\rmto{\evalE{\expr_{1}}}{\evalE{\expr_{2}}}) \ 
\end{array}
\]%
}%
\spaceshrink{-10pt}

\noindent 
The function $\mathsf{op}$,\! computing the fingerprint of primitive commands,\! is defined by:
%

\spaceshrink{-13pt}
{
\displaymathfont
\[%
\begin{aligned}
    \func{op}[\stk, \sn, \passign{\var}{\expr}] & \defeq  \emptyop 
    & 
    \func{op}[\stk, \sn, \passume(\expr)] & \defeq \emptyop 
    \\
    \func{op}[\stk, \sn,  \plookup{\var}{\expr}] & \defeq (\otR, \evalE{\expr}, \sn(\evalE{\expr})) 
    &
    \func{op}[\stk,  \sn, \pmutate{\expr_{1}}{\expr_{2}}] & \defeq (\otW, \evalE{\expr_{1}}, \evalE{\expr_{2}})
\end{aligned}
\]%
}
\spaceshrink{-10pt}

\noindent
The empty operation $\emptyop$ is used for those primitive commands that do not
contribute to the fingerprint.
\end{definition}
\spaceshrink{-5pt}


The conclusion of the \( \rl{TPrimitive}\)  rule uses the \emph{combination operator} $\addO: 
\pset{\Ops} \times (\Ops \uplus \Set{\emptyop}) \to \pset{\Ops}$, defined 
in \cref{fig:semantics-trans}, to extend the fingerprint $\fp$ accumulated with
operation $o$ associated with $\transpri$, as
appropriate: it adds  a read from $\key$  if $\fp$ 
contains no entry for $\key$; and it always updates the  write for 
$\key$ to $\fp$, removing previous writes to $\key$.


%\begin{definition}[Fingerprint Set]
%Given a client stack $\stk$ and a snapshot $\sn$, the \emph{fingerprint set} of a transaction $\trans$ is:
%\(
%\Fingerprints(\trans) \defeq \Set{\fp \mid \exsts{\stk,\stk',\sn,\sn'} (\stk, \sn, \emptyset), \trans \toTRANS^* (\stk', \sn', \fp), \pskip }
%\)
%where $\toTRANS^*$ denotes the reflexive, transitive closure of $\toTRANS$ given in \cref{fig:semantics-trans}.  
%A set $\fp \in \Fingerprints(\trans)$ is called a \emph{final fingerprint} of $\trans$. 
%\end{definition}
%\noindent 
%It is immediate to see that final fingerprints of $\trans$ 
%A final fingerprint of $\trans$ contains at most one read (resp.\ one write) operation per key.

\subsubsection{Command and Program Semantics}
We give the operational semantics of commands and programs in
\cref{fig:semantics-commands}.  
%
\begin{figure*}[!t]
\begin{center}
\scalebox{.9}{%
\begin{tabular}{@{}c@{\hspace{20pt}}c@{}}
    $
    \inferrule[\rl{CPrimitive}]{
		\stk \toLTS{\cmdpri} \stk'
    }{
        \cl \vdash 
        ( \mkvs, \vi, \stk ) , \cmdpri 
        \toCMD{(\cl,\iota)}_{\ET} 
        ( \mkvs, \vi, \stk' ) , \pskip
    }%
    $
    &
    $
 \begin{array}{r @{\ }c @{\ } l}
        \stk & \toLTS{\passign{\var}{\expr}} & \stk\rmto{\var}{\evalE{\expr}} \\
        \stk & \toLTS{\passume(\expr)} & \stk \ \text{where} \ \evalE{\expr} \neq 0
\end{array}
$
\\[15pt]
\multicolumn{2}{c}{
    $
     \inferrule[\rl{CAtomicTrans}]{%
        \vi \viewleq  \vi'' 
        \quad 
        \sn = \snapshot[\mkvs,\vi''] 
        \quad
        (\stk, \sn, \emptyset), \trans \toTRANS^{*}   (\stk', \stub,
  \fp) , \pskip
  \quad
	\cancommit{\mkvs}{\vi''}{\fp}
    \\\\
   \txid \in \nextTxid[\cl, \mkvs] 
\\
     \mkvs' = \updateKV[\mkvs, \vi'', \fp, \txid] 
\\
	\vshift{\mkvs}{\vi''}{\mkvs'}{\vi'}	
    }{%
        \cl \vdash 
        ( \mkvs, \vi, \stk ), \ptrans{\trans} 
        \toCMD{(\cl, \vi'', \fp)}_{\ET}
        (\mkvs',\vi', \stk' ) , \pskip
    }%
    $
    }




    \\[10pt]
    \multicolumn{2}{c}{
    $
    \inferrule[\rl{PProg}]{%
	    \vi = \vienv (\cl)
        \\
        \stk = \thdenv(\cl)
        \\
        \cmd = \prog(\cl)
        \\
		\cl \vdash 
		( \mkvs, \vi, \stk ) , \cmd
		\toCMD{\lambda}_{\ET} 
		( \mkvs', \vi', \stk' ) , \cmd'  
    }{%
		\vdash 
		(\mkvs, \vienv, \thdenv ), \prog
		\toCMD{\lambda}_{\ET} 
		( \mkvs', \vienv\rmto{\cl}{\vi'}, \thdenv\rmto{\cl}{\stk'} ) , \prog\rmto{\cl}{\cmd'} ) 
    }%
    $
    }\\[-10pt]
\end{tabular}
}
\end{center}

\hrulefill

\caption{Rules for sequential commands and programs}
\label{fig:semantics-commands}
\spaceshrink{-3pt}
\end{figure*}
%
The command semantics describes transitions of the form
$\cl \vdash (\mkvs, \vi, \stk), \cmd \ \toCMD{\lambda}_{\ET} \ (\mkvs', \vi', \stk') ,
\cmd'$, stating that given the kv-store $\mkvs$, view $\vi$ and stack $\stk$, 
a client $\cl$ may execute command $\cmd$ for one step, updating 
the kv-store to $\mkvs'$, the stack to $\stk'$, the view to \( \vi' \) and the command to its continuation $\cmd'$.
The label $\lambda$ is either of the form $(\cl, \iota)$ denoting that $\cl$ executed a primitive command
that required no access to $\mkvs$, 
or $(\cl, \vi'', \fp)$ denoting that $\cl$ committed an atomic transaction with final fingerprint $\fp$ under the view $\vi''$.
The semantics is parametric in the choice of \emph{execution test}
$\ET$ for kv-stores, which is used to generate 
the \emph{consistency model} on kv-stores
under which a 
transaction can execute.
In \cref{sec:cm}, we give many examples of execution tests for
well-known consistency models.
In \cref{sec:other_formalisms} and \cref{app:et_sound_complete}, we prove that our execution tests 
generate consistency models that are equivalent to the existing definitions of
consistency models (using execution graphs). 

The rules for the compound constructs are straightforward and given in \cref{sec:full-semantics}.
The rule for primitive commands, $\rl{CPrimitive}$, 
depends on the 
transition system $\toLTS{\cmdpri} \subseteq \Stacks \times \Stacks$ 
which simply describes how the primitive command $\cmdpri$ affects the local state of a client.
The rule \rl{CAtomicTrans}  describes the execution of an atomic 
transaction under the execution test $\ET$. 


We explain the \rl{CAtomicTrans} rule in detail. 
The first premise 
states that the current view $\vi$ of the executing command may be advanced to a newer  view $\vi''$ (see \cref{def:views}). 
Given the new view $\vi''$, the transaction obtains a snapshot $\sn$ of the kv-store $\mkvs$, 
and executes $\trans$ locally to completion ($\pskip$), updating the stack to $\stk'$, while accumulating the fingerprint $\fp$; 
this behaviour  is modelled in the second and third premises of \rl{CAtomicTrans}.
Note that the resulting snapshot is ignored 
as the effect of the transaction is recorded in the fingerprint $\fp$. 
The $\cancommit{\mkvs}{\vi''}{\fp}$ premise ensures that under the execution test $\ET$, 
the final fingerprint $\fp$ of the transaction is compatible with the (original) kv-store
$\mkvs$ and the client view $u''$, and thus the transaction \emph{can commit}. 
Note that the \cancommitname check is parametric in the execution test $\ET$.
This is because the conditions checked upon committing depend on the consistency model under which the transaction is to commit. 
In \cref{sec:cm}, we define \cancommitname for several execution tests associated with well-known consistency models.


Now we are ready for client $\cl$ to commit the transaction resulting 
in the kv-store $\mkvs'$ with the client view $\vi''$ \emph{shifting} to a new view $\vi'$: 
\begin{enumerate*}
\item pick a fresh transaction identifier $\txid \in \nextTxid[\cl, \mkvs]$;
\item compute the new kv-store $\mkvs' = \mathtt{update}(\mkvs, {\allowbreak \vi''},$
$\fp, \txid)$; and 
\item check if the \emph{view shift} is permitted under execution test $\ET$ using $\vshift{\mkvs}{\vi''}{\mkvs'}{\vi'}$. 
\end{enumerate*}
Observe that as with \(\cancommitname\), the \(\vshiftname\) check is parametric in the execution test $\ET$. 
Once again this is because the conditions checked for shifting the client view depend on the consistency model. 
In \cref{sec:cm}, we define \(\vshiftname\) for several execution tests associated with well-known consistency models.
The set $\nextTxid[\cl, \mkvs]$ is defined by
\(
\nextTxid[\cl, \mkvs] \defeq 
\{ \txid_{\cl}^{n} \mid %
\fora{m} \txid_{\cl}^{m} \in \mkvs \allowbreak \implies \allowbreak m < n \}
\).
The function $\updateKV[\mkvs, \vi, \fp, \txid]$
describes how the fingerprint $\fp$ of transaction $\txid$ executed under view $\vi$ updates kv-store $\mkvs$:
for each read $(\otR, \key, \val) \in \fp$, it adds $\txid$ 
to the reader set of the last version of $\key$ in $\vi$; 
for each write $(\otW, \key, \val)$, it appends a new version $(\val, \txid, \emptyset)$ 
to $\mkvs(\key)$. 


\spaceshrink{-3pt}
\begin{definition}[Transactional update]
\label{eq:updatekv}
\label{def:updatekv}
The function  $\updateKV[\mkvs, \vi, \fp, \txid]$,  is
defined by: \( \updateKV[\mkvs, \vi, \emptyset, \txid] \defeq \mkvs  \) and

\spaceshrink{-10pt}
{%
\displaymathfont
\[%
\begin{array}{lcl}
    \updateKV[\mkvs, \vi, \Set{(\otR, \key, \val)} \uplus \fp, \txid]
    & \defeq & \text{let} \ i = \max_{<}(\vi(\key)) \ \text{and} \ (\val,\txid',\txidset) = \mkvs(\key,i) \text{ in } \\
    && \;\;\;\; \updateKV[\mkvs\rmto{\key}{\mkvs(\key)\rmto{i}{(\val, \txid', \txidset \uplus \Set{\txid})}},\vi, \fp, \txid] \\
%	
	\updateKV[\mkvs, \vi, \Set{(\otW, \key, \val)} \uplus \fp, \txid]
    & \defeq & \text{let } \mkvs' = \mkvs\rmto{\key}{ \mkvs(\key) \lcat (\val, \txid, \emptyset) } \text{ in } \updateKV[\mkvs', \vi, \fp, \txid] 
\end{array}
\]%
\normalsize%
}%
\spaceshrink{-5pt}

\noindent 
where, given a list of versions $\vilist = \ver_0 \lcat \cdots \lcat \ver_n$ 
and an index $i: 0 \leq i \leq n$, 
then $\vilist\rmto{i}{\ver} \defeq \ver_0 \lcat \cdots \lcat \ver_{i-1} \lcat \ver \lcat \ver_{i+1} \cdots \ver_{n}$.

\end{definition}
\spaceshrink{-3pt}

The last rule, \( \rl{PProg} \) (\cref{fig:semantics-commands}),
captures the execution of a program step 
given a \emph{client environment} $\thdenv \in \ThdEnv$.
A client environment $\thdenv$ is a function from client identifiers to stacks, associating each client with its stack. 
We assume that the domain of client environment contains 
the domain of the program throughout the execution: 
$\dom(\prog) \subseteq \dom(\thdenv)$.
Program transitions are simply defined in terms of the transitions of
their constituent client commands. 
This yields an interleaving semantics for transactions of different clients:  
a client executes a transaction in an atomic step without
interference from the  other clients. 
