\section{Operational Model}
\label{sec:model}

\pg{Remember: section 3 is all about technical definitions; section 2
 is  about definitions. This saves space and clarifies the purpose of
 each section.}

We give the technical definitions of our operational model: 
the global, centralised key-value stores; the partial client views;  and the
operational semantics. 

\subsection{Key-value Stores}

Key-value stores are defined using client identifiers and transaction identifiers.
Let $\Clients \defeq \Set{\cl, \cl',\cdots}$ be a 
countably infinite set of \emph{client identifiers}. 
The set of \emph{transaction identifiers}, 
$\TxID \defeq  \Set{\txid_{0}} \uplus \Set{ \txid_{\cl}^{n} \mid \cl
  \in \Clients \wedge n \geq 0 }$, 
comprises an initialisation transaction 
 $\txid_0$ 
 and, for each $n \in \mathbb{N}$, an identifier $\txid_{\cl}^{n}$
which  identifies a transaction 
 committed by client $\cl$. Elements of $\TxID$ are ranged over by
$\txid, \txid', \cdots$, and subsets by $\txidset, \txidset', \cdots$. 
Let $\TxID_{0} \defeq \TxID \setminus \{ \txid_0\}$. 
The superscript $n$ records  the session order, $\PO$, associated with
the transaction and client: that is, 
$\PO \defeq \Set{ (\txid, \txid') \mid \exsts{ \cl, n,m } \txid =
  \txid_{\cl}^{n} \wedge \txid' = \txid_{\cl}^{m} \wedge n < m}$.
We write $\txid \xrightarrow{\PO} \txid'$ for
$(\txid, \txid') \in \PO$, and $\txid \xrightarrow{\PO ?} \txid'$ for
for its reflexive closure. 


\begin{definition}[Key-value Stores]
\label{def:his_heap}
\label{def:mkvs}
Assume a countably infinite set of \emph{keys} $\Keys = \Set{\ke, \ke', \cdots}$
and \emph{values} $\Val = \{\val, \val', \cdots\}$, including an
initialisation value    $\val_0 $.
A \emph{version} is a triple $\ver \in \Versions \defeq \Val \times \TxID \times \powerset{\TxID_{0}}$. 
A \emph{key-value store}, abbreviated to kv-store,  is a function $\hh: \Keys \rightarrow \Versions^{\ast}$, 
where $\Versions^{\ast}$ is the free monoid generated by $\Versions$. 
The \emph{initial}  key-value store is given by 
$\hh_{0}(\ke)\defeq  (\val_0, \txid_0, \emptyset)$ for
all $\ke \in \Keys$.



\end{definition}


A \emph{version} $\ver = (\val, \txid, \txidset)$ comprises  a value $\val$
and meta-data about the transactions that accessed it: the \emph{writer} $\txid$ identifies the transaction that wrote $\ver$;
and the set of \emph{readers} $\txidset$ identifies  the  transactions
that read from  $\ver$.
We write $\valueOf(\ver) \defeq \val$,
$\WTx(\ver) \defeq \txid$ and $\RTx(\ver) \defeq \txidset$. 
 Elements of $\Versions$ are ranged over by
$\ver, \ver',  \cdots$, and lists of versions (elements of $\Versions^{\ast}$) by $\vilist, \vilist',\cdots$.
Given kv-store $\hh$ and a transaction identifier $\txid$, we write 
$\txid \in \hh$ when $\txid$ appears as either the writer or reader of 
some version included in $\hh$.
%%%%
%%%%I don't think this technical detail is needed, it's clear
%%:  $\txid \in \hh \defeq 
%\exists \ke, i. 0 \leq i < \lvert \hh(\ke) \rvert \wedge (\txid \in
%\RTx(\hh(\ke, i)) \cup \{\WTx(\hh(\ke, i))\})$.
We also write 
$\hh(\ke, i)$ for the   $i$-th version of $\ke$ when defined, starting
from $0$ with  $i \geq 0$,   and $\lvert \hh(\ke) \rvert$ for   the length of
$\hh(\ke)$. 




We focus on key-value stores whose consistency model enforces the \emph{atomic visibility} of transactions~\cite{framework-concur}. 
This amounts to requiring that a transaction reads and writes at most
one version of each key.
We also assume that the list of versions for each key has an initial version carrying a default value $\val_0$, 
written by the designated initialisation transaction $\txid_0$ with an initial
empty set of readers.
Finally, we assume that the state of a key-value store is consistent with 
the session order of clients: a client cannot read a version of a key that has 
been installed by a future transaction within the same session;  and 
the order in which versions are installed by a single client must agree 
with its session order. When a 
kv-store satisfies  these conditions, we say that it is
\emph{well-formed}. 
Henceforth, we assume kv-stores are well-formed, and write  $\HisHeaps$ to denote 
the set of well-formed kv-stores.


\pg{Above, I've got rid of the technical details to make it more
  readable. I've also done a technical version, commented out, which
  should probably go in the appendix.}

%We focus on key-value stores whose consistency model enforces the
%\emph{atomic visibility} of transactions~\cite{framework-concur}.
%This amounts to requiring that a transaction reads and writes at most
%one version of each key: \textbf{(i)}
%$\fora{\ke, i, j} (o \leq i, j \leq \abs{ \hh(\ke) } \land
%\RTx(\hh(\ke, i)) \cap \RTx(\hh(\ke, j)) \neq \emptyset ) \implies i =
%j$, \textbf{(ii)}
%$\fora{\ke, i, j} (0 \leq i,j < \abs{ \hh(\ke) } \wedge \WTx(\hh(\ke,
%i)) = \WTx(\hh(\ke, j)) ) \implies i = j$.  We also assume that the
%list of versions for each key has an initial version carrying a
%default value $\val_0$, written by the designated initialisation
%transaction $\txid_0$: \textbf{(iii)}
%$\fora{\ke} \hh(\ke, 0) = (\val_0, \txid_0, \stub)$.  Finally, we
%assume that the state of a key-value store is consistent with the
%session order of clients: \textbf{(iv)}
%$\fora{ \ke, \cl, i,j, n, m} 0 \leq i < j < \abs{\hh(\ke)} \land
%\txid_{\cl}^{n} = \WTx(\hh(\ke,i)) {} \wedge \txid_{\cl}^{m} \in
%\Set{\WTx(\hh(\ke,j))} \cup \RTx(\hh(\ke, i)) \implies n < m $.  We
%say that kv-stores that satisfy the conditions
%\textbf{(i)}-\textbf{(iv)} above are \emph{well-formed}.  Henceforth,
%we always assume kv-stores are  well-formed, and  write 
%$\HisHeaps$ to denote the set of well-formed kv-stores.


\subsection{Client Views}

Clients often have a partial view of the key-value store, 
with different clients observing 
different version subsets  of the same key.
%To keep track of the versions observed by clients,
%we introduce the notion of \emph{client views} (\cref{def:view}). 

\begin{definition}[View]
\label{def:view}
\label{def:cuts}
\label{def:views}
\label{def:configuration}
A \emph{view} of a kv-store $\hh$ is a function
$\vi \in \Views(\hh) \defeq \Keys \to\powerset{\Nat}$ such that:
\begin{align}
    & \fora{ \ke } 
    0 \in \vi(\ke) 
    \wedge \fora{ i \in \vi(\ke) } 
    i < \abs{ \hh(\ke) } 
    \tag{well-formed}
    \label{eq:view.wf}\\
    & 
    \begin{array}{@{}l@{}}
    \fora{ \ke_1,\ke_2, i_1, i_2} 
	i_1 \in \vi(\ke_1)  \\
    \quad {} \land \WTx(\hh(\ke_1, i_1)) = \WTx(\hh(\ke_2, i_2))  \\
    \qquad {} \implies i_2 \in \vi(\ke_2)
    \end{array}
	\tag{atomic}
	\label{eq:view.atomic}
\end{align}
The initial view $\vi_{0}$ is defined as $\vi_{0}(\ke) = \{0\}$ for
each $\ke \in \Keys$. A \emph{configuration}, $\conf \in \Confs$  is a pair $ (\hh, \viewFun)$
with $\hh \in \HisHeaps$ and
$\viewFun : \Clients \parfinfun \Views(\hh)$. A configuration is an 
\emph{initial} configuration when $\hh$ is the initial key-value store
$\hh_{0}$. 
\end{definition}
%


In configuration $(\hh, \viewFun)$, the view of client 
$\cl$, given by $\vi = \viewFun(\cl)$ if defined,  determines, for each key $\ke
\in \Keys$, the sublist of versions in $\hh$
that the client can observe. If $i,j \in \vi(\ke)$ and $i < j$, then
the client knows about 
 the values and transaction meta-data associated with  versions
$\hh(\ke, i)$ and 
$\hh(\ke, j)$, and  knows that these versions are contained in 
 $\hh$ with  $\hh(\ke, j)$ more 
up-to-date than $\hh(\ke, i)$. 
Let $\Views = \bigcup_{\hh \in \HisHeaps} \Views(\hh)$ be the set of all views. 
Given a kv-store $\hh$ and two views $\vi, \vi' \in \Views(\hh)$, 
we write $\vi \viewleq \vi'$ when $\vi(k) \subseteq \vi'(\ke)$ for all $\ke \in \dom(\hh)$. 

Given configuration
$(\hh, \viewFun)$ and client $\cl$ such that $\viewFun(\cl)$ is
defined, it is possible to define a \emph{snapshot} of the
view $\viewFun  (\cl)$, which identifies the last write of a client
view. This definition assumes that the database satisfies the \emph{last write wins}
resolution policy~\cite{}. It  is straightforward to generalise our work
to other resolution policies~\cite{.}.

\begin{definition}[Snapshot]
\label{def:heaps}
\label{def:snapshot}
A \emph{snapshot}, \( \ss \in \Snapshots  \defeq \Keys \to
\Val\),  is a function  from keys to values.
Given $\hh \in \HisHeaps$ and $\vi \in \Views(\hh)$, the \emph{snapshot} of $\vi$ in 
$\hh$ is defined by  $\snapshot(\hh, \vi) \defeq \lambda \ke \ldotp \valueOf(\hh(\ke, \max_{<}(\vi(\ke)))$, 
where $\max_{<}(\vi(\ke))$ is the maximum element in $\vi(\ke)$ with respect to the natural 
order $<$ over $\mathbb{N}$.
\end{definition}
Given a kv-store $\hh$, key $\ke$ and view $\vi$, we abuse
notation, writing 
$\hh(\ke, \vi)$ as a shorthand for 
$\hh(\ke, \max_{<}(\vi(\ke))$. Thus, $\snapshot(\hh, \vi) = \lambda \ke \ldotp \valueOf(\hh(\ke, \vi))$. 

\begin{remark}
The client view describes the partial history of  the database that is
observed  by
the client. The client snapshot describes the most up-to-date
information about the state of the database known to the client. 
It is not possible to work with just the 
snapshot. The client view is essential to 
to express some of the standard consistency models (see
Section~\cref{........}). 
\end{remark}


\subsection{Operational Semantics}

\noindent {\bf Programming Language} A 
\emph{program} \( \prog \) comprises a finite number of clients,
where each client is associated with a unique identifier \( \thid \in \ThreadID \), 
and executes a sequential \emph{command} $\cmd$, given by the following grammar:
\begin{align*}
\cmd & ::=  
\pskip \mid 
\cmdpri \mid  
\ptrans{\trans} \mid 
\cmd \pseq \cmd \mid 
\cmd \pchoice \cmd \mid 
\cmd \prepeat  
\\
\cmdpri & ::=  
\pass{\txvar}{\expr} \mid 
\passume{\expr} 
\\
\trans & ::=
\pskip \mid
\transpri \mid 
\trans \pseq \trans \mid
\trans \pchoice \trans \mid
\trans\prepeat    
\\
\transpri & ::= 
\cmdpri \mid
\pderef{\txvar}{\expr} \mid
\pmutate{\expr}{\expr} 
\end{align*} 
%

\pg{Why are you using $a$ for variable. Later you use $x$. Just be
  consistent. I'm happy with $x$.} 
Sequential commands  comprise $\pskip$,  primitive commands $\cmdpri
$, atomic transactions
$\ptrans{\trans}$,  and standard
compound commands. 
Primitive commands--the variable assignment
$\pass{\txvar}{\expr}$ and the assume statement $\passume{\expr}$
used to encode conditionals--are used for computations based on 
client-local variables 
and can hence be invoked without restriction. The 
transaction commands, $T$, 
comprise primitive commands, 
primitive transactional commands $\transpri$,  and standard compound commands. 
Primitive transactional commands---the lookup $\pderef{\txvar}{\expr}$ and the mutation 
$\pmutate{\expr}{\expr}$---are used for reading and writing to kv-stores respectively and 
can be invoked  as part of an atomic transaction $\ptrans{\trans}$.

A {\em program} is a finite partial function from client identifiers to sequential
commands.
%%%%I don't think this is needed. 
%$\prog = \Set{\thid_{1} \mapsto \cmd_{1}, \dots, \thid_{n} \mapsto \cmd_{n} $.
For clarity, we often write \( \cmd_{1}\ppar \dots \ppar \cmd_{n}\) as syntactic sugar 
for a program \( \prog \) with $n$ clients associated with identifiers
$\thid_1 \dots \thid_n$, where each client $\thid_i$ executes
$\cmd_i$. Each client is associated with its own client-local stack,  \emph{stack} 
$\stk \in \Stacks \defeq \Vars \to \Val$,  mapping program variables
ranged over by $\pvar{x}, \pvar{y}, \cdots$
to values. 
We assume a language of expressions built from values
and program variables, 
\( \expr ::= \val \mid \var \mid \expr + \expr \mid \dots  \).
The evaluation $\evalE{\expr}$ of  expression $\expr$ is parametric in
the client-local stack:
\begin{gather*}
\evalE{\val} \defeq
\val
\quad
\evalE{\var} \defeq
\stk(\var)
\quad
\evalE{\expr_{1} + \expr_{2}} \defeq
\evalE{\expr_{1}} + \evalE{\expr_{2}}
\quad
\dots
\end{gather*}
\pg{In above, you could define stack  $s_i$ for client $cl_i$.}

\noindent {\bf Transaction Semantics}  
In our framework, transactions are executed atomically. 
Roughly speaking, given a configuration $\conf = (\hh, \viewFun)$, 
when a client $\cl$ executes some transactional code $\ptrans{\trans}$, 
 it performs the following steps: 
it constructs a snapshot of $\hh$ from the view $\viewFun(\cl)$ that 
the client has over $\hh$;  it executes the code $\trans$ in isolation, using the 
snapshot constructed in the previous step as the initial, local state
of the client, and it determines the observable 
effects that such an execution has on the key-value store; and  it incorporates 
the effects of executing the code $\trans$ in the initial state determined by the snapshot into 
the key-value store.

We capture  the behaviour of a  transaction,
$\trans$,
by describing how it updates the client stack and snapshot of 
the kv-store and by  identifying   its {\em fingerprint set}.
A fingerprint of a transaction
is a set of read and write operations that describes  the 
reads of a snapshot of the kv-store taken  at the beginning of
the invocation of the transaction, and the writes to be
commited to the kv-store  as long as certain
consistency conditions are met. A transaction can have more than one fingerprint due to
non-determinism. 




\pg{Above  might need to dovetail with what we say in section 2. 
Maybe give an example that relates to
section 2 that shows intuitively how fingerprint works, maybnot not
and instead jsut refer to section 2}




\begin{definition}[Read-Write Set]
\label{beebop}
Let 
%The set of \emph{operations} is
$\Ops \defeq \{(l, \ke, \val) \mid$ $ l \in \Set{\otW, \otR} \land \ke \in \Keys \wedge \val \in \Val \}$ 
be a set of operations. 
A \emph{Read-Write Set} $\opset$ is a subset of operations, $\opset \subseteq \Ops$,
such that, for all $\ke \in \Keys$ and \( l  \in \Set{\otW, \otR} \),
if $(l, \ke, \val_1), (l, \ke, \val_2) \in \opset$ then $\val_1 = \val_2$.
\end{definition}
Note that we have placed a constraint that, per key, a read-write set
contains at most one read operation and at most one write operation.
This reflects the fact that we work with transactions that are
atomically visible~\cite{laws} meaning here that the reads are taken
from a snapshot of the kv-store and, since clients observe either none
or all effects of a transaction, only the last writes are 
committed.

We provide an operational description of the behaviour of a transaction command, $T$,
starting 
from an intial client stack
$s$, a snapshot $ss$  and the empty read-write set. 
%Intuitively,  a fingerprint of a transaction  records, for each key $\ke$,
%the first value a transaction reads (before a subsequent write) for $\ke$, 
%and the last value the transaction writes for $\ke$.
First, we define a transition system which describes how the stack and snapshot are updated  using the
primitive transaction commands.


\begin{definition}
\label{foo}
The transition system, $\toLTS{\transpri}\; \subseteq (\Stacks \times \Heaps) \times (\Stacks \times \Heaps)$, 
is defined by:
\[
\begin{rclarray}
(\stk, \h)  & \toLTS{\passign{\var}{\expr}}          & (\stk\rmto{\var}{\evalE{\expr}}, \h)                  \\
(\stk, \h)  & \toLTS{\passume{\expr}}                & (\stk, \h) \text{ where } \evalE{\expr} \neq 0        \\
(\stk, \h)  
& \toLTS{ \pderef{\var}{\expr} } 
& (\stk\rmto{\var}{\h(\evalE{\expr})}, \h) 
\\
(\stk, \h)
& \toLTS{\pmutate{\expr_{1}}{\expr_{2}}  }
& (\stk, \h\rmto{\evalE{\expr_{1}}}{\evalE{\expr_{2}}}) \\
\end{rclarray}                                                                                               
\]
\end{definition}
%Second, we define the basic actions of the primitive commands and
%primitive transactional commands using  function
% I don't like this name fingerprint function, it's not, the action
% might not be in the fingerprint. we only begin to get the
% fingerprint intruition with the composition operator. 
%we define a \emph{fingerprint function}, 
%$\func{fp}{\stub} \; : \Stacks \times \Heaps \times \transpri \rightarrow \Ops \cup \{\varepsilon\}$:
%

%Note that  the primitive commands are associated with the empty operation $\varepsilon$,
%as they only access the local stack and do not access the kv-store.

\noindent Second, we provide a 
function, $\mathsf{op}$,  connecting the  primitive transaction commands to the
basic read and write operations given in Definition~\ref{beebop}: 
\[
\begin{array}{rcl @{\quad} rcl}
\func{op}{\stk, \h, \passign{\var}{\expr}}          & \defeq & \emptyop                                     \\
\func{op}{\stk, \h, \passume{\expr}}                & \defeq & \emptyop                                     \\
\func{op}{\stk, \h,  \pderef{\var}{\expr}}           & \defeq & (\etR, \evalE{\expr}, \h(\evalE{\expr}))    \\
\func{op}{\stk,  \h, \pmutate{\expr_{1}}{\expr_{2}}} & \defeq & (\etW, \evalE{\expr_{1}}, \evalE{\expr_{2}})\\
\end{array}
\]
The  empty operation $\emptyop$ is used for the primitive commands which do not
contribute to the fingerprint.

Third, we define an operator,
$\addO  : {\cal P}(\Ops)\times \Ops\rightarrow {\cal P}(\Ops)$,  which
adds a  basic
operation to  a read-write set and ignores the empty operation: 
%For instance, when executing $ \ptrans{\trans}$ with $\trans \eqdef \transpri^1; \cdots ; \transpri^n$,
%the effect of each $\transpri^i$ is calculated via the $\op_i = \func{fp}{-, -, \transpri^i}$ function, 
%with the overall fingerprint given as the $\addO$-composition of the constituent effects: $\op_1 \addO \cdots \addO \op_n$. 
\begin{align*}
    \opset \addO (\etR, \addr, \val)  
    & \defeq
    \begin{cases}
        \opset \cup \{(\etR, \addr, \val)\} & \text{if } \for{l, v'} (l, \addr, v') \notin \opset \\
        \opset &  \text{otherwise} \\
    \end{cases}  \\
    \opset \addO (\etW, \addr, \val) 
    & \defeq 
    \left( \opset \setminus \setcomp{(\etW, \addr, v')}{v' \in \Val} \right) 
    \cup \Set{(\etW, \addr, \val)}  \\
    \opset \addO \emptyop  & \defeq  \opset  \\
\end{align*}
A read gets added to the read-write set if there is no read or
write there already, thus only  recording the reads of the
original snapshot of the kv-store at invocation. 
A write always updates the read-write set, corresponding to a
transaction only commiting the final writes. 


%
\begin{figure*}[!t]
\hrulefill
\begin{mathpar}
    \inferrule[\rl{TPrimitive}]{%
        (\stk, \h) \toLTS{\transpri} (\stk', \h')
        \\ \op = \func{op}{\stk, \h, \transpri}
    }{%
        (\stk, \h, \opset) , \transpri \ \toL \  (\stk', \h', \opset \addO \op) , \pskip 
    }
    \\
    \inferrule[\rl{TChoice}]{
        i \in \Set{1,2}
    }{%
        (\stk, \h, \opset) , \trans_{1} \pchoice \trans_{2} \ \toL \  (\stk, \h, \opset) , \trans_{i}
    }
    \and
    \inferrule[\rl{TIter}]{ }{%
        (\stk, \h, \opset),  \trans\prepeat \ \toL \  (\stk, \h, \opset), \pskip \pchoice (\trans \pseq \trans\prepeat)
    } 
    \and
    \inferrule[\rl{TSeqSkip}]{ }{%
        (\stk, \h, \opset), \pskip \pseq \trans \ \toL \  (\stk, \h, \opset), \trans
    }
    \and
    \inferrule[\rl{TSeq}]{%
        (\stk, \h, \opset), \trans_{1} \ \toL \  (\stk', \h', \opset'), \trans_{1}'
    }{%
        (\stk, \h, \opset), \trans_{1} \pseq \trans_{2} \ \toL \  (\stk', \h', \opset'), \trans_{1}' \pseq \trans_{2}
    }
\end{mathpar}
\hrulefill
\caption{Transaction semantics.}
\label{fig:semantics-trans}
\end{figure*}

Finally, we have all the ingredients to describe the behaviour of  a
transaction command.  \cref{fig:semantics-trans} provides the
one-step transaction semantics: the  interesting rule is the \rl{TPrimitive}
rule which describes how a primitive transactional command updates
the client stack, the snapshot and the read-write set; the other
rules are standard rules for compound commands. 



\begin{definition}[Fingerprint Set]
Given client stack $s$
and snapshot $ss$, the \emph{fingerprint set } of $T$ is given by 
\[F \defeq
\{\opset : (\stk, \h, \emptyset), T \toL^* (\stk', \h', \opset),
\pskip \}
\]
 where $\toL^*$ is the transitive closure of $\toL$ given
 in~\cref{fig:semantics-trans}.  A set $\opset$ in $F$ is called a
 \emph{fingerprint} of $T$. 
\end{definition}
\noindent It is straightforward to prove that the  fingerprints of $T$ contain at most one read operation per key and
one write operation per key. \\

\pg{Below, there were some minor inconsistences. Needs checking.I've
  made lots of changes. It's a bit difficult to assess by laptop,
  really needs a paper copy. Hope ok.}

\noindent {\bf Operational Semantics.} We give the operational
semantics of commands and programs. The command semantics describes
transitions of the form
$(\hh, \vi, \stk), \cmd \ \toT{\lambda}_{\ET} \ (\hh', \vi', \stk') ,
\cmd'$, stating that, given the kv-store $\hh$, view $\vi$ and stack
$\stk$, a client $\cl$ may execute command $\cmd$ for one step, updating the
kv-store to $\hh'$, the stack to $\stk'$, and the command to its
continuation $\cmd'$.  The label $\lambda$ is either of the form
$(\cl, \iota)$ denoting that client $\cl$ used a primitive command
that did not require access to the kv-store, or
$(\cl, \vi'', \opset)$ denoting that client $\cl$ commited an atomic
transaction with fingerprint $\opset$ under the view $\vi''$.
Transitions are parametric in the choice of \emph{execution test},
$\ET$,  used to capture well-known consisitency models studied in the literature
(Sections~\ref{4}
and~\ref{5}). 

\pg{Sorry, I think it's better to revert back to the judgement $cl
  \vdash (\hh, \vi, \stk), \cmd \ \toT{\lambda}_{\ET} \ (\hh', \vi', \stk') ,
\cmd'$, please change everywhere. Why is the label needed? It's not clear.}



\begin{figure*}[t]
\hrulefill
\[
    \inferrule[\rl{CPrimitive}]{
        (\stk, \h)  \toLTS{\cmdpri} (\stk', \h)
        \qquad \h = \clpsHH{\hh,\vi}
    }{%
        ( \hh, \vi, \stk ) , \cmdpri \ \toT{(\cl,\iota)}_{\como} \  ( \hh, \vi, \stk' ) , \pskip
    }
\]
\[
    \inferrule[\rl{CAtomicTrans}]{%
        \vi \orderVI  \vi''
        \qquad \h = \clpsHH{\hh,\vi''}
        \qquad \txid \in \nextTxId(\cl, \hh)
        \\\\ (\stk, \h, \emptyset), \trans \ \toL^{*} \  (\stk', \stub,  \opset) , \pskip
        \\ \ET \vdash (\hh, \vi'') \triangleright \opset : \vi'
    }{%
        ( \hh, \vi, \stk ), \ptrans{\trans} \ \toT{(\cl, \vi'', \opset)}_{\ET} \ (\updateKV(\hh, \vi, \txid, \opset),\vi', \stk' ) , \pskip
    }
\]
%\hrulefill
\[
    \inferrule[\rl{PProg}]{%
        ( \mkvs, \vi, \thdenv(\thid) ) , \prog(\thid), \
        \toT{\lambda}_{\ET} \  ( \mkvs', \vi', \stk' ) , \cmd'  \qquad 
\viewFun (\thid) = \vi
    }{%
        (\mkvs, \viewFun, \thdenv ), \prog  \ \toT{\lambda}_{\ET} \  ( \mkvs', \viewFun, \thdenv\rmto{\thid}{\stk'} ) , \prog\rmto{\thid}{\cmd'} ) 
    }
\]
\hrulefill
\caption{Rules for primitive  commands, atomic
  transactions and programs.}
\label{fig:semantics}
\end{figure*}



Figure~\ref{.} contains the rules for primitive commands and atomic
transactions.  The rules for the compound commands are straightforward
and given in the appendix.
The rule for primtive commands, $\rl{CPrimitive}$,  uses the transition
system
 $\toLTS{\transpri}$, given in  Definition~\ref{foo}, applied to the primitive
commands which just affect the client stack. The rule for atomic
transactions, \rl{CAtomicTrans}, describes the execution of an atomic 
transaction under execution test $\ET$.  The first premise
states that the current view $\vi$ of the executing command maybe advanced to a newer atomic view $\vi''$ (see \cref{def:views}). 
The semantics only allows to advance the view to later versions, which corresponds to \emph{monotonic read} \cite{.......}.
Given the new view $\vi''$, the transaction proceeds by obtaining a snapshot $\sn$ of the kv-store $\hh$, and executing $\trans$ locally to completion ($\pskip$), updating the stack to $\stack'$, while accumulating the fingerprint $\opset$. Note that the resulting snapshot is ignored (denoted by $\stub$) as the effect of the transaction is recorded in the fingerprint $\opset$. 
%

The transaction is now ready to commit. The rule picks a fresh
transaction identifier using $\txid \in \nextTxId(\cl, \hh)$, judges
whether the commit is permitted by $\ET$ using the relation
$\ET \vdash (\hh, \vi'') \triangleright \opset : \vi'$, and updates
the kv-store using $\updKV{\hh, \vi, \txid, \opset}$.  The
set $\nextTxId(\cl, \hh)$ provides the  transactions identifiers
associated with $\cl$ that are fresh for  $\hh$:
$
\nextTxId(\cl, \hh) \defeq \Setcon{\txid_{\cl}^{n}}{\fora{m}
  \txid_{\cl}^{m} \in \hh \Rightarrow m < n }.
$
 By construction, all
elements of $\nextTxId(\cl, \hh)$ are greater, with respect to session
order $\xrightarrow{\PO}$,  than any transaction identifier previously
used by $\cl$. The judgement $\ET \vdash (\hh, \vi'') \triangleright
\opset : \vi'$
states that the fingerprint $\opset$ is compatible with kv-store $\hh$
and client view $\vi''$, and the resulting view  is $\vi'$. In
Section~\ref{4}, 
we give many examples of such relations.
Finally, we need to address how a fingerprint  $\opset$ of a
transaction executed
by client $\cl$  with view $\vi$  update a  key-value store $\hh$. 




Having selected a suitable transaction identifier $t$, we push the
fingerprint $\opset$ 
into $\hh$ as follows: for each read operation $(\otR, \ke, \_) \in \opset$, we add $t$ 
to the set of readers of the last version of $\ke$ that is included in the view $\vi$ of the client; 
for each write operation $(\otW, \ke, \val)$, we append a new version $(\val, t, \emptyset)$ 
to the tail of $\hh(\ke)$.
In the definition below, we use $\lcat$ to denote the concatenation of two lists; 
if $\vilist = \ver_0, \cdots, \ver_n$ and $i=0,\cdots,n$, 
$\vilist\rmto{i}{\ver}$ denotes the updated list 
$\vilist' = \ver_0, \cdots, \ver_{i-1}, \ver, \ver_{i+1}, \cdots,
\ver_{n}$. 

\begin{definition}[Transaction Update]
\label{eq:updatekv}
\label{def:updatekv}
Let $\hh \in \HisHeaps, \vi \in \Views(\hh)$, $t \in
\TxID_{0}$ and 
$\opset \subseteq \powerset{\Ops} $.  
The transaction update function,  $\updateKV(\hh, \vi, \txid, \opset) $,  is
defined by:
\begin{align*}         
    & \updateKV(\hh, \vi, \txid, \emptyset) \defeq  \hh \\
    & \updateKV(\hh, \vi, \txid, \opset \uplus \Set{(\otR, \ke, \val)}) \\
    & \quad \defeq 
    \begin{array}[t]{@{}l}
        \texttt{let} \ (\val, \txid', \txidset) = \hh(\ke, \max_{<}(\vi(\ke))), \\
        \vilist = \hh(\ke)\rmto{\max_{<}(\vi(\ke))}{(\val, \txid', \txidset \uplus \{ t \})}\\
        \quad \texttt{in} \ \updateKV(\hh\rmto{\ke}{\vilist}, \vi, \txid, \opset)
    \end{array} \\
    & \updateKV(\hh, \vi, \txid, \opset \uplus \Set{(\otW, \ke, \val)} ) \\
    & \quad \defeq 
    \begin{array}[t]{@{}l}
        \texttt{let } \hh' = \hh\rmto{\ke}{ ( \hh(\ke) \lcat (\val, \txid, \emptyset) ) } \\
        \quad \texttt{in } \updateKV(\hh', \vi, \txid, \opset)
    \end{array}  
\end{align*}
For client $\cl$, given $\hh \in \HisHeaps$, $\vi \in \Views(\hh)$ and 
$\opset \subseteq \powerset{\Ops} $, the transaction update set 
for $\cl$ is 
\begin{align*}
&\updateKV(\hh, \vi, \cl, \opset)  \\
 & \quad \defeq \Setcon{\updateKV(\hh, \vi, \txid, \opset)}{\txid \in
    \nextTxId(\hh, \cl)}
\end{align*}
\end{definition}

\pg{In above, can the layout be better? I would like to put the transaction identifier $t$ at the
  end of the arguments, since this is what's happening in the
  premises. I also don't like $\cl$ where it is.}

\pg{There is a disconnect. $\ET \vdash (\hh, \vi'') \triangleright
  \opset : \vi'$ says $\vi'$ makes sense with respect to $\hh, \vi'',
  \opset$. Nothing says that $\vi'$ makes sense with $\updateKV(\hh,
  \vi, \txid, \opset)$. I always find myself concerned about $\ET \vdash (\hh, \vi'') \triangleright
  \opset : \vi'$ because $\vi'$ is a view of an updated store which is
not part of the $\ET$ relation. I would add it.}


Note that,  under the assumption that read-write set  $\opset$ contains at most one read and one write 
operation per key and the identifier is fresh for $\hh$, 
the transaction update function and the transaction update set for
$\cl$ are well-defined. 


\pg{This is out of place here. It could go after ET-traces.Two read-write sets $\opset_1, \opset_2$ are \emph{conflict-free} if 
$\forall \ke \in \Keys.\; (\otW,\ke, \_) \in \opset_1 \implies \forall
\val \in \Val.\; (\otW, \ke, \val) \notin \opset_2$, and vice versa.
The commitment of two  conflict-free read-write sets 
$\opset_1, \opset_2$ into a key-value store $\hh$ does not depend on the order in which the commits 
are performed. }

Figure~\ref{.} also contains the rule for programs. 
A \emph{client environment}, $\thdenv \in \ThdEnv$, is a function  from client identifiers to stacks. 
We assume that the domain of client environments  is the same as the
the domain of the program throughout the execution: 
that is,
$\dom(\thdenv) = \dom(\prog)$.
Program transitions are simply defined in terms of the transitions of
their constituent client commands. 
This  yields a  standard interleaving semantics for concurrent
programs:
that is, 
a client performs a reduction in an atomic step without
affecting other clients. 
\pg{ I
  don't like the notation $\thdenv $, nothing else looks like this and
it's not as important as the notation suggests.}








\section{Consistency Guarantees}

Traditional concistency guarantees for distributed databases capture
what it means for distributed data to be consistent. They have been
formally described by axiomatic conditions associated with 
 dependency graphs~\cite{.} and abstract
execution graphs~\cite{.}. We provide consistency guarantees for our
centralised kv-stores by defining a {\em consistency model}, which is
a set of kv-stores representing the possible outcomes that can be
obtained as the result of multiple clients committing several
transactions each.  The set of such transactions of the client are  restricted
to those whose effects comply with the consistency guarantee of the model. To
achieve this, we define consistency models induced by an {\em
  execution test}, which is a relation which determines whether a
client may commit a transaction into a kv-store.  We demontrate in
Section~\ref{.} that our consistency models for kv-stores correspond
to the traditional consistency guarantees for distributed databases,
with different execution tests corresponding to different graph
axioms.




An execution test, formally a set $\ET$ of quadruples of the form $(\mkvs, \vi, \opset, \vi')$,
determines when a client with view $\vi$ on kv-store $\hh$ 
is allowed to commit an atomic  transaction with fingerprint
$\opset$  and obtain an updated view $\vi'$. 


\begin{definition}
\label{def:execution.test}
An \emph{execution test} is a set of tuples $\ET \subseteq \HisHeaps \times \Views \times \powerset{\Ops} \times \Views$ 
such that for all $(\hh, \vi, \opset, \vi') \in \ET$:
\begin{align}
    & 
    \begin{array}{@{}l@{}}
    \fora{\otR, \ke, \val} (\otR, \ke, \val) \in \opset \\
    \quad {} \implies 
    \hh(\ke, \max{}_{<}(\vi(\ke))) = \val  
    \end{array}
    \tag{Ext} \label{eq:read-external} \\
    & 
    \begin{array}{@{}l@{}}
    \fora{\ke} \vi(\ke) \neq \vi'(\ke) \\
    \quad {} \implies
    ( (\otR, \ke, \_) \in \opset \vee (\otW, \ke, \_)) \in \opset) 
    \end{array}
    \tag{ValidViewUpd} \label{eq:valid-view-update}
\end{align}
\end{definition}
\noindent 
The first condition enforces the last-write-wins policy~\cite{}: that
is, a transaction always reads the most recent writes from the initial
view.  The second condition states that a transaction is only allowed
to update the view for those keys that have been recorded in the
fingerprint. 
We often write   $\ET
\vdash (\hh, \vi) \triangleright \opset: \vi'$  for $(\mkvs, \vi,
\opset, \vi') \in \ET$. 

\noindent Given an execution set  $\ET$, we define the $\ET$-trace which is a
sequence of $\ET$-reductions on configutations that either moves the
client view to be a more up-to-date view, or commits a fingerprint of
a transaction. 

\begin{definition}[$\ET$-trace]
\label{def:reduction}
An \emph{action} $\alpha \in \Act$ has either the form $(\cl, \varepsilon)$, 
or $(\cl, \opset)$, for client 
$\cl$ and fingerprint $\opset$. 
Given an execution test $\ET$, the $\ET$-\emph{reduction relation},
$\xrightarrowtriangle{}_{\ET} \subseteq \Confs \times \Act \times \Confs$, 
is the smallest relation such that for all $\vi, \vi', \cl, \hh, \hh', \viewFun, \opset$:
\begin{enumerate}
	\item 
    $
    \viewFun(\cl) = \vi 
    \wedge \vi \sqsubseteq \vi' 
    \implies (\hh, \viewFun) \xrightarrowtriangle{(\cl, \varepsilon)}_{\ET} 
    (\hh, \viewFun\rmto{\cl}{\vi'})$; and
	\item 
    $\viewFun(\cl) = \vi
        \wedge ((\mkvs, \vi,
\opset, \vi') \in \ET
        \wedge \hh' \in \updateKV(\hh, \vi, \cl, \opset) \implies
	$  \\
	\phantom{a} \hfill 
	$(\hh, \viewFun) \xrightarrowtriangle{(\cl, \opset)}_{\ET} (\hh', \viewFun\rmto{\cl}{\vi'})$
\end{enumerate}
Given an execution test $\ET$, an \emph{$\ET$-trace} is a sequence of $\ET$-reductions of the form $\conf_{0} \xrightarrowtriangle{\alpha_{0}}_{\ET} \cdots 
\xrightarrow{\alpha_{n-1}} \conf_{n}$.
\end{definition}


\pg{In 2, how do we know that $\hh'$ and $\vi'$ are compatible?}



\noindent A consistency model induced by $\ET$ is a set of kv-stores
that  result from
an $\ET$-trace starting from the 
initial configuration (Definition~\ref{views??}). 

\begin{definition}[Consistency Model]
\label{def:cm}
Given an execution test $\ET$ and initial configuration $\conf_0$ (Definition~\ref{view??}),
the set of configurations $\Confs(\ET) $   induced by $\ET$ is   given by: 
\[
\Confs(\ET)\defeq \Setcon{ \conf_n}{ \exsts{\conf_0} \conf_0 \text{ 
    initial}  \wedge \conf_0 \xrightarrowtriangle{\stub}_{\ET} \cdots
  \xrightarrowtriangle{\stub}_{\ET} \conf_n }
\]
The \emph{consistency model} induced by $\ET$ is:
\( 
\CMs(\ET) \defeq \Setcon{ \hh }{ (\hh, \stub) \in \Confs(\ET) }
\)
\end{definition}
\pg{Above  definition of $\Confs(\ET) $ seems a slightly informal definition. Layout needs attention.}

\noindent In~\cref{sec:mono-et}, we prove that consistency models are 
\emph{monotonic}: 
if  $\ET_1 \subseteq \ET_2$ then $\CMs(\ET_1) \subseteq \CMs(\ET_2)$.
In~\cref{sec:et-comm}, we also prove that  $\CMs(\ET_1 \cap \ET_2) = \CMs(\ET_1) \cap
\CMs(\ET_2)$ under certain conditions on the $\ET_i$.


\pg{Theorem  commented out below must be incorrect. I think you will
  need both  $\ET_1$ and $\ET_2$ commutative.}



%\begin{definition}
%Two triples $(\cl_1, \opset_1)$ and $(\cl_2, \opset_2)$ are 
%conflicting if either $\cl_1 = \cl_2$, or there exists a key $\ke$ such that 
%$(\otW, \ke, \_) \in \opset_1, (\otW, \ke, \_) \in \opset_2$. 

%%An execution test is $\ET$ is \emph{commutative} if, whenever $(\cl_1, \vi_1, \opset_1)$, 
%$(\cl_2, \vi_2, \opset_2)$ are non-conflicting, and $\vi_1, \vi_2 \in \Views(\hh_0)$,  
%then for any $\hh_0, \hh', \viewFun, \viewFun'$ we have that 
%\[
%\begin{array}{lr}
%(\hh_0, \viewFun) \xrightarrowtriangle{(\cl_1, \opset_1)}_{\ET} 
%\_ \xrightarrowtriangle{(\cl_2, \opset_2)}_{\ET} (\hh', \viewFun') &\implies \\
% (\hh_0, \viewFun) \xrightarrowtriangle{(\cl_2, \opset_2)}_{\ET} 
%\_ \xrightarrowtriangle{(\cl_1, \opset_1)}_{\ET} (\hh', \viewFun')
%\end{array}
%\]
%%%%%\end{definition}

%\begin{definition}
%An execution test $\ET$ has \emph{no blind writes} if, whenever $\ET \vdash (\hh, \vi) \triangleright \opset \cup \{(\otW, \ke, \_)%\} : \vi'$, 
%then $(\otR, \ke, \_) \in \opset$.
%\end{definition}

%\begin{definition}
%An execution test $\ET$ has \emph{minimum footprints} if for any key-value store \( \hh \)
%views \( \vi, \vi',\vi''\) and fingerprint \( \f \),
%\[
%\begin{array}{@{}l@{}}
 %   ( \fora{ \ke} (\stub, \ke, \stub) \in \f \implies \vi(\ke) = \vi'(\ke) ) \land {} \\
 %   \quad \ET \vdash (\hh, \vi) \triangleright \opset : \vi'' \implies \ET \vdash (\hh, \vi') \triangleright \opset : \vi''
%\end{array}
%\]
%\end{definition}

%\begin{definition}
%An execution test $\ET$ has \emph{continuous post-views} if for any key-value store \( \hh \)
%views \( \vi, \vi',\vi''\) and fingerprint \( \f \), 
%\[
%\begin{array}{@{}l@{}}
 %   \quad \ET \vdash (\hh, \vi) \triangleright \opset : \vi' \land \vi' \sqsubseteq \vi'' \implies \ET \vdash (\hh, \vi) \triangleright \o%pset : \vi''
%\end{array}
%\]
%\end{definition}


%\begin{theorem}                                                                            
%Let $\ET_1, \ET_2$ be two execution tests has no blind writes, minimum footprints and continuous post-views.
%If $\ET_1$ is commutative, 
%then $\CMs(\ET_1 \cap \ET_2) = \CMs(\ET_1) \cap \CMs(\ET_2)$. 
%Furthermore, if $\ET_1, \ET_2$ are commutative, then $\ET_1 \cap \ET_2$ 
%is commutative.
%\end{theorem}
%\begin{proof}
  %  See \cref{sec:et-comm}.
%\end{proof}



\subsection{Example Execution Tests}

\begin{figure*}
\begin{tabular}{ l @{} r }
\hline
Consistency Model & Execution Test: \((\hh, \vi) \csat \opset : \vi'\)\\
\hline
\MRd & $\vi \viewleq \vi'$\\
\MW & 
$j \in \vi(\ke) \wedge \WTx(\hh(\ke', i)) \xrightarrow{\PO?} \WTx(\hh(\ke, j)) 
\implies i \in \vi(\ke')$
\\
\RYW & $ \mkvs' = \updateKV(\hh, \vi, \txid, \opset) \implies \WTx( \mkvs'(\ke, i) ) \leq \txid \implies i \in \vi'(\ke) $\\
\WFR & $j \in \vi(\ke) \wedge \txid \in \RTx(\hh(\ke', i)) \wedge \txid {\xrightarrow{\PO?}}
\WTx(\ke, j) ) \implies i \in \vi(\ke')$\\
\CC & $\ET_{\CC} = \ET_{\MRd} \cap \ET_{\MW} \cap \ET_{\RYW} \cap \ET_{\WFR}$\\
\hline
\hline
\UA & $(\otW, \ke,  \stub) \in \opset \land 0 \leq i < \lvert \hh(\ke)
      \rvert \implies i \in \vi(\ke) $\\
\PSI & $\ET_{\PSI} = \ET_{\CC} \cap \ET_{\UA}$\\
\CP & \( \Setcon{(\mkvs, \vi, \f, \vi')}{\ddagger} \cap \ET_\MRd \cap \ET_\RYW \) \\
$\SI$ & $\Setcon{(\mkvs, \vi, \f, \vi')}{\dagger} \cap \ET_\MRd \cap \ET_\RYW  \cap \ET_\UA $\\
\SER & $ 0 \leq i < \lvert \hh(\ke) \rvert \implies  \in \vi(\ke) $\\
\hline
\end{tabular}

\[
        \dagger \equiv 
        \fora{\ke, \ke', i, j}. 
                i \in \vi(\ke)  \wedge \WTx(\hh(\ke', j)) \toEdge{((\PO \cup \RF_{\hh} \cup \VO_{\hh}) ; \AD_{\hh}?)^{+}} \WTx(\hh(\ke, i))
          \implies j \in \vi(\ke')    
\]
\[  \ddagger  \equiv 
        \fora{\ke, \ke', i, j}.
             i \in \vi(\ke)  \wedge \WTx(\hh(\ke', j)) \toEdge{(((\PO \cup \RF_{\hh}) ; \AD_{\hh}?) \cup \VO_{\hh})^{+}} \WTx(\hh(\ke, i))
         \implies j \in \vi(\ke') 
\]
\caption{Execution tests for  client-centric (top) and data-centric 
 (bottom) consistency models. The relations 
  $\RF_{\hh} $, $\VO_{\hh} $ and $\AD_{\hh} $ are well-known relations
  on dependancy graphs~\cite{.}, and are given in the main
  text. The relation $  \PO $ is the session order given in
  Section~\ref{3.1}. 
All the free variables are universally quantified.
}
\label{fig:execution.tests}
\label{fig:execution-tests}
\end{figure*}

\begin{figure*}[t]
\captionsetup[subfigure]{aboveskip=-5pt, belowskip=5pt}
\begin{tabular}{@{} c | c | c @{}}
\hline
\phantom{-}& \phantom{-}& \phantom{-}\\
\begin{subfigure}{0.2\textwidth}
\begin{centertikz}

%Location x
\node(locx) {$\ke_1 \mapsto$};
\draw pic at ([xshift=\tikzkvspace]locx.east) {vlist={versionx}{%
    /$\val_0$/$\txid_0$/$\Set{\txid_\cl^2}$
    , /$\val_1$/$\txid_1$/$\Set{\txid_\cl^1}$
}};

\end{centertikz}\vspace{5pt}%
\caption{Disallowed by \(\MRd\)}
\label{fig:mr-disallowed}
\end{subfigure}
%\quad
&
\begin{subfigure}{0.36\textwidth}
\begin{centertikz}

%Location x

\node(locx) {$\ke_1 \mapsto$};
\draw pic at ([xshift=\tikzkvspace]locx.east) {vlist={versionx}{%
    /$\val_0$/$\txid_0$/$\Set{\txid'}$
    , /$\val_1$/$\txid_\cl^1$/$\emptyset$
}};

%Location y
\path (versionx.east) + (1,0) node (locy) {$\ke_2 \mapsto$};
\draw pic at ([xshift=\tikzkvspace]locy.east) {vlist={versiony}{%
    /$\val_0$/$\txid_0$/$\emptyset$
    , /$\val_2$/$\txid_\cl^2$/$\Set{\txid'}$
}};

\end{centertikz}\vspace{5pt}
\caption{Disallowed by \(\MW\)}
\label{fig:mw-disallowed}
\end{subfigure}
%\quad
&
\begin{subfigure}{0.39\textwidth}
\begin{centertikz}

%Location x
\node(locx) {$\ke_1 \mapsto$};
\draw pic at ([xshift=\tikzkvspace]locx.east) {vlist={versionx}{%
    /$\val_0$/$\txid_0$/$\Set{\txid}$
    , /$\val_1$/$\txid'$/$\Set{\txid_\cl^1}$
}};

%Location y
\path (versionx.east) + (1,0) node (locy) {$\ke_2 \mapsto$};
\draw pic at ([xshift=\tikzkvspace]locy.east) {vlist={versiony}{%
    /$\val_0$/$\txid_0$/$\emptyset$
    , /$\val_2$/$\txid_\cl^2$/$\Set{\txid}$
}};

\end{centertikz}

\vspace{5pt}
\caption{Disallowed by \(\WFR\)}
\label{fig:wfr-disallowed}
\end{subfigure}\\
\hline
\end{tabular}
%
%
%
%
\begin{tabular}{@{} c | c | c @{}}
\phantom{-}& \phantom{-}& \phantom{-}\\
\begin{subfigure}{0.25\textwidth}
\begin{centertikz}%

%Location x
\node(locx) {$\ke_1 \mapsto$};
\draw pic at ([xshift=\tikzkvspace]locx.east) {vlist={versionx}{%
    /$0$/$\txid_0$/$\Set{\txid_\cl^1,\txid_\cl^2}$
    , /$1$/$\txid_\cl^1$/$\emptyset$
    , /$1$/$\txid_\cl^2$/$\emptyset$
}};

\end{centertikz}%
\vspace{5pt}
\caption{Disallowed by \(\RYW\)}
\label{fig:ryw-disallowed}
\end{subfigure}
& 

\begin{subfigure}{0.40\textwidth}
\begin{centertikz}%

%Location x
\node(locx) {$\ke_1 \mapsto$};
\draw pic at ([xshift=\tikzkvspace]locx.east) {vlist={versionx}{%
    /$\val_0$/$\txid_0$/$\Set{\txid_2}$
    , /$\val_1$/$\txid_1$/$\emptyset$
}};

%Location y
\path (versionx.east) + (1,0) node (locy) {$\ke_2 \mapsto$};
\draw pic at ([xshift=\tikzkvspace]locy.east) {vlist={versiony}{%
    /$\val_0$/$\txid_0$/$\Set{\txid_1}$
    , /$\val_2$/$\txid_2$/$\emptyset$
}};

\end{centertikz}%
\vspace{5pt}
\caption{Write skew, disallowed by \(\SER\)}
\label{fig:ser-disallowed}
\end{subfigure}%
&
\begin{subfigure}{0.30\textwidth}
\begin{centertikz}

\node(locx) {$\ke_1 \mapsto$};
\draw pic at ([xshift=\tikzkvspace]locx.east) {vlist={versionx}{%
    /$0$/$\txid_0$/$\Set{\txid,\txid'}$
    , /$1$/$\txid$/$\emptyset$
    , /$1$/$\txid'$/$\emptyset$
}};

\end{centertikz}
\vspace{5pt}
\caption{Lost update, disallowed by \(\UA\)}
\end{subfigure}
\\
\hline
\end{tabular}
%
%
%
%
\phantom{x}\vspace{7pt}
\begin{tabular}{@{} c | c @{}}
\phantom{-}& \phantom{-} \\
\begin{subfigure}{0.42\textwidth}
\begin{centertikz}%
%Location x
\node(locx) {$\ke_1 \mapsto$};
\draw pic at ([xshift=\tikzkvspace]locx.east) {vlist={versionx}{%
    /$\val_0$/$\txid_0$/$\Set{\txid_{\cl_2}^1}$
    , /$\val_1$/$\txid$/$\Set{\txid_{\cl_1}^1}$
}};

%Location y
\path (versionx.east) + (1,0) node (locy) {$\ke_2 \mapsto$};
\draw pic at ([xshift=\tikzkvspace]locy.east) {vlist={versiony}{%
    /$\val_0$/$\txid_0$/$\Set{\txid_{\cl_1}^2}$
    , /$\val_1$/$\txid$/$\Set{\txid_{\cl_2}^2}$
}};

\end{centertikz}%
\vspace{5pt}
\caption{Long fork, disallowed by \(\CP\)}
\label{fig:cp-disallowed-2}
\label{fig:cp-disallowed}
\end{subfigure}
&
\begin{subfigure}{0.542\textwidth}%
\begin{centertikz}%
%Location x
\node(locx) {$\ke_1 \mapsto$};
\draw pic at ([xshift=\tikzkvspace]locx.east) {vlist={versionx}{%
    /$\val_0$/$\txid_0$/$\Set{\txid_4}$
    , /$\val_1$/$\txid_1$/$\emptyset$
    , /$\val_2$/$\txid_2$/$\emptyset$
}};

%Location y
\path (versionx.east) + (1,0) node (locy) {$\ke_2 \mapsto$};
\draw pic at ([xshift=\tikzkvspace]locy.east) {vlist={versiony}{%
    /$\val_0$/$\txid_0$/$\Set{\txid_2}$
    , /$\val_3$/$\txid_3$/$\Set{\txid_4}$
    , /$\val_4$/$\txid_4$/$\emptyset$
}};

%%Location z
%\path (versiony.east) + (1,0) node (locz) {$\ke_3 \mapsto$};
%\matrix(versionz) [version list,column 2/.style={text width=7mm}]
   %at ([xshift=\tikzkvspace]locz.east) {
 %{a} & $\txid_0$ & {a} & $\txid_3$ & {a} & $\txid_4$ \\
  %{a} & $\emptyset$ & {a} & $\emptyset$ & {a} & $\emptyset$\\
%};

%\tikzvalue{versionz-1-1}{versionz-2-1}{locz-v0}{$\val_0$};
%\tikzvalue{versionz-1-3}{versionz-2-3}{locz-v1}{$\val_1$};
%\tikzvalue{versionz-1-5}{versionz-2-5}{locz-v2}{$\val_2$};

%%Location w
%\path (versionz.east) + (1,0) node (locw) {$\ke_4 \mapsto$};
%\matrix(versionw) [version list,column 2/.style={text width=7mm}]
    %at ([xshift=\tikzkvspace]locw.east) {
    %{a} & $\txid_0$ & {a} & $\txid_1$ \\
    %{a} & $\{\txid_4\}$ & {a} & $\emptyset$ \\
%};
%\tikzvalue{versionw-1-1}{versionw-2-1}{locw-v0}{$\val_0$};
%\tikzvalue{versionw-1-3}{versionw-2-3}{locw-v1}{$\val_1$};
\end{centertikz}
\vspace{5pt}
\caption{Allowed by \( \UA \) and \( \CP \) but disallowed by \(\SI\)}%
\label{fig:si-disallowed}%
\end{subfigure} \\
\hline
\end{tabular}
%\begin{tabular}{@{}c @{} c @{}}
%\begin{minipage}{0.4\textwidth}
%\begin{subfigure}{\textwidth}
%\begin{centertikz}
%%Location x
%\node(locx) {$\ke_1 \mapsto$};
%
%\matrix(versionx) [version list,,column 2/.style={text width=8mm},column 4/.style={text width=7mm}]
%    at ([xshift=\tikzkvspace]locx.east) {
%    {a} & $\txid_0$ & {a} & $\txid_{\cl}^{1}$\\
%    {a} & $\left\{\txid_{\cl'}^{2}\right\}$ & {a} & $\emptyset$ \\
%};
%\tikzvalue{versionx-1-1}{versionx-2-1}{locx-v0}{$\val_0$};
%\tikzvalue{versionx-1-3}{versionx-2-3}{locx-v1}{$\val_1$};
%
%%Location y
%\path (locx.south) + (0,\tikzkeyspace) node (locy) {$\ke_2 \mapsto$};
%\matrix(versiony) [version list,column 2/.style={text width=8mm},column 4/.style={text width=7mm}]
%   at ([xshift=\tikzkvspace]locy.east) {
% {a} & $\txid_0$ & {a} & $\txid_{\cl'}^1$ \\
%  {a} & $\left\{\txid_{cl}^{2}\right\}$ & {a} & $\emptyset$\\
%};
%
%\tikzvalue{versiony-1-1}{versiony-2-1}{locy-v0}{$\val_0$};
%\tikzvalue{versiony-1-3}{versiony-2-3}{locy-v1}{$\val_2$};
%
%\end{centertikz}
%\caption{Disallowed by \(\CP\)}
%\label{fig:cp-disallowed-2}
%\end{subfigure}
%
%\begin{subfigure}{\textwidth}
%\begin{centertikz}
%%Location x
%\node(locx) {$\ke_1 \mapsto$};
%
%\matrix(versionx) [version list,column 2/.style={text width=7mm},column 4/.style={text width=7mm}]
%    at ([xshift=\tikzkvspace]locx.east) {
%    {a} & $\txid_0$ & {a} & $\txid_1$\\
%    {a} & $\left\{\txid_2\right\}$ & {a} & $\emptyset$ \\
%};
%\tikzvalue{versionx-1-1}{versionx-2-1}{locx-v0}{$\val_0$};
%\tikzvalue{versionx-1-3}{versionx-2-3}{locx-v1}{$\val_1$};
%
%%Location y
%\path (locx.south) + (0,\tikzkeyspace) node (locy) {$\ke_2 \mapsto$};
%\matrix(versiony) [version list,column 2/.style={text width=7mm},column 4/.style={text width=7mm}]
%   at ([xshift=\tikzkvspace]locy.east) {
% {a} & $\txid_0$ & {a} & $\txid_2$ \\
%  {a} & $\left\{\txid_1\right\}$ & {a} & $\emptyset$\\
%};
%
%\tikzvalue{versiony-1-1}{versiony-2-1}{locy-v0}{$\val_0$};
%\tikzvalue{versiony-1-3}{versiony-2-3}{locy-v1}{$\val_2$};
%\end{centertikz}
%\caption{Disallowed by \(\SER\)}
%\label{fig:ser-disallowed}
%\end{subfigure}
%
%\end{minipage}
%
%&
%\begin{subfigure}{0.55\textwidth}%
%\begin{centertikz}%
%%Location x
%\node(locx) {$\ke_1 \mapsto$};
%
%\matrix(versionx) [version list,column 2/.style={text width=7mm}]
%    at ([xshift=\tikzkvspace]locx.east) {
%    {a} & $\txid_0$ & {a} & $\txid_1$ & {a} & $\txid_2$\\
%    {a} & $\emptyset$ & {a} & $\emptyset$ & {a} & $\emptyset$\\
%};
%\tikzvalue{versionx-1-1}{versionx-2-1}{locx-v0}{$\val_0$};
%\tikzvalue{versionx-1-3}{versionx-2-3}{locx-v1}{$\val_1$};
%\tikzvalue{versionx-1-5}{versionx-2-5}{locx-v1}{$\val_2$};
%%Location y
%\path (locx.south) + (0,\tikzkeyspace) node (locy) {$\ke_2 \mapsto$};
%\matrix(versiony) [version list,column 2/.style={text width=7mm}]
%   at ([xshift=\tikzkvspace]locy.east) {
% {a} & $\txid_0$ & {a} & $\txid_3$ \\
%  {a} & $\left\{\txid_2\right\}$ & {a} & $\emptyset$\\
%};
%
%\tikzvalue{versiony-1-1}{versiony-2-1}{locy-v0}{$\val_0$};
%\tikzvalue{versiony-1-3}{versiony-2-3}{locy-v1}{$\val_1$};
%
%%Location z
%\path (locy.south) + (0,\tikzkeyspace) node (locz) {$\ke_3 \mapsto$};
%\matrix(versionz) [version list,column 2/.style={text width=7mm}]
%   at ([xshift=\tikzkvspace]locz.east) {
% {a} & $\txid_0$ & {a} & $\txid_3$ & {a} & $\txid_4$ \\
%  {a} & $\emptyset$ & {a} & $\emptyset$ & {a} & $\emptyset$\\
%};
%
%\tikzvalue{versionz-1-1}{versionz-2-1}{locz-v0}{$\val_0$};
%\tikzvalue{versionz-1-3}{versionz-2-3}{locz-v1}{$\val_1$};
%\tikzvalue{versionz-1-5}{versionz-2-5}{locz-v2}{$\val_2$};
%
%%Location w
%\path (locz.south) + (0,\tikzkeyspace) node (locw) {$\ke_4 \mapsto$};
%\matrix(versionw) [version list,column 2/.style={text width=7mm}]
%    at ([xshift=\tikzkvspace]locw.east) {
%    {a} & $\txid_0$ & {a} & $\txid_1$ \\
%    {a} & $\{\txid_4\}$ & {a} & $\emptyset$ \\
%};
%\tikzvalue{versionw-1-1}{versionw-2-1}{locw-v0}{$\val_0$};
%\tikzvalue{versionw-1-3}{versionw-2-3}{locw-v1}{$\val_1$};
%\end{centertikz}%
%\caption{Disallowed by \(\SI\)}%
%\label{fig:si-disallowed}%
%\end{subfigure} \\
%\end{tabular}
\hrulefill
\vspace*{-5pt}
\caption{Behaviours disallowed under different consistency models}
\label{fig:anomalies}
\vspace*{-10pt}
\end{figure*}


\pg{I am using distributed databases to mean replicated or sharded
  databases. This will be made clear, once and for all, in the
  introduction and ever afterwards we just refer to distributed databases.}

We now give examples of execution tests in~\cref{fig:execution.tests},
where the associated consistency models for kv-stores correspond to
widely adopted consistency guaranteees for distributed databases.
Following \cite{distrprinciples}, we distinguish between
client-centric and data-centric consistency models: the former
constrain the client views; and the latter impose conditions on the
structure of the kv-store.  In Figure~\ref{.}, we give illustrative
examples of allowed and disallowed key-value stores for our
consistency models.


\noindent {\bf Monotonic Reads ($\MRd$).}
It ensures that read operations from subsequent transactions always return a more up-to-date versions.
For example, the key-value store of \cref{fig:mr-disallowed} is disallowed by $\MRd$.
Because the client $\cl$ first observes the latest version of $\ke$ in $\txid_{\cl}^{1}$,
then it observes the initial version of $\ke$ in $\txid_{\cl}^{2}$.
The execution test $\ET_{\MRd}$ prevents this scenario by forcing clients to always update their views to newer ones. 

\noindent {\bf Monotonic Writes ($\MW$).}
It states that whenever a transaction observes the effects of a version installed by some client $\cl$,
then the transaction observes all the transactions executed by the client. 
It prevents the scenario of \cref{fig:mw-disallowed}, 
where transaction $\txid'$ observes the second version of $\ke_2$ carrying value $\val_2$, written by client $\cl$;
but it does not observe the second version of $\ke_1$ carrying value $\val_1$, previously written by the same client.
The execution test $\ET_{\MW}$ (\cref{fig:execution.tests}) ensures that, prior to executing a transaction,
the set of versions included in the view of the client must be prefix-closed with respect to the relation $\xrightarrow{\PO}$.

\noindent {\bf Read Your Writes (\RYW).}
It states that a client must always be able to read any version of a key that was previously written by the same client.
This prevents the key-value store of \cref{fig:ryw-disallowed}. 
In the \cref{fig:ryw-disallowed}, the initial version of $\ke$ carries value $0$, 
and the client $\cl$ tries to increment the value of $\ke$ by $1$ twice.
For the first time, it reads the initial value $0$ and then installing a new version carrying  value $1$ within a single transaction.
However, since the client does not need to read its own writes, 
the client might read the initial value $0$ again in the second increment transaction \( \txid_\cl^2 \),
and install a new version carrying value $1$.
The Read Your Writes ($\RYW$) (\cref{fig:execution.tests}) enforces that after committing a transaction, 
a client includes all the versions it wrote.  

\noindent {\bf Write Follows Reads (\WFR).}
It states that if a client \( \cl \) writes some version $\ver$ in a transaction,
following  another transaction (or in the same transaction of) who reads of some version $\ver'$, 
then a transaction may observe version $\ver$ only if it also observes $\ver'$. 
The Write Follow Reads ($\WFR$) disallows the scenario of \cref{fig:wfr-disallowed} 
where a transaction $\txid$ observes the version $\ver_2$ of $\ke_2$ carrying value $\val_2$ written by client $\cl$,
but the same transaction $\txid$ does not observe the version of $\ke_1$ carrying value $\val_1$, read by $\cl$ prior to writing $\ver$. 
The execution test $\ET_{\WFR}$ (\cref{fig:execution.tests}) prevents this scenarios 
by enforcing a view includes all the versions previous read by some client \( \cl \), 
if the view already include a write from that client \( \cl \).

\noindent {\bf Causal Consistency (\CC).}
Causal Consistency requires that if a client observes a version $\ver$, 
then it must also observe any version $\ver'$ from which $\ver$ potentially depends \cite{cops}. 
The dependency here means session order and write-read relation.
For session order, it means when a view includes some effect from a client, 
it must include previous effect from the same client.
For write-read relation, it means when a view includes a transaction (the versions it write),
it must include all the writes that the transaction read from.
A necessary and sufficient condition is to enforce the four session guarantees $\MRd, \MW, \RYW$ and $\WFR$ \cite{session2causal}.
Therefore, we let $\ET_{\CC} = \ET_{\MRd} \cap \ET_{\MW} \cap \ET_{\RYW} \cap \ET_{\WFR}$. 

\paragraph{Update Atomic ($\UA$).}
This consistency model has been proposed in \cite{framework-concur}, 
though we are not aware of any implementation. 
However, many implemented consistency models can be obtained by strengthening Update Atomic.
Update Atomic disallows concurrent transactions writing to the same key. 
This property is known as \emph{write conflict detection}.
For example, $\UA$ prevents the key-value store of \cref{fig:ua-disallowed},
where two transactions $\txid, \txid'$ concurrently increment the initial version of $\ke$ by $1$.
Note that this scenario generalises the one exhibited by $\RYW$, 
since we do not require $\txid, \txid'$ to be executed by the same client.
To prevent this scenario, the execution test $\ET_\UA$ requires that 
a client $\cl$ can write to key $\ke$ in a transaction,
only if its view prior to the execution includes the last version of
$\ke$.


\paragraph{Parallel Snapshot Isolation (\PSI)} 
Parallel Snapshot Isolation (\PSI) can be obtained by combining causal consistency with update atomic, 
$\ET_{\PSI} = \ET_{\CC} \cap \ET_{\UA}$, and $\ET_{\SI} = \ET_{\CC}
\cap \ET_{\UA}$.


\pg{This is where we should put these definitions.}

\[
    \begin{rclarray}
        \RF_{\hh} &\defeq& \{ (\txid, \txid') \mid \exists \ke, i.\; \txid = \WTx(\hh(\ke, i)) \wedge \txid' \in \RTx(\hh(\ke, i))\}\\
       \VO_{\hh} &\defeq& \{ (\txid, \txid') \mid \exists \ke, i, j.\; \txid = \WTx(\hh(\ke, i)) \wedge \txid' = \WTx(\hh(\ke, j)) \wedge i < j\}\\
        \AD_{\hh} &\defeq& \{ (\txid, \txid') \mid \exists \ke, i, j.\; \txid \in \RTx(\hh(\ke, i)) \wedge \txid' = \WTx(\hh(\ke, j)) \wedge i < j\}\\
        \PO &\defeq& \{ (\txid, \txid') \mid \exists \cl, m, n.\;
        \txid = \txid_{\cl}^{m} \wedge \txid' = \txid_{\cl}^{n} \wedge
        m < n \}
\end{rclarray}
\]


\paragraph{Consistent Prefix ($\CP$).}
\label{para:cp}
In centralised databases, where there is a total order in which transactions commit, 
Consistent Prefix is described by the following property: 
if a client observes the effect of a transaction $\txid$,
then it also observe the effect of any transaction $\txid'$ that commits before $\txid$.
It is difficult to formulate in key-value store,
because key-value stores do not contain the full information about the total order in which transactions committed. 
Inspired by dependency graph \cite{.....},
there are minimum observable transactions for each transaction derived from the following:
\[
    \SO  \subseteq  \VIS \qquad
    ( ( ( \SO \cup \WR ) ; \RW? )^* \cup \WW ) ; \VIS \subseteq \VIS
\]
where the \( R? \) is the reflexive closure of the relation \( R \) 
and \( R_1 ; R_2 \defeq \Setcon{(a,b)}{\exsts{c} (a,c) \in R_1 \land (c,b) \in R_2 } \) is the composition of the two relation.
The session order relation \( (\txid, \txid') \in \SO \) means the session order;
write-read relation \( (\txid, \txid') \in \WR \) means \( \txid' \) reads the write of \( \txid \);
read-write relation \( (\txid, \txid') \in \RW \) means \( \txid' \) read a old version of a key 
and \( \txid' \) installs a new version for the same key;
and \( (\txid, \txid') \in \VIS \) means the view exactly before \( \txid' \) should include all effect \( \txid \).
Given the minimum observable transactions, we can specify $\CP$. 
First, \( \SO \subseteq \VIS \) means a transaction observes all previous transactions from the same client,
and it is enforced by \( \ET_\RYW \).
Then the combination of \( \ddagger\) (\cref{fig:execution-tests}) and \( \ET_\MRd \) gives us \( ( ( ( \SO \cup \WR ) ; \RW? )^* \cup \WW ) ; \VIS \subseteq \VIS \).
Let consider a client \( \cl \) and the view \( \vi \).
Assume two transactions \( \txid, \txid' \)  such that \( \txid' \) is in the view \( \vi \) and \( \txid \toEdge{( ( ( \SO \cup \WR ) ; \RW? )^* \cup \WW )} \txid' \).
If \( \txid' \) is a transaction already observable by some previous transaction from the client \( \cl \), 
the transaction \( \txid \) must be observable by that time,
therefore by the \( \ET_\MRd \), the transaction \( \txid \) is in the current view \( \vi \).
Otherwise, if \( \txid' \) is a transaction that is first time observed by the client \( \cl \),
the \( \dagger \) predicate enforces \( \txid \) is also in the view \( \vi \).
Intuitively, the \( \CP \) disallows that a transaction observes updates in different order (\cref{fig:cp-disallowed-1}).
In \cref{fig:cp-disallowed-1}, transactions $\txid_{3}$ and \( \txid_4 \) observes updates in different order.
That is, \( \txid_3 \) observes that the update of $\ke_2$ carrying value $\val_2$ happens before the update of $\ke_1$ carrying value $\val_2$,
yet $\txid_{4}$ observes that the update of $\ke_1$ carrying value
$\val_1$ happens before the update of $\ke_2$ carrying value
$\val_2$. 

\paragraph{Snapshot Isolation (\SI)}
When the total order in which transactions commit is known,
SI can be specified as the weakest consistency model that guarantees both 
Consistency Prefix and Update Atomic \cite{gsi,framework-concur}. 
Yet, it is not true in our framework, since transactions are not totally ordered.
For example, the kv-store of \cref{fig:si-disallowed} is included in both $\CMs(\ET_{\CP})$ and $\CMs(\ET_{\UA})$, 
but it is forbidden by snapshot isolation in general.
The reason is \( \CMs(\ET_1 \cap \ET_2) = \CMs(\ET_1) \cap \CMs(\ET_2) \) holds 
only under some conditions of the \( \ET_1 \) and \( \ET_2 \) (more detail in \cref{thm:et-comm} \cref{sec:et-comp}).
\footnote{%
    This problem is not limited to our setting: 
    because kv-stores are isomorphic to Adya's dependency 
    graph, the same problem arises there.%
} 
We are inspired by the following constraint that has been proven satisfying \( \SI \) \cite{cerone:snapshot}:
\[
    (\SO \cup \WW) \subseteq \VIS \quad  ( (\SO \cup \WW \cup \WR) ; \RW? ) ; \VIS \subseteq \VIS
\]
where write-write relation \( (\txid, \txid') \in \WW \) means the transaction \( \txid \) installs a version for a key \( \ke \) following by \( \txid' \) installing a new version for the key \( \ke \).
The constraint \( \SO \subseteq \VIS \) coincides with \( \ET_\RYW \).
The \( \WW \subseteq \VIS \) means two transactions cannot concurrently write to the same key,
which is enforced by \( \ET_\UA \).
Let consider \( ( (\SO \cup \WW \cup \WR) ; \RW? ) ; \VIS \subseteq \VIS \).
Similar to the argument we made in Consistent Prefix (\pageref{para:cp}), 
let assume a client \( \cl \), its view \( \vi \) and two transactions \( \txid, \txid' \) such that 
\( \txid' \) is in the view \( \vi \)
and \( \txid \toEdge{(\SO \cup \WW \cup \WR) ; \RW?} \txid' \).
If \( \txid' \) is observable by any previous transaction of the client \( \cl \),
then \( \txid \) is also observable before.
By \( \ET_\MRd\), it is the case \( \txid \) is in the view \( \vi \).
If \( \txid' \) is a new transaction observed by the client \( \cl \),
the \( \dagger \) enforces that \( \txid \) should be included \( \vi \).

\paragraph{(Strict) serialisability (\SER)}
Serialisability is the strongest consistency model, 
which requires that there exists a serial or sequential schedule of transaction. 
This prevents scenarios of \cref{fig:ser-disallowed},
which is instead allowed by all the other execution tests that we have presented.
The execution test $\ET_{\SER}$ requires 
clients to execute transactions only when their view of the key-value
store is up-to-date.



An example of execution test is serialisability (\SER).
The execution test for \( \SER \) requires the initial view  \( \vi \) contains all the versions in the key-value store \( \mkvs \), that is, \( \fora{\ke, i < \abs{\mkvs(\ke)} } i \in \vi(\ke)\).
It matches the intuition that a transaction should observe the up-to-date state of the database.
Later in \cref{sec:spec} we present more examples of execution tests of well-known consistency models in the literature. 
 


------------------

%\subsection{Multi-version Key-value Stores and Views}
\label{sec:mkvs-view}

\subsubsection{Multi-version Key-value Stores} 
We assume a countably infinite set of \emph{client identifiers} $\Clients \defeq \Set{\cl, \cl',\cdots}$. 
We define the set of \emph{transaction identifiers} 
$\TxID \defeq  \Set{\txid_{0}} \uplus \Set{ \txid_{\cl}^{n} \mid \cl \in \Clients \wedge n \geq 0 }$, where 
 $\txid_0$ denotes a designated transaction used for initialisation, 
 and for each $n \in \mathbb{N}$, $\txid_{\cl}^{n}$ identifies a transaction 
 committed by client $\cl$.
% the $n$\textsuperscript{th} transaction of client $\cl$. 
Elements of $\TxID$ are ranged over by $\txid, \txid', \cdots$, 
while subsets of $\TxID$ are ranged over by $\txidset, \txidset', \cdots$. 
We define $\TxID_{0} \defeq \TxID \setminus \{ \txid_0\}$.
As we will see, we assume that each client is bound to a single session, and 
we use the superscript $n$, in a transaction identifier $\txid_{\cl}^{n}$, 
to embed the information about the session order $\PO$ in which clients execute 
transactions:  
%
%The structure of $\TxID$  
%embeds the transaction execution order for each client, or the \emph{session order} $\PO$. 
%More concretely, 
$\PO \defeq \Set{ (\txid, \txid') \mid \exsts{ \cl, n,m } \txid = \txid_{\cl}^{n} \wedge \txid' = \txid_{\cl}^{m} \wedge n < m}$.
%As such, $(\txid, \txid') \in \PO$ denotes that 
%client $\cl$ executes $\txid$ before $\txid'$.
For readability, we often write  $\txid \xrightarrow{\PO} \txid'$ for $(\txid, \txid') \in \PO$.

%Given a set $X$, 
%%then $\powerset{X}$ denotes 
%%the powerset of $X$,
%we write $X^{\ast}$ for the free monoid induced by $X$.
%We next define the notion of \emph{multi-version key-value stores}.


\begin{definition}[Multi-version Key-value Stores]
\label{def:his_heap}
\label{def:mkvs}
Assume a countably infinite set of \emph{keys} $\Keys = \Set{\ke, \ke', \cdots}$, 
and a set of \emph{values} $\Val = \{\val, \val', \cdots\}$.

A \emph{version} is a triple $\ver \in \Versions \defeq \Val \times \TxID \times \powerset{\TxID_{0}}$. 
%The set of versions is denoted by $$.
A \emph{key-value store} is a mapping $\hh: \Keys \rightarrow \Versions^{\ast}$, 
where we recall that $\Versions^{\ast}$ is the free monoid generated by $\Versions$.
%\ac{ The superscript fin over the $\rightharpoonup$ needs to be fixed. You may want to look at the package extpfeil.}
\end{definition}

%For simplicity, we instantiate the set of values as  $\Val \eqdef \Nat \uplus \Keys$,
Among the elements of $\Val$, we distinguish a default value $\val_0 \in \Val$. 
A \emph{version} $\ver = (\val, \txid, \txidset)$ comprises a value $\val$,
and the meta-data of the transactions that accessed it.
Specifically, the \emph{writer} $\txid$ identifies the transaction that wrote version $\ver$, 
and the \emph{readers} $\txidset$ denote the set of transactions that read from $\ver$.
Given a version $\ver = (\val, \txid, \txidset)$, we define $\valueOf(\ver) \defeq \val$,
$\WTx(\ver) \defeq \txid$ and $\RTx(\ver) \defeq \txidset$.
Lists of versions (\ie elements of $\Versions^{\ast}$) are ranged over by $\vilist, \vilist',\cdots$.

%A \emph{multi-version key-value store}, or \emph{kv-store}, 
%is a mapping from keys to lists of versions. 
Given kv-store $\hh$, key $\ke$ and index $i \geq 0$, 
we write $\hh(\ke, i)$ for the $i$-th version (starting from $0$) of $\ke$.
That is, if $\hh(\ke) = \ver_0 \cdots\ver_{n}$, then $\hh(\ke, i) \defeq \ver_{i}$ for $i \leq n$; 
and it is undefined otherwise. 
We write $\lvert \hh(\ke) \rvert$ for the length of $\hh(\ke)$. 

We focus on key-value stores whose consistency model enforces the \emph{atomic visibility} of transactions~\cite{framework-concur}. 
This amounts to requiring that transaction reads at most one version of each key, and similarly 
it writes at most one version for each key. From the point of view of key-value stores, 
these conditions amount to require that \textbf{(i)}
$\fora{\ke i, j} (o \leq i, j \abs{ \hh(\ke) } \land \RTx(\hh(\ke, i)) \cap \RTx(\hh(\ke, j)) \neq \emptyset ) \implies i = j$, 
\textbf{(ii)}
$\fora{\ke, i, j} (0 \leq i,j < \abs{ \hh(\ke) } \wedge \WTx(\hh(\ke, i)) = \WTx(\hh(\ke, j)) ) \implies i = j$. 
We also assume that the list of version for each key has an initial version carrying a default value $\val_0$, 
written by the designated initialisation transaction $\txid_0$: \textbf{(iii)} $\fora{\ke} \hh(\ke, 0) = (\val_0, \txid_0, \stub)$.
Finally, we assume that the state of a key-value store is consistent with 
the session order of clients; a client cannot read a version of a key that has 
been installed by a future transactions within the same session, and 
the order in which versions are installed by a single client must agree 
with its session order: \textbf{(iv)}
$\fora{ \ke, \cl, i,j, n, m} 0 \leq i < j < \abs{\hh(\ke)} 
    \land \txid_{\cl}^{n} = \WTx(\hh(\ke,i)) {} \wedge \txid_{\cl}^{m} \in \Set{\WTx(\hh(\ke,j))} \cup \RTx(\hh(\ke, i))
    \implies n < m $.
%
%Formally, we require the following well-formedness requirement from key-value stores: 
%
%\begin{enumerate}%[label=(\roman*)]
%\item\label{kv:wf.init} 
%    $\hh(\ke, 0) = (\val_0, \txid_0, \stub)$ for $\ke \in \dom(\hh)$, where $\val_0$ is the default value in $\Val$;
%\item\label{kv:wf.onewrite} 
%    transactions write at most one version for each key:
%\[
%\fora{\ke, i,j }
%0 \leq i, j < \abs{ \hh(\ke) }
%\land \WTx(\hh(\ke, i)) = \WTx(\hh(\ke, j))
%\implies i = j 
%\]
%\item\label{kv:wf.oneread} 
%    transactions read at most one version for each key:
%\[
%\fora{\ke, i,j } 
%0 \leq i, j < \abs{ \hh(\ke) }
%\land \RTx(\hh(\ke, i)) \cap \RTx(\hh(\ke, j)) \neq \emptyset 
%\implies i = j
%\]
%\item\label{kv:wf.so} 
%	transactions (of the same client) install different versions of a key in the session order; 
%%    the order in which transactions issued by the same client install different versions for a key $\ke$, 
%%    is consistent with the session order;
%    a client $\cl$ can read versions written by $\cl$ itself only after they have been installed:
%\begin{multline*}
%    \fora{ \ke, \cl, i,j, n, m} 
%    0 \leq i < j < \abs{\hh(\ke)} 
%    \land \txid_{\cl}^{n} = \WTx(\hh(\ke,i)) \\
%    {} \wedge \txid_{\cl}^{m} \in \Set{\WTx(\hh(\ke,j))} \cup \RTx(\hh(\ke, i))
%    \implies n < m
%\end{multline*}
%\end{enumerate}
%
We say that kv-stores that satisfy the conditions \textbf{(i)}-\textbf{(iv)} above are 
\emph{well-formed}.
Henceforth, we will always assume kv-stores tp be well-formed, and we use $\HisHeaps$ to denote 
the set of well-formed kv-stores.

\subsubsection{Views and Configurations}

Key-value stores track the global state of a database. 
However, clients do not need to agree on the portion of 
the state of the database that they observe. Different clients 
may observe different different subsets of versions of the same key.
when executing transactions,
%different \emph{clients} may observe different versions of the same key. 
To keep track of the versions observed by clients,
we introduce the notion of \emph{views} (\cref{def:view}). 

\begin{definition}[Views, configurations]
\label{def:view}
\label{def:cuts}
\label{def:views}
\label{def:configuration}
A \emph{view} of a kv-store $\hh$ is a mapping  
$\vi \in \Views(\hh) \defeq \Keys \to\powerset{\Nat}$ such that:
\begin{align}
    & \fora{ \ke } 
    0 \in \vi(\ke) 
    \wedge \fora{ i \in \vi(\ke) } 
    i < \abs{ \hh(\ke) } 
    \tag{WF}
    \label{eq:view.wf}\\
    %\Set{0} \subseteq \vi(\ke) \subseteq \Setcon{i}{ 0 \leq i < \abs{\mkvs(\ke)}}
    & 
    \fora{ \ke_1,\ke_2, i_1, i_2} 
	i_1 \in \vi(\ke_1) 
	\land \WTx(\hh(\ke_1, i_1)) = \WTx(\hh(\ke_2, i_2)) 
	\implies i_2 \in \vi(\ke_2)
	\tag{Atomic}
	\label{eq:view.atomic}
\end{align}
A \emph{configuration} $\conf \in \Confs$, is a pair $(\hh, \viewFun)$, 
where $\hh \in \HisHeaps$ and
$\viewFun : \Clients \parfinfun \Views(\hh)$. 
\end{definition}
Configurations extend key-value stores with the information of 
the views of each client. In a configuration $\conf = (\hh, \viewFun)$, the view of client 
$\cl$, $\vi = \viewFun(\cl)$ (if defined) determines for each key $\ke \in \Keys$ the sub-list of versions in $\hh$ 
that the client is aware of, or equivalently that it can observe. If $i,j \in \vi(\ke)$ and $i < j$, then the client is 
aware of the fact that $\hh$ contains the versions $\hh(\ke, i)$, $\hh(\ke, j)$, and that $\hh(\ke, j)$ is more 
up-to-date than $\hh(\ke, i)$. The client also observes the information relative to the versions $\hh(\ke, i)$ and 
$\hh(\ke, j)$, i.e. the value they carry and the meta-data relative to writing and reading transactions of such 
versions. 
Equation \eqref{eq:view.wf} in \cref{def:view} is a natural requirement, while \eqref{eq:view.atomic} 
models the atomic visibility of transactions: if a client observes the updates of a transaction $\txid$, then 
it must observe all the updates from $\txid$. 
We let $\Views = \bigcup_{\hh \in \HisHeaps} \Views(\hh)$ be the set of all view. 
Given a kv-store $\hh$ and two views $\vi, \vi' \in \Views(\hh)$, 
we write $\vi \viewleq \vi'$ when $\vi(k) \subseteq \vi'(\ke)$ for all $\ke \in \dom(\hh)$. 
The initial view $\vi_{0}$ is defined as $\vi_{0}(\ke) = \{0\}$ for each $\ke \in \Keys$.
A configuration $\conf_{0} = (\hh_{0}, \viewFun_{0})$ is
\emph{initial} if $\hh_{0}(\ke) = (\val_0, \txid_0, \emptyset)$ for all $\ke \in \Keys$.

Given a configuration $\conf = (\hh, \viewFun)$ and a client $\cl$ for which 
$\viewFun(\cl)$ is defined, the view $\vi(\cl)$ is used to determine a \emph{snapshot}, i.e. 
a mapping from each key to a unique value that the client observes when executing a transaction. 
In general, the snapshot of a transaction is also determined by a \emph{resolution policy} 
cite{}. Throughout this paper, we assume that the database employ the \emph{Last Writer 
Wins} \cite{} resolution policy to determine the snapshot of clients, although generalisation 
to different resolution policies is straightforward.

\begin{definition}[Snapshots]
\label{def:heaps}
\label{def:snapshot}
A snapshot is a mapping from keys to values \( \ss \in \Snapshots  \defeq \Keys \to \Val\).
Given $\hh \in \HisHeaps$ and $\vi \in \Views(\hh)$, the \emph{snapshot} of $\vi$ in 
$\hh$ is defined as $\snapshot(\hh, \vi) \defeq \lambda \ke \ldotp \valueOf(\hh(\ke, \max_{<}(\vi(\ke)))$, 
where we recall that $\max_{<}(\vi(\ke))$ is the maximum element in $\vi(\ke)$ with respect to the natural 
order $<$ over $\mathbb{N}$.
\end{definition}
Given a kv-store $\hh$, a key $\ke$ and a view $\vi$, we often commit an abuse of notation and write 
$\hh(\ke, vi)$ as a shorthand for 
$\hh(\ke, \max_{<}(\vi(\ke))$. Thus, $\snapshot(\hh, \vi) = \lambda \ke \ldotp \valueOf(\hh(\ke, \vi))$. 

\begin{remark}
Because the function $\snapshot(\hh, \vi)$ only selects the last version of $\hh$ comprised 
in $\vi$, one may wonder about the necessity of including multiple versions in the view of a 
key $\ke$.  Here we only point that requiring a view to contain a single version for each key 
would impair the expressiveness of our framework in terms of consistency models that it captures; 
unfortunately, we have to wait until \cref{} before giving more details on this issue.
\end{remark}

%Let $\vi$ be on a key-value store $\hh$; the view $\vi$ determines, 
%for each key $\ke \in \Keys$, the sub-list of versions in $\hh(\ke)$ 
%$\vi(\ke)$ determines the subset of versions that 
%  
%\emph{The set of views} 
%$
%\Views \defeq \bigcup_{\hh \in \HisHeaps} \Views(\hh)
%$.
%A \emph{configuration}, $\conf \in \Confs$, is a pair $(\hh, \viewFun)$, 
%where $\hh \in \HisHeaps$ and
%$\viewFun : \Clients \parfinfun \Views(\hh)$. 
%The $\conf_{0} = (\hh_{0}, \viewFun_{0})$ is an
%\emph{initial configuration} if, $\hh_{0}(\ke) = (\val_0, \txid_0, \emptyset)$ for all $\ke \in \Keys$.
%
%A view of a kv-store $\hh$ is a mapping from the keys in $\hh$ to a non-empty set of natural numbers. 
%For each $\ke \in \dom(\hh)$, $\vi(\ke)$ denotes the indices of versions in $\hh(\ke)$ recorded in $\vi$. 
%As such, when $i \in \vi(\ke)$ then $ i < \abs{ \hh(\ke) }$. 
%Moreover, the initialisation version (at index $0$) must be included in all views. 
%These two properties are captured by \eqref{eq:view.wf} in \cref{def:view} below. 
%Lastly, views cannot observe \emph{partial} effects of a given transaction. 
%That is, if a view includes a version written by a transaction $\txid$, it must include \emph{all} versions written by $\txid$. 
%This is formalised by \eqref{eq:view.atomic} in \cref{def:view} below, and captures the \emph{atomic visibility} of transactions. 


%At any point during execution, the overall state is captured by \emph{a configuration}. 
%A configuration includes a kv-store and a partial mapping from clients to views.
%The view of the client $\cl$ in $\hh$ reflects the set of versions for each key 
%that the client \(\cl \) observes upon executing a transaction. 
%The constraint of \cref{eq:view.atomic} establishes that if a client observes 
%a version of some key written by a transaction $\txid$, then it must observe all the versions of 
%all keys that $\txid$ wrote. 


%\begin{definition}[Views, configurations]
%\label{def:view}
%\label{def:cuts}
%\label{def:views}
%\label{def:configuration}
%A \emph{view} of a kv-store $\hh$ is a mapping  
%$\vi \in \Views(\mkvs) \defeq \dom(\hh) \to\powerset{\Nat}$ such that:
%\begin{align}
%    & \fora{ \ke } 
%    0 \in \vi(\ke) 
%    \wedge \fora{ i \in \vi(\ke) } 
%    i < \abs{ \hh(\ke) } 
%    \tag{WF}
%    \label{eq:view.wf}\\
%    %\Set{0} \subseteq \vi(\ke) \subseteq \Setcon{i}{ 0 \leq i < \abs{\mkvs(\ke)}}
%    & 
%    \fora{ \ke,\ke', i,j} 
%	j \in \vi(\ke) 
%	\land \WTx(\hh(\ke, j)) = \WTx(\hh(\ke', i) 
%	\implies i \in \vi(\ke')
%	\tag{Atomic}
%	\label{eq:view.atomic}
%\end{align}
%
%\emph{The set of views} is
%$
%\Views \defeq \bigcup_{\hh \in \HisHeaps} \Views(\hh)
%$.
%A \emph{configuration}, $\conf \in \Confs$, is a pair $(\hh, \viewFun)$, 
%where $\hh \in \HisHeaps$ and
%$\viewFun : \Clients \parfinfun \Views(\hh)$. 
%The $\conf_{0} = (\hh_{0}, \viewFun_{0})$ is an
%\emph{initial configuration} if, $\hh_{0}(\ke) = (\val_0, \txid_0, \emptyset)$ for all $\ke \in \Keys$.
%\end{definition}



%Given a kv-store $\hh$ and two views $\vi, \vi' \in \Views(\hh)$, 
%we write $\vi \viewleq \vi'$ when $\vi(k) \subseteq \vi'(\ke)$ for all $\ke \in \dom(\hh)$. 
%Also, we commit an abuse of notation and write $\hh(\ke, \vi)$ as a shorthand 
%for $\hh(\ke, \max_{<}(\vi(\ke)))$. 
%Note that such a version always exists as
%$\vi(\ke) \neq \emptyset$ (see \eqref{eq:view.wf} above).

%\subsubsection{Snapshots}
%Transactions are executed with respect to a \emph{snapshot} of a kv-store.
%A snapshot $\h$ is a mapping from keys to values (\cref{def:snapshot}). 
%Given a view $\vi$ of a transaction, a snapshot can be induced 
%by extracting the value of the latest observable version for each key $\ke \in \dom(\hh)$. 
%%Views are used to determine the snapshot in which a transaction 
%%is executed, according to the following definition.


%\begin{definition}[Snapshots]
%\label{def:heaps}
%\label{def:snapshot}
%A snapshot is a mapping from keys to values \( \ss \in \Snapshots  \defeq \Keys \to \Val\).
%Given $\hh \in \HisHeaps$ and $\vi \in \Views(\hh)$, the \emph{snapshot} of $\vi$ in 
%$\hh$ is defined as $\snapshot(\hh, \vi) \defeq \lambda \ke \ldotp \valueOf(\hh(\ke, \max_{<}(\vi(\ke)))$.
%\end{definition}

\paragraph{Fingerprints}
Once the execution of a transaction is completed, its effects are committed to the kv-store. 
The effects of transactions are modelled as a \emph{fingerprint} $\opset$. 
A finger print comprises a set of \emph{operations} $\Ops$: $\opset \subseteq \Ops$. 
An operation is a triple of the form $(l, \ke, \val)$ with $l \in \{\otR, \otW\}$.    
Intuitively, given the fingerprint $\opset$ of a transaction $\txid$, 
$(\otR, \ke, \val) \in \opset$ denotes that 
$\txid$ requested to read key $\ke$ from the kv-store, 
and it fetched a version carrying value $\val$.
Similarly, $(\otW, \ke, \val) \in \opset$ denotes that 
$\txid$ attempted to write value $v$ for key $\ke$. 
A fingerprint includes at most one read operation per key;
this formalises the intuition that, in our setting, 
transactions always read from an atomic snapshot of the kv-store. 
Analogously, a fingerprint includes at most one write operation per key.
%because a client either observes none or all the updates of a transaction.

\begin{definition}[Fingerprints]
The set of \emph{operations} is
$\Ops \defeq \{(l, \ke, \val) \mid$ $ l \in \Set{\otW, \otR} \land \ke \in \Keys \wedge \val \in \Val \}$.
A \emph{fingerprint} $\opset$ is a subset of operations, $\opset \subseteq \Ops$,
such that for all $\ke \in \Keys$ and \( l \in \Set{\otW, \otR} \),
if $(l, \ke, \val_1), (l, \ke, \val_2) \in \opset$, then $\val_1 = \val_2$.
\end{definition}



%\ac{Note that I now require fingerprints to be non-empty sets of transactions. This simplifies a lot the development of 
%the theory of kv-stores, and it fixes a problem that was spotted by Shale, that breaks the compositionality of 
%execution tests (see later). The main reason why we allowed empty fingerprints is that in the semantics, a client can 
%execute a transaction with no access to the memory. In practice, in the semantics we can require that at least 
%one access to the database must be performed in transactions. This can be checked syntactically, and nobody 
%should complain about that. I can put a remark about how this is a natural requirement that, if violated, 
%breaks the compositionality of consistency models.\\ 
%\textbf{Update 02/08/2018}: empty fingerprints are now allowed again. We still had some problems with compositionality, 
%one of which has to do with the fact that we allow the view of a client over some key to move freely after executing a transaction, 
%even if such a key was not accessed by the transaction. Later, I forbid this behaviour by requiring in execution tests that the 
%view of an client for a given key cannot be shifted if the transaction executed by the client did not access such a key.}
%$(\otW, \ke, \val) \in \opset$ means that the transaction writes a new version, carrying value $\val$, for key $\ke$. 




%of $\vi$ by accessing the value of 
%A view $\vi$ in $\hh$ naturally defines a snapshot $\snapshot(\hh, \nu)$
%A MKVS tracks the global state of the system; however, different \emph{clients} may observe different versions of the same key. 
%To model this, we introduce the notion of \emph{views} (\cref{def:views}). 
%A view $V$ reflects the particular version for each key that a client observes upon executing a transaction. 
%%We present an example of views in \cref{fig:hheap-a} with two views: $\client_1$ in red and $\client_2$ in blue.
%More concretely, the view for \( \client_1 \) is given formally as $\vi_1 = \Set{\key{k}_1 \mapsto 1, \key{k}_2 \mapsto 0}$.
%That is, the client with view $\vi_1$ observes the second version (at index 1) of key \( \ke_{1} \) with value $v_1$, and the first version (at index 0) of key \( \ke_2 \) with value $v'_0$.
%%, and 
%%the first version of $\key{k}_2$, carrying value $0$. Similarly, according to its view 
%%$V_2 = [\key{k}_1 \mapsto 2, \key{k}_2 \mapsto 2]$, the client $\txid_2$ observes 
%%in $\hh$the second and most up-to-date version for both $\key{k}_1$ and $\key{k}_2$.
%
%\begin{definition}[Views]
%\label{def:view}
%\label{def:cuts}
%\label{def:views}
%\emph{A view} is a partial finite function from keys to indexes:
%$
%\vi \in \Views \defeq \Addr \parfinfun \Nat 
%%\begin{rclarray}
%%    \vi \in \Views & \defeq & \Addr \parfinfun \Nat 
%%\end{rclarray}
%$.                                                                 
%The \emph{view composition}, $\composeVI: \Views \times \Views \rightharpoonup \Views$ is defined as the standard disjoint function union: $\composeVI \eqdef \uplus$. 
%% \( \vi \composeVI \vi' \defeq \vi \uplus \vi'\) 
%The \emph{unit view}, $\unitVI \in \Views$, is a function with an empty domain: $\unitVI \eqdef \emptyset$. 
%% and the unit is \( \unitVI \defeq \emptyset\).
%The \emph{order relation} on views, $\orderVI: \Views \times \Views$, is defined between two views with the same domain as the point-wise comparison of their indexes for each entry: 
%\[
%\begin{rclarray}
%    \vi \orderVI \vi' & \defiff & \dom(\vi) = \dom(\vi') \land \fora{\ke} \cu(\ke) \leq \cu'(\ke) \\
%\end{rclarray}
%\]
%\end{definition}
%%
%We say view $\vi$ is \emph{older} than view $\vi'$ (or $\vi'$ is \emph{newer} than $\vi$) whenever $\vi \orderVI \vi'$ holds.
%
%
%\mypar{Configurations} A \emph{configuration} comprises an MKVS, and the views associated with clients.
%In \cref{fig:hheap-a} we present an example of a configuration comprising an MKVS and the two views associated with clients $\client_1$ and $\client_2$. 
%We write $\version(\hh, \ke, \vi)$ for $\hh(\ke, \vi(\ke))$; 
%and write $\valueOf(\hh, \ke, \vi)$ as a shorthand for $ \valueOf(\version(\hh, \key{k}, V))$; similarly for $\WTx, \RTx$.
%%we commit an abuse of notation and often write $\valueOf(\hh, \ke, \vi)$ in lieu of $ \valueOf(\version(\hh, \key{k}, V))$, and similarly for $\WTx, \RTx$.
%When $\ver = \version(\hh, \ke, \vi)$, we say that \emph{$\vi$ $\ke$-points to $\ver$ in $\hh$}. 
%When $\ver = \hh(\ke, i)$ for some $0 \leq i \le \vi(\ke)$, we say that \emph{$\vi$ $\ke$-includes $\ver$ in $\hh$}.
%Lastly, we always assume that MKVSs, views, and configurations are well-formed, unless otherwise stated.
%
%
%
%\begin{definition}[Configurations]
%A view $\vi$ is \emph{well-formed with respect to an MKVS} $\mkvs$, written \( \wfV{\mkvs, \vi} \),  iff they have the same domain and every index from $\vi$ is within the range of the corresponding entry in $\mkvs$ and the view is \emph{atomic} with  respect to the key-value store: 
%\[
%\begin{rclarray}
%    \wfV{\mkvs, \vi} & \defeq & \dom(\mkvs) = \dom(\vi) \land \fora{\ke \in \dom(\vi)} 0 \leq \vi(\ke) < \lvert \mkvs(\ke) \rvert \\
%    \pred{atomic}{\vi ,\hh} & \eqdef & \fora{\txid } \exsts{\ke, i} i \leq \vi(\ke) \land \hh(\ke,i) = (\stub, \txid, \stub) \implies \pred{visible}{\txid, \vi, \hh} \\ 
%    \pred{visible}{\txid, \vi, \hh} & \eqdef & \fora{\ke, i} \hh(\ke,i) = (\stub, \txid, \stub) \implies i \leq \vi(\ke) 
%\end{rclarray}
%\]
%%
%\azalea{We need a symbol for this to fill the ???? above. Also ???? below. \sx{Done}}
%A \emph{configuration} $\conf$ is a pair of the form $(\hh, \viewFun)$, where $\hh$ denotes an MKVS, and $\viewFun: \Clients \parfinfun \Views$ is a partial finite function from clients to views. 
%A configuration $\conf = (\hh, \viewFun)$ is \emph{well-formed}, written \( \wfC{\conf}\), iff for all clients $\cl \in \dom(\viewFun)$, the view $\viewFun(\txid)$ is well-formed with respect to $\hh$. 
%%We say that a view $V$ is well-defined with respect to the 
%%MKVS $\hh$ if, $\forall \key{k} \in \ke. 0 < V(\key{k}) \leq 
%%\lvert \hh(\key{k}) \rvert$. 
%%Given a view $V$ that is well-defined 
%%with respect to a 
%
%\end{definition}
%
%\mypar{Snapshots} When a client executes a transaction on the $\mkvs$ MKVS, it extracts a \emph{snapshot} of it via the \( \func{snapshot}{\mkvs, \vi} \) function, extracting the values corresponding to the versions indexed by its view \( \vi \) (\cref{def:snapshot}).
%For instance, for client \( \client_1 \) in \cref{fig:hheap-a}, the $\func{snapshot}{\cdots}$ functions yields a state where key $\ke_1$ carries value $v_1$ and second key \( \ke_2 \) carries value $v'_0$.
%%The concrete state extracted in this way takes the name of the \emph{snapshot} of the transaction.
%%In general, the process of determining the view of a client, hence the snapshot in which such a client executes transactions, is non-deterministic.
%
%\azalea{Before in MKVSs we had values drawn from $\Nat$ in \cref{def:mkvs}. Now we use $\Val$. I think you mean to use $\Val$ in both places? \sx{I would say so} }
%\begin{definition}[Snapshots]
%\label{def:heaps}
%\label{def:snapshot}
%Given the sets of values $\Val$  and keys \( \Addr\)  (\cref{def:mkvs}), the set of \emph{snapshots} is:
%$
%    \h \in \Heaps \eqdef \Addr \parfinfun \Val
%$. 
%%\[
%%\begin{rclarray}
%%    \h \in \Heaps & \eqdef & \Addr \parfinfun \Val
%%\end{rclarray}
%%\]
%The \emph{snapshot composition function}, $\composeH: \Heaps \times \Heaps \parfun \Heaps$, is defined as $\composeH \eqdef \uplus$, where $\uplus$ denotes the standard disjoint function union. The \emph{ snapshot unit element} is $\unitH \eqdef \emptyset$, denoting a function with an empty domain.
%The \emph{partial commutative monoid of snapshots} is $(\Heaps, \composeH, \{\unitH\})$.
%Given an MKVS $\hh$ and a view $\vi$, the snapshot of $\vi$ in $\hh$, written $\snapshot(\hh, \vi) $, is defined as:
%$
%    \snapshot(\hh, \vi) \defeq \lambda \ke \ldotp \valueOf(\hh, \ke, \vi)
%$.
%%\[
%%\begin{rclarray}
%%    \snapshot(\hh, \vi) & \defeq & \lambda \ke \ldotp \valueOf(\hh, \ke, \vi).
%%\end{rclarray}
%%\]
%\end{definition}
%
%\sx{Need some explanation}
%\ac{General Comment on this Section: it is too abstract. We 
%should give either here or in the introduction an example of computation - 
%the write skew program should be okay that helps the reader understanding 
%what's going on. Also, it could be also good to illustrate the notions 
%of execution tests and consistency models.}
%
%\sx{From Andrea: introduce the execution test here with a table, also introduce fingerprint here}


%\subsection{Consistency Models and Execution Tests}
Formally, a \emph{consistency model} $\CMs$ is a set of key-value stores. 
Each $\hh \in \CMs$ represents a possible scenario that 
can be obtained as a result of multiple clients committing transactions. 
To specify consistency models we introduce the notion of \emph{execution tests}. 
\ac{
For example, \emph{serialisability} can be described as the set 
of key-value stores for which it is possible to recover a total schedule of transactions, 
such that each read operation on key $\ke$ fetches its value from the 
most recent write on the same key \cite{??????}.
In this sense, the kv-store $\hh$ from \cref{fig:hheap-a} is not serialisable: 
transaction $\txid_1$ reads the initial version carrying value $\val'_0$ for key $\ke_{2}$, 
and installs a new version of $\ke_{2}$ carrying value $\val_1$. The transaction $\txid_2$ 
reads the initial version carrying value $\val'_0$, and therefore, 
cannot be scheduled after $\txid_1$. Similarly, $\txid_2$ cannot be scheduled after $\txid_1$.
}
\begin{definition}
\label{def:execution.test}
An \emph{execution test} is a set of tuples $\ET \subseteq \HisHeaps \times \Views \times \powerset{\Ops} \times \Views$ 
such that for every element $(\hh, \vi, \opset, \vi') \in \ET$,
\textbf{(i)} for any read operations $(\otR, \ke, \val) \in \opset$ then $\hh(\ke, \max_{<}(\vi(\ke))) = \val$, 
and \textbf{(ii)}  for any key \( \ke \) such that $\vi(\ke) \neq \vi'(\ke)$, 
then $( (\otR, \ke, \_) \in \opset \vee (\otW, \ke, \_)) \in \opset)$.
%\textbf{(iii)} $\forall \opset' \subseteq \opset.\; (\hh, \vi, \opset', \vi') \in \ET$
\end{definition}
%\sx{The definition has a problem that for the subset \( \f'\) the post-view \( \vi' \) might point to an undefined version, I also thing it should satisfy \( \fora{\vi''} \vi \sqsubseteq \vi'' \implies (\hh, \vi, \opset', \vi'') \)}.
%\ac{I removed this condition, as I do not think that it was used anywhere. The new definition requires that 
%you cannot change the view for keys that you do not read nor write.}
%\sx{The current \CP will not satisfy the \textbf{(ii)}. }
Given an execution test $\ET$, 
then $(\hh, \vi, \opset, \vi') \in \ET$ means that 
a client whose view over the key-value store $\hh$ is $\vi$, 
can commit a transaction whose fingerprint is $\opset$;
as a result of this operation, the view of the client must be updated to $\vi'$.
Henceforth, we adopt the more suggestive notation $\ET \vdash (\hh, \vi) \triangleright \opset: \vi'$ 
in lieu of $(\hh, \vi, \opset, \vi') \in \ET$.
Execution tests induce \emph{consistency models} \( \CMs(\ET) \) as defined in \cref{def:reduction,def:cm}.
\begin{definition}[$\ET$-reductions]
\label{def:reduction}
Let $\cl$ be a client and $\opset$ be a fingerprint. 
An \emph{action} $\alpha \in \Act$ has either the form $(\cl, \varepsilon)$, 
or $(\cl, \opset)$. 
Given an execution test $\ET$ the action-labelled relation 
$\xrightarrowtriangle{}_{\ET} \subseteq \Confs \times \Act \times \Confs$ 
is defined as the smallest relation such that:
\begin{itemize}
\item 
    $\forall \vi, \vi', \cl, \hh, \viewFun.\; 
    \viewFun(\cl) = \vi 
    \wedge \vi \sqsubseteq \vi' 
    \implies (\hh, \viewFun) \xrightarrowtriangle{(\cl, \varepsilon)}_{\ET} 
    (\hh, \viewFun\rmto{\cl}{\vi'})$
\item 
    $\begin{array}[t]{@{}l@{}}
        \forall \vi, \vi', \cl, \opset, \hh, \hh', \viewFun.\; 
        \viewFun(\cl) = \vi
        \wedge (\ET \vdash (\hh, \vi) \triangleright \opset: \vi')  \\
        \quad {} \wedge \hh' \in \updateKV(\hh, \vi, \cl, \opset) 
        \implies (\hh, \viewFun) \xrightarrowtriangle{(\cl, \opset)}_{\ET} (\hh', \viewFun\rmto{\cl}{\vi'})
    \end{array}$
\end{itemize}
Such relations take the name of $\ET$-reductions, or simply reductions.
\end{definition}
Given an execution test $\ET$, sequences of $\ET$-reductions of the form $\conf_{0} \xrightarrowtriangle{\alpha_{0}}_{\ET} \cdots 
\xrightarrow{\alpha_{n-1}} \conf_{n}$ take the name of \emph{$\ET$-traces}.
\begin{definition}[Consistency Models]
\label{def:cm}
Given an execution test $\ET$, the set of configurations induced by $\ET$ is given by:
\[
\Confs(\ET) \defeq \Setcon{ \conf}{ \exsts{\conf_0} \conf_0 \text{ is initial } \wedge \conf_0 \xrightarrowtriangle{\stub}_{\ET} \cdots \xrightarrowtriangle{\stub}_{\ET} \conf }
\]
The \emph{consistency model} induced by $\ET$ is:
\( 
\CMs(\ET) \defeq \Setcon{ \hh }{ (\hh, \stub) \in \Confs(\ET) }
\)
\end{definition}
Thus, consistency models are computed from execution tests by closing the set of initial key-value stores with respect to two operations: 
\textbf{(i)} advancing the view of a client, 
and \textbf{(ii)} committing a fingerprint of a transaction. 

Last, for sanity check, consistency models induced by execution tests are monotonic in the following sense.
\begin{proposition}
\label{prop:mono-et}
Let $\ET_1 \subseteq \ET_2$. Then $\CMs(\ET_1) \subseteq \CMs(\ET_2)$.
\end{proposition}
\begin{proof}
    \ifTechReport
    It is sufficient to prove that \(\ET_1 \subseteq \ET_2 \implies \Confs(\ET_1) \subseteq\ Confs(\ET_2) \).
We prove it by induction on the length of the traces, \( n \).

\caseB{n = 0}
We have \( \conf_0 \in \Confs(\ET_1) \) and \( \conf_0 \in \Confs(\ET_2)\).
\caseI(n = i + 1)
Suppose identical traces of \( \ET_1 \) and \( \ET_2 \) respectively with length \( i \).
Let the final configuration be \( \conf_i = ( \mkvs_i, \viewFun_i ) \).
If the next step is a view shift or a step with empty fingerprint, it trivially holds.
If the next step is a step by a client \( \cl \) with fingerprint \( \f \),
we have \( \ET_1 \vdash \mkvs_i, \viewFun_i(\cl) \csat \f : vi' \).
The next configuration from \( \ET_1 \) is \( \conf_{i+1} = (\updateKV{ \mkvs_i, \viewFun_i(\cl), \f, \txid_\cl}) \).
Since \( \ET_1 \subseteq \ET_2 \), so \( \ET_1 \vdash \mkvs_i, \viewFun_i(\cl) \csat \f : vi' \) holds.
It is possible for \( \ET_2 \) to have the exactly same next configuration \( \conf_{n+1}\).

    \else
    See \cref{sec:mono-et}.
    \fi
\end{proof}

