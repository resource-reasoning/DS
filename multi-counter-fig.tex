\begin{figure}

\begin{tabularx}{\textwidth}{@{} c | c @{} }
\hline
\phantom{-}& \phantom{-}\\[-5pt]
\begin{subfigure}{0.39\textwidth}
\centering
\scalebox{.8}{%
\begin{tikzpicture}%
\KVMapping{x}{\key_1}{
    /0/\txidinit/\emptyset
};
\KVMapping[x]{y}{\key_2}{
    /0/\txidinit/\emptyset
};
\end{tikzpicture}%
}
\caption{Initial kv-store}
\label{fig:overview-sec-long-fork-init}
\end{subfigure}
& \begin{subfigure}{0.58\textwidth}
\centering
\scalebox{.8}{%
\begin{tikzpicture}%
\KVMapping{x}{\key_1}{
    /0/\txidinit/\Set{\txid,\txid_3}
    , /1/\txid/\Set{\txid_2}
};
\KVMapping[x]{y}{\key_2}{
    /0/\txidinit/\Set{\txid',\txid_2}
    , /1/\txid'/\Set{\txid_3}
};
\end{tikzpicture}%
}
\caption{Transactions \( \txid_2 \) and \( \txid_3\)
            observe the update to \( \key_1 \) and \( \key_2 \) 
            in different order (\emph{long fork} anomaly)}
\label{fig:overview-sec-long-fork}
\end{subfigure}
\\
\hline
\end{tabularx} 

\caption{Long fork anomaly: multiple counters}
\label{fig:mult-counter}
\end{figure}
