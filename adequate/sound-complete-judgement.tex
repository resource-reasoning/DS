\subsection{Soundness and Completeness Constructor}
\label{sec:kv2aexec-sound-complete}

We now show how all the results illustrated so far 
can be put together to show that the kv-store operational semantics 
is sound and complete with respect to abstract execution operational semantics.

\subsubsection{Soundness}
Recall that in the abstract execution operational semantics,
a client \( \cl \) loses information of the visible transactions immediately after it commits a transaction.
Yet such information is indirectly presented when the next transaction from the same client is committed.
To define the soundness judgement (\cref{def:et_sound}), we introduce a notation of \emph{invariant} (\cref{def:invariant-for-clients})
to encore constraints on the visible transactions after each commit.

\ac{The idea behind client-based invariant being that \(I(\aexec, \cl)\) represents 
the minimal set of transactions that \(\cl\) must see in \(\aexec\), before 
updating the view and performing a transaction. Such a set of transaction 
roughly correspond to the view of the client before performing a 
sequence of \emph{update view+execute transaction} operations, 
or equivalently from the view obtained after the execution of the 
last transaction from that client.}

\begin{definition}[Invariant for clients]
\label{def:invariant-for-clients}
A \emph{client-based invariant condition}, or simply \emph{invariant}, is a 
function \(I : \Aexecs \times \Clients \to \pset{\TxID}\) 
such that for any \(\cl\) we have that \(I(\aexec, \cl) \subseteq \txidset_{\aexec}\), and 
for any  \(\cl'\) such that \(\cl' \neq \cl\) we have that 
\(I(\extend[\aexec, \txid_{\cl'}^{\cdot}, \stub, \stub], \cl) = I(\aexec, \cl)\).
\end{definition}



\begin{definition}[Soundness judgement]
\label{def:et_sound}
An execution test \(\ET\) is sound with respect to an axiomatic 
definition \((\RP_{\LWW}, \Ax)\) if and only if
there exists an invariant condition \(I\) such that 
if assuming that
\begin{itemize}
    \item a client \( \cl \) having an initial view \( \vi \), 
        commits a transaction \( \txid \) with a fingerprint \( \fp \) and updates the view to \( \vi' \), 
        which is allowed by \( \ET \) \ie \(\ET \vdash (\mkvs, \vi) \csat \fp: (\mkvs',\vi')\) where \( \mkvs' = \updateKV[\mkvs, \vi ,\fp, \txid]\),
    \item a \(\aexec\) such that \(\mkvs_{\aexec} = \mkvs\) and \(I(\aexec, \cl) \subseteq \Tx[\mkvs, \vi]\),
\end{itemize}
then there exist a set of read-only transactions \(\txidset_{\rd}\) such that 
\begin{itemize}
    \item the new abstract execution \( \aexec'  = \extend[\aexec,\txid,\Tx[\mkvs,\vi] \cup \txidset_\rd, \fp]\),
    \item the view \( \vi \) satisfies \( \Ax \), \ie \(\fora{ \A \in \Ax } \Set{\txid' }[ (\txid', \txid) \in \A(\aexec')] \subseteq \Tx[\mkvs, \vi] \cup \txidset_{\rd}\), 
    \item the invariant is preserved, \ie \(I(\aexec', \cl) \subseteq \Tx[\mkvs', \vi']\).
\end{itemize}
\end{definition}

\begin{theorem}[Soundness]
\label{thm:et_soundness}
If \(\ET\) is sound with respect to \((\RP_{\LWW}, \Ax)\), then 
\[
    \CMs(\ET) \subseteq \Set{ \mkvs }[ \exsts{ \aexec \in \CMa(\RP_{\LWW}, \Ax)) }\mkvs_{\aexec} = \mkvs]
\]
\end{theorem}
\begin{proof}
Let \(\ET\) be an execution test that is sound with respect to an 
axiomatic definition \((\RP_{\LWW}, \Ax)\). Let \(I\) be 
the invariant that satisfies \cref{def:et_sound}. 
Let consider an \(\ET\)-trace \(\tr\).
We can assume that \(\tr\) is in normal form, 
a trace that every view shift of a client \( \cl \) is followed by a transaction from \( \cl \),
and any transaction from \( \cl \) must be after a view shift of \( \cl \).
Without lose generality, we can also assume that the trace does not have transitions labelled as \((\stub, \emptyset)\).
Thus we have that the following trace \( \tr \):
\begin{align*}
\tr & =  (\mkvs_{0}, \vienv_{0}) \toET{(\cl_{0}, \varepsilon)} (\mkvs_{0}, \vienv_{0}') 
\toET{(\cl_{0}, \fp_{0})} 
(\mkvs_1, \vienv_{1}) \toET{(\cl_1, \varepsilon)}  \cdots
\toET{(\cl_{n-1}, \fp_{n-1})} (\mkvs_{n}, \vienv_{n})
\end{align*}
For any \(i : 0 \leq i \leq n\), let \(\tr_{i}\) be the prefix of \(\tr\) that 
contains only the first \(2i\) transitions. 
Clearly \(\tr_{i}\) is a valid \(\ET\)-trace, and it is also a \(\ET_{\top}\)-trace. 
By \cref{prop:kvtrace2aexec}, 
any abstract execution \(\aexec_{i} \in \aeset(\tr_{i})\) satisfies the last write wins policy. 
We show by induction on \(i\) that we can always find 
an abstract execution \(\aexec_{i} \in \aeset(\tr_{i})\) such that \(\aexec_i \models \Ax\) and \(I(\aexec_{i}, \cl) \subseteq \txidset^{i}_{\cl}\)
for any client \(\cl\) and set of transactions 
\(\txidset^{i}_{\cl} = \Tx[\aexec_{i}, \vienv_{i}(\cl)] \cup \txidset^{i}_\rd\), 
and read-only transactions \(\txidset_\rd^{i}\) in \(\aexec_{i}\).
If so, because \(\aexec_{i}\) satisfies the last write wins policy,
then it must be the case that \(\aexec_{i} \models (\RP_{\LWW}, \Ax)\). 
Then by choosing \(i = n\), we will obtain that \(\aexec_{n} \models (\RP_{\LWW}, \Ax)\). 
Last, by \cref{prop:kvtrace2aexec}, \(\mkvs_{\aexec_{n}} = \mkvs_{n}\), and there is nothing left to prove.
Now let prove such \(\aexec_{i} \in \aeset(\tr_{i})\) always exists.

\caseB{\(i = 0\)} 
Let \(\aexec_{0}\) be the only abstract execution included in \(\aeset(\tr_{0})\), 
that is \(\aexec_{0} = ([], \emptyset, \emptyset)\). 
For any \(\A \in \Ax\), it must be the case that 
\(\A(\aexec_{0}) \subseteq \txidset_{\aexec_{0}} = \emptyset\), 
hence the inequation \(\A(\aexec_{0}) \subseteq \VIS_{\aexec_{0}}\) is trivially satisfies.
Furthermore, for the client invariant \(I\) we also require that \(I(\aexec_{0}, \stub) \subseteq \txidset_{\aexec_{0}} = \emptyset\); 
for any client \(\cl\) we can choose \(\txidset_{\cl}^{0} = \Tx[\mkvs_{\aexec_{0}},\vienv_{0}(\cl)] \cup \emptyset = \emptyset\). 
Therefore \(I(\aexec_{0}, \cl) = \emptyset \subseteq \emptyset = \txidset_{\cl}^{0}\).

\caseI{\(i' = i + 1\) where \(i < n\)}
By the inductive hypothesis, there exists an abstract execution \(\aexec_i\) such that  
\begin{itemize}
\item \(\aexec_{i} \models \A\) for all \(\A \in \Ax\), and 
\item \(I(\aexec, \cl) \subseteq \txidset_{\cl}^{i}\) for any client \(\cl\) and set of transactions \(\txidset_{\cl}^{i} = \Tx[\mkvs_{i}, \vienv_{i}(\cl)]\).
\end{itemize}

We have two transitions to check, the view shift and committing a transaction.
\begin{itemize}
\item the view shift transition \((\mkvs_{i}, \vienv_{i}) \toET{(\cl_{i}, \varepsilon)} (\mkvs_{i}, \vienv'_{i})\). 
By definition, it must be the case that \(\vienv'_{i} = \vienv_{i}\rmto{\cl}{\vi'_{i}}\) 
for some \(\vi'_{i}\) such that \(\vienv_{i}(\cl) \viewleq \vi'_{i}\).
Let \((\txidset_{\cl}^{i})' = \Tx[\mkvs_{i}, \vi'_{i}]\); then we have 
\(
\txidset_{\cl}^{i} = \Tx[\mkvs_{i}, \vienv_{i}(\cl)] \subseteq \Tx[\mkvs_{i}, \vi'_{i}] = (\txidset_{\cl}^{i})' \)
As a consequence, \(I(\aexec, \cl) \subseteq \txidset_{\cl}^{i} \subseteq (\txidset_{\cl}^{i})'\).

\item the commit transaction transition $(\mkvs_{i}, \vienv_{i}') \toET{(\cl_{i}, \fp_{i})}
(\mkvs_{i+1}, \vienv_{i+1})$.
A necessary condition for this transition 
to appear in \(\tr\) is that \(\ET \vdash (\mkvs_{i}, \vienv(\cl)) \csat \fp_{i}: (\mkvs_{i+1},\vienv_{i+1}(\cl))\). 
Because \(I\) is the invariant to derive that \(\ET\) is sound with respect to \(\Ax\), 
and because \(I(\aexec_{i}, \cl_{i}) \subseteq (\txidset^{i}_{\cl})'\), 
then by \cref{def:et_sound} we have the following:
\begin{itemize}
\item there exists a set of read-only transactions \(\txidset_\rd\) 
such that 
\[
    \Set{\txid' }[ (\txid', \txid_{(\cl, i)}) \in \A(\aexec_{i+1})] \subseteq {\txidset^{i}_{\cl}}' \cup \txidset_\rd
\]
where 
\(\txid_{(\cl, i)} \in \nextTxid[\mkvs_{i}, \cl]\)
and \(\aexec_{i+1} = \extend[\aexec_{i}, \txid_{(\cl, i)}, (\txidset^{i}_{\cl})' \cup \txidset_\rd, \fp_{i}]\),
\item  \(I(\aexec_{i+1}, \cl) \subseteq \Tx[\mkvs_{i+1}, \vienv_{i+1}(\cl)]\).
\end{itemize} 
Because \(\aexec_{i} \in \aeset(\tr_{i})\), by definition of \(\aeset(\stub)\) we have that 
\(\aexec_{i+1} \in \aeset(\tr)\) (under the assumption that \(\fp_{i} \neq \emptyset\)), 
and because \(\lastConf(\tr_{i+1}) = (\mkvs_{i+1}, \stub)\), then \(\mkvs_{\aexec_{i+1}} = \mkvs_{i+1}\). 

Now we need to check if \( \aexec_{i+1} \) satisfies \( \Ax\) and the invariant \( I \) is preserved.
\begin{itemize}
\item \(\A(\aexec_{i+1}) \subseteq \VIS_{\aexec}^{i+1}\) for any \(\A \in \Ax\).
Fix \(\A \in \Ax\) and \((\txid', \txid) \in \A(\aexec_{i+1})\). 
Because \(\aexec_{i+1} = \extend[\aexec_{i}, \txid_{(\cl, i)}, (\txidset_{\cl}^{i})' \cup \txidset_\rd, \fp_{i}]\), 
we distinguish between two cases.
\begin{itemize}
\item If \(\txid = \txid_{(\cl, i)}\), then it must be the case that \(\txid' \in (\txidset^{i}_{\cl})' \cup \txidset_\rd\), 
and by definition of \(\extend\) we have that \((\txid' ,\txid_{(\cl, i)}) \in \VIS_{\aexec_{i+1}}\). 
\item If \(\txid \neq \txid_{(\cl, i)}\), then we have that \(\txid, \txid' \in \txidset_{\aexec_{i}}\). 
Because \(\aexec_{i}\) and \(\aexec_{i+1}\) agree on \(\txidset_{\aexec_{i}}\), then \((\txid', \txid) \in \A(\aexec_{i})\).
Because \(\aexec_{i} \models \A\), then \((\txid', \txid) \in \VIS_{\aexec_{i}}\). 
By definition of \(\extend\), it follows that \((\txid', \txid) \in \VIS_{\aexec_{i+1}}\).
\end{itemize}

\item Finally, we show the invariant is preserved.
Fix a client \(\cl'\). 
\begin{itemize}
\item If \(\cl' = \cl\), then we have already proved that 
\(I(\aexec_{i+1}, \cl) \subseteq \txidset_{\cl}^{i+1}\). 
\item if \(\cl' \neq \cl\), then note that \(\vienv_{i}(\cl') = \vienv'_{i}(\cl') = \vienv_{i+1}(\cl')\), 
and in particular \((\txidset^{\cl'}_{i})' = \Tx[\aexec_{i}, \vienv'_{i}(\cl')] = \Tx[\aexec_{i+1}, \vienv_{i+1}(\cl')] =  \txidset_{\cl'}^{i+1}\).
By the inductive hypothesis we know that \(I(\aexec_{i}, \cl) \subseteq \txidset_{\cl'}^{i}\), 
and by the definition of invariant, we have \(I(\aexec_{i+1}, \cl) \subseteq \txidset_{\cl'}^{i} = \txidset_{\cl'}^{i+1}\). 
\end{itemize}
\end{itemize}
\end{itemize}
\end{proof}

\begin{corollary}
\label{cor:et-soundness}
If \(\ET\) is sound with respect to \((\RP_{\LWW}, \Ax)\), then 
for any program \(\prog\), \(\interpr{\prog}_{\ET} \subseteq \Set{ \mkvs_{\aexec} }[ \aexec \in \interpr{\prog}_{(\RP_{\LWW}, \Ax)} ]\).
\end{corollary}
\begin{proof}
\begin{align*}
\interpr{\prog}_{\ET} 
& \stackrel{%\cref{thm:consistency-intersect-permissive}
}{=} 
\interpr{\prog}_{\ET_\top} \cap \CMs(\ET) \\
& \stackrel{%\cref{cor:kvtrace2aexec}
}{=} 
\Set{\mkvs_{\aexec} }[ \aexec \text{ satisfies } \RP_{\LWW}] \cap \CMs(\ET) \\
& \stackrel{\cref{thm:et_soundness}}{\subseteq} 
\Set{\mkvs_{\aexec} }[ \aexec \text{ satisfies } \RP_{\LWW} \land \aexec \in \CMa(\RP_{\LWW}, \Ax) ] \\
& \stackrel{%\cref{thm:consistency-intersect-anarchic}
}{=}
\Set{ \mkvs_{\aexec} }[ \aexec \in \interpr{\prog}_{(\RP_{\LWW}, \Ax)} ]
\end{align*}
\end{proof}

\subsubsection{Completeness}
The Completeness judgement is in \cref{def:et_complete}.
Given a transaction \( \txid_i \) from client \( \cl \), it converts the visible transactions \( \VIS_{\aexec}^{-1}(\txid_{i}) \) into view  and such view should satisfy the \( \ET \).
Note that \( \aexec \) does not contain precise information about final view after update,
yet the visible transactions of the immediate next transaction from the same client \( \cl \) include those information.

\begin{definition}
\label{def:et_complete}
An execution test \(\ET\) is \emph{complete} with respect 
to an axiomatic definition \((\RP_{\LWW}, \Ax)\) if, for any abstract execution \(\aexec \in \CMa(\RP_{\LWW}, \Ax)\) 
and index \( i : 1 \leq i < \abs{\txidset_{\aexec}}\) such that \( \txid_{i} \toEDGE{\AR_{\aexec}} \txid_{i+1} \), there exist an initial view \(\vi_{i}\) and a final view \(\vi_{i}'\) where 
\begin{itemize}
\item \(\vi_{i} = \getView[\aexec, \VIS_{\aexec}^{-1}(\txid_{i})]\), 
\item let \(\txid_{i} = \txid_{\cl}^{n}\) for some \(\cl, n\); 
    \begin{itemize}
        \item if the transaction \(\txid_{i}' = \min_{\SO_{\aexec}}\Set{\txid' }[ \txid_i \toEDGE{\SO_{\aexec}} \txid']\) is defined, then \(\vi' = \getView[\aexec, \txidset_{i}]\) where \(\txidset_{i} \subseteq (\AR_{\aexec}^{-1})\rflx(\txid_{i}) \cap \VIS_{\aexec}^{-1}(\txid_{i}'))\); 
        \item otherwise \(\vi' = \getView[\aexec, \txidset_{i}]\) where \(\txidset_{i} \subseteq (\AR_{\aexec}^{-1})\rflx(\txid_{i})\), 
    \end{itemize}
\item \(\ET \vdash (\mkvs_{\cut[\aexec, i-1]}, \vi_{i}) \csat \TtoOp{T}_{\aexec}(\txid_{i}) : (\mkvs_{\cut[\aexec, i]},\vi_{i}')\).
\end{itemize}
\end{definition}

\begin{theorem}
\label{thm:et_complete}
Let \(\ET\) be an execution test that is complete with respect to an axiomatic definition \((\RP_{\LWW}, \Ax)\). 
Then \(\CMa(\RP_{\LWW}, \Ax) \subseteq \CMs(\ET)\).
\end{theorem}
\begin{proof}
Fix an abstract execution \(\aexec \in \CMa(\RP_{\LWW}, \Ax)\). 
For any \(i : 1 \leq i < \abs{\txidset_\aexec} \), suppose that \( \txid_i \) that is the i-\emph{th} transaction follows the arbitrary order, \ie \(\txid_{i} \toEDGE{\AR_{\aexec}} \txid_{i+1}\) 
and let \(\cl_{i}\) be the client of the i-\emph{th} step, \ie \(\txid_{i} = \txid_{\cl_{i}}^{\stub}\).
Because \(\ET\) is complete with respect to \((\RP_{\LWW}, \Ax)\), 
for any step \(i\) we can find an initial views \(\vi_i\),and a final view \(\vi'_{i}\) such that 
\begin{itemize}
\item \(\vi_i = \getView[\aexec, \VIS^{-1}_{\aexec}(\txid_{i})]\), 
\item there exists a set of transactions \(\txidset_{i}\) such that \(\getView[\aexec, \txidset_{i}] = \vi'_{i}\), and 
either \(\min_{\SO_{\aexec}}\Set{\txid' }[ \txid_{i} \toEDGE{\SO_{\aexec}} \txid']\) is 
is defined and \(\txidset_{i} \subseteq (\AR_{\aexec}^{-1})\rflx(\txid_{i}) \cap \VIS^{-1}_{\aexec}(\txid')\), 
or \(\txidset_{i} \subseteq (\AR_{\aexec}^{-1})\rflx(\txid_{i})\), 
\item \(\ET \vdash (\mkvs_{\cut[\aexec, i-1]}, \vi_i) \csat \TtoOp{T}_{\aexec}(\txid_{i}): (\mkvs_{\cut[\aexec, i]}, \vi'_{i})\).
\end{itemize}
Given above, let \(\mkvs_{i} = \cut[\aexec, i]\) and \(\fp_{i} = \TtoOp{T}_{\aexec}(\txid_{i})\). Define the views for clients as 
\[
\vienv_{0} = \lambda \cl \in \Set{\cl' }[ \exsts{ \txid \in \txidset_{\aexec} } \txid = \txid_{\cl'}] \ldotp \lambda \key \ldotp \Set{0}
\quad \vienv'_{i-1} = \vienv_{i}\rmto{\cl_{i}}{ \vi_i}
\quad \vienv_{i} = \vienv'_{i-1}\rmto{\cl_{i} }{\vi'_{i}}
\]
and the ke-stores as
\[
\mkvs_{0} = \lambda \key.(\val_{0}, \txid_{0}, \emptyset)
\quad \mkvs_{i} = \updateKV[\mkvs_{i-1}, \vi_i, \fp_{i}, \txid_{i}]
\]
Now by \cref{prop:aexec2kvtrace} we have that the following sequence of \(\ET_{\top}\)-reductions 
\[
\begin{array}{l}
(\mkvs_{0}, \vienv_{0}) \toET{(\cl_{1}, \varepsilon)}[\ET_{\top}] (\mkvs_{0}, \vienv'_{0}) 
\toET{(\cl_{1}, \fp_{1})}[\ET_{\top}] (\mkvs_{1}, \vienv_{1}) 
\toET{(\cl_{2}, \varepsilon)}[\ET_{\top}]
\cdots \toET{(\cl_{n}, \fp_{n})}[\ET_{\top}] (\mkvs_{n}, \vienv_{n})
\end{array}
\]
Note that \(\mkvs_{i} = \mkvs_{\cut[\aexec, i]}\). 
Because \(\ET \vdash ( \mkvs_{\cut[\aexec,i-1]}, \vi_i ) \csat \fp_{i} : (\mkvs_{i}, \vi'_{i})\), 
or equivalently \(\ET \vdash ( \mkvs_{\cut[\aexec, i-1]}, \vienv'_{i-1}(\cl_{i}) ) \csat \fp_{i} : ( \mkvs_{\cut[\aexec, i-1]}, \vienv_{i}(\cl_{i}) )\), therefore 
\[
\begin{array}{l}
(\mkvs_{0}, \vienv_{0}) \toET{(\cl_{1}, \varepsilon)} (\mkvs_{0}, \vienv'_{0}) 
\toET{(\cl_{1}, \fp_{1})} (\mkvs_{1}, \vienv_{1})
\toET{(\cl_{2}, \varepsilon)}
\cdots \toET{(\cl_{n}, \fp_{n})} (\mkvs_{n}, \vienv_{n})
\end{array}
\]
It follows that \(\mkvs_{n} \in \CMs(\ET)\) then \(\mkvs_{n} = \mkvs_{\cut[\aexec, n]} = \mkvs_{\aexec}\), and there is nothing left to prove.
\end{proof}

\begin{corollary}
\label{cor:et-completeness}
If \(\ET\) is complete with respect to \((\RP_{\LWW}, \Ax)\), then 
for any program \(\prog\), 
\[\Set{ \mkvs_{\aexec} }[ \aexec \in \interpr{\prog}_{(\RP_{\LWW}, \Ax)} ] \subseteq \interpr{\prog}_{\ET}\]
\end{corollary}
\begin{proof}
\begin{align*}
\Set{ \mkvs_{\aexec} }[ \aexec \in \interpr{\prog}_{(\RP_{\LWW}, \Ax)} ]
& \stackrel{%\cref{thm:consistency-intersect-anarchic}
}{=} 
\Set{\mkvs_{\aexec} }[ \aexec \text{ satisfies } \RP_{\LWW} \land \aexec \in \CMa(\RP_{\LWW}, \Ax) ] \\
& \stackrel{\cref{thm:et_complete}}{\subseteq} 
\Set{\mkvs_{\aexec} }[ \aexec \text{ satisfies } \RP_{\LWW}] \cap \CMs(\ET) \\
& \stackrel{%\cref{cor:kvtrace2aexec}
}{=} 
\interpr{\prog}_{\ET_\top} \cap \CMs(\ET) \\
& \stackrel{%\cref{thm:consistency-intersect-permissive}
}{=} 
\interpr{\prog}_{\ET} 
\end{align*}
\end{proof}
