\section{Formal Model}
\label{sec:model}
\label{sec:semantics}
\ac{I am going to split this section into two, for the sake of clarity. 
As of now, the structure that I have in mind for the paper is as follows: 
Section 2 (i.e. this section) contains the notion of key-value store, view, snapshot, 
execution tests, and consistency model  -i.e. sets of key-value stores- 
induced by an execution test). Then Section 3 discusses 
the semantics of programs, and possibly we state the result 
of adequateness. In Section 4 we discuss the equivalence of consistency 
models specifications with respect to axiomatic ones. In Section 5 
we present the logic and litmus tests examples.}

\begin{wrapfigure}[7]{r}{0.33\textwidth}
\vspace{-10pt}
\begin{verbatim}
interface Transaction {
	Start(); 
	Read(Key k);
	Write(Key k, Value v); 
	Commit();    }
\end{verbatim}
\vspace{-10pt}
\caption{Example of Transaction API.}
\label{fig:api}
\end{wrapfigure}
We focus on an abstract computational model where multiple client programs can access and update keys in a key-value store using atomic transactions. 
In general, clients are provided with a simple \textit{API} such as the one depicted in \cref{fig:api} \cite{gdur,physicsnmsi,clockSI}\footnote{It is 
often the case that key-value stores provide a mechanism to wrap more transactions inside a session, and give 
provide appropriate APIs to handle sessions. For the sake of simplicity, in this paper we assume that each client executes transactions 
within a single session.}
\textbf{AC: (not using boxes because they clash with wrapfig. Technically, clockSI also has a delete operation, this should be either pointed out 
or the citation be substituted with something else.)}, while both the implementation details and the system architecture are hidden from clients. 
Because (distributed) key-value stores only give weak consistency guarantees of the data to their clients, the latter are 
not ensured to read the most up-to-date version of a key.
%In an ideal world, when executing a transaction clients would read the most up-to-date version of a key. In a distributed setting 
%this approach, known as (strict) serialisability, would require a continuous synchronisation between the different components of 
%the system, which impacts performance and limits scalability. To this end, the database only provides weak consistency model
%\ac{This sentence should probably be in the introduction.}

Following these intuitions, we model a key-value stores, or \emph{kv-store}, as a centralised unit where multiple versions 
are stored for each key \cref{sec:mkvs}. Versions consist of a value and the meta-data relating to the transactions that wrote and 
read such a version. In practical systems, the meta-data is usually encoded using either timestamps 
\cite{physicsnmsi,clockSI} or vector clocks \cite{gdur}. We focus on key-value stores whose transactions 
enjoy \emph{atomic visibility}, meaning that \textbf{(i)} transactions read their data from an atomic 
snapshot of the key-value store, and \textbf{(ii)} a transaction can observe either none or all 
of the updates performed by another transaction. In other words, a transaction only reads (writes) at most 
one version for each key.
Because clients may observe potentially out-of-date versions of the system, we introduce the notion of \emph{views}. 
Intuitively, a view records the version of each key that a client observes at a given time. We use views 
to determine the snapshot taken by transactions executed by clients.

A consistency model is a contract between the key-value store and its clients. We distinguish 
between \emph{data-centric} consistency models \cite{framework-concur}, which impose constraints 
on the structure of the key value store, and \emph{client-centric} consistency models, 
which impose constraint on the observations and updates made by a single client 
\cite{terry1994session}. 
To specify weak consistency models, 
we introduce the notion of \emph{execution tests} \cref{sec:exectest}. An execution test is a predicate 
that specifies when a client is allowed to execute a transaction carrying a given 
set of read and write operations, or \emph{fingerprint}. Therefore, an execution 
test describes how the state of the key-value store may evolve;
by considering at all the possible evolutions of the key-value store under said execution test, 
we determine a consistency model. For example, the execution test for serialisability 
requires that a transaction can be executed by a client, only if it observes the most up-to-date 
version for each key. 
We give several examples of execution tests that capture both data-centric and 
client-centric consistency models. 

The idea of specifying consistency models using execution tests has been 
already proposed in \cite{seebelieve}; however, in that paper the author require 
the knowledge of the total order in which all past transactions have been 
executed, to determine whether a new transaction can commit.

%%\ac{Can't cite documentations of real databases, as they usually have a much more complicated API.}
%
%%Transactions in our model execute atomically, though they have different effects on the key-value stores depending on their associated \emph{consistency model}.
%%A consistency model controls how the key-value store evolves.
%%A common model is \emph{serialisability}, where transactions appear to execute one after another in a sequential order.
%%This notion of sequential execution is however not necessary for many weaker models. As such, upon commencing execution, a transaction may not observe the most up-to-date values for keys. 
%
%To address this, we 
%first model 
%the state of the system using \emph{multi-version key-value stores (MKVSs)} (\cref{sec:mkvs-view}). 
%An MKVS keeps track of all versions (values) written for keys, as well as the information about the transactions that read and wrote such versions. 
%To model the potential out-of-date observation, we introduce \emph{views}.
%A view decides the observable versions of keys for a client.
%Therefore, in order to execute a transaction, the client first takes \emph{a snapshot} of the system with the view, executes the transaction locally with respect to its snapshot (\cref{sec:trans-semantics}), and afterwards commits the effect of the transaction if the change is allowed by the underlying consistency model (\cref{sec:prog-semantics}).


%We first introduce MKVSs and views (\secref{sec:mkvs-view}) and.
%We starts with the syntax of programs followed by the semantics of transaction.
%Finally, we will give the semantics for the entire programs.


\subsection{Multi-version Key-value Stores and Views}
\label{sec:mkvs-view}
\ac{I'm going to leave this comment here, I know it's going to be harsh but needs to be said: 
If you are going to change the main text, please be sure that the English is correct, and that the 
writing is consistent! It took me several days to get this in a good shape, and now I have to go over all of it again!}
%\ac{
%We focus on a computational model where multiple client programs can access and update 
%locations in a key-value store using atomic transactions. Transactions in our model execute atomically, 
%though the consistency guarantees that they provide do not necessarily correspond to \emph{serialisability}. 
%This means that, at the moment of executing, a transaction may not observe the most up-to-date value 
%of a location. 

%To overcome this issue, we model the state of the system using \emph{multi-version key-value stores} 
%(MKVSs) and \emph{views}. A 
%MKVS keeps track of all the versions written for any key, as well as the information 
%about the transactions that read and wrote such versions.  Views keep track of the version observed 
%for each key by clients. 
%}

%\sx{ Value as natural number or natural number + key index? Is partial mkvs is a problem here??}


%We often depict MKVSs graphically. 
%One example is given by the MKVS $\hh_0$ of Figure \ref{fig:hheap}(a) (ignore for the moment 
%the straight lines labelled $\txid_1$ and $\txid_2$).

%\ac{\sx{Partial is better for logic}Maybe it's better to keep $\ke$ fixed and say that we look at only a 
%fragment of the key value store. Alternatively, we can go for partial mappings to 
%represent MKVSs, but still avoiding allocation and deallocation of keys.}



%\ac{
%We often depict MKVSs graphically. 
%One example is given by the MKVS $\hh_0$ of Figure \ref{fig:hheap}(a) (ignore for the moment 
%the straight lines labelled $\txid_1$ and $\txid_2$).

%To the left 
%we have the set of keys stored y $\hh_0$, in this case $\key{k}_1$ and 
%$\key{k}_2$. To the right, on the same line of a key, a matrix containing the
 %list of versions stored by such a key in $\hh_0$. Starting from the first column, 
 %each version is represented by two adjacent columns in the matrix: the 
 %column to the left gives the value of the version, while the column to the 
 %right contains the identifier of the transaction that wrote the version to the 
 %top, and the identifiers of the transactions that read such a version to the bottom.
%In the case of $\hh_0$, there are two versions stored for $\key{k}_1$: 
%the first one with value $0$, written by $\txid_0$ and read by $\txid_2$; 
%and the second one with value $1$, written by $\txid_2$ and read by no 
%transaction. 
%}
\azalea{I have changed the values to $v_0$ and $v_1$ (from $0$ and $1$) to help clarify the distinction between indexes and values.
I have also paraphrased, please double check.  }

\mypar{Key-value Stores} 

%We model the state of a system using \emph{multi-version key-value stores} (MKVSs). 
We assume a countably infinite set of \emph{keys} $\Keys \defeq \Set{\ke, \ke', \cdots}$, 
a set of \emph{values} $\Val \defeq \{\val, \val', \cdots\}$ which for simplicity we instantiate to be 
$\Nat \uplus \Keys$, a set of clients $\Clients \defeq \Set{\cl, \cl',\cdots}$. 
We also assume a set of transaction identifiers $\TxID \defeq \Set{ \txid_{\cl}^{n} \mid \cl \in \Clients \wedge n \geq 0 } 
\uplus \Set{\txid_{0}}$,
each of which is either a special transaction identifier $\txid_0$, 
or it is indexed by a client identifier and a natural number. 
Elements of $\TxID$ are ranged over by $\txid, \txid', \cdots$, 
while subsets of $\TxID$ are ranged over $\txidset, \txidset', \cdots$. 
We let $\TxID_{0} = \TxID \setminus \{ \txid_0\}$.
The structure of the set $\TxID$  
embeds the order in which transactions are executed by individual clients, or \emph{session order}. 
Specifically, we let $\PO = \Set{ (\txid, \txid') \mid \exsts{ \cl, n,m } \txid = \txid_{\cl}^{n} \wedge 
\txid' = \txid_{\cl}^{m} \wedge n < m}$; 
\ac{Put some space between the existential quantifier and the quantified variable in the \textbackslash exsts macro. 
Same for \textbackslash fora.}
$(\txid, \txid') \in \PO$ means that 
some client $\cl$ has executed $\txid$ prior to $\txid'$. For $\PO$ (and in general  
for relations between transaction identifiers) we will often adopt the more graphic notation 
$\txid \xrightarrow{\PO} \txid'$ in lieu of $(\txid, \txid') \in \PO$.

Given a set $X$, then $\powerset{X}$ denotes 
the powerset of $X$, while $X^{\ast}$ is the free monoid induced by $X$.
%We define \( \Versions \) and \( \MKVSs \) in \cref{def:mkvs}.


\begin{definition}[Multi-version Key-value Stores]
\label{def:his_heap}
\label{def:mkvs}
%Assuming a countably infinite set of keys $\Keys = \Set{\ke_1, \cdots}$, \emph{transactions identifiers} \( \TxID \defeq \Set{\txid_{1}, \cdots}\) and natural numbers \(\nat \in \Nat \), 
A \emph{version} is a triple $\ver = (\val, \txid, \txidset)$. The set of versions is denoted by $\Versions \defeq \Val \times \TxID \times \powerset{\TxID}$, 
and a \emph{key-value store} is a partial, finite mapping $\hh \in \MKVSs \defeq \Keys \parfinfun \Versions^{\ast}$. 
\ac{ The superscript fin over the $\rightharpoonup$ needs to be fixed. You may want to look at the package extpfeil.}
%The set of key-value stores is denoted as $\HisHeaps$.
\end{definition}
\ac{Also, I am not sure whether we want the kv-store to be a partial mapping. What happens if you try to access a location not 
defined in the kv-store in the semantics? Viktor in his Thesis had the notion of faulty programs, and triples of the form 
$\{P\} C \{Q\}$ in RGsep also imply that the command $C$ is not faulty. If this point  does not get clarified, 
I suggest that we keep kv-stores as total mappings from keys to lists of versions. Shale, 
your previous comment. \textbf{I think it has no semantics in our case} is not a clarification.}
%\ac{Is it safe to assume that key-value stores are partial functions? What happens if we execute a program 
%that tries to access a key that is not in the key-value store? In his Thesis, Viktor had the notion of fault to distinguish cases 
%where non-allocated variables were used. Check. \sx{I think it has no semantics in our case.}}
%\ac{Also, in $\parfinfun$ the text above is out of sync. Fix. You may want to look at the extensible arrow package 
%(I think it's extpfeil, but I'm not sure and I am currently on a plane with no internet. Will Check.}
\emph{A version} $\ver = (\val, \txid, \txidset)$ consists of a value $\val$, and the meta-data of the transactions 
that accessed the version; specifically, $\txid$ is the identifier of the transaction that wrote such a version, 
and $\txidset$ is the set of identifiers of transactions that read the version.
Given a version $\ver = (\val, \txid, \txidset)$, we let $\valueOf(\ver) = \val$. 
$\WTx(\ver) = \txid$ and $\RTx(\ver) = \txidset$.
% for its value ($n$),
%$\WTx(\ver)$ for its writer ($\txid$) and $\RTx(\ver)$ for its readers ($\txidset$).
%, $\txid = \WTx(\ver)$ denotes the transaction identifier corresponding to the 
%writer of the version, while $\txidset = \RTx(\ver)$ contains the set of the transaction identifiers that read a version. 
Lists of versions, that is elements of $\Versions^{\ast}$, are ranged over by $\vilist, \vilist',\cdots$.

\emph{A multi-version key-value store}, or \emph{kv-store}, 
is a mapping from keys to lists of versions. 
For a given kv-store, $\hh$, key $\ke$ and index $i \geq 0$, we use the notation $\hh(\ke, i)$ 
to denote the $i$-th version (starting from $0$) installed for $\ke$; that is, if $\hh(\ke) = \ver_0 \cdots\ver_{n}$, then 
$\hh(\ke, i) = \ver_{i}$ if $i \leq n$, it is undefined otherwise. We also let $\lvert \hh(\ke) \rvert = n +1 $ denote 
the length of $\hh(\ke)$.

It will be often convenient to depict key-value stores graphically: an 
example is given by the kv-store $\hh$ depicted in \cref{fig:hheap-a}
(ignore for the moment the vertical lines labelled $\client$ and $\client'$). 
It comprises two keys \( \ke_1\) and \( \ke_2 \), 
each of which is associated with two versions carrying values $\val_0$ and $\val_1$, and $\val'_0$ and $\val'_1$, respectively.
The versions of a key are listed in order from left to right. 
We represent each version as a three-cell box, with the left cell storing the value, the top right cell recording the writer, and the bottom right cell recording the readers. 
For example, the version carrying value $\val_0$ in $\ke_1$ has been written by $\txid_0$, and has been read by $\txid_{\cl'}^1$.

In this paper we focus on key-value stores whose consistency model enforces the  
\emph{atomic visibility} of transactions \cite{framework-concur}. 
We also assume that in kv-stores, keys with a defined list of versions, have an initial version carrying  a default value $\val_0 \in \Val$, 
written by the special transaction identifies $\txid_0$.
% an initial 
%initialised by the special transaction identifier $\tsid_0$, i.e.  
%install the initial version of each key. 
A \emph{well-formed} kv-store $\hh$ requires that:
\begin{enumerate}[(i)]
\item\label{kv:wf.init} for each key $\ke \in \dom(\hh)$, $\hh(\ke, 0) = (\val_0, \txid_0, \stub)$, where $\val_0$ is a default value from $\Val$;
\item\label{kv:wf.onewrite} transactions never write more than one version per key,  
\[
\fora{\ke \in \dom(\hh)} \fora{ i,j : 0 \leq i, j < \lvert \hh(\ke) \rvert}
\WTx(\hh(\ke, i)) = \WTx(\hh(\ke, j)) \implies i = j \]
\ac{Before the text $\WTx(\hh(\ke, i))$ was changed to $\WTx(\ke, i)$. The latter form is wrong, 
as there is no reference to the kv-store that we are considering.}
%\item transactions never read different versions 
%for the same key $\fora{ \ke \in \dom(\hh)} \fora{ i,j : 0 \leq i, j < \lvert \hh(\ke) \rvert}
%\RTx(\ke, i) \cap \RTx(\ke, j) \neq \emptyset \implies i = j$;
\item\label{kv:wf.oneread} transactions never read different versions for the same key, 
\[
\fora{\ke \in \dom(\hh)} \fora{ i,j : 0 \leq i, j < \lvert \hh(\ke) \rvert} 
\RTx(\hh(\ke, i)) \cap \RTx(\hh(\ke, j)) \neq \emptyset \implies i = j
\]
\ac{I am leaving this comment here: to change the macro from the standard \textbackslash forall 
to \textbackslash fora, you elided the comma that separated the formula from the rest of the text. 
This is bad writing. If you need to make changes, please be sure that the text remains consistent 
and nicely written. This happened also in the previous item.}
\ac{Also, why do you insist on using the \textbackslash fora and \textbackslash exsts macros, when LaTeX has 
its own predefined ones (that everybody understands and knows)? Remember that you are working with other people, 
that will prefer the standard, well-known alternative.}
%\item all the reads of a transaction precede the writes of the same transaction, 
%\[\fora{ \ke \in \dom(\hh)}
%\fora{ i : 0 \leq i < \lvert \hh(k) \rvert} \WTx(\hh, k) \notin \RTx(\hh,k)
%\]
\item\label{kv:wf.so} the order 
in which transactions issued by the same client install different versions for some key $\ke$, is consistent with the order in which 
such transactions have been invoked; similarly, a client can read the version of a key $\ke$ only after it installed it. 
\begin{multline*}
\fora{ \ke \in \dom(\hh), \cl \in \Clients} \fora{ i,j; 0 \leq i < j < \lvert \hh(\ke) \rvert}
\fora{ n, m \geq 0}\\ (\txid_{\cl}^{n} = \WTx(\hh(\ke,i)) \wedge \txid_{\cl}^{m} \in \{\WTx(\hh(\ke,j))\} \cup \RTx(\hh(\ke, i)) \implies n < m.
\end{multline*}
\end{enumerate}
We always assume that kv-stores are well-formed, and let $\HisHeaps$ be the set of well-formed kv-stores.

%A\emph{multi-version key-value store (MKVS)}, \( \mkvs \in \MKVSs \), is a partial finite function from keys to lists of \emph{versions}.
%A \emph{version} is a triple containing a program value \( \val \), a transaction identifier \( \txid \) and a set of transactions identifiers \( \txidset \):
%\[
%\begin{rclarray}
%    \ver \in \Versions & \defeq &  \Setcon{(\val, \txid, \txidset)}{\val \in \Val \land \txid \in \TxID \land \txidset \subseteq \TxID \land \txid \notin \txidset} \\
%    \HisHeaps & \defeq & 
%    \Setcon{ \mkvs }{%
%        \mkvs \in \Keys \parfinfun \Versions^{*} 
%        \land \fora{\ke, \txid, i, j} 
%        \WTx(\hh(\ke, i)) = \WTx(\hh(\ke, j)) \lor {} \\
%        (\txid \in \RTx(\hh(\ke, i)) \land \txid \in \RTx(\hh(\ke, j))
%        \implies i = j 
%    }
%\end{rclarray}
%\]
%
%We model a kv-store $\hh$ as a map from keys to lists of versions.
%%where each key is associated with a list \emph{versions} from the initial one to the latest one (\defref{def:mkvs}).
%%More concretely, versions associated with a key $\ke$ 
%Versions are tuples of the form \( \ver =( \val, \txid, \txidset ) \), where $\val$ denotes the \emph{current value} of $\ke$, 
%$\txid$ is the identifier of the transaction that wrote such a version, 
%%, identifying its \emph{writer}, i.e.\ the transaction responsible for writing value $\val$, 
%and $\txidset$ is the set of identifiers of transactions that read the version.
%%, denoting its \emph{readers}, i.e.\ those transactions who read from $\ke$ 
%(see \cref{def:mkvs}).
%
%\cref{fig:hheap-a} depicts an example of an MKVS. 
%Ignoring the lines labelled $\client_1$ and $\client_2$, the depicted MKVS contains two keys \( \ke_1\) and \( \ke_2 \), each of which associated with two versions (with values $v_0$ and $v_1$ for $\ke_1$, and values $v'_0$ and $v'_1$ for $\ke_2$).
%The versions of a key are listed in chronological order from left to right.
%We represent each version as a three-cell box, with the left cell storing the value, the top right cell recording the writer, and the bottom right cell recording the readers. 
%
%
%Given a version $\ver = (\val, \txid, \txidset)$, we write $\valueOf(\ver)$ for its value ($n$), $\WTx(\ver)$ for its writer ($\txid$), $\RTx(\ver)$ for its readers ($\txidset$), and \( \txid' \in \ver \) for \( \txid' = \txid \lor \txid' \in \txidset \).
%%A \emph{Multi-version Key-value Store (MKVS)} is a mapping $\hh : \Keys \parfinfun \Versions^{\ast}$ from keys to lists of versions. 
%Given a list of versions $L = \List{\ver_0, \cdots, \ver_{s-1}}$, we write  $\lvert L\rvert$ for its length ($s$). 
%Given an MKVS $\hh$ and a key $\ke$, we write $\hh(\ke)$ for the list of versions associated with $\ke$ in $\hh$, and write \(\hh(\ke,i) \) for the $i$\textsuperscript{th} entry (indexed from $0$) in $\hh(\ke)$. 
%%We assume the index starts from 0, so \( \hh(\ke)(\left|\hh(\ke)\right| - 1)\) is the latest versions of the key \( \ke \).
%
%\azalea{I think $\hh$ is too close to $\ke$ and we should use a different symbol. Maybe $H$? }
%
%We assume a countably infinite set of \emph{keys}, $\Keys $,
%a set of countably infinite set of \emph{program values}, \( \val \in \Val \),
%a countably infinite set of \emph{client identifiers}, \( \cl \in \Clients \),
%and a countably infinite set of \emph{transactions identifiers indexed by client}, $\txid^{\cl} \in \TxID$
%We also assume there are countably infinite elements for every client, and elements for each client are ordered.
%We use \( \txid^{\cl}_{i}\) for the \emph{i-th} identifier among the total order and it is indexed by the client \( \cl \). 
%For a client, the order among transaction identifiers corresponds to the session order of them.
%Sometime we omit the order and/or client index when they are irrelevant.
%We use $\ke$ and its variants (e.g.\ $\ke_1$, $\ke'$ and so forth) as meta-variables for keys in $\Keys$.
%%and use $\txid$ and its variants as meta-variables for transaction identifiers in $\TxID$. 
%
%A key-value store is \emph{well-formed} iff 
%a transaction identifier appears in all versions of a key at most twice, once as the writer and once as a reader.
%%(i) it does not contain circular dependencies across its versions; and 
%%
%%Given two versions $\ver_1, \ver_2$ in an MKVS $\hh$, the $\ver_2$ \emph{directly depends} on $\ver_1$, written $\ver_1 \xrightarrow{\ddep} \ver_2$, iff:
%%\[ 
%%\begin{rclarray}
%%\ver_1 \xrightarrow{\ddep} \ver_2 & \defeq &
%%\begin{array}[t]{@{}l}
%%\WTx(\ver_2) \in \RTx(\ver_1) 
%%%\exsts{\cl,i,j,\txid_{i}^{\cl},\txid_{j}^{\cl}} \\
%%%\quad \txid_{i}^{\cl} < \txid_{j}^{\cl}
%%%\land \txid_{i}^{\cl} \in \ver_{1}
%%%\land \txid_{j}^{\cl} \in \ver_{2}
%%\end{array}
%%\end{rclarray}
%%\]
%%that is, a transaction $\txid$ wrote the version $\ver_2$ after reading $\ver_1$;
%%%or two transactions from the same client access
%%%The $\ver_1 \xrightarrow{\ddep(\hh)} \ver_2$ denotes that $\ver_1, \ver_2$ appear as versions from the same KVMS $\hh$ and \( \ver_1\) depends on \(\ver_2 \).
%
%%A key-value store is \emph{well-formed}, written $\pred{wfMKVS}{\hh}$, iff 
%%(i) it does not contain circular dependencies across its versions; and 
%%(ii) a transaction identifier appears in all versions of a key at most twice, once as the writer and once as a reader.
%%% a transaction identifier appears in all the versions for an key at most twice, one as the writer and one as a reader.
%%%
%%%Also there are no circular dependencies in versions. 
%%%More concretely, the first well-formedness condition (i) of an KVMS ensures that the \emph{transitive closure} of the direct dependency relation $\left(\xrightarrow{\ddep(\hh)}\right)^{+}$ is acyclic:
%%%in $\hh$, i.e. $\left(\xrightarrow{\ddep(\hh)} \right)^{+} \cap \text{Id} = \emptyset$ (where $\text{Id}$ is 
%%%the identity relation).
%%%\[
%%%\begin{rclarray}
%    %%\pred{wfMKVS}{\hh} & \defeq &
%    %%\begin{array}[t]{@{}l}
%        %%\pred{acyclic}{\left(\xrightarrow{\ddep(\hh)}\right)^{+}}
%        %%\land \fora{\ke, \txid, i, j}
%        %%\begin{B}
%        %%\WTx(\hh(\ke, i)) = \WTx(\hh(\ke, j)) \lor {} \\ (\txid \in \RTx(\hh(\ke, i)) \land \txid \in \RTx(\hh(\ke, j))
%        %%\end{B} 
%        %%\implies i = j 
%    %%\end{array}
%%%\end{rclarray}
%%%\]
%%%
%The \emph{partial commutative monoid (pcm) of MKVSs} is $(\MKVSs, \composeHH, \{\unitHH\})$, where 
%$\composeHH:  \MKVSs \times \MKVSs \rightharpoonup \MKVSs$ denotes the \emph{pcm composition} defined as the standard function disjointed union: $\composeHH \eqdef \uplus$; and
%%\( \hh \composeHH \hh' \defeq \hh \uplus \hh' \)
%$\unitHH \in \MKVSs$ denotes the \emph{pcm unit element}:  $\unitHH \eqdef \emptyset$, where $\emptyset$ denotes a function with an empty domain.
%\end{definition}
%
%\ac{A point that this does not ensure a real causal dependency between the two versions, yet it is consistent with the notion of causality employed in databases, should be made}. 
%

\begin{figure}
\begin{center}
\hrule
\begin{tabular}{@{}c @{\qquad} c@{}}
\begin{halfsubfig}
\begin{tikzpicture}
\begin{pgfonlayer}{foreground}
%Uncomment line below for help lines
%\draw[help lines] grid(5,4);

%Location x
\node(locx) {$\ke_1 \mapsto$};

\matrix(versionx) [version list]
    at ([xshift=\tikzkvspace]locx.east) {
    {a} & $\txid_0$ & {a} & $\txid_{\cl}^{1}$\\
    {a} & $\left\{\txid_{\cl'}^{1}\right\}$ & {a} & $\emptyset$ \\
};
\tikzvalue{versionx-1-1}{versionx-2-1}{locx-v0}{$v_0$};
\tikzvalue{versionx-1-3}{versionx-2-3}{locx-v1}{$v_1$};

%Location y
\path (locx.south) + (0,\tikzkeyspace) node (locy) {$\ke_2 \mapsto$};
\matrix(versiony) [version list]
   at ([xshift=\tikzkvspace]locy.east) {
 {a} & $\txid_0$ & {a} & $\txid_{\cl'}^{1}$ \\
  {a} & $\left\{\txid_{\cl}^1\right\}$ & {a} & $\emptyset$\\
};

\tikzvalue{versiony-1-1}{versiony-2-1}{locy-v0}{$v'_0$};
\tikzvalue{versiony-1-3}{versiony-2-3}{locy-v1}{$v'_1$};

% \draw[-, red, very thick, rounded corners] ([xshift=-5pt, yshift=5pt]locx-v1.north east) |- 
%  ($([xshift=-5pt,yshift=-5pt]locx-v1.south east)!.5!([xshift=-5pt, yshift=5pt]locy-v0.north east)$) -| ([xshift=-5pt, yshift=5pt]locy-v0.south east);

%blue view - I should  check whether I can use pgfkeys to just declare the list of locations, and then add the view automatically.
\draw[-, blue, very thick, rounded corners=10pt]
 ([xshift=-3pt, yshift=20pt]locx-v1.north east) node (tid1start) {} -- 
 ([xshift=-3pt, yshift=-5pt]locx-v1.south east) --
 ([xshift=-3pt, yshift=5pt]locy-v0.north east) -- 
 ([xshift=-3pt, yshift=-5pt]locy-v0.south east);
 
 \path (tid1start) node[anchor=south, rectangle, fill=blue!20, draw=blue, font=\small, inner sep=1pt] {$\client$};

%red view
\draw[-, red, very thick, rounded corners = 10pt]
 ([xshift=-16pt, yshift=5pt]locx-v1.north east) node (tid2start) {}-- 
% ([xshift=-16pt, yshift=-5pt]locx-v0.south east) --
% ([xshift=-16pt, yshift=5pt]locy-v1.north east) -- 
 ([xshift=-16pt, yshift=-5pt]locy-v1.south east) node {};
 
\path (tid2start) node[anchor=south, rectangle, fill=red!20, draw=red, font=\small, inner sep=1pt] {$\client'$};

\end{pgfonlayer}
\end{tikzpicture}
\caption{A configuration with a well-formed kv-store $\hh$ and two views $\vi,\vi'$.}
\label{fig:hheap-a}
\end{halfsubfig}
&

\begin{halfsubfig} 
\begin{center}
\begin{tikzpicture}[scale=0.85, every node/.style={transform shape}]
%\draw[help lines] grid(6,4);

\node(t0wx) at (-1,2) {$(\otW, \ke_1, \val_0)$}; 
\path (t0wx.south) + (0,-0.2) node[anchor=north] (t0wy) {$(\otW, \ke_2, \val'_0)$};
\path (t0wx.north east) + (1,0.5) node[anchor = west] (t1ry) {$(\otR, \ke_2, \val'_0)$}; 
\path (t1ry.east) + (0.2,0) node[anchor = west] (t1wx) {$(\otW, \ke_1, \val_1)$};
\path (t0wy.south east) + (1,-0.5) node[anchor = west] (t2rx) {$(\otR, \ke_1, \val_0)$};
\path (t2rx.east) + (0.2,0) node[anchor = west] (t2wy) {$(\otW, \ke_2, \val'_1$)};

\begin{pgfonlayer}{background}
\node[background, fit=(t0wx) (t0wy)] (t0) {};
\node[background, fit= (t1ry) (t1wx)] (t1) {};
\node[background, fit= (t2rx) (t2wy)] (t2) {};

\path(t0.west) node[anchor=east] (t0lbl) {$\txid_0$};
\path(t1.north) node[anchor=south] (t1lbl) {$\txid_1$};
\path(t2.south) node[anchor=north] (t2lbl) {$\txid_2$};

\path[->]
(t0.north) edge[bend left=70] node[above, yshift=7pt, xshift=-1pt, pos=0.3] {$\RF(\ke_2), \VO(\ke_1)$} (t1.west)
(t0.south) edge[bend right=70] node[below, yshift=-8pt, xshift=-1pt, pos=0.3] {$\RF(\ke_1), \VO(\ke_2)$} (t2.west)
([xshift=-8pt]t2.north) edge[bend left=40] node[left] {$\AD(\ke_1)$} ([xshift=-8pt]t1.south) 
([xshift=8pt]t1.south) edge[bend left=40] node[right] {$\AD(\ke_2)$} ([xshift=8pt]t2.north);
\end{pgfonlayer}

%\begin{pgfonlayer}{foreground}
%%Uncomment line below for help lines
%%\draw[help lines] grid(5,4);
%
%%Location x
%\node(locx) {$\ke_1 \mapsto$};
%
%\matrix(versionx) [version list, text width=7mm, anchor=west]
%    at ([xshift=\tikzkvspace]locx.east) {
%    {a} & $\txid_0$ & {a} & $\txid_1$\\
%    {a} & $\emptyset$ & {a} & $\{\txid_2\}$ \\
%};
%\tikzvalue{versionx-1-1}{versionx-2-1}{locx-v0}{$v_0$};
%\tikzvalue{versionx-1-3}{versionx-2-3}{locx-v1}{$v_1$};
%
%%Location y
%\path (locx.south) + (0,\tikzkeyspace) node (locy) {$\ke_2 \mapsto$};
%\matrix(versiony) [version list, text width=7mm, anchor=west]
%    at ([xshift=\tikzkvspace]locy.east) {
%    {a} & $\txid_0$ & {a} & $\txid_2$ \\
%    {a} & $\emptyset$ & {a} & $\{\txid_1\}$\\
%};
%\tikzvalue{versiony-1-1}{versiony-2-1}{locy-v0}{$v'_0$};
%\tikzvalue{versiony-1-3}{versiony-2-3}{locu-v1}{$v'_1$};
%
%% \draw[-, red, very thick, rounded corners] ([xshift=-5pt, yshift=5pt]locx-v1.north east) |- 
%%  ($([xshift=-5pt,yshift=-5pt]locx-v1.south east)!.5!([xshift=-5pt, yshift=5pt]locy-v0.north east)$) -| ([xshift=-5pt, yshift=5pt]locy-v0.south east);
%
%%blue view - I should  check whether I can use pgfkeys to just declare the list of locations, and then add the view automatically.
%%\draw[-, blue, very thick, rounded corners=10pt]
%% ([xshift=-3pt, yshift=20pt]locx-v1.north east) node (tid1start) {} -- 
%% ([xshift=-3pt, yshift=-5pt]locx-v1.south east) --
%% ([xshift=-3pt, yshift=5pt]locy-v0.north east) -- 
%% ([xshift=-3pt, yshift=-5pt]locy-v0.south east);
%% 
%% \path (tid1start) node[anchor=south, rectangle, fill=blue!20, draw=blue, font=\small, inner sep=1pt] {$\txid_1$};
%%
%%%red view
%%\draw[-, red, very thick, rounded corners = 10pt]
%% ([xshift=-16pt, yshift=5pt]locx-v1.north east) node (tid2start) {}-- 
%%% ([xshift=-16pt, yshift=-5pt]locx-v0.south east) --
%%% ([xshift=-16pt, yshift=5pt]locy-v1.north east) -- 
%% ([xshift=-16pt, yshift=-5pt]locy-v1.south east) node {};
%% 
%%\path (tid2start) node[anchor=south, rectangle, fill=red!20, draw=red, font=\small, inner sep=1pt] {$\txid_2$};
%
%%%Stack for clients tid_1 and tid_2
%%
%%\draw[-, dashed] let 
%%   \p1 = ([xshift=0pt]locy.west),
%%   \p2 = ([yshift=-5pt]locycells.south),
%%   \p3 = ([xshift=10pt]locycells.east) in
%%   (\x1, \y2) -- (\x3, \y2);
%%   
%%\matrix(stacks) [
%%   matrix of nodes,
%%   anchor=north, 
%%   text=blue, 
%%   font=\normalsize, 
%%   row 1/.style = {text = blue}, 
%%   row 2/.style = {text = red}, 
%%   text width= 13mm ] 
%%   at ([xshift=-10pt,yshift=-8pt]locycells.south) {
%%   $\txid_1:$ & $\retvar = 0$\\
%%   $\txid_2:$ & $\retvar = 0$\\
%%   };
%\end{pgfonlayer}
\end{tikzpicture}
\end{center}
\caption{The dependency graph induced by $\hh$.}
\label{fig:hheap-b}
\end{halfsubfig} \\
\end{tabular}
\end{center}
\hrule
\caption{Multi-version key-value stores}
\label{fig:hheap}
\end{figure}

%Formally speaking, we assume a countably infinite set of keys $\ke = \{\key{k}_1, \cdots\}$, a set of transaction identifiers $\TxID = 
%\{ \txid_1, \cdots \}$, 
%a set of clients $\txidset = \{\txid_1, \cdots \}$.
%We also assume values to be natural numbers from the set $\nat$. 
%\ac{from now on, the sort font is used for sets.}
%\begin{definition}
%\label{def:hheap}
%A \emph{version} is a triple $\ver = ( n, \txid, \T)$, where, 
%$n$ is the value of the version, $\txid$ is the identifier of the transaction 
%that wrote the version, and $\T$ is a (possibly empty) set of identifiers of 
%the transactions that read the version.
%Given a version $\ver = (n, \txid, \T)$, 
%we let $\valueOf(\ver) = n$, $\WTx(\ver) = \txid$, $\RTx(\ver) = \T$. 
%The set of versions is denoted as $\Versions$.

%A \emph{Multi-version Key-value Store}, or MKVS, is a mapping  
%$\hh : \ke \rightarrow \Versions^{\ast}$ from keys to lists of versions. 
%\end{definition}
%Given a list of versions $\ver_1 \cdots \ver_{n}$, 
%we let $\lvert \ver_1 \cdots \ver_{n} \rvert = n$ be 
%its length. Also, let $\hh$ be a MKVS, $\key{k}$ be a key 
%such that $\hh(\key{k}) = \ver_1 \cdots \ver_n$, and 
%$i \leq n$ be a strictly positive natural number; then we let 
%$\hh(\key{k}, i) = \ver_i$. 

%\ac{Maybe it's better to keep $\ke$ fixed and say that we look at only a 
%fragment of the key value store. Alternatively, we can go for partial mappings to 
%represent MKVSs, but still avoiding allocation and deallocation of keys.}. To the left 
%we have the set of keys stored y $\hh_0$, in this case $\key{k}_1$ and 
%$\key{k}_2$. To the right, on the same line of a key, a matrix containing the
 %list of versions stored by such a key in $\hh_0$. Starting from the first column, 
 %each version is represented by two adjacent columns in the matrix: the 
 %column to the left gives the value of the version, while the column to the 
 %right contains the identifier of the transaction that wrote the version to the 
 %top, and the identifiers of the transactions that read such a version to the bottom.
%In the case of $\hh_0$, there are two versions stored for $\key{k}_1$: 
%the first one with value $0$, written by $\txid_0$ and read by $\txid_2$; 
%and the second one with value $1$, written by $\txid_2$ and read by no 
%transaction. 

%\ac{
%Throughout this paper, we focus on MKVSs that can be obtained in 
%databases whose consistency guarantees enjoy atomic visibility 
%\cite{framework-concur,SIanalysis,laws}. To this end, we impose 
%some well-formedness constraints on the MKVSs.

%\begin{definition}
%\label{def:hh.wellformed}
%\label{def:ddep}
%A MKVS $\hh$ is \emph{well-formed} if and only if 
%\begin{itemize}
%\item a transaction does not write two different versions for the same key: 
%$\forall \key{k} \in \ke.\;\forall i, j = 1,\cdots, \lvert \hh(\key{k}) \rvert. 
%\WTx(\hh(\key{k}, i)) = \WTx(\hh(\key{k}, j)) \implies i = j$, 
%\item a transaction does not read two different versions for the same key:  
%$\forall \key{k} \in \ke.\;\forall i, j = 1,\cdots, \lvert \hh(\key{k}) \rvert. 
%(\RTx(\hh(\key{k},i) \cap \RTx(\hh(\key{k}, j)) \neq \emptyset) \implies i = j$.
%\item There are no circular dependencies in versions. Given two versions 
%$\ver_1, \ver_2$, we say that $\ver_2$ \emph{direct dependency} from 
%$\ver_1$, written $\ver_1 \xrightarrow{\ddep} \ver_2$, if $\WTx(nu_2) \in \RTx(\ver_1)$; 
%that is, some transaction $\txid$ wrote the version $\ver_2$ after reading $\ver_1$ 
%\ac{A point that this does not ensure a real causal dependency between the 
%two versions, yet it is consistent with the notion of causality employed in databases, 
%should be made}. If $\ver_1, \ver_2$ appear as versions of some object in 
%$\hh$, then we write $\ver_1 \xrightarrow{\ddep(\hh)} \ver_2$. Then the relation $\left(\xrightarrow{\ddep(\hh)}\right)^{+}$ is acyclic in $\hh$, 
%i.e. $\left(\xrightarrow{\ddep(\hh)} \right)^{+} \cap \text{Id} = \emptyset$ (where $\text{Id}$ is 
%the identity relation).
%\end{itemize}
%\end{definition}
%}

%Let us elaborate on the first well-formedness constraint of an MKVS \( \mkvs \) in \cref{def:mkvs}. As stated above, this states that there is no circularity in the dependency relation.
%This in turn ensures that no versions are created \emph{out of thin-air}.
%An example of the out of thin-air anomaly is given in \cref{fig:hheap-b}, where
%transaction $\txid_2$ reads the value of $\ke_1$ written by $\txid_1$;
%conversely, transaction $\txid_1$ reads the value of $\ke_2$ written by $\txid_2$. 
%As we assume transactions read a state of the key-value store from an atomic snapshot fixed at the moment they execute, this situation cannot arise. 
%For $\txid_2$ to read the version written by $\txid_1$, transaction $\txid_2$ must start after $\txid_1$, \ie \( \hh(\ke_2, 1) \xrightarrow{\ddep(\hh_1)} \hh(\ke_1, 1) \).
%Similarly, $\txid_1$ must starts after $\txid_2$, \ie \( \hh(\ke_1, 1) \xrightarrow{\ddep(\hh_1)} \hh_1(\ke_2, 1) \).
%This however violates the well-formedness of MKVSs that $\xrightarrow{\ddep(\hh_1)}$ is acyclic. 

%\azalea{Why $\hh_1$ for the store and not $\hh$? Also I thought the versions are indexed from $0$ in which case index $2$ does not make sense here? \sx{\( \mkvs \) and index starts from 0.}}

%\azalea{common Latin abbreviations such as \ie, \eg, and et al. do not need to be italicised. I have adjusted the macros. I have also rephrased the definitions quite a bit. Make sure you're happy with this.}
%When introducing our semantics of clients in \S \ref{sec:semantics}, we show that (under reasonable conditions) it generates only well-formed MKVSs.

%\ac{
%A view $V$ defines the particular version of each key that a client 
%will observe when executing a transaction. A configuration consists 
%of a MKVS, and the views that a set of clients have each on the state 
%of the MKVS. An example of configuration is given in Figure \ref{fig:hheap}(a). 
%There are two clients, $\txid_1$ and $\txid_2$, each with their own view 
%(represented in the Figure by labelled lines crossing the MKVS at each location). 
%According to the view of $\txid_1$, formally defined as $V_1 = [\key{k}_1 \mapsto 2], 
%\key{k}_2 \mapsto 1]$, this client observes in $\hh$ the second version of key $\key{k}_1$, carrying 
%value $1$, and 
%the first version of $\key{k}_2$, carrying value $0$. Similarly, according to its view 
%$V_2 = [\key{k}_1 \mapsto 2, \key{k}_2 \mapsto 2]$, the client $\txid_2$ observes 
%in $\hh$the second and most up-to-date version for both $\key{k}_1$ and $\key{k}_2$.
%}

\mypar{Views, Configurations and Snapshots.}
%\ac{Configurations are trivial once that view are defined.}
%A MKVS tracks the global state of the system; however, different \emph{clients} may observe different versions of the same key. 
The key-value store tracks the global state, 
but when executing transactions, different \emph{clients} may observe 
different versions of the same key. To keep track of 
the versions they observe, clients are associated with \emph{views} (\cref{def:view}). 

%\begin{definition}[Views and configurations]
%\label{def:view}
%\label{def:cuts}
%\label{def:views}
%\label{def:configuration}
%Given a key-value store $\hh$, \emph{a view} in $\hh$ is a function  
%$\vi: \dom(\hh) \to \Nat$ such that, for any $\ke \in \dom(\hh)$, 
%$\vi(\ke) < \lvert \hh(\ke) \rvert$ and 
%\begin{equation}
%\label{eq:view.atomic}
%\fora{ \ke,\ke' \in \dom(\hh)} \fora{ i,j \in \Nat} (j \leq \vi(\ke) \wedge 
%\WTx(\hh(\ke, v(\ke))) = \WTx(\hh(\ke', i)) \implies i \leq \vi(\ke')
%\tag{Atomic}
%\end{equation}

\begin{definition}[Views and configurations]
\label{def:view}
\label{def:cuts}
\label{def:views}
\label{def:configuration}
Given a key-value store $\hh$, \emph{a view} in $\hh$ is a function  
$\vi: \dom(\hh) \to\powerset{\Nat}$ such that  
\[
\forall \ke \in \dom(\hh).\; 0 \in \vi(\ke) \wedge \forall i \in \vi(\ke).\; i < \lvert \hh(\ke) \rvert.
\]
and 
\begin{equation}
\label{eq:view.atomic}
\fora{ \ke,\ke' \in \dom(\hh)} \fora{ i,j \in \Nat} (j \in \vi(\ke) \wedge 
\WTx(\hh(\ke, \vi(\ke))) = \WTx(\hh(\ke', i)) \implies i \in \vi(\ke')
\tag{Atomic}
\end{equation}

\ac{Now views map keys to sets of indexes. In plain terms, we lifted 
an assumption in the model that required that once a client observes 
a version for key $\ke$, it must also observe any transaction that 
installed a previous version for $\ke$.\\
ALL THE FIGURES THAT USED VIEWS MUST BE DRAWN AGAIN. WE ALSO NEED A 
GRAPHIC FORMALISM FOR THE NEW NOTION OF VIEWS.}


%and \( \pred{atomic}{\mkvs,\vi} \),
%\[
%\begin{rclarray}
%\pred{atomic}{\mkvs,\vi} & \defeq &
%\begin{array}[t]{@{}l@{}}
%\fora{ \ke,\ke' \in \dom(\hh), i,j \in \Nat} \\
%\quad (j \leq \vi(\ke) \wedge 
%\WTx(\hh(\ke, v(\ke))) = \WTx(\hh(\ke', i)) \implies i \leq \vi(\ke')
%\end{array}
%\end{rclarray}
%\]
The set of views of $\hh$ is denoted 
as $\Views(\hh)$, and \emph{the set of views} is defined as:
\[
\Views \defeq \bigcup_{\hh \in \HisHeaps} \Views(\hh)
\]
A \emph{configuration} $\conf$ is a pair $(\hh, \viewFun)$, where $\viewFun: 
\Clients \parfinfun \Views(\hh)$. The configuration $\conf = (\hh, \stub)$ is 
initial if, for any $\ke$, $\hh(\ke) = (\val_0, \txid_0, \emptyset)$, for some 
initial value $\val_0$. The set of configurations is denoted as 
$\Confs$.
\end{definition}
\ac{Before I edited the definition was this one: 
Given a key-value store $\hh$, \emph{a view} in $\hh$ is function  
$\vi: \dom(\hh) \to \Nat$ such that, for any $\ke \in \dom(\hh)$, 
$\vi(\ke) < \lvert \hh(\ke) \rvert$ 
and \( \pred{atomic}{\mkvs,\vi} \),
\[
\begin{rclarray}
\pred{atomic}{\mkvs,\vi} & \defeq &
\begin{array}[t]{@{}l@{}}
\fora{ \ke,\ke' \in \dom(\hh), i,j \in \Nat} \\
\quad (j \leq \vi(\ke) \wedge 
\WTx(\hh(\ke, v(\ke))) = \WTx(\hh(\ke', i)) \implies i \leq \vi(\ke')
\end{array}
\end{rclarray}
\]
Before you used two different universal quantifiers for variables of different types, 
now you use a single universal quantifier for all variables. Make a choice, and be consistent 
with it. I don't think it is necessary to introduce the predicate atomic for the views, 
but we can just have a tagged equation. The less notation we have, the better it is. 
If you insist on using atomic as a predicate, then it should be introduced properly. 
For example: [..] and \( \pred{atomic}{\mkvs, \vi} \), \textbf{WHERE} \( \pred{atomic}{\mkvs, \vi} \defeq \cdots \)
}
Given $\hh \in \HisHeaps$ and two views $\vi, \vi' \in \Views(\hh)$, 
we let $\vi \viewleq \vi'$ if, for any $\ke \in \dom(\hh)$, $\vi(k) \subseteq \vi'(\ke)$. 
\ac{Note that the definition of $\viewleq$ has been changed to reflect the new definition of views.}
%In this case we say that $\vi$ is older than $\vi'$ ($\vi'$ is newer than $\vi$).

%\emph{A configuration} (\cref{def:configuration} augments the notion of kv-stores with the information 
%about the version observed by each client. 
\emph{A configuration}
includes a kv-store and a partial mapping from clients from clients to views.
\ac{Before it was: \emph{A configuration} (\cref{def:configuration}).\\
Configurations were introduced literally two lines ago. Do not include references when 
they are not necessary.}
%about the version observed by each client of its clients. 
The view of the client $\cl$ in $\hh$ reflects the set of versions for each key 
that the client \(\cl \) observes upon executing a transaction. 
The constraint of \cref{eq:view.atomic} establishes that if a client observes 
a version of some key written by a transaction $\txid$, then it must observe all the versions of 
all keys that $\txid$ wrote. This constraint captures the \emph{atomic visibility} of transactions.

{\color{red} We often depict views of clients graphically by drawing client-labelled lines crossing 
versions of key-value stores. A line crossing the $i$-th version of key $\ke$ defines a view 
$\vi$ for client $\cl$, with $\vi(\ke) = i$. One example is given by \cref{fig:hheap-a} where the configuration is
$\conf_0 = (\hh_0, \Set{\cl_1 \mapsto \vi_1, \cl_2 \mapsto \vi_2})$. 
There are two clients, 
$\cl_1$ and $\cl_2$, with views $\vi_1$, and $\vi_2$ respectively. $\vi_1$ crosses $\ke_1$ at its $0$-th 
version, and $\ke_2$ at its $1$-st version. Therefore we have $\vi_1 = \Set{\ke_1 \mapsto 0, \ke_2 \mapsto 1}$. 
Similarly, we have $\vi_2 = \Set{\ke_1 \mapsto 1, \ke_2 \mapsto 0}$. }
\ac{Paragraph Above must be rewritten when we decide on a graphic notation for 
the new notion of views.}
%\ac{Never refer to colors in text. People may read in black and white.}

Given a kv-store $\hh$, a view $\vi$ and a key $\ke \in \dom(\hh)$, 
we commit an abuse of notation and write $\hh(\ke, \vi)$ as a shorthand 
for $\hh(\ke, \max(\vi(\ke)))$. Note that such a version is well-defined because 
we are assuming that $\vi(\ke) \neq \emptyset$.
The view $\vi$ naturally induces a \emph{snapshot} 
by extracting the value of the most up-to-date version it observes for each key $\ke \in \dom(\hh)$. 
%As we will see presently, snapshots are used to determine the value of keys returned 
%by read operations of transactions. 
As we will see presently, transactions operate on their own snapshots.
Because of atomicity, the value of keys returned 
by read operations of transactions. 
\begin{definition}[Snapshots]
\label{def:heaps}
\label{def:snapshot}
Given $\hh \in \HisHeaps$ and $\vi \in \Views(\hh)$, the snapshot of $\vi$ in 
$\hh$ is defined as $\snapshot(\hh, \vi) = \lambda \ke. \valueOf(\hh(\ke, \vi))$.
\end{definition}

\ac{General comment: the explanations should be okay, but I wonder whether we now have 
too many definitions. We may consider putting all the definitions in a table, and 
go through there in the text. Though I am for using tables with notations and 
definitions only as a last resort - they tend to be scary.}

%of $\vi$ by accessing the value of 
%A view $\vi$ in $\hh$ naturally defines a snapshot $\snapshot(\hh, \nu)$
%A MKVS tracks the global state of the system; however, different \emph{clients} may observe different versions of the same key. 
%To model this, we introduce the notion of \emph{views} (\cref{def:views}). 
%A view $V$ reflects the particular version for each key that a client observes upon executing a transaction. 
%%We present an example of views in \cref{fig:hheap-a} with two views: $\client_1$ in red and $\client_2$ in blue.
%More concretely, the view for \( \client_1 \) is given formally as $\vi_1 = \Set{\key{k}_1 \mapsto 1, \key{k}_2 \mapsto 0}$.
%That is, the client with view $\vi_1$ observes the second version (at index 1) of key \( \ke_{1} \) with value $v_1$, and the first version (at index 0) of key \( \ke_2 \) with value $v'_0$.
%%, and 
%%the first version of $\key{k}_2$, carrying value $0$. Similarly, according to its view 
%%$V_2 = [\key{k}_1 \mapsto 2, \key{k}_2 \mapsto 2]$, the client $\txid_2$ observes 
%%in $\hh$the second and most up-to-date version for both $\key{k}_1$ and $\key{k}_2$.
%
%\begin{definition}[Views]
%\label{def:view}
%\label{def:cuts}
%\label{def:views}
%\emph{A view} is a partial finite function from keys to indexes:
%$
%\vi \in \Views \defeq \Addr \parfinfun \Nat 
%%\begin{rclarray}
%%    \vi \in \Views & \defeq & \Addr \parfinfun \Nat 
%%\end{rclarray}
%$.                                                                 
%The \emph{view composition}, $\composeVI: \Views \times \Views \rightharpoonup \Views$ is defined as the standard disjoint function union: $\composeVI \eqdef \uplus$. 
%% \( \vi \composeVI \vi' \defeq \vi \uplus \vi'\) 
%The \emph{unit view}, $\unitVI \in \Views$, is a function with an empty domain: $\unitVI \eqdef \emptyset$. 
%% and the unit is \( \unitVI \defeq \emptyset\).
%The \emph{order relation} on views, $\orderVI: \Views \times \Views$, is defined between two views with the same domain as the point-wise comparison of their indexes for each entry: 
%\[
%\begin{rclarray}
%    \vi \orderVI \vi' & \defiff & \dom(\vi) = \dom(\vi') \land \fora{\ke} \cu(\ke) \leq \cu'(\ke) \\
%\end{rclarray}
%\]
%\end{definition}
%%
%We say view $\vi$ is \emph{older} than view $\vi'$ (or $\vi'$ is \emph{newer} than $\vi$) whenever $\vi \orderVI \vi'$ holds.
%
%
%\mypar{Configurations} A \emph{configuration} comprises an MKVS, and the views associated with clients.
%In \cref{fig:hheap-a} we present an example of a configuration comprising an MKVS and the two views associated with clients $\client_1$ and $\client_2$. 
%We write $\version(\hh, \ke, \vi)$ for $\hh(\ke, \vi(\ke))$; 
%and write $\valueOf(\hh, \ke, \vi)$ as a shorthand for $ \valueOf(\version(\hh, \key{k}, V))$; similarly for $\WTx, \RTx$.
%%we commit an abuse of notation and often write $\valueOf(\hh, \ke, \vi)$ in lieu of $ \valueOf(\version(\hh, \key{k}, V))$, and similarly for $\WTx, \RTx$.
%When $\ver = \version(\hh, \ke, \vi)$, we say that \emph{$\vi$ $\ke$-points to $\ver$ in $\hh$}. 
%When $\ver = \hh(\ke, i)$ for some $0 \leq i \le \vi(\ke)$, we say that \emph{$\vi$ $\ke$-includes $\ver$ in $\hh$}.
%Lastly, we always assume that MKVSs, views, and configurations are well-formed, unless otherwise stated.
%
%
%
%\begin{definition}[Configurations]
%A view $\vi$ is \emph{well-formed with respect to an MKVS} $\mkvs$, written \( \wfV{\mkvs, \vi} \),  iff they have the same domain and every index from $\vi$ is within the range of the corresponding entry in $\mkvs$ and the view is \emph{atomic} with  respect to the key-value store: 
%\[
%\begin{rclarray}
%    \wfV{\mkvs, \vi} & \defeq & \dom(\mkvs) = \dom(\vi) \land \fora{\ke \in \dom(\vi)} 0 \leq \vi(\ke) < \lvert \mkvs(\ke) \rvert \\
%    \pred{atomic}{\vi ,\hh} & \eqdef & \fora{\txid } \exsts{\ke, i} i \leq \vi(\ke) \land \hh(\ke,i) = (\stub, \txid, \stub) \implies \pred{visible}{\txid, \vi, \hh} \\ 
%    \pred{visible}{\txid, \vi, \hh} & \eqdef & \fora{\ke, i} \hh(\ke,i) = (\stub, \txid, \stub) \implies i \leq \vi(\ke) 
%\end{rclarray}
%\]
%%
%\azalea{We need a symbol for this to fill the ???? above. Also ???? below. \sx{Done}}
%A \emph{configuration} $\conf$ is a pair of the form $(\hh, \viewFun)$, where $\hh$ denotes an MKVS, and $\viewFun: \Clients \parfinfun \Views$ is a partial finite function from clients to views. 
%A configuration $\conf = (\hh, \viewFun)$ is \emph{well-formed}, written \( \wfC{\conf}\), iff for all clients $\cl \in \dom(\viewFun)$, the view $\viewFun(\txid)$ is well-formed with respect to $\hh$. 
%%We say that a view $V$ is well-defined with respect to the 
%%MKVS $\hh$ if, $\forall \key{k} \in \ke. 0 < V(\key{k}) \leq 
%%\lvert \hh(\key{k}) \rvert$. 
%%Given a view $V$ that is well-defined 
%%with respect to a 
%
%\end{definition}
%
%\mypar{Snapshots} When a client executes a transaction on the $\mkvs$ MKVS, it extracts a \emph{snapshot} of it via the \( \func{snapshot}{\mkvs, \vi} \) function, extracting the values corresponding to the versions indexed by its view \( \vi \) (\cref{def:snapshot}).
%For instance, for client \( \client_1 \) in \cref{fig:hheap-a}, the $\func{snapshot}{\cdots}$ functions yields a state where key $\ke_1$ carries value $v_1$ and second key \( \ke_2 \) carries value $v'_0$.
%%The concrete state extracted in this way takes the name of the \emph{snapshot} of the transaction.
%%In general, the process of determining the view of a client, hence the snapshot in which such a client executes transactions, is non-deterministic.
%
%\azalea{Before in MKVSs we had values drawn from $\Nat$ in \cref{def:mkvs}. Now we use $\Val$. I think you mean to use $\Val$ in both places? \sx{I would say so} }
%\begin{definition}[Snapshots]
%\label{def:heaps}
%\label{def:snapshot}
%Given the sets of values $\Val$  and keys \( \Addr\)  (\cref{def:mkvs}), the set of \emph{snapshots} is:
%$
%    \h \in \Heaps \eqdef \Addr \parfinfun \Val
%$. 
%%\[
%%\begin{rclarray}
%%    \h \in \Heaps & \eqdef & \Addr \parfinfun \Val
%%\end{rclarray}
%%\]
%The \emph{snapshot composition function}, $\composeH: \Heaps \times \Heaps \parfun \Heaps$, is defined as $\composeH \eqdef \uplus$, where $\uplus$ denotes the standard disjoint function union. The \emph{ snapshot unit element} is $\unitH \eqdef \emptyset$, denoting a function with an empty domain.
%The \emph{partial commutative monoid of snapshots} is $(\Heaps, \composeH, \{\unitH\})$.
%Given an MKVS $\hh$ and a view $\vi$, the snapshot of $\vi$ in $\hh$, written $\snapshot(\hh, \vi) $, is defined as:
%$
%    \snapshot(\hh, \vi) \defeq \lambda \ke \ldotp \valueOf(\hh, \ke, \vi)
%$.
%%\[
%%\begin{rclarray}
%%    \snapshot(\hh, \vi) & \defeq & \lambda \ke \ldotp \valueOf(\hh, \ke, \vi).
%%\end{rclarray}
%%\]
%\end{definition}
%
%\sx{Need some explanation}
%\ac{General Comment on this Section: it is too abstract. We 
%should give either here or in the introduction an example of computation - 
%the write skew program should be okay that helps the reader understanding 
%what's going on. Also, it could be also good to illustrate the notions 
%of execution tests and consistency models.}
%
%\sx{From Andrea: introduce the execution test here with a table, also introduce fingerprint here}

\mypar{Relationship between kv-stores and dependency graphs.}
\ac{It could be that this subsection gets moved to a different section, where 
we also relate specifications of consistency models using dependency graphs
with execution tests.}
\emph{Dependency graphs} are a formalism  introduced by Adya to specify 
consistency models of transactional databases \cite{adya}. They are direct graphs consisting of transactions as nodes, 
each of which is labelled with the respective fingerprint, and labelled edges 
between transactions for specifying how information flows within a computation. 
Specifically, a transaction $\txid$ may read a version for a key $\ke$ that has been written by another transaction $\txid'$ 
(\emph{write-read dependency}), overwrite a version of $\ke$ written by $\txid'$ (\emph{write-write dependency}), or 
read a version of $\ke$ that is later overwritten by $\txid'$ (\emph{read-write anti-dependency}).
\begin{definition}
A \emph{dependency graph} is a quadruple $\Gr = (\TtoOp{T}, \RF, \VO, \AD)$, where
\begin{itemize}
\item $\TtoOp{T}: \TxID_{0} \rightharpoonup \powerset{\Ops}$ is a partial function 
mapping transaction identifiers to the set of operations they perform, 
\item $\RF : \Keys \rightarrow \dom(\TtoOp{T}) \times \dom(\TtoOp{T})$ is a function that 
maps each key $\ke$ into a relation between transactions, such that for any $\txid, \txid_1, \txid_2, 
\ke, \cl, m, n$: 
\begin{itemize}
\item if $(\otR, \ke, \val) \in \TtoOp{T}(\txid)$, either $\val = \val_0$ 
and there exists no $\txid'$ such that $\txid' \xrightarrow{\RF(\ke)} \txid$,  
or there exists $\txid'$ such that $(\otW, \ke, \val) \in \TtoOp{T}(\txid')$, and $\txid' \xrightarrow{\RF(\ke)} \txid$, 
\item if $\txid_1 \xrightarrow{\RF(\ke)} \txid$ and $\txid_2 \xrightarrow{\RF(\ke)} \txid$, then 
$\txid_2 = \txid_1$, 
\item if $\txid_{\cl}^{m} \xrightarrow{\RF(\ke)} \txid_{\cl}^{n}$, then $m < n$.
\end{itemize}
\item $\VO: \Keys \rightarrow \dom(\TtoOp{T}) \times \dom(\TtoOp{T})$ is a function 
that maps each key into an irreflexive relation between transactions, such that for any $\txid, \txid', \ke, \cl, m, n$, 
\begin{itemize}
\item if $\txid \xrightarrow{\VO(\ke)} \txid'$, then $(\otW, \ke, \_) \in \TtoOp{T}(\txid), (\otW, \ke, \_) \in \TtoOp{T}(\txid')$, 
\item if $(\otW, \ke, \_) \in \TtoOp{T}(\txid), (\otW, \ke, \_) \in \TtoOp{T}(\txid')$, then either $\txid = \txid'$, 
$\txid \xrightarrow{\VO(\ke)} \txid'$, or $\txid' \xrightarrow{\VO(\ke)} \txid$, 
\item if $\txid_{\cl}^{m} \xrightarrow{\RF(\ke)} \txid_{\cl}^{n}$, then $m < n$.
\end{itemize}
\item $\AD: \Keys \rightarrow \dom(\TtoOp{T}) \times \dom(\TtoOp{T})$ is defined 
by letting $\txid \xrightarrow{\AD(\ke)} \txid'$ if and only if $(\otR, \ke, \_) \in \TtoOp{T}(\txid)$, 
$(\otW, \ke, \_) \in \TtoOp{T}(\txid')$ and 
either there exists no $\txid''$ such that $\txid'' \xrightarrow{\RF(\ke)} \txid$, or 
$\txid'' \xrightarrow{\RF(\ke)} \txid$, $\txid'' \xrightarrow{\VO(\ke)} \txid'$ for 
some $\txid''$.
\end{itemize}
\end{definition}
Given a dependency graph $\Gr = (\TtoOp{T}, \RF, \VO, \AD)$, we often 
commit an abuse of notation and use $\RF$ to denote the relation 
$\bigcup_{\ke \in \Keys} \RF_{\ke}$; a similar notation is adopted for $\VO, \AD$. 
It will always be clear from the context whether the symbol $\RF$ refers to a function 
from keys to relations, or to a relation between transactions. 

%A dependency graph $\Gr = (\TtoOp{T}, \RF, \VO, \AD)$ is well-formed if 
%$(\PO \cup \RF \cup \VO)$ is acyclic, i.e. its transitive closure is irreflexive. 
%Henceforth, we always assume that dependency graphs are well-formed, 
%and we let 
We let $\Dgraphs$ be the set of all dependency graphs.
It is always possible to convert a kv-store $\hh$ into a well-formed dependency 
graph. For example, \cref{fig:hheap-b} illustrates the dependency graph constructed 
from the kv-store depicted in \cref{fig:hheap-a}.

\begin{definition}
\label{def:kv2graph}
Given a kv-store $\hh$, the \emph{dependency graph} $\Gr_{\hh} = (\TtoOp{T}_{\hh}, \RF_{\hh}, 
\VO_{\hh}, \AD_{\hh})$ is defined as follows: 
\begin{itemize}
\item for any $\txid \neq \txid_0$, $\TtoOp{T}_{\hh}(\txid)$ is defined if and only if there exists an index $i$ and a key 
$\ke$ such that either $\txid = \WTx(\hh(\ke, i))$, or $\txid \in \RTx(\hh(\ke,i))$; furthermore, 
$(\otW, \ke, \val) \in \TtoOp{T}(\txid)$ if and only 
if $\txid = \WTx(\hh(\ke, i))$ for some $i$, and 
$(\otR, \ke, \val) \in \TtoOp{T}(\txid)$ if and only if $\txid \in \RTx(\hh(\ke, i))$ for some $i$, 
\item $\txid \xrightarrow{\RF(\ke)} \txid'$ if and only if there exists an index $i: 0 < i < \lvert \hh(\ke) \rvert$ 
such that $\txid = \WTx(\hh(\ke, i))$, and $\txid' \in \RTx(\hh(\ke, i))$, 
\item $\txid \xrightarrow{\VO(\ke)} \txid'$ if and only if there exist two indexes $i,j$: $0 < i < j < \lvert \hh(\ke) \rvert$ 
such that $\txid = \WTx(\ke, i)$, $\txid' = \WTx(\ke, j)$, 
\item $\txid \xrightarrow{\AD(\ke)} \txid'$ if and only if there exist two indexes $i,j$: $0 < i < j < \lvert \hh(\ke) \rvert$ 
such that $\txid \in \RTx(\ke, i)$ and $\txid' = \WTx(\ke, j)$.
\end{itemize}
\end{definition}

\begin{theorem}
\label{thm:kv2graph}
The function $\Gr_{(\stub)}$ is a bijection between kv-stores and well-formed dependency graphs.
\end{theorem}

\ac{Both the formal definition of dependency graph and the function $\graphof$ will need to go in the appendix.}

\subsection{Programming Language and Operational Semantics}

We define a simple programming language for client programs interacting with an MKVS where clients may only interact with the MKVS via transactions. 
For simplicity, we abstract away from aborting transactions: rather than assuming that a transaction may abort due to a violation of the consistency guarantees given by the MKVS, we only allow the execution of a transaction when its effects are guaranteed to not violate the consistency model. This emulates the setting in which clients always restart a transaction if it aborts.
\sx{The sentence is too long. I think before the ``:'' is enough}
 
\subsubsection{Programming Language}

\emph{A program} \( \prog \) contains a fixed number of clients, where each client is associated with a unique identifier \( \thid \in \ThreadID \), and executes a sequential \emph{command}.
We thus model a program $\prog$ as a function from client identifiers  to commands (\cref{def:language}).
For clarity, we often write \( \cmd_{1}\ppar \dots \ppar \cmd_{n}\) as a syntactic sugar for a program \( \prog \) with $n$ implicit clients associated with identifiers, $\thid_1 \dots \thid_n$, where each client $\thid_i$ executes the command $\cmd_i$:  \( \prog = \Set{\thid_{1} \mapsto \cmd_{1}, \dots, \thid_{n} \mapsto \cmd_{n}  }\).
Sequential \emph{commands}, ranged over by $\cmd$, are defined by an inductive grammar comprising the standard constructs of $\pskip$, sequential composition ($\cmd; \cmd$), non-deterministic choice ($\cmd+\cmd$) and loops ($\cmd^*$).
To simulate conditional branching and loops, commands include the standard constructs of assume (\( \passume{\expr}\)), and assignment (\( \passign{\var}{\expr} \)), where \( \var \) denotes a local variable (on the stack), and \( \expr \) denotes an arithmetic expression with no side effect.  
evaluated with respect to a stack  with no side effect.
Arithmetic expressions are evaluated with respect to a local stack (variable store) -- see  \cref{def:stacks,def:eval-expr} below.
We assume a countably infinite set of local variables, $\Vars $, and use the \texttt{typewriter} font for its meta variables, \eg $\var$. 

Commands additionally include the \emph{transaction} construct, $\ptrans{\trans}$, denoting the \emph{atomic} execution of the transaction $\trans$. 
The atomicity guarantees the execution are dictated by the underlying consistency model.
\emph{Transactions}, ranged over by $\trans$, are similarly defined by an inductive grammar comprising $\pskip$, \emph{primitive commands} \( \transpri \), non-deterministic choice, loops and sequential composition.
Primitive commands include assignment (\( \passign{\var}{\expr}\)), lookup (\( \pderef{\expr}{\expr}\)), mutation (\( \pmutate{\expr}{\expr}\)) and assume (\( \passume{\expr}\)). 
Note that transactions do \emph{not} contain the \emph{parallel} composition construct ($\ppar$) as they are executed atomically.

%We assume a valid transactions codes must have the only return at the end.
%For better presentation, sometime we omit the default return zero \( \preturn{0} \).
%Transactions can only assign to their own variables, namely transaction variables (\defref{def:program_values}), but it can read from both the thread and transaction stacks.

%\begin{definition}[Program values]
%where $ \nat \in \Nat$ denotes the set of natural numbers.
%\end{definition}

\begin{definition}[Programming language]
\label{def:language}
\label{def:program_values}
A \emph{program}, $\prog \in \Programs$, is a partial finite function from client identifiers to commands.
The sequential \emph{commands}, \( \cmd \in \Commands \), are defined by the following grammar, where
$\val \in \Val \eqdef \Nat \cup \Addr$ denotes the set of \emph{program values}, and $\addr \in \Addr$ denotes the set of MKVS keys (\cref{def:mkvs}):
\[
\begin{array}{@{} l @{\hspace{20pt}}  l @{}}
    \begin{rclarray}
    \cmd & ::= &
        \pskip \mid 
        \passign{\thvar}{\expr} \mid
        \passume{\expr} \mid
        \ptrans{\trans} \mid 
        \cmd \pseq \cmd \mid 
        \cmd \pchoice \cmd \mid 
        \cmd \prepeat \\
%
	 \expr & ::= &
        \val \mid
        \var \mid
        \expr + \expr \mid
        \expr \times \expr \mid
        \dots  
       \end{rclarray} 
%    
	& 
%
	\begin{rclarray}        
	\trans & ::= &
        \pskip \mid
        \transpri \mid 
        \trans \pseq \trans \mid
        \trans \pchoice \trans \mid
        \trans\prepeat   \\
%        
	\transpri & ::= &
        \pass{\txvar}{\expr} \mid
        \pderef{\txvar}{\expr} \mid
        \pmutate{\expr}{\expr} \mid
        \passume{\expr} 
%
    \end{rclarray}
\end{array} 
\]
%The $\trans \in \Transactions$ in the grammar above denotes a \emph{transaction} defined by the following grammar including the primitive transactional commands \( \transpri\):
%The transaction codes \( \ptrans{\trans} \) satisfy a well-form condition that there is exactly a return at the end, \ie \( \ptrans{\trans} \iff \exsts{\trans', \expr} \trans \equiv ( \trans' \pseq \preturn{\expr} )  \land \pred{noRet}{\trans'} \)
%\[
%    \begin{rclarray}
%        \transpri & ::= &
%        \pass{\txvar}{\expr} \mid
%        \pderef{\txvar}{\expr} \mid
%        \pmutate{\expr}{\expr} \mid
%        \passume{\expr} \mid \\
%        %\preturn{\expr} \\
%        \trans & ::= &
%        \pskip \mid
%        \transpri \mid 
%        \trans \pseq \trans \mid
%        \trans \pchoice \trans \mid
%        \trans\prepeat
%    \end{rclarray}
%\]
%Given the set of \emph{keys}, $\addr \in \Addr$ (\cref{def:mkvs}), the set of \emph{program values} is $\val \in \Val \eqdef \Nat \cup \Addr$.
%The $\expr \in \Expressions$ denotes an \emph{arithmetic expression} defined by the grammar:
%\[
%    \begin{rclarray}
%        \expr & ::= &
%        \val \mid
%        \var \mid
%        \expr + \expr \mid
%        \expr \times \expr \mid
%        \dots 
%    \end{rclarray}
%\]
\end{definition}

\begin{definition}[Stacks]
\label{def:stacks}
\label{def:eval-expr}
A \emph{stack}, $\stk \in \Stacks$, is a partial finite function from variables to values: \(\Stacks \defeq \Vars \parfinfun \Val \).
Given a stack $\stk \in \Stacks$, %(\cref{def:stacks}), 
the \emph{arithmetic expression evaluation} function, $\evalE[(.)]{.}:\Expressions \times \Stacks \parfun \Val$, is defined inductively over the structure of expressions: 
%
\[
	\evalE{\val}  \defeq  \val 
	\qqquad 
	\evalE{\var} \defeq \stk(\var) 
	\qqquad 
	\evalE{\expr_{1} + \expr_{2}}  \defeq  \evalE{\expr_{1}} + \evalE{\expr_{2}} 
	\qqquad
	\evalE{\expr_{1} \times \expr_{2}}  \defeq  \evalE{\expr_{1}} \times \evalE{\expr_{2}} 
	\qqquad 
	\dots
\]
\end{definition}

%\begin{definition}[Evaluation of expression]
%\label{def:eval-expr}
%Given a stack $\stk \in \Stacks$, %(\cref{def:stacks}), 
%the \emph{arithmetic expression evaluation} function, $\evalE[(.)]{.}:\Expressions \times \Stacks \parfun \Val$, is defined inductively over the structure of expressions: 
%%
%\[
%	\evalE{\val}  \defeq  \val 
%	\qquad 
%	\evalE{\var} \defeq \stk(\var) 
%	\qquad 
%	\evalE{\expr_{1} + \expr_{2}}  \defeq  \evalE{\expr_{1}} + \evalE{\expr_{2}} 
%	\qquad
%	\evalE{\expr_{1} \times \expr_{2}}  \defeq  \evalE{\expr_{1}} \times \evalE{\expr_{2}} 
%	\qquad 
%	\dots
%\]
%%\[
%%    \begin{rclarray}
%%        \evalE{\val} & \defeq & \val \\
%%        \evalE{\var} & \defeq & \stk(\var) \\
%%        \evalE{\expr_{1} + \expr_{2}} & \defeq & \evalE{\expr_{1}} + \evalE{\expr_{2}} \\
%%        \evalE{\expr_{1} \times \expr_{2}} & \defeq & \evalE{\expr_{1}} \times \evalE{\expr_{2}} \\
%%        \dots & \eqdef & \dots \\
%%    \end{rclarray}
%%\]
%\end{definition}

\subsubsection{Semantics for transactions}
\label{sec:trans-semantics}

\paragraph{Operations and Fingerprints}
Recall that when a transaction starts, it extracts a local snapshot $\ss$ of the MKVS via its view. 
The execution of the transaction is then carried out \emph{locally} on $\ss$:  
read operations inspect $\ss$ and write operations update $\ss$. 
%, as we discuss in \cref{sec:prog-semantics}.
To track the effect of a transaction while executing, each transaction is associated with a \emph{fingerprint}, initially set to empty. 
The fingerprint of a transaction records its interactions (read or write operations) with the MKVS.
Each time a transaction executes a primitive command \( \transpri\) (locally on its snapshot), its fingerprint is accordingly updated to track the effect of $\transpri$.
%More concretely, every time the transaction performs a write operation, 
Once the local execution of a transaction is complete, its local snapshot is discarded, and the transaction is committed by propagating the effects in its fingerprint to the MKVS.\\
%
\indent The fingerprint of a transaction, $\opset$, is modelled as a set of \emph{operations}. 
Each operation $\op$ is a tuple of the form $(\mathit{tag}, \ke, \val)$, where $\mathit{tag}$ denotes the operation type and may be one of $\otW$ for \emph{write} operations, or $\otR$ for \emph{read} operations; 
the $\ke$ denotes the operation key, and  $\val$ denotes the operation value. 
For instance, $(\otW, \ke, \val)$ denotes the execution of an operation where value $\val$ is written to key $\ke$; and
$(\otR, \ke, \val)$ denotes the execution of an operation where value $\val$ is read from key $\ke$.\\
%
\indent When tracking the write operations (with tag $\otW$) in a fingerprint $\opset$, for each key $\ke$, only the effect of the \emph{last} write to $\ke$ is recorded in $\opset$. 
This is because while executing, a transaction may write to $\ke$ multiple times. Upon committing however, only the last value written to $\ke$ is propagated to the MKVS. 
As such, $\opset$ need not include duplicate writes operations for  $\ke$ and thus only records its latest write. 
This choice is motivated by the atomic visibility of transactions: the intermediate writes to a key are not visible from outside a transaction.\\
%
\indent Analogously, when tracking the read operations (with tag $\otR$), for each key $\ke$, the fingerprint $\opset$ includes a read operation for $\ke$ \emph{only if} the transaction reads from $\ke$ \emph{before} subsequently writing to it.
More concretely, a read $\op$ operation from $\ke$ is recorded in $\opset$ if $\op$ reads from the initial value recorded in the snapshot. % $\ss$. 
That is, read operations track those operations that read from the MKVS and not those that read internally from the updated snapshot.
These two principles, \emph{last-write} and \emph{read-before-write}, are formalised in the definition of {fingerprint update}, $\opset \addO \op$,  in \cref{def:ops} below. 
For convenience, we define $\opset \addO \emptyop \eqdef \opset$, to denote fingerprint update with respect to an operation with no effect.


In what follows we write $\op \projection{i}$ to denote the $i$\textsuperscript{th} projection of $\op$. 
We lift this notation to sets and write \eg $\{\op_1 \cdots \op_n\}\projection{i}$ for the set comprising the $i$\textsuperscript{th} projection of $\op_1 \cdots \op_n$. 




%The fingerprints include the \emph{first read preceding a write} and \emph{last write} for each key.
%This is because a transaction is executed atomically, all the intermediate steps are not observable from the outside world.
%The \emph{fingerprints} formally is a set of \emph{operations} \( \Ops \) which are either read \( (\etR, \addr, \val)\) from the \( \ke \) with the value \( \val \), or write \( (\etW, \addr, \val) \) to the key \( \ke \) with the value \( \val \) (\cref{def:ops}).


%
%Note that In the \cref{def:ops}, the \( (.)\projection{(.)} \) denotes projection.
%For a tuple, for example \( \op\projection{i} \), it gives the \emph{i-th} element of the tuple.
%It is lifted to a set of tuples, for example \( \opset\projection{i}\), which gives a set of all the \emph{i-th} elements.
%The well-formedness condition for fingerprints asserts it is a set of operations in which there are at most one read and one write for each key.
%The composition, then, is defined as set disjointed union as long as the result is well-formed.
 

\begin{definition}[Operations and fingerprints]
\label{def:ops}
The set of \emph{operations} is \( \op \in \Ops \eqdef \powerset{\Set{\otR, \otW}} \times \Addr \times \Val\).
%A \emph{transaction operation} is a tuple of \emph{an operation tag} that is either read or write, an key and a value.
%\[
%\begin{rclarray}
%\op \in \Ops & \defeq  & \powerset{\Set{\otR, \otW}} \times \Addr \times \Val
%\end{rclarray}
%\]
\emph{A fingerprint}, \( \opset \in \Opsets \), is a subset of \( \Ops \) in which any two elements contain either different tags or different keys:
\[
    \begin{rclarray}
        \Opsets & \defeq & \Setcon{\opset}{%
            \opset \subseteq \Events \land \fora{\op, \op' \in \opset} 
            \op\projection{1} \neq  \op'\projection{1} \lor \op\projection{2} \neq  \op'\projection{2}  } \\
    \end{rclarray}
\]
%
The \emph{fingerprint update function}, $\addO: \Opsets \times \Ops \cup \{\emptyop\} \rightarrow \Opsets$, is defined as follows: 
\[
\begin{array}{c @{\qqqquad} c}
\begin{rclarray}
    \opset \addO (\etR, \addr, \val) & \defeq & 
    \begin{cases}
        \opset \uplus \Set{(\etR, \addr, \val)} & (\stub, \addr, \stub) \notin \opset \\
        \opset &  \text{otherwise} \\
    \end{cases} \\
\end{rclarray}
&
\begin{rclarray}
    \opset \addO (\etW, \addr, \val) & \defeq & \left( \opset \setminus \Set{(\etW, \addr, \stub)} \right) \uplus \Set{(\etW, \addr, \val)} \\
    \opset \addO \emptyop & \defeq & \opset \\
\end{rclarray}
\end{array}
\]


The \emph{fingerprint unit element} is \( \unitE \defeq \emptyset\); 
the \emph{fingerprint composition}, $\composeO: \Opsets \times \Opsets \rightharpoonup \Opsets$,  is defined when two fingerprints contain operations with distinct keys: 
\[ 
\begin{rclarray}
    \opset \composeO \opset' & \defeq & 
    \begin{cases}
        \opset \uplus \opset' & \text{if } \opset\projection{2} \cap \opset'\projection{2} = \emptyset \\
        \text{undefined} & \text{otherwise}
    \end{cases}
\end{rclarray}
\]
\end{definition}
%\pg{ \(\opset \addO \opset'\)?. \sx{We can define this but not sure it is useful}}
%
%\begin{lemma}
%The well-formedness of fingerprints is closed under \( \addO \).
%\end{lemma}
%\azalea{What is $\addO$?? You have not defined this yet.}

\paragraph{Operational Semantics of Transactions}
The \emph{operational semantics of transactions} (\cref{def:language}) is given at the bottom of  \cref{fig:transaction_semantics},
described with respect to a transactional state of the form \((\stk, \sn, \opset)\), where $\stk$ denotes a local variable stack (\cref{def:stacks}), $\sn$ denotes an MKVS snapshot (\cref{def:snapshot}), and $\opset$ denotes a fingerprint (\cref{def:ops}).\\
%
\indent To this end, we first define a \emph{local state transformer} on pairs of stacks and snapshots for primitive commands \(\trans_{p}\) (top of  \cref{fig:transaction_semantics}).
More concretely, we write $(\stk, \h)  \toLTS{\trans_{p}} (\stk', \h')$ to denote that executing $\trans_p$ updates the stack $\stk$ and snapshot $\h$, to $\stk'$ and $\h'$, respectively.
Additionally, we compute the fingerprint (effect) of a primitive command via the $\funcFont{fp}$ function and write $\func{fp}{\stk, \h, \trans_p}$ to denote the effect of $\trans_p$ on stack $\stk$ and snapshot $\h$.\\
%
%We also define its fingerprint by \( \funcn{fp} \) function, which denotes the contribution of the primitive command that might observed by the external environment, \ie transactions from other threads.
%The \( \funcn{fp} \) extracts the read or write operation from loop-up and mutation respectively, otherwise \( \emptyop \).
%We also define a binary operator \( \opset \addO \op \) that specifies the effects of adding a new operation \( \op \) to the fingerprints \( \opset \).
%If the new operation is a read, for example \((\otR, \addr, \val)\) where \( \addr \) is the key and \( \val\) is the associated value, and there is no other operation related to the same key, this new read operation will be included in the result.
%Meanwhile, if the new operation is a write, it will overwrite all preview write operations to the same key.
%This ensures the fingerprints contains only the first read preceding a write, and only the last write for each key.
%This choice is motivated by the fact that we only focus on atomically visible transactions: keys are read from a snapshot of the database, and new version are written only at the moment the transaction commits.
%For technical reasons, if the right hand side is a special token \( \emptyop \) corresponding to a command does not result in an interaction with key-value store.
%Therefore, the semantics for primitive command \(\rl{TPrimitive}\) updates the stack and heap by the transformers relation and updates the operation set by first extracting the operation and adding it via \( \addO \) operator.
%The semantics for non-deterministic choices \(\rl{TChoice}\), sequential compositions \(\rl{TSeqSkip}\) and \(\rl{TSeq}\), and iteration \(\rl{TIter}\) have the expected behaviours.
%
\indent Each step of the operational semantics updates the stack and snapshot using the local state transformer, and updates the fingerprint via the \( \addO \) function.
The behaviour of all transitions in \cref{fig:transaction_semantics} is standard.


\begin{figure}[!t]
\hrule%\vspace{5pt}
%\begin{flushleft}
%The state transformers on pairs of stacks and snapshots for the primitive commands \(\trans_{p}\) (left), and the \( \funcn{op} \) for extracting the operation from the primitive commands (right):
%\end{flushleft}
\[
\begin{array}{@{} c @{\qquad} c @{}}
\begin{rclarray}
(\Stacks \times \Heaps)\!\!\! & \toLTS{\trans_p} &   (\Stacks \times \Heaps)  \vspace{5pt}\\
(\stk, \h)  & \toLTS{\passign{\var}{\expr}}          & (\stk\rmto{\var}{\evalE{\expr}}, \h)                  \\
(\stk, \h)  & \toLTS{\pderef{\var}{\expr}}           & (\stk\rmto{\var}{\h(\evalE{\expr})}, \h)              \\
(\stk, \h)  & \toLTS{\pmutate{\expr_{1}}{\expr_{2}}} & (\stk, \h\rmto{\evalE{\expr_{1}}}{\evalE{\expr_{2}}}) \\
(\stk, \h)  & \toLTS{\passume{\expr}}                & (\stk, \h) \text{ where } \evalE{\expr} \neq 0          
%(\stk, \h) & \toLTS{\preturn{\expr}}                & (\stk\rmto{\ret}{\evalE{\expr}}, \h)                 
\end{rclarray}                                                                                               
&
\begin{array}{@{} l @{}}
\funcFont{fp}: \Stacks \times \Heaps \times \trans_p \rightarrow \Ops \cup \{\emptyop\} \vspace{5pt} \\
\begin{rclarray}
\func{fp}{\stk, \h, \passign{\var}{\expr}}          & \defeq & \emptyop                                     \\
\func{fp}{\stk, \h, \pderef{\var}{\expr}}           & \defeq & (\etR, \evalE{\expr}, \h(\evalE{\expr}))     \\
\func{fp}{\stk, \h, \pmutate{\expr_{1}}{\expr_{2}}} & \defeq & (\etW, \evalE{\expr_{1}}, \evalE{\expr_{2}}) \\
\func{fp}{\stk, \h, \passume{\expr}}                & \defeq & \emptyop                                     \\
\func{fp}{\stk, \h, \preturn{\expr}}                & \defeq & \emptyop                                     
\end{rclarray}
\end{array}
\end{array}
\]

%\hrule\vspace{5pt}
%\begin{flushleft}
%The binary operator \( \opset \addO \op \) that specifies the effects of adding a new operation \( \op \) to the set \( \opset \):
%\end{flushleft}
%\[
%\begin{array}{c @{\qquad} c}
%\begin{rclarray}
%    \opset \addO (\etR, \addr, \val) & \defeq & 
%    \begin{cases}
%        \opset \uplus \Set{(\etR, \addr, \val)} & (\stub, \addr, \stub) \notin \opset \\
%        \opset &  \text{otherwise} \\
%    \end{cases} \\
%\end{rclarray}
%&
%\begin{rclarray}
%    \opset \addO (\etW, \addr, \val) & \defeq & \left( \opset \setminus \Set{(\etW, \addr, \stub)} \right) \uplus \Set{(\etW, \addr, \val)} \\
%    \opset \addO \emptyop & \defeq & \opset \\
%\end{rclarray}
%\end{array}
%\]

\hrule%\vspace{5pt}
%\begin{flushleft}
%Given the set of stacks \( \Stacks \) (\cref{def:stacks}), heaps \( \Heaps \) (\cref{def:heaps}) and transactions \( \Transactions \) (\cref{def:language}) and the arithmetic expression evaluation \( \evalE{\expr} \) (\cref{def:language}), the \emph{operational semantics of transactions}:
%\end{flushleft}
\[
\begin{rclarray}
\toL & : & ((\Stacks \times \Heaps \times \Opsets) \times \Transactions) \times ((\Stacks \times \Heaps \times \Opsets) \times \Transactions)
\end{rclarray}
\]

\begin{mathpar}
    \inferrule[\rl{TPrimitive}]{%
        (\stk, \h) \toLTS{\transpri} (\stk', \h')
        \\ \op = \func{fp}{\stk, \h, \transpri}
    }{%
        (\stk, \h, \opset) , \transpri \ \toL \  (\stk', \h', \opset \addO \op) , \pskip \rangle
    }
    \and
    \inferrule[\rl{TChoice}]{
        i \in \Set{1,2}
    }{%
        (\stk, \h, \opset) , \trans_{1} \pchoice \trans_{2} \ \toL \  (\stk, \h, \opset) , \trans_{i}
    }
    \and
    \inferrule[\rl{TIter}]{ }{%
        (\stk, \h, \opset),  \trans\prepeat \ \toL \  (\stk, \h, \opset), \pskip \pchoice (\trans \pseq \trans\prepeat)
    } 
    \and
    \inferrule[\rl{TSeqSkip}]{ }{%
        (\stk, \h, \opset), \pskip \pseq \trans \ \toL \  (\stk, \h, \opset), \trans
    }
    \and
    \inferrule[\rl{TSeq}]{%
        (\stk, \h, \opset), \trans_{1} \ \toL \  (\stk', \h', \opset'), \trans_{1}'
    }{%
        (\stk, \h, \opset), \trans_{1} \pseq \trans_{2} \ \toL \  (\stk', \h', \opset'), \trans_{1}' \pseq \trans_{2}
    }
\end{mathpar}
\hrule
\caption{Operational semantics of transactions}
\label{fig:transaction_semantics}
\end{figure}


\subsection{Operational Semantics of Programs}
\label{sec:prog-semantics}
\azalea{I rewrote most of this section.}
%\sx{
%For weak consistency models, or weak isolation levels as a common term used in database community, a thread is not necessary to work on the up-to-date version of a database in exchange for better performance. 
%Even in a single machine database, a thread running under weak consistency model can make less synchronisation with the hard drivers and other running threads, which means the thread could observe out-of-date state.
%Therefore, we introduce \emph{views} to model threads of a database.
%A \emph{view} is a cut in a history heap that corresponds the indexes of versions that a thread work with.
%We also define a order between two views, if they contain the same addresses and the indexes are ordered point-wise.
%This is to model the synchronisation between threads.
%For example \( \vi \orderVI \vi' \) could mean that a thread updates it view from \( \vi \) to \( \vi' \) by synchronisation with others.
%}

Before proceeding with the program operational semantics, we formalise the notion of \emph{execution tests}.
Execution tests are used to determine whether a transaction may commit its effects (fingerprint) to the MKVS by ensuring that its  effects comply with the underlying consistency model.

\mypar{Execution Tests}
An execution tests of a transaction is a quadruples of the form \( (\mkvs, \vi, \opset, \vi') \), where $\mkvs$ denotes the MKVS;
the $\vi$ denotes the \emph{initial} view, recorded at the beginning of the transactions; 
the $\opset$ denotes the fingerprint of the transaction; and 
$\vi'$ demotes the \emph{final} view of the transaction, obtained after committing the transaction. 
An execution test $(\mkvs, \vi, \opset, \vi')$ states that when the MKVS is described by $\mkvs$, a client with view $\vi$ is allowed to execute a single transaction with fingerprints $\opset$, commit the transaction and obtain an updated view $\vi'$. 
%
%
\azalea{What do we mean by at least $\vi'$? \sx{ meaning the lower bound of the view. The semantics  model this by shift the view at the beginning, so I think we should say update to the exact view.} }
%
%
\azalea{Why do we call these execution tests and not the consistency model? \sx{Andrea thinks it is more precise than consistency model, if I remember.}}
\begin{definition}[Execution Tests]
\label{def:consistency-models}
\label{def:executiontests}
Given the set of $\HisHeaps$ (\cref{def:his_heap}), fingerprints $\Opsets$ (\cref{def:ops}), and views $\Views$ (\cref{def:views}), the set of \emph{execution tests}, \( \como \in \Como \), is:
\[
        \ETS \eqdef  
		\setcomp{
			(\mkvs, \vi, \opset, \vi') \in \HisHeaps \times \Views \times \Opsets \times \Views
		} 
		{
		\opset\projection{2} \subseteq \dom(\vi) = \dom(\vi') = \dom(\hh)
		}       
\]
\end{definition}
%
We often write $(\mkvs, \vi) \etto \opset : \vi'$ for  $(\mkvs, \vi, \opset, \vi') \in \et$.
\sx{
    Maybe also some version of composition requirement.
    For example, the composition of two should be also included in the consistency model.
    \[
        \fora{m,m'} m \in \como \land m' \in \como \implies m \compose{} m' \in \como
    \]
    where \( \compose{} \defeq (\composeHH, \composeVI,\composeO, \composeVI)\).
}


%\begin{definition}[Execution tests]
%\label{def:consistency-models}
%\label{def:executiontests}
%Given the set of key-value stores \( \hh \in \HisHeaps \) (\cref{def:his_heap}), fingerprints \( \opset \in \Opsets \) (\cref{def:ops}) and views \( \vi, \vi' \in \Views \) (\cref{def:views}), \emph{execution tests} \( \como \in \Como \) is a set of quadruples in the form of \( ( \hh, \vi, \opset, \vi' ) \):
%\[
%    \begin{rclarray}
%        \et \in \ETS & \defeq & \powerset{\HisHeaps \times \Views \times \Opsets \times \Views}
%    \end{rclarray}
%\]
%Well-formed  execution tests \( \et \) require the domain of views and the key-value store are the same, and the fingerprints only have keys included in the previous domain:
%\[
%    \begin{rclarray}
%         &&\fora{\hh, \vi, \vi', \opset } (\hh, \vi, \opset, \vi') \in \como \implies \opset\projection{2} \subseteq \dom(\vi) = \dom(\vi') = \dom(\hh)
%    \end{rclarray}
%\]
%\sx{
%    Maybe also some version of composition requirement.
%    For example, the composition of two should be also included in the consistency model.
%    \[
%        \fora{m,m'} m \in \como \land m' \in \como \implies m \compose{} m' \in \como
%    \]
%    where \( \compose{} \defeq (\composeHH, \composeVI,\composeO, \composeVI)\).
%}
%\end{definition}
%
%

%Execution tests is a set of quadruples \( (\mkvs, \vi, \opset, \vi') \) consisting of a key-value store, a view before execution, a operation set and a view after committing of the operation set. 
%We often write \( (\mkvs, \vi) \etto \opset : \vi'\) in lieu of \( (\mkvs, \vi, \opset, \vi') \in \et\).
%The quadruple describes that when the state of the key-value store is \( \hh \), a client who has view \( \vi \) is allowed to execute a single transaction that has  the fingerprints \( \opset \), and then after the commit the thread view must be updated to at least \( \vi' \).

\ac{
There we also note that by tweaking the execution test used by the 
semantics, we capture different consistency models of 
key-value stores.
}


\begin{figure}[!t]
\sx{drop \( \func{updateView}{\hh', \vi'', \opset} \orderVI \vi'\) and use \( \vi'' \leq \vi' \) which corresponds to monotonic read. }
\hrule
%
\[
\begin{rclarray}
	\toT{}  & : &
    \ClientID \; \times \;
	\left( ( \HisHeaps \times \Views \times \Stacks ) \times \Commands \right) 
	\; \times\; \Como \;\times \;
	\left( ( \HisHeaps \times \Views \times \Stacks ) \times \Commands \right) 
	\vspace{5pt}
\end{rclarray}
\]
\begin{mathpar}
    \inferrule[\rl{PCommit}]{%
        \vi \orderVI  \vi''
        \\ \h = \clpsHH{\hh,\vi''}
        \\ (\stk, \h, \unitO), \trans \ \toL^{*} \  (\stk', \stub,  \opset) , \pskip
        \\ \txid \in \func{nextTxid}{\hh,\cl}  
        \\\\ \hh' = \func{updateMKVS}{\hh, \vi'', \txid, \opset}  
        \\ \func{updateView}{\hh', \vi'', \opset} \orderVI \vi'
        \\ (\hh, \vi) \csat \opset : \vi'
    }{%
        \cl \vdash ( \hh, \vi, \stk ), \ptrans{\trans} \ \toT{\como} \ ( \hh', \vi', \stk' ) , \pskip
    }
    \and
    \inferrule[\rl{PAssign}]{
        \val = \evalE{\expr}
    }{%
        \cl \vdash ( \hh, \vi, \stk ) , \passign{\var}{\expr} \ \toT{\como} \  ( \hh, \vi, \stk\rmto{\var}{\val} ) , \pskip
    }
    \and
    \inferrule[\rl{PAssume}]{%
        \evalE[\thstk]{\expr} \neq 0
    }{%
        \cl \vdash ( \hh, \vi, \stk ) , \passume{\expr} \ \toT{\como} \  ( \hh, \vi, \stk ) , \pskip
    }
    \and
    \inferrule[\rl{PChoice}]{
        i \in \Set{1,2}
    }{%
        \cl \vdash ( \hh, \vi, \stk ) , \cmd_{1} \pchoice \cmd_{2} \ \toT{\como} \  ( \hh, \vi, \stk ) , \cmd_{i}
    }
    \and
    \inferrule[\rl{PIter}]{ }{%
        \cl \vdash ( \hh, \vi, \stk ) , \cmd\prepeat \ \toT{\como} \  ( \hh, \vi, \stk ) , \pskip \pchoice (\cmd \pseq \cmd\prepeat)
    }
    \and
    \inferrule[\rl{PSeqSkip}]{ }{%
        \cl \vdash ( \hh, \vi, \stk ) , \pskip \pseq \cmd \ \toT{\como} \  ( \hh, \vi, \stk ) , \cmd
    }
    \and
    \inferrule[\rl{PSeq}]{% 
        \cl \vdash ( \hh, \vi, \stk ) , \cmd_{1} \ \toT{\como} \  ( \hh, \vi', \stk' ) , {\cmd_{1}}' 
    }{%
        \cl \vdash ( \hh, \vi, \stk ) , \cmd_{1} \pseq \cmd_{2} \ \toT{\como} \ ( \hh, \vi', \stk' ) , {\cmd_{1}}' \pseq \cmd_{2}
    }\vspace{5pt}
\end{mathpar}
\begin{flushleft} 
with
\quad
$
\func{nextTxid}{\hh,\cl}  \eqdef
\Setcon{ \txid^{\cl} }{ 
	\txid^{\cl} \in \TxID \land \fora{\addr, i, \txid} \txid^{\cl} > \WTx(\hh(\addr,i)) 
	\land \txid^{\cl} > \txid \land \txid \in \RTx(\hhR(\addr,i))
} 
$, and
\vspace{5pt}
 \end{flushleft}
%
\[
\begin{rclarray}         
%	 \func{updateMKVS}{., ., ., .} & : & \MKVSs \times \Views \times  \\                        
    \func{updateMKVS}{\hh, \vi, \txid, \unitO} & \defeq & \hh \\
    \func{updateMKVS}{\hh, \vi, \txid, \opset \uplus \Set{(\otR, \ke, \stub)}} & \defeq &  
    \begin{array}[t]{@{}l}
        \texttt{let } (\nat, \txid', \txidset) = \hh(\ke, \vi(\ke)) \\
        \texttt{and } \hh' = \hh\rmto{\ke}{%
            \hh(\ke)\rmto{\vi(\addr)}{%
                (\nat, \txid', \txidset \uplus \Set{\txid}) } } \\
        \texttt{ in } \func{updateMKVS}{\hh', \vi, \txid, \opset}
    \end{array} \\
    \func{updateMKVS}{\hh, \vi, \txid, \opset \uplus \Set{(\otW, \ke, \nat)}} & \defeq &  
    \begin{array}[t]{@{}l}
        \texttt{let } \hh' = \hh\rmto{\ke}{ ( \hh(\ke) \lcat \List{(\nat, \txid, \emptyset)} ) } \\
        \texttt{ in } \func{updateMKVS}{\hh', \vi, \txid, \opset}
    \end{array} 
%
\end{rclarray}
\]
\begin{flushleft} and \end{flushleft}
%
\[
\begin{rclarray}
    \func{updateView}{\hh, \vi, \unitO} & \defeq & \vi \\
    \func{updateView}{\hh, \vi, \opset \uplus \Set{(\otR, \ke, \stub)}} & \defeq & \func{updateView}{\hh, \vi, \opset}\\
    \func{updateView}{\hh, \vi, \opset \uplus \Set{(\otW, \ke, \stub)}} & \defeq & \func{updateView}{\hh, \vi\rmto{\addr}{(\left| \hh(\addr) \right| - 1)}, \opset}\\
%
%%              
%	\func{fresh}{\hh}  & \defeq & 
%	\Setcon{ \txid }{ 
%		\txid \in \TxID \land \fora{\addr, i} \txid \neq \WTx(\hh(\addr,i)) \\
%		\land\ \txid \notin \RTx(\hhR(\addr,i))
%	} 
\end{rclarray}
\]
\vspace{5pt}
\hrule%\vspace{5pt}
%\begin{flushleft}
%The thread environment is a partial function from thread identifiers to pairs of stacks and views \( \thdenv \in \ThdEnv \defeq \ThreadID \parfinfun \Stacks \times \Views \).
%Given the set of execution tests \( \ConsisModels \) (\cref{def:consistency-models}) and key-value stores \(\HisHeaps\) (\cref{def:mkvs}), the \emph{semantics for programs}:
%\end{flushleft}
\[
	\toG{} : 
    ( \Confs \times \ThdEnv \times \Programs) 
    \;\times\; \Como \;\times\;
    ( \Confs \times \ThdEnv \times \Programs) 
\]
\begin{mathpar}
    \inferrule[\rl{PSingleThread}]{%
        \cl \vdash ( \mkvs, \viewFun(\thid), \thdenv(\thid) ), \prog(\thid), \ \toT{\como} \  ( \mkvs', \vi', \stk' ) , \cmd'  
    }{%
        ( (\mkvs, \viewFun), \thdenv, \prog ) \ \toG{\como} \  ( \mkvs', \viewFun\rmto{\thid}{\vi'}) \thdenv\rmto{\thid}{\stk'} , \prog\rmto{\thid}{\cmd'} ) 
    }
\end{mathpar}
%
\hrule
\caption{Operational semantics of commands (above) and programs (below)}
\label{def:thread_semantics}
\label{fig:thread_semantics}
\label{def:thread_pool_semantics}
\label{fig:thread_pool_semantics}
\label{def:program_semantics}
\label{fig:program_semantics}
\end{figure}


\mypar{Operational Semantics of Commands}
The \emph{operational semantics of commands} is given at the top of \cref{fig:thread_semantics}. 
Command transitions are of the form $(\mkvs, \vi, \stk), \cmd \ \toT{\et} \ (\mkvs', \vi', \stk') , \cmd'$, stating that given the MKVS $\mkvs$, view $\vi$ and stack $\stk$, executing command $\cmd$ for one step under $\como$, updates the MKVS to $\mkvs'$, the stack to $\stk'$, and the command to its continuation $\cmd'$. 
Note that $\ET$ denotes the set of execution tests, prescribing the permissible executions of transactions. 
We keep the command operational semantics in \cref{fig:thread_semantics} parametric in the choice of execution tests. 
Later in \cref{sec:cmexamples} we present several examples of execution tests of well-known consistency models in the literature. 

With the exception of the \rl{PCommit} transition, the remaining command transitions are standard and behave as expected. We write 
In \cref{fig:thread_semantics} we write $\lcat$ for denotes list concatenation.
We write $f \rmto{a}{b}$ for function update: $f \rmto{a}{b}(a) = b$, and for all $c \ne a$, $f \rmto{a}{b}(c) = f(c)$.

%The notation \( l\rmto{i}{k} \) on a list \( l \) means the result by replacing the \emph{i-th} element to \( k \).
%%To record the index of list starts from 0.
%The \( \lcat \) denotes list concatenation.

The premise of \rl{PCommit} state that the current view $\vi$ of the executing command maybe advanced to a newer view $\vi''$ (see \cref{def:views}). 
Given the new view $\vi''$, the transaction proceeds by obtaining a snapshot $\sn$ of the MKVS $\mkvs$, and executing $\trans$ locally to completion ($\pskip$), updating the stack to $\stack'$, while accumulating the fingerprint $\opset$. Note that the resulting snapshot is ignored (denoted by $\stub$) as the effect of the transaction is recorded in the fingerprint $\opset$. 
%

The transaction is now ready to commit and may propagate its changes to $\mkvs$.
To this end, a \emph{fresh} transaction identifier $\txid$ is picked (\ie one that does not appear in $\mkvs$ as defined in \cref{fig:thread_semantics}) to identify the completed transaction, and the changes performed by $\txid$ are propagated to $\mkvs$. 
This is done via the $\func{updateMKVS}{\mkvs, \vi'', \txid, \opset}$ function (defined in \cref{fig:thread_semantics}) to update $\mkvs$ to  $\mkvs'$. 
As expected, for every read operation $(\otR, \ke, -)$ in fingerprint $\opset$, the readers of $\ke$ at index $\vi(\ke)$ are extended with $\txid$.
For every write operation $(\otW, \ke, \val)$ in fingerprint $\opset$, the $\mkvs(\ke)$ entry is extended with a new version $(\val, \txid, \emptyset)$, denoting that $\txid$ is responsible for creating this version which has no readers as of yet. 

Once the MKVS is updated to $\mkvs'$, the client subsequently updates its view $\vi''$ with respect to its fingerprint via the $\func{updateView}{\mkvs', \vi'', \opset}$ function, defined in \cref{fig:thread_semantics}.
Whilst the client does not need to update its view for those keys it has read from, 
%for each key $\ke$ that the client has written to, 
it must update its view to the \emph{latest} version available for those keys it has written to. %$\ke$. 
This definition of $\funcFont{updateView}$ imposes a lower bound on the updated view by ensuring that the view of the client is up-to-date for all those keys it has written to. 
This in turn guarantees a strong program order, meaning that the following transactions of the same client will at least read the previous writes of the client  itself.
Assuming that client commands are wrapped within a single session, this lower bound of the view corresponds to the strong session guarantees introduced by \cite{.........}.
%\azalea{I have rewritten this whole section. However, I don't quite understand the motivation behind $\func{updateView}{\mkvs', \vi'', \opset}$. Please elaborate.}

The updated view of the client may then be advanced to a newer view $\vi'$.
Lastly, to ensure that the effect of the transaction (its fingerprint  $\opset$) is permitted by the underlying consistency model, 
the final premise of the transition requires that the updates be permitted by the execution test $\ET$, \ie \( (\mkvs, \vi'') \etto \opset : \vi'\).





%First, the view can shift to later versions before executing the transaction to model the client might gain more information about the key-value store since its last commit.
%To recall The order between two views with the same domain, for example \( \vi'' \geq \vi \) in \rl{PCommit}, is defined by the order of the indexes (\cref{def:views}).
%This new local view \( \vi'' \) should also be consistent with the key-value stores, \ie it leads to a situation where the current client is allowed to execute the transaction.
%The transaction code \( \trans \) is executed locally given an initial snapshot \( \sn = \clpsHH{\hh, \vi''}\) (\cref{def:snapshot}) decided by the current state of key-value store \( \mkvs \) and the local view \( \vi'' \).
%The \( \funcn{localHeap} \) function uniquely determined a (local) heap from a history heap and a view by picking the versions of addresses indexed by the view.
%After local execution via the semantics for transactions (\cref{fig:transaction_semantics}), we propagate the stack \( \stk' \) and more importantly obtain the fingerprints \( \opset \), while the snapshot \( \sn' \) will be throw away.
%Then the transaction picks a fresh identifier \( \txid \), \ie one that does not appear in the key-value store, and commits the fingerprint \( \opset \), which will update the key-value store and local view.
%%The operation set includes the first read and last write of each key, which are the operations might affect the key-value store, because of the atomicity of transactions.
%The \funcn{updateMKVS} function updates the history heap using the fingerprint, \( \mkvs' = \func{updateMKVS}{\mkvs,\vi'',\txid, \opset}\).
%For read operations, it includes the new identifier to the read set of the version the is pointed by the local view \( \vi''\).
%For write operations of the form \( (\otW, \ke, \nat) \), it extends a new version written by the new transaction \( (\nat, \txid, \emptyset) \) to the tail of \( \mkvs(\ke) \).
%For updating the view, we set a lower bound for the new local view by \funcn{updateView} function.
%Assuming the commands executed by clients are wrapped with in a single session, the lower bound of the view corresponds to the strong session guarantees introduced by \cite{.........}.
%This function shifts the view to the up-to-date version in the new key-value store if the version is installed by the current transaction.
%This guarantees strong program order, meaning the following transaction will at least read its own write.
%Finally, the actual new local view \( \vi' \) is any view greater than the lower bound \( \vi' \geq \func{updateView}{\mkvs', \vi'', \opset}\).
%The overall execution satisfied the execution tests, \ie \( (\mkvs, \vi'') \etto \opset : \vi'\).

\ac{The paragraph below should probably go when discussing the rules of the semantics:

Note that the way in which MKVSs and views are updated ensure the following: 
$\bullet$ a client always reads its own preceding writes; 
$\bullet$ clients always read from an increasingly up-to-date state of the database; 
$\bullet$ the order in which clients update a key $\key{k}$ is consistent with the 
order of the versions for such keys in the MKVS; 
$\bullet$ writes take place after reads on which they depend. 
}



\mypar{Program Operational Semantics}
The \emph{operational semantics of programs} are given at the bottom of \cref{fig:program_semantics}. 
Programs transitions are of the form $(\conf,  \thdenv, \prog) \ \toG{\como} (\conf',  \thdenv', \prog')$,
stating that given the configuration $\mkvs$ and the \emph{client environment} $\thdenv$, executing program $\prog$ for one step under $\como$, updates the configuration to $\conf'$, the client environment to $\thdenv'$, and the program to its continuation $\prog'$. 
A \emph{client environment}, $\thdenv \in \ThdEnv$, tracks the local variable stack ($\stack$). 
That is, a client environment is a mapping from client identifiers to pairs of stacks and views. 
We assume that the client identifiers in the domain of client environments are are those in the domain of the program throughout the execution. 
%$\prog$: $\dom(\thdenv) = \dom(\prog)$; and that 
Program transitions are simply defined in terms of the transitions of their constituent client commands, defined by $\toT{\como}$. 
This in turn yields the standard interleaving semantics for concurrent programs. 

%Last, the program has standard interleaving semantics by picking a client and then progressing one step (\cref{fig:thread_pool_semantics}).
%To achieve that a thread environment holds the stacks and views associated with all active clients \( Env \in \sort{ThdEnv} \).
%We assume the client identifiers from client environment match with those in the program \( \prog \).
%We also assume all the stacks and views initially are the same respectively.



\begin{lemma}
\label{lem:hhupdate.welldefined}
%The function $\HHupdate$ is well-defined over well-formed MKVS, fingerprints and views. 
Given a well-formed MKVS $\hh$, a view $\vi$ that is well-formed with respect to \( \mkvs \), a fingerprint \( \opset \), and a $\txid$ that does not appear in $\hh$, then $\HHupdate(\hh, \opset, \tsid, \vi)$ is uniquely determined and yields a well-formed MKVS.
\end{lemma}

\begin{lemma}
%The function $\Vupdate$ is well-defined.
Given a well-formed MKVS $\hh$, a view $\vi$ that is well-formed with respect to \( \mkvs \), and a fingerprint \( \opset \), the $\Vupdate(\hh, \opset, \vi)$ uniquely determines a view which is well-formed with respect to $\mkvs$.
\end{lemma}

%\begin{lem}[Confluence of \funcn{updateMKVS} and \funcn{updateView}]
%Given a valid operation set \( \opset \), the order of applying \( \funcn{updateMKVS} \) function (\( \funcn{updateView} \) function) to the elements from operation set does not affect the final result.
%\end{lem}
                                                                                                         

\section{Examples\label{sec:example}}

\sx{New observation:
\begin{itemize}
\item We might want to have some assertion to say the initial values of a region.
\item What is the meaning of sequential composition since some consistency model does not have strong session guarantee, maybe the answer is the ``commit'' order.
\item Explain stablisation in a roughly syntactic level.
\end{itemize}
}

\subsection{Single increment and multi-reader.}
\[
    \begin{array}{@{}l@{}}
        \boxass{\V{x} \pt \V{\nat}}{\lrid}{\intass} \\
        \C{Inc} \composeK \C{Inc} \text{ is undefined} \\
        \C{Rd} \text{ is the unit element} \\
    \end{array}
\]
\subsubsection{SER}
\[
    \begin{array}{@{}l@{}}
        \intass : 
        \begin{rclarray}[t]
        \C{Inc} & : & \exsts{\V{m, k}} \Set{(\etR, \V{x}, \V{m}), (\etW, \V{x}, \V{m} + 1)} \mat \V{x} \pt \V{k} \oassto \V{x} \pt \V{k} \\
        \C{Rd}  & : & \exsts{\V{m, k, v}} \Set{(\etR, \V{x}, \V{m})} \mat \V{x} \pt \V{k} \oassto \V{x} \pt \V{k} \\ 
        \end{rclarray} \\
    \end{array}
\]

\[
\begin{session}
\specline{\boxass{\vx \pt 0}{\lrid}{\intass} \sep \cass{\C{Inc}}{\lrid} } \\
\specline{\boxass{\vx \pt 0}{\lrid}{\intass} \sep \cass{\C{Inc}}{\lrid} \sep \cass{\C{Rd}}{\lrid} \sep \cass{\C{Rd}}{\lrid} } \\
\begin{parl}
    \begin{session}
    \specline{\boxass{\vx \pt 0}{\lrid}{\intass} \sep \cass{\C{Inc}}{\lrid} \sep \cass{\C{Rd}}{\lrid} } \\
    \begin{transaction}
        \specline{ \vx \pt 0 } \\
        \pderef{\pvar{a}}{\vx} ; \\
        \specline{ \vx \pt 0 \land \pvar{a} = 0 \sep \Set{(\etR, \vx, 0)} } \\
        \pmutate{\vx}{\pvar{a} + 1} ; \\
        \specline{ \vx \pt 1 \land \pvar{a} = 0 \\
                {} \sep \Set{(\etR, \vx, 0), (\etW, \vx, 1)} } \\
    \end{transaction} \\
    \specline{\boxass{\vx \pt 1}{\lrid}{\intass} \sep \cass{\C{Inc}}{\lrid} \sep \cass{\C{Rd}}{\lrid} } \\
    \begin{transaction}
        \specline{ \vx \pt 1 } \\
        \pderef{\pvar{b}}{\vx} ; \\
        \specline{ \lor \vx \pt 1 \sep \Set{(\etR, \vx, 1)} } \\
    \end{transaction} \\
    \specline{\boxass{\vx \pt 1}{\lrid}{\intass} \sep \cass{\C{Inc}}{\lrid} \sep \cass{\C{Rd}}{\lrid} } \\
    \begin{transaction}
        \specline{ \vx \pt 1 } \\
        \pderef{\pvar{a}}{\vx} ; \\
        \specline{ \vx \pt 1 \land \pvar{a} = 1 \sep \Set{(\etR, \vx, 1)} } \\
        \pmutate{\vx}{\pvar{a} + 1} ; \\
        \specline{ \vx \pt 2 \land \pvar{a} = 1 \\
                {} \sep \Set{(\etR, \vx, 1), (\etW, \vx, 2)} } \\
    \end{transaction} \\
    \specline{\boxass{\vx \pt 2}{\lrid}{\intass} \sep \cass{\C{Inc}}{\lrid} \sep \cass{\C{Rd}}{\lrid} } \\
    \end{session}
    &
    \begin{session}
    \specline{\exsts{\V{n}}\boxass{\vx \pt \V{n}}{\lrid}{\intass} \land \V{n} \geq 0 \sep \cass{\C{Rd}}{\lrid} } \\
    \begin{transaction}
        \specline{ \exsts{ \V{v} } \vx \pt \V{v} { \color{gray} \land \V{v} = \V{n} } } \\
        \pderef{\pvar{c}}{\vx} ; \\
        \specline{ \exsts{ \V{v} } \vx \pt \V{v} \sep \Set{(\etR, \vx, \V{v})} { \color{gray} \land \V{v} = \V{n} } } \\
    \end{transaction} \\
    \specline{\exsts{\V{n}}\boxass{\vx \pt \V{n}}{\lrid}{\intass} \land \V{n} \geq 0 \sep \cass{\C{Rd}}{\lrid} } \\
    \end{session}
\end{parl} \\
\specline{\boxass{\vx \pt 2}{\lrid}{\intass} \sep \cass{\C{Inc}}{\lrid} \sep \cass{\C{Rd}}{\lrid} \sep \cass{\C{Rd}}{\lrid} } \\
\specline{\boxass{\vx \pt 2}{\lrid}{\intass} \sep \cass{\C{Inc}}{\lrid} } \\
\end{session}
\]
\subsubsection{SI/PSI}
\[
    \begin{array}{@{}l@{}}
        \intass : 
        \begin{rclarray}[t]
        \C{Inc} & : & \exsts{\V{m, k}} \Set{(\etR, \V{x}, \V{m}), (\etW, \V{x}, \V{m} + 1)} \mat \V{x} \pt \V{k} \oassto \V{x} \pt \V{k} \\
        \C{Rd}  & : & \exsts{\V{m, k, v}} \Set{(\etR, \V{x}, \V{m})} \mat \V{x} \pt \V{k} \oassto \V{x} \pt \V{v} \land \V{v} \leq \V{k} \\ 
        \end{rclarray} \\
        \C{Inc} \composeK \C{Inc} \text{ is undefined} \\
        \C{Rd} \text{ is the unit element} \\
    \end{array}
\]

\[
\begin{session}
\specline{\boxass{\vx \pt 0}{\lrid}{\intass} \sep \cass{\C{Inc}}{\lrid} } \\
\specline{\boxass{\vx \pt 0}{\lrid}{\intass} \sep \cass{\C{Inc}}{\lrid} \sep \cass{\C{Rd}}{\lrid} \sep \cass{\C{Rd}}{\lrid} } \\
\begin{parl}
    \begin{session}
    \specline{\boxass{\vx \pt 0}{\lrid}{\intass} \sep \cass{\C{Inc}}{\lrid} \sep \cass{\C{Rd}}{\lrid} } \\
    \begin{transaction}
        \specline{ \vx \pt 0 } \\
        \pderef{\pvar{a}}{\vx} ; \\
        \specline{ \vx \pt 0 \land \pvar{a} = 0 \sep \Set{(\etR, \vx, 0)} } \\
        \pmutate{\vx}{\pvar{a} + 1} ; \\
        \specline{ \vx \pt 1 \land \pvar{a} = 0 \\
                {} \sep \Set{(\etR, \vx, 0), (\etW, \vx, 1)} } \\
    \end{transaction} \\
    \specline{\boxass{\vx \pt 1}{\lrid}{\intass} \sep \cass{\C{Inc}}{\lrid} \sep \cass{\C{Rd}}{\lrid} } \\
    \begin{transaction}
        \specline{ {\color{purple} \vx \pt 0} \lor \vx \pt 1 } \\
        \pderef{\pvar{b}}{\vx} ; \\
        \specline{ { \color{purple} \vx \pt 0 \sep \Set{(\etR, \vx, 0)} }  \\
                    {} \lor \vx \pt 1 \sep \Set{(\etR, \vx, 1)} } \\
    \end{transaction} \\
    \specline{\boxass{\vx \pt 1}{\lrid}{\intass} \sep \cass{\C{Inc}}{\lrid} \sep \cass{\C{Rd}}{\lrid} } \\
    \begin{transaction}
        \specline{ \vx \pt 1 } \\
        \pderef{\pvar{a}}{\vx} ; \\
        \specline{ \vx \pt 1 \land \pvar{a} = 1 \sep \Set{(\etR, \vx, 1)} } \\
        \pmutate{\vx}{\pvar{a} + 1} ; \\
        \specline{ \vx \pt 2 \land \pvar{a} = 1 \\
                {} \sep \Set{(\etR, \vx, 1), (\etW, \vx, 2)} } \\
    \end{transaction} \\
    \specline{\boxass{\vx \pt 2}{\lrid}{\intass} \sep \cass{\C{Inc}}{\lrid} \sep \cass{\C{Rd}}{\lrid} } \\
    \end{session}
    &
    \begin{session}
    \specline{\exsts{\V{n}}\boxass{\vx \pt \V{n}}{\lrid}{\intass} \land \V{n} \geq 0 \sep \cass{\C{Rd}}{\lrid} } \\
    \begin{transaction}
        \specline{ \exsts{ \V{v} } \vx \pt \V{v} { \color{gray} \land \V{v} \leq \V{n} } } \\
        \pderef{\pvar{c}}{\vx} ; \\
        \specline{ \exsts{ \V{v} } \vx \pt \V{v} \sep \Set{(\etR, \vx, \V{v})} { \color{gray} \land \V{v} \leq \V{n} } } \\
    \end{transaction} \\
    \specline{\exsts{\V{n}}\boxass{\vx \pt \V{n}}{\lrid}{\intass} \land \V{n} \geq 0 \sep \cass{\C{Rd}}{\lrid} } \\
    \end{session}
\end{parl} \\
\specline{\boxass{\vx \pt 1}{\lrid}{\intass} \sep \cass{\C{Inc}}{\lrid} \sep \cass{\C{Rd}}{\lrid} \sep \cass{\C{Rd}}{\lrid} } \\
\specline{\boxass{\vx \pt 1}{\lrid}{\intass} \sep \cass{\C{Inc}}{\lrid} } \\
\end{session}
\]

\subsubsection{Causal}

\[
    \begin{array}{@{}l@{}}
        \intass : 
        \begin{rclarray}[t]
        \C{Inc} & : & \exsts{\V{m, k}} \Set{(\etR, \V{x}, \V{m}), (\etW, \V{x}, \V{m} + 1)} \mat \V{x} \pt \V{k} \oassto \V{x} \pt \V{v} \\
        \C{Rd}  & : & \exsts{\V{m, k, v}} \Set{(\etR, \V{x}, \V{m})} \mat \V{x} \pt \V{k} \oassto \V{x} \pt \V{v} \\ 
        \end{rclarray} \\
        \C{Inc} \composeK \C{Inc} \text{ is undefined} \\
        \C{Rd} \text{ is the unit element} \\
    \end{array}
\]

\[
\begin{session}
\specline{\boxass{\vx \pt 0}{\lrid}{\intass} \sep \cass{\C{Inc}}{\lrid} } \\
\specline{\boxass{\vx \pt 0}{\lrid}{\intass} \sep \cass{\C{Inc}}{\lrid} \sep \cass{\C{Rd}}{\lrid} \sep \cass{\C{Rd}}{\lrid} } \\
\begin{parl}
    \begin{session}
    \specline{\boxass{\vx \pt 0}{\lrid}{\intass} \sep \cass{\C{Inc}}{\lrid} \sep \cass{\C{Rd}}{\lrid} } \\
    \begin{transaction}
        \specline{ \exsts{ \V{v} } \vx \pt \V{v} } \\
        \pderef{\pvar{a}}{\vx} ; \\
        \specline{ \exsts{ \V{v} } \vx \pt \V{v} \land \pvar{a} = \V{v} \sep \Set{(\etR, \vx, \V{v})} } \\
        \pmutate{\vx}{\pvar{a} + 1} ; \\
        \specline{ \vx \pt \V{v} + 1 \land \pvar{a} = \V{v} \\
                {} \sep \Set{(\etR, \vx, \V{v}), (\etW, \vx, \V{v} + 1)} } \\
    \end{transaction} \\
    \specline{\exsts{\V{n}}\boxass{\vx \pt \V{n}}{\lrid}{\intass} \sep \cass{\C{Inc}}{\lrid} \sep \cass{\C{Rd}}{\lrid} } \\
    \begin{transaction}
        \specline{ \exsts{ \V{v} } \vx \pt \V{v} } \\
        \pderef{\pvar{c}}{\vx} ; \\
        \specline{ \exsts{ \V{v} } \vx \pt \V{v} \sep \Set{(\etR, \vx, \V{v})} } \\
    \end{transaction} \\
    \specline{\exsts{\V{n}}\boxass{\vx \pt \V{n}}{\lrid}{\intass} \sep \cass{\C{Inc}}{\lrid} \sep \cass{\C{Rd}}{\lrid} } \\
    \begin{transaction}
        \specline{ \exsts{ \V{v} } \vx \pt \V{v} } \\
        \pderef{\pvar{a}}{\vx} ; \\
        \specline{ \exsts{ \V{v} } \vx \pt \V{v} \land \pvar{a} = \V{v} \sep \Set{(\etR, \vx, \V{v})} } \\
        \pmutate{\vx}{\pvar{a} + 1} ; \\
        \specline{ \vx \pt \V{v} + 1 \land \pvar{a} = \V{v} \\
                {} \sep \Set{(\etR, \vx, \V{v}), (\etW, \vx, \V{v} + 1)} } \\
    \end{transaction} \\
    \specline{\exsts{\V{n}}\boxass{\vx \pt \V{n}}{\lrid}{\intass} \sep \cass{\C{Inc}}{\lrid} \sep \cass{\C{Rd}}{\lrid} } \\
    \end{session}
    &
    \begin{session}
    \specline{\exsts{\V{n}}\boxass{\vx \pt \V{n}}{\lrid}{\intass} \sep \cass{\C{Rd}}{\lrid} } \\
    \begin{transaction}
        \specline{ \exsts{ \V{v} } \vx \pt \V{v} { \color{gray} \land \V{v}, \V{n} } } \\
        \pderef{\pvar{c}}{\vx} ; \\
        \specline{ \exsts{ \V{v} } \vx \pt \V{v} \sep \Set{(\etR, \vx, \V{v})} { \color{gray} \land \V{v}, \V{n} } } \\
    \end{transaction} \\
    \specline{\exsts{\V{n}}\boxass{\vx \pt \V{n}}{\lrid}{\intass} \sep \cass{\C{Rd}}{\lrid} } \\
    \end{session}
\end{parl} \\
\specline{\exsts{\V{n}}\boxass{\vx \pt \V{n}}{\lrid}{\intass} \sep \cass{\C{Inc}}{\lrid} \sep \cass{\C{Rd}}{\lrid} \sep \cass{\C{Rd}}{\lrid} } \\
\specline{\exsts{\V{n}}\boxass{\vx \pt \V{n}}{\lrid}{\intass} \sep \cass{\C{Inc}}{\lrid} } \\
\end{session}
\]

\subsection{Two associated bank accounts}
\[
    \begin{array}{@{}l@{}}
        \boxass{\V{x} \pt \V{n} \sep \V{y} \pt \V{m} }{\lrid}{\intass} \\
        \C{xx} \composeK \C{xx} \text{ is undefined} \\
        \C{yy} \composeK \C{yy} \text{ is undefined} \\
        \C{Rd} \text{ is the unit} \\
    \end{array}          
\]

\subsubsection{SER}
\[
    \begin{array}{@{}l@{}}
        \intass : 
        \begin{rclarray}[t]
        \C{xx} & : & \exsts{\V{v, k, a, b }} \Setcon{(\etR, \V{x}, \V{v}), (\etR, \V{y}, \V{k}), (\etW, \V{x}, \V{v} - 100)}{\V{v} + \V{k} \geq 100} \\
        & & \qqquad \mat \V{x} \pt \V{a} \sep \V{y} \pt \V{b} \oassto \V{x} \pt \V{a} \sep \V{y} \pt \V{b} \land \V{a} + \V{b} \geq 0 \\
        \C{yy} & : & \exsts{\V{v, k, a, b }} \Setcon{(\etR, \V{x}, \V{v}), (\etR, \V{y}, \V{k}), (\etW, \V{y}, \V{k} - 100)}{\V{v} + \V{k} \geq 100} \\
        & & \qqquad \mat \V{x} \pt \V{a} \sep \V{y} \pt \V{b} \oassto \V{x} \pt \V{a} \sep \V{y} \pt \V{b} \land \V{a} + \V{b} \geq 0 \\
        \C{Rd} & : & \exsts{\V{v, k, a, b }} \Set{(\etR, \V{x}, \V{v}), (\etR, \V{y}, \V{k})} \\
        & & \qqquad \mat \V{x} \pt \V{a} \sep \V{y} \pt \V{b} \oassto \V{x} \pt \V{a} \sep \V{y} \pt \V{b} \land \V{a} + \V{b} \geq 0 \\
        \end{rclarray} \\
    \end{array}
\]

\[
\begin{session}
\specline{ \boxass{ \vx \pt 60 \sep \vy \pt 60 }{\lrid}{\intass} \sep \cass{\C{xx}}{\lrid} \sep \cass{\C{yy}}{\lrid} } \\
\begin{parl}
    \begin{session}
        \specline{ \boxass{ \vx \pt 60 \sep \vy \pt 60 \lor \vx \pt 60 \sep \vy \pt -40 }{\lrid}{\intass} \sep \cass{\C{xx}}{\lrid} } \\
        \begin{transaction}
            \specline{\vx \pt 60 \sep \vy \pt 60 \\ {} \lor \vx \pt 60 \sep \vy \pt -40} \\
            \pderef{\pvar{a}}{\vx}; \\
            \pderef{\pvar{b}}{\vy}; \\
            \pifs{\pvar{a} + \pvar{b} \geq 100} \\
            \quad \pmutate{\vx}{\pvar{a} - 100} ; \\
            \pife \\
            \specline{\vx \pt-40 \sep \vy \pt 60 \sep {} \\
            \Set{(\etR, \vx, 60), (\etR, \vy, 60), (\etW, \vx, -40)} \\ 
            {} \lor \vx \pt 60 \sep \vy \pt -40 \sep {} \\
            \Set{(\etR, \vx, 60), (\etR, \vy, -40)} }
        \end{transaction} \\
        \specline{ \boxass{ \vx \pt -40 \sep \vy \pt 60 \lor \vx \pt 60 \sep \vy \pt -40 }{\lrid}{\intass} \sep \cass{\C{xx}}{\lrid} } \\
    \end{session}
    &
    \begin{session}
        \specline{ \boxass{ \vx \pt -40 \sep \vy \pt 60 \lor \vx \pt 60 \sep \vy \pt 60 }{\lrid}{\intass} \sep \cass{\C{yy}}{\lrid} } \\
        \begin{transaction}
            \pderef{\pvar{a}}{\vx}; \\
            \pderef{\pvar{b}}{\vy}; \\
            \pifs{\pvar{a} + \pvar{b} \geq 100} \\
            \quad \pmutate{\vy}{\pvar{b} - 100} ; \\
            \pife 
        \end{transaction} \\
        \specline{ \boxass{ \vx \pt -40 \sep \vy \pt 60 \lor \vx \pt 60 \sep \vy \pt -40 }{\lrid}{\intass} \sep \cass{\C{yy}}{\lrid} } \\
    \end{session}
\end{parl} \\
\specline{ \boxass{ \vx \pt -40 \sep \vy \pt 60 \lor \vx \pt 60 \sep \vy \pt -40 }{\lrid}{\intass} \sep \cass{\C{xx}}{\lrid} \sep \cass{\C{yy}}{\lrid} } \\
\end{session}
\]

\subsubsection{SI/PSI}
\[
    \begin{array}{@{}l@{}}
        \intass : 
        \begin{rclarray}[t]
        \C{xx} & : & \exsts{\V{v, k, a, b, c }} \Setcon{(\etR, \V{x}, \V{v}), (\etR, \V{y}, \V{k}), (\etW, \V{x}, \V{v} - 100)}{\V{v} + \V{k} \geq 100} \\
        & & \qqquad \mat \V{x} \pt \V{a} \sep \V{y} \pt \V{b} \oassto \V{x} \pt \V{a} \sep \V{y} \pt \V{b} + ( \V{c} \times 100 ) \\
        \C{yy} & : & \exsts{\V{v, k, a, b, c }} \Setcon{(\etR, \V{x}, \V{v}), (\etR, \V{y}, \V{k}), (\etW, \V{y}, \V{k} - 100)}{\V{v} + \V{k} \geq 100} \\
        & & \qqquad \mat \V{x} \pt \V{a} \sep \V{y} \pt \V{b} \oassto \V{x} \pt \V{a} + ( \V{c} \times 100 ) \sep \V{y} \pt \V{b} \\
        \C{Rd} & : & \exsts{\V{v, k, a, b, c, d }} \Set{(\etR, \V{x}, \V{v}), (\etR, \V{y}, \V{k})} \\
        & & \qqquad \mat \V{x} \pt \V{a} \sep \V{y} \pt \V{b} \oassto \V{x} \pt \V{a} + ( \V{c} \times 100 ) \sep \V{y} \pt \V{b} + ( \V{d} \times 100 ) \\
        \end{rclarray} \\
    \end{array}
\]

\[
\begin{session}
\specline{ \boxass{\vx \pt 60 \sep \vy \pt 60 }{\lrid}{\intass} \sep \cass{\C{xx}}{\lrid} \sep \cass{\C{yy}}{\lrid} } \\
\begin{parl}
    \begin{session}
        \specline{ \exsts{ \V{n} } \boxass{\vx \pt 60 \sep \vy \pt 60 - \V{n} \times 100 }{\lrid}{\intass} \sep \cass{\C{xx}}{\lrid} } \\
        \begin{transaction}
            \specline{ \exsts{ \V{n}, \V{k} \geq 0 } \vx \pt 60 + \V{k} \times 100 \sep \vy \pt 60 + \V{n} \times 100 } \\
            \pderef{\pvar{a}}{\vx}; \\
            \pderef{\pvar{b}}{\vy}; \\
            \specline{ \exsts{ \V{n}, \V{k} \geq 0 } \vx \pt 60 + \V{k} \times 100 \sep \vy \pt 60 + \V{n} \times 100 \\
                        {} \sep \Set{ (\etR, \vx, 60 + \V{k} \times 100), (\etR, \vx, 60 + \V{n} \times 100) } \\
                        {} \land \pvar{a} = 60 + \V{k} \times 100 \land \pvar{b} = 60 + \V{n} \times 100 } \\
            \pifs{\pvar{a} + \pvar{b} \geq 100} \\
            \quad \specline{ \exsts{ \V{n}, \V{k} \geq 0 } \vx \pt 60 + \V{k} \times 100 \sep \vy \pt 60 + \V{n} \times 100 \\
                            {} \sep \Set{ (\etR, \vx, 60 + \V{k} \times 100), (\etR, \vx, 60 + \V{n} \times 100) } \\
                            {} \land \pvar{a} = 60 + \V{k} \times 100 \land \pvar{b} = 60 + \V{n} \times 100 \land \V{k} + \V{N} \geq 0} \\
            \quad \pmutate{\vx}{\pvar{a} - 100} ; \\
            \quad \specline{ \exsts{ \V{n}, \V{k} \geq 0 } \vx \pt -40 + \V{k} \times 100 \sep \vy \pt 60 + \V{n} \times 100 \\
                            {} \sep \Set{ (\etR, \vx, 60 + \V{k} \times 100), (\etR, \vx, 60 + \V{n} \times 100), \\ 
                                                    (\etW, \vx, -40 + \V{k} \times 100) } \\
                            {} \land \pvar{a} = 60 + \V{k} \times 100 \land \pvar{b} = 60 + \V{n} \times 100 \land \V{k} + \V{N} \geq 0} \\
            \pife \\
            \comment{Weaken the assertion by } \\
            \comment{throwing away program variables.} \\
            \specline{ \exsts{ \V{n}, \V{k} \geq 0 } \vy \pt 60 + \V{n} \times 100 \\
                        {} \sep 
                        \begin{formulea}
                        \vx \pt -40 + \V{k} \times 100 
                        \land \V{k} + \V{N} \geq 0 \\
                        {} \sep \Set{ (\etR, \vx, 60 + \V{k} \times 100), (\etR, \vx, 60 + \V{n} \times 100), \\
                                                (\etW, \vx, -40 + \V{k} \times 100) } \\
                        \end{formulea} \\
                        {} \lor 
                        \begin{formulea}
                        \vx \pt 60 + \V{k} \times 100 \\ 
                        {} \sep \Set{ (\etR, \vx, 60 + \V{k} \times 100), (\etR, \vx, 60 + \V{n} \times 100) } 
                        \end{formulea}
                    } \\
        \end{transaction} \\
        \comment{To allow the write to be committed,} \\
        \comment{the \V{K} must be 0 } \\
        \specline{ \exsts{ \V{n} } \boxass{ ( \vx \pt 60 \lor \vx \pt -40 ) \sep \vy \pt 60 - \V{n} \times 100 }{\lrid}{\intass} \sep \cass{\C{xx}}{\lrid} } \\
    \end{session}
    &
    \begin{session}
        \specline{ \exsts{ \V{n} } \boxass{\vx \pt 60 - \V{n} \times 100 \sep \vy \pt 60 }{\lrid}{\intass} \sep \cass{\C{yy}}{\lrid} } \\
        \begin{transaction}
            \pderef{\pvar{a}}{\vx}; \\
            \pderef{\pvar{b}}{\vy}; \\
            \pifs{\pvar{a} + \pvar{b} \geq 100} \\
            \quad \pmutate{\vy}{\pvar{b} - 100} ; \\
            \pife 
        \end{transaction} \\
        \specline{ \exsts{ \V{n} } \boxass{\vx \pt 60 - \V{n} \times 100 \sep ( \vy \pt 60 \lor \vy \pt -40 ) }{\lrid}{\intass} \sep \cass{\C{yy}}{\lrid} } \\
    \end{session}
\end{parl} \\
\specline{ \boxass{\vx \pt -40 \sep \vy \pt 60 \lor \vx \pt 60 \sep \vy \pt -40 \lor \vx \pt -40 \sep \vy \pt -40 }{\lrid}{\intass} \sep \cass{\C{xx}}{\lrid} \sep \cass{\C{yy}}{\lrid} } \\
\end{session}
\]

%A dummy bank transfer example that is not serialisable.
%\[
    %\begin{rclarray}
        %\intass(\rid) & = &
        %\begin{cases}
            %\unitelem{} : \exsts{x, m, k} x \fpt{\fp} n \sep \cass{S(m)}{\rid} \transfersto x \fpt{\addFPW{\fp}} n \pm k \sep  \cass{S(m \pm k)}{\rid} \\
            %\unitelem{} : \exsts{x} x \fpt{\fp} n \transfersto x \fpt{\addFPR{\fp}} n 
        %\end{cases} \\
        %S(m) \composeK S(n) & = & S(m+n) \\
        %S(0) & \in & \unitK \\
    %\end{rclarray}
%\]
%A dummy bank transfer example that is serialisable even under snapshot isolation.
%\[
    %\begin{rclarray}
        %\intass & = &
        %\begin{cases}
            %\unitelem{} : \exsts{x, y, m, k} x \fpt{\fp} n \sep  y \fpt{\fp'} m \transfersto x \fpt{\addFPW{\fp}} n - k \sep \fpt{\addFPW{\fp}} m + k  \\
            %\unitelem{} : \exsts{x} x \fptEMP n \transfersto x \fptR n 
        %\end{cases}
    %\end{rclarray}
%\]
%If x write y and if y write x example.
%\[
    %\begin{rclarray}
        %\intass(x,y) & = &
        %\begin{cases}
            %\perm{L} : x \fptEMP 0 \sep y \fptEMP 0 \transfersto x \fptW 1 \sep y \fptR 0 \\
            %\perm{R} : x \fptEMP 0 \sep y \fptEMP 0 \transfersto x \fptR 0 \sep y \fptW 1 \\
        %\end{cases}
    %\end{rclarray}
%\]

