\section{Formal Model}
\label{sec:model}
\label{sec:semantics}
\ac{I am going to split this section into two, for the sake of clarity. 
As of now, the structure that I have in mind for the paper is as follows: 
Section 2 (i.e. this section) contains the notion of key-value store, view, snapshot, 
execution tests, and consistency model  -i.e. sets of key-value stores- 
induced by an execution test). Then Section 3 discusses 
the semantics of programs, and possibly we state the result 
of adequateness. In Section 4 we discuss the equivalence of consistency 
models specifications with respect to axiomatic ones. In Section 5 
we present the logic and litmus tests examples.}

\begin{wrapfigure}[7]{r}{0.33\textwidth}
\vspace{-10pt}
\begin{verbatim}
interface Transaction {
    Start(); 
    Read(Key k);
    Write(Key k, Value v); 
    Commit();    }
\end{verbatim}
\vspace{-10pt}
\caption{Example of Transaction API.}
\label{fig:api}
\end{wrapfigure}
We focus on an abstract computational model where multiple client programs can access and update keys in a key-value store using atomic transactions. 
In general, clients are provided with a simple \textit{API} such as the one depicted in \cref{fig:api} \cite{gdur,physicsnmsi,clockSI}\footnote{It is 
often the case that key-value stores provide a mechanism to wrap more transactions inside a session, and give 
provide appropriate APIs to handle sessions. For the sake of simplicity, in this paper we assume that each client executes transactions 
within a single session.}
\textbf{AC: (not using boxes because they clash with wrapfig. Technically, clockSI also has a delete operation, this should be either pointed out 
or the citation be substituted with something else.)}, while both the implementation details and the system architecture are hidden from clients. 
Because (distributed) key-value stores only give weak consistency guarantees of the data to their clients, the latter are 
not ensured to read the most up-to-date version of a key.
%In an ideal world, when executing a transaction clients would read the most up-to-date version of a key. In a distributed setting 
%this approach, known as (strict) serialisability, would require a continuous synchronisation between the different components of 
%the system, which impacts performance and limits scalability. To this end, the database only provides weak consistency model
%\ac{This sentence should probably be in the introduction.}

Following these intuitions, we model a key-value stores, or \emph{kv-store}, as a centralised unit where multiple versions 
are stored for each key (\cref{sec:mkvs-view}). Versions consist of a value and the meta-data of the transactions that wrote and 
read such a version. In practical systems, the meta-data is usually encoded using either timestamps 
\cite{physicsnmsi,clockSI} or vector clocks \cite{gdur}. We focus on key-value stores whose transactions 
enjoy \emph{atomic visibility}, meaning that \textbf{(i)} transactions read their data from an atomic 
snapshot of the key-value store, and \textbf{(ii)} a transaction can observe either none or all 
of the updates performed by another transaction. In other words, a transaction only reads (writes) at most 
one version for each key.
Because clients may observe potentially out-of-date versions of the system, we introduce the notion of \emph{views}. 
Intuitively, a view records the version of each key that a client observes at a given time. We use views 
to determine the snapshot taken by transactions executed by clients.

A consistency model is a contract between the key-value store and its clients. We distinguish 
between \emph{data-centric} consistency models \cite{framework-concur}, which impose constraints 
on the structure of the key value store, and \emph{client-centric} consistency models, 
which impose constraints on the observations and updates made by a single client 
\cite{terry1994session}. 
To specify weak consistency models, 
we introduce the notion of \emph{execution tests} (sec:execution.tests). An execution test is a predicate 
that specifies when a client is allowed to execute a transaction carrying a given 
set of read and write operations, or \emph{fingerprint}. Therefore, an execution 
test constrains how the state of the key-value store may evolve;
by considering at all the possible evolutions of the key-value store under said execution test, 
we determine a consistency model. For example, an execution test for (strict) serialisability 
requires that a transaction can be executed by a client, only if it observes the most up-to-date 
version for each key. 
We give several examples of execution tests that capture both data-centric and 
client-centric consistency models. 

The idea of specifying consistency models using execution tests has been 
already proposed in \cite{seebelieve}; however, their notion of execution 
test is intrinsecally more complex than ours: to determine 
whether a transaction can commit, the total order in which all past transactions 
have committed must be known. This knowledge is not needed in our setting.
%the author require 
%the knowledge of the total order in which all past transactions have been 
%executed, to determine whether a new transaction can commit.

%%\ac{Can't cite documentations of real databases, as they usually have a much more complicated API.}
%
%%Transactions in our model execute atomically, though they have different effects on the key-value stores depending on their associated \emph{consistency model}.
%%A consistency model controls how the key-value store evolves.
%%A common model is \emph{serialisability}, where transactions appear to execute one after another in a sequential order.
%%This notion of sequential execution is however not necessary for many weaker models. As such, upon commencing execution, a transaction may not observe the most up-to-date values for keys. 
%
%To address this, we 
%first model 
%the state of the system using \emph{multi-version key-value stores (MKVSs)} (\cref{sec:mkvs-view}). 
%An MKVS keeps track of all versions (values) written for keys, as well as the information about the transactions that read and wrote such versions. 
%To model the potential out-of-date observation, we introduce \emph{views}.
%A view decides the observable versions of keys for a client.
%Therefore, in order to execute a transaction, the client first takes \emph{a snapshot} of the system with the view, executes the transaction locally with respect to its snapshot (\cref{sec:trans-semantics}), and afterwards commits the effect of the transaction if the change is allowed by the underlying consistency model (\cref{sec:prog-semantics}).


%We first introduce MKVSs and views (\secref{sec:mkvs-view}) and.
%We starts with the syntax of programs followed by the semantics of transaction.
%Finally, we will give the semantics for the entire programs.


\subsection{Multi-version Key-value Stores and Views}
\label{sec:mkvs-view}
\azalea{I rewrote most of this section. Please make sure you are happy with this.}
%\ac{
%We focus on a computational model where multiple client programs can access and update 
%locations in a key-value store using atomic transactions. Transactions in our model execute atomically, 
%though the consistency guarantees that they provide do not necessarily correspond to \emph{serialisability}. 
%This means that, at the moment of executing, a transaction may not observe the most up-to-date value 
%of a location. 

%To overcome this issue, we model the state of the system using \emph{multi-version key-value stores} 
%(MKVSs) and \emph{views}. A 
%MKVS keeps track of all the versions written for any key, as well as the information 
%about the transactions that read and wrote such versions.  Views keep track of the version observed 
%for each key by clients. 
%}

%\sx{ Value as natural number or natural number + key index? Is partial mkvs is a problem here??}


%We often depict MKVSs graphically. 
%One example is given by the MKVS $\hh_0$ of Figure \ref{fig:hheap}(a) (ignore for the moment 
%the straight lines labelled $\txid_1$ and $\txid_2$).
\ac{\sx{Partial is better for logic}Maybe it's better to keep $\ke$ fixed and say that we look at only a 
fragment of the key value store. Alternatively, we can go for partial mappings to 
represent MKVSs, but still avoiding allocation and deallocation of keys.}



%\ac{
%We often depict MKVSs graphically. 
%One example is given by the MKVS $\hh_0$ of Figure \ref{fig:hheap}(a) (ignore for the moment 
%the straight lines labelled $\txid_1$ and $\txid_2$).

%To the left 
%we have the set of keys stored y $\hh_0$, in this case $\key{k}_1$ and 
%$\key{k}_2$. To the right, on the same line of a key, a matrix containing the
 %list of versions stored by such a key in $\hh_0$. Starting from the first column, 
 %each version is represented by two adjacent columns in the matrix: the 
 %column to the left gives the value of the version, while the column to the 
 %right contains the identifier of the transaction that wrote the version to the 
 %top, and the identifiers of the transactions that read such a version to the bottom.
%In the case of $\hh_0$, there are two versions stored for $\key{k}_1$: 
%the first one with value $0$, written by $\txid_0$ and read by $\txid_2$; 
%and the second one with value $1$, written by $\txid_2$ and read by no 
%transaction. 
%}
\azalea{I have changed the values to $v_0$ and $v_1$ (from $0$ and $1$) to help clarify the distinction between indexes and values.
I have also paraphrased, please double check.  }

\mypar{Key-value Stores} 

%We model the state of a system using \emph{multi-version key-value stores} (MKVSs). 
We assume a countably infinite set of \emph{keys} $\Keys = \{\ke, \ke', \cdots\}$, 
a set of \emph{values} $\Val = \{\val, \val', \cdots\}$, a set of clients $\Clients = \{\cl, \cl',\cdots\}$. 
We also assume a set of transaction identifiers $\TxID = \{ \txid_{\cl}^{n} \mid \cl \in \Clients \wedge n \geq 0 \} 
\uplus \{\txid_{0}\}$,
each of which is either a special transaction identifier $\txid_0$, 
or it is indexed by a client identifier and a natural number. This particular structure is chosen to 
embed the session order of transactions executed by individual clients (\cref{sec:exectest}). Elements 
of $\TxID$ are ranged over by $\txid, \txid', \cdots$, while subsets of $TxID$ are ranged over 
$\txidset, \txidset', \cdots$. Given a set $X$, then $\powerset{X}$ denotes 
the powerset of $X$, while $X^{\ast}$ is the free monoid induced by $X$.

\begin{definition}[Multi-version key-value stores]
\label{def:his_heap}
\label{def:mkvs}
%Assuming a countably infinite set of keys $\Keys = \Set{\ke_1, \cdots}$, \emph{transactions identifiers} \( \TxID \defeq \Set{\txid_{1}, \cdots}\) and natural numbers \(\nat \in \Nat \), 
A \emph{version} is a triple $\ver = (\val, \txid, \txidset)$. The set of versions is denoted by $\Versions \defeq \Val \times \TxID \times \powerset{\TxID}$, 
where $\powerset{\cdot}$ is the powerset operator. 

A \emph{key-value store} is a partial, finite mapping $\hh : \Keys \parfinfun \Versions^{\ast}$. 
%The set of key-value stores is denoted as $\HisHeaps$.
\end{definition}
Elements of $\Versions^{\ast}$ are ranged over by $\vilist, \vilist',\cdots$.
\ac{Is it safe to assume that key-value stores are partial functions? What happens if we execute a program 
that tries to access a key that is not in the key-value store? In his Thesis, Viktor had the notion of fault to distinguish cases 
where non-allocated variables were used. Check.}
Intuitively, a version $\ver = (\val, \txid, \txidset)$ consists of a value and the meta-data of the transactions 
that accessed such a version; specifically, $\txid = \WTx(\ver)$ denotes the transaction identifier corresponding to the 
writer of the version, while $\txidset = \RTx(\ver)$ contains the set of the transaction identifier that read a version. 
A key-value store is then a mapping from keys to the list of versions that have been installed by 
transactions for such a key. For a given kv-store, $\hh$, key $\ke$ and index $i \geq 0$, we use the notation $\hh(\ke, i)$ 
the $i$-th version (starting from $0$) installed for $\ke$; that is, if $\hh(\ke) = \ver_0 \cdots\ver_{n}$, then 
$\hh(\ke, i) = \ver_{i}$ if $i \leq n$, it is undefined otherwise. We also let $\lvert \hh(\ke) \rvert = n +1 $ denote 
the length of $\hh(\ke)$.

It will be often convenient to depict key-value stores graphically: an 
example is given by the kv-store $\hh_{0}$ of Figure \cref{fig:hheap-a}.
Ignoring the lines labelled $\client_1$ and $\client_2$, $\hh_{0}$ contains two keys \( \ke_1\) and \( \ke_2 \), 
each of which associated with two versions (with values $v_0$ and $v_1$ for $\ke_1$, and values $v'_0$ and $v'_1$ for $\ke_2$).
The versions of a key are listed in %chronological 
order from left to right. 
We represent each version as a three-cell box, with the left cell storing the value, the top right cell recording the writer, and the bottom right cell recording the readers. 

In this paper we only focus on kv-stores whose consistency model ensures that transactions enjoy 
\emph{atomic visibility} \cite{framework-concur}. We also assume that the special transaction identifier $\tsid_0$ is used 
to install the initial version of a key. This amounts to require from a kv-store $\hh$ that \textbf{(i)} for each 
key $\ke \in \dom(\hh)$, $\hh(\ke, 0) = (v_0, \txid_0, \_)$ for some initial value $v_0$,
\textbf{(ii)} transactions never write more than one version per key - $\forall \ke \in \dom(\hh)\; \forall i,j : 0 \leq i, j < \lvert \hh(\ke) \rvert.\; 
\WTx(\ke, i) = \WTx(\ke, j) \implies i = j$, \textbf{(iii)} transactions never read different versions 
for the same key - $\forall \ke \in \dom(\hh).\; \forall i,j : 0 \leq i, j < \lvert \hh(\ke) \rvert.\; 
\RTx(\ke, i) \cap \RTx(\ke, j) \neq \emptyset \implies i = j$, and 
\textbf{(iv)} all the reads of a transactions precede the writes of the same transaction. $\forall \ke \in \dom(\hh). 
\forall i : 0 \leq i < \lvert \hh(k) \rvert.\; \WTx(\hh, k) \notin \RTx(\hh,k)$. We also require that \textbf{(v)} the order 
in which transactions issued by the same client install different versions for some key $\ke$, is consistent with the order in which 
such transactions have been invoked by the client  - $\forall \ke \in \dom(\hh), \cl \in \Clients.\;\forall i,j; 0 \leq i < j < \lvert \hh(\ke) \rvert. 
\forall n, m \geq 0.\; (\WTx(\ke(k,i)) = \txid_{\cl}^{n} \wedge \WTx(\ke(k,j)) = \txid_{\cl}^{m}) \implies n < m$.
We call kv-stores that satisfy the conditions (i-v) \emph{well-formed}. Henceforth, we always assume that kv-stores 
are well formed, and let $\HisHeaps$ be the set of well-formed kv-stores.


%A\emph{multi-version key-value store (MKVS)}, \( \mkvs \in \MKVSs \), is a partial finite function from keys to lists of \emph{versions}.
%A \emph{version} is a triple containing a program value \( \val \), a transaction identifier \( \txid \) and a set of transactions identifiers \( \txidset \):
%\[
%\begin{rclarray}
%    \ver \in \Versions & \defeq &  \Setcon{(\val, \txid, \txidset)}{\val \in \Val \land \txid \in \TxID \land \txidset \subseteq \TxID \land \txid \notin \txidset} \\
%    \HisHeaps & \defeq & 
%    \Setcon{ \mkvs }{%
%        \mkvs \in \Keys \parfinfun \Versions^{*} 
%        \land \fora{\ke, \txid, i, j} 
%        \WTx(\hh(\ke, i)) = \WTx(\hh(\ke, j)) \lor {} \\
%        (\txid \in \RTx(\hh(\ke, i)) \land \txid \in \RTx(\hh(\ke, j))
%        \implies i = j 
%    }
%\end{rclarray}
%\]
%
%We model a kv-store $\hh$ as a map from keys to lists of versions.
%%where each key is associated with a list \emph{versions} from the initial one to the latest one (\defref{def:mkvs}).
%%More concretely, versions associated with a key $\ke$ 
%Versions are tuples of the form \( \ver =( \val, \txid, \txidset ) \), where $\val$ denotes the \emph{current value} of $\ke$, 
%$\txid$ is the identifier of the transaction that wrote such a version, 
%%, identifying its \emph{writer}, i.e.\ the transaction responsible for writing value $\val$, 
%and $\txidset$ is the set of identifiers of transactions that read the version.
%%, denoting its \emph{readers}, i.e.\ those transactions who read from $\ke$ 
%(see \cref{def:mkvs}).
%
%\cref{fig:hheap-a} depicts an example of an MKVS. 
%Ignoring the lines labelled $\client_1$ and $\client_2$, the depicted MKVS contains two keys \( \ke_1\) and \( \ke_2 \), each of which associated with two versions (with values $v_0$ and $v_1$ for $\ke_1$, and values $v'_0$ and $v'_1$ for $\ke_2$).
%The versions of a key are listed in chronological order from left to right.
%We represent each version as a three-cell box, with the left cell storing the value, the top right cell recording the writer, and the bottom right cell recording the readers. 
%
%
%Given a version $\ver = (\val, \txid, \txidset)$, we write $\valueOf(\ver)$ for its value ($n$), $\WTx(\ver)$ for its writer ($\txid$), $\RTx(\ver)$ for its readers ($\txidset$), and \( \txid' \in \ver \) for \( \txid' = \txid \lor \txid' \in \txidset \).
%%A \emph{Multi-version Key-value Store (MKVS)} is a mapping $\hh : \Keys \parfinfun \Versions^{\ast}$ from keys to lists of versions. 
%Given a list of versions $L = \List{\ver_0, \cdots, \ver_{s-1}}$, we write  $\lvert L\rvert$ for its length ($s$). 
%Given an MKVS $\hh$ and a key $\ke$, we write $\hh(\ke)$ for the list of versions associated with $\ke$ in $\hh$, and write \(\hh(\ke,i) \) for the $i$\textsuperscript{th} entry (indexed from $0$) in $\hh(\ke)$. 
%%We assume the index starts from 0, so \( \hh(\ke)(\left|\hh(\ke)\right| - 1)\) is the latest versions of the key \( \ke \).
%
%\azalea{I think $\hh$ is too close to $\ke$ and we should use a different symbol. Maybe $H$? }
%
%We assume a countably infinite set of \emph{keys}, $\Keys $,
%a set of countably infinite set of \emph{program values}, \( \val \in \Val \),
%a countably infinite set of \emph{client identifiers}, \( \cl \in \Clients \),
%and a countably infinite set of \emph{transactions identifiers indexed by client}, $\txid^{\cl} \in \TxID$
%We also assume there are countably infinite elements for every client, and elements for each client are ordered.
%We use \( \txid^{\cl}_{i}\) for the \emph{i-th} identifier among the total order and it is indexed by the client \( \cl \). 
%For a client, the order among transaction identifiers corresponds to the session order of them.
%Sometime we omit the order and/or client index when they are irrelevant.
%We use $\ke$ and its variants (e.g.\ $\ke_1$, $\ke'$ and so forth) as meta-variables for keys in $\Keys$.
%%and use $\txid$ and its variants as meta-variables for transaction identifiers in $\TxID$. 
%
%A key-value store is \emph{well-formed} iff 
%a transaction identifier appears in all versions of a key at most twice, once as the writer and once as a reader.
%%(i) it does not contain circular dependencies across its versions; and 
%%
%%Given two versions $\ver_1, \ver_2$ in an MKVS $\hh$, the $\ver_2$ \emph{directly depends} on $\ver_1$, written $\ver_1 \xrightarrow{\ddep} \ver_2$, iff:
%%\[ 
%%\begin{rclarray}
%%\ver_1 \xrightarrow{\ddep} \ver_2 & \defeq &
%%\begin{array}[t]{@{}l}
%%\WTx(\ver_2) \in \RTx(\ver_1) 
%%%\exsts{\cl,i,j,\txid_{i}^{\cl},\txid_{j}^{\cl}} \\
%%%\quad \txid_{i}^{\cl} < \txid_{j}^{\cl}
%%%\land \txid_{i}^{\cl} \in \ver_{1}
%%%\land \txid_{j}^{\cl} \in \ver_{2}
%%\end{array}
%%\end{rclarray}
%%\]
%%that is, a transaction $\txid$ wrote the version $\ver_2$ after reading $\ver_1$;
%%%or two transactions from the same client access
%%%The $\ver_1 \xrightarrow{\ddep(\hh)} \ver_2$ denotes that $\ver_1, \ver_2$ appear as versions from the same KVMS $\hh$ and \( \ver_1\) depends on \(\ver_2 \).
%
%%A key-value store is \emph{well-formed}, written $\pred{wfMKVS}{\hh}$, iff 
%%(i) it does not contain circular dependencies across its versions; and 
%%(ii) a transaction identifier appears in all versions of a key at most twice, once as the writer and once as a reader.
%%% a transaction identifier appears in all the versions for an key at most twice, one as the writer and one as a reader.
%%%
%%%Also there are no circular dependencies in versions. 
%%%More concretely, the first well-formedness condition (i) of an KVMS ensures that the \emph{transitive closure} of the direct dependency relation $\left(\xrightarrow{\ddep(\hh)}\right)^{+}$ is acyclic:
%%%in $\hh$, i.e. $\left(\xrightarrow{\ddep(\hh)} \right)^{+} \cap \text{Id} = \emptyset$ (where $\text{Id}$ is 
%%%the identity relation).
%%%\[
%%%\begin{rclarray}
%    %%\pred{wfMKVS}{\hh} & \defeq &
%    %%\begin{array}[t]{@{}l}
%        %%\pred{acyclic}{\left(\xrightarrow{\ddep(\hh)}\right)^{+}}
%        %%\land \fora{\ke, \txid, i, j}
%        %%\begin{B}
%        %%\WTx(\hh(\ke, i)) = \WTx(\hh(\ke, j)) \lor {} \\ (\txid \in \RTx(\hh(\ke, i)) \land \txid \in \RTx(\hh(\ke, j))
%        %%\end{B} 
%        %%\implies i = j 
%    %%\end{array}
%%%\end{rclarray}
%%%\]
%%%
%The \emph{partial commutative monoid (pcm) of MKVSs} is $(\MKVSs, \composeHH, \{\unitHH\})$, where 
%$\composeHH:  \MKVSs \times \MKVSs \rightharpoonup \MKVSs$ denotes the \emph{pcm composition} defined as the standard function disjointed union: $\composeHH \eqdef \uplus$; and
%%\( \hh \composeHH \hh' \defeq \hh \uplus \hh' \)
%$\unitHH \in \MKVSs$ denotes the \emph{pcm unit element}:  $\unitHH \eqdef \emptyset$, where $\emptyset$ denotes a function with an empty domain.
%\end{definition}
%
%\ac{A point that this does not ensure a real causal dependency between the two versions, yet it is consistent with the notion of causality employed in databases, should be made}. 
%

\begin{figure}
\begin{center}
\hrule\vspace{5pt}
\begin{tabular}{@{}c c@{}}
\begin{halfsubfig}
\begin{centertikz}

\begin{pgfonlayer}{foreground}
%Uncomment line below for help lines
%\draw[help lines] grid(5,4);

%Location x
\node(locx) {$\ke_1 \mapsto$};

\matrix(versionx) [version list]
    at ([xshift=\tikzkvspace]locx.east) {
    {a} & $\txid_0$ & {a} & $\txid_1$\\
    {a} & $\{\txid_2\}$ & {a} & $\emptyset$ \\
};
\tikzvalue{versionx-1-1}{versionx-2-1}{locx-v0}{$v_0$};
\tikzvalue{versionx-1-3}{versionx-2-3}{locx-v1}{$v_1$};
%\node[version node, fit=(locxcells-1-1) (locxcells-2-1), fill=white, inner sep= 0cm, font=\Large] (locx-v0) {$0$};
%\node[version node, fit=(locxcells-1-3) (locxcells-2-3), fill=white, inner sep=0cm, font=\Large] (locx-v1) {$1$};

%Location y
\path (locx.south) + (0,\tikzkeyspace) node (locy) {$\ke_2 \mapsto$};
\matrix(versiony) [version list]
   at ([xshift=\tikzkvspace]locy.east) {
 {a} & $\txid_0$ & {a} & $\txid_2$ \\
  {a} & $\{\txid_1\}$ & {a} & $\emptyset$\\
};

\tikzvalue{versiony-1-1}{versiony-2-1}{locy-v0}{$v'_0$};
\tikzvalue{versiony-1-3}{versiony-2-3}{locy-v1}{$v'_1$};
%\node[version node, fit=(locycells-1-1) (locycells-2-1), fill=white, inner sep= 0cm, font=\Large] (locy-v0) {$0$};
%\node[version node, fit=(locycells-1-3) (locycells-2-3), fill=white, inner sep=0cm, font=\Large] (locy-v1) {$1$};

% \draw[-, red, very thick, rounded corners] ([xshift=-5pt, yshift=5pt]locx-v1.north east) |- 
%  ($([xshift=-5pt,yshift=-5pt]locx-v1.south east)!.5!([xshift=-5pt, yshift=5pt]locy-v0.north east)$) -| ([xshift=-5pt, yshift=5pt]locy-v0.south east);

%blue view - I should  check whether I can use pgfkeys to just declare the list of locations, and then add the view automatically.
\draw[-, blue, very thick, rounded corners=10pt]
 ([xshift=-3pt, yshift=20pt]locx-v1.north east) node (tid1start) {} -- 
 ([xshift=-3pt, yshift=-5pt]locx-v1.south east) --
 ([xshift=-3pt, yshift=5pt]locy-v0.north east) -- 
 ([xshift=-3pt, yshift=-5pt]locy-v0.south east);
 
 \path (tid1start) node[anchor=south, rectangle, fill=blue!20, draw=blue, font=\small, inner sep=1pt] {$\client_1$};

%red view
\draw[-, red, very thick, rounded corners = 10pt]
 ([xshift=-16pt, yshift=5pt]locx-v1.north east) node (tid2start) {}-- 
% ([xshift=-16pt, yshift=-5pt]locx-v0.south east) --
% ([xshift=-16pt, yshift=5pt]locy-v1.north east) -- 
 ([xshift=-16pt, yshift=-5pt]locy-v1.south east) node {};
 
\path (tid2start) node[anchor=south, rectangle, fill=red!20, draw=red, font=\small, inner sep=1pt] {$\client_2$};

%%Stack for clients tid_1 and tid_2
%
%\draw[-, dashed] let 
%   \p1 = ([xshift=0pt]locy.west),
%   \p2 = ([yshift=-5pt]locycells.south),
%   \p3 = ([xshift=10pt]locycells.east) in
%   (\x1, \y2) -- (\x3, \y2);
%   
%\matrix(stacks) [
%   matrix of nodes,
%   anchor=north, 
%   text=blue, 
%   font=\normalsize, 
%   row 1/.style = {text = blue}, 
%   row 2/.style = {text = red}, 
%   text width= 13mm ] 
%   at ([xshift=-10pt,yshift=-8pt]locycells.south) {
%   $\txid_1:$ & $\retvar = 0$\\
%   $\txid_2:$ & $\retvar = 0$\\
%   };
\end{pgfonlayer}
\end{centertikz}
\caption{A configuration with a well-formed MSKV and two views}
\label{fig:hheap-a}
\end{halfsubfig}
&

\begin{halfsubfig} 
\begin{centertikz}

\begin{pgfonlayer}{foreground}
%Uncomment line below for help lines
%\draw[help lines] grid(5,4);

%Location x
\node(locx) {$\ke_1 \mapsto$};

\matrix(versionx) [version list, text width=7mm, anchor=west]
    at ([xshift=\tikzkvspace]locx.east) {
    {a} & $\txid_0$ & {a} & $\txid_1$\\
    {a} & $\emptyset$ & {a} & $\{\txid_2\}$ \\
};
\tikzvalue{versionx-1-1}{versionx-2-1}{locx-v0}{$v_0$};
\tikzvalue{versionx-1-3}{versionx-2-3}{locx-v1}{$v_1$};

%Location y
\path (locx.south) + (0,\tikzkeyspace) node (locy) {$\ke_2 \mapsto$};
\matrix(versiony) [version list, text width=7mm, anchor=west]
    at ([xshift=\tikzkvspace]locy.east) {
    {a} & $\txid_0$ & {a} & $\txid_2$ \\
    {a} & $\emptyset$ & {a} & $\{\txid_1\}$\\
};
\tikzvalue{versiony-1-1}{versiony-2-1}{locy-v0}{$v'_0$};
\tikzvalue{versiony-1-3}{versiony-2-3}{locu-v1}{$v'_1$};

% \draw[-, red, very thick, rounded corners] ([xshift=-5pt, yshift=5pt]locx-v1.north east) |- 
%  ($([xshift=-5pt,yshift=-5pt]locx-v1.south east)!.5!([xshift=-5pt, yshift=5pt]locy-v0.north east)$) -| ([xshift=-5pt, yshift=5pt]locy-v0.south east);

%blue view - I should  check whether I can use pgfkeys to just declare the list of locations, and then add the view automatically.
%\draw[-, blue, very thick, rounded corners=10pt]
% ([xshift=-3pt, yshift=20pt]locx-v1.north east) node (tid1start) {} -- 
% ([xshift=-3pt, yshift=-5pt]locx-v1.south east) --
% ([xshift=-3pt, yshift=5pt]locy-v0.north east) -- 
% ([xshift=-3pt, yshift=-5pt]locy-v0.south east);
% 
% \path (tid1start) node[anchor=south, rectangle, fill=blue!20, draw=blue, font=\small, inner sep=1pt] {$\txid_1$};
%
%%red view
%\draw[-, red, very thick, rounded corners = 10pt]
% ([xshift=-16pt, yshift=5pt]locx-v1.north east) node (tid2start) {}-- 
%% ([xshift=-16pt, yshift=-5pt]locx-v0.south east) --
%% ([xshift=-16pt, yshift=5pt]locy-v1.north east) -- 
% ([xshift=-16pt, yshift=-5pt]locy-v1.south east) node {};
% 
%\path (tid2start) node[anchor=south, rectangle, fill=red!20, draw=red, font=\small, inner sep=1pt] {$\txid_2$};

%%Stack for clients tid_1 and tid_2
%
%\draw[-, dashed] let 
%   \p1 = ([xshift=0pt]locy.west),
%   \p2 = ([yshift=-5pt]locycells.south),
%   \p3 = ([xshift=10pt]locycells.east) in
%   (\x1, \y2) -- (\x3, \y2);
%   
%\matrix(stacks) [
%   matrix of nodes,
%   anchor=north, 
%   text=blue, 
%   font=\normalsize, 
%   row 1/.style = {text = blue}, 
%   row 2/.style = {text = red}, 
%   text width= 13mm ] 
%   at ([xshift=-10pt,yshift=-8pt]locycells.south) {
%   $\txid_1:$ & $\retvar = 0$\\
%   $\txid_2:$ & $\retvar = 0$\\
%   };
\end{pgfonlayer}
\end{centertikz}
\caption{An ill-formed MKVS}
\label{fig:hheap-b}
\end{halfsubfig} \\
\end{tabular}
\end{center}
\hrule\vspace{5pt}
\ac{Remove ill-formed heap. Insert Configuration with non-atomic view. Refer to the figure 
when explaining the notion of atomic view.}
\caption{Multi-version key-value stores}
\label{fig:hheap}
\end{figure}

%Formally speaking, we assume a countably infinite set of keys $\ke = \{\key{k}_1, \cdots\}$, a set of transaction identifiers $\TxID = 
%\{ \txid_1, \cdots \}$, 
%a set of clients $\txidset = \{\txid_1, \cdots \}$.
%We also assume values to be natural numbers from the set $\nat$. 
%\ac{from now on, the sort font is used for sets.}
%\begin{definition}
%\label{def:hheap}
%A \emph{version} is a triple $\ver = ( n, \txid, \T)$, where, 
%$n$ is the value of the version, $\txid$ is the identifier of the transaction 
%that wrote the version, and $\T$ is a (possibly empty) set of identifiers of 
%the transactions that read the version.
%Given a version $\ver = (n, \txid, \T)$, 
%we let $\valueOf(\ver) = n$, $\WTx(\ver) = \txid$, $\RTx(\ver) = \T$. 
%The set of versions is denoted as $\Versions$.

%A \emph{Multi-version Key-value Store}, or MKVS, is a mapping  
%$\hh : \ke \rightarrow \Versions^{\ast}$ from keys to lists of versions. 
%\end{definition}
%Given a list of versions $\ver_1 \cdots \ver_{n}$, 
%we let $\lvert \ver_1 \cdots \ver_{n} \rvert = n$ be 
%its length. Also, let $\hh$ be a MKVS, $\key{k}$ be a key 
%such that $\hh(\key{k}) = \ver_1 \cdots \ver_n$, and 
%$i \leq n$ be a strictly positive natural number; then we let 
%$\hh(\key{k}, i) = \ver_i$. 

%\ac{Maybe it's better to keep $\ke$ fixed and say that we look at only a 
%fragment of the key value store. Alternatively, we can go for partial mappings to 
%represent MKVSs, but still avoiding allocation and deallocation of keys.}. To the left 
%we have the set of keys stored y $\hh_0$, in this case $\key{k}_1$ and 
%$\key{k}_2$. To the right, on the same line of a key, a matrix containing the
 %list of versions stored by such a key in $\hh_0$. Starting from the first column, 
 %each version is represented by two adjacent columns in the matrix: the 
 %column to the left gives the value of the version, while the column to the 
 %right contains the identifier of the transaction that wrote the version to the 
 %top, and the identifiers of the transactions that read such a version to the bottom.
%In the case of $\hh_0$, there are two versions stored for $\key{k}_1$: 
%the first one with value $0$, written by $\txid_0$ and read by $\txid_2$; 
%and the second one with value $1$, written by $\txid_2$ and read by no 
%transaction. 

%\ac{
%Throughout this paper, we focus on MKVSs that can be obtained in 
%databases whose consistency guarantees enjoy atomic visibility 
%\cite{framework-concur,SIanalysis,laws}. To this end, we impose 
%some well-formedness constraints on the MKVSs.

%\begin{definition}
%\label{def:hh.wellformed}
%\label{def:ddep}
%A MKVS $\hh$ is \emph{well-formed} if and only if 
%\begin{itemize}
%\item a transaction does not write two different versions for the same key: 
%$\forall \key{k} \in \ke.\;\forall i, j = 1,\cdots, \lvert \hh(\key{k}) \rvert. 
%\WTx(\hh(\key{k}, i)) = \WTx(\hh(\key{k}, j)) \implies i = j$, 
%\item a transaction does not read two different versions for the same key:  
%$\forall \key{k} \in \ke.\;\forall i, j = 1,\cdots, \lvert \hh(\key{k}) \rvert. 
%(\RTx(\hh(\key{k},i) \cap \RTx(\hh(\key{k}, j)) \neq \emptyset) \implies i = j$.
%\item There are no circular dependencies in versions. Given two versions 
%$\ver_1, \ver_2$, we say that $\ver_2$ \emph{direct dependency} from 
%$\ver_1$, written $\ver_1 \xrightarrow{\ddep} \ver_2$, if $\WTx(nu_2) \in \RTx(\ver_1)$; 
%that is, some transaction $\txid$ wrote the version $\ver_2$ after reading $\ver_1$ 
%\ac{A point that this does not ensure a real causal dependency between the 
%two versions, yet it is consistent with the notion of causality employed in databases, 
%should be made}. If $\ver_1, \ver_2$ appear as versions of some object in 
%$\hh$, then we write $\ver_1 \xrightarrow{\ddep(\hh)} \ver_2$. Then the relation $\left(\xrightarrow{\ddep(\hh)}\right)^{+}$ is acyclic in $\hh$, 
%i.e. $\left(\xrightarrow{\ddep(\hh)} \right)^{+} \cap \text{Id} = \emptyset$ (where $\text{Id}$ is 
%the identity relation).
%\end{itemize}
%\end{definition}
%}

%Let us elaborate on the first well-formedness constraint of an MKVS \( \mkvs \) in \cref{def:mkvs}. As stated above, this states that there is no circularity in the dependency relation.
%This in turn ensures that no versions are created \emph{out of thin-air}.
%An example of the out of thin-air anomaly is given in \cref{fig:hheap-b}, where
%transaction $\txid_2$ reads the value of $\ke_1$ written by $\txid_1$;
%conversely, transaction $\txid_1$ reads the value of $\ke_2$ written by $\txid_2$. 
%As we assume transactions read a state of the key-value store from an atomic snapshot fixed at the moment they execute, this situation cannot arise. 
%For $\txid_2$ to read the version written by $\txid_1$, transaction $\txid_2$ must start after $\txid_1$, \ie \( \hh(\ke_2, 1) \xrightarrow{\ddep(\hh_1)} \hh(\ke_1, 1) \).
%Similarly, $\txid_1$ must starts after $\txid_2$, \ie \( \hh(\ke_1, 1) \xrightarrow{\ddep(\hh_1)} \hh_1(\ke_2, 1) \).
%This however violates the well-formedness of MKVSs that $\xrightarrow{\ddep(\hh_1)}$ is acyclic. 

%\azalea{Why $\hh_1$ for the store and not $\hh$? Also I thought the versions are indexed from $0$ in which case index $2$ does not make sense here? \sx{\( \mkvs \) and index starts from 0.}}

%\azalea{common Latin abbreviations such as \ie, \eg, and et al. do not need to be italicised. I have adjusted the macros. I have also rephrased the definitions quite a bit. Make sure you're happy with this.}
%When introducing our semantics of clients in \S \ref{sec:semantics}, we show that (under reasonable conditions) it generates only well-formed MKVSs.

%\ac{
%A view $V$ defines the particular version of each key that a client 
%will observe when executing a transaction. A configuration consists 
%of a MKVS, and the views that a set of clients have each on the state 
%of the MKVS. An example of configuration is given in Figure \ref{fig:hheap}(a). 
%There are two clients, $\txid_1$ and $\txid_2$, each with their own view 
%(represented in the Figure by labelled lines crossing the MKVS at each location). 
%According to the view of $\txid_1$, formally defined as $V_1 = [\key{k}_1 \mapsto 2], 
%\key{k}_2 \mapsto 1]$, this client observes in $\hh$ the second version of key $\key{k}_1$, carrying 
%value $1$, and 
%the first version of $\key{k}_2$, carrying value $0$. Similarly, according to its view 
%$V_2 = [\key{k}_1 \mapsto 2, \key{k}_2 \mapsto 2]$, the client $\txid_2$ observes 
%in $\hh$the second and most up-to-date version for both $\key{k}_1$ and $\key{k}_2$.
%}

\mypar{Views, Configurations and Snapshots.}
\ac{Configurations are trivial once that view are defined.}
The main idea behind our formal notion of key-value stores is that, 
when executing a transaction, different clients may observe 
different version of the same key, at any given time. To keep track of 
the versions they observe, clients are associated with a \emph{view}. 
\begin{definition}
\label{def:view}
\label{def:cuts}
\label{def:views}
\label{def:configuration}
Let $\hh$ be a key-value store; a view in $\hh$ is function  
$V: \dom(\hh) \rightarrow \Nat$ such that, for any $\ke \in \dom(\hh)$, 
$V(\ke) < \lvert \hh(\ke) \rvert$, and 
\begin{equation}
\label{eq:view.atomic}
\forall \ke,\ke' \in \dom(\hh), i,j \in \Nat.\; (j \leq  \vi(\ke) \wedge 
\WTx(\hh(\ke, v(\ke))) = \WTx(\hh(\ke', i)) \implies i \leq \vi(\ke').
\tag{AtomicView}
\end{equation}
The set of views of $\hh$ is denoted 
as $\Views_{\hh}$, and we let $\Views = \bigcup_{\hh \in \HisHeaps} 
\Views_{\hh}$.

A configuration $\conf$ is a pair $(\hh, \viewFun)$, where $\viewFun: 
\Clients \parfinfun \Views_{\hh}$.
\end{definition}
Given $\hh \in \HisHeaps$ and two views $\vi, \vi' \in \Views_{\hh}$, 
we let $\vi \viewleq \vi'$ if, for any $\ke \in \dom(\hh)$, $\vi(k) \leq \vi'(\ke)$. 
In this case we say that $\vi$ is older than $\vi'$ ($\vi'$ is newer than $\vi$).

Simply put, a configuration augments the notion of kv-stores with the information 
about the version observed by each client of its clients. The view of a client $\cl$ in $\hh$ 
keeps track of the most recent version of $\hh$ that $\cl$ can access when executing a 
transaction. The constraint of \cref{eq:view.atomic} establishes that clients observe 
either none or all the updates performed by a transaction, thus modelling atomic 
visibility.
We often depict configurations graphically by depicting views as client-labelled lines crossing 
versions  of key-value stores. A line labelled $\cl$ crossing the $i$-th version of key $\ke$ defines a view 
$\vi$ for client $\cl$, with $\vi(\ke) = i$. One example is given by the configuration 
$\conf_0 = (\hh_0, [\cl_1 \mapsto \vi_1, \cl_2 \mapsto \vi_2])$ of \cref{fig:hheap-a}. 
There are two clients, 
$\cl_1$ and $\cl_2$, with views $\vi_1, \vi_2$ respectively. $\vi_1$ Crosses $\ke_1$ at its $0$-th 
version, and $\ke_2$ at its $1$-st version. Therefore we have $\vi_1 = [\ke_1 \mapsto 0, \ke_2 \mapsto_1]$. 
Similarly, $\vi_2 = [\ke_1 \mapsto 1, \ke_2 \mapsto 0]$. 
\ac{Never refer to colors in text. People may read in black and white.}

Given a kv-store $\hh$, a view $\vi$ and a key $\ke \in \dom(\hh)$, 
we commit an abuse of notation and write $\hh(\ke, \vi)$ as a shorthand 
for $\hh(\ke, \vi(\ke))$. The view $\vi$ naturally induces a \emph{snapshot} 
by extracting the value of the $\vi(\ke)$-th version for each key $\ke \in \dom(\hh)$. 
As we will see presently, snapshots are used to determine the value of keys returned 
by read operations of transactions. 
\begin{definition}
\label{def:heaps}
\label{def:snapshot}
Let $\hh \in \HisHeaps, \vi \in \Views_{\hh}$. The snapshot of $\vi$ in 
$\hh$ is defined as $\snapshot(\hh, \vi) = \lambda \ke. \valueOf(\hh(\ke, \vi))$.
\end{definition}

\ac{General comment: the explanations should be okay, but I wonder whether we now have 
too many definitions. We may consider putting all the definitions in a table, and 
go through there in the text. Though I am for using tables with notations and 
definitions only as a last resort - they tend to be scary.}

%of $\vi$ by accessing the value of 
%A view $\vi$ in $\hh$ naturally defines a snapshot $\snapshot(\hh, \nu)$
%A MKVS tracks the global state of the system; however, different \emph{clients} may observe different versions of the same key. 
%To model this, we introduce the notion of \emph{views} (\cref{def:views}). 
%A view $V$ reflects the particular version for each key that a client observes upon executing a transaction. 
%%We present an example of views in \cref{fig:hheap-a} with two views: $\client_1$ in red and $\client_2$ in blue.
%More concretely, the view for \( \client_1 \) is given formally as $\vi_1 = \Set{\key{k}_1 \mapsto 1, \key{k}_2 \mapsto 0}$.
%That is, the client with view $\vi_1$ observes the second version (at index 1) of key \( \ke_{1} \) with value $v_1$, and the first version (at index 0) of key \( \ke_2 \) with value $v'_0$.
%%, and 
%%the first version of $\key{k}_2$, carrying value $0$. Similarly, according to its view 
%%$V_2 = [\key{k}_1 \mapsto 2, \key{k}_2 \mapsto 2]$, the client $\txid_2$ observes 
%%in $\hh$the second and most up-to-date version for both $\key{k}_1$ and $\key{k}_2$.
%
%\begin{definition}[Views]
%\label{def:view}
%\label{def:cuts}
%\label{def:views}
%\emph{A view} is a partial finite function from keys to indexes:
%$
%\vi \in \Views \defeq \Addr \parfinfun \Nat 
%%\begin{rclarray}
%%    \vi \in \Views & \defeq & \Addr \parfinfun \Nat 
%%\end{rclarray}
%$.                                                                 
%The \emph{view composition}, $\composeVI: \Views \times \Views \rightharpoonup \Views$ is defined as the standard disjoint function union: $\composeVI \eqdef \uplus$. 
%% \( \vi \composeVI \vi' \defeq \vi \uplus \vi'\) 
%The \emph{unit view}, $\unitVI \in \Views$, is a function with an empty domain: $\unitVI \eqdef \emptyset$. 
%% and the unit is \( \unitVI \defeq \emptyset\).
%The \emph{order relation} on views, $\orderVI: \Views \times \Views$, is defined between two views with the same domain as the point-wise comparison of their indexes for each entry: 
%\[
%\begin{rclarray}
%    \vi \orderVI \vi' & \defiff & \dom(\vi) = \dom(\vi') \land \fora{\ke} \cu(\ke) \leq \cu'(\ke) \\
%\end{rclarray}
%\]
%\end{definition}
%%
%We say view $\vi$ is \emph{older} than view $\vi'$ (or $\vi'$ is \emph{newer} than $\vi$) whenever $\vi \orderVI \vi'$ holds.
%
%
%\mypar{Configurations} A \emph{configuration} comprises an MKVS, and the views associated with clients.
%In \cref{fig:hheap-a} we present an example of a configuration comprising an MKVS and the two views associated with clients $\client_1$ and $\client_2$. 
%We write $\version(\hh, \ke, \vi)$ for $\hh(\ke, \vi(\ke))$; 
%and write $\valueOf(\hh, \ke, \vi)$ as a shorthand for $ \valueOf(\version(\hh, \key{k}, V))$; similarly for $\WTx, \RTx$.
%%we commit an abuse of notation and often write $\valueOf(\hh, \ke, \vi)$ in lieu of $ \valueOf(\version(\hh, \key{k}, V))$, and similarly for $\WTx, \RTx$.
%When $\ver = \version(\hh, \ke, \vi)$, we say that \emph{$\vi$ $\ke$-points to $\ver$ in $\hh$}. 
%When $\ver = \hh(\ke, i)$ for some $0 \leq i \le \vi(\ke)$, we say that \emph{$\vi$ $\ke$-includes $\ver$ in $\hh$}.
%Lastly, we always assume that MKVSs, views, and configurations are well-formed, unless otherwise stated.
%
%
%
%\begin{definition}[Configurations]
%A view $\vi$ is \emph{well-formed with respect to an MKVS} $\mkvs$, written \( \wfV{\mkvs, \vi} \),  iff they have the same domain and every index from $\vi$ is within the range of the corresponding entry in $\mkvs$ and the view is \emph{atomic} with  respect to the key-value store: 
%\[
%\begin{rclarray}
%    \wfV{\mkvs, \vi} & \defeq & \dom(\mkvs) = \dom(\vi) \land \fora{\ke \in \dom(\vi)} 0 \leq \vi(\ke) < \lvert \mkvs(\ke) \rvert \\
%    \pred{atomic}{\vi ,\hh} & \eqdef & \fora{\txid } \exsts{\ke, i} i \leq \vi(\ke) \land \hh(\ke,i) = (\stub, \txid, \stub) \implies \pred{visible}{\txid, \vi, \hh} \\ 
%    \pred{visible}{\txid, \vi, \hh} & \eqdef & \fora{\ke, i} \hh(\ke,i) = (\stub, \txid, \stub) \implies i \leq \vi(\ke) 
%\end{rclarray}
%\]
%%
%\azalea{We need a symbol for this to fill the ???? above. Also ???? below. \sx{Done}}
%A \emph{configuration} $\conf$ is a pair of the form $(\hh, \viewFun)$, where $\hh$ denotes an MKVS, and $\viewFun: \Clients \parfinfun \Views$ is a partial finite function from clients to views. 
%A configuration $\conf = (\hh, \viewFun)$ is \emph{well-formed}, written \( \wfC{\conf}\), iff for all clients $\cl \in \dom(\viewFun)$, the view $\viewFun(\txid)$ is well-formed with respect to $\hh$. 
%%We say that a view $V$ is well-defined with respect to the 
%%MKVS $\hh$ if, $\forall \key{k} \in \ke. 0 < V(\key{k}) \leq 
%%\lvert \hh(\key{k}) \rvert$. 
%%Given a view $V$ that is well-defined 
%%with respect to a 
%
%\end{definition}
%
%\mypar{Snapshots} When a client executes a transaction on the $\mkvs$ MKVS, it extracts a \emph{snapshot} of it via the \( \func{snapshot}{\mkvs, \vi} \) function, extracting the values corresponding to the versions indexed by its view \( \vi \) (\cref{def:snapshot}).
%For instance, for client \( \client_1 \) in \cref{fig:hheap-a}, the $\func{snapshot}{\cdots}$ functions yields a state where key $\ke_1$ carries value $v_1$ and second key \( \ke_2 \) carries value $v'_0$.
%%The concrete state extracted in this way takes the name of the \emph{snapshot} of the transaction.
%%In general, the process of determining the view of a client, hence the snapshot in which such a client executes transactions, is non-deterministic.
%
%\azalea{Before in MKVSs we had values drawn from $\Nat$ in \cref{def:mkvs}. Now we use $\Val$. I think you mean to use $\Val$ in both places? \sx{I would say so} }
%\begin{definition}[Snapshots]
%\label{def:heaps}
%\label{def:snapshot}
%Given the sets of values $\Val$  and keys \( \Addr\)  (\cref{def:mkvs}), the set of \emph{snapshots} is:
%$
%    \h \in \Heaps \eqdef \Addr \parfinfun \Val
%$. 
%%\[
%%\begin{rclarray}
%%    \h \in \Heaps & \eqdef & \Addr \parfinfun \Val
%%\end{rclarray}
%%\]
%The \emph{snapshot composition function}, $\composeH: \Heaps \times \Heaps \parfun \Heaps$, is defined as $\composeH \eqdef \uplus$, where $\uplus$ denotes the standard disjoint function union. The \emph{ snapshot unit element} is $\unitH \eqdef \emptyset$, denoting a function with an empty domain.
%The \emph{partial commutative monoid of snapshots} is $(\Heaps, \composeH, \{\unitH\})$.
%Given an MKVS $\hh$ and a view $\vi$, the snapshot of $\vi$ in $\hh$, written $\snapshot(\hh, \vi) $, is defined as:
%$
%    \snapshot(\hh, \vi) \defeq \lambda \ke \ldotp \valueOf(\hh, \ke, \vi)
%$.
%%\[
%%\begin{rclarray}
%%    \snapshot(\hh, \vi) & \defeq & \lambda \ke \ldotp \valueOf(\hh, \ke, \vi).
%%\end{rclarray}
%%\]
%\end{definition}
%
%\sx{Need some explanation}
%\ac{General Comment on this Section: it is too abstract. We 
%should give either here or in the introduction an example of computation - 
%the write skew program should be okay that helps the reader understanding 
%what's going on. Also, it could be also good to illustrate the notions 
%of execution tests and consistency models.}
%
%\sx{From Andrea: introduce the execution test here with a table, also introduce fingerprint here}

\subsection{Specification of Consistency Models using Key-value Stores}
\label{sec:execution.tests}

In this section we model how key-value stores evolve when a client commits 
the effects of a transaction.
Because we want to model different consistency models, 
this notion is parametrised by an \emph{execution test}, which determine whether 
a client may commit the effects of a transaction to the kv-store, at a given time.
Because we want to model different consistency models.

\mypar{Fingerprints} 
Formally, the effects of transactions are modelled as sets of reads \( \otR \) and writes \( \otW \)
over keys; we assume a set of \emph{operations} $\Ops \defeq \Setcon{(\otR, \ke, \val), (\otW, \ke, \val) }{ \ke \in \Keys \wedge 
\val \in \Val }$. The \emph{fingerprint} of a transaction is a set of operations $\opset \subseteq \Ops$, such that any key $\ke \in \Keys$,
\textbf{(i)} whenever $(\otR, \ke, \val_1) \in \opset$, $(\otR, \ke, \val_2) \in \opset$, then $\val_1 = \val_2$, 
and \textbf{(ii)} whenever $(\otW, \ke, \val_1) \in \opset$, $(\otW, \ke, \val_2) \in \opset$, then $\val_1 = \val_2$. 
Intuitively, $(\otR, \ke, \val) \in \opset$ means that the transaction requested to read key $\ke$ from the kv-store, 
and it fetched a version carrying value $\val$. Condition \textbf{(i)} states that there is at most one read operation per key; it  
formalises the intuition that, in our setting, 
transactions always read from an atomic snapshot of the kv-store, hence only one version will be read for key $\ke$. 
\ac{Note that I now require fingerprints to be non-empty sets of transactions. This simplifies a lot the development of 
the theory of kv-stores, and it fixes a problem that was spotted by Shale, that breaks the compositionality of 
execution tests (see later). The main reason why we allowed empty fingerprints is that in the semantics, a client can 
execute a transaction with no access to the memory. In practice, in the semantics we can require that at least 
one access to the database must be performed in transactions. This can be checked syntactically, and nobody 
should complain about that. I can put a remark about how this is a natural requirement that, if violated, 
breaks the compositionality of consistency models.\\ 
\textbf{Update 02/08/2018}: empty fingerprints are now allowed again. We still had some problems with compositionality, 
one of which has to do with the fact that we allow the view of a client over some key to move freely after executing a transaction, 
even if such a key was not accessed by the transaction. Later, I forbid this behaviour by requiring in execution tests that the 
view of an client for a given key cannot be shifted if the transaction executed by the client did not access such a key.}
%$(\otW, \ke, \val) \in \opset$ means that the transaction writes a new version, carrying value $\val$, for key $\ke$. 
Condition \textbf{(ii)} above is needed because we assume that a client either observes none or all the updates 
of a transaction; consequently, in the fingerprint of a transaction we require at most one write operation per key. 

\mypar{Committing Transactions.}
First we introduce how a client committing the effects of a transaction affects the kv-store. 
Suppose that a client $\cl$, with view $\vi$ in the kv-store $\hh$,
wants to commit the effects of a transaction whose fingerprint is $\opset$.
We model this via the function $\updateKV: \HisHeaps \times \Views \times \TxID \times \powerset{\Ops} 
\to \HisHeaps$, defined recursively below:
\begin{equation}
\begin{rclarray}         
%	 \func{updateMKVS}{., ., ., .} & : & \MKVSs \times \Views \times  \\                        
    \updateKV(\hh, \vi, \txid, \emptyset) &\defeq & \hh \\
    \updateKV(\hh, \vi, \txid, \opset \uplus \Set{(\otR, \ke, \stub)}) & \defeq &  
    \begin{array}[t]{@{}l}
        \texttt{let } (\val, \txid', \txidset) = \hh(\ke, \max_{<}(\vi(\ke))), \\
        \vilist = \hh(\ke)\rmto{\max_{<}(\vi(\ke))}{(\val, \txid', \txidset \uplus \{ t \})}\\
        \quad \texttt{in } \updateKV(\hh\rmto{\ke}{\vilist}, \vi, \txid, \opset)
    \end{array} \\
    \updateKV(\hh, \vi, \txid, \opset \uplus \Set{(\otW, \ke, \val)} )& \defeq &  
    \begin{array}[t]{@{}l}
        \texttt{let } \hh' = \hh\rmto{\ke}{ ( \hh(\ke) \lcat (\val, \txid, \emptyset) ) } \\
        \quad \texttt{in } \updateKV(\hh', \vi, \txid, \opset)
    \end{array} 
%
\end{rclarray}
\label{eq:updatekv}
\end{equation}
In the definition above, the operator $\lcat$ is used to append a new version at the tail of 
a list of versions. Given a list of versions $\vilist = \ver_0\cdots \ver_n$ and $i=0,\cdots,n$, 
$\vilist[i \mapsto \ver]$ denotes the list of versions $\vilist' = \ver_0 \cdots \ver_{i-1} \ver \ver_{i+1} \cdots 
\ver_{n}$. 
Note that, under the assumption that fingerprints contain at most one read and one write 
operation for each key, and assuming that $\txid \notin \hh$\footnote{
Here we are committing an abuse of notation, and let $\txid \in \hh$ if there exists a key 
$\ke$ and an index $i=0,\cdots, \lvert \hh(\ke) \rvert -1$ such that $\txid \in \{\WTx(\hh(\ke, i)\} 
\cup \RTx(\hh(\ke, i))$.}, 
the function $\updateKV$ is well defined. Furthermore, 
let $\nextTxId(\cl, \hh) \defeq
\Setcon{\txid_{\cl}^{n}}{\fora{m} \txid_{\cl}^{m} \in \hh \implies m < n }$, 
and suppose that $\txid \in \nextTxId(\cl, \hh)$ for some $\cl$;
then $\updateKV(\hh, \ver, \txid, \opset)$ produces a valid kv-store according to \cref{def:mkvs}.
In the following, we commit an abuse of notation and write $\updateKV(\hh, \vi, \cl, \opset)$ 
for the set $\{\hh' \mid \exsts { \txid \in \nextTxId(\cl, \mkvs) } \hh' = \updateKV(\hh, \vi, \txid, \opset)\}$.

Let us discuss how kv-stores in $\updateKV(\hh, \vi, \cl, \opset)$ are computed. 
Suppose that client \( \cl \) wants to commit a transaction.
First, we select a fresh transaction identifier $\txid \in \nextTxId(\cl, \hh)$ that we associate 
with the fingerprint to be committed into the kv-store. We choose any $\txid$ to be 
greater (w.r.t. the session order $\xrightarrow{\PO}$) than any transaction identifier 
of the form $\txid_{\cl}^{n}$ (indexed by the same client) appearing in $\hh$,
as to reflect the fact that $\txid$ is the most recent transaction executed by $\cl$.
Then for every read operation $(\otR, \ke, \stub)$ in $\opset$,
the set of transactions reading the  $\max_{<}(\vi(\ke))$-th version of $\ke$ is extended with $\txid$.
Because a transaction reads the values of keys from the atomic 
snapshot determined by the view of the client, the version read by $\txid$ for key $\ke$ 
corresponds to $\hh(\ke, \max_{<}(\vi(\ke)))$.
For every write operation $(\otW, \ke, \val)$ in fingerprint $\opset$, 
the list of versions $\mkvs(\ke)$ is extended with a new version $(\val, \txid, \emptyset)$, 
denoting that $\txid$ is responsible for creating this version which has no readers as of yet. 
Note that the assumption that versions of 
a given key are totally ordered and consistent with the order in which 
transactions commit is standard \cite{adya,framework-concur,seebelieve}. 

The \cref{prop:updatekv.comm} captures that
if two different clients $\cl_1$ and $\cl_2$ commit transactions 
whose fingerprints $\opset_1$ and $\opset_2$ do not contain a write 
to the same key, then the order in which the updates are executed is 
not relevant. 
\begin{proposition}
\label{prop:updatekv.comm}
\label{prop:swap-update}
Let $\hh \in \HisHeaps$, $\vi_1, \vi_2 \in \Views(\hh)$ and let $\cl_1, \cl_2 \in \Clients$ 
be such that $\cl_1 \neq \cl_2$. 
Let also $\opset_1, \opset_2 \in \powerset{\Ops}$ be such that 
whenever $(\otW, \ke, \_) \in \opset_1$ for some key $\ke$, then 
$(\otW, \ke, \val) \notin \opset_2$ for all $\val \in \Val$. Then 
\[
\begin{array}{l}
\{ \updateKV(\hh_1, \vi_2, \cl_2, \opset_2) \mid \hh_1 \in \updateKV(\hh, \vi_1, \cl_1, \opset_1)\} = \\
\{ \updateKV(\hh_2, \vi_1, \cl_1, \opset_1) \mid \hh_2 \in \updateKV(\hh, \vi_2, \cl_2, \opset_2)\}\\
\end{array}
\]
\end{proposition}
\begin{proof}
    See \cref{sec:comm-updatekv}.
\end{proof}

\mypar{Execution Tests and Specification of Consistency Models.}
Formally, a \emph{consistency model} $\CMs$ is a 
set of kv-stores. Each $\hh \in \CMs$ represents a possible scenario that 
can be obtained as a result of multiple clients committing transactions. 
For example, \emph{serialisability} can be described as the set 
of kv-stores for which it is possible to recover a total schedule of transactions, 
such that each read operation on key $\ke$ fetches its value from the 
most recent write on the same key \cite{??????}.
In this sense, the kv-store $\hh$ from \cref{fig:hheap-a} is not serialisable: 
transaction $\txid_1$ reads the initial version carrying value $\val'_0$ for key $\ke_{2}$, 
and installs a new version of $\ke_{2}$ carrying value $\val_1$. The transaction $\txid_2$ 
reads the initial version carrying value $\val'_0$, and therefore, 
cannot be scheduled after $\txid_1$. Similarly, $\txid_2$ cannot be scheduled after $\txid_1$.

To specify consistency models we introduce the notion of \emph{execution tests}. 
\begin{definition}
\label{def:execution.test}
\emph{An execution test} is a set of tuples $\ET \subseteq \HisHeaps \times \Views \times \powerset{\Ops} \times \Views$ 
such that whenever $(\hh, \vi, \opset, \vi') \in \ET$, then 
\textbf{(i)} for any read operations $(\otR, \ke, \val) \in \opset$ then $\hh(\ke, \max(\vi(\ke))) = \val$, 
and \textbf{(ii)}  $\forall \ke. \; \vi(\ke) \neq \vi'(\ke) \implies ( (\otR, \ke, \_) \in \opset \vee (\otW, \ke, \_)) \in \opset)$.
%\textbf{(iii)} $\forall \opset' \subseteq \opset.\; (\hh, \vi, \opset', \vi') \in \ET$
\end{definition}
\sx{The definition has a problem that for the subset \( \f'\) the post-view \( \vi' \) might point to an undefined version, I also thing it should satisfy \( \fora{\vi''} \vi \sqsubseteq \vi'' \implies (\hh, \vi, \opset', \vi'') \)}.
\ac{I removed this condition, as I do not think that it was used anywhere. The new definition requires that 
you cannot change the view for keys that you do not read nor write.}
\sx{The current \CP will not satisfy the \textbf{(ii)}. }
Let $\ET$ be an execution test, and then $(\hh, \vi, \opset, \vi') \in \ET$ means 
that a client whose view over the kv-store $\hh$ is $\vi$, can commit a 
transaction whose fingerprint is $\opset$; as a result of this operation, the 
view of the client must be updated to $\vi'$.
Henceforth, we adopt the 
more suggestive notation $\ET \vdash (\hh, \vi) \triangleright \opset: \vi'$ 
in lieu of $(\hh, \vi, \opset, \vi') \in \ET$.

Execution tests induce \emph{consistency models} \( \CMs(\ET) \) as defined in \cref{def:cm}.
\begin{definition}
\label{def:reduction}
Let $\cl$ be a client and $\opset$ be a fingerprint. 
An \emph{action} has either the form $(\cl, \varepsilon)$, 
or $(\cl, \opset)$. 
Let $\mathsf{Act}$ be the set of actions.
Given an execution test $\ET$, for any action $\alpha \in \Act$ the action-labelled 
relation $\xrightarrowtriangle{\alpha}_{\ET} \subseteq \Confs \times \Confs$ is defined
as the smallest relation such that:
\begin{itemize}
\item $\forall \vi, \vi', \cl, \hh, \viewFun.\; \viewFun(\cl) = \vi \wedge \vi \sqsubseteq \vi' \implies (\hh, \viewFun) \xrightarrowtriangle{(\cl, \varepsilon)}_{\ET} 
(\hh, \viewFun\rmto{\cl}{\vi'})$, 
\item $\forall \vi, \vi', \cl, \opset, \hh, \hh', \viewFun.\; \viewFun(\cl) = \vi \wedge (\ET \vdash (\hh, \vi) \triangleright \opset: \vi') \wedge 
\hh' \in \updateKV(\hh, \vi, \cl, \opset) \implies (\hh, \viewFun) \xrightarrowtriangle{(\cl, \opset)}_{\ET} (\hh', \viewFun\rmto{\cl}{\vi'})$.
\end{itemize}
Such relations take the name of $\ET$-reductions, or simply reductions.
\end{definition}
Given an execution test $\ET$, sequences of $\ET$-reductions of the form $\conf_{0} \xrightarrowtriangle{\alpha_{0}}_{\ET} \cdots 
\xrightarrow{\alpha_{n-1}} \conf_{n}$ take the name of \emph{$\ET$-traces}.
\begin{definition}
\label{def:cm}
Given an execution test $\ET$, the set of configurations induced by $\ET$ is given by:
\[
\Confs(\ET) \defeq \Setcon{ \conf}{ \exsts{\conf_0} \conf_0 \text{ is initial } \wedge \conf_0 \xrightarrowtriangle{\stub}_{\ET} \cdots \xrightarrowtriangle{\stub}_{\ET} \conf }
\]
The \emph{consistency model} induced by $\ET$ is defined as:
\( 
\CMs(\ET) \defeq \Setcon{ \hh }{ (\hh, \stub) \in \Confs(\ET) }
\)
\end{definition}
Thus, consistency models are computed from execution tests by closing the set of initial kv-stores with 
respect to two operations: \textbf{(i)} replacing the view of a client on the kv-store with a more up-to-date one, 
and \textbf{(ii)} committing the effects of a transaction. 

Consistency models induced by execution tests are monotonic in the following sense.
\begin{proposition}
\label{prop:mono-et}
Let $\ET_1 \subseteq \ET_2$. Then $\CMs(\ET_1) \subseteq \CMs(\ET_2)$.
\end{proposition}
\begin{proof}
    See \cref{sec:mono-et}.
\end{proof}

For technical reasons, it will be convenient to adopt a reduction strategy for inferring kv-stores induced by an 
execution test: such an execution strategy require that clients only commit transactions with non-empty fingerprints, 
and a client updates its view only immediately before committing a transaction. 
The next proposition states that all kv-stores induced by an execution test $\ET$ can be 
obtained via a sequence of reductions that adhere to the reduction strategy outlined above. 
\begin{definition}
Let $\ET$ be an execution test. The $\ET$-trace
\[
\conf_0 \xrightarrowtriangle{\alpha_0}_{\ET} \conf_1 \xrightarrowtriangle{\alpha_1}_{\ET} \cdots \xrightarrowtriangle{\alpha_{2n}}_{\ET} \conf_{2n + 1}
\]
is in \emph{normal form} if \textbf{(i)} $\conf_0$ is initial, and 
\textbf{(ii)} $\forall i=0,\cdots, n$ there exists a client $\cl_i$ and set of operations $\opset_{i}$ such that 
$\alpha_{2 \cdot i} = (\cl_{i}, \varepsilon)$, and $\alpha_{2 \cdot i + 1}$ is defined and equal to $(\cl_{i}, \opset_{i})$ where \( \f_i \neq \emptyset \).
\end{definition}

\begin{proposition}
\label{prop:et.normalform}
Let $\ET$ be an execution test, and suppose that $\hh \in \CMs(\ET)$. Then there exists a $\ET$-trace  
\[
(\hh_0, \viewFun_0) \xrightarrowtriangle{\stub}_{\ET} \cdots \xrightarrowtriangle{\stub}_{\ET} (\hh_n, \viewFun_{n})
\]
that is in normal form, and such that $\hh_{n} = \hh$.
\end{proposition}
\begin{proof}
    See \cref{sec:normal-form-exist}.
\end{proof}

%Note that specifications of consistency models using execution tests 
%are compositional in the following way: 
%
%\begin{proposition}
%\label{prop:et.compositional}
%Let $\ET_1, \ET_2$ be two execution tests. Then $\CMs(\ET_1 \cap \ET_2) = \CMs(\ET_1) \cap \CMs(\ET_2)$.
%\end{proposition}
%
%\begin{example}
%Let $\ET_{\top} = \HisHeaps \times \viewSet \times \powerset{\Ops} \times \viewSet$ 
%be the most permissive execution test. Note that \cref{prop:et.compositional} ensures 
%that, for any execution test $\ET$, then $\CMs(\ET) \subseteq \CMs(\ET_{\top})$. 
%Consider the kv-store $\hh$ from \cref{fig:hheap-a}. 
%If we assume that $\val'_0 = \val_0$, and $\val_0$ is the initial value of kv-stores, 
%then $\hh$ is included in $\CMs(\ET_{\top})$. 
%In fact, let $\opset_1 = \{(\otR, \ke_2, \val'_0), (\otW, \ke_1, \val_1)\}$, 
%$\opset_2 = \{(\otR, \ke_1, \val_0), (\otW, \ke_2, \val'_1)\}$, and 
%$\vi = [\ke_1 \mapsto 0, \ke_2 \mapsto 0]$. Assuming that 
%$\hh_0$ is the initial kv-store with $\dom(\hh_0) = \{ \ke_1, \ke_2\}$, 
%then $\updateKV(\updateKV(\hh_0, \vi, \_, \opset_1), \vi, \_, \opset_2)$ 
%generates exactly the kv-store $\hh$, provided that transaction identifiers 
%$\txid, \txid'$ are consistent with the clients that committed $\opset_1, \opset_2$, 
%respectively.
%\end{example}
%\ac{Looking at compositionality gets us into problems that we will never solve before 
%the deadline.}

\mypar{Compositionality for Execution Tests.} 
A desirable property that one would request from execution 
test is compositionality: the consistency model induced by 
a composite execution test can be recovered from the consistency 
models generated by each execution test: that is, 
\[ 
\forall \ET_1, \ET_2. \CMs(\ET_1 \cap \ET_2) = \CMs(\ET_1) \cap \CMs(\ET_2).
\]
Unfortunately, this property is not satisfied by execution tests in their 
most general setting, as the following example shows: 
\begin{example}
\label{ex:noncompositional.et}
Define the following terms: 
\[
\begin{array}{lcl}
\hh_0 &=& [\ke_1 \mapsto (\val_0, \txid_0, \emptyset) , \ke_2 \mapsto (\val_0, \txid_0, \emptyset)]\\
\hh_1 &=& \big[\ke_1 \mapsto \big( (\val_0, \txid_0, \emptyset) \lcat (\val_2, \txid_{\cl_1}^{1}, \emptyset)\big) , \ke_2 \mapsto (\val_0, \txid_0, \emptyset) \big]\\
\hh_2 &=& \big[\ke_1, \mapsto (\val_0, \txid_0, \emptyset), \ke_2 \mapsto \big( (\val_0, \txid_0, \emptyset) \lcat (\val_2, \txid_{\cl_2}^{1}, \emptyset) \big) \big]\\
\hh_3 &=& \big[\ke_1 \mapsto \big( (\val_0, \txid_0, \emptyset) \lcat (\val_2, \txid_{\cl_1}^{1}, \emptyset)\big), 
                         \ke_2 \mapsto \big( (\val_0, \txid_0, \emptyset) \lcat (\val_2, \txid_{\cl_2}^{1}, \emptyset) \big) \big]\\
&&\\
\vi_0 &=& [\ke_1 \mapsto \{0\}, \ke_2 \mapsto \{0\}]\\
\viewFun_0 &=& [\cl_1 \mapsto \vi_0, \cl_2 \mapsto \vi_0]\\
&&\\
\ET_1 &\vdash& (\hh_0, \vi_0) \triangleright \{(\otW, \ke_1, \val_1)\} : \vi_0\\
\ET_1 &\vdash& (\hh_1, \vi_0) \triangleright \{(\otW, \ke_2, \val_2)\} : \vi_0\\
&&\\
\ET_2 &\vdash& (\hh_0, \vi_0) \triangleright \{(\otW, \ke_2, \val_2)\} : \vi_0\\
\ET_2 &\vdash& (\hh_2, \vi_0) \triangleright \{(\otW, \ke_1, \val_1)\} : \vi_0.
\end{array}
\]
There are no further constraints on $\ET_1, \ET_2$.
For $\ET_1$ and $\ET_2$, we have that 
\[
\begin{array}{l}
(\hh_0, \viewFun_0) \xrightarrowtriangle{(\cl_1, \{(\otW, \ke_1, \val_1)\})}_{\ET_1} 
(\hh_1, \viewFun_0) \xrightarrowtriangle{(\cl_2, \{(\otW, \ke_2, \val_2)\})}_{\ET_1} (\hh_3, \viewFun_0), \\
(\hh_0, \viewFun_0) \xrightarrowtriangle{(\cl_2, \{(\otW, \ke_2, \val_2)\})}_{\ET_2} 
(\hh_2, \viewFun_0) \xrightarrowtriangle{(\cl_1, \{(\otW, \ke_1, \val_1)\})}_{\ET_2} (\hh_3, \viewFun_0).\\
\end{array}
\] 
Therefore, we have that $\hh_3 \in \CMs(\ET_1) \cap \CMs(\ET_2)$. On the other hand, it is immediate 
to observe that $\ET_1 \cap \ET_2 = \emptyset$, and therefore $\hh_3 \notin \CMs(\ET_1 \cap \ET_2)$.
\end{example}
The reason why compositionality fails, for the execution tests of \cref{ex:noncompositional.et}, 
is that both the execution tests $\ET_1, \ET_2$ require that the fingerprints 
$\{(\otW, \ke_1, \_)\}, \{(\otW, \ke_2, \_)\}$ commit in different order: in $\ET_1$, the write to $\ke_1$ must commit 
before the write to $\ke_2$, and vice versa for $\ET_2$. On the other hand, 
because the two fingerprints above do not write to the same key, 
the order in which they are committed should not be relevant: by changing the order 
in which different clients commit such fingerprints to a kv-store, the result stays the same. 
\begin{definition}
Two triples $(\cl_1, \opset_1)$ and $(\cl_2, \opset_2)$ are 
conflicting if either $\cl_1 = \cl_2$, or there exists a key $\ke$ such that 
$(\otW, \ke, \_) \in \opset_1, (\otW, \ke, \_) \in \opset_2$. 

An execution test is $\ET$ is \emph{commutative} if, whenever $(\cl_1, \vi_1, \opset_1)$, 
$(\cl_2, \vi_2, \opset_2)$ are non-conflicting, and $\vi_1, \vi_2 \in \Views(\hh_0)$,  
then for any $\hh_0, \hh', \viewFun, \viewFun'$ we have that 
\[
\begin{array}{lr}
(\hh_0, \viewFun) \xrightarrowtriangle{(\cl_1, \opset_1)}
\_ \xrightarrowtriangle{(\cl_2, \opset_2)}_{\ET} (\hh', \viewFun') &\implies \\
(\hh_0, \viewFun) \xrightarrowtriangle{(\cl_2, \opset_2)}_{\ET} 
\_ \xrightarrowtriangle{(\cl_1, \opset_1)}_{\ET} (\hh', \viewFun')
\end{array}
\]
\end{definition}
\ac{Note that the views are not part of actions anymore. Furthermore, the definition 
of $\ET$-reduction has been changed, so that no view shifts can be made prior to 
committing a transaction, in the reductions above.}

Requiring execution tests to be commutative is a necessary step for ensuring 
that the specification of consistency models are compositional. However, it 
is not sufficient. The next example shows how compositionality fails 
for commutative execution tests. 

\begin{example}
\label{ex:noblindwrites}
For any $n \in \Nat$, let $[n] = \{0,\cdots, n\}$.
Consider the execution tests $\ET_1, \ET_2$ defined below: 
\[
\begin{array}{lcl}
\ET_1 \vdash (\hh, \vi) \triangleright \opset : \vi' &\iff& 
\forall \ke.\;(\otW, \ke, \_) \in \opset \implies \vi(\ke) = \vi'(\ke) = [0]\\
\ET_2 \vdash (\hh, \vi) \triangleright \opset : \vi'(\ke) &\iff& 
\forall \ke. \;(\otW, \ke, \_ ) \in \opset \implies \vi(\ke) = [ \lvert \hh(\ke) \rvert - 1] \wedge \vi'(\ke) = [\lvert \hh(\ke) \rvert ] \\
\end{array}
\]
It is immediate to observe that both $\ET_1$ and $\ET_2$ are commutative. However, 
consider the kv-store $\hh_2 = [\ke \mapsto (\val_0, \txid_0, \emptyset) \lcat (\val_1, \txid_{\cl}^1, \emptyset) \lcat (\val_2, \txid_{\cl}^2, \emptyset)]$. 
We have that $\hh \in \CMs(\ET_1)$ and $\hh \in \CMs(\ET_2)$.
Let in fact $\hh_1 = [\ke \mapsto (\val_0, \txid_0, \emptyset) \lcat (\val_1, \txid_{\cl}^1, \emptyset)]$, $\vi_{i} = [\ke \mapsto [i] ]$.
\ac{Seriously, square brackets are being used everywhere (though all of this is standard notation. Maybe $\langle n \rangle$ for $\{0,\cdots, n\}$ is 
a better notation?
\sx{  \( \ke \mapsto \langle n \rangle \) is cool. }
}
We have the following sequences of reductions: 
\[
\begin{array}{l}
(\hh_0, \vi_0) \xrightarrowtriangle{(\cl, \{(\otW, \ke, \val_1)\})}_{\ET_1} 
(\hh_1, \vi_0) \xrightarrowtriangle{(\cl,\{(\otW, \ke, \val_2)\})}_{\ET_1} (\hh_2, \vi_0)\\
(\hh_0, \vi_0) \xrightarrow{(\cl, \{(\otW, \ke, \val_1)\})}_{\ET_2} (\hh_1, \vi_1) \xrightarrow{(\cl, \{(\otW, \ke, \val_2)\})}_{\ET_2} 
(\hh_2, \vi_2)
\end{array}
\]
On the other hand, we can observe that $\hh_2 \notin \CMs(\ET_1 \cap \ET_2)$. $\ET_1$ allows a client to 
commit a transaction if its view only includes the initial version of each key it writes. $\ET_2$ allows a client 
to commit a transaction when its view include all the versions for each key it writes. In $\ET_1 \cap \ET_2$ 
a client can commit a transaction only if the initial version of each key it writes is also the only version in the kv-store: 
as a result, $\CMs(\ET_1 \cap \ET_2)$ never contains a  kv-stores $\hh$ such that $\hh(\ke) > 1$ for some key $\ke$; 
in particular, $\hh_2 \notin \CMs(\ET_1 \cap \ET_2)$.
\end{example}
\ac{Two possible reasons why compositionality fails: because of blind writes, or because the test $\ET_1$ hinders progress, 
i.e. it is not possible to replace a view with a more up-to-date one to enable progress. We must choose which assumption 
we make on the consistency model.}

One reason why compositionality fails in \cref{ex:noblindwrites} is that the execution tests $\ET_1$ and $\ET_2$ do not contain 
any information about the the views that client $\cl$ used to commit the transactions $\txid_{\cl}^1, \txid_{\cl}^2$. 
To solve this problem we adopt the \emph{no blind writes} assumption, that requires that a client never commits 
a transaction that writes a key, without reading such a key beforehand. Many implementations of consistency models 
in distributed key-value stores respect the no blind writes assumption. 

\begin{definition}
\label{def:noblidwrites}
An execution test $\ET$ has \emph{no blind writes} if, whenever $\ET \vdash (\hh, \vi) \triangleright \opset \cup \{(\otW, \ke, \_)\} : \vi'$, 
then $(\otR, \ke, \_) \in \opset$.
\end{definition}

\begin{definition}
\label{def:et-minimum-footprint}
An execution test $\ET$ has \emph{minimum footprints} if for any \( \hh, \vi, \vi',\vi'', \f \),
\[
\begin{array}{@{}l@{}}
    ( \fora{ \ke} (\stub, \ke, \stub) \in \f \implies \vi(\ke) = \vi'(\ke) ) \land {} \\
    \quad \ET \vdash (\hh, \vi) \triangleright \opset : \vi'' \implies \ET \vdash (\hh, \vi') \triangleright \opset : \vi''
\end{array}
\]
\end{definition}

\begin{definition}
\label{def:et-continuous-postview}
An execution test $\ET$ has \emph{continuous post-views} if for any \( \hh, \vi, \vi',\vi'', \f \),
\[
\begin{array}{@{}l@{}}
    \quad \ET \vdash (\hh, \vi) \triangleright \opset : \vi' \land \vi' \sqsubseteq \vi'' \implies \ET \vdash (\hh, \vi) \triangleright \opset : \vi''
\end{array}
\]
\end{definition}

\begin{theorem}                                                                            
\label{thm:et-comm}                          
Let $\ET_1, \ET_2$ be two execution tests has no blind writes, minimum footprints and continuous post-views.
If $\ET_1$ is commutative, 
then $\CMs(\ET_1 \cap \ET_2) = \CMs(\ET_1) \cap \CMs(\ET_2)$. 
Furthermore, if $\ET_1, \ET_2$ are commutative, then $\ET_1 \cap \ET_2$ 
is commutative.
\end{theorem}
\begin{proof}
    See \cref{sec:et-comm}.
\end{proof}



\subsection{Programming language}

For simplify, \emph{a program} \( \prog \) contains fixed numbers of top level threads and there is no dynamic fork and join.
Each thread has a unique thread identifier \( \thid \in \ThreadID \) and associated \emph{commands}.
The \emph{commands}, ranged over by $\cmd$, are defined by an inductive grammar comprising the standard constructs of $\pskip$, sequential composition ($\cmd; \cmd$), non-deterministic choice ($\cmd+\cmd$) and loops ($\cmd^*$).
To simulate conditional branching and loops, we have primitive commands assume (\( \passume{\expr}\)) and assignment (\( \passign{\var}{\expr} \)), where \( \var \) denotes stack variable and \( \expr \) denotes arithmetic expressions which have no side effect.
Additionally, the programming language contains the \emph{transaction} construct $\ptrans{\trans}$ denoting the \emph{atomic} execution of the transaction $\trans$. 
The atomicity guarantees the execution are dictated by the underlying consistency model.
\emph{Transactions}, ranged over by $\trans$, are defined by a similar inductive grammar comprising $\pskip$, non-deterministic choice, loops and sequential composition, as well as primitive constructs, ranged over by \( \transpri \), including assignment (\( \passign{\var}{\expr}\)), assume (\( \passume{\expr}\)), lookup (\( \pderef{\expr}{\expr}\)), mutation (\( \pmutate{\expr}{\expr}\)) and return(\( \preturn{\expr}\)). 
Transactions do \emph{not} contain the \emph{parallel} composition construct ($\ppar$) as they are to be executed atomically.
We assume a valid transactions codes must have the only return at the end.
For better presentation, sometime we omit the default return zero \( \preturn{0} \).
%Transactions can only assign to their own variables, namely transaction variables (\defref{def:program_values}), but it can read from both the thread and transaction stacks.

\begin{defn}[Program values]
\label{def:program_values}
Assume a countably infinite set of \emph{addresses}, $\addr \in \Addr$. The set of \emph{program values} is $\val \in \Val \eqdef \Nat \cup \Addr$, where $ \nat \in \Nat$ denotes the set of natural numbers.
\end{defn}

\begin{defn}[Programming language]
\label{def:language}
A \emph{program}, $\prog \in \Programs$, is a partial function from thread identifiers to commands.
Assuming the set of \emph{variables} \( \var \in \Vars \), the commands \( \cmd \in \Commands \) are defined by the following grammar,
\[
    \begin{rclarray}
    \cmd & ::= &
        \pskip \mid 
        \passign{\thvar}{\expr} \mid
        \passume{\expr} \mid
        \ptrans{\trans} \mid 
        \cmd \pseq \cmd \mid 
        \cmd \pchoice \cmd \mid 
        \cmd \prepeat 
    \end{rclarray}
\]
The $\trans \in \Transactions$ in the grammar above denotes a \emph{transaction} defined by the following grammar.
The transaction codes \( \ptrans{\trans} \) satisfy a well-form condition that there is exactly a return at the end, \ie \( \ptrans{\trans} \iff \exsts{\trans', \expr} \trans \equiv ( \trans' \pseq \preturn{\expr} )  \land \pred{noRet}{\trans'} \).
\[
    \begin{rclarray}
        \transpri & ::= &
        \pass{\txvar}{\expr} \mid
        \pderef{\txvar}{\expr} \mid
        \pmutate{\expr}{\expr} \mid
        \passume{\expr} \mid
        \preturn{\expr} \\
        \trans & ::= &
        \pskip \mid
        \transpri \mid 
        \trans \pseq \trans \mid
        \trans \pchoice \trans \mid
        \trans\prepeat
    \end{rclarray}
\]
The $\expr \in \Expressions$ denotes an \emph{arithmetic expression} defined by the grammar below with $\val \in \Val$ (\defin\ref{def:program_values}),
\[
    \begin{rclarray}
        \expr & ::= &
        \val \mid
        \var \mid
        \expr + \expr \mid
        \expr \times \expr \mid
        \dots 
    \end{rclarray}
\]
\end{defn}

\subsection{Local/Transaction Semantics}

Each thread has it own stack, where local variables are stored.
A thread can access the stack inside or outside transactions.
When a transaction start, it determine a local heap from the current state of database and local view, which we will explain the process later.
Yet this local heap does not affect outside world, instead we use \emph{operation} to connect a transaction to outside world.
A operation could be read or write to a address with a value.
Intuitively, when a transaction is about to commit, it will have \emph{a set of operations} containing the first read and last write for each address.
Because a transaction is executed atomically, all the intermediate steps are not observable from the outside world.

we introduce \emph{a well-formed set of operations} \( \opset \in \Opsets\) that is a subset of operations in which there are at most one read and one write for each address.
The composition, then, is defined as set disjointed union as long as the result is well-formed.
To help us write down the semantics, we assume a binary operator \( \opset \addO \op \) that specifies the effects of adding a new operation \( \op \) to the set \( \opset \).
If the new operation is a read, for example \(\otR, \addr, \val\) that \( \addr \) is the address and \( \val\) is the value, and there is no other operation related to the same address, this new read operation will be included in the set.
Note that if there is already a write in the set, this mean the following reads are all local.
Meanwhile, if the new operation is a write, it will overwrite all preview write operations to the same address.

\sx{Single stack for simplicity. Design choice, total stack vs partial stack, make sure it is consistent.}

\begin{defn}[Stacks]
\label{def:stacks}
A \emph{stack} is a partial function from variables \( \Vars \) (\defref{def:language}) to values program values \( \Val \) (\defref{def:program_values}), this is \( \stk \in \Stacks \defeq \Vars \parfun \Val \).
\end{defn}

%{ \color{gray}
%\begin{defn}[Stacks]
%Given the program values (\defref{def:program_values}) and a set of \emph{transaction variables} \( \TxVars \defeq \Set{\txvar, \dots}\), a \emph{transaction stack} is a partial function from transaction variables to values \( \txstk \in \TxStacks \defeq \TxVars \parfun \Val \).
%Similarly, assuming a set of \emph{thread variables} \( \ThdVars \defeq \Set{\thvar, \dots}\), a \emph{thread stack} is defined as \( \thstk \in \ThdStacks \defeq \ThdVars \parfun \Val \).
%Then, the set of \emph{stacks} is defined as the union of transaction stacks and thread stacks \( \stk \in \Stacks \eqdef \TxStacks \uplus \ThdStacks \).
%\end{defn}
%}

\begin{definition}[Heaps]
\label{def:heaps}
Given the sets of program values $\Val$  and addresses \( \Addr\)  (\defin\ref{def:program_values}), the set of \emph{heaps} is: $\h \in \Heaps \eqdef \Addr \parfinfun \Val$.
The \emph{heap composition function}, $\composeH: \Heaps \times \Heaps \parfun \Heaps$, is defined as $\composeH \eqdef \uplus$, where $\uplus$ denotes the standard disjoint function union. The \emph{ heap unit element} is $\unitH \eqdef \emptyset$, denoting a function with an empty domain.
The \emph{partial commutative monoid of  heaps} is $(\Heaps, \composeH, \{\unitH\})$.
\end{definition}

\begin{defn}[Evaluation of expression]
Given a stack $\stk \in \Stacks$ (\defin\ref{def:stacks}), the \emph{arithmetic expression evaluation} function, $\evalE[(.)]{.}:\Expressions \times \Stacks \parfun \Val$, is defined inductively over the structure of expressions as follows: 
%
\[
    \begin{rclarray}
        \evalE{\val} & \defeq & \val \\
        \evalE{\var} & \defeq & \stk(\var) \\
        \evalE{\expr_{1} + \expr_{2}} & \defeq & \evalE{\expr_{1}} + \evalE{\expr_{2}} \\
        \evalE{\expr_{1} \times \expr_{2}} & \defeq & \evalE{\expr_{1}} \times \evalE{\expr_{2}} \\
        \dots & \eqdef & \dots \\
    \end{rclarray}
\]
\end{defn}


\begin{defn}[Operation and valid operations]
\label{def:transaction-event}
\label{def:transactions}
Assume a set of \emph{operations tags}, a \emph{transaction operation} \( \op \in \Ops \) is a tuple of an operation tag, an address and a value.
The tuple represents either read or write of the address.
The \emph{operation tags} \( \etR \) and \( \etW \) correspond to read and write respectively.
\[
\begin{rclarray}
\OTags & \defeq & \Set{\otR, \otW} \\
\op \in \Ops & \defeq  & \OTags \times \Addr \times \Val
\end{rclarray}
\]
\emph{a well-formed set of operations}, \( \opset \in \Opsets \), is a subset of \( \Ops \) in which any two elements contain either different tags or different address.
\[
    \begin{rclarray}
        \Opsets & \defeq & \Setcon{\opset}{\opset \subseteq \Events \land \wfO{\opset} } \\
        \wfO{\opset} & \defeq & \for{\op, \op' \in \opset} \op\projection{1} \neq  \op'\projection{1} \lor \op\projection{2} \neq  \op'\projection{2}
    \end{rclarray}
\]
The unit element is \( \unitE \defeq \emptyset\) and the composition of two set of operations is only defined when the two sets contains disjointed addresses,
\[ 
\begin{rclarray}
    \opset \composeO \opset' & \defeq & 
    \begin{cases}
        \opset \uplus \opset' & \text{if } \opset\projection{2} \cap \opset'\projection{2} = \emptyset \\
        \text{undefined} & \text{otherwise}
    \end{cases}
\end{rclarray}
\]
A partial binary operation \( \addO \) adds a new operation to valid operations \( \opset \) that ensures the set contains the first read and last write.
For technical reason, if the right hand side is a special token \( \emptyop \), which represents no operation, the operations remains the same.
\[
\begin{rclarray}
    \opset \addO (\etR, \addr, \val) & \defeq & 
    \begin{cases}
        \opset \uplus \Set{(\etR, \addr, \val)} & (\stub, \addr, \stub) \notin \opset \\
        \opset &  \text{otherwise} \\
    \end{cases} \\
    \opset \addO (\etW, \addr, \val) & \defeq & \left( \opset \setminus \Set{(\etW, \addr, \stub)} \right) \uplus \Set{(\etW, \addr, \val)} \\
    \opset \addO \emptyop & \defeq & \opset \\
\end{rclarray}
\]
\end{defn}

\begin{lem}
The \( \addO \) operator preserves the well-form property.
\end{lem}

\azalea{
\label{comm:operations}
This can all be very simplified as follows. Each location is associated with a value and a set of tags in $\pset{\Set{\otR, \otW}}$, where $\emptyset$ means no finger print and the rest have the obvious meaning. 
This way you won't need the well-formedness condition or the lemma. 
%
\begin{defn}[Operations]
An \emph{operation map} \( \opset \in \Opsets \eqdef \Addr \fm \Val \times \pset{\Set{\otR, \otW}} \) is function associating each address with its tags.
The \emph{operation tags} \( \etR \) and \( \etW \) correspond to read and write respectively.

The unit element is the function with empty domain, i.e.\ $\emptyset$. 

The composition of two operation maps is defined as the standard disjoint function union: 
\[ 
\begin{rclarray}
    (\opset \composeO \opset')(a) & \defeq & 
    \begin{cases}
        \opset(a)  & \text{if } a \in \dom(\opset) \land a \not\in \dom(\opset') \\
        \opset'(a)  & \text{if } a \in \dom(\opset') \land a \not\in \dom(\opset) \\
        \text{undefined} & \text{otherwise}
    \end{cases}
\end{rclarray}
\]
%
A partial binary operation \( \addO \) updates an operation map $\opset$ with a tuple $(t, \addr, \val)$ where $t \in \Set{\etR, \etW}$ as follows:
\[
\begin{rclarray}
    \opset \addO (\etR, \addr, \val) & \defeq & 
    \begin{cases}
        \opset[\addr \mapsto (\val, \{\etR\})] & \opset(\addr) = (\val, \emptyset) \\
        \opset &  \text{otherwise} \\
    \end{cases} \\
%
%
	\opset \addO (\etW, \addr, \val) & \defeq & 
    \begin{cases}
        \opset[\addr \mapsto (\val, S \cup \{\etW\})] & \opset(\addr) = (-, S) \\
        \opset &  \text{otherwise} \\
    \end{cases} \\
%      
    \opset \addO \emptyop & \defeq & \opset \\
\end{rclarray}
\]
%
For technical reason, if the right hand side is a special token \( \emptyop \), which represents no operation, the operations remains the same.
\end{defn}
}


%We define a labelled transition system for primitive transaction commands where the labels are commands.
The operational semantics for transactions \(\trans\) is defined with respect to a configuration of the form \((\stk, \h, \opset)\) comprising a stack, a (local) heap and a set of operations.
The operations are those that might affect other transactions, meaning the first read and the last write for each address.
We first define a transition between pairs of stacks and heaps for the primitive commands \( \trans_{p}\).
We also define its operation by \( \funcn{op} \) function, which denotes the contribution of the primitive command that might observed by the external environment, \ie transactions from other threads.
The \( \funcn{op} \) extracts the read or write operation from loop-up and mutation respectively, otherwise returns \( \emptyop \).
The semantics for non-deterministic choices, sequential compositions, and loops have the expected behaviours.


\begin{defn}[Transaction operational semantics]
Given the set of stacks \( \Stacks \) (\defref{def:stacks}), heaps \( \Heaps \) (\defin\ref{def:heaps}) and transactions \( \Transactions \) (\defin\ref{def:language}), the \emph{operational semantics of transactions}, 
\[
\begin{rclarray}
\toL & : & \ThdStacks \times \\
& & \quad ((\TxStacks \times \Heaps \times \Opsets) \times \Transactions) \times ((\TxStacks \times \Heaps \times \Opsets) \times \Transactions)
\end{rclarray}
\]
is given in \fig\ref{fig:transaction_semantics}.
Note that arithmetic expression evaluation \( \evalE{\expr} \) is defined in \defref{def:language} and the \( \addO \) operators is defined in \defref{def:transactions}.
\end{defn}

\begin{figure}[!t]
\hrule\vspace{5pt}
\[
\begin{array}{@{} r c l r  c l @{}}
    (\stk, \h) & \toLTS{\passign{\var}{\expr}} & (\stk\rmto{\var}{\evalE{\expr}}, \h) & \func{op}{\stk, \h, \passign{\var}{\expr}} & \defeq & \emptyop \\
    (\stk, \h) & \toLTS{\pderef{\var}{\expr}} & (\stk\rmto{\var}{\h(\evalE{\expr})}, \h) & \func{op}{\stk, \h, \pderef{\var}{\expr}} & \defeq & (\etR, \evalE{\expr}, \h(\evalE{\expr})) \\
    (\stk, \h) & \toLTS{\pmutate{\expr_{1}}{\expr_{2}}} & (\stk, \h\rmto{\evalE{\expr_{1}}}{\evalE{\expr_{2}}}) & \func{op}{\stk, \h, \pmutate{\expr_{1}}{\expr_{2}}} & \defeq & (\etW, \evalE{\expr_{1}}, \evalE{\expr_{2}}) \\
    (\stk, \h) & \toLTS{\passume{\expr}} & (\stk, \h) \text{ where } \evalE{\expr} = 0 & \func{op}{\stk, \h, \passume{\expr}} & \defeq & \emptyop \\
    (\stk, \h) & \toLTS{\preturn{\expr}} & (\stk\rmto{\ret}{\evalE{\expr}}, \h) & \func{op}{\stk, \h, \preturn{\expr}} & \defeq & \emptyop \\
\end{array}
\]
\hrule\vspace{5pt}
\[	
    \infer[\rl{TPrimitive}]{%
        \vdash (\stk, \h, \opset) , \transpri \ \toL \  (\stk', \h', \opset \addO \op) , \pskip
    }{%
        (\stk, \h) \toLTS{\transpri} (\stk', \h')
        && \op = \func{op}{\stk, \h, \transpri}
    }
\]

\[
    \infer[\rl{TChoice}]{%
        \stk \vdash (\txstk, \h, \opset) , \trans_{1} \pchoice \trans_{2} \ \toL \  (\stk, \h, \opset) , \trans'
    }{
        \trans' \in \Set{\trans_{1}, \trans_{2}}
    }
\]

\[
    \infer[\rl{TLoop}]{%
        \stk \vdash (\txstk, \h, \opset),  \trans\prepeat \ \toL \  (\stk, \h, \opset), \pskip \pchoice (\trans \pseq \trans\prepeat)
    }{}
\]


\[
    \infer[\rl{TSeqSkip}]{%
        \stk \vdash (\txstk, \h, \opset), \pskip \pseq \trans \ \toL \  (\stk, \h, \opset), \trans
    }{%
    }
\]

\[
    \infer[\rl{TSeq}]{%
        \stk \vdash (\txstk, \h, \opset), \trans_{1} \pseq \trans_{2} \ \toL \  (\stk', \h', \opset'), \trans_{1}' \pseq \trans_{2}
    }{%
        \stk \vdash (\txstk, \h, \opset), \trans_{1} \ \toL \  (\stk', \h', \opset'), \trans_{1}'
    }
\]

\hrule\vspace{5pt}
\caption{The transaction operational semantics}
\label{fig:transaction_semantics}
\end{figure}


%\subsection{Operational Semantics of Programs}
\label{sec:prog-semantics}
\azalea{I rewrote most of this section.}
%\sx{
%For weak consistency models, or weak isolation levels as a common term used in database community, a thread is not necessary to work on the up-to-date version of a database in exchange for better performance. 
%Even in a single machine database, a thread running under weak consistency model can make less synchronisation with the hard drivers and other running threads, which means the thread could observe out-of-date state.
%Therefore, we introduce \emph{views} to model threads of a database.
%A \emph{view} is a cut in a history heap that corresponds the indexes of versions that a thread work with.
%We also define a order between two views, if they contain the same addresses and the indexes are ordered point-wise.
%This is to model the synchronisation between threads.
%For example \( \vi \orderVI \vi' \) could mean that a thread updates it view from \( \vi \) to \( \vi' \) by synchronisation with others.
%}

Before proceeding with the program operational semantics, we formalise the notion of \emph{execution tests}.
Execution tests are used to determine whether a transaction may commit its effects (fingerprint) to the MKVS by ensuring that its  effects comply with the underlying consistency model.

\mypar{Execution Tests}
An execution tests of a transaction is a quadruples of the form \( (\mkvs, \vi, \opset, \vi') \), where $\mkvs$ denotes the MKVS;
the $\vi$ denotes the \emph{initial} view, recorded at the beginning of the transactions; 
the $\opset$ denotes the fingerprint of the transaction; and 
$\vi'$ demotes the \emph{final} view of the transaction, obtained after committing the transaction. 
An execution test $(\mkvs, \vi, \opset, \vi')$ states that when the MKVS is described by $\mkvs$, a client with view $\vi$ is allowed to execute a single transaction with fingerprints $\opset$, commit the transaction and obtain an updated view $\vi'$. 
%
%
\azalea{What do we mean by at least $\vi'$? \sx{ meaning the lower bound of the view. The semantics  model this by shift the view at the beginning, so I think we should say update to the exact view.} }
%
%
\azalea{Why do we call these execution tests and not the consistency model? \sx{Andrea thinks it is more precise than consistency model, if I remember.}}
\begin{definition}[Execution Tests]
\label{def:consistency-models}
\label{def:executiontests}
Given the set of $\HisHeaps$ (\cref{def:his_heap}), fingerprints $\Opsets$ (\cref{def:ops}), and views $\Views$ (\cref{def:views}), the set of \emph{execution tests}, \( \como \in \Como \), is:
\[
        \ETS \eqdef  
		\setcomp{
			(\mkvs, \vi, \opset, \vi') \in \HisHeaps \times \Views \times \Opsets \times \Views
		} 
		{
		\opset\projection{2} \subseteq \dom(\vi) = \dom(\vi') = \dom(\hh)
		}       
\]
\end{definition}
%
We often write $(\mkvs, \vi) \etto \opset : \vi'$ for  $(\mkvs, \vi, \opset, \vi') \in \et$.
\sx{
    Maybe also some version of composition requirement.
    For example, the composition of two should be also included in the consistency model.
    \[
        \fora{m,m'} m \in \como \land m' \in \como \implies m \compose{} m' \in \como
    \]
    where \( \compose{} \defeq (\composeHH, \composeVI,\composeO, \composeVI)\).
}


%\begin{definition}[Execution tests]
%\label{def:consistency-models}
%\label{def:executiontests}
%Given the set of key-value stores \( \hh \in \HisHeaps \) (\cref{def:his_heap}), fingerprints \( \opset \in \Opsets \) (\cref{def:ops}) and views \( \vi, \vi' \in \Views \) (\cref{def:views}), \emph{execution tests} \( \como \in \Como \) is a set of quadruples in the form of \( ( \hh, \vi, \opset, \vi' ) \):
%\[
%    \begin{rclarray}
%        \et \in \ETS & \defeq & \powerset{\HisHeaps \times \Views \times \Opsets \times \Views}
%    \end{rclarray}
%\]
%Well-formed  execution tests \( \et \) require the domain of views and the key-value store are the same, and the fingerprints only have keys included in the previous domain:
%\[
%    \begin{rclarray}
%         &&\fora{\hh, \vi, \vi', \opset } (\hh, \vi, \opset, \vi') \in \como \implies \opset\projection{2} \subseteq \dom(\vi) = \dom(\vi') = \dom(\hh)
%    \end{rclarray}
%\]
%\sx{
%    Maybe also some version of composition requirement.
%    For example, the composition of two should be also included in the consistency model.
%    \[
%        \fora{m,m'} m \in \como \land m' \in \como \implies m \compose{} m' \in \como
%    \]
%    where \( \compose{} \defeq (\composeHH, \composeVI,\composeO, \composeVI)\).
%}
%\end{definition}
%
%

%Execution tests is a set of quadruples \( (\mkvs, \vi, \opset, \vi') \) consisting of a key-value store, a view before execution, a operation set and a view after committing of the operation set. 
%We often write \( (\mkvs, \vi) \etto \opset : \vi'\) in lieu of \( (\mkvs, \vi, \opset, \vi') \in \et\).
%The quadruple describes that when the state of the key-value store is \( \hh \), a client who has view \( \vi \) is allowed to execute a single transaction that has  the fingerprints \( \opset \), and then after the commit the thread view must be updated to at least \( \vi' \).

\ac{
There we also note that by tweaking the execution test used by the 
semantics, we capture different consistency models of 
key-value stores.
}


\begin{figure}[!t]
\sx{change the order to key-value, view and stack, and the program semantics change to the type of configuration + a stack env.}
\hrule
%
\[
\begin{rclarray}
	\toT{}  & : &
	\left( ( \HisHeaps \times \Views \times \Stacks ) \times \Commands \right) 
	\; \times\; \Como \;\times \;
	\left( ( \HisHeaps \times \Views \times \Stacks ) \times \Commands \right) 
	\vspace{5pt}
\end{rclarray}
\]
\begin{mathpar}
    \inferrule[\rl{PCommit}]{%
        \vi \orderVI  \vi''
        \\ \h = \clpsHH{\hh,\vi''}
        \\ (\stk, \h, \unitO), \trans \ \toL^{*} \  (\stk', \stub,  \opset) , \pskip
        \\ \txid \in \func{fresh}{\hh}  
        \\\\ \hh' = \func{updateMKVS}{\hh, \vi'', \txid, \opset}  
        \\ \func{updateView}{\hh', \vi'', \opset} \orderVI \vi'
        \\ (\hh, \vi) \csat \opset : \vi'
    }{%
        ( \hh, \vi, \stk ), \ptrans{\trans} \ \toT{\como} \ ( \hh', \vi', \stk' ) , \pskip
    }
    \and
    \inferrule[\rl{PAssign}]{
        \val = \evalE{\expr}
    }{%
        ( \hh, \vi, \stk ) , \passign{\var}{\expr} \ \toT{\como} \  ( \hh, \vi, \stk\rmto{\var}{\val} ) , \pskip
    }
    \and
    \inferrule[\rl{PAssume}]{%
        \evalE[\thstk]{\expr} \neq 0
    }{%
        ( \hh, \vi, \stk ) , \passume{\expr} \ \toT{\como} \  ( \hh, \vi, \stk ) , \pskip
    }
    \and
    \inferrule[\rl{PChoice}]{
        i \in \Set{1,2}
    }{%
        ( \hh, \vi, \stk ) , \cmd_{1} \pchoice \cmd_{2} \ \toT{\como} \  ( \hh, \vi, \stk ) , \cmd_{i}
    }
    \and
    \inferrule[\rl{PIter}]{ }{%
        ( \hh, \vi, \stk ) , \cmd\prepeat \ \toT{\como} \  ( \hh, \vi, \stk ) , \pskip \pchoice (\cmd \pseq \cmd\prepeat)
    }
    \and
    \inferrule[\rl{PSeqSkip}]{ }{%
        ( \hh, \vi, \stk ) , \pskip \pseq \cmd \ \toT{\como} \  ( \hh, \vi, \stk ) , \cmd
    }
    \and
    \inferrule[\rl{PSeq}]{% 
        ( \hh, \vi, \stk ) , \cmd_{1} \ \toT{\como} \  ( \hh, \vi', \stk' ) , {\cmd_{1}}' 
    }{%
        ( \hh, \vi, \stk ) , \cmd_{1} \pseq \cmd_{2} \ \toT{\como} \ ( \hh, \vi', \stk' ) , {\cmd_{1}}' \pseq \cmd_{2}
    }\vspace{5pt}
\end{mathpar}
\begin{flushleft} 
with
\quad
$
\func{fresh}{\hh}  \eqdef
\Setcon{ \txid }{ 
	\txid \in \TxID \land \fora{\addr, i} \txid \neq \WTx(\hh(\addr,i)) 
	\land \txid \notin \RTx(\hhR(\addr,i))
} 
$, and
\vspace{5pt}
 \end{flushleft}
%
\[
\begin{rclarray}         
%	 \func{updateMKVS}{., ., ., .} & : & \MKVSs \times \Views \times  \\                        
    \func{updateMKVS}{\hh, \vi, \txid, \unitO} & \defeq & \hh \\
    \func{updateMKVS}{\hh, \vi, \txid, \opset \uplus \Set{(\otR, \ke, \stub)}} & \defeq &  
    \begin{array}[t]{@{}l}
        \texttt{let } (\nat, \txid', \txidset) = \hh(\ke, \vi(\ke)) \\
        \texttt{and } \hh' = \hh\rmto{\ke}{%
            \hh(\ke)\rmto{\vi(\addr)}{%
                (\nat, \txid', \txidset \uplus \Set{\txid}) } } \\
        \texttt{ in } \func{updateMKVS}{\hh', \vi, \txid, \opset}
    \end{array} \\
    \func{updateMKVS}{\hh, \vi, \txid, \opset \uplus \Set{(\otW, \ke, \nat)}} & \defeq &  
    \begin{array}[t]{@{}l}
        \texttt{let } \hh' = \hh\rmto{\ke}{ ( \hh(\ke) \lcat \List{(\nat, \txid, \emptyset)} ) } \\
        \texttt{ in } \func{updateMKVS}{\hh', \vi, \txid, \opset}
    \end{array} 
%
\end{rclarray}
\]
\begin{flushleft} and \end{flushleft}
%
\[
\begin{rclarray}
    \func{updateView}{\hh, \vi, \unitO} & \defeq & \vi \\
    \func{updateView}{\hh, \vi, \opset \uplus \Set{(\otR, \ke, \stub)}} & \defeq & \func{updateView}{\hh, \vi, \opset}\\
    \func{updateView}{\hh, \vi, \opset \uplus \Set{(\otW, \ke, \stub)}} & \defeq & \func{updateView}{\hh, \vi\rmto{\addr}{(\left| \hh(\addr) \right| - 1)}, \opset}\\
%
%%              
%	\func{fresh}{\hh}  & \defeq & 
%	\Setcon{ \txid }{ 
%		\txid \in \TxID \land \fora{\addr, i} \txid \neq \WTx(\hh(\addr,i)) \\
%		\land\ \txid \notin \RTx(\hhR(\addr,i))
%	} 
\end{rclarray}
\]
\vspace{5pt}
\hrule%\vspace{5pt}
%\begin{flushleft}
%The thread environment is a partial function from thread identifiers to pairs of stacks and views \( \thdenv \in \ThdEnv \defeq \ThreadID \parfinfun \Stacks \times \Views \).
%Given the set of execution tests \( \ConsisModels \) (\cref{def:consistency-models}) and key-value stores \(\HisHeaps\) (\cref{def:mkvs}), the \emph{semantics for programs}:
%\end{flushleft}
\[
	\toG{} : 
    ( \Confs \times \ThdEnv \times \Programs) 
    \;\times\; \Como \;\times\;
    ( \Confs \times \ThdEnv \times \Programs) 
\]
\begin{mathpar}
    \inferrule[\rl{PSingleThread}]{%
        ( \mkvs, \viewFun(\thid), \thdenv(\thid) ), \prog(\thid), \ \toT{\como} \  ( \mkvs', \vi', \stk' ) , \cmd'  
    }{%
        ( (\mkvs, \viewFun), \thdenv, \prog ) \ \toG{\como} \  ( \mkvs', \viewFun\rmto{\thid}{\vi'}) \thdenv\rmto{\thid}{\stk'} , \prog\rmto{\thid}{\cmd'} ) 
    }
\end{mathpar}
%
\hrule
\caption{Operational semantics of commands (above) and programs (below)}
\label{def:thread_semantics}
\label{fig:thread_semantics}
\label{def:thread_pool_semantics}
\label{fig:thread_pool_semantics}
\label{def:program_semantics}
\label{fig:program_semantics}
\end{figure}


\mypar{Operational Semantics of Commands}
The \emph{operational semantics of commands} is given at the top of \cref{fig:thread_semantics}. 
Command transitions are of the form $(\mkvs, \vi, \stk), \cmd \ \toT{\et} \ (\mkvs', \vi', \stk') , \cmd'$, stating that given the MKVS $\mkvs$, view $\vi$ and stack $\stk$, executing command $\cmd$ for one step under $\como$, updates the MKVS to $\mkvs'$, the stack to $\stk'$, and the command to its continuation $\cmd'$. 
Note that $\ET$ denotes the set of execution tests, prescribing the permissible executions of transactions. 
We keep the command operational semantics in \cref{fig:thread_semantics} parametric in the choice of execution tests. 
Later in \cref{sec:cmexamples} we present several examples of execution tests of well-known consistency models in the literature. 

With the exception of the \rl{PCommit} transition, the remaining command transitions are standard and behave as expected. We write 
In \cref{fig:thread_semantics} we write $\lcat$ for denotes list concatenation.
We write $f \rmto{a}{b}$ for function update: $f \rmto{a}{b}(a) = b$, and for all $c \ne a$, $f \rmto{a}{b}(c) = f(c)$.

%The notation \( l\rmto{i}{k} \) on a list \( l \) means the result by replacing the \emph{i-th} element to \( k \).
%%To record the index of list starts from 0.
%The \( \lcat \) denotes list concatenation.

The premise of \rl{PCommit} state that the current view $\vi$ of the executing command maybe advanced to a newer view $\vi''$ (see \cref{def:views}). 
Given the new view $\vi''$, the transaction proceeds by obtaining a snapshot $\sn$ of the MKVS $\mkvs$, and executing $\trans$ locally to completion ($\pskip$), updating the stack to $\stack'$, while accumulating the fingerprint $\opset$. Note that the resulting snapshot is ignored (denoted by $\stub$) as the effect of the transaction is recorded in the fingerprint $\opset$. 
%

The transaction is now ready to commit and may propagate its changes to $\mkvs$.
To this end, a \emph{fresh} transaction identifier $\txid$ is picked (\ie one that does not appear in $\mkvs$ as defined in \cref{fig:thread_semantics}) to identify the completed transaction, and the changes performed by $\txid$ are propagated to $\mkvs$. 
This is done via the $\func{updateMKVS}{\mkvs, \vi'', \txid, \opset}$ function (defined in \cref{fig:thread_semantics}) to update $\mkvs$ to  $\mkvs'$. 
As expected, for every read operation $(\otR, \ke, -)$ in fingerprint $\opset$, the readers of $\ke$ at index $\vi(\ke)$ are extended with $\txid$.
For every write operation $(\otW, \ke, \val)$ in fingerprint $\opset$, the $\mkvs(\ke)$ entry is extended with a new version $(\val, \txid, \emptyset)$, denoting that $\txid$ is responsible for creating this version which has no readers as of yet. 

Once the MKVS is updated to $\mkvs'$, the client subsequently updates its view $\vi''$ with respect to its fingerprint via the $\func{updateView}{\mkvs', \vi'', \opset}$ function, defined in \cref{fig:thread_semantics}.
Whilst the client does not need to update its view for those keys it has read from, 
%for each key $\ke$ that the client has written to, 
it must update its view to the \emph{latest} version available for those keys it has written to. %$\ke$. 
This definition of $\funcFont{updateView}$ imposes a lower bound on the updated view by ensuring that the view of the client is up-to-date for all those keys it has written to. 
This in turn guarantees a strong program order, meaning that the following transactions of the same client will at least read the previous writes of the client  itself.
Assuming that client commands are wrapped within a single session, this lower bound of the view corresponds to the strong session guarantees introduced by \cite{.........}.
%\azalea{I have rewritten this whole section. However, I don't quite understand the motivation behind $\func{updateView}{\mkvs', \vi'', \opset}$. Please elaborate.}

The updated view of the client may then be advanced to a newer view $\vi'$.
Lastly, to ensure that the effect of the transaction (its fingerprint  $\opset$) is permitted by the underlying consistency model, 
the final premise of the transition requires that the updates be permitted by the execution test $\ET$, \ie \( (\mkvs, \vi'') \etto \opset : \vi'\).





%First, the view can shift to later versions before executing the transaction to model the client might gain more information about the key-value store since its last commit.
%To recall The order between two views with the same domain, for example \( \vi'' \geq \vi \) in \rl{PCommit}, is defined by the order of the indexes (\cref{def:views}).
%This new local view \( \vi'' \) should also be consistent with the key-value stores, \ie it leads to a situation where the current client is allowed to execute the transaction.
%The transaction code \( \trans \) is executed locally given an initial snapshot \( \sn = \clpsHH{\hh, \vi''}\) (\cref{def:snapshot}) decided by the current state of key-value store \( \mkvs \) and the local view \( \vi'' \).
%The \( \funcn{localHeap} \) function uniquely determined a (local) heap from a history heap and a view by picking the versions of addresses indexed by the view.
%After local execution via the semantics for transactions (\cref{fig:transaction_semantics}), we propagate the stack \( \stk' \) and more importantly obtain the fingerprints \( \opset \), while the snapshot \( \sn' \) will be throw away.
%Then the transaction picks a fresh identifier \( \txid \), \ie one that does not appear in the key-value store, and commits the fingerprint \( \opset \), which will update the key-value store and local view.
%%The operation set includes the first read and last write of each key, which are the operations might affect the key-value store, because of the atomicity of transactions.
%The \funcn{updateMKVS} function updates the history heap using the fingerprint, \( \mkvs' = \func{updateMKVS}{\mkvs,\vi'',\txid, \opset}\).
%For read operations, it includes the new identifier to the read set of the version the is pointed by the local view \( \vi''\).
%For write operations of the form \( (\otW, \ke, \nat) \), it extends a new version written by the new transaction \( (\nat, \txid, \emptyset) \) to the tail of \( \mkvs(\ke) \).
%For updating the view, we set a lower bound for the new local view by \funcn{updateView} function.
%Assuming the commands executed by clients are wrapped with in a single session, the lower bound of the view corresponds to the strong session guarantees introduced by \cite{.........}.
%This function shifts the view to the up-to-date version in the new key-value store if the version is installed by the current transaction.
%This guarantees strong program order, meaning the following transaction will at least read its own write.
%Finally, the actual new local view \( \vi' \) is any view greater than the lower bound \( \vi' \geq \func{updateView}{\mkvs', \vi'', \opset}\).
%The overall execution satisfied the execution tests, \ie \( (\mkvs, \vi'') \etto \opset : \vi'\).

\ac{The paragraph below should probably go when discussing the rules of the semantics:

Note that the way in which MKVSs and views are updated ensure the following: 
$\bullet$ a client always reads its own preceding writes; 
$\bullet$ clients always read from an increasingly up-to-date state of the database; 
$\bullet$ the order in which clients update a key $\key{k}$ is consistent with the 
order of the versions for such keys in the MKVS; 
$\bullet$ writes take place after reads on which they depend. 
}



\mypar{Program Operational Semantics}
The \emph{operational semantics of programs} are given at the bottom of \cref{fig:program_semantics}. 
Programs transitions are of the form $(\conf,  \thdenv, \prog) \ \toG{\como} (\conf',  \thdenv', \prog')$,
stating that given the configuration $\mkvs$ and the \emph{client environment} $\thdenv$, executing program $\prog$ for one step under $\como$, updates the configuration to $\conf'$, the client environment to $\thdenv'$, and the program to its continuation $\prog'$. 
A \emph{client environment}, $\thdenv \in \ThdEnv$, tracks the local variable stack ($\stack$). 
That is, a client environment is a mapping from client identifiers to pairs of stacks and views. 
We assume that the client identifiers in the domain of client environments are are those in the domain of the program throughout the execution. 
%$\prog$: $\dom(\thdenv) = \dom(\prog)$; and that 
Program transitions are simply defined in terms of the transitions of their constituent client commands, defined by $\toT{\como}$. 
This in turn yields the standard interleaving semantics for concurrent programs. 

%Last, the program has standard interleaving semantics by picking a client and then progressing one step (\cref{fig:thread_pool_semantics}).
%To achieve that a thread environment holds the stacks and views associated with all active clients \( Env \in \sort{ThdEnv} \).
%We assume the client identifiers from client environment match with those in the program \( \prog \).
%We also assume all the stacks and views initially are the same respectively.



\begin{lemma}
\label{lem:hhupdate.welldefined}
%The function $\HHupdate$ is well-defined over well-formed MKVS, fingerprints and views. 
Given a well-formed MKVS $\hh$, a view $\vi$ that is well-formed with respect to \( \mkvs \), a fingerprint \( \opset \), and a $\txid$ that does not appear in $\hh$, then $\HHupdate(\hh, \opset, \tsid, \vi)$ is uniquely determined and yields a well-formed MKVS.
\end{lemma}

\begin{lemma}
%The function $\Vupdate$ is well-defined.
Given a well-formed MKVS $\hh$, a view $\vi$ that is well-formed with respect to \( \mkvs \), and a fingerprint \( \opset \), the $\Vupdate(\hh, \opset, \vi)$ uniquely determines a view which is well-formed with respect to $\mkvs$.
\end{lemma}

%\begin{lem}[Confluence of \funcn{updateMKVS} and \funcn{updateView}]
%Given a valid operation set \( \opset \), the order of applying \( \funcn{updateMKVS} \) function (\( \funcn{updateView} \) function) to the elements from operation set does not affect the final result.
%\end{lem}
                                                                                                         

%\subsection{Example of Consistency models}
\ac{This Section is going to become heavy in pictures, which should be organised into figures.}
In this Section we present different consistency models specifications. 
For each of them, we give: 
\begin{itemize}
\item the intuition of the commit tests for different consistency models, and the formal definitions with respect to the \(\Como\) (\defref{def:consistency-models}).
%describing the consistency guarantees that schedules of the database should have in plain English, 
%\item a formal consistency model specification, in the style described in \S \ref{sec:semantics.programs},
\item examples of litmus tests that, when executed, give rise to the anomalies that are forbidden from the consistency model, 
\item an explanation of why the consistency model forbids the litmus tests to exhibit the anomaly that should be forbidden. 
\end{itemize}
Later, we will show how to compare our consistency models specifications with those already existing in the 
literature.
\ac{There is still a long-way to go before proving correspondence with dependency graph specifications, 
but this should be mentioned here.}


\subsection{Read Atomic} 
Read atomic (RA) \cite{ramp} is the weakest consistency model among those that enjoy \emph{atomic visibility} \cite{framework-concur}. 
It requires of a transaction to read an atomic snapshot of the database and never observe the partial effects of other transactions.
This is also known as the \emph{all-or-nothing} property: a transaction observes either none or all the updates performed by other transactions. 

\sx{example RAMP, not sure the meaning}
One litmus test that should \textbf{not} be failed in RAMP consists of the program $\prog_1$ from \S \ref{sec:semantics.example}, which we already observed to produce a violation of atomic visibility if no constraints on the consistency model are placed.
\ac{not be failed. Double negation. Bad English.}

\sx{Rephrase}
Intuitively, in such a program, it violate the atomic visibility because we allowed to execute the transaction \( \trans_1^2\) in the thread-local configuration of $\mathcal{C}'$ relative to $\thid_2$, which is obtained by removing all the information about $\thid_1$ (view and stack) in Figure \ref{fig:opsem.example}(c).
\[
\trans_1^2 = \begin{array}{c} 
            \begin{transaction}
            		\pderef{\pvar{a}}{\vx};\\
            		\pderef{\pvar{b}}{\vy};\\
            		\pifs{\pvar{a}=1 \wedge \pvar{b}=0}\\
            			\quad \passign{\retvar}{\Large \frownie{}} \\
                    \pife
             \end{transaction}
     \end{array}
\]

\ac{
To avoid transactions to only observe the partial effects of other transactions, we 
must ensure that transactional code cannot be executed by a thread whose 
views is up-to-date with respect to some transaction $\tsid$ for some location $[\loc_x]$, 
but not for some other location $[\loc_y]$. This leads to the following definition.
}
To avoid a transaction to observe the partial effects of other transactions, we need to ensure that transactional code cannot be executed by a thread whose views is partially up-to-date with respect to some transactions. This leads to the following definition.
\begin{defn}
\label{def:readatomic}
%Let $\hh$ be a history heap,$V$ be a view, $[\loc_x]$ be 
%a location and $\nu$ be a version. We say that $V$ $[\loc_x]$-\emph{sees} version 
%$\nu$ if there exists an index $i \leq V([\loc_x])$ such that $V(i) = \nu$. 
%We say that $V$ $[\loc_x]$-\emph{sees} transaction $\tsid$ if 
%$V$ $[\loc_x]$-sees a version $\nu = (\_, \tsid, \_)$. 
Given a view $\vi \in \Views$, a history heap $\hh \in \HisHeaps$, and a transaction identifier $\txid \in \TxID$, the view \emph{sees} the transaction in the history heap, written $\pred{visible}{\txid, \vi, \hh}$, if the view sees all the writes from the transaction,
%We say that $V$ \emph{sees} transaction $\tsid$ in $\hh$, written 
%$\mathsf{Visible}(\tsid, V, \hh)$, iff 
\sx{\( \exsts{i} \) might be enough}
\[
\begin{rclarray}
\pred{visible}{\txid, \vi, \hh} & \eqdef & \fora{\addr, i} \hh(\addr)(i) = (\stub, \txid, \stub) \implies i \leq \vi(\addr).
\end{rclarray}
\]
\ac{In English: the view is up-to-date with respect to all the updates 
performed by transaction $\tsid$.}

Then given a history heap \( \hh \), the view $V$ is \emph{consistent} with respect to \emph{atomic visibility}, written $\pred{atomic}{\vi, \hh}$, if the view $V$ is up-to-date with some of the updates performed by $\txid$, then it should be up-to-date with all the updates performed by $\txid$,
\[
\begin{rclarray}
\pred{atomic}{\vi ,\hh} & \eqdef & \fora{\txid } \exsts{\addr, i} i \leq \vi(\addr) \land \hh(\addr)(i) = (\stub, \txid, \stub) \implies \pred{vusible}{\txid, \vi, \hh}
\end{rclarray}
\]
\ac{In English: if the view $V$ is up-to-date with some of the updates performed 
by $\tsid$, then it must be up-to-date with all the updates performed by $\tsid$. 
This is the all-or-nothing property.}
%for all location 
%$[\loc_x]$, if there exists an index $i = 0,\cdots, \lvert \hh([\loc_x]) \rvert - 1$, 
%such that $\hh([\loc_x])(i) = (\_, \tsid, \_)$, then $i \leq V([\loc_x])$.

The consistency model specification $\mathsf{RA}$ is defined as the smallest set such that  
\sx{what is the meaning of smallest?}
\[
\pred{atomic}{\hh, \vi} \implies (\hh, \vi) \csat[\mathsf{RA}] \stub: \stub
\]
\ac{In English: Before executing a transaction, either you observe all or none the 
updates of all other transactions. We may strengthen the consistency model and 
require that the same property must be satisfied at the end as well, though 
this is not strictly necessary. In this case the check becomes: 
\[
\mathsf{atomic}(\hh, V) \wedge \mathsf{atomic}(\hh, V') \wedge \mathsf{UpdateView}(\hh, V, \mathcal{O}) 
\sqsubseteq V' \implies (\hh, V) \triangleright_{\mathsf{RA}} \mathcal{O}: V'.
\]
}
%written $\mathsf{up-to-date}(\hh, V, \tsid, [\loc_x])$, 
%if either 
%
%\begin{itemize}
%\item for all indexes $i = 0,\cdots, \lvert \hh([\loc_x]) - 1 \rvert$, 
%$\WS(\hh([\loc_x])(i)) \neq \tsid)$, or 
%\item if $\WS(\hh([\loc_x])(i)) = \tsid$ for some $i = 0,\cdots, \lvert \hh([\loc_n]) -1 \rvert$, 
%then $i \leq V([\loc_n])$.
%\end{itemize}
\end{defn}

\sx{Not sure how to link the explanation from Andrea's document, sort out later }
Suppose that we execute the program $\prog_1$ under the consistency model specification $\mathsf{RA}$.
We can proceed as in Section \ref{sec:semantics.example} to infer the transition 
$\langle \mathcal{C_0}, \prog_1 \rangle \xrightarrow{\mathsf{RA}} \langle \mathcal{C}_1, \prog_1' \rangle$, 
where we recall that $\mathcal{C}_0$, $\mathcal{C}_1$ are depicted in Figure \ref{fig:opsem.exampe}(a), 
\ref{fig:opsem.example}(b), respectively. 

It is immediate to observe that the only way in which the execution of transaction $\ptrans{\trans}$ from $\thid_2$ in $\prog_1'$ can return value ${\Large \frownie}$ is the following: 
\begin{itemize}
\item first, push the view $V$ of thread $\txid_2$ in the configuration 
$\mathcal{C}_1$ of Figure \ref{fig:opsem.example}(b) to observe the update of location $[\vx]$, but not the update of 
$[\vy]$. This view is the one labelled with $\txid_2$ in Figure \ref{fig:opsem.example}(c), and we refer 
to it as $V'$;
\item then, execute the transaction $\ptrans{\trans}$ in $\thid_2$. 
\end{itemize}

\subsection{Causal Consistency}
%\begin{figure}
%\begin{tabular}{|c|c|}
%\hline
%\begin{tikzpicture}[font=\large]

%\begin{pgfonlayer}{foreground}
%%Uncomment line below for help lines
%%\draw[help lines] grid(5,4);

%%Location x
%\node(locx) at (1,3) {$[\loc_x] \mapsto$};

%\matrix(locxcells) [version list, text width=7mm, anchor=west]
   %at ([xshift=10pt]locx.east) {
 %{a} & $T_0$ \\
  %{a} & $\emptyset$ \\
%};
%\node[version node, fit=(locxcells-1-1) (locxcells-2-1), fill=white, inner sep= 0cm, font=\Large] (locx-v0) {$0$};

%%Location y
%\path (locx.south) + (0,-1.5) node (locy) {$[\loc_y] \mapsto$};
%\matrix(locycells) [version list, text width=7mm, anchor=west]
   %at ([xshift=10pt]locy.east) {
 %{a} & $T_0$ \\
  %{a} & $\emptyset$ \\
%};
%\node[version node, fit=(locycells-1-1) (locycells-2-1), fill=white, inner sep= 0cm, font=\Large] (locy-v0) {$0$};

%% \draw[-, red, very thick, rounded corners] ([xshift=-5pt, yshift=5pt]locx-v1.north east) |- 
%%  ($([xshift=-5pt,yshift=-5pt]locx-v1.south east)!.5!([xshift=-5pt, yshift=5pt]locy-v0.north east)$) -| ([xshift=-5pt, yshift=5pt]locy-v0.south east);

%%blue view - I should  check whether I can use pgfkeys to just declare the list of locations, and then add the view automatically.
%\draw[-, blue, very thick, rounded corners=10pt]
 %([xshift=-2pt, yshift=20pt]locx-v0.north east) node (tid1start) {} -- 
%% ([xshift=-2pt, yshift=-5pt]locx-v0.south east) --
%% ([xshift=-2pt, yshift=5pt]locy-v0.north east) -- 
 %([xshift=-2pt, yshift=-5pt]locy-v0.south east);
 
 %\path (tid1start) node[anchor=south, rectangle, fill=blue!20, draw=blue, font=\small, inner sep=1pt] {$\tid_3$};

%%red view
%\draw[-, red, very thick, rounded corners = 10pt]
 %([xshift=-5pt, yshift=5pt]locx-v0.north east) -- 
%% ([xshift=-8pt, yshift=-5pt]locx-v0.south east) --
%% ([xshift=-8pt, yshift=5pt]locy-v0.north east) -- 
 %([xshift=-5pt, yshift=-10pt]locy-v0.south east) node (tid2start) {};
 
%\path (tid2start) node[anchor=north, rectangle, fill=red!20, draw=red, font=\small, inner sep=1pt] {$\tid_2$};
 
 %%green view
%\draw[-, DarkGreen, very thick, rounded corners = 10pt]
 %([xshift=-16pt, yshift=8pt]locx-v0.north east) node (tid3start) {}-- 
%% ([xshift=-15pt, yshift=-5pt]locx-v0.south east) --
%% ([xshift=-15pt, yshift=5pt]locy-v0.north east) -- 
 %([xshift=-16pt, yshift=-5pt]locy-v0.south east);
 
 %\path (tid3start) node[anchor=south, rectangle, fill=DarkGreen!20, draw=DarkGreen, font=\small, inner sep=1pt] {$\tid_1$};

%\end{pgfonlayer}
%\end{tikzpicture}
%&
%\begin{tikzpicture}[font=\large]

%\begin{pgfonlayer}{foreground}
%%Uncomment line below for help lines
%%\draw[help lines] grid(5,4);

%%Location x
%\node(locx) at (1,3) {$[\loc_x] \mapsto$};

%\matrix(locxcells) [version list, text width=7mm, anchor=west]
   %at ([xshift=10pt]locx.east) {
 %{a} & $\tsid_0$ & {a} & $\tsid_1$\\
  %{a} & $\emptyset$ & {a} & $\emptyset$ \\
%};
%\node[version node, fit=(locxcells-1-1) (locxcells-2-1), fill=white, inner sep= 0cm, font=\Large] (locx-v0) {$0$};
%\node[version node, fit=(locxcells-1-3) (locxcells-2-3), fill=white, inner sep=0cm, font=\Large] (locx-v-1) {$1$};
%%Location y
%\path (locx.south) + (0,-1.5) node (locy) {$[\loc_y] \mapsto$};
%\matrix(locycells) [version list, text width=7mm, anchor=west]
   %at ([xshift=10pt]locy.east) {
 %{a} & $\tsid_0$ \\
   %{a} & $\emptyset$ \\
%};
%\node[version node, fit=(locycells-1-1) (locycells-2-1), fill=white, inner sep= 0cm, font=\Large] (locy-v0) {$0$};
%% \draw[-, red, very thick, rounded corners] ([xshift=-5pt, yshift=5pt]locx-v1.north east) |- 
%%  ($([xshift=-5pt,yshift=-5pt]locx-v1.south east)!.5!([xshift=-5pt, yshift=5pt]locy-v0.north east)$) -| ([xshift=-5pt, yshift=5pt]locy-v0.south east);

%%blue view - I should  check whether I can use pgfkeys to just declare the list of locations, and then add the view automatically.
%\draw[-, blue, very thick, rounded corners=10pt]
 %([xshift=-2pt, yshift=20pt]locx-v0.north east) node (tid1start) {} -- 
%% ([xshift=-2pt, yshift=-5pt]locx-v0.south east) --
%% ([xshift=-2pt, yshift=5pt]locy-v0.north east) -- 
 %([xshift=-2pt, yshift=-5pt]locy-v0.south east);
 
 %\path (tid1start) node[anchor=south, rectangle, fill=blue!20, draw=blue, font=\small, inner sep=1pt] {$\tid_3$};

%%red view
%\draw[-, red, very thick, rounded corners = 10pt]
 %([xshift=-5pt, yshift=5pt]locx-v0.north east) -- 
%% ([xshift=-8pt, yshift=-5pt]locx-v0.south east) --
%% ([xshift=-8pt, yshift=5pt]locy-v0.north east) -- 
 %([xshift=-5pt, yshift=-10pt]locy-v0.south east) node (tid2start) {};
 
%\path (tid2start) node[anchor=north, rectangle, fill=red!20, draw=red, font=\small, inner sep=1pt] {$\tid_2$};
 
 %%green view
%\draw[-, DarkGreen, very thick, rounded corners = 10pt]
 %([xshift=-16pt, yshift=8pt]locx-v1.north east) node (tid3start) {}-- 
 %([xshift=-16pt, yshift=-5pt]locx-v1.south east) --
 %([xshift=-16pt, yshift=5pt]locy-v0.north east) -- 
 %([xshift=-16pt, yshift=-5pt]locy-v0.south east);
 
 %\path (tid3start) node[anchor=south, rectangle, fill=DarkGreen!20, draw=DarkGreen, font=\small, inner sep=1pt] {$\tid_1$};

%\end{pgfonlayer}
%\end{tikzpicture}
%\\
%{\small (a)} & {\small (b)}\\
%\hline
%\begin{tikzpicture}[font=\large]

%\begin{pgfonlayer}{foreground}
%%Uncomment line below for help lines
%%\draw[help lines] grid(5,4);

%%Location x
%\node(locx) at (1,3) {$[\loc_x] \mapsto$};

%\matrix(locxcells) [version list, text width=7mm, anchor=west]
   %at ([xshift=10pt]locx.east) {
 %{a} & $\tsid_0$ & {a} & $\tsid_1$\\
  %{a} & $\emptyset$ & {a} & $\emptyset$ \\
%};
%\node[version node, fit=(locxcells-1-1) (locxcells-2-1), fill=white, inner sep= 0cm, font=\Large] (locx-v0) {$0$};
%\node[version node, fit=(locxcells-1-3) (locxcells-2-3), fill=white, inner sep=0cm, font=\Large] (locx-v-1) {$1$};
%%Location y
%\path (locx.south) + (0,-1.5) node (locy) {$[\loc_y] \mapsto$};
%\matrix(locycells) [version list, text width=7mm, anchor=west]
   %at ([xshift=10pt]locy.east) {
 %{a} & $\tsid_0$ \\
   %{a} & $\emptyset$ \\
%};
%\node[version node, fit=(locycells-1-1) (locycells-2-1), fill=white, inner sep= 0cm, font=\Large] (locy-v0) {$0$};
%% \draw[-, red, very thick, rounded corners] ([xshift=-5pt, yshift=5pt]locx-v1.north east) |- 
%%  ($([xshift=-5pt,yshift=-5pt]locx-v1.south east)!.5!([xshift=-5pt, yshift=5pt]locy-v0.north east)$) -| ([xshift=-5pt, yshift=5pt]locy-v0.south east);

%%blue view - I should  check whether I can use pgfkeys to just declare the list of locations, and then add the view automatically.
%\draw[-, blue, very thick, rounded corners=10pt]
 %([xshift=-2pt, yshift=20pt]locx-v0.north east) node (tid1start) {} -- 
%% ([xshift=-2pt, yshift=-5pt]locx-v0.south east) --
%% ([xshift=-2pt, yshift=5pt]locy-v0.north east) -- 
 %([xshift=-2pt, yshift=-5pt]locy-v0.south east);
 
 %\path (tid1start) node[anchor=south, rectangle, fill=blue!20, draw=blue, font=\small, inner sep=1pt] {$\tid_3$};

%%red view
%\draw[-, red, very thick, rounded corners = 10pt]
 %([xshift=-5pt, yshift=5pt]locx-v1.north east) -- 
 %([xshift=-5pt, yshift=-5pt]locx-v1.south east) --
 %([xshift=-5pt, yshift=3pt]locy-v0.north east) -- 
 %([xshift=-5pt, yshift=-10pt]locy-v0.south east) node (tid2start) {};
 
%\path (tid2start) node[anchor=north, rectangle, fill=red!20, draw=red, font=\small, inner sep=1pt] {$\tid_2$};
 
 %%green view
%\draw[-, DarkGreen, very thick, rounded corners = 10pt]
 %([xshift=-16pt, yshift=8pt]locx-v1.north east) node (tid3start) {}-- 
 %([xshift=-16pt, yshift=-5pt]locx-v1.south east) --
 %([xshift=-16pt, yshift=5pt]locy-v0.north east) -- 
 %([xshift=-16pt, yshift=-5pt]locy-v0.south east);
 
 %\path (tid3start) node[anchor=south, rectangle, fill=DarkGreen!20, draw=DarkGreen, font=\small, inner sep=1pt] {$\tid_1$};

%\end{pgfonlayer}
%\end{tikzpicture}
%&
%\begin{tikzpicture}[font=\large]

%\begin{pgfonlayer}{foreground}
%%Uncomment line below for help lines
%%\draw[help lines] grid(5,4);

%%Location x
%\node(locx) at (1,3) {$[\loc_x] \mapsto$};

%\matrix(locxcells) [version list, text width=7mm, anchor=west]
   %at ([xshift=10pt]locx.east) {
 %{a} & $\tsid_0$ & {a} & $\tsid_1$\\
  %{a} & $\emptyset$ & {a} & $\{\tsid_2\}$ \\
%};
%\node[version node, fit=(locxcells-1-1) (locxcells-2-1), fill=white, inner sep= 0cm, font=\Large] (locx-v0) {$0$};
%\node[version node, fit=(locxcells-1-3) (locxcells-2-3), fill=white, inner sep=0cm, font=\Large] (locx-v-1) {$1$};
%%Location y
%\path (locx.south) + (0,-1.5) node (locy) {$[\loc_y] \mapsto$};
%\matrix(locycells) [version list, text width=7mm, anchor=west]
   %at ([xshift=10pt]locy.east) {
 %{a} & $\tsid_0$ & {a} & $\tsid_2$ \\
   %{a} & $\emptyset$ & {a} & $\emptyset$\\
%};
%\node[version node, fit=(locycells-1-1) (locycells-2-1), fill=white, inner sep= 0cm, font=\Large] (locy-v0) {$0$};
%\node[version node, fit=(locycells-1-3) (locycells-2-3), fill=white, inner sep=0cm, font=\Large] (locy-v-1) {$1$};
%% \draw[-, red, very thick, rounded corners] ([xshift=-5pt, yshift=5pt]locx-v1.north east) |- 
%%  ($([xshift=-5pt,yshift=-5pt]locx-v1.south east)!.5!([xshift=-5pt, yshift=5pt]locy-v0.north east)$) -| ([xshift=-5pt, yshift=5pt]locy-v0.south east);

%%blue view - I should  check whether I can use pgfkeys to just declare the list of locations, and then add the view automatically.
%\draw[-, blue, very thick, rounded corners=10pt]
 %([xshift=-2pt, yshift=20pt]locx-v0.north east) node (tid1start) {} -- 
%% ([xshift=-2pt, yshift=-5pt]locx-v0.south east) --
%% ([xshift=-2pt, yshift=5pt]locy-v0.north east) -- 
 %([xshift=-2pt, yshift=-5pt]locy-v0.south east);
 
 %\path (tid1start) node[anchor=south, rectangle, fill=blue!20, draw=blue, font=\small, inner sep=1pt] {$\tid_3$};

%%red view
%\draw[-, red, very thick, rounded corners = 10pt]
 %([xshift=-5pt, yshift=5pt]locx-v1.north east) -- 
%% ([xshift=-5pt, yshift=-5pt]locx-v1.south east) --
%% ([xshift=-5pt, yshift=3pt]locy-v0.north east) -- 
 %([xshift=-5pt, yshift=-10pt]locy-v1.south east) node (tid2start) {};
 
%\path (tid2start) node[anchor=north, rectangle, fill=red!20, draw=red, font=\small, inner sep=1pt] {$\tid_2$};
 
 %%green view
%\draw[-, DarkGreen, very thick, rounded corners = 10pt]
 %([xshift=-16pt, yshift=8pt]locx-v1.north east) node (tid3start) {}-- 
 %([xshift=-16pt, yshift=-5pt]locx-v1.south east) --
 %([xshift=-16pt, yshift=5pt]locy-v0.north east) -- 
 %([xshift=-16pt, yshift=-5pt]locy-v0.south east);
 
 %\path (tid3start) node[anchor=south, rectangle, fill=DarkGreen!20, draw=DarkGreen, font=\small, inner sep=1pt] {$\tid_1$};

%\end{pgfonlayer}
%\end{tikzpicture}\\
%{\small (c)} & {\small (d)} \\
%\hline
%\begin{tikzpicture}[font=\large]

%\begin{pgfonlayer}{foreground}
%%Uncomment line below for help lines
%%\draw[help lines] grid(5,4);

%%Location x
%\node(locx) at (1,3) {$[\loc_x] \mapsto$};

%\matrix(locxcells) [version list, text width=7mm, anchor=west]
   %at ([xshift=10pt]locx.east) {
 %{a} & $\tsid_0$ & {a} & $\tsid_1$\\
  %{a} & $\emptyset$ & {a} & $\{\tsid_2\}$ \\
%};
%\node[version node, fit=(locxcells-1-1) (locxcells-2-1), fill=white, inner sep= 0cm, font=\Large] (locx-v0) {$0$};
%\node[version node, fit=(locxcells-1-3) (locxcells-2-3), fill=white, inner sep=0cm, font=\Large] (locx-v-1) {$1$};
%%Location y
%\path (locx.south) + (0,-1.5) node (locy) {$[\loc_y] \mapsto$};
%\matrix(locycells) [version list, text width=7mm, anchor=west]
   %at ([xshift=10pt]locy.east) {
 %{a} & $\tsid_0$ & {a} & $\tsid_2$ \\
   %{a} & $\emptyset$ & {a} & $\emptyset$\\
%};
%\node[version node, fit=(locycells-1-1) (locycells-2-1), fill=white, inner sep= 0cm, font=\Large] (locy-v0) {$0$};
%\node[version node, fit=(locycells-1-3) (locycells-2-3), fill=white, inner sep=0cm, font=\Large] (locy-v-1) {$1$};
%% \draw[-, red, very thick, rounded corners] ([xshift=-5pt, yshift=5pt]locx-v1.north east) |- 
%%  ($([xshift=-5pt,yshift=-5pt]locx-v1.south east)!.5!([xshift=-5pt, yshift=5pt]locy-v0.north east)$) -| ([xshift=-5pt, yshift=5pt]locy-v0.south east);

%%blue view - I should  check whether I can use pgfkeys to just declare the list of locations, and then add the view automatically.
%\draw[-, blue, very thick, rounded corners=10pt]
 %([xshift=-2pt, yshift=20pt]locx-v0.north east) node (tid1start) {} -- 
 %([xshift=-2pt, yshift=-5pt]locx-v0.south east) --
 %([xshift=-2pt, yshift=5pt]locy-v1.north east) -- 
 %([xshift=-2pt, yshift=-5pt]locy-v1.south east);
 
 %\path (tid1start) node[anchor=south, rectangle, fill=blue!20, draw=blue, font=\small, inner sep=1pt] {$\tid_3$};

%%red view
%\draw[-, red, very thick, rounded corners = 10pt]
 %([xshift=-5pt, yshift=5pt]locx-v1.north east) -- 
%% ([xshift=-5pt, yshift=-5pt]locx-v1.south east) --
%% ([xshift=-5pt, yshift=3pt]locy-v0.north east) -- 
 %([xshift=-5pt, yshift=-10pt]locy-v1.south east) node (tid2start) {};
 
%\path (tid2start) node[anchor=north, rectangle, fill=red!20, draw=red, font=\small, inner sep=1pt] {$\tid_2$};
 
 %%green view
%\draw[-, DarkGreen, very thick, rounded corners = 10pt]
 %([xshift=-16pt, yshift=8pt]locx-v1.north east) node (tid3start) {}-- 
 %([xshift=-16pt, yshift=-5pt]locx-v1.south east) --
 %([xshift=-16pt, yshift=5pt]locy-v0.north east) -- 
 %([xshift=-16pt, yshift=-5pt]locy-v0.south east);
 
 %\path (tid3start) node[anchor=south, rectangle, fill=DarkGreen!20, draw=DarkGreen, font=\small, inner sep=1pt] {$\tid_1$};

%\end{pgfonlayer}
%\end{tikzpicture}
%&
%\begin{tikzpicture}[font=\large]

%\begin{pgfonlayer}{foreground}
%%Uncomment line below for help lines
%%\draw[help lines] grid(5,4);

%%Location x
%\node(locx) at (1,3) {$[\loc_x] \mapsto$};

%\matrix(locxcells) [version list, text width=7mm, anchor=west]
   %at ([xshift=10pt]locx.east) {
 %{a} & $\tsid_0$ & {a} & $\tsid_1$\\
  %{a} & $\{\tsid_3\}$ & {a} & $\{\tsid_2\}$ \\
%};
%\node[version node, fit=(locxcells-1-1) (locxcells-2-1), fill=white, inner sep= 0cm, font=\Large] (locx-v0) {$0$};
%\node[version node, fit=(locxcells-1-3) (locxcells-2-3), fill=white, inner sep=0cm, font=\Large] (locx-v1) {$1$};
%%Location y
%\path (locx.south) + (0,-1.5) node (locy) {$[\loc_y] \mapsto$};
%\matrix(locycells) [version list, text width=7mm, anchor=west]
   %at ([xshift=10pt]locy.east) {
 %{a} & $\tsid_0$ & {a} & $\tsid_2$ \\
   %{a} & $\emptyset$ & {a} & $\{\tsid_3\}$\\
%};
%\node[version node, fit=(locycells-1-1) (locycells-2-1), fill=white, inner sep= 0cm, font=\Large] (locy-v0) {$0$};
%\node[version node, fit=(locycells-1-3) (locycells-2-3), fill=white, inner sep=0cm, font=\Large] (locy-v1) {$1$};
%% \draw[-, red, very thick, rounded corners] ([xshift=-5pt, yshift=5pt]locx-v1.north east) |- 
%%  ($([xshift=-5pt,yshift=-5pt]locx-v1.south east)!.5!([xshift=-5pt, yshift=5pt]locy-v0.north east)$) -| ([xshift=-5pt, yshift=5pt]locy-v0.south east);

%%blue view - I should  check whether I can use pgfkeys to just declare the list of locations, and then add the view automatically.
%\draw[-, blue, very thick, rounded corners=10pt]
 %([xshift=-2pt, yshift=20pt]locx-v0.north east) node (tid1start) {} -- 
 %([xshift=-2pt, yshift=-5pt]locx-v0.south east) --
 %([xshift=-2pt, yshift=5pt]locy-v1.north east) -- 
 %([xshift=-2pt, yshift=-5pt]locy-v1.south east);
 
 %\path (tid1start) node[anchor=south, rectangle, fill=blue!20, draw=blue, font=\small, inner sep=1pt] {$\tid_3$};

%%red view
%\draw[-, red, very thick, rounded corners = 10pt]
 %([xshift=-5pt, yshift=5pt]locx-v1.north east) -- 
%% ([xshift=-5pt, yshift=-5pt]locx-v1.south east) --
%% ([xshift=-5pt, yshift=3pt]locy-v0.north east) -- 
 %([xshift=-5pt, yshift=-10pt]locy-v1.south east) node (tid2start) {};
 
%\path (tid2start) node[anchor=north, rectangle, fill=red!20, draw=red, font=\small, inner sep=1pt] {$\tid_2$};
 
 %%green view
%\draw[-, DarkGreen, very thick, rounded corners = 10pt]
 %([xshift=-16pt, yshift=8pt]locx-v1.north east) node (tid3start) {}-- 
 %([xshift=-16pt, yshift=-5pt]locx-v1.south east) --
 %([xshift=-16pt, yshift=5pt]locy-v0.north east) -- 
 %([xshift=-16pt, yshift=-5pt]locy-v0.south east);
 
 %\path (tid3start) node[anchor=south, rectangle, fill=DarkGreen!20, draw=DarkGreen, font=\small, inner sep=1pt] {$\tid_1$};

%\end{pgfonlayer}
%\end{tikzpicture}
%\\
%{\small (e)} & {\small (f)} \\
%\hline
%\end{tabular}
%\caption{History heaps obtained in a execution of $\prog_2$.}
%\label{fig:cc.exec}
%\end{figure}


\sx{Should we give intuition about causal dependencies here ?}
The next consistency model that we are interested is \emph{transactional causal consistency} \cite{cops}. 
Intuitively, it ensures that versions read by transactions are closed with respect to \emph{causal dependencies}. 
Consider for example the following program: 
\[
    \prog_2 \equiv \begin{session}
        \begin{array}{@{}c || c || c@{}}
            \txid_{1} : 
            \begin{transaction}
                \pmutate{\vx}{1};\\
            \end{transaction} &
            \txid_{2} : 
            \begin{transaction} 
                \pderef{\pvar{a}}{\vx};\\
                \pmutate{\vy}{\pvar{a}};\\
            \end{transaction} &
            \txid_{3} :
             \begin{transaction}
               	   \pderef{\pvar{a}}{\vx};\\
               	   \pderef{\pvar{b}}{\vy};\\
               	   \pifs{\pvar{a}=0 \wedge \pvar{b}=1}\\
               			\quad \passign{\retvar}{\Large \frownie{}}
               		\pife
             \end{transaction}
        \end{array}
    \end{session}
 \]
For the sake of simplicity, we label the code of the three transactions above as $\txid_{1}, \txid_2, \txid_3$ from left to right.
It is easy to see that, if no constraints or even under read atomic, the third transaction $\txid_{3}$ can return ${\Large \frownie{}}$. 
%The same is true even if the consistency model specification $\mathsf{RA}$ is assumed. 
Informally, the return of value ${\Large \frownie{}}$ by $\txid_3$ can be obtained from the execution outlined below. 
\sx{Did not edit, change later}
\begin{itemize}
\item The initial configuration of this execution is depicted in Figure \ref{fig:cc.exec}(a).
\item $\ptrans{\trans_1}$ executes with the initial view, which points to the 
initial (and only) version for each location; after this transaction is 
executed, a new version $\langle 1, T_1, \emptyset \rangle$ is appended 
at the end of $\hh(\loc_{x})$. The resulting history heap is depicted in Figure \ref{fig:cc.exec}(b).
\item next, $\thid_2$ updates its view as to see the version of $\loc_x$ installed by $\thid_1$, after 
which it proceeds to execute $\ptrans{\trans_2}$. This results in a new version with value $1$ 
to be installed for $\loc_y$. The configurations before, and after the execution of $\ptrans{\trans_2}$, 
are depicted in figures \ref{fig:cc.exec}(c) and \ref{fig:cc.exec}(d), respectively.
\item Finally, thread $\thid_3$ updates its view to observe the update of location $[\loc_y]$, but not the update of 
location $[\loc_y]$, before executing transaction $\ptrans{\trans_3}$. The execution of $\ptrans{\trans_3}$ will 
return the value ${\Large \frownie{}}$. The history heaps immediately before and after 
the execution of $\thid_3$, are depicted in figures \ref{fig:cc.exec}(e) and \ref{fig:cc.exec}(f), respectively. 
\end{itemize}

\sx{Change above}

In the last step, the thread to the right commits the transaction $\txid_3$ in a state where its initial view observes the second version of the address $\vy$, which is created by \( \txid_{2} \).
However, because the transaction \( \txid_{2} \) read the second version of \( \vx \) and create the second version of \( \vy \), this means the latter depends on the former.
Yet the transaction \( \txid_{3} \) does not read from the second version of \( \vx \), which is disallowed by \emph{transactional causal consistency}.
Summarising, under transactional causal consistency if a transaction sees updates for an address \( \addr \), it should also observes those addresses that \( \addr \) depends on.

\ac{
but not the update to address $\vx$.
However, the update of $[\vy]$ committed by $\txid_{2}$, consisted in copying the value of the update 
of $[\loc_y]$: that is, the update of $[\loc_y]$ \textbf{depends} from the update of $[\loc_x]$. 
Summarising, the execution of transaction $\ptrans{\trans_3}$ resulted in a violation of 
causality: the update of $[\loc_y]$ is observed, but not the update of $[\loc_x]$ on which 
it depends.
}

\sx{in RA up-to-date view, and here consistent view, not sure are good words, let re-think later}
To formally specify transactional causal consistency, we inductively define the set of views that are consistent with respect to a history heap $\hh$. 
\sx{What do you mean by the word??}
The definition below models the fact that, if we start from a causally consistent view, and we wish to update the view for some location $\txid_2$, 

\begin{defn}
Given two versions $\ver_{1} = (\val_1, \txid_1, \txidset_1)$, $\ver_2 = (\val_2, \txid_2, \txidset_2)$, $\ver_{1}$ \emph{directly depends on} $\ver_2$, written $\pred{ddep}{\ver_{1}, \ver_{2}}$, if $\txid_1 \in \txidset_2$. 
\ac{Note to self: the notion of directly depends here has little to do with dependencies 
between transactions. $\nu_1 \xrightarrow{\mathsf{ddep}} \nu_2$ means that 
some transaction $\tsid$ touches both versions. However, it reads $\nu_2$ and 
writes $\nu_1$.}
Given $\hh \in \HisHeaps$, the set of views that are \emph{causally consistent} with respect to $\hh$, $\func{CCViews}{\hh}$, is defined as the smallest set such that: 
\begin{enumerate} 
\item the initial view \( \vi_0\)  is in the set, \ie $\vi_0 \in \func{CCViews}{\hh}$ where \( \fora{\addr \in \dom(\hh)} \vi_{0}(\addr) = 1 \),
\item assume any view $\vi \in \pred{CCViews}{\hh}$ and a new view \( \vi' \) by observing one more version for an address $\vi' = \vi\rmto{\addr}{\vi(\addr) + 1}$, where \( \vi'(\addr) \leq \left| \hh(\addr) \right| \).
If some versions directly depend on the version corresponding to \( \vi'(\addr)\) and those versions are aware by \( \vi'\), the new view is included in \( \func{CCViews}{\hh}\),
%for some $i: V([\loc_x]) < i \leq \lvert \hh \rvert -1$
%for some $[\loc_x]$ such that $V([\loc_x]) < \lvert \hh([\loc_x]) \rvert - 1$.
%Suppose that 
\[
\begin{array}{@{}l}
\fora{\vi,\vi'} \exsts{\addr}
\vi \in \func{CCViews}{\hh} 
\land \addr \in \dom(\vi)
\land \vi' = \vi\rmto{\addr}{\vi(\addr) + 1}
\land \vi'(\addr) \leq \left| \hh(\addr) \right|  \\
\quad {} \land 
\begin{B}
\fora{\addr', i}  
1 \leq i \leq \left| \hh(\addr') \right|
\land \pred{ddep}{\hh(\addr')(i), \hh(\addr)(\vi'(\addr))}
\implies i \leq \vi(\addr')
\end{B} \\
\qquad {} \implies \vi' \in \func{CCViews}{\hh}
\end{array}
\]
%for any location $[\loc_y]$ and 
%index $j = 0, \cdots, \lvert \hh([\loc_y]) \rvert -1$ such that $\hh([\loc_{x})(V'([\loc_x]))$ 
%directly depends on $\hh([\loc_y])(j)$, then $j \leq V([\loc_y])$. Then 
%$V' \in \mathsf{CCViews}(\hh)$.
% and suppose that $\hh([\loc_{x}])(V'([\loc_x])) = 
%(\_, \tsid, \_ )$ for some $\tsid$. If for all locations $\loc_{y}$ and 
%indexes $j$ such that $\hh([\loc_y])(j) = (\_, \_, \_ \cup \{\tsid\})$, 
%then $j \leq V'([\loc_y])$, then $V' \in \mathsf{CCViews}(\hh)$.
\end{enumerate}
\end{defn}

