\section{Semantics}
\label{sec:semantics}

%\sx{
    %For Philippa and others who read Andrea's report, this is a re-edit version of the semantics, mainly for the reason of properly definition blocks, references, etc.
    %Steal many words from that report and add more explanation.
    %Few notations difference,
    %\begin{itemize}
        %\item address \( \addr \) .
            %At some point we use \( l \)  for heap/history heap locations, yet there is a variable clash so using \( \addr \).
        %\item Transaction identifier \( \txid \).
        %\item \( \hh \)  history heap, and we assume index starts from 1 instead of 0.
            %We use superscripts \( \hhV(\addr)(3) \), \( \hhW(\addr)(3) \) and \( \hhR(\addr)(3) \) for the value, write, and reads of the version corresponding to the third version of the address \( \addr \) in the history \( \hh \).
        %\item \( \stk \) per-thread stack, and no transaction stack.
        %\item \( \vi \) single view, as \( \val \) is for value and \( V \) looks like a set.
        %\item \( (\otR, \addr, \val ) \) denotes a operation read address \( \addr \) with value is \( \val \) and similarly the write \( (\otW, \addr, \val ) \)
            %Also we use, for example \( \opset \addO (\otR, \stub, \stub) \) instead of \( \oplus \), as the latter looks like a commutative operator??
        %\item We call \( \funcn{op}\) instead of \( \funcn{Fprint} \) for the function extract the operation/fingerprint from the primitive commands.
            %Because we use \emph{fingerprint} to refer the \emph{fingerprint assertion} later in the logic.
        %\item We call \( \func{localHp}{\hh, \vi} \) instead of \( \func{snapshot}{\hh, \vi} \) as the latter is a bit misleading.
        %\item \( \func{updHisHp}{\hh, \vi, \txid, \opset} \) instead of \( \funcn{HHupdate}_{\txid}(\hh, \vi, \opset)\).
        %\( \func{updView}{\hh, \vi, \opset}\) instead of  \( \func{ViewUpdate}{\hh, \vi, \opset} \), but we also assume the \( \hh \) here is the new one not the old one.
        %\item Consistency model \( \como \) as it is a set of quadruples and \( C \) is used for command, so we pick a capital \( \como \).
    %\end{itemize}
    %\ac{
        %\( \hh(\addr, 3).\texttt{val} \) instead of \( \hhV(\addr)(3) \) 

        %\( \vartriangleleft \) to \( \addO \) so that not clash with \( \csat \)
        
        %\emph{Heaps} in general are not sure, maybe \emph{key-value stores}. Just a terminology change.

        %Commit tests/Execution tests \( \como \) than consistency models.
        %}
    %}

%\sx{The small intro is stolen from Andrea :)}
We focus on an abstract computational model for database where multi-threaded programs can access and update addresses in a heap through atomic transactions. 
Transactions in our model execute atomically, though they have different effect on database depending on the consistency model that does not necessarily correspond to \emph{serialisability}. 
This means that, at the moment of executing, a transaction may not observe the most up-to-date value of an address. 
To overcome this issue, we model the state of the database using \emph{history heaps}. 
A history heap keeps track of all the versions written for any address, as well as the information about the transactions that read and wrote such versions. 
To model the potential out-of-date observation, we use \emph{views}.
A view decides the observable versions of addresses for a thread.
When executing, a transaction extracts a local state from the history heap and the view, and afterwards the transaction commits a set of operations that might change the history heap and the view, if the change is allowed by the consistency model.

We starts with the syntax of programs followed by the semantics of transaction.
Then we will formally define history heaps and views, and how we specify consistency models.
Finally, we will give the semantics for the entire programs.


\subsection{Programming Language and Operational Semantics}

We define a simple programming language for client programs interacting with an MKVS where clients may only interact with the MKVS via transactions. 
For simplicity, we abstract away from aborting transactions: rather than assuming that a transaction may abort due to a violation of the consistency guarantees given by the MKVS, we only allow the execution of a transaction when its effects are guaranteed to not violate the consistency model. This emulates the setting in which clients always restart a transaction if it aborts.
\sx{The sentence is too long. I think before the ``:'' is enough}
 
\subsubsection{Programming Language}

\emph{A program} \( \prog \) contains a fixed number of clients, where each client is associated with a unique identifier \( \thid \in \ThreadID \), and executes a sequential \emph{command}.
We thus model a program $\prog$ as a function from client identifiers  to commands (\cref{def:language}).
For clarity, we often write \( \cmd_{1}\ppar \dots \ppar \cmd_{n}\) as a syntactic sugar for a program \( \prog \) with $n$ implicit clients associated with identifiers, $\thid_1 \dots \thid_n$, where each client $\thid_i$ executes the command $\cmd_i$:  \( \prog = \Set{\thid_{1} \mapsto \cmd_{1}, \dots, \thid_{n} \mapsto \cmd_{n}  }\).
Sequential \emph{commands}, ranged over by $\cmd$, are defined by an inductive grammar comprising the standard constructs of $\pskip$, sequential composition ($\cmd; \cmd$), non-deterministic choice ($\cmd+\cmd$) and loops ($\cmd^*$).
To simulate conditional branching and loops, commands include the standard constructs of assume (\( \passume{\expr}\)), and assignment (\( \passign{\var}{\expr} \)), where \( \var \) denotes a local variable (on the stack), and \( \expr \) denotes an arithmetic expression with no side effect.  
evaluated with respect to a stack  with no side effect.
Arithmetic expressions are evaluated with respect to a local stack (variable store) -- see  \cref{def:stacks,def:eval-expr} below.
We assume a countably infinite set of local variables, $\Vars $, and use the \texttt{typewriter} font for its meta variables, \eg $\var$. 

Commands additionally include the \emph{transaction} construct, $\ptrans{\trans}$, denoting the \emph{atomic} execution of the transaction $\trans$. 
The atomicity guarantees the execution are dictated by the underlying consistency model.
\emph{Transactions}, ranged over by $\trans$, are similarly defined by an inductive grammar comprising $\pskip$, \emph{primitive commands} \( \transpri \), non-deterministic choice, loops and sequential composition.
Primitive commands include assignment (\( \passign{\var}{\expr}\)), lookup (\( \pderef{\expr}{\expr}\)), mutation (\( \pmutate{\expr}{\expr}\)) and assume (\( \passume{\expr}\)). 
Note that transactions do \emph{not} contain the \emph{parallel} composition construct ($\ppar$) as they are executed atomically.

%We assume a valid transactions codes must have the only return at the end.
%For better presentation, sometime we omit the default return zero \( \preturn{0} \).
%Transactions can only assign to their own variables, namely transaction variables (\defref{def:program_values}), but it can read from both the thread and transaction stacks.

%\begin{definition}[Program values]
%where $ \nat \in \Nat$ denotes the set of natural numbers.
%\end{definition}

\begin{definition}[Programming language]
\label{def:language}
\label{def:program_values}
A \emph{program}, $\prog \in \Programs$, is a partial finite function from client identifiers to commands.
The sequential \emph{commands}, \( \cmd \in \Commands \), are defined by the following grammar, where
$\val \in \Val \eqdef \Nat \cup \Addr$ denotes the set of \emph{program values}, and $\addr \in \Addr$ denotes the set of MKVS keys (\cref{def:mkvs}):
\[
\begin{array}{@{} l @{\hspace{20pt}}  l @{}}
    \begin{rclarray}
    \cmd & ::= &
        \pskip \mid 
        \passign{\thvar}{\expr} \mid
        \passume{\expr} \mid
        \ptrans{\trans} \mid 
        \cmd \pseq \cmd \mid 
        \cmd \pchoice \cmd \mid 
        \cmd \prepeat \\
%
	 \expr & ::= &
        \val \mid
        \var \mid
        \expr + \expr \mid
        \expr \times \expr \mid
        \dots  
       \end{rclarray} 
%    
	& 
%
	\begin{rclarray}        
	\trans & ::= &
        \pskip \mid
        \transpri \mid 
        \trans \pseq \trans \mid
        \trans \pchoice \trans \mid
        \trans\prepeat   \\
%        
	\transpri & ::= &
        \pass{\txvar}{\expr} \mid
        \pderef{\txvar}{\expr} \mid
        \pmutate{\expr}{\expr} \mid
        \passume{\expr} 
%
    \end{rclarray}
\end{array} 
\]
%The $\trans \in \Transactions$ in the grammar above denotes a \emph{transaction} defined by the following grammar including the primitive transactional commands \( \transpri\):
%The transaction codes \( \ptrans{\trans} \) satisfy a well-form condition that there is exactly a return at the end, \ie \( \ptrans{\trans} \iff \exsts{\trans', \expr} \trans \equiv ( \trans' \pseq \preturn{\expr} )  \land \pred{noRet}{\trans'} \)
%\[
%    \begin{rclarray}
%        \transpri & ::= &
%        \pass{\txvar}{\expr} \mid
%        \pderef{\txvar}{\expr} \mid
%        \pmutate{\expr}{\expr} \mid
%        \passume{\expr} \mid \\
%        %\preturn{\expr} \\
%        \trans & ::= &
%        \pskip \mid
%        \transpri \mid 
%        \trans \pseq \trans \mid
%        \trans \pchoice \trans \mid
%        \trans\prepeat
%    \end{rclarray}
%\]
%Given the set of \emph{keys}, $\addr \in \Addr$ (\cref{def:mkvs}), the set of \emph{program values} is $\val \in \Val \eqdef \Nat \cup \Addr$.
%The $\expr \in \Expressions$ denotes an \emph{arithmetic expression} defined by the grammar:
%\[
%    \begin{rclarray}
%        \expr & ::= &
%        \val \mid
%        \var \mid
%        \expr + \expr \mid
%        \expr \times \expr \mid
%        \dots 
%    \end{rclarray}
%\]
\end{definition}

\begin{definition}[Stacks]
\label{def:stacks}
\label{def:eval-expr}
A \emph{stack}, $\stk \in \Stacks$, is a partial finite function from variables to values: \(\Stacks \defeq \Vars \parfinfun \Val \).
Given a stack $\stk \in \Stacks$, %(\cref{def:stacks}), 
the \emph{arithmetic expression evaluation} function, $\evalE[(.)]{.}:\Expressions \times \Stacks \parfun \Val$, is defined inductively over the structure of expressions: 
%
\[
	\evalE{\val}  \defeq  \val 
	\qqquad 
	\evalE{\var} \defeq \stk(\var) 
	\qqquad 
	\evalE{\expr_{1} + \expr_{2}}  \defeq  \evalE{\expr_{1}} + \evalE{\expr_{2}} 
	\qqquad
	\evalE{\expr_{1} \times \expr_{2}}  \defeq  \evalE{\expr_{1}} \times \evalE{\expr_{2}} 
	\qqquad 
	\dots
\]
\end{definition}

%\begin{definition}[Evaluation of expression]
%\label{def:eval-expr}
%Given a stack $\stk \in \Stacks$, %(\cref{def:stacks}), 
%the \emph{arithmetic expression evaluation} function, $\evalE[(.)]{.}:\Expressions \times \Stacks \parfun \Val$, is defined inductively over the structure of expressions: 
%%
%\[
%	\evalE{\val}  \defeq  \val 
%	\qquad 
%	\evalE{\var} \defeq \stk(\var) 
%	\qquad 
%	\evalE{\expr_{1} + \expr_{2}}  \defeq  \evalE{\expr_{1}} + \evalE{\expr_{2}} 
%	\qquad
%	\evalE{\expr_{1} \times \expr_{2}}  \defeq  \evalE{\expr_{1}} \times \evalE{\expr_{2}} 
%	\qquad 
%	\dots
%\]
%%\[
%%    \begin{rclarray}
%%        \evalE{\val} & \defeq & \val \\
%%        \evalE{\var} & \defeq & \stk(\var) \\
%%        \evalE{\expr_{1} + \expr_{2}} & \defeq & \evalE{\expr_{1}} + \evalE{\expr_{2}} \\
%%        \evalE{\expr_{1} \times \expr_{2}} & \defeq & \evalE{\expr_{1}} \times \evalE{\expr_{2}} \\
%%        \dots & \eqdef & \dots \\
%%    \end{rclarray}
%%\]
%\end{definition}

\subsubsection{Semantics for transactions}
\label{sec:trans-semantics}

\paragraph{Operations and Fingerprints}
Recall that when a transaction starts, it extracts a local snapshot $\ss$ of the MKVS via its view. 
The execution of the transaction is then carried out \emph{locally} on $\ss$:  
read operations inspect $\ss$ and write operations update $\ss$. 
%, as we discuss in \cref{sec:prog-semantics}.
To track the effect of a transaction while executing, each transaction is associated with a \emph{fingerprint}, initially set to empty. 
The fingerprint of a transaction records its interactions (read or write operations) with the MKVS.
Each time a transaction executes a primitive command \( \transpri\) (locally on its snapshot), its fingerprint is accordingly updated to track the effect of $\transpri$.
%More concretely, every time the transaction performs a write operation, 
Once the local execution of a transaction is complete, its local snapshot is discarded, and the transaction is committed by propagating the effects in its fingerprint to the MKVS.\\
%
\indent The fingerprint of a transaction, $\opset$, is modelled as a set of \emph{operations}. 
Each operation $\op$ is a tuple of the form $(\mathit{tag}, \ke, \val)$, where $\mathit{tag}$ denotes the operation type and may be one of $\otW$ for \emph{write} operations, or $\otR$ for \emph{read} operations; 
the $\ke$ denotes the operation key, and  $\val$ denotes the operation value. 
For instance, $(\otW, \ke, \val)$ denotes the execution of an operation where value $\val$ is written to key $\ke$; and
$(\otR, \ke, \val)$ denotes the execution of an operation where value $\val$ is read from key $\ke$.\\
%
\indent When tracking the write operations (with tag $\otW$) in a fingerprint $\opset$, for each key $\ke$, only the effect of the \emph{last} write to $\ke$ is recorded in $\opset$. 
This is because while executing, a transaction may write to $\ke$ multiple times. Upon committing however, only the last value written to $\ke$ is propagated to the MKVS. 
As such, $\opset$ need not include duplicate writes operations for  $\ke$ and thus only records its latest write. 
This choice is motivated by the atomic visibility of transactions: the intermediate writes to a key are not visible from outside a transaction.\\
%
\indent Analogously, when tracking the read operations (with tag $\otR$), for each key $\ke$, the fingerprint $\opset$ includes a read operation for $\ke$ \emph{only if} the transaction reads from $\ke$ \emph{before} subsequently writing to it.
More concretely, a read $\op$ operation from $\ke$ is recorded in $\opset$ if $\op$ reads from the initial value recorded in the snapshot. % $\ss$. 
That is, read operations track those operations that read from the MKVS and not those that read internally from the updated snapshot.
These two principles, \emph{last-write} and \emph{read-before-write}, are formalised in the definition of {fingerprint update}, $\opset \addO \op$,  in \cref{def:ops} below. 
For convenience, we define $\opset \addO \emptyop \eqdef \opset$, to denote fingerprint update with respect to an operation with no effect.


In what follows we write $\op \projection{i}$ to denote the $i$\textsuperscript{th} projection of $\op$. 
We lift this notation to sets and write \eg $\{\op_1 \cdots \op_n\}\projection{i}$ for the set comprising the $i$\textsuperscript{th} projection of $\op_1 \cdots \op_n$. 




%The fingerprints include the \emph{first read preceding a write} and \emph{last write} for each key.
%This is because a transaction is executed atomically, all the intermediate steps are not observable from the outside world.
%The \emph{fingerprints} formally is a set of \emph{operations} \( \Ops \) which are either read \( (\etR, \addr, \val)\) from the \( \ke \) with the value \( \val \), or write \( (\etW, \addr, \val) \) to the key \( \ke \) with the value \( \val \) (\cref{def:ops}).


%
%Note that In the \cref{def:ops}, the \( (.)\projection{(.)} \) denotes projection.
%For a tuple, for example \( \op\projection{i} \), it gives the \emph{i-th} element of the tuple.
%It is lifted to a set of tuples, for example \( \opset\projection{i}\), which gives a set of all the \emph{i-th} elements.
%The well-formedness condition for fingerprints asserts it is a set of operations in which there are at most one read and one write for each key.
%The composition, then, is defined as set disjointed union as long as the result is well-formed.
 

\begin{definition}[Operations and fingerprints]
\label{def:ops}
The set of \emph{operations} is \( \op \in \Ops \eqdef \powerset{\Set{\otR, \otW}} \times \Addr \times \Val\).
%A \emph{transaction operation} is a tuple of \emph{an operation tag} that is either read or write, an key and a value.
%\[
%\begin{rclarray}
%\op \in \Ops & \defeq  & \powerset{\Set{\otR, \otW}} \times \Addr \times \Val
%\end{rclarray}
%\]
\emph{A fingerprint}, \( \opset \in \Opsets \), is a subset of \( \Ops \) in which any two elements contain either different tags or different keys:
\[
    \begin{rclarray}
        \Opsets & \defeq & \Setcon{\opset}{%
            \opset \subseteq \Events \land \fora{\op, \op' \in \opset} 
            \op\projection{1} \neq  \op'\projection{1} \lor \op\projection{2} \neq  \op'\projection{2}  } \\
    \end{rclarray}
\]
%
The \emph{fingerprint update function}, $\addO: \Opsets \times \Ops \cup \{\emptyop\} \rightarrow \Opsets$, is defined as follows: 
\[
\begin{array}{c @{\qqqquad} c}
\begin{rclarray}
    \opset \addO (\etR, \addr, \val) & \defeq & 
    \begin{cases}
        \opset \uplus \Set{(\etR, \addr, \val)} & (\stub, \addr, \stub) \notin \opset \\
        \opset &  \text{otherwise} \\
    \end{cases} \\
\end{rclarray}
&
\begin{rclarray}
    \opset \addO (\etW, \addr, \val) & \defeq & \left( \opset \setminus \Set{(\etW, \addr, \stub)} \right) \uplus \Set{(\etW, \addr, \val)} \\
    \opset \addO \emptyop & \defeq & \opset \\
\end{rclarray}
\end{array}
\]


The \emph{fingerprint unit element} is \( \unitE \defeq \emptyset\); 
the \emph{fingerprint composition}, $\composeO: \Opsets \times \Opsets \rightharpoonup \Opsets$,  is defined when two fingerprints contain operations with distinct keys: 
\[ 
\begin{rclarray}
    \opset \composeO \opset' & \defeq & 
    \begin{cases}
        \opset \uplus \opset' & \text{if } \opset\projection{2} \cap \opset'\projection{2} = \emptyset \\
        \text{undefined} & \text{otherwise}
    \end{cases}
\end{rclarray}
\]
\end{definition}
%\pg{ \(\opset \addO \opset'\)?. \sx{We can define this but not sure it is useful}}
%
%\begin{lemma}
%The well-formedness of fingerprints is closed under \( \addO \).
%\end{lemma}
%\azalea{What is $\addO$?? You have not defined this yet.}

\paragraph{Operational Semantics of Transactions}
The \emph{operational semantics of transactions} (\cref{def:language}) is given at the bottom of  \cref{fig:transaction_semantics},
described with respect to a transactional state of the form \((\stk, \sn, \opset)\), where $\stk$ denotes a local variable stack (\cref{def:stacks}), $\sn$ denotes an MKVS snapshot (\cref{def:snapshot}), and $\opset$ denotes a fingerprint (\cref{def:ops}).\\
%
\indent To this end, we first define a \emph{local state transformer} on pairs of stacks and snapshots for primitive commands \(\trans_{p}\) (top of  \cref{fig:transaction_semantics}).
More concretely, we write $(\stk, \h)  \toLTS{\trans_{p}} (\stk', \h')$ to denote that executing $\trans_p$ updates the stack $\stk$ and snapshot $\h$, to $\stk'$ and $\h'$, respectively.
Additionally, we compute the fingerprint (effect) of a primitive command via the $\funcFont{fp}$ function and write $\func{fp}{\stk, \h, \trans_p}$ to denote the effect of $\trans_p$ on stack $\stk$ and snapshot $\h$.\\
%
%We also define its fingerprint by \( \funcn{fp} \) function, which denotes the contribution of the primitive command that might observed by the external environment, \ie transactions from other threads.
%The \( \funcn{fp} \) extracts the read or write operation from loop-up and mutation respectively, otherwise \( \emptyop \).
%We also define a binary operator \( \opset \addO \op \) that specifies the effects of adding a new operation \( \op \) to the fingerprints \( \opset \).
%If the new operation is a read, for example \((\otR, \addr, \val)\) where \( \addr \) is the key and \( \val\) is the associated value, and there is no other operation related to the same key, this new read operation will be included in the result.
%Meanwhile, if the new operation is a write, it will overwrite all preview write operations to the same key.
%This ensures the fingerprints contains only the first read preceding a write, and only the last write for each key.
%This choice is motivated by the fact that we only focus on atomically visible transactions: keys are read from a snapshot of the database, and new version are written only at the moment the transaction commits.
%For technical reasons, if the right hand side is a special token \( \emptyop \) corresponding to a command does not result in an interaction with key-value store.
%Therefore, the semantics for primitive command \(\rl{TPrimitive}\) updates the stack and heap by the transformers relation and updates the operation set by first extracting the operation and adding it via \( \addO \) operator.
%The semantics for non-deterministic choices \(\rl{TChoice}\), sequential compositions \(\rl{TSeqSkip}\) and \(\rl{TSeq}\), and iteration \(\rl{TIter}\) have the expected behaviours.
%
\indent Each step of the operational semantics updates the stack and snapshot using the local state transformer, and updates the fingerprint via the \( \addO \) function.
The behaviour of all transitions in \cref{fig:transaction_semantics} is standard.


\begin{figure}[!t]
\hrule%\vspace{5pt}
%\begin{flushleft}
%The state transformers on pairs of stacks and snapshots for the primitive commands \(\trans_{p}\) (left), and the \( \funcn{op} \) for extracting the operation from the primitive commands (right):
%\end{flushleft}
\[
\begin{array}{@{} c @{\qquad} c @{}}
\begin{rclarray}
(\Stacks \times \Heaps)\!\!\! & \toLTS{\trans_p} &   (\Stacks \times \Heaps)  \vspace{5pt}\\
(\stk, \h)  & \toLTS{\passign{\var}{\expr}}          & (\stk\rmto{\var}{\evalE{\expr}}, \h)                  \\
(\stk, \h)  & \toLTS{\pderef{\var}{\expr}}           & (\stk\rmto{\var}{\h(\evalE{\expr})}, \h)              \\
(\stk, \h)  & \toLTS{\pmutate{\expr_{1}}{\expr_{2}}} & (\stk, \h\rmto{\evalE{\expr_{1}}}{\evalE{\expr_{2}}}) \\
(\stk, \h)  & \toLTS{\passume{\expr}}                & (\stk, \h) \text{ where } \evalE{\expr} \neq 0          
%(\stk, \h) & \toLTS{\preturn{\expr}}                & (\stk\rmto{\ret}{\evalE{\expr}}, \h)                 
\end{rclarray}                                                                                               
&
\begin{array}{@{} l @{}}
\funcFont{fp}: \Stacks \times \Heaps \times \trans_p \rightarrow \Ops \cup \{\emptyop\} \vspace{5pt} \\
\begin{rclarray}
\func{fp}{\stk, \h, \passign{\var}{\expr}}          & \defeq & \emptyop                                     \\
\func{fp}{\stk, \h, \pderef{\var}{\expr}}           & \defeq & (\etR, \evalE{\expr}, \h(\evalE{\expr}))     \\
\func{fp}{\stk, \h, \pmutate{\expr_{1}}{\expr_{2}}} & \defeq & (\etW, \evalE{\expr_{1}}, \evalE{\expr_{2}}) \\
\func{fp}{\stk, \h, \passume{\expr}}                & \defeq & \emptyop                                     \\
\func{fp}{\stk, \h, \preturn{\expr}}                & \defeq & \emptyop                                     
\end{rclarray}
\end{array}
\end{array}
\]

%\hrule\vspace{5pt}
%\begin{flushleft}
%The binary operator \( \opset \addO \op \) that specifies the effects of adding a new operation \( \op \) to the set \( \opset \):
%\end{flushleft}
%\[
%\begin{array}{c @{\qquad} c}
%\begin{rclarray}
%    \opset \addO (\etR, \addr, \val) & \defeq & 
%    \begin{cases}
%        \opset \uplus \Set{(\etR, \addr, \val)} & (\stub, \addr, \stub) \notin \opset \\
%        \opset &  \text{otherwise} \\
%    \end{cases} \\
%\end{rclarray}
%&
%\begin{rclarray}
%    \opset \addO (\etW, \addr, \val) & \defeq & \left( \opset \setminus \Set{(\etW, \addr, \stub)} \right) \uplus \Set{(\etW, \addr, \val)} \\
%    \opset \addO \emptyop & \defeq & \opset \\
%\end{rclarray}
%\end{array}
%\]

\hrule%\vspace{5pt}
%\begin{flushleft}
%Given the set of stacks \( \Stacks \) (\cref{def:stacks}), heaps \( \Heaps \) (\cref{def:heaps}) and transactions \( \Transactions \) (\cref{def:language}) and the arithmetic expression evaluation \( \evalE{\expr} \) (\cref{def:language}), the \emph{operational semantics of transactions}:
%\end{flushleft}
\[
\begin{rclarray}
\toL & : & ((\Stacks \times \Heaps \times \Opsets) \times \Transactions) \times ((\Stacks \times \Heaps \times \Opsets) \times \Transactions)
\end{rclarray}
\]

\begin{mathpar}
    \inferrule[\rl{TPrimitive}]{%
        (\stk, \h) \toLTS{\transpri} (\stk', \h')
        \\ \op = \func{fp}{\stk, \h, \transpri}
    }{%
        (\stk, \h, \opset) , \transpri \ \toL \  (\stk', \h', \opset \addO \op) , \pskip \rangle
    }
    \and
    \inferrule[\rl{TChoice}]{
        i \in \Set{1,2}
    }{%
        (\stk, \h, \opset) , \trans_{1} \pchoice \trans_{2} \ \toL \  (\stk, \h, \opset) , \trans_{i}
    }
    \and
    \inferrule[\rl{TIter}]{ }{%
        (\stk, \h, \opset),  \trans\prepeat \ \toL \  (\stk, \h, \opset), \pskip \pchoice (\trans \pseq \trans\prepeat)
    } 
    \and
    \inferrule[\rl{TSeqSkip}]{ }{%
        (\stk, \h, \opset), \pskip \pseq \trans \ \toL \  (\stk, \h, \opset), \trans
    }
    \and
    \inferrule[\rl{TSeq}]{%
        (\stk, \h, \opset), \trans_{1} \ \toL \  (\stk', \h', \opset'), \trans_{1}'
    }{%
        (\stk, \h, \opset), \trans_{1} \pseq \trans_{2} \ \toL \  (\stk', \h', \opset'), \trans_{1}' \pseq \trans_{2}
    }
\end{mathpar}
\hrule
\caption{Operational semantics of transactions}
\label{fig:transaction_semantics}
\end{figure}


%\subsection{Program Semantics}

To model the global machine states, instead of heap-based states, we use \emph{abstract executions} (\defref{def:abs-exec}).
A \emph{abstract execution} is a graph where each node represents a committed transaction with a unique transaction identifier, and its associated events that have global effect, \ie the first reads and last writes.
There are three types of edges in the graph, a \emph{program order} that is a total order for transactions from the same thread, a \emph{visibility relation} that decides the observable history (a set of transactions) for each transaction, and an \emph{arbitration order} that decides the actual global state that is not necessary to be the same as the observable state for each transaction \cite{eventually-consistent-transactions,Burckhardt:2014:RDT:2535838.2535848,cerone_et_al:LIPIcs:2015:5375}.

\begin{defn}[Runtime abstract executions and abstract executions]
\label{def:run-abs-exec}
\label{def:abs-exec}
Assuming a set of \emph{transactions identifiers} \( \TxID \defeq \Set{\txid, \dots}\), the \emph{transactions} \( \tx \in \Tx \) is defined as a finite partial function from transactions identifiers \( \TxID \) to valid sets of events \( \Evsets \),
\[
\begin{rclarray}
\tx \in \Tx & \defeq & \TxID \parfinfun \Evsets
\end{rclarray}
\]
An \emph{runtime abstract execution} is a tuple \( \aexecrun = (\tx, \setthid, \porun, \vis, \ar) \in \Aexecrun \) that satisfies the following conditions,
\begin{itemize}
\item
The \emph{transactions} \( \tx \) is a partial function from transaction identifiers to their corresponding events (\defref{def:transactions}).
\item 
Assume a countably infinite set of thread identifiers \( \thid \in \setthid \subseteq \ThreadID \). 
The \emph{runtime threads} \( \setthid \) is a set of transaction identifiers.
\item 
The \emph{arbitration order} $\ar \subseteq \dom(\tx) \times \dom(\tx)$ is a strict, total order%
\footnote{Recall that a relation $R \subseteq A \times A$ is a strict partial order if it is irreflexive and transitive.
It is a strict total order if for any $a_1, a_2 \in A$, either $a_1 = a_2$, $(a_1, a_2) \in R$ or $(a_2, a_1) \in R$.}.
\item 
The \emph{runtime program order} $\porun \subseteq \dom(\tx) \times ( \dom(\tx) \uplus \setthid )$ is the union of several disjoint, strict total orders \( \porun_{i} \).
That is, there exists a partition $\Set{ \dom(\tx)_{i} }_{i \in I}$ of $\dom(\tx)$ such that $\porun = \biguplus_{i \in I} \porun_{i}$, where $\porun_{i}$ is a strict, total order.
It also requires \( \po \subseteq \ar\).
\item 
The \emph{visibility relation} $\vis \subseteq \dom(\tx) \times \dom(\tx)$ is a relation such that \( \vis \subseteq \ar \).
\end{itemize} 
The set of \emph{abstract executions} $\aexec  = (\tx, \po, \vis, \ar) \in \Aexecs$, where the \emph{program order} \( \po \subseteq \dom(\tx) \times \dom(\tx)\), is defined by erasing the runtime threads from the runtime abstract executions \( \Aexecrun \),
\[
\begin{rclarray}
    \Aexecs & \defeq & \Setcon{\eraseAEX{\aexecrun}}{\aexecrun \in \Aexecrun} \\
\end{rclarray} 
\]
where the erasing function \( \eraseAEX{.}: \Aexecrun \to \Aexecs \) converts a runtime abstract execution to an abstract execution by erasing the second element, \ie the set of thread identifiers, and also any thread identifiers from the runtime program order,
\[
    \begin{rclarray}
        \eraseAEX{(\tx, \setthid, \porun, \vis, \ar)} & \defeq & (\tx, \hat{\po} \setminus \Setcon{(\txid, \thid)}{ \txid \in \dom(\tx) \land \thid \in \setthid}, \vis, \ar)
    \end{rclarray}
\]
\begin{figure}
\centering
\begin{tikzpicture}
\node[draw] (n1) { \( \txid_{1}:\emptyset\) };
\node[draw,below=1cm of n1] (n2) { \( \txid_{2}:\emptyset\) };
\node[draw,below=1cm of n2] (n3) { \( \txid_{3}:\emptyset\) };
\coordinate (n1n3) at ($(n2) + (1.5,0)$);
\draw[->] (n1) -- (n2) node[pos=0.5]{\ar,\vis};
\draw[->] (n2) -- (n3) node[pos=0.5]{\ar};
\draw[->] (n1) to[out=-20, in=90] (n1n3) to[out=-90,in=20]  (n3);
\node at (n1n3) {\(\ar, \vis\)};
\node[below] at (current bounding box.south) {\(\aexec\)};
\end{tikzpicture}
%
\begin{tikzpicture}
\node[draw] (n1) { \( \txid_{1}:\Set{(\etW, 1, 0), (\etW, 2, 0)}\) };
\node[draw,below=1cm of n1] (n2) { \( \txid_{2}:\Set{(\etR, 1, 0), (\etW, 2, 1)}\) };
\node[draw,below=1cm of n2] (n3) { \( \txid_{3}:\Set{(\etW, 1, 1), (\etR, 2, 0)}\) };
\coordinate (n1n3) at ($(n2) + (2.5,0)$);
\draw[->] (n1) -- (n2) node[pos=0.5]{\ar,\vis};
\draw[->] (n2) -- (n3) node[pos=0.5]{\ar};
\draw[->] (n1) to[out=-20, in=90] (n1n3) to[out=-90,in=20]  (n3);
\node at (n1n3) {\(\ar, \vis\)};
\node[below] at (current bounding box.south) {\(\aexec'\)};
\end{tikzpicture}
\caption{The abstract execution \( \aexec \) is a element of the unit set \( \unitAEX\), and here \( \aexec \composeAEX \aexec' = \aexec' \) as the two abstract executions have the same structure the compositions of each nodes exist.}
\end{figure}
For brevity, the \( \aexecrun\prjT \), \( \aexecrun\prjI \), \( \aexecrun\prjP \), \( \aexecrun\prjV \) and \( \aexecrun\prjA \) denote the corresponding elements in the tuple and similarly for \( \aexec\projection{(.)} \).
The composition of two runtime abstract executions, \( \composeAEXRUN : \Aexecrun \times \Aexecrun \parfun \Aexecrun \), is defined as the follows,
\[
\begin{rclarray}
    \aexecrun_{1} \composeAEXRUN \aexecrun_{2} & \defeq & 
    \begin{cases}
        \left( \lambda \txid \ldotp \aexecrun_{1}\prjT(\txid) \composeE \aexecrun_{2}\prjT(\txid), \aexecrun_{1}\prjI, \aexecrun_{1}\prjP, \aexecrun_{1}\prjV, \aexecrun_{1}\prjA \right) & \dagger \\
        \text{undefined} & \text{ otherwise}
    \end{cases} \\
    \dagger & \equiv &  
    \begin{array}[t]{@{}l@{}}
        \dom(\aexecrun_{1}\prjT) = \dom(\aexecrun_{2}\prjT)
        \land \aexecrun_{1}\prjI = \aexecrun_{2}\prjI \\
        \quad {} \land \aexecrun_{1}\prjP = \aexecrun_{2}\prjP
        \land \aexecrun_{1}\prjV = \aexecrun_{2}\prjV
        \land \aexecrun_{1}\prjA = \aexecrun_{2}\prjA \\
    \end{array} \\
\end{rclarray}
\]
The set of the units is \( \unitAEX \defeq \Setcon{\aexec}{\for{\txid} \aexec\prjT(\txid) = \unitE} \).
Then, the order between two runtime abstract executions \( \aexecrun_{1} \) and \( \aexecrun_{2} \) is defined as point-wise set inclusions,
\[
\begin{rclarray}
\aexecrun_{1} \ordAEXRUN \aexecrun_{2} & \iffdef & 
    \begin{array}[t]{@{}l@{}}
        ( \for{\txid} \aexecrun_{1}\prjT(\txid) \implies \aexecrun_{2}\prjT(\txid) )  
        \land \aexecrun_{1}\prjI \subseteq  \aexecrun_{2}\prjI  \\
        {} \quad \land \aexecrun_{1}\prjP \subseteq  \aexecrun_{2}\prjP 
        \land \aexecrun_{1}\prjV \subseteq  \aexecrun_{2}\prjV 
        \land \aexecrun_{1}\prjA \subseteq  \aexecrun_{2}\prjA 
    \end{array}
\end{rclarray}
\]
Last, by erasing the runtime, the order between two abstract executions is defined as the follows,
\[
\begin{rclarray}
\aexec_{1} \ordAEX \aexec_{2} & \iffdef & 
    \exsts{\aexecrun_{1}, \aexecrun_{2}} \aexec_{1} = \eraseAEX{\aexecrun_{1}}
    \land \aexec_{2} = \eraseAEX{\aexecrun_{2}}
    \land \aexecrun_{1} \ordAEXRUN \aexecrun_{2}
\end{rclarray}
\]
\end{defn}

We parametrise the consistency models in our semantics.
A \emph{consistency model} contains two parts, a \emph{resolution policy} and a \emph{consistency guarantee} (\defref{def:consistency-models}) \cite{cerone_et_al:lipics:2017:7794}.
Given a abstract execution \( \aexec \), a set of observable transactions \( \txidset \) and an address \( \addr \), the \emph{resolution policy} \( \respo(\aexec, \txidset, \addr) \) decides the observable values for the address \( \addr \) through some computation on the observable transactions \( \txidset \).
A common resolution policy is \emph{last-write-win} that if a transaction observes several writes for the same address, it always reads the last write (by the arbitration order).
The consistency guarantee gives the minimum constraint for the visibility relation.


\begin{defn}[Consistency Models]
\label{def:consistency-models}
A \emph{Consistency model} is a tuple, \( \como = (\respo, \conguar) \in \Como\), including a \emph{resolution policy} and a \emph{consistency guarantee}.
A \emph{resolution policy} \( \respo \) is a function such that for a given address \( \addr \) and a set of observable transactions from a abstract execution, it returns a set of possible observable values.
\[
\begin{rclarray}
    \ResPos & \defeq & 
    \Setcon{%
        \respo
     }{%
        \respo \in \Aexecs \times \powerset{\TxID} \times \Addr \to \powerset{\Val}\\
        \quad {} \land \for{\aexec, \txidset, \addr } \respo(\aexec, \txidset, \addr)\isdef \implies \txidset \subseteq \dom(\aexec\prjT)
    }
\end{rclarray}
\]
A \emph{consistency guarantee} \( \conguar \) is a function such that for an abstract execution, it returns a relation which corresponds to the minimum visibility relation.
\[ 
\begin{rclarray}
\ConGuar & \defeq & 
\Setcon{%
        \conguar
    }{%
        \conguar \in \Aexecs \to \powerset{\TransID \times \TransID}
        \land \for{\aexec} \\
        \quad \conguar(\aexec) \subseteq \aexec\prjA
        \land \for{\txid, \txid'} (\txid, \txid') \in \conguar(\aexec) 
        \land \txid,\txid'  \in \dom(\aexec\prjT)
        
    }
\end{rclarray}
\]
Note that a well-formed consistency guarantee must not violate the arbitrary order.
The order \( \ordCOM \)  between two consistency model \( \como_{1}, \como_{2} \) is defined as the follows,
\[
\begin{rclarray}
    (\respo_{1}, \conguar_{1}) \ordCOM (\respo_{2}, \conguar_{2}) & \iffdef & 
    \begin{array}[t]{@{}l@{}}
    \for{\aexec, \txidset, \addr} \\
    \quad \respo_{2}(\aexec, \txidset, \addr) \subseteq \respo_{1}(\aexecrun, \txidset, \addr) \land \conguar_{1}(\aexec) \subseteq  \conguar_{2}(\aexec)
    \end{array} \\
\end{rclarray}
\]
The bottom element is \( \btmCOM \defeq (\lambda (\aexec, \txidset, \addr ) \ldotp \Val, \lambda (\aexec) \ldotp \emptyset) \), which means one is able to observe any arbitrary value for each address and there is no constraint for visibility relation.
\end{defn}

\begin{example}[Last write win]
\[
\begin{rclarray}
        \respo_{LWW}(\aexec, \txidset, \addr) & \defeq & 
        \Setcon{\val}{%
            \exsts{\ar = \aexec\prjA}
            \Set{(\etW, \addr, \val)} = \max_{\ar}\Setcon{\txid}{\txid \in \txidset \land (\etW, \addr, \stub) \in \tx(\txid) } \\
            \quad {} \lor \emptyset = \max_{\ar}\Setcon{\txid}{\txid \in \txidset \land (\etW, \addr, \stub) \in \tx(\txid) } \land v = 0
        }\\
\end{rclarray}
\]
\end{example}

\begin{example}[Write-write conflict]
\[
\begin{rclarray}
        \conguar_{WW}(\aexec) & \defeq & \Setcon{(\txid, \txid')}{ \exsts{ \addr } (\etW, \addr, \stub) \in \aexec\prjT(\txid) \land (\etW, \addr, \stub) \in \aexec\prjT(\txid') \land (\txid, \txid') \in \aexec\prjA } \\
\end{rclarray}
\]
\end{example}

Given two sets of relations \( A \) and \( B \), the notation \( A ; B \) denotes that \( A;B \defeq \Setcon{(a,b)}{(a,c) \in A \land (c,b) \in B} \).

\begin{example}[Serialisibility(SER)]
\[
    \begin{rclarray}                                   
        \respo_{SER} & \defeq & \respo_{LWW} \\
        \conguar_{SER}(\aexec) & \defeq & \aexec\prjA \\
    \end{rclarray}                                                      
\]
\end{example}
%\ac{Why $\como_{SER}$ and not $\respo_{SER}?$ Also, the resolution policy is the last write wins, which you need to define only once.}

\begin{example}[Snapshot isolation(SI)]
\[
    \begin{rclarray}                                   
        \respo_{SI} & \defeq & \respo_{LWW} \\
        \conguar_{SI}(\aexec) & \defeq & ( \aexec\prjA ; \aexec\prjV )  \cup \conguar_{WW}(\aexec) \\
    \end{rclarray}
\]
\end{example}
%\ac{Why not just say that $\conguar_{SI}(\aexec) = \aexec\prjV ; \aexec\prjA$? Also, you are missing 
%write conflict detection ($\conguar_{\mathsf{WWconf}} = \bigcup_{\addr \in \Addr} [\mathsf{Write}_\addr] ; \aexec\prjA ; [\mathsf{Write}_\addr]$.}

\begin{example}[Parallel snapshot isolation(PSI)]
\[
    \begin{rclarray}                                   
        \respo_{PSI} & \defeq & \respo_{LWW} \\
        \conguar_{PSI}(\aexec) & \defeq & ( \aexec\prjV ; \aexec\prjV ) \cup \conguar_{WW}(\aexec) \\
    \end{rclarray}
\]
\end{example}

\begin{defn}[Thread transition labels]
\label{def:label}
Given the set of thread identifiers \(\ThreadID\) (\defref{def:abs-exec}), the set of \emph{thread transition labels}, $\lb \in \Translabel$, is defined by the following grammar, where $\prog$ denotes a program (\defref{def:language}), the $\txid$ demotes a transaction identifier and $\thstk$ denotes a thread stack (\defref{def:stacks}),
\[
    \begin{rclarray}
	\iota \in \Translabel & ::= & \lbID \mid \lbC{\txid} \mid \lbF{\thid,\prog} \mid \lbJ{\thid,\thstk}
    \end{rclarray}
\]
\end{defn}

\begin{defn}[Thread semantics]
\label{def:thread_semantics}
Given the thread identifiers $\thid \in \ThreadID$, the set of \emph{intermediate programs}, $\iprog \in \IntermediatePrograms$, is defined by the following grammar:
\[
    \iprog ::= \prog \mid \iprog \pseq \pwait{\thid}
\]
Given the set of consistency model \( \ConsisModels \) (\defref{def:consistency-models}), the set of thread stacks \( \ThdStacks \) (\defref{def:stacks}) and runtime abstract executions \( \Aexecrun \) (\defref{def:run-abs-exec}), the \emph{per-thread operational semantics} of programs,
\[
\begin{rclarray}
	\toT{} & : &
    \begin{array}[t]{@{}l@{}}
    \Como \times \ThreadID 
    \times \\
	\quad \left( ( \ThdStacks \times \Aexecrun ) \times \IntermediatePrograms \right) 
	\times  \Translabel \times
	\left( ( \ThdStacks \times \Aexecrun ) \times \IntermediatePrograms \right) 
    \end{array}
\end{rclarray}
\]
is defined in \figref{fig:thread_semantics}.
\end{defn}

The \rl{PCommit} rule ``substitutes'' the dummy node for the thread \( \thid \) in the runtime abstract execution \( \aexecrun \) with a newly allocated transaction \( \txid \) with its associated events \( \evset \).
To obtain the events \( \evset \), it prophesies a set of observable transactions \( \txidset \), which will be added into the visibility relation later.
Given the observable transactions, the \( \obsstateName \) function computes possible initial heaps, by applying the resolution policy \( \respo \) for each address.
Note that the resolution policy might return more than one value for an address, so there are more than many possible initial heaps.
Given the transaction code \( \trans \), first pick an initial heap \( \h \) and then given the transaction semantics (\figref{fig:transaction_semantics}), we can get the events set \( \evset \).
To \emph{extend the runtime abstract execution}, it replaces the dummy node \( \thid \) with the new transaction \( \txid \), links all transaction from the observable transactions \( \txidset \) to the new transaction, and puts the new transaction at the end of arbitration order.
Then, it extends the program order by adding back the dummy node \( \thid \) after the transaction \( \txid \) for preserving program order for the future transactions from the same thread.

The \rl{Par} rule forks a new thread and inserts the appropriate joining point by appending the auxiliary \( \pwait{\thid} \) command, where the parameter \(\thid\) denotes the identifier of the newly forked thread.
The \rl{Wait} rule dually awaits the termination of the child thread \(\thid\) indicated by the auxiliary \( \pwait{\thid} \) command, and subsequently updates the thread stack.
Note that these two rules are labelled with the \(\lbF{\thid, \prog}\) and \(\lbJ{\thid, \thstk}\) which are used by the semantics of the thread pool described shortly (\figref{fig:thread_pool_semantics}).

\begin{figure}
%
\hrule
%
\[
    \infer[\rl{PCommit}]{%
        (\respo, \conguar), \thid \vdash ( \thstk, \aexecrun ), \ptrans{\trans} \ \toT{\lbC{\txid}} \ ( \thstk\rmto{\ret}{\txstk(\ret)}, \aexecrun' ) , \pskip
    }{%
        \begin{array}{c}
            \txid \in \func{fresh}{\aexecrun}
            %\quad \txidset \in \addpotvis{\aexecrun, \thid}
            \quad \txidset \subseteq \dom(\aexecrun\prjT)
            \quad \h \in \obsstate{\aexecrun, \txidset, \respo} \\
            %\quad \fph = \lambda \addr \ldotp  (\respo(\eraseAEX{\aexecrun}, \txidset , \addr), \emptyset) \\
            \thstk \vdash (\emptyset, \h, \emptyset) , \trans \ \toL^{*} \  (\txstk, \h', \evset) , \pskip 
            %\quad \evset = \getevent{\fph, \fph'}
            \quad \aexecrun' = \newaexec{\aexecrun, \thid, \evset, \txid, A, \conguar }
        \end{array}
    }
\]

\[
    \infer[\rl{PAssign}]{%
        \como, \thid \vdash ( \thstk, \aexecrun ) , \passign{\thvar}{\expr} \ \toT{\lbID} \  ( \thstk\rmto{\thvar}{\val}, \aexecrun  ) , \pskip
    }{
        \val = \evalE[\thstk]{\expr}
        && \thvar \in \ThdVars
    }
\]

\[
    \infer[\rl{PAssume}]{%
        \como, \thid \vdash ( \thstk, \aexecrun ) , \passume{\expr} \ \toT{\lbID} \  ( \thstk, \aexecrun ) , \pskip
    }{%
        \evalE[\thstk]{\expr} = 0
    }
\]

\[
    \infer[\rl{PChoice}]{%
        \como, \thid \vdash ( \thstk, \aexecrun ) , \prog_{1} \pchoice \prog_{2} \ \toT{\lbID} \  ( \thstk, \aexecrun ) , \prog'
    }{
        \prog' \in \Set{\prog_{1}, \prog_{2}}
    }
\]

\[
    \infer[\rl{PLoop}]{%
        \como, \thid \vdash ( \thstk, \aexecrun ) , \prog\prepeat \ \toT{\lbID} \  ( \thstk, \aexecrun ) , \pskip \pchoice (\prog \pseq \prog\prepeat)
    }{}
\]

\[
    \infer[\rl{PSeqSkip}]{%
        \como, \thid \vdash ( \thstk, \aexecrun ) , \pskip \pseq \iprog \ \toT{\lbID} \  ( \thstk, \aexecrun ) , \iprog
    }{}
\]

\[
    \infer[\rl{PSeq}]{%
        \como, \thid \vdash ( \thstk, \aexecrun ) , \iprog_{1} \pseq \iprog_{2} \ \toT{\lb} \ ( \thstk', \aexecrun' ) , {\iprog_{1}}' \pseq \iprog_{2}
    }{%
        \como, \thid \vdash ( \thstk, \aexecrun ) , \iprog_{1} \ \toT{\lb} \  ( \thstk', \aexecrun' ) , {\iprog_{1}}' 
    }
\]

\[
    \infer[\rl{PPar}]{%
        \como, \thid \vdash ( \thstk, \aexecrun ) , \prog_{1} \ppar \prog_{2} \ \toT{\lbF{\thid', \prog_{2}}} \  ( \thstk, \aexecrun' ) , \prog_{1} \pseq \pwait{\thid'}
    }{
        \aexecrun' = \func{extend\_thread}{\aexecrun, \thid, \thid'}
    }
\]

\[
    \infer[\rl{PWait}]{%
        \como, \thid \vdash ( \thstk, \aexecrun ) , \pwait{\thid'} \ \toT{\lbJ{\thid', \thstk'}} \  (  \thstk_{1} \uplus \thstk_{2} \uplus \thstk_{f}, \aexecrun' ) , \pskip 
    }{
        \thstk = \thstk_{1} \uplus \thstk_{f}
        && \thstk' = \thstk_{2} \uplus \thstk_{f}
        && \aexecrun = \func{erase\_thread}{\aexecrun', \thid'}
    }
\]

\sx{The stack could be polluted by the child thread, but they must agree?}
 
where,
\[
\begin{rclarray}                                 
    \consis{\aexecrun, \thid, \txid, \conguar} & \defeq & 
    \begin{array}[t]{@{}l@{}}
        \thid \in \aexecrun\prjI \land  \for{ \txid' } ( (\txid', \txid) \in \conguar(\eraseAEX{\aexecrun}) \implies (\txid', \txid) \in \aexecrun\prjV ) \\
    \end{array} \\
    \newaexecName & : & 
    \left(\begin{array}{l}
        \Aexecrun \times \ThreadID \times \TxID \times \\
        \quad  \powerset{\Events} \times \powerset{\TxID} \times \ConGuar \end{array} \right)
        \parfun \Aexecrun \\
    \newaexec{\aexecrun, \thid, \txid, \evset, \txidset, \conguar } & \defeq & 
    \begin{cases}
        \aexecrun' & \consis{\aexecrun', \thid, \txid, \conguar} \\
        \text{undefined} & \text{otherwise} \\
    \end{cases} \\
    \aexecrun' & \equiv & 
        \left(
        \begin{array}{@{}l@{}}
            \aexecrun\prjT \uplus \Set{\txid \mapsto \evset},
            \aexecrun\prjI, 
            \aexecrun\prjP \uplus \Setcon{(\txid', \txid)}{(\txid', \thid) \in \aexecrun\prjP} \uplus \Set{(\txid, \thid)}, \\
            \quad \aexecrun\prjV \uplus \Setcon{(\txid', \txid)}{\txid' \in \txidset}, 
            \aexecrun\prjA \uplus \Setcon{(\txid', \txid)}{\txid' \in \dom(\aexecrun\prjT)}
        \end{array}
        \right) \\
%	
%
    \obsstateName & : & \Aexecrun \times \powerset{\TxID} \times \ResPos \parfun \powerset{\FPHeaps} \\
    \obsstate{\aexecrun, \txidset, \respo} & \defeq & 
    \Setcon{%
        \h
    }{%            
        \for{\addr, \val}  \val \in \respo(\eraseAEX{\aexecrun}, \txidset, \addr) \iff \h(\addr) = \val \\
    } \\
%
%              
	\func{fresh}{\aexecrun}  & \defeq & \Setcon{ \txid }{ \neg \txid \in \dom(\aexecrun\prjT) } \\
%
%
    %\geteventName & : & \FPHeaps \times \FPHeaps \to \powerset{\Events} \\
    %\getevent{\fph, \fph'} & \defeq & 
    %\begin{array}[t]{@{}l@{}}
        %\Setcon{ (\etW, \addr, \val ) }{ \exsts{ \fp } \fph(\addr) = ( \val, \fp ) \land \fpW \in \fp} \\
        %\quad {} \uplus \Setcon{ (\etR, \addr, \val ) }{ \exsts{ \fp } \fph(\addr) = (\val, \stub) \land \fph(\addr) = ( \stub, \fp ) \land \fpR \in \fp} \\ 
    %\end{array} \\
%
%
    \func{extend\_thread}{\aexecrun, \thid, \thid'} & \defeq & (\aexecrun\prjT, \aexecrun\prjI \uplus \Set{\thid'}, \aexecrun\prjP \uplus \Setcon{(\txid, \thid')}{ (\txid, \thid) \in \aexecrun\prjP}, \aexecrun\prjV, \aexecrun\prjA) \\
    \func{erase\_thread}{\aexecrun, \thid} & \defeq & (\aexecrun\prjT, \aexecrun\prjI \setminus \Set{\thid}, \aexecrun\prjP \setminus \Setcon{(\txid, \thid)}{ \txid \in \dom(\aexecrun\prjT)}, \aexecrun\prjV, \aexecrun\prjA)                                                                                                                                                                                                                   
    \end{rclarray}
\]
\hrule
\caption{Per-thread operational semantics}
\label{fig:thread_semantics}
\end{figure}

In order to model concurrency, we use thread pools.
A \emph{thread pool} is a finite partial function from thread identifiers to triples of the form \((\thstk, \iprog)\). That is, each thread is associated with a thread stack \(\thstk\) and an intermediate program \(\iprog\) to be executed. 

\begin{defn}[Thread pools]
\label{def:thread_pools}
Given the set of thread stacks $\ThdStacks$ (\defref{def:stacks}) and intermediate programs $\IntermediatePrograms$ (\defref{def:thread_semantics}), a \emph{thread pool} is a a finite partial function from thread identifiers to triples of thread stacks and intermediate programs, \(\thpl \in \TPool \eqdef \ThreadID \parfinfun \ThdStacks \times \IntermediatePrograms\).
\end{defn}
 
\begin{defn}[Thread pool semantics] 
\label{def:thread_pool_semantics}
Given the set of consistent models \( \ConsisModels \) (\defref{def:consistency-models}), runtime abstract executions \(\Aexecrun\) (\defref{def:abs-exec}), transition labels \( \Translabel \) (\defref{def:label}) and thread pools  \( \TPool \) (\defref{def:thread_pools}), the \emph{thread pool semantics}, 
\[
	\toG{} : \Como \times (\Aexecrun \times \TPool) \times \Translabel \times (\Aexecrun \times \TPool) 
\]
is defined in \figref{fig:thread_pool_semantics}.
\end{defn}
 
The thread pool operational semantics is given in \figref{fig:thread_pool_semantics}, where an arbitrary thread in the pool \(\thpl\) is picked to run for one step.
If the next execution step is a thread fork, then a new thread \(\thid'\) is allocated in the pool to be executed with its thread stack copied from its parent (forking) thread.
Conversely, when the next execution step is the joining of thread \(\thid'\), then \(\thid'\) is removed from the thread pool and the stack from the child thread merges into the parent thread.

\begin{figure}
\hrule\vspace{5pt}
%
\[
    \infer[\rl{PSingle}]{%
        \como \vdash ( \aexecrun, \thpl \uplus \Set{ \thid \mapsto (\thstk, \iprog) } ) \ \toG{\lb} \  ( \aexecrun, \thpl \uplus \Set{ \thid \mapsto (\thstk', {\iprog}') } ) 
    }{%
        \como, \thid \vdash ( \thstk, \aexecrun ) , \iprog \ \toT{\lb} \  ( \thstk', \aexecrun ' ) , {\iprog}' 
        \quad \lb \in \Set{ \lbID, \lbC{\stub} }
    }
\]

\[
    \infer[\rl{PFork}]{%
        \como \vdash ( \aexecrun, \thpl \uplus \Set{ \thid \mapsto (\thstk, \iprog) } ) \ \toG{\lbF{\thid', \prog}} \  ( \aexecrun', \thpl \uplus \Set{ \thid \mapsto (\thstk, {\iprog}'), \thid' \mapsto (\thstk, \prog) } )
    }{%
        \como, \thid \vdash ( \thstk, \aexecrun ) , \iprog \ \toT{\lbF{\thid', \prog}} \  ( \thstk, \aexecrun' ) , {\iprog}' 
    }
\]

\[
    \infer[\rl{PJoin}]{%
        \como \vdash ( \aexecrun, \thpl \uplus \Set{ \thid \mapsto (\thstk, \iprog), \thid' \mapsto (\thstk', \pskip) } )  \ \toG{\lbJ{\thid',\thstk''}} \ ( \aexecrun', \thpl \uplus \Set{ \thid \mapsto (\thstk'', {\iprog}')} )
    }{%
        \como, \thid \vdash ( \thstk, \aexecrun ) , \iprog \ \toT{\lbJ{\thid',\thstk'}} \  ( \thstk'', \aexecrun' ) , {\iprog}' 
    }
\]
%
\hrule\vspace{5pt}
\caption{Thread pool semantics}
\label{fig:thread_pool_semantics}
\end{figure}

\subsection{Soundness and Completeness}

Given a consistency model, we can define the set of abstract executions that satisfy a consistency model (\defref{def:valid-aexec}) \cite{cerone:snapshot,cerone_et_al:lipics:2017:7794,cerone_et_al:LIPIcs:2015:5375}.
The \lemref{lem:consistency-monotonicity} is for sanity check.

\begin{defn}[Valid abstract executions]
\label{def:valid-aexec}
Given a consistency model \( (\respo, \conguar) = \como \in \Como \) (\defref{def:consistency-models}), the set of valid abstract executions under the model, denoted by \( \evalCOM{\como} \),  is defined as the follows,
\[
    \begin{rclarray}
        \evalCOM{(\respo, \conguar)} & \defeq & 
        \Setcon{%
            \aexec = (\tx, \po, \vis, \ar)
        }{%
            \conguar(\aexec) \subseteq \vis 
            \land \for{\txid, \addr, \val}  \\
            \qquad (\etR, \addr, \val) \in \tx(\txid) 
            \implies \val \in \respo(\aexec, \aexec\prjV(\txid), \addr)
        }
    \end{rclarray}
\]
where \( \aexec\prjV(\txid) \) returns all the predecessors of \( \txid \) with respect to the visibility relation.
This is, for any relation \( R \),
\[
\begin{rclarray}
    R(x) & \defeq & \Setcon{ x' }{ (x', x) \in R}
\end{rclarray}
\]
\end{defn}    
 
\begin{lem}[Consistency models monotonicity]
\label{lem:consistency-include}
\label{lem:consistency-monotonicity}
The abstract executions allowed by a stronger consistency model is allowed by a weaker consistency model, this is,
\[
    \como_{1} \ordCOM \como_{2} \implies \evalCOM{\como_{2}} \subseteq \evalCOM{\como_{1}}
\]
\end{lem}
\begin{proof}
For any abstract execution \( \aexec \) that satisfies the stronger consistency model \( \como_{2} = (\respo_{2}, \conguar_{2}) \), first, it has \( \conguar_{2}(\aexec) \subseteq \aexec\prjV \) by the \defref{def:valid-aexec}.
Given the hypothesis \( \como_{1} \ordAEX \como_{2} \) and the order definition (\defref{def:consistency-models}), we know \( \conguar_{1}(\aexec)  \subseteq \conguar_{2}(\aexec)  \) so that \( \conguar_{1}(\aexec) \subseteq \aexec\prjV \).
Second, assume a transaction \( \txid \) and any of its read event with an address \( \addr \) and a \( \val \) such that \( (\etR, \addr, \val) \in \tx(\txid) \) therefore \( \val \in \respo_{2}(\aexec, \aexec\prjV(\txid), \addr) \).
Similarly by the hypothesis \( \como_{1} \ordAEX \como_{2} \), we know \( \respo_{2}(\aexec, \aexec\prjV(\txid), \addr) \subseteq \respo_{1}(\aexec, \aexec\prjV(\txid), \addr)\), thus \( \val \in \respo_{1}(\aexec, \aexec\prjV(\txid), \addr)\).
Therefore we have \( \aexec \in \evalCOM{\como_{1}} \).
\end{proof}

\begin{thm}[Soundness of the semantics]
\label{thm:soundness-semantics}
For any runtime abstract abstract execution \( \aexecrun \) that satisfies a consistency model \( \como \), if the semantics under consistency model \( \como \) take one step  to  a new abstract execution \( \aexecrun' \), the new execution should satisfy the consistency model.
This is,
 \[
 \begin{array}{@{}l@{}}
    \for{\como, \thid, \aexecrun, \aexecrun', \thstk, \thstk', \iprog, {\iprog}', \lb} \\
    \qquad \eraseAEX{\aexecrun} \in \evalCOM{\como}
    \land \como, \thid \vdash (\aexecrun, \thstk), \iprog \toT{\lb} (\aexecrun', \thstk'), {\iprog}' 
    \implies \eraseAEX{\aexecrun'} \in \evalCOM{\como}
 \end{array}
 \]
\end{thm}
\begin{proof}
We prove it by induction on the derivations.

\caseB{\rl{PCommit}}

By the \rl{PCommit} rule, it has \( \iprog = \ptrans{\trans} \), \( \iprog' = \pskip \) and \( \lb = \lbC{\txid} \), for some transaction code \( \trans \) and identifier \( \txid \).
Let variables \( \aexec = \eraseAEX{\aexecrun} \) and \( \aexec' = \eraseAEX{\aexecrun'} \) in the following discussion.
We need to prove the follows,
\begin{align}
    & \for{ \txid', \txid'', \addr, \val}  \nonumber \\
    & \quad (\etR, \addr, \val) \in \aexec'\prjT(\txid') \implies \val \in \respo(\aexec',\aexec'\prjV(\txid'), \addr) \label{equ:res_policy}\\
    & \quad (\txid', \txid'') \in \conguar(\aexec') \implies (\txid', \txid'') \in \aexec'\prjV \label{equ:con_guarantee}
\end{align}
First for the \equref{equ:res_policy}, it only needs to check the new transaction \( \txid \) as others are proved directly from the hypothesis.
Given an initial heap \( \h \), a set of events \( \evset \) associated with the new transaction \( \txid \), a set of transactions \( \txidset \) observed by the new transaction \( \txid \), assume these variables satisfy the follows,
\[
\begin{array}{@{}l@{}}
    \exsts{\thstk, \txstk, \h'} \nonumber \\
    \quad \thstk \vdash (\emptyset, \h, \emptyset), \trans \toL^{*} (\txstk, \h', \evset), \pskip 
    \land \h \in \obsstate{\aexecrun, \txidset, \respo} \nonumber \\
\end{array}
\]
therefore the following hold,
\[
    (\etR, \addr, \val) \in \evset \implies \h(\addr) = \val
\]
Because by the transaction semantics (\figref{fig:transaction_semantics}), a transaction only records the first read event for each address.
It can be proved by induction on the derivations for transaction operational semantics, where the only rule that involves read event is \rl{TRead}.
If the new read event is included in the events set after flushing read (\defref{def:transaction-event}), \ie \( (\etR, \addr, \val) \in ( \evset \flushR (\etR, \addr, \val) ) \), this means there is no other read and write to the same address before, so that the value \( \val  \) associate with the address \( \addr \) is the initial value, \ie \( \h(\addr)\), and after that no other read event can over-write.
By the \( \obsstateName \) function (\figref{fig:thread_semantics}), where \( \h(\addr) = \val \iff \respo(\aexec, \txidset, \addr) \), the following hold,
\[
    (\etR, \addr, \val) \in \evset \implies \respo(\aexec, \txidset, \addr)
\]
Since the \( \newaexecName \) function only extends abstract execution, which means \( \aexec \ordAEX \aexec' \), so that,
\[
    (\etR, \addr, \val) \in \evset \implies \respo(\aexec', \txidset, \addr)
\]
By the \( \obsstateName \) function, we have \( \evset = \aexec'\prjT(\txid)\) and \( \txidset = \aexec'\prjV(\txid)\) therefore we prove \equref{equ:res_policy}.

Second for \equref{equ:con_guarantee}, the visibility relation of the new abstract execution \( \aexec'\prjV \) contains the minimum relation required by the consistency guarantee.
Similarly, it is sufficient to consider those visibility edges related to the new transaction \( \txid \).
Note that for any transaction \( \txid'' \in \dom(\aexec'\prjT) \), it has \( (\tsid, \tsid'') \notin \conguar(\aexec') \).
Given \( \newaexecName \) function, so that \( (\tsid'', \tsid) \in \aexec'\prjA \).
Then, given the consistency guarantee (\defref{def:consistency-models}), it cannot violate arbitration order, this is, \( \conguar(\aexec') \subseteq \aexec'\prjA\).
Thus, it is safe to assume a transaction \( \tsid' \in \dom(\aexec'\prjT) \) such that \( (\tsid', \tsid) \in \conguar(\aexec') \).
By the \( \predn{consis}\) predicate, we have \( (\tsid', \tsid) \in \conguar(\aexec') \implies (\tsid', \tsid) \in \aexec'\prjV\), so that we prove \equref{equ:con_guarantee}.

\caseB{\rl{PAssign}, \rl{PAssume}, \rl{PChoice}, \rl{PLoop}, \rl{PSeqSkipS}}

For these base cases, the runtime abstract execution remains the same, \ie \( \aexecrun = \aexecrun' \), so they trivially hold because of the hypothesis.

\caseB{\rl{PPar}, \rl{PWait}}

For these two base cases, since the \( \funcn{extend\_thread} \)  and \( \funcn{erase\_thread} \) functions only change relations related to the corresponding threads, therefore \ie \( \eraseAEX{\aexecrun} = \eraseAEX{\aexecrun'} \), so they hold because of the hypothesis.

\caseI{\rl{PSeq}}

It is proved directly by applying the \ih
\end{proof}

For sanity check and also proving the completeness of this semantics, we first prove the semantics is monotonic (\lemref{lem:semantics-monotonicity}), which means that for any reduction that can happen in the stronger consistency model, it can also happen in the weaker one.

\begin{lem}[Semantics monotonicity]
\label{lem:semantics-monotonicity}
Given an initial runtime abstract execution \( \aexecrun \), if it can transfer to an abstract execution \( \aexecrun' \) by reducing one step of the semantics under stronger consistency model \( \como_{2}\), it is also possible by reducing one step of the semantics under a weaker consistency model \( \como_{1} \).
\[
\begin{array}{@{}l@{}}
    \for{\como_{1}, \como_{2}, \thid, \aexecrun, \aexecrun', \iprog, \iprog', \thstk, \thstk', \lb}  \\
    \quad \como_{2}, \thid \vdash ( \aexecrun, \thstk ), \iprog \toT{\lb} ( \aexecrun', \thstk' ), \iprog'
    \land \como_{1} \ordCOM \como_{2} \\
    \quad \implies \como_{1}, \thid \vdash ( \aexecrun, \thstk ), \iprog \toT{\lb} ( \aexecrun', \thstk' ), \iprog'
\end{array}
\]
\end{lem}
\begin{proof}
We prove it by induction on the derivations.
The only interesting case is the \rl{PCommit} rule.

\caseB{\rl{PCommit}}

Let variables \( (\respo_{1}, \conguar_{1}) = \como_{1} \) and  \( (\respo_{2}, \conguar_{2}) = \como_{2} \) respectively.
Given an initial runtime abstract execution \( \aexecrun \), a set of observable transactions \( \txidset \), a new transaction identifier \( \txid \) and a thread identifier \( \thid \), by the \rl{PCommit} rule (\figref{fig:thread_pool_semantics}), it is sufficient to prove, first, all the observable states under the stronger consistency model can also be observed under the weaker one,
\begin{equation}
    \label{equ:obs-state-included}
    \obsstate{\aexecrun, \txidset, \respo_{2}} \subseteq  \obsstate{\aexecrun, \txidset, \respo_{1}} 
\end{equation}
and second if the new runtime abstract execution \( \aexecrun' \) exists under the stronger concurrency model, it should also exist under weaker one,
\begin{equation}
    \label{equ:consis-both-exist}
    \consis{\aexecrun', \txid, \thid, \conguar_{2}} \implies \consis{\aexecrun', \txid, \thid, \conguar_{1}}
\end{equation}
To prove \equref{equ:obs-state-included}, assume an observable heap \( \h \) under stronger consistency model \( \como_{2}\), which means \( \h \in \obsstate{\aexecrun, \txidset, \respo_{2}} \).
Then, assume an address \( \addr\) and the corresponding value \( \val \) such that \(  \h(\addr) = \val \).
By the \( \obsstateName \) function  (\figref{fig:thread_semantics}) and resolution policy (\defref{def:consistency-models}), it is known that \( \val \in \respo_{2}(\eraseAEX{\aexecrun}, \txidset, \addr)  \).
Because of \( \respo_{2}(\eraseAEX{\aexecrun}, \txidset, \addr) \subseteq \respo_{1}(\eraseAEX{\aexecrun}, \txidset, \addr) \) (\defref{def:consistency-models}), we have \(  \val \in \respo_{1}(\eraseAEX{\aexecrun}, \txidset, \addr) \), so \( \h \in \obsstate{\aexecrun, \txidset, \respo_{1}} \).
For \equref{equ:consis-both-exist}, assume the \( \consis{\aexecrun, \txid, \thid, \conguar_{2}} \) predicate holds and assume an edge \( (\txid', \txid) \in \conguar_{2}(\eraseAEX{\aexecrun'}) \) for some \( \txid' \).
Thus, the edge \( (\txid', \txid) \) will be included in the visibility relation of the new runtime abstract execution, \ie \( (\txid', \txid) \in \aexecrun'\prjV \).
Since \( \conguar_{1}(\eraseAEX{\aexecrun}) \subseteq \conguar_{2}(\eraseAEX{\aexecrun})\), so \( (\txid', \txid) \in \conguar_{1}(\eraseAEX{\aexecrun'}) \implies (\txid', \txid) \in \aexecrun'\prjV \) holds.
This means for any edges that satisfy the consistency guarantee for stronger model, they also satisfy the weaker consistency guarantee, thus the \equref{equ:consis-both-exist} holds.
Combining \equref{equ:obs-state-included} and \equref{equ:consis-both-exist}, we prove \rl{PCommit}.

\caseB{\rl{PAssign}, \rl{PAssume}, \rl{PChoice}, \rl{PLoop}, \rl{PSeqSkipS}, \rl{PPar}, \rl{PWait}}

These base cases do not depend on the consistency model, so they trivial hold because of the hypothesis.

\caseI{\rl{PSeq}}

It is proved directly by applying the \ih
\end{proof}

\begin{lem}[Preservation of the consistency model]
\label{lem:preserve-of-consistency}
\[
 \begin{array}{@{}l@{}}
    \for{\como_{1}, \como_{2}, \thid, \aexecrun, \aexecrun', \thstk, \thstk', \iprog, {\iprog}', \lb} \\
    \qquad \como_{1}, \thid \vdash (\aexecrun, \thstk), \iprog \toT{\lb} (\aexecrun', \thstk'), {\iprog}'
    \land \eraseAEX{\aexecrun'} \in \evalCOM{\como_{2}}
    \implies \eraseAEX{\aexecrun} \in \evalCOM{\como_{2}}
 \end{array}
\]
\end{lem}
\begin{proof}
We prove it by induction on the derivations.
The only interesting case is the \rl{Commit}.

\caseB{\rl{Commit}}

By the rule it has \( \iprog = \ptrans{\trans} \), \( \iprog' = \pskip \) and \( \lb = \lbC{\txid} \).
We prove this case by deriving contradiction.
Assume \( \eraseAEX{\aexecrun} \notin \evalCOM{\como_{2}} \), which means that there exists an edge \( (\txid', \txid'') \) such that it is in the consistency guarantee, \( (\txid', \txid'') \in \conguar_{2}(\eraseAEX{\aexecrun}) \) but not is not included in the visibility relation.
\begin{equation}
    \label{equ:not-in-vis}
    (\txid', \txid'') \notin \aexecrun\prjV 
\end{equation}
Another possibility is that there is a read event from a transaction, \( (\etW, \addr, \val) \in \aexecrun\prjT(\txid') \) where the value is not observable under the stronger consistency model, \ie
\begin{equation}
    \label{equ:not-observable}
    \val \notin \respo_{2}(\eraseAEX{\aexecrun}, \aexecrun\prjV(\txid'), \addr) 
\end{equation}
Because the rule only extend the runtime abstract execution \( \aexecrun \ordAEXRUN \aexecrun'\), this means the edge \( (\txid', \txid'') \) is not in the runtime abstract execution after reduction \( \aexecrun'\), or the transaction \( \txid' \) reads a unobservable value, 
\[
    (\txid', \txid'') \notin \aexecrun'\prjV \lor \val \notin \respo_{2}(\eraseAEX{\aexecrun'}, \aexecrun'\prjV(\txid'), \addr) 
\]
Both cases lead to \( \aexecrun' \notin \evalCOM{\como_{2}} \) so there is contradiction to the hypothesis.
Therefore we have the proof for this base case.

\caseB{\rl{PAssign}, \rl{PAssume}, \rl{PChoice}, \rl{PLoop}, \rl{PSeqSkipS}}

For these base cases, the runtime abstract execution remains the same, \ie \( \aexecrun = \aexecrun' \), so they trivially hold because of the hypothesis.

\caseB{\rl{PPar}, \rl{PWait}}

For these two base cases, since the \( \funcn{extend\_thread} \)  and \( \funcn{erase\_thread} \) functions change edges only related to the corresponding threads, therefore \ie \( \eraseAEX{\aexecrun} = \eraseAEX{\aexecrun'} \), so it holds because of the hypothesis.

\caseI{\rl{PSeq}}

It is proved directly by applying the \ih
\end{proof}

The completeness means that if an abstract execution \( \aexecrun \) satisfies a consistency model, it always is possible to produce such execution through the semantics under the corresponding consistency model for some initial configurations.
To define the completeness, we introduce \emph{anarchic semantics}, which is the semantics under the bottom element for consistency model \( \btmCOM \) as for the fact that there is no constraint for the visibility relations and for the observable value for each address.
\sx{citation for anarchic semantics?}
The \thmref{thm:semantics-completeness} says, given some initial configurations, after one step under the anarchic semantics, if one is ``lucky'' that it ends up with a runtime abstract execution \( \aexecrun'\) that satisfies a consistency model \( \como \), it is possible to get the same result using the semantics specifically for the consistency model \( \como \).

\begin{thm}[Completeness of the semantics]
\label{thm:semantics-completeness}
For any initial configuration \( ( (\aexecrun, \thstk), \iprog ) \), after one step under the \emph{anarchic semantics}, \ie the semantics under the bottom element \( \btmCOM \), it ends up with \( \aexecrun' \), and if the new runtime abstract execution \( \aexecrun' \) satisfies a consistency model \( \como \), then there is a corresponding step using semantics under consistency model \( \como \).
 \[
 \begin{array}{@{}l@{}}
    \for{\como, \thid, \aexecrun, \aexecrun', \thstk, \thstk', \iprog, {\iprog}', \lb} \\
    \qquad \btmCOM, \thid \vdash (\aexecrun, \thstk), \iprog \toT{\lb} (\aexecrun', \thstk'), {\iprog}' \land \eraseAEX{\aexecrun'} \in \evalCOM{\como} \\
    \qqquad \implies \como, \thid \vdash (\aexecrun, \thstk), \iprog \toT{\lb} (\aexecrun', \thstk'), {\iprog}' 
 \end{array}
 \]
\end{thm}
\begin{proof}
For any runtime abstract execution after one step such that \( \eraseAEX{\aexecrun'} \in \evalCOM{\como} \), by the \lemref{lem:preserve-of-consistency}, it is known that the initial configuration also satisfies the consistency model, this is, \( \eraseAEX{\aexecrun} \in \evalCOM{\como} \).
Since \( \btmCOM \) is the bottom element such that \( \btmCOM \ordCOM \como \), because the semantics are monotonic with respect to the order of consistency model (\lemref{lem:semantics-monotonicity}), we have the proof.
\end{proof}

\subsection{Program Semantics}

We model the state of database as a history heap \( \hh \), where each address is associated with a list \emph{versions} from the initial one to the latest one.
Each version \( ( \val, \txid, \txidset ) \) contains a value \( \val \) and a transaction identifier \( \txid \) who writes it and a set of transactions \( \txidset \) who read it.
Let \( \hh(\addr)(i)\) denotes the \emph{i-th} version associated with address \( \addr \), and \( \hh(\addr)(i).\texttt{val} \), \( \hh(\addr)(i).\texttt{write} \) and \( \hh(\addr)(i).\texttt{read} \) denote the first (value), second (write) and third element (read) of the version.
For better presentation, we assume the index starts from 1, so \( \hh(\addr)(1)\) and \( \hh(\addr)(\left|\hh(\addr)\right|)\) denote the first and the latest versions of the address \( \addr \).

\ac{
    Notation for version, \( \ver\) , history heap \( \hh\) vs \( h\!h \) (shrike the space between two h).
}
\sx{Need a well-formed condition here but not sure how strong it will be.}
\begin{defn}[History Heaps]
\label{def:his_heap}
Assuming a set of \emph{transactions identifiers} \( \TxID \defeq \Set{\txid, \dots}\), a \emph{history heap}, \( \hh \in \HisHeaps \), is a partial finite function from addresses to lists of \emph{versions}.
Each version is a tuple containing a value, a transaction (identifier) and a set of transactions (identifiers).
\[
\begin{rclarray}
    \ver \in \Versions & \defeq &  \Val \times \TxID \times \powerset{\TxID} \\
    \pred{wfV}{(\val, \txid, \txidset)} & \defeq & \txid \notin \txidset \\
    \hh \in \HisHeaps & \defeq & \Addr \parfinfun \Versions^{*}
\end{rclarray}
\]
\sx{Need to think here}
The well-formed condition for a history heap asserts a transaction identifier appears in all the versions from an address at most twice, one as a write and one as a read,
\[
\begin{rclarray}
    \pred{wfHH}{\hh} & \defeq &
    \begin{array}[t]{@{}l}
        \fora{\addr, \txid, i, j}  \\
        \quad \hhW(\addr)(i)  = \hhW(\addr)(j) \lor (\txid \in  \hhR(\addr)(i) \land \txid \in \hhR(\addr)(j)) \implies i = j \\
    \end{array}
\end{rclarray}
\]
The composition of two history heaps is disjointed union, \ie when the domains are disjointed \( \hh \composeHH \hh' \defeq \hh \uplus \hh' \) and the unit element is \( \unitHH \defeq \emptyset \), which together form \emph{the partial commutative monoid of history heaps}.
\end{defn}
 

\pg{Settle down the term, and what we are aiming for? Explain some in related works}
For weak consistency models, or weak isolation levels as a common term used in database community, a thread is not necessary to work on the up-to-date version of a database in exchange for better performance. 
Even in a single machine database, a thread running under weak consistency model can make less synchronisation with the hard drivers and other running threads, which means the thread could observe out-of-date state.
Therefore, we introduce \emph{views} to model threads of a database.
A \emph{view} is a cut in a history heap that corresponds the indexes of versions that a thread work with.
We also define a order between two views, if they contain the same addresses and the indexes are ordered point-wise.
This is to model the synchronisation between threads.
For example \( \vi \orderVI \vi' \) could mean that a thread updates it view from \( \vi \) to \( \vi' \) by synchronisation with others.

\begin{defn}[Views]
\label{def:cuts}
\label{def:views}
\emph{A view of a history heap}, or \emph{a view}, is a partial finite function from addresses to indexes,
%\sx{Need standard notation for multi set}
\[
\begin{rclarray}
    \vi \in \Views & \defeq & \Addr \parfinfun \Nat \\
    %\viset & \in & \func{multiSet}{\Views} \\
\end{rclarray}
\]                                                                     
The composition is \( \vi \composeVI \vi' \defeq \vi \uplus \vi'\) and the unit is \( \unitVI \defeq \emptyset\).
The order between two views with the same domain is defined by the order of the indexes, 
\[
\begin{rclarray}
    \cu \orderCU \cu' & \defiff & \dom(\cu) = \dom(\cu') \land \fora{\addr} \cu(\addr) \leq \cu'(\addr) \\
\end{rclarray}
\]
%Assuming a set of thread identifiers \( \setthid \subseteq \ThreadID \), \emph{a view environment} is a function from thread identifiers to views, this is, \( \vienv \in \ViEnv \defeq \ThreadID \parfun \Views \).
\end{defn}

A consistency model \( \como \) is specified as a set of quadruples in the form of \( ( \hh, \vi, \opset, \vi' ) \) that describes a thread who has view \( \vi \) is allowed to commit a single transaction that can be abstracted to the operation set \( \opset \) when the state of the database is \( \hh \), and then after the commit the thread view must be updated to at least \( \vi' \).
We often write \( (\hh, \vi) \csat \opset : \vi' \) in lieu of \( (\hh, \vi, \opset, \vi') \in \como \).

%The external environment \( \viset \) is a multi-set includes all the views of threads who may be interacting with the database.

\ac{Terminology: commit test instead of consistency model}

\begin{defn}[Consistency Models]
\label{def:consistency-models}
Given the set of history heaps \( \hh \in \HisHeaps \) (\defref{def:his_heap}), operation sets \( \opset \in \Opsets \) (\defref{def:ops}) and views \( \vi, \vi' \in \Views \) (\defref{def:views}), \emph{a consistency model} \( \como \in \Como \) is a set of quadruples in the form of \( ( \hh, \vi, \opset, \vi' ) \),
\[
    \begin{rclarray}
        \como \in \Como & \defeq & \powerset{\HisHeaps \times \Views \times \Opsets \times \Views}
    \end{rclarray}
\]
A well-formed consistency model, written \( \pred{wfC}{\como}\), requires the domain of pre- and post views are are the same as the domain of the history heap, and the operation set has no garbage record.
\[
    \begin{rclarray}
        \pred{wfC}{\como} & \defeq & \fora{\hh, \vi, \vi', \opset } (\hh, \vi, \opset, \vi') \in \como \implies \opset\projection{2} \subseteq \dom(\vi) = \dom(\vi') = \dom(\hh)
    \end{rclarray}
\]
\sx{
    Maybe also some version of composition requirement.
    For example, the composition of two should be also included in the consistency model.
    \[
        \fora{m,m'} m \in \como \land m' \in \como \implies m \compose{} m' \in \como
    \]
    where \( \compose{} \defeq (\composeHH, \composeVI,\composeO, \composeVI)\).
}
\end{defn}


%\begin{defn}[Consistency Models]
%\label{def:consistency-models}
%A \emph{Consistency model} \( \como \) is a \emph{reflexive and transitive relation} over pairs of history heaps and cut environments, written \( ((\hh,\viset),(\hh',\viset')) \in \como \) or \( (\hh, \viset) \toCO{\como} (\hh', \viset' )\), that the transition from the state \( (\hh,\viset)\) to the state \( (\hh',\viset') \) satisfies certain constraints.
%Each tuple in the set should satisfy a well-formed condition that the cut environment should contain no more address than the history heap,
%\[
    %\bigwedge\limits_{\vi \in \viset} \dom(\vi) \subseteq \dom(\hh)
%\]
%\sx{Might be useful constraint for soundness of logic, if never used, just delete this.}
%The specification of a certain consistency model should be local to the owned history heaps, so that the combination of two disjointed parts also satisfies the specification.
%\[
    %\begin{array}{@{}l}
    %\fora{\hh,\hh',\hh'',\hh''', \viset, \viset', \viset'', \viset'''} \\
        %\quad ((\hh, \viset),(\hh',\viset')) \in \como 
        %\land ((\hh'', \viset''),(\hh''',\viset''')) \in\como 
        %\land ( \hh \composeHH \hh'' )\isdef
        %\land ( \hh' \composeHH \hh''' )\isdef \\
        %\qquad \implies  ((\hh \composeHH \hh'', \viset \uplus \viset''),(\hh' \composeHH \hh''',\viset' \uplus \viset''')) \in \como
    %\end{array}
%\]
%\sx{The following says the consistency model transfer from a valid state to another valid state. }
%A consistency model also requires to transfer from a \emph{valid} state to another \emph{valid} state where the validity is independent from the transition.
%\[
    %\fora{\como} \exsts{\funcn{valid}} (\hh, \viset) \in \como \implies \func{valid}{\hh, \viset}
%\]
%where \( (\hh , \viset) \in \como\) means the state \((\hh, \viset)\) appears in the relation.
%\end{defn}

%\sx{Need Andrea for more clarify about defining consistency model}

%There are two ways to specify consistency models, one by converting the history heaps to dependant graphs and checking if the cycles satisfy certain property, while another by checking if the view environment satisfy certain property.
%We introduce \( \funcn{graph} : \HisHeaps \to \sort{DGraph} \) function for converting history heaps to dependant graphs.
%\[
%\begin{rclarray}
    %\DGraph & \defeq & \Setcon{(\txidset, \ww, \wrr, \rw)}{ \txidset \subseteq \TxID \land \ww, \wrr, \rw \subseteq (\txidset \times \txidset)} \\
    %\func{graph}{\hh} & \defeq & \left(
    %\begin{array}{@{}l}
        %\Setcon{\txid}{\exsts{\addr, i} \txid = \hhW(\addr)(i) \lor \txid \in \hhR(\addr)(i) }, \\
        %\Setcon{(\txid, \txid')}{\exsts{\addr, i, i'} \txid = \hhW(\addr)(i) \lor \txid' = \hhW(\addr)(i') \land i < i' }, \\
        %\Setcon{(\txid, \txid')}{\exsts{\addr, i} \txid = \hhW(\addr)(i) \lor \txid' \in \hhR(\addr)(i) }, \\
        %\Setcon{(\txid, \txid')}{\exsts{\addr, i, i'} \txid \in \hhR(\addr)(i) \lor \txid' = \hhW(\addr)(i') \land i < i' }, \\
    %\end{array}
    %\right)
%\end{rclarray}
%\]

%\begin{example}[Serialisibility]
%\end{example}

%\begin{example}[Snapshot]
%Any cycle in the corresponding dependant graph of the history heap after the transition must have adjacent anti-dependent edges (\( \rw \) edges).
%\[
    %\begin{rclarray}
    %\SI & \defeq & \Setcon{((\hh, \thcu),(\hh', \thcu'))}{\exsts{\ww, \rw, \wrr} (\stub, \ww, \wrr, \rw)  = \func{graph}{\hh'} \land (\wrr \cup \ww) ; \rw? \text{ is acyclic}}
    %\end{rclarray}
%\]
%\end{example}

%{ \color{gray}
%\begin{defn}[Thread transition labels]
%\label{def:label}
%Given the set of thread identifiers \(\thid \in \ThreadID\), the set of \emph{thread transition labels}, $\lb \in \Translabel$, is defined by the following grammar, where $\prog$ denotes a program (\defref{def:language}), the $\txid$ demotes a transaction identifier and $\thstk$ denotes a thread stack (\defref{def:stacks}),
%\[
    %\begin{rclarray}
	%\iota \in \Translabel & ::= & \lbID \mid \lbC{\txid} \mid \lbF{\thid,\prog} \mid \lbJ{\thid,\thstk}
    %\end{rclarray}
%\]
%\end{defn}
%}

%{ \color{gray}
%A spacial version of this function only takes a history heap, and it returns a normal heap that is the latest state of the database.
%}

%{ \color{gray}
%The function is overloaded with one parameter version that collapses history heaps by picking the last values,
%\[
%\begin{rclarray}
    %\clpsHH{\hh} & \defeq & \lambda \addr \ldotp \hhV(\addr)(\left|\hh(\addr)\right|) 
%\end{rclarray}
%\]
%}
%\end{defn}

\sx{Re-assemble some words from Andrea. Need more work on the words }

A new transaction is committed through the \rl{PCommit} rule.
First, to model synchronisation between threads, the \rl{PCommit} rule has a view shift before executing the transaction.
This new local view should also be consistent with the history heap, \ie it leads to a situation where the current thread is allowed to commit the effect of the immediate next transaction.
The transaction code \( \trans \) is executed locally given the local state that is decided by the current state of database and the local view, \ie the history heap \( \hh \) and view \( \vi \) respectively.
The \( \funcn{localHeap} \) function uniquely determined a (local) heap from a history heap and a view by picking the versions of addresses indexed by the view.
After local execution, we obtain a operation set \( \opset \).
Then the transaction picks a fresh identifier \( \txid \) and commits the operation set \( \opset \).
The operations are the first read and last write of each address, which are the operations might affect the database, because of the atomicity of transactions.
The \funcn{updHisHp} function updates the history heap.
If there is a read operation, it includes the new identifier to the version relates to the address and the local view \( \vi \).
If there is a write operation, it extends a new version to the end and puts the new identifier as the writer.
Since we assume strong program order, we set the lower bound for the new local view by \funcn{updView} function.
This function shifts the view to the latest version if the version is written by the current transaction.
This guarantees strong program order, meaning the following transaction will at least read its own write.
Since it is a lower bound, the new local view \( \vi' \) can advance the view as long as it is not out-of-bound.
Finally, the transaction is allowed by the consistency model.

\begin{defn}[Thread semantics]
\label{def:thread_semantics}
%{ \color{gray}
%Given the thread identifiers \(\thid \in \ThreadID\), the set of \emph{intermediate programs}, \(\iprog \in \IntermediatePrograms\), is defined by the following grammar,
%\[
    %\iprog ::= \prog \mid \iprog \pseq \pwait{\thid}
%\]
%}
Given the set of consistency model \( \ConsisModels \) (\defref{def:consistency-models}), stacks \( \Stacks \) (\defref{def:stacks}), history heaps \( \HisHeaps \) (\defref{def:his_heap}) and views \( \Views \) (\defref{def:cuts}), the \emph{per-thread operational semantics} of programs,
\[
\begin{rclarray}
	(.)\toT{(.)} (.) & : &
    %\begin{array}[t]{@{}l@{}}
	\left( ( \HisHeaps \times \Stacks \times \Views ) \times \Commands \right) 
	\times \Como \times 
	\left( ( \HisHeaps \times \Stacks \times \Views ) \times \Commands \right) 
    %\end{array}
\end{rclarray}
\]
is defined in \figref{fig:thread_semantics}.
The well-formed condition for the relation asserts the views and history heaps contains the same addresses.
\[
\fora{\hh, \vi, \hh', \vi'} (\stub, \hh, \vi, \stub) \toT{\stub} (\stub, \hh', \vi', \stub)  \implies \dom(\hh) = \dom(\vi) \land \dom(\hh') = \dom(\vi')
\]
%\sx{
%The labels are for technical reason because some steps need to satisfy certain constraints related other threads. 
%The labels includes commit \( \lbC{\txid} \), view shift \( \lbV \) and identify \( \lbID \).
%}
\end{defn}
For any lists or tuples \( l \), the notation \( l\rmto{i}{k} \) means the result by replacing the \emph{i-th} element to \( k \), where the index starts from 1.
The \( \lcat \) denotes list concatenation.

\begin{figure}[!t]
%
\hrule\vspace{5pt}
%
\[
    \infer[\rl{PCommit}]{%
        ( \hh, \stk, \vi ), \ptrans{\trans} \ \toT{\como} \ ( \hh', \stk', \vi' ) , \pskip
    }{%
        \begin{array}{c}
            \vi'' \geq \vi
            \quad \txid \in \func{fresh}{\hh}  
            \quad \h = \clpsHH{\hh,\vi''}
            \quad (\stk, \h, \unitO), \trans \ \toL^{*} \  (\stk', \h', \opset) , \pskip \\
            \hh' = \func{updHisHp}{\hh, \vi'', \txid, \opset}  
            \quad \vi' \geq \func{updView}{\hh', \vi'', \opset}
            \quad (\hh, \vi) \csat \opset : \vi'
        \end{array}
    }
\]

\[
    \infer[\rl{PAssign}]{%
        ( \hh, \stk, \vi ) , \passign{\var}{\expr} \ \toT{\como} \  ( \hh, \stk\rmto{\var}{\val}, \vi ) , \pskip
    }{
        \val = \evalE{\expr}
    }
\]

\[
    \infer[\rl{PAssume}]{%
        ( \hh, \stk, \vi ) , \passume{\expr} \ \toT{\como} \  ( \hh, \stk, \vi ) , \pskip
    }{%
        \evalE[\thstk]{\expr} = 0
    }
\]

%\[
    %\infer[\rl{PViewLShift}]{
        %\como \vdash ( \hh, \stk, \vi ) , \cmd \ \toT{\como} \  ( \hh, \stk, \cu' ) , \cmd
    %}{
        %\vi \orderVI \vi'
    %}
%\]

\[
    \infer[\rl{PChoice}]{%
        ( \hh, \stk, \vi ) , \cmd_{1} \pchoice \cmd_{2} \ \toT{\como} \  ( \hh, \stk, \vi ) , \cmd_{i}
    }{
        i \in \Set{1,2}
    }
\]

\[
    \infer[\rl{PIter}]{%
        ( \hh, \stk, \vi ) , \cmd\prepeat \ \toT{\como} \  ( \hh, \stk, \vi ) , \pskip \pchoice (\cmd \pseq \cmd\prepeat)
    }{}
\]

\[
    \infer[\rl{PSeqSkip}]{%
        ( \hh, \stk, \vi ) , \pskip \pseq \cmd \ \toT{\como} \  ( \hh, \stk, \vi ) , \cmd
    }{}
\]

\[
    \infer[\rl{PSeq}]{%
        ( \hh, \stk, \vi ) , \cmd_{1} \pseq \cmd_{2} \ \toT{\como} \ ( \hh, \stk', \vi' ) , {\cmd_{1}}' \pseq \cmd_{2}
    }{% 
        ( \hh, \stk, \vi ) , \cmd_{1} \ \toT{\como} \  ( \hh, \stk', \vi' ) , {\cmd_{1}}' 
    }
\]

%{ \color{gray}
%\[
    %\infer[\rl{PPar}]{%
        %\como, \thid \vdash ( \thstk, \hh, \thcu ) , \prog_{1} \ppar \prog_{2} \ \toT{\lbF{\thid', \prog_{2}}} \  \left( \thstk, \hh, \thcu \uplus \Set{\thid' \mapsto \thcu(\thid)} \right) , \prog_{1} \pseq \pwait{\thid'}
    %}{}
%\]

%\[
    %\infer[\rl{PWait}]{%
        %\como, \thid \vdash \left( \thstk_{1} \uplus \thstk_{f}, \hh, \thcu \uplus \Set{\thid' \mapsto \thcu(\thid)} \right) , \pwait{\thid'} \ \toT{\lbJ{\thid', \thstk_{2} \uplus \thstk_{f}}} \  ( \thstk_{1} \uplus \thstk_{2} \uplus \thstk_{f}, \hh, \thcu ) , \pskip 
    %}{}
%\]
%}

\hrule
 
\[
\begin{rclarray}                                 
    \clpsHH{\hh, \cu} & \defeq & 
    \begin{cases}
        \lambda \addr \ldotp \hh(\addr)(\cu(\addr)).\texttt{val} & \text{if } \dom(\hh) = \dom(\cu) \land \fora{\addr' \in \dom(\vi)} 1 \leq \cu(\addr') \leq \left|\hh(\addr')\right| \\
        \text{undefined} & \text{otherwise}
    \end{cases} \\
    \func{updHisHp}{\hh, \vi, \txid, \opset} & \defeq & 
    \left\{ \begin{array}{@{} l l}
        \hh & \text{if } \opset = \unitO \\
        \func{updHisHp}{\hh', \vi, \txid, \opset'} & \text{if } \opset = \opset' \uplus (\otR, \addr, \val) \\
        \multicolumn{2}{@{}l}{\quad \text{and } \hh' \equiv \hh\rmto{\addr}{\hh(\addr)\rmto{\vi(\addr)}{ \hh(\addr)(\vi(\addr))\rmto{3}{\hh(\addr)(\vi(\addr)).\texttt{read} \uplus \Set{\txid}} }}}\\
        \func{updHisHp}{\hh'', \vi, \txid, \opset'} & \text{if } \opset = \opset' \uplus (\otW, \addr, \val) \\
        \multicolumn{2}{@{}l}{\quad \text{and } \hh'' \equiv \hh\rmto{\addr}{\hh(\addr) \lcat \List{(\val, \txid, \emptyset)} } }\\
    \end{array} 
    \right. \\
    %\hh' & \equiv & \hh\rmto{\addr}{\hh(\addr)\rmto{\vi(\addr)}{ \hh(\addr)(\vi(\addr))\rmto{3}{\hhR(\addr)(\vi(\addr)) \uplus \Set{\txid}} }} \\
    %\hh'' & \equiv & \hh\rmto{\addr}{\hh(\addr) \lcat \List{(\val, \txid, \emptyset)} } \\
%
%
    \func{updView}{\hh, \vi, \opset} & \defeq &
    \begin{cases}
        \vi & \text{if } \opset = \unitO \\
        \func{updView}{\hh, \vi, \opset'} & \text{if } \opset = \opset' \uplus (\otR, \addr, \val) \\
        \func{updView}{\hh, \vi\rmto{\addr}{\left| \hh(\addr) \right|}, \opset'} & \text{if } \opset = \opset' \uplus (\otW, \addr, \val) \\
    \end{cases} \\
%
%              
	\func{fresh}{\hh}  & \defeq & \Setcon{ \txid }{ \txid \in \TxID \land \fora{\addr, i} \txid \neq \hh(\addr)(i).\texttt{write} \land \txid \notin \hhR(\addr)(i).\texttt{read}} \\
\end{rclarray}
\]
\hrule\vspace{5pt}
\[
    \infer[\rl{PSingleThread}]{%
        ( \hh, \thdenv, \prog ) \ \toG{\como} \  ( \hh', \thdenv\rmto{\thid}{(\stk',\vi')} , \prog\rmto{\thid}{\cmd'} ) 
    }{%
        (\stk,\vi) = \thdenv(\thid)
        && \cmd = \prog(\thid) 
        && ( \hh, \stk, \vi ), \cmd, \ \toT{\como} \  ( \stk', \hh', \vi' ) , \cmd'  
    }
\]
\hrule\vspace{5pt}
\caption{operational semantics for threads and programs}
\label{fig:thread_semantics}
\label{fig:thread_pool_semantics}
\end{figure}


\begin{lem}[Confluence of \funcn{updHisHp} and \funcn{updView}]
Given a valid operation set \( \opset \), all the computation paths for \( \funcn{updHisHp} \) function (\( \funcn{updView} \) function) yield the same result.
\end{lem}
\begin{proof}
\end{proof}

%\sx{The full version should support dynamic thread, but we might only need fixed thread so that the thread pool semantics will be pick one thread and run one step}

%{ \color{gray}
%\begin{defn}[Thread pools]
%\label{def:thread_pools}
%Given the set of thread stacks $\ThdStacks$ (\defref{def:stacks}) and intermediate programs $\IntermediatePrograms$ (\defref{def:thread_semantics}), a \emph{thread pool} is a a finite partial function from thread identifiers to triples of thread stacks and intermediate programs, \(\thpl \in \TPool \eqdef \ThreadID \parfinfun \ThdStacks \times \IntermediatePrograms\).
%\end{defn}
%}
 
\sx{Later, a better way to say the initial config. to cover history heap stacks.}

The program has standard interleaving semantics by picking a thread and then progressing one step.
To achieve that a thread environment holds the stacks and views associated with all running threads \( Env \in \sort{ThdEnv} \).
We assume the thread identifiers in the thread environment match with those in the program \( \prog \).
\sx{For the logic, we need at beginning all the stacks are the same and at the end stacks are compatible with each others. Need to find a way to say that here.}
We also assume all the stacks initially are the same and when all threads finish, their stacks are compatible which means that the same variables have the same values.


%If the step is commit \( \lbC{\txid} \), the overall transition of history heap and view environment should satisfy consistency model.
%In anther words, the view that has been changed should be compatible with those has not been changed.
%It is necessary to check the compatibility because some consistency models are specified as constraints among all views.
%We write \( \hh, \vienv \toCO{\como} \hh', \vienv' \) as short-hand for \( \hh, \func{range}{\vienv} \toCO{\como} \hh', \func{range}{\vienv'} \).
%If the step is view shift \( \lbV \), it should shift to a valid state\footnote{It is not necessary to always shift to a valid state, yet it is a design choice for abstract operational semantics.}.

\begin{defn}[Programs semantics] 
\label{def:thread_pool_semantics}
\label{def:program_semantics}
Given the set of consistent models \( \ConsisModels \) (\defref{def:consistency-models}) and history heaps \(\HisHeaps\) (\defref{def:his_heap}) , the \emph{semantics for programs}, 
\[
	(.) \toG{(.)} (.): 
    ( \HisHeaps \times \ThdEnv \times \Programs) 
    \times \Como 
    \times ( \HisHeaps \times \ThdEnv \times \Programs) 
\]
is defined in \figref{fig:thread_pool_semantics}.
The thread environment is a partial function from thread identifiers to pairs of stacks and views \( \thdenv \in \ThdEnv \defeq \ThreadID \parfinfun \Stacks \times \Views \).
\end{defn}
 
%The thread pool operational semantics is given in \figref{fig:thread_pool_semantics}, where an arbitrary thread in the pool \(\thpl\) is picked to run for one step.
%If the next execution step is a thread fork, then a new thread \(\thid'\) is allocated in the pool to be executed with its thread stack copied from its parent (forking) thread.
%Conversely, when the next execution step is the joining of thread \(\thid'\), then \(\thid'\) is removed from the thread pool and the stack from the child thread merges into the parent thread.

%\begin{figure}[!t]
%\hrule\vspace{5pt}
%
%\[
    %\infer[\rl{PSingle}]{%
        %\como \vdash ( \stkenv, \hh, \vienv, \prog ) \ \toG \  ( \stkenv\rmto{\thid}{\stk'}, \hh', \vienv\rmto{\thid}{\vi'}, \prog\rmto{\thid}{\cmd'} ) 
    %}{%
    %\begin{array}{c}
        %\stk = \stkenv(\thid)
        %\quad \vi = \vienv(\thid)
        %\quad \cmd = \prog(\thid) \\
        %\como \vdash ( \hh, \stk, \vi ), \cmd, \ \toT{\lb} \  ( \stk', \hh', \vi' ) , \cmd'  \\
        %\begin{B}
        %\lb = \lbID \lor (\lb = \lbC{\stub} \land \hh, \vienv \toCO{\como} \hh', \vienv') \lor \begin{B}\lb = \lbV \land \pred{valid}{\hh',\vienv'}\end{B}
        %\end{B}
    %\end{array}
    %}
%\]
%
%{ \color{gray}
%\[
    %\infer[\rl{PFork}]{%
        %\como \vdash ( \hh, \thcu, \thpl \uplus \Set{ \thid \mapsto (\thstk, \iprog) } ) \ \toG{\lbF{\thid', \prog}} \  ( \hh', \thcu', \thpl \uplus \Set{ \thid \mapsto (\thstk, {\iprog}'), \thid' \mapsto (\thstk, \prog) } )
    %}{%
        %\como, \thid \vdash ( \thstk, \hh, \thcu ) , \iprog \ \toT{\lbF{\thid', \prog}} \  ( \thstk, \hh', \thcu' ) , {\iprog}' 
    %}
%\]
%
%\[
    %\infer[\rl{PJoin}]{%
        %\como \vdash ( \hh, \thcu, \thpl \uplus \Set{ \thid \mapsto (\thstk, \iprog), \thid' \mapsto (\thstk', \pskip) } )  \ \toG{\lbJ{\thid',\thstk''}} \ ( \hh', \thcu', \thpl \uplus \Set{ \thid \mapsto (\thstk'', {\iprog}')} )
    %}{%
        %\como, \thid \vdash ( \thstk, \hh, \thcu ) , \iprog \ \toT{\lbJ{\thid',\thstk'}} \  ( \thstk'', \hh', \thcu' ) , {\iprog}' 
    %}
%\]
%}
%
%\hrule\vspace{5pt}
%\caption{Thread pool semantics}
%\label{fig:thread_pool_semantics}
%\end{figure}

\subsubsection{Example of Operational Semantics} 
\label{sec:semantics.example}
\label{sec:semantics-example}
To conclude our discussion on the operational semantics, we show in detail one possible computation of a program \( \prog_{1} \) consisting of two transactions executing in parallel:
%The program $\prog_1$ that we consider is illustrated below: 
\[
    \prog_{1} \equiv 
    \begin{session}
        \begin{array}{@{}c || c@{}}
            \txid_{1} : 
            \begin{transaction}
                \pmutate{\vx}{1};\\
            	\pmutate{\vy}{1};
            \end{transaction} &
            \txid_{2} : 
            \begin{transaction}
                \pderef{\pvar{a}}{\vx};\\
            	\pderef{\pvar{b}}{\vy};\\
            	\pifs{\pvar{a}=1 \wedge \pvar{b}=0}\\
            		\quad \passign{\ret}{\sadface}
            	\pife
            \end{transaction}
        \end{array}
    \end{session}
 \]
The \( \pifs{\expr} \cmd_{1} \pifm \cmd_{2} \pife \) is encoded as \( (\passume{\expr} \pseq \cmd_{1}) \pchoice (\neg\passume{\expr} \pseq \cmd_{2} )\).
To recall, we often write \( \cmd_{1} \ppar \cmd_{2} \ppar \dots \ppar \cmd_{n}\) as a syntactic sugar for a program \( \prog \) with implicit unique thread identifiers \( \prog = \Set{\thid_{1} \mapsto \cmd_{1}, \thid_{2} \mapsto \cmd_{2}, \dots, \thid_{n} \mapsto \cmd_{n}  }\).
For better presentation, we annotated transactions with unique identifiers, yet they are allocated dynamically in the semantics.
We also treat the value assigned to the \( \ret \) variable as \emph{returned value}.
Assume the variables \( \vx \) and \( \vy \) refer to two key, and \( \va \) and \( \vb \) are local variables to threads.

The special symbol \(\sadface\), for example the returned value by the transaction $\txid_2$, is to emphasise some undesirable behaviour of a transaction.
In this case, the undesirable behaviour corresponds to the transaction to the right \( \txid_{2} \) observing only one of the updates from \( \txid_{1} \). 
Intuitively, this behaviour violates the constraints that transactions should be executed atomically (further discussed in \cref{......}), we want to show that if no restrictions are placed on the consistency model specification, it is possible for $\prog_1$ to reach a configuration where the second transaction $\txid_2$ returns $\sadface$. 
To illustrate this and also explain the semantics, we instantiate the operation semantics with the most permissive execution tests \( \csatP \), \ie the view after \( \vi' \) at least observes its own writes and no other constraint:
\[
\begin{rclarray}
    (\hh, \vi) \csatP \opset : \vi' & \defeq & \func{updView}{\hh, \vi, \opset} \leq \vi'
\end{rclarray}
\]

\ac{The condition on $V'$ is not really needed.}

\begin{figure}[!t]

\hrule\vspace{5pt}
\begin{center}
\begin{tabular}{@{}c@{}@{}c@{}}
\begin{halfsubfig}
\begin{centertikz}
\begin{pgfonlayer}{foreground}

%Location x
\node(locx) {$\ke_{\vx} \mapsto$};

\matrix(initx) [version list] 
    at ([xshift=\tikzkvspace]locx.east) {
    {a} & $\txid_0$ \\ 
    {a} & $\emptyset$ \\
};  
\tikzvalue{initx-1-1}{initx-2-1}{locx-v0}{0};

%Location y
\path (locx.south) + (0,\tikzkeyspace) node (locy) {$\ke_\vy \mapsto$};
\matrix(inity) [version list] 
    at ([xshift=\tikzkvspace]locy.east) {
    {a} & $\txid_0$ \\
    {a} & $\emptyset$ \\
};
\tikzvalue{inity-1-1}{inity-2-1}{locy-v0}{0};

%blue view - I should  check whether I can use pgfkeys to just declare the list of locations, and then add the view automatically.
\draw[-, blue, very thick, rounded corners=10pt]
([xshift=-3pt, yshift=20pt]locx-v0.north east) node (tid1start) {} -- 
([xshift=-3pt, yshift=-5pt]locy-v0.south east);
 
\path (tid1start) node[anchor=south, rectangle, fill=blue!20, draw=blue, font=\small, inner sep=1pt] {$\thid_1$};

%red view
\draw[-, red, very thick, rounded corners = 10pt]
([xshift=-16pt, yshift=5pt]locx-v0.north east) node (tid2start) {}-- 
([xshift=-16pt, yshift=-5pt]locy-v0.south east) node {};
 
\path (tid2start) node[anchor=south, rectangle, fill=red!20, draw=red, font=\small, inner sep=1pt] {$\thid_2$};

%Stack for threads tid_1 and tid_2

%\draw[-, dashed] let 
   %\p1 = ([xshift=0pt]locy.west),
   %\p2 = ([yshift=-5pt]inity.south),
   %\p3 = ([xshift=10pt]inity.east) in
   %(\x1, \y2) -- (\x3, \y2);
   
%\matrix(stacks) [
   %matrix of nodes,
   %anchor=north, 
   %text=blue, 
   %font=\normalsize, 
   %row 1/.style = {text = blue}, 
   %row 2/.style = {text = red}, 
   %text width= 13mm ] 
   %at ([xshift=-10pt,yshift=-8pt]inity.south) {
   %$\thid_1:$ & $\retvar = 0$\\
   %$\thid_2:$ & $\retvar = 0$\\
   %};
\end{pgfonlayer}
\end{centertikz}
\caption{Initial state}
\label{fig:opsem-example-a}
\end{halfsubfig}
&
\begin{halfsubfig}
\begin{centertikz}

\begin{pgfonlayer}{foreground}
%Uncomment line below for help lines

%Location x
\node(locx) {$\ke_{\vx} \mapsto$};

\matrix(versionx) [version list] 
    at ([xshift=\tikzkvspace]locx.east) { 
    {a} & $\txid_{0}$ &{a} & $\txid_{1}$\\
    {a} & $\emptyset$ & {a} & $\emptyset$ \\
};
\tikzvalue{versionx-1-1}{versionx-2-1}{locx-v0}{0};
\tikzvalue{versionx-1-3}{versionx-2-3}{locx-v1}{1};

%Location y
\path (locx.south) + (0,\tikzkeyspace) node (locy) {$\ke_{\vy} \mapsto$};
\matrix(versiony) [version list]
   at ([xshift=\tikzkvspace]locy.east) {
 {a} & $\txid_0$ & {a} & $\txid_1$\\
  {a} & $\emptyset$ & {a} & $\emptyset$ \\
};
\tikzvalue{versiony-1-1}{versiony-2-1}{locy-v0}{0};
\tikzvalue{versiony-1-3}{versiony-2-3}{locy-v1}{1};

% \draw[-, red, very thick, rounded corners] ([xshift=-5pt, yshift=5pt]locx-v1.north east) |- 
%  ($([xshift=-5pt,yshift=-5pt]locx-v1.south east)!.5!([xshift=-5pt, yshift=5pt]locy-v0.north east)$) -| ([xshift=-5pt, yshift=5pt]locy-v0.south east);

%blue view - I should  check whether I can use pgfkeys to just declare the list of locations, and then add the view automatically.
\draw[-, blue, very thick, rounded corners=10pt]
([xshift=-3pt, yshift=20pt]locx-v1.north east) node (tid1start) {} -- 
([xshift=-3pt, yshift=-5pt]locy-v1.south east);
 
\path (tid1start) node[anchor=south, rectangle, fill=blue!20, draw=blue, font=\small, inner sep=1pt] {$\thid_1$};

%red view
\draw[-, red, very thick, rounded corners = 10pt]
([xshift=-16pt, yshift=5pt]locx-v0.north east) node (tid2start) {}-- 
([xshift=-16pt, yshift=-5pt]locy-v0.south east) node {};
 
\path (tid2start) node[anchor=south, rectangle, fill=red!20, draw=red, font=\small, inner sep=1pt] {$\thid_2$};

%Stack for threads tid_1 and tid_2

%\draw[-, dashed] let 
   %\p1 = ([xshift=0pt]locy.west),
   %\p2 = ([yshift=-5pt]locycells.south),
   %\p3 = ([xshift=10pt]locycells.east) in
   %(\x1, \y2) -- (\x3, \y2);
   
%\matrix(stacks) [
   %matrix of nodes,
   %anchor=north, 
   %text=blue, 
   %font=\normalsize, 
   %row 1/.style = {text = blue}, 
   %row 2/.style = {text = red}, 
   %text width= 13mm ] 
   %at ([xshift=-10pt,yshift=-8pt]locycells.south) {
   %$\thid_1:$ & $\retvar = 0$\\
   %$\thid_2:$ & $\retvar = 0$\\
   %};
\end{pgfonlayer}
\end{centertikz}
\caption{After transaction \( \txid_{1}\)} 
\label{fig:opsem-example-b}
\end{halfsubfig}
\\
\begin{halfsubfig}
\begin{centertikz}

\begin{pgfonlayer}{foreground}
%Uncomment line below for help lines
%\draw[help lines] grid(5,4);

%Location x
\node(locx) {$\ke_{\vx} \mapsto$};

\matrix(versionx) [version list] 
    at ([xshift=\tikzkvspace]locx.east) { 
    {a} & $\txid_{0}$ &{a} & $\txid_{1}$\\
    {a} & $\emptyset$ & {a} & $\emptyset$ \\
};
\tikzvalue{versionx-1-1}{versionx-2-1}{locx-v0}{0};
\tikzvalue{versionx-1-3}{versionx-2-3}{locx-v1}{1};

%Location y
\path (locx.south) + (0,\tikzkeyspace) node (locy) {$\ke_{\vy} \mapsto$};
\matrix(versiony) [version list]
   at ([xshift=\tikzkvspace]locy.east) {
 {a} & $\txid_0$ & {a} & $\txid_1$\\
  {a} & $\emptyset$ & {a} & $\emptyset$ \\
};
\tikzvalue{versiony-1-1}{versiony-2-1}{locy-v0}{0};
\tikzvalue{versiony-1-3}{versiony-2-3}{locy-v1}{1};

%blue view - I should  check whether I can use pgfkeys to just declare the list of locations, and then add the view automatically.
\draw[-, blue, very thick, rounded corners=10pt]
 ([xshift=-3pt, yshift=20pt]locx-v1.north east) node (tid1start) {} -- 
% ([xshift=-2pt, yshift=-5pt]locx-v0.south east) --
% ([xshift=-2pt, yshift=5pt]locy-v0.north east) -- 
 ([xshift=-3pt, yshift=-5pt]locy-v1.south east);
 
 \path (tid1start) node[anchor=south, rectangle, fill=blue!20, draw=blue, font=\small, inner sep=1pt] {$\thid_1$};

%red view
\draw[-, red, very thick, rounded corners = 10pt]
 ([xshift=-16pt, yshift=5pt]locx-v1.north east) node (tid2start) {}-- 
 ([xshift=-16pt, yshift=-5pt]locx-v1.south east) --
 ([xshift=-16pt, yshift=5pt]locy-v0.north east) -- 
 ([xshift=-16pt, yshift=-5pt]locy-v0.south east) node {};
 
\path (tid2start) node[anchor=south, rectangle, fill=red!20, draw=red, font=\small, inner sep=1pt] {$\thid_2$};

%Stack for threads tid_1 and tid_2

%\draw[-, dashed] let 
   %\p1 = ([xshift=0pt]locy.west),
   %\p2 = ([yshift=-5pt]locycells.south),
   %\p3 = ([xshift=10pt]locycells.east) in
   %(\x1, \y2) -- (\x3, \y2);
   
%\matrix(stacks) [
   %matrix of nodes,
   %anchor=north, 
   %text=blue, 
   %font=\normalsize, 
   %row 1/.style = {text = blue}, 
   %row 2/.style = {text = red}, 
   %text width= 13mm ] 
   %at ([xshift=-10pt,yshift=-8pt]locycells.south) {
   %$\thid_1:$ & $\retvar = 0$\\
   %$\thid_2:$ & $\retvar = 0$\\
   %};
\end{pgfonlayer}
\end{centertikz}
\caption{When \( \txid_{2}\) starts}
\label{fig:opsem-example-c}
\end{halfsubfig}
&
\begin{halfsubfig}
\begin{centertikz}

\begin{pgfonlayer}{foreground}
%Uncomment line below for help lines
%\draw[help lines] grid(5,4);

%Location x
\node(locx) {$\ke_{\vx} \mapsto$};

\matrix(versionx) [version list] 
    at ([xshift=\tikzkvspace]locx.east) { 
    {a} & $\txid_{0}$ &{a} & $\txid_{1}$\\
    {a} & $\emptyset$ & {a} & $\Set{\txid_{2}}$ \\
};
\tikzvalue{versionx-1-1}{versionx-2-1}{locx-v0}{0};
\tikzvalue{versionx-1-3}{versionx-2-3}{locx-v1}{1};

%Location y
\path (locx.south) + (0,\tikzkeyspace) node (locy) {$\ke_{\vy} \mapsto$};
\matrix(versiony) [version list]
   at ([xshift=\tikzkvspace]locy.east) {
 {a} & $\txid_0$ & {a} & $\txid_1$\\
  {a} & $\Set{\txid_{2}}$ & {a} & $\emptyset$ \\
};
\tikzvalue{versiony-1-1}{versiony-2-1}{locy-v0}{0};
\tikzvalue{versiony-1-3}{versiony-2-3}{locy-v1}{1};

% \draw[-, red, very thick, rounded corners] ([xshift=-5pt, yshift=5pt]locx-v1.north east) |- 
%  ($([xshift=-5pt,yshift=-5pt]locx-v1.south east)!.5!([xshift=-5pt, yshift=5pt]locy-v0.north east)$) -| ([xshift=-5pt, yshift=5pt]locy-v0.south east);

%blue view - I should  check whether I can use pgfkeys to just declare the list of locations, and then add the view automatically.
\draw[-, blue, very thick, rounded corners=10pt]
 ([xshift=-3pt, yshift=20pt]locx-v1.north east) node (tid1start) {} -- 
% ([xshift=-2pt, yshift=-5pt]locx-v0.south east) --
% ([xshift=-2pt, yshift=5pt]locy-v0.north east) -- 
 ([xshift=-3pt, yshift=-5pt]locy-v1.south east);
 
 \path (tid1start) node[anchor=south, rectangle, fill=blue!20, draw=blue, font=\small, inner sep=1pt] {$\thid_1$};

%red view
\draw[-, red, very thick, rounded corners = 10pt]
 ([xshift=-16pt, yshift=5pt]locx-v1.north east) node (tid2start) {}-- 
 ([xshift=-16pt, yshift=-5pt]locx-v1.south east) --
 ([xshift=-16pt, yshift=5pt]locy-v0.north east) -- 
 ([xshift=-16pt, yshift=-5pt]locy-v0.south east) node {};
 
\path (tid2start) node[anchor=south, rectangle, fill=red!20, draw=red, font=\small, inner sep=1pt] {$\thid_2$};

%Stack for threads tid_1 and tid_2

%\draw[-, dashed] let 
   %\p1 = ([xshift=0pt]locy.west),
   %\p2 = ([yshift=-5pt]locycells.south),
   %\p3 = ([xshift=10pt]locycells.east) in
   %(\x1, \y2) -- (\x3, \y2);
   
%\matrix(stacks) [
   %matrix of nodes,
   %anchor=north, 
   %text=blue, 
   %font=\normalsize, 
   %row 1/.style = {text = blue}, 
   %row 2/.style = {text = red}, 
   %text width= 15mm ] 
   %at ([xshift=-10pt,yshift=-8pt]locycells.south) {
   %$\thid_1:$ & $\retvar = 0$\\
   %$\thid_2:$ & $\retvar = {\Large \frownie}$\\
   %};
   \end{pgfonlayer}
\end{centertikz}
   \caption{Transaction \( \txid_{2}\) returns \( \sadface \)}
    \label{fig:opsem-example-d}
\end{halfsubfig}
\\
\end{tabular}
\end{center}
\hrule\vspace{5pt}
\caption{Graphical Representation of configurations 
obtained through the execution of $\prog_1$.}
\label{fig:opsem.example}
\label{fig:opsem-example}
\end{figure}

Before any computation, the initial configuration for $\prog_1$ is the one in which there are two keys \( \ke_{\vx}\) and \( \ke_{\vy} \) where each key is associated with an initial version with value zero written by an initialisation transaction $\txid_0$, \( \hh_{0} = \Set{\ke_{\vx} \mapsto \List{(0, \txid, \emptyset)}, \ke_{\vy} \mapsto \List{(0, \txid, \emptyset)}} \).
The initial view of each thread points to the initial version of each key, \( \vi^{1}_{0} = \vi^{2}_{0} = \Set{\ke_{\vx} \mapsto 1, \ke_{\vy} \mapsto 1}\).
The two threads have the same initial stack containing two variables \( \vx \) and \( \vy \) referring to the only keys in the key-value store respectively, \ie \( \stk^{1}_{0} = \stk^{2}_{0} = \Set{\vx \mapsto \ke_{\vx}, \vy \mapsto \ke_{\vy}}\).
Therefore the initial configuration is \( (\hh_{0}, \thdenv_{0}, \prog_{1}) \) where \( \thdenv_{0} = \Set{\thid_{1} \mapsto (\stk_{0}^{1}, \vi_{0}^{1}), \thid_{2} \mapsto (\stk_{0}^{2}, \vi_{0}^{2})}\).
\cref{fig:opsem-example-a} gives a graphical representation of the initial configuration.
 
%\ac{
%Before showing the computation of $\prog_1$ that leads to the transaction of 
%$\thid_2$ to return ${\Large \frownie{}}$, we need to introduce some 
%definitions and  notation.
%The initial configuration in which $\prog_1$ is executed is the one in which 
%each location has an initial version written by some initialisation transaction $\tsid_0$, 
%the view of each thread points to the initial version of each location, and all the  
%thread stacks are initialised to $0$. Assuming that the only key in the database 
%are $[\loc_x], [\loc_y]$, and the only variable in the thread stack is $\retvar$, 
%the initial configuration is then given by $\mathcal{C}_0 = (\hh_{0}, \mathsf{Env}_0)$, 
%where $\hh_{0}([\loc_{x}) = \hh_{0}([\loc_y]) =  (0, \tsid_0, \emptyset)$,  
%$\mathsf{Env}_0(\thid_1) = \mathsf{Env}_0(\thid_2) = ([\retvar \mapsto 0], V_0)$, 
%and $V_0([\loc_{x}]) = V_0([\loc_{y}]) = 0$. Here and in the following, we prefer 
%to adopt a more graphical notation for configurations. For example, the initial configuration 
%defined is represented graphically in Figure \ref{fig:opsem.example}(a). 

%In the picture above, the part above the dashed line represents the history heap and 
%view of each thread, while the part below the dashed line contains the thread-stack 
%of each thread. History heaps are represented as mappings to locations to lists of cells, 
%each of which represents a version and has three component: the value of the version 
%to the left, the identifier of the transaction that wrote it to the top right, and the 
%set of transactions that read the version to the bottom right. Vertical lines labelled 
%with thread identifiers are used to represent the views. The position of the view of 
%thread $\thid_1$ relatively to the location $[\loc_x]$ is determined by the version 
%at which the vertical line labelled $\thid_1$ crosses the list of versions for $[\loc_x]$,
 %and similarly for $[\loc_y]$.
%}

We are now ready to show how to derive a computation of $\prog_1$ that violates \emph{atomic visibility} and it will be explained formally in \cref{sec:example-commit-test}.
In the specific computation, we choose to let the transaction $\thid_1$ commit first.
According to rule $\rl{PCommit}$, we need to perform the following steps: 
\sx{Not sure the bullet points work here? Just typesetting. }
\begin{itemize}
\item
Arbitrarily shift the view $\vi_{0}^{1}$ for thread $\thid_1$ to the right as long as it is within the bound of key-value store and obtain a view \(\vi'' \) such that \( \vi'' \geq \vi_{0}^{1} \). 
Because $\hh_0$ contains only one version per key, here the only possibility is that $\vi'' = \vi_{0}^{1}$.
\item
Determine the initial snapshot $\sn = \clpsHH{\hh_0, \vi''}$ for the transaction $\thid_1$.
In this case, we have that $\h = \Set{\loc_x \mapsto 0, \loc_y \mapsto 0}$.
\item 
Given the initial snapshot \( \h \), initially empty fingerprint \( \unitO \) and the stack \( \stk_{0}^{1}\), by the operational semantics for transaction (\cref{fig:transaction_semantics}), after executing the transactional codes \( \pmutate{\vx}{1}; \pmutate{\vy}{1} \), the finial fingerprints include two write operations as $\opset = \Set{(\otW, \ke_{\vx}, 1), (\otW, \ke_{\vy}, 1)}$.
\ac{
\sx{Not sure those details are necessary.}
This amounts to execute the transaction in isolation from the external environment, using the rules in the operational semantics for 
 transactions. Because this execution must match the premiss of Rule $(C-Tx)$,  The code is run using $h$ as the initial heap, $\sigma_0$ as the initial 
 thread stack, $\tau_0 = \lambda_a.0$ as the initial transaction stack, and $\emptyset$ as the 
 initial fingerprint. We only need to apply 
 Rule $(Tx-prim)$ twice, in which case we obtain
 \begin{equation}
\label{eq:tx1}
\begin{array}{lcr}
& \sigma_0 \vdash \left\langle h_0, \_, \emptyset, \begin{array}{l}
\pmutate{\loc_x}{1};\\ \pmutate{\loc_y}{1} \end{array} \right\rangle 
&\rightarrow \\
\rightarrow & 
\left \langle h_0[ [\loc_x] \mapsto 1], \_, \big( \emptyset \oplus \text{WR}\; [\loc_x]: 1 \big), 
\pmutate{\loc_y}{1} \right\rangle &= \\
=&\left \langle h_0[[\loc_x] \mapsto 1], \_, \{\text{WR}\; [\loc_x]: 1\}, 
\pmutate{\loc_y}{1} \right\rangle 
&\rightarrow\\ 
\rightarrow & 
\left\langle h_0[[\loc_x] \mapsto_1, [\loc_y] \mapsto 1], \_, \big( \{\text{WR}\;[\loc_x]:1\} \oplus  (\text{WR}\;[\loc_y]: 1) \big), 
\stub \right\rangle & = \\
=& \left \langle \_, \_, \{\text{WR}\;[\loc_x]: 1, \text{WR}\;[\loc_y]:1 \}, \stub \right\rangle
\end{array}
\end{equation}
Therefore, we conclude $\mathcal{O} = \{\text{WR}\; [\loc_x] : 1, \text{WR}\;[\loc_y]:1\}$.
}
\item 
The transaction throws away the local snapshot and commits the fingerprints to the key-value store.
A fresh transaction identifier \( \txid_{1} \) is picked.
The new key-value store \( \hh_{1} \) is determined by the functions $\hh_{1} = \func{updHisHp}{\hh_0, \vi'',\txid_1, \opset}$ and the lower bound of the new view is given by the function \( \func{updView}{\hh_1, \vi'' ,\opset}\).
\item Last, the local view shift to the right so that it satisfies the execution tests, \( \vi' \geq \func{updView}{\hh_1, \vi'' ,\opset} \land (\hh_{0}, \vi'') \csatP \vi' : \opset \).
In this case, the permissive model does not constraint the view at all.
Therefore the overall final state is in \cref{fig:opsem-example-b}.
\end{itemize}

Next, thread $\thid_2$ executes its own transaction.
Note that the view for \( \thid_2\) still points to initial versions and the semantics allows the view get updated arbitrarily before executing the transaction.
Because the key-value store now has two versions for each key, there are exactly four possible views for the transaction \( \txid_{2} \).
In particular, assume it updates the view for \( \ke_{x} \) but not \( \ke_y \), \ie \( \Set{\ke_\vx \mapsto 1, \ke_\vx \mapsto 0} \) (\cref{fig:opsem-example-c}).
Given the view, the transaction \( \txid_{2} \) will assign \(\sadface\) to the \( \ret\) and this transaction is allowed to commit since the commit test does not stop this (\cref{fig:opsem-example-d}).

%\subsection{Example of Consistency models}
%\label{sec:example-commit-test}
%\ac{This Section is going to become heavy in pictures, which should be organised into figures.}
%In this Section we present different consistency models specifications. 
%For each of them, we give: 
%\begin{itemize}
%\item the intuition of the commit tests for different consistency models, and the formal definitions with respect to the \(\Como\) (\defref{def:consistency-models}).
%%describing the consistency guarantees that schedules of the database should have in plain English, 
%%\item a formal consistency model specification, in the style described in \S \ref{sec:semantics.programs},
%\item examples of litmus tests that, when executed, give rise to the anomalies that are forbidden from the consistency model, 
%\item an explanation of why the consistency model forbids the litmus tests to exhibit the anomaly that should be forbidden. 
%\end{itemize}
%Later, we will show how to compare our consistency models specifications with those already existing in the 
%literature.
%\ac{There is still a long-way to go before proving correspondence with dependency graph specifications, 
%but this should be mentioned here.}

%\subsection{Read Atomic} 
\sx{revisit and take out of the updateview }
\sx{This section might take out as the atomic constraint appears before. }
\label{sec:read-atomic}
\label{sec:semantics.example}
\label{sec:semantics-example}
Read atomic (RA) \cite{ramp} is the weakest consistency model among those that enjoy \emph{atomic visibility} \cite{framework-concur}. 
It requires of a transaction to read an atomic snapshot of the database and never observe the partial effects of other transactions.
This is also known as the \emph{all-or-nothing} property: a transaction observes either none or all the updates performed by other transactions. 
To illustrate that, we show in detail one possible computation of a program \( \prog_{1} \) consisting of two transactions executing in parallel if there is no atomic view constraint:
\[
    \prog_{1} \equiv 
    \begin{session}
        \begin{array}{@{}c || c@{}}
            \txid_{1} : 
            \begin{transaction}
                \pmutate{\vx}{1};\\
            	\pmutate{\vy}{1};
            \end{transaction} &
            \txid_{2} : 
            \begin{transaction}
                \pderef{\pvar{a}}{\vx};\\
            	\pderef{\pvar{b}}{\vy};\\
            	\pifs{\pvar{a}=1 \wedge \pvar{b}=0}\\
            		\quad \passign{\ret}{\sadface}
            	\pife
            \end{transaction}
        \end{array}
    \end{session}
 \]
The \( \pifs{\expr} \cmd_{1} \pifm \cmd_{2} \pife \) is encoded as \( (\passume{\expr} \pseq \cmd_{1}) \pchoice (\neg\passume{\expr} \pseq \cmd_{2} )\).
To recall, we often write \( \cmd_{1} \ppar \cmd_{2} \ppar \dots \ppar \cmd_{n}\) as a syntactic sugar for a program \( \prog \) with implicit unique thread identifiers \( \prog = \Set{\thid_{1} \mapsto \cmd_{1}, \thid_{2} \mapsto \cmd_{2}, \dots, \thid_{n} \mapsto \cmd_{n}  }\).
For better presentation, we annotated transactions with unique identifiers, yet they are allocated dynamically in the semantics.
We also treat the value assigned to the \( \ret \) variable as \emph{returned value}.
Assume the variables \( \vx \) and \( \vy \) refer to two key, and \( \va \) and \( \vb \) are local variables to threads.

The special symbol \(\sadface\), for example the returned value by the transaction $\txid_2$, is to emphasise some undesirable behaviour of a transaction.
In this case, the undesirable behaviour corresponds to the transaction to the right \( \txid_{2} \) observing only one of the updates from \( \txid_{1} \). 
%Intuitively, this behaviour violates the constraints that transactions should be executed atomically (further discussed in \cref{......}), we want to show that if no restrictions are placed on the consistency model specification, it is possible for $\prog_1$ to reach a configuration where the second transaction $\txid_2$ returns $\sadface$. 
To illustrate this, we instantiate the operation semantics with the most permissive execution tests \( \csatP \): \( (\hh, \vi) \csatP \opset : \vi'  \defeq  \true \) and we assume there is no \( \predn{atomic} \) constraints for views.

\begin{figure}[!t]
\hrule
\begin{center}
\begin{tabular}{@{}c@{}@{}c@{}}
\begin{halfsubfig}
\begin{centertikz}
\begin{pgfonlayer}{foreground}

%Location x
\node(locx) {$\ke_{\vx} \mapsto$};

\matrix(initx) [version list] 
    at ([xshift=\tikzkvspace]locx.east) {
    {a} & $\txid_0$ \\ 
    {a} & $\emptyset$ \\
};  
\tikzvalue{initx-1-1}{initx-2-1}{locx-v0}{0};

%Location y
\path (locx.south) + (0,\tikzkeyspace) node (locy) {$\ke_\vy \mapsto$};
\matrix(inity) [version list] 
    at ([xshift=\tikzkvspace]locy.east) {
    {a} & $\txid_0$ \\
    {a} & $\emptyset$ \\
};
\tikzvalue{inity-1-1}{inity-2-1}{locy-v0}{0};

%blue view - I should  check whether I can use pgfkeys to just declare the list of locations, and then add the view automatically.
\draw[-, blue, very thick, rounded corners=10pt]
([xshift=-3pt, yshift=20pt]locx-v0.north east) node (tid1start) {} -- 
([xshift=-3pt, yshift=-5pt]locy-v0.south east);
 
\path (tid1start) node[anchor=south, rectangle, fill=blue!20, draw=blue, font=\small, inner sep=1pt] {$\thid_1$};

%red view
\draw[-, red, very thick, rounded corners = 10pt]
([xshift=-16pt, yshift=5pt]locx-v0.north east) node (tid2start) {}-- 
([xshift=-16pt, yshift=-5pt]locy-v0.south east) node {};
 
\path (tid2start) node[anchor=south, rectangle, fill=red!20, draw=red, font=\small, inner sep=1pt] {$\thid_2$};

\end{pgfonlayer}
\end{centertikz}
\caption{Initial state}
\label{fig:opsem-example-a}
\end{halfsubfig}
&
\begin{halfsubfig}
\begin{centertikz}

\begin{pgfonlayer}{foreground}
%Uncomment line below for help lines

%Location x
\node(locx) {$\ke_{\vx} \mapsto$};

\matrix(versionx) [version list] 
    at ([xshift=\tikzkvspace]locx.east) { 
    {a} & $\txid_{0}$ &{a} & $\txid_{1}$\\
    {a} & $\emptyset$ & {a} & $\emptyset$ \\
};
\tikzvalue{versionx-1-1}{versionx-2-1}{locx-v0}{0};
\tikzvalue{versionx-1-3}{versionx-2-3}{locx-v1}{1};

%Location y
\path (locx.south) + (0,\tikzkeyspace) node (locy) {$\ke_{\vy} \mapsto$};
\matrix(versiony) [version list]
   at ([xshift=\tikzkvspace]locy.east) {
 {a} & $\txid_0$ & {a} & $\txid_1$\\
  {a} & $\emptyset$ & {a} & $\emptyset$ \\
};
\tikzvalue{versiony-1-1}{versiony-2-1}{locy-v0}{0};
\tikzvalue{versiony-1-3}{versiony-2-3}{locy-v1}{1};

%blue view - I should  check whether I can use pgfkeys to just declare the list of locations, and then add the view automatically.
\draw[-, blue, very thick, rounded corners=10pt]
([xshift=-3pt, yshift=20pt]locx-v1.north east) node (tid1start) {} -- 
([xshift=-3pt, yshift=-5pt]locy-v1.south east);
 
\path (tid1start) node[anchor=south, rectangle, fill=blue!20, draw=blue, font=\small, inner sep=1pt] {$\thid_1$};

%red view
\draw[-, red, very thick, rounded corners = 10pt]
([xshift=-16pt, yshift=5pt]locx-v0.north east) node (tid2start) {}-- 
([xshift=-16pt, yshift=-5pt]locy-v0.south east) node {};
 
\path (tid2start) node[anchor=south, rectangle, fill=red!20, draw=red, font=\small, inner sep=1pt] {$\thid_2$};

\end{pgfonlayer}
\end{centertikz}
\caption{After transaction \( \txid_{1}\)} 
\label{fig:opsem-example-b}
\end{halfsubfig}
\\
\begin{halfsubfig}
\begin{centertikz}

\begin{pgfonlayer}{foreground}

%Location x
\node(locx) {$\ke_{\vx} \mapsto$};

\matrix(versionx) [version list] 
    at ([xshift=\tikzkvspace]locx.east) { 
    {a} & $\txid_{0}$ &{a} & $\txid_{1}$\\
    {a} & $\emptyset$ & {a} & $\emptyset$ \\
};
\tikzvalue{versionx-1-1}{versionx-2-1}{locx-v0}{0};
\tikzvalue{versionx-1-3}{versionx-2-3}{locx-v1}{1};

%Location y
\path (locx.south) + (0,\tikzkeyspace) node (locy) {$\ke_{\vy} \mapsto$};
\matrix(versiony) [version list]
    at ([xshift=\tikzkvspace]locy.east) {
    {a} & $\txid_0$ & {a} & $\txid_1$\\
    {a} & $\emptyset$ & {a} & $\emptyset$ \\
};
\tikzvalue{versiony-1-1}{versiony-2-1}{locy-v0}{0};
\tikzvalue{versiony-1-3}{versiony-2-3}{locy-v1}{1};

%blue view - I should  check whether I can use pgfkeys to just declare the list of locations, and then add the view automatically.
\draw[-, blue, very thick, rounded corners=10pt]
([xshift=-3pt, yshift=20pt]locx-v1.north east) node (tid1start) {} -- 
([xshift=-3pt, yshift=-5pt]locy-v1.south east);

\path (tid1start) node[anchor=south, rectangle, fill=blue!20, draw=blue, font=\small, inner sep=1pt] {$\thid_1$};

%red view
\draw[-, red, very thick, rounded corners = 10pt]
([xshift=-16pt, yshift=5pt]locx-v1.north east) node (tid2start) {}-- 
([xshift=-16pt, yshift=-5pt]locx-v1.south east) --
([xshift=-16pt, yshift=5pt]locy-v0.north east) -- 
([xshift=-16pt, yshift=-5pt]locy-v0.south east) node {};
 
\path (tid2start) node[anchor=south, rectangle, fill=red!20, draw=red, font=\small, inner sep=1pt] {$\thid_2$};

\end{pgfonlayer}
\end{centertikz}
\caption{When \( \txid_{2}\) starts}
\label{fig:opsem-example-c}
\end{halfsubfig}
&
\begin{halfsubfig}
\begin{centertikz}

\begin{pgfonlayer}{foreground}
%Uncomment line below for help lines
%\draw[help lines] grid(5,4);

%Location x
\node(locx) {$\ke_{\vx} \mapsto$};

\matrix(versionx) [version list] 
    at ([xshift=\tikzkvspace]locx.east) { 
    {a} & $\txid_{0}$ &{a} & $\txid_{1}$\\
    {a} & $\emptyset$ & {a} & $\Set{\txid_{2}}$ \\
};
\tikzvalue{versionx-1-1}{versionx-2-1}{locx-v0}{0};
\tikzvalue{versionx-1-3}{versionx-2-3}{locx-v1}{1};

%Location y
\path (locx.south) + (0,\tikzkeyspace) node (locy) {$\ke_{\vy} \mapsto$};
\matrix(versiony) [version list]
   at ([xshift=\tikzkvspace]locy.east) {
 {a} & $\txid_0$ & {a} & $\txid_1$\\
  {a} & $\Set{\txid_{2}}$ & {a} & $\emptyset$ \\
};
\tikzvalue{versiony-1-1}{versiony-2-1}{locy-v0}{0};
\tikzvalue{versiony-1-3}{versiony-2-3}{locy-v1}{1};


%blue view - I should  check whether I can use pgfkeys to just declare the list of locations, and then add the view automatically.
\draw[-, blue, very thick, rounded corners=10pt]
([xshift=-3pt, yshift=20pt]locx-v1.north east) node (tid1start) {} -- 
([xshift=-3pt, yshift=-5pt]locy-v1.south east);

\path (tid1start) node[anchor=south, rectangle, fill=blue!20, draw=blue, font=\small, inner sep=1pt] {$\thid_1$};

%red view
\draw[-, red, very thick, rounded corners = 10pt]
([xshift=-16pt, yshift=5pt]locx-v1.north east) node (tid2start) {}-- 
([xshift=-16pt, yshift=-5pt]locx-v1.south east) --
([xshift=-16pt, yshift=5pt]locy-v0.north east) -- 
([xshift=-16pt, yshift=-5pt]locy-v0.south east) node {};

\path (tid2start) node[anchor=south, rectangle, fill=red!20, draw=red, font=\small, inner sep=1pt] {$\thid_2$};

\end{pgfonlayer}
\end{centertikz}
\caption{Transaction \( \txid_{2}\) returns \( \sadface \)}
\label{fig:opsem-example-d}
\end{halfsubfig}
\\
\end{tabular}
\end{center}
\hrule
\caption{Graphical Representation of configurations 
obtained through the execution of $\prog_1$.}
\label{fig:opsem.example}
\label{fig:opsem-example}
\end{figure}

Before any computation, the initial configuration for $\prog_1$ is the one in which there are two keys \( \ke_{\vx}\) and \( \ke_{\vy} \) where each key is associated with an initial version with value zero written by an initialisation transaction $\txid_0$, \( \hh_{0} = \Set{\ke_{\vx} \mapsto \List{(0, \txid, \emptyset)}, \ke_{\vy} \mapsto \List{(0, \txid, \emptyset)}} \).
The initial view of each thread points to the initial version of each key, \( \vi^{1}_{0} = \vi^{2}_{0} = \Set{\ke_{\vx} \mapsto 1, \ke_{\vy} \mapsto 1}\).
The two threads have the same initial stack containing two variables \( \vx \) and \( \vy \) referring to the only keys in the key-value store respectively, \ie \( \stk^{1}_{0} = \stk^{2}_{0} = \Set{\vx \mapsto \ke_{\vx}, \vy \mapsto \ke_{\vy}}\).
Therefore the initial configuration is \( (\hh_{0}, \thdenv_{0}, \prog_{1}) \) where \( \thdenv_{0} = \Set{\thid_{1} \mapsto (\stk_{0}^{1}, \vi_{0}^{1}), \thid_{2} \mapsto (\stk_{0}^{2}, \vi_{0}^{2})}\).
\cref{fig:opsem-example-a} gives a graphical representation of the initial configuration.
 
%\ac{
%Before showing the computation of $\prog_1$ that leads to the transaction of 
%$\thid_2$ to return ${\Large \frownie{}}$, we need to introduce some 
%definitions and  notation.
%The initial configuration in which $\prog_1$ is executed is the one in which 
%each location has an initial version written by some initialisation transaction $\tsid_0$, 
%the view of each thread points to the initial version of each location, and all the  
%thread stacks are initialised to $0$. Assuming that the only key in the database 
%are $[\loc_x], [\loc_y]$, and the only variable in the thread stack is $\retvar$, 
%the initial configuration is then given by $\mathcal{C}_0 = (\hh_{0}, \mathsf{Env}_0)$, 
%where $\hh_{0}([\loc_{x}) = \hh_{0}([\loc_y]) =  (0, \tsid_0, \emptyset)$,  
%$\mathsf{Env}_0(\thid_1) = \mathsf{Env}_0(\thid_2) = ([\retvar \mapsto 0], V_0)$, 
%and $V_0([\loc_{x}]) = V_0([\loc_{y}]) = 0$. Here and in the following, we prefer 
%to adopt a more graphical notation for configurations. For example, the initial configuration 
%defined is represented graphically in Figure \ref{fig:opsem.example}(a). 

%In the picture above, the part above the dashed line represents the history heap and 
%view of each thread, while the part below the dashed line contains the thread-stack 
%of each thread. History heaps are represented as mappings to locations to lists of cells, 
%each of which represents a version and has three component: the value of the version 
%to the left, the identifier of the transaction that wrote it to the top right, and the 
%set of transactions that read the version to the bottom right. Vertical lines labelled 
%with thread identifiers are used to represent the views. The position of the view of 
%thread $\thid_1$ relatively to the location $[\loc_x]$ is determined by the version 
%at which the vertical line labelled $\thid_1$ crosses the list of versions for $[\loc_x]$,
 %and similarly for $[\loc_y]$.
%}

We are now ready to show how to derive a computation of $\prog_1$ that violates \emph{atomic visibility} and it will be explained formally in \cref{sec:example-commit-test}.
In the specific computation, we choose to let the transaction $\thid_1$ commit first.
According to rule $\rl{PCommit}$, we need to perform the following steps: 
\sx{Not sure the bullet points work here? Just typesetting. }
\begin{itemize}
\item
Arbitrarily shift the view $\vi_{0}^{1}$ for thread $\thid_1$ to the right as long as it is within the bound of key-value store and obtain a view \(\vi'' \) such that \( \vi'' \geq \vi_{0}^{1} \). 
Because $\hh_0$ contains only one version per key, here the only possibility is that $\vi'' = \vi_{0}^{1}$.
\item
Determine the initial snapshot $\sn = \clpsHH{\hh_0, \vi''}$ for the transaction $\thid_1$.
In this case, we have that $\h = \Set{\loc_x \mapsto 0, \loc_y \mapsto 0}$.
\item 
Given the initial snapshot \( \h \), initially empty fingerprint \( \unitO \) and the stack \( \stk_{0}^{1}\), by the operational semantics for transaction (\cref{fig:transaction_semantics}), after executing the transactional codes \( \pmutate{\vx}{1}; \pmutate{\vy}{1} \), the finial fingerprints include two write operations as $\opset = \Set{(\otW, \ke_{\vx}, 1), (\otW, \ke_{\vy}, 1)}$.
%\ac{
%\sx{Not sure those details are necessary.}
%This amounts to execute the transaction in isolation from the external environment, using the rules in the operational semantics for 
 %transactions. Because this execution must match the premiss of Rule $(C-Tx)$,  The code is run using $h$ as the initial heap, $\sigma_0$ as the initial 
 %thread stack, $\tau_0 = \lambda_a.0$ as the initial transaction stack, and $\emptyset$ as the 
 %initial fingerprint. We only need to apply 
 %Rule $(Tx-prim)$ twice, in which case we obtain
 %\begin{equation}
%\label{eq:tx1}
%\begin{array}{lcr}
%& \sigma_0 \vdash \left\langle h_0, \_, \emptyset, \begin{array}{l}
%\pmutate{\loc_x}{1};\\ \pmutate{\loc_y}{1} \end{array} \right\rangle 
%&\rightarrow \\
%\rightarrow & 
%\left \langle h_0[ [\loc_x] \mapsto 1], \_, \big( \emptyset \oplus \text{WR}\; [\loc_x]: 1 \big), 
%\pmutate{\loc_y}{1} \right\rangle &= \\
%=&\left \langle h_0[[\loc_x] \mapsto 1], \_, \{\text{WR}\; [\loc_x]: 1\}, 
%\pmutate{\loc_y}{1} \right\rangle 
%&\rightarrow\\ 
%\rightarrow & 
%\left\langle h_0[[\loc_x] \mapsto_1, [\loc_y] \mapsto 1], \_, \big( \{\text{WR}\;[\loc_x]:1\} \oplus  (\text{WR}\;[\loc_y]: 1) \big), 
%\stub \right\rangle & = \\
%=& \left \langle \_, \_, \{\text{WR}\;[\loc_x]: 1, \text{WR}\;[\loc_y]:1 \}, \stub \right\rangle
%\end{array}
%\end{equation}
%Therefore, we conclude $\mathcal{O} = \{\text{WR}\; [\loc_x] : 1, \text{WR}\;[\loc_y]:1\}$.
%}
\item 
The transaction throws away the local snapshot and commits the fingerprints to the key-value store.
A fresh transaction identifier \( \txid_{1} \) is picked.
The new key-value store \( \hh_{1} \) is determined by the functions $\hh_{1} = \func{updHisHp}{\hh_0, \vi'',\txid_1, \opset}$ and the lower bound of the new view is given by the function \( \func{updView}{\hh_1, \vi'' ,\opset}\).
\item Last, the local view shift to the right so that it satisfies the execution tests, \( \vi' \geq \func{updView}{\hh_1, \vi'' ,\opset} \land (\hh_{0}, \vi'') \csatP \vi' : \opset \).
In this case, the permissive model does not constraint the view at all.
Therefore the overall final state is in \cref{fig:opsem-example-b}.
\end{itemize}

Next, thread $\thid_2$ executes its own transaction.
Note that the view for \( \thid_2\) still points to initial versions and the semantics allows the view get updated arbitrarily before executing the transaction.
Because the key-value store now has two versions for each key, there are exactly four possible views for the transaction \( \txid_{2} \).
In particular, assume it updates the view for \( \ke_{x} \) but not \( \ke_y \), \ie \( \Set{\ke_\vx \mapsto 1, \ke_\vx \mapsto 0} \) (\cref{fig:opsem-example-c}).
Given the view, the transaction \( \txid_{2} \) will assign \(\sadface\) to the \( \ret\) and this transaction is allowed to commit since the commit test does not stop this (\cref{fig:opsem-example-d}).

%\subsection{Example of Consistency models}
%\label{sec:example-commit-test}
%\ac{This Section is going to become heavy in pictures, which should be organised into figures.}
%In this Section we present different consistency models specifications. 
%For each of them, we give: 
%\begin{itemize}
%\item the intuition of the commit tests for different consistency models, and the formal definitions with respect to the \(\Como\) (\defref{def:consistency-models}).
%%describing the consistency guarantees that schedules of the database should have in plain English, 
%%\item a formal consistency model specification, in the style described in \S \ref{sec:semantics.programs},
%\item examples of litmus tests that, when executed, give rise to the anomalies that are forbidden from the consistency model, 
%\item an explanation of why the consistency model forbids the litmus tests to exhibit the anomaly that should be forbidden. 
%\end{itemize}
%Later, we will show how to compare our consistency models specifications with those already existing in the 
%literature.
%\ac{There is still a long-way to go before proving correspondence with dependency graph specifications, 
%but this should be mentioned here.}







Intuitively, in such a program, it violate the atomic visibility because it is allowed to execute the second transaction \( \trans_2\) when the client $\thid_2$ only observes partial effect from transaction \( \txid_1 \).
%\ac{, which is obtained by removing all the information about $\thid_1$ (view and stack) in Figure \ref{fig:opsem.example}(c).
%\[
%\prog_1 \equiv 
    %\begin{array}{c} 
    %\begin{transaction} 
        %\pderef{\pvar{a}}{\vx};\\
        %\pderef{\pvar{b}}{\vy};\\
        %\pifs{\pvar{a}=1 \wedge \pvar{b}=0}\\
        %\quad \passign{\retvar}{\Large \frownie{}} \\
        %\pife
    %\end{transaction}
%\end{array}
%\]
%}

%\ac{
%To avoid transactions to only observe the partial effects of other transactions, we 
%must ensure that transactional code cannot be executed by a client whose 
%views is up-to-date with respect to some transaction $\tsid$ for some location $[\loc_x]$, 
%but not for some other location $[\loc_y]$. This leads to the following definition.
%}
To avoid a transaction to observe the partial effects of other transactions, it needs to ensure that transactional code cannot be executed by a client whose views is partially up-to-date with respect to some transactions.
This leads to the following definition.
\begin{definition}[Read atomic]
\label{def:readatomic}
%Let $\hh$ be a history heap,$V$ be a view, $[\loc_x]$ be 
%a location and $\nu$ be a version. We say that $V$ $[\loc_x]$-\emph{sees} version 
%$\nu$ if there exists an index $i \leq V([\loc_x])$ such that $V(i) = \nu$. 
%We say that $V$ $[\loc_x]$-\emph{sees} transaction $\tsid$ if 
%$V$ $[\loc_x]$-sees a version $\nu = (\_, \tsid, \_)$. 
Given a view $\vi \in \Views$, a history heap $\hh \in \HisHeaps$, and a transaction identifier $\txid \in \TxID$, the view \emph{sees} the transaction in the history heap, written $\pred{visible}{\txid, \vi, \hh}$, if the view sees all the writes from the transaction,
%We say that $V$ \emph{sees} transaction $\tsid$ in $\hh$, written 
%$\mathsf{Visible}(\tsid, V, \hh)$, iff 
\sx{\( \exsts{i} \) might be enough}
\[
\begin{rclarray}
\pred{visible}{\txid, \vi, \hh} & \eqdef & \fora{\ke, i} \hh(\ke)(i) = (\stub, \txid, \stub) \implies i \leq \vi(\ke).
\end{rclarray}
\]
%\ac{In English: the view is up-to-date with respect to all the updates 
%performed by transaction $\tsid$.

%In English: if the view $V$ is up-to-date with some of the updates performed 
%by $\tsid$, then it must be up-to-date with all the updates performed by $\tsid$. 
%This is the all-or-nothing property.

%In English: Before executing a transaction, either you observe all or none the 
%updates of all other transactions. We may strengthen the consistency model and 
%require that the same property must be satisfied at the end as well, though 
%this is not strictly necessary. In this case the check becomes: 
%\[
%\mathsf{atomic}(\hh, V) \wedge \mathsf{atomic}(\hh, V') \wedge \mathsf{UpdateView}(\hh, V, \mathcal{O}) 
%\sqsubseteq V' \implies (\hh, V) \triangleright_{\mathsf{RA}} \mathcal{O}: V'.
%\]
%}

\sx{consistent has been used in many place, might mislead?}
Then given a history \( \hh \), the view $\vi$ is \emph{consistent} with respect to \emph{atomic visibility}, written $\pred{atomic}{\vi, \hh}$, if the view $\vi$ is up-to-date with some of the updates performed by transaction $\txid$, then it should be up-to-date with all the updates performed by $\txid$,
\[
\begin{rclarray}
\pred{atomic}{\vi ,\hh} & \eqdef & \fora{\txid } \exsts{\ke, i} i \leq \vi(\ke) \land \hh(\ke)(i) = (\stub, \txid, \stub) \implies \pred{visible}{\txid, \vi, \hh}
\end{rclarray}
\]
The commit test for \emph{read atomic} $\csatRA$ is defined as,
\[
\begin{rclarray}
(\hh, \vi) \csatRA \stub: \stub & \defeq & \pred{atomic}{\hh, \vi} 
\end{rclarray}
\]
%written $\mathsf{up-to-date}(\hh, V, \tsid, [\loc_x])$, 
%if either 
%
%\begin{itemize}
%\item for all indexes $i = 0,\cdots, \lvert \hh([\loc_x]) - 1 \rvert$, 
%$\WS(\hh([\loc_x])(i)) \neq \tsid)$, or 
%\item if $\WS(\hh([\loc_x])(i)) = \tsid$ for some $i = 0,\cdots, \lvert \hh([\loc_n]) -1 \rvert$, 
%then $i \leq V([\loc_n])$.
%\end{itemize}
\end{definition}

Suppose that we execute the program $\prog_1$ under read atomic $\comoRA$.
Assume the \( \txid_{1}\) is still committed first as in Section \ref{sec:semantics.example}, yielding the result shown in \cref{fig:opsem-example-b}.
Yet the second transaction \( \txid_{2} \) cannot starts with a view as in \cref{fig:opsem-example-c}, because later when the transaction try to commit, the read atomic commit test will stop it.
%\ac{
%$\langle \mathcal{C_0}, \prog_1 \rangle \xrightarrow{\mathsf{RA}} \langle \mathcal{C}_1, \prog_1' \rangle$, 
%where we recall that $\mathcal{C}_0$, $\mathcal{C}_1$ are depicted in Figure \ref{fig:opsem.exampe}(a), 
%\ref{fig:opsem.example}(b), respectively. 

%It is immediate to observe that the only way in which the execution of transaction $\ptrans{\trans}$ from $\thid_2$ in $\prog_1'$ can return value ${\Large \frownie}$ is the following: 
%\begin{itemize}
%\item first, push the view $V$ of client $\txid_2$ in the configuration 
%$\mathcal{C}_1$ of Figure \ref{fig:opsem.example}(b) to observe the update of location $[\vx]$, but not the update of 
%$[\vy]$. This view is the one labelled with $\txid_2$ in Figure \ref{fig:opsem.example}(c), and we refer 
%to it as $V'$;
%\item then, execute the transaction $\ptrans{\trans}$ in $\thid_2$. 
%\end{itemize}
%}

One may wonder whether it is the case that an execution of a command may get stuck because of a ill-defined execution test.
For example, for a given execution test $\ET$, it may be possible to reach a state of the system, with MKVS $\hh$ and view $\vi$ for client $\cl$ , such that for any $\vi': \vi \sqsubseteq \vi'$ we have that $ \comoRA \not\vdash (\hh, \vi') \csat[] \stub : \stub $.
We show that this is not the case for the execution test $\comoRA$. 

\ac{This does not mean that progress is ensured. A transaction may very well 
not terminate. Rather, this says that the inability of a program to execute a 
transaction does not depend on the execution test itself.}
\sx{what do you mean???}
\begin{proposition}
\label{prop:ra.progress}
The execution test $\comoRA$ \emph{does not hinder progress}: for any MKVS $\mkvs$, views $\vi$, and fingerprint $\opset$, there exist two views $\vi': \vi \sqsubseteq \vi'$, $\vi'': \Vupdate(\hh, \vi', \opset) \sqsubseteq \vi''$ such that $(\hh, \vi) \csatRA \opset : \vi'$.
\end{proposition}

\begin{proof}
For a given MKVS $\hh$, let $\func{up-to-date}{\hh} = \lambda \key{k}. (\lvert \hh \rvert - 1)$ be the view that points to the most recent version of each object. It is immediate to note that $\pred{atomic}{\hh, \func{up-to-date}{\hh}}$, and therefore $(\hh, \vi) \csatRA \opset : \vi'$. 
\end{proof}

%\subsection{Transactional Causal Consistency}
\begin{figure}
\centering
\hrule
\begin{tabular}{@{} c @{} c @{} c @{}}
\begin{halfsubfig}
\begin{centertikz}

\begin{pgfonlayer}{foreground}
%Uncomment line below for help lines

%Location x
\node(locx) {$\ke_\vx \mapsto$};

\matrix(versionx) [version list]
    at ([xshift=\tikzkvspace]locx.east) {
    {a} & $\txid_0$ \\
    {a} & $\emptyset$ \\
};
\tikzvalue{versionx-1-1}{versionx-2-1}{locx-v0}{0};

%Location y
\path (locx.south) + (0,\tikzkeyspace) node (locy) {$\ke_{\vy} \mapsto$};
\matrix(versiony) [version list]
    at ([xshift=\tikzkvspace]locy.east) {
    {a} & $\txid_0$ \\
    {a} & $\emptyset$ \\
};
\tikzvalue{versiony-1-1}{versiony-2-1}{locy-v0}{0};

% \draw[-, red, very thick, rounded corners] ([xshift=-5pt, yshift=5pt]locx-v1.north east) |- 
%  ($([xshift=-5pt,yshift=-5pt]locx-v1.south east)!.5!([xshift=-5pt, yshift=5pt]locy-v0.north east)$) -| ([xshift=-5pt, yshift=5pt]locy-v0.south east);

%blue view - I should  check whether I can use pgfkeys to just declare the list of locations, and then add the view automatically.
\draw[-, blue, very thick, rounded corners=10pt]
([xshift=-2pt, yshift=20pt]locx-v0.north east) node (tid1start) {} -- 
([xshift=-2pt, yshift=-5pt]locy-v0.south east);
 
\path (tid1start) node[anchor=south, rectangle, fill=blue!20, draw=blue, font=\small, inner sep=1pt] {$\thid_3$};

%red view
\draw[-, red, very thick, rounded corners = 10pt]
([xshift=-5pt, yshift=5pt]locx-v0.north east) -- 
([xshift=-5pt, yshift=-10pt]locy-v0.south east) node (tid2start) {};
 
\path (tid2start) node[anchor=north, rectangle, fill=red!20, draw=red, font=\small, inner sep=1pt] {$\thid_2$};
 
 %green view
\draw[-, DarkGreen, very thick, rounded corners = 10pt]
([xshift=-16pt, yshift=8pt]locx-v0.north east) node (tid3start) {}-- 
([xshift=-16pt, yshift=-5pt]locy-v0.south east);
 
\path (tid3start) node[anchor=south, rectangle, fill=DarkGreen!20, draw=DarkGreen, font=\small, inner sep=1pt] {$\thid_1$};

\end{pgfonlayer}
\end{centertikz}
\caption{}
\label{fig:cc-exec-a}
\end{halfsubfig}
&
\begin{halfsubfig}
\begin{centertikz}

\begin{pgfonlayer}{foreground}
%Uncomment line below for help lines
%\draw[help lines] grid(5,4);

%Location x
\node(locx) {$\ke_{\vx} \mapsto$};

\matrix(versionx) [version list]
    at ([xshift=\tikzkvspace]locx.east) {
    {a} & $\tsid_0$ & {a} & $\tsid_1$\\
    {a} & $\emptyset$ & {a} & $\emptyset$ \\
};
\tikzvalue{versionx-1-1}{versionx-2-1}{locx-v0}{0};
\tikzvalue{versionx-1-3}{versionx-2-3}{locx-v1}{1};

%Location y
\path (locx.south) + (0,\tikzkeyspace) node (locy) {$\ke_{\vy} \mapsto$};
\matrix(versiony) [version list]
   at ([xshift=\tikzkvspace]locy.east) {
 {a} & $\tsid_0$ \\
   {a} & $\emptyset$ \\
};
\tikzvalue{versiony-1-1}{versiony-2-1}{locy-v0}{0};

%blue view - I should  check whether I can use pgfkeys to just declare the list of locations, and then add the view automatically.
\draw[-, blue, very thick, rounded corners=10pt]
([xshift=-2pt, yshift=20pt]locx-v0.north east) node (tid1start) {} -- 
([xshift=-2pt, yshift=-5pt]locy-v0.south east);
 
\path (tid1start) node[anchor=south, rectangle, fill=blue!20, draw=blue, font=\small, inner sep=1pt] {$\thid_3$};

%red view
\draw[-, red, very thick, rounded corners = 10pt]
([xshift=-5pt, yshift=5pt]locx-v0.north east) -- 
([xshift=-5pt, yshift=-10pt]locy-v0.south east) node (tid2start) {};
 
\path (tid2start) node[anchor=north, rectangle, fill=red!20, draw=red, font=\small, inner sep=1pt] {$\thid_2$};
 
 %green view
\draw[-, DarkGreen, very thick, rounded corners = 10pt]
([xshift=-16pt, yshift=8pt]locx-v1.north east) node (tid3start) {}-- 
([xshift=-16pt, yshift=-5pt]locx-v1.south east) --
([xshift=-16pt, yshift=5pt]locy-v0.north east) -- 
([xshift=-16pt, yshift=-5pt]locy-v0.south east);
 
\path (tid3start) node[anchor=south, rectangle, fill=DarkGreen!20, draw=DarkGreen, font=\small, inner sep=1pt] {$\thid_1$};

\end{pgfonlayer}
\end{centertikz}
\caption{}
\label{fig:cc-exec-b}
\end{halfsubfig}
&
\begin{halfsubfig}
\begin{centertikz}
\begin{pgfonlayer}{foreground}
%Uncomment line below for help lines
%\draw[help lines] grid(5,4);

%Location x
\node(locx) {$\ke_{\vx} \mapsto$};

\matrix(versionx) [version list]
    at ([xshift=\tikzkvspace]locx.east) {
    {a} & $\tsid_0$ & {a} & $\tsid_1$\\
    {a} & $\emptyset$ & {a} & $\emptyset$ \\
};
\tikzvalue{versionx-1-1}{versionx-2-1}{locx-v0}{0};
\tikzvalue{versionx-1-3}{versionx-2-3}{locx-v1}{1};

%Location y
\path (locx.south) + (0,\tikzkeyspace) node (locy) {$\ke_{\vy} \mapsto$};
\matrix(versiony) [version list]
   at ([xshift=\tikzkvspace]locy.east) {
 {a} & $\tsid_0$ \\
   {a} & $\emptyset$ \\
};
\tikzvalue{versiony-1-1}{versiony-2-1}{locy-v0}{0};

%blue view - I should  check whether I can use pgfkeys to just declare the list of locations, and then add the view automatically.
\draw[-, blue, very thick, rounded corners=10pt]
([xshift=-2pt, yshift=20pt]locx-v0.north east) node (tid1start) {} -- 
([xshift=-2pt, yshift=-5pt]locy-v0.south east);
 
\path (tid1start) node[anchor=south, rectangle, fill=blue!20, draw=blue, font=\small, inner sep=1pt] {$\thid_3$};

%red view
\draw[-, red, very thick, rounded corners = 10pt]
([xshift=-5pt, yshift=5pt]locx-v1.north east) -- 
([xshift=-5pt, yshift=-5pt]locx-v1.south east) --
([xshift=-5pt, yshift=3pt]locy-v0.north east) -- 
([xshift=-5pt, yshift=-10pt]locy-v0.south east) node (tid2start) {};
 
\path (tid2start) node[anchor=north, rectangle, fill=red!20, draw=red, font=\small, inner sep=1pt] {$\thid_2$};
 
%green view
\draw[-, DarkGreen, very thick, rounded corners = 10pt]
([xshift=-16pt, yshift=8pt]locx-v1.north east) node (tid3start) {}-- 
([xshift=-16pt, yshift=-5pt]locx-v1.south east) --
([xshift=-16pt, yshift=5pt]locy-v0.north east) -- 
([xshift=-16pt, yshift=-5pt]locy-v0.south east);
 
\path (tid3start) node[anchor=south, rectangle, fill=DarkGreen!20, draw=DarkGreen, font=\small, inner sep=1pt] {$\thid_1$};

\end{pgfonlayer}
\end{centertikz}
\caption{}
\label{fig:cc-exec-c}
\end{halfsubfig}
\\
\begin{halfsubfig}
\begin{centertikz}

\begin{pgfonlayer}{foreground}
%Uncomment line below for help lines
%\draw[help lines] grid(5,4);

%Location x
\node(locx) {$\ke_{\vx} \mapsto$};

\matrix(versionx) [version list]
    at ([xshift=\tikzkvspace]locx.east) {
    {a} & $\tsid_0$ & {a} & $\tsid_1$\\
    {a} & $\emptyset$ & {a} & $\Set{\txid_{2}}$ \\
};
\tikzvalue{versionx-1-1}{versionx-2-1}{locx-v0}{0};
\tikzvalue{versionx-1-3}{versionx-2-3}{locx-v1}{1};

%Location y
\path (locx.south) + (0,\tikzkeyspace) node (locy) {$\ke_{\vy} \mapsto$};
\matrix(versiony) [version list]
    at ([xshift=\tikzkvspace]locy.east) {
    {a} & $\tsid_0$ & {a} & $\tsid_2$ \\
    {a} & $\emptyset$ & {a} & $\emptyset$\\
};
\tikzvalue{versiony-1-1}{versiony-2-1}{locy-v0}{0};
\tikzvalue{versiony-1-3}{versiony-2-3}{locy-v1}{1};

%blue view - I should  check whether I can use pgfkeys to just declare the list of locations, and then add the view automatically.
\draw[-, blue, very thick, rounded corners=10pt]
([xshift=-2pt, yshift=20pt]locx-v0.north east) node (tid1start) {} -- 
([xshift=-2pt, yshift=-5pt]locy-v0.south east);
 
\path (tid1start) node[anchor=south, rectangle, fill=blue!20, draw=blue, font=\small, inner sep=1pt] {$\thid_3$};

%red view
\draw[-, red, very thick, rounded corners = 10pt]
([xshift=-5pt, yshift=5pt]locx-v1.north east) -- 
([xshift=-5pt, yshift=-10pt]locy-v1.south east) node (tid2start) {};
 
\path (tid2start) node[anchor=north, rectangle, fill=red!20, draw=red, font=\small, inner sep=1pt] {$\thid_2$};
 
 %green view
\draw[-, DarkGreen, very thick, rounded corners = 10pt]
([xshift=-16pt, yshift=8pt]locx-v1.north east) node (tid3start) {}-- 
([xshift=-16pt, yshift=-5pt]locx-v1.south east) --
([xshift=-16pt, yshift=5pt]locy-v0.north east) -- 
([xshift=-16pt, yshift=-5pt]locy-v0.south east);
 
\path (tid3start) node[anchor=south, rectangle, fill=DarkGreen!20, draw=DarkGreen, font=\small, inner sep=1pt] {$\thid_1$};

\end{pgfonlayer}
\end{centertikz}
\caption{}
\label{fig:cc-exec-d}
\end{halfsubfig}
&
\begin{halfsubfig}
\begin{centertikz}

\begin{pgfonlayer}{foreground}
%Uncomment line below for help lines
%\draw[help lines] grid(5,4);

%Location x
\node(locx) {$\ke_{\vx} \mapsto$};

\matrix(versionx) [version list]
    at ([xshift=\tikzkvspace]locx.east) {
    {a} & $\tsid_0$ & {a} & $\tsid_1$\\
    {a} & $\emptyset$ & {a} & $\Set{\txid_{2}}$ \\
};
\tikzvalue{versionx-1-1}{versionx-2-1}{locx-v0}{0};
\tikzvalue{versionx-1-3}{versionx-2-3}{locx-v1}{1};

%Location y
\path (locx.south) + (0,\tikzkeyspace) node (locy) {$\ke_{\vy} \mapsto$};
\matrix(versiony) [version list]
    at ([xshift=\tikzkvspace]locy.east) {
    {a} & $\tsid_0$ & {a} & $\tsid_2$ \\
    {a} & $\emptyset$ & {a} & $\emptyset$\\
};
\tikzvalue{versiony-1-1}{versiony-2-1}{locy-v0}{0};
\tikzvalue{versiony-1-3}{versiony-2-3}{locy-v1}{1};

%blue view - I should  check whether I can use pgfkeys to just declare the list of locations, and then add the view automatically.
\draw[-, blue, very thick, rounded corners=10pt]
([xshift=-2pt, yshift=20pt]locx-v0.north east) node (tid1start) {} -- 
([xshift=-2pt, yshift=-5pt]locx-v0.south east) --
([xshift=-2pt, yshift=5pt]locy-v1.north east) -- 
([xshift=-2pt, yshift=-5pt]locy-v1.south east);
 
\path (tid1start) node[anchor=south, rectangle, fill=blue!20, draw=blue, font=\small, inner sep=1pt] {$\thid_3$};

%red view
\draw[-, red, very thick, rounded corners = 10pt]
([xshift=-5pt, yshift=5pt]locx-v1.north east) -- 
([xshift=-5pt, yshift=-10pt]locy-v1.south east) node (tid2start) {};
 
\path (tid2start) node[anchor=north, rectangle, fill=red!20, draw=red, font=\small, inner sep=1pt] {$\thid_2$};
 
%green view
\draw[-, DarkGreen, very thick, rounded corners = 10pt]
([xshift=-16pt, yshift=8pt]locx-v1.north east) node (tid3start) {}-- 
([xshift=-16pt, yshift=-5pt]locx-v1.south east) --
([xshift=-16pt, yshift=5pt]locy-v0.north east) -- 
([xshift=-16pt, yshift=-5pt]locy-v0.south east);
 
\path (tid3start) node[anchor=south, rectangle, fill=DarkGreen!20, draw=DarkGreen, font=\small, inner sep=1pt] {$\thid_1$};

\end{pgfonlayer}
\end{centertikz}
\caption{}
\label{fig:cc-exec-e}
\end{halfsubfig}
&
\begin{halfsubfig}
\begin{centertikz}

\begin{pgfonlayer}{foreground}
%Uncomment line below for help lines
%\draw[help lines] grid(5,4);

%Location x
\node(locx) {$\ke_{\vx} \mapsto$};

\matrix(versionx) [version list]
    at ([xshift=\tikzkvspace]locx.east) {
    {a} & $\tsid_0$ & {a} & $\tsid_1$\\
    {a} & $\Set{\txid_{3}}$ & {a} & $\Set{\txid_{2}}$ \\
};
\tikzvalue{versionx-1-1}{versionx-2-1}{locx-v0}{0};
\tikzvalue{versionx-1-3}{versionx-2-3}{locx-v1}{1};

%Location y
\path (locx.south) + (0,\tikzkeyspace) node (locy) {$\ke_{\vy} \mapsto$};
\matrix(versiony) [version list]
    at ([xshift=\tikzkvspace]locy.east) {
    {a} & $\tsid_0$ & {a} & $\tsid_2$ \\
    {a} & $\emptyset$ & {a} & $\Set{\txid_{3}}$\\
};
\tikzvalue{versiony-1-1}{versiony-2-1}{locy-v0}{0};
\tikzvalue{versiony-1-3}{versiony-2-3}{locy-v1}{1};

%blue view - I should  check whether I can use pgfkeys to just declare the list of locations, and then add the view automatically.
\draw[-, blue, very thick, rounded corners=10pt]
([xshift=-2pt, yshift=20pt]locx-v0.north east) node (tid1start) {} -- 
([xshift=-2pt, yshift=-5pt]locx-v0.south east) --
([xshift=-2pt, yshift=5pt]locy-v1.north east) -- 
([xshift=-2pt, yshift=-5pt]locy-v1.south east);
 
\path (tid1start) node[anchor=south, rectangle, fill=blue!20, draw=blue, font=\small, inner sep=1pt] {$\thid_3$};

%red view
\draw[-, red, very thick, rounded corners = 10pt]
([xshift=-5pt, yshift=5pt]locx-v1.north east) -- 
([xshift=-5pt, yshift=-10pt]locy-v1.south east) node (tid2start) {};
 
\path (tid2start) node[anchor=north, rectangle, fill=red!20, draw=red, font=\small, inner sep=1pt] {$\thid_2$};
 
%green view
\draw[-, DarkGreen, very thick, rounded corners = 10pt]
([xshift=-16pt, yshift=8pt]locx-v1.north east) node (tid3start) {}-- 
([xshift=-16pt, yshift=-5pt]locx-v1.south east) --
([xshift=-16pt, yshift=5pt]locy-v0.north east) -- 
([xshift=-16pt, yshift=-5pt]locy-v0.south east);
 
\path (tid3start) node[anchor=south, rectangle, fill=DarkGreen!20, draw=DarkGreen, font=\small, inner sep=1pt] {$\thid_1$};

\end{pgfonlayer}
\end{centertikz}
\caption{}
\label{fig:cc-exec-f}
\end{halfsubfig}
\end{tabular}
\hrule\vspace{5pt}
\caption{History heaps obtained in a execution of $\prog_2$.}
\label{fig:cc.exec}
\label{fig:cc-exec}
\end{figure}


\sx{Should we give intuition about causal dependencies here ?}
The next consistency model that we are interested is \emph{transactional causal consistency} \cite{cops}. 
Intuitively, it ensures that versions read by transactions are closed with respect to \emph{causal dependencies}. 
Consider for example the following program: 
\[
    \prog_2 \equiv \begin{session}
        \begin{array}{@{}c || c || c@{}}
            \txid_{1} : 
            \begin{transaction}
                \pmutate{\vx}{1};\\
            \end{transaction} &
            \txid_{2} : 
            \begin{transaction} 
                \pderef{\pvar{a}}{\vx};\\
                \pmutate{\vy}{\pvar{a}};\\
            \end{transaction} &
            \txid_{3} :
             \begin{transaction}
               	   \pderef{\pvar{a}}{\vx};\\
               	   \pderef{\pvar{b}}{\vy};\\
               	   \pifs{\pvar{a}=0 \wedge \pvar{b}=1}\\
               			\quad \passign{\retvar}{\sadface}
               		\pife
             \end{transaction}
        \end{array}
    \end{session}
 \]
For the sake of simplicity, we label the code of the three transactions above as $\txid_{1}, \txid_2, \txid_3$ from left to right.
It is easy to see that, if no constraints or even under read atomic, the third transaction $\txid_{3}$ can return ${\sadface{}}$. 
%The same is true even if the consistency model specification $\mathsf{RA}$ is assumed. 
Informally, the return of value ${\sadface}$ by $\txid_3$ can be obtained from the execution outlined below. 
\begin{itemize}
\item The initial configuration of this execution is depicted in \cref{fig:cc-exec-a}.
\item The transaction $\txid_{1}$ executes with the initial view, which points to the initial (and only) version for each location; after this executing the transaction, a new version $( 1, \txid_1, \emptyset )$ is appended to \( \ke_\vx\) (\cref{fig:cc-exec-b}).
\item The second client $\thid_2$ updates its view as to see the last version of $\ke_\vx$ installed by $\thid_1$ (\cref{fig:cc-exec-c}), after which it proceeds to execute transaction $\txid_{2}$. 
This results in a new version with value $1$ to be installed to the end of $\ke_\vy$ (\cref{fig:cc-exec-d}). 
\item Finally, the client $\thid_3$ updates its view so to observe the update of location $\ke_{\vy}$, but not the update of 
location $\ke_{\vx}$ (\cref{fig:cc-exec-e}).
Then, it executes $\txid_{3}$ which will return the value ${\sadface{}}$ (\cref{fig:cc-exec-f}).
\end{itemize}

In the last step, the client to the right commits the transaction $\txid_3$ in a state where its initial view observes the second version of the address $\vy$, which is created by \( \txid_{2} \).
However, because the transaction \( \txid_{2} \) read the second version of \( \ke_\vx \) and create the second version of \( \ke_\vy \), this means the latter depends on the former.
Yet the transaction \( \txid_{3} \) does not read from the second version of \( \ke_\vx \), which is disallowed by \emph{transactional causal consistency}.
Summarising, under transactional causal consistency if a transaction sees updates for a key \( \ke \), it should also observes those keys that \( \ke \) depends on.

%\ac{
%but not the update to address $\vx$.
%However, the update of $[\vy]$ committed by $\txid_{2}$, consisted in copying the value of the update 
%of $\ke_{\vy}$: that is, the update of $\ke_{\vy}$ \textbf{depends} from the update of $\ke_{\vx}$. 
%Summarising, the execution of transaction $\ptrans{\trans_3}$ resulted in a violation of 
%causality: the update of $\ke_{\vy}$ is observed, but not the update of $\ke_{\vx}$ on which 
%it depends.
%}

To formally specify \emph{transactional causal consistency}, we inductively define the set of views that are consistent with respect to a history heap $\hh$. 
The definition below models the fact that, if we start from a causally consistent view, and we wish to update the view for some location $\txid_2$, 

\begin{definition}[Transactional causal dependency]
\label{def:causal}
Given two versions $\ver_{1} = (\val_1, \txid_1, \txidset_1)$, $\ver_2 = (\val_2, \txid_2, \txidset_2)$, $\ver_{1}$ \emph{directly depends on} $\ver_2$, written $\ver_1 \xrightarrow{\ddep} \ver_2$, iff:
\[
\begin{rclarray}
    (\val_1, \txid_1, \txidset_1) \xrightarrow{\ddep} (\val_2, \txid_2, \txidset_2) & \defeq & \txid_1 \in \txidset_2 \lor  \exsts{\txid, \txid'} \txid \in \Set{\txid_1} \cup \txidset_1 \land \txid' \in \Set{\txid_2} \cup \txidset_2 \land \txid \leq \txid'
\end{rclarray}
\]
%\ac{Note to self: the notion of directly depends here has little to do with dependencies 
%between transactions. $\nu_1 \xrightarrow{\mathsf{ddep}} \nu_2$ means that 
%some transaction $\tsid$ touches both versions. However, it reads $\nu_2$ and 
%writes $\nu_1$.}
Given $\hh \in \HisHeaps$, a view \( \vi \) are \emph{causally consistent} with the key-value store $\hh$, $\pred{CCViews}{\hh, \vi}$: 
\[
\begin{rclarray}
    \pred{CCViews}{\hh, \vi} & \defeq & \fora{\ke, \ke', i, j} i \leq \vi(\ke) \land \mkvs(\ke, i) \xrightarrow{\ddep}^{*} \mkvs(\ke, j) \implies j \leq \vi(\ke')
\end{rclarray}
\]
%\begin{itemize} 
%\item the initial view \( \vi_0\)  is in the set, \ie $\vi_0 \in \func{CCViews}{\hh}$ where \( \fora{\ke \in \dom(\hh)} \vi_{0}(\ke) = 1 \),
%\item assume any view $\vi \in \pred{CCViews}{\hh}$ and a new view \( \vi' \) by observing one more version for an address $\vi' = \vi\rmto{\ke}{\vi(\ke) + 1}$, where \( \vi'(\ke) < \left| \hh(\ke) \right| \).
%If some versions directly depend on the version corresponding to \( \vi'(\ke)\) and those versions are aware by \( \vi'\), the new view is included in \( \func{CCViews}{\hh}\),
%\[
%\begin{array}{@{}l}
%\fora{\vi,\vi'} \exsts{\ke}
%\vi \in \func{CCViews}{\hh} 
%\land \ke \in \dom(\vi)
%\land \vi' = \vi\rmto{\ke}{\vi(\ke) + 1}
%\land \vi'(\ke) < \left| \hh(\ke) \right|  \\
%\quad {} \land 
%\begin{B}
%\fora{\ke', i}  
%0 \leq i < \left| \hh(\ke') \right|
%\land \pred{ddep}{\hh(\ke')(i), \hh(\ke)(\vi'(\ke))}
%\implies i \leq \vi(\ke')
%\end{B} \\
%\qquad {} \implies \vi' \in \func{CCViews}{\hh}
%\end{array}
%\]
%\end{itemize}
Causal consistency is stronger than read atomic (\cref{def:readatomic}) by further ensuring that a transaction can only observe a causally consistent state of the database, i.e. the view when the transaction starts is causally consistent with the history heap,
\[
\begin{rclarray}
    (\hh, \vi) \csatCC \opset : \vi' & \defeq & (\hh, \vi) \csatRA \opset : \vi' \land \pred{CCViews}{\hh,\vi}
\end{rclarray}
\]
\end{definition}

%\begin{figure}
%\centering
%\hrule\vspace{5pt}
%\begin{tabular}{@{} c @{} c @{}} 


%\begin{halfsubfig}
%\begin{centertikz}

%\begin{pgfonlayer}{foreground}
%%Uncomment line below for help lines
%%\draw[help lines] grid(5,4);

%%Location x
%\node(locx) {$\ke_{\vx} \mapsto$};

%\matrix(versionx) [version list]
   %at ([xshift=\tikzkvspace]locx.east) {
 %{a} & $\tsid_0$ & {a} & $\tsid_1$\\
  %{a} & $\emptyset$ & {a} & $\Set{\tsid_2}$ \\
%};
%\tikzvalue{versionx-1-1}{versionx-2-1}{locx-v0}{\stub};
%\tikzvalue{versionx-1-3}{versionx-2-3}{locx-v1}{\stub};

%%Location y
%\path (locx.south) + (0,\tikzkeyspace) node (locy) {$\ke_{\vy} \mapsto$};
%\matrix(versiony) [version list]
    %at ([xshift=\tikzkvspace]locy.east) {
    %{a} & $\tsid_0$ & {a} & $\tsid_2$ \\
    %{a} & $\emptyset$ & {a} & $\Set{\tsid_3}$\\
%};

%\tikzvalue{versiony-1-1}{versiony-2-1}{locy-v0}{\stub};
%\tikzvalue{versiony-1-3}{versiony-2-3}{locy-v1}{\stub};

%%Location z

%\path (locy.south) + (0,\tikzkeyspace) node (locz) {$\ke_\vz \mapsto$};
%\matrix(versionz) [version list]
    %at ([xshift=\tikzkvspace]locz.east) {
    %{a} & $\tsid_0$ & {a} & $\tsid_3$ \\
    %{a} & $\emptyset$ & {a} & $\emptyset$\\
%};

%\tikzvalue{versionz-1-1}{versionz-2-1}{locz-v0}{\stub};
%\tikzvalue{versionz-1-3}{versionz-2-3}{locz-v1}{\stub};

%%blue view - I should  check whether I can use pgfkeys to just declare the list of locations, and then add the view automatically.
%\draw[-, blue, very thick, rounded corners=10pt]
%([xshift=-5pt, yshift=5pt]locx-v0.north east) node (tid1start) {} -- 
%([xshift=-5pt, yshift=-5pt]locz-v0.south east);
 
%% \path (tid1start) node[anchor=south, rectangle, fill=blue!20, draw=blue, font=\small, inner sep=1pt] {$\thid_3$};
%\end{pgfonlayer}
%\end{centertikz}
%\caption{}
%\label{fig:cc-view-a}
%\end{halfsubfig}
%&
%\begin{halfsubfig}
%\begin{centertikz}

%\begin{pgfonlayer}{foreground}
%%Uncomment line below for help lines
%%\draw[help lines] grid(5,4);

%%Location x
%\node(locx) {$\ke_{\vx} \mapsto$};

%\matrix(versionx) [version list]
   %at ([xshift=\tikzkvspace]locx.east) {
 %{a} & $\tsid_0$ & {a} & $\tsid_1$\\
  %{a} & $\emptyset$ & {a} & $\Set{\tsid_2}$ \\
%};
%\tikzvalue{versionx-1-1}{versionx-2-1}{locx-v0}{\stub};
%\tikzvalue{versionx-1-3}{versionx-2-3}{locx-v1}{\stub};

%%Location y
%\path (locx.south) + (0,\tikzkeyspace) node (locy) {$\ke_{\vy} \mapsto$};
%\matrix(versiony) [version list]
    %at ([xshift=\tikzkvspace]locy.east) {
    %{a} & $\tsid_0$ & {a} & $\tsid_2$ \\
    %{a} & $\emptyset$ & {a} & $\Set{\tsid_3}$\\
%};

%\tikzvalue{versiony-1-1}{versiony-2-1}{locy-v0}{\stub};
%\tikzvalue{versiony-1-3}{versiony-2-3}{locy-v1}{\stub};

%%Location z

%\path (locy.south) + (0,\tikzkeyspace) node (locz) {$\ke_\vz \mapsto$};
%\matrix(versionz) [version list]
    %at ([xshift=\tikzkvspace]locz.east) {
    %{a} & $\tsid_0$ & {a} & $\tsid_3$ \\
    %{a} & $\emptyset$ & {a} & $\emptyset$\\
%};

%\tikzvalue{versionz-1-1}{versionz-2-1}{locz-v0}{\stub};
%\tikzvalue{versionz-1-3}{versionz-2-3}{locz-v1}{\stub};

%%blue view - I should  check whether I can use pgfkeys to just declare the list of locations, and then add the view automatically.
%\draw[-, blue, very thick, rounded corners=10pt]
%([xshift=-5pt, yshift=5pt]locx-v1.north east) node (tid1start) {} -- 
%([xshift=-5pt, yshift=-5pt]locx-v1.south east) --
%([xshift=-5pt, yshift=5pt]locy-v0.north east) -- 
%([xshift=-5pt, yshift=-5pt]locz-v0.south east);
 
%% \path (tid1start) node[anchor=south, rectangle, fill=blue!20, draw=blue, font=\small, inner sep=1pt] {$\thid_3$};
%\end{pgfonlayer}
%\end{centertikz}
%\caption{}
%\label{fig:cc-view-b}
%\end{halfsubfig}
%\\
%\begin{halfsubfig}
%\begin{centertikz}

%\begin{pgfonlayer}{foreground}
%%Uncomment line below for help lines
%%\draw[help lines] grid(5,4);

%%Location x
%\node(locx) {$\ke_{\vx} \mapsto$};

%\matrix(versionx) [version list]
   %at ([xshift=\tikzkvspace]locx.east) {
 %{a} & $\tsid_0$ & {a} & $\tsid_1$\\
  %{a} & $\emptyset$ & {a} & $\Set{\tsid_2}$ \\
%};
%\tikzvalue{versionx-1-1}{versionx-2-1}{locx-v0}{\stub};
%\tikzvalue{versionx-1-3}{versionx-2-3}{locx-v1}{\stub};

%%Location y
%\path (locx.south) + (0,\tikzkeyspace) node (locy) {$\ke_{\vy} \mapsto$};
%\matrix(versiony) [version list]
    %at ([xshift=\tikzkvspace]locy.east) {
    %{a} & $\tsid_0$ & {a} & $\tsid_2$ \\
    %{a} & $\emptyset$ & {a} & $\Set{\tsid_3}$\\
%};

%\tikzvalue{versiony-1-1}{versiony-2-1}{locy-v0}{\stub};
%\tikzvalue{versiony-1-3}{versiony-2-3}{locy-v1}{\stub};

%%Location z

%\path (locy.south) + (0,\tikzkeyspace) node (locz) {$\ke_\vz \mapsto$};
%\matrix(versionz) [version list]
    %at ([xshift=\tikzkvspace]locz.east) {
    %{a} & $\tsid_0$ & {a} & $\tsid_3$ \\
    %{a} & $\emptyset$ & {a} & $\emptyset$\\
%};

%\tikzvalue{versionz-1-1}{versionz-2-1}{locz-v0}{\stub};
%\tikzvalue{versionz-1-3}{versionz-2-3}{locz-v1}{\stub};

%%blue view - I should  check whether I can use pgfkeys to just declare the list of locations, and then add the view automatically.
%\draw[-, blue, very thick, rounded corners=10pt]
%([xshift=-5pt, yshift=5pt]locx-v1.north east) node (tid1start) {} -- 
%([xshift=-5pt, yshift=-5pt]locy-v1.south east) --
%([xshift=-5pt, yshift=5pt]locz-v0.north east) -- 
%([xshift=-5pt, yshift=-5pt]locz-v0.south east);
 
%% \path (tid1start) node[anchor=south, rectangle, fill=blue!20, draw=blue, font=\small, inner sep=1pt] {$\thid_3$};
%\end{pgfonlayer}
%\end{centertikz}
%\caption{}
%\label{fig:cc-view-c}
%\end{halfsubfig}
%&
%\begin{halfsubfig}
%\begin{centertikz}

%\begin{pgfonlayer}{foreground}
%%Uncomment line below for help lines
%%\draw[help lines] grid(5,4);

%%Location x
%\node(locx) {$\ke_{\vx} \mapsto$};

%\matrix(versionx) [version list]
   %at ([xshift=\tikzkvspace]locx.east) {
 %{a} & $\tsid_0$ & {a} & $\tsid_1$\\
  %{a} & $\emptyset$ & {a} & $\Set{\tsid_2}$ \\
%};
%\tikzvalue{versionx-1-1}{versionx-2-1}{locx-v0}{\stub};
%\tikzvalue{versionx-1-3}{versionx-2-3}{locx-v1}{\stub};

%%Location y
%\path (locx.south) + (0,\tikzkeyspace) node (locy) {$\ke_{\vy} \mapsto$};
%\matrix(versiony) [version list]
    %at ([xshift=\tikzkvspace]locy.east) {
    %{a} & $\tsid_0$ & {a} & $\tsid_2$ \\
    %{a} & $\emptyset$ & {a} & $\Set{\tsid_3}$\\
%};

%\tikzvalue{versiony-1-1}{versiony-2-1}{locy-v0}{\stub};
%\tikzvalue{versiony-1-3}{versiony-2-3}{locy-v1}{\stub};

%%Location z

%\path (locy.south) + (0,\tikzkeyspace) node (locz) {$\ke_\vz \mapsto$};
%\matrix(versionz) [version list]
    %at ([xshift=\tikzkvspace]locz.east) {
    %{a} & $\tsid_0$ & {a} & $\tsid_3$ \\
    %{a} & $\emptyset$ & {a} & $\emptyset$\\
%};

%\tikzvalue{versionz-1-1}{versionz-2-1}{locz-v0}{\stub};
%\tikzvalue{versionz-1-3}{versionz-2-3}{locz-v1}{\stub};

%%blue view - I should  check whether I can use pgfkeys to just declare the list of locations, and then add the view automatically.
%\draw[-, blue, very thick, rounded corners=10pt]
%([xshift=-5pt, yshift=5pt]locx-v1.north east) node (tid1start) {} -- 
%([xshift=-5pt, yshift=-5pt]locz-v1.south east);
 
%% \path (tid1start) node[anchor=south, rectangle, fill=blue!20, draw=blue, font=\small, inner sep=1pt] {$\thid_3$};
%\end{pgfonlayer}
%\end{centertikz}
%\caption{}
%\label{fig:cc-view-d}
%\end{halfsubfig}
%\end{tabular}
%\hrule
%\caption{Building a causally consistent view.}
%\label{fig:cc.view}
%\label{fig:cc-view}
%\end{figure}

%\ac{Note to self: I got this example and the definition wrong several times before getting them
%right. Which means that inductive definition of causally dependent views is not really that 
%intuitive after all...}
%Let consider the history $\hh$ in \cref{fig:cc-view-a}, where $\pred{ddep}{\hh(\ke_\vz)(1),\hh(\ke_{\vy})(1)}$, and $\pred{ddep}{\hh(\ke_{\vy})(1) ,\hh(\ke_{\vx})(1)}$.
%Since the values are irrelevant, we ignore the values.
%We want to find a view $\vi$ that is causal consistent and is up-to-date with the last version of key $[\ke_{\vz}]$, \ie \( \vi(\ke_\vz) = 1\).
%We can construct such a view incrementally by the definition of \( \funcn{CCViews} \) function.

%We start from the initial view $\vi_0$ pointing to initial versions of all keys (\cref{fig:cc-view-a}).
%This view is causally consistent by definition.
%As a first try, one could immediately consider a view $\vi'= \vi_0\rmto{\ke_\vz}{1}$ where the index of version for key $\ke_\vz$ is updated to $1$, but this view is not causally consistent due to \( \pred{ddep}{\hh(\ke_\vz)(1),\hh(\ke_{\vy})(1)} \) and \( \vi'(\ke_\vy) = 0\).
%That is, the version of $\ke_\vz$ observed by $\vi'$ directly depends on the second version of $\ke_\vy$ that is not observed by $\vi'$.
%Similarly one cannot advance the view to observe the latest version of \( \ke_\vy\) without knowing the latest version of \( \ke_\vx\).
%Therefore we can only update the view $V_0$ by including the second version of $\ke_{\vx}$, \ie the new view \(\vi_1 = \vi_0\rmto{\ke_\vx}{1} \) (\cref{fig:cc-view-b}).
%The new view \( \vi_1\) is causally consistent as no version directly depend on the second version of \( \ke_\vx\).
%We can now update the view $\vi_1$ to include the latest version of $\ke_{\vy}$, resulting in the view $\vi_2 = \vi_1\rmto{\ke_{\vy}}{1}$ (\cref{fig:cc-view-c}). 
%The view \( \vi_2\) is causally consistent as the second version of \( \ke_\vx\) is already included in the view.
%Finally, the view includes the latest version of \( \ke_\vz \) (\cref{fig:cc-view-d}).
%The view \( \vi_3\) is causally consistent as the second version of \( \ke_\vy\) is already included in the view.

%\ac{Note to self: here there is something subtle going on. We also need to ensure that dependencies 
%caused by the information flow of the program are tracked down. For example, we could have a 
%transaction returning the value of a location $\ke_{\vx}$, and then another transaction copy such a value 
%into another location $\ke_{\vy}$. There is a notion of dependency between $\ke_{\vx}, \ke_{\vy}$ that 
%is not captured by the notion of \emph{directly depends}. On the other hand, the fact that 
%stacks are client local, and we have per-client view monotonicity, should ensure that also program 
%dependencies are preserved. A definitive proof that $\mathsf{CC}$ is equivalent to caual consistency 
%specified either in terms of abstract executions or dependency graphs, would settle the argument.}

Note that, the view of $\thid_3$ in \cref{fig:cc-exec-e} is not causally consistent.
Because $V(\ke_{\vy}) = 1$, and $\vi(\ke_{\vx}) = 0$. 
However, $\hh(\ke_{\vy},1)$ directly depends on $\hh(\ke_{\vx},1)$, which is not included in the view. 
In general, the only case in which an execution of $\prog_2$ causes transaction $\txid_3$ to return value $\sadface$ is when such a transaction is executed using a snapshot determined by a non-causally consistent view. 
There exists no execution of $\prog_2$ under $\mathsf{CC}$ that causes $\txid_3$ to return value $\sadface$.
%if we execute the program $\prog_2$, illustrated previously in this section, under $\mathsf{CCViews}$,
%it is not possible any more to have the transaction $\ptrans{\trans_3}$ return value ${\sadface}$. 
%This is because, in order for $\ptrans{\trans_3}$ to return such a value, it must be executed in a 
%configuration such as the one of Figure \ref{fig:cc.view}(e) (only the view of $\thid_3$ is relevant here), 
%%state where the view client of $\thid_3$ observes the update to location $\ke_{\vy}$, but not the 
%%update of $\ke_{\vx}$ from which the latter causally depends. 

%\ac{In theory, I can do better, and require that I see only the causal dependencies 
%of what I read. But at the end of the day, who cares?}

%\begin{definition}
%Let $\hh, V$ and $\ke_{\vx}$ be a history heap, a view, and a location, respectively. 
%Given two transactions $\tsid_1, \tsid_2$, we say that $\tsid_2$ write-read depends 
%on $\tsid_1$ via $\ke_{\vx}$, written $\tsid_1 \xrightarrow{\RF(\ke_{\vx})_{\hh}} \tsid_2$, 
%if there exists an index $i = 0,\cdots, \llvert \hh, \rvert -1$ such that $\hh([\loc_n])(x) = 
%(\_, \tsid_1, \mathcal{T})$, and $\tsid_2 \in \mathcal{T}$.
%\end{definition}
%
%\begin{definition}
%Let $\hh$ be a history heap, and $V$ be a view.
%Let $\ke_{\vx}$ be a location, and let 
%$(\_, \tsid, \_) := \hh([\loc_{n}])(V([\loc_{x}]))$. 
%We say that $V$ respects causality for $[\loc_{x}]$ if, 
%whenever $\tsid' \xrightarrow{\RF([\loc_{x}])}_{\tsid}'$, 
%$\hh(\ke_{\vx})(i)$ as follows: 
%\begin{enumerate}
%\end{definition}

%\subsection{Update Atomic \( \UA \)}
\label{sec:sound-complete-ua}

Given abstract execution \( \aexec \), we define write-write relation for a key \( \ke \) as the following:
\[ 
    \WW(\aexec,\ke) \defeq \Setcon{(\txid, \txid')}{\txid \toEdge{\AR_\aexec} \txid' \land (\otW,\ke, \stub ) \in \txid \land (\otW,\ke, \stub ) \in \txid'  } 
\]
Then, the notation \( \WW_\aexec \defeq \bigcup_{\ke \in \Keys} \WW(\aexec, \ke) \).
Note that for a kv-store \( \mkvs \) such that \( \mkvs = \mkvs_\aexec \),
by the definition of  \(  \mkvs = \mkvs_\aexec \), 
the following holds:
\[
    \WW_\aexec = \Setcon{(\txid, \txid')}{\exsts{\ke, i,j } \txid = \WTx(\mkvs(\ke, i)) \land \txid' = \WTx(\mkvs(\ke, j)) \land i < j}
\]
Also the \( \WW_\aexec \) coincides with \( \WW_\Gr \) and \( \WW_\mkvs \).

The execution test $\ET_\UA$ is sound with respect to the axiomatic specification \( (\RP_{\LWW}, \Set{\lambda \aexec. \WW_\aexec }) \).
We pick the invariant as \( I( \aexec, \cl ) = \emptyset \), given the fact of no constraint on the final view.
Assume a kv-store $\hh$, an initial and a final view $\vi, \vi'$  a fingerprint $\opset$ 
such that $\ET_{\UA} \vdash (\hh, \vi) \csat \opset: (\hh', \vi')$. 
Also choose an arbitrary $\cl$, a transaction identifier $\txid \in \nextTxId(\hh, \cl)$, 
and an abstract execution $\aexec$ such that $\hh_{\aexec} = \hh$ and 
\( I(\aexec, \cl) =  \emptyset \subseteq \Tx(\hh, \vi) \).
Let \( \aexec' = \extend(\aexec, \txid, \Tx(\mkvs, \vi), \f ) \).
Note that since the invariant is empty set, it remains to prove the following:
\[
    \begin{array}{@{}l@{}}
        \fora{ \txid' } \txid' \toEdge{\WW_{\aexec'}} \txid \implies \txid' \in \Tx(\mkvs, \vi)
    \end{array}
\]
Assume a transaction \( \txid' \) that writes to a key \( \ke \) as \( \txid \), \ie \( \txid' \toEdge{\WW_{\aexec'}} \txid \).
Since that \( \txid' \) is a transaction already existing in \( \mkvs\),
we have \( \WTx(\mkvs(\ke, i)) = \txid' \) for some index \( i \).
By the execution test of \( \UA \), we know \( i \in \vi(\ke) \) therefore \( \txid' \in \Tx(\mkvs, \vi) \).

The execution test $\ET_\UA$ is complete with respect to 
the axiomatic specification \( (\RP_{\LWW}, \Set{\lambda \aexec. \WW_\aexec }) \).
Assume i-\emph{th} transaction \( \txid_i \) in the arbitrary order,
and let view \( \vi_{i} = \getView(\aexec, \VIS^{-1}_{\aexec}(\txid_{i}) ) \).
We also pick any final view such that \( \vi'_{i} \subseteq \getView(\aexec, (\AR^{-1}_{\aexec})?(\txid_{i}) ) \).
Note that there is nothing to prove for \( \vi'_i \),
so it is sufficient to prove the following:
\[
    \fora{\ke} (\otW, \ke, \stub) \in \TtoOp{T}_{\aexec}(\txid_{i}) 
    \implies 
    \fora{j : 0 \leq j < \abs{\mkvs_{\cut(\aexec, i-1)}(\ke)}} j \in \vi_i(\ke)
\]
Let consider a key \( \ke \) that have been overwritten by the transaction \( \txid_i \).
By the constraint of \( \aexec \) that \( \WW_\aexec \subseteq \VIS_\aexec \),
for any transaction \( \txid \) that writes to the same key \( \ke \) and committed before \( \txid_i \), 
they are included in the visible set \(\txid \in \VIS^{-1}_{\aexec}(\txid_{i}) \).
Note that \( \txid \toEdge{\WW_\aexec} \txid_i \implies \txid \toEdge{\AR_\aexec} \txid_i \implies \txid \in \mkvs_{\cut(\aexec,i-1)}\).
Since that the transaction \( \txid \) write to the key \( \ke \),
it means \( \WTx(\mkvs_{\cut(\aexec, i-1)}(\ke,j)) = \txid \) for some index \( j \).
Then by the definition of \( \getView \), we have \( j \in \vi_i(\ke)\).

%\subsection{Consistent Prefix} 
%\begin{figure}
%\begin{center}
%\begin{tabular}{|@{}c|c@{}|}
%\hline
%\begin{tikzpicture}[font=\large]

%\begin{pgfonlayer}{foreground}
%%Uncomment line below for help lines
%%\draw[help lines] grid(5,4);

%%Location x
%\node(locx) at (1,3) {$[\loc_x] \mapsto$};

%\matrix(locxcells) [version list, text width=7mm, anchor=west]
   %at ([xshift=10pt]locx.east) {
 %{a} & $T_0$ \\
  %{a} & $\emptyset$ \\
%};
%\node[version node, fit=(locxcells-1-1) (locxcells-2-1), fill=white, inner sep= 0cm, font=\Large] (locx-v0) {$0$};
%%\node[version node, fit=(locxcells-1-3) (locxcells-2-3), fill=white, inner sep=0cm, font=\Large] (locx-v1) {$1$};

%%Location y
%\path (locx.south) + (0,-1.5) node (locy) {$[\loc_y] \mapsto$};
%\matrix(locycells) [version list, text width=7mm, anchor=west]
   %at ([xshift=10pt]locy.east) {
 %{a} & $T_0$ \\
  %{a} & $\emptyset$ \\
%};
%\node[version node, fit=(locycells-1-1) (locycells-2-1), fill=white, inner sep= 0cm, font=\Large] (locy-v0) {$0$};
%%\node[version node, fit=(locycells-1-3) (locycells-2-3), fill=white, inner sep=0cm, font=\Large] (locy-v1) {$1$};

%% \draw[-, red, very thick, rounded corners] ([xshift=-5pt, yshift=5pt]locx-v1.north east) |- 
%%  ($([xshift=-5pt,yshift=-5pt]locx-v1.south east)!.5!([xshift=-5pt, yshift=5pt]locy-v0.north east)$) -| ([xshift=-5pt, yshift=5pt]locy-v0.south east);

%%blue view - I should  check whether I can use pgfkeys to just declare the list of locations, and then add the view automatically.
%\draw[-, blue, very thick, rounded corners=10pt]
 %([xshift=-3pt, yshift=20pt]locx-v0.north east) node (tid1start) {} -- 
%% ([xshift=-2pt, yshift=-5pt]locx-v0.south east) --
%% ([xshift=-2pt, yshift=5pt]locy-v0.north east) -- 
 %([xshift=-3pt, yshift=-5pt]locy-v0.south east);
 
 %\path (tid1start) node[anchor=south, rectangle, fill=blue!20, draw=blue, font=\small, inner sep=1pt] {$\thid_1$};

%%red view
%\draw[-, red, very thick, rounded corners = 10pt]
 %([xshift=-16pt, yshift=5pt]locx-v0.north east) node (tid2start) {}-- 
%% ([xshift=-8pt, yshift=-5pt]locx-v0.south east) --
%% ([xshift=-8pt, yshift=5pt]locy-v0.north east) -- 
 %([xshift=-16pt, yshift=-5pt]locy-v0.south east) node {};
 
%\path (tid2start) node[anchor=south, rectangle, fill=red!20, draw=red, font=\small, inner sep=1pt] {$\thid_2$};

%%%Stack for threads tid_1 and tid_2
%%
%%\draw[-, dashed] let 
%%   \p1 = ([xshift=0pt]locy.west),
%%   \p2 = ([yshift=-5pt]locycells.south),
%%   \p3 = ([xshift=10pt]locycells.east) in
%%   (\x1, \y2) -- (\x3, \y2);
%%   
%%\matrix(stacks) [
%%   matrix of nodes,
%%   anchor=north, 
%%   text=blue, 
%%   font=\normalsize, 
%%   row 1/.style = {text = blue}, 
%%   row 2/.style = {text = red}, 
%%   text width= 13mm ] 
%%   at ([xshift=-10pt,yshift=-8pt]locycells.south) {
%%   $\thid_1:$ & $\retvar = 0$\\
%%   $\thid_2:$ & $\retvar = 0$\\
%%   };
%\end{pgfonlayer}
%\end{tikzpicture} 
%%
%&
%%
%\begin{tikzpicture}[font=\large]

%\begin{pgfonlayer}{foreground}
%%Uncomment line below for help lines
%%\draw[help lines] grid(5,4);

%%Location x
%\node(locx) at (1,3) {$[\loc_x] \mapsto$};

%\matrix(locxcells) [version list, text width=7mm, anchor=west]
   %at ([xshift=10pt]locx.east) {
 %{a} & $T_0$ & {a} & $\tsid_1$\\
  %{a} & $\emptyset$ & {a} & $\emptyset$ \\
%};
%\node[version node, fit=(locxcells-1-1) (locxcells-2-1), fill=white, inner sep= 0cm, font=\Large] (locx-v0) {$0$};
%\node[version node, fit=(locxcells-1-3) (locxcells-2-3), fill=white, inner sep=0cm, font=\Large] (locx-v1) {$1$};

%%Location y
%\path (locx.south) + (0,-1.5) node (locy) {$[\loc_y] \mapsto$};
%\matrix(locycells) [version list, text width=7mm, anchor=west]
   %at ([xshift=10pt]locy.east) {
 %{a} & $T_0$ \\
  %{a} & $\emptyset$ \\
%};
%\node[version node, fit=(locycells-1-1) (locycells-2-1), fill=white, inner sep= 0cm, font=\Large] (locy-v0) {$0$};
%%\node[version node, fit=(locycells-1-3) (locycells-2-3), fill=white, inner sep=0cm, font=\Large] (locy-v1) {$1$};

%% \draw[-, red, very thick, rounded corners] ([xshift=-5pt, yshift=5pt]locx-v1.north east) |- 
%%  ($([xshift=-5pt,yshift=-5pt]locx-v1.south east)!.5!([xshift=-5pt, yshift=5pt]locy-v0.north east)$) -| ([xshift=-5pt, yshift=5pt]locy-v0.south east);

%%blue view - I should  check whether I can use pgfkeys to just declare the list of locations, and then add the view automatically.
%\draw[-, blue, very thick, rounded corners=10pt]
 %([xshift=-3pt, yshift=20pt]locx-v1.north east) node (tid1start) {} -- 
 %([xshift=-3pt, yshift=-5pt]locx-v1.south east) --
 %([xshift=-3pt, yshift=5pt]locy-v0.north east) -- 
 %([xshift=-3pt, yshift=-5pt]locy-v0.south east);
 
 %\path (tid1start) node[anchor=south, rectangle, fill=blue!20, draw=blue, font=\small, inner sep=1pt] {$\thid_1$};

%%red view
%\draw[-, red, very thick, rounded corners = 10pt]
 %([xshift=-16pt, yshift=5pt]locx-v0.north east) node (tid2start) {}-- 
%% ([xshift=-8pt, yshift=-5pt]locx-v0.south east) --
%% ([xshift=-8pt, yshift=5pt]locy-v0.north east) -- 
 %([xshift=-16pt, yshift=-5pt]locy-v0.south east) node {};
 
%\path (tid2start) node[anchor=south, rectangle, fill=red!20, draw=red, font=\small, inner sep=1pt] {$\thid_2$};

%%%Stack for threads tid_1 and tid_2
%%
%%\draw[-, dashed] let 
%%   \p1 = ([xshift=0pt]locy.west),
%%   \p2 = ([yshift=-5pt]locycells.south),
%%   \p3 = ([xshift=10pt]locycells.east) in
%%   (\x1, \y2) -- (\x3, \y2);
%%   
%%\matrix(stacks) [
%%   matrix of nodes,
%%   anchor=north, 
%%   text=blue, 
%%   font=\normalsize, 
%%   row 1/.style = {text = blue}, 
%%   row 2/.style = {text = red}, 
%%   text width= 13mm ] 
%%   at ([xshift=-10pt,yshift=-8pt]locycells.south) {
%%   $\thid_1:$ & $\retvar = 0$\\
%%   $\thid_2:$ & $\retvar = 0$\\
%%   };
%\end{pgfonlayer}
%\end{tikzpicture}
%\\
%{\small(a)} & {\small(b)}\\
%\hline

%\begin{pgfonlayer}{foreground}
%%Uncomment line below for help lines
%%\draw[help lines] grid(5,4);

%%Location x
%\node(locx) at (1,3) {$[\loc_x] \mapsto$};

%\matrix(locxcells) [version list, text width=7mm, anchor=west]
   %at ([xshift=10pt]locx.east) {
 %{a} & $T_0$ & {a} & $\tsid_1$\\
  %{a} & $\emptyset$ & {a} & $\emptyset$ \\
%};
%\node[version node, fit=(locxcells-1-1) (locxcells-2-1), fill=white, inner sep= 0cm, font=\Large] (locx-v0) {$0$};
%\node[version node, fit=(locxcells-1-3) (locxcells-2-3), fill=white, inner sep=0cm, font=\Large] (locx-v1) {$1$};

%%Location y
%\path (locx.south) + (0,-1.5) node (locy) {$[\loc_y] \mapsto$};
%\matrix(locycells) [version list, text width=7mm, anchor=west]
   %at ([xshift=10pt]locy.east) {
 %{a} & $T_0$ & {a} & $\tsid_2$ \\
  %{a} & $\emptyset$ & {a} & $\emptyset$\\
%};
%\node[version node, fit=(locycells-1-1) (locycells-2-1), fill=white, inner sep= 0cm, font=\Large] (locy-v0) {$0$};
%\node[version node, fit=(locycells-1-3) (locycells-2-3), fill=white, inner sep=0cm, font=\Large] (locy-v1) {$1$};

%% \draw[-, red, very thick, rounded corners] ([xshift=-5pt, yshift=5pt]locx-v1.north east) |- 
%%  ($([xshift=-5pt,yshift=-5pt]locx-v1.south east)!.5!([xshift=-5pt, yshift=5pt]locy-v0.north east)$) -| ([xshift=-5pt, yshift=5pt]locy-v0.south east);

%%blue view - I should  check whether I can use pgfkeys to just declare the list of locations, and then add the view automatically.
%\draw[-, blue, very thick, rounded corners=10pt]
 %([xshift=-3pt, yshift=20pt]locx-v1.north east) node (tid1start) {} -- 
 %([xshift=-3pt, yshift=-5pt]locx-v1.south east) --
 %([xshift=-3pt, yshift=5pt]locy-v0.north east) -- 
 %([xshift=-3pt, yshift=-5pt]locy-v0.south east);
 
 %\path (tid1start) node[anchor=south, rectangle, fill=blue!20, draw=blue, font=\small, inner sep=1pt] {$\thid_1$};

%%red view
%\draw[-, red, very thick, rounded corners = 10pt]
 %([xshift=-16pt, yshift=5pt]locx-v0.north east) node (tid2start) {}-- 
 %([xshift=-16pt, yshift=-5pt]locx-v0.south east) --
 %([xshift=-16pt, yshift=5pt]locy-v1.north east) -- 
 %([xshift=-16pt, yshift=-5pt]locy-v1.south east) node {};
 
%\path (tid2start) node[anchor=south, rectangle, fill=red!20, draw=red, font=\small, inner sep=1pt] {$\thid_2$};

%%%Stack for threads tid_1 and tid_2
%%
%%\draw[-, dashed] let 
%%   \p1 = ([xshift=0pt]locy.west),
%%   \p2 = ([yshift=-5pt]locycells.south),
%%   \p3 = ([xshift=10pt]locycells.east) in
%%   (\x1, \y2) -- (\x3, \y2);
%%   
%%\matrix(stacks) [
%%   matrix of nodes,
%%   anchor=north, 
%%   text=blue, 
%%   font=\normalsize, 
%%   row 1/.style = {text = blue}, 
%%   row 2/.style = {text = red}, 
%%   text width= 13mm ] 
%%   at ([xshift=-10pt,yshift=-8pt]locycells.south) {
%%   $\thid_1:$ & $\retvar = 0$\\
%%   $\thid_2:$ & $\retvar = 0$\\
%%   };
%\end{pgfonlayer}
%\end{tikzpicture}
%%
%&
%%
%\begin{tikzpicture}[font=\large]

%\begin{pgfonlayer}{foreground}
%%Uncomment line below for help lines
%%\draw[help lines] grid(5,4);

%%Location x
%\node(locx) at (1,3) {$[\loc_x] \mapsto$};

%\matrix(locxcells) [version list, text width=7mm, anchor=west]
   %at ([xshift=10pt]locx.east) {
 %{a} & $T_0$ & {a} & $\tsid_1$\\
  %{a} & $\{\tsid_4\}$ & {a} & $\emptyset$ \\
%};
%\node[version node, fit=(locxcells-1-1) (locxcells-2-1), fill=white, inner sep= 0cm, font=\Large] (locx-v0) {$0$};
%\node[version node, fit=(locxcells-1-3) (locxcells-2-3), fill=white, inner sep=0cm, font=\Large] (locx-v1) {$1$};

%%Location y
%\path (locx.south) + (0,-1.5) node (locy) {$[\loc_y] \mapsto$};
%\matrix(locycells) [version list, text width=7mm, anchor=west]
   %at ([xshift=10pt]locy.east) {
 %{a} & $T_0$ & {a} & $\tsid_2$ \\
  %{a} & $\{\tsid_3\}$ & {a} & $\emptyset$\\
%};
%\node[version node, fit=(locycells-1-1) (locycells-2-1), fill=white, inner sep= 0cm, font=\Large] (locy-v0) {$0$};
%\node[version node, fit=(locycells-1-3) (locycells-2-3), fill=white, inner sep=0cm, font=\Large] (locy-v1) {$1$};

%% \draw[-, red, very thick, rounded corners] ([xshift=-5pt, yshift=5pt]locx-v1.north east) |- 
%%  ($([xshift=-5pt,yshift=-5pt]locx-v1.south east)!.5!([xshift=-5pt, yshift=5pt]locy-v0.north east)$) -| ([xshift=-5pt, yshift=5pt]locy-v0.south east);

%%blue view - I should  check whether I can use pgfkeys to just declare the list of locations, and then add the view automatically.
%\draw[-, blue, very thick, rounded corners=10pt]
 %([xshift=-3pt, yshift=20pt]locx-v1.north east) node (tid1start) {} -- 
%% ([xshift=-3pt, yshift=-5pt]locx-v1.south east) --
%% ([xshift=-3pt, yshift=5pt]locy-v0.north east) -- 
 %([xshift=-3pt, yshift=-5pt]locy-v1.south east);
 
 %\path (tid1start) node[anchor=south, rectangle, fill=blue!20, draw=blue, font=\small, inner sep=1pt] {$\thid_1$};

%%red view
%\draw[-, red, very thick, rounded corners = 10pt]
 %([xshift=-16pt, yshift=5pt]locx-v1.north east) node (tid2start) {}-- 
%% ([xshift=-16pt, yshift=-5pt]locx-v0.south east) --
%% ([xshift=-16pt, yshift=5pt]locy-v1.north east) -- 
 %([xshift=-16pt, yshift=-5pt]locy-v1.south east) node {};
 
%\path (tid2start) node[anchor=south, rectangle, fill=red!20, draw=red, font=\small, inner sep=1pt] {$\thid_2$};

%%%Stack for threads tid_1 and tid_2
%%
%%\draw[-, dashed] let 
%%   \p1 = ([xshift=0pt]locy.west),
%%   \p2 = ([yshift=-5pt]locycells.south),
%%   \p3 = ([xshift=10pt]locycells.east) in
%%   (\x1, \y2) -- (\x3, \y2);
%%   
%%\matrix(stacks) [
%%   matrix of nodes,
%%   anchor=north, 
%%   text=blue, 
%%   font=\normalsize, 
%%   row 1/.style = {text = blue}, 
%%   row 2/.style = {text = red}, 
%%   text width= 13mm ] 
%%   at ([xshift=-10pt,yshift=-8pt]locycells.south) {
%%   $\thid_1:$ & $\retvar = 0$\\
%%   $\thid_2:$ & $\retvar = 0$\\
%%   };
%\end{pgfonlayer}
%\end{tikzpicture}
%\\
%{\small(c)} & {\small(d)}\\
%\hline
%\begin{tikzpicture}[font=\large]

%\begin{pgfonlayer}{foreground}
%%Uncomment line below for help lines
%%\draw[help lines] grid(5,4);

%%Location x
%\node(locx) at (1,3) {$[\loc_x] \mapsto$};

%\matrix(locxcells) [version list, text width=7mm, anchor=west]
   %at ([xshift=10pt]locx.east) {
 %{a} & $T_0$ & {a} & $\tsid_1$\\
  %{a} & $\emptyset$ & {a} & $\emptyset$ \\
%};
%\node[version node, fit=(locxcells-1-1) (locxcells-2-1), fill=white, inner sep= 0cm, font=\Large] (locx-v0) {$0$};
%\node[version node, fit=(locxcells-1-3) (locxcells-2-3), fill=white, inner sep=0cm, font=\Large] (locx-v1) {$1$};

%%Location y
%\path (locx.south) + (0,-1.5) node (locy) {$[\loc_y] \mapsto$};
%\matrix(locycells) [version list, text width=7mm, anchor=west]
   %at ([xshift=10pt]locy.east) {
 %{a} & $T_0$ & {a} & $\tsid_2$ \\
  %{a} & $\emptyset$ & {a} & $\emptyset$\\
%};
%\node[version node, fit=(locycells-1-1) (locycells-2-1), fill=white, inner sep= 0cm, font=\Large] (locy-v0) {$0$};
%\node[version node, fit=(locycells-1-3) (locycells-2-3), fill=white, inner sep=0cm, font=\Large] (locy-v1) {$1$};

%% \draw[-, red, very thick, rounded corners] ([xshift=-5pt, yshift=5pt]locx-v1.north east) |- 
%%  ($([xshift=-5pt,yshift=-5pt]locx-v1.south east)!.5!([xshift=-5pt, yshift=5pt]locy-v0.north east)$) -| ([xshift=-5pt, yshift=5pt]locy-v0.south east);

%%blue view - I should  check whether I can use pgfkeys to just declare the list of locations, and then add the view automatically.
%\draw[-, blue, very thick, rounded corners=10pt]
 %([xshift=-3pt, yshift=20pt]locx-v1.north east) node (tid1start) {} -- 
 %([xshift=-3pt, yshift=-5pt]locx-v1.south east) --
 %([xshift=-3pt, yshift=5pt]locy-v0.north east) -- 
 %([xshift=-3pt, yshift=-5pt]locy-v0.south east);
 
 %\path (tid1start) node[anchor=south, rectangle, fill=blue!20, draw=blue, font=\small, inner sep=1pt] {$\thid_1$};

%%red view
%\draw[-, red, very thick, rounded corners = 10pt]
 %([xshift=-16pt, yshift=5pt]locx-v1.north east) node (tid2start) {}-- 
%% ([xshift=-16pt, yshift=-5pt]locx-v0.south east) --
%% ([xshift=-16pt, yshift=5pt]locy-v1.north east) -- 
 %([xshift=-16pt, yshift=-5pt]locy-v1.south east) node {};
 
%\path (tid2start) node[anchor=south, rectangle, fill=red!20, draw=red, font=\small, inner sep=1pt] {$\thid_2$};

%%%Stack for threads tid_1 and tid_2
%%
%%\draw[-, dashed] let 
%%   \p1 = ([xshift=0pt]locy.west),
%%   \p2 = ([yshift=-5pt]locycells.south),
%%   \p3 = ([xshift=10pt]locycells.east) in
%%   (\x1, \y2) -- (\x3, \y2);
%%   
%%\matrix(stacks) [
%%   matrix of nodes,
%%   anchor=north, 
%%   text=blue, 
%%   font=\normalsize, 
%%   row 1/.style = {text = blue}, 
%%   row 2/.style = {text = red}, 
%%   text width= 13mm ] 
%%   at ([xshift=-10pt,yshift=-8pt]locycells.south) {
%%   $\thid_1:$ & $\retvar = 0$\\
%%   $\thid_2:$ & $\retvar = 0$\\
%%   };
%\end{pgfonlayer}
%\end{tikzpicture}
%&
%\begin{tikzpicture}[font=\large]

%\begin{pgfonlayer}{foreground}
%%Uncomment line below for help lines
%%\draw[help lines] grid(5,4);

%%Location x
%\node(locx) at (1,3) {$[\loc_x] \mapsto$};

%\matrix(locxcells) [version list, text width=7mm, anchor=west]
   %at ([xshift=10pt]locx.east) {
 %{a} & $T_0$ & {a} & $\tsid_1$\\
  %{a} & $\emptyset$ & {a} & $\{\tsid_4\}$ \\
%};
%\node[version node, fit=(locxcells-1-1) (locxcells-2-1), fill=white, inner sep= 0cm, font=\Large] (locx-v0) {$0$};
%\node[version node, fit=(locxcells-1-3) (locxcells-2-3), fill=white, inner sep=0cm, font=\Large] (locx-v1) {$1$};

%%Location y
%\path (locx.south) + (0,-1.5) node (locy) {$[\loc_y] \mapsto$};
%\matrix(locycells) [version list, text width=7mm, anchor=west]
   %at ([xshift=10pt]locy.east) {
 %{a} & $T_0$ & {a} & $\tsid_2$ \\
  %{a} & $\{\tsid_3\}$ & {a} & $\emptyset$\\
%};
%\node[version node, fit=(locycells-1-1) (locycells-2-1), fill=white, inner sep= 0cm, font=\Large] (locy-v0) {$0$};
%\node[version node, fit=(locycells-1-3) (locycells-2-3), fill=white, inner sep=0cm, font=\Large] (locy-v1) {$1$};

%% \draw[-, red, very thick, rounded corners] ([xshift=-5pt, yshift=5pt]locx-v1.north east) |- 
%%  ($([xshift=-5pt,yshift=-5pt]locx-v1.south east)!.5!([xshift=-5pt, yshift=5pt]locy-v0.north east)$) -| ([xshift=-5pt, yshift=5pt]locy-v0.south east);

%%blue view - I should  check whether I can use pgfkeys to just declare the list of locations, and then add the view automatically.
%\draw[-, blue, very thick, rounded corners=10pt]
 %([xshift=-3pt, yshift=20pt]locx-v1.north east) node (tid1start) {} -- 
%% ([xshift=-3pt, yshift=-5pt]locx-v1.south east) --
%% ([xshift=-3pt, yshift=5pt]locy-v0.north east) -- 
 %([xshift=-3pt, yshift=-5pt]locy-v1.south east);
 
 %\path (tid1start) node[anchor=south, rectangle, fill=blue!20, draw=blue, font=\small, inner sep=1pt] {$\thid_1$};

%%red view
%\draw[-, red, very thick, rounded corners = 10pt]
 %([xshift=-16pt, yshift=5pt]locx-v1.north east) node (tid2start) {}-- 
%% ([xshift=-16pt, yshift=-5pt]locx-v0.south east) --
%% ([xshift=-16pt, yshift=5pt]locy-v1.north east) -- 
 %([xshift=-16pt, yshift=-5pt]locy-v1.south east) node {};
 
%\path (tid2start) node[anchor=south, rectangle, fill=red!20, draw=red, font=\small, inner sep=1pt] {$\thid_2$};

%%%Stack for threads tid_1 and tid_2
%%
%%\draw[-, dashed] let 
%%   \p1 = ([xshift=0pt]locy.west),
%%   \p2 = ([yshift=-5pt]locycells.south),
%%   \p3 = ([xshift=10pt]locycells.east) in
%%   (\x1, \y2) -- (\x3, \y2);
%%   
%%\matrix(stacks) [
%%   matrix of nodes,
%%   anchor=north, 
%%   text=blue, 
%%   font=\normalsize, 
%%   row 1/.style = {text = blue}, 
%%   row 2/.style = {text = red}, 
%%   text width= 13mm ] 
%%   at ([xshift=-10pt,yshift=-8pt]locycells.south) {
%%   $\thid_1:$ & $\retvar = 0$\\
%%   $\thid_2:$ & $\retvar = 0$\\
%%   };
%\end{pgfonlayer}
%\end{tikzpicture}
%\\
%{\small(e)} & {\small(f)}\\
%\hline
%\end{tabular}
%\end{center}
%\caption{Configurations obtained throughout an execution of 
%$\prog_4$.}
%\label{fig:cp.exec}
%\end{figure}

The next consistency model that we illustrate is \emph{consistent prefix}. 
It ensures that once a thread observes the effects of some transaction $\txid$, it also observes all the transactions that were committed before $\txid$. 
\sx{why different locations}
Another way to express this property is that two different transactions never observe the updates to different addresses in a different order.

Consider the program $\prog_4$ below,
 \[
    \prog_4 \equiv  
    \left( 
    \begin{session} 
        \begin{array}{@{}c || c @{}}
            \begin{session} 
            \txid_1 : 
            \begin{transaction}
                \pmutate{\vx}{1};\\
            \end{transaction}; \\
            
            \txid_3 :
            \begin{transaction}
              	\pderef{\pvar{a}}{\vy};\\
              	\pifs{\pvar{a}=0}\\
                    \quad \passign{\retvar}{\Large \frownie{}} 
                \pife 
            \end{transaction}
            \end{session}
            &
            \begin{session}
            \txid_2 :
            \begin{transaction}
                \pmutate{\vy}{1};\\
            \end{transaction} ; \\
            
            \txid_4 :
            \begin{transaction}
              	\pderef{\pvar{a}}{\vx};\\
              	\pifs{\pvar{a}=0}\\
              		\quad \passign{\retvar}{\Large \frownie{}} 
                \pife
            \end{transaction}
            \end{session}
        \end{array}
    \end{session}
    \right)
 \]

We argue that, under $\mathsf{RA}$, it is possible to obtain an execution 
of program $\prog_4$ where both $\ptrans{\trans_2}$ and $\ptrans{\trans_4}$ 
return value ${\Large \frownie{}}$.
Such an execution can be summarised as follows: 

\begin{itemize}
\item initially, $\thid_1$ executes transaction $\ptrans{\trans_1}$, 
leading to the configuration of Figure \ref{fig:cp.exec}(b), and the program 
$\ptrans{\trans_2} \ppar \ptrans{\trans_3} ; \ptrans{\trans_4}$ to be 
executed, 
\item then $\thid_2$ executes transaction $\ptrans{\trans_3}$, leading 
to the configuration of Figure \ref{fig:cp.exec}(c); the remaining 
program to be executed is $\ptrans{\trans_3} \ppar \ptrans{\trans_4}$, 
\item without changing its view, $\thid_1$ executes transaction $\ptrans{\trans_2}$. 
The code $\trans_2$ is executed using the heap $[ [\loc_x] \mapsto 1, [\loc_y] \mapsto 0]$ 
as a snapshot; this means that, by executing $\trans_2$, the variable $\retvar$ of the thread-local 
stack of $\thid_1$ is set to ${\Large \frownie{}}$. Next, the thread $\thid_2$ executes $\ptrans{\trans_4}$ 
without altering its view. Similarly to the execution of $\ptrans{\trans_2}$ in $\thid_1$, this causes the 
$\retvar$ variable of the thread-local stack of $\thid_2$ to be set to ${\Large \frownie{}}$. At this point, 
the final configuration of the program is the one given in Figure \ref{fig:cp.exec}(d).
\end{itemize}

\sx{What is different update ?}
To avoid different threads to observe different updates in different orders, we impose a constraint known as \emph{consistent prefix}.
\sx{ why centralised? 
in a centralised database, where 
transactions have a start and a commit point, 
}
Assuming a transaction has a start and commit point, \emph{consistent prefix} requires that if a transaction $\txid_1$ observes the effects of another transaction $\txid_2$, then it must observe the effects of any other transaction that committed before $\txid_2$.
In the history heaps framework, transactions are executed in a single step in the semantics,
\sx{
this constraint can be formalised as follows: 
\emph{If a transaction $\tsid_1$ observes the effects of another transaction $\tsid_2$, then it must 
observe the effects of any transaction that committed before $\tsid_2$.}
}
In the history heaps framework, transactions are executed in a single step and the step corresponds to the commit order.
Thus, upon the read atomic (\defref{def:readatomic}), \emph{consistent prefix} requires once a thread commits a transaction, it pushes the view to be up-to-date with the state of the database, so the following transactions from the thread will observe the effects of all transactions committed before \( \txid \) included.
\sx{It is easy to see the definition match the intuition when inside a thread, but not easy to see cross-thread. There is something I feel subtle but dont know what it is and I feel it is actually already included in the definition.}

\sx{ dont understand:
In the history heaps framework, transactions are executed in a single step; 
however, one may think of the order in which transactions execute in our 
operational semantics to be consistent with the order in which 
transactions commit (this correspondence will be made precise later in 
the document, when we will relate executions in our operational semantics 
to abstract executions used in the declarative style for specifications of 
consistency models). By requiring that, after a thread $\thid$ executes 
a transaction $\tsid$, it pushes its own view to be up-to-date with the state of 
the system, we model the fact that any future transaction executed 
by $\thid$ will observe the effects of anything that committed before 
$\tsid$ (included).
}

\begin{defn}[consistent prefix]
\label{def:consistent-prefix}
The \emph{consistent prefix} is stronger than read atomic (\defref{def:readatomic}) by further requiring the view after pushes to the latest for all addresses, 
\[
\begin{rclarray}
    (\hh, \vi) \csat[\mathsf{CP}] \opset: \vi' & \defeq &  (\hh, \vi) \csat[\mathsf{RA}] \opset: \vi' \land \fora{\addr \in \dom(\hh)} \vi'(\addr) = \left| \hh(\addr) \right| \\
\end{rclarray}
\]
\sx{
We say that $(hh, V) \triangleright_{\mathsf{CP}} \mathcal{O}: V'$ 
if and only if $(\hh, V) \triangleright_{\mathsf{RA}} \mathcal{O}: V'$, 
and for any location $[\loc_x]$, $V'([\loc_x]) = \lvert \mathsf{HHeapUpdate}(\hh, V, \mathcal{O}) \rvert -1$. 
}
\end{defn}


\sx{did not read} 
Consider again th program $\prog_4$, ths time to be executed 
under $\mathsf{CP}$. We argue that in this case it is not possible to have both 
threads $\thid_1$ and $\thid_2$ to set the value of $\retvar$ to ${\Large \frownie{}}$. 
The initial configuration in which the program $\prog_4$ is executed is 
the one depicted in Figure \ref{fig:cp.exec}(a). Initially, either thread $\thid_1$ executes 
the code $\ptrans{\trans_1}$, or thread $\thid_2$ executes transaction $\ptrans{\trans_3}$; without 
loss of generality, we consider the former option (the case in which $\thid_2$ executes 
first is symmetric). Upon executing the code $\ptrans{\trans_1}$, we obtain the 
configuration of Figure \ref{fig:cp.exec}(b), with the program $\ptrans{\trans_2} \ppar (\ptrans{\trans_3} ; \ptrans{\trans_4})$ 
to be executed next.
At this point, note that under $\mathsf{CP}$ it is not possible for $\thid_2$ to execute $\ptrans{\trans_3}$ and obtain 
the configuration of Figure \ref{fig:cp.exec}(c) as a result. This is because, in $\mathsf{CP}$, we require the view of $\thid_2$ 
\textbf{after} executing $\ptrans{\trans_3}$ to be up-to-date, i.e. to point to the last location of each version. That is, 
assuming that $\thid_2$ executes $\ptrans{\trans_3}$ next, we obtain the configuration of Figure \ref{fig:cp.exec}(e). 
From this point on, whenever $\thid_2$ will execute transaction $\ptrans{\trans_4}$, it will read value $1$ for 
location $[\loc_x]$, hence it won't be able to set the value of $\retvar$ to ${\Large \frownie{}}$. One possible 
final configuration for the program is given in Figure \ref{fig:cp.exec}(f).

%\begin{definition}
%$(\hh, V, \mathcal{V}) \triangleright_{\mathsf{CP}} \mathcal{O}$ iff 
%$(\hh, V, \mathcal{V}) \triangleright_{\mathsf{RA}} \mathcal{O}$, 
%and for any $V' \in \mathcal{V}$, either $V' \sqsubseteq V$ or 
%$V \sqsubseteq V'$.
%\end{definition}
%\ac{I found that this is a very easy way to encode consistent 
%prefix. In English, a thread can execute a transaction if its view does 
%not cross with the views of any other thread.}

%\ac{Very Important - note to self: It seems that the condition of 
%requiring that views do not cross before executing a transaction 
%does not suffice to model snapshot isolation. In fact, it seems that 
%consistent prefix (when transaction are limited to either one read 
%or one write) coincides with TSO.
%Update, it seems that the condition that I need for Snapshot Isolation 
%(besides write confict detection) is that, after you execute a transaction, 
%you bring your view up-to-date. (So here I have to concede that I was wrong, 
%and the state of the view after you execute a transaction is actually important).\\
%
%}

%\subsection{Parallel Snapshot Isolation and Snapshot Isolation}
\sx{Need some citations here, what is geo-replicated, distributed database vs distributed system}
\emph{Snapshot Isolation} (SI) is a consistency model that has been widely employed in both centralised and distributed databases. 
Because snapshot isolation does not scale well to geo-replicated and distributed systems, a weaker model called \emph{Parallel Snapshot Isolation} (PSI) has been recently proposed. 

%\sx{
%Both SI and PSI can be specified in the history heap framework by combining the consistency models that we have already introduced. 
%In short, SI combines atomic visibility and the snapshot monotonicity property from consistent prefix property (if a transaction $\tsid_1$ 
%observes the effects of another transaction $\tsid_2$, then it also observes 
%the effects of any transaction that committed before), and the write-conflict 
%detection from Update Atomic (two committing transactions do not write 
%concurrently to the same location). In contrast, PSI only requires atomic 
%visibility, causal consistency and write-conflict detection. Formally, we have 
%the following:
%}
Both SI and PSI can be specified in the history heap framework by combining the consistency models that we have already introduced. 
In short, SI combines consistent prefix property and update atomic, while PSI only requires causal consistency and update atomic.
\begin{definition}
The \emph{parallel snapshot isolation} combines causal consistency (\cref{def:causal})  and update atomic (\cref{def:update-atomic}): $\mathsf{PSI} = \mathsf{CC} \cap \mathsf{UA}$.
The {snapshot isolation} combines consistent prefix (\cref{def:consistent-prefix}) and update atomic (\cref{def:update-atomic}): $\mathsf{SI} = \mathsf{CP} \cap \mathsf{UA}$.
\end{definition}



%\subsection{Serialisability \( \SER \)}
\label{sec:sound-complete-ser}

The execution test $\ET_\SER$ is sound with respect to the axiomatic definition 
\[ 
    (\RP_{\LWW}, \Set{\lambda \aexec. \AR })
\]
We pick the invariant as \( I( \aexec, \cl ) = \emptyset \), given the fact of no constraint on the view after update.
Assume a kv-store $\mkvs$, an initial and a final view $\vi, \vi'$  a fingerprint $\fp$ 
such that $\ET_{\SER} \vdash (\mkvs, \vi) \csat \fp: (\mkvs',\vi')$. 
Also choose an arbitrary $\cl$, a transaction identifier $\txid \in \nextTxid(\mkvs, \cl)$, 
and an abstract execution $\aexec$ such that $\mkvs_{\aexec} = \mkvs$ and 
\( I(\aexec, \cl) =  \emptyset \subseteq \Tx[\mkvs, \vi] \).
Let \( \aexec' = \extend[\aexec, \txid, \Tx[\mkvs, \vi], \fp] \).
Note that since the invariant is empty set, it remains to prove there exists a set of read-only transactions \( \txidset_\rd \) such that:
\[
    \begin{array}{@{}l@{}}
        \fora{ \txid' } 
        \txid' \toEDGE{\AR_{\aexec'}} \txid \implies \txid' \in \Tx[\mkvs, \vi] \cup \txidset_\rd
    \end{array}
\]
Since the abstract execution satisfies the constraint for \( \SER \), \ie \( \AR \subseteq \VIS \), we know \( \AR = \VIS \).
Since \( \Tx[\mkvs, \vi]  \) contains all transactions that write at least a key, 
we can pick a \( \txidset_\rd \) such that \( \Tx[\mkvs, \vi] \cup \txidset_\rd = \txidset_\aexec\),
which gives us the proof.


The execution test $\ET_\UA$ is complete with respect to the axiomatic definition \( (\RP_{\LWW}, \Set{\lambda \aexec. \AR_\aexec }) \).
Assume i-\emph{th} transaction \( \txid_i \) in the arbitrary order,
and let view \( \vi_{i} = \getView[\aexec, \VIS^{-1}_{\aexec}(\txid_{i})] \).
We also pick any final view such that \( \vi'_{i} \subseteq \getView[\aexec, (\AR^{-1}_{\aexec})\rflx(\txid_{i})] \).
Note that there is nothing to prove for \( \vi'_i \),
Now we need to prove the following:
\[
    \fora{\key, j}  0 \leq j < \abs{\mkvs_{\cut[\aexec, i-1]}(\key)} \implies j \in \vi_i(\key)
\]
Because \( \VIS^{-1}(\txid_i) = \AR^{-1}(\txid_i) = \Set{\txid }[\txid \in \mkvs_{\cut[\aexec, i-1]} ]\),
so for any key \( \key \) and index \( j \) such that \( 0 \leq j < \abs{\mkvs_{\cut[\aexec, i-1]}(\key)} \),
the j-\emph{th} version of the key contains in the view, \ie \( j \in \vi(\key)\).



