\vspace*{\fill}
\section*{Abstract}

Transactions are the main units of execution in database systems that employ concurrency to achieve better performance. They are managed by protocols that are able to guarantee different degrees of consistency based on the application's requirements. One of the strong consistency properties is serializability, which requires the outcome of any schedule of operations, coming from concurrent transactions, to be equivalent to a serial one in terms of the effects on the database. Surprisingly, modern program logics that enable the verification of fine-grained concurrency have not been applied to the context of database transactions yet. \\ \\
We present a program logic for serializable transactions that are able to manipulate a shared storage. The logic is proven to be sound with respect to operational semantics that treat transactions as real atomic blocks. Nevertheless, we are able to model the atomicity of transactions even if it is only apparent, and concurrent interleavings do actually occur. We show this by providing the first application of our logic in terms of the \textit{Two-phase locking} (\tpl) protocol which ensures serializability.
We first define a formal model and semantics that fully express the two-phase locking behaviours as part of a generic software system, then we show the equivalence between its operational semantics and the truly atomic ones. This result creates the necessary link to enable client reasoning in the \tpl\ setting through our program logic for transactions.
\vspace*{\fill}