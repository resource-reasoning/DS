\section{Introduction}
Transactions are the \emph{de facto} synchronisation mechanism in modern distributed databases.
To achieve scalability and performance, distributed databases  
often use weak transactional consistency guarantees. 
Much work has been done to formalise the semantics of such consistency guarantees, both
declaratively and operationally.
On the declarative side, several \emph{general} formalisms have been proposed, 
such as dependency graphs~\cite{adya} and abstract executions~\cite{ev_transactions}, to provide a unified
semantics for formulating different consistency models.  
On the operational side, the semantics of \emph{specific} consistency models have
been captured using reference implementations~\cite{si,PSI,PSI-RA}. 
However, unlike declarative approaches, there has been
little work on \emph{general} operational semantics for describing a range
of consistency models.

%There has been some work on general operational semantics, summarised
Let us summarise briefly the existing work on general operational semantics, 
which we  
%here and 
discuss further in \cref{sec:conclusions}.
%~\cite{sureshConcur,alonetogether,seebelieve}. 
%\citeauthor{alonetogether} 
\citet{alonetogether} propose an operational
semantics for describing the behaviour of  atomic transactions
accessing a global, centralised store 
in order to develop 
a program logic  for reasoning about client programs. They capture 
 several consistency models, although not some of the  weaker ones
 such as parallel snapshot isolation (\PSI)  \cite{PSI}
 which are important 
for distributed data-stores.
%Moreover, they do not establish the equivalence of their definitions
%of consistency model
%with existing declarative definitions in the literature. 
%\citeauthor{sureshConcur} 
\citet{sureshConcur} propose an operational semantics over abstract
execution graphs, rather than   a concrete centralised store, in order to
prove the robustness of applications against
a given consistency model. They are able to 
capture weaker consistency models
such as \(\PSI\). However, although they focus on consistency models with snapshot 
property\footnote{\emph{Snapshot property}, also known as \emph{atomic visibility},
means that a transaction reads from an atomic snapshot of the database, and commits atomically.},
%\footnote{\emph{Atomic visibility} means that a transaction 
%observes either all or none the updates of another transaction.},
their semantics allows for the fine-grained interleaving of operations in different
transactions. We believe that this results  in an unnecessarily complicated semantics.
%However, this semantics cannot model client sessions.
\citet{seebelieve} provide a trace semantics over a global
centralised store, where the behaviour of clients is formalised by the   
observations they make on the totally-ordered history of states. 
They 
%show  that implementations are correct  with respect to the
%appropriate 
%consistency model, and 
demonstrate  the equivalence of several
implementation-specific definitions of
consistency model  in the literature. 
%but it does little towards proving the correctness of protocols employed by distributed databases. 
However, the usefulness  of their approach for analysing client programs is not clear, 
%program analysis for clients; in particular, we believe that in their framework the latter task would be 
%difficult,
since observations made by their clients  involve information that is not generally 
available to real-world clients,  such as the total order in which transactions commit.
%or to p
%They focus on proving implementations correct. They do not consider program analysis for clients;
%indeed we believe it would be difficult  given their choice to
%keep track of the entire system.
In summary, there has been no work on a general operational semantics 
for both verifying protocols of distributed databases and 
analysing  client programs. 


In this paper, we introduce a general operational semantics for describing the
client-observable behaviour of distributed {atomic} transactions
(\cref{sec:overview}, \cref{sec:model}), while successfully abstracting from the 
internal details of protocols of geo-replicated and partitioned
databases. In our semantics, 
transactions execute atomically, preventing  fine-grained interleaving of the  
operations they perform. This makes it feasible 
both to verify that a distributed protocol correctly
implements a consistency model and 
to  analyse client applications.
Our semantics comprises a global, centralised key-value store (kv-store)
with {\em multi-versioning}, which records all the versions of a key,
and partial {\em client views}, which let clients see only a subset of
the versions,  and is inspired by the views in the  C11 operational semantics
in~\cite{promises}. 
Like \cite{seebelieve}, our operational semantics is parametric in the notion of \emph{execution test},
which determines if a client with a given view is allowed to commit a transaction.
Just as with standard operational semantics, our  next transaction step is determined by the current kv-store and client view.
This is contrast with Crook's trace semantics, where the next step requires an analysis of the whole trace.
%   if a client with a given view is
%allowed to commit a transaction. Unlike \cite{seebelieve},
%our execution tests rely on a partial client view rather than the
%client requiring complete 
%knowledge of the state history. 
%Different execution tests give rise
%to different consistency models in our semantics. In contrast with \cite{seebelieve}, 
%we restrict observations of clients  to the current state of the system: this leads to 
%execution tests being a more faithful abstraction of the behaviour of real distributed databases protocols.
We  capture  most of the well-known consistency models in
a uniform way (\cref{sec:cm}): e.g,  causal consistency (\CC) \cite{cops}, parallel snapshot isolation (\PSI),
snapshot isolation (\SI) \cite{fekete-tods} and serialisability (\SER), and also identify 
a new one that sits in-between \PSI\; and \SI\; and retains good properties of both. Since we focus on
atomic visibility,  we are not able to capture popular 
  consistency models such as \emph{read committed}. 
Because our focus is on protocols and applications employed  by distributed databases, 
  most of which  guarantee atomic visibility, we do not find this constraint to be a severe
  limitation.
%We define these models using kv-stores and views, 
%provide a correspondence between our kv-stores and dependency graphs, 
%We introduce novel proof techniques for demonstrating that our
%definitions of consistency models 
%are equivalent to existing declarative definitions (\cref{sec:other_formalisms}).
%

We believe it is crucial to link our definitions of consistency model for
kv-stores to definitions of consistency model given in the
literature. 
This has not been done in \cite{alonetogether} but has been done in
\cite{seebelieve}. 
We  provide (\cref{sec:other_formalisms} and appendix) a general proof technique for proving the correspondence between 
our  execution tests and the axiomatic specifications of consistency
models using abstract executions.  Using
this technique, we 
%are able to 
demonstrate 
that our consistency models are equivalent to the  declarative definitions of
consistency model  for
abstract executions~\cite{framework-concur}, and hence for  dependency
graphs~\cite{adya,laws}. 







With our operational semantics, we are able to verify protocols of
distributed databases and analyse client programs. By contrast, in
existing literature,  these two tasks are carried out in
\emph{different} semantics: for example, protocols are verified using
abstract execution graphs~\cite{repldatatypes,framework-concur}; clients are analysed using dependency
graphs~\cite{fekete-tods,SIanalysis,giovanni_concur16,psi-chopping,sureshConcur}; and equivalence results are used to move between
the two~\cite{laws}. Using our semantics, we establish the correctness of two database protocols: 
the COPS protocol for the fully replicated kv-stores~\cite{cops}  
which 
satisfies CC;  and the Clock-SI   protocol for
partitioned kv-stores~\cite{clocksi}  which satisfies SI.  We are also
able to  
establish the robustness of simple client applications: in 
particular, we prove the robustness of a single counter against \PSI,
and the robustness of multiple counters against  \SI\;
(\cref{sec:applications}). 
%
%For the former, we show that the COPS protocol of a 
%replicated database satisfies our definition of $\CC$ and that the Clock-SI protocol of a partitioned database satisfies our definition of $\SI$.  
%For the latter, we show the robustness of applications against our consistency models: 
%we prove that a transactional library comprising a single counter is robust against $\PSI$; 
%and  that the library with multiple counters is robust against $\SI$,
%but not $\PSI$. 
To our knowledge, our robustness results are the first to take into account client sessions.
Interestingly, with sessions we show that  
multiple counters are not
robust against \(\PSI\),
whereas without sessions it can be shown that 
multiple counters are  robust against \(\PSI\) using 
static-analysis techniques from \cite{giovanni_concur16}. 
\sx{The reviewer from OOPSLA confused about client sessions here.
    Multi-counters are actually robust against \( \WSI \).
    It immediately gives some idea why \( \WSI \) is a interesting consistency model.
}
%
%\mypar{Outline}
%The remainder of this article is organised as follows.
%In \cref{sec:overview} we give an intuitive overview of our ideas. 
%In \cref{sec:model} we present our general operational semantics. 
%In \cref{sec:cm} we show how we encode consistency models in our semantics.
%In \cref{sec:other_formalisms} we relate our formalism to existing declarative formalisms.
%In \cref{sec:applications} we showcase the applications of our semantics.  
%In \cref{sec:conclusions} we discuss related work and conclude. 
%
