\section{Introduction}
\label{sec:intro}

Transactions are the \emph{de facto} synchronisation mechanism in
modern distributed databases.  To achieve scalability and performance,
distributed databases often use weak transactional consistency
guarantees. These weak guarantees pose two main challenges: the
verification that database protocols satisfy a particular consistency
guarantee; and the analysis of client behaviour with such guarantees.

These consistency guarantees are known as \emph{consistency models}
Much work has been done to formalise the semantics of such consistency models:
declaratively, several \emph{general} formalisms have been proposed, such as
dependency graphs \cite{adya} and abstract
execution \cite{ev_transactions}, to provide a unified axiomatic semantics for
formulating different consistency models;
operationally, although {specific} consistency models have been captured 
using reference implementations \cite{si,PSI,PSI-RA,cops,ramp,bayou},
there has been little work on \emph{general} operational semantics
for describing a range of consistency models.
Our goal is to provide such an general operational semantics, 
to verify database protocols of particular consistency models 
and to analyse client programs with respect to such consistency models.  

The declarative approach has been substantially
studied~\cite{adya,ev_transactions,framework-concur,laws}. 
It uses  a two-step procedure:
\begin{enumerate*}
\item construct {\em candidate executions} of the whole program comprising
transactions in which reads may contain an arbitrary value; and 
\item apply a number of axioms on such executions to rule out invalid executions. 
Such axioms may state, for example, that every read is
validated by a write that has written the read value. 
\end{enumerate*}
%Of particular interest are axioms that rule out certain cycles.
This axiomatic approach is unsuitable for invariant-based program analysis 
which requires reasoning about the step-wise execution of a program. 

The operational approach has been less studied \cite{sureshConcur,alonetogether,seebelieve,push-pull}.
\citet{sureshConcur} introduce an operational semantics over abstract
execution graphs, and prove robustness of client programs using model
checking. Each execution step adds a new transaction node in the graph
by first constructing candidate steps and then applying axioms to the
whole graph to rule out invalid candidate steps.  This approach is
unlikely to be suitable for invariant-based analysis associated with
the state.  \citet{alonetogether} propose an operational semantics for
SQL transaction programs over an abstract centralised store with the consistency
models given by the standard ANSI/SQL isolation levels \cite{si}. They
develop a program logic and prototype tool for reasoning about client
programs, and so can capture invariant properties of the state. They can
express snapshot isolation \cite{si}, but the consensus is
that they cannot
capture the weaker consistency models such as parallel snapshot isolation \cite{PSI} 
important for distributed databases. 
\citet{seebelieve} provide a trace semantics
over a global centralised store, where an execution step involves
analysing the totally-ordered
history of states and the read-write set of the transaction.
They show the equivalence of several
implementation-specific definitions of consistency model. However, their
approach does not lend itself to  analysing client programs,
since observations made by clients require the total order in
which transactions commit which is not realistic for clients. 
\citet{push-pull} provide a log-based abstract operational semantics for verifying 
several implementations of serilisability.
However, 
it is unknown that they could be easily extended to tackle weaker consistency models.



We introduce an \emph{interleaving} operational semantics for
describing the client-observable behaviour of atomic transactions on
distributed key-value stores (\cref{sec:model}). Our semantics uses abstract states
comprising \emph{a global, centralised key-value store} (kv-store) with
\emph{multi-versioning}, which records all the versions of a key, and
\emph{partial client views}, which let clients see only a subset of the
versions.  Our client views partly inspired by the views in the C11
operational semantics in~\cite{promises}.  An execution step depends
simply on the abstract state, the read-write set of the transaction, and an \emph{execution test} which
determines if a client with a given view is allowed to commit a
transaction. Different execution tests give rise to different consistency models, % fr our abstract states, 
which we show to be equivalent to well-known
declarative definitions of consistency model for execution graphs (\cref{sec:other_formalisms}).
Our execution tests resemble the  approach taken in~\cite{seebelieve},
except that they require an analysis of the whole trace for an
execution step whereas we require the current abstract state. 




We make the assumption that our transactions satisfy the \emph{snapshot property}, 
also known as \emph{atomic visibility}. This means that
the client observes that a transaction reads from an atomic snapshot
of the database and commits atomically, even if the underlying
distributed protocol has a more fine-grained behaviour. This
assumption is reasonable in distributed databases: for example, with
on-line shopping application, a client only sees one snapshot of the database and
only has knowledge that their transaction has successfully committed.
Our execution tests  uniformly capture  the well-known consistency models 
that satisfy this snapshot property (\cref{sec:cm}): \eg, 
causal consistency (\(\CC\)) \cite{cops,bayou}, parallel snapshot isolation (\(\PSI\)) \cite{PSI,PSI-RA}, 
snapshot isolation (\(\SI\)) \cite{si} and serialisability \cite{si}. 
We also identify a new consistency model, called \emph{weak snapshot isolation} (\(\WSI)\), 
that interestingly sits between \(\PSI\) and \(\SI\) and retains good properties of both.


Using our operational semantics, we demonstrate that we can verify
that database protocols satisfy particular consistency models and
prove invariant properties of client programs with respect to such
consistency models (\cref{sec:applications,sec:program-analysis}). In particular, we establish the correctness of two database
protocols: the COPS protocol for the fully replicated kv-stores~\cite{cops} 
which satisfies causal consistency (\cref{sec:verify-impl}); 
and the Clock-SI protocol for partitioned kv-stores \cite{clocksi} 
which satisfies snapshot isolation (\cref{app:implementation-verification}). 
We prove invariant properties of client programs including general robustness results
for client programs with certain patterns of transaction and
invariant properties of specific programs not satisfying such robustness results (\cref{sec:invariant-client-programs}). 
We believe out robustness results are the first to take into account client
sessions: with sessions, we show that multiple counters {\em are not} robust against \(\PSI\);
interestingly, without sessions, \citet{giovanni_concur16} show that multiple counters {\em are}
robust against \(\PSI\) using static-analysis techniques which are
known not to be applicable to sessions.  


With our operational semantics, we believe that we have identified an interesting \emph{mid-point}
between distributed databases and clients.
As far as we are aware, there has been no previous work on transactions
that, in the same general semantics, verifies distributed protocols and analyses client programs. 
this was an important goal for us, resonating with recent work
that does just this for standard shared-memory concurrency \cite{cap,tada,iris,fcsl}. 


%{\color{red}
%----------below is good stuff to put somewhere else, the invariant
%property blurb might  change depending on what you find--------

%For weak consistency models, this goal has not been explored in the proposed  operational
%semantics~\cite{1,2,3}. With the declarative formalisms, 
%protocols are typically verified using abstract execution
%graphs~\cite{repldatatypes,framework-concur} and 
%clients are analysed using dependency graphs~\cite{fekete-tods,SIanalysis,giovanni_concur16,psi-chopping,sureshConcur};
%and equivalence results are used to move between the two~\cite{laws}. 

%Following this work, \citet{sureshConcur} propose an operational semantics over abstract
%execution graphs, rather than   a concrete centralised store, in order to
%prove the robustness of applications against
%a given consistency model. They are able to 
%capture weaker consistency models
%such as \(\PSI\) and \(\CC\) . However, although they focus on consistency models with snapshot 
%property,
%their semantics allows for the fine-grained interleaving of operations in different
%transactions. We believe that this results  in an unnecessarily complicated semantics.


%We prove that our operational definitions of consistency models for
%kv-stores are equivalent to the well-known declarative definitions
%of consistency model for execution graphs.  Such results have not
%been given in \cite{alonetogether}, but have been given in
%\cite{seebelieve}.  We provide a general proof technique which
%captures the correspondence between our execution tests and the
%axiomatic specifications of consistency models for abstract
%executions (\cref{sec:other_formalisms} and \cref{sec:et-sound-complete-constructor}). 
%Using this technique, we prove that 
%our definitions of consistency model for kv-stores are equivalent to 
%the  declarative definitions of consistency model for
%abstract executions~\cite{framework-concur}, and hence for dependency graphs~\cite{adya,laws}. 
%}
