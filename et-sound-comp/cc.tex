\subsection{Causal Consistency \( \CC \)}
\label{sec:sound-complete-cc}

The wildly used definition on abstract executions for causal consistency is that \( \VIS \) is transitive.
Yet it is for the sack of elegant definition, while there is a minimum visibility relation given by \( (\WR_\aexec \cup \SO_\aexec)^{+} ; \VIS_\aexec \subseteq \VIS_\aexec \) (\cref{lem:aexec-spec-cc}).

\begin{lemma}
    \label{lem:aexec-spec-cc}
    For any abstract execution \( \aexec \) under last-write-win, if it satisfies the following:
    \[
        (\WR_\aexec \cup \SO_\aexec)^{+} ; \VIS_\aexec \subseteq \VIS_\aexec \quad \SO_\aexec \subseteq \VIS_\aexec
    \]
    There exists a new abstract execution \( \aexec' \) where \( \txidset_\aexec = \txidset_{\aexec'} \), \( \AR_\aexec = \AR_{\aexec'} \),
    \( \VIS_{\aexec'} ; \VIS_{\aexec'} \subseteq \VIS_{\aexec'} \), and
    under last-write-win \( \TtoOp{T}_{\aexec}(\txid) = \TtoOp{T}_{\aexec'}(\txid) \) for all transactions \( \txid \).
\end{lemma}
\begin{proof}
    To recall, the write-read relation under a key \( \WR(\aexec, \key) \) is defined as 
    \( \WR(\aexec, \key) \defeq \Set{ (\txid, \txid') }[ \exsts{\val} (\otW, \key, \val) \in_\aexec \txid \land (\otR, \key, \val) \in_\aexec \txid' \land \txid = \max_\AR(\VIS^{-1}(\txid')) ]\).
    Given an \( \aexec \) that satisfies the following
    \[
        (\WR_\aexec \cup \SO_\aexec )^{+} ; \VIS_\aexec \subseteq \VIS_\aexec \quad \SO_\aexec \subseteq \VIS_\aexec
    \]
    we erase some visibility relation for each transaction following the order of arbitration \( \AR \) until the visibility is transitive.
    Assume the i-\emph{th} transaction \( \txid_i \)  with respect to the arbitration order.
    Let \( R_i \) denote a new visibility for transaction \( \txid_i \) such that
    \( R_i\projection{2} = \Set{\txid_i}\)
    and the visibility relation before (including) \( \txid_i \) is transitive.
    Let \( \aexec_i = \mkvs_{\cut[\aexec, i]} \) and \( \VIS_i = \bigcup_{0 \leq k \leq i} R_i \).
    For each step, says i-\emph{th} step, we  preserve the following:
    \begin{gather}
        \VIS_i ; \VIS_i \subseteq \VIS_i \label{equ:vis-i-transitive} \\
        \fora{\txid} (\txid,\txid_i) \in R_i \implies (\txid, \txid_i) \in (\WR_i \cup \SO_i)^{+}
        \label{equ:last-read-correct}
    \end{gather}
    
    \begin{itemize}
    \item \caseB{\( i = 1 \) and \( R_1 = \emptyset \)}
    Assume it is from client \( \cl \).
    There is no transaction committed before, so \( \VIS_1 = \emptyset \) and \( \VIS_1 ; \VIS_1 \subseteq \VIS_1 \) as \cref{equ:vis-i-transitive}.

    \item \caseI{i-\emph{th} step}
    Suppose the (i-1)-\emph{th} step satisfies \cref{equ:vis-i-transitive} and \cref{equ:last-read-correct}.
    Let consider i-\emph{th} step and the transaction \( \txid_i \).
    Initially we take \( R_i \) as empty set.
    We first extend \( R_i \) by closing with respect to \( \WR_i \)
    and prove that it does not affect any read from the transaction \( \txid_i \).
    Then we will do the same for \( \SO_i \).
    \begin{itemize}
        \item \( \WR_i\). For any read \( (\otR, \key, \val ) \in \txid_i \),
        there must be a transaction \( \txid_j \) that \( \txid_j \toEDGE{\WR(\aexec_i,\key), \AR} \txid_i \) and \( j < i \).
        We include \( (\txid_j, \txid_i) \in R_i \).
        Let consider all the visible transactions of \( \txid_j \).
        Assume a transaction \( \txid' \in \VIS_{i-1}^{-1}(\txid_j) \), 
        thus \( \txid' \in \VIS_{j}^{-1}(\txid_j) = R_j^{-1}(\txid_j) \).
        It is safe to include \( (\txid', \txid_i) \in R_i \) without affecting the read result,
        because those transaction \( \txid' \) is already visible for \( \txid_i \) in the abstract execution \( \aexec \):
        by \cref{equ:last-read-correct} we know \( R_j \subseteq (\WR_j \cup \SO_j)^{+} \subseteq (\WR_\aexec \cup \SO_\aexec)^{+}\),
        and by the definition of \( \WR(\aexec_i,\key) \) we know \( \WR(\aexec_i,\key) \subseteq \VIS_\aexec\).

        \item Given \( \SO_\aexec \subseteq \VIS_\aexec \), we include \( (\txid_j,\txid_i) \) for some \( \txid_j \)
        such that \( \txid_j \toEDGE{\SO_\aexec} \txid_i\).
        For the similar reason as \( \WR \),
        it is safe to includes all the visible transactions \( \txid' \) for \( \txid_j \), \ie \( \txid' \in R_j^{-1}\).
        \end{itemize}
        
    By the construction, both \cref{equ:vis-i-transitive} and \cref{equ:last-read-correct} are preserved. 
    Thus we have the proof.
    \end{itemize}
\end{proof}
\begin{proposition}
    \label{prop:cc-vis}
    For any abstract execution \( \aexec \) under last-write-win, if it satisfies the following:
    \[
        (\WR_\aexec \cup \SO_\aexec)^{+} ; \VIS_\aexec \subseteq \VIS_\aexec \quad \SO_\aexec \subseteq \VIS_\aexec
    \]
    then
    \[
        \exsts{R \subseteq \AR_\aexec} \VIS = (\WR_\aexec \cup \SO_\aexec \cup R)^{+}
    \]
\end{proposition}

By \cref{lem:aexec-spec-cc}, the execution test $\ET_\CC$ is sound with respect to the axiomatic definition 
\( (\RP_{\LWW}, \Set{\lambda \aexec. ( \SO_{\aexec} \cup \WR_{\aexec} )^{+} ; \VIS_{\aexec}, \lambda \aexec \ldotp \SO_\aexec })\).
We pick an invariant for the \( \ET_\CC \) as the union of those for \( \MR\) and \( \RYW \) shown in the following:
\begin{align*}
    I_1(\aexec, \cl) & =  \left( \bigcup_{\Set{\txid_{\cl}^{n} \in \txidset_{\aexec} }[ n \in \Nat ]} \VIS_{\aexec}^{-1}(\txid^n_\cl) \right) \setminus \txidset_\rd \\
    I_2(\aexec, \cl) & =  \left( \bigcup_{\Set{\txid_{\cl}^{n} \in \txidset_{\aexec} }[ n \in \Nat ]} (\SO_{\aexec}^{-1})\rflx(\txid^n_\cl) \right) \setminus \txidset_\rd
\end{align*}
where \( \txidset_\rd \) is all the read-only transactions included in both 
\( \left( \bigcup_{\Set{\txid_{\cl}^{n} \in \txidset_{\aexec} }[ n \in \Nat ]} \VIS_{\aexec}^{-1}(\txid^n_\cl) \right)\) 
and \( \left( \bigcup_{\Set{\txid_{\cl}^{n} \in \txidset_{\aexec} }[ n \in \Nat ]} (\SO_{\aexec}^{-1})\rflx(\txid^n_\cl) \right) \).
Assume a kv-store $\mkvs$, an initial and a final view $\vi, \vi'$  a fingerprint $\fp$ 
such that $\ET_{\CC} \vdash (\mkvs, \vi) \csat \fp: (\mkvs',\vi')$. 
Also choose an arbitrary $\cl$, a transaction identifier $\txid_\cl^n \in \nextTxid(\mkvs, \cl)$, 
and an abstract execution $\aexec$ such that $\mkvs_{\aexec} = \mkvs$ and 
\( I_1(\aexec, \cl) \cup I_2(\aexec, \cl) \subseteq \Tx[\mkvs, \vi] \).
We are about to prove there exists an extra set of read-only transactions \( \txidset'_\rd \) such that
the new abstract execution \( \aexec' = \extend[\aexec, \txid_\cl^n, \fp, \Tx[\mkvs, \vi] \cup \txidset_\rd \cup \txidset'_\rd] \) and:
\begin{gather}
    \fora{\txid} (\txid, \txid_\cl^n) \in \SO_{\aexec'} \implies \txid \in \Tx[\mkvs, \vi] \cup \txidset_\rd \cup \txidset'_\rd \label{equ:cc-sound-update-so}\\
    \fora{\txid} (\txid, \txid_\cl^n) \in ( \SO_{\aexec'} \cup \WR_{\aexec'} ) ; \VIS_{\aexec'} \implies \txid \in \Tx[\mkvs, \vi] \cup \txidset_\rd \cup \txidset'_\rd \label{equ:cc-sound-update-visvis}\\
    I_1(\aexec',\cl) \cup I_2(\aexec',\cl) \subseteq \Tx[\mkvs_{\aexec'}, \vi'] \label{equ:cc-sound-inv} 
\end{gather}

\begin{itemize}
\item The invariant \( I_2 \) implies \cref{equ:cc-sound-update-so} as the same as \( \RYW \) in \cref{sec:sound-complete-ryw}.
\item \cref{equ:cc-sound-update-visvis}.
    Note that \( (\txid, \txid_\cl^n) \in ( \SO_{\aexec'} \cup \WR_{\aexec'} ); \VIS_{\aexec'} \implies (\txid, \txid_\cl^n) \in ( \SO_{\aexec} \cup \WR_{\aexec} ) ; \VIS_{\aexec'}\).
    Also, recall that \( \SO_\aexec = \SO_\mkvs \) and \( \WR_\aexec = \WR_\mkvs \).
    Let \( \txidset'_\rd = \lfpTx[\mkvs,\vi,\SO_{\mkvs} \cup \WR_{\mkvs}] \). 
    This means that \( \aexec' = \extend[\aexec, \txid_\cl^n, \fp, \lfpTx[\mkvs, \vi, \SO_{\mkvs} \cup \WR_{\mkvs}] \cup \txidset_\rd ] \).
    Let assume \( \txid \toEDGE{\SO_{\mkvs} \cup \WR_{\mkvs}} \txid' \) and \( \txid' \in \lfpTx[\mkvs, \vi, \SO_{\mkvs} \cup \WR_{\mkvs}] \cup \txidset_\rd \).
    We have two possible cases:
    \begin{itemize}
        \item If \( \txid' \in \lfpTx[\mkvs, \vi, \SO_{\mkvs} \cup \WR_{\mkvs}] \), by  \cref{thm:view-vis-relation}, we know \( \txid \in \lfpTx[\mkvs, \vi, \SO_{\mkvs} \cup \WR_{\mkvs}] \).
        \item If \( \txid' \in \txidset_\rd \), there are two cases:
        \begin{itemize}
            \item \( \txid' \in  \left( \bigcup_{\Set{\txid_{\cl}^{n} \in \txidset_{\aexec} }[ n \in \Nat ]} \VIS_{\aexec}^{-1}(\txid^n_\cl) \right) \).
                By the property of \( \aexec \) (before update) that \( \SO \cup \WR_\aexec \in \VIS_\aexec \), it is known that \( \txid \in \left( \bigcup_{\Set{\txid_{\cl}^{n} \in \txidset_{\aexec} }[ n \in \Nat ]} \VIS_{\aexec}^{-1}(\txid^n_\cl) \right) \), that is, \( \txid \in \Tx[\mkvs,\vi] \cup \txidset_\rd\).

            \item \( \txid' \in  \left( \bigcup_{\Set{\txid_{\cl}^{n} \in \txidset_{\aexec} }[ n \in \Nat ]} \SO_{\aexec}^{-1}(\txid^n_\cl) \right) \).
                Then we know \( \txid \in (\SO \cup \WR_\aexec)^{-1} \left( \bigcup_{\Set{\txid_{\cl}^{n} \in \txidset_{\aexec} }[ n \in \Nat ]} \SO_{\aexec}^{-1}(\txid^n_\cl) \right) \).
                By the property of \( \aexec \) (before update) that \( \SO \cup \WR_\aexec \in \VIS_\aexec \),
                it follows:
                \begin{align*}
                    \txid & \in VIS_\aexec^{-1} \left( \bigcup_{\Set{\txid_{\cl}^{n} \in \txidset_{\aexec} }[ n \in \Nat ]} \SO_{\aexec}^{-1}(\txid^n_\cl) \right) \\
                          & = \left( \bigcup_{\Set{\txid_{\cl}^{n} \in \txidset_{\aexec} }[ n \in \Nat ]} \VIS_{\aexec}^{-1}(\txid^n_\cl) \right)  \\
                          & = \Tx[\mkvs,\vi] \cup \txidset_\rd
                \end{align*}
                
        \end{itemize}
    \end{itemize}
\item Finally the new abstract execution preserves the invariant \( I_1 \) and \( I_2 \) 
because  \( \CC \) satisfies \( \MW \) and \( \RYW \).
The proofs are the same as those in \cref{sec:sound-complete-mr} and \cref{sec:sound-complete-ryw}.

\end{itemize}

The execution test $\ET_\CC$ is complete with respect to the axiomatic definition 
\( (\RP_{\LWW}, \Set{\lambda \aexec.  (\SO \cup \WR_{\mkvs})^{+} ; \VIS_{\aexec}, \lambda \aexec \ldotp \SO_\aexec })\).
Assume i-\emph{th} transaction \( \txid_i \) in the arbitrary order,
and let view \( \vi_{i} = \getView[\aexec, \VIS^{-1}_{\aexec}(\txid_{i})] \).
We pick final view as \( \vi'_{i} = \getView[\aexec, (\AR^{-1}_{\aexec})\rflx(\txid_{i}) \cap \VIS_{\aexec}^{-1}(\txid'_i)] \),
if \( \txid'_i = \min_{\SO}\Set{\txid'}[\txid_i \toEDGE{\SO} \txid' ]\) is defined,
otherwise  \( \vi'_{i} = \getView[\aexec, (\AR^{-1}_{\aexec})\rflx(\txid_{i})]\).
Let the \( \mkvs = \mkvs_{\cut[\aexec, i-1]} \).
Now we prove the three parts separately.
\begin{itemize}
    \item \( \MR \).  By \cref{prop:cc-vis} since 
    \( \VIS_\aexec  = (\WR_\aexec \cup \SO_\aexec \cup \rel)^{+} \) for some \( \rel \)
    so it follows \(  \VIS_\aexec ; \SO_\aexec \subseteq \VIS_{\aexec} \),
    and the proof is as the same proof as in \cref{sec:sound-complete-mr}.
    \item \( \RYW \). For \( \RYW \), since \( \SO_\aexec \subseteq \VIS_\aexec \), 
    the proof is as the same proof as in \cref{sec:sound-complete-ryw}.
    \item \( \allowed[\WR_\mkvs \cup \SO]\). It is derived from \cref{thm:view-vis-relation}
\end{itemize}
