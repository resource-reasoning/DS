\subsection{Causal Consistency \( \CC \)}
\label{sec:sound-complete-cc}

The wildly used definition on abstract executions for causal consistency is that \( \VIS \) is transitive.
Yet it is for the sack of elegant definition, while there is a minimum visibility relation given by \( (\WR_\aexec \cup \SO_\aexec)^{+} ; \VIS_\aexec \subseteq \VIS_\aexec \) (\cref{lem:aexec-spec-cc}).

\begin{lemma}
    \label{lem:aexec-spec-cc}
    For any abstract execution \( \aexec \) under last-write-win, if it satisfies the following:
    \[
        (\WR_\aexec \cup \SO_\aexec)^{+} ; \VIS_\aexec \subseteq \VIS_\aexec \quad \SO_\aexec \subseteq \VIS_\aexec
    \]
    There exists a new abstract execution \( \aexec' \) where \( \txidset_\aexec = \txidset_{\aexec'} \), \( \AR_\aexec = \AR_{\aexec'} \),
    \( \VIS_{\aexec'} ; \VIS_{\aexec'} \subseteq \VIS_{\aexec'} \), and
    under last-write-win \( \TtoOp{T}_{\aexec}(\txid) = \TtoOp{T}_{\aexec'}(\txid) \) for all transactions \( \txid \).
\end{lemma}
\begin{proof}
    To recall, the write-read relation under a key \( \WR(\aexec, \key) \) is defined as 
    \( \WR(\aexec, \key) \defeq \Setcon{ (\txid, \txid') }{ \exsts{\val} (\otW, \key, \val) \in_\aexec \txid \land (\otR, \key, \val) \in_\aexec \txid' \land \txid = \max_\AR(\VIS^{-1}(\txid')) }\).
    Given an \( \aexec \) that satisfies the following
    \[
        (\WR_\aexec \cup \SO_\aexec )^{+} ; \VIS_\aexec \subseteq \VIS_\aexec \quad \SO_\aexec \subseteq \VIS_\aexec
    \]
    we erase some visibility relation for each transaction following the order of arbitration \( \AR \) until the visibility is transitive.
    Assume the i-\emph{th} transaction \( \txid_i \)  with respect to the arbitration order.
    Let \( R_i \) denote a new visibility for transaction \( \txid_i \) such that
    \( R_i\projection{2} = \Set{\txid_i}\)
    and the visibility relation before (including) \( \txid_i \) is transitive.
    Let \( \aexec_i = \mkvs_{\cut[\aexec, i]} \) and \( \VIS_i = \bigcup_{0 \leq k \leq i} R_i \).
    For each step, says i-\emph{th} step, we  preserve the following:
    \begin{gather}
        \VIS_i ; \VIS_i \subseteq \VIS_i \label{equ:vis-i-transitive} \\
        \fora{\txid} (\txid,\txid_i) \in R_i \implies (\txid, \txid_i) \in (\WR_i \cup \SO_i)^{+}
        \label{equ:last-read-correct}
    \end{gather}
    
    \begin{itemize}
    \item \caseB{\( i = 1 \) and \( R_1 = \emptyset \)}
    Assume it is from client \( \cl \).
    There is no transaction committed before, so \( \VIS_1 = \emptyset \) and \( \VIS_1 ; \VIS_1 \subseteq \VIS_1 \) as \cref{equ:vis-i-transitive}.

    \item \caseI{i-\emph{th} step}
    Suppose the (i-1)-\emph{th} step satisfies \cref{equ:vis-i-transitive} and \cref{equ:last-read-correct}.
    Let consider i-\emph{th} step and the transaction \( \txid_i \).
    Initially we take \( R_i \) as empty set.
    We first extend \( R_i \) by closing with respect to \( \WR_i \)
    and prove that it does not affect any read from the transaction \( \txid_i \).
    Then we will do the same for \( \SO_i \).
    \begin{itemize}
        \item \( \WR_i\). For any read \( (\otR, \key, \val ) \in \txid_i \),
        there must be a transaction \( \txid_j \) that \( \txid_j \toEDGE{\WR(\aexec_i,\key), \AR} \txid_i \) and \( j < i \).
        We include \( (\txid_j, \txid_i) \in R_i \).
        Let consider all the visible transactions of \( \txid_j \).
        Assume a transaction \( \txid' \in \VIS_{i-1}^{-1}(\txid_j) \), 
        thus \( \txid' \in \VIS_{j}^{-1}(\txid_j) = R_j^{-1}(\txid_j) \).
        It is safe to include \( (\txid', \txid_i) \in R_i \) without affecting the read result,
        because those transaction \( \txid' \) is already visible for \( \txid_i \) in the abstract execution \( \aexec \):
        by \cref{equ:last-read-correct} we know \( R_j \subseteq (\WR_j \cup \SO_j)^{+} \subseteq (\WR_\aexec \cup \SO_\aexec)^{+}\),
        and by the definition of \( \WR(\aexec_i,\key) \) we know \( \WR(\aexec_i,\key) \subseteq \VIS_\aexec\).

        \item Given \( \SO_\aexec \subseteq \VIS_\aexec \), we include \( (\txid_j,\txid_i) \) for some \( \txid_j \)
        such that \( \txid_j \toEDGE{\SO_\aexec} \txid_i\).
        For the similar reason as \( \WR \),
        it is safe to includes all the visible transactions \( \txid' \) for \( \txid_j \), \ie \( \txid' \in R_j^{-1}\).
        \end{itemize}
        
    By the construction, both \cref{equ:vis-i-transitive} and \cref{equ:last-read-correct} are preserved. 
    Thus we have the proof.
    \end{itemize}
\end{proof}

By \cref{lem:aexec-spec-cc}, the execution test $\ET_\CC$ is sound with respect to the axiomatic definition 
\( (\RP_{\LWW}, \Set{\lambda \aexec. ( \SO_{\aexec} \cup \WR_{\aexec} )^{+} ; \VIS_{\aexec}, \lambda \aexec \ldotp \SO_\aexec })\).
We pick an invariant for the \( \ET_\CC \) as the union of those for \( \MRd\) and \( \RYW \) shown in the following:
\[  
\begin{rclarray}
    I_1(\aexec, \cl) & = & \left( \bigcup_{\Setcon{\txid_{\cl}^{n} \in \txidset_{\aexec} }{ n \in \Nat }} \VIS_{\aexec}^{-1}(\txid^n_\cl) \right) \setminus \txidset_\rd \\
    I_2(\aexec, \cl) & = & \left( \bigcup_{\Setcon{\txid_{\cl}^{n} \in \txidset_{\aexec} }{ n \in \Nat }} (\SO_{\aexec}^{-1})?(\txid^n_\cl) \right) \setminus \txidset_\rd
\end{rclarray}
\]
where \( \txidset_\rd \) is all the read-only transactions included in both 
\( \left( \bigcup_{\Setcon{\txid_{\cl}^{n} \in \txidset_{\aexec} }{ n \in \Nat }} \VIS_{\aexec}^{-1}(\txid^n_\cl) \right)\) 
and \( \left( \bigcup_{\Setcon{\txid_{\cl}^{n} \in \txidset_{\aexec} }{ n \in \Nat }} (\SO_{\aexec}^{-1})?(\txid^n_\cl) \right) \).
Assume a kv-store $\mkvs$, an initial and a final view $\vi, \vi'$  a fingerprint $\fp$ 
such that $\ET_{\CC} \vdash (\mkvs, \vi) \triangleright \fp: \vi'$. 
Also choose an arbitrary $\cl$, a transaction identifier $\txid_\cl^n \in \nextTxid(\mkvs, \cl)$, 
and an abstract execution $\aexec$ such that $\mkvs_{\aexec} = \mkvs$ and 
\( I_1(\aexec, \cl) \cup I_2(\aexec, \cl) \subseteq \Tx(\mkvs, \vi) \).
Let a new abstract execution \( \aexec' = \extend(\aexec, \txid_\cl^n, \fp, \Tx(\mkvs, \vi) \cup \txidset_\rd) \).
We are about to prove there exists an extra set of read-only transactions \( \txidset'_\rd \) such that:
\begin{gather}
    \fora{\txid} (\txid, \txid_\cl^n) \in \SO_{\aexec'} \implies \txid \in \Tx(\mkvs, \vi) \cup \txidset_\rd \cup \txidset'_\rd \label{equ:cc-sound-update-so}\\
    \fora{\txid} (\txid, \txid_\cl^n) \in ( \SO_{\aexec'} \cup \WR_{\aexec'} )^{+} ; \VIS_{\aexec'} \implies \txid \in \Tx(\mkvs, \vi) \cup \txidset_\rd \cup \txidset'_\rd \label{equ:cc-sound-update-visvis}\\
    I_1(\aexec',\cl) \cup I_2(\aexec',\cl) \subseteq \Tx(\mkvs_{\aexec'}, \vi') \label{equ:cc-sound-inv} 
\end{gather}

\begin{itemize}
\item The invariant \( I_2 \) implies \cref{equ:cc-sound-update-so} as the same as \( \RYW \) in \cref{sec:sound-complete-ryw}.
\item To prove \cref{equ:cc-sound-update-visvis}, let \( \txidset'_\rd = \emptyset \) initially,
and more read-only transactions will be added in \( \txidset'_\rd \) until the \cref{equ:cc-sound-update-visvis} holds.
Assume transaction \( \txid \) such that \( ((\txid, \txid_\cl^n) \in ( \SO_{\aexec'} \cup \WR_{\aexec'} )^{+} ; \VIS_{\aexec'}) \).
That is, there exists some transaction \( \txid' \) such that
\( \txid \toEDGE{( \SO_{\aexec'} \cup \WR_{\aexec'} )^+}  \txid' \toEDGE{\VIS_{\aexec'}} \txid_\cl^n\).
We consider two cases for \( \txid' \): \( \txid' \) is also visible by previous transactions from the same client; 
or \( \txid' \) is a newly visible transaction for the client.
\begin{itemize}
    \item if \( \txid' \) is also visible by previous transactions from the same client, it means \( \txid' \toEDGE{\VIS_{\aexec'}} \txid_\cl^m \) for some \( m < n \).
    The edge already exists before, therefore \( \txid' \toEDGE{\VIS_\aexec} \txid_\cl^m \).
    Since \( \txid \toEDGE{( \SO_{\aexec} \cup \WR_{\aexec} )^{+}} \txid' \) and \( ( \SO_{\aexec} \cup \WR_{\aexec} )^{+} ; \VIS_{\aexec} \subseteq \VIS_{\aexec} \),
    we know \( \txid \toEDGE{ \VIS_{\aexec} }  \txid_\cl^m  \).
    Because of \( I_1 \) and \( \txid_\cl^m \toEDGE{\SO} \txid_\cl^n \), then \( \txid \in I_1 \cup \txidset_\rd \subseteq \Tx(\mkvs, \vi) \cup \txidset_\rd \).
    
    \item if \( \txid' \) is a newly visible transaction for the client \( \cl \),
    it suffices to prove \(\txid \in \Tx(\mkvs, \vi) \cup \txidset_\rd \cup \txidset'_\rd \) 
    for some \( \txid \toEDGE{ ( \SO_{\aexec'} \cup \WR_{\aexec'} ) ; \VIS_{\aexec'} } \txid_\cl^n  \)%
    \footnote{For two relation \( R_1, R_2\), \( R_1^* ; R_2 \subseteq R_2 \iff R_1 ; R_2 \subseteq R_2 \) }.
    Since \( \txid' \toEDGE{\VIS_{\aexec'}} \txid_\cl^n \) so \( \txid' \in Tx(\mkvs, \vi) \cup \txidset_\rd \cup \txidset'_\rd\).
    More specifically, \( \txid' \) is not visible for the client before, we know \( \txid' \in Tx(\mkvs, \vi) \cup \txidset'_\rd\).
    We perform case analysis if \( \txid' \) has write.
    \begin{itemize}

        \item If \( \txid' \) writes to some keys, \ie \( \txid' \in Tx(\mkvs, \vi) \), 
        because  \( \CC \) satisfies \( \MW \) and \( \WFR \),
        and by the execution tests for \( \MW \) and \( \WFR \) (the proofs follows \cref{sec:sound-complete-mw} and \cref{sec:sound-complete-wfr}),
        the \( \txid \) is either already in \( \Tx(\mkvs, \vi) \), 
        or \( \txid \) is a read-only and we include it in \( \txidset'_\rd \).

        \item If \( \txid' \) is a read-only transaction,
        given that \( \txidset'_\rd \) initially is empty set,
        we know there exists a third transaction \( \txid'' \) that writes to some keys
        and it satisfies \( \txid \toEDGE{( \SO_{\aexec'} \cup \WR_{\aexec'} )}  \txid' \toEDGE{( \SO_{\aexec'} \cup \WR_{\aexec'} )}  \txid'' \toEDGE{\VIS_{\aexec'}} \txid_\cl^n \).
        Since \( \txid'' \) has a write, it means 
        \( \txid \toEDGE{( \SO_{\aexec'} \cup \WR_{\aexec'} )}  \txid' \toEDGE{( \SO_{\aexec'} )}  \txid'' \toEDGE{\VIS_{\aexec'}} \txid_\cl^n \).
        \begin{itemize}
            \item if \( \txid \toEDGE{\WR_{\aexec'} }  \txid' \toEDGE{ \SO_{\aexec'} }  \txid'' \toEDGE{\VIS_{\aexec'}} \txid_\cl^n \),
            this is exactly \( \WFR \).
            Therefore,the \( \txid \) is either already in \( \Tx(\mkvs, \vi) \), 
            or \( \txid \) is a read-only and we include it in \( \txidset'_\rd \).

            \item if \( \txid \toEDGE{\SO_{\aexec'} }  \txid' \toEDGE{ \SO_{\aexec'} }  \txid'' \toEDGE{\VIS_{\aexec'}} \txid_\cl^n \),
            because \( \SO \) is transitive, we have \( \txid \toEDGE{\SO_{\aexec'}} \txid'' \toEDGE{\VIS_{\aexec'}} \txid_\cl^n \).
            By previous case we already know \( \txid \in \Tx(\mkvs, \vi) \cup \txidset'_\rd \).
        \end{itemize}
    \end{itemize}
    \end{itemize}
    \item Finally the new abstract execution preserves the invariant \( I_1 \) and \( I_2 \) 
    because  \( \CC \) satisfies \( \MW \) and \( \RYW \).
    The proofs are the same as those in \cref{sec:sound-complete-mr} and \cref{sec:sound-complete-ryw}.

\end{itemize}

The execution test $\ET_\CC$ is complete with respect to the axiomatic definition 
\( (\RP_{\LWW}, \Set{\lambda \aexec. \VIS_{\aexec} ; \VIS_{\aexec}, \lambda \aexec \ldotp \SO_\aexec })\).

For \( \MR \), since \(  \VIS_\aexec ; \SO_\aexec \subseteq  \VIS_\aexec ; \VIS_{\aexec} \subseteq \VIS_\aexec \),
the proof is as the same proof as in \cref{sec:sound-complete-mr}.
For \( \MW \), since \( \SO_\aexec ; \VIS_\aexec \subseteq  \VIS_\aexec ; \VIS_{\aexec} \subseteq \VIS_\aexec \),
the proof is as the same proof as in \cref{sec:sound-complete-mw}.
For \( \RYW \), since \( \SO_\aexec \subseteq \VIS_\aexec \),
the proof is as the same proof as in \cref{sec:sound-complete-ryw}.
For \( \WFR \), since \( \WR_\aexec ; \SO_\aexec? ; \VIS_\aexec \subseteq \VIS_\aexec ; \VIS_{\aexec}? ; \VIS_\aexec \subseteq \VIS_\aexec \),
the proof is as the same proof as in \cref{sec:sound-complete-wfr}.
