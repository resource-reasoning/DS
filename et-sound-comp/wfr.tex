\subsection{Write Following Read \( \WFR \) }
\label{sec:sound-complete-wfr}

The write-read relation  on \( \aexec \) is defined as the following:
\[
    \WR(\aexec, \key) \defeq \Set{ (\txid, \txid') }[ \exsts{\val} (\otW, \key, \val) \in_\aexec \txid \land (\otR, \key, \val) \in_\aexec \txid' \land \txid = \max_\AR(\VIS^{-1}(\txid')) ]
\]
The notation \( \WR_\aexec \) is defined as \( \WR_\aexec \defeq \bigcup_{\key \in \Keys} \WR(\aexec, \key) \).
Note that for a kv-store \( \mkvs \) such that \( \mkvs = \mkvs_\aexec \),
by the definition of  \(  \mkvs = \mkvs_\aexec \), 
the following holds:
\[
    \WR_\aexec = \Set{(\txid, \txid')}[\exsts{\key, i } \mkvs(\key, i) = (\stub, \txid, \txid'\cup \stub)]
\]
Note that such \( \WR_\aexec \) coincides with \( \WR_\Gr \) and \( \WR_\mkvs \).

The execution test $\ET_\WFR$ is sound with respect to the axiomatic definition 
\( (\RP_{\LWW}, \Set{\lambda \aexec. \WR_\aexec ; \SO_{\aexec}\rflx ; \VIS_{\aexec} })\).
We pick the invariant as \( I( \aexec, \cl ) = \emptyset \), given the fact of no constraint on the view after update.
Assume a kv-store $\mkvs$, an initial and a final view $\vi, \vi'$  a fingerprint $\fp$ 
such that $\ET_{\WFR} \vdash (\mkvs, \vi) \csat \fp: (\mkvs', \vi')$. 
Also choose an arbitrary $\cl$, a transaction identifier $\txid \in \nextTxid(\mkvs, \cl)$, 
and an abstract execution $\aexec$ such that $\mkvs_{\aexec} = \mkvs$ and 
\( I(\aexec, \cl) =  \emptyset \subseteq \Tx[\mkvs, \vi] \).
Note that since the invariant is empty set, it remains to prove there is a set of read-only transactions \( \txidset_\rd \) such that
Let \( \aexec' = \extend[\aexec, \txid, \Tx[\mkvs, \vi] \cup \txidset_\rd, \fp] \) and
\[
    \begin{array}{@{}l@{}}
        \fora{ \txid' } 
        (\txid' ,\txid)  \in \WR_{\aexec'} ; \SO_{\aexec'}\rflx ; \VIS_{\aexec'} 
        \implies \txid' \in \Tx[\mkvs, \vi]
    \end{array}
\]
which can be derived from \cref{thm:view-vis-relation}.

The execution test $\ET_\WFR$ is complete with respect to the axiomatic definition 
\( (\RP_{\LWW}, \Set{\lambda \aexec. \WR(\aexec', \key) ; \SO_{\aexec'}\rflx ; \VIS_{\aexec'} })\).
Assume i-\emph{th} transaction \( \txid_i \) in the arbitrary order,
and let view \( \vi_{i} = \getView[\mkvs_{\cut[\aexec, i-1]}, \VIS^{-1}_{\aexec}(\txid_{i})] \).
We also pick any final view such that \( \vi'_{i} \subseteq \getView[\mkvs_{\cut[\aexec, i]}, (\AR^{-1}_{\aexec})\rflx(\txid_{i})] \).
Note that there is nothing to prove for \( \vi'_i \),
so it is sufficient to prove the following:
\[
    \vi_i = \getView[\mkvs_{\cut[\aexec, i-1]}, \lfpTx[\mkvs_{\cut[\aexec, i-1]},\vi,\WR_{\mkvs_{\cut[\aexec, i-1]}}; \SO]]
\]
which can be derived from \cref{thm:view-vis-relation}.
