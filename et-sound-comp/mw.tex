\subsection{Monotonic Write \( \MW \)}
\label{sec:sound-complete-mw}

The execution test $\ET_\MW$ is sound with respect to the axiomatic definition 
\[(\RP_{\LWW}, \Set{\lambda \aexec. \SO_{\aexec} ; \VIS_{\aexec} })\]
We pick the invariant as empty set given the fact of no constraint on the view after update:
\[ 
    I( \aexec, \cl ) = \emptyset 
\]
Assume a kv-store $\mkvs$, an initial and a final view $\vi, \vi'$  a fingerprint $\fp$ 
such that $\ET_{\MW} \vdash (\mkvs, \vi) \csat \fp: (\mkvs',\vi')$. 
Also choose an arbitrary $\cl$, a transaction identifier $\txid \in \nextTxid(\mkvs, \cl)$, 
and an abstract execution $\aexec$ such that $\mkvs_{\aexec} = \mkvs$ and 
\( I(\aexec, \cl) =  \emptyset \subseteq \Tx[\mkvs, \vi] \).
Note that since the invariant  is empty set, it remains to prove that there exists a set of read-only transactions \( \txidset_\rd \) such that
\( \aexec' = \extend(\aexec, \txid, \Tx[\mkvs, \vi] \cup \txidset_\rd, \fp ) \) and:
\[
    \begin{array}{@{}l@{}}
        \fora{ \txid' }  (\txid' ,\txid)  \in \SO_{\aexec'} ; \VIS_{\aexec'}
        \implies \txid' \in \Tx[\mkvs, \vi] \cup \txidset_\rd
    \end{array}
\]
which can be derived from \cref{thm:view-vis-relation}.

The execution test $\ET_{\MW}$ is complete with respect to 
the axiomatic definition
\[(\RP_{\LWW}, \Set{\lambda \aexec.(\SO_{\aexec} ; \VIS_{\aexec})})\]
Let $\aexec$ be an abstract execution that satisfies the definition
$\CMa(\RP_{\LWW}, \Set{\lambda \aexec.(\SO_{\aexec} ; \VIS_{\aexec})})$, 
and consider a transaction $\txid \in \txidset_{\aexec}$. 
Let \( \mkvs = \mkvs_\aexec \).
Assume i-\emph{th} transaction \( \txid_i \) in the arbitrary order,
and let view \( \vi_{i} = \getView[\mkvs, \VIS^{-1}_{\aexec}(\txid_{i})] \).
We also pick any final view such that \( \vi'_{i} \subseteq \getView[\mkvs, (\AR^{-1}_{\aexec})\rflx(\txid_{i})] \).
It suffices to prove \( \ET_\MW \vdash (\mkvs_{\cut[\aexec, i-1]}, \vi_i ) \csat  \TtoOp{T}_{\aexec}(\txid_{i}) : (\mkvs_{\cut[\aexec, i-1]}, \vi'_{i}) \).
It means to prove the following:
\[
    \vi_i = \getView[\mkvs_{\cut[\aexec, i-1]}, \lfpTx[\mkvs_{\cut[\aexec, i-1]},\vi,\SO]]
\]
which can be derived from \cref{thm:view-vis-relation}.
