\subsection{Monotonic Write \( \MW \)}
\label{sec:sound-complete-mw}

The execution test $\ET_\MW$ is sound with respect to the axiomatic definition 
$(\RP_{\LWW}, \Set{\lambda \aexec. \SO_{\aexec} ; \VIS_{\aexec} })$.
We pick the invariant as empty set given the fact of no constraint on the view after update:
\[ 
    I( \aexec, \cl ) = \emptyset 
\]
Assume a kv-store $\mkvs$, an initial and a final view $\vi, \vi'$  a fingerprint $\fp$ 
such that $\ET_{\MW} \vdash (\mkvs, \vi) \csat \fp: (\mkvs',\vi')$. 
Also choose an arbitrary $\cl$, a transaction identifier $\txid \in \nextTxid(\mkvs, \cl)$, 
and an abstract execution $\aexec$ such that $\mkvs_{\aexec} = \mkvs$ and 
\( I(\aexec, \cl) =  \emptyset \subseteq \Tx[\mkvs, \vi] \).
Let \( \aexec' = \extend(\aexec, \txid, \Tx[\mkvs, \vi] \cup \txidset_\rd, \fp ) \).
Note that since the invariant  is empty set, it remains to prove that there exists a set of read-only transactions \( \txidset_\rd \) such that:
\[
    \begin{array}{@{}l@{}}
        \fora{ \txid' }  (\txid' ,\txid)  \in \SO_{\aexec'} ; \VIS_{\aexec'}
        \implies \txid' \in \Tx[\mkvs, \vi] \cup \txidset_\rd
    \end{array}
\]
which can be derived from \cref{thm:view-vis-relation}.
%Initially we take \( \txidset_\rd = \emptyset \), 
%and by closing the \( \Tx[\mkvs, \vi] \) with respect to the relation \( \SO_{\aexec'} ; \VIS_{\aexec'} \),
%we will add more read-only transactions into the set \( \txidset_\rd\).
%Suppose \( (\txid' ,\txid)  \in \SO_{\aexec'} ; \VIS_{\aexec'} \), 
%that is, \( \txid' \toEDGE{\SO_{\aexec'}} \txid'' \toEDGE{\VIS_{\aexec'}} \txid \).
%We perform a case analysis on if \( \txid'' \) has write:
%\begin{itemize}
%\item If the transaction \( \txid'' \) writes to a key.
%For the new abstract execution \( \aexec' \), the visible transactions for \( \txid \) must come from \( \Tx[\mkvs, \vi] \cup \txidset_\rd \).
%It means \( \txid'' \in \Tx[\mkvs, \vi] \cup \txidset_\rd  \).
%Then given that \( \txid'' \) is not a read-only transaction, we have \( \txid'' \in \Tx[\mkvs, \vi] \).
%Now there are two cases:
%\begin{itemize}
    %\item if \( \txid' \) is a read-only transaction, we include \( \txid' \in \txidset_{\rd} \).
    %\item if \( \txid' \) has at least one write, it is easy to see \( \txid' \in \Tx[\mkvs, \vi] \) since \( j \in \vi(\key) \land \wtOf(\mkvs(\key', i)) \toEDGE{\SO\rflx} \wtOf(\mkvs(\key, j)) \implies i \in \vi(\key') \).
%\end{itemize}
%\item If the transaction \( \txid'' \in \txidset_\rd \) is a read-only transaction, 
%since \( \txidset_\rd \) is initial empty, there must exist a later transaction \( \txid''' \) from the same client that writes to a key,
%and such transaction \( \txid''' \) is included in \( \Tx[\mkvs, \vi] \):
%\[
    %\txid' \toEDGE{\SO_{\aexec'}} \txid'' 
    %\toEDGE{\SO_{\aexec'}} \txid''' \toEDGE{\VIS_{\aexec'}} \txid 
    %\land \txid''' \in \Tx[\mkvs,\vi]
%\]
%Since \( \SO \) is transitive, 
%therefore \( \txid' \toEDGE{\SO_{\aexec'}} \txid''' \toEDGE{\VIS_{\aexec'}} \txid \),
%which we have already proven \( \txid' \in \Tx[\mkvs, \vi] \) or we will include \( \txid' \) in \( \txidset_\rd \).
%Since there are finite transactions from a client in a trace, there must exist a \( \txidset_\rd \) in the end.
%\end{itemize}

The execution test $\ET_{\MW}$ is complete with respect to 
the axiomatic definition $(\RP_{\LWW}, \Set{\lambda \aexec.(\SO_{\aexec} ; \VIS_{\aexec})})$. 
Let $\aexec$ be an abstract execution that satisfies the definition
$\CMa(\RP_{\LWW}, \Set{\lambda \aexec.(\SO_{\aexec} ; \VIS_{\aexec})})$, 
and consider a transaction $\txid \in \txidset_{\aexec}$. 
Assume i-\emph{th} transaction \( \txid_i \) in the arbitrary order,
and let view \( \vi_{i} = \getView[\aexec, \VIS^{-1}_{\aexec}(\txid_{i})] \).
We also pick any final view such that \( \vi'_{i} \subseteq \getView[\aexec, (\AR^{-1}_{\aexec})\rflx(\txid_{i})] \).
It suffices to prove \( \ET_\MW \vdash (\mkvs_{\cut[\aexec, i-1]}, \vi_i ) \csat  \TtoOp{T}_{\aexec}(\txid_{i}) : (\mkvs_{\cut[\aexec, i-1]}, \vi'_{i}) \).
It means to prove the follows:
\begin{equation}
\label{equ:mw-complete}
\begin{array}{@{}l@{}}
    \fora{j,m,\key, \key' } j \in \vi(\key)  
    \land \wtOf(\mkvs_{\cut[\aexec, i-1]}(\key', m)) \toEDGE{\SO\rflx} \wtOf(\mkvs_{\cut[\aexec, i-1]}(\key, j))  
    \implies m \in \vi(\key')
\end{array}
\end{equation}
which can be derived from \cref{thm:view-vis-relation}.
%Assume \( j \) and \( \key' \) such that \( j \in \vi(\key')\), which means \( \wtOf(\mkvs_{\cut[\aexec, i-1]}(\key', j)) \in \VIS^{-1}_{\aexec}(\txid_{i}) \).
%Now let consider transaction \( \txid \) that commits before \( \txid \) from the same session, \ie \( \txid \toEDGE{\SO} \wtOf(\mkvs_{\cut[\aexec, i-1]}(\key, j)) \).
%By the constraint \( \lambda \aexec.(\SO_{\aexec} ; \VIS_{\aexec}) \), the transaction \( \txid \in \VIS^{-1}_{\aexec}(\txid_{i}) \).
%It means that in the kv-store \(  \mkvs_{\cut[\aexec, i-1]} \) every version written by \( \txid =  \wtOf(\mkvs_{\cut[\aexec, i-1]}(\key', m)) \) should be included in the view \( m \in \vi_i(\key') \).
%Thus we have the proof of \cref{equ:mw-complete}.
