\subsection{Consistency Prefix \( \CP \) }
\label{sec:sound-complete-cp}

Given abstract execution \( \aexec \), we define read-write read-write relation:
\[
    \RW(\aexec,\key) \defeq \Set{(\txid, \txid')}[\txid \toEDGE{\AR_\aexec} \txid' \land (\otR,\key, \stub ) \in \txid \land (\otW,\key, \stub ) \in \txid' ] 
\]
It is easy to see \( \RW(\aexec,\key) \)  can be derived from \( \WW(\aexec,\key) \) and \( \WR(\aexec, \key ) \) as the following:
\[
    \RW(\aexec,\key) = \Set{(\txid, \txid')}[ \exsts{\txid'' } (\txid'', \txid) \in \WR(\aexec, \key) \land (\txid'', \txid') \in \WW(\aexec, \key) ]
\]
Then, the notation \( \RW_\aexec \defeq \bigcup_{\key \in \Keys} \RW(\aexec, \key) \).
Note that for a kv-store \( \mkvs \) such that \( \mkvs = \mkvs_\aexec \),
by the definition of  \(  \mkvs = \mkvs_\aexec \), 
the following holds:
\[
    \RW_\aexec = \Set{(\txid, \txid')}[\exsts{\key, i,j } \txid \in \rsOf(\mkvs(\key, i)) \land \txid' = \wtOf(\mkvs(\key, j)) \land i < j]
\]
The \( \RW_\aexec \) also coincides with \( \RW_\Gr \) and \( \RW_\mkvs \).


An abstract execution \( \aexec \) satisfies consistency prefix (\(\CP\)), 
if it satisfies \( \AR_\aexec ; \VIS_\aexec \subseteq \VIS_\aexec \) and \( \SO_\aexec \subseteq \VIS_\aexec \).
Given the definition, there is a corresponding definition on dependency graph by solve the following inequalities:
\[
    \begin{array}{@{}l@{}}
        \WR \subseteq \VIS \\
        \WW \subseteq \AR \\
        \VIS \subseteq \AR \\
        \VIS ; \RW \subseteq \AR \\
        \AR ; \AR \subseteq \AR  \\
        \SO \subseteq \VIS \\
        \AR ; \VIS \subseteq \VIS
    \end{array}
\]
By solving the inequalities the visibility and arbitration relations are:
\begin{align*}
    \AR  & \defeq \left( (\SO \cup \WR ) ; \RW\rflx \cup \WW \cup R \right)^+ \\
    \VIS & \defeq \left( (\SO \cup \WR ) ; \RW\rflx \cup \WW \cup R \right)^* ; (\SO \cup \WR )
\end{align*}
for some relation \( R \subseteq \AR \).
When \( R = \emptyset \), it is the smallest solution therefore the minimum visibility required.

\sx{A bit verbal}
\begin{lemma}
    \label{lem:cp-eauiv-spec}
    For any abstract execution \( \aexec \),
    if it satisfies 
    \[
        \left( (\SO \cup \WR ) ; \RW\rflx \cup \WW \right) ; \VIS_\aexec \subseteq \VIS_\aexec 
        \qquad \SO_\aexec \subseteq \VIS_\aexec
    \]
    then there exists a new \( \aexec' \) such that \( \txidset_\aexec = \txidset_{\aexec'} \), 
    under last-write-win \( \TtoOp{T}_{\aexec}(\txid) = \TtoOp{T}_{\aexec'}(\txid) \) for all transactions \( \txid \),
    and the relations satisfy the following:
    \[ 
        \AR_{\aexec'} ; \VIS_{\aexec'} \subseteq \VIS_{\aexec'}  \qquad \SO_{\aexec'} \subseteq \VIS_{\aexec'}
    \]
    and vice versa.
\end{lemma}
\begin{proof}
Assume abstract execution \( \aexec' \) that satisfies \( \AR_{\aexec'} ; \VIS_{\aexec'} \subseteq \VIS_{\aexec'} \)
and  \( \SO_{\aexec'} \subseteq \VIS_{\aexec'} \).
We already show that:
\begin{align*}
    \AR_{\aexec'} & = \left( (\SO_\aexec \cup \WR_\aexec ) ; \RW_\aexec\rflx \cup \WW_\aexec \cup R \right)^+ \\
    \VIS_{\aexec'} & = \left( (\SO_\aexec \cup \WR_\aexec ) ; \RW_\aexec\rflx \cup \WW_\aexec \cup R \right)^* ; (\SO_\aexec \cup \WR_\aexec )
\end{align*}
for some relation \( R \subseteq \AR_{\aexec'} \).
If we take \( R  = \emptyset \), we have the proof for:
\[
        \SO \subseteq \VIS_\aexec \qquad 
        \left( (\SO_\aexec \cup \WR_\aexec ) ; \RW_\aexec\rflx \cup \WW_\aexec \right) ; \VIS_\aexec \subseteq \VIS_\aexec
\]
For another way, we pick the \( R \) that extends
\( \left( (\SO_\aexec \cup \WR_\aexec ) ; \RW_\aexec\rflx \cup \WW_\aexec \cup R \right)^+ \) 
to a total order.
\end{proof}

By \cref{lem:cp-eauiv-spec} to prove soundness and completeness of \( \ET_\CP \), it is sufficient to use the definition:
\[
    (\RP_{\LWW}, \Set{\lambda \aexec. \left( (\SO \cup \WR ) ; \RW\rflx \cup \WW \right) ; \VIS_\aexec, \lambda \aexec \ldotp \SO_\aexec }) 
\]

For the soundness, we pick the invariant as the following:
\begin{align*}
    I_1(\aexec, \cl) & = \left( \bigcup_{\Set{\txid_{\cl}^{i} \in \txidset_{\aexec} }[ i \in \Nat ]} \VIS_{\aexec}^{-1}(\txid^i_\cl) \right) \setminus \txidset_\rd \\
    I_2(\aexec, \cl) & = \left( \bigcup_{\Set{\txid_{\cl}^{i} \in \txidset_{\aexec} }[ i \in \Nat ]} (\SO_{\aexec}^{-1})\rflx(\txid^i_\cl) \right) \setminus \txidset_\rd
\end{align*}
where \( \txidset_\rd \) is all the read-only transactions included in both 
\[ \txidset_\rd \in \left( \bigcup_{\Set{\txid_{\cl}^{i} \in \txidset_{\aexec} }[ i \in \Nat ]} \VIS_{\aexec}^{-1}(\txid^i_\cl) \right)\] 
and \[ \txidset_\rd \in \left( \bigcup_{\Set{\txid_{\cl}^{i} \in \txidset_{\aexec} }[ i \in \Nat ]} (\SO_{\aexec}^{-1})\rflx(\txid^i_\cl) \right) \]
Assume a key-value store $\mkvs$, an initial and a final view $\vi, \vi'$  a fingerprint $\fp$ 
such that $\ET_{\CP} \vdash (\mkvs, \vi) \csat \fp: ( \mkvs',\vi')$. 
Also choose an arbitrary $\cl$, a transaction identifier $\txid_\cl^n \in \nextTxid(\mkvs, \cl)$, 
and an abstract execution $\aexec$ such that $\mkvs_{\aexec} = \mkvs$ and 
\( I_1(\aexec, \cl) \cup I_2(\aexec, \cl) \subseteq \Tx[\mkvs, \vi] \).
We are about to prove that there exists an extra set of read-only transaction \( \txidset'_\rd \) such that
the new abstract execution \( \aexec' = \extend[\aexec, \txid_\cl^n, \fp, \Tx[\mkvs, \vi] \cup \txidset_\rd \cup \txidset'_\rd] \) and
\begin{gather}
    \fora{\txid} (\txid, \txid_\cl^n) \in \SO_{\aexec'} \implies \txid \in \Tx[\mkvs, \vi] \cup \txidset_\rd \cup \txidset'_\rd \label{equ:cp-sound-update-so}\\
    \begin{array}{l}
        \fora{\txid} (\txid, \txid_\cl^n) \in \left( (\SO_{\aexec'} \cup \WR_{\aexec'} ) ; \RW_{\aexec'}\rflx \cup \WW_{\aexec'} \right) ; \VIS_{\aexec'} 
    \implies \txid \in \Tx[\mkvs, \vi] \cup \txidset_\rd \cup \txidset'_\rd 
    \end{array}
    \label{equ:cp-sound-update-arvis}\\
    I_1(\aexec',\cl) \cup I_2(\aexec',\cl) \subseteq \Tx[\mkvs_{\aexec'}, \vi'] \label{equ:cp-sound-inv} 
\end{gather}
\begin{itemize}
\item the invariant \( I_2 \) implies the \cref{equ:cp-sound-update-so} where the proof is the same as \( \RYW \) in \cref{sec:sound-complete-ryw}.

\item \cref{equ:cp-sound-update-arvis}.
    Note that 
    \begin{centermultline} 
        (\txid, \txid_\cl^n) \in \left( (\SO_{\aexec'} \cup \WR_{\aexec'} ) ; \RW_{\aexec'}\rflx \cup \WW_{\aexec'} \right) ; \VIS_{\aexec'} \\
        {} \implies (\txid, \txid_\cl^n) \in \left( (\SO_{\aexec} \cup \WR_{\aexec} ) ; \RW_{\aexec}\rflx \cup \WW_{\aexec} \right) ; \VIS_{\aexec'}
    \end{centermultline}
    Also, recall that \( \rel_\aexec = \rel_\mkvs \) for \( \rel \in \Set{\SO, \WR, \WW, \RW} \).
    Let \[ \txidset'_\rd = \lfpTx[\mkvs,\vi,(\SO_{\mkvs} \cup \WR_{\mkvs} ) ; \RW_{\mkvs}\rflx \cup \WW_{\mkvs} ] \]
    Let assume \( \txid \toEDGE{(\SO_{\mkvs} \cup \WR_{\mkvs} ) ; \RW_{\mkvs}\rflx \cup \WW_{\mkvs}} \txid' \) and \( \txid' \in \lfpTx[\mkvs, \vi, (\SO_{\mkvs} \cup \WR_{\mkvs} ) ; \RW_{\mkvs}\rflx \cup \WW_{\mkvs}] \cup \txidset_\rd \).
    We have two possible cases:
    \begin{itemize}
        \item If \( \txid' \in \lfpTx[\mkvs, \vi, (\SO_{\mkvs} \cup \WR_{\mkvs} ) ; \RW_{\mkvs}\rflx \cup \WW_{\mkvs}] \), by \cref{thm:view-vis-relation}, we know \[ \txid \in \lfpTx[\mkvs, \vi, (\SO_{\mkvs} \cup \WR_{\mkvs} ) ; \RW_{\mkvs}\rflx \cup \WW_{\mkvs}] \]
        \item If \( \txid' \in \txidset_\rd \), there are two cases:
        \begin{itemize}
            \item \( \txid' \in  \left( \bigcup_{\Set{\txid_{\cl}^{n} \in \txidset_{\aexec} }[ n \in \Nat ]} \VIS_{\aexec}^{-1}(\txid^n_\cl) \right) \).
                Note that \( \txid' \) is a read-only transaction,
                which means \( \txid \toEDGE{\SO_{\mkvs} \cup \WR_{\mkvs} } \txid' \).
                By the property of \( \aexec \) (before update) that \( ( \SO \cup \WR_\aexec ) ; \VIS_\aexec  \in \VIS_\aexec \), it is known that \( \txid \in \left( \bigcup_{\Set{\txid_{\cl}^{n} \in \txidset_{\aexec} }[ n \in \Nat ]} \VIS_{\aexec}^{-1}(\txid^n_\cl) \right) \), that is, \( \txid \in \Tx[\mkvs,\vi] \cup \txidset_\rd\).

            \item \( \txid' \in  \left( \bigcup_{\Set{\txid_{\cl}^{n} \in \txidset_{\aexec} }[ n \in \Nat ]} \SO_{\aexec}^{-1}(\txid^n_\cl) \right) \) and it is a read only transaction.
                Then we know \( \txid \in (\SO \cup \WR_\aexec)^{-1} \left( \bigcup_{\Set{\txid_{\cl}^{n} \in \txidset_{\aexec} }[ n \in \Nat ]} \SO_{\aexec}^{-1}(\txid^n_\cl) \right) \).
                By the property of \( \aexec \) (before update) that \( \SO \cup \WR_\aexec \in \VIS_\aexec \),
                it follows:
                \begin{align*}
                    \txid & \in VIS_\aexec^{-1} \left( \bigcup_{\Set{\txid_{\cl}^{n} \in \txidset_{\aexec} }[ n \in \Nat ]} \SO_{\aexec}^{-1}(\txid^n_\cl) \right) \\
                          & = \left( \bigcup_{\Set{\txid_{\cl}^{n} \in \txidset_{\aexec} }[ n \in \Nat ]} \VIS_{\aexec}^{-1}(\txid^n_\cl) \right)  \\
                          & = \Tx[\mkvs,\vi] \cup \txidset_\rd
                \end{align*}
                
        \end{itemize}
    \end{itemize}

\item Since \( \CP \) satisfies \( \RYW \) and \( \MR \), thus invariants \( I_1 \) and  \( I_2 \) are preserved after update.

\end{itemize}
    


The execution test $\ET_\CP$ is complete with respect to the axiomatic definition 
\[ (\RP_{\LWW}, \Set{\lambda \aexec.  \AR_\aexec ; \VIS_{\aexec}, \lambda \aexec \ldotp \SO_\aexec })\]
Assume i-\emph{th} transaction \( \txid_i \) in the arbitrary order,
and let view \( \vi_{i} = \getView[\aexec, \VIS^{-1}_{\aexec}(\txid_{i})] \).
We pick final view as \( \vi'_{i} = \getView[\aexec, (\AR^{-1}_{\aexec})\rflx(\txid_{i}) \cap \VIS_{\aexec}^{-1}(\txid'_i)] \),
if \( \txid'_i = \min_{\SO}\Set{\txid'}[\txid_i \toEDGE{\SO} \txid' ]\) is defined,
otherwise  \( \vi'_{i} = \getView[\aexec, (\AR^{-1}_{\aexec})\rflx(\txid_{i})]\).
Let the \( \mkvs = \mkvs_{\cut[\aexec, i-1]} \).
Now we prove the three parts separately.

\begin{itemize}
    \item \( \MR \).  By \cref{prop:cc-vis} since 
    \( \VIS_\aexec ; \SO_\aexec \subseteq \AR_\aexec ; \VIS_\aexec \subseteq \VIS_\aexec \)
    so it follows as in \cref{sec:sound-complete-mr}.
    \item \( \RYW \). For \( \RYW \), since \( \WR_\aexec ; \SO_\aexec ; \VIS_\aexec \subseteq \AR_\aexec ; \AR_\aexec ; \VIS_\aexec \subseteq \VIS_\aexec\), 
    the proof is as the same proof as in \cref{sec:sound-complete-ryw}.

    \item \( \allowed[(\SO ; \RW_{\mkvs}\rflx) \cup (\WR_{\mkvs} ; \RW_{\mkvs}\rflx) \cup \WW_\mkvs] \) can be derived 
    from \cref{thm:view-vis-relation} and \[ (\SO ; \RW_{\aexec}\rflx) \cup (\WR_{\aexec} ; \RW_{\aexec}\rflx) \cup \WW_\aexec ; \VIS_\aexec \subseteq \AR_\aexec ; \VIS_\aexec \subseteq \VIS_\aexec \]
\end{itemize}
