\subsection{Serialisability \( \SER \)}
\label{sec:sound-complete-ser}

The execution test $\ET_\UA$ is sound with respect to the axiomatic definition 
\[ 
    (\RP_{\LWW}, \Set{\lambda \aexec. \AR })
\]
We pick the invariant as \( I( \aexec, \cl ) = \emptyset \), given the fact of no constraint on the view after update.
Assume a kv-store $\hh$, an initial and a final view $\vi, \vi'$  a fingerprint $\opset$ 
such that $\ET_{\SER} \vdash (\hh, \vi) \csat \opset: (\hh',\vi')$. 
Also choose an arbitrary $\cl$, a transaction identifier $\txid \in \nextTxId(\hh, \cl)$, 
and an abstract execution $\aexec$ such that $\hh_{\aexec} = \hh$ and 
\( I(\aexec, \cl) =  \emptyset \subseteq \Tx(\hh, \vi) \).
Let \( \aexec' = \extend(\aexec, \txid, \Tx(\mkvs, \vi), \f ) \).
Note that since the invariant is empty set, it remains to prove there exists a set of read-only transactions \( \T_\rd \) such that:
\[
    \begin{array}{@{}l@{}}
        \fora{ \txid' } 
        \txid' \toEdge{\AR_{\aexec'}} \txid \implies \txid' \in \Tx(\mkvs, \vi) \cup \T_\rd
    \end{array}
\]
Since the abstract execution satisfies the constraint for \( \SER \), \ie \( \AR \subseteq \VIS \), we know \( \AR = \VIS \).
Since \( \Tx(\mkvs, \vi)  \) contains all transactions that write at least a key, 
we can pick a \( \T_\rd \) such that \( \Tx(\mkvs, \vi) \cup \T_\rd = \T_\aexec\),
which gives us the proof.


The execution test $\ET_\UA$ is complete with respect to the axiomatic definition \( (\RP_{\LWW}, \Set{\lambda \aexec. \AR_\aexec }) \).
Assume i-\emph{th} transaction \( \txid_i \) in the arbitrary order,
and let view \( \vi_{i} = \getView(\aexec, \VIS^{-1}_{\aexec}(\txid_{i}) ) \).
We also pick any final view such that \( \vi'_{i} \subseteq \getView(\aexec, (\AR^{-1}_{\aexec})?(\txid_{i}) ) \).
Note that there is nothing to prove for \( \vi'_i \),
Now we need to prove the following:
\[
    \fora{\ke, j}  0 \leq j < \abs{\mkvs_{\cut(\aexec, i-1)}(\ke)} \implies j \in \vi_i(\ke)
\]
Because \( \VIS^{-1}(\txid_i) = \AR^{-1}(\txid_i) = \Setcon{\txid }{\txid \texttt{ appears in } \mkvs_{\cut(\aexec, i-1)} }\),
so for any key \( \ke \) and index \( j \) such that \( 0 \leq j < \abs{\mkvs_{\cut(\aexec, i-1)}(\ke)} \),
the j-\emph{th} version of the key contains in the view, \ie \( j \in \vi(\ke)\).

