\subsection{Monotonic Read \( \MR \)}
\label{sec:sound-complete-mr}

The execution test $\ET_\MR$ is sound with respect to the axiomatic definition \cite{surech-session-guarantee}
\[(\RP_{\LWW}, \Set{\lambda \aexec. \VIS_{\aexec} ; \SO_{\aexec} })\] 
We choose an invariant as the following,  
\[
    I(\aexec, \cl) = \left( \bigcup_{\Set{\txid_{\cl}^{n} \in \txidset_{\aexec} }[ n \in \Nat]} \VIS_{\aexec}^{-1}(\txid^n_\cl) \right) \setminus \txidset_\rd
\]
where \( \txidset_\rd \) is all the read-only transactions in 
\( \bigcup_{\Set{\txid_{\cl}^{n} \in \txidset_{\aexec} }[ n \in \Nat]} \VIS_{\aexec}^{-1}(\txid^n_\cl) \).
Assume a kv-store $\mkvs$, an initial and a final view $\vi, \vi'$  a fingerprint $\fp$ 
such that $\ET_{\MR} \vdash (\mkvs, \vi) \csat \fp: (\mkvs',\vi')$. 
Also choose an arbitrary $\cl$, a transaction identifier $\txid \in \nextTxid(\mkvs, \cl)$, 
and an abstract execution $\aexec$ such that $\mkvs_{\aexec} = \mkvs$ and 
\begin{equation}
I(\aexec, \cl) \subseteq \Tx[\mkvs, \vi]
\label{eq:mr_invariant}
\end{equation}
Let \( \aexec' = \extend[\aexec, \txid, \fp, \Tx[\mkvs, \vi] \cup \txidset_\rd] \).
We now check if \( \aexec' \) satisfies the axiomatic definition and the invariant is preserved:
\begin{itemize}
    \item $\Set{\txid' }[ (\txid', \txid) \in \VIS_{\aexec'} ; \SO_{\aexec'} ] \subseteq \Tx[\mkvs, \vi] \cup \txidset_\rd$. 
Suppose that $\txid' \toEDGE{\VIS_{\aexec'}} \txid'' \toEDGE{\SO_{\aexec'}} \txid$ 
for some $\txid', \txid''$. We show that $\txid' \in I(\aexec, \cl)$, and then \cref{eq:mr_invariant} ensures 
that $\txid' \in \Tx[\mkvs, \vi] \cup \txidset_{\mathsf{rd}}$. 
Suppose $\txid'' \toEDGE{\SO_{\aexec'}} \txid$, then $\txid'' = \txid_{\cl}^{n}$ for some $n \in \Nat$.
Because $\txid'' \neq \txid$ and $\txidset_{\aexec'} \setminus \txidset_{\aexec} = \Set{ \txid }$, we also 
have that $\txid'' \in \aexec$. By the invariant of $I(\aexec, \cl)$, 
we have that $\VIS^{-1}_{\aexec}(\cl) \subseteq I(\aexec, \cl)$:
because $\txid' \toEDGE{\VIS_{\aexec'}} \txid''$ and $\txid'' \neq \txid$ we have 
that $\txid' \toEDGE{\VIS_{\aexec}} \txid''$ and therefore $\txid' \in I(\aexec, \cl)$. 

\item $I(\aexec', \cl) \subseteq \Tx[\aexec', \vi'] = \Tx[\mkvs', \vi']$. 
    In this case, because $\ET_{\MR} \vdash (\mkvs, \vi) \csat \fp: (\mkvs',\vi')$, 
then it must be the case that $\vi \viewleq \vi'$. 
A trivial consequence of this fact is that $\Tx[\mkvs, \vi] \subseteq \Tx[\mkvs, \vi']$.
Also, because $\aexec' = \extend[\aexec, \txid, \Tx[\mkvs, \vi] \cup \txidset_{\rd}]$, 
we have that $\Tx[\mkvs_{\aexec}, \vi] = \Tx[\mkvs_{\aexec'}, \vi]$. 
%
%
%
\ac{to infer this there should be a Lemma that states that if $\vi \in \Views(\mkvs)$, 
then $\Tx[\updateKV[\mkvs, \vi', \fp, \txid], \vi] = \Tx[\mkvs, \vi]$.}
%
%
%
Finally, note that $\Set{\txid_{\cl}^{n} \in \aexec' }[ n \in \Nat] = 
\Set{ \txid_{\cl}^{n} \in \txidset_{\aexec} }[ n \in \Nat] \cup \txid$, that for any 
$\txid_{\cl}^{n} \in \txidset_{\aexec}$ we have that $\VIS^{-1}_{\aexec'}(\txid_{\cl}^{n}) = 
\VIS^{-1}_{\aexec}(\txid_{\cl}^{n})$, and that 
$\VIS_{\aexec'}^{-1}(\txid) = \Tx[\mkvs, \vi] \cup \txidset_{\mathsf{rd}}$. 
Using all these facts, we obtain 
\begin{align*}
    I(\aexec', \cl) 
    &= \left( \bigcup_{\Set{\txid_{\cl}^{n} \in \aexec' }[ n \in \Nat]} \VIS_{\aexec'}^{-1}(\txid_{\cl}^{n}) \right) \setminus \txidset_\rd \\
    &= \left( \left( \bigcup_{\Set{\txid_{\cl}^{n} \in \aexec }[ n \in \Nat]} \VIS_{\aexec}^{-1}(\txid_{\cl}^{n}) \right) \setminus \txidset_\rd  \right) \cup \left( \VIS^{-1}_{\aexec'}(\txid) \setminus \txidset_\rd  \right) \\
    &= I(\aexec, \cl) \cup \Tx[\mkvs, \vi] \\
    &\stackrel{\eqref{eq:mr_invariant}}{\subseteq} \Tx[\mkvs, \vi] \\
    &= \Tx[\mkvs_\aexec, \vi] \\
    &= \Tx[\mkvs_{\aexec'}, \vi] \\
    &\subseteq \Tx[\mkvs_{\aexec'}, \vi']
\end{align*}
\end{itemize}

We show that the execution test $\ET_{\MR}$ is complete 
with respect to the axiomatic definition 
\[(\RP_{\LWW}, \Set{\lambda \aexec.(\VIS_{\aexec};\SO_{\aexec})})\]
Let $\aexec$ be an abstract execution that satisfies the definition
$\CMa(\RP_{\LWW}, \Set{\lambda \aexec.(\VIS_{\aexec};\SO_{\aexec})})$, 
and consider a transaction $\txid \in \txidset_{\aexec}$. 
Assume i-\emph{th} transaction \( \txid_i \) in the arbitrary order,
and let $\vi_i = \getView[\aexec, \VIS^{-1}_{\aexec}(\txid_{i})]$.
We have two possible cases: 
\begin{itemize}
    \item the transaction $\txid'_{i} = \min_{\SO_{\aexec}}\Set{\txid' }[ \txid_{i} \toEDGE{\SO_{\aexec}} \txid']$ is 
defined. In this case let 
\[\vi'_{i} =\getView[\aexec, (\AR^{-1}_{\aexec})\rflx(\txid_{i}) \cap \VIS^{-1}_{\aexec}(\txid'_{i})]\]
Note that $\txid_{i} \toEDGE{\SO_{\aexec}} \txid'_{i}$, and because $\aexec \models \VIS_{\aexec} ; \SO_{\aexec}$, 
it follows that $\VIS^{-1}_{\aexec}(\txid_{i}) \subseteq \VIS^{-1}_{\aexec}(\txid'_{i})$. 
We also have that $\VIS^{-1}_{\aexec}(\txid_{i}) \subseteq (\AR^{-1}_{\aexec})\rflx(\txid_{i})$ because of 
the definition of abstract execution. It follows that 
\[
\VIS^{-1}_{\aexec}(\txid_{i}) \subseteq (\AR^{-1}_{\aexec})\rflx(\txid_{i}) \cap \VIS^{-1}_{\aexec}(\txid'_{i}),
\]
Recall that  $\vi_i = \getView[\aexec, \VIS^{-1}_{\aexec}(\txid_{i})]$,
and $\vi'_{i} =\getView[\aexec, (\AR^{-1}_{\aexec})\rflx(\txid_{i}) \cap \VIS^{-1}_{\aexec}(\txid'_{i})]$.
Thus we have that $\vi_i \viewleq \vi'_{i}$, and therefore $\ET_{\MR} \vdash (\mkvs_{\cut[\aexec, i]}, \vi_i) 
\csat \TtoOp{T}_{\aexec}(\txid_{i}) : (\mkvs_{\cut[\aexec, i+1]},\vi'_{i})$. 
\item the transaction $\txid'_{i} = \min_{\SO_{\aexec}}\Set{\txid' }[ \txid_{i} \toEDGE{\SO_{\aexec}} \txid_{i}]$ 
is not defined. In this case, let 
\[\vi'_{i} = \getView[\aexec, (\AR^{-1}_{\aexec})\rflx(\txid_{i})]\]
As for the case above, we have that $\vi_i \viewleq \vi'_{i}$, and therefore 
$\ET_{\MR} \vdash (\mkvs_{\cut[\aexec, i]}, \vi_i) \csat \TtoOp{T}_{\aexec}(\txid_{i}) : (\mkvs_{\cut[\aexec, i+1]},\vi'_{i},\vi'_{i})$. 
\end{itemize}
