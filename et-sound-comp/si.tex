\subsection{Snapshot Isolation \( \SI \)}
\label{sec:sound-complete-si}

The axiomatic definition for \( \SI \) is 
\[ 
(\RP_{\LWW}, \Set{\lambda \aexec. \AR_\aexec ; \VIS_\aexec, \lambda \aexec \ldotp \SO_\aexec, \lambda \aexec \ldotp \WW_\aexec }) 
\]
By a lemma proven in \cite{SIanalysis}, for any \( \aexec \) satisfies the \( \SI \)
there exists an equivalent \( \aexec' \) with minimum visibility \( \VIS_{\aexec'} \subseteq \VIS_\aexec \) satisfying 
\[ 
    \left( \RP_{\LWW}, \Set{\lambda \aexec. \left( (\SO_{\aexec'} \cup \WW_{\aexec'} \cup \WR_{\aexec'} ) ; \RW_{\aexec'}\rflx \right) ; \VIS'_{\aexec'}, 
    \lambda \aexec \ldotp (\WW_{\aexec'} \cup \SO_{\aexec'}) } \right) 
\]
Under the minimum visibility \( \VIS \) all the transactions still have the same behaviour as before,
meaning they do not violate last-write-win.

To prove the soundness, we pick the invariant as the following:
\begin{align*}
    I_1(\aexec, \cl) & = \left( \bigcup_{\Set{\txid_{\cl}^{i} \in \txidset_{\aexec} }[ i \in \Nat ]} \VIS_{\aexec}^{-1}(\txid^i_\cl) \right) \setminus \txidset_\rd \\
    I_2(\aexec, \cl) & = \left( \bigcup_{\Set{\txid_{\cl}^{i} \in \txidset_{\aexec} }[ i \in \Nat ]} (\SO_{\aexec}^{-1})\rflx(\txid^i_\cl) \right) \setminus \txidset_\rd
\end{align*}
where \( \txidset_\rd \) is all the read-only transactions included in both 
\[ \txidset_\rd \in \left( \bigcup_{\Set{\txid_{\cl}^{i} \in \txidset_{\aexec} }[ i \in \Nat ]} \VIS_{\aexec}^{-1}(\txid^i_\cl) \right)\]
and \[ \txidset_\rd \in \left( \bigcup_{\Set{\txid_{\cl}^{i} \in \txidset_{\aexec} }[ i \in \Nat ]} (\SO_{\aexec}^{-1})\rflx(\txid^i_\cl) \right) \]
Assume a kv-store $\mkvs$, an initial and a final view $\vi, \vi'$  a fingerprint $\fp$ 
such that $\ET_{\SI} \vdash (\mkvs, \vi) \csat \fp: (\mkvs',\vi')$. 
Also choose an arbitrary $\cl$, a transaction identifier $\txid_\cl^n \in \nextTxid(\mkvs, \cl)$, 
and an abstract execution $\aexec$ such that $\mkvs_{\aexec} = \mkvs$ and 
\( I_1(\aexec, \cl) \cup I_2(\aexec, \cl) \subseteq \Tx[\mkvs, \vi] \).
We are about to prove there exists an extra set of read-only transaction \( \txidset'_\rd \) such that
the new abstract execution \( \aexec' = \extend[\aexec, \txid_\cl^n, \fp, \Tx[\mkvs, \vi] \cup \txidset_\rd \cup \txidset'_\rd]\) and 
\begin{gather}
    \fora{\txid} (\txid, \txid_\cl^n) \in \SO_{\aexec'} \implies \txid \in \Tx[\mkvs, \vi] \cup \txidset_\rd \cup \txidset'_\rd \label{equ:si-sound-update-so}\\
    \fora{\txid} (\txid, \txid_\cl^n) \in \WW_{\aexec'} \implies \txid \in \Tx[\mkvs, \vi] \cup \txidset_\rd \cup \txidset'_\rd \label{equ:si-sound-update-ww}\\
    \begin{array}{l}
    \fora{\txid} (\txid, \txid_\cl^n) \in \left( (\SO_{\aexec'} \cup \WW_{\aexec'} \cup \WR_{\aexec'} ) ; \RW_{\aexec'}\rflx \right) ; \VIS_{\aexec'} 
    \implies \txid \in \Tx[\mkvs, \vi] \cup \txidset_\rd \cup \txidset'_\rd 
    \end{array}
    \label{equ:si-sound-update-arvis}\\
    I_1(\aexec',\cl) \cup I_2(\aexec',\cl) \subseteq \Tx[\mkvs_{\aexec'}, \vi'] \label{equ:si-sound-inv} 
\end{gather}
\begin{itemize}
\item The invariant \( I_2 \) implies \cref{equ:si-sound-update-so} as the same as \( \RYW \) in \cref{sec:sound-complete-ryw}.
\item Since \( \SI \) also satisfies \( \UA \), the \cref{equ:si-sound-update-ww} can be proven as the same as \( \UA \) in \cref{sec:sound-complete-ua}.
\item \cref{equ:si-sound-update-arvis}. 
    Note that 
    \begin{centermultline}
        (\txid, \txid_\cl^n) \in \left( (\SO_{\aexec'} \cup \WW_{\aexec'} \cup \WR_{\aexec'} ) ; \RW_{\aexec'}\rflx \right); \VIS_{\aexec'} \\
        {} \implies (\txid, \txid_\cl^n) \in \left( (\SO_{\aexec} \cup \WW_{\aexec} \cup \WR_{\aexec} ) ; \RW_{\aexec}\rflx \right) ; \VIS_{\aexec'}
    \end{centermultline}
    Also, recall that \( \rel_\aexec = \rel_\mkvs \) for \( \rel \in \Set{\SO, \WR, \WW, \RW} \).
    Let \[ \txidset'_\rd = \lfpTx[\mkvs,\vi, \left( (\SO_{\aexec'} \cup \WW_{\aexec'} \cup \WR_{\aexec'} ) ; \RW_{\aexec'}\rflx \right)] \]
    This means that \[ \aexec' = \extend[\aexec, \txid_\cl^n, \fp, \lfpTx[\mkvs, \vi, \left( (\SO_{\aexec'} \cup \WW_{\aexec'} \cup \WR_{\aexec'} ) ; \RW_{\aexec'}\rflx \right)] \cup \txidset_\rd ] \]
    Let assume \( \txid \toEDGE{\left( (\SO_{\aexec'} \cup \WW_{\aexec'} \cup \WR_{\aexec'} ) ; \RW_{\aexec'}\rflx \right)} \txid' \) and \[ \txid' \in \lfpTx[\mkvs, \vi, \left( (\SO_{\aexec'} \cup \WW_{\aexec'} \cup \WR_{\aexec'} ) ; \RW_{\aexec'}\rflx \right)] \cup \txidset_\rd \]
    We have two possible cases:
    \begin{itemize}
        \item If \( \txid' \in \lfpTx[\mkvs, \vi, \left( (\SO_{\aexec'} \cup \WW_{\aexec'} \cup \WR_{\aexec'} ) ; \RW_{\aexec'}\rflx \right)] \), by  \cref{thm:view-vis-relation}, we know \[ \txid \in \lfpTx[\mkvs, \vi, \left( (\SO_{\aexec'} \cup \WW_{\aexec'} \cup \WR_{\aexec'} ) ; \RW_{\aexec'}\rflx \right)] \]
        \item If \( \txid' \in \txidset_\rd \), there are two cases:
        \begin{itemize}
            \item \( \txid' \in  \left( \bigcup_{\Set{\txid_{\cl}^{n} \in \txidset_{\aexec} }[ n \in \Nat ]} \VIS_{\aexec}^{-1}(\txid^n_\cl) \right) \).
                Since \( \txid' \) is a read-only transaction, it means \( \txid \toEDGE{\SO_{\mkvs} \cup \WR_{\mkvs} } \txid' \).
                By the property of \( \aexec \) (before update) that \( ( \SO \cup \WR_\aexec  ) ; \VIS_\aexec \in \VIS_\aexec \), it is known that \( \txid \in \left( \bigcup_{\Set{\txid_{\cl}^{n} \in \txidset_{\aexec} }[ n \in \Nat ]} \VIS_{\aexec}^{-1}(\txid^n_\cl) \right) \), that is, \( \txid \in \Tx[\mkvs,\vi] \cup \txidset_\rd\).

            \item \( \txid' \in  \left( \bigcup_{\Set{\txid_{\cl}^{n} \in \txidset_{\aexec} }[ n \in \Nat ]} \SO_{\aexec}^{-1}(\txid^n_\cl) \right) \).
                Given that \( \txid' \) is a read only transaction, we know \( \txid \in (\SO \cup \WR_\aexec)^{-1} \left( \bigcup_{\Set{\txid_{\cl}^{n} \in \txidset_{\aexec} }[ n \in \Nat ]} \SO_{\aexec}^{-1}(\txid^n_\cl) \right) \).
                By the property of \( \aexec \) (before update) that \( \SO \cup \WR_\aexec \in \VIS_\aexec \),
                it follows:
                \begin{align*}
                    \txid & \in VIS_\aexec^{-1} \left( \bigcup_{\Set{\txid_{\cl}^{n} \in \txidset_{\aexec} }[ n \in \Nat ]} \SO_{\aexec}^{-1}(\txid^n_\cl) \right) \\
                          & = \left( \bigcup_{\Set{\txid_{\cl}^{n} \in \txidset_{\aexec} }[ n \in \Nat ]} \VIS_{\aexec}^{-1}(\txid^n_\cl) \right)  \\
                          & = \Tx[\mkvs,\vi] \cup \txidset_\rd
                \end{align*}
        \end{itemize}
    \end{itemize}
\item Since \( \SI \) satisfies \( \RYW \) and \( \MR \), thus invariants \( I_1 \) and  \( I_2 \) are preserved, that is, \cref{equ:si-sound-inv}.
\end{itemize}

The execution test $\ET_\SI$ is complete with respect to the axiomatic definition 
\[ (\RP_{\LWW}, \Set{\lambda \aexec.  \AR_\aexec ; \VIS_{\aexec}, \lambda \aexec \ldotp \SO_\aexec , \lambda \aexec \ldotp \WW_\aexec })\]
Assume i-\emph{th} transaction \( \txid_i \) in the arbitrary order,
and let view \( \vi_{i} = \getView[\aexec, \VIS^{-1}_{\aexec}(\txid_{i})] \).
We pick final view as \( \vi'_{i} = \getView[\aexec, (\AR^{-1}_{\aexec})\rflx(\txid_{i}) \cap \VIS_{\aexec}^{-1}(\txid'_i)] \),
if \( \txid'_i = \min_{\SO}\Set{\txid'}[\txid_i \toEDGE{\SO} \txid' ]\) is defined,
otherwise  \( \vi'_{i} = \getView[\aexec, (\AR^{-1}_{\aexec})\rflx(\txid_{i})]\).
Let the \( \mkvs = \mkvs_{\cut[\aexec, i-1]} \).
Now we prove the three parts separately.
\begin{itemize}
    \item \( \MR \).  By \cref{prop:cc-vis} since 
    \( \VIS_\aexec ; \SO_\aexec \subseteq \VIS_\aexec ; \VIS_\aexec \subseteq \VIS_\aexec \)
    so it follows as in \cref{sec:sound-complete-mr}.
    \item \( \RYW \). For \( \RYW \), since \( \WR_\aexec ; \SO_\aexec ; \VIS_\aexec \subseteq \VIS_\aexec ; \VIS_\aexec ; \VIS_\aexec \subseteq \VIS_\aexec\), 
    the proof is as the same proof as in \cref{sec:sound-complete-ryw}.
    \item \( \UA \). Since \( \WW_\aexec \subseteq \VIS_\aexec\), 
    the proof is as the same proof as in \cref{sec:sound-complete-ua}.
    \item \( \allowed[\left( (\SO_{\mkvs} \cup \WW_{\mkvs} \cup \WR_{\mkvs} ) ; \RW_{\mkvs}\rflx \right)]\). It is derived from \cref{thm:view-vis-relation} and 
        \[ \left( (\SO_{\aexec} \cup \WW_{\aexec} \cup \WR_{\aexec} ) ; \RW_{\aexec}\rflx \right) ; \VIS_\aexec \subseteq \AR_\aexec ; \VIS_\aexec \subseteq \VIS_\aexec\]
\end{itemize}
