\subsection{Sanity Check for \( \ET \)}
\label{sec:mono-et}
\begin{proposition}
\label{prop:mono-et}
if $\ET_1 \subseteq \ET_2$ then $\CMs(\ET_1) \subseteq \CMs(\ET_2)$.
\end{proposition}
\begin{proof}
It is sufficient to prove that \(\ET_1 \subseteq \ET_2 \implies \Confs(\ET_1) \subseteq \Confs(\ET_2) \).
We prove it by induction on the length of the traces, \( i \).

\caseB{i = 0}
We have \( \conf_0 \in \Confs(\ET_1) \) and \( \conf_0 \in \Confs(\ET_2)\).
\caseI{i + 1}
Suppose identical traces of \( \ET_1 \) and \( \ET_2 \) respectively with length \( i \).
Let the final configuration be \( \conf_i = ( \mkvs_i, \vienv_i ) \).
If the next step is a view shift or a step with empty fingerprint, it trivially holds.
If the next step is a step by a client \( \cl \) with fingerprint \( \fp \),
we have \( \ET_1 \vdash (\mkvs_i, \vienv_i(\cl)) \csat \fp : (\mkvs_{i+1}, \vi') \),
where \( \mkvs_{i+1} \in \updKV{ \mkvs_i, \vienv_i(\cl), \fp, \cl} \)
The next configuration from \( \ET_1 \) is \( \conf_{i+1} =  (\mkvs, \vienv_i\rmto{\cl}{\vi'})\).
Since \( \ET_1 \subseteq \ET_2 \), so \( \ET_2 \vdash (\mkvs_i, \vienv_i(\cl)) \csat \fp : (\mkvs_{i+1}, \vi') \) holds.
It is possible for \( \ET_2 \) to have the exactly same next configuration \( \conf_{n+1}\).
\end{proof}
