\subsection{Compositionality of \( \ET \)}
\label{sec:et-comm}
\label{sec:et-comp}

To make two execution tests \( \ET_1 \) \( \ET_2 \) compositional with respect to to function \( \CMs \),
they need to satisfy \cref{def:conflict-commit,def:noblidwrites,def:et-minimum-footprint,def:et-continuous-postview}.
For all the specification we have in \cref{fig:execution.tests},
It is easy to adapt so that they satisfy \cref{def:noblidwrites,def:et-minimum-footprint,def:et-continuous-postview},
but not all of them can be adapted so to satisfy \cref{def:conflict-commit}, for example \( \CP \).

\begin{definition}
\label{def:noblidwrites}
An execution test $\ET$ has \emph{no blind writes} if, whenever $\ET \vdash (\hh, \vi) \triangleright \opset \cup \{(\otW, \ke, \_)\} : \vi'$, 
then $(\otR, \ke, \_) \in \opset$.
\end{definition}

\begin{definition}
\label{def:et-minimum-footprint}
An execution test $\ET$ has \emph{minimum footprints} if for any key-value store \( \hh \)
views \( \vi, \vi',\vi''\) and fingerprint \( \f \),
\[
\begin{array}{@{}l@{}}
\ET \vdash (\hh, \vi) \triangleright \opset : \vi''     
\land \fora{ \ke} \left( (\stub, \ke, \stub) \in \f \implies \vi(\ke) = \vi'(\ke) \right) 
\implies \ET \vdash (\hh, \vi') \triangleright \opset : \vi''
\end{array}
\]
\end{definition}

\begin{definition}
\label{def:et-continuous-postview}
An execution test $\ET$ has \emph{continuous post-views} if for any key-value store \( \hh \)
views \( \vi, \vi',\vi''\) and fingerprint \( \f \), 
\[
\begin{array}{@{}l@{}}
    \quad \ET \vdash (\hh, \vi) \triangleright \opset : \vi' \land \vi' \sqsubseteq \vi'' \implies \ET \vdash (\hh, \vi) \triangleright \opset : \vi''
\end{array}
\]
\end{definition}

Now we can prove compositionality of \( \ET \) (\cref{thm:et-comm}).

\begin{theorem}                                                                            
\label{thm:et-comm}                          
Let $\ET_1, \ET_2$ be two execution tests has no blind writes, minimum footprints and continuous post-views.
If $\ET_1$ is commutative, 
then $\CMs(\ET_1 \cap \ET_2) = \CMs(\ET_1) \cap \CMs(\ET_2)$. 
Furthermore, if $\ET_1, \ET_2$ are commutative, then $\ET_1 \cap \ET_2$ 
is commutative.
\end{theorem}
\begin{proof}
Given the definition of the \( \CMs(.) \) function (\cref{def:cm}), 
it suffices to prove that \( \CMs(\ET_{1} \cap \ET{2}) \subseteq \CMs(\ET_1) \cap \CMs(\ET_2) \)
and \( \CMs(\ET_1) \cap \CMs(\ET_2) \subseteq \CMs(\ET_{1} \cap \ET{2}) \).
The former is proven by the \cref{lem:et12-in-et1-et2} and the later is proven by \cref{lem:et1-et2-in-et12}.
\end{proof}

\begin{lemma}
\label{lem:et12-in-et1-et2}
\( \CMs(\ET_{1} \cap \ET_{2}) \subseteq \CMs(\ET_1) \cap \CMs(\ET_2) \).
\end{lemma}
\begin{proof}
It suffices to prove a stronger result that \( \Confs(\ET_{1} \cap \ET_{2}) \subseteq \Confs(\ET_1) \cap \Confs(\ET_2) \).
By the definition of \Confs (\cref{def:cm}), it suffices to prove for configurations \( \conf_0 \) to \( \conf_n \) 
\begin{equation}
    \label{equ:et12-in-et1-et2}
    \begin{array}{@{}l}
    \conf_0 \text{ is initial } 
    \land \conf_0 \xrightarrowtriangle{\stub}_{\ET_1 \cap \ET_2} \cdots \xrightarrowtriangle{\stub}_{\ET_1 \cap \ET_2} \conf_n \implies {} \\
    \quad \conf_0 \xrightarrowtriangle{\stub}_{\ET_{1}} \cdots \xrightarrowtriangle{\stub}_{\ET_{1}} \conf_n \land \conf_0 \xrightarrowtriangle{\stub}_{\ET_{2}} \cdots \xrightarrowtriangle{\stub}_{\ET_{2}} \conf_n 
    \end{array}
\end{equation}
We prove the \cref{equ:et12-in-et1-et2} by induction on the number \( n \).
\begin{itemize}
\item Base case: \(n = 0\). 
The \cref{equ:et12-in-et1-et2} holds when \( n = 0 \), because all initial configurations \( \conf_0 \) are included in the \( \Confs(\ET_1)\) and \( \Confs(\ET_2) \) by the definition of the \( \Confs \) function (\cref{def:cm}).

\item Inductive case: \(n = i+1\). Suppose the \cref{equ:et12-in-et1-et2} holds when \( n = i \) for some \( i \).
Let consider \( n = i + 1 \) and specifically the last step.
For any \( \conf_{i+1} = (\mkvs_{i+1}, \viewFun_{i+1}) \) induced by \( \ET_{1} \cap \ET_2 \), 
there exist some client \( \cl \), views \( \vi, \vi' \) and fingerprint \( \f \) such that:
\[
    \begin{array}{l}
    (\mkvs_i, \viewFun_i) \xrightarrowtriangle{\cl, \opset}_{\ET_{1} \cap \ET_{2}} (\mkvs_{i+1}, \viewFun_{i+1}) 
    \land \viewFun_{i+1} = \viewFun_{i}\rmto{\cl}{\vi'} \land (\mkvs_i, \vi, \f, \vi' ) \in \ET_{1} \cap \ET_{2}
    \end{array}
\]
Thus, it is easy to see that \( \conf_i \xrightarrowtriangle{\cl, \opset}_{\ET_{1}} \conf_{i+1} \) and \( \conf_i \xrightarrowtriangle{\cl, \opset}_{\ET_{2}} \conf_{i+1} \) by the \cref{lem:mono-et}.
\end{itemize}
\end{proof}

\begin{lemma}
\label{lem:mono-et}
If $\conf \xrightarrowtriangle{\cl, \opset}_{\ET} \conf'$ and $\ET \subseteq \ET'$, 
then $\conf \xrightarrowtriangle{\cl, \opset}_{\ET'} \conf'$.
\end{lemma}
\begin{proof}
    Let \((\mkvs, \viewFun)  = \conf \), \( (\mkvs', \viewFun') = \conf' \) and \( \vi  =\viewFun(\cl) \)
    By the definition of  $\conf \xrightarrowtriangle{\cl, \opset}_{\ET} \conf'$ (\cref{def:cm}), we have \(\mkvs' \in \updKV{\mkvs, \vi, \cl, \f}\) and  \( \viewFun' = \viewFun\rmto{\cl}{\vi'} \) for some \( \vi' \) such that \( \ET \vdash (\mkvs, \vi) \csat \f : \vi' \).
    Given that \( \ET \subseteq \ET'\), we know \( \ET' \vdash (\mkvs, \vi) \csat \f : \vi' \) and so $\conf \xrightarrowtriangle{\cl, \opset}_{\ET'} \conf'$.
\end{proof}

\begin{lemma}
\label{lem:et1-et2-in-et12}
\( \CMs(\ET_1) \cap \CMs(\ET_2) \subseteq \CMs(\ET_{1} \cap \ET_{2}) \).
\end{lemma}
\begin{proof}
    By the definition of \( \CMs\) and \( \Confs\) (\cref{def:cm}), we prove a stronger result that
    for an initial configuration \( \conf_0 \), 
    configurations \( \conf_1 \) to \( \conf_n \) from trace \( \ET_1 \), 
    configurations \( \conf'_1 \) to \( \conf'_m \) from trace \( \ET_2 \),
    \[
    \begin{array}{@{}l}
    \conf_0 \xrightarrowtriangle{\stub}_{\ET_{1}} \conf_1 \xrightarrowtriangle{\stub}_{\ET_{1}} \cdots \xrightarrowtriangle{\stub}_{\ET_{1}} \conf_n 
    \land \conf_0 \xrightarrowtriangle{\stub}_{\ET_{2}} \conf'_1 \xrightarrowtriangle{\stub}_{\ET_{2}}  \cdots \xrightarrowtriangle{\stub}_{\ET_{2}} \conf'_m 
    \land \conf_n\projection{1} = \conf'_m\projection{1} \\
    \end{array}
    \]
    there exists configurations from \( \conf''_1\)  to \( \conf''_k \) from trace \( \ET_1 \cap \ET_2 \):
\begin{equation}
    \label{equ:et1-et2-in-et12}
    \begin{array}{@{}l}
    \conf_0 \xrightarrowtriangle{\stub}_{\ET_1 \cap \ET_2} \conf''_1 \xrightarrowtriangle{\stub}_{\ET_1 \cap \ET_2} \cdots \xrightarrowtriangle{\stub}_{\ET_1 \cap \ET_2} \conf''_k 
    \land \conf_n\projection{1} = \conf'_m\projection{1} = \conf''_k\projection{1}  \\
    \quad {} \land \fora{\cl \in \dom(\conf''_k\projection{2}),\ke \in (\conf''_k\projection{1})}
    \conf''_k\projection{2}(\cl)(\ke) = \max\Set{\conf_n\projection{2}(\cl)(\ke), \conf'_m\projection{2}(\cl)(\ke)}
    \end{array}
\end{equation}
We prove \cref{equ:et1-et2-in-et12} by induction on the length \( m \) of the trace of \( \ET_2 \).
\begin{itemize}
    \item \caseB{\(m = 0\)}
We have the trace of \( \ET_1 \):
\begin{equation}
    \label{equ:trace-view-shift-et1}
    \conf_0 \text{ is initial } \land \conf_0 \toET{\stub}{\ET_1} \dots \toET{\stub}{\ET_1} \conf_n
\end{equation}
for some number \( n \) and configurations from \( \conf_0 \) to \( \conf_n \) and the trace of \( \ET_2 \) with only one configuration:
\begin{equation}
    \label{equ:trace-singleton-et2}
    \conf_0
\end{equation}
By the hypothesis we have \( \conf_0\projection{1} = \conf_n\projection{1} \), which means that all the steps from the trace of \( \ET_1 \) are view shift.
We can pick the trace of \( \ET_1 \) (\cref{equ:trace-view-shift-et1}) as the trace of \( \ET_1 \cap \ET_2 \):
\begin{equation}
    \label{equ:trace-view-shift-et12}
    \conf_0 \toET{\stub}{\ET_1 \cap \ET_2} \dots \toET{\stub}{\ET_1 \cap \ET_2} \conf''_k \land  k = n \land \bigwedge_{ 0 < i \leq k} \conf_i = \conf''_i
\end{equation}
It is easy to see:
\begin{equation}
    \label{equ:max-et1-et2}
    \begin{array}{l}
    \fora{\cl \in \dom(\conf_k\projection{2}), \ke \in \dom(\conf_k\projection{1})} 
    \conf_0\projection{2}(\cl)(\ke) = \max\Set{\conf_0\projection{2}(\cl)(\ke), \conf_n\projection{2}(\cl)(\ke)}
\end{array}
\end{equation}
Combine \cref{equ:trace-view-shift-et12} and \cref{equ:max-et1-et2}, we prove the \cref{equ:et1-et2-in-et12}.

\item \caseI{\(m = i + 1\)}
Suppose that \cref{equ:et1-et2-in-et12} holds when \( m = i \).
Let consider \( m = i + 1 \).
We have the trace for \( \ET_1 \):
\begin{equation}
    \conf_0 \xrightarrowtriangle{\stub}_{\ET_{1}} \conf_1 \xrightarrowtriangle{\stub}_{\ET_{1}} \cdots \xrightarrowtriangle{\stub}_{\ET_{1}} \conf'_{n} 
\end{equation}
for some number \( n \) and the configurations from \(\conf_0\) to \( \conf_n \), and the trace of \(\ET_2\):
\begin{equation}
    \conf_0 \xrightarrowtriangle{\stub}_{\ET_{2}} \conf'_1 \xrightarrowtriangle{\stub}_{\ET_{2}} \cdots \xrightarrowtriangle{\stub}_{\ET_{2}} \conf'_{i+1} 
\end{equation}
It is safe to assume these two traces are in normal form by \cref{prop:et.normalform}.
Assume a client \( \cl'_{i} \), views \( \vi'_{i}, \vi'_{i+1} \) and a fingerprint \( \opset'_{i} \) that commit to the second last configuration \( (\mkvs'_i, \viewFun'_i) = \conf'_i \) in the trace of \( \ET_2 \) which yields the final configuration \( (\mkvs'_{i+1}, \viewFun'_{i+1}) = \conf_{i+1} \):
\begin{equation}
    \label{equ:last-et-2}
    \begin{array}{@{}l @{}}
    (\mkvs'_i, \viewFun'_i) \toET{\cl'_{i}, \opset'_{i}}{\ET_2} (\mkvs'_{i+1}, \viewFun'_{i+1}) \land \ET_2 \vdash (\mkvs'_i, \vi'_i) \csat \f'_i  : \vi'_{i+1} 
    \land \vi' = \viewFun'_i(\cl'_i) \land \viewFun'_{i+1} = \viewFun'_i\rmto{\cl'_i}{\vi'_{i+1}}
    \end{array}
\end{equation}
There are three cases: \textbf{(i)} \( \f'_i = \unitO \), \textbf{(i)} \( \f'_i = \epsilon \) and \textbf{(ii)} \( \f'_i \neq \unitO \land \f'_i \neq \epsilon \).
\begin{itemize}
    \item If \( \f'_i = \epsilon \) or \( \f'_i = \unitO \), by the \cref{lem:no-effect-for-empty-fingerprint} we know \( \conf'_{i}\projection{1} = \conf'_{i+1}\projection{1}\) from the trace of \( \ET_2 \).
Since \( \conf'_{i+1}\projection{1} = \conf_n\projection{1}\) where \( \conf_n \) is the final configuration of the trace of \( \ET_1 \), we now have \( \conf'_{i}\projection{1} = \conf_n\projection{1}\).
Applying \ih that \cref{equ:et1-et2-in-et12} holds when \( m = i \), so there exist configurations from \( \conf''_1 \) to \( \conf''_k \):
\begin{equation}
    \label{equ:ih-for-k-length}
    \begin{array}{@{}l@{}}
    \quad \conf_0 \xrightarrowtriangle{\stub}_{\ET_1 \cap \ET_2} \conf''_1 \xrightarrowtriangle{\stub}_{\ET_1 \cap \ET_2} \cdots \xrightarrowtriangle{\stub}_{\ET_1 \cap \ET_2} \conf''_k
    \land \conf_n\projection{1} = \conf'_i\projection{1} = \conf''_k\projection{1} \\
    \quad {} \land \fora{\cl \in \dom(\conf''_k\projection{2}),\ke \in (\conf''_k\projection{1})} 
    \conf''_k\projection{2}(\cl)(\ke) = \max\Set{\conf_n\projection{2}(\cl)(\ke), \conf'_i\projection{2}(\cl)(\ke)}
\end{array}
\end{equation}
Given the definition of the reduction (\cref{def:reduction}), when \( \f = \epsilon \) or \( \f = \unitO \) we know \( \conf'_i\projection{2}(\cl_{i+1}) \sqsubseteq  \conf'_{i+1}\projection{2}(\cl_{i+1})\) thus:
\begin{equation}
    \label{equ:preserve-max-view}
    \begin{array}{l}
    \max\Set{\conf_n\projection{2}(\cl_{i+1}), \conf'_i\projection{2}(\cl_{i+1})} 
    \sqsubseteq \max\Set{\conf_n\projection{2}(\cl_{i+1}), \conf'_{i+1}\projection{2}(\cl_{i+1})} 
    \end{array}
\end{equation}
Therefore \cref{equ:et1-et2-in-et12} holds when \( m = i + 1\) by appending a view shift to the end of the trace in \cref{equ:ih-for-k-length}:
\[
    \begin{array}{@{}l}
    \conf_0 \xrightarrowtriangle{\stub}_{\ET_1 \cap \ET_2} \conf''_1 \xrightarrowtriangle{\stub}_{\ET_1 \cap \ET_2} \cdots
    \xrightarrowtriangle{\stub}_{\ET_1 \cap \ET_2} \conf''_k \toET{\cl_{j}, \epsilon }{\ET_1 \cap \ET_2} \\
    \qquad {} \land \conf''_k\rmto{2}{\conf''_k\projection{2}\rmto{\cl_{i}}{\max\Set{\conf_n\projection{2}(\cl_{i+1}), \conf'_{i+1}\projection{2}(\cl_{i+1})} }}
    \end{array}
\]

    \item If \( \f'_i \neq \unitO  \land \f' \neq \epsilon \), by \cref{lem:identical-step} there exists a step \( (\cl_j, \f_j) \) from the trace of \( \ET_1 \) such that:
\begin{equation}
    \label{equ:j-th-step}
    \begin{array}{l}
    (\mkvs_{j}, \viewFun_{j}) \xrightarrowtriangle{\cl_{j}, \opset_{j}}_{\ET_1} (\mkvs_{j + 1}, \viewFun_{j + 1}) 
    \land \ET \vdash (\mkvs_{j}, \vi_j) \csat \f_j : \viewFun_{j + 1}(\cl_{j}) \land \vi_j = \viewFun_{j}(\cl_j)
\end{array}
\end{equation}
for some \( j, \cl_j, \vi_j\) and \( \f_j \) such that \( 0 \leq  j < n \), \( \cl_j = \cl'_{i}\), \( \f_j = \f'_{i}\), and
\[ 
    \fora{\ke} (\stub, \ke, \stub ) \in \f_j \implies \vi_j(\ke) = \vi'_{i}(\ke)
\]
We apply the commutativity of \( \ET_1 \) until the step shown in \cref{equ:j-th-step} is at the end or the second end of the trace of \( \ET_1 \).
Let consider the next two steps, (j+1)-\emph{th} and (j+2)-\emph{th} step.
Since the trace is in normal form, the (j+1)-\emph{th} step is a view shift by a client \( \cl_{j+2} \) and (j+2)-\emph{th} step is a concrete step issued by the same client \( \cl_{j+2} \) under the view \( \vi_{j+2} \):
\begin{equation}
    \label{equ:j-plus-1-th-step}
    \begin{array}{@{}l@{}}
        (\mkvs_{j+1}, \viewFun_{j+1}) \xrightarrowtriangle{\cl_{j+2}, \epsilon}_{\ET_1}
        (\mkvs_{j+1}, \viewFun_{j+1}\rmto{\cl_{j+2}}{\vi_{j+2}}) \toET{\cl_{j+2}, \f_{j+2}}{\ET_1} (\mkvs_{j+3}, \viewFun_{j+3}) \\
        \qquad \land \ET \vdash (\mkvs_{j+1},\vi_{j+2}) \csat \f_{j+2} : \viewFun_{j+2}(\cl_{j+2}) 
    \end{array}
\end{equation}
It is known that the client  \( \cl_{j+2} \) is different from \( \cl_j \) (\cref{lem:different-cl}) and \( \f_{j+2} \) writes different keys from \( \f_j\) (\cref{lem:different-writes}). 
Because \( \cl_j \neq \cl_{j+2} \) we can swap the view shift step shown in \cref{equ:j-plus-1-th-step} before the j-\emph{th} step shown in \cref{equ:j-th-step} which gives the following:
\begin{equation}
    \label{equ:swap-the-view-shift-et1}
    \begin{array}{@{}l@{}}
    (\mkvs_{j}, \viewFun_{j}) \toET{\cl_{j+2}, \epsilon}{\ET_1} (\mkvs_{j}, \viewFun_{j}\rmto{\cl_{j+2}}{\vi_{j+2}}) \toET{\cl_{j}, \opset_{j}}{\ET_1} \\
    \quad (\mkvs_{j + 1}, \viewFun_{j + 1}\rmto{\cl_{j+2}}{\vi_{j+2}}) \toET{\cl_{j+2}, \f_{j+2}}{\ET_1} (\mkvs_{j+3}, \viewFun_{j+3})
    \end{array}
\end{equation}
Now let discuss the (j+2)-\emph{th} step.
Similarly by the \cref{lem:identical-step}, there is a step \((\cl_p, \f_p)\) from the trace of \( \ET_2 \) such that \( \cl_p = \cl_{j+2}\) and \( \f_p = \f_{j+2}\) and \( p < i \).
Note that the last step from \( \ET_2 \), \ie (i+1)-\emph{th} step, is not a view shift therefore the i-\emph{th} step must be a view shift so the p-\emph{th} step must be before  i-\emph{th} step.
This means the fingerprint \( \f_p \) does not observe any change by (i+1)-\emph{th} step from the trace of \( \ET_2 \).
Therefore \( \vi_{j+2} \) does not observe any change by j-\emph{th} step from the trance of \( \ET_1\), \ie \( \vi_{j+2} \in \Views(\mkvs_j) \).
By \cref{prop:swap-update}, that allows to swap the two adjacent non-conflict steps from \cref{equ:swap-the-view-shift-et1}, \ie the last two steps.
It follows a new kv-stores \( \mkvs'''_{j+2}\) and a new view environment \( \viewFun'''_{j+2} \) such that:
\begin{equation}
    \label{equ:swap-step-et1}
    \begin{array}{@{}l@{}}
    (\mkvs_{j}, \viewFun_{j}) \toET{\cl_{j+2}, \epsilon}{\ET_1} (\mkvs_{j}, \viewFun_{j}\rmto{\cl_{j+2}}{\vi_{j+2}}) \toET{\cl_{j+2}, \opset_{j+2}}{\ET_1} \\
    \quad (\mkvs_{j + 2}''', \viewFun_{j + 2}''') \toET{\cl_{j+2}, \f_{j+2}}{\ET_1} (\mkvs_{j+3}, \viewFun_{j+3})
    \end{array}
\end{equation}
In the \cref{equ:swap-step-et1} the j-\emph{th} step moves to the right of (j+2)-\emph{th} step.
We continuously move the j-\emph{th} step until it is at the end or the second end of trace of \( \ET_1 \):
\[
    \begin{array}{@{}l}
        \conf_0 \xrightarrowtriangle{\stub}_{\ET_{1}} \cdots \xrightarrowtriangle{\stub}_{\ET_{1}} \conf_{j-1} \toET{\stub}{\ET_{1}} 
        \conf'''_{j} \toET{\stub}{\ET_{1}} \dots \toET{\stub}{\ET_{1}} \conf'''_{n-1} \toET{\cl_j, \f_j }{\ET_{1}} \conf_{n} \lor {} \\
        \conf_0 \xrightarrowtriangle{\stub}_{\ET_{1}} \cdots \xrightarrowtriangle{\stub}_{\ET_{1}} \conf_{j-1} \toET{\stub}{\ET_{1}} 
        \conf'''_{j} \toET{\stub}{\ET_{1}} \dots \toET{\stub}{\ET_{1}} \conf'''_{n-2} \toET{\cl_j, \f_j }{\ET_{1}} \conf'''_{n-1} \toET{\cl_{n-1}, \epsilon }{\ET_{1}} \conf_{n}  \\ 
    \end{array}
\]
for some new configurations from \( \conf'''_{j}\) to \( \conf'''_{n-1} \).
Note that if it is the second end, the last step must be a view shift step as shown in \cref{equ:new-et-1}.
\begin{itemize}
    \item If the j-\emph{th} step is at the end of the new trace of \( \ET_1 \), we have the trace:
\begin{equation}
    \label{equ:new-et-1}
    \begin{array}{@{}l}
        \conf_0 \xrightarrowtriangle{\stub}_{\ET_{1}} \cdots \xrightarrowtriangle{\stub}_{\ET_{1}} \conf_{j-1} \toET{\stub}{\ET_{1}} 
        \conf'''_{j} \toET{\stub}{\ET_{1}} \dots \toET{\stub}{\ET_{1}} \conf'''_{n-1} \toET{\cl_j, \f_j }{\ET_{1}} \conf_{n}  \\
    \end{array}
\end{equation}
Given the hypothesis that \( \conf_{n}\projection{1} = \conf'_{i+1}\projection{1} \) and the fact that the last step of the new trace of \( \ET_1 \) (\cref{equ:new-et-1}) and the last step the trace of \( \ET_2 \) (\cref{equ:last-et-2}) are the same step, the kv-stores of the second last configurations the new trace of \( \ET_1 \) (\cref{equ:new-et-1}) and the one from the trace of \( \ET_2 \) (\cref{equ:last-et-2}) are the same \(  \conf'''_{n-1}\projection{1} = \conf'_{i}\projection{1} \).
Then by applying \ih that \cref{equ:et1-et2-in-et12} holds when \( m = i \), there exists a trace of \( \ET_1 \cap \ET_2 \):
\begin{equation}
    \label{equ:ih-for-merge-two-trace}
    \begin{array}{@{}l}
        \conf_0 \toET{\stub}{\ET_1 \cap \ET_2} \dots \toET{\stub}{\ET_1 \cap \ET_2} \conf''_{k-1} 
        \land \conf'''_{n-1}\projection{1} = \conf'_{i}\projection{1} = \conf''_{k-1}\projection{1}  \\
        \quad {} \land \fora{\cl \in \dom(\conf''_{k-1}\projection{2}),\ke \in (\conf''_{k-1}\projection{1})} 
        \conf''_{k-1}\projection{2}(\cl)(\ke) = \max\Set{\conf_n\projection{2}(\cl)(\ke), \conf'_i\projection{2}(\cl)(\ke)}
\end{array}
\end{equation}
for some number \( k \) and configurations from \( \conf''_1 \) to \( \conf''_{k-1} \).
By \cref{equ:last-et-2} and \cref{equ:new-et-1}, we have:
\begin{equation}
    \label{equ:et1-et2-csat}
    \begin{array}{@{} l@{}}
    \ET_1 \vdash ( \conf'_{i}\projection{1}, \conf'_{i}\projection{2}(\cl_{i}) )  \csat \f_{i}, \conf'_{i+1}\projection{2}(\cl_{i})  
    \land \ET_2 \vdash ( \conf'''_{n-1}\projection{1}, \conf'''_{n-1}\projection{2}(\cl_{i}) )  \csat \f_{i}, \conf_{n}\projection{2}(\cl_{i})
    \end{array}
\end{equation}
First, for any quadraple in \( \ET_1 \) and \( \ET_2 \), it does not constrain the view for keys that are not appear in the fingerprint before update.
That is:
\[
    \begin{array}{@{}l@{}}
    \for{ \mkvs, \vi, \vi', \vi'', \f, \ke } 
    (\stub, \ke, \stub) \in \f \land \vi(\ke) = \vi'(\ke) \land (\mkvs, \vi, \f, \vi'') \in \ET 
    \implies (\mkvs, \vi', \f, \vi'') \in \ET
    \end{array}
\]
Given above and \cref{equ:ih-for-merge-two-trace}, we can substitute the configurations \( \conf'_{i} \) and  \( \conf'''_{n-1} \) from \cref{equ:et1-et2-csat} by \( \conf''_{k-1}\).
Then,  because for any \( \mkvs, \vi, \vi', \vi'' \) and \( \f \), if \( (\mkvs, \vi, \f, \vi' ) \in \ET_1 \) and \( \vi' \sqsubseteq \vi'' \) then \( (\mkvs, \vi, \f, \vi'' ) \in \ET_1 \), and similarly for \( \ET_2 \).
It means:
\begin{equation}
    \label{equ:et12-csat}
    \begin{array}{l}
    \ET_1 \cap \ET_2 \vdash ( \conf''_{k-1}\projection{1}, \conf''_{k-1}\projection{2}(\cl_{i}) ) \csat
    \f_{i}, \max\Set{\conf'_{i+1}\projection{2}(\cl_{i}), \conf_{n}\projection{2}(\cl_{i})}
    \end{array}
\end{equation}
Therefore the \cref{equ:et1-et2-in-et12} holds when \( m = i + 1\) by appending the shown in \cref{equ:et12-csat} to the end of the trace shown in \cref{equ:ih-for-merge-two-trace}:
\[
\begin{array}{@{}l}
    \conf_0 \toET{\stub}{\ET_1 \cap \ET_2} \dots \toET{\stub}{\ET_1 \cap \ET_2} \conf''_{k-1} \toET{cl_{i}, \f_{i}}{\ET_1 \cap \ET_2} \\
    \quad \left( \conf_n\projection{1},\conf''_{k-1}\projection{2}\rmto{\cl_{i}}{\max\Set{\conf'_{i+1}\projection{2}(\cl_{i}), \conf_{n}\projection{2}(\cl_{i})} } \right)
\end{array}
\]
    \item If the j-\emph{th} step is the second last step of the new trace of \( \ET_1 \), we have the trace:
\begin{equation}
    \label{equ:new-et-1-with-view-shift-tail}
    \begin{array}{@{}l}
        \conf_0 \xrightarrowtriangle{\stub}_{\ET_{1}} \cdots \xrightarrowtriangle{\stub}_{\ET_{1}} \conf_{j-1} \toET{\stub}{\ET_{1}} 
        \conf'''_{j} \toET{\stub}{\ET_{1}} \dots \toET{\stub}{\ET_{1}} \conf'''_{n-2} \toET{\cl_j, \f_j }{\ET_{1}} 
        \conf'''_{n-1} \toET{\cl_{n-1}, \epsilon }{\ET_{1}} \conf_{n}  \\ 
    \end{array}
\end{equation}
Since the last step is a view shift, we know \( \conf_n\projection{1} = \conf'''_{n-1}\projection{1}\), and the rest of proof is the same as the case where j-\emph{th} is the last step as shown in \cref{equ:new-et-1}.
\end{itemize}
\end{itemize}
\end{itemize}
\end{proof}

\begin{lemma}[No effect from empty fingerprint and epsilon reduction]
    \label{lem:no-effect-for-empty-fingerprint}
    \label{lem:no-effect-for-view-shift}
    \[
    \fora{\conf, \conf', \cl,\vi} \conf \toET{\cl, \unitO}{\ET} \conf' \lor \conf \toET{\cl, \epsilon}{\ET} \conf' \implies \conf\projection{1} = \conf'\projection{1}
    \]
\end{lemma}
\begin{proof}
    Let \((\mkvs, \viewFun)  = \conf \) and \( (\mkvs', \viewFun') = \conf' \).
    For the case of empty fingerprint,
    by the definition of  $\conf \xrightarrowtriangle{\cl, \unitO}_{\ET} \conf'$ (\cref{def:reduction}), we have \(\mkvs' \in \updKV{\mkvs, \vi, \cl, \unitO}\), and therefore \( \mkvs' = \mkvs \).
    For the case of view shift, by the definition of  $\conf \xrightarrowtriangle{\cl, \epsilon}_{\ET} \conf'$ (\cref{def:reduction}) it is easy to see \( \mkvs' = \mkvs \).
\end{proof}

We define a \(  \mkvs(\txid) \) function that returns the fingerprint associate with the transaction identifier \( \txid \):
\[
    \begin{rclarray}
        \mkvs(\txid) & \defeq & \Setcon{(\otW, \ke, \val)}{\exsts{i} \mkvs(\ke)(i) = (\val, \txid, \stub)} \cup  \Setcon{(\otR, \ke, \val)}{\exsts{i,\txidset} \mkvs(\ke)(i) = (\val, \stub, \txidset) \land \txid \in \txidset}
    \end{rclarray}
\]

\begin{lemma}[Transactions persistence]
    \label{lem:mono-fingerprint}
    \[
        \fora{\ET,\conf,\conf',\txid,\f} \conf\projection{1}(\txid) = \f \land \conf \toET{\stub}{\ET} \conf' \implies \conf'\projection{1}(\txid) = \f
    \]
\end{lemma}
\begin{proof}
    It is easy to prove this by case analysis on the reduction relation.
\end{proof}

\begin{lemma}[Same steps]
\label{lem:identical-step}
Given a trace of \( \ET_1 \) and a trace of \( \ET_2 \),
if the have the same final key-value store,
the trace contains the same concrete steps (free variables are globally quantified):
\[
\begin{array}{@{}l}
    \conf_0 \xrightarrowtriangle{\cl_1, \f_1}_{\ET_{1}} \cdots \xrightarrowtriangle{\cl_n, \f_n}_{\ET_{1}} \conf_n \land
    \conf_0 \xrightarrowtriangle{\cl'_1, \f'_1}_{\ET_{2}} \cdots \xrightarrowtriangle{\cl'_m, \f'_m}_{\ET_{2}} \conf'_m 
    \land \conf_n\projection{1} = \conf'_m\projection{1} \\
    \quad \implies \fora{i: 0 < i \leq n} 
    \f_i = \unitO 
    \lor \f_i = \epsilon 
    \lor \exsts{j: 0 < j \leq m} 
    \cl_i = \cl'_j \land \f_i = \f'_j \land ( \fora{\ke} (\stub, \ke, \stub) \in \f_i \implies \vi_i(\ke) = \vi'_j(\ke) )
\end{array}
\]
\end{lemma} 
\begin{proof}
    We prove by contradiction.
    First because \( \mkvs_n = \mkvs'_m \), we know that:
    \begin{equation}
        \label{equ:same-kv-store}
        \fora{\txid, \f} \mkvs_n(\txid) = \f \iff \mkvs'_m(\txid) = \f
    \end{equation}
    Let \(\conf_n = (\mkvs_n,\viewFun_n) \) and \(\conf'_m = (\mkvs'_m,\viewFun'_m) \).
    Assume a step \( \conf_i \toET{\cl, \f }{\ET_1} \conf_{i+1} \)  from the trace of \( \ET_1 \) where the transaction identifier is \( \txid \) and \( \f \neq \unitO \).
    It must have a step from the trace of \( \ET_2 \), which commits some fingerprint via the same transaction identifier  \( \txid \).
    We know \( \mkvs_n(\txid) = \f \) by \cref{lem:mono-fingerprint}, thus \( \mkvs'_m(\txid) = \f \) by \cref{equ:same-kv-store}.
    Let assume a key \( \ke \) that \( (\stub, \ke, \stub) \in \f \land \vi_i(\ke) \neq \vi'_j(\ke)\) where \( \vi_i\) and \( \vi'_j\) are the views immediate before the commit of the fingerprint \( \f \) in traces of \( \ET_1\) and \( \ET_2 \) respectively.
    Since the no blind write assumption, it is safe to assume it is a read operation on the key \( \ke \).
    By the definition of the reduction (\cref{def:reduction}) and \cref{lem:mono-fingerprint}, we know \( \func{read}{\mkvs_n(\ke)(\vi_i(\ke))} \neq \func{read}{\mkvs'_m(\ke)(\vi_i(\ke))} \), which contradicts with \( \mkvs_n = \mkvs'_m \).
\end{proof}

We define \( \max_\cl(\conf) \) function that returns the most recent transaction identifier for client \( \cl \) in the configuration \( \conf \) 
\[
\begin{rclarray}
    \max_\cl((\mkvs, \viewFun)) & \defeq & \max\Setcon{\txid^{n}_\cl}{\txid^{n}_\cl \text{ appear in } \mkvs} \\
\end{rclarray}
\]

\begin{lemma}[Transactions from different clients]
\label{lem:different-cl}
Given a trace of \( \ET_1 \) and a trace of \( \ET_2 \),
if the i-\emph{th} step from \( \ET_1 \) issued by the client \( \cl_i \) 
is the same as the last step from \( \ET_2 \),
then in the trace of \( \ET_1 \) 
there is no concrete step issued by the client \(\cl_i \) after the i-\emph{th} step (free variables are globally quantified):
\[
\begin{array}{@{}l}
    \conf_0 \xrightarrowtriangle{\cl_1, \f_1}_{\ET_{1}} \cdots \xrightarrowtriangle{\cl_n, \f_n}_{\ET_{1}} \conf_n 
    \land \conf_0 \xrightarrowtriangle{\cl'_1, \f'_1}_{\ET_{2}} \cdots \xrightarrowtriangle{\cl'_m, \f'_m}_{\ET_{2}} \conf'_m 
    \land \conf_n\projection{1} = \conf'_m\projection{1} 
    \land \f'_m \neq \unitO \\
    \quad {} \land \exsts{i}  
    \cl_i = \cl'_m
    \land \f_i = \f'_m 
    \implies \fora{j > i} 
    \f_j = \epsilon \lor \f_j = \unitO \lor \cl_i \neq \cl_j
\end{array}
\]
\end{lemma}
\begin{proof}
    We prove by deriving contradiction.
    Assume the last step of the trace of \( \ET_2 \) is:
    \begin{equation}
        \label{equ:last-step-for-cl-et2}
        \conf'_{m-1} \xrightarrowtriangle{\cl'_m, \f'_m}_{\ET_{2}} \conf'_m
    \end{equation}
    Assume a step of the trace of \( \ET_1 \):
    \begin{equation}
        \label{equ:identical-step-for-cl-et1}
        \conf_{i-1} \xrightarrowtriangle{\cl_i, \f_i}_{\ET_{1}} \conf_i
    \end{equation}
    where \( \cl_i = \cl'_m \) and \( \f_i = \f'_m \).
    Because these two steps (\cref{equ:last-step-for-cl-et2} and \cref{equ:identical-step-for-cl-et1}) are issued by the same transaction identifier,
    we know \( \max{}_{\cl_m}(\conf'_m) = \max{}_{\cl_m}(\conf_i) \).
    Assume that there exists a step from the trace of \( \ET_1 \), says j-\emph{th} step, such that:
    \[
        \conf_{j-1} \xrightarrowtriangle{\cl_j, \f_h}_{\ET_{1}} \conf_j \land j > i \land \f_j \neq \unitO \land \cl_i = \cl_j 
    \]
    Therefore we have \( \max{}_{\cl_m}(\conf_j) > \max{}_{\cl_m}(\conf_i) \) by \cref{lem:kv-max-cl}.
    That means \( \max{}_{\cl_m}(\conf_n) > \max{}_{\cl_m}(\conf_j) > \max{}_{\cl_m}(\conf_i) = \max{}_{\cl_m}(\conf'_m) \), which contradicts to \( \conf_n\projection{1} = \conf'_m\projection{1}\).
\end{proof}

\begin{lemma}[Reduction following session order]
\label{lem:kv-max-cl}
\[
\begin{array}{@{}l}
    \fora{\conf, \conf' ,\cl, \f, \ET}
    \conf \xrightarrowtriangle{\cl, \f}_{\ET}  \conf' 
    \land 
    \left( 
        \begin{array}{l}
        \f \neq \unitO \implies \max{}_\cl(\conf) < \max{}_\cl(\conf') )
        \lor ( \f = \unitO \implies \max{}_\cl(\conf) = \max{}_\cl(\conf')
        \end{array}
    \right)
\end{array}
\]
\end{lemma}
\begin{proof}
    Assume a step \( (\mkvs, \viewFun) \xrightarrowtriangle{\cl, \f}_{\ET} (\mkvs', \viewFun') \).
    By the definition of \( \toET{\stub}{\ET}\) (\cref{def:reduction}), we know \( \mkvs' \in \updKV{\mkvs, \vi, \cl, \f} \).
    The \( \updKV{\mkvs, \vi, \cl, \f} \) picks a fresh transaction identifier \( \txid_\cl^{m} \) that is greater than any transaction identifiers \( \txid_\cl^{n} \) in \( \mkvs \) via \( \nextTxId \) function, \ie \( m > n \).
    If the fingerprint \( \f \) is not empty, the new identifier appears in \( \mkvs' \), so \( \max{}_\cl(\conf) < \max{}_\cl(\conf') \).
    Otherwise  the fingerprint is empty, the new identifier will not appear anywhere in \( \mkvs' \), so \( \max{}_\cl(\conf) = \max{}_\cl(\conf') \). 
\end{proof}

\begin{lemma}[Writing different keys]
\label{lem:different-writes}
Given a trace of \( \ET_1 \) and a trace of \( \ET_2 \),
if the i-\emph{th} step from \( \ET_1 \) that writes to key \( \ke \) 
is the same as the last step from \( \ET_2 \),
then in the trace of \( \ET_1 \) 
there is no concrete step writing to the key \(\ke\) after the i-\emph{th} step (free variables are globally quantified):
\[
\begin{array}{@{}l}
    \conf_0 \xrightarrowtriangle{\cl_1, \f_1}_{\ET_{1}} \cdots \xrightarrowtriangle{\cl_n, \f_n}_{\ET_{1}} \conf_n \land \conf_0 \xrightarrowtriangle{\cl'_1, \f'_1}_{\ET_{2}} \cdots \xrightarrowtriangle{\cl'_m, \f'_m}_{\ET_{2}} \conf'_m 
    \land \conf_n\projection{1} = \conf'_m\projection{1} 
    \land \f'_m \neq \unitO \\
    \quad {} \land \exsts{i} 
    \cl_i = \cl'_m
    \land \f_i = \f'_m
    \implies \fora{j > i} \nexists{\ke} \ldotp (\otW, \ke, \stub) \in \f_j \cap \f_i
\end{array}
\]
\end{lemma}
\begin{proof}
    We prove this by deriving contradiction.
    Assume the last step from the trace of \( \ET_2 \):
    \begin{equation}
        \label{equ:last-step-for-write-et2}
        \conf'_{m-1} \xrightarrowtriangle{\cl'_m, \f'_m}_{\ET_{2}} (\mkvs'_m, \viewFun'_m )
    \end{equation}
    Assume the transaction identifier for the \cref{equ:last-step-for-write-et2} is \( \txid \), and by the definition of \( \toET{}{\ET}\) (\cref{def:reduction}) we know:
    \begin{equation}
        \label{equ:write-fingerprint}
        \fora{\ke} (\otW, \ke, \stub) \in \f'_m \implies \mkvs'_m(\ke)(\lvert\mkvs'_m(\ke)\rvert - 1) = (\stub, \txid, \stub)
    \end{equation}
    Assume a step of the trace of \( \ET_1 \) that is issued by the same transaction identifier with the same fingerprint:
    \begin{equation}
        \label{equ:identical-step-for-write-et1}
        \conf_{i-1} \xrightarrowtriangle{\cl_i, \f_i}_{\ET_{1}} (\mkvs_i, \viewFun_i)
    \end{equation}
    where \( \cl_i = \cl'_m \) and \( \f_i = \f'_m \).
    Given \cref{equ:write-fingerprint} and \cref{equ:identical-step-for-write-et1}, it follows:
    \[
        \fora{\ke} (\otW, \ke, \stub) \in \f_i \implies \mkvs_i(\ke)(\lvert\mkvs_i(\ke)\rvert - 1) = (\stub, \txid, \stub)
    \]
    Assume a step, says j-\emph{th}, after i-\emph{th} step that writes to the same key:
    \[
        \conf_{j-1} \xrightarrowtriangle{\cl_j, \f_j}_{\ET_{1}} (\mkvs_j, \viewFun_j) 
        \land j > i
        \land \exsts{\ke} (\otW, \ke, \stub) \in \f_i \cap \f_j
    \]
    Therefore, by \cref{lem:unique-writer} we have:
    \[
        \exsts{\ke,i} (\otW, \ke, \stub) \in \f_i \cap \f_j \land \mkvs_j(\ke)(i) = \txid \land \mkvs_j(\ke)(\lvert \mkvs_j(\ke) \rvert - 1) \neq \txid
    \]
    Note that \( \f_i = \f'_m\).
    Since the writer of a version cannot be overwritten, for the final configuration of the trace of \( \ET_1 \) \((\mkvs_n, \viewFun_n)\), we know:
    \[
        \exsts{\ke,i} (\otW, \ke, \stub) \in \f_i \cap \f_j \land \mkvs_n(\ke)(i) = \txid \land \mkvs_n(\ke)(\lvert \mkvs_n(\ke) \rvert - 1) \neq \txid
    \]
    Last, by \cref{equ:write-fingerprint} and \( \f_i = \f'_m\), it follows:
    \[
        \exsts{\ke} (\otW, \ke, \stub) \in \f'_m \land \mkvs_n(\ke)(\mkvs_n(\ke) - 1) \neq \txid \land \mkvs'_m(\ke)(\lvert\mkvs'_m(\ke)\rvert - 1)\projection{2} = \txid
    \]
    which contradicts with \( \mkvs'_m = \mkvs_n\).
\end{proof}

\begin{lemma}[Version persistence]
    \label{lem:unique-writer}
    \[
    \begin{array}{@{}l}
        \fora{\mkvs, \mkvs',\viewFun,\viewFun', \cl, \vi, \f, i} 
        (\mkvs, \viewFun) \toET{\cl, \f}{\ET} (\mkvs', \viewFun')
        \land (\otW, \ke, \stub) \in \f  
        \land 0 \leq i < \lvert \mkvs'(\ke) \rvert - 1 \\
        \quad {} \implies \mkvs'(\ke)(i)\projection{2} \neq \mkvs'(\ke)(\lvert \mkvs'(\ke) \rvert - 1)\projection{2}
    \end{array}
    \]
\end{lemma}
\begin{proof}
    By the definition of \( \toET{\stub}{\ET} \) (\cref{def:reduction}), the \( \mkvs' \in \updKV{\mkvs, \vi, \cl, \f} \).
    Given the definition of \( \updKV{\mkvs, \vi, \cl, \f}\), it the picks a fresh transaction identifier \( \txid \) such that does not appear in \( \mkvs \).
    For any write fingerprint \( (\otW, \ke, \stub) \in \f \), a new version is appended to the end of the key \( \ke \) and the writer (the second projection) is assigned to be the fresh identifier \( \txid \).
    Thus we have the proof.
\end{proof}

\begin{proposition}
\label{thm:appendix-et-composition-2}
\label{prop:appendix-et-composition-2}
if $\ET_1, \ET_2$ are commutative, then $\ET_1 \cap \ET_2$ is commutative.
\end{proposition}
\begin{proof}
Let \( \ET_{12} = \ET_1 \cap \ET_2 \).
Assume \(\conf_1, \conf_2, \conf_3, \cl, \cl', \vi, \vi', \opset, \opset' \) such that:
\[
    \conf_1 \xrightarrowtriangle{\cl, \vi, \opset}_{\ET_{12}} \conf_2 \xrightarrowtriangle{\cl', \vi', \opset'}_{\ET_{12}} \conf_3
\]
Therefore, we have:
\[
    \conf_1 \xrightarrowtriangle{\cl, \vi, \opset}_{\ET_{1}} \conf_2 \xrightarrowtriangle{\cl', \vi', \opset'}_{\ET_{1}} \conf_3 \land 
    \conf_1 \xrightarrowtriangle{\cl, \vi, \opset}_{\ET_{2}} \conf_2 \xrightarrowtriangle{\cl', \vi', \opset'}_{\ET_{2}} \conf_3
\]
Because \( ET_1 \)  and \( \ET_2 \) are commutative, there exists a configuration \( \conf'_2 \) such that:
\[
    \conf_1 \xrightarrowtriangle{\cl', \vi', \opset'}_{\ET_{1}} \conf'_2 \xrightarrowtriangle{\cl, \vi, \opset}_{\ET_{1}} \conf_3 \land 
    \conf_1 \xrightarrowtriangle{\cl', \vi', \opset'}_{\ET_{2}} \conf'_2 \xrightarrowtriangle{\cl, \vi, \opset}_{\ET_{2}} \conf_3
\]
so we have the proof that: 
\[
    \conf_1 \xrightarrowtriangle{\cl', \vi', \opset'}_{\ET_{12}} \conf'_2 \xrightarrowtriangle{\cl, \vi, \opset}_{\ET_{12}} \conf_3
\]
\end{proof}

