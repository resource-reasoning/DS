\subsection{Compositionality of \( \ET \)}
\label{sec:et-comm}
\label{sec:et-comp}

To make two execution tests \( \ET_1 \) \( \ET_2 \) compositional with respect to to function \( \CMs \),
they need to satisfy \cref{def:app-et-comm,def:match-view}.
%For all the definitions we have in \cref{fig:execution.tests},
%It is easy to adapt so that they satisfy \cref{def:noblidwrites,def:et-minimum-footprint,def:et-monotonic-postview},
Note that but \( \CP \) and \( \SI \) do not  satisfy \cref{def:conflict-commit}.
Now we can prove compositionality of \( \ET \) (\cref{thm:et-comm}).

\begin{theorem}                                                                            
\label{thm:et-comm}                          
Let $\ET_1, \ET_2$ be two execution tests have matching views
If $\ET_1$ is commutative, 
then $\CMs(\ET_1 \cap \ET_2) = \CMs(\ET_1) \cap \CMs(\ET_2)$. 
%Furthermore, if $\ET_1, \ET_2$ are commutative, then $\ET_1 \cap \ET_2$ 
%is commutative.
\end{theorem}
\begin{proof}
Given the definition of the \( \CMs(.) \) function (\cref{def:cm}), 
it suffices to prove that \( \CMs(\ET_{1} \cap \ET_{2}) \subseteq \CMs(\ET_1) \cap \CMs(\ET_2) \)
and \( \CMs(\ET_1) \cap \CMs(\ET_2) \subseteq \CMs(\ET_{1} \cap \ET{2}) \).
The former is proven by the \cref{lem:et12-in-et1-et2} and the later is proven by \cref{lem:et1-et2-in-et12}.
\end{proof}

\begin{lemma}
\label{lem:et12-in-et1-et2}
\( \CMs(\ET_{1} \cap \ET_{2}) \subseteq \CMs(\ET_1) \cap \CMs(\ET_2) \).
\end{lemma}
\begin{proof}
It suffices to prove a stronger result that \( \confOf[\ET_{1} \cap \ET_{2}] \subseteq \confOf[\ET_1] \cap \confOf[\ET_2] \).
By the definition of \confOf (\cref{def:cm}), it suffices to prove for configurations \( \conf_0 \) to \( \conf_n \) 
\begin{equation}
    \label{equ:et12-in-et1-et2}
    \begin{multlined}
        \conf_0 \in \Confs_0
    \land \conf_0 \toET{\stub}[\ET_1 \cap \ET_2] \cdots \toET{\stub}[\ET_1 \cap \ET_2] \conf_n \implies 
    \conf_0 \toET{\stub}[\ET_{1}] \cdots \toET{\stub}[\ET_{1}] \conf_n \land \conf_0 \toET{\stub}[\ET_{2}] \cdots \toET{\stub}[\ET_{2}] \conf_n 
    \end{multlined}
\end{equation}
We prove the \cref{equ:et12-in-et1-et2} by induction on the number \( n \).
\begin{itemize}
\item Base case: \(n = 0\). 
The \cref{equ:et12-in-et1-et2} holds when \( n = 0 \), because all initial configurations \( \conf_0 \) are included in the \( \confOf[\ET_1]\) and \( \confOf[\ET_2] \) by the definition of the \( \confOf \) function (\cref{def:cm}).

\item Inductive case: \(n = i+1\). Suppose the \cref{equ:et12-in-et1-et2} holds when \( n = i \) for some \( i \).
Let consider \( n = i + 1 \) and specifically the last step.
For any \( \conf_{i+1} = (\mkvs_{i+1}, \vienv_{i+1}) \) induced by \( \ET_{1} \cap \ET_2 \), 
there exist some client \( \cl \), views \( \vi, \vi' \) and fingerprint \( \fp \) such that:
\[
    \begin{array}{l}
    (\mkvs_i, \vienv_i) \toET{\cl, \fp}[\ET_{1} \cap \ET_{2}] (\mkvs_{i+1}, \vienv_{i+1}) 
    \land \vienv_{i+1} = \vienv_{i}\rmto{\cl}{\vi'} \land (\mkvs_i, \vi, \fp, \vi' ) \in \ET_{1} \cap \ET_{2}
    \end{array}
\]
Thus, it is easy to see that \( \conf_i \toET{\cl, \fp}[\ET_{1}] \conf_{i+1} \) and \( \conf_i \toET{\cl, \fp}[\ET_{2}] \conf_{i+1} \) by the \cref{lem:mono-et}. \qedhere
\end{itemize}
\end{proof}

\begin{lemma}
\label{lem:mono-et}
If $\conf \toET{\cl, \fp}[\ET] \conf'$ and $\ET \subseteq \ET'$, 
then $\conf \toET{\cl, \fp}[\ET'] \conf'$.
\end{lemma}
\begin{proof}
    Let \((\mkvs, \vienv)  = \conf \), \( (\mkvs', \vienv') = \conf' \) and \( \vi  =\vienv(\cl) \)
    By the definition of  $\conf \toET{\cl, \fp}[\ET] \conf'$ (\cref{def:cm}), we have \(\mkvs' \in \updateKV[\mkvs, \vi, \fp, \cl]\) and  \( \vienv' = \vienv\rmto{\cl}{\vi'} \) for some \( \vi' \) such that \( \ET \vdash (\mkvs, \vi) \csat \fp : (\mkvs',\vi') \).
    Given that \( \ET \subseteq \ET'\), we know \( \ET' \vdash (\mkvs, \vi) \csat \fp : (\mkvs',\vi') \) and so $\conf \toET{\cl, \fp}[\ET'] \conf'$.
\end{proof}

For each normal trace \( \tr \), there exists a new trace \( \func{minitrace}[\tr] \) such that each step takes minimum views:
\begin{align*}
    \func{minitrace}[\tr \toET{\cl,\vi} \stub \toEDGE{\cl,\fp} \mkvs', \vienv'] & \defeq 
    \begin{array}[t]{@{}l@{}}
        \mathtt{let} \ (\mkvs,\vienv) = \func{last}[\func{minitrace}[\tr]]  \\
        \mathtt{in} \ \tr \toET{\cl,\vi'} \stub \toEDGE{\cl,\fp} \mkvs', \vienv\rmto{\cl}{\vi''} \\
        \mathtt{where} \ \begin{array}[t]{@{}l@{}}
            (\vi',\vi'') = \func{minFpViews}[\ET,\mkvs,\fp,\mkvs',\cl] \\
            {} \land \vi'' = \func{minClientViews}[\ET,\mkvs,\cl]
        \end{array} 
    \end{array}
\end{align*}

\begin{lemma}
\label{lem:et1-et2-in-et12}
\( \CMs(\ET_1) \cap \CMs(\ET_2) \subseteq \CMs(\ET_{1} \cap \ET_{2}) \).
\end{lemma}
\begin{proof}
Assume \( \mkvs \in \CMs(\ET_{1} \cap \ET_{2}) \).
There are traces from both execution tests \( \tr_1, \dots, \tr_n \in \confOf[\ET_1] \) and \( \tr'_1, \dots, \tr'_m \in \confOf[\ET_2] \),
where the final kv-stores are the same one, \( \bigwedge_{ 1 \leq i \leq n, 1 \leq j \leq m} \lastConf[\tr_i]\projection{1} = \lastConf[\tr_j]\projection{1} = \mkvs \).
It suffices to pick specific two traces \( \tr \in \confOf[\ET_1]\) and \( \tr' \in \confOf[\ET_2] \),
and from which we construct a trace \( \tr'' \in \confOf[\ET_1 \cap \ET_2] \).
By \cref{prop:et.normalform} and \( \func{minitrace} \) function, we pick normal and minimum traces \( \tr \in \confOf[\ET_1]\) and \( \tr' \in \confOf[\ET_2] \).
Let \( \conf_1 \) to \( \conf_n \) be the configurations from trace \( \tr \), 
and \( \conf'_1 \) to \( \conf'_m \) be configurations from trace \( \tr' \):
\[
\conf_0 \toET{\stub}[\ET_{1}] \conf_1 \toET{\stub}[\ET_{1}] \cdots \toET{\stub}[\ET_{1}] \conf_n 
\land \conf_0 \toET{\stub}[\ET_{2}] \conf'_1 \toET{\stub}[\ET_{2}]  
\cdots \toET{\stub}[\ET_{2}] \conf'_m 
\land \conf_n\projection{1} = \conf'_m\projection{1} 
\]
There exists configurations from \( \conf''_1\)  to \( \conf''_k \) from trace \( \ET_1 \cap \ET_2 \):
\begin{centermultline}[equ:et1-et2-in-et12]
\conf_0 \toET{\stub}[\ET_1 \cap \ET_2] \conf''_1 \toET{\stub}[\ET_1 \cap \ET_2] \cdots \toET{\stub}[\ET_1 \cap \ET_2] \conf''_k 
\land \conf_n\projection{1} = \conf'_m\projection{1} = \conf''_k\projection{1}
\end{centermultline}
and assuming that the last fingerprint \( \fp \) committed of an arbitrary client \( cl \) is at i-\emph{th} step ( \( i \leq k\) ) in the trace \( \tr'' \), let
\begin{itemize}
\item \( \mkvs = \conf''_i\projection{1} \)
\item \( \mkvs' = \conf''_{i+1}\projection{1} \) 
\item \( \vi = \conf''_i\projection{2}(\cl) \), and
\item \( \vi' = \conf''_{i+1}\projection{2}(\cl) \), 
\end{itemize}
then the views \( \vi, \vi' \) are the one given by matching view definition (\cref{def:match-view}):
\begin{centermultline}[eq:minimum-view-for-last-fp]
    (\vi,\vi') = \func{minFpView}[\ET_1 \cap \ET_2, \mkvs,\fp,\mkvs',\cl] \land \vi' = \func{minClientViews}[\ET_1 \cap \ET_2, \mkvs, \cl]
\end{centermultline}
and since i-\emph{th} step the view for the client \( \cl \) never changes:
\begin{centermultline}[eq:minimum-view-fix-after-last]
    \conf''_k\projection{2}(\cl) = \vi'
\end{centermultline}
\Cref{eq:minimum-view-for-last-fp,eq:minimum-view-fix-after-last} mean that the view for any client will fix to be the one given by \cref{def:match-view} if there is no further fingerprint.
We prove \cref{equ:et1-et2-in-et12} by induction on the length \( m \) of the trace of \( \ET_2 \).
\begin{itemize}
    \item \caseB{\(m = 0\)}
We have the trace of \( \ET_1 \):
\begin{equation}
    \label{equ:trace-view-shift-et1}
    \conf_0 \in \Confs_0 \land \conf_0 \toET{\stub}[\ET_1] \dots \toET{\stub}[\ET_1] \conf_n
\end{equation}
for some number \( n \) and configurations from \( \conf_0 \) to \( \conf_n \) and the trace of \( \ET_2 \) with only one configuration:
\begin{equation}
    \label{equ:trace-singleton-et2}
    \conf_0
\end{equation}
By the hypothesis we have \( \conf_0\projection{1} = \conf_n\projection{1} \), which means that all the steps from the trace of \( \ET_1 \) are view shift.
We can pick the trace of \( \ET_2 \) (\cref{equ:trace-view-shift-et1}) as the trace of \( \ET_1 \cap \ET_2 \):
\begin{equation}
    \label{equ:trace-view-shift-et12}
    \conf_0 
\end{equation}
It is easy to see it satisfies \cref{eq:minimum-view-for-last-fp,eq:minimum-view-fix-after-last}, since nothing has been committed.

\item \caseI{\(m + 1\)}
Suppose that \cref{equ:et1-et2-in-et12} holds when the trace length is \( m \).
Let consider \( m + 1 \).
We have the trace for \( \ET_1 \):
\begin{equation}
    \conf_0 \toET{\stub}[\ET_{1}] \conf_1 \toET{\stub}[\ET_{1}] \cdots \toET{\stub}[\ET_{1}] \conf'_{n} 
\end{equation}
for some number \( n \) and the configurations from \(\conf_0\) to \( \conf_n \), and the trace of \(\ET_2\):
\begin{equation}
    \conf_0 \toET{\stub}[\ET_{2}] \conf'_1 \toET{\stub}[\ET_{2}] \cdots \toET{\stub}[\ET_{2}] \conf'_{m+1} 
\end{equation}
Assume a client \( \cl'_{m} \), views \( \vi'_{m}, \vi'_{m+1} \) and a fingerprint \( \fp'_{m} \) that is committed to the second last configuration \( (\mkvs'_m, \vienv'_m) = \conf'_m \) in the trace of \( \ET_2 \) which yields the final configuration \( (\mkvs'_{m+1}, \vienv'_{m+1}) = \conf_{m+1} \):
\begin{centermultline}
        (\mkvs'_i, \vienv'_i) \toET{\cl'_{m}, \fp'_{m}}[\ET_2] (\mkvs'_{m+1}, \vienv'_{m+1}) 
\end{centermultline}
There are two cases: \textbf{(i)} \( \fp'_m = \epsilon \) and \textbf{(ii)} \( \fp'_m \neq \epsilon \).
\begin{itemize}
    \item If \( \fp'_m = \epsilon \), by the \cref{lem:no-effect-for-empty-fingerprint} we know \( \conf'_{m}\projection{1} = \conf'_{m+1}\projection{1}\) from the trace of \( \ET_2 \).
Since \( \conf'_{m+1}\projection{1} = \conf_n\projection{1}\) where \( \conf_n \) is the final configuration of the trace of \( \ET_1 \), we now have \( \conf'_{m}\projection{1} = \conf_n\projection{1}\).
Applying \ih that \cref{equ:et1-et2-in-et12} holds when the length is \( m \), so there exist an trace \( \tr'' \) from configurations from \( \conf''_1 \) to \( \conf''_k \) that satisfies \cref{eq:minimum-view-for-last-fp,eq:minimum-view-fix-after-last}:
\begin{centermultline}[equ:ih-for-k-length]
    \tr'' = \conf_0 \toET{\stub}[\ET_1 \cap \ET_2] \conf''_1 \toET{\stub}[\ET_1 \cap \ET_2] \cdots \toET{\stub}[\ET_1 \cap \ET_2] \conf''_k
    \land \conf_n\projection{1} = \conf'_m\projection{1} = \conf''_k\projection{1} 
\end{centermultline}
Given the definition of the reduction (\cref{def:reduction}), when \( \fp'_m = \epsilon \) we know \( \conf'_m\projection{2}(\cl_{m+1}) \sqsubseteq  \conf'_{m+1}\projection{2}(\cl_{m+1})\).
Yet we can still pick trace \( \tr'' \) which satisfies \cref{equ:et1-et2-in-et12,eq:minimum-view-for-last-fp,eq:minimum-view-fix-after-last}.

\item If \( \fp' \neq \epsilon \), 
    the final step of trace \( \tr' \) is:
    \begin{centermultline}[equ:last-et-2]
        (\mkvs'_i, \vienv'_i) \toET{\cl'_{m}, \fp'_{m}}[\ET_2] (\mkvs'_{m+1}, \vienv'_{m+1}) \land \ET_2 \vdash (\mkvs'_m, \vi'_m) \csat \fp'_m  : (\mkvs'_{m+1},\vi'_{m+1}) \\
        \quad {} \land \vi_i' = \vienv'_m(\cl'_m) \land \vienv'_{m+1} = \vienv'_i\rmto{\cl'_i}{\vi'_{m+1}}
    \end{centermultline}
    and by \cref{lem:identical-step} there exists a step \( (\cl_j, \fp_j) \) from the trace of \( \ET_1 \) such that:
\begin{centermultline}[equ:j-th-step]
    (\mkvs_{j}, \vienv_{j}) \toET{\cl_{j}, \fp_{j}}[\ET_1] (\mkvs_{j + 1}, \vienv_{j + 1}) 
    \land \ET_1 \vdash (\mkvs_{j}, \vi_j) \csat \fp_j : (\mkvs_{j+1},\vienv_{j + 1}(\cl_{j}) ) \land \vi_j = \vienv_{j}(\cl_j)
\end{centermultline}
for some \( j, \cl_j, \vi_j\) and \( \fp_j \) such that \( 0 \leq  j < n \), \( \cl_j = \cl'_{m}\), \( \fp_j = \fp'_{m}\), and
\[ 
    \fora{\key} (\stub, \key, \stub ) \in \fp_j \implies \vi_j(\key) = \vi'_{m}(\key)
\]
We apply the commutativity of \( \ET_1 \) until the step shown in \cref{equ:j-th-step} is at the end or the second end of the trace of \( \ET_1 \).
Let consider the next two steps, (j+1)-\emph{th} and (j+2)-\emph{th} step.
Since the trace is in normal form, the (j+1)-\emph{th} step is a view shift by a client \( \cl_{j+2} \) and (j+2)-\emph{th} step is a concrete step issued by the same client \( \cl_{j+2} \) under the view \( \vi_{j+2} \):
\begin{equation}
    \label{equ:j-plus-1-th-step}
    \begin{array}{@{}l@{}}
        (\mkvs_{j+1}, \vienv_{j+1}) \toET{\cl_{j+2}, \epsilon}[\ET_1]
        (\mkvs_{j+1}, \vienv_{j+1}\rmto{\cl_{j+2}}{\vi_{j+2}}) \toET{\cl_{j+2}, \fp_{j+2}}[\ET_1] (\mkvs_{j+3}, \vienv_{j+3}) \\
        \qquad \land \ET_1 \vdash (\mkvs_{j+1},\vi_{j+2}) \csat \fp_{j+2} : (\mkvs_{j+3},\vienv_{j+3}(\cl_{j+2}) )
    \end{array}
\end{equation}
It is known that the client  \( \cl_{j+2} \) is different from \( \cl_j \) (\cref{lem:different-cl}) and \( \fp_{j+2} \) writes different keys from \( \fp_j\) (\cref{lem:different-writes}). 
Because \( \cl_j \neq \cl_{j+2} \) we can swap the view shift step shown in \cref{equ:j-plus-1-th-step} before the j-\emph{th} step shown in \cref{equ:j-th-step} which gives the following:
\begin{centermultline}[equ:swap-the-view-shift-et1]
    (\mkvs_{j}, \vienv_{j}) \toET{\cl_{j+2}, \epsilon}[\ET_1] (\mkvs_{j}, \vienv_{j}\rmto{\cl_{j+2}}{\vi_{j+2}}) \toET{\cl_{j}, \fp_{j}}[\ET_1] \\
    (\mkvs_{j + 1}, \vienv_{j + 1}\rmto{\cl_{j+2}}{\vi_{j+2}}) \toET{\cl_{j+2}, \fp_{j+2}}[\ET_1] (\mkvs_{j+3}, \vienv_{j+3})
\end{centermultline}
Now let discuss the (j+2)-\emph{th} step.
Similarly by the \cref{lem:identical-step}, there is a step \((\cl_p, \fp_p)\) from the trace of \( \ET_2 \) such that \( \cl_p = \cl_{j+2}\) and \( \fp_p = \fp_{j+2}\) and \( p < m \).
Note that the last step from \( \ET_2 \), \ie (m+1)-\emph{th} step, is not a view shift therefore the m-\emph{th} step must be a view shift so the p-\emph{th} step must be before  m-\emph{th} step.
This means the fingerprint \( \fp_p \) does not observe any change by (m+1)-\emph{th} step from the trace of \( \ET_2 \).
Therefore \( \vi_{j+2} \) does not observe any change by j-\emph{th} step from the trance of \( \ET_1\), \ie \( \vi_{j+2} \in \Views(\mkvs_j) \).
By \cref{prop:et-comm-matching-vi-env}, that allows to swap the two adjacent non-conflict steps from \cref{equ:swap-the-view-shift-et1}.
It follows a new kv-stores \( \mkvs'''_{j+2}\) and a new view environment \( \vienv'''_{j+2} \) such that:
\begin{centermultline}[equ:swap-step-et1]
    (\mkvs_{j}, \vienv_{j}) \toET{\cl_{j+2}, \epsilon}[\ET_1] (\mkvs_{j}, \vienv_{j}\rmto{\cl_{j+2}}{\vi_{j+2}}) \toET{\cl_{j+2}, \fp_{j+2}}[\ET_1] {} \\
    (\mkvs_{j + 2}''', \vienv_{j + 2}''') \toET{\cl_{j}, \fp_{j}}[\ET_1] \stub \toET{\cl_{j}, \epsilon}[\ET_1] (\mkvs_{j+3}, \vienv_{j+3})
\end{centermultline}
In the \cref{equ:swap-step-et1} the j-\emph{th} step moves to the right of (j+2)-\emph{th} step.
Note that the last view shift of \( \cl_j \) will be combined with another view shift appear later in the trace 
(more details shown in the proof of normal trace in \cref{sec:normal-form-exist}).
We keep moving the j-\emph{th} step until it is at the end or the second end of trace of \( \ET_1 \):
\[
    \begin{array}{@{}l}
        \conf_0 \toET{\stub}[\ET_{1}] \cdots \toET{\stub}[\ET_{1}] \conf_{j-1} \toET{\stub}[\ET_{1}]
        \conf'''_{j} \toET{\stub}[\ET_{1}] \dots \toET{\stub}[\ET_{1}] \conf'''_{n-1} \toET{\cl_j, \fp_j }[\ET_{1}] \conf_{n} \lor {} \\
        \conf_0 \toET{\stub}[\ET_{1}] \cdots \toET{\stub}[\ET_{1}] \conf_{j-1} \toET{\stub}[\ET_{1}] 
        \conf'''_{j} \toET{\stub}[\ET_{1}] \dots \toET{\stub}[\ET_{1}] \conf'''_{n-2} \toET{\cl_j, \fp_j }[\ET_{1}] \conf'''_{n-1} \toET{\cl_{n-1}, \epsilon }[\ET_{1}] \conf_{n}  
    \end{array}
\]
for some new configurations from \( \conf'''_{j}\) to \( \conf'''_{n-1} \).
Note that if it is the second end, the last step must be a view shift step as shown in \cref{equ:new-et-1}.
\begin{itemize}
\item If the j-\emph{th} step is at the end of the new trace of \( \ET_1 \), we have the trace:
\begin{centermultline}[equ:new-et-1]
    \conf_0 \toET{\stub}[\ET_{1}] \cdots \toET{\stub}[\ET_{1}] \conf_{j-1} \toET{\stub}[\ET_{1}] 
    \conf'''_{j} \toET{\stub}[\ET_{1}] \dots \toET{\stub}[\ET_{1}] \conf'''_{n-1} \toET{\cl_j, \fp_j }[\ET_{1}] \conf_{n}  
\end{centermultline}
Given the hypothesis that \( \conf_{n}\projection{1} = \conf'_{m+1}\projection{1} \) and the fact that the last step of the new trace of \( \ET_1 \) (\cref{equ:new-et-1}) and the last step the trace of \( \ET_2 \) (\cref{equ:last-et-2}) are the same step, the kv-stores of the second last configurations the new trace of \( \ET_1 \) (\cref{equ:new-et-1}) and the one from the trace of \( \ET_2 \) (\cref{equ:last-et-2}) are the same \(  \conf'''_{n-1}\projection{1} = \conf'_{m}\projection{1} \).
Then by applying \ih that \cref{equ:et1-et2-in-et12} holds when the length is \( m \), there exists a trace  \( \tr'' \) of \( \ET_1 \cap \ET_2 \) that satisfies \cref{eq:minimum-view-for-last-fp,eq:minimum-view-fix-after-last}:
\begin{centermultline}[equ:ih-for-merge-two-trace]
        \tr'' = \conf_0 \toET{\stub}[\ET_1 \cap \ET_2] \dots \toET{\stub}[\ET_1 \cap \ET_2] \conf''_{k-1} 
        \land \conf'''_{n-1}\projection{1} = \conf'_{m}\projection{1} = \conf''_{k-1}\projection{1}  
\end{centermultline}
for some number \( k \) and configurations from \( \conf''_1 \) to \( \conf''_{k-1} \).
Now let consider the last step.
Let relabel the subscripts of new trace of \( \ET_1 \), from \( 0 \) to \( n \).
Since the kv-stores are the same before and after the transaction from two trace,
let \( \mkvs_m = \conf'''_{n-1}\projection{1} = \conf'_{m}\projection{1} \) and 
\( \mkvs_{m+1} = \conf_{n}\projection{1} =  \conf'_{m+1}\projection{1}\).
By \cref{equ:last-et-2} and \cref{equ:new-et-1}, we have:
\begin{centermultline}[equ:et1-et2-csat]
    \ET_1 \vdash ( \mkvs_m, \conf'''_{n-1}\projection{2}(\cl_{m}) )  \csat \fp_{m} : (\mkvs_{m+1} , \conf'''_{n}\projection{2}(\cl_{m}) ) \\
        {} \land \ET_2 \vdash ( \mkvs_m, \conf'_{m}\projection{2}(\cl_{m}) )  \csat \fp_{m} : ( \mkvs_{m+1} , \conf'_{m+1}\projection{2}(\cl_{m}) ) 
\end{centermultline}
where the views have the following properties:
\begin{itemize}
    \item \( (\conf'''_{n-1}\projection{2}(\cl_{m}), \conf'''_{n}\projection{2}(\cl_{m}) )  = \func{minFpViews}[\ET_1, \mkvs_m, \fp_m,\mkvs_{m+1},\cl_m] \)
    \item \( \conf'''_{n}\projection{2}(\cl_{m}) = \func{minClientViews}[\ET_1,\mkvs_{m+1},\cl_m] \)
    \item \( (\conf'_{m}\projection{2}(\cl_{m}), \conf'_{m+1}\projection{2}(\cl_{m}) )  = \func{minFpViews}[\ET_2, \mkvs_m, \fp_m,\mkvs_{m+1},\cl_m] \)
    \item \( \conf'_{m+1}\projection{2}(\cl_{m}) = \func{minClientViews}[\ET_2,\mkvs_{m+1},\cl_m] \)
\end{itemize}
Then by the matching view definition (\cref{def:match-view})
there exists views \( \vi, \vi' \) such that:
\begin{centermultline}[equ:et12-csat]
    (\vi,\vi') = \func{minFpViews}[\ET_1 \cap \ET_2, \mkvs_m,\fp,\mkvs_{m+1},\cl_m]  \\
    {} \land \ET_1 \cap \ET_2 \vdash (\mkvs_m,\vi) \csat \fp : (\mkvs_{m+1}, \vi' ) \\
    {} \land \conf'''_{n-1}\projection{2}(\cl_{m}) \viewcup  \conf'_{m}\projection{2}(\cl_{m}) \viewleq \vi \\
    {} \land \conf'''_{n}\projection{2}(\cl_{m}) \viewcup  \conf'_{m+1}\projection{2}(\cl_{m}) \viewleq \vi' = \func{minClientViews}[\ET_1 \cap \ET_2, \mkvs_{m+1},\cl]
\end{centermultline}
By \cref{eq:minimum-view-for-last-fp,eq:minimum-view-fix-after-last}, there exists an configuration, says the p-\emph{th} where \( p \leq (k - 1) \):
\begin{centermultline}
    \conf''_{p}\projection{2}(\cl_m) = \conf''_{k-1}\projection{2}(\cl_m)
\end{centermultline}
Yet the p-\emph{th} kv-store contains less information than (k-1)-\emph{th} one:
\begin{centermultline}
    \min(\ET_1,\mkvs_p,\cl_m) \viewcup \min(\ET_2,\mkvs_p,\cl_m) \viewleq \min(\ET_1 \cap \ET_2,\mkvs_p,\cl_m)  \viewleq \min(\ET_1 \cap \ET_2,\mkvs_{k-1},\cl_m) 
\end{centermultline}
Given above and \cref{equ:et12-csat} (especially the definition of \func{minFpViews}), it is known:
\begin{centermultline}
    \conf''_{k-1}(\cl_m) \viewleq \vi
\end{centermultline}
Therefore the \cref{equ:et1-et2-in-et12} holds when length is \( m + 1\) by appending a view shift of client \( \cl_m \) and the fingerprint shown in \cref{equ:et12-csat} to the trace shown in \cref{equ:ih-for-merge-two-trace}:
\begin{centermultline}
    \conf_0 \toET{\stub}[\ET_1 \cap \ET_2] \dots \toET{\stub}[\ET_1 \cap \ET_2] (\mkvs''_{k-1}, \vienv''_{k-1}) \toET{\cl_m, \vi}[\ET_1 \cap \ET_2] \stub \toET{\cl_m, \fp_m}[\ET_1 \cap \ET_2] (\mkvs''_{k+1}, \vienv''_{k+1}) \\
    {} \land \conf'''_{n}\projection{1} = \conf'_{m+1}\projection{1} = \mkvs''_{k+1}
    \land  \vienv''_{k+1} = \vienv''_{k-1}\rmto{\cl_m}{\vi'}
\end{centermultline}

\item If the j-\emph{th} step is the second last step of the new trace of \( \ET_1 \), we have the trace:
\begin{centermultline}[equ:new-et-1-with-view-shift-tail]
\conf_0 \toET{\stub}[\ET_{1}] \cdots \toET{\stub}[\ET_{1}] \conf_{j-1} \toET{\stub}[\ET_{1}] 
\conf'''_{j} \toET{\stub}[\ET_{1}] \dots 
\toET{\stub}[\ET_{1}] \conf'''_{n-2} \toET{\cl_j, \fp_j }[\ET_{1}] 
\conf'''_{n-1} \toET{\cl_{n-1}, \epsilon }[\ET_{1}] \conf_{n}
\end{centermultline}
Since the last step is a view shift, we know \( \conf_n\projection{1} = \conf'''_{n-1}\projection{1}\), and the rest of proof is the same as the case where j-\emph{th} is the last step as shown in \cref{equ:new-et-1}. \qedhere
\end{itemize}
\end{itemize}
\end{itemize}
\end{proof}

\begin{lemma}[No effect from empty fingerprint and epsilon reduction]
    \label{lem:no-effect-for-empty-fingerprint}
    \label{lem:no-effect-for-view-shift}
    \[
    \fora{\conf, \conf', \cl,\vi} \conf \toET{\cl, \unitO}[\ET] \conf' \lor \conf \toET{\cl, \epsilon}[\ET] \conf' \implies \conf\projection{1} = \conf'\projection{1}
    \]
\end{lemma}
\begin{proof}
    Let \((\mkvs, \vienv)  = \conf \) and \( (\mkvs', \vienv') = \conf' \).
    For the case of empty fingerprint,
    by the definition of  $\conf \toET{\cl, \unitO} \conf'$ (\cref{def:reduction}), we have \(\mkvs' \in \updateKV[\mkvs, \vi, \unitO, \cl]\), and therefore \( \mkvs' = \mkvs \).
    For the case of view shift, by the definition of  $\conf \toET{\cl, \epsilon} \conf'$ (\cref{def:reduction}) it is easy to see \( \mkvs' = \mkvs \).
\end{proof}

We define a \(  \mkvs(\txid) \) function that returns the fingerprint associate with the transaction identifier \( \txid \):
\begin{align*}
    \mkvs(\txid) & \defeq \Set{(\otW, \key, \val)}[\exsts{i} \mkvs(\key)(i) = (\val, \txid, \stub)] \cup  \Set{(\otR, \key, \val)}[\exsts{i,\txidset} \mkvs(\key)(i) = (\val, \stub, \txidset) \land \txid \in \txidset]
\end{align*}

\begin{lemma}[Transactions persistence]
    \label{lem:mono-fingerprint}
    \[
        \fora{\ET,\conf,\conf',\txid,\fp} \conf\projection{1}(\txid) = \fp \land \conf \toET{\stub}[\ET] \conf' \implies \conf'\projection{1}(\txid) = \fp
    \]
\end{lemma}
\begin{proof}
    It is easy to prove this by case analysis on the reduction relation.
\end{proof}

\begin{lemma}[Same steps]
\label{lem:identical-step}
Given a trace of \( \ET_1 \) and a trace of \( \ET_2 \),
if the have the same final kv-store,
the trace contains the same concrete steps (free variables are globally quantified):
\begin{multline*}
    \conf_0 \toET{\cl_1, \fp_1}[\ET_{1}] \cdots \toET{\cl_n, \fp_n}[\ET_{1}] \conf_n \land
    \conf_0 \toET{\cl'_1, \fp'_1}[\ET_{2}] \cdots \toET{\cl'_m, \fp'_m}[\ET_{2}] \conf'_m 
    \land \conf_n\projection{1} = \conf'_m\projection{1} \\
    \quad \implies \fora{i: 0 < i \leq n} 
    \fp_i = \epsilon 
    \lor \exsts{j: 0 < j \leq m}  \\
    \cl_i = \cl'_j \land \fp_i = \fp'_j \land ( \fora{\key} (\otR, \key, \stub) \in \fp_i \implies \max(\vi_i(\key)) = \max(\vi'_j(\key)) )
\end{multline*}
\end{lemma} 
\begin{proof}
    We prove by contradiction.
    First because \( \mkvs_n = \mkvs'_m \), we know that:
    \begin{equation}
        \label{equ:same-kv-store}
        \fora{\txid, \fp} \mkvs_n(\txid) = \fp \iff \mkvs'_m(\txid) = \fp
    \end{equation}
    Let \(\conf_n = (\mkvs_n,\vienv_n) \) and \(\conf'_m = (\mkvs'_m,\vienv'_m) \).
    Assume a step \( \conf_i \toET{\cl, \fp }[\ET_1] \conf_{i+1} \)  from the trace of \( \ET_1 \) where the transaction identifier is \( \txid \) and \( \fp \neq \unitO \).
    It must have a step from the trace of \( \ET_2 \), which commits some fingerprint via the same transaction identifier  \( \txid \).
    We know \( \mkvs_n(\txid) = \fp \) by \cref{lem:mono-fingerprint}, thus \( \mkvs'_m(\txid) = \fp \) by \cref{equ:same-kv-store}.
    Let assume a key \( \key \) that \( (\otR, \key, \stub) \in \fp\).
    %where \( \vi_i\) and \( \vi'_j\) are the views immediate before the commit of the fingerprint \( \fp \) in traces of \( \ET_1\) and \( \ET_2 \) respectively.
    Since that the final kv-stores are the same, that means the new transaction added itself to the reader set of the same version.
    Thus, \( \max(\vi(\key)) = \max(\vi'(\key)) \).
\end{proof}

We define \( \max_\cl(\conf) \) function that returns the most recent transaction identifier for client \( \cl \) in the configuration \( \conf \) 
\begin{align*}
    \max_\cl((\mkvs, \vienv)) & \defeq \max\Set{\txid^{n}_\cl}[\txid^{n}_\cl \text{ appear in } \mkvs] \\
\end{align*}

\begin{lemma}[Transactions from different clients]
\label{lem:different-cl}
Given a trace of \( \ET_1 \) and a trace of \( \ET_2 \),
if the i-\emph{th} step from \( \ET_1 \) issued by the client \( \cl_i \) 
is the same as the last step from \( \ET_2 \),
then in the trace of \( \ET_1 \) 
there is no concrete step issued by the client \(\cl_i \) after the i-\emph{th} step (free variables are globally quantified):
\[
\begin{array}{@{}l}
    \conf_0 \toET{\cl_1, \fp_1}[\ET_{1}] \cdots \toET{\cl_n, \fp_n}[\ET_{1}] \conf_n 
    \land \conf_0 \toET{\cl'_1, \fp'_1}[\ET_{2}] \cdots \toET{\cl'_m, \fp'_m}[\ET_{2}] \conf'_m 
    \land \conf_n\projection{1} = \conf'_m\projection{1} 
    \land \fp'_m \neq \unitO \\
    \quad {} \land \exsts{i}  
    \cl_i = \cl'_m
    \land \fp_i = \fp'_m 
    \implies \fora{j > i} 
    \fp_j = \epsilon \lor \fp_j = \unitO \lor \cl_i \neq \cl_j
\end{array}
\]
\end{lemma}
\begin{proof}
    We prove by deriving contradiction.
    Assume the last step of the trace of \( \ET_2 \) is:
    \begin{equation}
        \label{equ:last-step-for-cl-et2}
        \conf'_{m-1} \toET{\cl'_m, \fp'_m}[\ET_{2}] \conf'_m
    \end{equation}
    Assume a step of the trace of \( \ET_1 \):
    \begin{equation}
        \label{equ:identical-step-for-cl-et1}
        \conf_{i-1} \toET{\cl_i, \fp_i}[\ET_{1}] \conf_i
    \end{equation}
    where \( \cl_i = \cl'_m \) and \( \fp_i = \fp'_m \).
    Because these two steps (\cref{equ:last-step-for-cl-et2} and \cref{equ:identical-step-for-cl-et1}) are issued by the same transaction identifier,
    we know \( \max{}_{\cl_m}(\conf'_m) = \max{}_{\cl_m}(\conf_i) \).
    Assume that there exists a step from the trace of \( \ET_1 \), says j-\emph{th} step, such that:
    \[
        \conf_{j-1} \toET{\cl_j, \fp_h}[\ET_{1}] \conf_j \land j > i \land \fp_j \neq \unitO \land \cl_i = \cl_j 
    \]
    Therefore we have \( \max{}_{\cl_m}(\conf_j) > \max{}_{\cl_m}(\conf_i) \) by \cref{lem:kv-max-cl}.
    That means \( \max{}_{\cl_m}(\conf_n) > \max{}_{\cl_m}(\conf_j) > \max{}_{\cl_m}(\conf_i) = \max{}_{\cl_m}(\conf'_m) \), which contradicts to \( \conf_n\projection{1} = \conf'_m\projection{1}\).
\end{proof}

\begin{lemma}[Reduction following session order]
\label{lem:kv-max-cl}
\begin{centermultline}
    \fora{\conf, \conf' ,\cl, \fp, \ET}
    \conf \toET{\cl, \fp}  \conf'  \\
    {} \land 
    \left( 
        \fp \neq \unitO \implies \max{}_\cl(\conf) < \max{}_\cl(\conf') )
        \lor ( \fp = \unitO \implies \max{}_\cl(\conf) = \max{}_\cl(\conf')
    \right)
\end{centermultline}
\end{lemma}
\begin{proof}
    Assume a step \( (\mkvs, \vienv) \toET{\cl, \fp} (\mkvs', \vienv') \).
    By the definition of \( \toET{\stub}[\ET]\) (\cref{def:reduction}), we know \( \mkvs' \in \updateKV[\mkvs, \vi, \fp, \cl] \).
    The \( \updateKV[\mkvs, \vi, \fp, \cl] \) picks a fresh transaction identifier \( \txid_\cl^{m} \) that is greater than any transaction identifiers \( \txid_\cl^{n} \) in \( \mkvs \) via \( \nextTxid \) function, \ie \( m > n \).
    If the fingerprint \( \fp \) is not empty, the new identifier appears in \( \mkvs' \), so \( \max{}_\cl(\conf) < \max{}_\cl(\conf') \).
    Otherwise  the fingerprint is empty, the new identifier will not appear anywhere in \( \mkvs' \), so \( \max{}_\cl(\conf) = \max{}_\cl(\conf') \). 
\end{proof}

\begin{lemma}[Writing different keys]
\label{lem:different-writes}
Given a trace of \( \ET_1 \) and a trace of \( \ET_2 \),
if the i-\emph{th} step from \( \ET_1 \) that writes to key \( \key \) 
is the same as the last step from \( \ET_2 \),
then in the trace of \( \ET_1 \) 
there is no concrete step writing to the key \(\key\) after the i-\emph{th} step (free variables are globally quantified):
\[
\begin{array}{@{}l}
    \conf_0 \toET{\cl_1, \fp_1}[\ET_{1}] \cdots \toET{\cl_n, \fp_n}[\ET_{1}] \conf_n \land \conf_0 \toET{\cl'_1, \fp'_1}[\ET_{2}] \cdots \toET{\cl'_m, \fp'_m}[\ET_{2}] \conf'_m 
    \land \conf_n\projection{1} = \conf'_m\projection{1} 
    \land \fp'_m \neq \unitO \\
    \quad {} \land \exsts{i} 
    \cl_i = \cl'_m
    \land \fp_i = \fp'_m
    \implies \fora{j > i} \nexists{\key} \ldotp (\otW, \key, \stub) \in \fp_j \cap \fp_i
\end{array}
\]
\end{lemma}
\begin{proof}
    We prove this by deriving contradiction.
    Assume the last step from the trace of \( \ET_2 \):
    \begin{equation}
        \label{equ:last-step-for-write-et2}
        \conf'_{m-1} \toET{\cl'_m, \fp'_m}[\ET_{2}] (\mkvs'_m, \vienv'_m )
    \end{equation}
    Assume the transaction identifier for the \cref{equ:last-step-for-write-et2} is \( \txid \), and by the definition of \( \toET{}[\ET]\) (\cref{def:reduction}) we know:
    \begin{equation}
        \label{equ:write-fingerprint}
        \fora{\key} (\otW, \key, \stub) \in \fp'_m \implies \mkvs'_m(\key)(\lvert\mkvs'_m(\key)\rvert - 1) = (\stub, \txid, \stub)
    \end{equation}
    Assume a step of the trace of \( \ET_1 \) that is issued by the same transaction identifier with the same fingerprint:
    \begin{equation}
        \label{equ:identical-step-for-write-et1}
        \conf_{i-1} \toET{\cl_i, \fp_i}[\ET_{1}] (\mkvs_i, \vienv_i)
    \end{equation}
    where \( \cl_i = \cl'_m \) and \( \fp_i = \fp'_m \).
    Given \cref{equ:write-fingerprint} and \cref{equ:identical-step-for-write-et1}, it follows:
    \[
        \fora{\key} (\otW, \key, \stub) \in \fp_i \implies \mkvs_i(\key)(\lvert\mkvs_i(\key)\rvert - 1) = (\stub, \txid, \stub)
    \]
    Assume a step, says j-\emph{th}, after i-\emph{th} step that writes to the same key:
    \[
        \conf_{j-1} \toET{\cl_j, \fp_j}[\ET_{1}] (\mkvs_j, \vienv_j) 
        \land j > i
        \land \exsts{\key} (\otW, \key, \stub) \in \fp_i \cap \fp_j
    \]
    Therefore, by \cref{lem:unique-writer} we have:
    \[
        \exsts{\key,i} (\otW, \key, \stub) \in \fp_i \cap \fp_j \land \mkvs_j(\key)(i) = \txid \land \mkvs_j(\key)(\lvert \mkvs_j(\key) \rvert - 1) \neq \txid
    \]
    Note that \( \fp_i = \fp'_m\).
    Since the writer of a version cannot be overwritten, for the final configuration of the trace of \( \ET_1 \) \((\mkvs_n, \vienv_n)\), we know:
    \[
        \exsts{\key,i} (\otW, \key, \stub) \in \fp_i \cap \fp_j \land \mkvs_n(\key)(i) = \txid \land \mkvs_n(\key)(\lvert \mkvs_n(\key) \rvert - 1) \neq \txid
    \]
    Last, by \cref{equ:write-fingerprint} and \( \fp_i = \fp'_m\), it follows:
    \[
        \exsts{\key} (\otW, \key, \stub) \in \fp'_m \land \mkvs_n(\key)(\mkvs_n(\key) - 1) \neq \txid \land \mkvs'_m(\key)(\lvert\mkvs'_m(\key)\rvert - 1)\projection{2} = \txid
    \]
    which contradicts with \( \mkvs'_m = \mkvs_n\).
\end{proof}

\begin{lemma}[Version persistence]
    \label{lem:unique-writer}
    \[
    \begin{array}{@{}l}
        \fora{\mkvs, \mkvs',\vienv,\vienv', \cl, \vi, \fp, i} 
        (\mkvs, \vienv) \toET{\cl, \fp}[\ET] (\mkvs', \vienv')
        \land (\otW, \key, \stub) \in \fp  
        \land 0 \leq i < \lvert \mkvs'(\key) \rvert - 1 \\
        \quad {} \implies \mkvs'(\key)(i)\projection{2} \neq \mkvs'(\key)(\lvert \mkvs'(\key) \rvert - 1)\projection{2}
    \end{array}
    \]
\end{lemma}
\begin{proof}
    By the definition of \( \toET{\stub}[\ET] \) (\cref{def:reduction}), the \( \mkvs' \in \updateKV[\mkvs, \vi, \fp, \cl] \).
    Given the definition of \( \updateKV[\mkvs, \vi, \fp, \cl]\), it the picks a fresh transaction identifier \( \txid \) such that does not appear in \( \mkvs \).
    For any write fingerprint \( (\otW, \key, \stub) \in \fp \), a new version is appended to the end of the key \( \key \) and the writer (the second projection) is assigned to be the fresh identifier \( \txid \).
    Thus we have the proof.
\end{proof}
