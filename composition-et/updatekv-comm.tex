\subsection{Commutativity \( \updateKV \)}
\label{sec:updatekv-comm}

By \cref{cor:updatekv.singlecell}, then \cref{prop:updatekv.comm} holds, that is,
given two non-conflict commits, it is possible to swap the commit order as shown in \cref{prop:updatekv.comm}.


\begin{proposition}
\label{prop:updatekv.comm}
\label{prop:swap-update}
Let $\mkvs \in \MKVSs$, $\vi_1, \vi_2 \in \Views(\mkvs)$ and let $\cl_1, \cl_2 \in \Clients$ 
be such that $\cl_1 \neq \cl_2$. 
Let also $\fp_1, \fp_2 \in \pset{\Ops}$ be such that 
whenever $(\otW, \key, \stub) \in \fp_1$ for some key $\key$, then 
$(\otW, \key, \val) \notin \fp_2$ for all $\val \in \Val$. Then 
\begin{centermultline}
\Set{ \updateKV[\mkvs_1, \vi_2, \fp_2, \cl_2]}[\mkvs_1 \in \updateKV[\mkvs, \vi_1, \fp_1, \cl_1]] =  \\
\Set{ \updateKV[\mkvs_2, \vi_1, \fp_1, \cl_1]}[\mkvs_2 \in \updateKV[\mkvs, \vi_2, \fp_2, \cl_2]]
\end{centermultline}
\end{proposition}

\begin{proof}
Assume $\mkvs_1 = \updateKV[\mkvs, \vi_1, \fp_1, \txid_1]$, $\mkvs_2 = \updateKV[\mkvs, \vi_2, \fp_2, \txid_2]$. 
It suffices to show that for any key $\key$:
\[\lvert \updateKV[\mkvs_1, \vi_2, \fp_2, \txid_2](\key) \rvert = \lvert 
\updateKV[\mkvs_2, \vi_1, \fp_1, \txid_1](\key) \rvert
\]
and for any index $i$ such that \( 0 \leq i < \lvert \updateKV[\mkvs_1, \vi_2, \fp_2, \txid_2](\key) \rvert \):
\[
\updateKV[\mkvs_1, \vi_2,\fp_2, \txid_2](\key, i) = \updateKV[\mkvs_2, \vi_1, \fp_1, \txid_1](\key_1)
\]

First, fix a key $\key \in \Keys$. Note that if $(\otW, \key, \stub) \in \fp_1$, then 
by Corollary \ref{cor:updatekv.singlecell} we have that $\lvert \updateKV[\mkvs, \vi_1, \fp_1, \txid_1](\key) \rvert = 
\lvert \mkvs(\key) \rvert$. Because $\fp_1$ is not conflicting with $\fp_2$, it must be the case 
that $\fora{ \val } (\otW,\key,\val) \notin \fp_2$, and therefore by \cref{cor:updatekv.singlecell} 
we have that 
\[
\lvert \updateKV[\mkvs_1, \vi_2, \fp_2, \txid_2](\key) \rvert = \lvert \mkvs_1(\key) \rvert = \lvert \mkvs(\key) \rvert + 1.
\] 
Similarly, because $\fora{ \val } (\otW,\key,\val)\notin \fp_2$ 
and $(\otW,\key,\stub) \in \fp_1$, then 
\[
\lvert \updateKV[\mkvs_2, \vi_1, \fp_1, \txid_1](\key) \rvert = \lvert \mkvs_2(\key) \rvert + 1 
= \lvert \updateKV[\mkvs, \vi_2, \fp_2, \txid_2](\key) \rvert = \lvert \mkvs(\key) \rvert + 1
\]
Therefore, if $(\otW, \key, \stub) \in \fp_1$, we have that 
\[ 
\lvert \updateKV[\mkvs_2, \vi_1, \fp_1, \txid_1](\key) \rvert = 
\lvert \updateKV[\mkvs_1(\key), \vi_2, \fp_2, \txid_2](\key) \rvert
\]

Analogously, we can prove that this claim holds also when $(\otW, \key, \stub) \in \fp_2$. 
Finally, if $(\fora{ \val } (\otW,\key,\stub) \notin \fp_1) \land (\fora{ \val } (\otW,\key,\val) \notin \fp_2)$, 
then by \cref{cor:updatekv.singlecell} we have that 
\[
\lvert \updateKV[\mkvs_1, \vi_2, \fp_2, \txid_2](\key) \rvert = 
\lvert \mkvs_1(\key) \rvert = \lvert \updateKV[\mkvs, \vi_1, \fp_1, \txid_1](\key) \rvert = \lvert \mkvs(\key) \rvert
\]
and
\[
\lvert \updateKV[\mkvs_2, \vi_1, \fp_1, \txid_1](\key) \rvert = 
\lvert \mkvs_2(\key) \rvert = \lvert \updateKV[\mkvs, \vi_2, \fp_2, \txid_2](\key) \rvert = \lvert \mkvs(\key) \rvert
\]
This concludes the proof that, for any key $\key \in \Keys$,
\[ \lvert \updateKV[\mkvs_1,\vi_2,\fp_2,\txid_2] \rvert = 
\lvert \updateKV[\mkvs_2,\vi_1,\fp_1,\txid_1] \rvert
\]

Next, fix a key $\key$ and a index $i$ such that $0 \leq i < \abs{ \mkvs(\key) } - 1$. 
We show that:
\[ 
    \updateKV[\mkvs_1,\vi_2,\fp_2,\txid_2](\key, i) = \updateKV[\mkvs_2, \vi_1, \fp_1, \txid_1](\key, i)
\]
by performing a case analysis on $\vi_1$: 
\begin{enumerate}
    \item $i \neq \max_{<}(\vi_1(\key))$. 
In this case, by \cref{cor:updatekv.singlecell}\cref{item:updatekv.singlecell.noview}, 
we have that 
\begin{equation}
\mkvs_1(\key, i) = \updateKV[\mkvs, \vi_1, \fp_1, \txid_1](\key, i) = \mkvs(\key, i)
\label{eq:v1.nord.hh1}
\end{equation}
and 
\begin{equation}
\updateKV[\mkvs_2, \vi_1, \fp_1, \txid_1](\key, i) = \mkvs_2(\key, i)
\label{eq:v1.nord.uhh2}
\end{equation}
Then, we have three possible sub-cases: 
\begin{enumerate}
    \item $i \neq \max_{<}(\vi_2(\key))$: in this case, by \cref{cor:updatekv.singlecell}\cref{item:updatekv.singlecell.noview} we have that 
\[
\updateKV[\mkvs_1, \vi_2, \fp_2, \txid_2](\key, i) = 
\mkvs_1(\key, i) \stackrel{\cref{eq:v1.nord.hh1}}{=} \mkvs(\key, i)
\]
and
\[
\updateKV[\mkvs_2, \vi_1, \fp_1, \txid_1] \stackrel{\cref{eq:v1.nord.uhh2}}{=} \mkvs_2(\key,i) = 
\updateKV[\mkvs, \vi_2, \fp_2, \txid_2](\key, i) = \mkvs(\key, i)
\]
\item $i = \max_{<}(\vi_2(\key))$, and $(\otR, \key, \stub) \notin \fp_2$. In this case the proof is analogous to the previous case, 
only \cref{cor:updatekv.singlecell}\cref{item:updatekv.singlecell.nord} needs to be applied in place 
of \cref{cor:updatekv.singlecell}\cref{item:updatekv.singlecell.noview}.
\item $i = \max_{<}(\vi_2(\key))$, and $(\otR, \key, \stub) \in \fp_2$. In this case we can apply \cref{cor:updatekv.singlecell}\cref{item:updatekv.singlecell.rd}, 
and deduce that 
\begin{equation}
\updateKV[\mkvs_1, \vi_2, \fp_2, \txid_2](\key, i) = \mkvs_1(\key, i) \oplus \Set{\txid_2}
\label{eq:v1.nord.v2.rd.uhh1}
\end{equation}
\begin{equation}
\mkvs_2(\key, i) = \updateKV[\mkvs, \vi_2, \fp_2, \txid_2](\key, i) = \mkvs(\key,i) \oplus \Set{\txid_2}
\label{eq:v1.nord.v2.rd.hh2}
\end{equation}
It follows that 
\[
\begin{array}{l}
\updateKV[\mkvs_1, \vi_2, \fp_2, \txid_2](\key, i) \stackrel{\cref{eq:v1.nord.v2.rd.uhh1}}{=} \mkvs_1(\key, i) \oplus \Set{\txid_2} \stackrel{\cref{eq:v1.nord.hh1}}{=} \mkvs(\key, i) \oplus \Set{\txid_2}\\
\updateKV[\mkvs_2,\vi_1,\fp_1, \txid_1](\key, i) \stackrel{\cref{eq:v1.nord.uhh2}}{=} \mkvs_2(\key, i) \stackrel{\cref{eq:v1.nord.v2.rd.hh2}}{=} \mkvs(\key, i) \oplus \Set{\txid_2}
\end{array}
\]
\end{enumerate}
\item $i = \max_{<}(\vi_1(\key))$, $(\otR, \key, \stub) \notin \fp_1$. This case is similar to the previous one: we can infer 
that Equations \cref{eq:v1.nord.hh1} and \cref{eq:v1.nord.uhh2} are valid in this case using \cref{cor:updatekv.singlecell}
\cref{item:updatekv.singlecell.nord}, then we can proceed by performing a case analysis on $\vi_2$ and $\fp_2$ as in the previous case.
\item $i = \max_{<}(\vi_1(\key))$, $\otR,\ ke, \stub) \in \fp_1$. We can apply \cref{cor:updatekv.singlecell}\cref{item:updatekv.singlecell.rd} 
to deduce the following: 
\begin{equation}
\mkvs_1(\key, i) = \updateKV[\mkvs, \vi_1, \fp_1, \txid_1](\key, i) = \mkvs(\key, i) \oplus \Set{\txid_1}
\label{eq:v1.rd.hh1}
\end{equation}
and
\begin{equation}
\updateKV[\mkvs_2, \vi_1, \fp_1, \txid_1](\key, i) = \mkvs_2(\key, i) \oplus \Set{\txid_1}
\label{eq:v1.rd.uhh2}
\end{equation}
We have two different sub-cases to consider: 
\begin{enumerate}
\item $i \neq \max_{<}(\vi_2(\key))$, or $i = \max_{<}(\vi_2(\key))$ with $(\otR,\key,\stub) \notin \fp_2$. In this case, we can apply either 
\cref{cor:updatekv.singlecell}\cref{item:updatekv.singlecell.noview} (if $i \neq \max_{<}(\vi_2(\key))$ ), or 
\cref{cor:updatekv.singlecell} \cref{item:updatekv.singlecell.nord} (if $i = \max_{<}(\vi_2(\key))$ and $(\otR, \key, \stub) \notin \fp_2$), 
to obtain 
\[
\updateKV[\mkvs_1, \vi_2, \fp_2, \txid_2](\key, i) = \mkvs_1(\key, i) \stackrel{\cref{eq:v1.rd.hh1}}{=} \mkvs(\key, i) \oplus \Set{\txid_1}
\]
and
\[
\updateKV[\mkvs_2, \vi_1, \fp_1, \txid_1](\key, i) \stackrel{\cref{eq:v1.rd.uhh2}}{=} \mkvs_2(\key, i) \oplus \Set{ \txid_1 } = 
\mkvs(\key, i) \oplus \Set{ \txid_1 }
\]
\item if $i = \max_{<}(\vi_2(\key))$ and $(\otR, \key, \stub) \in \fp_2$, then by \cref{cor:updatekv.singlecell}\cref{item:updatekv.singlecell.rd} 
we obtain that 
\begin{equation}
\mkvs_2(\key, i) = \updateKV[\mkvs, \vi_2, \fp_2, \txid_2](\key, i) = \mkvs(\key, i) \oplus \Set{\txid_2}
\label{eq:v1.rd.v2.rd.hh2}
\end{equation}
\begin{equation}
\updateKV[\mkvs_1, \vi_2, \fp_2, \txid_2](\key, i) = \mkvs_1(\key, i) \oplus \Set{ \txid_2}
\label{eq:v1.rd.v2.rd.uhh1}
\end{equation}
From these facts it follows that
\[
\begin{array}{l}
\updateKV[\mkvs_1, \vi_2, \fp_2, \txid_2](\key, i) \stackrel{\cref{eq:v1.rd.v2.rd.uhh1}}{=} 
\mkvs_1(\key, i) \oplus \Set{\txid_2} \stackrel{\cref{eq:v1.rd.hh1}}{=} 
(\mkvs(\key, i) \oplus \Set{\txid_1}) \oplus \Set{\txid_2 } = \mkvs(\key, i) \oplus \Set{\txid_1, \txid_2}\\
\updateKV[\mkvs_2, \vi_1, \fp_1, \txid_1](\key, i) \stackrel{\cref{eq:v1.rd.uhh2}}{=} 
\mkvs_2(\key, i) \oplus \Set{\txid_1} 
\stackrel{\cref{eq:v1.rd.v2.rd.hh2}}{=} (\mkvs(\key, i) \oplus \Set{\txid_2}) \oplus \Set{\txid_1} = \mkvs(\key, i) \oplus \Set{\txid_1, \txid_2}
\end{array}
\]
\end{enumerate}
\end{enumerate}

Next, note that if $\fora{ \val \in \Val } (\otW,\key,\val) \notin \fp_1 \land (\otW, \key, \val) \notin 
\fp_2$, then 
\[
\lvert \updateKV[\mkvs_1, \vi_2, \fp_2, \txid_2](\key) \rvert = \lvert \mkvs(\key) \rvert = 
\lvert \updateKV[\mkvs_2, \vi_1, \fp_1, \txid_1](\key) \rvert
\]
Because we have already proved that 
\[
    \fora{ i = 0,\cdots, \lvert \mkvs(\key) \rvert } \updateKV[\mkvs_1, \vi_2, \fp_2, \txid_2](\key, i) = \updateKV[\mkvs_2, \vi_1, \fp_1, \txid_1](\key, i)
\]
It follows that
\[ 
    \updateKV[\mkvs_1, \vi_2, \fp_2, \txid_2](\key) = \updateKV[\mkvs_2,\vi_1,\fp_1,\txid_1](\key)
\]
and there is nothing left to prove.

Suppose then that  either $(\otW, \key, \val) \in \fp_1$ or $(\otW,\key, \val) \in \fp_2$ 
for some $\val$. Without loss of generality, let $(\otW,\key,\val) \in \fp_1$ for some $\val \in \Val$; 
because we are assuming that $\fp_1$ does not conflict with $\fp_2$, then 
it must be the case that $\fora{ \val' \in \Val } (\otW,\key,\val') \notin \fp_2$. 
Using \cref{cor:updatekv.singlecell}\cref{item:updatekv.singlecell.nowr} and 
\cref{cor:updatekv.singlecell}\cref{item:updatekv.singlecell.wr}, 
\[
\begin{array}{l}
\updateKV[\mkvs_1, \vi_2, \fp_2, \txid_2](\key, \lvert \mkvs(\key) \rvert) = 
\mkvs_1(\key, \lvert \mkvs(\key) \rvert) = \updateKV[\mkvs, \vi_1, \fp_1, \txid_1](\lvert \mkvs(\key) \rvert) = (\val, \txid_1, \emptyset)\\
{} \land \updateKV[\mkvs_2, \vi_1,\fp_1, \txid_1](\key, \lvert, \mkvs(\key) \rvert) = (\val, \txid_1, \emptyset)
\end{array}
\]
We have now proved that if $(\otW,\key,\val) \in \fp_1$, then $\lvert \updateKV[\mkvs_1, \vi_2, \fp_2, \txid_2] \rvert = 
\lvert \updateKV[\mkvs_2, \vi_1, \fp_1, \txid_1] \rvert$, and for all 
$i=0,\cdots, \lvert \updateKV[\mkvs_1, \vi_2, \fp_2, \txid_2] \rvert - 1$, 
$\updateKV[\mkvs_1,\vi_2, \fp_2, \txid_2](\key, i) = \updateKV[\mkvs_2, \vi_1, \fp_1, \txid_1](\key, i)$. 
This concludes the proof that for any key \( \key \), $\updateKV[\mkvs_1,\vi_2,\fp_2,\txid_2](\key) = 
\updateKV[\mkvs_2,\vi_1,\fp_1,\txid_1](\key)$, and therefore 
$\updateKV[\mkvs_1, \vi_2, \fp_2, \txid_2] = \updateKV[\mkvs_2, \vi_1,\fp_1,\txid_1]$.
\end{proof}
